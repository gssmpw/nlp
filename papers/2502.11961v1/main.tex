\documentclass[a4paper,UKenglish,cleveref, autoref, thm-restate]{lipics-v2021}
%This is a template for producing LIPIcs articles. 
%See lipics-v2021-authors-guidelines.pdf for further information.
%for A4 paper format use option "a4paper", for US-letter use option "letterpaper"
%for british hyphenation rules use option "UKenglish", for american hyphenation rules use option "USenglish"
%for section-numbered lemmas etc., use "numberwithinsect"
%for enabling cleveref support, use "cleveref"
%for enabling autoref support, use "autoref"
%for anonymousing the authors (e.g. for double-blind review), add "anonymous"
%for enabling thm-restate support, use "thm-restate"
%for enabling a two-column layout for the author/affilation part (only applicable for > 6 authors), use "authorcolumns"
%for producing a PDF according the PDF/A standard, add "pdfa"

\pdfoutput=1 %uncomment to ensure pdflatex processing (mandatatory e.g. to submit to arXiv)
\hideLIPIcs  %uncomment to remove references to LIPIcs series (logo, DOI, ...), e.g. when preparing a pre-final version to be uploaded to arXiv or another public repository
 
%\graphicspath{{./graphics/}}%helpful if your graphic files are in another directory
 
\bibliographystyle{plainurl}% the mandatory bibstyle

\title{Parameterised algorithms for temporal reconfiguration problems} %TODO Please add

%\titlerunning{Dummy short title} %TODO optional, please use if title is longer than one line
\author{Tom Davot}{Univ Angers, LERIA, SFR MATHSTIC, F-49000 Angers, France}{tom.davot@univ-angers.fr}{0000-0003-4203-5140}{Supported by EPSRC grant EP/T004878/1. Work completed at University of Glasgow.}
\author{Jessica Enright}{School of Computing Science, University of Glasgow, UK}{jessica.enright@glasgow.ac.uk}{0000-0002-0266-3292}{Supported by EPSRC grant EP/T004878/1.}
\author{Laura Larios-Jones}{School of Computing Science, University of Glasgow, UK}{Laura.Larios-Jones@glasgow.ac.uk}{0000-0003-3322-0176}{}
%TODO mandatory, please use full name; only 1 author per \author macro; first two parameters are mandatory, other parameters can be empty. Please provide at least the name of the affiliation and the country. The full address is optional. Use additional curly braces to indicate the correct name splitting when the last name consists of multiple name parts.

% \author{Joan R. Public\footnote{Optional footnote, e.g. to mark corresponding author}}{Department of Informatics, Dummy College, [optional: Address], Country}{joanrpublic@dummycollege.org}{[orcid]}{[funding]}

\authorrunning{T. Davot, J. Enright, and L. Larios-Jones}

\Copyright{Tom Davot, Jessica Enright, and Laura Larios-Jones} %TODO mandatory, please use full first names. LIPIcs license is "CC-BY";  http://creativecommons.org/licenses/by/3.0/

\begin{CCSXML}
<ccs2012>
   <concept>
       <concept_id>10002950.10003624.10003633.10010917</concept_id>
       <concept_desc>Mathematics of computing~Graph algorithms</concept_desc>
       <concept_significance>500</concept_significance>
       </concept>
   <concept>
       <concept_id>10003752.10003809.10010052</concept_id>
       <concept_desc>Theory of computation~Parameterized complexity and exact algorithms</concept_desc>
       <concept_significance>500</concept_significance>
       </concept>
 </ccs2012>
\end{CCSXML}

\ccsdesc[500]{Mathematics of computing~Graph algorithms}
\ccsdesc[500]{Theory of computation~Parameterized complexity and exact algorithms}

%\supplement{}%optional, e.g. related research data, source code, ... hosted on a repository like zenodo, figshare, GitHub, ...
%\supplementdetails[linktext={opt. text shown instead of the URL}, cite=DBLP:books/mk/GrayR93, subcategory={Description, Subcategory}, swhid={Software Heritage Identifier}]{General Classification (e.g. Software, Dataset, Model, ...)}{URL to related version} %linktext, cite, and subcategory are optional

%\funding{(Optional) general funding statement \dots}%optional, to capture a funding statement, which applies to all authors. Please enter author specific funding statements as fifth argument of the \author macro.

\acknowledgements{For the purpose of open access, the author(s) has applied a Creative Commons Attribution (CC BY) licence to any Author Accepted Manuscript version arising from this submission.}%optional

\nolinenumbers %uncomment to disable line numbering



%Editor-only macros:: begin (do not touch as author)%%%%%%%%%%%%%%%%%%%%%%%%%%%%%%%%%%
\EventEditors{John Q. Open and Joan R. Access}
\EventNoEds{2}
\EventLongTitle{42nd Conference on Very Important Topics (CVIT 2016)}
\EventShortTitle{CVIT 2016}
\EventAcronym{CVIT}
\EventYear{2016}
\EventDate{December 24--27, 2016}
\EventLocation{Little Whinging, United Kingdom}
\EventLogo{}
\SeriesVolume{42}
\ArticleNo{23}
%
%\usepackage[a4paper,margin=1in]{geometry}

%%%%%%%%%%%%%%%%%%%%%%%%%%%%%%%%%%%%%%%%%%%%%%%%%%%%%%%%%%%%%%%%%%%%%%%%%%%%%%

%% Beautiful mathematics
\usepackage{amsmath, amssymb, amsfonts} 
\usepackage{nicefrac}
\usepackage{mathtools}
\usepackage{bm, bbm}
\usepackage[scr=boondoxo,scrscaled=1.05]{mathalfa}

%% References in the correct format 
%\usepackage[square,numbers]{natbib}
%\def\bibfont{\footnotesize} % fix to have the same font size as without natbib

\usepackage[sort, compress, space]{cite}            


%% Enumerate nicely 
\usepackage{enumitem}

%% Different color comments and commenting large parts of the text
\usepackage{xcolor}
\usepackage{comment}
\usepackage{soul}

%% Hyper references
\usepackage{hyperref}
\usepackage{cleveref}
%\usepackage[numbers]{natbib}

\usepackage{tikz}
%\usepackage{thm-restate}
%% Appendix package
%\usepackage{appendix}

%% Random text to test spacing 
\usepackage{blindtext}

\usepackage{afterpage}

\usepackage{algorithm, algorithmic}    



\usepackage{dsfont}

\usepackage{tikz}
\usepackage{graphicx}
\usepackage{tikzscale}
\usepackage{pgfplots}
\pgfplotsset{compat=newest}
\usepackage{xfrac}

\usepackage{thm-restate}

%\usepackage{subcaption}

\usepackage{balance}

\usepackage{cite}
\usepackage{amsmath,amssymb,amsfonts}
\usepackage{balance}
\usepackage{algorithmic}
\usepackage{graphicx}
\usepackage{textcomp}
\usepackage{xcolor}
\usepackage{amsmath}
\usepackage{amssymb}
\usepackage[mathscr]{euscript}
\usepackage{comment}
\usepackage{xcolor}
\usepackage{enumitem} 
\usepackage{amsthm}


%\usepackage{graphicx}
% Used for displaying a sample figure. If possible, figure files should
% be included in EPS format.
%
% If you use the hyperref package, please uncomment the following line
% to display URLs in blue roman font according to Springer's eBook style:
% \renewcommand\UrlFont{\color{blue}\rmfamily}

%%% REVIEW
\newcommand{\tocite}{{\color{red}CITE} }
\newcommand{\toref}{{\color{red}REF} }

%%% LOGO
\newcommand{\usc}{\raisebox{-1pt}{\includegraphics[height=0.8em]{figures/usc_logo.png}}}
\newcommand{\vuam}{\raisebox{-1pt}{\includegraphics[height=0.8em]{figures/vu_logo.png}}}

%%% SIGNS and SYMBOLS
\newcommand{\grad}{\texttt{grad-CROP}}
\newcommand{\att}{\texttt{att-CROP}}
\newcommand{\seg}{\texttt{seg}}
\newcommand{\clip}{\texttt{clip-CROP}}
\newcommand{\sam}{\texttt{sam-CROP}}
\newcommand{\yolo}{\texttt{yolo-CROP}}
\newcommand{\hc}{\texttt{human-CROP}}
\newcommand{\zsvqa}{\texttt{ZSVQA}}
\newcommand{\vic}{\textbf{ViCrop}}
\newcommand{\xmark}{\text{\ding{55}}}
\newcommand{\cmark}{\text{\ding{51}}}
\newcommand{\success}{\texttt{\color{green} \cmark}}
\newcommand{\failure}{\texttt{\color{red} \xmark}}
\newcommand{\rel}{\texttt{rel-att}}
\newcommand{\gra}{\texttt{grad-att}}
\newcommand{\pgra}{\texttt{pure-grad}}
\newcommand{\relh}{\texttt{rel-att$^h$}}
\newcommand{\grah}{\texttt{grad-att$^h$}}
\newcommand{\pgrah}{\texttt{pure-grad$^h$}}


%%% Text Abb.
\makeatletter
\DeclareRobustCommand\onedot{\futurelet\@let@token\@onedot}
\def\@onedot{\ifx\@let@token.\else.\null\fi\xspace}

\def\aka{\emph{a.k.a}\onedot} \def\Eg{\emph{E.g}\onedot}
\def\eg{\emph{e.g}\onedot} \def\Eg{\emph{E.g}\onedot}
\def\ie{\emph{i.e}\onedot} \def\Ie{\emph{I.e}\onedot}
\def\cf{\emph{c.f}\onedot} \def\Cf{\emph{C.f}\onedot}
\def\etc{\emph{etc}\onedot} \def\vs{\emph{vs}\onedot}
\def\wrt{w.r.t\onedot} \def\dof{d.o.f\onedot}
\def\etal{\emph{et al}\onedot}
\makeatletter



\definecolor{myred}{HTML}{FF8577}
\definecolor{mygreen}{HTML}{0FA958}
\definecolor{myblue}{HTML}{1982C4}
\definecolor{codegreen}{rgb}{0,0.5,0}
\definecolor{codegray}{rgb}{0.5,0.5,0.5}
\definecolor{codepurple}{rgb}{0.07,0,0.53}
\definecolor{codered}{RGB}{189,41,0}
\definecolor{codecomment}{RGB}{153,153,153}
\definecolor{backcolour}{rgb}{0.96,0.96,0.96}
\definecolor{royalblue}{rgb}{0.0, 0.14, 0.4}
\definecolor{egyptianblue}{rgb}{0.06, 0.2, 0.65}
\definecolor{royalazure}{rgb}{0.0, 0.22, 0.66}
\definecolor{portlandorange}{rgb}{1.0, 0.35, 0.21}
\definecolor{sienna}{RGB}{183,105,68}
\definecolor{saddlebrown}{RGB}{139,69,19}
\definecolor{mediumbrown}{RGB}{83,41,11}
\definecolor{darkbrown}{RGB}{58,28,7}
\hypersetup{
    colorlinks=true,
    linkcolor=sienna,
    urlcolor=royalblue,
    citecolor=royalblue,
}




\begin{document} 
\maketitle              % typeset the header of the contribution
%
\begin{abstract}
Given a static vertex-selection problem (e.g. independent set, dominating set) on a graph, we can define a corresponding temporal reconfiguration problem on a temporal graph which asks for a sequence of solutions to the vertex-selection problem at each time such that we can reconfigure from one solution to the next. We can think of each solution in the sequence as a set of vertices with tokens placed on them; our reconfiguration model allows us to slide tokens along active edges of a temporal graph.  

We show that it is possible to efficiently check whether one solution can be reconfigured to another, and show that approximation results on the static vertex-selection problem can be adapted with a lifetime factor to the reconfiguration version.
Our main contributions are fixed-parameter tractable algorithms with respect to: enumeration time of the related static problem; the combination of temporal neighbourhood diversity and lifetime of the input graph; and the combination of lifetime and treewidth of the footprint graph.

\keywords{parameterised algorithms, temporal graphs, reconfiguration}

\end{abstract}
\documentclass[../main.tex]{subfiles}
\graphicspath{{../images/}}
\makeatletter
\def\input@path{{../images/}}
\makeatother
\begin{document}
\section{Introduction}
\begin{figure}
\centering
\begin{tikzpicture}
\node[inner sep=0pt] (ws) at (0, 0) {
\includegraphics[height=.4\textwidth, trim={10cm 0 10cm 0},clip]{world_space.png}};
\node[inner sep=0pt] (cs) at (6,0) {\includegraphics[height=.4\textwidth, trim={10cm 1cm 10cm 4cm},clip]{conf_space.png}};
\end{tikzpicture}
\vspace{-5pt}
\label{fig:pbrm_intro}
\caption{\textbf{Left}: Shows world space obstacles as grey spheres. Robots start and goal configuration is colored red and green, respectively. Configurations along the computed path are colored transparent blue. \textbf{Right:} Mapped world space scenario to configuration space. Obstacle region is the grey mesh. Red spheres are collision-free regions computed by the neural SCDF. The optimized shortest path in the convex corridor is the blue curve.}
\vspace{-25pt}
\end{figure}
Motion planning is the problem of finding a collision-free trajectory that connects a given start and goal configuration. The planning takes place in the configuration space of the robot. For single body robots, like mobile robots or drones, the configuration space and the world space are usually the same. This simplifies the planning, since explicit obstacle representations are available which enables geometrical tools like separating hyperplanes, smallest distance to obstacles etc., to be used when designing motion planning algorithms. For multi-body robots like manipulators, the situation is completely different. The world space obstacles are usually mapped to non-convex regions, and to make the problem even harder, the mapping is usually not known. Forming explicit representations of the obstacle region in the configuration space is usually too expensive or intractable. Despite all of this, sampling based planners are used with great success, which mainly is due to their use of implicit representations of the obstacle region. The basic idea is to construct a graph in the configuration space that covers and connects the collision-free region. From this graph, a path can be extracted that connects a given start and goal configuration. The approach is computationally expensive, since the graph is constructed with the smallest geometrical building block available, points, which represents a collision-check. Furthermore, the extracted paths from the graph are non-smooth and jagged due to the stochastic nature of the approach. This adds an additional post-processing step to the process, where the paths are shortcutted and smoothened, before the path can be used for tracking. Clearly a lot of time is invested to form this graph and produce smooth paths. Thus, if the obstacles start to move, then all of this work is done in no use, since all points that make up this graph need to be re-verified, which is simply too time consuming to be done in real time.
\\\\
In this work, we want to address the existing drawbacks of the sampling based planners. Our main contribution is an improved motion planner where each vertex in the graph covers a collision-free region in the form of a sphere instead of a point and where the edges are formed with neighboring intersecting spheres. This representation has the advantage of instead of returning piecewise linear paths, returning a sequence of overlapping spheres, i.e. a convex corridor, that connects a given start and goal configuration, illustrated in Figure \ref{fig:pbrm_intro}. This convex corridor allows us to use convex optimization to produce smooth trajectories, instead of computationally expensive post-processing methods. The representation further allows us to estimate the coverage of the collision-free space, which gives us awareness and feedback in the offline roadmap construction phase. Finally, our representation is simple to adapt to moving obstacles, simply requery for the new radii and recheck for intersections. 
\\\\
The spherical collision-free regions are formed using a signed distance function (SDF), which is a function that returns the smallest distance from an arbitrary point to the boundary of an obstacle. As the name implies, the distance is signed, thus if the point is inside the obstacle it is negative otherwise positive. If the distance is positive, a sphere with radius equal to the distance is guaranteed to cover a collision-free region. Using an SDF in motion planning is not new, but what is novel about our approach is that we express the distance in the configuration space instead of the world space and by doing so allows us to form these convex collision-free regions. We refer to the resulting SDF as a signed configuration distance function (SCDF). Computing an SCDF analytically is non-trivial, our approach is therefore to parameterize the SCDF with a deep neural network and learn the mapping by supervised learning. Our resulting neural SCDF can compute distances for different parameter values of obstacle shapes and we also show how multiple distances can be combined, thus making our approach flexible.
\section{Related work}
Motion planning algorithms can roughly be divided into three families, grid-based, sampling based and optimization based methods. Grid-based methods (GBM) discretize the planning space from which a graph is then compiled. A standard search method is A$^\star$ \citep{a_star}, which is classified as an \textit{informed} search method, since it employs a heuristic function to speed up the search. A$^\star$ guarantees to return an optimal path at the level of discretization used. GBMs usually discretize the planning space by a regular lattice and this limits the GBMs to problems with low dimensionality due to the curse of dimensionality. Thus, GBMs are usually limited to single-body robots where the degrees of freedom (DOF) are low. To overcome the inherent scaling problem with the GBMs, stochastic methods are usually used for multi-body robots. These methods are termed as sampling-based methods (SBM) and core members within this family are the rapidly-exploring random trees (RRT) \citep{rrt} and the probabilistic roadmap (PRM) \citep{prm}. RRT grows a tree from the start configuration and explores the collision-free region in a rapid way until it is able to connect to the goal region. RRT is usually improved by bi-directional planning \citep{rrt_connect}, i.e. an additional tree is grown from the goal configuration and the trees are tested for connection after any tree has been expanded. RRT is a single-query method, thus it searches for a path from scratch each time it is queried. Contrary to this, PRM is a multi-query method, which solves for multiple queries without starting from scratch. PRM does this by creating a roadmap (graph) that covers the collision-free space as an offline step. The graph is then used to solve for multiple queries. PRMs are used in cases where the environment does not change since the extra offline step is too computationally costly and needs to be re-done if the environment is changed. In our work, we address this inherent issue by using a different roadmap representation. Our vertices in the graph cover a collision-free region in the form of spheres and we form the edges by checking for intersecting spheres. If something in the environment changes, we recompute the spheres radii and recheck the intersections, without relying on collision detection. We use a trained neural network to compute the sphere radius, therefore querying for the radius can be done fast, hence our representation enables the PRM for dynamic environments.
\\\\
In the recent decades, optimization based methods (OBM) \citep{chomp, schulman, itomp, stomp} have been introduced as an alternative to SBM for multi-body robots. Like the SBM, the OBMs scale well to higher dimensional problems and produce smoother motion. It is common to use a SDF in the optimization since it is a smooth function, thus enabling gradient-based methods. However, the standard way of expressing the SDF is in world space. The distance therefore needs to be mapped to the configuration space by the forward kinematics. This mapping makes the optimization problem a non-linear program (NLP), which is computationally expensive to solve. Recently, a different approach has been proposed. In \cite{mp_gcs} motion planning is formulated as a convex optimization problem by using the graph of convex sets framework \citep{gcs}. The underlying idea is to decompose the collision-free space into intersecting convex sets from which a convex optimization problem is formulated. In cases where an explicit representation of the obstacles in the configuration space exists, like for single-body robots, creating collision-free convex regions can be done fast \citep{iris}. For multi-body robots, this is non-trivial. Existing work does this successfully \citep{iris_nlp, iris_c} by an optimization based approach, but the methods are still too time consuming to be used in the presence of moving obstacles. Our approach is instead to use deep learning to learn an SDF expressed in the configuration space. With this, we can query for shortest distances to the collision boundary, which allows us to expand spherical regions which are collision-free. Our approach is fast and therefore enables our suggested roadmap planner to be used in dynamic environments.
\\\\
Recent research has focused on learning collision detection \citep{fk_kernel_distance, diffco, graphdistnet} by predicting the signed distance between the robot links and the surrounding obstacles in the world space. The learned SDF is used in trajectory optimization but since the distance is expressed in the world space, the problem becomes an NLP and therefore takes a long time to solve. We take a novel approach and suggest to instead express the signed distance in the configuration space. This allows us to improve the PRM at the same time as it enables convex optimization for trajectory optimization, which runs faster and is more reliable than NLP solvers. In \cite{cspf} a learned signed distance function in the configuration space is proposed similar to our approach. However, their approach is restricted to point cloud representations, while we propose to represent the obstacles as parameterized geometric shapes, e.g. spheres. Furthermore, we also show how to use our learned SCDF to improve an existing roadmap planner.
\section{Problem formulation}
A robot is located in the world space, $\W \subset \R^3 $. The unique location of the robot is given by its configuration $\q \in \C$, where $\C$ is the configuration space. The set of points covered by the robots bodies at a certain configuration is expressed as $\B(\q) \subset \W$. The robot is surrounded by $\NrObst$ obstacles $\O = \bigcup_{i=1}^{\NrObst} \O_i$, where  $\O_i \subset \W$. The representation of the obstacle in the configuration space is the set $\C\O_i = \{\q \in \C \: |\: \B(\q) \cap \O_i \neq \emptyset \}$. The obstacle space is formed as $\Co = \bigcup_{i=1}^{\NrObst} \C \O_i$. The complement is referred to as the free space, $\Cf = \C \setminus \Co$. The path planning problem is a tuple, ($\Cf$, $\qStart$, $\qGoal$), where we want to connect a query pair, consisting of a start, $\qStart$, and goal configuration, $\qGoal$, with a geometric path, $\q(s): [0, 1] \mapsto \Cf$, such that $\q(0)=\qStart$ and $\q(1)=\qGoal$, or report correctly when such a path does not exist.
\end{document}

\newcommand{\ours}{$\text{Q}$LASS}

%\input{max-edge}
\section{Preliminary results}
\label{sec:polynomial}
In the remainder of this paper we present a number of algorithmic results; here we start with several initial results first showing that we can efficiently check if a given sequence is reconfigurable, and then presenting a condition for a reconfiguration problem to belong to the same approximation class as its static version.

\subsection{Checking if a sequence is reconfigurable}

We show that testing if a sequence $(T_1,\dots,T_\tau)$ is a reconfigurable sequence of a temporal graph $\mathcal{G}=(G,\lambda)$ can be done in polynomial-time. We begin with the smaller problem of checking whether one state can be reconfigured into another. This can be done by computing a perfect matching in a bipartite graph that captures what set changes are possible between time-steps.  Details are deferred to the appendix. 

\begin{lemmarep}
  \label{lemma:check reconfiguration two sets}
Let $G$ be a static graph and let $T_1$ and $T_2$ be two subsets of vertices. We can determine in $\mathcal{O}((|V(G)| + |E(G)|) \cdot \sqrt{|V(G)|})$ time if $T_1$ is reconfigurable into $T_2$.
\end{lemmarep}  
\begin{proof}
  We first contruct a bipartite graph $H$ with vertex set $V(H) = \{v_i \mid x_i \in T_1\} \cup \{u_i \mid x_i \in T_2\}$ and edge set $E(H) = \{v_iu_i \mid x_i \in T_1\cap T_2\} \cup \{v_iu_j \mid x_i \in T_1, x_j\in T_2, x_ix_j \in E(G)\}$.
  We show that there is a reconfiguration bijection between $T_1$ and $T_2$ if and only if there is there is a perfect matching in $H$.
  Let $f$ be a reconfiguration bijection between $T_1$ and $T_2$. Consider the set of edges $M = \{ v_iu_j \mid x_i \in T_1, f(x_i) = x_j\} \cup \{ v_iu_i \mid x_i \in T_1, f(x_i) = x_i\}$. If $f(x_i)=x_i$, then $x_i \in T_1 \cap T_2$ and by construction of $H$, the edge $v_iu_i$ belongs to $H$. If $f(x_i)=x_j$, then $v_i \in T_1$, $v_2 \in T_2$ and $x_ix_j \in E(G)$ and again by construction of $H$, $v_iv_j$ belongs to $H$. Hence, $M \subseteq E(H)$ and since $f$ is a bijection between $T_1$ and $T_2$, $M$ is a perfect matching in $H$.
  %
  Now let $M$ be a perfect matching in $H$. For each edge $v_iu_j \in M$ (possibly $i=j$), we set $f(x_i)=x_j$. Clearly, each vertex of $T_1$ has eaxctly one image and each vertex of $T_2$ has exactly one inverse image, thus $f$ is a bijection. For each $f(x_i)=x_j$ (with $i\neq j$), we have $v_iu_j \in M$ and by construction of $H$, there is an edge between $x_i$ and $x_j$ in $G$. Hence, $f$ is a reconfiguration bijection.
  %
  Finally, the graph $H$ can be constructed in $\mathcal{O}(|E(G)|+|V(G)|)$ and the perfect matching can be computed in $\mathcal{O}((|E(H)|+|V(H)|) \cdot \sqrt{|V(H)|})$~\cite{Hopcroft1971ANA}. We obtain an overall complexity of $\mathcal{O}((|V(G)| + |E(G)|) \cdot \sqrt{|V(G)|})$.
\end{proof}

We call a sequence $T=(T_1,\dots,T_\tau)$ in temporal graph $\mathcal{G}$ \emph{reconfigurable} if for every $1 \leq i  < \tau$, the set $T_i$ is reconfigurable in $\mathcal{G}$ to $T_{i+1}$.  We now extend the previous result to show that testing if a sequence is reconfigurable can also be done in polynomial-time.  


\begin{corollaryrep}
   \label{cor:check reconfiguration sequence}
Let $\mathcal{G}=(G,\lambda)$ be a temporal graph. Let $T=(T_1,\dots,T_\tau)$ be a sequence such that for all $i \in [\tau]$, $T_i \subseteq V(G)$. We can determine whether $T$ is a reconfigurable sequence in $\mathcal{O}(\tau \cdot (|V(G)|+|E(G)|\cdot\sqrt{|V(G)|}))$ time.
\end{corollaryrep}

\begin{proof}
A sequence $(T_1,\dots,T_\tau)$ is reconfigurable if and only if for each $i \in [\tau-1]$, $T_i$ is reconfigurable into $T_{i+1}$ in $G_i$. Hence, by \Cref{lemma:check reconfiguration two sets}, we can test if $(T_1,\dots,T_\tau)$ is reconfigurable in $\mathcal{O}(\tau \cdot (|V(G)|+|E(G)|\cdot\sqrt{|V(G)|}))$.
\end{proof}
 

\subsection{Approximation}

We now turn our attention to the approximation of reconfiguration problems. An optimisation problem belongs to the approximation class $f(n)$-APX if it is possible to approximate this problem in polynomial-time with a $\mathcal{O}(f(n))$ approximation factor. Let $\Pi$ be a minimisation static problem. A $f$-approximation algorithm for $\Pi$ is a polynomial-time algorithm that, given a graph $G$, returns an approximate solution $X_{app}$ such that, we have $|X_{app}|\leq f(G) \cdot |X_{opt}|$, where $X_{opt}$ is an optimal solution for $\Pi$ in $G$. 
 \begin{theoremrep}
   \label{th:approx}
Let $\Pi_S$ be a minimisation static graph problem such that for any two solutions $S_1$ and $S_2$, $S_1 \cup S_2$ is also a solution. Let $\Pi_T$ be the corresponding reconfiguration version of $\Pi_S$. If $\Pi_S$ is $f$-approximable, then $\Pi_T$ is $\tau\cdot f$-approximable.
\end{theoremrep}

\begin{proof}
  Let $app_S$ be a polynomial-time approximation algorithm for $\Pi_S$.
  %
  Let $\mathcal{G}=(G,\lambda)$ be a temporal graph. Let $X_{app} = \bigcup\limits_{t\in [1,\tau]} app_S(G_t)$ be the union of every approximate solution of $\Pi_S$ in all snapshots of $\mathcal{G}$.
  We show that the algorithm $app_T$ that returns the reconfigurable sequence $T_{app} = (X_{app},\dots,X_{app})$ (i.e. $T_{app}$ always contains $X_{app}$ as set of selected vertices) is a $\tau \cdot f(G)$ approximation algorithm for $\Pi_T$.  
  %
 $T_{app}$ is obviously a reconfigurable sequence for $\Pi_T$, thus we only need to show that it achieves the desired ratio.
  For each time-step $t \in [1,\tau]$, let $X^t_{opt}$ denote the optimal solution for $\Pi_S$ in $G_t$. Let $T_{opt}=(T_1,\dots,T_\tau)$ be an optimal reconfiguration sequence for $\Pi_T$ in $\mathcal{G}$. Notice that for each time-step $t$, we have $|X^t_{opt}| \leq |T_t|$.  For all $t$, we have $|app_S(G_t)| \leq f(|G_t|) \cdot |Opt_t|$ and thus, $|app_S(G_t)| \leq f(|G_t|) \cdot |T_t|$. It follows that $|X_{app}| = |T_{app}| \leq \tau \cdot f(G) \cdot |T_{opt}|$. We can conclude that $app_T$ is a $\tau\cdot f(G)$ approximation algorithm.
\end{proof}


{\sc Dominating Set} is known to be Log-APX-complete~\cite{EscoffierP06}, \textit{i.e.} there is a polynomial $\mathcal{O}(log)$-approximation algorithm and there is no polynomial-time approximation algorithm with a constant ratio. Thus, we obtain the following result.

\begin{corollary}
{\sc Temporal Dominating Set Reconfiguration} is Log-APX-complete.
\end{corollary}



\section{Fixed-parameter tractability with respect to the enumeration time of the static version}
\label{sec:enum}

We show in this section that if the number of solutions for a static problem $\Pi_S$ can be bounded by some function $f(G)$, then the reconfiguration version is in FPT with respect to $f(G)$.

\begin{lemmarep}
  \label{lemma:enumeration}
  Let $\Pi_S$ be a static problem and let $\Pi_T$ be the reconfiguration version of $\Pi_S$. Let $\mathcal{G}=(G,\lambda)$ be a temporal graph such that for all $t \in [\tau]$, all solutions in $G_t$ for $\Pi_S$ can be enumerated in $\mathcal{O}(f(G))$. We can solve $\Pi_T$ in $\mathcal{O}(f(G) \cdot \tau \cdot (|V(G)|+|E(G)|)\cdot\sqrt{|V(G)|})$.
\end{lemmarep}


\begin{proof}
  For all $t \in [\tau]$, let $X_t$ denote the set of solutions of $\Pi_S$ in $G_t$. For each $t \in [\tau]$, denote by $Y_t \subseteq X_t$ such that for all $T_t \in Y_t$, there exists a reconfigurable sequence $(T_1,\dots,T_t)$  in the temporal graph containing the $t$ first snapshots of $\mathcal{G}$.
  Notice that $Y_\tau$ contains all set $T_\tau$ such that there is a reconfigurable sequence ending with $T_\tau$ that is a solution for $\Pi_T$. Hence, to obtain the optimal size for $\Pi_T$ in $\mathcal{G}$, it suffices to take the size of the best (minimised or maximised as required by $\Pi_S$) set  $T_\tau \in Y_\tau$ for $\Pi_S$ in $G_\tau$.
  %
  
  We now show by induction how to construct $Y_t$ for each $t \in [\tau]$.
  For the base case, we set $Y_1 = X_1$. Clearly $Y_1$ respects the induction hypothesis.
  Now, for each $t \in [2,\dots,\tau]$, we set $Y_t = \{ T_t \in X_t \mid \exists T_{t-1} \in Y_{t-1} \texttt{ st $T_{t-1}$ is reconfigurable into $T_t$ in $G_{t-1}$}\}$. Let $T_t \in Y_t$. First, since $T_t \in X_t$, $T_t$ is a solution for $\Pi_S$ in $G_t$. Then, for each $T_t \in Y_t$, there is $T_{t-1} \in Y_{t-1}$ such that $T_{t-1}$ is reconfigurable into $T_t$ in $G_{t-1}$. By induction hypothesis, there is a reconfigurable sequence $(T_1,\dots,T_{t-1})$ for $\Pi_T$ in the temporal graph containing the $t-1$ first snapshots of $\mathcal{G}$. Hence, we can conclude thay the sequence $(T_1,\dots,T_{t-1},T_t)$ is a reconfigurable sequence for $\Pi_T$ in the temporal graph containing the $t$ first snapshots of $\mathcal{G}$. Hence, all sets in $Y_t$ respect the induction hypothesis.
  Now let $(T_1,\dots,T_t)$ be a reconfigurable sequence for $\Pi_T$ in the temporal graph containing the $t$ first snapshots of $\mathcal{G}$. That is, $T_{t-1}$ is reconfigurable into $T_t$ in $G_t$ and $(T_1,\dots,T_{t-1})$ be a reconfigurable sequence for $\Pi_T$ in the temporal graph containing the $t-1$ first snapshots of $\mathcal{G}$. By the inductive hypothesis, $T_{t-1} \in Y_{t-1}$ and so, by construction of $Y_t$, $T_t$ belongs to $Y_t$. We can conclude that $Y_t$ respects the recurrence hypothesis.
  
  Since $|X_t| \leq \mathcal{O}(f(G))$, by \Cref{lemma:check reconfiguration two sets} each $Y_t$ can be computed in $\mathcal{O}(f(G)\cdot (|V(G)| + |E(G)|)\cdot \sqrt{|V(G)|})$. Hence, computing $Y_\tau$ can be done in $\mathcal{O}(f(G)\cdot \tau \cdot (|V(G)| + |E(G))|\cdot \sqrt{|V(G)|})$. Notice that it is possible to associate each set $T_t$ with a sequence $(T_1,\dots,T_t)$ to make a constructive algorithm.
\end{proof}


% \subsection{FPT with respect to maximum number of edges in any snapshot}
% \label{sec:FPT with max edges}

Let $|E|_{\max}$ denote $\max_{t\in |\tau|}|E_t|$, the maximum number of edges in any snapshot of the temporal graph $\mathcal{G}$.
We use \Cref{lemma:enumeration} to show inclusion of \textsc{Temporal Dominating Set Reconfiguration} in FPT with respect to $|E|_{\max}$. First, we show that the number of dominating sets is bounded by a function of $|E|_{\max}$ in each snapshot.

\begin{lemmarep}
  Let $\mathcal{G}=(G,\lambda)$ be a temporal graph. For each $t \in [\tau]$, all dominating sets of $G_t$ can be enumerated in $\mathcal{O}(2^{|E|_{max}} \cdot |E|_{max})$.
\end{lemmarep}

\begin{proof}
  Let $G_t$ be a snapshot of $\mathcal{G}$, $X$ be the set of vertices of degree zero in $G_t$ and $Y = V(G) \setminus X$. 
  Notice that in any dominating set $D$ of $G_t$, $X \subseteq D$. Hence, to enumerate all dominating sets in $G_t$, it suffices to list every dominating set in the subgraph induced by $Y$.  For each enumerated dominating set $D$ for $G_t[Y]$, a dominating set for $G_t$ can be obtained by taking the union of $X$ and $D$.
  To list all dominating sets $D$ for $G_t$, we enumerate all subsets of $Y$ and check for each one whether it forms a dominating set. The enumeration of all subsets of $Y$ can be done in $\mathcal{O}(2^{|Y|})$ and checking whether a subset of $Y$ is a dominating set can be done in $\mathcal{O}(|Y| + |E|_{max})$.
  Since we have $|Y| \leq |E|_{max} +1$, we can conclude that all dominating sets of $G_t$ can be enumerated in $\mathcal{O}(2^{|E|_{max}} \cdot |E|_{max})$.
\end{proof}

\noindent Now, we can conclude that \textsc{Temporal Dominating Set Reconfiguration} is in FPT with respect to $|E|_{\max}$.

\begin{corollary}
{\sc Temporal Dominating Set Reconfiguration} can be solved in $\mathcal{O}(2^{|E|_{max}} \cdot \tau \cdot (|V(G)|+|E(G)|\cdot\sqrt{|V(G)|}))$.
\end{corollary}

% \subsection{FPT with respect to minimum-size clique cover}

% Let $G$ be a static graph. A \emph{clique cover} $P = \{X_1,\dots,X_k\}$ of $G$ is a partition of its vertices into cliques. We 

% \begin{lemma}
% Let $\mathcal{G}=(G,\lambda)$ be a temporal graph such that there are two constants $k_1$ and $k_2$ such that for each $t \in [\tau]$, the vertices of $G_t$ can be partitionned into at most $k_1$ vertex-disjoint cliques of size at most $k_2$. For each $t \in [\tau]$, all independent sets of $G_t$ can be enumerated in $\mathcal{O}(k_2^{k_1} \cdot |E(G)|)$.
% \end{lemma}


% \begin{corollary}
% {\sc Reconfiguration Independent Set} can be solved in $\mathcal{O}(k_1^{k_2} \cdot \tau \cdot (|V(G)|+|E(G)|\cdot\sqrt{|V(G)|}))$.
% \end{corollary}

%%% Local Variables:
%%% mode: LaTeX
%%% TeX-master: "main"
%%% End:


\section{Fixed parameter tractability by lifetime and temporal neighbourhood diversity}
\label{sec:ndiversity}
\newcommand{\tndg}[1][\mathcal{G}]{\ensuremath{TND_{#1}}\xspace}
\newcommand{\ndg}[1][\mathcal{G}]{\ensuremath{ND_{#1}}\xspace}
In this section we present a fixed-parameter algorithm, parametrised by lifetime and the temporal neighbourhood diversity of the temporal graph, to solve a class of reconfiguration problems that we call \emph{temporal neighbourhood diversity locally decidable}. 

We need a number of algorithmic tools to build toward this overall result.  First, in \Cref{subsec:definitions}, we give definitions and notation necessary for this section, including defining temporal neighbourhood diversity, as well as the class of temporal neighbourhood diversity locally decidable problems. These build on analogous definitions in static graphs. 

%In \Cref{subsec:reconfiguration}, we introduce a key subroutine of the parameterized algorithm, which we then use in Subsection \Cref{subsec:algo}
Then, in \Cref{subsec:algo} we describe an overall algorithm to solve our restricted class of problems that is in FPT with respect to temporal neighbourhood diversity and lifetime. This algorithm uses a critical subroutine that constitutes the majority of the technical detail, and is presented in \Cref{subsec:reconfiguration}.  The subroutine uses a reduction to the efficiently-solvable circulation problem to give the key result (in \Cref{lemma:compute reconfigurable sequence}) The result allows us to efficiently generate an optimal reconfigurable sequence that is compatible with that candidate sequence given a candidate reconfiguration sequence of a type specific to temporal neighbourhood diversity locally decidable problems. 


% Then, we present the intuition of the algorithm in \Cref{subsec:algo}. Finally, we describe and show the correctness of the main subroutine of the algorithm in \Cref{subsec:reconfiguration}.


\subsection{Definitions}
\label{subsec:definitions}

% \subsubsection{In static graphs}

Neighbourhood diversity is a static graph parameter
% We first state the formal definitions of neighbourhood diversity and neighbourhood diversity locally decidable problem. This parameter has been
introduced by Lampis~\cite{Lampis12}:


\begin{definition}[Neighbourhood Diversity~\cite{Lampis12}]
  The \emph{neighbourhood diversity} of a static graph $G$ is the minimum value $k$ such that the vertices of $G$ can be partitioned into $k$ classes $V_1,\dots,V_k$ such that 
%
  for each pair of vertices $u$ and $v$ in a same class $V_i$, we have $N(u) \setminus \{v\} = N(v) \setminus \{u\}$.  We call $V_1,\dots,V_k$ a \emph{neigbourhood diversity partition} of $\mathcal{G}$.
\end{definition}

Notice that each set $V_i$ of $P$ forms either an independent set or a clique. Moreover, for any pair of sets $V_i$ and $V_j$ either no vertex of $V_i$ is adjacent to any vertex of $V_j$ or every vertex of $V_i$ is adjacent to every vertex of $V_j$. We distinguish two types of classes: $V_i$ is a \emph{clique-class} if $G[V_i]$ is a clique and $V_i$ is an \emph{independent-class} otherwise.
%
% Observe that the definition of temporal neigbourhood diversity is a generalisation of the neighbourhood diversity in static graphs since the temporal neighbourhood diversity partition of a temporal graph with lifetime one and the neighbourhood diversity partition of its footprint are the same. 

%


 \begin{definition}[Neighbourhood diversity graph]
 Let $G$ be a static graph with neighbourhood diversity partition $V_1,\dots,V_k$. The \emph{neighbourhood diversity graph} of $G$, denoted $\ndg[G]$ is the graph obtained by merging each class $V_i$ into a single vertex. Formally, we have $V(\ndg[G]) = \{V_1,\dots,V_k\}$ and $E(\ndg[G]) = \{V_iV_j \mid \forall v_i \in V_i, \forall v_j\in V_j, v_iv_j \in E(G)\}$.
\end{definition}


%
For clarity, we use the term class when referring to a vertex of \ndg[G], in order to distinguish between the vertices of $G$ and the vertices of \ndg[G].
%
% The definition of temporal neighbourhood diversity in temporal graphs is a generalisation of the neighbourhood diversity which is a parameter defined in static graphs. The neighbourhood diversity of a static graph $G$ is equivalent to the temporal neighbourhood diversity of the temporal graph of lifetime one and containing $G$ as unique snapshot. Hence, in the following, given a static graph $G$, we also use the notation $\tndg[G]$ to refer to the temporal neighbourhood diversity graph of the temporal graph containing $G$ as unique snapshot.

%
Let $X$ be a set of vertices and $Y$ be a subset of classes. We say that $X$ and $Y$ are \emph{compatible} if for all classes $V_i$, $V_i$ belongs to $Y$ if and only if $V_i$ intersects $X$, that is, $\forall V_i, (V_i \cap X \neq \emptyset \Leftrightarrow V_i \in Y)$.
% we have $X \cap V_i \neq \emptyset \Leftrightarrow V_i \in Y$
Notice that there is exactly one subset of classes that is compatible with a set of vertices $X$ whereas several subsets of vertices of $G$ can be compatible with a set of classes $Y$.
%

We now introduce the concept of a neighbourhood diversity locally decidable problem -- we do this first in the static setting in order to build into the temporal setting. 
Intuitively, these are problems for which, given a set of classes $Y$, we can determine the minimum and maximum number of vertices to select in each class that are realised by at least one solution compatible with $Y$, if such a solution exists. The formal definition is as follows:

\begin{definition}[Neighbourhood diversity locally decidable]
  \label{def:ndld}
  A static graph problem $\Pi$ is $f(n)$-\emph{neighbourhood diversity locally decidable} if
  for any static graph $G$ with $n$ vertices and every subset of classes $Y$ of $\ndg[G]$, the following two conditions hold:
  \begin{enumerate}[(a)]
    \item there is a computable function $check_\Pi(Y)$ with time complexity $\mathcal{O}(f(n))$ that determines if there is a solution for $\Pi$ in $G$ that is compatible with $Y$,
    \item if there is such a solution, then
      there exist two computable functions \\$low_\pi : \mathcal{P}(V(\ndg[G])) \times V(\ndg[G]) \to \mathbb{N}$ and $up_\pi : \mathcal{P}(V(\ndg[G])) \times  V(\ndg[G]) \to \mathbb{N}$ with time complexity $\mathcal{O}(f(n))$ such that, for all subsets $X \subseteq V(G)$, we have
\begin{gather*}
  \forall V_i \in V(\ndg[G]), low_\pi(Y,V_i) \leq |V_i \cap X| \leq up_\pi(Y,V_i)\\
  \Leftrightarrow \\
  \text{$X$ is solution for $\Pi$ and is compatible with $Y$}. 
\end{gather*}
  \end{enumerate}
\end{definition}

  In other words, $low_\pi$ and $up_\pi$ are necessary and sufficient lower and upper bounds for the number of selected vertices in each class in a solution to our problem in the graph of low neighbourhood diversity.

Notice that it is easy to compute a solution of minimum (respectively maximum) size that is compatible with a subset of classes $Y$ (if such a solution exists), by arbitrarily selecting exactly $low_\pi(V_i)$ (resp. $up_\pi(V_i)$) vertices inside each class.
%
Hence, $\Pi$ is solvable in $\mathcal{O}(2^k \cdot f(n))$, where $k$ is the neighbourhood diversity of the graph. Indeed, it suffices to enumerate every subset of classes and keep the best solution. It follows that if $check_\pi, low_\pi$ and $up_\pi$ are polynomial-time functions, $\Pi$ is in FPT when parameterised by neighbourhood diversity.

%\subsubsection{Example problems that are neighbourhood diversity locally decidable}

We show that {\sc Dominating Set} and {\sc Independent Set} are $f(n)$-neighbourhood diversity locally decidable.

\begin{lemmarep}
  \label{lemma:ds is ndld} {\sc Dominating set} is $f(n)$-neighbourhood diversity locally decidable where $f(n) \in O(n)$.
\end{lemmarep}

\begin{proof}
  Let $G$ be a static graph and let $Y \subseteq V(\ndg[G])$ be a subset of classes. We show that the following two conditions hold.
  \begin{enumerate}[(a)]
    \item there is a dominating set compatible with $Y$ if and only if $Y$ is a dominating set in $\ndg[G]$, and
    \item if $Y$ is a dominating set in $\ndg[G]$, then for each $V_i \in \ndg[G]$ the lower and upper bounds functions $low : V(\ndg[G])  \to \mathbb{N}$ and $up : V(\ndg[G]) \to \mathbb{N}$ are given by
    \begin{enumerate}[(i)]
      \item $low(V_i) = up(V_i) =0$, if $V_i\not\in Y$,
      \item $low(V_i) = 1$ and $up(V_i) = |V_i|$, if $V_i\in Y$ and $V_i$ is a clique-class or has a neighbour in $Y$ and,
      \item $low(V_i) = up(V_i) =|V_i|$, otherwise.
    \end{enumerate}
  \end{enumerate}
  
  First, we show that there is a dominating set in $G$ compatible with $Y$ if and only if $Y$ is a dominating set in $\ndg[G]$.
  Let $Y$ be a dominating set in $\ndg[G]$, we show that the subset of vertices $X = \bigcup\limits_{V_i \in Y} V_i$ is a dominating set of $G$. For any vertex $v \not \in X$ of $G$, the class $V_i \in N(\ndg[G])$ to which $v$ belongs has a neighbour $V_j$ in $\ndg[G]$ that belongs to $Y$ since otherwise $Y$ would not be a dominating set of $\ndg[G]$. Then, any vertex of $V_j$ belongs to $X$ and is a neighbour of $v$ in $G$ and so, $v$ is dominated. Hence, $X$ is a dominating set of $G$.
  %
  
  Now, let $X$ be a dominating set of $G$, we show that the subset of classes $Y$ compatible with $X$ is a dominating set of $\ndg[G]$. For any class $V_i \not \in Y$ and any vertex $v \in V_i$, $v$ is dominated by a vertex $u$ that belongs to a class $V_j \neq V_i$. Then, $V_j \in Y$ and $V_j$ is adjacent to $V_i$ in $\ndg[G]$ and so, $V_i$ is dominated. Hence, any class that does not belong to $Y$ has a neighbour in $Y$ and so, $Y$ is a dominating set in $\ndg[G]$.

  
Further, suppose that $Y$ is a dominating set in $\ndg[G]$, % we show that the depicted bounds are necessary and sufficient.
and let $X$ be a dominating set of $G$ that is compatible with $Y$, we show that for all classes $V_i$ of $G$, we have $low_\pi(V_i) \leq |X \cap V_i| \leq up_\pi(V_i)$.
  Let $V_i$ be a class of $G$.
  %
  If $V_i\not\in Y$, then we have $|X \cap V_i| = 0$ since otherwise $X$ would not be compatible with $Y$. 
  %
  If $V_i\in Y$, since $X$ is compatible with $Y$, $X$ should contain at least one vertex in $V_i$. Moreover, since $X$ cannot contain more that $|V_i|$ vertices in $V_i$, we have $1 \leq |V_i \cap X| \leq |V_i|$.
  %
  If $V_i$ is not a clique-class or has no neighbour in $Y$, then any vertex $v \in V_i$ is isolated in the subgraph induced by $X \cap \bigcup_{V_j \in Y} V_j$. Hence, $X$ contains necessarily every vertex of $V_i$ and so, $|X \cap V_i| = |V_i|$.

  
  Finally, we show that if $Y$ is a dominating set in $\ndg[G]$, then any set of vertices $X$ such that for all classes $V_i$, we have $low_\pi(V_i) \leq |X \cap V_i| \leq up_\pi(V_i)$,  is a dominating set in $G$ compatible with $Y$. First, since for any class $V_i$ we have $|V_i \cap X| = 0$ if $V_i \not\in Y$ and $1 \leq |V_i \cap X|$ otherwise, $X$ is compatible with $Y$. It remains to show that $X$ is also a dominating set.
  Let $v \not\in X$ be a vertex of $G$ that belongs to a class $V_i$. If $V_i \not\in Y$, then since $Y$ is a dominating set of $\ndg[G]$,  there is a class $V_j \in Y$ that is adjacent to $V_i$ in $\ndg[G]$. Thus, $V_j$ contains a vertex $u \in X$ and $u$ is adjacent to $v$ in $G$ and dominates $v$.
  If $V_i \in Y$ and is a clique-class, then there is a vertex $u \in V_i \cap X$ and $u$ is adjacent to $v$ in $G$. If $V_i$ is an independent-class, then since $v$ does not belong to $X$, $v$, there is necessarily another class $V_j \in Y$ that is adjacent to $V_i$ in $\ndg[G]$, since otherwise, every vertex of $V_i$ would belong to $X$. Thus, $V_j$ contains a vertex $u \in X$ and $u$ is adjacent to $v$ in $G$ and dominates $v$. Hence, in any case, $v$ is dominated by another vertex and thus, $X$ is a dominating set of $G$ that is compatible with $Y$.
\end{proof}

\begin{lemmarep}
  \label{lemma:is is ndld} {\sc Independent Set} is $f(n)$-neighbourhood diversity locally decidable where $f(n) \in O(n)$. 
\end{lemmarep}

\begin{proof}
    Let $G$ be a static graph and let $Y \subseteq V(\ndg[G])$ be a subset of classes. We show the following two conditions.
  \begin{enumerate}[(a)]
    \item there is an independent set compatible with $Y$ if and only if $Y$ is an independent set in $\ndg[G]$, and
    \item if $Y$ is a independent set in $\ndg[G]$, then the lower and upper bounds functions $low : V(\ndg[G])  \to \mathbb{N}$ and $up : V(\ndg[G]) \to \mathbb{N}$ are given by
    \begin{enumerate}[(i)]
      \item $low(V_i) = up(V_i) =0$, if $V_i\not\in Y$,
      \item $low(V_i) = up(V_i) = 1$, if $V_i\in Y$ and $V_i$ is a clique-class and,
      \item $low(V_i)=1$ and $up(V_i) =|V_i|$, otherwise.
    \end{enumerate}

  \end{enumerate}
  
  First, we show that there is an independent set in $G$ compatible with a subset of classes $Y$ if and only if $Y$ is an independent set in $\ndg[G]$.
  Let $Y$ be an independent set in $\ndg[G]$, let $X \subseteq V(G)$ be any subset of vertices constructed by arbitrarily choosing one vertex in each $V_i \in Y$.
  We show that $X$ is an independent set of $G$. Let $u$ be a vertex of $X$ that belongs to some class $V_i$. Per construction of $X$, we have $V_i \in Y$. Let $v$ be a neighbour of $u$. If $v \in V_i$, then $v$ does not belong to $X$ since $X$ contains at most one vertex per class in $Y$. If $v$ belongs to another class $V_j$, then $V_i$ and $V_j$ are adjacent in \ndg[G] and since $Y$ is an independent set, $V_j \not \in Y$ and so, by construction of $X$, $v \not\in X$. So, $X$ is an independent set.
  %
  Now, let $X$ be an independent set of $G$, we show that the subset of classes $Y$ compatible with $X$ is an independent set of $\ndg[G]$. Suppose there exist two adjacent classes $V_i$ and $V_j$ that belong to $Y$. Then, $V_i$ contains a vertex $u \in X$ and $V_j$ contains a vertex $v \in V_j$. Since $u$ and $v$ are adjacent, it contradicts $X$ being an independent set. Hence, $Y$ does not contain two adjacent classes and so, $Y$ is an independent set in \ndg[G].
  %
  
Further, suppose that $Y$ is an independent set in $\ndg[G]$, % we show that the depicted bounds are necessary and sufficient.
and let $X$ be an independent set of $G$ that is compatible with $Y$, we show that for all classes $V_i$ of $G$, we have $low_\pi(V_i) \leq |X \cap V_i| \leq up_\pi(V_i)$.
  Let $V_i$ be a class of $G$.
  %
  If $V_i\not\in Y$, then we have $|X \cap V_i| = 0$ since otherwise $X$ would not be compatible with $Y$. 
  %
  If $V_i\in Y$, since $X$ is compatible with $Y$, $X$ should contain at least one vertex in $V_i$. Moreover, since $X$ cannot contain more that $|V_i|$ vertices in $V_i$, we have $1 \leq |V_i \cap X| \leq |V_i|$.
  %
  If $V_i$ is a clique-class, then $X$ cannot contain more two vertices in $V_i$ since they would be adjacent.
  Hence, $|V_i \cap X|=1$ .
  %
  
  Finally, we show that if $Y$ is an independent set in $\ndg[G]$, then any set of vertices $X$ such that for all classes $V_i$, we have $low_\pi(V_i) \leq |X \cap V_i| \leq up_\pi(V_i)$, is an independent set in $G$ compatible with $Y$. First, since for any class $V_i$ we have $|V_i \cap X| = 0$ if $V_i \not\in Y$ and $1 \leq |V_i \cap X|$ otherwise, $X$ is compatible with $Y$. It remains to show that $X$ is also an independent set.
  Suppose there exist two adjacent vertices $u$ and $v$ that both belong to $X$. The two vertices cannot belong to different classes $V_i$ and $V_j$ since otherwise both $V_i$ and $V_j$ would belong to $Y$, contradicting $Y$ being an independent set in \ndg[G]. Thus, $u$ and $v$ belong to the same class $V_i$. Since, $u$ and $v$ are adjacent, $V_i$ is not an independent class. But then, we have $|X \cap V_i| > up(V_i)$ which is a contradiction. Hence, $X$ does not contain two adjacent vertices and therefore $X$ is an independent set. 
\end{proof}


\subsubsection{In temporal graphs}
We now extend the concept of neighbourhood diversity locally decidable problems to a temporal setting. We use temporal neighbourhood diversity introduced by Enright et al.~\cite{abs-2404-19453}.


\begin{definition}[Temporal Neighbourhood Diversity~\cite{abs-2404-19453}]
The \emph{temporal neighbourhood diversity} of a temporal graph $\mathcal{G}=(G,\lambda)$ is the minimum value $k$ such that the vertices of $G$ can be partitioned into $k$ classes $V_1,\dots,V_k$ such that 
%
  for each pair of vertices $u$ and $v$ in a same class $V_i$, we have $N_t(u) \setminus \{v\} = N_t(v) \setminus \{u\}$ at each time-step $t \in [1,\tau]$.  We call $V_1,\dots,V_k$ a \emph{temporal neigbourhood diversity partition} of $\mathcal{G}$.
\end{definition}

\noindent In the same way as for the static parameter, we can use a graph to represent the partition.

\begin{definition}[Temporal neighbourhood diversity graph]
 Let $\mathcal{G}=(G,\lambda)$ be a temporal graph with temporal neighbourhood partition $V_1,\dots,V_k$. The \emph{temporal neighbourhood diversity graph} of $\mathcal{G}$, denoted $\tndg$ is the temporal graph obtained by merging every class $V_i$ into a single vertex. Formally, we have $V(\tndg) = \{V_1,\dots,V_k\}$, $E(\tndg) = \{V_iV_j \mid \forall v_i \in V_i, \forall v_j\in V_j, v_iv_j \in E(G)\}$ and $\lambda(V_iV_j) = \lambda(u_iu_j)$ for any vertices $u_i \in V_i$ and $u_j \in V_j$.
\end{definition}

\begin{figure}
  \centering
  \scalebox{0.4}{
    \begin{tikzpicture}
      \newcounter{vertex}
      \setcounter{vertex}{1}

      \newcounter{tmp}
      \setcounter{tmp}{1}
      
    \foreach[count=\l from 1] \x/\T in {3/{C,I,C},4/{I,I,C},6/{I,I,I},6/{C,I,I},5/{C,C,I},2/{I,I,C}}{
      
      \setcounter{tmp}{\thevertex}

      \foreach[count=\X from 1] \i in \T {
        \setcounter{vertex}{\thetmp}
        \node[draw,circle,minimum width=2.1cm,label=\l*60+30:{\LARGE $V_\l$}] (\l\X) at ($(\X*10,0)+(\l*60:3)$) {};
        \pgfmathsetmacro{\a}{360/\x}% 
        \foreach \v in {1,...,\x}{
          \node[smallvertex,label=\v*\a:{\letter{\thevertex}}] (\v) at ($(\X*10,0)+(\l*60:3)+(\v*\a:0.5)$) {};
          \stepcounter{vertex}
        }
        \ifthenelse{\equal{\i}{C}}{
          \foreach \a in {1,...,\x}{
            \foreach \b in {1,...,\a} {
              \draw (\a)--(\b);
            }
          }
        }{}
      }
    }
    \foreach \a/\b in {21/31,41/51,41/11,41/61,51/61,11/61,11/51,22/12,62/12,22/62,62/52,22/42,42/52,23/33,53/63,53/13,53/43}
    \draw (\a)--(\b);
    \foreach \L in {1,...,3}
    \node at (10*\L,-5) {\LARGE $G_\L$};
  \end{tikzpicture}
  }
  \caption{\label{fig:tmp neighbourhood diversity} Example of temporal neighbourhood diversity graph \tndg of a temporal graph $\mathcal{G}=(G,\lambda)$. An edge between two sets $V_i$ and $V_j$ indicates that each vertex of $V_i$ is adjacent to all vertices of $V_j$.} 
\end{figure}
 
An example of a temporal neighbourhood diversity graph is depicted in \Cref{fig:tmp neighbourhood diversity}.
%
Let $\Pi$ be a reconfiguration problem. We call $\Pi$ \emph{$f(n)$-temporal neighbourhood diversity locally decidable} if the static version of $\Pi$ is $f(n)$-neighbourhood diversity locally decidable. In both of the problems we consider $f(n)=n$, and we simply write ``temporal neighbourhood diversity locally decidable'' for convenience. For a temporal graph $\mathcal{G}=(G,\lambda)$, we denote by $low^t_\pi$ and $up^t_\pi$ the functions giving the lower and upper bounds on the number of selected vertices in a class $V_i$ needed to obtain solution for the static version of $\Pi$ in $G_t$, as defined in \Cref{def:ndld}.


\subsection{FPT algorithm with respect to temporal neighbourhood diversity and lifetime}
\label{subsec:algo}

We now formulate an FPT algorithm using temporal neighbourhood diversity and lifetime to solve a neighbourhood diversity locally decidable reconfiguration problem $\Pi$ in a temporal graph $\mathcal{G}=(G,\lambda)$. 
A \emph{candidate sequence} is a sequence $Y = (Y_1,\dots,Y_\tau)$ such that for each $t \in [1,\tau]$, $Y_t$ is a subset of classes of $\tndg$. 
We say that $Y$ is \emph{valid} if $check_\Pi(Y_t)= \texttt{true}$ for all $t \in [\tau]$. Let $T=(T_1,\dots,T_\tau)$ be a reconfigurable sequence for $\mathcal{G}$. For the sake of simplicity, we say that the two sequences $Y$ and $T$ are compatible if for each $t \in [\tau], T_t$ and $Y_t$ are compatible. 

The principle of the algorithm is to iterate through each candidate sequence $Y$ and, if $Y$ is valid, compute an optimal reconfigurable sequence $T$ compatible with $Y$. Then, we return the optimal solution among all solutions associated to valid candidate sequences.
Notice that there are at most $2^{tnd \cdot\tau}$ candidate sequences to consider. Hence, we can expect the algorithm to be efficient if the temporal graph has small temporal neighbourhood diversity and short lifetime.
% This algorithm is depicted in \Cref{alg:fpt nd}.
If $\Pi$ is $f(n)$-neighbourhood diversity locally decidable, we show in \Cref{subsec:reconfiguration} how to compute an optimal reconfigurable sequence compatible with a candidate sequence, if one exists, by solving an instance of the circulation problem. Combining the observation that there are at most $2^{tnd \cdot\tau}$ candidate sequences to consider and the result from Subsection~\ref{subsec:reconfiguration} gives us the following result, which we restate at the end of that subsection.

% \begin{algorithm}
%   \SetAlgoLined
%   \KwData{A temporal graph $\mathcal{G}$ and a reconfiguration problem $\Pi$}
%   \KwResult{The score of the optimal solution for $\Pi$ in $\mathcal{G}$}
%   \ForEach{candidate sequence $Y = (Y_1,\dots,T_\tau)$}{
%     \If{$check(Y)$}{
%       \tcc{compute an optimal sequence compatible with $Y$}
%       $T \gets compute\_sequence(\mathcal{G},Y)$\;
%       \lIf{$\Pi$ is a minimisation problem}{$score \gets \min(score,|T|)$}
%       \lElse{$score \gets \max(score,|T|)$}
%     }
%   }
%     \Return score\;
%   \caption{}
%   \label{alg:fpt nd}
% \end{algorithm}

\begin{lemma*}
    Let $\mathcal{G}=(G,\lambda)$ be a temporal graph and let $\Pi$ be an $f(n)$-temporal neighbourhood diversity locally decidable reconfiguration problem. 
 $\Pi$ is solvable in $\mathcal{O}(2^{(tnd\cdot \tau)} \cdot (\tau\cdot tnd \cdot f(n) + \tau^{\nicefrac{11}{2}} \cdot tnd^8))$ where $tnd$ is the temporal neighbourhood diversity of $\mathcal{G}=(G,\lambda)$.
\end{lemma*}








%%% Local Variables:
%%% mode: LaTeX
%%% TeX-master: "main"
%%% End:


\subsection{Computing an optimal reconfigurable sequence compatible with a candidate sequence}
\label{subsec:reconfiguration}

In this subsection, we present an efficient method for computing an optimal reconfigurable sequence $T$ compatible with a particular valid sequence $Y$. 
 Note that,  we cannot compute $T$ by exhaustively generating all possible sequences in $\mathcal{G}$ and selecting the best one compatible with $Y$. This approach would have time complexity of $\mathcal{O}(2^{\tau\cdot|V(G)|})$ for each candidate sequence making it impractical.
%
 % Given a valid candidate sequence $Y$, we describe how to compute an optimal reconfigurable sequence $T = (T_1,\dots,T_\tau)$ for $\Pi$ compatible with $Y$ in $\mathcal{O}(\tau\cdot tnd \cdot f(n) + \tau^{\nicefrac{11}{2}} \cdot tnd^8)$ time. % It corresponds to the function compute\_sequence of \Cref{alg:fpt nd}.

%

As a tool to build our algorithm we construct and solve an instance of the circulation problem: this is a variation of the network flow problem for which there are upper and lower bounds on the capacities of the arcs. Formally, a \emph{circulation graph} $\overrightarrow{F}=(F,c,l,u)$ consists of a directed graph $F$, along with a cost function $c : A(F) \to \mathbb{N}$ on the arcs and two functions on the arcs $l : A(F) \to \mathbb{N}$ and $u : A(F) \to \mathbb{N}$, representing the lower and upper bounds of the flow capacities. The digraph $F$ also has two designated vertices $sou$ and $tar$ representing the source and the target vertices of the flow, respectively.

In the circulation problem, we want to find a \emph{circulation flow} between $sou$ and $tar$, which is a function $g : A(F) \to \mathbb{N}$ that respects two constraints:
\begin{enumerate}[(a)]
  \item \emph{Flow preservation}: the flow entering in any vertex $v \in V(F) \setminus \{sou,tar\}$ is equal to the flow leaving it, that is, $\sum\limits_{u \in N^+(v)} g(u,v) = \sum\limits_{u \in N^-(v)} g(v,u)$.
  \item \emph{Capacity:} for all arcs $(u,v) \in A(F)$, the amount of flow over $(u,v)$ must respect the capacity of $(u,v)$, that is, $l(u,v) \leq g(u,v) \leq u(u,v)$.
\end{enumerate}
%
Finally, the circulation problem is defined as follows.
\prob{Circulation problem}{
A circulation graph $\overrightarrow{F}$, a source vertex $sou$ and a target vertex $tar$
}{
  Find a circulation flow $g$ that minimises  $\sum\limits_{a \in A(F)}g(a)\cdot c(a)$.
}
The circulation problem is known to be solvable in $\mathcal{O}(|A(F)|^{\nicefrac{5}{2}}\cdot |V(F)|^3)$~\cite{Tardos85}. Notice that the circulation problem can easily be transformed into a maximisation problem by modifying all positive costs into negative costs.

We now show that the problem of finding an optimal reconfigurable sequence compatible with $Y$ is reducible to the circulation problem.

% \begin{construction}
%   \TD{constructive version\\}
%   \label{const:flow}
%   Let $\mathcal{G}=(G,\lambda)$ be a temporal graph, let $Y=(Y_1,\dots,Y_\tau)$ be a valid candidate sequence.
%   We construct the following circulation graph $\overrightarrow{F}=(F,c,l,u)$.
%   \begin{itemize}
%     \item For each $Y_t$ and each set $V_i \in Y_t$, introduce two vertices $x^t_i, y^t_i$ along with the arc $(x^t_i,y^t_i)$ and set $c(x^t_i,y^t_i)=0$, $l(x^t_i,y^t_i)=low_\pi(Y_t,V_i)$ and $u(x^t_i,y^t_i)=up_\pi(Y_t,V_i)$.
%     \item Introduce a source vertex $sou$ and for each set $V_i \in Y_1$, introduce the arc $(sou,x^1_i)$ and set $c(sou,x^1_i)=1, l(sou,x^1_i)=0$ and $u(sou,x^1_i)=+\infty$.
%     \item Introduce a target vertex $tar$ and for each set $V_i \in Y_\tau$, introduce the arc $(x^\tau_i,tar)$ and set $c(x^\tau_i,tar)=l(x^\tau_i,tar)=0$ and $u(x^\tau_i,tar)=+\infty$.
%     \item For each $t\in[1,\tau-1]$, and each pair $V_i \in Y_t, V_j \in Y_{t+1}$ such that either $V_i=V_j$ or $V_i$ and $V_j$ are adjacent in $\tndg$ at time step $t$, introduce an arc $(y^t_i,x^{t+1}_j)$ and set $c(y^t_i,x^{t+1}_j)=0, l(y^t_i,x^{t+1}_j)=0$ and $u(y^t_i,x^{t+1}_j)=+\infty$.  
%   \end{itemize}
% \end{construction}


\begin{construction}
  \label{const:flow}
  Let $\mathcal{G}=(G,\lambda)$ be a temporal graph, let $Y=(Y_1,\dots,Y_\tau)$ be a valid candidate sequence.  
  The reconfiguring circulation graph $\overrightarrow{RF}(\mathcal{G},Y)=(F,c,l,u)$ is the circulation graph with vertex set $V(F) = \{sou,tar\} \cup \{x_i^t \mid V_i \in V(\tndg), t \in [1,\tau]\}$ and arc set $E(F) = S \cup \{C_t \mid t \in [1,\tau]\} \cup \{R_t \mid t \in [1,\tau-1]\}  \cup T$ where the arc sets $S,C_t ,R_t$ and $T$ are defined as follows: 
  \begin{itemize}
    \item $S= \{ (sou,x_i^1) \mid V_i \in Y_1\}$ with $c(a)=1$ and $l(a)=0$ and $u(a)=+\infty$ for each a $\in S$,
    \item $C_t= \{(x^t_i,y^t_i) \mid V_i \in Y_t\}$ with $c(x^t_i,y^t_i) = 0$, $l(x^t_i,y^t_i) = low^t_\pi(V_i)$ and  $u(x^t_i,y^t_i) = up^t_\pi(V_i)$ for each $(x^t_i,y^t_i)\in C_t$,
    \item $R_t= \{(y^t_i,v^{t+1}_i) \mid V_i \in Y_t \cap Y_{t+1}\} \cup \{(y^t_i,v^{t+1}_j) \mid V_i \in Y_t,V_j \in Y_{t+1}, V_i$ is adjacent to $V_j$ in $\tndg$ at time $t\}$ with $c(a)=l(a)= 0$ and $u(a)=+\infty$ for each $a\in R_t$,
    \item $T = \{(y_i^\tau,tar) \mid V_i \in V(\tndg)\}$ with $c(a)=l(a)=0$ and $u(a)=+\infty$ for each $a\in T$.
  \end{itemize}
  \end{construction}


An example construction can be found in Figure~\ref{fig:flow}. The idea of the reduction is to represent each token by a unit value of the flow.
  At each time-step $t$, we can decompose the reconfiguring circulation graph into two parts: the checking part $C_t$ and the reconfiguring part $R_t$ (which only exists if $t \neq \tau$). In the checking part, each class $V_i$ is represented by an arc $(x^t_i,y^t_i) \in C_t$ and the value of the flow passing by this arc corresponds to the number of tokens inside $V_i$ at time-step $t$. Since the lower and upper bounds of the arc are the same as those described in \Cref{def:ndld}, we ensure that at each step, the set of selected vertices forms a solution. In the 
reconfiguring part, an arc $(y^t_i,x^t_j) \in R_t$ indicates that the tokens contained in the class $V_i$ can move to the class $V_j$ during time-step $t$. An arc $(y^t_i,x^t_i) \in R_t$ indicates that the tokens in the class $V_i$ can stay in the same vertex during time-step $t$.

% \begin{definition}
%   \label{def:flow}
%   Let $\mathcal{G}=(G,\lambda)$ be a temporal graph and let $\mathcal{Y}=(Y_1,\dots,Y_\tau)$ be a candidate sequence in $\mathcal{G}$ for some reconfiguration problem $\Pi$. The \emph{reconfiguration flow graph} of $\mathcal{G}$ and $\mathcal{Y}$, denoted $\overrightarrow{F}_\Pi(\mathcal{G},\mathcal{Y})=(F,c,l,u)$ is a circulation graph with vertex set $V(F)= \{sou,tar\} \cup \{x^t_i,y^t_i \mid i \in [1,\tau], V^t_i \in Y_t\} \cup \{w^t_i,z^t_i \mid i \in [1,\tau], \}$ 
%   We construct the following circulation graph $\overrightarrow{F}=(F,c,l,u)$. In the following, for each arc $e$, if $c(e),l(e)$ or $u(e)$ are not specified, we set $c(e)=0$, $l(e)=0$ and $u(e) +\infty$.
%   %
%   \begin{itemize}
%     \item For each time-step $t \in [1,\tau]$ each class $V^t_i \in Y^t$, introduce two vertices $x^t_i$ and $y^t_i$ along with the arc $(x^t_i,y^t_i)$ and set $l(x^t_i,y^t_i) = low_{\Pi}(V^t_i)$, $u(x^t_i,y^t_i) = up_{\Pi}(V^t_i)$. 
%     \item For each time-step $t \in [1,\tau-1]$ and each class $V^t_i$ of $G_t$ such that there exists $V^{t+1}_j \in Y^{t+1}$ with $V^t_i \cap V^{t+1}_j \neq \emptyset$, introduce two vertices $w^t_i$ and $z^t_i$ along with the arc $(w^t_i,z^t_i)$ and set $up(w^t_i,z^t_i)=|V^t_i|$. Then introduce the following arcs.
%     \begin{itemize}
%       \item For each $V^t_j \in Y^t$ such that $V^t_j=V^t_i$ or $V^t_j$ is adjacent to $V^t_i$ in $NDG(G_t)$, introduce the arc $(y^t_i,w^t_i)$.
%       \item For each $V^{t+1}_j \in Y^{t+1}$ such that $V^t_i \cap V^{t+1}_j \neq \emptyset$, introduce the arc $(z^t_i,x^{t+1}_j)$ and set $u(z^t_i,x^{t+1}_j) = |V^t_i \cap V^{t+1}_j|$.
%     \end{itemize}
%     \item   Finally, introduce one source vertex $sou$ and one target vertex $tar$. For each class $V^1_i \in Y^t$, introduce the arc $(sou,x^1_i)$ and set $c(sou,x^1_i) = 1$. For each class $V^\tau_i \in Y^\tau$, introduce the arc $(z^\tau_i,tar)$. 
%   \end{itemize}
% \end{definition}



% \begin{construction}
%   \label{const:flow}
%   Let $\mathcal{G}=(G,\lambda)$ be a temporal graph, let $\mathcal{H}=(H,\lambda)$ be a temporal neigbourhood diversity graph of $\mathcal{G}$ and let $S=(S_1,\dots,S_\tau)$ be a sliding dominating set of $\mathcal{H}$.
%   We construct the following circulation graph $\overrightarrow{F}=(F,c,l,u)$.
%   \begin{itemize}
%     \item For each $S_t$ and each set $V_i \in S_t$, introduce two vertices $x^t_i, y^t_i$ along with the arc $(x^t_i,y^t_i)$ and set $c(x^t_i,y^t_i)=0$, $l(x^t_i,y^t_i)=min\_tok(V_i,H_t,S_t)$ and $u(x^t_i,y^t_i)=|V_i|$.
%     \item Introduce a source vertex $sou$ and for each set $V_i \in S_1$, introduce the arc $(s,x^1_i)$ and set $c(sou,x^1_i)=1, l(sou,x^1_i)=0$ and $u(sou,x^1_i)=+\infty$.
%     \item Introduce a target vertex $tar$ and for each set $V_i \in S_\tau$, introduce the arc $(x^\tau_i,tar)$ and set $c(x^\tau_i,tar)=l(x^\tau_i,tar)=0$ and $u(x^\tau_i,tar)=+\infty$.
%     \item For each $t<\tau$, and each pair $V_i \in S_t, V_j \in S_{t+1}$ such that either $V_i=V_j$ or $V_i$ and $V_j$ are adjacent in $H_t$, introduce an arc $(y^t_i,x^{t+1}_j)$ and set $c(y^t_i,x^{t+1}_j)=0, l(y^t_i,x^{t+1}_j)=0$ and $u(y^t_i,x^{t+1}_j)=+\infty$.  
%   \end{itemize}
% \end{construction}





% \begin{figure}
%   \centering
%   \scalebox{0.55}{
%   \begin{tikzpicture}
%     \foreach[count=\l from 1] \x/\T in {3/{C,I,C},4/{I,I,C},6/{I,I,I},6/{C,I,I},5/{C,C,I},3/{I,I,C}}{
%       \foreach[count=\X from 1] \i in \T {
%         \node[draw,circle,minimum width=2cm,label=\l*60+30:{\LARGE $V_\l$}] (\l\X) at ($(\X*10,0)+(\l*60:3)$) {};
%         \pgfmathsetmacro{\a}{360/\x}% 
%         \foreach \v in {1,...,\x}{
%           \node[smallvertex] (\v) at ($(\X*10,0)+(\l*60:3)+(\v*\a:0.5)$) {};
%         }
%         \ifthenelse{\equal{\i}{C}}{
%           \foreach \a in {1,...,\x}{
%             \foreach \b in {1,...,\a} {
%               \draw (\a)--(\b);
%             }
%           }
%         }{}
%       }
%     }
%     \foreach \a/\b in {21/31,41/51,41/11,41/61,51/61,11/61,11/51,22/12,62/12,22/62,62/52,22/42,42/52,23/33,53/63,53/13,53/43}
%     \draw (\a)--(\b);
%     \foreach \L in {1,...,3}
%     \node at (10*\L,-5) {\LARGE $G_\L$};
%   \end{tikzpicture}
%   }
%   \caption{\label{fig:neighbourhood diversity} Example of temporal neigbourhood diversity graph. An edge between two sets $V_i$ and $V_j$ indicates that each vertex of $V_i$ is adjacent to all vertices of $V_j$. } 
% \end{figure}
 

\begin{figure}[ht]
  \centering
  \scalebox{0.75}{
    \begin{tikzpicture}[
tap/.style args = {#1}{decoration={raise=5,
                                      text along path,
                                      text align={align=center},
                                      text={#1}
                                      },
              postaction={decorate},
              },
                   ]
   
     \foreach \x/\y/\l/\c/\u/\t in {0/0/1/1/3/1,0/-2/2/1/4/1,0/-4/3/1/6/1,
       4/1/6/1/3/2,4/-2/2/1/4/2,4/-4/3/6/6/2,
       8/-2/2/1/4/3,8/-4/3/1/6/3,8/1/6/1/3/3,8/-1/5/1/5/3}
     {
       \draw[->,>=stealth] (\x,\y) node[smallvertex,label=90:{$x_\l^\t$}] (x\l\t) {} ->
       node[midway,above] {0/\c/\u}
       ++(1.9,0) node[smallvertex,label=90:{$y_\l^\t$}] (y\l\t) at ++(0.1,0) {}
       ;
     }
     \foreach \a/\b in {21/22,22/23,31/32,32/33,21/32,31/22,11/62,22/63,22/53,62/53}
     \draw[->] (y\a)->(x\b);

     \node[smallvertex,label=90:{$sou$}] (s) at (-3,-2) {};
     \foreach \a in {11,21,31}
     \draw[->,tap={1/0/${+\infty}${}}] (s) -- node[midway,above] {} (x\a);

     \node[smallvertex,label=90:{$tar$}] (t) at (13,-2) {};
     \foreach \a in {63,53,23,33}
     \draw[->] (y\a)--(t);
   \end{tikzpicture}
   }
   \caption{\label{fig:flow} Example of flow graph produced by \Cref{const:flow}. The input temporal graph is the one depicted in \Cref{fig:tmp neighbourhood diversity}. We want to compute a minimum dominating set compatible with the candidate sequence $Y = (Y_1 = \{V_1,V_2,V_3\}, Y_2= \{V_2,V_3,V_6\},Y_3 = \{V_2,V_3,V_5,V_6\})$.}
 \end{figure}


 
\begin{lemmarep}
  \label{lemma:flow reconfiguration} Let $\mathcal{G}$ be a temporal graph with temporal neighbourhood graph \tndg and let $Y=(Y_1,\dots,Y_\tau)$ be a valid candidate sequence for some $f(n)$-neighbourhood diversity locally decidable reconfiguration problem $\Pi$. Let $\overrightarrow{RF}(\mathcal{G},Y)=(F,c,l,u)$ be the reconfiguring circulation graph as described in \Cref{const:flow}.
  There is a reconfigurable sequence $T$ compatible with $Y$ of size $\ell$ if and only if there is a circulation flow $g$ in $\overrightarrow{RF}(\mathcal{G},Y)$ with cost $\ell$.
  % $\mathcal{Y}$ is valid if there is a feasible flow $f$ in $\overrightarrow{F}$ and in that case, the minimum size of a reconfiguration sequence associated to $\mathcal{Y}$ is given by the minimum cost value for 

  % There is a sliding dominating set $(S,M,tok)$ in $\mathcal{G}$ with $\ell$ tokens if and only if there is a circulation flow $f$ with cost $\ell$ in $\overrightarrow{F}$.
\end{lemmarep}


\begin{proof}
  For any vertex $v$ of $\overrightarrow{RF}(\mathcal{G},Y)$, we denote $g^+(v)=\sum\limits_{u \in N^+(v)} g(u,v)$ and $g^-(v)= \sum\limits_{u \in N^-(v)} g(v,u)$, the flow entering in $v$ and leaving $v$, respectively.

  Let $T = (T_1,\dots,T_\tau)$ be a reconfigurable sequence of size $\ell$. For each time-step $t \in [1,\tau-1]$, let $b_t$ be a reconfiguration bijection between $T_t$ and $T_{t+1}$.  We construct a circulation flow $g$ between $sou$ and $tar$ as follows.
  %
  Let \tndg$=(\{V_1,\ldots,V_k\}, \{V_iV_j | \forall v_i\in V_i, \forall v_j\in V_j, v_iv_j\in E(G)\})$. First, we set $g(sou,x^1_i):= |T_1 \cap V_i|$.
  %
  Then, for each time-step $t \in [1,\tau]$, we set $g(x^t_i,y^t_i) := |V_i \cap T_t|$ (i.e.\ the number of tokens contained in the class $V_i$ at time-step $t$).
  %
  For each time-step $t \in [1,\tau-1]$, we set
      % 
    $g(y^t_i,x^{t+1}_j) := | \{ (u,v) \in b_t \mid u \in T_t \cap V_i, v \in T_{t+1} \cap V_j\} |$ (i.e.\ the number of tokens moving from the class $V_i$ to another class $V_j$ during time-step $t$ if $i \neq j$ or the number of tokens moving inside the class $V_i$ or staying in the same vertex in $V_i$ during time-step $t$ if $i = j$).
    %
    % $f(y^t_i,w^t_i) := | \{ uv \in M_t \mid u \in T_t \cap V_i, v \in T_{t+1} \cap V_i\} \cup T_t \cap T_{t+1} \cap V_i|$ (\textit{i.e.} the number of tokens moving inside the class $V_i$ or staying in the same vertex in $V_i$ during time-step $t$)
    %
    Finally, we set $g(y^\tau_i,tar):= |V_i \cap T_\tau|$. 
      % 
    We now show that $g$ is a circulation flow of cost $\ell$. First, notice that the only non-zero cost arcs are those leaving $sou$. Hence, the cost of $g$ is $\sum\limits_{V_i\in Y_1} g(sou,x^1_i) = \sum\limits_{V_i\in Y_1} |V_i \cap T_1| = \ell$.
    Further, we show that the capacity constraints are respected. For each arc $(x^t_i,y^t_i)$, since $\Pi$ is $f(n)$-neighbourhood diversity locally decidable, we have $low^t_\pi(Y_t,V_i) = l(x^t_i,y^t_i) \leq g(x^t_i,y^t_i) \leq up^t_\pi(Y_t,V_i) = u(x^t_i,y^t_i)$. For any other arc $a$, we have $l(a)=0$ and $u(a)=+\infty$ and so, the capacity constraint is necessarily respected. Hence, the capacity constraints are respected for all arcs.
    It remains to show that $g$ respects the flow conservation constraint. 
    %
    For any vertex $x^t_i$, we have $g^-(x^t_i) = g(x^t_i,y^t_i) = |T_t \cap V_i|$. If $t=1$, then $g^+(x^1_i)=g(sou,x^1_i)= |T_1 \cap V^1_i|= g^+(x^t_i)$. Otherwise, by construction of $g$, $g^+(x^t_i)$ is equal to the sum of the number of tokens moving from another class $V_j$ to the class $V_i$ plus the number tokens staying inside the class $V_i$ during the time-step $t$, that is, the number of tokens inside $V_i$ at time-step $t$. So, $g^-(x^t_i) = |T_t \cap V_i| = g^+(x^t_i)$. 
    % 
    For any vertex $y^t_i$, we have $g^+(y^t_i) = g(x^t_i,y^t_i) = |T_t \cap V^t_i|$. If $t = \tau$, then $g^-(y^t_i) = g(y^t_i,tar) = |T_\tau \cap V^\tau_i| = g^+(y^t_i)$. Otherwise, by construction of $g$, $g^-(y^t_i)$ is equal to the number of tokens moving from the class $V_i$ to another class $V_j$ plus the number of tokens staying inside the class $V_i$ during the time-step $t$, that is, the number of tokens inside $V_i$ at time-step $t$. So, $g^+(y^t_i) = |T_t \cap V_i| = g^-(y^t_i)$.
    %

    We now show the reverse.
    Let $g$ be a circulation flow with cost $\ell$ in $\overrightarrow{RF}(\mathcal{G},Y)$.
    We construct a reconfigurable sequence $T = (T_1,\dots,T_\tau)$ compatible with $Y$ of size $\ell$ as follows. First, for each class $V_i \in Y_1$, we construct $T_1$, by selecting $g(x^1_i,y^1_i)$ vertices in the class $V_i$. Since $g$ respects the flow conservation constraint, we have $|T_1 \cap V_i| = g(x^1_i,y^1_i)$ for each class $V_i$. Moreover, since $low^1_\pi(Y_1,V_i) = l(x^1_i,y^1_i) \leq g(x^1_i,y^1_i) < u(x^1_i,y^1_i) = up^1_\pi(Y_1,V_i)$ and since $\Pi$ is $f(n)$-neighbourhood diversity locally decidable, $T_1$ is a solution for the static version of $\Pi$ in $G_1$.
    %
    Now suppose that at time-step $t<\tau-1$, we also have $|T_t \cap V_i| = g(x^t_i,y^t_i)$ for each class $V_i$. We construct $T_{t+1}$ by moving $g(y^t_i,x^{t+1}_j)$ tokens from the class $V_i$ to the class $V_j$ for each pair $i,j$ such that $i\neq j$ and $(y^t_i,x^{t+1}_j)$ is an arc in $\overrightarrow{RF}(\mathcal{G},Y)$. Observe that inside $V_i$, exactly $g(y^t_i,x^{t+1}_i)$ tokens remain on the same vertex. The number of tokens leaving $V_i$ or staying inside $V_i$ is equal to $g^-(y^t_i)$ and since $g$ respects the flow conservation constraint, we have $g^+(y^t_i)= |T_t \cap V_i| = g^-(y^t_i)$. Thus, we have exactly the correct number of tokens in $V_i$ to make the moves. Moreover, we have  $f^+(x^{t+1}_j)= f(x^{t+1}_j,y^{t+1}_j)$ and so, we have $|T_{t+1} \cap V_j| = f(x^{t+1}_j,y^{t+1}_j)$. Since $low^{t+1}_\pi(Y_{t+1},V_j) = l(x^{t+1}_j,y^{t+1}_j) \leq f(x^{t+1}_j,y^{t+1}_j) < u(x^{t+1}_j,y^{t+1}_j) = up^{t+1}_\pi(Y_{t+1},V_i)$ and since $\Pi$ is $f(n)$-neighbourhood diversity locally decidable, $T_{t+1}$ is a solution for the static version of $\Pi$ in $G_{t+1}$. Hence, inductively, we have constructed a reconfigurable sequence $T = (T_1,\dots,T_\tau)$ compatible with $Y$.
\end{proof}


We can now conclude that it is possible to compute an optimal reconfigurable sequence compatible with a candidate sequence in polynomial time.
 
\begin{lemmarep}
  \label{lemma:compute reconfigurable sequence} Let $\Pi$ be a $f(n)$-neighbourhood diversity locally decidable problem, let $\mathcal{G}=(G,\lambda)$ be a temporal graph with lifetime $\tau$ and temporal neighbourhood diversity $tnd$ and let $Y$ be a valid candidate sequence. It is possible to compute an optimal reconfigurable sequence $T$ compatible with $Y$ in
$\mathcal{O}(\tau\cdot tnd \cdot f(n) + \tau^{\nicefrac{11}{2}} \cdot tnd^8)$.
\end{lemmarep}

\begin{proof}
  We first construct the reconfiguring circulation graph $\overrightarrow{RF}(\mathcal{G},Y)=(F,c,l,u)$, as depicted in \Cref{const:flow}. We have $|V(\overrightarrow{RF})|=\mathcal{O}(\tau\cdot \sum\limits_{Y_i \in Y}|Y_i|) = \mathcal{O}(\tau \cdot tnd)$ and $|A(\overrightarrow{RF})|= \mathcal{O}(\tau \cdot \sum\limits_{Y_i \in Y}|Y_i| + \sum\limits_{Y_i \in Y \setminus Y_\tau}(|Y_i|\cdot |Y_{i+1}|)) = \mathcal{O}(\tau \cdot tnd^2)$. Moreover, we call the two functions $low^t_\pi$ and $up^t_\pi$ exactly once per arc $(x^t_i,y^t_i)$ \textit{i.e.}, $\mathcal{O}(\tau\cdot tnd)$ calls in total. Hence,  the construction of $\overrightarrow{RF}$ can be done in $\mathcal{O}(\tau\cdot tnd \cdot f(n) + |A(\overrightarrow{RF})| + |V(\overrightarrow{RF})|)$
  Then, we compute an optimal circulation flow $f$ in $\overrightarrow{RF}$ which can be done in $\mathcal{O}(|A(\overrightarrow{RF})|^{\nicefrac{5}{2}} \cdot |V(\overrightarrow{RF})|^3) = \mathcal{O}((\tau\cdot tnd^2)^{\nicefrac{5}{2}}\cdot (\tau\cdot tnd)^3) = \mathcal{O}(\tau^{\nicefrac{11}{2}}\cdot tnd^8)$~\cite{Tardos85}. We can then construct an optimal reconfigurable sequence $T$ compatible with $Y$, as described in the proof of \Cref{lemma:flow reconfiguration}. We obtain an overall complexity of $\mathcal{O}(\tau\cdot tnd \cdot f(n) + \tau^{\nicefrac{11}{2}} \cdot tnd^8)$.
\end{proof}

We now have the necessary tools for the overall algorithmic result:

\begin{lemmarep}   \label{lemma:fpt nd}
  Let $\mathcal{G}=(G,\lambda)$ be a temporal graph and let $\Pi$ be an $f(n)$-temporal neighbourhood diversity locally decidable reconfiguration problem. 
 $\Pi$ is solvable in $\mathcal{O}(2^{(tnd\cdot \tau)} \cdot (\tau\cdot tnd \cdot f(n) + \tau^{\nicefrac{11}{2}} \cdot tnd^8))$ where $tnd$ is the temporal neighbourhood diversity of $\mathcal{G}=(G,\lambda)$.
\end{lemmarep}

\begin{proof}
  
  For each candidate sequence $Y = (Y_1,\dots,Y_\tau)$, we call the function $check_\Pi$ and we compute an optimal reconfigurable sequence compatible with it. Thus, each iteration can be done in $\mathcal{O}(\tau\cdot tnd \cdot f(n) + \tau^{\nicefrac{11}{2}} \cdot tnd^8)$. For each time-step $t$, there exists $2^{|V(\tndg|)} = 2^{(tnd)}$ possibilities to choose $Y_t$. Hence, there are $2^{tnd\cdot \tau}$ candidate sequences. We obtain an overall complexity of $\mathcal{O}(2^{tnd\cdot\tau} \cdot (f(n)))$.
\end{proof}

\begin{corollaryrep}
  Let $\mathcal{G}=(G,\lambda)$ be a temporal graph with temporal neighbourhood diversity $tnd$ and lifetime $\tau$.
  {\sc Temporal Dominating Set Reconfiguration} and {\sc Temporal Independent Set Reconfiguration} are solvable in $\mathcal{O}(2^{(tnd\cdot\tau)}\cdot(\tau^{\nicefrac{11}{2}} \cdot tnd^8))$ in $\mathcal{G}$.
  
\end{corollaryrep}

\begin{proof}
By \Cref{lemma:ds is ndld,lemma:is is ndld}, the complexity of the functions $check$ for $low_\pi$ and $up_\pi$ {\sc Dominating Set} and {\sc Independent Set} is $\mathcal{O}(|V(\tndg)|+|E(\tndg)|) = \mathcal{O}(tnd^2)$. Hence, by \Cref{lemma:compute reconfigurable sequence}, we can compute and check an optimal sequence in time $\mathcal{O}(\tau^{\nicefrac{11}{2}} \cdot tnd^8)$. Hence, by \Cref{lemma:fpt nd}, {\sc Temporal Dominating Set Reconfiguration} and {\sc Temporal Independent Set} are solvable in $\mathcal{O}(2^{(tnd\cdot\tau)}\cdot(\tau^{\nicefrac{11}{2}} \cdot tnd^8))$.
\end{proof}

\section{An FPT algorthm with respect to the lifetime of the temporal graph and the treewidth of the underlying graph}
\label{sec:tw algo}
In this section we show that if a static problem is FPT in the treewidth, then its reconfiguration version is FPT in the treewidth of the footprint (the union of all time-steps) and the lifetime of the temporal graph.
We first introduce definitions specific to this section. Then, using Courcelle's theorem, we show that if the static version is expressable in MSO then the reconfiguration version can be parameterized by the lifetime and the treewidth.

%Finally, we show how to adapt a generic algorithm on a tree decomposition for the static version to the reconfiguration version.


% \subsection{Definitions}

\subsection{Treewidth and tree decompositions}

Tree decompositions are widely used to solve a large class of combinatorial problems efficiently by dynamic programming when the graph has low treewidth. 

\begin{definition}[Treewidth, tree decomposition \cite{BodlaenderK96,Kloks94}]\label{def:TD}
  Given a static graph $G$, a \emph{tree decomposition} of $G$ is a pair $(\mathcal{T},\mathcal{X})$
  where $\mathcal{T}$ is a tree and 
  $\mathcal{X}=\{B_i\mid i\in V(\mathcal{T})\}$ is a multiset of subsets of $V(G)$ (called ``bags'') such that
  \begin{enumerate}[(a)]
    \item for each $uv \in E(G)$, there is some $i$ with $uv \subseteq B_i$ and
    \item for each $v \in V(G)$, the bags $B_i$ containing $v$ form a connected subtree of $\mathcal{T}$.
    \end{enumerate}
    \vspace{0.1cm}
  The width of $(\mathcal{T},\mathcal{X})$ is $\max_{B_i \in \mathcal{X}} |B_i|-1$.
%   Further, $(\mathcal{T},\mathcal{X})$ is called \emph{nice} if
%   \begin{enumerate}[(i)]
%     \item $\mathcal{T}$ is rooted at bag $B_r$, with $B_r = \emptyset$ and each bag has at most two children.
%     \item Each bag $B_i$ of $\mathcal{T}$ has one of the four types:
%       \begin{itemize}
%       \item \textbf{Leaf bag:} $i$ has no children and $B_i = \emptyset$.
%         \item \textbf{Join bag:} $i$ has two children $j$ and $k$ and $B_i = B_j = B_k$.
%         \item \textbf{Introduce vertex $u$ bag:} $i$ has only one child $j$ and $B_j = B_i\setminus\{u\}$.
% %      \item \textbf{Introduce edge $uv$ bag:} $i$ has only one child $j$ and $B_j = B_i$.
%       \item \textbf{Forget $u$ bag:} $i$ has only one child $j$ and $B_j = B_i\cup\{u\}$.
%     \end{itemize}
%   \end{enumerate}
%   \vspace{0.1cm}
% 
  % For any bag $B_i$ of $\mathcal{T}$, we let $\mathcal{T}_i$ denote the subgraph of $G$ containing all vertices and edges that have been introduced ``below'' $B_i$
  % (that is, in a bag of the subtree of $T$ that is rooted at $i$).
\end{definition}

\noindent It is NP-complete to determine whether a graph has treewidth at most $k$~\cite{Arnborg87}. However, there exists a linear-time algorithm that, for any constant $k$, computes a tree-decomposition with treewidth at most $k$, if there is one~\cite{Bodlaender96}.
% \noindent
% Note that for each vertex $u$ of $G$, $(\mathcal{T},\mathcal{X})$ contains exactly one forget $u$ bag.  Bodlaender and Kloks shown that a nice tree decomposition contains at most $\mathcal{O}(|E|\cdot |V|)$ bags~\cite{BodlaenderK96}.
%
% Let $\mathcal{G}=(G,\lambda)$ be a temporal graph. A \emph{temporal tree decomposition} for $\mathcal{G}$ is a tree decomposition $(\mathcal{G},\mathcal{X})$ of the footprint $G$ in which we have replaced every introduce edge $uv$ bag by $|\lambda(uv)|$ introduce lambda edge $(uv,t)$ bag. Formally, let $B_i$ be an introduce edge $uv$ bag with child $B_j$ and parent $B_\ell$ and let $\lambda(uv)= \{t_1,\dots,t_k\}$. We replace remove $B_i$ from the tree decomposition and we introduce $k$ introduce lambda edge $(uv,t)$ bags $B_{t_1},\dots,B_{t_k}$ such that $B_j=B_{t_1}=\dots=B_{t_k}$, $B_{t_1}$ has child $B_j$, $B_\ell$ has child $B_{t_k}$ and for each $t_{x} \in \lambda(uv) \setminus \{t_1\}$, $B_{t_x}$ has child $B_{t_{x-1}}$.
%Notice that a nice tree decomposition contains at most $\mathcal{O}(\tau\cdot|E(G)|\cdot|V(G)|)$.


\paragraph{Monadic second order logic.} % A \emph{relational vocabulary} $\mathcal{R}$ is a set of relation symbols. Each relation symbol $R$  has an arity, denoted $arity(R) > 1$. A structure $\mathcal{A}$ of vocabulary $\mathcal{R}$ , or $\mathcal{R}$-structure, consists of a set $A$, called the \emph{universe}, and an interpretation $R^\mathcal{A} \subseteq A^{arity(R)}$ of each relation symbol $R \in \mathcal{R}$. We write $\bar{a} \in R^\mathcal{A}$ or $R^\mathcal{A}(a)$ to denote that the tuple
%  $a \in A^{arity(R)}$ belongs to the relation $R^\mathcal{A}$.
% We briefly recall the syntax and semantics of first-order logic. We fix a countably infinite set of (individual) variables, for
% which we use small letters. Atomic formulas of vocabulary $\mathcal{R}$ are of the form: $x = y$ or $R(x_1,\dots,x_r)$,  where $R \in \mathcal{R}$ is r-ary and $x_1,\dots,x_r,x,y$ are variables. \emph{First-order formulas} of vocabulary $\mathcal{R}$ are built from the atomic formulas using the boolean connectives $\neg, \vee,\wedge$ and existential and universal quantifiers $\exists$, $\forall$. The difference between first-order and second-order

Monadic second-order logic (MSO logic) is a fragment of second-order logic where quantification is restricted to sets. Importantly, if a graph property is expressible in MSO logic, Courcelle's theorem states that there exists fixed-parameter tractable algorithm in the treewidth of the graph and in the length of the MSO expression.
%
In graphs, we are allowed to use the following variables and relations to express a property in MSO logic:

\begin{compactitem}
  \item standard boolean connectives: $\neg$ (negation), $\wedge$ (and), $\vee$ (or), $\Rightarrow$ (implication),
  \item standard quantifiers: $\exists$ (existential quantifier), $\forall$ (universal quantifier), which can be applied to any variable used to represent vertices, edges, sets of vertices or sets of edges of a graph. By convention, lower-case letters are used to represent vertices and edges, and upper-case letters are used to represent sets of vertices or edges,
  \item the binary equality relation $=$, the binary inclusion relation $\in$, the binary incidence relation $inc(e,v)$ which encodes that an edge $e$ is incident to a vertex $v$.
\end{compactitem}

A \emph{free variable} is a variable not bound by quantifiers. An MSO \emph{sentence} is an MSO formula with no free variables. Let $G$ be a graph, the notation $G\models \phi$ indicates that $G$ verifies the formula. Courcelle's theorem is stated as follows.


\begin{theorem}[Courcelle's theorem~\cite{Courcelle86a,Courcelle90}]
  \label{theorem:courcelle} Let $G$ be a simple graph of treewidth $tw$ and a fixed MSO sentence $\phi$, there exists an algorithm that tests if $G\models \phi$ and runs in $\mathcal{O}(f(tw,|\phi|) \cdot |G|)$, where $f$ is a computable function.
\end{theorem}


\subsection{MSO formulation}

We show, using Courcelle's theorem, that if a static problem $\Pi_S$ is definable in the monadic second-order logic, then its reconfiguration version $\Pi_T$ is FPT when parameterized by the treewidth of the footprint and the lifetime of the temporal graph.

In order to do that, we first convert the temporal graph $\mathcal{G}=(G,\lambda)$ into a static graph $H$ as defined in the following construction.



\begin{construction}
  \label{const:sliding}
  Given a temporal graph $\mathcal{G}=(G,\lambda)$ with bounded lifetime $\tau$ and vertex set $V(G) = \{u_1,\dots,u_n\}$, we consider the following static graph $H$ with vertex set $V(H) = \{ v^t_i \mid u_i \in V(G), t \in [1,\tau]\}$ and such that $H$ contains the following edges:% and edge set $E(H) = \bigcup\limits_{t \in [1,\tau]} E_t \bigcup\limits_{t \in [1,\tau-1]} E_{t,t+1}$ where $E_t = \{v^t_iv^t_j \mid u_iu_j \in G_t\}$ and $E_{t,t+1} = \{v^t_iv^{t+1}_j \mid u_iu_j \in G_t\} \cup \{v^t_iv^{t+1}_i \mid u_i \in V(G)\}$.
  \begin{itemize}
    \item for each $t \in [1,\tau]$, $H$ contains the set of edges $E_t = \{ v^t_iv^t_j \mid u_iu_j \in E(G_t)\}$ (\textit{i.e.} $H$ contains the disjoint union of each snapshot $G_t$), and
    \item for each $t \in [1,\tau-1]$, $H$ contains the edge set $E_{t,t+1} = \{v^t_iv^{t+1}_j \mid u_iu_j \in E(G_t)\} \cup \{v^t_iv^{t+1}_i \mid u_i \in V(G)\}$ (\textit{i.e.} there is an edge between $v^t_i$ and $v^{t+1}_j$ if it is possible to move a token from $u_i$ to $u_j$ at time-step $t$).  
  \end{itemize}
\end{construction}
Note that this construction is related to but is not the same as the time-expanded graph of a temporal graph (as used in, e.g.~\cite{fluschnik_as_2020}).

\begin{lemmarep}
Let $\mathcal{G}=(G,\lambda)$ be a temporal graph, and let $H$ be the graph described in \Cref{const:sliding}. The treewidth of $H$ is at most $\tau\cdot tw$ where $tw$ is the treewidth of the footprint $G$.
\end{lemmarep}
\begin{proof}
  Let $(T,\mathcal{X})$ be a tree decomposition of $G$ of width $tw$. We construct a tree decomposition $(T,\mathcal{Y})$ for $H$ as follows. For each bag $B_x \in \mathcal{X}$, we subtsitute each vertex $u_i \in B_x$ by the set of vertices $\{v^t_i \mid t \in [1,\tau]\}$. Clearly, for any vertex $v^t_i \in V(H)$, the bags in $\mathcal{Y}$ containing $v^t_i$ are connected in $T$ since the bags in $\mathcal{X}$ containing $u_i$ are connected in $T$.
For each edge $v^t_iv^t_j \in E(H)$ (respectively $v^t_iv^{t+1}_j$), let $B_x$ be a bag containing $u_iu_j$ in $\mathcal{X}$, after the substitution, $B_x$ contains the edge $v^t_iv^t_j$ (resp. $v^t_iv^{t+1}_j$). For each edge $v^t_iv^{t+1}_i$, any bag $B_x$ containing the vertex $u_i \in \mathcal{X}$ contains the edge $v^t_iv^{t+1}_i$. Hence, $(T,\mathcal{Y})$ is a tree decomposition of $H$ and since each vertex is replaced by $\tau$ vertices in each bag, the width of $(T,\mathcal{Y})$ is $\tau \cdot tw$.
\end{proof}
 

\noindent An example of a graph produced by \Cref{const:sliding} is depicted in \Cref{fig:to static}.

\begin{figure}[ht]
  \centering
  \scalebox{0.75}{
  \begin{tikzpicture}
    \foreach \Y in {1,2} {
      \foreach \X in {1,...,3} {
        \foreach[count=\l from 1] \x/\y/\a in {0/0/-90,1/-1/-90,0/2/90,1/1/-135,1/3/90}{
          \ifthenelse{\equal{\Y}{2}}{
            \node[smallvertex,label=90+90*\X:{$v^\X_\l$}] (\X\l) at (\X*2.5+\x+\Y*9,\y*0.75) {};
          }{
            \node[smallvertex,label=90+90*\X:{$u_\l$}] (\X\l) at (\X*2.5+\x+\Y*9,\y*0.75) {};
          }
        }

        \ifthenelse{\equal{\Y}{1}}{
          \node at (\X*2.5+0.5+\Y*9,-2) {$G_\X$};
        }
        {}
      }
      
      \foreach \a/\b/\L in {2/1/{2,3},1/3/{1,3},3/5/{1},3/4/{2}} {
        \foreach \X in \L {
          \draw (\X\a) -- (\X\b);
          \ifthenelse{\equal{\Y}{2}}{
          \ifthenelse{\equal{\X}{3}}{}{
            \pgfmathsetmacro{\B}{int(\X + 1)}
            \draw[draw=iris,dashed] (\X\a) -- (\B\b);
            \draw[draw=iris,dashed] (\B\a) -- (\X\b);
          }
          }{}
        }
      }
      \ifthenelse{\equal{\Y}{2}}{
        \foreach[count=\B from 2] \X in {1,2}{
          \foreach \a in {1,...,5}{
          \draw[draw=iris,dashed] (\X\a) -- (\B\a);
        }
        }
      }{}
      
    }
 % \foreach \a in {12,22,33}
 % \node[smallvertex,fill=gray] at (\a.center) {};
    
  \end{tikzpicture}
}
  \caption{\label{fig:to static} Example of a graph produced by \Cref{const:sliding}. The edges in $E_{t,t+1}$ sets are depicted in blue/dashed.}
\end{figure}


Let $\mathcal{G}$ be a temporal graph and let $H$ be the static graph produced by \Cref{const:sliding} with $\mathcal{G}$ given in input. for convenience, we denote by $V_t$ the set of vertices in $H$ with superscript $t$. Notice that finding a solution for $\Pi_T$ in $\mathcal{G}$ is equivalent to finding a vertex set $X$ in $H$ such that:
\begin{compactitem}
  \item for each $t \in [\tau]$, $X \cap V_t$ is a solution for $\Pi_S$ in $H[V_t]$, and
  \item for each $t \in [\tau-1]$, there is a perfect matching between $X \cap V_t$ and $X \cap V_{t+1}$.
  \end{compactitem}
We show that if $\Pi_S$ can be expressed in MSO logic, then we can also express the reconfiguration version $\Pi_T$ in MSO logic, with an increase in formula size by a factor $\tau$.
By Courcelle's theorem, it follows that $\Pi_T$ is FPT in the lifetime of the temporal graph and the treewidth of the footprint combined.

\begin{theoremrep}
  \label{theorem:mso}
  Let $\Pi_T$ be a reconfiguration problem such that its static version $\Pi_S$ is expressable with an MSO formula $\phi(H)$.
  Let $\mathcal{G}=(G,\lambda)$ be a temporal graph such that the treewidth of $G$ is $tw$. There is an algorithm to determine if there is a reconfigurable sequence of size $k$ for $\Pi_T$ in $\mathcal{G}$ in $\mathcal{O}(f(tw,\tau,|\phi|) \cdot \tau\cdot|G|)$.
\end{theoremrep}

\begin{proof}

  Let $H$ be a graph produced by \Cref{const:sliding} on $\mathcal{G}=(G,\lambda)$. Since $\Pi_S$ is expressable in MSO logic, there is a predicate $\phi(G_t,X)$ indicating if a subset of vertices $X$ is a solution for $\Pi_S$ in $G_t$ 
  % Let $\phi$ be an MSO formula such that G
  We construct an MSO formula to express a set $X$ such that $X \cap V_t$ is a solution for $\Pi_S$ in $H[V_t]$ for each $t \in [\tau]$ and such that there is a perfect matching between $X \cap V_t$ and $X \cap V_{t+1}$ for each $t \in [\tau-1]$.


We introduce the following predicate that given a set of edges $M$, two sets of vertices $X$ and $Y$ and a set of edges $E$, $match\_edge$ returns \texttt{true} if
$v$ belongs to $X$,
there is exactly one edge $e\in M$ incident to $v$ and
the other endpoint of $e$ belongs to $Y$. 


\begin{eqnarray*}
  match\_edge(M,X,v,Y) = &  v \in X \wedge \exists e, \exists u, (e \in M) \wedge (u \in Y) &\wedge  \\
  & inc(e,v) \wedge inc(e,u) &\wedge  \\
  & \forall e', (e'\in M \wedge inc(e',v)) \Rightarrow e = e'&
  \end{eqnarray*}

  Notice that given two disjoint vertex sets $X$ and $Y$, a set of edges $M$ forms a perfect matching between $X$ and $Y$ if for every $v$ in $X$ we have $match\_edge(M,X,v,Y)= \texttt{true}$ and for every $u$ in $Y$ we have $match\_edge(M,Y,u,X) = \texttt{true}$.
  Hence, we can formulate the following predicate that, given two vertex sets $X$ and $Y$ and an edge set $E$ returns \texttt{true} if $E$ contains a perfect matching between $X$ and $Y$.


\begin{eqnarray*}
  Matching(X,Y,E) =
  \exists M, \forall v & &\\
                       & (v\not\in X  \wedge v \not\in Y ) &\vee\\
                       & match\_edge(M,X,v,Y) &\vee \\
                       &match\_edge(M,Y,v,X) &\\
\end{eqnarray*}

Hence, we can encode $\Pi_T$ with an MSO formula as follows:
\begin{eqnarray}
\exists X_1 \subseteq V_1,\dots, X_\tau \subseteq V_\tau \bigwedge\limits_{t\leq\tau} \phi(G_t,X_t) \bigwedge\limits_{t<\tau} Matching(X_t,X_{t+1},E_{t,t+1})  
\end{eqnarray}
The size of the formula is in $\mathcal{O}(\tau |\phi|)$ so, we can conclude by \Cref{theorem:courcelle} that $\Pi_T$ is solvable in $\mathcal{O}(f(tw,\tau,|\phi|) \cdot \tau\cdot|G|)$.
\end{proof}

\noindent Let $G$ be a static graph. {\sc Dominating Set} and {\sc Independent Set} can formulated in $G$ as follows:
  \[
    \exists D \subseteq V(G), \forall v \in V(G), v \in D \vee (\exists u\in D, uv\in E(G)) \tag{{\sc Dominating Set}}
  \]
  \[
    \exists I \subseteq V(G), \forall v \in I, \forall u \in I, uv \not\in E(G) \tag{{\sc Independent Set}}. 
  \]
  Hence, by \Cref{theorem:mso}, we can deduce the following result for {\sc Temporal Dominating Set Reconfiguration} and {\sc Temporal Independent Set Reconfiguration}.

\begin{corollary}
{\sc Temporal Dominating Set Reconfiguration} and {\sc Temporal Independent Set Reconfiguration} are solvable in $\mathcal{O}(f(tw,\tau) \cdot \tau\cdot|G|)$.
\end{corollary}




%%% Local Variables:
%%% mode: LaTeX
%%% TeX-master: "main"
%%% End:



\section{Conclusion and future work}
Motivated by both the ability of temporal graphs to model real-world processes and a gap in the theoretical literature, we have defined a general framework for formulating vertex-selection optimisation problems as a temporal reconfiguration problems, and have described several associated algorithmic tools.  

While hardness results on static vertex selection problems will straightforwardly imply hardness for their corresponding reconfiguration versions, we have described several algorithmic approaches, including an approximation algorithm and several fixed-parameter tractable algorithms.  

Several areas of future work present themselves: first, further investigation of which problems can be solved using our results, or which other temporal parameters are useful here.  
Secondly, because temporal problems are so frequently harder than corresponding static ones, it may be interesting to establish negative results that are stronger than those in the static setting, such as $W[k]$-completeness results that consider the lifetime as a parameter.
Finally, we could explore a more restrictive version of the model, where the number of tokens allowed to move at each time step is bounded, or there are other restrictions on the speed of change of the vertex set. 



% ---- Bibliography ----
%
% BibTeX users should specify bibliography style 'splncs04'.
% References will then be sorted and formatted in the correct style.
% 
%\bibliographystyle{splncs04}
\bibliography{biblio}
%
\end{document}
