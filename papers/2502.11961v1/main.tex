\documentclass[a4paper,UKenglish,cleveref, autoref, thm-restate]{lipics-v2021}
%This is a template for producing LIPIcs articles. 
%See lipics-v2021-authors-guidelines.pdf for further information.
%for A4 paper format use option "a4paper", for US-letter use option "letterpaper"
%for british hyphenation rules use option "UKenglish", for american hyphenation rules use option "USenglish"
%for section-numbered lemmas etc., use "numberwithinsect"
%for enabling cleveref support, use "cleveref"
%for enabling autoref support, use "autoref"
%for anonymousing the authors (e.g. for double-blind review), add "anonymous"
%for enabling thm-restate support, use "thm-restate"
%for enabling a two-column layout for the author/affilation part (only applicable for > 6 authors), use "authorcolumns"
%for producing a PDF according the PDF/A standard, add "pdfa"

\pdfoutput=1 %uncomment to ensure pdflatex processing (mandatatory e.g. to submit to arXiv)
\hideLIPIcs  %uncomment to remove references to LIPIcs series (logo, DOI, ...), e.g. when preparing a pre-final version to be uploaded to arXiv or another public repository
 
%\graphicspath{{./graphics/}}%helpful if your graphic files are in another directory
 
\bibliographystyle{plainurl}% the mandatory bibstyle

\title{Parameterised algorithms for temporal reconfiguration problems} %TODO Please add

%\titlerunning{Dummy short title} %TODO optional, please use if title is longer than one line
\author{Tom Davot}{Univ Angers, LERIA, SFR MATHSTIC, F-49000 Angers, France}{tom.davot@univ-angers.fr}{0000-0003-4203-5140}{Supported by EPSRC grant EP/T004878/1. Work completed at University of Glasgow.}
\author{Jessica Enright}{School of Computing Science, University of Glasgow, UK}{jessica.enright@glasgow.ac.uk}{0000-0002-0266-3292}{Supported by EPSRC grant EP/T004878/1.}
\author{Laura Larios-Jones}{School of Computing Science, University of Glasgow, UK}{Laura.Larios-Jones@glasgow.ac.uk}{0000-0003-3322-0176}{}
%TODO mandatory, please use full name; only 1 author per \author macro; first two parameters are mandatory, other parameters can be empty. Please provide at least the name of the affiliation and the country. The full address is optional. Use additional curly braces to indicate the correct name splitting when the last name consists of multiple name parts.

% \author{Joan R. Public\footnote{Optional footnote, e.g. to mark corresponding author}}{Department of Informatics, Dummy College, [optional: Address], Country}{joanrpublic@dummycollege.org}{[orcid]}{[funding]}

\authorrunning{T. Davot, J. Enright, and L. Larios-Jones}

\Copyright{Tom Davot, Jessica Enright, and Laura Larios-Jones} %TODO mandatory, please use full first names. LIPIcs license is "CC-BY";  http://creativecommons.org/licenses/by/3.0/

\begin{CCSXML}
<ccs2012>
   <concept>
       <concept_id>10002950.10003624.10003633.10010917</concept_id>
       <concept_desc>Mathematics of computing~Graph algorithms</concept_desc>
       <concept_significance>500</concept_significance>
       </concept>
   <concept>
       <concept_id>10003752.10003809.10010052</concept_id>
       <concept_desc>Theory of computation~Parameterized complexity and exact algorithms</concept_desc>
       <concept_significance>500</concept_significance>
       </concept>
 </ccs2012>
\end{CCSXML}

\ccsdesc[500]{Mathematics of computing~Graph algorithms}
\ccsdesc[500]{Theory of computation~Parameterized complexity and exact algorithms}

%\supplement{}%optional, e.g. related research data, source code, ... hosted on a repository like zenodo, figshare, GitHub, ...
%\supplementdetails[linktext={opt. text shown instead of the URL}, cite=DBLP:books/mk/GrayR93, subcategory={Description, Subcategory}, swhid={Software Heritage Identifier}]{General Classification (e.g. Software, Dataset, Model, ...)}{URL to related version} %linktext, cite, and subcategory are optional

%\funding{(Optional) general funding statement \dots}%optional, to capture a funding statement, which applies to all authors. Please enter author specific funding statements as fifth argument of the \author macro.

\acknowledgements{For the purpose of open access, the author(s) has applied a Creative Commons Attribution (CC BY) licence to any Author Accepted Manuscript version arising from this submission.}%optional

\nolinenumbers %uncomment to disable line numbering



%Editor-only macros:: begin (do not touch as author)%%%%%%%%%%%%%%%%%%%%%%%%%%%%%%%%%%
\EventEditors{John Q. Open and Joan R. Access}
\EventNoEds{2}
\EventLongTitle{42nd Conference on Very Important Topics (CVIT 2016)}
\EventShortTitle{CVIT 2016}
\EventAcronym{CVIT}
\EventYear{2016}
\EventDate{December 24--27, 2016}
\EventLocation{Little Whinging, United Kingdom}
\EventLogo{}
\SeriesVolume{42}
\ArticleNo{23}
%
%\usepackage[a4paper,margin=1in]{geometry}

% basic
%\usepackage{color,xcolor}
\usepackage{color}
\usepackage{epsfig}
\usepackage{graphicx}
\usepackage{algorithm,algorithmic}
% \usepackage{algpseudocode}
%\usepackage{ulem}

% figure and table
\usepackage{adjustbox}
\usepackage{array}
\usepackage{booktabs}
\usepackage{colortbl}
\usepackage{float,wrapfig}
\usepackage{framed}
\usepackage{hhline}
\usepackage{multirow}
% \usepackage{subcaption} % issues a warning with CVPR/ICCV format
% \usepackage[font=small]{caption}
\usepackage[percent]{overpic}
%\usepackage{tikz} % conflict with ECCV format

% font and character
\usepackage{amsmath,amsfonts,amssymb}
% \let\proof\relax      % for ECCV llncs class
% \let\endproof\relax   % for ECCV llncs class
\usepackage{amsthm} 
\usepackage{bm}
\usepackage{nicefrac}
\usepackage{microtype}
\usepackage{contour}
\usepackage{courier}
%\usepackage{palatino}
%\usepackage{times}

% layout
\usepackage{changepage}
\usepackage{extramarks}
\usepackage{fancyhdr}
\usepackage{lastpage}
\usepackage{setspace}
\usepackage{soul}
\usepackage{xspace}
\usepackage{cuted}
\usepackage{fancybox}
\usepackage{afterpage}
%\usepackage{enumitem} % conflict with IEEE format
%\usepackage{titlesec} % conflict with ECCV format

% ref
% commenting these two out for this submission so it looks the same as RSS example
% \usepackage[breaklinks=true,colorlinks,backref=True]{hyperref}
% \hypersetup{colorlinks,linkcolor={black},citecolor={MSBlue},urlcolor={magenta}}
\usepackage{url}
\usepackage{quoting}
\usepackage{epigraph}

% misc
\usepackage{enumerate}
\usepackage{paralist,tabularx}
\usepackage{comment}
\usepackage{pdfpages}
% \usepackage[draft]{todonotes} % conflict with CVPR/ICCV/ECCV format



% \usepackage{todonotes}
% \usepackage{caption}
% \usepackage{subcaption}

\usepackage{pifont}% http://ctan.org/pkg/pifont

% extra symbols
\usepackage{MnSymbol}


%\usepackage{graphicx}
% Used for displaying a sample figure. If possible, figure files should
% be included in EPS format.
%
% If you use the hyperref package, please uncomment the following line
% to display URLs in blue roman font according to Springer's eBook style:
% \renewcommand\UrlFont{\color{blue}\rmfamily}

%%%%%%%%%%%---SETME-----%%%%%%%%%%%%%
%replace @@ with the submission number submission site.
\newcommand{\thiswork}{INF$^2$\xspace}
%%%%%%%%%%%%%%%%%%%%%%%%%%%%%%%%%%%%


%\newcommand{\rev}[1]{{\color{olivegreen}#1}}
\newcommand{\rev}[1]{{#1}}


\newcommand{\JL}[1]{{\color{cyan}[\textbf{\sc JLee}: \textit{#1}]}}
\newcommand{\JW}[1]{{\color{orange}[\textbf{\sc JJung}: \textit{#1}]}}
\newcommand{\JY}[1]{{\color{blue(ncs)}[\textbf{\sc JSong}: \textit{#1}]}}
\newcommand{\HS}[1]{{\color{magenta}[\textbf{\sc HJang}: \textit{#1}]}}
\newcommand{\CS}[1]{{\color{navy}[\textbf{\sc CShin}: \textit{#1}]}}
\newcommand{\SN}[1]{{\color{olive}[\textbf{\sc SNoh}: \textit{#1}]}}

%\def\final{}   % uncomment this for the submission version
\ifdefined\final
\renewcommand{\JL}[1]{}
\renewcommand{\JW}[1]{}
\renewcommand{\JY}[1]{}
\renewcommand{\HS}[1]{}
\renewcommand{\CS}[1]{}
\renewcommand{\SN}[1]{}
\fi

%%% Notion for baseline approaches %%% 
\newcommand{\baseline}{offloading-based batched inference\xspace}
\newcommand{\Baseline}{Offloading-based batched inference\xspace}


\newcommand{\ans}{attention-near storage\xspace}
\newcommand{\Ans}{Attention-near storage\xspace}
\newcommand{\ANS}{Attention-Near Storage\xspace}

\newcommand{\wb}{delayed KV cache writeback\xspace}
\newcommand{\Wb}{Delayed KV cache writeback\xspace}
\newcommand{\WB}{Delayed KV Cache Writeback\xspace}

\newcommand{\xcache}{X-cache\xspace}
\newcommand{\XCACHE}{X-Cache\xspace}


%%% Notions for our methods %%%
\newcommand{\schemea}{\textbf{Expanding supported maximum sequence length with optimized performance}\xspace}
\newcommand{\Schemea}{\textbf{Expanding supported maximum sequence length with optimized performance}\xspace}

\newcommand{\schemeb}{\textbf{Optimizing the storage device performance}\xspace}
\newcommand{\Schemeb}{\textbf{Optimizing the storage device performance}\xspace}

\newcommand{\schemec}{\textbf{Orthogonally supporting Compression Techniques}\xspace}
\newcommand{\Schemec}{\textbf{Orthogonally supporting Compression Techniques}\xspace}



% Circular numbers
\usepackage{tikz}
\newcommand*\circled[1]{\tikz[baseline=(char.base)]{
            \node[shape=circle,draw,inner sep=0.4pt] (char) {#1};}}

\newcommand*\bcircled[1]{\tikz[baseline=(char.base)]{
            \node[shape=circle,draw,inner sep=0.4pt, fill=black, text=white] (char) {#1};}}




\begin{document} 
\maketitle              % typeset the header of the contribution
%
\begin{abstract}
Given a static vertex-selection problem (e.g. independent set, dominating set) on a graph, we can define a corresponding temporal reconfiguration problem on a temporal graph which asks for a sequence of solutions to the vertex-selection problem at each time such that we can reconfigure from one solution to the next. We can think of each solution in the sequence as a set of vertices with tokens placed on them; our reconfiguration model allows us to slide tokens along active edges of a temporal graph.  

We show that it is possible to efficiently check whether one solution can be reconfigured to another, and show that approximation results on the static vertex-selection problem can be adapted with a lifetime factor to the reconfiguration version.
Our main contributions are fixed-parameter tractable algorithms with respect to: enumeration time of the related static problem; the combination of temporal neighbourhood diversity and lifetime of the input graph; and the combination of lifetime and treewidth of the footprint graph.

\keywords{parameterised algorithms, temporal graphs, reconfiguration}

\end{abstract}
\section{Introduction}
\label{sec:introduction}
The business processes of organizations are experiencing ever-increasing complexity due to the large amount of data, high number of users, and high-tech devices involved \cite{martin2021pmopportunitieschallenges, beerepoot2023biggestbpmproblems}. This complexity may cause business processes to deviate from normal control flow due to unforeseen and disruptive anomalies \cite{adams2023proceddsriftdetection}. These control-flow anomalies manifest as unknown, skipped, and wrongly-ordered activities in the traces of event logs monitored from the execution of business processes \cite{ko2023adsystematicreview}. For the sake of clarity, let us consider an illustrative example of such anomalies. Figure \ref{FP_ANOMALIES} shows a so-called event log footprint, which captures the control flow relations of four activities of a hypothetical event log. In particular, this footprint captures the control-flow relations between activities \texttt{a}, \texttt{b}, \texttt{c} and \texttt{d}. These are the causal ($\rightarrow$) relation, concurrent ($\parallel$) relation, and other ($\#$) relations such as exclusivity or non-local dependency \cite{aalst2022pmhandbook}. In addition, on the right are six traces, of which five exhibit skipped, wrongly-ordered and unknown control-flow anomalies. For example, $\langle$\texttt{a b d}$\rangle$ has a skipped activity, which is \texttt{c}. Because of this skipped activity, the control-flow relation \texttt{b}$\,\#\,$\texttt{d} is violated, since \texttt{d} directly follows \texttt{b} in the anomalous trace.
\begin{figure}[!t]
\centering
\includegraphics[width=0.9\columnwidth]{images/FP_ANOMALIES.png}
\caption{An example event log footprint with six traces, of which five exhibit control-flow anomalies.}
\label{FP_ANOMALIES}
\end{figure}

\subsection{Control-flow anomaly detection}
Control-flow anomaly detection techniques aim to characterize the normal control flow from event logs and verify whether these deviations occur in new event logs \cite{ko2023adsystematicreview}. To develop control-flow anomaly detection techniques, \revision{process mining} has seen widespread adoption owing to process discovery and \revision{conformance checking}. On the one hand, process discovery is a set of algorithms that encode control-flow relations as a set of model elements and constraints according to a given modeling formalism \cite{aalst2022pmhandbook}; hereafter, we refer to the Petri net, a widespread modeling formalism. On the other hand, \revision{conformance checking} is an explainable set of algorithms that allows linking any deviations with the reference Petri net and providing the fitness measure, namely a measure of how much the Petri net fits the new event log \cite{aalst2022pmhandbook}. Many control-flow anomaly detection techniques based on \revision{conformance checking} (hereafter, \revision{conformance checking}-based techniques) use the fitness measure to determine whether an event log is anomalous \cite{bezerra2009pmad, bezerra2013adlogspais, myers2018icsadpm, pecchia2020applicationfailuresanalysispm}. 

The scientific literature also includes many \revision{conformance checking}-independent techniques for control-flow anomaly detection that combine specific types of trace encodings with machine/deep learning \cite{ko2023adsystematicreview, tavares2023pmtraceencoding}. Whereas these techniques are very effective, their explainability is challenging due to both the type of trace encoding employed and the machine/deep learning model used \cite{rawal2022trustworthyaiadvances,li2023explainablead}. Hence, in the following, we focus on the shortcomings of \revision{conformance checking}-based techniques to investigate whether it is possible to support the development of competitive control-flow anomaly detection techniques while maintaining the explainable nature of \revision{conformance checking}.
\begin{figure}[!t]
\centering
\includegraphics[width=\columnwidth]{images/HIGH_LEVEL_VIEW.png}
\caption{A high-level view of the proposed framework for combining \revision{process mining}-based feature extraction with dimensionality reduction for control-flow anomaly detection.}
\label{HIGH_LEVEL_VIEW}
\end{figure}

\subsection{Shortcomings of \revision{conformance checking}-based techniques}
Unfortunately, the detection effectiveness of \revision{conformance checking}-based techniques is affected by noisy data and low-quality Petri nets, which may be due to human errors in the modeling process or representational bias of process discovery algorithms \cite{bezerra2013adlogspais, pecchia2020applicationfailuresanalysispm, aalst2016pm}. Specifically, on the one hand, noisy data may introduce infrequent and deceptive control-flow relations that may result in inconsistent fitness measures, whereas, on the other hand, checking event logs against a low-quality Petri net could lead to an unreliable distribution of fitness measures. Nonetheless, such Petri nets can still be used as references to obtain insightful information for \revision{process mining}-based feature extraction, supporting the development of competitive and explainable \revision{conformance checking}-based techniques for control-flow anomaly detection despite the problems above. For example, a few works outline that token-based \revision{conformance checking} can be used for \revision{process mining}-based feature extraction to build tabular data and develop effective \revision{conformance checking}-based techniques for control-flow anomaly detection \cite{singh2022lapmsh, debenedictis2023dtadiiot}. However, to the best of our knowledge, the scientific literature lacks a structured proposal for \revision{process mining}-based feature extraction using the state-of-the-art \revision{conformance checking} variant, namely alignment-based \revision{conformance checking}.

\subsection{Contributions}
We propose a novel \revision{process mining}-based feature extraction approach with alignment-based \revision{conformance checking}. This variant aligns the deviating control flow with a reference Petri net; the resulting alignment can be inspected to extract additional statistics such as the number of times a given activity caused mismatches \cite{aalst2022pmhandbook}. We integrate this approach into a flexible and explainable framework for developing techniques for control-flow anomaly detection. The framework combines \revision{process mining}-based feature extraction and dimensionality reduction to handle high-dimensional feature sets, achieve detection effectiveness, and support explainability. Notably, in addition to our proposed \revision{process mining}-based feature extraction approach, the framework allows employing other approaches, enabling a fair comparison of multiple \revision{conformance checking}-based and \revision{conformance checking}-independent techniques for control-flow anomaly detection. Figure \ref{HIGH_LEVEL_VIEW} shows a high-level view of the framework. Business processes are monitored, and event logs obtained from the database of information systems. Subsequently, \revision{process mining}-based feature extraction is applied to these event logs and tabular data input to dimensionality reduction to identify control-flow anomalies. We apply several \revision{conformance checking}-based and \revision{conformance checking}-independent framework techniques to publicly available datasets, simulated data of a case study from railways, and real-world data of a case study from healthcare. We show that the framework techniques implementing our approach outperform the baseline \revision{conformance checking}-based techniques while maintaining the explainable nature of \revision{conformance checking}.

In summary, the contributions of this paper are as follows.
\begin{itemize}
    \item{
        A novel \revision{process mining}-based feature extraction approach to support the development of competitive and explainable \revision{conformance checking}-based techniques for control-flow anomaly detection.
    }
    \item{
        A flexible and explainable framework for developing techniques for control-flow anomaly detection using \revision{process mining}-based feature extraction and dimensionality reduction.
    }
    \item{
        Application to synthetic and real-world datasets of several \revision{conformance checking}-based and \revision{conformance checking}-independent framework techniques, evaluating their detection effectiveness and explainability.
    }
\end{itemize}

The rest of the paper is organized as follows.
\begin{itemize}
    \item Section \ref{sec:related_work} reviews the existing techniques for control-flow anomaly detection, categorizing them into \revision{conformance checking}-based and \revision{conformance checking}-independent techniques.
    \item Section \ref{sec:abccfe} provides the preliminaries of \revision{process mining} to establish the notation used throughout the paper, and delves into the details of the proposed \revision{process mining}-based feature extraction approach with alignment-based \revision{conformance checking}.
    \item Section \ref{sec:framework} describes the framework for developing \revision{conformance checking}-based and \revision{conformance checking}-independent techniques for control-flow anomaly detection that combine \revision{process mining}-based feature extraction and dimensionality reduction.
    \item Section \ref{sec:evaluation} presents the experiments conducted with multiple framework and baseline techniques using data from publicly available datasets and case studies.
    \item Section \ref{sec:conclusions} draws the conclusions and presents future work.
\end{itemize}
% !TeX root = main.tex 


\newcommand{\lnote}{\textcolor[rgb]{1,0,0}{Lydia: }\textcolor[rgb]{0,0,1}}
\newcommand{\todo}{\textcolor[rgb]{1,0,0.5}{To do: }\textcolor[rgb]{0.5,0,1}}


\newcommand{\state}{S}
\newcommand{\meas}{M}
\newcommand{\out}{\mathrm{out}}
\newcommand{\piv}{\mathrm{piv}}
\newcommand{\pivotal}{\mathrm{pivotal}}
\newcommand{\isnot}{\mathrm{not}}
\newcommand{\pred}{^\mathrm{predict}}
\newcommand{\act}{^\mathrm{act}}
\newcommand{\pre}{^\mathrm{pre}}
\newcommand{\post}{^\mathrm{post}}
\newcommand{\calM}{\mathcal{M}}

\newcommand{\game}{\mathbf{V}}
\newcommand{\strategyspace}{S}
\newcommand{\payoff}[1]{V^{#1}}
\newcommand{\eff}[1]{E^{#1}}
\newcommand{\p}{\vect{p}}
\newcommand{\simplex}[1]{\Delta^{#1}}

\newcommand{\recdec}[1]{\bar{D}(\hat{Y}_{#1})}





\newcommand{\sphereone}{\calS^1}
\newcommand{\samplen}{S^n}
\newcommand{\wA}{w}%{w_{\mathfrak{a}}}
\newcommand{\Awa}{A_{\wA}}
\newcommand{\Ytil}{\widetilde{Y}}
\newcommand{\Xtil}{\widetilde{X}}
\newcommand{\wst}{w_*}
\newcommand{\wls}{\widehat{w}_{\mathrm{LS}}}
\newcommand{\dec}{^\mathrm{dec}}
\newcommand{\sub}{^\mathrm{sub}}

\newcommand{\calP}{\mathcal{P}}
\newcommand{\totspace}{\calZ}
\newcommand{\clspace}{\calX}
\newcommand{\attspace}{\calA}

\newcommand{\Ftil}{\widetilde{\calF}}

\newcommand{\totx}{Z}
\newcommand{\classx}{X}
\newcommand{\attx}{A}
\newcommand{\calL}{\mathcal{L}}



\newcommand{\defeq}{\mathrel{\mathop:}=}
\newcommand{\vect}[1]{\ensuremath{\mathbf{#1}}}
\newcommand{\mat}[1]{\ensuremath{\mathbf{#1}}}
\newcommand{\dd}{\mathrm{d}}
\newcommand{\grad}{\nabla}
\newcommand{\hess}{\nabla^2}
\newcommand{\argmin}{\mathop{\rm argmin}}
\newcommand{\argmax}{\mathop{\rm argmax}}
\newcommand{\Ind}[1]{\mathbf{1}\{#1\}}

\newcommand{\norm}[1]{\left\|{#1}\right\|}
\newcommand{\fnorm}[1]{\|{#1}\|_{\text{F}}}
\newcommand{\spnorm}[2]{\left\| {#1} \right\|_{\text{S}({#2})}}
\newcommand{\sigmin}{\sigma_{\min}}
\newcommand{\tr}{\text{tr}}
\renewcommand{\det}{\text{det}}
\newcommand{\rank}{\text{rank}}
\newcommand{\logdet}{\text{logdet}}
\newcommand{\trans}{^{\top}}
\newcommand{\poly}{\text{poly}}
\newcommand{\polylog}{\text{polylog}}
\newcommand{\st}{\text{s.t.~}}
\newcommand{\proj}{\mathcal{P}}
\newcommand{\projII}{\mathcal{P}_{\parallel}}
\newcommand{\projT}{\mathcal{P}_{\perp}}
\newcommand{\projX}{\mathcal{P}_{\mathcal{X}^\star}}
\newcommand{\inner}[1]{\langle #1 \rangle}

\renewcommand{\Pr}{\mathbb{P}}
\newcommand{\Z}{\mathbb{Z}}
\newcommand{\N}{\mathbb{N}}
\newcommand{\R}{\mathbb{R}}
\newcommand{\E}{\mathbb{E}}
\newcommand{\F}{\mathcal{F}}
\newcommand{\var}{\mathrm{var}}
\newcommand{\cov}{\mathrm{cov}}


\newcommand{\calN}{\mathcal{N}}

\newcommand{\jccomment}{\textcolor[rgb]{1,0,0}{C: }\textcolor[rgb]{1,0,1}}
\newcommand{\fracpar}[2]{\frac{\partial #1}{\partial  #2}}

\newcommand{\A}{\mathcal{A}}
\newcommand{\B}{\mat{B}}
%\newcommand{\C}{\mat{C}}

\newcommand{\I}{\mat{I}}
\newcommand{\M}{\mat{M}}
\newcommand{\D}{\mat{D}}
%\newcommand{\U}{\mat{U}}
\newcommand{\V}{\mat{V}}
\newcommand{\W}{\mat{W}}
\newcommand{\X}{\mat{X}}
\newcommand{\Y}{\mat{Y}}
\newcommand{\mSigma}{\mat{\Sigma}}
\newcommand{\mLambda}{\mat{\Lambda}}
\newcommand{\e}{\vect{e}}
\newcommand{\g}{\vect{g}}
\renewcommand{\u}{\vect{u}}
\newcommand{\w}{\vect{w}}
\newcommand{\x}{\vect{x}}
\newcommand{\y}{\vect{y}}
\newcommand{\z}{\vect{z}}
\newcommand{\fI}{\mathfrak{I}}
\newcommand{\fS}{\mathfrak{S}}
\newcommand{\fE}{\mathfrak{E}}
\newcommand{\fF}{\mathfrak{F}}

\newcommand{\Risk}{\mathcal{R}}

\renewcommand{\L}{\mathcal{L}}
\renewcommand{\H}{\mathcal{H}}

\newcommand{\cn}{\kappa}
\newcommand{\nn}{\nonumber}


\newcommand{\Hess}{\nabla^2}
\newcommand{\tlO}{\tilde{O}}
\newcommand{\tlOmega}{\tilde{\Omega}}

\newcommand{\calF}{\mathcal{F}}
\newcommand{\fhat}{\widehat{f}}
\newcommand{\calS}{\mathcal{S}}

\newcommand{\calX}{\mathcal{X}}
\newcommand{\calY}{\mathcal{Y}}
\newcommand{\calD}{\mathcal{D}}
\newcommand{\calZ}{\mathcal{Z}}
\newcommand{\calA}{\mathcal{A}}
\newcommand{\fbayes}{f^B}
\newcommand{\func}{f^U}


\newcommand{\bayscore}{\text{calibrated Bayes score}}
\newcommand{\bayrisk}{\text{calibrated Bayes risk}}

\newtheorem{example}{Example}[section]
\newtheorem{exc}{Exercise}[section]
%\newtheorem{rem}{Remark}[section]

\newtheorem{theorem}{Theorem}[section]
\newtheorem{definition}{Definition}
\newtheorem{proposition}[theorem]{Proposition}
\newtheorem{corollary}[theorem]{Corollary}

\newtheorem{remark}{Remark}[section]
\newtheorem{lemma}[theorem]{Lemma}
\newtheorem{claim}[theorem]{Claim}
\newtheorem{fact}[theorem]{Fact}
\newtheorem{assumption}{Assumption}

\newcommand{\iidsim}{\overset{\mathrm{i.i.d.}}{\sim}}
\newcommand{\unifsim}{\overset{\mathrm{unif}}{\sim}}
\newcommand{\sign}{\mathrm{sign}}
\newcommand{\wbar}{\overline{w}}
\newcommand{\what}{\widehat{w}}
\newcommand{\KL}{\mathrm{KL}}
\newcommand{\Bern}{\mathrm{Bernoulli}}
\newcommand{\ihat}{\widehat{i}}
\newcommand{\Dwst}{\calD^{w_*}}
\newcommand{\fls}{\widehat{f}_{n}}


\newcommand{\brpi}{\pi^{br}}
\newcommand{\brtheta}{\theta^{br}}

% \newcommand{\M}{\mat{M}}
% \newcommand\Mmh{\mat{M}^{-1/2}}
% \newcommand{\A}{\mat{A}}
% \newcommand{\B}{\mat{B}}
% \newcommand{\C}{\mat{C}}
% \newcommand{\Et}[1][t]{\mat{E_{#1}}}
% \newcommand{\Etp}{\Et[t+1]}
% \newcommand{\Errt}[1][t]{\mat{\bigtriangleup_{#1}}}
% \newcommand\cnM{\kappa}
% \newcommand{\cn}[1]{\kappa\left(#1\right)}
% \newcommand\X{\mat{X}}
% \newcommand\fstar{f_*}
% \newcommand\Xt[1][t]{\mat{X_{#1}}}
% \newcommand\ut[1][t]{{u_{#1}}}
% \newcommand\Xtinv{\inv{\Xt}}
% \newcommand\Xtp{\mat{X_{t+1}}}
% \newcommand\Xtpinv{\inv{\left(\mat{X_{t+1}}\right)}}
% \newcommand\U{\mat{U}}
% \newcommand\UTr{\trans{\mat{U}}}
% \newcommand{\Ut}[1][t]{\mat{U_{#1}}}
% \newcommand{\Utinv}{\inv{\Ut}}
% \newcommand{\UtTr}[1][t]{\trans{\mat{U_{#1}}}}
% \newcommand\Utp{\mat{U_{t+1}}}
% \newcommand\UtpTr{\trans{\mat{U}_{t+1}}}
% \newcommand\Utptild{\mat{\widetilde{U}_{t+1}}}
% \newcommand\Us{\mat{U^*}}
% \newcommand\UsTr{\trans{\mat{U^*}}}
% \newcommand{\Sigs}{\mat{\Sigma}}
% \newcommand{\Sigsmh}{\Sigs^{-1/2}}
% \newcommand{\eye}{\mat{I}}
% \newcommand{\twonormbound}{\left(4+\DPhi{\M}{\Xt[0]}\right)\twonorm{\M}}
% \newcommand{\lamj}{\lambda_j}

% \renewcommand\u{\vect{u}}
% \newcommand\uTr{\trans{\vect{u}}}
% \renewcommand\v{\vect{v}}
% \newcommand\vTr{\trans{\vect{v}}}
% \newcommand\w{\vect{w}}
% \newcommand\wTr{\trans{\vect{w}}}
% \newcommand\wperp{\vect{w}_{\perp}}
% \newcommand\wperpTr{\trans{\vect{w}_{\perp}}}
% \newcommand\wj{\vect{w_j}}
% \newcommand\vj{\vect{v_j}}
% \newcommand\wjTr{\trans{\vect{w_j}}}
% \newcommand\vjTr{\trans{\vect{v_j}}}

% \newcommand{\DPhi}[2]{\ensuremath{D_{\Phi}\left(#1,#2\right)}}
% \newcommand\matmult{{\omega}}


%\input{max-edge}
\section{Proposition: FlavorDiffusion}

\subsection{Sub-Graph Sampling}

FlavorDiffusion is built upon the DIFUSCO Gaussian noise-based diffusion model, extending its capabilities to structured food-chemical graphs. The core objective is to train a model capable of reconstructing subgraphs sampled from the full heterogeneous graph \( G = (V, E) \) while leveraging node attributes as guidance. 

The full graph consists of a diverse set of nodes \( V \), including hub ingredients, non-hub ingredients, flavor compounds, and drug compounds, with edges \( E \) encoding the strength of their relationships as continuous values in \( [0,1] \). We define a dataset of subgraphs, where each sample contains \( m \) nodes selected from \( G \). These subgraphs are denoted as:

\begin{equation*}
    \mathcal{D}_m = \{ G_i = (V_i, E_i) \}_{i=1}^{N},
\end{equation*}

where each subgraph \( G_i \) has \( |V_i| = m \) nodes and an adjacency matrix \( E_i \) of size \( m \times m \), representing pairwise edge scores. The dataset is partitioned into training (\( N_t \)) and validation (\( N_v \)) subsets.

\subsection{Forward Diffusion Process}

For a single data point \( G_i = (V_i, E_i) \) sampled from the dataset, we define the diffusion process over its edge set \( E_i \). By convention, we denote the corrupted version of \( E_i \) at timestep \( t \) as \( x_t \), aligning with standard diffusion formalisms. The node representations, encompassing all vertex features, are denoted as \( \mathbf{Emb} \).

The forward diffusion process follows a Markovian Gaussian noise injection, progressively perturbing the edges \( x_t \) while preserving node representations:

\begin{equation*}
    q(x_t | x_{t-1}) = \mathcal{N}(x_t; \sqrt{1 - \beta_t} x_{t-1}, \beta_t I),
\end{equation*}

where \( \beta_t \) is a predefined noise variance at timestep \( t \). Given an initial clean edge matrix \( x_0 = E_i \), we can analytically express the direct corruption of \( x_0 \) at any timestep \( t \) as:

\begin{equation*}
    q(x_t | x_0) = \mathcal{N}(x_t; \sqrt{\bar{\alpha}_t} x_0, (1 - \bar{\alpha}_t) I),
\end{equation*}

where \( \bar{\alpha}_t = \prod_{s=1}^{t} (1 - \beta_s) \) represents the cumulative noise effect over time. This formulation allows direct sampling of \( x_t \) from \( x_0 \), bypassing iterative updates.

In this framework, the edge structure is progressively degraded into Gaussian noise, while node representations \( \mathbf{Emb} \) remain unchanged, ensuring that denoising relies on learned node attributes.

\subsection{Reverse Denoising Process}

The reverse process seeks to recover \( x_0 \) from the fully corrupted state \( x_T \), learning to remove noise in a stepwise manner. The key assumption is that the forward process follows a Gaussian transition, enabling an analytically derived reverse process.

Given the Markovian nature of the diffusion process, we define the true posterior:

\begin{equation*}
    q(x_{t-1} | x_t, x_0) = \mathcal{N}(x_{t-1}; \tilde{\mu}_t(x_t, x_0), \tilde{\beta}_t I),
\end{equation*}

where the posterior mean and variance are derived as:

\begin{equation*}
    \tilde{\mu}_t(x_t, x_0) = \frac{\sqrt{\bar{\alpha}_{t-1}} \beta_t}{1 - \bar{\alpha}_t} x_0 + \frac{\sqrt{\alpha_t} (1 - \bar{\alpha}_{t-1})}{1 - \bar{\alpha}_t} x_t,
\end{equation*}

\begin{equation*}
    \tilde{\beta}_t = \frac{1 - \bar{\alpha}_{t-1}}{1 - \bar{\alpha}_t} \beta_t.
\end{equation*}

Since \( x_0 \) is unknown, we train a model \( p_{\theta}(x_0 | x_t) \) to approximate it. Substituting the predicted \( x_0 \), the learned reverse process is modeled as:

\begin{align*}
    p_{\theta}(x_{t-1} | x_t, \mathbf{Emb}) &= \\
    \mathcal{N}\big(x_{t-1}; &\mu_{\theta}(x_t, t, \mathbf{Emb}), \Sigma_{\theta}(x_t, t)\big),
\end{align*}

where \( \mu_{\theta} \) is the learned estimate for \( \tilde{\mu}_t(x_t, x_0) \), and the variance term is fixed as \( \Sigma_{\theta}(x_t, t) = \tilde{\beta}_t I \), avoiding the need for explicit learning. The function \( \mu_{\theta} \) is now conditioned on the node representations (\(\mathbf{Emb}\)) of the two vertices forming the edge.

Using the DDPM convention, we parameterize \( \mu_{\theta} \) as:

\begin{align*}
    \mu_{\theta}(x_t, t, \mathbf{Emb}) &= \frac{1}{\sqrt{\alpha_t}} \Bigg( x_t \\
    &\quad - \frac{\beta_t}{\sqrt{1 - \bar{\alpha}_t}} \epsilon_{\theta}(x_t, t, \mathbf{Emb}) \Bigg),
\end{align*}

where \( \epsilon_{\theta}(x_t, t, \mathbf{Emb}) \) is the learned noise estimate, which is now explicitly conditioned on the representations of the two nodes forming the edge. The node representations provide additional context for denoising by leveraging node-specific features.

\subsection{Optimization via Variational Lower Bound}

To train the reverse model, we maximize the variational lower bound (ELBO), decomposed as:

\begin{align*}
    \mathcal{L}_{\text{ELBO}} = E_q \Bigg[
    \log p_{\theta}(x_0 | x_1, \mathbf{Emb}) \\
    - \sum_{t=1}^{T} D_{\text{KL}}\big(q(x_{t-1} | x_t, x_0) \| p_{\theta}(x_{t-1} | x_t, \mathbf{Emb})\big) 
    \Bigg].
\end{align*}

Here, \( T \) represents the total number of diffusion steps, defining the depth of the forward and reverse process. The KL divergence encourages the learned transitions to match the true posterior. Since \( q(x_t | x_0) \) is Gaussian, minimizing \( D_{\text{KL}} \) is equivalent to predicting the noise component \( \epsilon \) added during diffusion. Thus, the training objective simplifies to:

\begin{equation*}
    \mathcal{L}_{\text{recon}} = E_{t, x_0, \epsilon} \left[ \| \epsilon - \epsilon_{\theta}(x_t, t, \mathbf{Emb}) \|^2 \right].
\end{equation*}

This loss ensures that \( \epsilon_{\theta} \) effectively estimates the noise introduced in the forward process while incorporating node representations. By iteratively refining the denoising function, FlavorDiffusion reconstructs the original ingredient-ingredient graph from noisy subgraphs, leveraging both the structural edge information and node attributes to enhance predictive modeling for food pairing analysis.

\subsection{Inference}

Graph reconstruction follows Denoising Diffusion Implicit Models (DDIM) for efficient and deterministic sampling. Unlike DDPM, DDIM removes noise via a non-Markovian update, accelerating inference.

Starting from \( x_T \sim \mathcal{N}(0, I) \), the reverse process iterates:

\begin{equation*}
    x_{t-1} = \sqrt{\bar{\alpha}_{t-1}} \hat{x}_0 + \sqrt{1 - \bar{\alpha}_{t-1}} \cdot \epsilon_{\theta}(x_t, t, \mathbf{Emb}),
\end{equation*}

where the predicted clean graph is:

\begin{equation*}
    \hat{x}_0 = \frac{x_t - \sqrt{1 - \bar{\alpha}_t} \epsilon_{\theta}(x_t, t, \mathbf{Emb})}{\sqrt{\bar{\alpha}_t}}.
\end{equation*}

Iterating from \( T \) to \( 0 \), the model refines \( x_t \) to recover ingredient-ingredient relationships. DDIM ensures fast, stable, and chemically meaningful reconstructions.

\subsection{Model Architecture}

The noise prediction network \( \epsilon_{\theta}(x_t, t, \mathbf{V}) \) employs an anisotropic GNN to iteratively refine node and edge embeddings. Let \( h_i^\ell \in \mathbf{R}^d \) and \( e_{ij}^\ell \in \mathbf{R}^{d_e} \) denote the node and edge features at layer \( \ell \), respectively. The refinement process updates both edge and node embeddings through the following operations:

\paragraph{Edge Refinement} The initial edge embeddings \( e_{ij}^0 \) are set as the corresponding values from the noisy edge representation \( x_t \). At each layer \( \ell \), the intermediate edge embeddings \( \hat{e}_{ij}^\ell \) are updated as:

\begin{equation*}
    \hat{e}_{ij}^\ell = P^\ell e_{ij}^\ell + Q^\ell h_i^\ell + R^\ell h_j^\ell,
\end{equation*}

where \( P^\ell, Q^\ell, R^\ell \in \mathbf{R}^{d_e \times d_e} \) are learnable parameters. The refined edge embedding \( e_{ij}^{\ell+1} \) is then computed as:

\begin{equation*}
    e_{ij}^{\ell+1} = e_{ij}^\ell + \text{MLP}_e\big(\text{BN}(\hat{e}_{ij}^\ell)\big) + \text{MLP}_t(t),
\end{equation*}

where \( \text{MLP}_e \) is a 2-layer perceptron and \( \text{MLP}_t \) embeds the diffusion timestep \( t \) using sinusoidal features.

\paragraph{Node Refinement}
The node embeddings \( h_i^\ell \) are refined by aggregating information from neighboring nodes and their associated edges. The update rule for \( h_i^{\ell+1} \) is given by:

\begin{equation*}
    h_i^{\ell+1} = h_i^\ell + \alpha \cdot \text{BN}\Big(U^\ell h_i^\ell + \sum_{j \in \mathcal{N}(i)} \sigma(\hat{e}_{ij}^\ell) \odot V^\ell h_j^\ell\Big),
\end{equation*}

where \( U^\ell, V^\ell \in \mathbf{R}^{d \times d} \) are learnable parameter matrices, \( \sigma \) is the sigmoid activation function used for edge gating, \( \odot \) denotes the Hadamard (element-wise) product, \( \mathcal{N}(i) \) represents the set of neighbors for node \( i \), and \( \alpha \) is the ReLU activation applied after aggregation.

\paragraph{Final Prediction}
After \( L \) GNN layers, the final refined edge embeddings \( E^{(L)} \in \mathbf{R}^{N \times N \times d_e} \) are passed through a ReLU activation and a multi-layer perceptron (MLP) to predict the noise:

\begin{equation*}
    \epsilon_{\theta}(x_t, t, \mathbf{V}) = \text{MLP}\big(\text{ReLU}(E^{(L)})\big).
\end{equation*}

This formulation ensures that both node and edge embeddings are iteratively refined to capture local and global graph structure, enabling robust denoising and reconstruction of ingredient-ingredient relationships.


\section{Fixed parameter tractability by lifetime and temporal neighbourhood diversity}
\label{sec:ndiversity}
\newcommand{\tndg}[1][\mathcal{G}]{\ensuremath{TND_{#1}}\xspace}
\newcommand{\ndg}[1][\mathcal{G}]{\ensuremath{ND_{#1}}\xspace}
In this section we present a fixed-parameter algorithm, parametrised by lifetime and the temporal neighbourhood diversity of the temporal graph, to solve a class of reconfiguration problems that we call \emph{temporal neighbourhood diversity locally decidable}. 

We need a number of algorithmic tools to build toward this overall result.  First, in \Cref{subsec:definitions}, we give definitions and notation necessary for this section, including defining temporal neighbourhood diversity, as well as the class of temporal neighbourhood diversity locally decidable problems. These build on analogous definitions in static graphs. 

%In \Cref{subsec:reconfiguration}, we introduce a key subroutine of the parameterized algorithm, which we then use in Subsection \Cref{subsec:algo}
Then, in \Cref{subsec:algo} we describe an overall algorithm to solve our restricted class of problems that is in FPT with respect to temporal neighbourhood diversity and lifetime. This algorithm uses a critical subroutine that constitutes the majority of the technical detail, and is presented in \Cref{subsec:reconfiguration}.  The subroutine uses a reduction to the efficiently-solvable circulation problem to give the key result (in \Cref{lemma:compute reconfigurable sequence}) The result allows us to efficiently generate an optimal reconfigurable sequence that is compatible with that candidate sequence given a candidate reconfiguration sequence of a type specific to temporal neighbourhood diversity locally decidable problems. 


% Then, we present the intuition of the algorithm in \Cref{subsec:algo}. Finally, we describe and show the correctness of the main subroutine of the algorithm in \Cref{subsec:reconfiguration}.


\subsection{Definitions}
\label{subsec:definitions}

% \subsubsection{In static graphs}

Neighbourhood diversity is a static graph parameter
% We first state the formal definitions of neighbourhood diversity and neighbourhood diversity locally decidable problem. This parameter has been
introduced by Lampis~\cite{Lampis12}:


\begin{definition}[Neighbourhood Diversity~\cite{Lampis12}]
  The \emph{neighbourhood diversity} of a static graph $G$ is the minimum value $k$ such that the vertices of $G$ can be partitioned into $k$ classes $V_1,\dots,V_k$ such that 
%
  for each pair of vertices $u$ and $v$ in a same class $V_i$, we have $N(u) \setminus \{v\} = N(v) \setminus \{u\}$.  We call $V_1,\dots,V_k$ a \emph{neigbourhood diversity partition} of $\mathcal{G}$.
\end{definition}

Notice that each set $V_i$ of $P$ forms either an independent set or a clique. Moreover, for any pair of sets $V_i$ and $V_j$ either no vertex of $V_i$ is adjacent to any vertex of $V_j$ or every vertex of $V_i$ is adjacent to every vertex of $V_j$. We distinguish two types of classes: $V_i$ is a \emph{clique-class} if $G[V_i]$ is a clique and $V_i$ is an \emph{independent-class} otherwise.
%
% Observe that the definition of temporal neigbourhood diversity is a generalisation of the neighbourhood diversity in static graphs since the temporal neighbourhood diversity partition of a temporal graph with lifetime one and the neighbourhood diversity partition of its footprint are the same. 

%


 \begin{definition}[Neighbourhood diversity graph]
 Let $G$ be a static graph with neighbourhood diversity partition $V_1,\dots,V_k$. The \emph{neighbourhood diversity graph} of $G$, denoted $\ndg[G]$ is the graph obtained by merging each class $V_i$ into a single vertex. Formally, we have $V(\ndg[G]) = \{V_1,\dots,V_k\}$ and $E(\ndg[G]) = \{V_iV_j \mid \forall v_i \in V_i, \forall v_j\in V_j, v_iv_j \in E(G)\}$.
\end{definition}


%
For clarity, we use the term class when referring to a vertex of \ndg[G], in order to distinguish between the vertices of $G$ and the vertices of \ndg[G].
%
% The definition of temporal neighbourhood diversity in temporal graphs is a generalisation of the neighbourhood diversity which is a parameter defined in static graphs. The neighbourhood diversity of a static graph $G$ is equivalent to the temporal neighbourhood diversity of the temporal graph of lifetime one and containing $G$ as unique snapshot. Hence, in the following, given a static graph $G$, we also use the notation $\tndg[G]$ to refer to the temporal neighbourhood diversity graph of the temporal graph containing $G$ as unique snapshot.

%
Let $X$ be a set of vertices and $Y$ be a subset of classes. We say that $X$ and $Y$ are \emph{compatible} if for all classes $V_i$, $V_i$ belongs to $Y$ if and only if $V_i$ intersects $X$, that is, $\forall V_i, (V_i \cap X \neq \emptyset \Leftrightarrow V_i \in Y)$.
% we have $X \cap V_i \neq \emptyset \Leftrightarrow V_i \in Y$
Notice that there is exactly one subset of classes that is compatible with a set of vertices $X$ whereas several subsets of vertices of $G$ can be compatible with a set of classes $Y$.
%

We now introduce the concept of a neighbourhood diversity locally decidable problem -- we do this first in the static setting in order to build into the temporal setting. 
Intuitively, these are problems for which, given a set of classes $Y$, we can determine the minimum and maximum number of vertices to select in each class that are realised by at least one solution compatible with $Y$, if such a solution exists. The formal definition is as follows:

\begin{definition}[Neighbourhood diversity locally decidable]
  \label{def:ndld}
  A static graph problem $\Pi$ is $f(n)$-\emph{neighbourhood diversity locally decidable} if
  for any static graph $G$ with $n$ vertices and every subset of classes $Y$ of $\ndg[G]$, the following two conditions hold:
  \begin{enumerate}[(a)]
    \item there is a computable function $check_\Pi(Y)$ with time complexity $\mathcal{O}(f(n))$ that determines if there is a solution for $\Pi$ in $G$ that is compatible with $Y$,
    \item if there is such a solution, then
      there exist two computable functions \\$low_\pi : \mathcal{P}(V(\ndg[G])) \times V(\ndg[G]) \to \mathbb{N}$ and $up_\pi : \mathcal{P}(V(\ndg[G])) \times  V(\ndg[G]) \to \mathbb{N}$ with time complexity $\mathcal{O}(f(n))$ such that, for all subsets $X \subseteq V(G)$, we have
\begin{gather*}
  \forall V_i \in V(\ndg[G]), low_\pi(Y,V_i) \leq |V_i \cap X| \leq up_\pi(Y,V_i)\\
  \Leftrightarrow \\
  \text{$X$ is solution for $\Pi$ and is compatible with $Y$}. 
\end{gather*}
  \end{enumerate}
\end{definition}

  In other words, $low_\pi$ and $up_\pi$ are necessary and sufficient lower and upper bounds for the number of selected vertices in each class in a solution to our problem in the graph of low neighbourhood diversity.

Notice that it is easy to compute a solution of minimum (respectively maximum) size that is compatible with a subset of classes $Y$ (if such a solution exists), by arbitrarily selecting exactly $low_\pi(V_i)$ (resp. $up_\pi(V_i)$) vertices inside each class.
%
Hence, $\Pi$ is solvable in $\mathcal{O}(2^k \cdot f(n))$, where $k$ is the neighbourhood diversity of the graph. Indeed, it suffices to enumerate every subset of classes and keep the best solution. It follows that if $check_\pi, low_\pi$ and $up_\pi$ are polynomial-time functions, $\Pi$ is in FPT when parameterised by neighbourhood diversity.

%\subsubsection{Example problems that are neighbourhood diversity locally decidable}

We show that {\sc Dominating Set} and {\sc Independent Set} are $f(n)$-neighbourhood diversity locally decidable.

\begin{lemmarep}
  \label{lemma:ds is ndld} {\sc Dominating set} is $f(n)$-neighbourhood diversity locally decidable where $f(n) \in O(n)$.
\end{lemmarep}

\begin{proof}
  Let $G$ be a static graph and let $Y \subseteq V(\ndg[G])$ be a subset of classes. We show that the following two conditions hold.
  \begin{enumerate}[(a)]
    \item there is a dominating set compatible with $Y$ if and only if $Y$ is a dominating set in $\ndg[G]$, and
    \item if $Y$ is a dominating set in $\ndg[G]$, then for each $V_i \in \ndg[G]$ the lower and upper bounds functions $low : V(\ndg[G])  \to \mathbb{N}$ and $up : V(\ndg[G]) \to \mathbb{N}$ are given by
    \begin{enumerate}[(i)]
      \item $low(V_i) = up(V_i) =0$, if $V_i\not\in Y$,
      \item $low(V_i) = 1$ and $up(V_i) = |V_i|$, if $V_i\in Y$ and $V_i$ is a clique-class or has a neighbour in $Y$ and,
      \item $low(V_i) = up(V_i) =|V_i|$, otherwise.
    \end{enumerate}
  \end{enumerate}
  
  First, we show that there is a dominating set in $G$ compatible with $Y$ if and only if $Y$ is a dominating set in $\ndg[G]$.
  Let $Y$ be a dominating set in $\ndg[G]$, we show that the subset of vertices $X = \bigcup\limits_{V_i \in Y} V_i$ is a dominating set of $G$. For any vertex $v \not \in X$ of $G$, the class $V_i \in N(\ndg[G])$ to which $v$ belongs has a neighbour $V_j$ in $\ndg[G]$ that belongs to $Y$ since otherwise $Y$ would not be a dominating set of $\ndg[G]$. Then, any vertex of $V_j$ belongs to $X$ and is a neighbour of $v$ in $G$ and so, $v$ is dominated. Hence, $X$ is a dominating set of $G$.
  %
  
  Now, let $X$ be a dominating set of $G$, we show that the subset of classes $Y$ compatible with $X$ is a dominating set of $\ndg[G]$. For any class $V_i \not \in Y$ and any vertex $v \in V_i$, $v$ is dominated by a vertex $u$ that belongs to a class $V_j \neq V_i$. Then, $V_j \in Y$ and $V_j$ is adjacent to $V_i$ in $\ndg[G]$ and so, $V_i$ is dominated. Hence, any class that does not belong to $Y$ has a neighbour in $Y$ and so, $Y$ is a dominating set in $\ndg[G]$.

  
Further, suppose that $Y$ is a dominating set in $\ndg[G]$, % we show that the depicted bounds are necessary and sufficient.
and let $X$ be a dominating set of $G$ that is compatible with $Y$, we show that for all classes $V_i$ of $G$, we have $low_\pi(V_i) \leq |X \cap V_i| \leq up_\pi(V_i)$.
  Let $V_i$ be a class of $G$.
  %
  If $V_i\not\in Y$, then we have $|X \cap V_i| = 0$ since otherwise $X$ would not be compatible with $Y$. 
  %
  If $V_i\in Y$, since $X$ is compatible with $Y$, $X$ should contain at least one vertex in $V_i$. Moreover, since $X$ cannot contain more that $|V_i|$ vertices in $V_i$, we have $1 \leq |V_i \cap X| \leq |V_i|$.
  %
  If $V_i$ is not a clique-class or has no neighbour in $Y$, then any vertex $v \in V_i$ is isolated in the subgraph induced by $X \cap \bigcup_{V_j \in Y} V_j$. Hence, $X$ contains necessarily every vertex of $V_i$ and so, $|X \cap V_i| = |V_i|$.

  
  Finally, we show that if $Y$ is a dominating set in $\ndg[G]$, then any set of vertices $X$ such that for all classes $V_i$, we have $low_\pi(V_i) \leq |X \cap V_i| \leq up_\pi(V_i)$,  is a dominating set in $G$ compatible with $Y$. First, since for any class $V_i$ we have $|V_i \cap X| = 0$ if $V_i \not\in Y$ and $1 \leq |V_i \cap X|$ otherwise, $X$ is compatible with $Y$. It remains to show that $X$ is also a dominating set.
  Let $v \not\in X$ be a vertex of $G$ that belongs to a class $V_i$. If $V_i \not\in Y$, then since $Y$ is a dominating set of $\ndg[G]$,  there is a class $V_j \in Y$ that is adjacent to $V_i$ in $\ndg[G]$. Thus, $V_j$ contains a vertex $u \in X$ and $u$ is adjacent to $v$ in $G$ and dominates $v$.
  If $V_i \in Y$ and is a clique-class, then there is a vertex $u \in V_i \cap X$ and $u$ is adjacent to $v$ in $G$. If $V_i$ is an independent-class, then since $v$ does not belong to $X$, $v$, there is necessarily another class $V_j \in Y$ that is adjacent to $V_i$ in $\ndg[G]$, since otherwise, every vertex of $V_i$ would belong to $X$. Thus, $V_j$ contains a vertex $u \in X$ and $u$ is adjacent to $v$ in $G$ and dominates $v$. Hence, in any case, $v$ is dominated by another vertex and thus, $X$ is a dominating set of $G$ that is compatible with $Y$.
\end{proof}

\begin{lemmarep}
  \label{lemma:is is ndld} {\sc Independent Set} is $f(n)$-neighbourhood diversity locally decidable where $f(n) \in O(n)$. 
\end{lemmarep}

\begin{proof}
    Let $G$ be a static graph and let $Y \subseteq V(\ndg[G])$ be a subset of classes. We show the following two conditions.
  \begin{enumerate}[(a)]
    \item there is an independent set compatible with $Y$ if and only if $Y$ is an independent set in $\ndg[G]$, and
    \item if $Y$ is a independent set in $\ndg[G]$, then the lower and upper bounds functions $low : V(\ndg[G])  \to \mathbb{N}$ and $up : V(\ndg[G]) \to \mathbb{N}$ are given by
    \begin{enumerate}[(i)]
      \item $low(V_i) = up(V_i) =0$, if $V_i\not\in Y$,
      \item $low(V_i) = up(V_i) = 1$, if $V_i\in Y$ and $V_i$ is a clique-class and,
      \item $low(V_i)=1$ and $up(V_i) =|V_i|$, otherwise.
    \end{enumerate}

  \end{enumerate}
  
  First, we show that there is an independent set in $G$ compatible with a subset of classes $Y$ if and only if $Y$ is an independent set in $\ndg[G]$.
  Let $Y$ be an independent set in $\ndg[G]$, let $X \subseteq V(G)$ be any subset of vertices constructed by arbitrarily choosing one vertex in each $V_i \in Y$.
  We show that $X$ is an independent set of $G$. Let $u$ be a vertex of $X$ that belongs to some class $V_i$. Per construction of $X$, we have $V_i \in Y$. Let $v$ be a neighbour of $u$. If $v \in V_i$, then $v$ does not belong to $X$ since $X$ contains at most one vertex per class in $Y$. If $v$ belongs to another class $V_j$, then $V_i$ and $V_j$ are adjacent in \ndg[G] and since $Y$ is an independent set, $V_j \not \in Y$ and so, by construction of $X$, $v \not\in X$. So, $X$ is an independent set.
  %
  Now, let $X$ be an independent set of $G$, we show that the subset of classes $Y$ compatible with $X$ is an independent set of $\ndg[G]$. Suppose there exist two adjacent classes $V_i$ and $V_j$ that belong to $Y$. Then, $V_i$ contains a vertex $u \in X$ and $V_j$ contains a vertex $v \in V_j$. Since $u$ and $v$ are adjacent, it contradicts $X$ being an independent set. Hence, $Y$ does not contain two adjacent classes and so, $Y$ is an independent set in \ndg[G].
  %
  
Further, suppose that $Y$ is an independent set in $\ndg[G]$, % we show that the depicted bounds are necessary and sufficient.
and let $X$ be an independent set of $G$ that is compatible with $Y$, we show that for all classes $V_i$ of $G$, we have $low_\pi(V_i) \leq |X \cap V_i| \leq up_\pi(V_i)$.
  Let $V_i$ be a class of $G$.
  %
  If $V_i\not\in Y$, then we have $|X \cap V_i| = 0$ since otherwise $X$ would not be compatible with $Y$. 
  %
  If $V_i\in Y$, since $X$ is compatible with $Y$, $X$ should contain at least one vertex in $V_i$. Moreover, since $X$ cannot contain more that $|V_i|$ vertices in $V_i$, we have $1 \leq |V_i \cap X| \leq |V_i|$.
  %
  If $V_i$ is a clique-class, then $X$ cannot contain more two vertices in $V_i$ since they would be adjacent.
  Hence, $|V_i \cap X|=1$ .
  %
  
  Finally, we show that if $Y$ is an independent set in $\ndg[G]$, then any set of vertices $X$ such that for all classes $V_i$, we have $low_\pi(V_i) \leq |X \cap V_i| \leq up_\pi(V_i)$, is an independent set in $G$ compatible with $Y$. First, since for any class $V_i$ we have $|V_i \cap X| = 0$ if $V_i \not\in Y$ and $1 \leq |V_i \cap X|$ otherwise, $X$ is compatible with $Y$. It remains to show that $X$ is also an independent set.
  Suppose there exist two adjacent vertices $u$ and $v$ that both belong to $X$. The two vertices cannot belong to different classes $V_i$ and $V_j$ since otherwise both $V_i$ and $V_j$ would belong to $Y$, contradicting $Y$ being an independent set in \ndg[G]. Thus, $u$ and $v$ belong to the same class $V_i$. Since, $u$ and $v$ are adjacent, $V_i$ is not an independent class. But then, we have $|X \cap V_i| > up(V_i)$ which is a contradiction. Hence, $X$ does not contain two adjacent vertices and therefore $X$ is an independent set. 
\end{proof}


\subsubsection{In temporal graphs}
We now extend the concept of neighbourhood diversity locally decidable problems to a temporal setting. We use temporal neighbourhood diversity introduced by Enright et al.~\cite{abs-2404-19453}.


\begin{definition}[Temporal Neighbourhood Diversity~\cite{abs-2404-19453}]
The \emph{temporal neighbourhood diversity} of a temporal graph $\mathcal{G}=(G,\lambda)$ is the minimum value $k$ such that the vertices of $G$ can be partitioned into $k$ classes $V_1,\dots,V_k$ such that 
%
  for each pair of vertices $u$ and $v$ in a same class $V_i$, we have $N_t(u) \setminus \{v\} = N_t(v) \setminus \{u\}$ at each time-step $t \in [1,\tau]$.  We call $V_1,\dots,V_k$ a \emph{temporal neigbourhood diversity partition} of $\mathcal{G}$.
\end{definition}

\noindent In the same way as for the static parameter, we can use a graph to represent the partition.

\begin{definition}[Temporal neighbourhood diversity graph]
 Let $\mathcal{G}=(G,\lambda)$ be a temporal graph with temporal neighbourhood partition $V_1,\dots,V_k$. The \emph{temporal neighbourhood diversity graph} of $\mathcal{G}$, denoted $\tndg$ is the temporal graph obtained by merging every class $V_i$ into a single vertex. Formally, we have $V(\tndg) = \{V_1,\dots,V_k\}$, $E(\tndg) = \{V_iV_j \mid \forall v_i \in V_i, \forall v_j\in V_j, v_iv_j \in E(G)\}$ and $\lambda(V_iV_j) = \lambda(u_iu_j)$ for any vertices $u_i \in V_i$ and $u_j \in V_j$.
\end{definition}

\begin{figure}
  \centering
  \scalebox{0.4}{
    \begin{tikzpicture}
      \newcounter{vertex}
      \setcounter{vertex}{1}

      \newcounter{tmp}
      \setcounter{tmp}{1}
      
    \foreach[count=\l from 1] \x/\T in {3/{C,I,C},4/{I,I,C},6/{I,I,I},6/{C,I,I},5/{C,C,I},2/{I,I,C}}{
      
      \setcounter{tmp}{\thevertex}

      \foreach[count=\X from 1] \i in \T {
        \setcounter{vertex}{\thetmp}
        \node[draw,circle,minimum width=2.1cm,label=\l*60+30:{\LARGE $V_\l$}] (\l\X) at ($(\X*10,0)+(\l*60:3)$) {};
        \pgfmathsetmacro{\a}{360/\x}% 
        \foreach \v in {1,...,\x}{
          \node[smallvertex,label=\v*\a:{\letter{\thevertex}}] (\v) at ($(\X*10,0)+(\l*60:3)+(\v*\a:0.5)$) {};
          \stepcounter{vertex}
        }
        \ifthenelse{\equal{\i}{C}}{
          \foreach \a in {1,...,\x}{
            \foreach \b in {1,...,\a} {
              \draw (\a)--(\b);
            }
          }
        }{}
      }
    }
    \foreach \a/\b in {21/31,41/51,41/11,41/61,51/61,11/61,11/51,22/12,62/12,22/62,62/52,22/42,42/52,23/33,53/63,53/13,53/43}
    \draw (\a)--(\b);
    \foreach \L in {1,...,3}
    \node at (10*\L,-5) {\LARGE $G_\L$};
  \end{tikzpicture}
  }
  \caption{\label{fig:tmp neighbourhood diversity} Example of temporal neighbourhood diversity graph \tndg of a temporal graph $\mathcal{G}=(G,\lambda)$. An edge between two sets $V_i$ and $V_j$ indicates that each vertex of $V_i$ is adjacent to all vertices of $V_j$.} 
\end{figure}
 
An example of a temporal neighbourhood diversity graph is depicted in \Cref{fig:tmp neighbourhood diversity}.
%
Let $\Pi$ be a reconfiguration problem. We call $\Pi$ \emph{$f(n)$-temporal neighbourhood diversity locally decidable} if the static version of $\Pi$ is $f(n)$-neighbourhood diversity locally decidable. In both of the problems we consider $f(n)=n$, and we simply write ``temporal neighbourhood diversity locally decidable'' for convenience. For a temporal graph $\mathcal{G}=(G,\lambda)$, we denote by $low^t_\pi$ and $up^t_\pi$ the functions giving the lower and upper bounds on the number of selected vertices in a class $V_i$ needed to obtain solution for the static version of $\Pi$ in $G_t$, as defined in \Cref{def:ndld}.


\subsection{FPT algorithm with respect to temporal neighbourhood diversity and lifetime}
\label{subsec:algo}

We now formulate an FPT algorithm using temporal neighbourhood diversity and lifetime to solve a neighbourhood diversity locally decidable reconfiguration problem $\Pi$ in a temporal graph $\mathcal{G}=(G,\lambda)$. 
A \emph{candidate sequence} is a sequence $Y = (Y_1,\dots,Y_\tau)$ such that for each $t \in [1,\tau]$, $Y_t$ is a subset of classes of $\tndg$. 
We say that $Y$ is \emph{valid} if $check_\Pi(Y_t)= \texttt{true}$ for all $t \in [\tau]$. Let $T=(T_1,\dots,T_\tau)$ be a reconfigurable sequence for $\mathcal{G}$. For the sake of simplicity, we say that the two sequences $Y$ and $T$ are compatible if for each $t \in [\tau], T_t$ and $Y_t$ are compatible. 

The principle of the algorithm is to iterate through each candidate sequence $Y$ and, if $Y$ is valid, compute an optimal reconfigurable sequence $T$ compatible with $Y$. Then, we return the optimal solution among all solutions associated to valid candidate sequences.
Notice that there are at most $2^{tnd \cdot\tau}$ candidate sequences to consider. Hence, we can expect the algorithm to be efficient if the temporal graph has small temporal neighbourhood diversity and short lifetime.
% This algorithm is depicted in \Cref{alg:fpt nd}.
If $\Pi$ is $f(n)$-neighbourhood diversity locally decidable, we show in \Cref{subsec:reconfiguration} how to compute an optimal reconfigurable sequence compatible with a candidate sequence, if one exists, by solving an instance of the circulation problem. Combining the observation that there are at most $2^{tnd \cdot\tau}$ candidate sequences to consider and the result from Subsection~\ref{subsec:reconfiguration} gives us the following result, which we restate at the end of that subsection.

% \begin{algorithm}
%   \SetAlgoLined
%   \KwData{A temporal graph $\mathcal{G}$ and a reconfiguration problem $\Pi$}
%   \KwResult{The score of the optimal solution for $\Pi$ in $\mathcal{G}$}
%   \ForEach{candidate sequence $Y = (Y_1,\dots,T_\tau)$}{
%     \If{$check(Y)$}{
%       \tcc{compute an optimal sequence compatible with $Y$}
%       $T \gets compute\_sequence(\mathcal{G},Y)$\;
%       \lIf{$\Pi$ is a minimisation problem}{$score \gets \min(score,|T|)$}
%       \lElse{$score \gets \max(score,|T|)$}
%     }
%   }
%     \Return score\;
%   \caption{}
%   \label{alg:fpt nd}
% \end{algorithm}

\begin{lemma*}
    Let $\mathcal{G}=(G,\lambda)$ be a temporal graph and let $\Pi$ be an $f(n)$-temporal neighbourhood diversity locally decidable reconfiguration problem. 
 $\Pi$ is solvable in $\mathcal{O}(2^{(tnd\cdot \tau)} \cdot (\tau\cdot tnd \cdot f(n) + \tau^{\nicefrac{11}{2}} \cdot tnd^8))$ where $tnd$ is the temporal neighbourhood diversity of $\mathcal{G}=(G,\lambda)$.
\end{lemma*}








%%% Local Variables:
%%% mode: LaTeX
%%% TeX-master: "main"
%%% End:


\subsection{Computing an optimal reconfigurable sequence compatible with a candidate sequence}
\label{subsec:reconfiguration}

In this subsection, we present an efficient method for computing an optimal reconfigurable sequence $T$ compatible with a particular valid sequence $Y$. 
 Note that,  we cannot compute $T$ by exhaustively generating all possible sequences in $\mathcal{G}$ and selecting the best one compatible with $Y$. This approach would have time complexity of $\mathcal{O}(2^{\tau\cdot|V(G)|})$ for each candidate sequence making it impractical.
%
 % Given a valid candidate sequence $Y$, we describe how to compute an optimal reconfigurable sequence $T = (T_1,\dots,T_\tau)$ for $\Pi$ compatible with $Y$ in $\mathcal{O}(\tau\cdot tnd \cdot f(n) + \tau^{\nicefrac{11}{2}} \cdot tnd^8)$ time. % It corresponds to the function compute\_sequence of \Cref{alg:fpt nd}.

%

As a tool to build our algorithm we construct and solve an instance of the circulation problem: this is a variation of the network flow problem for which there are upper and lower bounds on the capacities of the arcs. Formally, a \emph{circulation graph} $\overrightarrow{F}=(F,c,l,u)$ consists of a directed graph $F$, along with a cost function $c : A(F) \to \mathbb{N}$ on the arcs and two functions on the arcs $l : A(F) \to \mathbb{N}$ and $u : A(F) \to \mathbb{N}$, representing the lower and upper bounds of the flow capacities. The digraph $F$ also has two designated vertices $sou$ and $tar$ representing the source and the target vertices of the flow, respectively.

In the circulation problem, we want to find a \emph{circulation flow} between $sou$ and $tar$, which is a function $g : A(F) \to \mathbb{N}$ that respects two constraints:
\begin{enumerate}[(a)]
  \item \emph{Flow preservation}: the flow entering in any vertex $v \in V(F) \setminus \{sou,tar\}$ is equal to the flow leaving it, that is, $\sum\limits_{u \in N^+(v)} g(u,v) = \sum\limits_{u \in N^-(v)} g(v,u)$.
  \item \emph{Capacity:} for all arcs $(u,v) \in A(F)$, the amount of flow over $(u,v)$ must respect the capacity of $(u,v)$, that is, $l(u,v) \leq g(u,v) \leq u(u,v)$.
\end{enumerate}
%
Finally, the circulation problem is defined as follows.
\prob{Circulation problem}{
A circulation graph $\overrightarrow{F}$, a source vertex $sou$ and a target vertex $tar$
}{
  Find a circulation flow $g$ that minimises  $\sum\limits_{a \in A(F)}g(a)\cdot c(a)$.
}
The circulation problem is known to be solvable in $\mathcal{O}(|A(F)|^{\nicefrac{5}{2}}\cdot |V(F)|^3)$~\cite{Tardos85}. Notice that the circulation problem can easily be transformed into a maximisation problem by modifying all positive costs into negative costs.

We now show that the problem of finding an optimal reconfigurable sequence compatible with $Y$ is reducible to the circulation problem.

% \begin{construction}
%   \TD{constructive version\\}
%   \label{const:flow}
%   Let $\mathcal{G}=(G,\lambda)$ be a temporal graph, let $Y=(Y_1,\dots,Y_\tau)$ be a valid candidate sequence.
%   We construct the following circulation graph $\overrightarrow{F}=(F,c,l,u)$.
%   \begin{itemize}
%     \item For each $Y_t$ and each set $V_i \in Y_t$, introduce two vertices $x^t_i, y^t_i$ along with the arc $(x^t_i,y^t_i)$ and set $c(x^t_i,y^t_i)=0$, $l(x^t_i,y^t_i)=low_\pi(Y_t,V_i)$ and $u(x^t_i,y^t_i)=up_\pi(Y_t,V_i)$.
%     \item Introduce a source vertex $sou$ and for each set $V_i \in Y_1$, introduce the arc $(sou,x^1_i)$ and set $c(sou,x^1_i)=1, l(sou,x^1_i)=0$ and $u(sou,x^1_i)=+\infty$.
%     \item Introduce a target vertex $tar$ and for each set $V_i \in Y_\tau$, introduce the arc $(x^\tau_i,tar)$ and set $c(x^\tau_i,tar)=l(x^\tau_i,tar)=0$ and $u(x^\tau_i,tar)=+\infty$.
%     \item For each $t\in[1,\tau-1]$, and each pair $V_i \in Y_t, V_j \in Y_{t+1}$ such that either $V_i=V_j$ or $V_i$ and $V_j$ are adjacent in $\tndg$ at time step $t$, introduce an arc $(y^t_i,x^{t+1}_j)$ and set $c(y^t_i,x^{t+1}_j)=0, l(y^t_i,x^{t+1}_j)=0$ and $u(y^t_i,x^{t+1}_j)=+\infty$.  
%   \end{itemize}
% \end{construction}


\begin{construction}
  \label{const:flow}
  Let $\mathcal{G}=(G,\lambda)$ be a temporal graph, let $Y=(Y_1,\dots,Y_\tau)$ be a valid candidate sequence.  
  The reconfiguring circulation graph $\overrightarrow{RF}(\mathcal{G},Y)=(F,c,l,u)$ is the circulation graph with vertex set $V(F) = \{sou,tar\} \cup \{x_i^t \mid V_i \in V(\tndg), t \in [1,\tau]\}$ and arc set $E(F) = S \cup \{C_t \mid t \in [1,\tau]\} \cup \{R_t \mid t \in [1,\tau-1]\}  \cup T$ where the arc sets $S,C_t ,R_t$ and $T$ are defined as follows: 
  \begin{itemize}
    \item $S= \{ (sou,x_i^1) \mid V_i \in Y_1\}$ with $c(a)=1$ and $l(a)=0$ and $u(a)=+\infty$ for each a $\in S$,
    \item $C_t= \{(x^t_i,y^t_i) \mid V_i \in Y_t\}$ with $c(x^t_i,y^t_i) = 0$, $l(x^t_i,y^t_i) = low^t_\pi(V_i)$ and  $u(x^t_i,y^t_i) = up^t_\pi(V_i)$ for each $(x^t_i,y^t_i)\in C_t$,
    \item $R_t= \{(y^t_i,v^{t+1}_i) \mid V_i \in Y_t \cap Y_{t+1}\} \cup \{(y^t_i,v^{t+1}_j) \mid V_i \in Y_t,V_j \in Y_{t+1}, V_i$ is adjacent to $V_j$ in $\tndg$ at time $t\}$ with $c(a)=l(a)= 0$ and $u(a)=+\infty$ for each $a\in R_t$,
    \item $T = \{(y_i^\tau,tar) \mid V_i \in V(\tndg)\}$ with $c(a)=l(a)=0$ and $u(a)=+\infty$ for each $a\in T$.
  \end{itemize}
  \end{construction}


An example construction can be found in Figure~\ref{fig:flow}. The idea of the reduction is to represent each token by a unit value of the flow.
  At each time-step $t$, we can decompose the reconfiguring circulation graph into two parts: the checking part $C_t$ and the reconfiguring part $R_t$ (which only exists if $t \neq \tau$). In the checking part, each class $V_i$ is represented by an arc $(x^t_i,y^t_i) \in C_t$ and the value of the flow passing by this arc corresponds to the number of tokens inside $V_i$ at time-step $t$. Since the lower and upper bounds of the arc are the same as those described in \Cref{def:ndld}, we ensure that at each step, the set of selected vertices forms a solution. In the 
reconfiguring part, an arc $(y^t_i,x^t_j) \in R_t$ indicates that the tokens contained in the class $V_i$ can move to the class $V_j$ during time-step $t$. An arc $(y^t_i,x^t_i) \in R_t$ indicates that the tokens in the class $V_i$ can stay in the same vertex during time-step $t$.

% \begin{definition}
%   \label{def:flow}
%   Let $\mathcal{G}=(G,\lambda)$ be a temporal graph and let $\mathcal{Y}=(Y_1,\dots,Y_\tau)$ be a candidate sequence in $\mathcal{G}$ for some reconfiguration problem $\Pi$. The \emph{reconfiguration flow graph} of $\mathcal{G}$ and $\mathcal{Y}$, denoted $\overrightarrow{F}_\Pi(\mathcal{G},\mathcal{Y})=(F,c,l,u)$ is a circulation graph with vertex set $V(F)= \{sou,tar\} \cup \{x^t_i,y^t_i \mid i \in [1,\tau], V^t_i \in Y_t\} \cup \{w^t_i,z^t_i \mid i \in [1,\tau], \}$ 
%   We construct the following circulation graph $\overrightarrow{F}=(F,c,l,u)$. In the following, for each arc $e$, if $c(e),l(e)$ or $u(e)$ are not specified, we set $c(e)=0$, $l(e)=0$ and $u(e) +\infty$.
%   %
%   \begin{itemize}
%     \item For each time-step $t \in [1,\tau]$ each class $V^t_i \in Y^t$, introduce two vertices $x^t_i$ and $y^t_i$ along with the arc $(x^t_i,y^t_i)$ and set $l(x^t_i,y^t_i) = low_{\Pi}(V^t_i)$, $u(x^t_i,y^t_i) = up_{\Pi}(V^t_i)$. 
%     \item For each time-step $t \in [1,\tau-1]$ and each class $V^t_i$ of $G_t$ such that there exists $V^{t+1}_j \in Y^{t+1}$ with $V^t_i \cap V^{t+1}_j \neq \emptyset$, introduce two vertices $w^t_i$ and $z^t_i$ along with the arc $(w^t_i,z^t_i)$ and set $up(w^t_i,z^t_i)=|V^t_i|$. Then introduce the following arcs.
%     \begin{itemize}
%       \item For each $V^t_j \in Y^t$ such that $V^t_j=V^t_i$ or $V^t_j$ is adjacent to $V^t_i$ in $NDG(G_t)$, introduce the arc $(y^t_i,w^t_i)$.
%       \item For each $V^{t+1}_j \in Y^{t+1}$ such that $V^t_i \cap V^{t+1}_j \neq \emptyset$, introduce the arc $(z^t_i,x^{t+1}_j)$ and set $u(z^t_i,x^{t+1}_j) = |V^t_i \cap V^{t+1}_j|$.
%     \end{itemize}
%     \item   Finally, introduce one source vertex $sou$ and one target vertex $tar$. For each class $V^1_i \in Y^t$, introduce the arc $(sou,x^1_i)$ and set $c(sou,x^1_i) = 1$. For each class $V^\tau_i \in Y^\tau$, introduce the arc $(z^\tau_i,tar)$. 
%   \end{itemize}
% \end{definition}



% \begin{construction}
%   \label{const:flow}
%   Let $\mathcal{G}=(G,\lambda)$ be a temporal graph, let $\mathcal{H}=(H,\lambda)$ be a temporal neigbourhood diversity graph of $\mathcal{G}$ and let $S=(S_1,\dots,S_\tau)$ be a sliding dominating set of $\mathcal{H}$.
%   We construct the following circulation graph $\overrightarrow{F}=(F,c,l,u)$.
%   \begin{itemize}
%     \item For each $S_t$ and each set $V_i \in S_t$, introduce two vertices $x^t_i, y^t_i$ along with the arc $(x^t_i,y^t_i)$ and set $c(x^t_i,y^t_i)=0$, $l(x^t_i,y^t_i)=min\_tok(V_i,H_t,S_t)$ and $u(x^t_i,y^t_i)=|V_i|$.
%     \item Introduce a source vertex $sou$ and for each set $V_i \in S_1$, introduce the arc $(s,x^1_i)$ and set $c(sou,x^1_i)=1, l(sou,x^1_i)=0$ and $u(sou,x^1_i)=+\infty$.
%     \item Introduce a target vertex $tar$ and for each set $V_i \in S_\tau$, introduce the arc $(x^\tau_i,tar)$ and set $c(x^\tau_i,tar)=l(x^\tau_i,tar)=0$ and $u(x^\tau_i,tar)=+\infty$.
%     \item For each $t<\tau$, and each pair $V_i \in S_t, V_j \in S_{t+1}$ such that either $V_i=V_j$ or $V_i$ and $V_j$ are adjacent in $H_t$, introduce an arc $(y^t_i,x^{t+1}_j)$ and set $c(y^t_i,x^{t+1}_j)=0, l(y^t_i,x^{t+1}_j)=0$ and $u(y^t_i,x^{t+1}_j)=+\infty$.  
%   \end{itemize}
% \end{construction}





% \begin{figure}
%   \centering
%   \scalebox{0.55}{
%   \begin{tikzpicture}
%     \foreach[count=\l from 1] \x/\T in {3/{C,I,C},4/{I,I,C},6/{I,I,I},6/{C,I,I},5/{C,C,I},3/{I,I,C}}{
%       \foreach[count=\X from 1] \i in \T {
%         \node[draw,circle,minimum width=2cm,label=\l*60+30:{\LARGE $V_\l$}] (\l\X) at ($(\X*10,0)+(\l*60:3)$) {};
%         \pgfmathsetmacro{\a}{360/\x}% 
%         \foreach \v in {1,...,\x}{
%           \node[smallvertex] (\v) at ($(\X*10,0)+(\l*60:3)+(\v*\a:0.5)$) {};
%         }
%         \ifthenelse{\equal{\i}{C}}{
%           \foreach \a in {1,...,\x}{
%             \foreach \b in {1,...,\a} {
%               \draw (\a)--(\b);
%             }
%           }
%         }{}
%       }
%     }
%     \foreach \a/\b in {21/31,41/51,41/11,41/61,51/61,11/61,11/51,22/12,62/12,22/62,62/52,22/42,42/52,23/33,53/63,53/13,53/43}
%     \draw (\a)--(\b);
%     \foreach \L in {1,...,3}
%     \node at (10*\L,-5) {\LARGE $G_\L$};
%   \end{tikzpicture}
%   }
%   \caption{\label{fig:neighbourhood diversity} Example of temporal neigbourhood diversity graph. An edge between two sets $V_i$ and $V_j$ indicates that each vertex of $V_i$ is adjacent to all vertices of $V_j$. } 
% \end{figure}
 

\begin{figure}[ht]
  \centering
  \scalebox{0.75}{
    \begin{tikzpicture}[
tap/.style args = {#1}{decoration={raise=5,
                                      text along path,
                                      text align={align=center},
                                      text={#1}
                                      },
              postaction={decorate},
              },
                   ]
   
     \foreach \x/\y/\l/\c/\u/\t in {0/0/1/1/3/1,0/-2/2/1/4/1,0/-4/3/1/6/1,
       4/1/6/1/3/2,4/-2/2/1/4/2,4/-4/3/6/6/2,
       8/-2/2/1/4/3,8/-4/3/1/6/3,8/1/6/1/3/3,8/-1/5/1/5/3}
     {
       \draw[->,>=stealth] (\x,\y) node[smallvertex,label=90:{$x_\l^\t$}] (x\l\t) {} ->
       node[midway,above] {0/\c/\u}
       ++(1.9,0) node[smallvertex,label=90:{$y_\l^\t$}] (y\l\t) at ++(0.1,0) {}
       ;
     }
     \foreach \a/\b in {21/22,22/23,31/32,32/33,21/32,31/22,11/62,22/63,22/53,62/53}
     \draw[->] (y\a)->(x\b);

     \node[smallvertex,label=90:{$sou$}] (s) at (-3,-2) {};
     \foreach \a in {11,21,31}
     \draw[->,tap={1/0/${+\infty}${}}] (s) -- node[midway,above] {} (x\a);

     \node[smallvertex,label=90:{$tar$}] (t) at (13,-2) {};
     \foreach \a in {63,53,23,33}
     \draw[->] (y\a)--(t);
   \end{tikzpicture}
   }
   \caption{\label{fig:flow} Example of flow graph produced by \Cref{const:flow}. The input temporal graph is the one depicted in \Cref{fig:tmp neighbourhood diversity}. We want to compute a minimum dominating set compatible with the candidate sequence $Y = (Y_1 = \{V_1,V_2,V_3\}, Y_2= \{V_2,V_3,V_6\},Y_3 = \{V_2,V_3,V_5,V_6\})$.}
 \end{figure}


 
\begin{lemmarep}
  \label{lemma:flow reconfiguration} Let $\mathcal{G}$ be a temporal graph with temporal neighbourhood graph \tndg and let $Y=(Y_1,\dots,Y_\tau)$ be a valid candidate sequence for some $f(n)$-neighbourhood diversity locally decidable reconfiguration problem $\Pi$. Let $\overrightarrow{RF}(\mathcal{G},Y)=(F,c,l,u)$ be the reconfiguring circulation graph as described in \Cref{const:flow}.
  There is a reconfigurable sequence $T$ compatible with $Y$ of size $\ell$ if and only if there is a circulation flow $g$ in $\overrightarrow{RF}(\mathcal{G},Y)$ with cost $\ell$.
  % $\mathcal{Y}$ is valid if there is a feasible flow $f$ in $\overrightarrow{F}$ and in that case, the minimum size of a reconfiguration sequence associated to $\mathcal{Y}$ is given by the minimum cost value for 

  % There is a sliding dominating set $(S,M,tok)$ in $\mathcal{G}$ with $\ell$ tokens if and only if there is a circulation flow $f$ with cost $\ell$ in $\overrightarrow{F}$.
\end{lemmarep}


\begin{proof}
  For any vertex $v$ of $\overrightarrow{RF}(\mathcal{G},Y)$, we denote $g^+(v)=\sum\limits_{u \in N^+(v)} g(u,v)$ and $g^-(v)= \sum\limits_{u \in N^-(v)} g(v,u)$, the flow entering in $v$ and leaving $v$, respectively.

  Let $T = (T_1,\dots,T_\tau)$ be a reconfigurable sequence of size $\ell$. For each time-step $t \in [1,\tau-1]$, let $b_t$ be a reconfiguration bijection between $T_t$ and $T_{t+1}$.  We construct a circulation flow $g$ between $sou$ and $tar$ as follows.
  %
  Let \tndg$=(\{V_1,\ldots,V_k\}, \{V_iV_j | \forall v_i\in V_i, \forall v_j\in V_j, v_iv_j\in E(G)\})$. First, we set $g(sou,x^1_i):= |T_1 \cap V_i|$.
  %
  Then, for each time-step $t \in [1,\tau]$, we set $g(x^t_i,y^t_i) := |V_i \cap T_t|$ (i.e.\ the number of tokens contained in the class $V_i$ at time-step $t$).
  %
  For each time-step $t \in [1,\tau-1]$, we set
      % 
    $g(y^t_i,x^{t+1}_j) := | \{ (u,v) \in b_t \mid u \in T_t \cap V_i, v \in T_{t+1} \cap V_j\} |$ (i.e.\ the number of tokens moving from the class $V_i$ to another class $V_j$ during time-step $t$ if $i \neq j$ or the number of tokens moving inside the class $V_i$ or staying in the same vertex in $V_i$ during time-step $t$ if $i = j$).
    %
    % $f(y^t_i,w^t_i) := | \{ uv \in M_t \mid u \in T_t \cap V_i, v \in T_{t+1} \cap V_i\} \cup T_t \cap T_{t+1} \cap V_i|$ (\textit{i.e.} the number of tokens moving inside the class $V_i$ or staying in the same vertex in $V_i$ during time-step $t$)
    %
    Finally, we set $g(y^\tau_i,tar):= |V_i \cap T_\tau|$. 
      % 
    We now show that $g$ is a circulation flow of cost $\ell$. First, notice that the only non-zero cost arcs are those leaving $sou$. Hence, the cost of $g$ is $\sum\limits_{V_i\in Y_1} g(sou,x^1_i) = \sum\limits_{V_i\in Y_1} |V_i \cap T_1| = \ell$.
    Further, we show that the capacity constraints are respected. For each arc $(x^t_i,y^t_i)$, since $\Pi$ is $f(n)$-neighbourhood diversity locally decidable, we have $low^t_\pi(Y_t,V_i) = l(x^t_i,y^t_i) \leq g(x^t_i,y^t_i) \leq up^t_\pi(Y_t,V_i) = u(x^t_i,y^t_i)$. For any other arc $a$, we have $l(a)=0$ and $u(a)=+\infty$ and so, the capacity constraint is necessarily respected. Hence, the capacity constraints are respected for all arcs.
    It remains to show that $g$ respects the flow conservation constraint. 
    %
    For any vertex $x^t_i$, we have $g^-(x^t_i) = g(x^t_i,y^t_i) = |T_t \cap V_i|$. If $t=1$, then $g^+(x^1_i)=g(sou,x^1_i)= |T_1 \cap V^1_i|= g^+(x^t_i)$. Otherwise, by construction of $g$, $g^+(x^t_i)$ is equal to the sum of the number of tokens moving from another class $V_j$ to the class $V_i$ plus the number tokens staying inside the class $V_i$ during the time-step $t$, that is, the number of tokens inside $V_i$ at time-step $t$. So, $g^-(x^t_i) = |T_t \cap V_i| = g^+(x^t_i)$. 
    % 
    For any vertex $y^t_i$, we have $g^+(y^t_i) = g(x^t_i,y^t_i) = |T_t \cap V^t_i|$. If $t = \tau$, then $g^-(y^t_i) = g(y^t_i,tar) = |T_\tau \cap V^\tau_i| = g^+(y^t_i)$. Otherwise, by construction of $g$, $g^-(y^t_i)$ is equal to the number of tokens moving from the class $V_i$ to another class $V_j$ plus the number of tokens staying inside the class $V_i$ during the time-step $t$, that is, the number of tokens inside $V_i$ at time-step $t$. So, $g^+(y^t_i) = |T_t \cap V_i| = g^-(y^t_i)$.
    %

    We now show the reverse.
    Let $g$ be a circulation flow with cost $\ell$ in $\overrightarrow{RF}(\mathcal{G},Y)$.
    We construct a reconfigurable sequence $T = (T_1,\dots,T_\tau)$ compatible with $Y$ of size $\ell$ as follows. First, for each class $V_i \in Y_1$, we construct $T_1$, by selecting $g(x^1_i,y^1_i)$ vertices in the class $V_i$. Since $g$ respects the flow conservation constraint, we have $|T_1 \cap V_i| = g(x^1_i,y^1_i)$ for each class $V_i$. Moreover, since $low^1_\pi(Y_1,V_i) = l(x^1_i,y^1_i) \leq g(x^1_i,y^1_i) < u(x^1_i,y^1_i) = up^1_\pi(Y_1,V_i)$ and since $\Pi$ is $f(n)$-neighbourhood diversity locally decidable, $T_1$ is a solution for the static version of $\Pi$ in $G_1$.
    %
    Now suppose that at time-step $t<\tau-1$, we also have $|T_t \cap V_i| = g(x^t_i,y^t_i)$ for each class $V_i$. We construct $T_{t+1}$ by moving $g(y^t_i,x^{t+1}_j)$ tokens from the class $V_i$ to the class $V_j$ for each pair $i,j$ such that $i\neq j$ and $(y^t_i,x^{t+1}_j)$ is an arc in $\overrightarrow{RF}(\mathcal{G},Y)$. Observe that inside $V_i$, exactly $g(y^t_i,x^{t+1}_i)$ tokens remain on the same vertex. The number of tokens leaving $V_i$ or staying inside $V_i$ is equal to $g^-(y^t_i)$ and since $g$ respects the flow conservation constraint, we have $g^+(y^t_i)= |T_t \cap V_i| = g^-(y^t_i)$. Thus, we have exactly the correct number of tokens in $V_i$ to make the moves. Moreover, we have  $f^+(x^{t+1}_j)= f(x^{t+1}_j,y^{t+1}_j)$ and so, we have $|T_{t+1} \cap V_j| = f(x^{t+1}_j,y^{t+1}_j)$. Since $low^{t+1}_\pi(Y_{t+1},V_j) = l(x^{t+1}_j,y^{t+1}_j) \leq f(x^{t+1}_j,y^{t+1}_j) < u(x^{t+1}_j,y^{t+1}_j) = up^{t+1}_\pi(Y_{t+1},V_i)$ and since $\Pi$ is $f(n)$-neighbourhood diversity locally decidable, $T_{t+1}$ is a solution for the static version of $\Pi$ in $G_{t+1}$. Hence, inductively, we have constructed a reconfigurable sequence $T = (T_1,\dots,T_\tau)$ compatible with $Y$.
\end{proof}


We can now conclude that it is possible to compute an optimal reconfigurable sequence compatible with a candidate sequence in polynomial time.
 
\begin{lemmarep}
  \label{lemma:compute reconfigurable sequence} Let $\Pi$ be a $f(n)$-neighbourhood diversity locally decidable problem, let $\mathcal{G}=(G,\lambda)$ be a temporal graph with lifetime $\tau$ and temporal neighbourhood diversity $tnd$ and let $Y$ be a valid candidate sequence. It is possible to compute an optimal reconfigurable sequence $T$ compatible with $Y$ in
$\mathcal{O}(\tau\cdot tnd \cdot f(n) + \tau^{\nicefrac{11}{2}} \cdot tnd^8)$.
\end{lemmarep}

\begin{proof}
  We first construct the reconfiguring circulation graph $\overrightarrow{RF}(\mathcal{G},Y)=(F,c,l,u)$, as depicted in \Cref{const:flow}. We have $|V(\overrightarrow{RF})|=\mathcal{O}(\tau\cdot \sum\limits_{Y_i \in Y}|Y_i|) = \mathcal{O}(\tau \cdot tnd)$ and $|A(\overrightarrow{RF})|= \mathcal{O}(\tau \cdot \sum\limits_{Y_i \in Y}|Y_i| + \sum\limits_{Y_i \in Y \setminus Y_\tau}(|Y_i|\cdot |Y_{i+1}|)) = \mathcal{O}(\tau \cdot tnd^2)$. Moreover, we call the two functions $low^t_\pi$ and $up^t_\pi$ exactly once per arc $(x^t_i,y^t_i)$ \textit{i.e.}, $\mathcal{O}(\tau\cdot tnd)$ calls in total. Hence,  the construction of $\overrightarrow{RF}$ can be done in $\mathcal{O}(\tau\cdot tnd \cdot f(n) + |A(\overrightarrow{RF})| + |V(\overrightarrow{RF})|)$
  Then, we compute an optimal circulation flow $f$ in $\overrightarrow{RF}$ which can be done in $\mathcal{O}(|A(\overrightarrow{RF})|^{\nicefrac{5}{2}} \cdot |V(\overrightarrow{RF})|^3) = \mathcal{O}((\tau\cdot tnd^2)^{\nicefrac{5}{2}}\cdot (\tau\cdot tnd)^3) = \mathcal{O}(\tau^{\nicefrac{11}{2}}\cdot tnd^8)$~\cite{Tardos85}. We can then construct an optimal reconfigurable sequence $T$ compatible with $Y$, as described in the proof of \Cref{lemma:flow reconfiguration}. We obtain an overall complexity of $\mathcal{O}(\tau\cdot tnd \cdot f(n) + \tau^{\nicefrac{11}{2}} \cdot tnd^8)$.
\end{proof}

We now have the necessary tools for the overall algorithmic result:

\begin{lemmarep}   \label{lemma:fpt nd}
  Let $\mathcal{G}=(G,\lambda)$ be a temporal graph and let $\Pi$ be an $f(n)$-temporal neighbourhood diversity locally decidable reconfiguration problem. 
 $\Pi$ is solvable in $\mathcal{O}(2^{(tnd\cdot \tau)} \cdot (\tau\cdot tnd \cdot f(n) + \tau^{\nicefrac{11}{2}} \cdot tnd^8))$ where $tnd$ is the temporal neighbourhood diversity of $\mathcal{G}=(G,\lambda)$.
\end{lemmarep}

\begin{proof}
  
  For each candidate sequence $Y = (Y_1,\dots,Y_\tau)$, we call the function $check_\Pi$ and we compute an optimal reconfigurable sequence compatible with it. Thus, each iteration can be done in $\mathcal{O}(\tau\cdot tnd \cdot f(n) + \tau^{\nicefrac{11}{2}} \cdot tnd^8)$. For each time-step $t$, there exists $2^{|V(\tndg|)} = 2^{(tnd)}$ possibilities to choose $Y_t$. Hence, there are $2^{tnd\cdot \tau}$ candidate sequences. We obtain an overall complexity of $\mathcal{O}(2^{tnd\cdot\tau} \cdot (f(n)))$.
\end{proof}

\begin{corollaryrep}
  Let $\mathcal{G}=(G,\lambda)$ be a temporal graph with temporal neighbourhood diversity $tnd$ and lifetime $\tau$.
  {\sc Temporal Dominating Set Reconfiguration} and {\sc Temporal Independent Set Reconfiguration} are solvable in $\mathcal{O}(2^{(tnd\cdot\tau)}\cdot(\tau^{\nicefrac{11}{2}} \cdot tnd^8))$ in $\mathcal{G}$.
  
\end{corollaryrep}

\begin{proof}
By \Cref{lemma:ds is ndld,lemma:is is ndld}, the complexity of the functions $check$ for $low_\pi$ and $up_\pi$ {\sc Dominating Set} and {\sc Independent Set} is $\mathcal{O}(|V(\tndg)|+|E(\tndg)|) = \mathcal{O}(tnd^2)$. Hence, by \Cref{lemma:compute reconfigurable sequence}, we can compute and check an optimal sequence in time $\mathcal{O}(\tau^{\nicefrac{11}{2}} \cdot tnd^8)$. Hence, by \Cref{lemma:fpt nd}, {\sc Temporal Dominating Set Reconfiguration} and {\sc Temporal Independent Set} are solvable in $\mathcal{O}(2^{(tnd\cdot\tau)}\cdot(\tau^{\nicefrac{11}{2}} \cdot tnd^8))$.
\end{proof}

\section{An FPT algorthm with respect to the lifetime of the temporal graph and the treewidth of the underlying graph}
\label{sec:tw algo}
In this section we show that if a static problem is FPT in the treewidth, then its reconfiguration version is FPT in the treewidth of the footprint (the union of all time-steps) and the lifetime of the temporal graph.
We first introduce definitions specific to this section. Then, using Courcelle's theorem, we show that if the static version is expressable in MSO then the reconfiguration version can be parameterized by the lifetime and the treewidth.

%Finally, we show how to adapt a generic algorithm on a tree decomposition for the static version to the reconfiguration version.


% \subsection{Definitions}

\subsection{Treewidth and tree decompositions}

Tree decompositions are widely used to solve a large class of combinatorial problems efficiently by dynamic programming when the graph has low treewidth. 

\begin{definition}[Treewidth, tree decomposition \cite{BodlaenderK96,Kloks94}]\label{def:TD}
  Given a static graph $G$, a \emph{tree decomposition} of $G$ is a pair $(\mathcal{T},\mathcal{X})$
  where $\mathcal{T}$ is a tree and 
  $\mathcal{X}=\{B_i\mid i\in V(\mathcal{T})\}$ is a multiset of subsets of $V(G)$ (called ``bags'') such that
  \begin{enumerate}[(a)]
    \item for each $uv \in E(G)$, there is some $i$ with $uv \subseteq B_i$ and
    \item for each $v \in V(G)$, the bags $B_i$ containing $v$ form a connected subtree of $\mathcal{T}$.
    \end{enumerate}
    \vspace{0.1cm}
  The width of $(\mathcal{T},\mathcal{X})$ is $\max_{B_i \in \mathcal{X}} |B_i|-1$.
%   Further, $(\mathcal{T},\mathcal{X})$ is called \emph{nice} if
%   \begin{enumerate}[(i)]
%     \item $\mathcal{T}$ is rooted at bag $B_r$, with $B_r = \emptyset$ and each bag has at most two children.
%     \item Each bag $B_i$ of $\mathcal{T}$ has one of the four types:
%       \begin{itemize}
%       \item \textbf{Leaf bag:} $i$ has no children and $B_i = \emptyset$.
%         \item \textbf{Join bag:} $i$ has two children $j$ and $k$ and $B_i = B_j = B_k$.
%         \item \textbf{Introduce vertex $u$ bag:} $i$ has only one child $j$ and $B_j = B_i\setminus\{u\}$.
% %      \item \textbf{Introduce edge $uv$ bag:} $i$ has only one child $j$ and $B_j = B_i$.
%       \item \textbf{Forget $u$ bag:} $i$ has only one child $j$ and $B_j = B_i\cup\{u\}$.
%     \end{itemize}
%   \end{enumerate}
%   \vspace{0.1cm}
% 
  % For any bag $B_i$ of $\mathcal{T}$, we let $\mathcal{T}_i$ denote the subgraph of $G$ containing all vertices and edges that have been introduced ``below'' $B_i$
  % (that is, in a bag of the subtree of $T$ that is rooted at $i$).
\end{definition}

\noindent It is NP-complete to determine whether a graph has treewidth at most $k$~\cite{Arnborg87}. However, there exists a linear-time algorithm that, for any constant $k$, computes a tree-decomposition with treewidth at most $k$, if there is one~\cite{Bodlaender96}.
% \noindent
% Note that for each vertex $u$ of $G$, $(\mathcal{T},\mathcal{X})$ contains exactly one forget $u$ bag.  Bodlaender and Kloks shown that a nice tree decomposition contains at most $\mathcal{O}(|E|\cdot |V|)$ bags~\cite{BodlaenderK96}.
%
% Let $\mathcal{G}=(G,\lambda)$ be a temporal graph. A \emph{temporal tree decomposition} for $\mathcal{G}$ is a tree decomposition $(\mathcal{G},\mathcal{X})$ of the footprint $G$ in which we have replaced every introduce edge $uv$ bag by $|\lambda(uv)|$ introduce lambda edge $(uv,t)$ bag. Formally, let $B_i$ be an introduce edge $uv$ bag with child $B_j$ and parent $B_\ell$ and let $\lambda(uv)= \{t_1,\dots,t_k\}$. We replace remove $B_i$ from the tree decomposition and we introduce $k$ introduce lambda edge $(uv,t)$ bags $B_{t_1},\dots,B_{t_k}$ such that $B_j=B_{t_1}=\dots=B_{t_k}$, $B_{t_1}$ has child $B_j$, $B_\ell$ has child $B_{t_k}$ and for each $t_{x} \in \lambda(uv) \setminus \{t_1\}$, $B_{t_x}$ has child $B_{t_{x-1}}$.
%Notice that a nice tree decomposition contains at most $\mathcal{O}(\tau\cdot|E(G)|\cdot|V(G)|)$.


\paragraph{Monadic second order logic.} % A \emph{relational vocabulary} $\mathcal{R}$ is a set of relation symbols. Each relation symbol $R$  has an arity, denoted $arity(R) > 1$. A structure $\mathcal{A}$ of vocabulary $\mathcal{R}$ , or $\mathcal{R}$-structure, consists of a set $A$, called the \emph{universe}, and an interpretation $R^\mathcal{A} \subseteq A^{arity(R)}$ of each relation symbol $R \in \mathcal{R}$. We write $\bar{a} \in R^\mathcal{A}$ or $R^\mathcal{A}(a)$ to denote that the tuple
%  $a \in A^{arity(R)}$ belongs to the relation $R^\mathcal{A}$.
% We briefly recall the syntax and semantics of first-order logic. We fix a countably infinite set of (individual) variables, for
% which we use small letters. Atomic formulas of vocabulary $\mathcal{R}$ are of the form: $x = y$ or $R(x_1,\dots,x_r)$,  where $R \in \mathcal{R}$ is r-ary and $x_1,\dots,x_r,x,y$ are variables. \emph{First-order formulas} of vocabulary $\mathcal{R}$ are built from the atomic formulas using the boolean connectives $\neg, \vee,\wedge$ and existential and universal quantifiers $\exists$, $\forall$. The difference between first-order and second-order

Monadic second-order logic (MSO logic) is a fragment of second-order logic where quantification is restricted to sets. Importantly, if a graph property is expressible in MSO logic, Courcelle's theorem states that there exists fixed-parameter tractable algorithm in the treewidth of the graph and in the length of the MSO expression.
%
In graphs, we are allowed to use the following variables and relations to express a property in MSO logic:

\begin{compactitem}
  \item standard boolean connectives: $\neg$ (negation), $\wedge$ (and), $\vee$ (or), $\Rightarrow$ (implication),
  \item standard quantifiers: $\exists$ (existential quantifier), $\forall$ (universal quantifier), which can be applied to any variable used to represent vertices, edges, sets of vertices or sets of edges of a graph. By convention, lower-case letters are used to represent vertices and edges, and upper-case letters are used to represent sets of vertices or edges,
  \item the binary equality relation $=$, the binary inclusion relation $\in$, the binary incidence relation $inc(e,v)$ which encodes that an edge $e$ is incident to a vertex $v$.
\end{compactitem}

A \emph{free variable} is a variable not bound by quantifiers. An MSO \emph{sentence} is an MSO formula with no free variables. Let $G$ be a graph, the notation $G\models \phi$ indicates that $G$ verifies the formula. Courcelle's theorem is stated as follows.


\begin{theorem}[Courcelle's theorem~\cite{Courcelle86a,Courcelle90}]
  \label{theorem:courcelle} Let $G$ be a simple graph of treewidth $tw$ and a fixed MSO sentence $\phi$, there exists an algorithm that tests if $G\models \phi$ and runs in $\mathcal{O}(f(tw,|\phi|) \cdot |G|)$, where $f$ is a computable function.
\end{theorem}


\subsection{MSO formulation}

We show, using Courcelle's theorem, that if a static problem $\Pi_S$ is definable in the monadic second-order logic, then its reconfiguration version $\Pi_T$ is FPT when parameterized by the treewidth of the footprint and the lifetime of the temporal graph.

In order to do that, we first convert the temporal graph $\mathcal{G}=(G,\lambda)$ into a static graph $H$ as defined in the following construction.



\begin{construction}
  \label{const:sliding}
  Given a temporal graph $\mathcal{G}=(G,\lambda)$ with bounded lifetime $\tau$ and vertex set $V(G) = \{u_1,\dots,u_n\}$, we consider the following static graph $H$ with vertex set $V(H) = \{ v^t_i \mid u_i \in V(G), t \in [1,\tau]\}$ and such that $H$ contains the following edges:% and edge set $E(H) = \bigcup\limits_{t \in [1,\tau]} E_t \bigcup\limits_{t \in [1,\tau-1]} E_{t,t+1}$ where $E_t = \{v^t_iv^t_j \mid u_iu_j \in G_t\}$ and $E_{t,t+1} = \{v^t_iv^{t+1}_j \mid u_iu_j \in G_t\} \cup \{v^t_iv^{t+1}_i \mid u_i \in V(G)\}$.
  \begin{itemize}
    \item for each $t \in [1,\tau]$, $H$ contains the set of edges $E_t = \{ v^t_iv^t_j \mid u_iu_j \in E(G_t)\}$ (\textit{i.e.} $H$ contains the disjoint union of each snapshot $G_t$), and
    \item for each $t \in [1,\tau-1]$, $H$ contains the edge set $E_{t,t+1} = \{v^t_iv^{t+1}_j \mid u_iu_j \in E(G_t)\} \cup \{v^t_iv^{t+1}_i \mid u_i \in V(G)\}$ (\textit{i.e.} there is an edge between $v^t_i$ and $v^{t+1}_j$ if it is possible to move a token from $u_i$ to $u_j$ at time-step $t$).  
  \end{itemize}
\end{construction}
Note that this construction is related to but is not the same as the time-expanded graph of a temporal graph (as used in, e.g.~\cite{fluschnik_as_2020}).

\begin{lemmarep}
Let $\mathcal{G}=(G,\lambda)$ be a temporal graph, and let $H$ be the graph described in \Cref{const:sliding}. The treewidth of $H$ is at most $\tau\cdot tw$ where $tw$ is the treewidth of the footprint $G$.
\end{lemmarep}
\begin{proof}
  Let $(T,\mathcal{X})$ be a tree decomposition of $G$ of width $tw$. We construct a tree decomposition $(T,\mathcal{Y})$ for $H$ as follows. For each bag $B_x \in \mathcal{X}$, we subtsitute each vertex $u_i \in B_x$ by the set of vertices $\{v^t_i \mid t \in [1,\tau]\}$. Clearly, for any vertex $v^t_i \in V(H)$, the bags in $\mathcal{Y}$ containing $v^t_i$ are connected in $T$ since the bags in $\mathcal{X}$ containing $u_i$ are connected in $T$.
For each edge $v^t_iv^t_j \in E(H)$ (respectively $v^t_iv^{t+1}_j$), let $B_x$ be a bag containing $u_iu_j$ in $\mathcal{X}$, after the substitution, $B_x$ contains the edge $v^t_iv^t_j$ (resp. $v^t_iv^{t+1}_j$). For each edge $v^t_iv^{t+1}_i$, any bag $B_x$ containing the vertex $u_i \in \mathcal{X}$ contains the edge $v^t_iv^{t+1}_i$. Hence, $(T,\mathcal{Y})$ is a tree decomposition of $H$ and since each vertex is replaced by $\tau$ vertices in each bag, the width of $(T,\mathcal{Y})$ is $\tau \cdot tw$.
\end{proof}
 

\noindent An example of a graph produced by \Cref{const:sliding} is depicted in \Cref{fig:to static}.

\begin{figure}[ht]
  \centering
  \scalebox{0.75}{
  \begin{tikzpicture}
    \foreach \Y in {1,2} {
      \foreach \X in {1,...,3} {
        \foreach[count=\l from 1] \x/\y/\a in {0/0/-90,1/-1/-90,0/2/90,1/1/-135,1/3/90}{
          \ifthenelse{\equal{\Y}{2}}{
            \node[smallvertex,label=90+90*\X:{$v^\X_\l$}] (\X\l) at (\X*2.5+\x+\Y*9,\y*0.75) {};
          }{
            \node[smallvertex,label=90+90*\X:{$u_\l$}] (\X\l) at (\X*2.5+\x+\Y*9,\y*0.75) {};
          }
        }

        \ifthenelse{\equal{\Y}{1}}{
          \node at (\X*2.5+0.5+\Y*9,-2) {$G_\X$};
        }
        {}
      }
      
      \foreach \a/\b/\L in {2/1/{2,3},1/3/{1,3},3/5/{1},3/4/{2}} {
        \foreach \X in \L {
          \draw (\X\a) -- (\X\b);
          \ifthenelse{\equal{\Y}{2}}{
          \ifthenelse{\equal{\X}{3}}{}{
            \pgfmathsetmacro{\B}{int(\X + 1)}
            \draw[draw=iris,dashed] (\X\a) -- (\B\b);
            \draw[draw=iris,dashed] (\B\a) -- (\X\b);
          }
          }{}
        }
      }
      \ifthenelse{\equal{\Y}{2}}{
        \foreach[count=\B from 2] \X in {1,2}{
          \foreach \a in {1,...,5}{
          \draw[draw=iris,dashed] (\X\a) -- (\B\a);
        }
        }
      }{}
      
    }
 % \foreach \a in {12,22,33}
 % \node[smallvertex,fill=gray] at (\a.center) {};
    
  \end{tikzpicture}
}
  \caption{\label{fig:to static} Example of a graph produced by \Cref{const:sliding}. The edges in $E_{t,t+1}$ sets are depicted in blue/dashed.}
\end{figure}


Let $\mathcal{G}$ be a temporal graph and let $H$ be the static graph produced by \Cref{const:sliding} with $\mathcal{G}$ given in input. for convenience, we denote by $V_t$ the set of vertices in $H$ with superscript $t$. Notice that finding a solution for $\Pi_T$ in $\mathcal{G}$ is equivalent to finding a vertex set $X$ in $H$ such that:
\begin{compactitem}
  \item for each $t \in [\tau]$, $X \cap V_t$ is a solution for $\Pi_S$ in $H[V_t]$, and
  \item for each $t \in [\tau-1]$, there is a perfect matching between $X \cap V_t$ and $X \cap V_{t+1}$.
  \end{compactitem}
We show that if $\Pi_S$ can be expressed in MSO logic, then we can also express the reconfiguration version $\Pi_T$ in MSO logic, with an increase in formula size by a factor $\tau$.
By Courcelle's theorem, it follows that $\Pi_T$ is FPT in the lifetime of the temporal graph and the treewidth of the footprint combined.

\begin{theoremrep}
  \label{theorem:mso}
  Let $\Pi_T$ be a reconfiguration problem such that its static version $\Pi_S$ is expressable with an MSO formula $\phi(H)$.
  Let $\mathcal{G}=(G,\lambda)$ be a temporal graph such that the treewidth of $G$ is $tw$. There is an algorithm to determine if there is a reconfigurable sequence of size $k$ for $\Pi_T$ in $\mathcal{G}$ in $\mathcal{O}(f(tw,\tau,|\phi|) \cdot \tau\cdot|G|)$.
\end{theoremrep}

\begin{proof}

  Let $H$ be a graph produced by \Cref{const:sliding} on $\mathcal{G}=(G,\lambda)$. Since $\Pi_S$ is expressable in MSO logic, there is a predicate $\phi(G_t,X)$ indicating if a subset of vertices $X$ is a solution for $\Pi_S$ in $G_t$ 
  % Let $\phi$ be an MSO formula such that G
  We construct an MSO formula to express a set $X$ such that $X \cap V_t$ is a solution for $\Pi_S$ in $H[V_t]$ for each $t \in [\tau]$ and such that there is a perfect matching between $X \cap V_t$ and $X \cap V_{t+1}$ for each $t \in [\tau-1]$.


We introduce the following predicate that given a set of edges $M$, two sets of vertices $X$ and $Y$ and a set of edges $E$, $match\_edge$ returns \texttt{true} if
$v$ belongs to $X$,
there is exactly one edge $e\in M$ incident to $v$ and
the other endpoint of $e$ belongs to $Y$. 


\begin{eqnarray*}
  match\_edge(M,X,v,Y) = &  v \in X \wedge \exists e, \exists u, (e \in M) \wedge (u \in Y) &\wedge  \\
  & inc(e,v) \wedge inc(e,u) &\wedge  \\
  & \forall e', (e'\in M \wedge inc(e',v)) \Rightarrow e = e'&
  \end{eqnarray*}

  Notice that given two disjoint vertex sets $X$ and $Y$, a set of edges $M$ forms a perfect matching between $X$ and $Y$ if for every $v$ in $X$ we have $match\_edge(M,X,v,Y)= \texttt{true}$ and for every $u$ in $Y$ we have $match\_edge(M,Y,u,X) = \texttt{true}$.
  Hence, we can formulate the following predicate that, given two vertex sets $X$ and $Y$ and an edge set $E$ returns \texttt{true} if $E$ contains a perfect matching between $X$ and $Y$.


\begin{eqnarray*}
  Matching(X,Y,E) =
  \exists M, \forall v & &\\
                       & (v\not\in X  \wedge v \not\in Y ) &\vee\\
                       & match\_edge(M,X,v,Y) &\vee \\
                       &match\_edge(M,Y,v,X) &\\
\end{eqnarray*}

Hence, we can encode $\Pi_T$ with an MSO formula as follows:
\begin{eqnarray}
\exists X_1 \subseteq V_1,\dots, X_\tau \subseteq V_\tau \bigwedge\limits_{t\leq\tau} \phi(G_t,X_t) \bigwedge\limits_{t<\tau} Matching(X_t,X_{t+1},E_{t,t+1})  
\end{eqnarray}
The size of the formula is in $\mathcal{O}(\tau |\phi|)$ so, we can conclude by \Cref{theorem:courcelle} that $\Pi_T$ is solvable in $\mathcal{O}(f(tw,\tau,|\phi|) \cdot \tau\cdot|G|)$.
\end{proof}

\noindent Let $G$ be a static graph. {\sc Dominating Set} and {\sc Independent Set} can formulated in $G$ as follows:
  \[
    \exists D \subseteq V(G), \forall v \in V(G), v \in D \vee (\exists u\in D, uv\in E(G)) \tag{{\sc Dominating Set}}
  \]
  \[
    \exists I \subseteq V(G), \forall v \in I, \forall u \in I, uv \not\in E(G) \tag{{\sc Independent Set}}. 
  \]
  Hence, by \Cref{theorem:mso}, we can deduce the following result for {\sc Temporal Dominating Set Reconfiguration} and {\sc Temporal Independent Set Reconfiguration}.

\begin{corollary}
{\sc Temporal Dominating Set Reconfiguration} and {\sc Temporal Independent Set Reconfiguration} are solvable in $\mathcal{O}(f(tw,\tau) \cdot \tau\cdot|G|)$.
\end{corollary}




%%% Local Variables:
%%% mode: LaTeX
%%% TeX-master: "main"
%%% End:



\section{Conclusion and future work}
Motivated by both the ability of temporal graphs to model real-world processes and a gap in the theoretical literature, we have defined a general framework for formulating vertex-selection optimisation problems as a temporal reconfiguration problems, and have described several associated algorithmic tools.  

While hardness results on static vertex selection problems will straightforwardly imply hardness for their corresponding reconfiguration versions, we have described several algorithmic approaches, including an approximation algorithm and several fixed-parameter tractable algorithms.  

Several areas of future work present themselves: first, further investigation of which problems can be solved using our results, or which other temporal parameters are useful here.  
Secondly, because temporal problems are so frequently harder than corresponding static ones, it may be interesting to establish negative results that are stronger than those in the static setting, such as $W[k]$-completeness results that consider the lifetime as a parameter.
Finally, we could explore a more restrictive version of the model, where the number of tokens allowed to move at each time step is bounded, or there are other restrictions on the speed of change of the vertex set. 



% ---- Bibliography ----
%
% BibTeX users should specify bibliography style 'splncs04'.
% References will then be sorted and formatted in the correct style.
% 
%\bibliographystyle{splncs04}
\bibliography{biblio}
%
\end{document}
