\section{Related Works}
\label{sec:related}

The Hyperledger Fabric platform has been extensively discussed in previous research, focusing on performance metrics.
This paper seeks to update the state of the art and fill gaps left by existing studies.

In \cite{melo_computing2022}, we introduced models to assess the use of computational resources in the context of HLF.
Using Continuous Time Markov Chains (CTMCs) and Stochastic Petri Nets (SPNs), we demonstrated the effectiveness of these formalisms in modeling and evaluating HLF-based applications, particularly in detecting infrastructure bottlenecks.
However, the model proposed in this paper extends the scope of the previous work, assessing metrics such as throughput and transaction latency not addressed by previously proposed models.

In other studies like \cite{wu_acm_ease2022, jiang2020performance, ke_springer_cc2022}, the authors modeled general system performance metrics from the perspective of HLF.
Jiang et al.\cite{jiang2020performance}, for example, used a hierarchical modeling approach for HLF 1.4, analyzing indicators such as throughput, latency, and system utilization.
Whereas Wu et al.\cite{wu_acm_ease2022} developed a queue theory-based model focused on the flow of a transaction within the scope of HLF 2.0.
Ke and Park~\cite{ke_springer_cc2022} proposed queue models considering different service rates.
% However, these studies did not include a formal bottleneck detection method, such as applying sensitivity analysis to quantify the impact caused by each platform component on the metric of interest.

Finally, in other works like \cite{sukhwani2018performance, yuan2020performance}, the authors evaluated the influence of arrival rates and block sizes on HLF throughput and latency. Still, as we will see in the present work, these are not the only factors significantly impacting these metrics.
Furthermore, in Yuan et al.\cite{yuan2020performance}, the authors modeled the platform using Generalized Stochastic Petri Nets (GSPN). In contrast, in Sukhwani et al.\cite{sukhwani2018performance}, the authors used Stochastic Reward Networks (SRN); both models are isomorphic to the SPNs presented in this work, that is, they have the same symbolic power; however, they do not contemplate the transaction flow for more recent versions of HLF.

This work extends those studies, especially directly extends our previous work \cite{silva2023avaliaccao}, by presenting an SPN model for the Hyperledger Fabric platform, focusing on the HLF 2.5+ version addressing the concept of the gateway and the possibility of it being a single point of failure and a pre-platform bottleneck.
% The methodology proposed in developing this work includes a formal validation and bottleneck identification in the proposed model, increasing the practical relevance of the findings through case studies.

%---------------------