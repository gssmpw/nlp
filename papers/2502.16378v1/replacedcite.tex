\section{Related Work}
We divide this section into two subsections. First, we describe evolutionary algorithms that are used to solve computational chemistry problems, which is an area that encompasses chemical ADMET property prediction. Next, we analyse a few works on AutoML related to computational chemistry, focusing on recent works that build and recommend customised predictive pipelines based on (bio)chemical data.

\subsection{Evolutionary Algorithms for Computation Chemistry}

There have been several efforts to solve computational chemistry or cheminformatics problems using evolutionary computation (EC) problems. The survey of Yu et al.____ mainly explores the use of EC for drug discovery, including the development of EC methods for docking, lead compound generation (such as ligands) and exploring the quantitative structure-activity relationships (QSAR) of compounds. 

Nevertheless, this section focuses on using EC for molecular generation and optimisation, and machine learning tasks, which are more related to Auto-ADMET's goals. The works of Soto et al. ____ and ____ utilise single-objective and multiple-objective genetic algorithms for selecting the best set of descriptors (or features) in ADMET property prediction tasks, respectively. They tested different machine learning models for these studies, including decision trees, k nearest neighbours and polynomic non-linear function regression models to estimate the quality of a given feature set. 

Liu et al. ____, in turn, proposed and developed ECoFFeS, which is an evolutionary-based feature selection software designed for drug discovery. ECOFFeS encompasses both single-objective and multi-objective bioinspired or evolutionary algorithms. Whereas ECoFFeS' single-objective algorithms include ant colony optimization (ACO)____,  differential evolution (DE)____,  genetic algorithm (GA) ____ and particle swarm optimization (PSO) ____, its multi-objective counterpart supports only two well-known Multi-Objective Evolutionary Algorithms (MOEAs) -- i.e., MOEA/D ____ and NSGA-II____.

EC may also be used to optimise chemical compounds, where evolutionary operators can be applied to molecules to derive new ones ____. For example, in Fromer and Coley ____, it is argued that during the evolutionary process aiming to optimise new molecules, a mutation operator might be used to add or remove atoms, bonds or molecular fragments. On the other hand, the crossover operator may be used to exchange molecular fragments among molecules ____. 

In terms of evolutionary computation, molecular optimisation and generation, and large language models, we have the work of da Silva et al. ____, which modelled \emph{de novo} drug design as a many-objective optimization problem (MaOP) since in drug discovery we do have problems with several conflicting objectives (e.g., potency versus safety versus proper pharmacokinetic properties). da Silva et al.'s work involved in developing genAI approaches combined with multi- and many-objective evolutionary algorithms (MOEAs and MaOEAs) for drug development, specifically for combining a generative deep learning model’s latent space with MOEA/MaOEA (NSGA-II/NSGA-III) for designing new and diverse molecules.


\subsection{AutoML for Cheminformatics}

Several ML methods have been proposed for dealing with cheminformatics tasks, including but not limited to pharmacokinetics, human and environmental toxicity, pharmacodynamics and pharmacogenetics ____. As aforementioned, the main issue in using the ML models derived from these methods is that they are static and non-customisable, leading to possible biases and the lack of predictive generalisation in cases where the input chemical molecules differ from those used to train the ML models. 

Recently, a few works have made efforts to automate and, consequently, personalise cheminformatics or computational chemistry pipelines through search and optimisation, such as the work of de S\'a\ and Ascher ____, AutoQSAR ____,  ZairaChem ____, Uni-QSAR ____, QSARtuna ____, and Deepmol ____.

de S\'a\ and Ascher ____ introduced the first evolutionary-based AutoML algorithm to build and recommend tailored predictive pipelines for small molecule pharmacokinetic data. These pipelines included feature definition, scaling and selection, and machine learning algorithms and hyperparameter optimisation. All steps followed by this AutoML algorithm are within a context-free grammar, which is followed to generate individuals, perform genetic operations and guide the evolutionary algorithm to produce only valid solutions. 

AutoQSAR ____, on the other hand, utilises an accuracy score to rank ML pipelines that are aiming to solve a QSAR problem. Nevertheless, AutoQSAR relies on an exhaustive search, not being able to scale in larger datasets. Following a distinct approach, ZairaChem ____ follows open-source ideas to deliver a robust AutoML package for drug screening, employing five optimisation methods for this purpose independently and targetting different objectives (e.g., predictive performance, interpretability and robustness).

Uni-QSAR ____ and QSARtuna ____ are both automated QSAR frameworks for molecule property prediction. In the case of Uni-QSAR, a stacking method is employed to combine the solutions of several ML models and predict molecule properties as a result. Differently, QSARtuna takes advantage of Bayesian optimisation for the same task.

Finally, Correia et al. ____ and Li et al. ____ proposed Deepmol and Model Training Engine (MTE), respectively. Both Deepmol and MTE are AutoML frameworks for computational chemistry considering both traditional machine learning and deep learning models. These frameworks are defined by Bayesian optimisation algorithms that search for and optimise pipelines in the context of drug discovery problems. Deepmol's and MTE's search spaces incorporate a list of options, such as standardisation, feature extraction, feature scaling and selection, machine learning modelling and imbalance learning.