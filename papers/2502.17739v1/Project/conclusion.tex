\section{Conclusion and Future Work}
In this work, we investigated the influence of graph topology on the behavior of GNNs, with a focus on how local topological features, such as $k$-hop similarity, interact with the message-passing scheme to impact global learning dynamics. Our theoretical framework, grounded in the concept of $k$-hop similarity, provides new insights into how locally similar neighborhoods induce consistent learning patterns across graphs. This interaction leads to either effective learning or inevitable oversmoothing, depending on the graph's inherent properties. Through empirical validation, we demonstrated that $k$-hop similar graphs lead to consistent GNN predictions, reinforcing our conjecture of weight consistency under $k$-hop similarity. Furthermore, we offered a topological explanation for oversmoothing, showing how deep GNNs trained on graphs with large $k$ values effectively transform each connected component into a complete graph, resulting in node representations that converge to identical values within each component. This sheds light on why oversmoothing occurs in deep GNNs but also in shallow ones, particularly in graphs with community structures or disconnected components.

Despite these insights, our framework still faces certain limitations, particularly in the scalability of our $k$-hop similar graph generation algorithm and the comprehensiveness of our empirical validation. Future work should focus on developing alternative algorithms that can efficiently handle large graphs. Additionally, more extensive empirical validation is crucial. This includes conducting ablation studies to assess the impact of different $k$ values, evaluating performance across various downstream tasks and GNN architectures, and testing on real-world datasets to further substantiate our findings.