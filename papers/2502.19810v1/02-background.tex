\section{Background}
\label{sec:background}

In this section, we present a brief review of Rust's borrow mechanisms in \cref{sec:background:rust} and the key concepts for automatic amortized resource analysis (AARA) in \cref{sec:overview:AARA}.

\begin{figure}[t]
\footnotesize
\centering
\hrule
\begin{subfigure}[b]{0.49\textwidth}
\begin{lstlisting}[language=Rust, style=colouredRust]
let mut n = 0;
let p1 = &n; // OK, shared borrow from n
// --- p1 lifetime begins ---
let p2 = &n; // OK, shared borrow from n
// --- p2 lifetime begins ---
// --- p2 lifetime ends ---
// --- p1 lifetime ends ---
\end{lstlisting}
\caption{Shared \& Shared}\label{fig:shared-shared}
\end{subfigure}
%
\begin{subfigure}[b]{0.49\textwidth}
\begin{lstlisting}[language=Rust, style=colouredRust]
let mut n = 0;
let p1 = &n; // OK, shared borrow from n
// --- p1 lifetime begins ---
*p1 = 42; // ERROR, p1 is not mutable
/* drop(p1) */
// --- p1 lifetime ends ---
let p2 = &mut n; // OK, mutable borrow from n
*p2 = 42; // OK, p2 is mutable
\end{lstlisting}
\caption{Shared \& Mutable}\label{fig:shared-mutable}
\end{subfigure}
%
\hrule
%
\begin{subfigure}[b]{0.49\textwidth}
\begin{lstlisting}[language=Rust, style=colouredRust]
let mut n = 0;
let p1 = &mut n; // OK, mutable borrow from n
// --- p1 lifetime begins ---
let p2 = &mut n; // ERROR
// n is already mutable-borrowed
*p1 = 42;
// --- p1 lifetime ends ---    
\end{lstlisting}
\caption{Mutable \& Mutable}\label{fig:mutable-mutable}
\end{subfigure}
%
\begin{subfigure}[b]{0.49\textwidth}
\begin{lstlisting}[language=Rust, style=colouredRust]
let mut n = 0;
let p1 = &mut n; // OK, mutable borrow from n
// --- p1 lifetime begins ---
let p2 = &n; // ERROR
// n is already mutable-borrowed
let m = do_something_with(p2);
*p1 = 42;
// --- p1 lifetime ends ---
\end{lstlisting}
\caption{Mutable \& Shared}\label{fig:mutable-shared}
\end{subfigure}
%
\hrule
%
\begin{subfigure}[b]{0.53\textwidth}
\begin{lstlisting}[language=Rust, style=colouredRust]
fn consume(p: &mut i32){ *p = 1; }
fn reborrow() {
  let mut n = 0;
  let p1 = &mut n; // OK, mutable borrow from n
  // --- p1 lifetime begins ---
  let p2 = &mut *p1; // OK, mutable reborrow from p1
  // --- p2 lifetime begins , p1 NOT accessible ---
  consume(p2);
  // --- p2 lifetime ends   , p1 accessible ---
  *p1 = 42;
  // --- p1 lifetime ends ---
}
\end{lstlisting}
\caption{Reborrow}\label{fig:reborrow}
\end{subfigure}
%
\begin{subfigure}[b]{0.46\textwidth}
\begin{lstlisting}[language=Rust, style=colouredRust]
enum List { Nil, Cons(i32, Box<List>) }
fn iter(l: &List) {
  match l {
    Nil => {
      tick(1); // consume 1 unit of resource
    }, 
    Cons(hd, tl) => { 
      tick(2); // consume 2 units of resource
      iter(&**tl); 
    }, 
  }
}
\end{lstlisting}
\caption{List Iteration}\label{fig:list-iteration}
\end{subfigure}
\hrule
\caption{Examples to Demonstrate Rust's Borrow Mechanisms and Automatic Amortized Resource Analysis}
\label{fig:borrows}
\end{figure}

\subsection{Rust's Borrow Mechanisms}
\label{sec:background:rust}

Rust's type system enforces ownership principles, e.g., declaring a variable \verb|x|
of some type means that \verb|x| is the \emph{exclusive} owner of the memory locations indicated by the type.
%
Similarly to linear or affine typing~\cite{kn:Walker02}, such exclusive ownership requires a \emph{move} semantics, which could soon be too restrictive in the common call-by-value paradigm.
%
Fortunately, as C has pointers and C++ has references, Rust introduces \emph{borrow} mechanisms around reference types to allow temporary accesses to the memory locations owned by others.
%
Rust employs a complex borrow checker to ensure that the borrows in a program work in a memory-safe way.

% A borrow in Rust works like a pointer in C, except that it comes with some restrictions. The Rust borrow checker will check whether a Rust program satisfies those restrictions. Our work assumes the program is always well-borrowed, checked by Rust's borrow checker.

Rust provides two main kinds of borrows: one is called \textbf{shared} or immutable borrows, the other exclusive or \textbf{mutable} borrows.
%
We demonstrate those borrow mechanisms using the example programs listed in \cref{fig:borrows}.
%
Bindings \verb|let| and \verb|let mut| introduce immutable and mutable variables, respectively, and the variables own the memory that stores the data.
%
\cref{fig:shared-shared} shows that one can use \verb|&n| to create a shared borrow from \verb|n|, and there can be multiple shared borrow from the same variable simultaneously.
%
\cref{fig:shared-mutable} shows that one can use \verb|&mut n| to create a mutable borrow from \verb|n|, and one cannot mutate the value of \verb|n| via shared borrows.
%
\cref{fig:mutable-mutable} shows that there can be at most one mutable borrow from the same variable at the same time.
%
\cref{fig:mutable-shared} shows that if there exists a mutable borrow, there cannot be any shared borrows from the same variable.
%
Rust's borrow checker introduces a notion of \emph{lifetimes} to track and check borrows.
%
In \cref{fig:borrows}, we annotate comments to indicate the lifetimes of certain borrows, and Rust's borrow checker performs lifetime inference to automatically derive the lifetimes.\footnote{The demonstration here is simplified for brevity and it is not consistent with the actual Rust's borrow checker.}
%
Note that in \cref{fig:shared-mutable}, Rust's borrow checker would implicitly insert \verb|drop(p1)| to end the lifetime of the shared reference \verb|p1|, enabling the next \verb|let| binding to create a mutable borrow \verb|p2| from the same variable.

% (1) Shared borrows share the value, and there can be many shared borrows towards one value, as \cref{fig:shared-shared} shows. (2) Shared borrows are immutable, and only mutable borrows are allowed to mutate the borrowed value, as \cref{fig:shared-mutable} shows. (3) Mutable borrows are exclusive towards the same value. The Rust compiler would implicitly insert \verb|drop(p1)| in \cref{fig:shared-mutable} and raise a compilation error when checking \verb|p2| in \cref{fig:mutable-mutable}. 

% Rust's borrow checker uses the notation of \emph{lifetimes} to track and check borrows, which are simply presented as implicit \verb|begins| and \verb|ends| scopes in \cref{fig:borrows}.\footnote{The demonstration here is simplified for brevity and it is not the actual implementation of Rust's compiler.}

% In practice, Rust programmers could still find out that the borrow checks are too strict.
% %
% For example, \cref{fig:mutable-shared} is totally fine because the mutable reference \verb|p1| is not used for mutation until the lifetime of the shared
% reference \verb|p2| ends.
% %
% Rust now supports \textbf{two-phase} borrows, which act as shared borrows initially and upgrade to mutable borrows later when the first mutation happens.
% %
% Therefore, Rust's borrow checker would not raise an error on \cref{fig:mutable-shared}, as it interprets the mutable borrow \verb|&mut n| as a two-phase borrow.

% However, in practice, programmers could soon find out that the lifetimes are over-approximation. In \cref{fig:mutable-shared}, the shared borrow \verb|p2| should be accepted, as if the mutable borrow \verb|p1| does not occurs until \verb|p2| ends. Rust therefore introduces \textbf{two-phase} borrows into borrow mechanism. Two-phase borrows first act as shared borrows, and will upgrade to mutable borrows when mutation occurs. Rust compiler will first try to emit two-phase borrows when mutable borrows occurs as \verb|&mut x|. 

The exclusiveness of mutable borrows could also be too strict: it is not uncommon to lend a mutable reference to a function call and then continue to use the mutable reference after the function returns.
%
Rust features a mechanism named \textbf{reborrowing}, which is done by creating borrows from dereferences, e.g., \verb|&mut *p1| as shown in \cref{fig:reborrow}.
%
In this example, \verb|p2| creates a mutable reborrow from \verb|p1|, and \verb|p1| becomes temporarily inaccessible during \verb|p2|'s lifetime because
\verb|p2| is mutable.
%
This does not break the exclusiveness of mutable borrows, because there is still at most one mutable reference can mutate the data.
%
Note that creating shared reborrows from dereferences is permitted, whereas
creating mutable reborrows from dereferences of shared references is not.

% Though mutable borrows are exclusive, they can be multiply consumed via reborrow mechanism, as in \cref{fig:reborrow}. Reborrowing is done by mutable reference of dereference of borrows, and will introduces a cascading of borrows with sub-lifetime. In another aspect, the dereference of mutable borrow can be viewed as a mutable variable. Shared borrows over the dereference of mutable borrows are also permitted, while mutable borrows over the derefernce of shared borrows are forbidden.

% The Rust borrow checker handles borrow mechanism, and we will assume restrictions described above as good properties of a well checked Rust program, to propose our new approach of resource analysis for Rust.

\subsection{Automatic Amortized Resource Analysis}
\label{sec:overview:AARA}

Automatic amortized resource analysis (AARA) is initially proposed by \citet{AARA-Linear} as a type system that derives linear bounds on the heap-space usage of functional programs.
%
In the past two decades, AARA has been extended with many features from different perspectives, and becomes a systematic methodology for resource analysis~\cite{JMSCS:HJ22}.
%
Although most of AARA type systems are presented with pure functional languages, e.g., Resource-aware ML~\cite{RaML},
in this section, we review the key concepts for AARA with a Rust program that uses  shared borrows.
%
As demonstrated in \cref{sec:background:rust}, Rust permits multiple simultaneous shared borrows from the same variable but disallows mutation via shared borrows; such behavior is similar to pure functional programming.

% has been applied to ML-like functional languages~\cite{RaML}. However, instead of in ML, we review the core idea of AARA in Rust, focusing solely on shared borrows. This is because, as stated in previous subsection,  shared borrows will never mutate values, similar to pure functional languages.

\cref{fig:list-iteration} implements list iteration via shared references.
%
A value of the \verb|Box| type owns a heap-allocated piece of data, adhering to Rust's ownership principles.
%
The \verb|Box| type provides a convenient way to represent heap-allocated data structures, such as lists and trees.
%
We use the statement \verb|tick(q)|  to indicate \verb|q| units of resource consumption.\footnote{In our current formalization and implementation, we use explicit $\kwd{tick}(q)$ as the cost model. It is also possible to adopt a parametric cost model that maps an expression or a statement to its resource consumption~\cite{RaML}.}
%
Because the \verb|iter| function consumes two units of resource per \verb|Cons| node and one unit for the \verb|Nil| node, the total resource consumption of applying \verb|iter| to a list of length $n$ is $2n+1$.
%
To automatically infer such a symbolic resource bound, AARA relies on the \emph{potential method} of amortized complexity analysis~\cite{JADM:Tarjan85},
i.e., AARA relies on the derivation of potential functions that map program states to non-negative numbers, in the sense that the potential of a state is sufficient to pay for the current step's resource consumption and the next state's potential.
%
AARA introduces resource-annotated types to encode certain kinds of potential functions in type signatures, and then reduces type inference to constraint solving (e.g., linear programming) to automate the bound inference.

% Suppose the length of list $l$ is $|l| = n$, then the resource consumption of function application \verb|iter(l)| is $2n+1$. 

To demonstrate the inference process, let us extend the list type
$\kwd{list}$ to a resource-annotated type $\kwd{list}(\alpha)$,
where $\alpha$ is a non-negative number and the type encodes a
linear potential function that assigns $\alpha$ units of potential to
each list element, i.e., the potential for a list of length $n$ is $\alpha \cdot n$.
%
Readers might realize the inconsistency between the abstract syntax $\kwd{list}$ and the concrete syntax:
\begin{lstlisting}[language=Rust, style=colouredRust, xleftmargin=\leftmargini]
enum List { Nil, Cons(i32, Box<List>) }
\end{lstlisting}
Here, we only annotate $\kwd{list}$ with one symbol $\alpha$ instead of two corresponding to the two constructors \verb|Nil| and \verb|Cons| to simplify the presentation.
%
Our implementation uses the multiple-symbols version, annotating the list type as follows:
\begin{lstlisting}[language=Rust, style=colouredRust, xleftmargin=\leftmargini, mathescape]
enum List { Nil : $\alpha_{\texttt{Nil}}$, Cons(i32, Box<List>) : $\alpha_{\texttt{Cons}}$ }
\end{lstlisting}
AARA systems can generally analyze different user-defined data types by annotating each constructor with one symbol denoting its potential.
%
To review AARA, we use the simplified version $\text{list}(\alpha)$, where $\alpha$ is essentially the annotation $\alpha_\texttt{Cons}$ for the \verb|Cons| constructor.

To initiate the analysis of \verb|iter|, we create a signature with two numeric unknowns $\alpha$ and $\delta$:
\[
\verb|iter| : \kwd{fn}(\verb|l|:\&\kwd{list}(\alpha)) \to () | \delta ,
\]
where $\delta$ denotes additional potential needed by \verb|iter| that is not associated with the list's potential function.
%
We then analyze the function body of \verb|iter| to collects constraints on $\alpha$ and $\delta$.
%
The pattern match on \verb|l| peeks the list structure; thus, according
to the definition of $\kwd{list}(\alpha)$, the \verb|Cons| branch would obtain $\alpha$ units of potential.
%
We proceed to analyze the two branches.
\begin{itemize}
    \item In the \verb|Nil| branch, we have $\delta$ units of potential and the statement \verb|tick(1)| consumes one unit of resource.
    %
    According to the potential method, we collect one constraint $\delta \ge 1$.

    \item In the \verb|Cons| branch, we have $\delta + \alpha$ units of potential, and the variables \verb|hd| and \verb|tl| have type $\&\kwd{i32}$ and $\&\kwd{box}~\kwd{list}(\alpha)$, respectively.
    %
    The statement \verb|tick(2)| consumes two units of resource, followed by a recursive call with \verb|&**tl|.
    %
    We now use the signature of \verb|iter| to analyze this recursive call; that is, it requires \verb|&**tl| to have type $\&\kwd{list}(\alpha)$ as well as $\delta$ units of additional potential.
    %
    The first requirement is fulfilled immediately as \verb|tl| has type $\&\kwd{box}~\kwd{list}(\alpha)$, and the second requirement can be satisfied if $\delta + \alpha \ge 2 + \delta$.
\end{itemize}
%
Noting that the constraints $\delta \ge 1, \delta + \alpha \ge 2 + \delta$ are linear constraints, we can employ a linear programming (LP) solver to automatically find instances of $\alpha$ and $\delta$.
%
For example, with a proper objective, an LP solver would find the assignments $\alpha = 2, \delta=1$, which yield the following signature:
\[
\verb|iter| : \kwd{fn}(\verb|l|:\&\kwd{list}(2)) \to () | 1 ,
\]
which coincides with the bound $2n+1$ where $n$ is the length of the list \verb|l|.

% \textbf{Linear Constraints and Programming:} 
% The AARA approach is to enrich type with resource potential annotation, i.e. resource type, and then collect linear constraints among those potentials during type checking:

% $
% \Gamma \vdash f : \kwd{fn}(l:\&\kwd{list}(\alpha)) \to () | \delta
% $

% In above, $\alpha$ is potential for each \verb|Cons|, $\delta$ is additional resource for function application. 

% When pattern matching list $l$, we will release resource $\alpha$ in the \verb|Cons(hd, tl)| branch. With this potential method from AARA, we can generate the following linear constraints during type checking.
% \begin{enumerate}
%     \item {$\delta \geq 1$ by \verb|Nil => { tick(1); }|, \\
%     where \verb|tick(1);| consumes 1 unit of resource, should be covered by $\delta$; }
%     \item {$\delta + \alpha \geq 2 + \delta$ by \verb|Cons(hd, tl) => { tick(2); iter(&*tl); }|, \\
%     where \verb|iter(&*tl)| consumes $\delta$ unit, to pay for function application.}
% \end{enumerate}

% Powered by linear programming solvers, and offering a heuristic object function, we can solve that $\delta = 1, \alpha = 2$, making the type of function $f$ as follow, therefore with $2|l| + 1$ as consumption:

% $
% \Gamma \vdash f : \kwd{fn}(l:\&\kwd{list}(2)) \to () | 1
% $

% The example is sufficient to show the nature of AARA, while leaving some problems not revealed.

% \textbf{Resource Subtyping:} 
% One problem not mentioned above is about how we check arguments and parameters for function application. AARA introduces resource aware subtyping for this, ensuring the resource of actual arguments is more than that of formal parameters. Usually, subtyping relation $T_1 <: T_2$ states term $t$ typed with $T_1$ can also be typed with $T_2$. However, for resource subtyping, we reverse the direction. $\kwd{list}(\alpha_1) \preceq \kwd{list}(\alpha_2) \iff \alpha_1 \leq \alpha_2$ states list $l$ typed with $\kwd{list}(\alpha_2)$ can also be typed with $\kwd{list}(\alpha_1)$, sacrificing resource $\alpha_2 - \alpha_1$ for each \lstinline|Cons|. With resource subtyping, we can simply process the relation between arguments and parameters.

% \textbf{Type of Substructures:}
% Another problem is about how we connect the resource type between \verb|l| and its substructure \verb|tl|. For linear resource case, their type are just the same; for polynomial case, it should follow the additive shifting operation~\cite{AARA-Poly}. However, linear or polynomial resource analysis is orthogonal to our research. therefore our choice is the simpler one, linear resource analysis. If $\Gamma \vdash l : \kwd{list}(\alpha)$, then we will use $\Gamma' := \Gamma, tl : \kwd{box} ~ \kwd{list}(\alpha)$ as a new context to check \verb|Cons(_, tl) => ...| branch. 

% With two problems handled, when we check $f(\&*tl)$, we should get the tautological constraint $\alpha = \alpha$, the first $\alpha$ from actual argument $\&*tl:\&\kwd{list}(\alpha)$, the second from formal parameter $l:\&\kwd{list}(\alpha)$ of the function $f$.