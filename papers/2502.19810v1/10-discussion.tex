\section{Discussion}
\label{sec:discussion}

In this section, we discuss some limitations of \rarust{} mentioned in \cref{section:introduction} and propose possible pathways towards overcoming them to improve the capability of \rarust{} in future work.

\paragraph{Unsafe code, interior mutability, vectors, reference counting, and cyclic data structures}
%
We focus on safe Rust programs because our design of \rarust{} relies on guarantees provided by Rust's borrow mechanisms, e.g., aliasing and mutation cannot happen simultaneously.
%
However, Rust programs cannot avoid unsafe code in general, because many standard libraries---including cells (\verb|Cell|), vectors (\verb|Vec|), and reference counting (\verb|Rc|)---rely on unsafe code to allow shared mutable states, e.g., interior mutability.
%
The unsafe code can operate C-style pointers in an unrestricted way and compromise Rust's memory safety; as a result, \rarust{}'s resource analysis cannot handle unsafe code.
%
This is actually a common limitation of advanced type systems and verification frameworks for Rust, including Flux~\cite{Flux}, Aeneas~\cite{Aeneas}, and Prusti~\cite{OOPSLA:AMP19}.
%
Nevertheless, there have been efforts to support unsafe code in formal reasoning about Rust programs.
%
RustBelt~\cite{RustBelt} pioneers a line of work on semantic typing and separation-logic-based verification of Rust programs with unsafe code. 
%
Verus~\cite{OOPSLA:LHC23} supports some unsafe features by providing specifications for unsafe memory operations to be memory safe and employing SMT solvers to check those specifications automatically.
%
However, it is unclear if one can integrate AARA type systems with those techniques.

One common method to support unsafe code is based on Rust's design philosophy: unsafe operations should be properly \emph{encapsulated} by safe APIs, and the developers of those unsafe operations take charge of ensuring the unsafe code does not break Rust's memory safety.
%
In terms of type systems, this amounts to assigning types to the safe APIs instead of inferring types from the unsafe code body.
%
Therefore, it would be possible for \rarust{} to analyze Rust programs with unsafe code, if all the unsafe code is encapsulated by resource-annotated safe APIs.
%
Rust's \verb|Cell| features interior mutability by providing operations for both getting and setting the content of a memory location.
%
At the API level, the \verb|Cell| type works similarly to references in an ML-like functional programming language, so it would be possible to adapt an AARA approach for supporting references~\cite{FSCD:LH17}.
%
For example, the resource-annotated APIs shown below can be used to manipulate cells storing lists:
%
\begin{align*}
    \texttt{new}: & \kwd{fn}(\texttt{l}: \kwd{list}(\alpha) ) \to \texttt{Cell<}\kwd{list}(\alpha)\texttt{>} | 0, \\
    \texttt{replace} : & \kwd{fn}(\texttt{self}: \&\texttt{Cell<}\kwd{list}(\alpha)\texttt{>}, \texttt{l}: \kwd{list}(\alpha) ) \to \kwd{list}(\alpha) | 0 ,
\end{align*}
in the sense that the potential type $\kwd{list}(\alpha)$ is an invariant for a cell type, and operations should maintain the invariant, e.g., \verb|replace| should store another list of the same type $\kwd{list}(\alpha)$.
%
Rust's \verb|Vec| also makes use of unsafe code to allow accessing uninitialized memory.
%
At the API level, we can treat vectors as abstract dynamic arrays, which fit nicely into the AARA framework because of their amortized complexity.
%
For example, we can declare the following APIs for integer vectors: 
\begin{align*}
    \texttt{new}: & \kwd{fn}() \to \texttt{Vec<}\kwd{i32},\alpha\texttt{>} | 0, \\
    \texttt{push} : & \kwd{fn}(\texttt{self}: \&\kwd{mut}~\texttt{Vec<}\kwd{i32},\alpha\texttt{>}, \texttt{n}: \kwd{i32} ) \to () | \alpha + 4,
\end{align*}
where the resource-annotated type $\texttt{Vec<}\kwd{i32},\alpha\texttt{>}$ denotes the potential function $\mathit{len} \cdot \alpha + (4 \cdot \mathit{len} - 2 \cdot \mathit{cap})$, with $\mathit{len}$ and $\mathit{cap}$ being the length and capacity of the vector, respectively. 
%
Intuitively, the potential function states that every vector element carries $\alpha$ units of potential and we need to store $(4 \cdot \mathit{len} - 2 \cdot \mathit{cap})$ units of extra potential for vector resizing, which would consume $2 \cdot \mathit{len}$ units of resource to extend the vector's capacity when the vector becomes full.
%

Rust's implementation of \verb|Rc| uses unsafe code, so we would annotate \verb|Rc| APIs with resource-annotated types. \verb|Rc| itself does not permit mutation, so we could model its behavior as if it is a shared reference: \verb|Rc::new()| should store potentials (e.g., \verb|Rc<list(4)>|) and \verb|Rc::clone()| should split potentials (e.g., splitting \verb|Rc<list(4)>| as \verb|Rc<list(1)>| and \verb|Rc<list(3)>|). We have not yet considered multithreading, and supporting \verb|Arc| would be interesting future work.

It is non-trivial to handle cyclic data structures. Rust provides \verb|Weak| pointers to accompany \verb|Rc| pointers. However, creating cyclic data structures usually requires using interior mutability (e.g., \verb|Cell| or \verb|RefCell|). The interaction between reference counting and interior mutability seems quite non-trivial. In the future, we may adapt \citet{ESOP:Atkey10}'s work on integrating AARA with separation logic.

\paragraph{Generic types, higher-order functions (closures), and trait objects}
%
Current \rarust{} does not support generic types like \verb|List<T>|. This is not a fundamental limitation, because we can always instantiate generic types. We will spare engineering efforts to support them in the future.
%

Our work currently only considers top-level functions, but Rust does support higher-order functions and closures to enable functional programming style. 
%
Fortunately, many AARA approaches support higher-order functions~\cite{AARA-HigherOrder,POPL:HDW17,ICFP:KWR20,ICFP:KH21}.
%
Conceptually,
it would be possible to adapt AARA's methodology of handling closures to \rarust{}
by extending the type system to deal with \emph{capturing} properly.
%
One simple extension is to enforce that closures cannot consume potentials stored in captured variables; in this way, it is sound to apply a closure multiple times.
%
It would be interesting future research to investigate how the interaction of borrow mechanisms (especially mutable borrows) and closures would affect AARA-style resource analysis.

Rust supports a form of dynamic dispatch through trait objects, in which the compiler knows an object's trait but not its actual type. One possible workaround is to annotate trait methods with resource annotations and require them not to change the resource type of \verb|Self|. In this way, even though we do not know an object's type, we know how calling its trait methods affects the resource-annotated context. Another possibility is to adapt \citet{AARA-OOP}'s work on integrating AARA with objective-oriented programming. 

\paragraph{Non-linear resource bounds}
%
Both our formalization and implementation of \rarust{} currently only consider linear resource bounds, which are too restrictive to analyze real-world programs.
%
Current \rarust{} can only support pattern matching of one single variable, without primitive support for pattern matching of tuple types, because tuple types usually introduce multivariate polynomial resource bounds like $\textit{first} \times \textit{second}$. 
%
This is not a fundamental limitation because it has been shown that AARA type systems can support polynomial bounds~\cite{AARA-Poly,AARA-Poly-Multivar}, exponential bounds~\cite{AARA-Exp}, logarithmic bounds~\cite{AARA-Log,CAV:LMZ21}, and value-dependent bounds~\cite{ICFP:KWR20,PLDI:KWP19}.
%
All of those approaches amount to devising proper type annotations that specify particular kinds of potential functions and developing constraint-based type-inference algorithms.
%
We plan to spare engineering efforts to extend our prototype implementation of \rarust{} to support various classes of resource bounds in the future.

% \todo{limitation also here, and some idea to solve limitation} 
% The limitation is linear and imprecision. (1) Prototype \rarust{} can only analyze linear bound of resource consumption, as $a + b\cdot n$. However, we can extend the bound to polynomials\cite{AARA-Poly} and \cite{AARA-Poly-Multivar}. (2) Recall the weak updates in \cref{sec:overview}. It will introduce inevitable imprecision. Our implementation introduce it when merging mutable borrows. Though inevitable, it can be delayed to actual writing on borrows, using similar techniques as \cite{CapTypes}. We leave extension as future works.