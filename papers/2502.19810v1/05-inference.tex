\section{Resource Aware Type System and Inference} \label{sec:inference}

In this section, we present the resource-aware type system based on RABC introduced in \cref{sec:calculus} and a type-inference algorithm based on the AARA methodology.
%
\cref{sec:inference:types} introduces resource-enriched types, which augment the types of RABC with resource annotations.
%
% Rich type $\tau$ is plain type $t$ enriched with potential annotation $\alpha$, e.g. $\kwd{list}(\alpha)$ for $\kwd{list}$. Then, we introduce the typing context and its read/write operation used for type checking, followed by the definition of function signatures. 
%
\cref{sec:inference:subtyping} formulates a subtyping relation among resource-enriched types and uses the relation to construct a lattice of types sketched in \cref{sec:overview:Lattice}.
%
% With subtyping, we can tell what rich types are well formed. The subtyping relation over rich types actually forms a lattice with meet and join operations. We define the lattice operations and extend it to the typing context.
%
\cref{sec:inference:eval} and \cref{sec:inference:exec} present the resource-aware typing rules for expressions and statements, respectively.
%
\cref{sec:inference:infer} discusses a type-inference algorithm for the resource-aware type system.

\subsection{Rich Types, Contexts, and Signatures} \label{sec:inference:types}
\begin{figure}[t]
\small
    \begin{align*}
    \tag{undefined} \textbf{RichType}~ \tau &::= \bot \\
    \tag{atom types} &|~ \kwd{i32} ~|~ \kwd{bool} \\
    \tag{list} &|~ \kwd{list}(\alpha)\\
    \tag{box} &|~ \kwd{box}(\kwd{list}(\alpha)) \\
    \tag{shared borrow} &|~ \&^\kwd{s}(\tau) \\
    \tag{mutable borrow} &|~ \&^\kwd{m}(\tau_\text{c}, \tau_\text{p})
    \end{align*}
    \caption{Rich Types}
    \label{fig:rich-type}
\end{figure}
\begin{figure}[t]
\small
    \judgement{Enrich (Selected)}{$\textit{enrich}~ t ~\textit{as}~ \tau$}
    \begin{mathpar}
    \inferrule[Enrich-List]
    {\alpha~\text{fresh}}
    {\textit{enrich}~ \kwd{list} ~\textit{as}~  \kwd{list}(\alpha)}
    \and
    \inferrule[Enrich-Shared]
    {\textit{enrich}~ t ~\textit{as}~ \tau}
    {\textit{enrich}~ \&^\kwd{s}(t) ~\textit{as}~ \&^\kwd{s}(\tau)}
    \and
    \inferrule[Enrich-Mutable]
    {\textit{enrich}~ t ~\textit{as}~ \tau_\text{c}
    \\ \textit{enrich}~ t ~\textit{as}~ \tau_\text{p}
    }
    {\textit{enrich}~ \&^\kwd{m}(t) ~\textit{as}~ \&^\kwd{m}(\tau_\text{c}, \tau_\text{p})}
    \end{mathpar}
    \caption{Enrichment}
    \label{fig:enrich}
\end{figure}

\begin{figure}[t]
\small
    \judgement{Context Reading}{$\Gamma\vdash p \hookrightarrow \tau$}
    \begin{mathpar}
    \inferrule[$\Gamma$-Rd-Var]
    {\Gamma(x)=\tau}
    {\Gamma\vdash x \hookrightarrow \tau}
    \and
    \inferrule[$\Gamma$-Rd-Box]
    {\Gamma\vdash p \hookrightarrow \kwd{box}(\tau)}
    {\Gamma\vdash * p\hookrightarrow \tau}
    \and
    \inferrule[$\Gamma$-Rd-Shared]
    {\Gamma\vdash p \hookrightarrow \&^\kwd{s}(\tau)}
    {\Gamma\vdash *p \hookrightarrow \tau}
    \and
    \inferrule[$\Gamma$-Rd-Mutable]
    {\Gamma\vdash p \hookrightarrow \&^\kwd{m}(\tau_\text{c},\tau_\text{p})}
    {\Gamma\vdash * p\hookrightarrow \tau_\text{c}}    
    \end{mathpar}
    
    \judgement{Context Writing}{$\GWt{\Gamma}{p}{\tau}{\Gamma'}$}
    \begin{mathpar}
    \inferrule[$\Gamma$-Wt-Var]
    {\forall y\not=x, \Gamma'(y)=\Gamma(y) 
    \\ \Gamma'(x) = \tau}
    {\GWt{\Gamma}{x}{\tau}{\Gamma'}}
    \and
    \inferrule[$\Gamma$-Wt-Box]
    {\Gamma\vdash p\hookrightarrow \kwd{box}(\_)
    \\ \GWt{\Gamma}{p}{\kwd{box}(\kwd{list}(\alpha))}{\Gamma'}
    }
    {\GWt{\Gamma}{*p}{\kwd{list}(\alpha)}{\Gamma'}}
    \\
    \inferrule*[Right=\rulename{$\Gamma$-Wt-Shared}]
    {\Gamma\vdash p\hookrightarrow \&^\kwd{s}(\_)
    \\ \GWt{\Gamma}{p}{\&^\kwd{s}(\tau)}{\Gamma'}
    }
    {\GWt{\Gamma}{*p}{\tau}{\Gamma'}}
    \\
    \inferrule*[Right=\rulename{$\Gamma$-Wt-Mutable}]
    {\Gamma\vdash p\hookrightarrow \&^\kwd{m}(\tau_\text{c}, \tau_\text{p})
    \\ \vdash \tau_\text{c}
    \\ \GWt{\Gamma}{p}{\&^\kwd{m}(\tau, \tau_\text{p})}{\Gamma'}
    }
    {\GWt{\Gamma}{*p}{\tau}{\Gamma'}}
    \end{mathpar}
    \caption{Context Reading and Writing}
    \label{fig:sta-rw}
\end{figure}

\begin{figure}[t]
\small
    \judgement{Signatures}{$\vdash f \Rightarrow (\Gamma_f, \delta_f)$}
    \begin{mathpar}
    \inferrule
    {\text{fn}~ f ~(\vec{x}_\text{param}:\vec{t}_\text{param}, \vec{x}_\text{local}:\vec{t}_\text{local}, x_\text{ret}:t_\text{ret}) \{~ s ~\}
    \\\\ \textit{enrich}~ \vec{t}_\text{param} ~\textit{as}~ \vec{\tau}_\text{param}
    \\ \textit{enrich}~ \vec{t}_\text{local} ~\textit{as}~ \vec{\tau}_\text{local}
    \\ \textit{enrich}~ t_\text{ret} ~\textit{as}~ \tau_\text{ret}
    \\\\ \GWt{\emptyset}{\vec{x}_\text{param}}{\vec{\tau}_\text{param}}{\Gamma_1} 
    \\ \GWt{\Gamma_1}{\vec{x}_\text{local}}{\vec{\tau}_\text{local}}{\Gamma_2}
    \\ \GWt{\Gamma_2}{x_\text{ret}}{\tau_\text{ret}}{\Gamma_f}
    \\ \delta_f ~\text{fresh} }
    {\vdash f \Rightarrow (\Gamma_f, \delta_f)}
    \end{mathpar}
    \caption{Function Signatures}
    \label{fig:fun-sig}
\end{figure}

\textbf{Rich types} are types enriched with potential annotation $\alpha$ as in \cref{fig:rich-type} and \cref{fig:enrich}. 
%
The rich type $\bot$ denotes zero potential as the minimum among all rich types. 
%
The rich type $\kwd{list}(\alpha)$, represents the potential function $\alpha\cdot n$ for list $l$ with length $n$.
%
In shared borrows $\&^\kwd{s}(\tau)$, $\tau$ represents the potential function of borrowed value. 
% 
Mutable borrows $\&^\kwd{m}(\tau_\text{c}, \tau_\text{p})$ contains 2 components. $\tau_\text{c}$ is the \textbf{c}urrent type, which denotes the current potential of mutable borrow. $\tau_\text{p}$ is the \textbf{p}rophecy type, which denotes the prophecy potential when the mutable borrow ends.  

Typing \textbf{context} $\Gamma : \mathbf{Variable}\to\mathbf{RichType}$ is a partial map, where unused variables can be mapped to $\bot$. Similarly, in \cref{fig:sta-rw}, we extend the reading and writing operation on typing context from variable $x$ to place $p$. It is worth noting that rules \rulename{$\Gamma$-Rd-Mutable} and \rulename{$\Gamma$-Wt-Mutable} indicate to read and write the mutable borrow on its current component $\tau_\text{c}$. We explicitly point out that $\vdash \tau_\text{c}$ in the premise of rule \rulename{$\Gamma$-Wt-Mutable} is \textbf{dropping condition} for soundness, detailed in \cref{sec:inference:subtyping}. Because when we update $\tau_\text{c}$, the old $\tau_\text{c}$ should be restored if it is a mutable borrow.

\textbf{Signature} $(\Sigma_f, \delta_f)$ of a function $f$ compose a typing context $\Sigma_f$ and a resource unknown variable $\delta_f \in \ZZ$. As shown in \cref{fig:fun-sig}, context $\Gamma_f$ contains rich types for parameters, local variables, and the return variable. $\delta_f$ indicates the resource consumption irrelevant to parameters.

\subsection{Subtyping, Well-formedness, and Merging} \label{sec:inference:subtyping}
\begin{figure}[t]
\small
    \judgement{Subtyping}{$\tau_1 \preceq \tau_2$}
    \begin{mathpar}
    \inferrule*[Right=\rulename{S-Bot}]
    {~}
    {\bot\preceq\tau}
    \and
    \inferrule*[Right=\rulename{S-Int}]
    {~}
    {\kwd{i32}\preceq\kwd{i32}}
    \and
    \inferrule*[Right=\rulename{S-Bool}]
    {~}
    {\kwd{bool}\preceq\kwd{bool}}
    \and
    \inferrule*[Right=\rulename{S-List}]
    {\alpha_1 \leq \alpha_2}
    {\kwd{list}(\alpha_1)\preceq\kwd{list}(\alpha_2)}
    \\
    \inferrule[S-Box]
    {\alpha_1 \leq \alpha_2}
    {\kwd{box}(\kwd{list}(\alpha_1))\preceq\kwd{box}(\kwd{list}(\alpha_2))}
    \and
    \inferrule[S-Shared]
    {\tau_1 \preceq \tau_2}
    {\&^\kwd{s}(\tau_1)\preceq\&^\kwd{s}(\tau_2)}
    \and
    \inferrule[S-Mutable]
    {\tau_{\text{c}, 1}\preceq \tau_{\text{c}, 2}
    \\ \tau_{\text{p}, 2}\preceq \tau_{\text{p}, 1} }
    {\&^\kwd{m}(\tau_{\text{c}, 1}, \tau_{\text{p}, 1})\preceq\&^\kwd{m}(\tau_{\text{c}, 2}, \tau_{\text{p}, 2})}
    \end{mathpar}
    \caption{Rich Subtyping}
    \label{fig:rich-subtyping}
\end{figure}
\begin{figure}[t]
\small
    \judgement{Well-formedness}{$\vdash \tau$}
    \begin{mathpar}

    \inferrule[WF-Bot]
    {~}
    {\vdash \bot}
    \and
    \inferrule[WF-Int]
    {~}
    {\vdash \kwd{i32}}
    \and
    \inferrule[WF-Bool]
    {~}
    {\vdash \kwd{bool}}
    \and
    \inferrule[WF-List]
    {\alpha \geq 0}
    {\vdash \kwd{list}(\alpha)}
    \and
    \inferrule[WF-Box]
    {\alpha \geq 0}
    {\vdash \kwd{box}(\kwd{list}(\alpha))}
    \\
    \inferrule*[Right=\rulename{WF-Shared}]
    {\vdash \tau
    }
    {\vdash \&^\kwd{s}(\tau)}
    \and
    \inferrule*[Right=\rulename{WF-Mutable}]
    {\tau_\text{p} \preceq \tau_\text{c}
    \\ \vdash \tau_\text{c}
    \\ \vdash \tau_\text{p}
    }
    {\vdash \&^\kwd{m}(\tau_\text{c}, \tau_\text{p})}
    \end{mathpar}
    \caption{Well-formedness}
    \label{fig:rich-type-wf}
\end{figure}
\begin{figure}[t]
\small
    \judgement{Context Merging}{$\Gamma_1 \sqcap \Gamma_2 = \{ x \hookrightarrow \Gamma_1(x)\cap\Gamma_2(x) : x \in \mathbf{dom}(\Gamma_1)=\mathbf{dom}(\Gamma_2)\}$}
    \judgement{Meet/Join (Selected)}{$\tau_1\cap\tau_2 / \tau_1\cup\tau_2$}
    \begin{mathpar}        
    \inferrule*[Right=Meet-List]
    {\min(\alpha_1, \alpha_2)=\alpha}
    {\kwd{list}(\alpha_1)\cap\kwd{list}(\alpha_2)=\kwd{list}(\alpha)}
    \and
    \inferrule*[Right=Join-List]
    {\max(\alpha_1, \alpha_2)=\alpha}
    {\kwd{list}(\alpha_1)\cup\kwd{list}(\alpha_2)=\kwd{list}(\alpha)}
    \\
    
    \inferrule*[Right=Meet-Shared]
    {\tau_1 \cap \tau_2=\tau}
    {\&^\kwd{s}(\tau_1)\cap\&^\kwd{s}(\tau_2)=\&^\kwd{s}(\tau)}
    \and
    \inferrule*[Right=Join-Shared]
    {\tau_1 \cup \tau_2=\tau}
    {\&^\kwd{s}(\tau_1)\cup\&^\kwd{s}(\tau_2)=\&^\kwd{s}(\tau)}
    \\

    \inferrule*[Right=Meet-Mutable]
    {\tau_{\text{c}, 1} \cap \tau_{\text{c}, 2}=\tau_\text{c}
    \\ \tau_{\text{p}, 1} \cup \tau_{\text{p}, 2}=\tau_\text{p}
    \\ \tau_{\text{p}, 1} \preceq \tau_{\text{c}, 1}
    \\ \tau_{\text{p}, 2} \preceq \tau_{\text{c}, 2}
    }
    {\&^\kwd{m}(\tau_{\text{c}, 1}, \tau_{\text{p}, 1})\cap\&^\kwd{m}(\tau_{\text{c}, 2}, \tau_{\text{p}, 2})=\&^\kwd{m}(\tau_\text{c}, \tau_\text{p})}
    \\
    \inferrule*[Right=Join-Mutable]
    {\tau_{\text{c}, 1} \cup \tau_{\text{c}, 2}=\tau_\text{c}
    \\ \tau_{\text{p}, 1} \cap \tau_{\text{p}, 2}=\tau_\text{p}
    \\ \tau_{\text{p}, 1} \preceq \tau_{\text{c}, 1}
    \\ \tau_{\text{p}, 2} \preceq \tau_{\text{c}, 2}
    }
    {\&^\kwd{m}(\tau_{\text{c}, 1}, \tau_{\text{p}, 1})\cup\&^\kwd{m}(\tau_{\text{c}, 2}, \tau_{\text{p}, 2})=\&^\kwd{m}(\tau_\text{c}, \tau_\text{p})}
    \end{mathpar}
    \caption{Merging}
    \label{fig:sta-merging}
\end{figure}

The order relation $\leq$ on resources derives another order relation on rich types, the \emph{subtyping} relation in \cref{fig:rich-subtyping}. The interpretation of subtyping $\tau_1 \preceq \tau_2$ is that the value $v$ typed with $\tau_1$ has \textbf{less} resource than the value $v$ typed with $\tau_2$. 
%
The rich type $\bot$ is a subtype of any type because $\bot$ denotes zero potential.
%
It is worth noting that in \rulename{S-Mutable}, $\tau_\text{p}$ is contravariant because prophecy type $\tau_\text{p}$ denotes the prophecy potential to return. 
%
The reflexive rule and the transitive rule are derivable.

A well-formed rich type always denotes a non-negative potential function. Our type system can drop well-formed types without sacrificing soundness.
%
\rulename{WF-List} and \rulename{WF-Box} request $\alpha \geq 0$, which makes $\alpha \cdot n \geq 0$ for list $l$ with length $n\geq0$. 
%
\rulename{WF-Shared} is a structural rule. For example, if $\kwd{list}(\alpha)$ is well-formed, so is $\&^\kwd{s}(\kwd{list}(\alpha))$. Rust borrow checker ensures that $\tau$ in $\&^\kwd{s}(\tau)$ satisfies $\tau\not=\&^\kwd{m}(\_, \_)$. Our type system supports nested borrows.
%
Besides structural premises $\vdash \tau_\text{c}$ and $\vdash \tau_\text{p}$, \rulename{WF-Mutable} demands \textbf{dropping condition} $\tau_\text{p} \preceq \tau_\text{c}$ in $\&^\kwd{m}(\tau_\text{c}, \tau_\text{p})$. The condition is called dropping condition because it works as dropping mutable borrows in \cref{fig:ex-prophecy}. The dropping condition makes sure that mutable borrow types denote non-negative potentials, as illustrated in \cref{sec:soundness}. 


\textbf{Merging} is a conservative approximation of resource potentials after conditional branching. Under typing context $\Gamma$, the type system checks statements $s_1$ and $s_2$ in different branches and gets remainder contexts $\Gamma_1$ and $\Gamma_2$.
The type system should merge them to check continuation. 
%
As illustrated in \cref{fig:sta-merging}, to merge typing contexts is to merge rich types at each $x$ in the domain of two contexts. 
% 
Because the prophecy type $\tau_\text{p}$ in mutable borrow is contravariant, we need to define not only the meet of types but also the join of types. Our purpose is to construct a \emph{lattice} with the property that $\tau_1\cap\tau_2\preceq \tau_i \preceq\tau_1\cup\tau_2, \forall i=1, 2$. Hence, merging over contexts is non-increasing on resources to conservatively fulfill soundness. 
%
The lattice operations of $\kwd{list}(\alpha_1)$ and $\kwd{list}(\alpha_1)$ are inherited from the resource's $\min$ and $\max$, so it is readily comprehensible. 
%
Notice that dropping conditions appear in rules \rulename{Meet-Mutable} and \rulename{Join-Mutable}. They are to fulfill soundness for weak updates, which is mentioned in \cref{sec:overview:Lattice}. Recall that dropping borrows without these conditions may increase resources in both original places indicated by $\tau_{\text{p}, 1}$ and $\tau_{\text{p}, 2}$, to break soundness. 

% In practical implementation, the well-formedness here $\vdash\&^\kwd{m}(\tau_{\text{c}, i}, \tau_{\text{p}, i})$ can be exchanged to $\tau_{\text{p}, i}\preceq\tau_{\text{c}, i}$, due to redundant well-formedness for substructures, $\vdash\tau_{\text{c}, i}$ and $\vdash\tau_{\text{p}, i}$ already covered by induction hypothesis on $\tau_{\text{c}, 1}\cap/\cup\tau_{\text{c}, 2}$ and $\tau_{\text{c}, 1}\cup/\cap\tau_{\text{p}, 2}$. 

It is worth noting that we support nested borrows like $\&^\kwd{s}(\&^\kwd{s}(\tau))$, $\&^\kwd{m}(\&^\kwd{s}(\tau_\text{c}), \&^\kwd{s}(\tau_\text{p}))$ and $\&^\kwd{m}(\&^\kwd{m}(\tau_\text{cc}, \tau_\text{cp}), \&^\kwd{m}(\tau_\text{cc}, \tau_\text{pp}))$. Rust's borrow mechanisms exclude nested borrows like shared borrows of mutable borrows $\&^\kwd{s}(\&^\kwd{m}(\tau_\text{c}, \tau_\text{p}))$, because they violate the property that at most one mutable borrow from the same piece of data is live at the same time.

\subsection{Typing Expressions} \label{sec:inference:eval}\
\begin{figure}[t]
\small
\judgement{Typing Expressions (Selected)}{$\Gamma\vdash e \hookrightarrow \tau\dashv\Gamma'$}
    \begin{mathpar}
    \inferrule*[Right=\rulename{$\Gamma$-Ev-Nil}]
    {\alpha ~\text{fresh}}
    {\Gamma\vdash \kwd{nil} \hookrightarrow \kwd{list}(\alpha)\vdash\Gamma}
    \and
    \inferrule*[Right=\rulename{$\Gamma$-Ev-Move}]
    {\Gamma\vdash p \hookrightarrow \tau
    \\ \GWt{\Gamma}{p}{\bot}{\Gamma'}
    }
    {\Gamma\vdash \kwd{move}~p \hookrightarrow \tau\dashv\Gamma'}
    \\

    \inferrule*[Right=\rulename{$\Gamma$-Ev-Shared}]
    {\Gamma\vdash p \hookrightarrow \tau
    \\ \textit{share}~ \tau ~\textit{as}~\tau_1, \tau_2
    \\ \GWt{\Gamma}{p}{\tau_1}{\Gamma'}
    }
    {\Gamma\vdash \&^\kwd{s}~p \hookrightarrow \&^\kwd{s}(\tau_2)\dashv\Gamma'}
    \\
    
    \inferrule*[Right=\rulename{$\Gamma$-Ev-Mutable}]
    {\Gamma\vdash p \hookrightarrow \tau
    \\ \textit{prophesy}~ \tau ~\textit{as}~ \tau_\text{p} 
    \\ \GWt{\Gamma}{p}{\tau_\text{p}}{\Gamma'}
    }
    {\Gamma\vdash \&^\kwd{m}~p \hookrightarrow \&^\kwd{m}(\tau, \tau_\text{p})\dashv\Gamma'}
    \end{mathpar}
    \caption{Typing Expressions}
    \label{fig:sta-eval}
\end{figure}

\begin{figure}[t]
\small
    \judgement{Sharing (Selected)}{$\textit{share}~ \tau ~\textit{as}~\tau_1, \tau_2$}
    \begin{mathpar}
    \inferrule*[Right=\rulename{Share-List}]
    {\alpha_1, \alpha_2 ~\text{fresh}
    \\\alpha = \alpha_1 + \alpha_2}
    {\textit{share}~ \kwd{list}(\alpha) ~\textit{as}~\kwd{list}(\alpha_1), \kwd{list}(\alpha_2)}
    \end{mathpar}
    \judgement{Prophesying (Selected)}{$\textit{prophesy}~ \tau_\text{c} ~\textit{as}~\tau_\text{p}$}
    \begin{mathpar}
    \inferrule*[Right=\rulename{Prophesy-List}]
    {\alpha_\text{p}~\text{fresh}}
    {\textit{prophesy}~ \kwd{list}(\alpha) ~\textit{as}~ \kwd{list}(\alpha_\text{p})}
    \end{mathpar}
    \caption{Sharing and Prophesying}
    \label{fig:sta-sharing-prophesying}
\end{figure}

\cref{fig:sta-eval} presents how to type check expressions via judgement $\Gamma\vdash e\hookrightarrow \tau\dashv\Gamma'$. Unlike the dynamic evaluation $V\vdash e \rightsquigarrow v$, checking expressions may modify $\Gamma$ to the remainder context $\Gamma'$. 
%
Rule \rulename{$\Gamma$-Ev-Nil} introduces a fresh unknown potential annotation $\alpha$ for $\kwd{nil}$. 
%
Rule \rulename{$\Gamma$-Ev-Move} explicitly moves the type $\tau$ out from place $p$, making $\GWt{\Gamma}{p}{\bot}{\Gamma'}$. 
%

Shared and mutable borrows modify typing context, as illustrated in rule \rulename{$\Gamma$-Ev-Shared} and \rulename{$\Gamma$-Ev-Mutable} with sharing and prophesying. \cref{fig:sta-sharing-prophesying} selects essential rules of $\textit{share}~ \tau ~\textit{as}~ \tau_1, \tau_2$ and $\textit{prophesy}~ \tau ~\textit{as}~ \tau_\text{p}$ for borrows. 

\textbf{Shared borrows} are handled with sharing $\textit{share}~ \tau ~\textit{as}~ \tau_1, \tau_2$. Recall the example in \cref{fig:ex-sharing}. We select the rule \rulename{Share-List} to reveal the essence of sharing. Sharing is splitting resource annotation $\alpha$ into $\alpha_1$ and $\alpha_2$ with linear constraint $\alpha = \alpha_1 + \alpha_2$. In rule \rulename{$\Gamma$-Ev-Shared}, we write $\tau_1$ back to original place $p$, with $\tau_2$ lent out. There is no sharing of mutable borrows as $\textit{share}~ \&^\kwd{m}(\_, \_) ~\textit{as}~ \tau_1, \tau_2$, because a well-checked program will never incur shared borrows of mutable borrows $\&^\kwd{s}(\&^\kwd{m}(\tau_\text{c}, \tau_\text{p}))$.

\textbf{Mutable borrows} are handled with prophesying $\textit{prophesy}~ \tau ~\textit{as}~ \tau_\text{p}$. Recall the example in \cref{fig:ex-prophecy}. The selected rule \rulename{Prophesy-List} prophesy $\alpha_\text{p}$ as the prophecy potential when the mutable borrow ends. In rule \rulename{$\Gamma$-Ev-Mutable}, we write prophecy type $\tau_\text{p}$ to the place $p$. Once the borrow ends, the dropping condition $\vdash \&^\kwd{m}(\tau, \tau_\text{p})$ ensures that the prophecy type $\tau_\text{p}$ is bounded by current type $\tau$.

\subsection{Typing Statements} \label{sec:inference:exec}
\begin{figure}[t]
\small
    \judgement{Typing Statements (Selected)}{$\Gamma\vdash s \hookrightarrow^\delta \dashv\Gamma'$}
    \begin{mathpar}
    \inferrule*[Right=\rulename{$\Gamma$-Ex-Tick}]
    {~}
    {\Gamma\vdash\kwd{tick}(\delta)\hookrightarrow^\delta\vdash\Gamma}
    \and
    \inferrule*[Right=\rulename{$\Gamma$-Ex-Drop}]
    {\Gamma\vdash p\hookrightarrow \tau
    \\ \vdash \tau
    \\ \GWt{\Gamma}{p}{\bot}{Gamma'}
    }
    {\Gamma\vdash \kwd{drop}~p \hookrightarrow^0\dashv \Gamma'}
    \\

    \inferrule*[Right=\rulename{$\Gamma$-Ex-Cons}]
    {\Gamma\vdash e_1\hookrightarrow \kwd{i32} \dashv \Gamma_1
    \\ \Gamma_1\vdash e_2\hookrightarrow \kwd{box}(\kwd{list}(\alpha'))\dashv\Gamma_2
    \\ \GWt{\Gamma_2}{p}{\kwd{list}(\alpha')}{\Gamma'}}
    {\Gamma\vdash p\from \kwd{cons}(e_1, e_2)\hookrightarrow^{\alpha'}\dashv\Gamma'}
    \\

    \inferrule*[Right=\rulename{$\Gamma$-Ex-If}]
    {\Gamma\vdash p\hookrightarrow \kwd{bool}
    \\ \Gamma\vdash s_1\hookrightarrow^{\delta_1}\dashv\Gamma_1
    \\ \Gamma\vdash s_2\hookrightarrow^{\delta_2}\dashv\Gamma_2
    \\ \max(\delta_1, \delta_2)=\delta
    \\ \Gamma_1\sqcap\Gamma_2=\Gamma' }
    {\Gamma\vdash \kwd{if}~ p ~\kwd{then}~ s_1 ~\kwd{else}~ s_2 ~\kwd{end} \hookrightarrow^\delta \dashv\Gamma'}
    \\
    \inferrule*[Right=\rulename{$\Gamma$-Ex-Mat}]
    {\Gamma\vdash p\hookrightarrow \kwd{list}(\alpha)
    \\ \Gamma\vdash s_1\hookrightarrow^{\delta_1}\dashv\Gamma_1
    \\\\ \GWt{\Gamma}{p}{\bot}{\Gamma_{\text{b}, 1}}
    \\ \GWt{\Gamma_{\text{b}, 1}}{x_\text{hd}}{\kwd{i32}}{\Gamma_{\text{b}, 2}}
    \\ \GWt{\Gamma_{\text{b}, 2}}{x_\text{tl}}{\kwd{box}(\kwd{list}(\alpha))}{\Gamma_\text{b}}
    \\ \Gamma_\text{b}\vdash s_2\hookrightarrow^{\delta_2}\dashv\Gamma'_\text{b}
    \\\\ \Gamma'_\text{b}\vdash x_\text{tl}\hookrightarrow \kwd{list}(\beta)
    \\ \GWt{\Gamma'_\text{b}}{x_\text{hd}}{\bot}{\Gamma'_{\text{b}, 1}}
    \\ \GWt{\Gamma'_{\text{b}, 1}}{x_\text{tl}}{\bot}{\Gamma'_{\text{b}, 2}}
    \\ \GWt{\Gamma'_{\text{b}, 2}}{p}{\kwd{list}(\beta)}{\Gamma_2}
    \\\\ \max(\delta_1, \delta_2-(\alpha-\beta))=\delta
    \\ \Gamma_1\sqcap\Gamma_2=\Gamma'}
    {\Gamma\vdash \kwd{match}~ p ~ \{\kwd{nil}\mapsto s_1, \kwd{cons}(x_\text{hd}, x_\text{tl})\mapsto s_2\} \hookrightarrow^\delta \dashv\Gamma'}
    \\

    \inferrule*[Right=\rulename{$\Gamma$-Ex-App}]
    {\text{fn}~ f ~(\vec{x}_\text{param}:\vec{t}_\text{param}, \vec{x}_\text{local}:\vec{t}_\text{local}, x_\text{ret}:t_\text{ret}) \{~ s ~\}
    \\\\ \vdash f \Leftarrow (\Gamma_f, \delta_f)
    \\ \Gamma_f\vdash x_\text{ret} \hookrightarrow \tau_\text{ret}, (\forall x_i\in\vec{x}_\text{param}, i=1, ..., n) \Gamma_f \vdash x_i \hookrightarrow \tau_{\text{param}, i}
    \\\\ \Gamma_0=\Gamma, (\forall e_i\in \vec{e}, i=1, ..., n) \Gamma_{i-1}\vdash e_i\hookrightarrow\tau_{\text{arg}, i}\dashv\Gamma_i
    \\ (\forall i=1,..,n)~ \tau_{\text{param}, i} = \tau_{\text{arg}, i}
    \\ \Gamma_n \vdash p \hookrightarrow \tau
    \\ \vdash \tau
    \\ \GWt{\Gamma_n}{p}{\tau_\text{ret}}{\Gamma'}
    }
    {\Gamma\vdash p\from f(\vec{e})\hookrightarrow^{\delta_f}\dashv\Gamma'}
    \end{mathpar}
    \caption{Typing Statements}
    \label{fig:sta-exec}
\end{figure}

\begin{figure}[t]
\small
    \judgement{Function Analysis}{$\vdash f \Leftarrow (\Gamma_f, \delta_f)$}
    \begin{mathpar}
    \inferrule
    {\text{fn}~ f ~(\vec{x}_\text{param}:\vec{t}_\text{param}, \vec{x}_\text{local}:\vec{t}_\text{local}, x_\text{ret}:t_\text{ret}) \{~ s ~\}
    \\  \vdash f \Rightarrow (\Gamma_f, \delta_f)
    \\ \Gamma_f\vdash s\hookrightarrow^\delta\dashv\Gamma'_f
    \\\\ \forall x \in \textbf{dom}(\Gamma'_f), \vdash \Gamma'_f(x)
    \\ \Gamma'_f \vdash x_\text{ret} \hookrightarrow \tau'_\text{ret}
    \\ \Gamma_f \vdash x_\text{ret} \hookrightarrow \tau_\text{ret}
    \\ \tau'_\text{ret} = \tau_\text{ret}
    \\ \delta = \delta_f}
    {\vdash f \Leftarrow (\Gamma_f, \delta_f)}
    \end{mathpar}
    \caption{Function Analysis}
    \label{fig:fun-anal}
\end{figure}

\cref{fig:sta-exec} presents how to type check statements as judgement $\Gamma\vdash s \hookrightarrow^\delta \dashv\Gamma'$. Under context $\Gamma$, the statement $s$ is checked with resource consumption $\delta$, and context becomes $\Gamma'$. 

Rule \rulename{$\Gamma$-Ex-Tick} indicates $\kwd{tick}(\delta)$ consumes $\delta$ unit of resource. Rule \rulename{$\Gamma$-Ex-Drop} drops the type $\tau$ with well-formedness $\vdash\tau$ as the dropping condition. Rule \rulename{$\Gamma$-Ex-Cons} indicates that $\kwd{cons}$ will consume $\alpha$ unit of resource for continuation payment, when the tail $e_2$ is typed with $\kwd{box}(\kwd{list}(\alpha))$. 

\textbf{Branching statements} require context merging, as in \rulename{$\Gamma$-Ex-If} and \rulename{$\Gamma$-Ex-Mat}. Rule \rulename{$\Gamma$-Ex-If} is simpler to merge contexts with the consumption as the maximum of those branches. Rule \rulename{$\Gamma$-Ex-Mat} is more intricate, due to resource potential stored in \kwd{cons}. The $\kwd{cons}$ branch will obtain $\alpha-\beta$ units of potential, therefore the net consumption is $\delta_2-(\alpha-\beta)$. Given $\Gamma\vdash p \hookrightarrow \kwd{list}(\alpha)$, The potential is not $\alpha$ but $\alpha-\beta$. $\beta$ is the remainder potential, indicated by $\Gamma'_\text{b} \vdash x_\text{tl} \hookrightarrow \kwd{list}(\beta)$. The subscript $\text{b}$ of $\Gamma_\text{b}$ means \textbf{b}inding, similar to rule \rulename{V-Ex-Mat}.

\textbf{Function application} is intractable because of recursive functions. Rule \rulename{$\Gamma$-Ex-App} assumes the function $f$ has a well-checked signature $(\Gamma_f, \delta_f)$, with judgement $\vdash f \Leftarrow (\Gamma_f, \delta_f)$ in \cref{fig:fun-anal}, different from $\vdash f \Rightarrow (\Gamma_f, \delta_f)$. Other premises are to ensure that the resources of actual arguments are equal to those of formal parameters. 

\subsection{Type Inference} \label{sec:inference:infer}
To this point, our type system has been primarily declarative because the well-checked signature in rule \rulename{$\Gamma$-Ex-App} is assumed to be pre-existent. Same as other AARA systems (such as Resource-aware ML~\cite{RaML}), we use linear programming to convert the declarative type system to an algorithmic version. The type system creates symbolic variables to denote unknown annotations in rich types and signatures. The type system then collects linear constraints among those symbolic variables and finally solves them via linear programming solvers. 

Readers might have perceived that a recursive function requires a well-checked signature during checking and that a function can exhibit multiple signatures at different call sites. To automatically analyze functions, we need to preprocess the call graph. First, we group recursive functions as strongly connected components. Second, we topologically sort groups to determine an order to analyze. For each group, we predefine signatures of functions in the group via the judgement $\textit{enrich}~ t ~\textit{as}~ \tau$ in \cref{fig:enrich}. During function analysis, the signature $(\Gamma_f, \delta_f)$ in \rulename{$\Gamma$-Ex-App} should be replaced with the predefined one if $f$ is in the group. Otherwise, $f$ is in the previously analyzed group, so we should clone that group's signature and linear constraints. It is necessary to clone instead of copy them because annotations in signatures and constraints are sensitive to actual arguments of function calls.

With linear constraints collected during function analysis and a heuristic objective, we can employ a linear programming solver to find instances of annotations that satisfy those constraints automatically. The inferred annotations in signatures will characterize functions' resource consumption. 

% The $\vdash f \Rightarrow (\Gamma_f, \delta_f)$ in premises of $\vdash f \Leftarrow (\Gamma_f, \delta_f)$, and function application in body $s$, require a topological order of functions to analyze. And (mutual) recursive functions require a strongly connected grouping of functions. In our implementation, all these requirements turn into strongly connected component analysis and topological sorting.