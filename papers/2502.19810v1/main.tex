%%
%% This is file `sample-manuscript.tex',
%% generated with the docstrip utility.
%%
%% The original source files were:
%%
%% samples.dtx  (with options: `manuscript')
%% 
%% IMPORTANT NOTICE:
%% 
%% For the copyright see the source file.
%% 
%% Any modified versions of this file must be renamed
%% with new filenames distinct from sample-manuscript.tex.
%% 
%% For distribution of the original source see the terms
%% for copying and modification in the file samples.dtx.
%% 
%% This generated file may be distributed as long as the
%% original source files, as listed above, are part of the
%% same distribution. (The sources need not necessarily be
%% in the same archive or directory.)
%%
%% Commands for TeXCount
%TC:macro \cite [option:text,text]
%TC:macro \citep [option:text,text]
%TC:macro \citet [option:text,text]
%TC:envir table 0 1
%TC:envir table* 0 1
%TC:envir tabular [ignore] word
%TC:envir displaymath 0 word
%TC:envir math 0 word
%TC:envir comment 0 0
%%
%%
%% The first command in your LaTeX source must be the \documentclass command. This is the generic manuscript mode required for submission and peer review.
\documentclass[acmsmall,screen,nonacm]{acmart} % review,anonymous,
%% To ensure 100% compatibility, please check the white list of
%% approved LaTeX packages to be used with the Master Article Template at
%% https://www.acm.org/publications/taps/whitelist-of-latex-packages 
%% before creating your document. The white list page provides 
%% information on how to submit additional LaTeX packages for 
%% review and adoption.
%% Fonts used in the template cannot be substituted; margin 
%% adjustments are not allowed.

%%
%% \BibTeX command to typeset BibTeX logo in the docs
\AtBeginDocument{%
  \providecommand\BibTeX{{%
    \normalfont B\kern-0.5em{\scshape i\kern-0.25em b}\kern-0.8em\TeX}}}

%% Rights management information.  This information is sent to you
%% when you complete the rights form.  These commands have SAMPLE
%% values in them; it is your responsibility as an author to replace
%% the commands and values with those provided to you when you
%% complete the rights form.
\setcopyright{acmlicensed}
\copyrightyear{2024}
\acmYear{2024}
\acmDOI{XXXXXXX.XXXXXXX}

%% These commands are for a PROCEEDINGS abstract or paper.
% \acmConference[Conference acronym 'XX]{Make sure to enter the correct
%   conference title from your rights confirmation emai}{June 03--05,
%   2018}{Woodstock, NY}
%
%  Uncomment \acmBooktitle if th title of the proceedings is different
%  from ``Proceedings of ...''!
%
%\acmBooktitle{Woodstock '18: ACM Symposium on Neural Gaze Detection,
%  June 03--05, 2018, Woodstock, NY} 

%% These commands are for a JOURNAL article.
\acmJournal{JACM}
\acmVolume{37}
\acmNumber{4}
\acmArticle{111}
\acmMonth{10}

\acmISBN{978-1-4503-XXXX-X/18/06}


%%
%% Submission ID.
%% Use this when submitting an article to a sponsored event. You'll
%% receive a unique submission ID from the organizers
%% of the event, and this ID should be used as the parameter to this command.
%%\acmSubmissionID{123-A56-BU3}

%%
%% For managing citations, it is recommended to use bibliography
%% files in BibTeX format.
%%
%% You can then either use BibTeX with the ACM-Reference-Format style,
%% or BibLaTeX with the acmnumeric or acmauthoryear sytles, that include
%% support for advanced citation of software artefact from the
%% biblatex-software package, also separately available on CTAN.
%%
%% Look at the sample-*-biblatex.tex files for templates showcasing
%% the biblatex styles.
%%

%%
%% The majority of ACM publications use numbered citations and
%% references.  The command \citestyle{authoryear} switches to the
%% "author year" style.
%%
%% If you are preparing content for an event
%% sponsored by ACM SIGGRAPH, you must use the "author year" style of
%% citations and references.
%% Uncommenting
%% the next command will enable that style.
%%\citestyle{acmauthoryear}

%%
%% end of the preamble, start of the body of the document source.

\usepackage{multicol}
\usepackage{multirow}
\usepackage{graphicx}
\usepackage{arydshln}
\usepackage{listings, listings-rust}
\usepackage{xspace}
\usepackage{cleveref}
\crefname{theorem}{Thm.}{Thms.}
\crefname{lemma}{Lem.}{Lemmas}
\crefname{corollary}{Cor.}{Cors.}
\crefname{figure}{Fig.}{Figs.}
\crefname{definition}{Defn.}{Defns.}
\crefname{table}{Tab.}{Tabs.}
\crefname{appendix}{Appendix}{Appendices}
\crefformat{section}{\S#2#1#3}
\crefmultiformat{section}{\S#2#1#3}{ and~\S#2#1#3}{, \S#2#1#3}{ and~\S#2#1#3}
\crefname{example}{Ex.}{Exs.}
\crefname{item}{item}{items}
\crefname{footnote}{footnote}{footnotes}
\crefname{observation}{Obs.}{Obs.}
\crefname{remark}{Remark}{Remarks}
\crefname{proposition}{Prop.}{Props.}
\crefname{equation}{Eqn.}{Eqns.}
\crefname{counterexample}{Counterexample}{Counterexamples}
\crefname{property}{Property}{Properties}
\crefname{algorithm}{Algorithm}{Algorithms}
\usepackage{subcaption}
\usepackage[shortlabels]{enumitem}
\usepackage{commands}
\usepackage{mathpartir}
\newcommand{\from}{:=}




\def\vDash{\vdash}
\def\Dashv{\dashv}

% \def\VWt#1#2#3#4{#4=#1\cup\{#2 \rightsquigarrow #3\}}
\def\VWt#1#2#3#4{\{#1\}#2 \from #3\{#4\}}
% \def\GWt#1#2#3#4{#4=#1\cup\{#2 \hookrightarrow #3\}}
\def\GWt#1#2#3#4{\{#1\}#2 \from #3\{#4\}}




\lstset{
  basicstyle=\ttfamily,
}
\def\kwd#1{\textbf{\texttt{#1}}}
\def\todo#1{\textcolor{red}{\textbf{#1}}}

\citestyle{acmauthoryear}


\AtBeginDocument{%
  \setlength\abovedisplayskip{0.5\abovedisplayskip}%
  \setlength\belowdisplayskip{0.5\belowdisplayskip}%
  \setlength\abovedisplayshortskip{0.5\abovedisplayshortskip}%
  \setlength\belowdisplayshortskip{0.5\belowdisplayshortskip}%
  \setlength\floatsep{0.5\floatsep}%
  \def\MathparLineskip{\lineskip=0.29cm}%
  \setlength\abovecaptionskip{0.5\abovecaptionskip}%
}
\setlist{leftmargin=*}

\newenvironment{DIFnomarkup}{}{}

\sloppy
\begin{document}

%%
%% The "title" command has an optional parameter,
%% allowing the author to define a "short title" to be used in page headers.
% \title{Automatic Amortized Resource Analysis with Borrow Mechanism}
% \title{Towards Automatic Amortized Resource Analysis for Rust}
\title{Automatic Linear Resource Bound Analysis for Rust via Prophecy Potentials}

%%
%% The "author" command and its associated commands are used to define
%% the authors and their affiliations.
%% Of note is the shared affiliation of the first two authors, and the
%% "authornote" and "authornotemark" commands
%% used to denote shared contribution to the research.
\author{Qihao Lian}
\affiliation{%
  \department{Key Laboratory of High Confidence Software Technologies (Peking University), Ministry of Education; School of Computer Science}
  \institution{Peking University}
  \country{China}
}
\email{mepy@stu.pku.edu.cn}
\author{Di Wang}
\affiliation{%
  \department{Key Laboratory of High Confidence Software Technologies (Peking University), Ministry of Education; School of Computer Science}
  \institution{Peking University}
  \country{China}
}
\email{wangdi95@pku.edu.cn}
%%
%% By default, the full list of authors will be used in the page
%% headers. Often, this list is too long, and will overlap
%% other information printed in the page headers. This command allows
%% the author to define a more concise list
%% of authors' names for this purpose.
%% \renewcommand{\shortauthors}{Q. Lian and D. Wang}

%%
%% The abstract is a short summary of the work to be presented in the
%% article.
Humor is a social binding agent. It is an act of creativity that can provoke emotional reactions on a broad range of topics. Humor has long been thought to be “too human” for AI to generate. However, humans are complex, and humor requires our complex set of skills: cognitive reasoning, social understanding, a broad base of knowledge, creative thinking, and audience understanding. We explore whether giving AI such skills enables it to write humor. We target one audience: Gen Z humor fans. We ask people to rate meme caption humor from three sources: highly upvoted human captions, 2) basic LLMs, and 3) LLMs captions with humor skills. We find that users like LLMs captions with humor skills more than basic LLMs and almost on par with top-rated humor written by people. We discuss how giving AI human-like skills can help it generate communication that resonates with people. 



%\received{20 February 2007}
%\received[revised]{12 March 2009}
%\received[accepted]{5 June 2009}

%%
%% This command processes the author and affiliation and title
%% information and builds the first part of the formatted document.
\maketitle


The increasing reliance on LLMs for multimodal tasks across far-reaching sectors such as healthcare, finance, and manufacturing underscores the need to assess the accuracy and reliability of the information they generate. Vision-Language Models (VLM) have achieved state-of-the-art (SoTA) performance on Visual Question-Answering (VQA) benchmarks, and these models often utilize Retrieval-Augmented Generation (RAG) to maintain factual accuracy and relevance in a dynamic information environment. However, this has led to uncertainty in the information the LLM bases its answer on, as it may choose between parametric memory and retrieved sources. When models rely on memorized information instead of dynamically retrieving information, they may inadvertently propagate outdated or incorrect information, causing serious legal and ethical risks and undermining trust and reliability in AI systems \citep{huang2023survey}.
% The ability to strike a balance between generalization and specialization in AI systems is therefore crucial for ensuring the safe, reliable use of these technologies in real-world applications.

Despite these concerns, the way that Vision-Language models (VLMs) memorize and retrieve information, particularly in complex multimodal tasks, remains under-explored. Current research often focuses on either the general capabilities of large language models (LLMs) or the specialized retrieval mechanisms in retrieval augmented generation systems (RAG) \citep{incontext_rag,chen_murag_2022,liu_universal_2023}. Particularly in the context of multimodal retrieval and multihop reasoning, few studies analyze the tradeoff between finetuning for specialized tasks and zero-shot prompting for general-purpose vision-language capabilities. A lack of consensus on how to approach this tradeoff motivates the development of measures to quantify reliance on parametric memory, as well as metrics for quantifying the potential performance impact of extending LLMs with RAG systems.

To address this gap, we investigate how multimodal QA models balance accuracy with memorization on the WebQA benchmark. We compare finetuned multimodal systems against zero-shot VLMs, analyzing how retrieval performance influences QA accuracy. In particular, we focus on cases where retrieval fails, allowing us to measure reliance on parametric memory through two proposed metrics---the \ppr (\PPR) which quantifies how much model accuracy is influenced by retrieval quality, contrasting performance in best-case versus worst-case retrieval scenarios, and the \ucr (\UCR) which measures how often correct QA responses are generated when the retriever fails, providing a proxy for memorization.

To enable this analysis, we make several methodological contributions. For the finetuned QA models, we investigate Vision-Transformer (ViT) architectures, which allow for multihop reasoning over multiple sources. To investigate the impact of retrieval performance on trained LMs, we propose a variable-input Fusion-in-Decoder (FiD) model \cite{tanaka_slidevqa_2023, nlvr2}, building upon the VoLTA architecture \citep{pramanick_volta_2023}. For the zero-shot case, we build upon previous research on In-Context Retrieval \citep{incontext_rag} by demonstrating that LLMs such as GPT-4o are capable of performing the final ranking step of the retrieval process. In doing so, we find that GPT-4o, a general-purpose LLM, achieves SoTA performance on the WebQA task, outperforming existing finetuned RAG models by a significant margin (7\% higher accuracy). 

Crucially, our results reveal that while retrieval-augmented models reduce memorization, the training paradigm plays an important role. Finetuned models exhibit higher reliance on parametric memory, whereas zero-shot RAG approaches have lower memorization scores at the cost of accuracy. This suggests that while retrieval modules may mitigate the risks associated with outdated or incorrect information, SoTA performance requires that they be coupled with specialized QA models. Our memorization measures contribute to the development of transparent and reliable AI systems, particularly in applications where the sourcing of up-to-date, factual information is critical.



% We investigate the impact of question complexity on the ability of these models to integrate multiple data sources—such as images, text, and external retrievers—and produce coherent and accurate answers. We also explore whether in-context retrieval can be a viable alternative to traditional retrieval-augmented systems, offering a more streamlined approach to multimodal QA.

% To achieve this, we first compare zero-shot prompting multimodal LLMs with finetuned multimodal systems. We evaluate both types of models on the WebQA benchmark, a dataset designed for complex question answering that requires reasoning across both image and text sources. For the finetuned models, we use a Fusion-in-Decoder (FiD) architecture, which allows for multihop reasoning over multiple sources. Additionally, we introduce the concept of In-Context Retrieval Language Modeling (RLM), where the LLM itself performs retrieval tasks without the need for external retrievers. This method builds upon existing research in in-context learning  and aims to explore the viability of LLMs retrieving relevant sources and generating accurate answers directly from their context window.

% In order to investigate source utilization in finetuned multimodal models and LLMs, three lines of inquiry are established; 
% \begin{itemize}
%     \item Study 1: retrieval vs QA performance on webQA (motivating example, does QA answer correctly even with incorrect sources?)
%     \item Study 2: performance on adversarial examples where parametric knowledge would be incorrect by design
%     \item Study 3: improving performance on adversarial examples by fine-tuning (i.e model robustness)
% \end{itemize}

% Note, there is one weakness in this plan which is tying in the work we've already done. 
% If we added something from adversarial generation to the retrieval experiment (like a combination of study 1 + 3) it would be complete. So for instance we could try fine-tuning the retriever with adversarial examples (and not just the QA model)

% \begin{figure}
%     \centering
%     \includegraphics[width=0.95\linewidth]{figures/segmentation/webqa_segment_infill.png}
%     \caption{Example of the segmentation substitution pipeline from the WebQA task.}
%     % d5c76d760dba11ecb1e81171463288e9
%     \label{fig:seg_sub_pipeline}
% \end{figure}



% Retrieval augmented generation (RAG) with zero-shot prompting and fine-tuning Large Language Models (LLMs) have become the go-to methods for tasks relying on information retrieval and text generation. In many cases the LLMs parametric memory can sufficiently generalize to answer questions without being provided with retrieval mechanisms for out-of-domain knowledge. However, LLMs often hallucinate and provide wrong information in certain scenarios. This problem is amplified even further on open-domain Question Answering (QA) tasks involving multiple modalities. Grounded text generation using retrieved sources \citep{lewis2021retrievalaugmented} has been extensively studied for text-to-text QA tasks, but its application in multimodal settings has not been studied as much.


% Multimodal reasoning and question answering have gained prominence in recent research endeavors, with an increasing emphasis on handling various forms of data, particularly text and images. In this study, we address a specific gap in the existing literature by focusing on the development of a versatile multihop model capable of accommodating varying numbers of input images.

% Our motivation for this research lies in the growing complexity of answering questions using information on the web, where the challenge of navigating the open-domain setting is further complicated by the presence of multiple modalities and sometimes requires reasoning over multiple sources. WebQA is an ideal dataset on which to compare performance of finetuned RAG systems against general purpose LLMs; it is multimodal, with correct answers requiring reasoning over image and text sources. It is multihop, requiring a complex reasoning process over multiple sources. Finally, WebQA questions from different categories can be broken down into subdomains to analyze performance over domains of varying cardinality.

% Motivated by the real-world challenges of building retrieval and question answering (QA) systems, we design and finetune a closed domain, multimodal, multihop QA model, that is capable of reasoning over a varying number of sources taken as input from an external retriever module. This research contributes to the relatively underexplored domain of multihop reasoning across various input sources and modalities. Our goal is to explore the challenges posed by these scenarios and develop strategies that enable QA models to retrieve relevant information, conduct logical or numerical reasoning across diverse modalities, and generate coherent responses in natural language. To our knowledge, this is the first application of the Fusion-in-Decoder (FiD) architecture \cite{tanaka_slidevqa_2023, nlvr2} that is shown to work with a variable number of inputs, enabling multi-hop reasoning over sources.

% In-Context Learning refers to the ability of LLMs to perform any task by simply providing examples in the input prompt \citep{dong2022survey,min2022rethinking}. Inspired by this research, we propose a method to use the LLM itself as a multimodal retriever, potentially eschewing the requirement of a distinct retrieval module, thereby allowing the design of simpler retrieval-augmented QA systems. We dub this method In-Context Retrieval Language Modeling (RLM). To the best of the authors knowledge, In-Content RLM is disparate from other retrieval augmented approaches which utilize external retrieval modules \citep{incontext_rag,chen_murag_2022,liu_universal_2023}. Despite being a natural extension of In-Context learning, In-Context RLM has not yet been studied empirically.

% To expand on our contribution of In-Context Retrieval, this stems from the well-researched in-context learning of LLMs. In-context learning is the ability of a model to perform any task given a sufficient context window \citep{dong2022survey,min2022rethinking}. Such tasks could include retrieval and ranking, but typically, the go-to solution for tasks requiring retrieval has been RAG. To the best of the authors knowledge, In-Context Retrieval is distinct from In-Context Retrieval Augmented Language Modelling (RALM), and despite being a natural extension of In-Context learning, In-Context Retrieval has not yet been shown empirically.

% Finally, we explore the tradeoff between using zero-shot prompting LLMs and the fine-tuning approach. While we find that, overall, GPT-4o obtains SoTA performance on the WebQA task, outperforming the accuracy of existing finetuned RAG approaches by 7\%, finetuned approaches still perform better on more restricted subdomains\footnote{``In-Context RLM" @ \url{https://eval.ai/web/challenges/challenge-page/1255/leaderboard/3168}}. Finally, we validate that GPT-4o is relying on retrieval abilities to solve the task; we find that GPT-4o is capable of retrieving relevant sources in the presence of distractors and furthermore, when GPT-4o fails to retrieve correct sources, it answers incorrectly 75\% of the time, meaning that it is not relying on parametric memory for this task.

% \paragraph{Contributions}
% Based on our experimentation and analysis on the WebQA benchmark, we make the following contributions:
% \begin{itemize}
%     \item Propose a new architecture for multimodal multihop QA that takes variable number of input sources inspired by the Fusion-in-Decoder method.
%     \item Comparison of general purpose LLMs vs specialized models on the WebQA benchmark.
%     \item Observation of In-Context Multimodal Retrieval abilities of GPT-4o and that it does not rely on parametric memory for multimodal QA.
%     \item Analysis of relationship between retrieval and QA task performance.
%     \item Analysis of task and query complexity on the performance of retrieval and QA tasks.
% \end{itemize}
















% Throughout this paper, we will present our methodology, experiments, and findings, emphasizing our approach to multihop reasoning over varying numbers of input images. We believe that our work contributes to a deeper understanding of multimodal reasoning and has the potential to enhance the capabilities of question-answering systems in the intricate, multimodal landscape of web-based information.
\section{Background and Motivation}
\label{sec:background}

We introduce the background on serverless workload serving and motivate the use of runtime resource adaptation to address resource inefficiency in existing serverless platforms.

\subsection{Resource Inefficiency with Early Binding}
% In current serverless platforms, developers are required to specify immutable sizes for their deployed functions.
% Then, providers consider functions' runtime workloads  (e.g., concurrency)  and resource usage to scale out/in their instances.
% Moreover, due to high runtime variability, functions must size their functions for worst-case scenarios.
% This, however, incurs considerable resource inefficiency.
Current serverless workflow platforms (e.g., AWS Step Functions~\cite{aws-step-function} and Azure Durable Functions~\cite{azure-durable-function}) offer the opportunity for developers to build various applications with advanced logic like chaining, branching, and parallel execution.
These applications can be defined by JSON-based structured languages (e.g., Amazon States Language) or other programming languages.
Meanwhile, developers require to specify resource configurations, including memory size, CPU cores, and scaling options, for individual functions---an early-binding approach.
The serverless platform is responsible for monitoring the workload intensity and resource usage at runtime and scaling out/in function instances accordingly.
To account for potential runtime variability, developers must size the functions in their application workflow accounting for the worst case in order to provide SLO guarantees over the end-to-end delay of request processing, e.g., the 99th percentile (P99) of the end-to-end delay must be within a given target. 
After deployment, the function sizes become immutable. The worst case is not representative and over-shoots most of the time, leading to resource inefficiency. 


To verify this claim, we conduct a data-driven analysis with a dataset from Microsoft Azure Functions~\cite{azure-dataset} to explicitly demonstrate the resource inefficiency issue. % , deriving from the worst-case based early bind.
To quantify the inefficiency, we define a metric called \emph{slack}---the margin between the actual execution time and the SLO, which is calculated as $1-l/T$ with $l$ and $T$ representing end-to-end latency and SLO, respectively.
Under certain SLO defined with P99 latency as done by existing works (e.g., \cite{osdi22-orion,mac22-wisefuse}),  we can see from Figure \ref{fig:bg:slack} that more than 60\% function invocations have slacks over 60\%.
Particularly, we analyze slacks of the top 100 most popular functions, whose invocations account for 81.6\% of the total function invocations. % (depicted in Figure~\ref{fig:bg:popular_func}) of overall invocations.
The result shows that only 20\% of the invocations of the popular functions (blue line in Figure~\ref{fig:bg:slack}) have slacks less than 40\%.
This means the majority of requests are processed faster than necessary.
Notably, in DAG-based workloads (i.e., Azure Durable Functions), the resource inefficiency further deteriorates wherein the ratio between the 95th percentile and 50th percentile is by up to three times \cite{mac22-wisefuse}.

% \begin{figure}[t!]
% \centering
% \includegraphics[width=0.25\textwidth]{./figure/motivation/Average_P99_cdf_top=100.pdf}
% \vspace{-0.3cm}
% \caption{Sufficient function slacks in production traces.}
% \label{fig:bg:slack}
% \end{figure}

\subsection{Runtime Dynamics}
\label{sec:bg:worst-case}

The resource inefficiency caused by the large slack can be mainly attributed to the over-provisioning of resources by the developer. This is to ensure that the SLO is guaranteed even in the worst case (i.e., P99). However, normal cases deviate from the worst case significantly due to runtime dynamics. 
In particular, we observe that functions face two major dynamic factors at runtime: varying working sets and inevitable performance interference. These two factors contribute significantly to the variance of the function execution time. 
% Functions face two remarkably dynamic factors at runtime: working sets and performance interference, which lead to considerable variance of execution latency.

\begin{figure*}[!t]
	\centering
	\subfloat[]{
		\includegraphics[width=0.24\textwidth]{./figure/motivation/Average_P99_cdf_top=100.pdf}
		\label{fig:bg:slack}
	}
	\hspace{8mm}
	\subfloat[]{
		\includegraphics[width=0.25\textwidth]{./figure/motivation/function-latency-ml-analyze-varying-worksets.pdf}
		\label{fig:bg:ml-func-latency}
	}
	\hspace{8mm}
	\subfloat[]{
	\includegraphics[width=0.28\textwidth]{./figure/motivation/coresident-perf.pdf}   
	\label{fig:bg:perf-inteference}
	}
	%\vspace{-0.1cm}
	\caption{(a) slacks of function invocations in production traces, (b) function latency variance caused by varying input worksets for functions object detection (OD), question answering (QA), and and text-to-speech (TS), respectively,
 (c) performance interference attributed to co-location of homogeneous function with different dominant resource demands.}
 %\vspace{-0.4cm}
\end{figure*}

%'ml-analyze':{'text-to-speech': 'text-to-speech', 'question-answer': 'question answer',
%                      'object-detection': 'object detection'
\textbf{\textit{Varying working sets.}} 
The working set, i.e., input data like videos, audios, and texts, can have varying sizes.
Taking Microsoft Azure Function Blobs (storage service) as an example, their data size difference can be as high as nine orders of magnitude~\cite{azure-function-blob}.
Such a large difference results in substantial variance of the execution time even for the same function~\cite{socc21-faast,eurosys21-ofc}.
Specifically, we measure the execution time of three functions under different working sets (detailed in \S\ref{exp:setup}).
Figure~\ref{fig:bg:ml-func-latency} illustrates the results, where we can observe a variance of up to 3.8 times in function execution caused by varying working set sizes.

% \begin{figure}[t!]
% \centering
% \includegraphics[width=0.25\textwidth]{././figure/motivation/function-latency-ml-analyze-varying-worksets.pdf}
% \vspace{-0.3cm}
% \caption{Function latency variance caused by varying input worksets}
% \label{fig:bg:ml-func-latency}
% \end{figure}	

\textbf{\textit{Performance interference.}}
% On the other hand, function deployment, which decides when and where to deploy functions, is completely undertaken by providers.
For simplicity and security, commercial serverless platforms, such as Alibaba Function Compute, Microsoft Azure, and AWS Lambda, exclusively deploy function instances belonging to the same tenant, or even belonging to the same function, in the same virtual machine~\cite{socc22-owl,atc18-peek-bench}.
For example, the empirical study in~\cite{socc22-owl} shows that in Alibaba Function Compute 65\% of the virtual machines exclusively deploy instances of the same function.
This co-location of homogeneous function instances, however, can incur severe resource contention on the same resource dimensions, particularly for network bandwidth and memory bandwidth of virtual machines~\cite{sc21-gsight,micro19-faaSprofiler,socc22-owl,atc18-peek-bench}.
To verify this observation, we use a virtual machine to run a function increasing the number of co-located instances from one to six while measuring the execution time of four different functions with resource dominance on different dimensions namely computing, I/O, network, and memory, respectively (detailed in \S\ref{exp:setup}). 
As shown in Figure~\ref{fig:bg:perf-inteference}, the co-location of homogeneous functions leads to substantial resource contention and performance interference, prolonging the function execution time up to 8.1 times. The performance interference is often hard to model and predict.

% this co-residency results in substantial increase of execution latency by up to 8.1 times,leading to considerable variance in function execution time.
% when compared with that with concurrency as one.

%for CPU-, IO-, network- and memory-intensive functions as the concurrency rises from one to six.
%Figure shows that significant performance interference can be observed, . 
%compared with the inclusive deployment (concurrency as one), 
% this exclusive deployment (gray bar) results in substantial increase of execution latency by up to 8.1$\times$ for CPU-, IO-, network- and memory-intensive functions as the concurrency rises from one to six.

% this exclusive deployment (gray bar) results in substantial increase of execution latency by up to 8.1$\times$ for CPU-, IO-, network- and memory-intensive functions as the concurrency rises from one to six.
% As depicted in Figure~\ref{fig:bg:concurrent_latency}, with the concurrency rising  from one to six,  the exclusive deployment results in substantial increase of execution latency by up to 8.1$\times$.
% This significantly magnifies execution latency variance.

% \begin{figure}[t!]
% \centering
% \includegraphics[width=0.25\textwidth]{./figure/motivation/coresident-perf.pdf}
% \vspace{-0.3cm}
% \caption{Performance interference attributed to co-residency of homogeneous function.}
% \label{fig:bg:perf-inteference}
% \end{figure}




\subsection{Runtime Resource Adaptation}
\label{sec:bg:adaptive-allocation}
To tackle the aforementioned resource inefficiency issue, we can adopt a late-binding approach through \emph{runtime resource adaptation}, which resizes functions on the fly based on runtime information (e.g., function slacks), achieving higher resource efficiency without violating SLO. For example, given a workflow as a chain of functions, the resource allocation of the downstream functions can be adjusted when the first function finishes execution. This way, the slack from the first function can be exploited to optimize resource efficiency. 

The idea sounds straightforward and has been considered in some existing works \cite{infocom22-stepconf,middleware20-fifer,socc21-llama,socc21-kraken,middleware20-xanadu}.
However, most of these works make an unrealistic assumption that either the developer performs the adaptation decision with access to runtime information or the serverless platform provider performs the adaptation with domain knowledge of the application workflow. These assumptions render these solutions impractical to deploy in real-world serverless systems. The information barrier between the developer and the provider calls for a new solution. 

We identify the following challenges and opportunities for a full-fledged design for runtime resource adaptation. 

\textbf{\textit{Skewed function execution time distribution.}} 
Resource allocation for a serverless workflow is typically done by leveraging performance profiles of all the functions in the workflow. 
During the offline profiling, the execution time distribution for each function is first obtained by running the function with a variety of sample inputs under different resource conditions. Then, given a time budget, existing approaches typically use P99 of the function execution time as a target and calculate the corresponding resource demands. However, due to the high runtime variability, the distribution of the function execution time is highly skewed where the difference between P50 and P99 can be as high as 100 times~\cite{socc23-huawei-cloud}. This means that if only the function execution time at a single percentile (P50 or P99) is used for resource allocation, there will be significant resource under-provisioning and over-provisioning for most requests at runtime. To address this issue, our idea is to allow for the exploration of the function execution time at diverse percentiles during resource allocation. 


% It is a prerequisite to profile execution latency for adaptive resource allocation.  
% As aforementioned, owing to a variety of unexpected runtime dynamics,  execution latency demonstrates skewed distributions, by up to 100$\times$ between 99\% percentile and 50\% percentile on Huawei cloud serverless~\cite{socc23-huawei-cloud} .
% This makes the current a single statistic (e.g., mean) or 99\% percentile distribution based profiling suffer significant under- and over-estimation.
% To fix this issue, our insight is to \textit{introduce more diverse percentiles to profile execution latency}. 

\textbf{\textit{Dependencies of adaptation decisions.}}
As the function execution progresses, a sub-workflow will be generated by removing the finished function(s) from the workflow. Within each sub-workflow, the resource adaptation decisions for remaining functions are dependent on each other due to the constraint imposed by the end-to-end latency SLO. For example, under-provisioning a function will result in a reduction of the time budget for executing its downstream functions, thus calling for more resources for these downstream functions to avoid SLO violations. Meanwhile, the selection of the percentile for the execution time of each function dictates resource-latency tradeoff for that function. For example, a higher percentile means that more resources will be allocated to ensure that more requests processed by the function will finish within the given time budget. On the contrary, a lower percentile means that more requests will risk SLO violation, but at the benefits of reduced resource consumption. To address such complex dependencies, we propose the following ideas: (1) We introduce two metrics (i.e., the timeout metric and the resilience metric detailed in \S\ref{sec:profilier}) to balance the resource adaptation decisions of the head function of the current sub-workflow and those of the remaining downstream functions. These metrics help us connect the decision making across sub-workflows and avoids sub-optimal adaptation decisions in each sub-workflow. 
(2) We explore lower percentiles for the head function and a high percentile (i.e., P99) for other functions in each sub-workflow. Using lower percentiles maximizes the opportunity for resource optimization since any over-time execution of the head function can later be compensated by resource adaptation in the next round. The high percentile ensures that the resource adaptation is not too radical to cause SLO violations. 

% Each workflow generates multiple sub-workflows as the execution moves forwards. 
% Within sub-workflows, the provisioning is inter-corrected.
% For instance, under-provisioning upstream functions may directly shrink the time budget for downstream functions, which dictates more resources required by the latter against (sub-) SLO violation. 
% This makes sub-workflows generally adopted as the basic unit to make adaptation decisions~\cite{socc21-llama,rtas22-fa2}. 
%  Moreover,  due to the high variance of execution performance, runtime adaptation requires to carry out function by function, i.e.,  discrete adaptation.
%  This, however, can easily lead to a sub-optimal (analyzed in~\S~\ref{sec:synthesizer:generate}).
% Our insight is to \emph{introduce a metric (i.e., resilience detailed in \S~\ref{sec:profilier}) to quantify the inter-correlation as well as a heuristic design (i.e., heavier head explained in \S~\ref{sec:synthesizer:generate})  to calibrate the sub-optimal,  such that resource adaptation can explore higher resource efficiency without SLO guarantee}.

% In particular, latency percentiles (introduced by the profiling)  involves resource adaptation as a new knob.
% Specifically, higher percentile earns  stronger guarantees in SLOs but may be highly prone to resource over-allocation because of its latency over-estimation, impairing resource efficiency.
% In contrast, decreasing percentiles offers the opportunity to explore higher resource efficiency, but suffers the risk of timeout, i.e., execution latency beyond specified time budget, and  may thus incur  SLO violations.
% Here, our insight is to \emph{moderately explore percentiles (detailed in~\S~\ref{sec:synthesizer:generate}), where head functions of  (sub-)workflows can explore lower percentiles because this creates the opportunity to reap higher resource efficiency while possible timeout can be recovered by subsequent functions' re-adaptive allocation.
% On the other head, non head functions maintain percentiles as 99\%}.
% This can well keep the trade-off between opportunities of exploring higher resource efficiency and risks of SLO violations. 
% Additionally, it effectively shrinks the searching space, benefiting the adaptation with higher time-efficiency.


\textbf{\textit{Tight resource adaptation window.}}
Runtime resource adaptation requires to calculate a new resource allocation decision for the remaining sub-workflow immediately when a function finishes execution. Since serverless functions are typically short-lived (less than 1s on average)~\cite{atc18-peek-bench,socc22-owl,atc20-serverless-in-the-wild,socc23-huawei-cloud}, the window for resource adaptation is quite tight. Assuming the serverless platform will perform the runtime adaptation on behalf of the developer since the platform has access to full runtime information, the resource adaptation decision making should be fast without involving complex calculations and logic or exploring a large space. As discussed before, the serverless platform provider does not have domain knowledge of the serverless workflow. Hence, the developer must pass the necessary information to the serverless platform for runtime adaptation decision making. Our idea is to let the developer synthesize critical hints containing resource allocation rules and options, which the serverless platform provider utilizes to perform runtime resource adaptation. The hints should be highly condensed so the serverless platform can make adaptation decisions quickly enough. 


% Apart from highly varying execution performance, serverless functions are also short-living (less than 1s on average)~\cite{atc18-peek-bench,socc22-owl,atc20-serverless-in-the-wild,socc23-huawei-cloud}, so is the window for adaptive allocation. 
% This variance and volatility calls for a well-preparation of hints for all possible runtime situations while promising them compact and straightforward enough for providers to easily take action.

% Here, our insight is to \emph{holistically synthesize hints in an offline manner, and then utilize the discreteness of adaptive allocation in both decision-making and decision-executing (detailed in~\S~\ref{sec:synthesizer:condense}) to fully condense the hints.
% Finally, hints are warped into a simple and compact table.
% Base on that, providers can accomplish the runtime adaption promptly and properly}.

To demonstrate the potential of runtime resource adaptation incorporating all the above ideas, we take a real-world serverless workflow (explained in \S\ref{exp:setup}) as an example, and evaluate its end-to-end latency (denoted by E2E) and resource consumption (CPU cores).
As illustrated in Figure~\ref{fig:bg:size}, the late-binding (blue triangle) reduces the resource consumption by up to 42.2\% compared with existing early-binding solutions (orange circle), while ensuring SLO guarantees. This highlights the significant gains from runtime resource adaptation. 


\begin{figure}[t!]
\centering
\includegraphics[width=0.45\textwidth]{./figure/motivation/size_early_bind_vs_ours.pdf}
%\vspace{-0.1cm}
\caption{Performance comparison between early-binding (left)~\cite{eurosys19-grandslam} and late-binding (runtime resource adaptation), where the CPU consumption (right) is normalized by the optimal obtained with exhaustive search.} 
%\vspace{-0.3cm}
\label{fig:bg:size}
\end{figure}

   
	







\section{Overview}
\label{sec:overview}

In this section, we use a few running examples to explain
the core ideas behind \rarust{} that integrates AARA with Rust's
borrow mechanisms.
%
\cref{sec:overview:Shared} shows how \rarust{} deals with \textbf{shared} borrows.
%
\cref{sec:overview:Mutable} shows how \rarust{} deals with \textbf{mutable} borrows.
%
\cref{sec:overview:Lattice} shows how \rarust{} deals with the aliasing problem.
%
Recall that we propose a lightweight design for \rarust{}, which assumes the programs already pass Rust's borrow checking so that \rarust{} works directly on \emph{well-borrowed} and \emph{well-typed} Rust programs.
%
Concretely, \rarust{} assumes that all borrows in the analyzed program have known lifetimes (the span they live) and satisfy the following properties:
\begin{itemize}
    \item multiple shared but no mutable borrows from the same piece of data are live at the same time; or
    \item no shared but at most one mutable borrow from the same piece of data are live at the same time.
\end{itemize}
We will show how \rarust{} exploits those properties to carry out AARA for Rust programs.

% With some running examples, we first informally present the core idea of our type inference system, which will be formally presented in other sections. Examples are for indicating critical problems to solve. Instead of showing only the solution we worked out, we will explain our design step by step, with some failed attempts, to clearly convey the reason why our system formulates as it is.

% We organize subsections as follow: We first analyze shared borrows with sharing \ref{sec:overview:Shared} and then mutable borrows with prophecy \ref{sec:overview:Mutable}; we finally conclude the lattice algebra of resource types and discuss weak update and how it introduces inaccuracy \ref{sec:overview:Lattice}.

\begin{figure}[t]
\centering
\footnotesize
\hrule
\begin{subfigure}[b]{0.52\textwidth}
\begin{lstlisting}[language=Rust, style=colouredRust]
fn iter_twice(l: &List) {
  // l : &list(4)
  iter(&*l); // share 4 as 2 + 2, &*l : &list(2)
  // l : &list(2)
  iter(&*l); // share 2 as 2 + 0, &*l : &list(2)
  // l : &list(0)
}
\end{lstlisting}
\caption{Shared Reborrowing}
\label{fig:ex-sharing}
\end{subfigure}
%
\begin{subfigure}[b]{0.46\textwidth}
\begin{lstlisting}[language=Rust, style=colouredRust]
fn update(l: &mut List) {
  iter(&*l);
  // l : &mut list(0)
  *l = Cons(3, Box::new(Nil));
  // l : &mut list(4)
  iter(&*l); iter(&*l);
}
\end{lstlisting}
\caption{Mutating A Mutable Borrow}
\label{fig:ex-mut-borrow}
\end{subfigure}
%
\hrule
%
\begin{subfigure}[b]{0.48\textwidth}
\begin{lstlisting}[language=Rust, style=colouredRust]
fn prophecy() {
  let mut l = Cons(3, Box::new(Nil));
  // l : list(p)
  let x = &mut l;
  // l : list(q),  x : &mut(list(p ), list(q))
  update(x);    // x : &mut(list(p'), list(q))
  /* drop(x) */ // p' >= q
}
\end{lstlisting}
\caption{Creating \& Dropping A Mutable Borrow}
\label{fig:ex-prophecy}
\end{subfigure}
%
\begin{subfigure}[b]{0.51\textwidth}
\begin{lstlisting}[language=Rust, style=colouredRust]
fn weak(b: bool, l1: &mut List, l2: &mut List) {
  let l = if b { 
    &mut *l1 // : &mut(list(p1), list(q1))
  } else { 
    &mut *l2 // : &mut(list(p2), list(q2))
  }; // : &mut(list(min(p1,p2)), list(max(q1,q2)))
  update(l);
}
\end{lstlisting}
\caption{Mutable Reborrowing \& Aliasing}
\label{fig:ex-weak}
\end{subfigure}
\hrule
\caption{Examples to Demonstrate How \rarust{} Works}
\label{fig:running-examples}
\end{figure}

\subsection{Dealing with Shared Borrows via Shared Potentials}
\label{sec:overview:Shared}

In \cref{sec:overview:AARA}, we demonstrate the key concepts of AARA via
the Rust program shown in \cref{fig:list-iteration}, which already features shared borrows.
%
However, this is not the end of the story: multiple shared borrows from the same memory location can exist simultaneously; for example, the function
\verb|iter_twice| shown in \cref{fig:ex-sharing} uses the reborrowing mechanism
to create two more shared borrows by \verb|&*l|.
%
If we still follow the methodology presented in \cref{sec:overview:AARA},
suppose that the function parameter \verb|l| has type $\&\kwd{list}(\alpha)$,
it would be unsound to type the two shared reborrows \verb|&*l| as $\&\kwd{list}(\alpha)$, because it would double the potentials stored in \verb|l|.

% The story of \cref{fig:list-iteration} does not end, because we actually do not follow the intuition : simply typing the shared borrow $\&t$ with $\& T$ when $t$ is typed with type $T$. Remember that there could be many shared borrows for one variable.

% \textbf{Sharing Operation:} 

Fortunately, prior research on AARA has proposed a notion of \emph{sharing} to allow multiple uses of a variable in a linear or affine type system~\cite{AARA-Poly-Multivar,AARA-Poly}. 
%
This is because AARA for functional programs needs shared potentials to pass value by reference.
%
The idea is to replace a variable $x$ of resource-annotated type $T$
with two fresh variables $x_1,x_2$ of types $T_1,T_2$, such that the potential function denoted by $T$ equals the sum of the potential functions denoted by $T_1,T_2$.
%
In RaRust, we reuse the sharing mechanism to handle shared borrows and prove it is sound with respect to the safe Rust semantics.
% 
In our setting, this indicates that we can replace a typing context
$x : \kwd{list}(\alpha)$ with $x_1 : \kwd{list}(\alpha_1), x_2:\kwd{list}(\alpha_2)$ such that $\alpha = \alpha_1 + \alpha_2$.

In \rarust{}, we adopt a more imperative design inspired by \emph{remainder contexts}~\cite{kn:Walker02,ICFP:KH21}.
%
We associate every program point with a typing context, and when a statement performs a shared (re)borrow, we split the potentials into two parts by splitting the resource-annotated type into two types as shown above: one becomes the type
of the shared (re)borrow, and the other is put back into the remainder context, i.e., the typing context after the statement.
%
For example, in \cref{fig:ex-sharing}, suppose that the function parameter \verb|l| has type $\&\kwd{list}(4)$, the first function call to \verb|iter|
performs a shared reborrow and we split the type $\&\kwd{list}(4)$ to $\&\kwd{list}(2)$ (for the function call) and $\&\kwd{list}(2)$ (for the remainder context).
%
The second function call also performs a shared reborrow, but this time the typing context indicates that \verb|l| has type $\&\kwd{list}(2)$, so we split it to $\&\kwd{list}(2)$ (for the function call) and $\&\kwd{list}(0)$.
%
Observing that the function \verb|iter| requires one unit of additional potentials, we derive the following signature for \verb|iter_twice|:
\[
\verb|iter_twice| : \kwd{fn}(\verb|l|:\&\kwd{list}(4)) \to () | 2 .
\]

% In original AARA, there exists the sharing operation for multiple usage of one variable, sharing $l:\kwd{list}(\alpha)$ as $l_1:\kwd{list}(\alpha_1)$ and $l_2:\kwd{list}(\alpha_2)$, with constraints $\alpha = \alpha_1+\alpha_2$ and that all potentials are non-negative, $\alpha_i\geq 0, i=1, 2$. In our calculus, we have a similar operation for shared borrows. When a shared borrow occurs, we will share a type as two parts, one for the borrow, another back to original, as is shown in the comment of Figure \ref{fig:ex-sharing}.

It is worth noting that Rust's shared borrows are \emph{immutable}.
However, \rarust{}'s analysis \emph{mutates} the resource-annotated types of shared borrows because shared borrows could consume resources when being accessed. 
%
This is safe unless the value of the borrow is mutated.
%
For example, a shared borrow \verb|l| points to a list of length 10 that carries 2 units of potentials per element; thus, the total potentials are 20 units.
%
We then create another shared borrow \verb|&*l| and split the potentials to let \verb|&*l| carry one unit of potentials per element.
%
Now suppose we mutate the value via the borrow \verb|l| to increment the list length by one.
%
The type of \verb|&*l| still indicates one unit of potentials per element, thus indicating 11 units of total potentials, but it only has 10 units.
%
To tackle the problem, \rarust{} exploits Rust's borrow mechanisms to render the reasoning sound:
mutable borrows and shared borrows from the same memory location cannot exist simultaneously.
%
Thus, if a Rust program mutates the list via a mutable reference \verb|l|, then the previous shared borrow \verb|&*l| must have ended its lifetime. 

% \textbf{Resource Characterization:}
% Until now, our approach seems nearly the same as original AARA. We need to point out our insight that the sharing operation characterizes the shared borrows, in the prospect of resource analysis. In purely functional world, this insight is just abstract nonsense, whereas in Rust's land, it is precious due to the distinction between shared and mutable borrows, handled by borrow mechanism. 

% \textbf{Updates on Shared Borrows:}
% Also, our version of sharing is more imperative. We will update the resource typing context along the checking, making it more similar to symbolic execution. We therefore need to define not only reading but also writing towards typing context.  One interesting observation is that even shared borrow can mutate corresponding type, but only decrease its resource. It indicates that the borrow mechanism is too coarse to differentiate value mutation and resource mutation. Resource sensitive languages, especially those for smart contracts, expects a more precise type system.

\subsection{Dealing with Mutable Borrows via Prophecy Potentials}
\label{sec:overview:Mutable}

It might seem straightforward to support mutable borrows in the approach sketched in \cref{sec:overview:Shared}.
%
For example, \cref{fig:ex-mut-borrow} implements a function \verb|update| that manipulates a mutable reference \verb|l|.
%
Suppose that the function parameter \verb|l| has type $\&\kwd{mut}~\kwd{list}(2)$.
%
For the first function call to \verb|iter| with a shared reborrow \verb|&*l|, we split the type to $\&\kwd{list}(2)$ and $\&\kwd{mut}~\kwd{list}(0)$.
%
The next assignment statement mutates the list stored in the location referenced by \verb|l|, so we mutate its type in the typing context accordingly.
%
To obtain enough potential for the remaining two function calls to \verb|iter|, we set the type of \verb|l| to $\&\kwd{mut}~\kwd{list}(4)$.
%
Because the new list \verb|*l| is a singleton list, the mutation itself consumes 4 units of potentials.
%
Similarly to the reasoning in \cref{sec:overview:Shared}, the potential is sufficient to perform the remaining two function calls, and the final remainder context is $\verb|l|: \&\kwd{mut}~\kwd{list}(0)$.
%
Noting that three calls to \verb|iter| need three units of additional potentials, we derive the following signature for \verb|update|:
\[
\verb|update|: \kwd{fn}(\verb|l|:\&\kwd{mut}~\kwd{list}(2)) \to () | 7 .
\]

A tricky issue arises when one considers creating and dropping mutable borrows.
%
\cref{fig:ex-prophecy} shows an example where the program creates a mutable borrow \verb|x| from a mutable list \verb|l|.
%
Note that it is no longer sound to split the potentials of \verb|l| into two parts and store one part in \verb|x|: the reason is stated already at the end of \cref{sec:overview:Shared}; that is, the program can later mutate the list \verb|l| through the mutable reference \verb|x|, making the remainder type of \verb|l| unsound.
%
Fortunately, Rust's borrow mechanisms ensure a good property that
at most, one mutable borrow from the same memory location can be live simultaneously, so in principle, it would be possible to track the change in the type of the mutable borrow \verb|x| and pass the change back to \verb|l| when \verb|x| gets dropped.
%
It is worth noting that our type system is aware of when a borrow $x$ gets dropped via an explicit statement $\kwd{drop}~ x$, which is generated according to lifetime constraints given by the Rust compiler.

% \textbf{Mutation might increase resource.}
% We start by comparing shared borrows with mutable borrows. According to borrow properties, when lifetime of shared borrows does not end, the value is immutable and the resource can only monotonically decrease(non strict). But if mutable borrow exists, the resource can increase at need. Figure \ref{fig:ex-mut-borrow} presents such an example. Before assignment, the mutable borrow has $0$ resource and the value is to remove, while after it, the borrow should have at least $4$ per \lstinline|Cons| to pay for two iterations. Note that the value, or specifically the length, of the list has changed, therefore the example is different from just iterating 3 times.

% \textbf{Mutable borrow is the true borrow to restore.}
% Because the mutable borrow might increase resource, i.e. $\alpha - \alpha_2 < 0$, the sharing technique $\alpha = \alpha_1 + \alpha_2$ with $\alpha_1 \geq 0$ mentioned in the previous subsection is no longer applicable.  The mutable borrow should take away the resource from original place, and give back when it drops. Mutable borrow is the true borrow because it really borrow the resource instead of sharing a part, and will finally restore. This is the main difference between mutable borrows and shared borrows from the aspect of resource analysis. 

% \textbf{To attempt to mutably borrow with places is doomed to fail.} 

% Mutable borrow needs to restore its resource when it drops. 

One idea might emerge immediately that the resource-annotated type of a mutable borrow keeps the location where it borrows from, as $\verb|l|:\&\kwd{mut}(\kwd{list}(\alpha), \verb|p|)$ with \verb|p| be the location such that \lstinline|l = &mut p|.
%
However, this design essentially embeds a pointer analysis in the type system, and one would soon find out that every mutable reference type needs to record a \emph{set} of possible locations.
%
\cref{fig:ex-weak} exemplifies this case and we will revisit this example in \cref{sec:overview:Lattice}.
%
Because those locations are usually local variables, it becomes unclear how to carry out inter-procedural analysis in a compositional way, and we certainly do not want function signatures to reveal local variables.

% It is a bad design because the mutable borrow type disastrously depends to places, or said term variables. The dependent type will definitely increase the complexity of analysis. We just enumerate some simple but non-trivial problems. 
% \begin{enumerate}
%     \item {\textbf{Parameters:}
%     When the mutable borrow types appear as types of functions' formal parameters, the interpretation of places is confusing, and like a placeholder. When analyzing recursive functions, it is hard to assume a set of places as signatures. And such a analysis will soon degenerate to point-to analysis and may diverge.
%     }
%     \item {\textbf{Subtyping:} 
%     Recall that AARA approach needs a resource subtyping relation. There exists research\cite{CapTypes} on subtyping over resource dependent types, while it is complex. 
%     }
%     \item{\textbf{Identification:}
%     Places are usually local variables. When analyzing inter-procedural, it will be a big problem to identify whether two places are the same.
%     }
% \end{enumerate}

Such an issue is not uncommon in the studies of advanced type systems or verification frameworks for Rust.
%
\citet{ProphecyInSepLogic} adapt \emph{prophecy variables}---which were originally proposed by \citet{LICS:AL88} to talk about future's program states during reasoning---to integrate separation logic with prophecies.
%
RustHorn~\cite{RustHorn} and RefinedRust~\cite{RefinedRust} also use prophecy variables in their verification frameworks.
%
The high-level idea of using prophecy variables to analyze Rust's mutable borrows is to record additional information which corresponds to the final value of a reference, i.e., the value when the reference gets dropped. 
%
However, prophecy variables usually record the final values, which are too heavy for AARA.

In \rarust{}, we propose \emph{prophecy potentials}, a novel adaption of prophecy variables to the AARA methodology to reason about the future's potential functions.
%
\cref{fig:ex-prophecy} shows how prophecy potentials work with mutable borrows.
%
We now represent the type of mutable reference as $\&\kwd{mut}(\tau_1,\tau_2)$, where $\tau_1$ denotes the current potential function and $\tau_2$ denotes the prophecy potential function, i.e., the expected potential function when the lifetime of the reference ends.
%
In \cref{fig:ex-prophecy}, suppose that the initial type of \verb|l| is $\kwd{list}(p)$ with some $p \ge 0$.
%
To create a mutable borrow \verb|x| from \verb|l|, we generate a prophecy type $\kwd{list}(q)$ with some $q \ge 0$, which indicates the final resource-annotated type of \verb|x| when \verb|x| gets dropped.
%
The mutable reference type of \verb|x| is initialized to $\&\kwd{mut}(\kwd{list}(p), \kwd{list}(q))$.
%
After the call to the function \verb|update|, the type of \verb|x| becomes $\&\kwd{mut}(\kwd{list}(p'), \kwd{list}(q))$.
%
Note that the prophecy type should remain unchanged.
%
When the mutable reference \verb|x| drops, \rarust{} emits a constraint that the potentials indicated by $\kwd{list}(p')$ are no less than the potentials indicated by the prophecy type $\kwd{list}(q)$, i.e., $p' \ge q$.
%
If \verb|l| would later be used again, we can start from the type $\kwd{list}(q)$.
%
In this way,
prophecy potentials enable \emph{compositional} reasoning about mutable borrows.

% \textbf{Prophecy variables characterize restoring.} 
% Among problems listed above, identification might be the most pathological, yet shedding light on global demands. We need a global staff to indicate where the borrows come from. Note that AARA utilizes linear programming, which contains lots of linear variables global to the analyzed program. Just as it is used to indicate future values in research \cite{RustHorn} in program verification, prophecy variables can be used to indicate the resource when the mutable borrow drops and restore. In such a meaning, we say that prophecy variables characterize the prophetic restoring at the time when the borrow occurs. 

% \textbf{Restoring captured as linear constraints.}
% Figure \ref{fig:ex-prophecy} shows how prophecy variables are correlated with places. When the mutable borrow occurs, resource type in the original place $l$ will be replaced with prophecy type $\kwd{list}(p)$, and mutable borrow takes away original type $\kwd{list}(q)$; when the borrow drops, type checker will generate linear constraints from subtyping relation $\kwd{list}(p) \preceq \kwd{list}(q')$, to ensure the prophetic is bound by the final resource $\kwd{list}(q')$. We explicitly point out that the restoring is captured as linear constraints, fully utilized by linear programming solvers.

% \textbf{Higher order borrows can be characterized by subtyping over mutable borrow types.}
% It is obviously that prophecy types are contravariant for subtyping, i.e.$\&^\kwd{m}(\tau_{\text{c}, 1}, \tau_{\text{p}, 1}) \preceq \&^\kwd{m}(\tau_{\text{c}, 2}, \tau_{\text{p}, 2})$ if and only if $\tau_{\text{c}, 1} \preceq \tau_{\text{c}, 2}, \tau_{\text{p}, 2} \preceq \tau_{\text{p}, 1}$. With subtyping over mutable borrow types, borrows of borrows, or said higher order borrows can be easily characterized by prophetic version of mutable borrow types. 

\subsection{Dealing with Aliasing via A Lattice of Resource-Annotated Types}
\label{sec:overview:Lattice}

A type system sometimes cannot precisely determine where a mutable reference is borrowed from.
%
For example, \cref{fig:ex-weak} uses a conditional expression to assign a mutable reference \verb|l| to a mutable reborrow from either \verb|l1| or \verb|l2|, depending on the runtime value of the Boolean-valued variable \verb|b|.
%
As the example shows, although Rust's borrow mechanisms enjoy some good properties that aid our design of \rarust{}, we still face the problem of \emph{aliasing}, as other static analyses of heap-manipulating programs would also face.

In our work, we can still exploit Rust's borrow mechanisms, which ensure that aliasing and mutation cannot happen at the same time.
%
Therefore, we only need to consider \emph{control-flow aliasing}; that is,
when the control-flow paths merge at a program point, we need to \emph{merge} the types---including the mutable reference types---of a variable from different paths.
%
The merging here needs to be conservative, similarly to the \emph{weak updates} usually seen in pointer analyses.
%
In \rarust{}, we formulate a subtyping relation among resource-annotated types to formalize the notion of ``conservative'' and then construct a \emph{lattice} of resource-annotated types based on subtyping to carry out merging.
%
For example, in \cref{fig:ex-weak}, suppose that the mutable reborrows \verb|&mut *l1| has type $\&\kwd{mut}(\kwd{list}(p_1),\kwd{list}(q_1))$ and \verb|&mut *l2| has type $\&\kwd{mut}(\kwd{list}(p_2),\kwd{list}(q_2))$.
%
Recall that $\kwd{list}(q_1)$ and $\kwd{list}(q_2)$ are prophecy types.
%
To obtain the type of the mutable reference \verb|l|, we need to merge the two types above.
%
Thinking about the merging conservatively, one can derive that
\verb|l| can hold potentials no more than the potentials indicated by
$\kwd{list}(p_1)$ and $\kwd{list}(p_2)$, so \verb|l|'s current potential type is
at most $\kwd{list}(\min(p_1,p_2))$.
%
Meanwhile, to ensure that the prophecies are sound no matter which branch is executed, \verb|l|'s prophecy potential type should be at least $\kwd{list}(\max(q_1,q_2))$.
%
In addition, because the type system cannot know which of \verb|l1| and \verb|l2| would be mutated later or which of them would remain unchanged, \rarust{} enforces that $p_1 \ge q_1$ and $p_2 \ge q_2$.

As illustrated above, \rarust{}'s current mechanism of handling aliasing compromises the precision of the resource analysis, mostly due to weak updates.
%
On the one hand, such precision loss is unavoidable---at least for \cref{fig:ex-weak}---due to insufficient information about runtime values during type checking.
%
On the other hand, recent work such as Flux~\cite{Flux} introduces \emph{strong references} to perform strong updates, and it is interesting future research to adapt them in \rarust{}.

% \textbf{Merging of Typing Context:} 
% Recall that resource typing context will be updated during type checking, for shared borrows, also for mutable borrows. Together with branching statements, it brings a new problem that after branching there would be more than one typing contexts to merge as one. With the help of resource subtyping, exactly the lattice algebra of resource types, we can use meet operation in algebra to merge contexts. 

% \textbf{Weak Update:} 
% Branching statements will introduce weak mutable borrows, those pointing to multiple possible places. Actually, it is also one reason why we give up mutable borrow with places. The updates on weak borrows are what called weak updates in literates of program verification. It is dangerous to update all possible places, because it might generate resource from the vacuum. As the example in Figure \ref{fig:ex-weak}, when updating $l$ increasingly, it is definitely wrong to increase resource at both $l1$ and $l2$, because in runtime, there always only one place to increase. To ensure soundness, our choice is to force non-increasing when merge mutable borrows, therefore additional non-increasing constraints introducing inaccuracy into our sound analysis. Besides, in presence of nondeterministic boolean values, or said static unknown values, weak updates and inaccuracy are unavoidable, in the sense of analysis. The inaccuracy of our sound analysis are mainly introduced by weak updates.
\section{Resource Aware Borrow Calculus}
\label{sec:calculus}

This section introduces Resource-Aware Borrow Calculus (RABC) and resource-aware dynamic semantics.
%
RABC is a resource-aware variant of Low-Level Borrow Calculus (LLBC)~\cite{Aeneas}, which is based on Rust's MIR but keeps high-level information such as structured control flow and a structured memory model.
%
RABC includes essential features such as mutation, borrow mechanisms, integer lists with explicit boxing, and recursive top-level functions.\footnote{We include only integer lists as heap-allocated data structures and exclude loops in RABC for ease of presentation. Our implementation supports user-defined inductive data types using structs and enums, as well as \texttt{while true} loops with break and continue.}
%
RABC also includes $\kwd{tick}(\cdot)$ statements to annotate resource consumption.
%
\cref{sec:syntax} presents the syntax and discusses properties of a well-borrowed RABC program, guaranteed by Rust's borrow checker.
%
\cref{sec:semantics} formalizes the resource-aware dynamic semantics of RABC, which captures the resource consumption during the execution of an RABC program.

\subsection{Syntax}
\label{sec:syntax}

\cref{fig:syntax} summarizes the syntax of RABC. For the convenience of formalization, we distinguish between expressions and statements.
%
We then describe each syntactical construction separately. 

\begin{figure}[t]
\small
    \begin{align*}
    \textbf{Type}~ t &::= \\
        \tag{atom} &|~ \kwd{i32} ~|~ \kwd{bool} \\
        \tag{list} &|~ \kwd{list} ~|~ \kwd{box}~\kwd{list}\\
        \tag{borrow} &|~ \&^\kwd{s}~t ~|~ \&^\kwd{m}~t \\
    \textbf{Place}~ p &::= \\
        \tag{variable} &|~ x \\
        \tag{dereference} &|~ * p \\
    \textbf{Expression}~ e &::= \\
        &|~ \kwd{n}_\text{i32} ~|~ \kwd{true} ~|~ \kwd{false} ~|~ \kwd{nil} ~|~ \kwd{box}(e) \\
        \tag{integer} &|~ e_1 ~\kwd{op}~ e_2 \\
        \tag{scalar copy} &|~ \kwd{copy}~ p \\
        \tag{borrow} &|~ \&^\kwd{s}~ p ~|~ \&^\kwd{m}~ p \\
        \tag{move ownership} &|~ \kwd{move}~ p \\
    \textbf{Statement}~ s &::= \\
        \tag{resource cost} &|~ \kwd{tick}(\delta) \\
        \tag{return} &|~ \kwd{return} \\
        \tag{sequence} &|~ s_1; s_2 \\
        \tag{drop} &|~ \kwd{drop}~ p\\
        \tag{if bool} &|~ \kwd{if}~ p ~\kwd{then}~ s_1 ~\kwd{else}~ s_2 ~\kwd{end}\\
        \tag{match list} &|~ \kwd{match}~ p ~ \{\kwd{nil}\mapsto s_1, \kwd{cons}(x_\text{hd}, x_\text{tl})\mapsto s_2\} \\
        \tag{assignment} &|~ p \from e \\
        \tag{list constructor} &|~ p \from \kwd{cons}(e_1, e_2) \\
        \tag{function call} &|~ p \from f(\vec{e}) \\
    \textbf{Toplevel}~ tl &::= \\
        \tag{sequence} &|~ tl_1 ~ tl_2 \\
        \tag{function} &|~ \kwd{fn}~ f ~(\vec{x}_\text{param}:\vec{t}_\text{param}, \vec{x}_\text{local}:\vec{t}_\text{local}, x_\text{ret}:t_\text{ret}) \{~ s ~\}
    \end{align*}
    \caption{Syntax}
    \label{fig:syntax}
\end{figure}


\textbf{Types} are simple, without resource annotations; we will present the types with annotations in \cref{sec:inference}. Integer $\kwd{i32}$ and Boolean $\kwd{bool}$ are atom types. Lists $\kwd{list}$ and the box type of lists $\kwd{box}~\kwd{list}$ are types for functional lists defined in Rust. The box type is required for a list's tail, which is usually heap-allocated. The reference type is for borrows with different modes: $\&^\kwd{s}~ t$ is for the shared borrow, and $\&^\kwd{m}~ t$ is for the mutable borrow. These borrow modes are notions from Rust explained as follows: mutation is forbidden on shared borrows and only allowed on mutable borrows.

\textbf{Places} are memory locations storing values, including program variables $x, y, \ldots$ and dereferences $* p$ of borrows or boxes stored in $p$. We will soon show their role in the dynamic semantics in \cref{sec:semantics}. 

\textbf{Expressions} represent resource-free evaluation. Integer literals $\kwd{n}_\text{i32}$ and Boolean literals $\kwd{true}$, $\kwd{false}$ are atom values. The $\kwd{nil}$ constructor stands for empty lists. The boxing expression $\kwd{box}(e)$ allocates memory in a heap to store the value of $e$, resembling \lstinline|Box::new(e)| of Rust. Arithmetic expressions $e_1 ~\kwd{op}~ e_2$ operate on integer-valued operands with the binary operator $\kwd{op}$. For an atom value stored in a place $p$, we use $\kwd{copy}~p$ to make a copy of it.
% The scalar copies are for the smaller data like integers and Boolean values, while the borrows are usually used for those larger data structures like lists.
Given a place $p$, we can borrow from it with different modes: $\&^\kwd{s}~ p$ creates a shared borrow, and $\&^\kwd{m}~ p$ creates a mutable borrow.
%
For a borrow stored in a place $p$, we can use $\kwd{move}~ p$ to move ownership out from the original place $p$. 

\textbf{Statements} represent resource-aware evaluation. The statement $\kwd{tick}(\delta)$ with $\delta \in \ZZ$ is the explicit annotation for consuming $\delta$ units of resource.
%
Statements $\kwd{return}$ and $s_1; s_2$ usually form a function body such as $s_1; s_2; \ldots, s_n; \kwd{return}$. In RABC, we introduce $\kwd{drop}~ p$ to drop the borrow stored in $p$ explicitly. The conditional and pattern-match statements perform case analysis on Boolean values and lists, respectively. Note that we use place $p$ instead of the expression $e$ to indicate Boolean values and lists because we only need to peak the value instead of copying Boolean values or moving ownership of lists. The assignment has three variants: one for assigning atom values and borrows, one for constructing lists, and another for function applications; the latter two variants do not reside in expressions because they need to be resource-aware as they incur resource consumption.

\textbf{Toplevels} define a sequence of (possibly recursive) top-level functions like $\kwd{fn} ~f_1, \ldots, \kwd{fn}~ f_n$. Each function contains one statement as its body and variables with corresponding type declarations, including function parameters $\vec{x}_\text{param}:\vec{t}_\text{param}$, local variables $\vec{x}_\text{local}:\vec{t}_\text{local}$ used in the function body, and a distinguished variable $x_\text{ret} : t_\text{ret}$ for the returned value. The $\vec{\bullet}$ notation represents vectors.

\subsection{Resource Aware Dynamic Semantics}
\label{sec:semantics}

\begin{figure}[t]
\small
    \begin{align*}
    \tag{undefined} \textbf{Value}~ v &::= \bot \\
    \tag{atoms} &|~ \kwd{n}_\text{i32} ~|~ \kwd{true} ~|~ \kwd{false} \\
    \tag{list} &|~  lv \\
    \tag{box} &|~ \kwd{box}(lv) \\
    \tag{borrow} &|~ \&(p, v)\\
    \textbf{List Value}~ lv &::= \kwd{nil} ~|~ \kwd{cons}(\kwd{n}_\text{i32}, \kwd{box}(lv))
    \end{align*}
    \caption{Value}
    \label{fig:dyn-value}
\end{figure}

\begin{figure}[t]
\small
    \judgement{Store Reading}{$V\vDash p \rightsquigarrow v$}
    \begin{mathpar}
    \inferrule*[Right=\rulename{V-Rd-Var}]
    {V(x)=v}
    {V\vDash x \rightsquigarrow v}
    \and
    \inferrule*[Right=\rulename{V-Rd-Box}]
    {V\vDash p \rightsquigarrow \kwd{box}(v)}
    {V\vDash * p\rightsquigarrow v}
    \and
    \inferrule*[Right=\rulename{V-Rd-Borrow}]
    {V\vDash p \rightsquigarrow \&(\_, v)}
    {V\vDash *p \rightsquigarrow v}
    \end{mathpar}

    \judgement{Store Writing}{$\VWt{V}{p}{v}{V'}$}
    \begin{mathpar}
    \inferrule[V-Wt-Var]
    {\forall y\not=x, V'(y)=V(y) 
    \\ V'(x) = v}
    {\VWt{V}{x}{v}{V'}}
    \and
    \inferrule[V-Wt-Box]
    {V\vDash p\rightsquigarrow \kwd{box}(\_)
    \\ \VWt{V}{p}{\kwd{box}(v)}{V'}}
    {\VWt{V}{* p}{v}{V'}}
    \and
    \inferrule*[Right=\rulename{V-Wt-Borrow}]
    {V\vDash p\rightsquigarrow \&(q, \_)
    \\ \VWt{V}{q}{v}{V'}
    \\ \VWt{V'}{p}{\&(q, v)}{V''}}
    {\VWt{V}{*p}{v}{V''}}
    \end{mathpar}

    \caption{Store Reading and Writing}
    \label{fig:dyn-rw}
\end{figure}

\begin{figure}[t]
\small
\judgement{Expression Evaluation (Selected)}{$V\vDash e \rightsquigarrow v$}
    \begin{mathpar}
    \inferrule*[Right=\rulename{V-Ev-Borrow}]
    {V\vDash p \rightsquigarrow v}
    {V\vDash \&^{\kwd{s}/\kwd{m}} p \rightsquigarrow \&(p, v)}
    \end{mathpar}

\judgement{Statement Execution (Selected)}{$V\vDash e \rightsquigarrow^\delta \Dashv V'$}
    \begin{mathpar}
    \inferrule*[Right=\rulename{V-Ex-Tick}]
    {~}
    {V\vDash \kwd{tick}(\delta)\rightsquigarrow^\delta \Dashv V}
    \and
    \inferrule*[Right=\rulename{V-Ex-Drop}]
    {~}
    {V\vDash \kwd{drop}~p\rightsquigarrow^0\Dashv V}
    \\
    \inferrule*[Right=\rulename{V-Ex-Cons}]
    {V\vDash e_1 \rightsquigarrow v_1
    \\ V\vDash e_2 \rightsquigarrow v_2
    \\ \VWt{V}{p}{\kwd{cons}(v_1, v_2)}{V'} }
    {V\vDash p\from \kwd{cons}(e_1, e_2)\rightsquigarrow^0 \Dashv V'}
    \\
    \inferrule*[Right=\rulename{V-Ex-IfT}]
    {V\vDash p\rightsquigarrow \kwd{true}
    \\ V\vDash s_1\rightsquigarrow^\delta \Dashv V'}
    {V\vDash \kwd{if}~ p ~\kwd{then}~ s_1 ~\kwd{else}~ s_2 ~\kwd{end} \rightsquigarrow^\delta \Dashv V'}
    \and
    \inferrule*[Right=\rulename{V-Ex-IfF}]
    {V\vDash p\rightsquigarrow \kwd{false}
    \\ V\vDash s_2\rightsquigarrow^\delta \Dashv V'}
    {V\vDash \kwd{if}~ p ~\kwd{then}~ s_1 ~\kwd{else}~ s_2 ~\kwd{end} \rightsquigarrow^\delta \Dashv V'}
    \\
    \inferrule*[Right=\rulename{V-Ex-Mat-Nil}]
    {V\vDash p\rightsquigarrow \kwd{nil}
    \\ V\vDash s_1\rightsquigarrow^\delta \Dashv V'}
    {V\vDash \kwd{match}~ p ~ \{\kwd{nil}\mapsto s_1, \kwd{cons}(x_\text{hd}, x_\text{tl})\mapsto s_2\} \rightsquigarrow^\delta \Dashv V'}

    \inferrule*[Right=\rulename{V-Ex-Mat-Cons}]
    {V\vDash p\rightsquigarrow \kwd{cons}(hd, tl)
    \\ \VWt{V}{p}{\bot}{V_1}
    \\ \VWt{V_1}{x_\text{hd}}{hd}{V_2}
    \\ \VWt{V_2}{x_\text{tl}}{tl}{V_\text{b}}
    \\\\ V_\text{b}\vDash s_2\rightsquigarrow^\delta \Dashv V'_\text{b}
    \\ V'_\text{b}\vDash x_\text{hd}\rightsquigarrow hd'
    \\ V'_\text{b}\vDash x_\text{tl}\rightsquigarrow tl'
    \\\\ \VWt{V'_\text{b}}{x_\text{hd}}{\bot}{V'_1}
    \\ \VWt{V'_1}{x_\text{tl}}{\bot}{V'_2}
    \\ \VWt{V'_2}{p}{\kwd{cons}(hd', tl')}{V'} 
    }
    {V\vDash \kwd{match}~ p ~ \{\kwd{nil}\mapsto s_1, \kwd{cons}(x_\text{hd}, x_\text{tl})\mapsto s_2\} \rightsquigarrow^\delta \Dashv V'}
    \\
    
    \inferrule*[Right=\rulename{V-Ex-App}]
    {\kwd{fn}~ f ~(\vec{x}_\text{param}:\vec{t}_\text{param}, \vec{x}_\text{local}:\vec{t}_\text{local}, x_\text{ret}:t_\text{ret}) \{~ s ~\}
    \\ V\vDash \vec{e}\rightsquigarrow \vec{v}
    \\\\ \VWt{V}{\vec{x}_\text{param}}{\vec{v}}{V_1}
    \\ \VWt{V_1}{\vec{x}_\text{local}}{\bot}{V_2}
    \\ \VWt{V_2}{x_\text{ret}}{\bot}{V_\text{b}}
    \\\\ V_\text{b}\vDash s\rightsquigarrow^\delta \Dashv V'_\text{b}
    \\ V'_\text{b}\vDash x_\text{ret} \rightsquigarrow v_\text{ret}
    \\\\ \VWt{V'_\text{b}}{\vec{x}_\text{param}}{\bot}{V'_1}
    \\ \VWt{V'_1}{\vec{x}_\text{local}}{\bot}{V'_2}
    \\ \VWt{V'_2}{x_\text{ret}}{\bot}{V'_3}
    \\ \VWt{V'_3}{p}{v_\text{ret}}{V'}
    } 
    {V\vDash p\from f(\vec{e})\rightsquigarrow^\delta \Dashv V'}
    \end{mathpar}
    \caption{Resource Aware Dynamic Semantics}
    \label{fig:dyn-eval-exec}
\end{figure}


\cref{fig:dyn-value}, \cref{fig:dyn-rw}, and \cref{fig:dyn-eval-exec} define a  resource-aware big-step dynamic semantics for RABC.
%
\cref{fig:dyn-value} defines values of RABC, including atom values, list values, box values, borrow values, and a distinguished undefined value $\bot$. 
%
Note that borrow values take the form $\&(p,v)$, denoting a value $v$ borrowed from a place $p$, but do not record the borrow mode ($\&^\kwd{s}$ or $\&^\kwd{m}$).
%
The design is reasonable because we work on well-borrowed programs; thus, we do not need to track the borrow modes during runtime.

A \emph{store} is a mapping $V : \mathbf{Variable}\to\mathbf{Value}$, where unused variables can be mapped to $\bot$.
%
\cref{fig:dyn-rw} formalizes reading from and writing to a store.
%
Judgement $V \vDash p \rightsquigarrow v$ means that under a store $V$, the place $p$ records a value $v$.
%
Judgement $\VWt{V}{p}{v}{V'}$ means that starting from a store $V$, writing a value $v$ to the place $p$ yields a new store $V'$.
%
Note that the rule \rulename{V-Wt-Borrow} may not terminate in general for heap-manipulating languages like C.
%
In our setting, we exploit Rust's borrow mechanisms that ensure that one cannot construct cyclic reference relations using borrows.

% We can read or write on a variable $x$. Due to $\mathbf{Place}$ syntax, we extend it to judgement $V\vDash p \rightsquigarrow v$ and $\VWt{V}{p}{v}{V'}$, the former reading and the latter writing, as \cref{fig:dyn-rw}. All rules are nearly trivial except that rule \rulename{V-Wt-Borrow} immediately updates value in original place $q$, as $\VWt{V}{q}{v}{V'}$. The termination and safety of rule \rulename{V-Wt-Borrow} is guaranteed by the Rust borrow checker again.

\cref{fig:dyn-eval-exec} presents selected evaluation rules for expressions and statements.
%
Judgement $V\vDash e \rightsquigarrow v$ indicates that under a store $V$, the expression $e$ evaluates to the value $v$. Recall that expressions denote resource-free computation, so we do not record resource information.
%
% The rule \rulename{V-Ev-Copy} is restricted to scalar values, and the rule \rulename{V-Ev-Move} is restricted to borrows.
%
The rule \rulename{V-Ev-Borrow} reflects the design that the runtime does not need to track borrow modes for well-borrowed programs.
%
Judgement $V \vDash s \rightsquigarrow^\delta \Dashv V'$ means that starting from a store $V$, the statement $s$ executes with $\delta$ units of resource consumption and ends in the store $V'$.
%
The rule \rulename{V-Ex-Tick} introduces $\delta$ unit of resource consumption; this is the only rule to incur actual resource uses.
%
The rule \rulename{V-Ex-Drop} does nothing, i.e., it does need to put the value back to the borrowed place, because we immediately update values when writing through borrows, as indicated by the rule \rulename{V-Wt-Borrow} in \cref{fig:dyn-rw}.
%
Also, because of such immediate updates, it is necessary to make sure that original places and variables should be passed to function applications; the subscript $\textbf{b}$ of the store $V_\text{b}$ stands for \textbf{b}inding in the rule \rulename{V-Ex-App}.
%
% Though global uniqueness of variables is a requirement in dynamic semantics, this feature does not add complexity to the type system, which will be elaborated in the subsequent section. This is because dynamic semantics serves merely as a resource-aware semantic instrument in the formalization to prove the soundness of our type system; it is not executed in practice.



\section{Resource Aware Type System and Inference} \label{sec:inference}

In this section, we present the resource-aware type system based on RABC introduced in \cref{sec:calculus} and a type-inference algorithm based on the AARA methodology.
%
\cref{sec:inference:types} introduces resource-enriched types, which augment the types of RABC with resource annotations.
%
% Rich type $\tau$ is plain type $t$ enriched with potential annotation $\alpha$, e.g. $\kwd{list}(\alpha)$ for $\kwd{list}$. Then, we introduce the typing context and its read/write operation used for type checking, followed by the definition of function signatures. 
%
\cref{sec:inference:subtyping} formulates a subtyping relation among resource-enriched types and uses the relation to construct a lattice of types sketched in \cref{sec:overview:Lattice}.
%
% With subtyping, we can tell what rich types are well formed. The subtyping relation over rich types actually forms a lattice with meet and join operations. We define the lattice operations and extend it to the typing context.
%
\cref{sec:inference:eval} and \cref{sec:inference:exec} present the resource-aware typing rules for expressions and statements, respectively.
%
\cref{sec:inference:infer} discusses a type-inference algorithm for the resource-aware type system.

\subsection{Rich Types, Contexts, and Signatures} \label{sec:inference:types}
\begin{figure}[t]
\small
    \begin{align*}
    \tag{undefined} \textbf{RichType}~ \tau &::= \bot \\
    \tag{atom types} &|~ \kwd{i32} ~|~ \kwd{bool} \\
    \tag{list} &|~ \kwd{list}(\alpha)\\
    \tag{box} &|~ \kwd{box}(\kwd{list}(\alpha)) \\
    \tag{shared borrow} &|~ \&^\kwd{s}(\tau) \\
    \tag{mutable borrow} &|~ \&^\kwd{m}(\tau_\text{c}, \tau_\text{p})
    \end{align*}
    \caption{Rich Types}
    \label{fig:rich-type}
\end{figure}
\begin{figure}[t]
\small
    \judgement{Enrich (Selected)}{$\textit{enrich}~ t ~\textit{as}~ \tau$}
    \begin{mathpar}
    \inferrule[Enrich-List]
    {\alpha~\text{fresh}}
    {\textit{enrich}~ \kwd{list} ~\textit{as}~  \kwd{list}(\alpha)}
    \and
    \inferrule[Enrich-Shared]
    {\textit{enrich}~ t ~\textit{as}~ \tau}
    {\textit{enrich}~ \&^\kwd{s}(t) ~\textit{as}~ \&^\kwd{s}(\tau)}
    \and
    \inferrule[Enrich-Mutable]
    {\textit{enrich}~ t ~\textit{as}~ \tau_\text{c}
    \\ \textit{enrich}~ t ~\textit{as}~ \tau_\text{p}
    }
    {\textit{enrich}~ \&^\kwd{m}(t) ~\textit{as}~ \&^\kwd{m}(\tau_\text{c}, \tau_\text{p})}
    \end{mathpar}
    \caption{Enrichment}
    \label{fig:enrich}
\end{figure}

\begin{figure}[t]
\small
    \judgement{Context Reading}{$\Gamma\vdash p \hookrightarrow \tau$}
    \begin{mathpar}
    \inferrule[$\Gamma$-Rd-Var]
    {\Gamma(x)=\tau}
    {\Gamma\vdash x \hookrightarrow \tau}
    \and
    \inferrule[$\Gamma$-Rd-Box]
    {\Gamma\vdash p \hookrightarrow \kwd{box}(\tau)}
    {\Gamma\vdash * p\hookrightarrow \tau}
    \and
    \inferrule[$\Gamma$-Rd-Shared]
    {\Gamma\vdash p \hookrightarrow \&^\kwd{s}(\tau)}
    {\Gamma\vdash *p \hookrightarrow \tau}
    \and
    \inferrule[$\Gamma$-Rd-Mutable]
    {\Gamma\vdash p \hookrightarrow \&^\kwd{m}(\tau_\text{c},\tau_\text{p})}
    {\Gamma\vdash * p\hookrightarrow \tau_\text{c}}    
    \end{mathpar}
    
    \judgement{Context Writing}{$\GWt{\Gamma}{p}{\tau}{\Gamma'}$}
    \begin{mathpar}
    \inferrule[$\Gamma$-Wt-Var]
    {\forall y\not=x, \Gamma'(y)=\Gamma(y) 
    \\ \Gamma'(x) = \tau}
    {\GWt{\Gamma}{x}{\tau}{\Gamma'}}
    \and
    \inferrule[$\Gamma$-Wt-Box]
    {\Gamma\vdash p\hookrightarrow \kwd{box}(\_)
    \\ \GWt{\Gamma}{p}{\kwd{box}(\kwd{list}(\alpha))}{\Gamma'}
    }
    {\GWt{\Gamma}{*p}{\kwd{list}(\alpha)}{\Gamma'}}
    \\
    \inferrule*[Right=\rulename{$\Gamma$-Wt-Shared}]
    {\Gamma\vdash p\hookrightarrow \&^\kwd{s}(\_)
    \\ \GWt{\Gamma}{p}{\&^\kwd{s}(\tau)}{\Gamma'}
    }
    {\GWt{\Gamma}{*p}{\tau}{\Gamma'}}
    \\
    \inferrule*[Right=\rulename{$\Gamma$-Wt-Mutable}]
    {\Gamma\vdash p\hookrightarrow \&^\kwd{m}(\tau_\text{c}, \tau_\text{p})
    \\ \vdash \tau_\text{c}
    \\ \GWt{\Gamma}{p}{\&^\kwd{m}(\tau, \tau_\text{p})}{\Gamma'}
    }
    {\GWt{\Gamma}{*p}{\tau}{\Gamma'}}
    \end{mathpar}
    \caption{Context Reading and Writing}
    \label{fig:sta-rw}
\end{figure}

\begin{figure}[t]
\small
    \judgement{Signatures}{$\vdash f \Rightarrow (\Gamma_f, \delta_f)$}
    \begin{mathpar}
    \inferrule
    {\text{fn}~ f ~(\vec{x}_\text{param}:\vec{t}_\text{param}, \vec{x}_\text{local}:\vec{t}_\text{local}, x_\text{ret}:t_\text{ret}) \{~ s ~\}
    \\\\ \textit{enrich}~ \vec{t}_\text{param} ~\textit{as}~ \vec{\tau}_\text{param}
    \\ \textit{enrich}~ \vec{t}_\text{local} ~\textit{as}~ \vec{\tau}_\text{local}
    \\ \textit{enrich}~ t_\text{ret} ~\textit{as}~ \tau_\text{ret}
    \\\\ \GWt{\emptyset}{\vec{x}_\text{param}}{\vec{\tau}_\text{param}}{\Gamma_1} 
    \\ \GWt{\Gamma_1}{\vec{x}_\text{local}}{\vec{\tau}_\text{local}}{\Gamma_2}
    \\ \GWt{\Gamma_2}{x_\text{ret}}{\tau_\text{ret}}{\Gamma_f}
    \\ \delta_f ~\text{fresh} }
    {\vdash f \Rightarrow (\Gamma_f, \delta_f)}
    \end{mathpar}
    \caption{Function Signatures}
    \label{fig:fun-sig}
\end{figure}

\textbf{Rich types} are types enriched with potential annotation $\alpha$ as in \cref{fig:rich-type} and \cref{fig:enrich}. 
%
The rich type $\bot$ denotes zero potential as the minimum among all rich types. 
%
The rich type $\kwd{list}(\alpha)$, represents the potential function $\alpha\cdot n$ for list $l$ with length $n$.
%
In shared borrows $\&^\kwd{s}(\tau)$, $\tau$ represents the potential function of borrowed value. 
% 
Mutable borrows $\&^\kwd{m}(\tau_\text{c}, \tau_\text{p})$ contains 2 components. $\tau_\text{c}$ is the \textbf{c}urrent type, which denotes the current potential of mutable borrow. $\tau_\text{p}$ is the \textbf{p}rophecy type, which denotes the prophecy potential when the mutable borrow ends.  

Typing \textbf{context} $\Gamma : \mathbf{Variable}\to\mathbf{RichType}$ is a partial map, where unused variables can be mapped to $\bot$. Similarly, in \cref{fig:sta-rw}, we extend the reading and writing operation on typing context from variable $x$ to place $p$. It is worth noting that rules \rulename{$\Gamma$-Rd-Mutable} and \rulename{$\Gamma$-Wt-Mutable} indicate to read and write the mutable borrow on its current component $\tau_\text{c}$. We explicitly point out that $\vdash \tau_\text{c}$ in the premise of rule \rulename{$\Gamma$-Wt-Mutable} is \textbf{dropping condition} for soundness, detailed in \cref{sec:inference:subtyping}. Because when we update $\tau_\text{c}$, the old $\tau_\text{c}$ should be restored if it is a mutable borrow.

\textbf{Signature} $(\Sigma_f, \delta_f)$ of a function $f$ compose a typing context $\Sigma_f$ and a resource unknown variable $\delta_f \in \ZZ$. As shown in \cref{fig:fun-sig}, context $\Gamma_f$ contains rich types for parameters, local variables, and the return variable. $\delta_f$ indicates the resource consumption irrelevant to parameters.

\subsection{Subtyping, Well-formedness, and Merging} \label{sec:inference:subtyping}
\begin{figure}[t]
\small
    \judgement{Subtyping}{$\tau_1 \preceq \tau_2$}
    \begin{mathpar}
    \inferrule*[Right=\rulename{S-Bot}]
    {~}
    {\bot\preceq\tau}
    \and
    \inferrule*[Right=\rulename{S-Int}]
    {~}
    {\kwd{i32}\preceq\kwd{i32}}
    \and
    \inferrule*[Right=\rulename{S-Bool}]
    {~}
    {\kwd{bool}\preceq\kwd{bool}}
    \and
    \inferrule*[Right=\rulename{S-List}]
    {\alpha_1 \leq \alpha_2}
    {\kwd{list}(\alpha_1)\preceq\kwd{list}(\alpha_2)}
    \\
    \inferrule[S-Box]
    {\alpha_1 \leq \alpha_2}
    {\kwd{box}(\kwd{list}(\alpha_1))\preceq\kwd{box}(\kwd{list}(\alpha_2))}
    \and
    \inferrule[S-Shared]
    {\tau_1 \preceq \tau_2}
    {\&^\kwd{s}(\tau_1)\preceq\&^\kwd{s}(\tau_2)}
    \and
    \inferrule[S-Mutable]
    {\tau_{\text{c}, 1}\preceq \tau_{\text{c}, 2}
    \\ \tau_{\text{p}, 2}\preceq \tau_{\text{p}, 1} }
    {\&^\kwd{m}(\tau_{\text{c}, 1}, \tau_{\text{p}, 1})\preceq\&^\kwd{m}(\tau_{\text{c}, 2}, \tau_{\text{p}, 2})}
    \end{mathpar}
    \caption{Rich Subtyping}
    \label{fig:rich-subtyping}
\end{figure}
\begin{figure}[t]
\small
    \judgement{Well-formedness}{$\vdash \tau$}
    \begin{mathpar}

    \inferrule[WF-Bot]
    {~}
    {\vdash \bot}
    \and
    \inferrule[WF-Int]
    {~}
    {\vdash \kwd{i32}}
    \and
    \inferrule[WF-Bool]
    {~}
    {\vdash \kwd{bool}}
    \and
    \inferrule[WF-List]
    {\alpha \geq 0}
    {\vdash \kwd{list}(\alpha)}
    \and
    \inferrule[WF-Box]
    {\alpha \geq 0}
    {\vdash \kwd{box}(\kwd{list}(\alpha))}
    \\
    \inferrule*[Right=\rulename{WF-Shared}]
    {\vdash \tau
    }
    {\vdash \&^\kwd{s}(\tau)}
    \and
    \inferrule*[Right=\rulename{WF-Mutable}]
    {\tau_\text{p} \preceq \tau_\text{c}
    \\ \vdash \tau_\text{c}
    \\ \vdash \tau_\text{p}
    }
    {\vdash \&^\kwd{m}(\tau_\text{c}, \tau_\text{p})}
    \end{mathpar}
    \caption{Well-formedness}
    \label{fig:rich-type-wf}
\end{figure}
\begin{figure}[t]
\small
    \judgement{Context Merging}{$\Gamma_1 \sqcap \Gamma_2 = \{ x \hookrightarrow \Gamma_1(x)\cap\Gamma_2(x) : x \in \mathbf{dom}(\Gamma_1)=\mathbf{dom}(\Gamma_2)\}$}
    \judgement{Meet/Join (Selected)}{$\tau_1\cap\tau_2 / \tau_1\cup\tau_2$}
    \begin{mathpar}        
    \inferrule*[Right=Meet-List]
    {\min(\alpha_1, \alpha_2)=\alpha}
    {\kwd{list}(\alpha_1)\cap\kwd{list}(\alpha_2)=\kwd{list}(\alpha)}
    \and
    \inferrule*[Right=Join-List]
    {\max(\alpha_1, \alpha_2)=\alpha}
    {\kwd{list}(\alpha_1)\cup\kwd{list}(\alpha_2)=\kwd{list}(\alpha)}
    \\
    
    \inferrule*[Right=Meet-Shared]
    {\tau_1 \cap \tau_2=\tau}
    {\&^\kwd{s}(\tau_1)\cap\&^\kwd{s}(\tau_2)=\&^\kwd{s}(\tau)}
    \and
    \inferrule*[Right=Join-Shared]
    {\tau_1 \cup \tau_2=\tau}
    {\&^\kwd{s}(\tau_1)\cup\&^\kwd{s}(\tau_2)=\&^\kwd{s}(\tau)}
    \\

    \inferrule*[Right=Meet-Mutable]
    {\tau_{\text{c}, 1} \cap \tau_{\text{c}, 2}=\tau_\text{c}
    \\ \tau_{\text{p}, 1} \cup \tau_{\text{p}, 2}=\tau_\text{p}
    \\ \tau_{\text{p}, 1} \preceq \tau_{\text{c}, 1}
    \\ \tau_{\text{p}, 2} \preceq \tau_{\text{c}, 2}
    }
    {\&^\kwd{m}(\tau_{\text{c}, 1}, \tau_{\text{p}, 1})\cap\&^\kwd{m}(\tau_{\text{c}, 2}, \tau_{\text{p}, 2})=\&^\kwd{m}(\tau_\text{c}, \tau_\text{p})}
    \\
    \inferrule*[Right=Join-Mutable]
    {\tau_{\text{c}, 1} \cup \tau_{\text{c}, 2}=\tau_\text{c}
    \\ \tau_{\text{p}, 1} \cap \tau_{\text{p}, 2}=\tau_\text{p}
    \\ \tau_{\text{p}, 1} \preceq \tau_{\text{c}, 1}
    \\ \tau_{\text{p}, 2} \preceq \tau_{\text{c}, 2}
    }
    {\&^\kwd{m}(\tau_{\text{c}, 1}, \tau_{\text{p}, 1})\cup\&^\kwd{m}(\tau_{\text{c}, 2}, \tau_{\text{p}, 2})=\&^\kwd{m}(\tau_\text{c}, \tau_\text{p})}
    \end{mathpar}
    \caption{Merging}
    \label{fig:sta-merging}
\end{figure}

The order relation $\leq$ on resources derives another order relation on rich types, the \emph{subtyping} relation in \cref{fig:rich-subtyping}. The interpretation of subtyping $\tau_1 \preceq \tau_2$ is that the value $v$ typed with $\tau_1$ has \textbf{less} resource than the value $v$ typed with $\tau_2$. 
%
The rich type $\bot$ is a subtype of any type because $\bot$ denotes zero potential.
%
It is worth noting that in \rulename{S-Mutable}, $\tau_\text{p}$ is contravariant because prophecy type $\tau_\text{p}$ denotes the prophecy potential to return. 
%
The reflexive rule and the transitive rule are derivable.

A well-formed rich type always denotes a non-negative potential function. Our type system can drop well-formed types without sacrificing soundness.
%
\rulename{WF-List} and \rulename{WF-Box} request $\alpha \geq 0$, which makes $\alpha \cdot n \geq 0$ for list $l$ with length $n\geq0$. 
%
\rulename{WF-Shared} is a structural rule. For example, if $\kwd{list}(\alpha)$ is well-formed, so is $\&^\kwd{s}(\kwd{list}(\alpha))$. Rust borrow checker ensures that $\tau$ in $\&^\kwd{s}(\tau)$ satisfies $\tau\not=\&^\kwd{m}(\_, \_)$. Our type system supports nested borrows.
%
Besides structural premises $\vdash \tau_\text{c}$ and $\vdash \tau_\text{p}$, \rulename{WF-Mutable} demands \textbf{dropping condition} $\tau_\text{p} \preceq \tau_\text{c}$ in $\&^\kwd{m}(\tau_\text{c}, \tau_\text{p})$. The condition is called dropping condition because it works as dropping mutable borrows in \cref{fig:ex-prophecy}. The dropping condition makes sure that mutable borrow types denote non-negative potentials, as illustrated in \cref{sec:soundness}. 


\textbf{Merging} is a conservative approximation of resource potentials after conditional branching. Under typing context $\Gamma$, the type system checks statements $s_1$ and $s_2$ in different branches and gets remainder contexts $\Gamma_1$ and $\Gamma_2$.
The type system should merge them to check continuation. 
%
As illustrated in \cref{fig:sta-merging}, to merge typing contexts is to merge rich types at each $x$ in the domain of two contexts. 
% 
Because the prophecy type $\tau_\text{p}$ in mutable borrow is contravariant, we need to define not only the meet of types but also the join of types. Our purpose is to construct a \emph{lattice} with the property that $\tau_1\cap\tau_2\preceq \tau_i \preceq\tau_1\cup\tau_2, \forall i=1, 2$. Hence, merging over contexts is non-increasing on resources to conservatively fulfill soundness. 
%
The lattice operations of $\kwd{list}(\alpha_1)$ and $\kwd{list}(\alpha_1)$ are inherited from the resource's $\min$ and $\max$, so it is readily comprehensible. 
%
Notice that dropping conditions appear in rules \rulename{Meet-Mutable} and \rulename{Join-Mutable}. They are to fulfill soundness for weak updates, which is mentioned in \cref{sec:overview:Lattice}. Recall that dropping borrows without these conditions may increase resources in both original places indicated by $\tau_{\text{p}, 1}$ and $\tau_{\text{p}, 2}$, to break soundness. 

% In practical implementation, the well-formedness here $\vdash\&^\kwd{m}(\tau_{\text{c}, i}, \tau_{\text{p}, i})$ can be exchanged to $\tau_{\text{p}, i}\preceq\tau_{\text{c}, i}$, due to redundant well-formedness for substructures, $\vdash\tau_{\text{c}, i}$ and $\vdash\tau_{\text{p}, i}$ already covered by induction hypothesis on $\tau_{\text{c}, 1}\cap/\cup\tau_{\text{c}, 2}$ and $\tau_{\text{c}, 1}\cup/\cap\tau_{\text{p}, 2}$. 

It is worth noting that we support nested borrows like $\&^\kwd{s}(\&^\kwd{s}(\tau))$, $\&^\kwd{m}(\&^\kwd{s}(\tau_\text{c}), \&^\kwd{s}(\tau_\text{p}))$ and $\&^\kwd{m}(\&^\kwd{m}(\tau_\text{cc}, \tau_\text{cp}), \&^\kwd{m}(\tau_\text{cc}, \tau_\text{pp}))$. Rust's borrow mechanisms exclude nested borrows like shared borrows of mutable borrows $\&^\kwd{s}(\&^\kwd{m}(\tau_\text{c}, \tau_\text{p}))$, because they violate the property that at most one mutable borrow from the same piece of data is live at the same time.

\subsection{Typing Expressions} \label{sec:inference:eval}\
\begin{figure}[t]
\small
\judgement{Typing Expressions (Selected)}{$\Gamma\vdash e \hookrightarrow \tau\dashv\Gamma'$}
    \begin{mathpar}
    \inferrule*[Right=\rulename{$\Gamma$-Ev-Nil}]
    {\alpha ~\text{fresh}}
    {\Gamma\vdash \kwd{nil} \hookrightarrow \kwd{list}(\alpha)\vdash\Gamma}
    \and
    \inferrule*[Right=\rulename{$\Gamma$-Ev-Move}]
    {\Gamma\vdash p \hookrightarrow \tau
    \\ \GWt{\Gamma}{p}{\bot}{\Gamma'}
    }
    {\Gamma\vdash \kwd{move}~p \hookrightarrow \tau\dashv\Gamma'}
    \\

    \inferrule*[Right=\rulename{$\Gamma$-Ev-Shared}]
    {\Gamma\vdash p \hookrightarrow \tau
    \\ \textit{share}~ \tau ~\textit{as}~\tau_1, \tau_2
    \\ \GWt{\Gamma}{p}{\tau_1}{\Gamma'}
    }
    {\Gamma\vdash \&^\kwd{s}~p \hookrightarrow \&^\kwd{s}(\tau_2)\dashv\Gamma'}
    \\
    
    \inferrule*[Right=\rulename{$\Gamma$-Ev-Mutable}]
    {\Gamma\vdash p \hookrightarrow \tau
    \\ \textit{prophesy}~ \tau ~\textit{as}~ \tau_\text{p} 
    \\ \GWt{\Gamma}{p}{\tau_\text{p}}{\Gamma'}
    }
    {\Gamma\vdash \&^\kwd{m}~p \hookrightarrow \&^\kwd{m}(\tau, \tau_\text{p})\dashv\Gamma'}
    \end{mathpar}
    \caption{Typing Expressions}
    \label{fig:sta-eval}
\end{figure}

\begin{figure}[t]
\small
    \judgement{Sharing (Selected)}{$\textit{share}~ \tau ~\textit{as}~\tau_1, \tau_2$}
    \begin{mathpar}
    \inferrule*[Right=\rulename{Share-List}]
    {\alpha_1, \alpha_2 ~\text{fresh}
    \\\alpha = \alpha_1 + \alpha_2}
    {\textit{share}~ \kwd{list}(\alpha) ~\textit{as}~\kwd{list}(\alpha_1), \kwd{list}(\alpha_2)}
    \end{mathpar}
    \judgement{Prophesying (Selected)}{$\textit{prophesy}~ \tau_\text{c} ~\textit{as}~\tau_\text{p}$}
    \begin{mathpar}
    \inferrule*[Right=\rulename{Prophesy-List}]
    {\alpha_\text{p}~\text{fresh}}
    {\textit{prophesy}~ \kwd{list}(\alpha) ~\textit{as}~ \kwd{list}(\alpha_\text{p})}
    \end{mathpar}
    \caption{Sharing and Prophesying}
    \label{fig:sta-sharing-prophesying}
\end{figure}

\cref{fig:sta-eval} presents how to type check expressions via judgement $\Gamma\vdash e\hookrightarrow \tau\dashv\Gamma'$. Unlike the dynamic evaluation $V\vdash e \rightsquigarrow v$, checking expressions may modify $\Gamma$ to the remainder context $\Gamma'$. 
%
Rule \rulename{$\Gamma$-Ev-Nil} introduces a fresh unknown potential annotation $\alpha$ for $\kwd{nil}$. 
%
Rule \rulename{$\Gamma$-Ev-Move} explicitly moves the type $\tau$ out from place $p$, making $\GWt{\Gamma}{p}{\bot}{\Gamma'}$. 
%

Shared and mutable borrows modify typing context, as illustrated in rule \rulename{$\Gamma$-Ev-Shared} and \rulename{$\Gamma$-Ev-Mutable} with sharing and prophesying. \cref{fig:sta-sharing-prophesying} selects essential rules of $\textit{share}~ \tau ~\textit{as}~ \tau_1, \tau_2$ and $\textit{prophesy}~ \tau ~\textit{as}~ \tau_\text{p}$ for borrows. 

\textbf{Shared borrows} are handled with sharing $\textit{share}~ \tau ~\textit{as}~ \tau_1, \tau_2$. Recall the example in \cref{fig:ex-sharing}. We select the rule \rulename{Share-List} to reveal the essence of sharing. Sharing is splitting resource annotation $\alpha$ into $\alpha_1$ and $\alpha_2$ with linear constraint $\alpha = \alpha_1 + \alpha_2$. In rule \rulename{$\Gamma$-Ev-Shared}, we write $\tau_1$ back to original place $p$, with $\tau_2$ lent out. There is no sharing of mutable borrows as $\textit{share}~ \&^\kwd{m}(\_, \_) ~\textit{as}~ \tau_1, \tau_2$, because a well-checked program will never incur shared borrows of mutable borrows $\&^\kwd{s}(\&^\kwd{m}(\tau_\text{c}, \tau_\text{p}))$.

\textbf{Mutable borrows} are handled with prophesying $\textit{prophesy}~ \tau ~\textit{as}~ \tau_\text{p}$. Recall the example in \cref{fig:ex-prophecy}. The selected rule \rulename{Prophesy-List} prophesy $\alpha_\text{p}$ as the prophecy potential when the mutable borrow ends. In rule \rulename{$\Gamma$-Ev-Mutable}, we write prophecy type $\tau_\text{p}$ to the place $p$. Once the borrow ends, the dropping condition $\vdash \&^\kwd{m}(\tau, \tau_\text{p})$ ensures that the prophecy type $\tau_\text{p}$ is bounded by current type $\tau$.

\subsection{Typing Statements} \label{sec:inference:exec}
\begin{figure}[t]
\small
    \judgement{Typing Statements (Selected)}{$\Gamma\vdash s \hookrightarrow^\delta \dashv\Gamma'$}
    \begin{mathpar}
    \inferrule*[Right=\rulename{$\Gamma$-Ex-Tick}]
    {~}
    {\Gamma\vdash\kwd{tick}(\delta)\hookrightarrow^\delta\vdash\Gamma}
    \and
    \inferrule*[Right=\rulename{$\Gamma$-Ex-Drop}]
    {\Gamma\vdash p\hookrightarrow \tau
    \\ \vdash \tau
    \\ \GWt{\Gamma}{p}{\bot}{Gamma'}
    }
    {\Gamma\vdash \kwd{drop}~p \hookrightarrow^0\dashv \Gamma'}
    \\

    \inferrule*[Right=\rulename{$\Gamma$-Ex-Cons}]
    {\Gamma\vdash e_1\hookrightarrow \kwd{i32} \dashv \Gamma_1
    \\ \Gamma_1\vdash e_2\hookrightarrow \kwd{box}(\kwd{list}(\alpha'))\dashv\Gamma_2
    \\ \GWt{\Gamma_2}{p}{\kwd{list}(\alpha')}{\Gamma'}}
    {\Gamma\vdash p\from \kwd{cons}(e_1, e_2)\hookrightarrow^{\alpha'}\dashv\Gamma'}
    \\

    \inferrule*[Right=\rulename{$\Gamma$-Ex-If}]
    {\Gamma\vdash p\hookrightarrow \kwd{bool}
    \\ \Gamma\vdash s_1\hookrightarrow^{\delta_1}\dashv\Gamma_1
    \\ \Gamma\vdash s_2\hookrightarrow^{\delta_2}\dashv\Gamma_2
    \\ \max(\delta_1, \delta_2)=\delta
    \\ \Gamma_1\sqcap\Gamma_2=\Gamma' }
    {\Gamma\vdash \kwd{if}~ p ~\kwd{then}~ s_1 ~\kwd{else}~ s_2 ~\kwd{end} \hookrightarrow^\delta \dashv\Gamma'}
    \\
    \inferrule*[Right=\rulename{$\Gamma$-Ex-Mat}]
    {\Gamma\vdash p\hookrightarrow \kwd{list}(\alpha)
    \\ \Gamma\vdash s_1\hookrightarrow^{\delta_1}\dashv\Gamma_1
    \\\\ \GWt{\Gamma}{p}{\bot}{\Gamma_{\text{b}, 1}}
    \\ \GWt{\Gamma_{\text{b}, 1}}{x_\text{hd}}{\kwd{i32}}{\Gamma_{\text{b}, 2}}
    \\ \GWt{\Gamma_{\text{b}, 2}}{x_\text{tl}}{\kwd{box}(\kwd{list}(\alpha))}{\Gamma_\text{b}}
    \\ \Gamma_\text{b}\vdash s_2\hookrightarrow^{\delta_2}\dashv\Gamma'_\text{b}
    \\\\ \Gamma'_\text{b}\vdash x_\text{tl}\hookrightarrow \kwd{list}(\beta)
    \\ \GWt{\Gamma'_\text{b}}{x_\text{hd}}{\bot}{\Gamma'_{\text{b}, 1}}
    \\ \GWt{\Gamma'_{\text{b}, 1}}{x_\text{tl}}{\bot}{\Gamma'_{\text{b}, 2}}
    \\ \GWt{\Gamma'_{\text{b}, 2}}{p}{\kwd{list}(\beta)}{\Gamma_2}
    \\\\ \max(\delta_1, \delta_2-(\alpha-\beta))=\delta
    \\ \Gamma_1\sqcap\Gamma_2=\Gamma'}
    {\Gamma\vdash \kwd{match}~ p ~ \{\kwd{nil}\mapsto s_1, \kwd{cons}(x_\text{hd}, x_\text{tl})\mapsto s_2\} \hookrightarrow^\delta \dashv\Gamma'}
    \\

    \inferrule*[Right=\rulename{$\Gamma$-Ex-App}]
    {\text{fn}~ f ~(\vec{x}_\text{param}:\vec{t}_\text{param}, \vec{x}_\text{local}:\vec{t}_\text{local}, x_\text{ret}:t_\text{ret}) \{~ s ~\}
    \\\\ \vdash f \Leftarrow (\Gamma_f, \delta_f)
    \\ \Gamma_f\vdash x_\text{ret} \hookrightarrow \tau_\text{ret}, (\forall x_i\in\vec{x}_\text{param}, i=1, ..., n) \Gamma_f \vdash x_i \hookrightarrow \tau_{\text{param}, i}
    \\\\ \Gamma_0=\Gamma, (\forall e_i\in \vec{e}, i=1, ..., n) \Gamma_{i-1}\vdash e_i\hookrightarrow\tau_{\text{arg}, i}\dashv\Gamma_i
    \\ (\forall i=1,..,n)~ \tau_{\text{param}, i} = \tau_{\text{arg}, i}
    \\ \Gamma_n \vdash p \hookrightarrow \tau
    \\ \vdash \tau
    \\ \GWt{\Gamma_n}{p}{\tau_\text{ret}}{\Gamma'}
    }
    {\Gamma\vdash p\from f(\vec{e})\hookrightarrow^{\delta_f}\dashv\Gamma'}
    \end{mathpar}
    \caption{Typing Statements}
    \label{fig:sta-exec}
\end{figure}

\begin{figure}[t]
\small
    \judgement{Function Analysis}{$\vdash f \Leftarrow (\Gamma_f, \delta_f)$}
    \begin{mathpar}
    \inferrule
    {\text{fn}~ f ~(\vec{x}_\text{param}:\vec{t}_\text{param}, \vec{x}_\text{local}:\vec{t}_\text{local}, x_\text{ret}:t_\text{ret}) \{~ s ~\}
    \\  \vdash f \Rightarrow (\Gamma_f, \delta_f)
    \\ \Gamma_f\vdash s\hookrightarrow^\delta\dashv\Gamma'_f
    \\\\ \forall x \in \textbf{dom}(\Gamma'_f), \vdash \Gamma'_f(x)
    \\ \Gamma'_f \vdash x_\text{ret} \hookrightarrow \tau'_\text{ret}
    \\ \Gamma_f \vdash x_\text{ret} \hookrightarrow \tau_\text{ret}
    \\ \tau'_\text{ret} = \tau_\text{ret}
    \\ \delta = \delta_f}
    {\vdash f \Leftarrow (\Gamma_f, \delta_f)}
    \end{mathpar}
    \caption{Function Analysis}
    \label{fig:fun-anal}
\end{figure}

\cref{fig:sta-exec} presents how to type check statements as judgement $\Gamma\vdash s \hookrightarrow^\delta \dashv\Gamma'$. Under context $\Gamma$, the statement $s$ is checked with resource consumption $\delta$, and context becomes $\Gamma'$. 

Rule \rulename{$\Gamma$-Ex-Tick} indicates $\kwd{tick}(\delta)$ consumes $\delta$ unit of resource. Rule \rulename{$\Gamma$-Ex-Drop} drops the type $\tau$ with well-formedness $\vdash\tau$ as the dropping condition. Rule \rulename{$\Gamma$-Ex-Cons} indicates that $\kwd{cons}$ will consume $\alpha$ unit of resource for continuation payment, when the tail $e_2$ is typed with $\kwd{box}(\kwd{list}(\alpha))$. 

\textbf{Branching statements} require context merging, as in \rulename{$\Gamma$-Ex-If} and \rulename{$\Gamma$-Ex-Mat}. Rule \rulename{$\Gamma$-Ex-If} is simpler to merge contexts with the consumption as the maximum of those branches. Rule \rulename{$\Gamma$-Ex-Mat} is more intricate, due to resource potential stored in \kwd{cons}. The $\kwd{cons}$ branch will obtain $\alpha-\beta$ units of potential, therefore the net consumption is $\delta_2-(\alpha-\beta)$. Given $\Gamma\vdash p \hookrightarrow \kwd{list}(\alpha)$, The potential is not $\alpha$ but $\alpha-\beta$. $\beta$ is the remainder potential, indicated by $\Gamma'_\text{b} \vdash x_\text{tl} \hookrightarrow \kwd{list}(\beta)$. The subscript $\text{b}$ of $\Gamma_\text{b}$ means \textbf{b}inding, similar to rule \rulename{V-Ex-Mat}.

\textbf{Function application} is intractable because of recursive functions. Rule \rulename{$\Gamma$-Ex-App} assumes the function $f$ has a well-checked signature $(\Gamma_f, \delta_f)$, with judgement $\vdash f \Leftarrow (\Gamma_f, \delta_f)$ in \cref{fig:fun-anal}, different from $\vdash f \Rightarrow (\Gamma_f, \delta_f)$. Other premises are to ensure that the resources of actual arguments are equal to those of formal parameters. 

\subsection{Type Inference} \label{sec:inference:infer}
To this point, our type system has been primarily declarative because the well-checked signature in rule \rulename{$\Gamma$-Ex-App} is assumed to be pre-existent. Same as other AARA systems (such as Resource-aware ML~\cite{RaML}), we use linear programming to convert the declarative type system to an algorithmic version. The type system creates symbolic variables to denote unknown annotations in rich types and signatures. The type system then collects linear constraints among those symbolic variables and finally solves them via linear programming solvers. 

Readers might have perceived that a recursive function requires a well-checked signature during checking and that a function can exhibit multiple signatures at different call sites. To automatically analyze functions, we need to preprocess the call graph. First, we group recursive functions as strongly connected components. Second, we topologically sort groups to determine an order to analyze. For each group, we predefine signatures of functions in the group via the judgement $\textit{enrich}~ t ~\textit{as}~ \tau$ in \cref{fig:enrich}. During function analysis, the signature $(\Gamma_f, \delta_f)$ in \rulename{$\Gamma$-Ex-App} should be replaced with the predefined one if $f$ is in the group. Otherwise, $f$ is in the previously analyzed group, so we should clone that group's signature and linear constraints. It is necessary to clone instead of copy them because annotations in signatures and constraints are sensitive to actual arguments of function calls.

With linear constraints collected during function analysis and a heuristic objective, we can employ a linear programming solver to find instances of annotations that satisfy those constraints automatically. The inferred annotations in signatures will characterize functions' resource consumption. 

% The $\vdash f \Rightarrow (\Gamma_f, \delta_f)$ in premises of $\vdash f \Leftarrow (\Gamma_f, \delta_f)$, and function application in body $s$, require a topological order of functions to analyze. And (mutual) recursive functions require a strongly connected grouping of functions. In our implementation, all these requirements turn into strongly connected component analysis and topological sorting.
\section{Soundness} \label{sec:soundness}

In this section, we define potential functions indicated by the resource-enriched types from \cref{sec:inference} in \cref{sec:pot-funcs} and then sketch the soundness proof of the resource-aware type system in \cref{sec:proof-sketch}. We include the detailed proofs in the \cref{sec:proof}.

\subsection{Potential Functions}
\label{sec:pot-funcs}
 
% We always assume that $v$ is well typed with $\tau$, stated as $\vdash v : \tau$, when we notate $\phi(v:\tau)$. We omit the definition in literal, as well as well typing over store and context as $\vdash V : \Gamma$, both guaranteed by Rust type system.

% \begin{definition}
%    $ \Phi(V:\Gamma) = \sum_{x\in\textbf{dom}(\Gamma)} \phi(V(x):\Gamma(x)) $ .
% \end{definition}
\begin{figure}[t]
\small
    \judgement{Potential Functions}{$\Phi(V:\Gamma) = \sum_{x\in\textbf{dom}(\Gamma)} \phi(V(x):\Gamma(x))$} \\
    \judgement{Potential Functions (Selected)}{$\phi(v:\tau)$}
    \begin{mathpar}
    \inferrule[$\phi$-Nil]
    {~}
    {\phi(\kwd{nil}:\kwd{list}(\alpha))=0}
    \and
    \inferrule[$\phi$-Cons]
    {~}
    {\phi(\kwd{cons}(\kwd{n}_\text{i32}, \kwd{box}(lv)):\kwd{list}(\alpha))=\alpha+\phi(\kwd{box}(lv):\kwd{box}(\kwd{list}(\alpha)))}
    \\
    \inferrule[$\phi$-Shared]
    {~}
    {\phi(\&(\_, v):\&^\kwd{s}(\tau))=\phi(v:\tau)}
    \and
    \inferrule[$\phi$-Mutable]
    {~}
    {\phi(\&(\_, v):\&^\kwd{m}(\tau_\text{c}, \tau_\text{p}))=\phi(v:\tau_\text{c})-\phi(v:\tau_\text{p})}
    \end{mathpar}
    \caption{Potential Functions}
    \label{fig:sound-potential}
\end{figure}

\cref{fig:sound-potential} defines potential function $\phi(v:\tau)$ and $\Phi(V:\Gamma)$.
% 
\rulename{$\phi$-Nil} and \rulename{$\phi$-Cons} define the potential of a list $l : \kwd{list}(\alpha)$ with length $n$ to be $\alpha \cdot n$.
%
\rulename{$\phi$-Shared} defines the potential of shared borrows to be the potential of borrowed values and borrowed rich types. 
%
\rulename{$\phi$-Mutable} defines the potential of mutable borrows to be the difference between current and prophecy potential. It is worth noting that when the program incurs a mutable borrow, the potential of context will not change. The dropping condition $\tau_\text{p} \preceq \tau_\text{c}$ in $\vdash \&^\kwd{m}(\tau_\text{c}, \tau_\text{p})$, ensures the potential is non-negative. We, therefore have the following lemma about potential:

\begin{lemma}
    Potential is non-negative and keeps subtyping:
    \begin{mathpar}
    \inferrule
    {  \vdash \tau_1
    \\ \vdash \tau_2
    \\ \tau_1 \preceq \tau_2
    }
    { 0 \leq \phi(v:\tau_1) \leq \phi(v:\tau_2)}
    \end{mathpar}
\end{lemma}
\begin{proof}
    First define the size of rich types structural inductively $\mathbf{size} : \mathbf{RichType} \to \NN$, \ 
    and prove by induction on $\mathbf{size}(\tau_1) + \mathbf{size}(\tau_2)$. The well-formedness will be used to prove potential of mutable borrows is non-negative.
\end{proof}
\begin{corollary}
    Potential is non-negative $\dfrac{\vdash \tau}{0\leq \phi(v:\tau)}$ due to the derivable rule \rulename{S-Refl} $\dfrac{~}{\tau\preceq\tau}$.
\end{corollary}

The subtyping relation and well-formedness can be extended to typing context, which can be conceptualized as record types. Lemmas about the lattice operation and potential can be proved with definition and simple induction.

\begin{definition}
    Subcontext $\Gamma_1 \subseteq \Gamma_2$ if and only if $\forall x \in \textbf{dom}(\Gamma_1), x \in \textbf{dom}(\Gamma_2), \Gamma_1(x)\preceq \Gamma_2(x)$. \\
    Context well formed $\vdash \Gamma$ if and only if $\forall x \in \textbf{dom}(\Gamma), \vdash \Gamma(x)$.
\end{definition}
\begin{lemma}[]
    For any store $V$ and any context $\Gamma_1, \Gamma_2$, 
    \begin{enumerate}
        \item {$\Gamma_1\sqcap\Gamma_2 \subseteq \Gamma_1$ and $\Gamma_1\sqcap\Gamma_2 \subseteq \Gamma_2$; }
        \item {if $\vdash \Gamma_1, \vdash \Gamma_2$, then $\vdash \Gamma_1\sqcap\Gamma_2$; }
        \item {if $\vdash \Gamma_1, \vdash \Gamma_2, \Gamma_1 \subseteq \Gamma_2$, then $0\leq \Phi(V:\Gamma_1) \leq \Phi(V:\Gamma_2)$. }
    \end{enumerate}
\end{lemma}


\subsection{Soundness Theorem}
\label{sec:proof-sketch}

With potential, we are able to formulate soundness theorem, which states that resource consumption of dynamics, together with potential difference, is bounded by type system.
\begin{theorem}[Soundness]
Our type system is sound towards resource aware dynamic semantics: \\
If $V\vDash s \rightsquigarrow^{\delta_V} \Dashv V'$ and $\Gamma \vdash s \hookrightarrow^{\delta_\Gamma} \dashv \Gamma'$, then $\Phi(V':\Gamma') - \Phi(V:\Gamma)+\delta_V \leq \delta_\Gamma$.
\end{theorem}
\begin{proof}
By induction on $V\vDash s \rightsquigarrow^{\delta_V} \Dashv V'$, with the help of following lemmas.
\end{proof}

% Lemmas are listed as follows, and proof is simply done by induction with borrow properties.

\begin{lemma}[Update]
If store or context is written with new value or new type, the difference of potential over store or context is equal to that over new value or new type.
\begin{enumerate}
    \item {If $V\vDash p\rightsquigarrow v$, $\Gamma\vdash p\hookrightarrow \tau$ and $\VWt{V}{p}{v'}{V'}$,
    then $\Phi(V':\Gamma) - \Phi(V:\Gamma)=\Phi(v':\tau)-\Phi(v:\tau)$;}
    \item {If $V\vDash p\rightsquigarrow v$, $\Gamma\vdash p\hookrightarrow \tau$ and $\GWt{\Gamma}{p}{\tau'}{\Gamma'}$, 
    then $\Phi(V:\Gamma')-\Phi(V:\Gamma)=\Phi(v:\tau')-\Phi(v:\tau)$.}
\end{enumerate}
\end{lemma}
\begin{proof}
By induction on $\VWt{V}{p}{v'}{V'}$ and $\GWt{\Gamma}{p}{\tau'}{\Gamma'}$.
\end{proof}
\begin{lemma}[Evaluation]
If $V\vDash e\rightsquigarrow v, \Gamma\vdash e\hookrightarrow \tau\dashv\Gamma'$, then $\Phi(V:\Gamma')-\Phi(V:\Gamma)= -\phi(v:\tau)$.
\end{lemma}
\begin{proof}
By induction on $e$.
\end{proof}

Towards soundness proof for statements, especially for the rule \rulename{$\Gamma$-Ex-App}, we will need one series of lemmas about context extension. Though similar, context extension is different from subcontext. Context extension necessitates strict type equality, whereas subcontext only demands subtyping. The weakening rules are intuitive about context extension, as they merely involve adding variables and types that the program will not utilize. 

\begin{definition}
    Extension $\Gamma_1\sqsubseteq\Gamma_2$ if and only if $\forall x \in\textbf{dom}(\Gamma_1), x\in\textbf{dom}(\Gamma_2), \Gamma_1(x) = \Gamma_2(x)$.
\end{definition}
\begin{lemma}[Weakening]
In following rules, $\dfrac{A ~ B}{C ~ D}$ means if $A$ and $B$ then $C$ and $D$.
\begin{mathpar}
    \inferrule[$\Gamma$-Rd-Weaken]
    {\Gamma_1 \sqsubseteq \Gamma_2 
    \\ \Gamma_1 \vdash p \hookrightarrow \tau_1
    }
    {\Gamma_2 \vdash p \hookrightarrow \tau_2
    \\ \tau_1 = \tau_2
    }
    \and
    \inferrule[$\Gamma$-Wt-Weaken]
    {\Gamma_1 \sqsubseteq \Gamma_2
    \\ \GWt{\Gamma_1}{p}{\tau_1}{\Gamma'_1}
    \\ \tau_1 = \tau_2
    }
    { \GWt{\Gamma_2}{p}{\tau_2}{\Gamma'_2}
    \\ \Gamma'_1 \sqsubseteq \Gamma'_2 
    }
    \\
    \inferrule[$\Gamma$-Ev-Weaken]
    {\Gamma_1 \sqsubseteq \Gamma_2
    \\ \Gamma_1 \vdash e \hookrightarrow \tau_1 \dashv \Gamma'_1 
    }
    {\Gamma_2 \vdash e \hookrightarrow \tau_2 \dashv \Gamma'_2
    \\ \tau_1 = \tau_2 
    \\ \Gamma'_1 \sqsubseteq \Gamma'_2
    }
    \and
    \inferrule[$\Gamma$-Ex-Weaken]
    {\Gamma_1 \sqsubseteq \Gamma_2
    \\ \Gamma_1 \vdash s \hookrightarrow^{\delta_1} \dashv \Gamma'_1
    }
    { \Gamma_2 \vdash s \hookrightarrow^{\delta_2} \dashv \Gamma'_2
    \\ \delta_1 = \delta_2
    \\ \Gamma'_1 \sqsubseteq \Gamma'_2
    }
\end{mathpar}
\end{lemma}
\begin{proof}
    By induction on $\Gamma_1\vdash p\hookrightarrow \tau_1$, $\GWt{\Gamma_1}{p}{\tau_1}{\Gamma'_1}$, $\Gamma_1\vdash e\hookrightarrow\tau_1\dashv\Gamma'_1$, $\Gamma_1\vdash s\hookrightarrow^{\delta_1}\dashv\Gamma'_1$.
\end{proof}


\section{Experimental Evaluation} \label{sec:impl}

In this section, we present an experimental evaluation of \rarust{}.
%
\cref{sec:proto} describes our prototype implementation of \rarust{}.
%
\cref{sec:eval} presents the evaluation results of \rarust{} on a suite of benchmarks.

\subsection{Implementation}
\label{sec:proto}

% In this section, we will describe the work flow, and supported features of implementation. 

We have implemented a prototype linear resource analyzer, \rarust{}. It takes raw Rust programs (within only $\kwd{tick}$ annotation) as inputs, and prints functions' signatures with resource annotation as output.
%
\rarust{} analyzes the whole program regardless of whether there are annotations or not. We currently use explicit manually annotated $\kwd{tick}$ as the cost model, but it is straightforward to support parametric cost models, which assign a cost to each kind of statements, as prior AARA systems (e.g., \textsc{RaML}~\cite{RaML}) do.
(1) \rarust{} first obtains the typed IR of the borrow calculus with explicit drops via Charon \cite{Aeneas} as a plugin of Rust compiler. Rust compiler guarantees that the compiled IR is well-checked, and \rarust{} will utilize properties straightforwardly. 
(2) Towards IR, \rarust{} analyzes strongly connected components of function call graph and topologically sort components, generating precedence of type checking. 
(3) \rarust{} enriches function signatures with fresh linear variables as resource annotation and assigns each component a linear programming context to record linear constraints. 
(4) As stated in \cref{sec:inference:infer}, \rarust{} type-checks functions' bodies, generating linear constraints.
% For function call, when the function is in the same component, \rarust{} will only use the signature; when the function is in previous component, \rarust{} will clone previous linear programming context, add to current context, and use the cloned signature.
(5) \rarust{} finally solves these constraints, with its heuristic linear objective, by invoking linear programming solvers. We can utilize different solvers due to the separation of (4) and (5), and we select CoinCbc \cite{CoinCbc} for demonstration.

Our implementation supports some features, based on core calculus formalized in \cref{sec:calculus} and some practical extensions. We list as follows:
\begin{enumerate}
    \item{
    \rarust{} supports reborrows and nested borrows like \lstinline|& & T|, \lstinline|&mut & T| and \lstinline|&mut &mut T|.
    }
    \item {
    \rarust{} supports user-defined data types instead of only built-in lists, as \cref{tab:user-defined} shows. 
    We annotate potential for each constructor, \lstinline|Nil|, \lstinline|Cons|, \lstinline|Leaf|, \lstinline|Node|, \lstinline|(_, _)|, \lstinline|Record(_, _)|, \lstinline|NListNode| and \lstinline|NList|. It is worth noting that \lstinline|NListNode| and \lstinline|NList| are mutually recursive.
    }
    \item {
    \rarust{} supports looping statements including $\kwd{while true}(s), \kwd{break}(i), \kwd{continue}(i)$, where $i$ represents the $i$-th loop outward from $\kwd{break}$ or $\kwd{continue}$. \rarust{} supports this directly without transforming loops into recursive functions.
    }
\end{enumerate}


\begin{DIFnomarkup}
\begin{table}[t]
\centering
\caption{User-defined Data Types} \label{tab:user-defined}
\footnotesize
\begin{tabular}{|l|l|}
\hline
\begin{lstlisting}[language=Rust, style=colouredRust]
pub enum List {
    Nil,
    Cons(i32, Box<List>),
}
\end{lstlisting} 
&
\begin{lstlisting}[language=Rust, style=colouredRust]
pub enum Tree {
    Leaf,
    Node(i32, Box<Tree>, Box<Tree>),
}
\end{lstlisting} 
\\
\hline
\begin{lstlisting}[language=Rust, style=colouredRust]
pub type Tuple = (List, Tree);
\end{lstlisting}
&
\begin{lstlisting}[language=Rust, style=colouredRust]
pub struct Record{pub l:List, pub t:Tree}
\end{lstlisting}
\\
\hline
\begin{lstlisting}[language=Rust, style=colouredRust]
pub struct NListNode {
    pub value : i32,
    pub next : NList
}
\end{lstlisting} 
&
\begin{lstlisting}[language=Rust, style=colouredRust]
pub enum NList {
    None,
    Some(Box<NListNode>)
}
\end{lstlisting} 
\\
\hline
\end{tabular}
\end{table}
\end{DIFnomarkup}

\subsection{Evaluation}
\label{sec:eval}

\begin{DIFnomarkup}
\begin{table}[t]
    \centering
    \caption{Benchmarks}
    \label{tab:eval}
    \small
    \scalebox{0.67}{
    \begin{tabular}{|c|c|c|c|c|c|}
    \hline
    case & type & description & complexity & lines of code & size of constraints  \\
    \hline
    \multicolumn{6}{|c|}{\textbf{feature(s): mutable borrows}} \\ \hdashline
    end\_m & {\lstinline|fn(l:&m List)->&m List|} & find the mutable borrow of the {\lstinline|Nil|} of a list & $1+3|l|$ & 14 & 71 \\
    end\_c & {\lstinline|fn(l:&m List,o:&m&m List)|} & c style end\_m to write to {\lstinline|o|} & $2+3|l|$ & 13 & 96 \\ 
    append & {\lstinline|fn(l:&m List,x:i32)|} & append $x$ to $l$ by {\lstinline|end_m|} & $5+3|l|$ & 5 & 89 \\
    concat & {\lstinline|fn(l1:&m List,l2:List)|} & $l'_1 = l_1@l_2$ by {\lstinline|end_c|} & $6+3|l_1|$ & 8 & 144 \\
    \hline
    \multicolumn{6}{|c|}{\textbf{feature(s): shared borrows, mutable borrows, recursive functions, and loop statements}} \\ \hdashline
    sum\_rec & {\lstinline|fn(l:& List)->i32|} & sum up integers, recursively & $1 + 6|l|$ & 14 & 19 \\
    sum\_loop & {\lstinline|fn(l:& List)->i32|} & sum up integers, via loops & $1 + 6|l|$ & 24 & 54 \\
    
    rev\_rec & {\lstinline|fn(l:&m List, r: List)->List|} & reverse $l$ to head of $r$ & $1 + 9|l|$ & 15 & 57 \\
    
    rev\_loop & {\lstinline|fn(l:&m List)|} & reverse $l$ mutably via loops & $1 + 9|l|$ & 21 & 164 \\
    % remove rich
    % \hline
    % sum & \lstinline|fn(l:& List)->i32| & sum up integers in list & $1 + 6|l|$ \\
    % rich & \lstinline|fn(l:&m List)| & iterate $l$ with $-1, -6$ as $\kwd{tick}$ & $-6|l|$ \\
    % rich\_sum & \lstinline|fn(l:&m List)->i32| & \lstinline|rich(l)|, and then \lstinline|sum(l)| & $1$ \\
    \hline
    \multicolumn{6}{|c|}{\textbf{feature(s): function calls}} \\ \hdashline
    sum2 & {\lstinline|fn(l:& List)|} & {\lstinline|sum_rec(l);sum_loop(l)|} & $2 + 12|l|$ & 4 & 82 \\
    rev & {\lstinline|fn(l:&m List)|} & reverse $l$ mutably via {\lstinline|rev_rec|} & $4 + 9|l|$ & 5 & 71 \\
    rev2 & {\lstinline|fn(l:&m List)|} & apply {\lstinline|rev|} to $l$ twice & $8 + 18|l|$ & 4 & 170 \\
    dup & {\lstinline|fn(l: List)->List|} & duplicate each element in $l$ & $1+11|l|$ & 16 & 37 \\
    dup2 & {\lstinline|fn(l: List)->List|} & apply {\lstinline|dup|} to $l$ twice & $2+33|l|$ & 4 & 81 \\
    \hline
    \multicolumn{6}{|c|}{\textbf{feature(s): reborrow and nested borrows}} \\ \hdashline
    reborrow\_s & {\lstinline|fn(l:& List)|} & reborrow $l$ as $ll$, {\lstinline|sum2(ll);sum2(l)|}& $4+24|l|$ & 6 & 176 \\
    reborrow\_m & {\lstinline|fn(l:&mut List)|} & reborrow $l$ as $ll$, {\lstinline|rev2(ll);rev2(l)|}& $16+36|l|$ & 6 & 364 \\
    nested\_s\_s & {\lstinline|fn(l:& & List)|} & {\lstinline|sum2(*l);|} & $2+12|l|$ & 4 & 88 \\
    nested\_m\_s & {\lstinline|fn(l:&m & List)|} & {\lstinline|sum2(*l);|} & $2+12|l|$ & 4 & 90 \\
    nested\_m\_m & {\lstinline|fn(l:&m &m List)|} & {\lstinline|rev2(*l);|} & $8+18|l|$ & 4 & 188 \\
    \hline
    \multicolumn{6}{|c|}{\textbf{feature(s): user-defined datatypes}} \\ \hdashline
    min & {\lstinline|fn(t:&Tree,d:i32)->i32|} & find min in $t$, $d$ as default & $1+5|t|$ & 14 & 19 \\
    max & {\lstinline|fn(t:&Tree,d:i32)->i32|} & find max in $t$, $d$ as default & $1+5|t|$ & 14 & 19 \\
    find & {\lstinline|fn(t:&Tree,x:i32)->bool|} & find whether $x$ is in $t$ & $1+8|t|$ & 29 & 31 \\
    add & {\lstinline|fn(t:&m Tree,x:i32)|} & add up $x$ to each element of $t$ & $1+10|t|$ & 15 & 53\\
    tuple & {\lstinline|fn(x:&m Tuple)|} & {\lstinline|rev2(x.0);min(x.1, 0);|} & $9+18|x.0|+5|x.1|$ & 4 & 216 \\
    record & {\lstinline|fn(x:&m Record)|} & {\lstinline|rev2(x.l);min(x.t, 0);|} & $9+18|x.l|+5|x.t|$ & 4 & 216 \\
    sum\_rec\_nlist & {\lstinline|fn(l:&NList) -> i32|} & sum up {\lstinline|NList|} as \lstinline|sum_rec| & $1+5|l|$ & 16 & 26 \\
    rev\_rec\_nlist & {\lstinline|fn(l:&mut NList,r:NList)->NList| } & reverse {\lstinline|NList|} as \lstinline|rev_rec| & $1+7|l|$ & 21 & 89 \\
    \hline
    \end{tabular}
    }
\end{table}
\end{DIFnomarkup}

We used the prototype implementation of \rarust{} to perform an experimental evaluation of resource analysis on typical examples concerning Rust's borrow mechanism. We adapted several pure functional examples from \textsc{RaML} to their Rust counterparts employing borrows; by using borrows we can do in-place updates in Rust. Some examples were deliberately crafted to demonstrate the prototype’s capability to support the aforementioned features. Due to the linear limitation, we select those examples with linear complexity. The experiments were performed on a laptop with 2.20 GHz Intel Core i9-13900HX as CPU and WSL 2.3.24.0/Ubuntu 22.04.3 LTS as operation system. The compiling of the benchmarks was done in 0.15s and the analysis was done in 0.3s.

\cref{tab:eval} shows our experimental results. We manually checked the analysis results on the benchmarks and confirmed that the derived bounds are correct (but not tight for some benchmarks such as \lstinline|min| and \lstinline|max|). We encode some cases with the derived bounds in \textsc{Flux}~\cite{Flux} to check that the derived bounds are correct. Our encoding introduces a global counter to accumulate resource consumption. The encoding is shipped into the artifact and it includes cases \lstinline|append|, \lstinline|concat|, \lstinline|sum_rec|, \lstinline|rev_rec|, \lstinline|sum2|, \lstinline|rev|, \lstinline|rev2|, \lstinline|dup|, \lstinline|dup2|, \lstinline|min|, \lstinline|max|, \lstinline|find|, \lstinline|add|, \lstinline|tuple|, \lstinline|record|, \lstinline|sum_rec_nlist|, and \lstinline|rev_rec_nlist|.

\cref{tab:eval} gives out 5 groups of test cases, and each group exercises some features. For each analyzed function as a case, \cref{tab:eval} first declares the function type in a simplified syntax of Rust, writing \lstinline|&mut T| as \lstinline|&m T| for short. The description column provides a concise explanation of the function's semantics. We plot the complexity in a more readable format in the table, where $|l|$ represents the length or count of \lstinline|Cons| constructors of $l$ and $|t|$ represents the count of \lstinline|Node| constructors of the tree $t$. The concrete coefficients are inferred by \rarust{} according to annotation $\kwd{tick}(\delta)$ in examples. In our evaluation, we set those different concrete numbers of $\delta$ for two purposes: (i) we roughly add one $\kwd{tick}(\delta)$ around one statement to account for the number of operations by the statement, and (ii) we can use different numbers to test multiple times to check if our implementation is correct.
%
We will explain each group in detail in the rest of this section.

% \textbf{Mutable Borrows and Nested Borrows:} 
To show that \rarust{} can handle mutable borrows, we construct cases \lstinline|end_m| and \lstinline|end_c|. They are recursively to find the mutable borrow of the nil of a list $l$. For example, in ML syntax, the nil of the list \verb|1::2::3::4::[]| is \verb|[]|. \lstinline|end_m| returns the borrow, while  \lstinline|end_c| storing it in the parameter \lstinline|o:&m&m List|. The returned mutable borrow of \lstinline|end_m| works as a closure function with type \lstinline|List->List|, therefore it is non-trivial for resource analysis. We use cases \lstinline|append| and \lstinline|concat| to show the resource correctness as well as the compositionality of the analysis.

%\textbf{Recursive Functions and Loop Statements:} 
Rust programmers are able to write code with loop statements. We construct cases \lstinline|sum_rec|, \lstinline|sum_loop|, \lstinline|rev_rec| and \lstinline|rev_loop|. We focus on shared borrows in \lstinline|sum| and on mutable borrows in \lstinline|rev|. The suffix \lstinline|rec| means recursive function and \lstinline|loop| means loop statements \lstinline|while true { ... }|. The same analysis results reveal that both are supported by \rarust{}. 

% \textbf{Sharing and Prophesying:} 
We construct multiple calls of function for shared borrows and mutable borrows, to demonstrate sharing and prophesying. The suffix \lstinline|2| means twice in cases \lstinline|sum2|, \lstinline|rev2| and \lstinline|dup2|. The coefficients in the complexity of \lstinline|sum2| are exactly 2 times of \lstinline|sum|, testing the sharing for shared borrows. \lstinline|rev2| is similar but for the prophesying of mutable borrows. \lstinline|dup2| is made to point out the difference between sharing and prophesying. The function \lstinline|dup| mutates the length of list $l$, therefore the second call of \lstinline|dup| is with length $2|l|$, making the linear coefficient of \lstinline|dup2| be $33$, 3 times $11$.


% \textbf{Reborrows and Nested Borrows:} 
\rarust{} also supports reborrows and nested borrows. We construct cases \lstinline|reborrow_s|, \lstinline|reborrow_m|, \lstinline|nested_s_s|, \lstinline|nested_m_s| and \lstinline|nested_m_m|. The suffix \lstinline|s| denotes shared borrows, while \lstinline|m| denotes mutable borrows.

% \textbf{User-defined Data Types:} 
Besides \lstinline|List|, we construct simple examples for other safe user-defined data types like trees, tuples, records, and C-style lists. The sizes of trees are the counts of internal nodes \lstinline|Node|, instead of the intuitive measure, their depths. \lstinline|Tuple| and \lstinline|Record| are data types to compose \lstinline|List| and \lstinline|Tree|, with complexity as the linear composition of their fields', such as \lstinline|x.0| and \lstinline|x.t|. \rarust{} also supports mutually recursive data types like \lstinline|NList| and \lstinline|NListNode|; they are C-style lists, with the former as the nullable pointer, the latter as the internal node of lists.
\section{Discussion}
\label{sec:discussion}

In this section, we discuss some limitations of \rarust{} mentioned in \cref{section:introduction} and propose possible pathways towards overcoming them to improve the capability of \rarust{} in future work.

\paragraph{Unsafe code, interior mutability, vectors, reference counting, and cyclic data structures}
%
We focus on safe Rust programs because our design of \rarust{} relies on guarantees provided by Rust's borrow mechanisms, e.g., aliasing and mutation cannot happen simultaneously.
%
However, Rust programs cannot avoid unsafe code in general, because many standard libraries---including cells (\verb|Cell|), vectors (\verb|Vec|), and reference counting (\verb|Rc|)---rely on unsafe code to allow shared mutable states, e.g., interior mutability.
%
The unsafe code can operate C-style pointers in an unrestricted way and compromise Rust's memory safety; as a result, \rarust{}'s resource analysis cannot handle unsafe code.
%
This is actually a common limitation of advanced type systems and verification frameworks for Rust, including Flux~\cite{Flux}, Aeneas~\cite{Aeneas}, and Prusti~\cite{OOPSLA:AMP19}.
%
Nevertheless, there have been efforts to support unsafe code in formal reasoning about Rust programs.
%
RustBelt~\cite{RustBelt} pioneers a line of work on semantic typing and separation-logic-based verification of Rust programs with unsafe code. 
%
Verus~\cite{OOPSLA:LHC23} supports some unsafe features by providing specifications for unsafe memory operations to be memory safe and employing SMT solvers to check those specifications automatically.
%
However, it is unclear if one can integrate AARA type systems with those techniques.

One common method to support unsafe code is based on Rust's design philosophy: unsafe operations should be properly \emph{encapsulated} by safe APIs, and the developers of those unsafe operations take charge of ensuring the unsafe code does not break Rust's memory safety.
%
In terms of type systems, this amounts to assigning types to the safe APIs instead of inferring types from the unsafe code body.
%
Therefore, it would be possible for \rarust{} to analyze Rust programs with unsafe code, if all the unsafe code is encapsulated by resource-annotated safe APIs.
%
Rust's \verb|Cell| features interior mutability by providing operations for both getting and setting the content of a memory location.
%
At the API level, the \verb|Cell| type works similarly to references in an ML-like functional programming language, so it would be possible to adapt an AARA approach for supporting references~\cite{FSCD:LH17}.
%
For example, the resource-annotated APIs shown below can be used to manipulate cells storing lists:
%
\begin{align*}
    \texttt{new}: & \kwd{fn}(\texttt{l}: \kwd{list}(\alpha) ) \to \texttt{Cell<}\kwd{list}(\alpha)\texttt{>} | 0, \\
    \texttt{replace} : & \kwd{fn}(\texttt{self}: \&\texttt{Cell<}\kwd{list}(\alpha)\texttt{>}, \texttt{l}: \kwd{list}(\alpha) ) \to \kwd{list}(\alpha) | 0 ,
\end{align*}
in the sense that the potential type $\kwd{list}(\alpha)$ is an invariant for a cell type, and operations should maintain the invariant, e.g., \verb|replace| should store another list of the same type $\kwd{list}(\alpha)$.
%
Rust's \verb|Vec| also makes use of unsafe code to allow accessing uninitialized memory.
%
At the API level, we can treat vectors as abstract dynamic arrays, which fit nicely into the AARA framework because of their amortized complexity.
%
For example, we can declare the following APIs for integer vectors: 
\begin{align*}
    \texttt{new}: & \kwd{fn}() \to \texttt{Vec<}\kwd{i32},\alpha\texttt{>} | 0, \\
    \texttt{push} : & \kwd{fn}(\texttt{self}: \&\kwd{mut}~\texttt{Vec<}\kwd{i32},\alpha\texttt{>}, \texttt{n}: \kwd{i32} ) \to () | \alpha + 4,
\end{align*}
where the resource-annotated type $\texttt{Vec<}\kwd{i32},\alpha\texttt{>}$ denotes the potential function $\mathit{len} \cdot \alpha + (4 \cdot \mathit{len} - 2 \cdot \mathit{cap})$, with $\mathit{len}$ and $\mathit{cap}$ being the length and capacity of the vector, respectively. 
%
Intuitively, the potential function states that every vector element carries $\alpha$ units of potential and we need to store $(4 \cdot \mathit{len} - 2 \cdot \mathit{cap})$ units of extra potential for vector resizing, which would consume $2 \cdot \mathit{len}$ units of resource to extend the vector's capacity when the vector becomes full.
%

Rust's implementation of \verb|Rc| uses unsafe code, so we would annotate \verb|Rc| APIs with resource-annotated types. \verb|Rc| itself does not permit mutation, so we could model its behavior as if it is a shared reference: \verb|Rc::new()| should store potentials (e.g., \verb|Rc<list(4)>|) and \verb|Rc::clone()| should split potentials (e.g., splitting \verb|Rc<list(4)>| as \verb|Rc<list(1)>| and \verb|Rc<list(3)>|). We have not yet considered multithreading, and supporting \verb|Arc| would be interesting future work.

It is non-trivial to handle cyclic data structures. Rust provides \verb|Weak| pointers to accompany \verb|Rc| pointers. However, creating cyclic data structures usually requires using interior mutability (e.g., \verb|Cell| or \verb|RefCell|). The interaction between reference counting and interior mutability seems quite non-trivial. In the future, we may adapt \citet{ESOP:Atkey10}'s work on integrating AARA with separation logic.

\paragraph{Generic types, higher-order functions (closures), and trait objects}
%
Current \rarust{} does not support generic types like \verb|List<T>|. This is not a fundamental limitation, because we can always instantiate generic types. We will spare engineering efforts to support them in the future.
%

Our work currently only considers top-level functions, but Rust does support higher-order functions and closures to enable functional programming style. 
%
Fortunately, many AARA approaches support higher-order functions~\cite{AARA-HigherOrder,POPL:HDW17,ICFP:KWR20,ICFP:KH21}.
%
Conceptually,
it would be possible to adapt AARA's methodology of handling closures to \rarust{}
by extending the type system to deal with \emph{capturing} properly.
%
One simple extension is to enforce that closures cannot consume potentials stored in captured variables; in this way, it is sound to apply a closure multiple times.
%
It would be interesting future research to investigate how the interaction of borrow mechanisms (especially mutable borrows) and closures would affect AARA-style resource analysis.

Rust supports a form of dynamic dispatch through trait objects, in which the compiler knows an object's trait but not its actual type. One possible workaround is to annotate trait methods with resource annotations and require them not to change the resource type of \verb|Self|. In this way, even though we do not know an object's type, we know how calling its trait methods affects the resource-annotated context. Another possibility is to adapt \citet{AARA-OOP}'s work on integrating AARA with objective-oriented programming. 

\paragraph{Non-linear resource bounds}
%
Both our formalization and implementation of \rarust{} currently only consider linear resource bounds, which are too restrictive to analyze real-world programs.
%
Current \rarust{} can only support pattern matching of one single variable, without primitive support for pattern matching of tuple types, because tuple types usually introduce multivariate polynomial resource bounds like $\textit{first} \times \textit{second}$. 
%
This is not a fundamental limitation because it has been shown that AARA type systems can support polynomial bounds~\cite{AARA-Poly,AARA-Poly-Multivar}, exponential bounds~\cite{AARA-Exp}, logarithmic bounds~\cite{AARA-Log,CAV:LMZ21}, and value-dependent bounds~\cite{ICFP:KWR20,PLDI:KWP19}.
%
All of those approaches amount to devising proper type annotations that specify particular kinds of potential functions and developing constraint-based type-inference algorithms.
%
We plan to spare engineering efforts to extend our prototype implementation of \rarust{} to support various classes of resource bounds in the future.

% \todo{limitation also here, and some idea to solve limitation} 
% The limitation is linear and imprecision. (1) Prototype \rarust{} can only analyze linear bound of resource consumption, as $a + b\cdot n$. However, we can extend the bound to polynomials\cite{AARA-Poly} and \cite{AARA-Poly-Multivar}. (2) Recall the weak updates in \cref{sec:overview}. It will introduce inevitable imprecision. Our implementation introduce it when merging mutable borrows. Though inevitable, it can be delayed to actual writing on borrows, using similar techniques as \cite{CapTypes}. We leave extension as future works.
\section{related work}
Code refinement is a critical core component in the code review process, and numerous scholars have conducted research on the automation of code refinement. Tufano M.~\cite{tufano2019learning} were the pioneers in proposing the use of Neural Machine Translation (NMT) to learn the automated modification of Java methods based on review comments. Subsequently, Tufano R.~\cite{tufano2021towards} and Thongtanunam~\cite{thongtanunam2022autotransform} employed transformer~\cite{vaswani2017attention} models to train and enhance the original task’s performance. 
Tufano R.~\cite{tufano2022using} and Li~\cite{li2022automating} further advanced this field by using the Text-To-Text Transfer Transformer (T5) model~\cite{raffel2020exploring} and CodeT5 model~\cite{wang2021codet5}, pre-training code review-related tasks to enable the model to comprehend the meaning of the code and review comments. These approaches yielded significant improvements in downstream code refinement tasks.

% Compared to pre-trained models, large models exhibit significant advantages in understanding instructions and generating code. 
With the rise of LLMs, many researchers have attempted to leverage them in software engineering \cite{ma2024specgen, ma2024speceval, kong2024contrastrepair, guo2024ft2ra, xia2023keep}. In particular, Guo~\cite{guo2024exploring} explored using ChatGPT for code refinement tasks, uncovering some prompt design techniques.
% as well as the strengths and weaknesses of Large Language Models (LLMs) in this context. 
Tufano R.~\cite{tufano2024code} manually analyzed over 2,000 code refinement examples, evaluating three code review models~\cite{hong2022commentfinder, li2022automating, tufano2022using} and comparing their performance to ChatGPT. This analysis revealed that ChatGPT is highly competitive compared to previous methods. 
Pornprasit~\cite{pornprasit2024fine} experimented with various prompt strategies for LLMs in code refinement. They also fine-tuned ChatGPT using an API~\cite{ChatGPTblog}, enhancing the effectiveness of LLMs in this task.
% Pornprasit~\cite{pornprasit2024fine} experimented with various prompt strategies for LLM in code refinement, including zero-shot, few-shot, and persona-based approaches. Additionally, Pornprasit fine-tuned ChatGPT using an API~\cite{ChatGPTblog}, which further improved the effectiveness of LLMs in this task.

Some studies have also involved classifying code refinement tasks.
Tufano~\cite{tufano2024code} categorizes code refinement based on the type of task and examines the performance of pre-trained models on different task types. 
Kononenko~\cite{kononenko2016code} studied the time and effort required by programmers for different types of code review tasks. Bacchelli~\cite{bacchelli2013expectations} investigated the categories of code review tasks, focusing on developer motivation and response speed. Pascarella~\cite{pascarella2018information} explored the information needed for different types of code refinement tasks, but their classification method is more oriented toward human understanding rather than guiding model modifications.

% 另外,将code Refinement分类,也有很多研究工作。Tufano是从任务的类型角度来分类,查看pre-trained模型在不同任务类型上的效果。Kononenko[kononenko2016code]研究了不同类别的code review任务所消耗程序员的时间和精力。Bacchelli[bacchelli2013expectations]调研了不同code review任务类别,开发者的motivation和响应速度。Pascarella[pascarella2018information]研究了不同类别的code Refinement任务的所需信息,不过他们的分类方式更偏重于让人类理解,而不是指导模型进行修改。


% Code Refinement自动化任务:code Refinement任务作为code review流程中的关键核心任务,有很多学者进行过code Refinement自动化相关研究。最早由Tufano M.等人[tufano2019learning]提出使用NMT去学习根据review comment自动化的修改java method。而后Tufano R.等人[tufano2021towards],Thongtanunam等人[thongtanunam2022autotransform]分别使用transformer模型训练并提升了原任务的效果。Tufano R.[tufano2022using]进一步使用Text-To-Text Transfer Transformer (T5)模型[raffel2020exploring],allowing the model to work with raw source code by keeping under control the vocabulary size, 解决了之前工作需要将变量名做化简代替的问题。而后,Li等人[li2022automating]首先使用了pre-trained模型,设计了 Diff Tag Prediction,Denoising Objective,Review Comment Generation等三个预训练任务,在CodeT5模型[wang2021codet5]的基础上预训练code review相关任务,让模型理解代码的含义,以及代码和review comment的对应关系。并在下游code Refinement任务上取得了很好的效果。

% 大模型for code Refinement任务:相较于预训练模型,大模型在理解指令,和生成代码方面具明显优势。随着大模型的兴起,很多研究者也尝试用大模型解决code Refinement任务。Guo[guo2024exploring]尝试用ChatGPT处理code Refinement任务,发现了一些prompt设计技巧,以及LLM在此任务上的优势和不足。Tufano R.[tufano2024code]手工分析了2000多个code Refinement例子,评估了三种code review模型[hong2022commentfinder, li2022automating, tufano2022using]并与ChatGPT的效果对比,发现ChatGPT相较于之前的方法具有很强的竞争力,并且通过引导模型先去思考问题类型,再做修改的COT方法,可以进一步提升修复效果。Pornprasit[pornprasit2024fine]尝试了大模型解决code Refinement问题时的多种prompt策略,包括zeroshot,fewshot和Persona,并且用API的方式finetune了ChatGPT,可以进一步提升大模型的效果。
\section{Conclusion}



This paper introduces \sysname, an AI-assisted system designed to enhance the process of visual blend ideation by leveraging metaphors. 
%Our system utilizes large language models and commonsense knowledge bases to explore objects and their associated attributes, forming metaphorical connections with abstract concepts.
Our system utilizes LLMs and commonsense knowledge bases to explore objects and their associated attributes, forming metaphorical connections with abstract concepts. 
It offers the capability to automatically generate blending proposals based on user selections, facilitating rapid creative realization for verification through the T2I model.
To evaluate the system, we conducted a user study involving 24 participants who had AI experience. The findings demonstrate that \sysname\ has the potential to enhance the creativity of the generated ideation results and enable the expression of abstract concepts more metaphorically.
Additionally, this research offers insights into user preferences regarding visual blend design and potential future approaches for supporting design with generative AI.



\section*{Data-Availability Statement}
The source code of the \rarust{} implementation and benchmarks referenced in \cref{sec:impl} are publicly available in the Zenodo \cite{RaRustArtifact}. The artifact contains necessary scripts and step-by-step guides to reproduce the experimental results.

\begin{acks}
We are grateful to Xuanyu Peng for the early discussion and investigation. We would like to thank the anonymous reviewers for their valuable feedback on our paper and the anonymous artifact reviewers for their suggestions for our artifact.
\end{acks}



\bibliographystyle{ACM-Reference-Format}
\bibliography{main,db}

\newpage
\appendix
\section{Judgements}
\centering
\judgement{Expression Evaluation}{$V\vDash e \rightsquigarrow v$}
    \begin{mathpar}
    \inferrule*[Right=\rulename{V-Ev-Int}]
    {~}
    {V\vDash \kwd{n}_\text{i32} \rightsquigarrow \kwd{n}_\text{i32}}
    \and
    \inferrule*[Right=\rulename{V-Ev-Op}]
    {V\vDash e_1 \rightsquigarrow \kwd{n}_1
    \\ V\vDash e_2 \rightsquigarrow \kwd{n}_2}
    {V\vDash e_1~\kwd{op}~e_2 \rightsquigarrow \kwd{n}_1~\kwd{op}~\kwd{n}_2}
    \\
    \inferrule*[Right=\rulename{V-Ev-True}]
    {~}
    {V\vDash \kwd{true} \rightsquigarrow \kwd{true}}
    \and
    \inferrule*[Right=\rulename{V-Ev-False}]
    {~}
    {V\vDash \kwd{false} \rightsquigarrow \kwd{false}}
    \\
    \inferrule*[Right=\rulename{V-Ev-Nil}]
    {~}
    {V\vDash \kwd{nil} \rightsquigarrow \kwd{nil}}
    \and
    \inferrule*[Right=\rulename{V-Ev-Box}]
    {V\vDash e \rightsquigarrow v}
    {V\vDash \kwd{box}(e) \rightsquigarrow \kwd{box}(v)}
    \\
    \inferrule*[Right=\rulename{V-Ev-Copy}]
    {V\vDash p \rightsquigarrow v}
    {V\vDash \kwd{copy}~p \rightsquigarrow v}
    \and
    \inferrule*[Right=\rulename{V-Ev-Move}]
    {V\vDash p \rightsquigarrow v}
    {V\vDash \kwd{move}~p \rightsquigarrow v}
    \and
    \inferrule*[Right=\rulename{V-Ev-Borrow}]
    {V\vDash p \rightsquigarrow v}
    {V\vDash \&^{\kwd{s}/\kwd{m}/\kwd{2}} p \rightsquigarrow \&(p, v)}
\end{mathpar}

\centering
\judgement{Statement Execution}{$V\vDash e \rightsquigarrow^\delta \Dashv V'$}
\begin{mathpar}
    \inferrule*[Right=\rulename{V-Ex-Ret}]
    {~}
    {V\vDash \kwd{return} \rightsquigarrow^0 \Dashv V}
    \and
    \inferrule*[Right=\rulename{V-Ex-Seq}]
    {V\vDash s_1\rightsquigarrow^{\delta_1}\Dashv V'
    \\ V'\vDash s_2\rightsquigarrow^{\delta_2}\Dashv V''}
    {V\vDash s_1; s_2\rightsquigarrow^{\delta_1+\delta_2}\Dashv V''}
    \\
    \inferrule*[Right=\rulename{V-Ex-Tick}]
    {~}
    {V\vDash \kwd{tick}(\delta)\rightsquigarrow^\delta \Dashv V}
    \and
    \inferrule*[Right=\rulename{V-Ex-Drop}]
    {~}
    {V\vDash \kwd{drop}~p\rightsquigarrow^0\Dashv V}
    \\
    \inferrule*[Right=\rulename{V-Ex-Assign}]
    {V\vDash e \rightsquigarrow v'
    \\ \VWt{V}{p}{v'}{V'}}
    {V\vDash p\from e\rightsquigarrow^0 \Dashv V'}
    \\
    \inferrule*[Right=\rulename{V-Ex-Cons}]
    {V\vDash e_1 \rightsquigarrow v_1
    \\ V\vDash e_2 \rightsquigarrow v_2
    \\ \VWt{V}{p}{\kwd{cons}(v_1, v_2)}{V'} }
    {V\vDash p\from \kwd{cons}(e_1, e_2)\rightsquigarrow^0 \Dashv V'}
    \\
    \inferrule*[Right=\rulename{V-Ex-IfT}]
    {V\vDash p\rightsquigarrow \kwd{true}
    \\ V\vDash s_1\rightsquigarrow^\delta \Dashv V'}
    {V\vDash \kwd{if}~ p ~\kwd{then}~ s_1 ~\kwd{else}~ s_2 ~\kwd{end} \rightsquigarrow^\delta \Dashv V'}
    \and
    \inferrule*[Right=\rulename{V-Ex-IfF}]
    {V\vDash p\rightsquigarrow \kwd{false}
    \\ V\vDash s_2\rightsquigarrow^\delta \Dashv V'}
    {V\vDash \kwd{if}~ p ~\kwd{then}~ s_1 ~\kwd{else}~ s_2 ~\kwd{end} \rightsquigarrow^\delta \Dashv V'}
    \\
    \inferrule*[Right=\rulename{V-Ex-Mat-Nil}]
    {V\vDash p\rightsquigarrow \kwd{nil}
    \\ V\vDash s_1\rightsquigarrow^\delta \Dashv V'}
    {V\vDash \kwd{match}~ p ~ \{\kwd{nil}\mapsto s_1, \kwd{cons}(x_\text{hd}, x_\text{tl})\mapsto s_2\} \rightsquigarrow^\delta \Dashv V'}

    \inferrule*[Right=\rulename{V-Ex-Mat-Cons}]
    {V\vDash p\rightsquigarrow \kwd{cons}(hd, tl)
    \\ \VWt{V}{p}{\bot}{V_1}
    \\ \VWt{V_1}{x_\text{hd}}{hd}{V_2}
    \\ \VWt{V_2}{x_\text{tl}}{tl}{V_\text{b}}
    \\\\ V_\text{b}\vDash s_2\rightsquigarrow^\delta \Dashv V'_\text{b}
    \\ V'_\text{b}\vDash x_\text{hd}\rightsquigarrow hd'
    \\ V'_\text{b}\vDash x_\text{tl}\rightsquigarrow tl'
    \\\\ \VWt{V'_\text{b}}{x_\text{hd}}{\bot}{V'_1}
    \\ \VWt{V'_1}{x_\text{tl}}{\bot}{V'_2}
    \\ \VWt{V'_2}{p}{\kwd{cons}(hd', tl')}{V'} 
    }
    {V\vDash \kwd{match}~ p ~ \{\kwd{nil}\mapsto s_1, \kwd{cons}(x_\text{hd}, x_\text{tl})\mapsto s_2\} \rightsquigarrow^\delta \Dashv V'}
    \\
    
    \inferrule*[Right=\rulename{V-Ex-App}]
    {\kwd{fn}~ f ~(\vec{x}_\text{param}:\vec{t}_\text{param}, \vec{x}_\text{local}:\vec{t}_\text{local}, x_\text{ret}:t_\text{ret}) \{~ s ~\}
    \\ V\vDash \vec{e}\rightsquigarrow \vec{v}
    \\\\ \VWt{V}{\vec{x}_\text{param}}{\vec{v}}{V_1}
    \\ \VWt{V_1}{\vec{x}_\text{local}}{\bot}{V_2}
    \\ \VWt{V_2}{x_\text{ret}}{\bot}{V_\text{b}}
    \\\\ V_\text{b}\vDash s\rightsquigarrow^\delta \Dashv V'_\text{b}
    \\ V'_\text{b}\vDash x_\text{ret} \rightsquigarrow v_\text{ret}
    \\\\ \VWt{V'_\text{b}}{\vec{x}_\text{param}}{\bot}{V'_1}
    \\ \VWt{V'_1}{\vec{x}_\text{local}}{\bot}{V'_2}
    \\ \VWt{V'_2}{x_\text{ret}}{\bot}{V'_3}
    \\ \VWt{V'_3}{p}{v_\text{ret}}{V'}
    } 
    {V\vDash p\from f(\vec{e})\rightsquigarrow^\delta \Dashv V'}
\end{mathpar}

\centering
\judgement{Enrich}{$\textit{enrich}~ t ~\textit{as}~ \tau$}
\begin{mathpar}
    \inferrule*[Right=\rulename{Enrich-Int}]
    {~}
    {\textit{enrich}~ \kwd{i32} ~\textit{as}~\kwd{i32}}
    \and
    \inferrule*[Right=\rulename{Enrich-Bool}]
    {~}
    {\textit{enrich}~ \kwd{bool} ~\textit{as}~ \kwd{bool}}
    \\

    \inferrule*[Right=\rulename{Enrich-List}]
    {\alpha~\text{fresh}}
    {\textit{enrich}~ \kwd{list} ~\textit{as}~  \kwd{list}(\alpha)}
    \and
    \inferrule*[Right=\rulename{Enrich-Box}]
    {\alpha~\text{fresh}}
    {\textit{enrich}~ \kwd{box}(\kwd{list}) ~\textit{as}~ \kwd{box}(\kwd{list}(\alpha))}
    \\
    
    \inferrule*[Right=\rulename{Enrich-Shared}]
    {\textit{enrich}~ t ~\textit{as}~ \tau}
    {\textit{enrich}~ \&^\kwd{s}(t) ~\textit{as}~ \&^\kwd{s}(\tau)}
    \\
    \inferrule*[Right=\rulename{Enrich-Mutable}]
    {\textit{enrich}~ t ~\textit{as}~ \tau_\text{c}
    \\ \textit{enrich}~ t ~\textit{as}~ \tau_\text{p}
    }
    {\textit{enrich}~ \&^\kwd{m}(t) ~\textit{as}~ \&^\kwd{m}(\tau_\text{c}, \tau_\text{p})}
\end{mathpar}

\centering
\judgement{Sharing}{$\textit{share}~ \tau ~\textit{as}~\tau_1, \tau_2$}
\begin{mathpar}
    \inferrule*[Right=\rulename{Share-Int}]
    {~}
    {\textit{share}~ \kwd{i32} ~\textit{as}~\kwd{i32}, \kwd{i32}}
    \and
    \inferrule*[Right=\rulename{Share-Bool}]
    {~}
    {\textit{share}~ \kwd{bool} ~\textit{as}~\kwd{bool}, \kwd{bool}}
    \\
    \inferrule*[Right=\rulename{Share-List}]
    {\alpha_1, \alpha_2 ~\text{fresh}
    \\\alpha = \alpha_1 + \alpha_2}
    {\textit{share}~ \kwd{list}(\alpha) ~\textit{as}~\kwd{list}(\alpha_1), \kwd{list}(\alpha_2)}
    \\
    \inferrule*[Right=\rulename{Share-Box}]
    {\textit{share}~ \tau ~\textit{as}~\tau_1, \tau_2}
    {\textit{share}~ \kwd{box}(\tau) ~\textit{as}~\kwd{box}(\tau_1), \kwd{box}(\tau_2)}
    \\
    \inferrule*[Right=\rulename{Share-Shared}]
    {\textit{share}~ \tau ~\textit{as}~\tau_1, \tau_2}
    {\textit{share}~ \&^\kwd{s}(\tau) ~\textit{as}~ \&^\kwd{s}(\tau_1), \&^\kwd{s}(\tau_2)}
\end{mathpar}

\centering
\judgement{Prophesying}{$\textit{prophesy}~ \tau_\text{c} ~\textit{as}~\tau_\text{p}$}
\begin{mathpar}
    \inferrule*[Right=\rulename{Prophesy-Int}]
    {~}
    {\textit{prophesy}~ \kwd{i32} ~\textit{as}~ \kwd{i32}}
    \and
    \inferrule*[Right=\rulename{Prophesy-Bool}]
    {~}
    {\textit{prophesy}~ \kwd{bool} ~\textit{as}~ \kwd{bool}}
    \\
    \inferrule*[Right=\rulename{Prophesy-List}]
    {\alpha_\text{p}~\text{fresh}}
    {\textit{prophesy}~ \kwd{list}(\alpha) ~\textit{as}~ \kwd{list}(\alpha_\text{p})}
    \\
    \inferrule*[Right=\rulename{Prophesy-Box}]
    {\textit{prophesy}~ \tau ~\textit{as}~ \tau_\text{p} }
    {\textit{prophesy}~ \kwd{box}(\tau) ~\textit{as}~ \kwd{box}(\tau_\text{p})}
    \\
    \inferrule*[Right=\rulename{Prophesy-Shared}]
    {\textit{prophesy}~ \tau ~\textit{as}~ \tau_\text{p}}
    {\textit{prophesy}~ \&^\kwd{s}(\tau) ~\textit{as}~ \&^\kwd{s}(\tau_\text{p})}
    \\
    \inferrule*[Right=\rulename{Prophesy-Mutable}]
    {\textit{prophesy}~ \tau_\text{c} ~\textit{as}~ \tau_\text{cp}
    \\ \textit{prophesy}~ \tau_\text{p} ~\textit{as}~ \tau_\text{pp}
    }
    {\textit{prophesy}~ \&^\kwd{m}(\tau_\text{c}, \tau_\text{p}) ~\textit{as}~ \&^\kwd{m}(\tau_\text{cp}, \tau_\text{pp})}
\end{mathpar}

\centering
\judgement{Meet/Join}{$\tau_1\cap\tau_2 / \tau_1\cup\tau_2$}
\begin{mathpar}    
    \inferrule*[Right=Meet-Int]
    {~}
    {\kwd{i32}\cap\kwd{i32}=\kwd{i32}}
    \and
    \inferrule*[Right=Join-Int]
    {~}
    {\kwd{i32}\cup\kwd{i32}=\kwd{i32}}
    \\
    \inferrule*[Right=Meet-Bool]
    {~}
    {\kwd{bool}\cap\kwd{bool}=\kwd{bool}}
    \and
    \inferrule*[Right=Join-Bool]
    {~}
    {\kwd{bool}\cup\kwd{bool}=\kwd{bool}}
    \\
    
    \inferrule*[Right=Meet-List]
    {\min(\alpha_1, \alpha_2)=\alpha}
    {\kwd{list}(\alpha_1)\cap\kwd{list}(\alpha_2)=\kwd{list}(\alpha)}
    \and
    \inferrule*[Right=Join-List]
    {\max(\alpha_1, \alpha_2)=\alpha}
    {\kwd{list}(\alpha_1)\cup\kwd{list}(\alpha_2)=\kwd{list}(\alpha)}
    \\
    
    \inferrule*[Right=Meet-Box]
    {\tau_1 \cap \tau_2=\tau}
    {\kwd{box}(\tau_1)\cap\kwd{box}(\tau_2)=\kwd{box}(\tau)}
    \and
    \inferrule*[Right=Join-Box]
    {\tau_1 \cup \tau_2=\tau}
    {\kwd{box}(\tau_1)\cup\kwd{box}(\tau_2)=\kwd{box}(\tau)}
    \\
    
    \inferrule*[Right=Meet-Shared]
    {\tau_1 \cap \tau_2=\tau}
    {\&^\kwd{s}(\tau_1)\cap\&^\kwd{s}(\tau_2)=\&^\kwd{s}(\tau)}
    \and
    \inferrule*[Right=Join-Shared]
    {\tau_1 \cup \tau_2=\tau}
    {\&^\kwd{s}(\tau_1)\cup\&^\kwd{s}(\tau_2)=\&^\kwd{s}(\tau)}
    \\

    \inferrule*[Right=Meet-Mutable]
    {\tau_{\text{c}, 1} \cap \tau_{\text{c}, 2}=\tau_\text{c}
    \\ \tau_{\text{p}, 1} \cup \tau_{\text{p}, 2}=\tau_\text{p}
    \\ \vdash \&^\kwd{m}(\tau_{\text{c}, 1}, \tau_{\text{p}, 1})
    \\ \vdash \&^\kwd{m}(\tau_{\text{c}, 2}, \tau_{\text{p}, 2})
    }
    {\&^\kwd{m}(\tau_{\text{c}, 1}, \tau_{\text{p}, 1})\cap\&^\kwd{m}(\tau_{\text{c}, 2}, \tau_{\text{p}, 2})=\&^\kwd{m}(\tau_\text{c}, \tau_\text{p})}
    \\
    \inferrule*[Right=Join-Mutable]
    {\tau_{\text{c}, 1} \cup \tau_{\text{c}, 2}=\tau_\text{c}
    \\ \tau_{\text{p}, 1} \cap \tau_{\text{p}, 2}=\tau_\text{p}
    \\ \vdash \&^\kwd{m}(\tau_{\text{c}, 1}, \tau_{\text{p}, 1})
    \\ \vdash \&^\kwd{m}(\tau_{\text{c}, 2}, \tau_{\text{p}, 2})
    }
    {\&^\kwd{m}(\tau_{\text{c}, 1}, \tau_{\text{p}, 1})\cup\&^\kwd{m}(\tau_{\text{c}, 2}, \tau_{\text{p}, 2})=\&^\kwd{m}(\tau_\text{c}, \tau_\text{p})}
\end{mathpar}

\centering
\judgement{Typing Expression}{$\Gamma\vdash e \hookrightarrow \tau\dashv\Gamma'$}
\begin{mathpar}
    \inferrule*[Right=\rulename{$\Gamma$-Ev-Int}]
    {~}
    {\Gamma\vdash \kwd{n}_\text{i32} \hookrightarrow \kwd{i32}\dashv\Gamma}
    \and
    \inferrule*[Right=\rulename{$\Gamma$-Ev-Op}]
    {\Gamma\vdash e_1\hookrightarrow \kwd{i32}\dashv\Gamma
    \\ \Gamma\vdash e_2\hookrightarrow \kwd{i32}\dashv\Gamma}
    {\Gamma\vdash e_1~\kwd{op}~e_2 \hookrightarrow \kwd{i32}\dashv\Gamma}
    \\
    
    \inferrule*[Right=\rulename{$\Gamma$-Ev-True}]
    {~}
    {\Gamma\vdash \kwd{true} \hookrightarrow \kwd{bool}\dashv\Gamma}
    \and
    \inferrule*[Right=\rulename{$\Gamma$-Ev-False}]
    {~}
    {\Gamma\vdash \kwd{false} \hookrightarrow \kwd{bool}\dashv\Gamma}
    \\
    \inferrule*[Right=\rulename{$\Gamma$-Ev-Copy}]
    {\Gamma\vdash p \hookrightarrow \tau
    \\ \tau = \kwd{i32} ~\text{or}~ \tau = \kwd{bool} }
    {\Gamma\vdash \kwd{copy}~p \hookrightarrow \tau\dashv\Gamma}
    \and
    \inferrule*[Right=\rulename{$\Gamma$-Ev-Box}]
    {\Gamma\vdash e \hookrightarrow \tau\vdash\Gamma'}
    {\Gamma\vdash \kwd{box}(e) \hookrightarrow \kwd{box}(\tau)\vdash\Gamma'}
    \\

    \inferrule*[Right=\rulename{$\Gamma$-Ev-Nil}]
    {\alpha ~\text{fresh}}
    {\Gamma\vdash \kwd{nil} \hookrightarrow \kwd{list}(\alpha)\vdash\Gamma}
    \and
    \inferrule*[Right=\rulename{$\Gamma$-Ev-Move}]
    {\Gamma\vdash p \hookrightarrow \tau
    \\ \GWt{\Gamma}{p}{\bot}{\Gamma'}}
    {\Gamma\vdash \kwd{move}~p \hookrightarrow \tau\dashv\Gamma'}
    \\
    
    \inferrule*[Right=\rulename{$\Gamma$-Ev-Shared}]
    {\Gamma\vdash p \hookrightarrow \tau
    \\ \textit{share}~ \tau ~\textit{as}~\tau_1, \tau_2
    \\ \GWt{\Gamma}{p}{\tau_1}{\Gamma'}
    }
    {\Gamma\vdash \&^\kwd{s}~p \hookrightarrow \&^\kwd{s}(\tau_2)\dashv\Gamma'}
    \\
    
    \inferrule*[Right=\rulename{$\Gamma$-Ev-Mutable}]
    {\Gamma\vdash p \hookrightarrow \tau
    \\ \textit{prophesy}~ \tau ~\textit{as}~ \tau_\text{p} 
    \\ \GWt{\Gamma}{p}{\tau_\text{p}}{\Gamma'}
    }
    {\Gamma\vdash \&^\kwd{m}~p \hookrightarrow \&^\kwd{m}(\tau, \tau_\text{p})\dashv\Gamma'}
\end{mathpar}

\centering
\judgement{Typing Statements}{$\Gamma\vdash s \hookrightarrow^\delta \dashv\Gamma'$}
\begin{mathpar}
    \inferrule*[Right=\rulename{$\Gamma$-Ex-Ret}]
    {~}
    {\Gamma\vdash \kwd{return} \hookrightarrow^0\vdash\Gamma}
    \and
    \inferrule*[Right=\rulename{$\Gamma$-Ex-Seq}]
    {\Gamma_1\vdash s_1\hookrightarrow^{\delta_1}\dashv\Gamma_2
    \\ \Gamma_2\vdash s_2\hookrightarrow^{\delta_2}\dashv\Gamma_3}
    {\Gamma_1\vdash s_1; s_2\hookrightarrow^{\delta_1+\delta_2}\dashv\Gamma_3}
    \\
    
    \inferrule*[Right=\rulename{$\Gamma$-Ex-Tick}]
    {~}
    {\Gamma\vdash\kwd{tick}(\delta)\hookrightarrow^\delta\vdash\Gamma}
    \and
    \inferrule*[Right=\rulename{$\Gamma$-Ex-Drop}]
    {\Gamma\vdash p\hookrightarrow \tau
    \\ \vdash \tau
    \\ \GWt{\Gamma}{p}{\bot}{\Gamma'}
    }
    {\Gamma\vdash \kwd{drop}~p \hookrightarrow^0\dashv \Gamma'}
    \\

    \inferrule*[Right=\rulename{$\Gamma$-Ex-Assign}]
    {\Gamma\vdash e\hookrightarrow \tau'\dashv\Gamma_1
    \\ \Gamma_1\vdash p \hookrightarrow \tau
    \\ \vdash \tau
    \\ \GWt{\Gamma_1}{p}{\tau'}{\Gamma'}}
    {\Gamma\vdash p\from e \hookrightarrow^0\dashv\Gamma'}
    \\

    \inferrule*[Right=\rulename{$\Gamma$-Ex-Cons}]
    {\Gamma\vdash e_1\hookrightarrow \kwd{i32} \dashv \Gamma_1
    \\ \Gamma_1\vdash e_2\hookrightarrow \kwd{box}(\kwd{list}(\alpha'))\dashv\Gamma_2
    \\ \GWt{\Gamma_2}{p}{\kwd{list}(\alpha')}{\Gamma'}}
    {\Gamma\vdash p\from \kwd{cons}(e_1, e_2)\hookrightarrow^{\alpha'}\dashv\Gamma'}
    \\

    \inferrule*[Right=\rulename{$\Gamma$-Ex-If}]
    {\Gamma\vdash p\hookrightarrow \kwd{bool}
    \\ \Gamma\vdash s_1\hookrightarrow^{\delta_1}\dashv\Gamma_1
    \\ \Gamma\vdash s_2\hookrightarrow^{\delta_2}\dashv\Gamma_2
    \\ \max(\delta_1, \delta_2)=\delta
    \\ \Gamma_1\sqcap\Gamma_2=\Gamma' }
    {\Gamma\vdash \kwd{if}~ p ~\kwd{then}~ s_1 ~\kwd{else}~ s_2 ~\kwd{end} \hookrightarrow^\delta \dashv\Gamma'}
    \\
    \inferrule*[Right=\rulename{$\Gamma$-Ex-Mat}]
    {\Gamma\vdash p\hookrightarrow \kwd{list}(\alpha)
    \\ \Gamma\vdash s_1\hookrightarrow^{\delta_1}\dashv\Gamma_1
    \\\\ \GWt{\Gamma}{p}{\bot}{\Gamma_{\text{b}, 1}}
    \\ \GWt{\Gamma_{\text{b}, 1}}{x_\text{hd}}{\kwd{i32}}{\Gamma_{\text{b}, 2}}
    \\ \GWt{\Gamma_{\text{b}, 2}}{x_\text{tl}}{\kwd{box}(\kwd{list}(\alpha))}{\Gamma_\text{b}}
    \\ \Gamma_\text{b}\vdash s_2\hookrightarrow^{\delta_2}\dashv\Gamma'_\text{b}
    \\\\ \Gamma'_\text{b}\vdash x_\text{tl}\hookrightarrow \kwd{list}(\beta)
    \\ \GWt{\Gamma'_\text{b}}{x_\text{hd}}{\bot}{\Gamma'_{\text{b}, 1}}
    \\ \GWt{\Gamma'_{\text{b}, 1}}{x_\text{tl}}{\bot}{\Gamma'_{\text{b}, 2}}
    \\ \GWt{\Gamma'_{\text{b}, 2}}{p}{\kwd{list}(\beta)}{\Gamma_2}
    \\\\ \max(\delta_1, \delta_2-(\alpha-\beta))=\delta
    \\ \Gamma_1\sqcap\Gamma_2=\Gamma'}
    {\Gamma\vdash \kwd{match}~ p ~ \{\kwd{nil}\mapsto s_1, \kwd{cons}(x_\text{hd}, x_\text{tl})\mapsto s_2\} \hookrightarrow^\delta \dashv\Gamma'}
    \\

    \inferrule*[Right=\rulename{$\Gamma$-Ex-App}]
    {\text{fn}~ f ~(\vec{x}_\text{param}:\vec{t}_\text{param}, \vec{x}_\text{local}:\vec{t}_\text{local}, x_\text{ret}:t_\text{ret}) \{~ s ~\}
    \\\\ \vdash f \Leftarrow (\Gamma_f, \delta_f)
    \\ \Gamma_f\vdash x_\text{ret} \hookrightarrow \tau_\text{ret}, (\forall x_i\in\vec{x}_\text{param}, i=1, ..., n) \Gamma_f \vdash x_i \hookrightarrow \tau_{\text{param}, i}
    \\\\ \Gamma_0=\Gamma, (\forall e_i\in \vec{e}, i=1, ..., n) \Gamma_{i-1}\vdash e_i\hookrightarrow\tau_{\text{arg}, i}\dashv\Gamma_i
    \\ (\forall i=1,..,n)~ \tau_{\text{param}, i} = \tau_{\text{arg}, i}
    \\ \Gamma_n \vdash p \hookrightarrow \tau
    \\ \vdash \tau
    \\ \GWt{\Gamma_n}{p}{\tau_\text{ret}}{\Gamma'}
    }
    {\Gamma\vdash p\from f(\vec{e})\hookrightarrow^{\delta_f}\dashv\Gamma'}
\end{mathpar}

\centering
\judgement{Function Analysis}{$\vdash f \Leftarrow (\Gamma_f, \delta_f)$}
\begin{mathpar}
    \inferrule
    {\text{fn}~ f ~(\vec{x}_\text{param}:\vec{t}_\text{param}, \vec{x}_\text{local}:\vec{t}_\text{local}, x_\text{ret}:t_\text{ret}) \{~ s ~\}
    \\  \vdash f \Rightarrow (\Gamma_f, \delta_f)
    \\ \Gamma_f\vdash s\hookrightarrow^\delta\dashv\Gamma'_f
    \\\\ \forall x \in \textbf{dom}(\Gamma'_f), \vdash \Gamma'_f(x)
    \\ \Gamma'_f \vdash x_\text{ret} \hookrightarrow \tau'_\text{ret}
    \\ \Gamma_f \vdash x_\text{ret} \hookrightarrow \tau_\text{ret}
    \\ \tau'_\text{ret} = \tau_\text{ret}
    \\ \delta = \delta_f}
    {\vdash f \Leftarrow (\Gamma_f, \delta_f)}
\end{mathpar}
    

\centering
\judgement{Potential Function}{$\phi(v:\tau)$}
\begin{mathpar}
    \inferrule*[Right=$\phi$-Bot]
    {~}
    {\phi(\_:\bot)=0}
    \and
    \inferrule*[Right=$\phi$-Int]
    {~}
    {\phi(\kwd{n}_\text{i32}:\kwd{i32})=0}
    \\
    \inferrule*[Right=$\phi$-True]
    {~}
    {\phi(\kwd{true}:\kwd{bool})=0}
    \and
    \inferrule*[Right=$\phi$-False]
    {~}
    {\phi(\kwd{false}:\kwd{bool})=0}
    \\
    
    \inferrule*[Right=$\phi$-Nil]
    {~}
    {\phi(\kwd{nil}:\kwd{list}(\alpha))=0}
    \\
    \inferrule*[Right=$\phi$-Cons]
    {~}
    {\phi(\kwd{cons}(\kwd{n}_\text{i32}, \kwd{box}(lv)):\kwd{list}(\alpha))=\alpha+\phi(\kwd{box}(lv):\kwd{box}(\kwd{list}(\alpha)))}
    \\
    \inferrule*[Right=$\phi$-Box]
    {~}
    {\phi(\kwd{box}(lv):\kwd{box}(\kwd{list}(\alpha)))=\phi(lv:\kwd{list}(\alpha))}
    \\
    \inferrule*[Right=$\phi$-Shared]
    {~}
    {\phi(\&(\_, v):\&^\kwd{s}(\tau))=\phi(v:\tau)}
    \\
    \inferrule*[Right=$\phi$-Mutable]
    {~}
    {\phi(\&(\_, v):\&^\kwd{m}(\tau_\text{c}, \tau_\text{p}))=\phi(v:\tau_\text{c})-\phi(v:\tau_\text{p})}
\end{mathpar}
\newpage
\section{Proof of Soundness}
\label{sec:proof}

\begin{lemma}
    Potential is non-negative and keeps subtyping:
    \begin{mathpar}
    \inferrule
    {  \vdash \tau_1
    \\ \vdash \tau_2
    \\ \tau_1 \preceq \tau_2
    }
    { 0 \leq \phi(v:\tau_1) \leq \phi(v:\tau_2)}
    \end{mathpar}
\end{lemma}
\begin{proof}
    First define the size of rich types structural inductively:
    \begin{align*}
    \textbf{size} &: \textbf{RichType} \to \NN \\
    &|~ \bot ~|~ \kwd{i32} ~|~ \kwd{bool} ~|~ \kwd{list}(\_) ~|~ \kwd{box}(\kwd{list}(\_))\mapsto 0 \\
    &|~ \&^\kwd{s}(\tau) \mapsto \textbf{size}(\tau) + 1 \\
    &|~ \&^\kwd{m}(\tau_\text{c}, \tau_\text{p}) \mapsto \textbf{size}(\tau_\text{c}) + \textbf{size}(\tau_\text{p}) + 1
    \end{align*}
    Prove by induction on $\textbf{size}(\tau_1) + \textbf{size}(\tau_2)$, and case on $\tau_1\preceq\tau_2$:
    \begin{enumerate}
        \item {\rulename{S-Bot} $\tau_1=\bot$: \textit{exfalso} due to no derivation for $\vdash \bot$; }
        \item {\rulename{S-Int}, \rulename{S-Bool} : $0 = \phi(v:\tau_1) = \phi(v:\tau_2)$ by definition;}
        \item {\rulename{S-List}, \rulename{S-Box} : $\phi(v:\tau_1) = \alpha_1\cdot n \leq \alpha_2\cdot n$, where $\alpha_1 \leq \alpha_2$ from $\tau_1\preceq \tau_2$ and $n$ is length of $v$;}
        \item {\rulename{S-Shared} : $\tau_1=\&^\kwd{s}(\tau'_1), \tau_2=\&^\kwd{s}(\tau'_2)$ by induction hypothesis on $\tau'_1, \tau'_2$, because $\textbf{size}(\tau'_1)+\textbf{size}(\tau'_2) < \textbf{size}(\tau_1) + \textbf{size}(\tau_2) = \textbf{size}(\tau'_1) + \textbf{size}(\tau'_2) + 2$, $\tau'_1\preceq\tau'_2$ from $\tau_1\preceq\tau_2$ and $\vdash \tau'_1, \vdash \tau'_2$ from $\vdash \tau_1, \vdash \tau_2$;}
        \item {\rulename{S-Mutable} : $\tau_1=\&^\kwd{m}(\tau_{\text{c}, 1}, \tau_{\text{p}, 1}), \tau_2=\&^\kwd{m}(\tau_{\text{c}, 2}, \tau_{\text{p}, 2})$, $v=\&(\_, v')$
        \begin{enumerate}
            \item {$0\leq\phi(v:\tau_1) = \phi(v':\tau_{\text{c}, 1}) - \phi(v':\tau_{\text{p}, 1})$ by induction hypothesis on $\tau_{\text{p}, 1}, \tau_{\text{c}, 1}$, because $\textbf{size}(\tau_{\text{p}, 1})+\textbf{size}(\tau_{\text{c}, 1}) < \textbf{size}(\tau_1)+\textbf{size}(\tau_2)$ and $\tau_{\text{p}, 1}\preceq \tau_{\text{c}, 1}, \vdash \tau_{\text{c}, 1}, \vdash \tau_{\text{p}, 1}$ from $\vdash \tau_1$;}
            \item {$\phi(v:\tau_1)\leq\phi(v:\tau_2)$, i.e. $\phi(v:\tau_{\text{c}, 1})-\phi(v:\tau_{\text{p}, 1})\leq \phi(v:\tau_{\text{c}, 2})-\phi(v:\tau_{\text{p}, 2})$ if and only if $\phi(v:\tau_{\text{c}, 1})\leq \phi(v:\tau_{\text{c}, 2}), \phi(v:\tau_{\text{p}, 2})\leq\phi(v:\tau_{\text{p}, 1})$. The goal can be proved by induction hypothesis because size reducing, subtyping and well-formedness inherited:
            \begin{itemize}
                \item {let $\text{x} = \text{c}$ or $\text{p}$, then $\textbf{size}(\tau_{\text{x}, 1}) + \textbf{size}(\tau_{\text{x}, 2}) < \textbf{size}(\tau_1) + \textbf{size}(\tau_2)$,\\
                the latter $= \textbf{size}(\tau_{\text{c}, 1}) + \textbf{size}(\tau_{\text{p}, 1}) + \textbf{size}(\tau_{\text{c}, 2}) + \textbf{size}(\tau_{\text{p}, 2}) + 2$; }
                \item {$\tau_{\text{c}, 1}\preceq\tau_{\text{c}, 2}, \tau_{\text{p}, 2}\preceq \tau_{\text{p}, 1}$ from $\tau_1\preceq\tau_2$;}
                \item {$\vdash \tau_{\text{c}, 1}, \vdash \tau_{\text{c}, 2}, \vdash \tau_{\text{p}, 1}, \vdash \tau_{\text{p}, 2}$ from $\vdash \tau_1, \vdash \tau_2$.}
            \end{itemize}
            }
        \end{enumerate}
        }
    \end{enumerate}
\end{proof}

\begin{lemma}[Update]
If store or context is written with new value or new type, the difference of potential over store or context is equal to that over new value or new type.
\begin{enumerate}
    \item {If $V\vDash p\rightsquigarrow v$, $\Gamma\vdash p\hookrightarrow \tau$ and $\VWt{V}{p}{v'}{V'}$,
    then $\Phi(V':\Gamma) - \Phi(V:\Gamma)=\Phi(v':\tau)-\Phi(v:\tau)$;}
    \item {If $V\vDash p\rightsquigarrow v$, $\Gamma\vdash p\hookrightarrow \tau$ and $\GWt{\Gamma}{p}{\tau'}{\Gamma'}$, 
    then $\Phi(V:\Gamma')-\Phi(V:\Gamma)=\Phi(v:\tau')-\Phi(v:\tau)$.}
\end{enumerate}
\end{lemma}
\begin{proof}
We first prove the update lemma on $V$, and then on $\Gamma$. \\
By induction on $\VWt{V}{p}{v'}{V'}$:
\begin{enumerate}
    \item {\rulename{V-Wt-Var} : We know from premise that $p = x$, $\forall y \neq x, V'(y) = V(y)$, $V'(x) = v', V(x) = v, \Gamma(x) = \tau$, then we reach
    \begin{align*}
        & \Phi(V':\Gamma)-\Phi(V:\Gamma) \\
        & = \left[\phi(V'(x):\Gamma(x)) + \sum_{y\neq x}\phi(V'(y):\Gamma(y))\right] - \left[\phi(V(x):\Gamma(x)) + \sum_{y\neq x}\phi(V(y):\Gamma(y))\right] \\
        & = \phi(v':\tau) -\phi(v:\tau)
    \end{align*}
    }
    \item {\rulename{V-Wt-Box} : We know from premise that $p = * p_1, V\vDash p_1\rightsquigarrow\kwd{box}(v), \VWt{V}{p_1}{\kwd{box}(v')}{V'}, \Gamma\vdash p_1 \hookrightarrow \kwd{box}(\tau)$, then we reach from hypothesis 
    \begin{align*}
        & \Phi(V':\Gamma)-\Phi(V:\Gamma) \\
        & = \phi(\kwd{box}(v'):\kwd{box}(\tau))-\phi(\kwd{box}(v):\kwd{box}(\tau)) \\
        & = \phi(v':\tau) - \phi(v:\tau)
    \end{align*} 
    }
    \item {\rulename{V-Wt-Borrow} : We know from premise that $p = * p_1, V\vDash p_1\rightsquigarrow\&(q, v), \VWt{V}{q}{v'}{V'}, \VWt{V'}{p_1}{\&(q, v')}{V''}$, and $p_1, q$ are separate and will not form a circle. Also we know $\Gamma\vdash p_1 \hookrightarrow \&^\kwd{m}(\tau, \tau_\text{p})$ \textbf{because only mutable borrows can be updated with values}, $\Gamma\vdash q\hookrightarrow \tau_\text{p}$ because $\Gamma$ is well formed by borrow checker. Therefore by induction hypothesis, we reach 
    \begin{align*}
    &\Phi(V'':\Gamma)-\Phi(V:\Gamma) \\
    &=\Phi(V'':\Gamma)-\Phi(V':\Gamma)+\Phi(V':\Gamma)-\Phi(V:\Gamma) \\
    &= \phi(v':\tau_\text{p})-\phi(v:\tau_\text{p}) + \phi(\&(q, v'):\&^\kwd{m}(\tau, \tau_\text{p}))-\phi(\&(q, v):\&^\kwd{m}(\tau, \tau_\text{p})) \\
    &= \phi(v':\tau_\text{p})-\phi(v:\tau_\text{p}) + [ \phi(v':\tau)-\phi(v':\tau_\text{p})]-[\phi(v:\tau)-\phi(v:\tau_\text{p})] \\
    &= \phi(v':\tau)-\phi(v:\tau)
    \end{align*}
    }
\end{enumerate}
By induction on $\GWt{\Gamma}{p}{\tau'}{\Gamma'}$:
\begin{enumerate}
    \item {\rulename{$\Gamma$-Wt-Var} : We know from premise that $p = x$, $\forall y \neq x, \Gamma'(y) = \Gamma(y)$, $\Gamma'(x) = \tau', \Gamma(x) = \tau, V(x) = v$, then we reach 
    \begin{align*}
        & \Phi(V:\Gamma')-\Phi(V:\Gamma) \\
        & = [\phi(V(x):\Gamma'(x)) + \sum_{y\neq x}\phi(V(y):\Gamma'(y))] - [\phi(V(x):\Gamma(x)) + \sum_{y\neq x}\phi(V(y):\Gamma(y))]\\
        & = \phi(v:\tau') -\phi(v:\tau)
    \end{align*}
    }
    \item {\rulename{$\Gamma$-Wt-Box} : We know from premise that $p = * p_1$, $\Gamma\vDash p_1\rightsquigarrow\kwd{box}(\tau)$, $\GWt{\Gamma}{p_1}{\kwd{box}(\tau')}{\Gamma'}$, and $V\vDash p_1 \rightsquigarrow \kwd{box}(v)$, then we reach from hypothesis 
    \begin{align*}
        & \Phi(V:\Gamma')-\Phi(V:\Gamma) \\
        & = \phi(\kwd{box}(v):\kwd{box}(\tau'))-\phi(\kwd{box}(v):\kwd{box}(\tau)) \\
        & = \phi(v:\tau') - \phi(v:\tau)
    \end{align*}
    }
    \item {\rulename{$\Gamma$-Wt-Shared} : We know from premise that $p = * p_1$, $\Gamma\vdash p_1 \hookrightarrow \&^\kwd{s}(\tau)$, $\GWt{\Gamma}{p_1}{\&^\kwd{s}(\tau')}{\Gamma'}$, and $V\vDash p_1 \rightsquigarrow \&(\_, v)$, then we reach from hypothesis 
    \begin{align*}
        & \Phi(V:\Gamma')-\Phi(V:\Gamma) \\
        & = \phi(\&(\_, v): \&^\kwd{s}(\tau')) - \phi(\&(\_, v): \&^\kwd{s}(\tau)) \\
        & = \phi(v:\tau') - \phi(v:\tau)
    \end{align*}
    }
    \item {\rulename{$\Gamma$-Wt-Mutable} : We know from premise that $p = * p_1$, $\Gamma\vdash p_1 \hookrightarrow \&^\kwd{m}(\tau, \tau_\text{p})$, $ \GWt{\Gamma}{p_1}{\&^\kwd{m}(\tau', \tau_\text{p})}{\Gamma'}$, and $V\vDash p_1 \rightsquigarrow \&(\_, v)$, then we reach from hypothesis 
    \begin{align*}
        & \Phi(V:\Gamma')-\Phi(V:\Gamma) \\
        & = \phi(\&(\_, v): \&^\kwd{m}(\tau', \tau_\text{p})) - \phi(\&(\_, v): \&^\kwd{m}(\tau, \tau_\text{p})) \\
        & = [\phi(v:\tau')-\phi(v:\tau_\text{p}] - [\phi(v:\tau)-\phi(v:\tau_\text{p})] \\
        & = \phi(v:\tau') - \phi(v:\tau)
    \end{align*}
    }
\end{enumerate}
\end{proof}

\newpage
\begin{lemma}[Evaluation]
If $V\vDash e\rightsquigarrow v, \Gamma\vdash e\hookrightarrow \tau\dashv\Gamma'$, then $\Phi(V:\Gamma')-\Phi(V:\Gamma)= -\phi(v:\tau)$.
\end{lemma}
\begin{proof}
By induction on $e$:
\begin{enumerate}
    \item {$e=\kwd{n}_\text{i32}, e_1 ~\kwd{op}~ e_2, \kwd{true}, \kwd{false}, \kwd{copy}~ p, \kwd{nil}$ : $\Gamma'=\Gamma$ and $\phi(v:\tau) = 0$;}
    \item {$e=\kwd{move}~p$ : We know that $V\vDash p\rightsquigarrow v$, $\Gamma\vdash p\hookrightarrow \tau$ and $\GWt{\Gamma}{p}{\bot}{\Gamma'}$, hence $\Phi(V:\Gamma')-\Phi(V:\Gamma)=\Phi(v:\bot)-\Phi(v:\tau)=-\Phi(v:\tau)$;} 
    \item {$e=\kwd{box}(e_1)$: simply by induction, $\Phi(V:\Gamma')-\Phi(V:\Gamma)=\Phi(v_1:\tau_1)=\Phi(\kwd{box}(v_1):\kwd{box}(\tau_1))$;}
    \item {$e=\&^\kwd{s}~p$: We assert that for all $\textit{share}~\tau~\textit{as}~\tau_1, \tau_2$, $\phi(v:\tau)=\phi(v:\tau_1)+\phi(v:\tau_2)$, then $\Phi(V:\Gamma')-\Phi(V:\Gamma)=\phi(v:\tau)-\phi(v:\tau_1)=\phi(v:\tau_2)=\phi(\&(p, v):\&^\kwd{s}(\tau_2))$, where the assertion can be proved by simple induction on sharing;}
    \item {$e=\&^\kwd{m}~p$: $V\vDash p\rightsquigarrow v, \Gamma\vdash p\hookrightarrow \tau, \GWt{\Gamma}{p}{\tau_\text{p}}{\Gamma'}$, with $\textit{prophecy}~ \tau ~\textit{as}~ \tau_\text{p}$, then $\Phi(V:\Gamma')-\Phi(V:\Gamma)=\phi(v:\tau_\text{p})-\phi(v:\tau)= - \phi(\&(p, v):\&^\text{m}(\tau, \tau_\text{p}))$.}
\end{enumerate}
\end{proof}

\begin{theorem}[Soundness]
If $V\vDash s \rightsquigarrow^{\delta_V} \Dashv V', \Gamma\vdash s \hookrightarrow^{\delta_\Gamma}\dashv\Gamma'$, then $\Phi(V':\Gamma')-\Phi(V:\Gamma)+\delta_V\leq\delta_\Gamma$.
\end{theorem}
\begin{proof}
By induction on $V\vDash s \rightsquigarrow^{\delta_V} \Dashv V'$:
\begin{enumerate}
    \item { \rulename{V-Ex-Ret} $s=\kwd{return}$ : $\delta_V=\delta_\Gamma=0, V'=V, \Gamma'=\Gamma$;}
    \item { \rulename{V-Ex-Seq} $s=s_1; s_2$ : By induction, we have $V\vDash s_1\rightsquigarrow^{\delta_{V, 1}}\Dashv V_1\vDash s_2\rightsquigarrow^{\delta_{V, 2}}\Dashv V'$, $\Gamma\vdash s_1\hookrightarrow^{\delta_{\Gamma, 1}}\dashv \Gamma_1\vdash s_2\hookrightarrow^{\delta_{\Gamma, 1}}\dashv \Gamma'$ and $\Phi(V_1:\Gamma_1)-\Phi(V:\Gamma)+\delta_{V, 1}\leq\delta_{\Gamma, 1}, \Phi(V':\Gamma')-\Phi(V_1:\Gamma_1)+\delta_{V, 2}\leq\delta_{\Gamma, 2}$. It is obvious that $\Phi(V:\Gamma)-\Phi(V':\Gamma')+\delta_{\Gamma, 1}+\delta_{\Gamma, 2}\geq\delta_{V, 1}+\delta_{V, 2}$;}
    \item {\rulename{V-Ex-Tick} $s=\kwd{tick}(\delta)$ : $\delta_V=\delta_\Gamma=\delta, V'=V, \Gamma'=\Gamma$;}
    \item {\rulename{V-Ex-Drop} $s=\kwd{drop}~p$ : $V=V', \delta_V=\delta_\Gamma=0$, we only need to prove that $\Phi(V:\Gamma')-\Phi(V:\Gamma')\leq 0$, but we know that $\GWt{\Gamma}{p}{\bot}{\Gamma'}$, then $\Phi(V:\Gamma')-\Phi(V:\Gamma)=\phi(v:\bot)-\phi(v:\tau)=-\phi(v:\tau)\leq0$, with the help of $\vdash \tau$ from premise of $\Gamma \vdash s \hookrightarrow^{\delta_\Gamma}\dashv \Gamma'$;}
    
    \item {\rulename{V-Ex-Assign} $s=p\from e$ : $\delta_V=\delta_\Gamma=0$, and we have $\Gamma\vdash e\hookrightarrow\tau'\dashv\Gamma_1$, $\Gamma_1\vdash p\hookrightarrow\tau$, $\GWt{\Gamma_1}{p}{\tau'}{\Gamma'}$, $V\vDash e\rightsquigarrow v'$, $V\vDash p\rightsquigarrow v$ and $\VWt{V}{p}{v'}{V'}$. With lemmas, we can imply that $\Phi(V:\Gamma_1)-\Phi(V:\Gamma)=-\phi(v':\tau')$, $\Phi(V:\Gamma')-\Phi(V:\Gamma_1)=\phi(v:\tau')-\phi(v:\tau)$ and $\Phi(V':\Gamma')-\Phi(V:\Gamma')=\phi(v':\tau')-\phi(v:\tau')$. Therefore $\Phi(V':\Gamma')-\Phi(V:\Gamma)=-\phi(v:\tau)\leq0$, with the help of $\vdash \tau$ from premise of $\Gamma \vdash s \hookrightarrow^{\delta_\Gamma}\dashv \Gamma'$; }
    
    \item{\rulename{V-Ex-Cons} $s=p\from \kwd{cons}(e_1; e_2)$ : $\delta_V=0, \delta_\Gamma=\alpha'$, and we have $\Gamma\vdash e_1\hookrightarrow\kwd{i32}$, $\Gamma\vdash e_2\hookrightarrow\kwd{box}(\kwd{list}(\alpha'))\dashv\Gamma_1$, $\Gamma_1\vdash p\hookrightarrow\kwd{list}(\alpha)$, $\GWt{\Gamma_1}{p}{\kwd{list}(\alpha')}{\Gamma'}$, $V\vDash p\rightsquigarrow v$, $V\vDash e_1\rightsquigarrow \kwd{n}_\text{i32}$, $V\vDash e_2\rightsquigarrow \kwd{box}(lv)$ and $\VWt{V}{p}{\kwd{cons}(n, \kwd{box}(lv))}{V'}$. With lemmas, we can imply that $\Phi(V:\Gamma_2)-\Phi(V:\Gamma)=-\phi(\kwd{box}(lv):\kwd{box}(\kwd{list}(\alpha)))$, $\Phi(V:\Gamma')-\Phi(V:\Gamma_2)=\phi(v:\kwd{list}(\alpha'))-\phi(v:\kwd{list}(\alpha))$, $\Phi(V':\Gamma')-\Phi(V:\Gamma')=\phi(\kwd{cons}(\kwd{n}_\text{i32}, \kwd{box}(lv)):\kwd{list}(\alpha'))-\phi(v:\kwd{list}(\alpha'))$. Therefore, we have $\Phi(V':\Gamma')-\Phi(V:\Gamma)=\alpha-\phi(v:\kwd{list}(\alpha'))\leq\alpha=\delta_\Gamma$, with the help of $\vdash \kwd{list}(\alpha')$;}
    
    \item {\rulename{V-Ex-IfT/F}$s=\kwd{if}~p~\kwd{then}~s_1~\kwd{else}~s_2~\kwd{end}$ : If $V\vDash p\rightsquigarrow\kwd{true}$, then we have $V\vDash s_1\rightsquigarrow^{\delta_{V, 1}}\Dashv V'$, $\Gamma\vdash s_1\hookrightarrow^{\delta_{\Gamma, 1}}\dashv\Gamma_1$ and $\Phi(V':\Gamma_1)-\Phi(V:\Gamma)+{\delta_{V, 1}}\leq{\delta_{\Gamma, 1}}$. $\Gamma_1\sqcap\Gamma_2=\Gamma'$ indicates $\Phi(V':\Gamma')\leq\Phi(V':\Gamma_1)$. From premise of statics, $\delta_{\Gamma, 1}\leq\delta_\Gamma$, hence $\Phi(V':\Gamma')-\Phi(V:\Gamma)+\delta_{V, 1}\leq\delta_\Gamma$. If $V\vDash p\rightsquigarrow\kwd{false}$, it is similar to prove;}
    \item {\rulename{V-Ex-Mat-Nil/Cons}$s=\kwd{match}~p~\{\kwd{nil}\mapsto s_1, \kwd{cons}(x_\text{hd}, x_\text{tl})\mapsto s_2\}$ : If $V\vDash p\rightsquigarrow\kwd{nil}$, it is similar to $\kwd{if}$ statement. We now turn to the possibility $V\vDash p\rightsquigarrow\kwd{cons}(n, \kwd{box}(lv))$. In such a case, we can obviously get that $\Phi(V_\text{b}:\Gamma_\text{b})-\Phi(V:\Gamma)=-\alpha$. From $V_\text{b}\vDash s_2\rightsquigarrow^{\delta_V}\Dashv V'_\text{b}, \Gamma_\text{b}\vdash s_2\hookrightarrow^{\delta_{\Gamma, 2}}\dashv \Gamma'_\text{b}$, we have that $\Phi(V'_\text{b}:\Gamma'_\text{b})-\Phi(V_\text{b}:\Gamma_\text{b})+\delta_V\leq\delta_{\Gamma, 2}$. Also, $\Phi(V':\Gamma_2)-\Phi(V'_\text{b}:\Gamma'_\text{b})=\beta$. Sum up inequalities, we have $\Phi(V':\Gamma_2)-\Phi(V:\Gamma)+\delta_V\leq\delta_{\Gamma, 2}-\alpha+\beta$. Similarly with $\Gamma'=\Gamma_1\sqcap\Gamma_2$ and $\delta_{\Gamma, 2}-\alpha+\beta\leq\delta_\Gamma$, we final reach that $\Phi(V':\Gamma')-\Phi(V:\Gamma)+\delta_V\leq\delta_\Gamma$; }
    \item {\rulename{V-Ex-App} $s=p\from f(\vec{e})$ : Assume $\text{fn}~ f ~(\vec{x}_\text{param}:\vec{t}_\text{param}, \vec{x}_\text{local}:\vec{t}_\text{local}, x_\text{ret}:t_\text{ret}) \{~ s_f ~\}, V\vDash p\rightsquigarrow v$, $\Gamma_n\vdash p\hookrightarrow \tau, \vdash \tau$, $V\vDash p\from f(\vec{e})\rightsquigarrow^{\delta_V}\Dashv V'$, and $\Gamma\vdash p\from f(\vec{e})\hookrightarrow^{\delta_\Gamma}\dashv\Gamma'$.
    
    To use induction hypothesis on $V_\text{b}\vDash s_f\rightsquigarrow^{\delta_V} V'_\text{b}$ from premise of dynamics, we need statics $\Gamma_\text{b}\vdash s_f\hookrightarrow^{\delta_\text{b}}\dashv \Gamma'_\text{b}$, where $\GWt{\Gamma_n}{\vec{x}_\text{param}}{\vec{\tau}_\text{arg}}{\Gamma_{\text{b}, 1}}$, $\GWt{\Gamma_{\text{b}, 1}}{\vec{x}_\text{local}}{\vec{\tau}_\text{local}}{\Gamma_{\text{b}, 2}}$ and $\GWt{{\Gamma_{\text{b}, 2}}}{x_\text{ret}}{\tau_\text{ret}}{\Gamma_\text{b}}$. However, it is not found but $\Gamma_f\vdash s_f \hookrightarrow^\delta \dashv\Gamma'_f$ appears in premise of $\vdash f \Leftarrow (\Gamma_f, \delta_f)$. Via context extension and weakening rules, we can reach our goal by proving $\Gamma_f \sqsubseteq \Gamma_\text{b}$. It is true because $\forall i=1, ..., n, \tau_{\text{param}, i} = \tau_{\text{arg}, i}$.
    
    \begin{enumerate}
        \item {$\Phi(V:\Gamma_n)-\Phi(V:\Gamma) = -\sum_i\phi(v_i:\tau_{\text{arg}, i})$;}
        \item {$\Phi(V_\text{b}:\Gamma_\text{b}) - \Phi(V:\Gamma_n) = \sum_i\phi(v_i:\tau_{\text{arg}, i})$; }

        \item {$\Phi(V'_\text{b}:\Gamma'_\text{b}) - \Phi(V_\text{b}:\Gamma_\text{b}) + \delta_V \leq \delta_\text{b}$; }
        
        \item {$\Phi(V':\Gamma')-\Phi(V'_\text{b}:\Gamma'_\text{b}) = -\sum_{x\in\vec{x}_\text{param} \text{or} x\in\vec{x}_\text{local}}
        \phi(v_x:\tau_x)-\phi(v:\tau) \leq 0$, \\
        considering that it does not affect resource to move $v_\text{ret}:\tau'_\text{ret}$ from $x_\text{ret}$ to $p$, \\
        while the erasure at parameters, local variables and $p$ affects; }

        \item {$\delta_\text{b} = \delta$ by weakening rules;}
        \item {$\delta = \delta_f$ from premise of $\vdash f \Leftarrow (\Gamma_f, \delta_f)$; }
        \item {$\delta_f = \delta_\Gamma$ from statics.}
    \end{enumerate}
    Sum up inequalities above, we reach $\Phi(V':\Gamma')-\Phi(V:\Gamma)+\delta_V\leq\delta_\Gamma$.
    }
\end{enumerate}
\end{proof}




\end{document}
\endinput
%%
%% End of file `sample-manuscript.tex'.
