%%
%% This is file `sample-manuscript.tex',
%% generated with the docstrip utility.
%%
%% The original source files were:
%%
%% samples.dtx  (with options: `manuscript')
%% 
%% IMPORTANT NOTICE:
%% 
%% For the copyright see the source file.
%% 
%% Any modified versions of this file must be renamed
%% with new filenames distinct from sample-manuscript.tex.
%% 
%% For distribution of the original source see the terms
%% for copying and modification in the file samples.dtx.
%% 
%% This generated file may be distributed as long as the
%% original source files, as listed above, are part of the
%% same distribution. (The sources need not necessarily be
%% in the same archive or directory.)
%%
%% Commands for TeXCount
%TC:macro \cite [option:text,text]
%TC:macro \citep [option:text,text]
%TC:macro \citet [option:text,text]
%TC:envir table 0 1
%TC:envir table* 0 1
%TC:envir tabular [ignore] word
%TC:envir displaymath 0 word
%TC:envir math 0 word
%TC:envir comment 0 0
%%
%%
%% The first command in your LaTeX source must be the \documentclass command. This is the generic manuscript mode required for submission and peer review.
\documentclass[acmsmall,screen,nonacm]{acmart} % review,anonymous,
%% To ensure 100% compatibility, please check the white list of
%% approved LaTeX packages to be used with the Master Article Template at
%% https://www.acm.org/publications/taps/whitelist-of-latex-packages 
%% before creating your document. The white list page provides 
%% information on how to submit additional LaTeX packages for 
%% review and adoption.
%% Fonts used in the template cannot be substituted; margin 
%% adjustments are not allowed.

%%
%% \BibTeX command to typeset BibTeX logo in the docs
\AtBeginDocument{%
  \providecommand\BibTeX{{%
    \normalfont B\kern-0.5em{\scshape i\kern-0.25em b}\kern-0.8em\TeX}}}

%% Rights management information.  This information is sent to you
%% when you complete the rights form.  These commands have SAMPLE
%% values in them; it is your responsibility as an author to replace
%% the commands and values with those provided to you when you
%% complete the rights form.
\setcopyright{acmlicensed}
\copyrightyear{2024}
\acmYear{2024}
\acmDOI{XXXXXXX.XXXXXXX}

%% These commands are for a PROCEEDINGS abstract or paper.
% \acmConference[Conference acronym 'XX]{Make sure to enter the correct
%   conference title from your rights confirmation emai}{June 03--05,
%   2018}{Woodstock, NY}
%
%  Uncomment \acmBooktitle if th title of the proceedings is different
%  from ``Proceedings of ...''!
%
%\acmBooktitle{Woodstock '18: ACM Symposium on Neural Gaze Detection,
%  June 03--05, 2018, Woodstock, NY} 

%% These commands are for a JOURNAL article.
\acmJournal{JACM}
\acmVolume{37}
\acmNumber{4}
\acmArticle{111}
\acmMonth{10}

\acmISBN{978-1-4503-XXXX-X/18/06}


%%
%% Submission ID.
%% Use this when submitting an article to a sponsored event. You'll
%% receive a unique submission ID from the organizers
%% of the event, and this ID should be used as the parameter to this command.
%%\acmSubmissionID{123-A56-BU3}

%%
%% For managing citations, it is recommended to use bibliography
%% files in BibTeX format.
%%
%% You can then either use BibTeX with the ACM-Reference-Format style,
%% or BibLaTeX with the acmnumeric or acmauthoryear sytles, that include
%% support for advanced citation of software artefact from the
%% biblatex-software package, also separately available on CTAN.
%%
%% Look at the sample-*-biblatex.tex files for templates showcasing
%% the biblatex styles.
%%

%%
%% The majority of ACM publications use numbered citations and
%% references.  The command \citestyle{authoryear} switches to the
%% "author year" style.
%%
%% If you are preparing content for an event
%% sponsored by ACM SIGGRAPH, you must use the "author year" style of
%% citations and references.
%% Uncommenting
%% the next command will enable that style.
%%\citestyle{acmauthoryear}

%%
%% end of the preamble, start of the body of the document source.

\usepackage{multicol}
\usepackage{multirow}
\usepackage{graphicx}
\usepackage{arydshln}
\usepackage{listings, listings-rust}
\usepackage{xspace}
\usepackage{cleveref}
\crefname{theorem}{Thm.}{Thms.}
\crefname{lemma}{Lem.}{Lemmas}
\crefname{corollary}{Cor.}{Cors.}
\crefname{figure}{Fig.}{Figs.}
\crefname{definition}{Defn.}{Defns.}
\crefname{table}{Tab.}{Tabs.}
\crefname{appendix}{Appendix}{Appendices}
\crefformat{section}{\S#2#1#3}
\crefmultiformat{section}{\S#2#1#3}{ and~\S#2#1#3}{, \S#2#1#3}{ and~\S#2#1#3}
\crefname{example}{Ex.}{Exs.}
\crefname{item}{item}{items}
\crefname{footnote}{footnote}{footnotes}
\crefname{observation}{Obs.}{Obs.}
\crefname{remark}{Remark}{Remarks}
\crefname{proposition}{Prop.}{Props.}
\crefname{equation}{Eqn.}{Eqns.}
\crefname{counterexample}{Counterexample}{Counterexamples}
\crefname{property}{Property}{Properties}
\crefname{algorithm}{Algorithm}{Algorithms}
\usepackage{subcaption}
\usepackage[shortlabels]{enumitem}
\usepackage{commands}
\usepackage{mathpartir}
\newcommand{\from}{:=}




\def\vDash{\vdash}
\def\Dashv{\dashv}

% \def\VWt#1#2#3#4{#4=#1\cup\{#2 \rightsquigarrow #3\}}
\def\VWt#1#2#3#4{\{#1\}#2 \from #3\{#4\}}
% \def\GWt#1#2#3#4{#4=#1\cup\{#2 \hookrightarrow #3\}}
\def\GWt#1#2#3#4{\{#1\}#2 \from #3\{#4\}}




\lstset{
  basicstyle=\ttfamily,
}
\def\kwd#1{\textbf{\texttt{#1}}}
\def\todo#1{\textcolor{red}{\textbf{#1}}}

\citestyle{acmauthoryear}


\AtBeginDocument{%
  \setlength\abovedisplayskip{0.5\abovedisplayskip}%
  \setlength\belowdisplayskip{0.5\belowdisplayskip}%
  \setlength\abovedisplayshortskip{0.5\abovedisplayshortskip}%
  \setlength\belowdisplayshortskip{0.5\belowdisplayshortskip}%
  \setlength\floatsep{0.5\floatsep}%
  \def\MathparLineskip{\lineskip=0.29cm}%
  \setlength\abovecaptionskip{0.5\abovecaptionskip}%
}
\setlist{leftmargin=*}

\newenvironment{DIFnomarkup}{}{}

\sloppy
\begin{document}

%%
%% The "title" command has an optional parameter,
%% allowing the author to define a "short title" to be used in page headers.
% \title{Automatic Amortized Resource Analysis with Borrow Mechanism}
% \title{Towards Automatic Amortized Resource Analysis for Rust}
\title{Automatic Linear Resource Bound Analysis for Rust via Prophecy Potentials}

%%
%% The "author" command and its associated commands are used to define
%% the authors and their affiliations.
%% Of note is the shared affiliation of the first two authors, and the
%% "authornote" and "authornotemark" commands
%% used to denote shared contribution to the research.
\author{Qihao Lian}
\affiliation{%
  \department{Key Laboratory of High Confidence Software Technologies (Peking University), Ministry of Education; School of Computer Science}
  \institution{Peking University}
  \country{China}
}
\email{mepy@stu.pku.edu.cn}
\author{Di Wang}
\affiliation{%
  \department{Key Laboratory of High Confidence Software Technologies (Peking University), Ministry of Education; School of Computer Science}
  \institution{Peking University}
  \country{China}
}
\email{wangdi95@pku.edu.cn}
%%
%% By default, the full list of authors will be used in the page
%% headers. Often, this list is too long, and will overlap
%% other information printed in the page headers. This command allows
%% the author to define a more concise list
%% of authors' names for this purpose.
%% \renewcommand{\shortauthors}{Q. Lian and D. Wang}

%%
%% The abstract is a short summary of the work to be presented in the
%% article.

During the early stages of interface design, designers need to produce multiple sketches to explore a design space.  Design tools often fail to support this critical stage, because they insist on specifying more details than necessary. Although recent advances in generative AI have raised hopes of solving this issue, in practice they fail because expressing loose ideas in a prompt is impractical. In this paper, we propose a diffusion-based approach to the low-effort generation of interface sketches. It breaks new ground by allowing flexible control of the generation process via three types of inputs: A) prompts, B) wireframes, and C) visual flows. The designer can provide any combination of these as input at any level of detail, and will get a diverse gallery of low-fidelity solutions in response. The unique benefit is that large design spaces can be explored rapidly with very little effort in input-specification. We present qualitative results for various combinations of input specifications. Additionally, we demonstrate that our model aligns more accurately with these specifications than other models. 

% OLD ABSTRACT
%When sketching Graphical User Interfaces (GUIs), designers need to explore several aspects of visual design simultaneously, such as how to guide the user’s attention to the right aspects of the design while making the intended functionality visible. Although current Large Language Models (LLMs) can generate GUIs, they do not offer the finer level of control necessary for this kind of exploration. To address this, we propose a diffusion-based model with multi-modal conditional generation. In practice, our model optionally takes semantic segmentation, prompt guidance, and flow direction to generate multiple GUIs that are aligned with the input design specifications. It produces multiple examples. We demonstrate that our approach outperforms baseline methods in producing desirable GUIs and meets the desired visual flow.

% Designing visually engaging Graphical User Interfaces (GUIs) is a challenge in HCI research. Effective GUI design must balance visual properties, like color and positioning, with user behaviors to ensure GUIs easy to comprehend and guide attention to critical elements. Modern GUIs, with their complex combinations of text, images, and interactive components, make it difficult to maintain a coherent visual flow during design.
% Although current Large Language Models (LLMs) can generate GUIs, they often lack the fine control necessary for ensuring a coherent visual flow. To address this, we propose a diffusion-based model that effectively handles multi-modal conditional generation. Our model takes semantic segmentation, optional prompt guidance, and ordered viewing elements to generate high-fidelity GUIs that are aligned with the input design specifications.
% We demonstrate that our approach outperforms baseline methods in producing desirable GUIs and meets the desired visual flow. Moreover, a user study involving XX designers indicates that our model enhances the efficiency of the GUI design ideation process and provides designers with greater control compared to existing methods.    



% %%%%%%%%%%%%%%%%%%%%%%%%%%%%%%%%%%%%%%%%%%%%%%%%%%%%%%
% % Writing Clinic Comments:
% %%%%%%%%%%%%%%%%%%%%%%%%%%%%%%%%%%%%%%%%%%%%%%%%%%%%%%
% % Define: Effective UI design
% % Motivate GANs and write in full form.
% % LLMs vs ControlNet vs GANs
% % Say something about the Figma plugin?
% % Write the work is novel or what has been done before
% % What is desirable UI and how to evalutate that?
% % Visual Flow - main theme (center around it)
% % Re-Title: use word Flow!
% % Use ControlNet++ & SPADE for abstract.
% % Write about input/output. 
% % Why better than previous work?
% %%%%%%%%%%%%%%%%%%%%%%%%%%%%%%%%%%%%%%%%%%%%%%%%%%%%%

% % v2:
% % \noindent \textcolor{red}{\textbf{NEW Abstract!} (Post Writing Clinic 1 - 25-Jun)}

% % \noindent \textcolor{red}{----------------------------------------------------------------------}

% % \noindent Designing user interfaces (UIs) is a time-consuming process, particularly for novice designers. 
% % Creating UI designs that are effective in market funneling or any other designer defined goal requires a good understanding of the visual flow to guide users' attention to UI elements in the desired order. 
% % While current Large Language Models (LLMs) can generate UIs from just prompts, they often lack finer pixel-precise control and fail to consider visual flow. 
% % In this work, we present a UI synthesis method that incorporates visual flow alongside prompts and semantic layouts. 
% % Our efficient approach uses a carefully designed Generative Adversarial Network (GAN) optimized for scenarios with limited data, making it more suitable than diffusion-based and large vision-language models.
% % We demonstrate that our method produces more "desirable" UIs according to the well-known contrast, repetition, alignment, and proximity principles of design. 
% % We further validate our method through comprehensive automatic non-reference, human-preference aligned network scoring and subjective human evaluations.
% % Finally, an evaluation with xx non-expert designers using our contributed Figma plugin shows that <method-name> improves the time-efficiency as well as the overall quality of the UI design development cycle.

% % \noindent \textcolor{red}{----------------------------------------------------------------------}


% \noindent \textcolor{blue}{\textbf{NEW Abstract!} (Pre Writing Clinic 9-July)}

% \noindent \textcolor{blue}{----------------------------------------------------------------------}

% \noindent Exploring different graphical user interface (GUI) design ideas is time-consuming, particularly for novice designers. 
% Given the segmentation masks, design requirement as prompt, and/or preferred visual flow, we aim to facilitate creative exploration for GUI design and generate different UI designs for inspiration.
% While current Vision Language Models (VLMs) can generate GUIs from just prompts, they often lack control over visual concepts and flow that are difficult to convey through language during the generation process. 
% In this work, we present FlowGenUI, a semantic map-guided GUI synthesis method that optionally incorporates visual flow information based on the user's choice alongside language prompts. 
% We demonstrate that our model not only creates more realistic GUIs but also creates "predictable" (how users pay attention to and order of looking at GUI elements) GUIs.
% Our approach uses Stable Diffusion (SD), a large paired image-text pretrained diffusion model with a rich latent space that we steer toward realistic GUIs using a trainable copy of SD's encoder for every condition (segmentation masks, prompts, and visual flow). 
% We further provide a semantic typography feature to create custom text-fonts and styles while also alleviating SD's inherent limitations in drawing coherent, meaningful and correct aspect-ratio text. 
% Finally, a subjective evaluation study of XX non-expert and expert designers demonstrates the efficiency and fidelity of our method.


% This process encourages creativity and prevents designers from falling into habitual patterns.


% ------------------------------------------------------------------
% Joongi Why is it important to create realistic GUI?
% I do not see how the Visual Flow given on the left hand side is reflected in the results on the right hand side. 
% I’d avoid making unsubstantiated claims about designers (falling into habitual patterns).
% The UIs you generate do not “align with users’ attention patterns” but rather try to control it (that’s what visual flow means)
% ------------------------------------------------------------------
% Comments - Writing Clinic - 9th July:
% Improve title. More names: FlowGen
% Figure 1: Use an inference time hand-drawn mask
% Figure 1: Show both workflows. Add a designer --> Input.
% Figure 1: Make them more diverse
% ------------------------------------------------------------------
% Designing graphical user interfaces (GUIs) requires human creativity and time. Designers often fall into habitual patterns, which can limit the exploration of new ideas. 
% To address this, we introduce FlowGenUI, a method that facilitates creative exploration and generates diverse GUI designs for inspiration. By using segmentation masks, design requirements as prompts, and/or selected visual flows, our approach enhances control over the visual concepts and flows during the generation process, which current Vision Language Models (VLMs) often lack.
% FlowGenUI uses Stable Diffusion (SD), a largely pretrained text-to-image diffusion model, and guides it to create realistic GUIs. 
% We achieve this by using a trainable copy of SD's encoder for each condition (segmentation masks, prompts, and visual flow). 
% This method enables the creation of more realistic and predictable GUIs that align with users' attention patterns and their preferred order of viewing elements.
% We also offer a semantic typography feature that creates custom text fonts and styles while addressing SD's limitations in generating coherent, meaningful, and correctly aspect-ratio text.
% Our approach's efficiency and fidelity are evaluated through a subjective user study involving XX designers. 
% The results demonstrate the effectiveness of FlowGenUI in generating high-quality GUI designs that meet user requirements and visual expectations.

% ---------------------------------------


%A critical and general issue remains while using such deep generative priors: creating coherent, meaningful and correct aspect-ratio text. 
%We tackle this issue within our framework and additionally provide a semantic typography feature to create custom text-fonts and styles. 


% %Creating UI designs that are effective in market funneling or any other designer-defined goal requires a good understanding of the visual flow to guide users' attention to UI elements in the desired order. 
% %While current largely pre-trained Vision Language Models (VLMs) can generate GUIs from just prompts, they often lack finer or pixel-precise control which can be crucial for many easy-to-understand visual concepts but difficult to convey through language. 
% % However, obtaining such pixe-level labels is an extremely expensive so we
% % For example - overlaying text on images with certain aspect ratios and two equally separated buttons 
% Additionally, all prior GUI generation work fails to consider visual flow information during the generation process. 
% We demonstrate that visual flow-informed generation not only creates more realistic and human-friendly GUIs but also creates "predictable" (how users pay attention to and order of looking at GUI elements) UIs that could be beneficial for designers for tasks like creating effective market funnels.
% In this work, we present a semantic map-guided GUI synthesis method that optionally incorporates visual flow information based on the user's choice alongside language prompts. 
% Our approach uses Stable Diffusion, a large (billions) paired image-text pretrained diffusion model with a rich latent space that we steer toward realistic GUIs using an ensemble of ControlNets. 
% % TODO: Mention it in 1 sentence:
% A critical and general issue remains while using such deep generative priors: creating coherent, meaningful and correct aspect-ratio text. 
% We tackle this issue within our framework and additionally provide a semantic typography feature to create custom text-fonts and styles. 
% To evaluate our method, we demonstrate that our method produces more "desirable" UIs according to the well-known contrast, repetition, alignment, and proximity principles of design. 
% % We further validate our method through comprehensive automatic non-reference and human-preference aligned scores. (TODO: Maybe Unskip if we get UIClip from Jason!)
% % TODO: Re-word this and only keep ideation cycles and time-efficiency.
% Finally, a subjective evaluation study of XX non-expert and expert designers demonstrates the efficiency and fidelity of our method.
% % improves the time-efficiency by quick iterations of the UI design ideation process.
% %Finally, an evaluation with xx non-expert designers using our contributed <method-name> improves the time-efficiency by quick iterations of the UI design ideation cycle.

%\noindent \textcolor{blue}{----------------------------------------------------------------------}


%In an evaluation with xx designers, we found that GenerativeLayout: 1) enhances designers' exploration by expanding the coverage of the design space, 2) reduces the time required for exploration, and 3) maintains a perceived level of control similar to that of manual exploration.



% Present-day graphical user interfaces (GUIs) exhibit diverse arrangements of text, graphics, and interactive elements such as buttons and menus, but representations of GUIs have not kept up. They do not encapsulate both semantic and visuo-spatial relationships among elements. %\color{red} 
% To seize machine learning's potential for GUIs more efficiently, \papername~ exploits graph neural networks to capture individual elements' properties and their semantic—visuo-spatial constraints in a layout. The learned representation demonstrated its effectiveness in multiple tasks, especially generating designs in a challenging GUI autocompletion task, which involved predicting the positions of remaining unplaced elements in a partially completed GUI. The new model's suggestions showed alignment and visual appeal superior to the baseline method and received higher subjective ratings for preference. 
% Furthermore, we demonstrate the practical benefits and efficiency advantages designers perceive when utilizing our model as an autocompletion plug-in.


% Overall pipeline: Maybe drop semantic typography / visual flow?


%\received{20 February 2007}
%\received[revised]{12 March 2009}
%\received[accepted]{5 June 2009}

%%
%% This command processes the author and affiliation and title
%% information and builds the first part of the formatted document.
\maketitle


% humans are sensitive to the way information is presented.

% introduce framing as the way we address framing. say something about political views and how information is represented.

% in this paper we explore if models show similar sensitivity.

% why is it important/interesting.



% thought - it would be interesting to test it on real world data, but it would be hard to test humans because they come already biased about real world stuff, so we tested artificial.


% LLMs have recently been shown to mimic cognitive biases, typically associated with human behavior~\citep{ malberg2024comprehensive, itzhak-etal-2024-instructed}. This resemblance has significant implications for how we perceive these models and what we can expect from them in real-world interactions and decisionmaking~\citep{eigner2024determinants, echterhoff-etal-2024-cognitive}.

The \textit{framing effect} is a well-known cognitive phenomenon, where different presentations of the same underlying facts affect human perception towards them~\citep{tversky1981framing}.
For example, presenting an economic policy as only creating 50,000 new jobs, versus also reporting that it would cost 2B USD, can dramatically shift public opinion~\cite{sniderman2004structure}. 
%%%%%%%% 图1:  %%%%%%%%%%%%%%%%
\begin{figure}[t]
    \centering
    \includegraphics[width=\columnwidth]{Figs/01.pdf}
    \caption{Performance comparison (Top-1 Acc (\%)) under various open-vocabulary evaluation settings where the video learners except for CLIP are tuned on Kinetics-400~\cite{k400} with frozen text encoders. The satisfying in-context generalizability on UCF101~\cite{UCF101} (a) can be severely affected by static bias when evaluating on out-of-context SCUBA-UCF101~\cite{li2023mitigating} (b) by replacing the video background with other images.}
    \label{fig:teaser}
\end{figure}


Previous research has shown that LLMs exhibit various cognitive biases, including the framing effect~\cite{lore2024strategic,shaikh2024cbeval,malberg2024comprehensive,echterhoff-etal-2024-cognitive}. However, these either rely on synthetic datasets or evaluate LLMs on different data from what humans were tested on. In addition, comparisons between models and humans typically treat human performance as a baseline rather than comparing patterns in human behavior. 
% \gabis{looks good! what do we mean by ``most studies'' or ``rarely'' can we remove those? or we want to say that we don't know of previous work doing both at the same time?}\gili{yeah the main point is that some work has done each separated, but not all of it together. how about now?}

In this work, we evaluate LLMs on real-world data. Rather than measuring model performance in terms of accuracy, we analyze how closely their responses align with human annotations. Furthermore, while previous studies have examined the effect of framing on decision making, we extend this analysis to sentiment analysis, as sentiment perception plays a key explanatory role in decision-making \cite{lerner2015emotion}. 
%Based on this, we argue that examining sentiment shifts in response to reframing can provide deeper insights into the framing effect. \gabis{I don't understand this last claim. Maybe remove and just say we extend to sentiment analysis?}

% Understanding how LLMs respond to framing is crucial, as they are increasingly integrated into real-world applications~\citep{gan2024application, hurlin2024fairness}.
% In some applications, e.g., in virtual companions, framing can be harnessed to produce human-like behavior leading to better engagement.
% In contrast, in other applications, such as financial or legal advice, mitigating the effect of framing can lead to less biased decisions.
% In both cases, a better understanding of the framing effect on LLMs can help develop strategies to mitigate its negative impacts,
% while utilizing its positive aspects. \gabis{$\leftarrow$ reading this again, maybe this isn't the right place for this paragraph. Consider putting in the conclusion? I think that after we said that people have worked on it, we don't necessarily need this here and will shorten the long intro}


% If framing can influence their outputs, this could have significant societal effects,
% from spreading biases in automated decision-making~\citep{ghasemaghaei2024understanding} to reducing public trust in AI-generated content~\citep{afroogh2024trust}. 
% However, framing is not inherently negative -- understanding how it affects LLM outputs can offer valuable insights into both human and machine cognition.
% By systematically investigating the framing effect,


%It is therefore crucial to systematically investigate the framing effect, to better understand and mitigate its impact. \gabis{This paragraph is important - I think that right now it's saying that we don't want models to be influenced by framing (since we want to mitigate its impact, right?) When we talked I think we had a more nuanced position?}




To better understand the framing effect in LLMs in comparison to human behavior,
we introduce the \name{} dataset (Section~\ref{sec:data}), comprising 1,000 statements, constructed through a three-step process, as shown in Figure~\ref{fig:fig1}.
First, we collect a set of real-world statements that express a clear negative or positive sentiment (e.g., ``I won the highest prize'').
%as exemplified in Figure~\ref{fig:fig1} -- ``I won the highest prize'' positive base statement. (2) next,
Second, we \emph{reframe} the text by adding a prefix or suffix with an opposite sentiment (e.g., ``I won the highest prize, \emph{although I lost all my friends on the way}'').
Finally, we collect human annotations by asking different participants
if they consider the reframed statement to be overall positive or negative.
% \gabist{This allows us to quantify the extent of \textit{sentiment shifts}, which is defined as labeling the sentiment aligning with the opposite framing, rather then the base sentiment -- e.g., voting ``negative'' for the statement ``I won the highest prize, although I lost all my friends on the way'', as it aligns with the opposite framing sentiment.}
We choose to annotate Amazon reviews, where sentiment is more robust, compared to e.g., the news domain which introduces confounding variables such as prior political leaning~\cite{druckman2004political}.


%While the implications of framing on sensitive and controversial topics like politics or economics are highly relevant to real-world applications, testing these subjects in a controlled setting is challenging. Such topics can introduce confounding variables, as annotators might rely on their personal beliefs or emotions rather than focusing solely on the framing, particularly when the content is emotionally charged~\cite{druckman2004political}. To balance real-world relevance with experimental reliability, we chose to focus on statements derived from Amazon reviews. These are naturally occurring, sentiment-rich texts that are less likely to trigger strong preexisting biases or emotional reactions. For instance, a review like ``The book was engaging'' can be framed negatively without invoking specific cultural or political associations. 

 In Section~\ref{sec:results}, we evaluate eight state-of-the-art LLMs
 % including \gpt{}~\cite{openai2024gpt4osystemcard}, \llama{}~\cite{dubey2024llama}, \mistral{}~\cite{jiang2023mistral}, \mixtral{}~\cite{mistral2023mixtral}, and \gemma{}~\cite{team2024gemma}, 
on the \name{} dataset and compare them against human annotations. We find  that LLMs are influenced by framing, somewhat similar to human behavior. All models show a \emph{strong} correlation ($r>0.57$) with human behavior.
%All models show a correlation with human responses of more than $0.55$ in Pearson's $r$ \gabis{@Gili check how people report this?}.
Moreover, we find that both humans and LLMs are more influenced by positive reframing rather than negative reframing. We also find that larger models tend to be more correlated with human behavior. Interestingly, \gpt{} shows the lowest correlation with human behavior. This raises questions about how architectural or training differences might influence susceptibility to framing. 
%\gabis{this last finding about \gpt{} stands in opposition to the start of the statement, right? Even though it's probably one of the largest models, it doesn't correlate with humans? If so, better to state this explicitly}

This work contributes to understanding the parallels between LLM and human cognition, offering insights into how cognitive mechanisms such as the framing effect emerge in LLMs.\footnote{\name{} data available at \url{https://huggingface.co/datasets/gililior/WildFrame}\\Code: ~\url{https://github.com/SLAB-NLP/WildFrame-Eval}}

%\gabist{It also raises fundamental philosophical and practical questions -- should LLMs aim to emulate human-like behavior, even when such behavior is susceptible to harmful cognitive biases? or should they strive to deviate from human tendencies to avoid reproducing these pitfalls?}\gabis{$\leftarrow$ also following Itay's comment, maybe this is better in the dicsussion, since we don't address these questions in the paper.} %\gabis{This last statement brings the nuance back, so I think it contradicts the previous parapgraph where we talked about ``mitigating'' the effect of framing. Also, I think it would be nice to discuss this a bit more in depth, maybe in the discussion section.}






\section{Background on Causal Inference}
\label{sec:background-causal} 



 \newtextold{In this section, we 
 %formalize the notion of {\em Average Treatment Effect and understand the 
 review the basic concepts and key assumptions for inferring the effects of an intervention on the outcome on collected datasets without performing randomized controlled experiments. 
We use {\em Pearl's graphical causal model} for {\em observational causal analysis} \cite{pearl2009causal} to define these concepts.}


\par
\paratitle{Causal Inference and Causal DAGs} The primary goal of causal inference is to model causal dependencies between attributes and evaluate how changing one variable (referred to as intervention) would affect the other.
Pearl's Probabilistic Graphical Causal Model \cite{pearl2009causal} can be written as a tuple $(\exo, \edvar, Pr_{\exo}, \psi)$, where $\exo$ is a set of {\em exogenous} variables, $\Pr_{\exo}$ is the joint distribution of \exo, and $\edvar$ is a set of observed {\em endogenous variables}.
Here $\psi$ is a set of structural equations that encode dependencies among variables. The equation for $A \in \edvar$ takes the following form:
%that encode the dependencies among the variables.  These equations are of the form 
$$\psi_{A}: 
\dom(Pa_{\exo}(A)) {\times} \dom(Pa_{\edvar}(A)) \to \dom(A)$$
Here $Pa_{\exo}(A) {\subseteq} {\exo}$ and $Pa_{\edvar}(A) {\subseteq} \edvar \setminus \{A\}$ respectively denote the exogenous and endogenous parents of $A$. A causal relational model is associated with a directed acyclic graph ({\em causal DAG}) $G$, whose nodes are the endogenous variables $\edvar$ and there is a directed edge from $X$ to $O$ if  $X {\in} Pa_{\edvar}(O)$. The causal DAG obfuscates exogenous variables as they are unobserved. %Any given set of values for the exogenous variables completely determines the values of the endogenous variables by the structural equations (we do not need any known closed-form expressions of the structural equations in this work). 
The probability distribution $\Pr_{\exo}$ on exogenous variables $\exo$ induces a probability distribution  
on the endogenous variables $\edvar$ by the structural equations $\psi$.  A causal DAG can be constructed by a domain expert as in the above example, or using existing {\em causal discovery} algorithms~\cite{glymour2019review}. 



\begin{figure}
    \centering
    \small
    \begin{tikzpicture}[node distance=0.6cm and 1cm, every node/.style={minimum size=0.5cm}]
        \tikzset{vertex/.style = {draw, circle, align=center}}

        \node[vertex] (Ethnicity) {\bf\scriptsize{{Ethnicity}}};
        \node[vertex, right=0.3cm of Ethnicity] (Gender) {\bf{\scriptsize{Gender}}};
        \node[vertex, right=0.3cm of Gender] (Age) {\bf{\scriptsize{Age}}};
        \node[vertex, below=0.3cm of Gender] (Role) {\bf{\scriptsize{Role}}};
        \node[vertex, right=0.3cm of Role] (Education) {\bf{\small{\scriptsize{Education}}}};
        \node[vertex, below=0.3cm of Role] (Salary) {\bf{\scriptsize{Salary}}};

        \draw[->] (Ethnicity) -- (Salary);
        \draw[->] (Gender) -- (Role);
        \draw[->] (Age) -- (Role);
         \draw[->] (Education) -- (Role);
           \draw[->] (Education) -- (Salary);
             \draw[->] (Ethnicity) -- (Education);
                \draw[->] (Ethnicity) -- (Role);
             \draw[->] (Gender) -- (Education);
               \draw[->] (Age) -- (Education);
                 \draw[->] (Role) -- (Salary);
        \draw[->] (Gender) to[bend right] (Salary);
        \draw[->] (Age) -- (Salary);
    \end{tikzpicture}
    \caption{Partial causal DAG for the Stack Overflow dataset.}
    \label{fig:causal_DAG}
\end{figure}



 \begin{example}
Figure \ref{fig:causal_DAG} depicts a partial causal DAG for the SO dataset over the attributes in Table \ref{tab:data} as endogenous variables (we use a larger causal DAG with all 20 attributes in our experiments). 
  Given this causal DAG, we can observe that the role that a coder has in their company depends on their education, age gender and ethnicity.
\end{example}
\par


\par
\paratitle{Intervention} In Pearl's model, a treatment $T = t$ (on one or more variables) is considered as an {\em intervention} to a causal DAG by mechanically changing the DAG such that the values of node(s) of $T$ in $G$ are set to the value(s) in $t$, which is denoted by $\doop(T = t)$. Following this operation, the probability distribution of the nodes in the graph changes as the treatment nodes no longer depend on the values of their parents. Pearl's model gives an approach to estimate the new probability distribution by identifying the confounding factors $Z$ described earlier using conditions such as {\em d-separation} and {\em backdoor criteria} \cite{pearl2009causal}, which we do not discuss in this paper.


\par
\paratitle{Average Treatment Effect} The effects of an intervention are often measured by evaluating
% \par
% \paratitle{Causal inference, Treatment, ATE, and CATE}
% \newtextold{One of the primary goals  of {\em causal inference} is to estimate the effect of making a change in terms of a {\em treatment} $T$ (often referred to as an intervention)
% on the outcome $O$. 
% %A variable that is modified is often referred to as the treatment variable $T$ and the metric used to captures 
% The effect of treatment $T$ on outcome $O$ is measured by 
% %is known as 
{\em Conditional Average treatment effect (CATE)}, 
%a {\em treatment variable} $T$ on an outcome variable $O$ (e.g., what is the effect of higher \verb|Education| on \verb|Salary|). 
measuring the effect of an intervention on a subset of records~\cite{rubin1971use,holland1986statistics} by calculating the difference in average outcomes between the group that receives the treatment and the group that does not (called the {\em control} group), providing an estimate of how the intervention by $T$ influences an outcome $O$ for a given subpopulation. 
% Mathematically,
% \begin{equation}
%     %{\small ATE(T,O) = \mathbb{E}[O \mid \doop(T=1)] -      \mathbb{E}[O \mid \doop(T=0)]}
%     {\small ATE(T, O) = \mathbb{E}[O \mid \doop(T=1)] -  
%     \mathbb{E}[O \mid \doop(T=0)]}
% \label{eq:ate}
% \end{equation}
% In our work, where the treatment with maximum effect may vary among different subpopulations, we are interested in computing the \emph{Conditional Average Treatment Effect} (CATE), which measures the effect of a treatment on an outcome on \emph{a subset of input units}~\cite{rubin1971use,holland1986statistics}. 
Given a subset of the records defined by (a vector of) attributes $B$ and their values $b$, 
%g {\in} \Qagg(\db)$ defined by a predicate $G {=} g$ 
we can compute $CATE(T,O \mid B = b)$ as:
{
\begin{eqnarray}    
    %CATE(T,O \mid G=g) = \mathbb{E}[O \mid \doop(T=1)&, G=g] -  \mathbb{E}[O \mid \doop(T=0), G=g] 
   % CATE(T,O \mid B = b) = 
    \mathbb{E}[O \mid \doop(T=1), B = b] -  
    \mathbb{E}[O \mid \doop(T=0), B = b]\label{eq:cate}
\end{eqnarray}
}
Setting $B=\phi$ is equivalent to the ATE estimate.
The above definitions assumes that the treatment assigned to one unit does not affect the outcome of another unit (called the {Stable Unit Treatment Value Assumption (SUTVA)) \cite{rubin2005causal}}\footnote{This assumption does not hold for causal inference on multiple tables and even on a single table where tuples depend on each other.}. 


The ideal way of estimating the ATE and CATE is through {\em randomized controlled experiments}, 
where the population is randomly divided into two groups (treated and control, for binary treatments): 
%treated group that receives the treatment and control group that does not (denoted by 
%{the \em treated} group 
denoted by 
$\doop(T = 1)$ 
%for a binary treatment)  (the {\em control} group, 
and $\doop(T = 0)$ resp.)~\cite{pearl2009causal}.
%\sr{edited up to here, going to read the rest first, this section should not look like causumx}
%\par
%\par
However, randomized experiments cannot always be performed due to ethical or feasibility issues. In these scenarios, observational data is used to estimate the treatment effect, which requires the following additional assumptions. 
% {\em Observational Causal Analysis} still allows sound causal inference under additional assumptions. Randomization in controlled trials mitigates the effect of {\em confounding factors}, i.e., attributes that can affect the treatment assignment and outcome. Suppose we want to understand the causal effect of \verb|Education| on \verb|Salary| from the SO dataset.  %in Example~\ref{ex:running_example}. 
% We no longer apply Eq. (\ref{eq:ate}) since the values of \verb|Education| were not assigned at random in this data, and obtaining higher education largely depends on other attributes like \verb|Gender|, \verb|Age|, and \verb|Country|. 
% Pearl's model provides ways to account for these confounding attributes $Z$ to get an unbiased causal estimate from observational data under the following assumptions ($\independent$ denotes independence):
% \vspace{-2mm}
\newtextold{
The first assumption is called {\em unconfoundedness} or {\em strong ignorability}  \cite{rosenbaum1983central} says that the independence of outcome $O$ and treatment $T$ conditioning on a set of confounder variables  (covariates) $Z$, i.e.,
%\begin{eqnarray}
 $    O \independent T | Z {=} z$.
 %\label{eq:unconfoundedness}
%\end{eqnarray}
The second assumption called {\em overlap or positivity} says that there is a chance of observing individuals in both the treatment and control groups for every combination of covariate values, i.e., 
%\begin{eqnarray}
   $ 0 < Pr(T {=} 1 ~~|~~Z {=} z)< 1 $.
   %\label{eq:overlap}
%\end{eqnarray}
}
%\sg{Is this overlap or positivity? maybe both are the same?} \sr{yeah - same - from Google AI - The overlap assumption, also known as the positivity assumption, is a key assumption in causal inference that states that there is a chance of observing individuals in both the treatment and control groups for every combination of covariate values.}
% The above conditions are known as {\em Strong Ignorability} in Rubin's model \cite{rubin2005causal}.
The unconfoundedness assumption requires that the treatment $T$ and the outcome $O$ be independent when conditioned on a set of variables $Z$. In SO, assuming that only $Z$ =\{\verb|Gender|, \verb|Age|, \verb|Country|\} affects $T = $ \verb|Education|, if we condition on a fixed set of values of $Z$, i.e., consider people of a given gender, from a given country, and at a given age, then $T = $ \verb|Education| and $O = $ \verb|Salary| are independent. For such confounding factors $Z$,  Eq. (\ref{eq:cate}) reduces to the following form 
(positivity 
gives the feasibility of the expectation difference): 
 \vspace{-1mm}
{\small
\begin{flalign}    
% \begin{eqnarray}
   % % & ATE(T,O) = \mathbb{E}_Z \left[\mathbb{E}[O \mid T=1, Z = z] -  
   %  \mathbb{E}[O \mid T=0, Z = z] \right] \label{eq:conf-ate}\\
 & CATE(T,O {\mid} B {=} b) {=} \nonumber
    \mathbb{E}_Z \left[\mathbb{E}[O {\mid} T{=}1, B {=} b, Z {=} z] {-}  
    \mathbb{E}[O {\mid} T{=}0, B {=} b, Z {=} z]\right]\label{eq:conf-cate}
\end{flalign}
% \end{eqnarray}
}
% \vspace{-4mm}
This equation contains conditional probabilities and not $\doop(T = b)$, which can be estimated from an observed data. 
Pearl's model gives a systematic way to find such a $Z$ when a causal DAG is available. 






\section{Overview}
\label{sec:label}

\methodname guides an LLM through the process of developing a library of procedural functions that matches an input design intent.
In our problem framing, we assume that a user has a procedural modeling domain in mind (e.g., a particular category of shapes).
The user will communicate their design intent to our system, which is then tasked with producing a fully realized library of abstraction functions that meet our desiderata: (a) they should generalize, (b) they should be interpretable, and (c) they should produce plausible outputs. 

Our system receives a number of benefits from the prior knowledge encoded in LLMs.
Since LLMs have been trained extensively on human-written code, they are able to author functions with meaningful names and parameters.
This exposes an interface that a person can easily work with and understand.
However, we also find that LLMs are prone to hallucinate, generating mismatches from `real' distributions of shapes (e.g.,  collections of 3D assets).

To overcome this issue, we guide and ground the LLM outputs under the supervision of the user provided design intent, consisting of a textual description and a set of seed shapes. 
Textual descriptions of desired function properties help constrain the interface design, prompting the semantic prior of the LLM to attune towards a particular modeling task.
Each seed set we consider is composed of twenty 3D shapes with part-level semantic segmentations and textured renders.
Our system validates the plausibility of its productions by searching for function implementations and applications that can explain sub-structures in these exemplars.

In the following, we explain how~\methodname~solves this problem.
In Section~\ref{sec:lib_design}, we describe how we convert design intent into a fully realized library of abstraction functions.
In Section~\ref{sec:lib_usage}, we describe how we can expand the usage of this library beyond the seed set by training a recognition network on synthetic data. 

\section{Resource Aware Borrow Calculus}
\label{sec:calculus}

This section introduces Resource-Aware Borrow Calculus (RABC) and resource-aware dynamic semantics.
%
RABC is a resource-aware variant of Low-Level Borrow Calculus (LLBC)~\cite{Aeneas}, which is based on Rust's MIR but keeps high-level information such as structured control flow and a structured memory model.
%
RABC includes essential features such as mutation, borrow mechanisms, integer lists with explicit boxing, and recursive top-level functions.\footnote{We include only integer lists as heap-allocated data structures and exclude loops in RABC for ease of presentation. Our implementation supports user-defined inductive data types using structs and enums, as well as \texttt{while true} loops with break and continue.}
%
RABC also includes $\kwd{tick}(\cdot)$ statements to annotate resource consumption.
%
\cref{sec:syntax} presents the syntax and discusses properties of a well-borrowed RABC program, guaranteed by Rust's borrow checker.
%
\cref{sec:semantics} formalizes the resource-aware dynamic semantics of RABC, which captures the resource consumption during the execution of an RABC program.

\subsection{Syntax}
\label{sec:syntax}

\cref{fig:syntax} summarizes the syntax of RABC. For the convenience of formalization, we distinguish between expressions and statements.
%
We then describe each syntactical construction separately. 

\begin{figure}[t]
\small
    \begin{align*}
    \textbf{Type}~ t &::= \\
        \tag{atom} &|~ \kwd{i32} ~|~ \kwd{bool} \\
        \tag{list} &|~ \kwd{list} ~|~ \kwd{box}~\kwd{list}\\
        \tag{borrow} &|~ \&^\kwd{s}~t ~|~ \&^\kwd{m}~t \\
    \textbf{Place}~ p &::= \\
        \tag{variable} &|~ x \\
        \tag{dereference} &|~ * p \\
    \textbf{Expression}~ e &::= \\
        &|~ \kwd{n}_\text{i32} ~|~ \kwd{true} ~|~ \kwd{false} ~|~ \kwd{nil} ~|~ \kwd{box}(e) \\
        \tag{integer} &|~ e_1 ~\kwd{op}~ e_2 \\
        \tag{scalar copy} &|~ \kwd{copy}~ p \\
        \tag{borrow} &|~ \&^\kwd{s}~ p ~|~ \&^\kwd{m}~ p \\
        \tag{move ownership} &|~ \kwd{move}~ p \\
    \textbf{Statement}~ s &::= \\
        \tag{resource cost} &|~ \kwd{tick}(\delta) \\
        \tag{return} &|~ \kwd{return} \\
        \tag{sequence} &|~ s_1; s_2 \\
        \tag{drop} &|~ \kwd{drop}~ p\\
        \tag{if bool} &|~ \kwd{if}~ p ~\kwd{then}~ s_1 ~\kwd{else}~ s_2 ~\kwd{end}\\
        \tag{match list} &|~ \kwd{match}~ p ~ \{\kwd{nil}\mapsto s_1, \kwd{cons}(x_\text{hd}, x_\text{tl})\mapsto s_2\} \\
        \tag{assignment} &|~ p \from e \\
        \tag{list constructor} &|~ p \from \kwd{cons}(e_1, e_2) \\
        \tag{function call} &|~ p \from f(\vec{e}) \\
    \textbf{Toplevel}~ tl &::= \\
        \tag{sequence} &|~ tl_1 ~ tl_2 \\
        \tag{function} &|~ \kwd{fn}~ f ~(\vec{x}_\text{param}:\vec{t}_\text{param}, \vec{x}_\text{local}:\vec{t}_\text{local}, x_\text{ret}:t_\text{ret}) \{~ s ~\}
    \end{align*}
    \caption{Syntax}
    \label{fig:syntax}
\end{figure}


\textbf{Types} are simple, without resource annotations; we will present the types with annotations in \cref{sec:inference}. Integer $\kwd{i32}$ and Boolean $\kwd{bool}$ are atom types. Lists $\kwd{list}$ and the box type of lists $\kwd{box}~\kwd{list}$ are types for functional lists defined in Rust. The box type is required for a list's tail, which is usually heap-allocated. The reference type is for borrows with different modes: $\&^\kwd{s}~ t$ is for the shared borrow, and $\&^\kwd{m}~ t$ is for the mutable borrow. These borrow modes are notions from Rust explained as follows: mutation is forbidden on shared borrows and only allowed on mutable borrows.

\textbf{Places} are memory locations storing values, including program variables $x, y, \ldots$ and dereferences $* p$ of borrows or boxes stored in $p$. We will soon show their role in the dynamic semantics in \cref{sec:semantics}. 

\textbf{Expressions} represent resource-free evaluation. Integer literals $\kwd{n}_\text{i32}$ and Boolean literals $\kwd{true}$, $\kwd{false}$ are atom values. The $\kwd{nil}$ constructor stands for empty lists. The boxing expression $\kwd{box}(e)$ allocates memory in a heap to store the value of $e$, resembling \lstinline|Box::new(e)| of Rust. Arithmetic expressions $e_1 ~\kwd{op}~ e_2$ operate on integer-valued operands with the binary operator $\kwd{op}$. For an atom value stored in a place $p$, we use $\kwd{copy}~p$ to make a copy of it.
% The scalar copies are for the smaller data like integers and Boolean values, while the borrows are usually used for those larger data structures like lists.
Given a place $p$, we can borrow from it with different modes: $\&^\kwd{s}~ p$ creates a shared borrow, and $\&^\kwd{m}~ p$ creates a mutable borrow.
%
For a borrow stored in a place $p$, we can use $\kwd{move}~ p$ to move ownership out from the original place $p$. 

\textbf{Statements} represent resource-aware evaluation. The statement $\kwd{tick}(\delta)$ with $\delta \in \ZZ$ is the explicit annotation for consuming $\delta$ units of resource.
%
Statements $\kwd{return}$ and $s_1; s_2$ usually form a function body such as $s_1; s_2; \ldots, s_n; \kwd{return}$. In RABC, we introduce $\kwd{drop}~ p$ to drop the borrow stored in $p$ explicitly. The conditional and pattern-match statements perform case analysis on Boolean values and lists, respectively. Note that we use place $p$ instead of the expression $e$ to indicate Boolean values and lists because we only need to peak the value instead of copying Boolean values or moving ownership of lists. The assignment has three variants: one for assigning atom values and borrows, one for constructing lists, and another for function applications; the latter two variants do not reside in expressions because they need to be resource-aware as they incur resource consumption.

\textbf{Toplevels} define a sequence of (possibly recursive) top-level functions like $\kwd{fn} ~f_1, \ldots, \kwd{fn}~ f_n$. Each function contains one statement as its body and variables with corresponding type declarations, including function parameters $\vec{x}_\text{param}:\vec{t}_\text{param}$, local variables $\vec{x}_\text{local}:\vec{t}_\text{local}$ used in the function body, and a distinguished variable $x_\text{ret} : t_\text{ret}$ for the returned value. The $\vec{\bullet}$ notation represents vectors.

\subsection{Resource Aware Dynamic Semantics}
\label{sec:semantics}

\begin{figure}[t]
\small
    \begin{align*}
    \tag{undefined} \textbf{Value}~ v &::= \bot \\
    \tag{atoms} &|~ \kwd{n}_\text{i32} ~|~ \kwd{true} ~|~ \kwd{false} \\
    \tag{list} &|~  lv \\
    \tag{box} &|~ \kwd{box}(lv) \\
    \tag{borrow} &|~ \&(p, v)\\
    \textbf{List Value}~ lv &::= \kwd{nil} ~|~ \kwd{cons}(\kwd{n}_\text{i32}, \kwd{box}(lv))
    \end{align*}
    \caption{Value}
    \label{fig:dyn-value}
\end{figure}

\begin{figure}[t]
\small
    \judgement{Store Reading}{$V\vDash p \rightsquigarrow v$}
    \begin{mathpar}
    \inferrule*[Right=\rulename{V-Rd-Var}]
    {V(x)=v}
    {V\vDash x \rightsquigarrow v}
    \and
    \inferrule*[Right=\rulename{V-Rd-Box}]
    {V\vDash p \rightsquigarrow \kwd{box}(v)}
    {V\vDash * p\rightsquigarrow v}
    \and
    \inferrule*[Right=\rulename{V-Rd-Borrow}]
    {V\vDash p \rightsquigarrow \&(\_, v)}
    {V\vDash *p \rightsquigarrow v}
    \end{mathpar}

    \judgement{Store Writing}{$\VWt{V}{p}{v}{V'}$}
    \begin{mathpar}
    \inferrule[V-Wt-Var]
    {\forall y\not=x, V'(y)=V(y) 
    \\ V'(x) = v}
    {\VWt{V}{x}{v}{V'}}
    \and
    \inferrule[V-Wt-Box]
    {V\vDash p\rightsquigarrow \kwd{box}(\_)
    \\ \VWt{V}{p}{\kwd{box}(v)}{V'}}
    {\VWt{V}{* p}{v}{V'}}
    \and
    \inferrule*[Right=\rulename{V-Wt-Borrow}]
    {V\vDash p\rightsquigarrow \&(q, \_)
    \\ \VWt{V}{q}{v}{V'}
    \\ \VWt{V'}{p}{\&(q, v)}{V''}}
    {\VWt{V}{*p}{v}{V''}}
    \end{mathpar}

    \caption{Store Reading and Writing}
    \label{fig:dyn-rw}
\end{figure}

\begin{figure}[t]
\small
\judgement{Expression Evaluation (Selected)}{$V\vDash e \rightsquigarrow v$}
    \begin{mathpar}
    \inferrule*[Right=\rulename{V-Ev-Borrow}]
    {V\vDash p \rightsquigarrow v}
    {V\vDash \&^{\kwd{s}/\kwd{m}} p \rightsquigarrow \&(p, v)}
    \end{mathpar}

\judgement{Statement Execution (Selected)}{$V\vDash e \rightsquigarrow^\delta \Dashv V'$}
    \begin{mathpar}
    \inferrule*[Right=\rulename{V-Ex-Tick}]
    {~}
    {V\vDash \kwd{tick}(\delta)\rightsquigarrow^\delta \Dashv V}
    \and
    \inferrule*[Right=\rulename{V-Ex-Drop}]
    {~}
    {V\vDash \kwd{drop}~p\rightsquigarrow^0\Dashv V}
    \\
    \inferrule*[Right=\rulename{V-Ex-Cons}]
    {V\vDash e_1 \rightsquigarrow v_1
    \\ V\vDash e_2 \rightsquigarrow v_2
    \\ \VWt{V}{p}{\kwd{cons}(v_1, v_2)}{V'} }
    {V\vDash p\from \kwd{cons}(e_1, e_2)\rightsquigarrow^0 \Dashv V'}
    \\
    \inferrule*[Right=\rulename{V-Ex-IfT}]
    {V\vDash p\rightsquigarrow \kwd{true}
    \\ V\vDash s_1\rightsquigarrow^\delta \Dashv V'}
    {V\vDash \kwd{if}~ p ~\kwd{then}~ s_1 ~\kwd{else}~ s_2 ~\kwd{end} \rightsquigarrow^\delta \Dashv V'}
    \and
    \inferrule*[Right=\rulename{V-Ex-IfF}]
    {V\vDash p\rightsquigarrow \kwd{false}
    \\ V\vDash s_2\rightsquigarrow^\delta \Dashv V'}
    {V\vDash \kwd{if}~ p ~\kwd{then}~ s_1 ~\kwd{else}~ s_2 ~\kwd{end} \rightsquigarrow^\delta \Dashv V'}
    \\
    \inferrule*[Right=\rulename{V-Ex-Mat-Nil}]
    {V\vDash p\rightsquigarrow \kwd{nil}
    \\ V\vDash s_1\rightsquigarrow^\delta \Dashv V'}
    {V\vDash \kwd{match}~ p ~ \{\kwd{nil}\mapsto s_1, \kwd{cons}(x_\text{hd}, x_\text{tl})\mapsto s_2\} \rightsquigarrow^\delta \Dashv V'}

    \inferrule*[Right=\rulename{V-Ex-Mat-Cons}]
    {V\vDash p\rightsquigarrow \kwd{cons}(hd, tl)
    \\ \VWt{V}{p}{\bot}{V_1}
    \\ \VWt{V_1}{x_\text{hd}}{hd}{V_2}
    \\ \VWt{V_2}{x_\text{tl}}{tl}{V_\text{b}}
    \\\\ V_\text{b}\vDash s_2\rightsquigarrow^\delta \Dashv V'_\text{b}
    \\ V'_\text{b}\vDash x_\text{hd}\rightsquigarrow hd'
    \\ V'_\text{b}\vDash x_\text{tl}\rightsquigarrow tl'
    \\\\ \VWt{V'_\text{b}}{x_\text{hd}}{\bot}{V'_1}
    \\ \VWt{V'_1}{x_\text{tl}}{\bot}{V'_2}
    \\ \VWt{V'_2}{p}{\kwd{cons}(hd', tl')}{V'} 
    }
    {V\vDash \kwd{match}~ p ~ \{\kwd{nil}\mapsto s_1, \kwd{cons}(x_\text{hd}, x_\text{tl})\mapsto s_2\} \rightsquigarrow^\delta \Dashv V'}
    \\
    
    \inferrule*[Right=\rulename{V-Ex-App}]
    {\kwd{fn}~ f ~(\vec{x}_\text{param}:\vec{t}_\text{param}, \vec{x}_\text{local}:\vec{t}_\text{local}, x_\text{ret}:t_\text{ret}) \{~ s ~\}
    \\ V\vDash \vec{e}\rightsquigarrow \vec{v}
    \\\\ \VWt{V}{\vec{x}_\text{param}}{\vec{v}}{V_1}
    \\ \VWt{V_1}{\vec{x}_\text{local}}{\bot}{V_2}
    \\ \VWt{V_2}{x_\text{ret}}{\bot}{V_\text{b}}
    \\\\ V_\text{b}\vDash s\rightsquigarrow^\delta \Dashv V'_\text{b}
    \\ V'_\text{b}\vDash x_\text{ret} \rightsquigarrow v_\text{ret}
    \\\\ \VWt{V'_\text{b}}{\vec{x}_\text{param}}{\bot}{V'_1}
    \\ \VWt{V'_1}{\vec{x}_\text{local}}{\bot}{V'_2}
    \\ \VWt{V'_2}{x_\text{ret}}{\bot}{V'_3}
    \\ \VWt{V'_3}{p}{v_\text{ret}}{V'}
    } 
    {V\vDash p\from f(\vec{e})\rightsquigarrow^\delta \Dashv V'}
    \end{mathpar}
    \caption{Resource Aware Dynamic Semantics}
    \label{fig:dyn-eval-exec}
\end{figure}


\cref{fig:dyn-value}, \cref{fig:dyn-rw}, and \cref{fig:dyn-eval-exec} define a  resource-aware big-step dynamic semantics for RABC.
%
\cref{fig:dyn-value} defines values of RABC, including atom values, list values, box values, borrow values, and a distinguished undefined value $\bot$. 
%
Note that borrow values take the form $\&(p,v)$, denoting a value $v$ borrowed from a place $p$, but do not record the borrow mode ($\&^\kwd{s}$ or $\&^\kwd{m}$).
%
The design is reasonable because we work on well-borrowed programs; thus, we do not need to track the borrow modes during runtime.

A \emph{store} is a mapping $V : \mathbf{Variable}\to\mathbf{Value}$, where unused variables can be mapped to $\bot$.
%
\cref{fig:dyn-rw} formalizes reading from and writing to a store.
%
Judgement $V \vDash p \rightsquigarrow v$ means that under a store $V$, the place $p$ records a value $v$.
%
Judgement $\VWt{V}{p}{v}{V'}$ means that starting from a store $V$, writing a value $v$ to the place $p$ yields a new store $V'$.
%
Note that the rule \rulename{V-Wt-Borrow} may not terminate in general for heap-manipulating languages like C.
%
In our setting, we exploit Rust's borrow mechanisms that ensure that one cannot construct cyclic reference relations using borrows.

% We can read or write on a variable $x$. Due to $\mathbf{Place}$ syntax, we extend it to judgement $V\vDash p \rightsquigarrow v$ and $\VWt{V}{p}{v}{V'}$, the former reading and the latter writing, as \cref{fig:dyn-rw}. All rules are nearly trivial except that rule \rulename{V-Wt-Borrow} immediately updates value in original place $q$, as $\VWt{V}{q}{v}{V'}$. The termination and safety of rule \rulename{V-Wt-Borrow} is guaranteed by the Rust borrow checker again.

\cref{fig:dyn-eval-exec} presents selected evaluation rules for expressions and statements.
%
Judgement $V\vDash e \rightsquigarrow v$ indicates that under a store $V$, the expression $e$ evaluates to the value $v$. Recall that expressions denote resource-free computation, so we do not record resource information.
%
% The rule \rulename{V-Ev-Copy} is restricted to scalar values, and the rule \rulename{V-Ev-Move} is restricted to borrows.
%
The rule \rulename{V-Ev-Borrow} reflects the design that the runtime does not need to track borrow modes for well-borrowed programs.
%
Judgement $V \vDash s \rightsquigarrow^\delta \Dashv V'$ means that starting from a store $V$, the statement $s$ executes with $\delta$ units of resource consumption and ends in the store $V'$.
%
The rule \rulename{V-Ex-Tick} introduces $\delta$ unit of resource consumption; this is the only rule to incur actual resource uses.
%
The rule \rulename{V-Ex-Drop} does nothing, i.e., it does need to put the value back to the borrowed place, because we immediately update values when writing through borrows, as indicated by the rule \rulename{V-Wt-Borrow} in \cref{fig:dyn-rw}.
%
Also, because of such immediate updates, it is necessary to make sure that original places and variables should be passed to function applications; the subscript $\textbf{b}$ of the store $V_\text{b}$ stands for \textbf{b}inding in the rule \rulename{V-Ex-App}.
%
% Though global uniqueness of variables is a requirement in dynamic semantics, this feature does not add complexity to the type system, which will be elaborated in the subsequent section. This is because dynamic semantics serves merely as a resource-aware semantic instrument in the formalization to prove the soundness of our type system; it is not executed in practice.



\section{Resource Aware Type System and Inference} \label{sec:inference}

In this section, we present the resource-aware type system based on RABC introduced in \cref{sec:calculus} and a type-inference algorithm based on the AARA methodology.
%
\cref{sec:inference:types} introduces resource-enriched types, which augment the types of RABC with resource annotations.
%
% Rich type $\tau$ is plain type $t$ enriched with potential annotation $\alpha$, e.g. $\kwd{list}(\alpha)$ for $\kwd{list}$. Then, we introduce the typing context and its read/write operation used for type checking, followed by the definition of function signatures. 
%
\cref{sec:inference:subtyping} formulates a subtyping relation among resource-enriched types and uses the relation to construct a lattice of types sketched in \cref{sec:overview:Lattice}.
%
% With subtyping, we can tell what rich types are well formed. The subtyping relation over rich types actually forms a lattice with meet and join operations. We define the lattice operations and extend it to the typing context.
%
\cref{sec:inference:eval} and \cref{sec:inference:exec} present the resource-aware typing rules for expressions and statements, respectively.
%
\cref{sec:inference:infer} discusses a type-inference algorithm for the resource-aware type system.

\subsection{Rich Types, Contexts, and Signatures} \label{sec:inference:types}
\begin{figure}[t]
\small
    \begin{align*}
    \tag{undefined} \textbf{RichType}~ \tau &::= \bot \\
    \tag{atom types} &|~ \kwd{i32} ~|~ \kwd{bool} \\
    \tag{list} &|~ \kwd{list}(\alpha)\\
    \tag{box} &|~ \kwd{box}(\kwd{list}(\alpha)) \\
    \tag{shared borrow} &|~ \&^\kwd{s}(\tau) \\
    \tag{mutable borrow} &|~ \&^\kwd{m}(\tau_\text{c}, \tau_\text{p})
    \end{align*}
    \caption{Rich Types}
    \label{fig:rich-type}
\end{figure}
\begin{figure}[t]
\small
    \judgement{Enrich (Selected)}{$\textit{enrich}~ t ~\textit{as}~ \tau$}
    \begin{mathpar}
    \inferrule[Enrich-List]
    {\alpha~\text{fresh}}
    {\textit{enrich}~ \kwd{list} ~\textit{as}~  \kwd{list}(\alpha)}
    \and
    \inferrule[Enrich-Shared]
    {\textit{enrich}~ t ~\textit{as}~ \tau}
    {\textit{enrich}~ \&^\kwd{s}(t) ~\textit{as}~ \&^\kwd{s}(\tau)}
    \and
    \inferrule[Enrich-Mutable]
    {\textit{enrich}~ t ~\textit{as}~ \tau_\text{c}
    \\ \textit{enrich}~ t ~\textit{as}~ \tau_\text{p}
    }
    {\textit{enrich}~ \&^\kwd{m}(t) ~\textit{as}~ \&^\kwd{m}(\tau_\text{c}, \tau_\text{p})}
    \end{mathpar}
    \caption{Enrichment}
    \label{fig:enrich}
\end{figure}

\begin{figure}[t]
\small
    \judgement{Context Reading}{$\Gamma\vdash p \hookrightarrow \tau$}
    \begin{mathpar}
    \inferrule[$\Gamma$-Rd-Var]
    {\Gamma(x)=\tau}
    {\Gamma\vdash x \hookrightarrow \tau}
    \and
    \inferrule[$\Gamma$-Rd-Box]
    {\Gamma\vdash p \hookrightarrow \kwd{box}(\tau)}
    {\Gamma\vdash * p\hookrightarrow \tau}
    \and
    \inferrule[$\Gamma$-Rd-Shared]
    {\Gamma\vdash p \hookrightarrow \&^\kwd{s}(\tau)}
    {\Gamma\vdash *p \hookrightarrow \tau}
    \and
    \inferrule[$\Gamma$-Rd-Mutable]
    {\Gamma\vdash p \hookrightarrow \&^\kwd{m}(\tau_\text{c},\tau_\text{p})}
    {\Gamma\vdash * p\hookrightarrow \tau_\text{c}}    
    \end{mathpar}
    
    \judgement{Context Writing}{$\GWt{\Gamma}{p}{\tau}{\Gamma'}$}
    \begin{mathpar}
    \inferrule[$\Gamma$-Wt-Var]
    {\forall y\not=x, \Gamma'(y)=\Gamma(y) 
    \\ \Gamma'(x) = \tau}
    {\GWt{\Gamma}{x}{\tau}{\Gamma'}}
    \and
    \inferrule[$\Gamma$-Wt-Box]
    {\Gamma\vdash p\hookrightarrow \kwd{box}(\_)
    \\ \GWt{\Gamma}{p}{\kwd{box}(\kwd{list}(\alpha))}{\Gamma'}
    }
    {\GWt{\Gamma}{*p}{\kwd{list}(\alpha)}{\Gamma'}}
    \\
    \inferrule*[Right=\rulename{$\Gamma$-Wt-Shared}]
    {\Gamma\vdash p\hookrightarrow \&^\kwd{s}(\_)
    \\ \GWt{\Gamma}{p}{\&^\kwd{s}(\tau)}{\Gamma'}
    }
    {\GWt{\Gamma}{*p}{\tau}{\Gamma'}}
    \\
    \inferrule*[Right=\rulename{$\Gamma$-Wt-Mutable}]
    {\Gamma\vdash p\hookrightarrow \&^\kwd{m}(\tau_\text{c}, \tau_\text{p})
    \\ \vdash \tau_\text{c}
    \\ \GWt{\Gamma}{p}{\&^\kwd{m}(\tau, \tau_\text{p})}{\Gamma'}
    }
    {\GWt{\Gamma}{*p}{\tau}{\Gamma'}}
    \end{mathpar}
    \caption{Context Reading and Writing}
    \label{fig:sta-rw}
\end{figure}

\begin{figure}[t]
\small
    \judgement{Signatures}{$\vdash f \Rightarrow (\Gamma_f, \delta_f)$}
    \begin{mathpar}
    \inferrule
    {\text{fn}~ f ~(\vec{x}_\text{param}:\vec{t}_\text{param}, \vec{x}_\text{local}:\vec{t}_\text{local}, x_\text{ret}:t_\text{ret}) \{~ s ~\}
    \\\\ \textit{enrich}~ \vec{t}_\text{param} ~\textit{as}~ \vec{\tau}_\text{param}
    \\ \textit{enrich}~ \vec{t}_\text{local} ~\textit{as}~ \vec{\tau}_\text{local}
    \\ \textit{enrich}~ t_\text{ret} ~\textit{as}~ \tau_\text{ret}
    \\\\ \GWt{\emptyset}{\vec{x}_\text{param}}{\vec{\tau}_\text{param}}{\Gamma_1} 
    \\ \GWt{\Gamma_1}{\vec{x}_\text{local}}{\vec{\tau}_\text{local}}{\Gamma_2}
    \\ \GWt{\Gamma_2}{x_\text{ret}}{\tau_\text{ret}}{\Gamma_f}
    \\ \delta_f ~\text{fresh} }
    {\vdash f \Rightarrow (\Gamma_f, \delta_f)}
    \end{mathpar}
    \caption{Function Signatures}
    \label{fig:fun-sig}
\end{figure}

\textbf{Rich types} are types enriched with potential annotation $\alpha$ as in \cref{fig:rich-type} and \cref{fig:enrich}. 
%
The rich type $\bot$ denotes zero potential as the minimum among all rich types. 
%
The rich type $\kwd{list}(\alpha)$, represents the potential function $\alpha\cdot n$ for list $l$ with length $n$.
%
In shared borrows $\&^\kwd{s}(\tau)$, $\tau$ represents the potential function of borrowed value. 
% 
Mutable borrows $\&^\kwd{m}(\tau_\text{c}, \tau_\text{p})$ contains 2 components. $\tau_\text{c}$ is the \textbf{c}urrent type, which denotes the current potential of mutable borrow. $\tau_\text{p}$ is the \textbf{p}rophecy type, which denotes the prophecy potential when the mutable borrow ends.  

Typing \textbf{context} $\Gamma : \mathbf{Variable}\to\mathbf{RichType}$ is a partial map, where unused variables can be mapped to $\bot$. Similarly, in \cref{fig:sta-rw}, we extend the reading and writing operation on typing context from variable $x$ to place $p$. It is worth noting that rules \rulename{$\Gamma$-Rd-Mutable} and \rulename{$\Gamma$-Wt-Mutable} indicate to read and write the mutable borrow on its current component $\tau_\text{c}$. We explicitly point out that $\vdash \tau_\text{c}$ in the premise of rule \rulename{$\Gamma$-Wt-Mutable} is \textbf{dropping condition} for soundness, detailed in \cref{sec:inference:subtyping}. Because when we update $\tau_\text{c}$, the old $\tau_\text{c}$ should be restored if it is a mutable borrow.

\textbf{Signature} $(\Sigma_f, \delta_f)$ of a function $f$ compose a typing context $\Sigma_f$ and a resource unknown variable $\delta_f \in \ZZ$. As shown in \cref{fig:fun-sig}, context $\Gamma_f$ contains rich types for parameters, local variables, and the return variable. $\delta_f$ indicates the resource consumption irrelevant to parameters.

\subsection{Subtyping, Well-formedness, and Merging} \label{sec:inference:subtyping}
\begin{figure}[t]
\small
    \judgement{Subtyping}{$\tau_1 \preceq \tau_2$}
    \begin{mathpar}
    \inferrule*[Right=\rulename{S-Bot}]
    {~}
    {\bot\preceq\tau}
    \and
    \inferrule*[Right=\rulename{S-Int}]
    {~}
    {\kwd{i32}\preceq\kwd{i32}}
    \and
    \inferrule*[Right=\rulename{S-Bool}]
    {~}
    {\kwd{bool}\preceq\kwd{bool}}
    \and
    \inferrule*[Right=\rulename{S-List}]
    {\alpha_1 \leq \alpha_2}
    {\kwd{list}(\alpha_1)\preceq\kwd{list}(\alpha_2)}
    \\
    \inferrule[S-Box]
    {\alpha_1 \leq \alpha_2}
    {\kwd{box}(\kwd{list}(\alpha_1))\preceq\kwd{box}(\kwd{list}(\alpha_2))}
    \and
    \inferrule[S-Shared]
    {\tau_1 \preceq \tau_2}
    {\&^\kwd{s}(\tau_1)\preceq\&^\kwd{s}(\tau_2)}
    \and
    \inferrule[S-Mutable]
    {\tau_{\text{c}, 1}\preceq \tau_{\text{c}, 2}
    \\ \tau_{\text{p}, 2}\preceq \tau_{\text{p}, 1} }
    {\&^\kwd{m}(\tau_{\text{c}, 1}, \tau_{\text{p}, 1})\preceq\&^\kwd{m}(\tau_{\text{c}, 2}, \tau_{\text{p}, 2})}
    \end{mathpar}
    \caption{Rich Subtyping}
    \label{fig:rich-subtyping}
\end{figure}
\begin{figure}[t]
\small
    \judgement{Well-formedness}{$\vdash \tau$}
    \begin{mathpar}

    \inferrule[WF-Bot]
    {~}
    {\vdash \bot}
    \and
    \inferrule[WF-Int]
    {~}
    {\vdash \kwd{i32}}
    \and
    \inferrule[WF-Bool]
    {~}
    {\vdash \kwd{bool}}
    \and
    \inferrule[WF-List]
    {\alpha \geq 0}
    {\vdash \kwd{list}(\alpha)}
    \and
    \inferrule[WF-Box]
    {\alpha \geq 0}
    {\vdash \kwd{box}(\kwd{list}(\alpha))}
    \\
    \inferrule*[Right=\rulename{WF-Shared}]
    {\vdash \tau
    }
    {\vdash \&^\kwd{s}(\tau)}
    \and
    \inferrule*[Right=\rulename{WF-Mutable}]
    {\tau_\text{p} \preceq \tau_\text{c}
    \\ \vdash \tau_\text{c}
    \\ \vdash \tau_\text{p}
    }
    {\vdash \&^\kwd{m}(\tau_\text{c}, \tau_\text{p})}
    \end{mathpar}
    \caption{Well-formedness}
    \label{fig:rich-type-wf}
\end{figure}
\begin{figure}[t]
\small
    \judgement{Context Merging}{$\Gamma_1 \sqcap \Gamma_2 = \{ x \hookrightarrow \Gamma_1(x)\cap\Gamma_2(x) : x \in \mathbf{dom}(\Gamma_1)=\mathbf{dom}(\Gamma_2)\}$}
    \judgement{Meet/Join (Selected)}{$\tau_1\cap\tau_2 / \tau_1\cup\tau_2$}
    \begin{mathpar}        
    \inferrule*[Right=Meet-List]
    {\min(\alpha_1, \alpha_2)=\alpha}
    {\kwd{list}(\alpha_1)\cap\kwd{list}(\alpha_2)=\kwd{list}(\alpha)}
    \and
    \inferrule*[Right=Join-List]
    {\max(\alpha_1, \alpha_2)=\alpha}
    {\kwd{list}(\alpha_1)\cup\kwd{list}(\alpha_2)=\kwd{list}(\alpha)}
    \\
    
    \inferrule*[Right=Meet-Shared]
    {\tau_1 \cap \tau_2=\tau}
    {\&^\kwd{s}(\tau_1)\cap\&^\kwd{s}(\tau_2)=\&^\kwd{s}(\tau)}
    \and
    \inferrule*[Right=Join-Shared]
    {\tau_1 \cup \tau_2=\tau}
    {\&^\kwd{s}(\tau_1)\cup\&^\kwd{s}(\tau_2)=\&^\kwd{s}(\tau)}
    \\

    \inferrule*[Right=Meet-Mutable]
    {\tau_{\text{c}, 1} \cap \tau_{\text{c}, 2}=\tau_\text{c}
    \\ \tau_{\text{p}, 1} \cup \tau_{\text{p}, 2}=\tau_\text{p}
    \\ \tau_{\text{p}, 1} \preceq \tau_{\text{c}, 1}
    \\ \tau_{\text{p}, 2} \preceq \tau_{\text{c}, 2}
    }
    {\&^\kwd{m}(\tau_{\text{c}, 1}, \tau_{\text{p}, 1})\cap\&^\kwd{m}(\tau_{\text{c}, 2}, \tau_{\text{p}, 2})=\&^\kwd{m}(\tau_\text{c}, \tau_\text{p})}
    \\
    \inferrule*[Right=Join-Mutable]
    {\tau_{\text{c}, 1} \cup \tau_{\text{c}, 2}=\tau_\text{c}
    \\ \tau_{\text{p}, 1} \cap \tau_{\text{p}, 2}=\tau_\text{p}
    \\ \tau_{\text{p}, 1} \preceq \tau_{\text{c}, 1}
    \\ \tau_{\text{p}, 2} \preceq \tau_{\text{c}, 2}
    }
    {\&^\kwd{m}(\tau_{\text{c}, 1}, \tau_{\text{p}, 1})\cup\&^\kwd{m}(\tau_{\text{c}, 2}, \tau_{\text{p}, 2})=\&^\kwd{m}(\tau_\text{c}, \tau_\text{p})}
    \end{mathpar}
    \caption{Merging}
    \label{fig:sta-merging}
\end{figure}

The order relation $\leq$ on resources derives another order relation on rich types, the \emph{subtyping} relation in \cref{fig:rich-subtyping}. The interpretation of subtyping $\tau_1 \preceq \tau_2$ is that the value $v$ typed with $\tau_1$ has \textbf{less} resource than the value $v$ typed with $\tau_2$. 
%
The rich type $\bot$ is a subtype of any type because $\bot$ denotes zero potential.
%
It is worth noting that in \rulename{S-Mutable}, $\tau_\text{p}$ is contravariant because prophecy type $\tau_\text{p}$ denotes the prophecy potential to return. 
%
The reflexive rule and the transitive rule are derivable.

A well-formed rich type always denotes a non-negative potential function. Our type system can drop well-formed types without sacrificing soundness.
%
\rulename{WF-List} and \rulename{WF-Box} request $\alpha \geq 0$, which makes $\alpha \cdot n \geq 0$ for list $l$ with length $n\geq0$. 
%
\rulename{WF-Shared} is a structural rule. For example, if $\kwd{list}(\alpha)$ is well-formed, so is $\&^\kwd{s}(\kwd{list}(\alpha))$. Rust borrow checker ensures that $\tau$ in $\&^\kwd{s}(\tau)$ satisfies $\tau\not=\&^\kwd{m}(\_, \_)$. Our type system supports nested borrows.
%
Besides structural premises $\vdash \tau_\text{c}$ and $\vdash \tau_\text{p}$, \rulename{WF-Mutable} demands \textbf{dropping condition} $\tau_\text{p} \preceq \tau_\text{c}$ in $\&^\kwd{m}(\tau_\text{c}, \tau_\text{p})$. The condition is called dropping condition because it works as dropping mutable borrows in \cref{fig:ex-prophecy}. The dropping condition makes sure that mutable borrow types denote non-negative potentials, as illustrated in \cref{sec:soundness}. 


\textbf{Merging} is a conservative approximation of resource potentials after conditional branching. Under typing context $\Gamma$, the type system checks statements $s_1$ and $s_2$ in different branches and gets remainder contexts $\Gamma_1$ and $\Gamma_2$.
The type system should merge them to check continuation. 
%
As illustrated in \cref{fig:sta-merging}, to merge typing contexts is to merge rich types at each $x$ in the domain of two contexts. 
% 
Because the prophecy type $\tau_\text{p}$ in mutable borrow is contravariant, we need to define not only the meet of types but also the join of types. Our purpose is to construct a \emph{lattice} with the property that $\tau_1\cap\tau_2\preceq \tau_i \preceq\tau_1\cup\tau_2, \forall i=1, 2$. Hence, merging over contexts is non-increasing on resources to conservatively fulfill soundness. 
%
The lattice operations of $\kwd{list}(\alpha_1)$ and $\kwd{list}(\alpha_1)$ are inherited from the resource's $\min$ and $\max$, so it is readily comprehensible. 
%
Notice that dropping conditions appear in rules \rulename{Meet-Mutable} and \rulename{Join-Mutable}. They are to fulfill soundness for weak updates, which is mentioned in \cref{sec:overview:Lattice}. Recall that dropping borrows without these conditions may increase resources in both original places indicated by $\tau_{\text{p}, 1}$ and $\tau_{\text{p}, 2}$, to break soundness. 

% In practical implementation, the well-formedness here $\vdash\&^\kwd{m}(\tau_{\text{c}, i}, \tau_{\text{p}, i})$ can be exchanged to $\tau_{\text{p}, i}\preceq\tau_{\text{c}, i}$, due to redundant well-formedness for substructures, $\vdash\tau_{\text{c}, i}$ and $\vdash\tau_{\text{p}, i}$ already covered by induction hypothesis on $\tau_{\text{c}, 1}\cap/\cup\tau_{\text{c}, 2}$ and $\tau_{\text{c}, 1}\cup/\cap\tau_{\text{p}, 2}$. 

It is worth noting that we support nested borrows like $\&^\kwd{s}(\&^\kwd{s}(\tau))$, $\&^\kwd{m}(\&^\kwd{s}(\tau_\text{c}), \&^\kwd{s}(\tau_\text{p}))$ and $\&^\kwd{m}(\&^\kwd{m}(\tau_\text{cc}, \tau_\text{cp}), \&^\kwd{m}(\tau_\text{cc}, \tau_\text{pp}))$. Rust's borrow mechanisms exclude nested borrows like shared borrows of mutable borrows $\&^\kwd{s}(\&^\kwd{m}(\tau_\text{c}, \tau_\text{p}))$, because they violate the property that at most one mutable borrow from the same piece of data is live at the same time.

\subsection{Typing Expressions} \label{sec:inference:eval}\
\begin{figure}[t]
\small
\judgement{Typing Expressions (Selected)}{$\Gamma\vdash e \hookrightarrow \tau\dashv\Gamma'$}
    \begin{mathpar}
    \inferrule*[Right=\rulename{$\Gamma$-Ev-Nil}]
    {\alpha ~\text{fresh}}
    {\Gamma\vdash \kwd{nil} \hookrightarrow \kwd{list}(\alpha)\vdash\Gamma}
    \and
    \inferrule*[Right=\rulename{$\Gamma$-Ev-Move}]
    {\Gamma\vdash p \hookrightarrow \tau
    \\ \GWt{\Gamma}{p}{\bot}{\Gamma'}
    }
    {\Gamma\vdash \kwd{move}~p \hookrightarrow \tau\dashv\Gamma'}
    \\

    \inferrule*[Right=\rulename{$\Gamma$-Ev-Shared}]
    {\Gamma\vdash p \hookrightarrow \tau
    \\ \textit{share}~ \tau ~\textit{as}~\tau_1, \tau_2
    \\ \GWt{\Gamma}{p}{\tau_1}{\Gamma'}
    }
    {\Gamma\vdash \&^\kwd{s}~p \hookrightarrow \&^\kwd{s}(\tau_2)\dashv\Gamma'}
    \\
    
    \inferrule*[Right=\rulename{$\Gamma$-Ev-Mutable}]
    {\Gamma\vdash p \hookrightarrow \tau
    \\ \textit{prophesy}~ \tau ~\textit{as}~ \tau_\text{p} 
    \\ \GWt{\Gamma}{p}{\tau_\text{p}}{\Gamma'}
    }
    {\Gamma\vdash \&^\kwd{m}~p \hookrightarrow \&^\kwd{m}(\tau, \tau_\text{p})\dashv\Gamma'}
    \end{mathpar}
    \caption{Typing Expressions}
    \label{fig:sta-eval}
\end{figure}

\begin{figure}[t]
\small
    \judgement{Sharing (Selected)}{$\textit{share}~ \tau ~\textit{as}~\tau_1, \tau_2$}
    \begin{mathpar}
    \inferrule*[Right=\rulename{Share-List}]
    {\alpha_1, \alpha_2 ~\text{fresh}
    \\\alpha = \alpha_1 + \alpha_2}
    {\textit{share}~ \kwd{list}(\alpha) ~\textit{as}~\kwd{list}(\alpha_1), \kwd{list}(\alpha_2)}
    \end{mathpar}
    \judgement{Prophesying (Selected)}{$\textit{prophesy}~ \tau_\text{c} ~\textit{as}~\tau_\text{p}$}
    \begin{mathpar}
    \inferrule*[Right=\rulename{Prophesy-List}]
    {\alpha_\text{p}~\text{fresh}}
    {\textit{prophesy}~ \kwd{list}(\alpha) ~\textit{as}~ \kwd{list}(\alpha_\text{p})}
    \end{mathpar}
    \caption{Sharing and Prophesying}
    \label{fig:sta-sharing-prophesying}
\end{figure}

\cref{fig:sta-eval} presents how to type check expressions via judgement $\Gamma\vdash e\hookrightarrow \tau\dashv\Gamma'$. Unlike the dynamic evaluation $V\vdash e \rightsquigarrow v$, checking expressions may modify $\Gamma$ to the remainder context $\Gamma'$. 
%
Rule \rulename{$\Gamma$-Ev-Nil} introduces a fresh unknown potential annotation $\alpha$ for $\kwd{nil}$. 
%
Rule \rulename{$\Gamma$-Ev-Move} explicitly moves the type $\tau$ out from place $p$, making $\GWt{\Gamma}{p}{\bot}{\Gamma'}$. 
%

Shared and mutable borrows modify typing context, as illustrated in rule \rulename{$\Gamma$-Ev-Shared} and \rulename{$\Gamma$-Ev-Mutable} with sharing and prophesying. \cref{fig:sta-sharing-prophesying} selects essential rules of $\textit{share}~ \tau ~\textit{as}~ \tau_1, \tau_2$ and $\textit{prophesy}~ \tau ~\textit{as}~ \tau_\text{p}$ for borrows. 

\textbf{Shared borrows} are handled with sharing $\textit{share}~ \tau ~\textit{as}~ \tau_1, \tau_2$. Recall the example in \cref{fig:ex-sharing}. We select the rule \rulename{Share-List} to reveal the essence of sharing. Sharing is splitting resource annotation $\alpha$ into $\alpha_1$ and $\alpha_2$ with linear constraint $\alpha = \alpha_1 + \alpha_2$. In rule \rulename{$\Gamma$-Ev-Shared}, we write $\tau_1$ back to original place $p$, with $\tau_2$ lent out. There is no sharing of mutable borrows as $\textit{share}~ \&^\kwd{m}(\_, \_) ~\textit{as}~ \tau_1, \tau_2$, because a well-checked program will never incur shared borrows of mutable borrows $\&^\kwd{s}(\&^\kwd{m}(\tau_\text{c}, \tau_\text{p}))$.

\textbf{Mutable borrows} are handled with prophesying $\textit{prophesy}~ \tau ~\textit{as}~ \tau_\text{p}$. Recall the example in \cref{fig:ex-prophecy}. The selected rule \rulename{Prophesy-List} prophesy $\alpha_\text{p}$ as the prophecy potential when the mutable borrow ends. In rule \rulename{$\Gamma$-Ev-Mutable}, we write prophecy type $\tau_\text{p}$ to the place $p$. Once the borrow ends, the dropping condition $\vdash \&^\kwd{m}(\tau, \tau_\text{p})$ ensures that the prophecy type $\tau_\text{p}$ is bounded by current type $\tau$.

\subsection{Typing Statements} \label{sec:inference:exec}
\begin{figure}[t]
\small
    \judgement{Typing Statements (Selected)}{$\Gamma\vdash s \hookrightarrow^\delta \dashv\Gamma'$}
    \begin{mathpar}
    \inferrule*[Right=\rulename{$\Gamma$-Ex-Tick}]
    {~}
    {\Gamma\vdash\kwd{tick}(\delta)\hookrightarrow^\delta\vdash\Gamma}
    \and
    \inferrule*[Right=\rulename{$\Gamma$-Ex-Drop}]
    {\Gamma\vdash p\hookrightarrow \tau
    \\ \vdash \tau
    \\ \GWt{\Gamma}{p}{\bot}{Gamma'}
    }
    {\Gamma\vdash \kwd{drop}~p \hookrightarrow^0\dashv \Gamma'}
    \\

    \inferrule*[Right=\rulename{$\Gamma$-Ex-Cons}]
    {\Gamma\vdash e_1\hookrightarrow \kwd{i32} \dashv \Gamma_1
    \\ \Gamma_1\vdash e_2\hookrightarrow \kwd{box}(\kwd{list}(\alpha'))\dashv\Gamma_2
    \\ \GWt{\Gamma_2}{p}{\kwd{list}(\alpha')}{\Gamma'}}
    {\Gamma\vdash p\from \kwd{cons}(e_1, e_2)\hookrightarrow^{\alpha'}\dashv\Gamma'}
    \\

    \inferrule*[Right=\rulename{$\Gamma$-Ex-If}]
    {\Gamma\vdash p\hookrightarrow \kwd{bool}
    \\ \Gamma\vdash s_1\hookrightarrow^{\delta_1}\dashv\Gamma_1
    \\ \Gamma\vdash s_2\hookrightarrow^{\delta_2}\dashv\Gamma_2
    \\ \max(\delta_1, \delta_2)=\delta
    \\ \Gamma_1\sqcap\Gamma_2=\Gamma' }
    {\Gamma\vdash \kwd{if}~ p ~\kwd{then}~ s_1 ~\kwd{else}~ s_2 ~\kwd{end} \hookrightarrow^\delta \dashv\Gamma'}
    \\
    \inferrule*[Right=\rulename{$\Gamma$-Ex-Mat}]
    {\Gamma\vdash p\hookrightarrow \kwd{list}(\alpha)
    \\ \Gamma\vdash s_1\hookrightarrow^{\delta_1}\dashv\Gamma_1
    \\\\ \GWt{\Gamma}{p}{\bot}{\Gamma_{\text{b}, 1}}
    \\ \GWt{\Gamma_{\text{b}, 1}}{x_\text{hd}}{\kwd{i32}}{\Gamma_{\text{b}, 2}}
    \\ \GWt{\Gamma_{\text{b}, 2}}{x_\text{tl}}{\kwd{box}(\kwd{list}(\alpha))}{\Gamma_\text{b}}
    \\ \Gamma_\text{b}\vdash s_2\hookrightarrow^{\delta_2}\dashv\Gamma'_\text{b}
    \\\\ \Gamma'_\text{b}\vdash x_\text{tl}\hookrightarrow \kwd{list}(\beta)
    \\ \GWt{\Gamma'_\text{b}}{x_\text{hd}}{\bot}{\Gamma'_{\text{b}, 1}}
    \\ \GWt{\Gamma'_{\text{b}, 1}}{x_\text{tl}}{\bot}{\Gamma'_{\text{b}, 2}}
    \\ \GWt{\Gamma'_{\text{b}, 2}}{p}{\kwd{list}(\beta)}{\Gamma_2}
    \\\\ \max(\delta_1, \delta_2-(\alpha-\beta))=\delta
    \\ \Gamma_1\sqcap\Gamma_2=\Gamma'}
    {\Gamma\vdash \kwd{match}~ p ~ \{\kwd{nil}\mapsto s_1, \kwd{cons}(x_\text{hd}, x_\text{tl})\mapsto s_2\} \hookrightarrow^\delta \dashv\Gamma'}
    \\

    \inferrule*[Right=\rulename{$\Gamma$-Ex-App}]
    {\text{fn}~ f ~(\vec{x}_\text{param}:\vec{t}_\text{param}, \vec{x}_\text{local}:\vec{t}_\text{local}, x_\text{ret}:t_\text{ret}) \{~ s ~\}
    \\\\ \vdash f \Leftarrow (\Gamma_f, \delta_f)
    \\ \Gamma_f\vdash x_\text{ret} \hookrightarrow \tau_\text{ret}, (\forall x_i\in\vec{x}_\text{param}, i=1, ..., n) \Gamma_f \vdash x_i \hookrightarrow \tau_{\text{param}, i}
    \\\\ \Gamma_0=\Gamma, (\forall e_i\in \vec{e}, i=1, ..., n) \Gamma_{i-1}\vdash e_i\hookrightarrow\tau_{\text{arg}, i}\dashv\Gamma_i
    \\ (\forall i=1,..,n)~ \tau_{\text{param}, i} = \tau_{\text{arg}, i}
    \\ \Gamma_n \vdash p \hookrightarrow \tau
    \\ \vdash \tau
    \\ \GWt{\Gamma_n}{p}{\tau_\text{ret}}{\Gamma'}
    }
    {\Gamma\vdash p\from f(\vec{e})\hookrightarrow^{\delta_f}\dashv\Gamma'}
    \end{mathpar}
    \caption{Typing Statements}
    \label{fig:sta-exec}
\end{figure}

\begin{figure}[t]
\small
    \judgement{Function Analysis}{$\vdash f \Leftarrow (\Gamma_f, \delta_f)$}
    \begin{mathpar}
    \inferrule
    {\text{fn}~ f ~(\vec{x}_\text{param}:\vec{t}_\text{param}, \vec{x}_\text{local}:\vec{t}_\text{local}, x_\text{ret}:t_\text{ret}) \{~ s ~\}
    \\  \vdash f \Rightarrow (\Gamma_f, \delta_f)
    \\ \Gamma_f\vdash s\hookrightarrow^\delta\dashv\Gamma'_f
    \\\\ \forall x \in \textbf{dom}(\Gamma'_f), \vdash \Gamma'_f(x)
    \\ \Gamma'_f \vdash x_\text{ret} \hookrightarrow \tau'_\text{ret}
    \\ \Gamma_f \vdash x_\text{ret} \hookrightarrow \tau_\text{ret}
    \\ \tau'_\text{ret} = \tau_\text{ret}
    \\ \delta = \delta_f}
    {\vdash f \Leftarrow (\Gamma_f, \delta_f)}
    \end{mathpar}
    \caption{Function Analysis}
    \label{fig:fun-anal}
\end{figure}

\cref{fig:sta-exec} presents how to type check statements as judgement $\Gamma\vdash s \hookrightarrow^\delta \dashv\Gamma'$. Under context $\Gamma$, the statement $s$ is checked with resource consumption $\delta$, and context becomes $\Gamma'$. 

Rule \rulename{$\Gamma$-Ex-Tick} indicates $\kwd{tick}(\delta)$ consumes $\delta$ unit of resource. Rule \rulename{$\Gamma$-Ex-Drop} drops the type $\tau$ with well-formedness $\vdash\tau$ as the dropping condition. Rule \rulename{$\Gamma$-Ex-Cons} indicates that $\kwd{cons}$ will consume $\alpha$ unit of resource for continuation payment, when the tail $e_2$ is typed with $\kwd{box}(\kwd{list}(\alpha))$. 

\textbf{Branching statements} require context merging, as in \rulename{$\Gamma$-Ex-If} and \rulename{$\Gamma$-Ex-Mat}. Rule \rulename{$\Gamma$-Ex-If} is simpler to merge contexts with the consumption as the maximum of those branches. Rule \rulename{$\Gamma$-Ex-Mat} is more intricate, due to resource potential stored in \kwd{cons}. The $\kwd{cons}$ branch will obtain $\alpha-\beta$ units of potential, therefore the net consumption is $\delta_2-(\alpha-\beta)$. Given $\Gamma\vdash p \hookrightarrow \kwd{list}(\alpha)$, The potential is not $\alpha$ but $\alpha-\beta$. $\beta$ is the remainder potential, indicated by $\Gamma'_\text{b} \vdash x_\text{tl} \hookrightarrow \kwd{list}(\beta)$. The subscript $\text{b}$ of $\Gamma_\text{b}$ means \textbf{b}inding, similar to rule \rulename{V-Ex-Mat}.

\textbf{Function application} is intractable because of recursive functions. Rule \rulename{$\Gamma$-Ex-App} assumes the function $f$ has a well-checked signature $(\Gamma_f, \delta_f)$, with judgement $\vdash f \Leftarrow (\Gamma_f, \delta_f)$ in \cref{fig:fun-anal}, different from $\vdash f \Rightarrow (\Gamma_f, \delta_f)$. Other premises are to ensure that the resources of actual arguments are equal to those of formal parameters. 

\subsection{Type Inference} \label{sec:inference:infer}
To this point, our type system has been primarily declarative because the well-checked signature in rule \rulename{$\Gamma$-Ex-App} is assumed to be pre-existent. Same as other AARA systems (such as Resource-aware ML~\cite{RaML}), we use linear programming to convert the declarative type system to an algorithmic version. The type system creates symbolic variables to denote unknown annotations in rich types and signatures. The type system then collects linear constraints among those symbolic variables and finally solves them via linear programming solvers. 

Readers might have perceived that a recursive function requires a well-checked signature during checking and that a function can exhibit multiple signatures at different call sites. To automatically analyze functions, we need to preprocess the call graph. First, we group recursive functions as strongly connected components. Second, we topologically sort groups to determine an order to analyze. For each group, we predefine signatures of functions in the group via the judgement $\textit{enrich}~ t ~\textit{as}~ \tau$ in \cref{fig:enrich}. During function analysis, the signature $(\Gamma_f, \delta_f)$ in \rulename{$\Gamma$-Ex-App} should be replaced with the predefined one if $f$ is in the group. Otherwise, $f$ is in the previously analyzed group, so we should clone that group's signature and linear constraints. It is necessary to clone instead of copy them because annotations in signatures and constraints are sensitive to actual arguments of function calls.

With linear constraints collected during function analysis and a heuristic objective, we can employ a linear programming solver to find instances of annotations that satisfy those constraints automatically. The inferred annotations in signatures will characterize functions' resource consumption. 

% The $\vdash f \Rightarrow (\Gamma_f, \delta_f)$ in premises of $\vdash f \Leftarrow (\Gamma_f, \delta_f)$, and function application in body $s$, require a topological order of functions to analyze. And (mutual) recursive functions require a strongly connected grouping of functions. In our implementation, all these requirements turn into strongly connected component analysis and topological sorting.
\section{Soundness} \label{sec:soundness}

In this section, we define potential functions indicated by the resource-enriched types from \cref{sec:inference} in \cref{sec:pot-funcs} and then sketch the soundness proof of the resource-aware type system in \cref{sec:proof-sketch}. We include the detailed proofs in the \cref{sec:proof}.

\subsection{Potential Functions}
\label{sec:pot-funcs}
 
% We always assume that $v$ is well typed with $\tau$, stated as $\vdash v : \tau$, when we notate $\phi(v:\tau)$. We omit the definition in literal, as well as well typing over store and context as $\vdash V : \Gamma$, both guaranteed by Rust type system.

% \begin{definition}
%    $ \Phi(V:\Gamma) = \sum_{x\in\textbf{dom}(\Gamma)} \phi(V(x):\Gamma(x)) $ .
% \end{definition}
\begin{figure}[t]
\small
    \judgement{Potential Functions}{$\Phi(V:\Gamma) = \sum_{x\in\textbf{dom}(\Gamma)} \phi(V(x):\Gamma(x))$} \\
    \judgement{Potential Functions (Selected)}{$\phi(v:\tau)$}
    \begin{mathpar}
    \inferrule[$\phi$-Nil]
    {~}
    {\phi(\kwd{nil}:\kwd{list}(\alpha))=0}
    \and
    \inferrule[$\phi$-Cons]
    {~}
    {\phi(\kwd{cons}(\kwd{n}_\text{i32}, \kwd{box}(lv)):\kwd{list}(\alpha))=\alpha+\phi(\kwd{box}(lv):\kwd{box}(\kwd{list}(\alpha)))}
    \\
    \inferrule[$\phi$-Shared]
    {~}
    {\phi(\&(\_, v):\&^\kwd{s}(\tau))=\phi(v:\tau)}
    \and
    \inferrule[$\phi$-Mutable]
    {~}
    {\phi(\&(\_, v):\&^\kwd{m}(\tau_\text{c}, \tau_\text{p}))=\phi(v:\tau_\text{c})-\phi(v:\tau_\text{p})}
    \end{mathpar}
    \caption{Potential Functions}
    \label{fig:sound-potential}
\end{figure}

\cref{fig:sound-potential} defines potential function $\phi(v:\tau)$ and $\Phi(V:\Gamma)$.
% 
\rulename{$\phi$-Nil} and \rulename{$\phi$-Cons} define the potential of a list $l : \kwd{list}(\alpha)$ with length $n$ to be $\alpha \cdot n$.
%
\rulename{$\phi$-Shared} defines the potential of shared borrows to be the potential of borrowed values and borrowed rich types. 
%
\rulename{$\phi$-Mutable} defines the potential of mutable borrows to be the difference between current and prophecy potential. It is worth noting that when the program incurs a mutable borrow, the potential of context will not change. The dropping condition $\tau_\text{p} \preceq \tau_\text{c}$ in $\vdash \&^\kwd{m}(\tau_\text{c}, \tau_\text{p})$, ensures the potential is non-negative. We, therefore have the following lemma about potential:

\begin{lemma}
    Potential is non-negative and keeps subtyping:
    \begin{mathpar}
    \inferrule
    {  \vdash \tau_1
    \\ \vdash \tau_2
    \\ \tau_1 \preceq \tau_2
    }
    { 0 \leq \phi(v:\tau_1) \leq \phi(v:\tau_2)}
    \end{mathpar}
\end{lemma}
\begin{proof}
    First define the size of rich types structural inductively $\mathbf{size} : \mathbf{RichType} \to \NN$, \ 
    and prove by induction on $\mathbf{size}(\tau_1) + \mathbf{size}(\tau_2)$. The well-formedness will be used to prove potential of mutable borrows is non-negative.
\end{proof}
\begin{corollary}
    Potential is non-negative $\dfrac{\vdash \tau}{0\leq \phi(v:\tau)}$ due to the derivable rule \rulename{S-Refl} $\dfrac{~}{\tau\preceq\tau}$.
\end{corollary}

The subtyping relation and well-formedness can be extended to typing context, which can be conceptualized as record types. Lemmas about the lattice operation and potential can be proved with definition and simple induction.

\begin{definition}
    Subcontext $\Gamma_1 \subseteq \Gamma_2$ if and only if $\forall x \in \textbf{dom}(\Gamma_1), x \in \textbf{dom}(\Gamma_2), \Gamma_1(x)\preceq \Gamma_2(x)$. \\
    Context well formed $\vdash \Gamma$ if and only if $\forall x \in \textbf{dom}(\Gamma), \vdash \Gamma(x)$.
\end{definition}
\begin{lemma}[]
    For any store $V$ and any context $\Gamma_1, \Gamma_2$, 
    \begin{enumerate}
        \item {$\Gamma_1\sqcap\Gamma_2 \subseteq \Gamma_1$ and $\Gamma_1\sqcap\Gamma_2 \subseteq \Gamma_2$; }
        \item {if $\vdash \Gamma_1, \vdash \Gamma_2$, then $\vdash \Gamma_1\sqcap\Gamma_2$; }
        \item {if $\vdash \Gamma_1, \vdash \Gamma_2, \Gamma_1 \subseteq \Gamma_2$, then $0\leq \Phi(V:\Gamma_1) \leq \Phi(V:\Gamma_2)$. }
    \end{enumerate}
\end{lemma}


\subsection{Soundness Theorem}
\label{sec:proof-sketch}

With potential, we are able to formulate soundness theorem, which states that resource consumption of dynamics, together with potential difference, is bounded by type system.
\begin{theorem}[Soundness]
Our type system is sound towards resource aware dynamic semantics: \\
If $V\vDash s \rightsquigarrow^{\delta_V} \Dashv V'$ and $\Gamma \vdash s \hookrightarrow^{\delta_\Gamma} \dashv \Gamma'$, then $\Phi(V':\Gamma') - \Phi(V:\Gamma)+\delta_V \leq \delta_\Gamma$.
\end{theorem}
\begin{proof}
By induction on $V\vDash s \rightsquigarrow^{\delta_V} \Dashv V'$, with the help of following lemmas.
\end{proof}

% Lemmas are listed as follows, and proof is simply done by induction with borrow properties.

\begin{lemma}[Update]
If store or context is written with new value or new type, the difference of potential over store or context is equal to that over new value or new type.
\begin{enumerate}
    \item {If $V\vDash p\rightsquigarrow v$, $\Gamma\vdash p\hookrightarrow \tau$ and $\VWt{V}{p}{v'}{V'}$,
    then $\Phi(V':\Gamma) - \Phi(V:\Gamma)=\Phi(v':\tau)-\Phi(v:\tau)$;}
    \item {If $V\vDash p\rightsquigarrow v$, $\Gamma\vdash p\hookrightarrow \tau$ and $\GWt{\Gamma}{p}{\tau'}{\Gamma'}$, 
    then $\Phi(V:\Gamma')-\Phi(V:\Gamma)=\Phi(v:\tau')-\Phi(v:\tau)$.}
\end{enumerate}
\end{lemma}
\begin{proof}
By induction on $\VWt{V}{p}{v'}{V'}$ and $\GWt{\Gamma}{p}{\tau'}{\Gamma'}$.
\end{proof}
\begin{lemma}[Evaluation]
If $V\vDash e\rightsquigarrow v, \Gamma\vdash e\hookrightarrow \tau\dashv\Gamma'$, then $\Phi(V:\Gamma')-\Phi(V:\Gamma)= -\phi(v:\tau)$.
\end{lemma}
\begin{proof}
By induction on $e$.
\end{proof}

Towards soundness proof for statements, especially for the rule \rulename{$\Gamma$-Ex-App}, we will need one series of lemmas about context extension. Though similar, context extension is different from subcontext. Context extension necessitates strict type equality, whereas subcontext only demands subtyping. The weakening rules are intuitive about context extension, as they merely involve adding variables and types that the program will not utilize. 

\begin{definition}
    Extension $\Gamma_1\sqsubseteq\Gamma_2$ if and only if $\forall x \in\textbf{dom}(\Gamma_1), x\in\textbf{dom}(\Gamma_2), \Gamma_1(x) = \Gamma_2(x)$.
\end{definition}
\begin{lemma}[Weakening]
In following rules, $\dfrac{A ~ B}{C ~ D}$ means if $A$ and $B$ then $C$ and $D$.
\begin{mathpar}
    \inferrule[$\Gamma$-Rd-Weaken]
    {\Gamma_1 \sqsubseteq \Gamma_2 
    \\ \Gamma_1 \vdash p \hookrightarrow \tau_1
    }
    {\Gamma_2 \vdash p \hookrightarrow \tau_2
    \\ \tau_1 = \tau_2
    }
    \and
    \inferrule[$\Gamma$-Wt-Weaken]
    {\Gamma_1 \sqsubseteq \Gamma_2
    \\ \GWt{\Gamma_1}{p}{\tau_1}{\Gamma'_1}
    \\ \tau_1 = \tau_2
    }
    { \GWt{\Gamma_2}{p}{\tau_2}{\Gamma'_2}
    \\ \Gamma'_1 \sqsubseteq \Gamma'_2 
    }
    \\
    \inferrule[$\Gamma$-Ev-Weaken]
    {\Gamma_1 \sqsubseteq \Gamma_2
    \\ \Gamma_1 \vdash e \hookrightarrow \tau_1 \dashv \Gamma'_1 
    }
    {\Gamma_2 \vdash e \hookrightarrow \tau_2 \dashv \Gamma'_2
    \\ \tau_1 = \tau_2 
    \\ \Gamma'_1 \sqsubseteq \Gamma'_2
    }
    \and
    \inferrule[$\Gamma$-Ex-Weaken]
    {\Gamma_1 \sqsubseteq \Gamma_2
    \\ \Gamma_1 \vdash s \hookrightarrow^{\delta_1} \dashv \Gamma'_1
    }
    { \Gamma_2 \vdash s \hookrightarrow^{\delta_2} \dashv \Gamma'_2
    \\ \delta_1 = \delta_2
    \\ \Gamma'_1 \sqsubseteq \Gamma'_2
    }
\end{mathpar}
\end{lemma}
\begin{proof}
    By induction on $\Gamma_1\vdash p\hookrightarrow \tau_1$, $\GWt{\Gamma_1}{p}{\tau_1}{\Gamma'_1}$, $\Gamma_1\vdash e\hookrightarrow\tau_1\dashv\Gamma'_1$, $\Gamma_1\vdash s\hookrightarrow^{\delta_1}\dashv\Gamma'_1$.
\end{proof}


\section{Experimental Evaluation} \label{sec:impl}

In this section, we present an experimental evaluation of \rarust{}.
%
\cref{sec:proto} describes our prototype implementation of \rarust{}.
%
\cref{sec:eval} presents the evaluation results of \rarust{} on a suite of benchmarks.

\subsection{Implementation}
\label{sec:proto}

% In this section, we will describe the work flow, and supported features of implementation. 

We have implemented a prototype linear resource analyzer, \rarust{}. It takes raw Rust programs (within only $\kwd{tick}$ annotation) as inputs, and prints functions' signatures with resource annotation as output.
%
\rarust{} analyzes the whole program regardless of whether there are annotations or not. We currently use explicit manually annotated $\kwd{tick}$ as the cost model, but it is straightforward to support parametric cost models, which assign a cost to each kind of statements, as prior AARA systems (e.g., \textsc{RaML}~\cite{RaML}) do.
(1) \rarust{} first obtains the typed IR of the borrow calculus with explicit drops via Charon \cite{Aeneas} as a plugin of Rust compiler. Rust compiler guarantees that the compiled IR is well-checked, and \rarust{} will utilize properties straightforwardly. 
(2) Towards IR, \rarust{} analyzes strongly connected components of function call graph and topologically sort components, generating precedence of type checking. 
(3) \rarust{} enriches function signatures with fresh linear variables as resource annotation and assigns each component a linear programming context to record linear constraints. 
(4) As stated in \cref{sec:inference:infer}, \rarust{} type-checks functions' bodies, generating linear constraints.
% For function call, when the function is in the same component, \rarust{} will only use the signature; when the function is in previous component, \rarust{} will clone previous linear programming context, add to current context, and use the cloned signature.
(5) \rarust{} finally solves these constraints, with its heuristic linear objective, by invoking linear programming solvers. We can utilize different solvers due to the separation of (4) and (5), and we select CoinCbc \cite{CoinCbc} for demonstration.

Our implementation supports some features, based on core calculus formalized in \cref{sec:calculus} and some practical extensions. We list as follows:
\begin{enumerate}
    \item{
    \rarust{} supports reborrows and nested borrows like \lstinline|& & T|, \lstinline|&mut & T| and \lstinline|&mut &mut T|.
    }
    \item {
    \rarust{} supports user-defined data types instead of only built-in lists, as \cref{tab:user-defined} shows. 
    We annotate potential for each constructor, \lstinline|Nil|, \lstinline|Cons|, \lstinline|Leaf|, \lstinline|Node|, \lstinline|(_, _)|, \lstinline|Record(_, _)|, \lstinline|NListNode| and \lstinline|NList|. It is worth noting that \lstinline|NListNode| and \lstinline|NList| are mutually recursive.
    }
    \item {
    \rarust{} supports looping statements including $\kwd{while true}(s), \kwd{break}(i), \kwd{continue}(i)$, where $i$ represents the $i$-th loop outward from $\kwd{break}$ or $\kwd{continue}$. \rarust{} supports this directly without transforming loops into recursive functions.
    }
\end{enumerate}


\begin{DIFnomarkup}
\begin{table}[t]
\centering
\caption{User-defined Data Types} \label{tab:user-defined}
\footnotesize
\begin{tabular}{|l|l|}
\hline
\begin{lstlisting}[language=Rust, style=colouredRust]
pub enum List {
    Nil,
    Cons(i32, Box<List>),
}
\end{lstlisting} 
&
\begin{lstlisting}[language=Rust, style=colouredRust]
pub enum Tree {
    Leaf,
    Node(i32, Box<Tree>, Box<Tree>),
}
\end{lstlisting} 
\\
\hline
\begin{lstlisting}[language=Rust, style=colouredRust]
pub type Tuple = (List, Tree);
\end{lstlisting}
&
\begin{lstlisting}[language=Rust, style=colouredRust]
pub struct Record{pub l:List, pub t:Tree}
\end{lstlisting}
\\
\hline
\begin{lstlisting}[language=Rust, style=colouredRust]
pub struct NListNode {
    pub value : i32,
    pub next : NList
}
\end{lstlisting} 
&
\begin{lstlisting}[language=Rust, style=colouredRust]
pub enum NList {
    None,
    Some(Box<NListNode>)
}
\end{lstlisting} 
\\
\hline
\end{tabular}
\end{table}
\end{DIFnomarkup}

\subsection{Evaluation}
\label{sec:eval}

\begin{DIFnomarkup}
\begin{table}[t]
    \centering
    \caption{Benchmarks}
    \label{tab:eval}
    \small
    \scalebox{0.67}{
    \begin{tabular}{|c|c|c|c|c|c|}
    \hline
    case & type & description & complexity & lines of code & size of constraints  \\
    \hline
    \multicolumn{6}{|c|}{\textbf{feature(s): mutable borrows}} \\ \hdashline
    end\_m & {\lstinline|fn(l:&m List)->&m List|} & find the mutable borrow of the {\lstinline|Nil|} of a list & $1+3|l|$ & 14 & 71 \\
    end\_c & {\lstinline|fn(l:&m List,o:&m&m List)|} & c style end\_m to write to {\lstinline|o|} & $2+3|l|$ & 13 & 96 \\ 
    append & {\lstinline|fn(l:&m List,x:i32)|} & append $x$ to $l$ by {\lstinline|end_m|} & $5+3|l|$ & 5 & 89 \\
    concat & {\lstinline|fn(l1:&m List,l2:List)|} & $l'_1 = l_1@l_2$ by {\lstinline|end_c|} & $6+3|l_1|$ & 8 & 144 \\
    \hline
    \multicolumn{6}{|c|}{\textbf{feature(s): shared borrows, mutable borrows, recursive functions, and loop statements}} \\ \hdashline
    sum\_rec & {\lstinline|fn(l:& List)->i32|} & sum up integers, recursively & $1 + 6|l|$ & 14 & 19 \\
    sum\_loop & {\lstinline|fn(l:& List)->i32|} & sum up integers, via loops & $1 + 6|l|$ & 24 & 54 \\
    
    rev\_rec & {\lstinline|fn(l:&m List, r: List)->List|} & reverse $l$ to head of $r$ & $1 + 9|l|$ & 15 & 57 \\
    
    rev\_loop & {\lstinline|fn(l:&m List)|} & reverse $l$ mutably via loops & $1 + 9|l|$ & 21 & 164 \\
    % remove rich
    % \hline
    % sum & \lstinline|fn(l:& List)->i32| & sum up integers in list & $1 + 6|l|$ \\
    % rich & \lstinline|fn(l:&m List)| & iterate $l$ with $-1, -6$ as $\kwd{tick}$ & $-6|l|$ \\
    % rich\_sum & \lstinline|fn(l:&m List)->i32| & \lstinline|rich(l)|, and then \lstinline|sum(l)| & $1$ \\
    \hline
    \multicolumn{6}{|c|}{\textbf{feature(s): function calls}} \\ \hdashline
    sum2 & {\lstinline|fn(l:& List)|} & {\lstinline|sum_rec(l);sum_loop(l)|} & $2 + 12|l|$ & 4 & 82 \\
    rev & {\lstinline|fn(l:&m List)|} & reverse $l$ mutably via {\lstinline|rev_rec|} & $4 + 9|l|$ & 5 & 71 \\
    rev2 & {\lstinline|fn(l:&m List)|} & apply {\lstinline|rev|} to $l$ twice & $8 + 18|l|$ & 4 & 170 \\
    dup & {\lstinline|fn(l: List)->List|} & duplicate each element in $l$ & $1+11|l|$ & 16 & 37 \\
    dup2 & {\lstinline|fn(l: List)->List|} & apply {\lstinline|dup|} to $l$ twice & $2+33|l|$ & 4 & 81 \\
    \hline
    \multicolumn{6}{|c|}{\textbf{feature(s): reborrow and nested borrows}} \\ \hdashline
    reborrow\_s & {\lstinline|fn(l:& List)|} & reborrow $l$ as $ll$, {\lstinline|sum2(ll);sum2(l)|}& $4+24|l|$ & 6 & 176 \\
    reborrow\_m & {\lstinline|fn(l:&mut List)|} & reborrow $l$ as $ll$, {\lstinline|rev2(ll);rev2(l)|}& $16+36|l|$ & 6 & 364 \\
    nested\_s\_s & {\lstinline|fn(l:& & List)|} & {\lstinline|sum2(*l);|} & $2+12|l|$ & 4 & 88 \\
    nested\_m\_s & {\lstinline|fn(l:&m & List)|} & {\lstinline|sum2(*l);|} & $2+12|l|$ & 4 & 90 \\
    nested\_m\_m & {\lstinline|fn(l:&m &m List)|} & {\lstinline|rev2(*l);|} & $8+18|l|$ & 4 & 188 \\
    \hline
    \multicolumn{6}{|c|}{\textbf{feature(s): user-defined datatypes}} \\ \hdashline
    min & {\lstinline|fn(t:&Tree,d:i32)->i32|} & find min in $t$, $d$ as default & $1+5|t|$ & 14 & 19 \\
    max & {\lstinline|fn(t:&Tree,d:i32)->i32|} & find max in $t$, $d$ as default & $1+5|t|$ & 14 & 19 \\
    find & {\lstinline|fn(t:&Tree,x:i32)->bool|} & find whether $x$ is in $t$ & $1+8|t|$ & 29 & 31 \\
    add & {\lstinline|fn(t:&m Tree,x:i32)|} & add up $x$ to each element of $t$ & $1+10|t|$ & 15 & 53\\
    tuple & {\lstinline|fn(x:&m Tuple)|} & {\lstinline|rev2(x.0);min(x.1, 0);|} & $9+18|x.0|+5|x.1|$ & 4 & 216 \\
    record & {\lstinline|fn(x:&m Record)|} & {\lstinline|rev2(x.l);min(x.t, 0);|} & $9+18|x.l|+5|x.t|$ & 4 & 216 \\
    sum\_rec\_nlist & {\lstinline|fn(l:&NList) -> i32|} & sum up {\lstinline|NList|} as \lstinline|sum_rec| & $1+5|l|$ & 16 & 26 \\
    rev\_rec\_nlist & {\lstinline|fn(l:&mut NList,r:NList)->NList| } & reverse {\lstinline|NList|} as \lstinline|rev_rec| & $1+7|l|$ & 21 & 89 \\
    \hline
    \end{tabular}
    }
\end{table}
\end{DIFnomarkup}

We used the prototype implementation of \rarust{} to perform an experimental evaluation of resource analysis on typical examples concerning Rust's borrow mechanism. We adapted several pure functional examples from \textsc{RaML} to their Rust counterparts employing borrows; by using borrows we can do in-place updates in Rust. Some examples were deliberately crafted to demonstrate the prototype’s capability to support the aforementioned features. Due to the linear limitation, we select those examples with linear complexity. The experiments were performed on a laptop with 2.20 GHz Intel Core i9-13900HX as CPU and WSL 2.3.24.0/Ubuntu 22.04.3 LTS as operation system. The compiling of the benchmarks was done in 0.15s and the analysis was done in 0.3s.

\cref{tab:eval} shows our experimental results. We manually checked the analysis results on the benchmarks and confirmed that the derived bounds are correct (but not tight for some benchmarks such as \lstinline|min| and \lstinline|max|). We encode some cases with the derived bounds in \textsc{Flux}~\cite{Flux} to check that the derived bounds are correct. Our encoding introduces a global counter to accumulate resource consumption. The encoding is shipped into the artifact and it includes cases \lstinline|append|, \lstinline|concat|, \lstinline|sum_rec|, \lstinline|rev_rec|, \lstinline|sum2|, \lstinline|rev|, \lstinline|rev2|, \lstinline|dup|, \lstinline|dup2|, \lstinline|min|, \lstinline|max|, \lstinline|find|, \lstinline|add|, \lstinline|tuple|, \lstinline|record|, \lstinline|sum_rec_nlist|, and \lstinline|rev_rec_nlist|.

\cref{tab:eval} gives out 5 groups of test cases, and each group exercises some features. For each analyzed function as a case, \cref{tab:eval} first declares the function type in a simplified syntax of Rust, writing \lstinline|&mut T| as \lstinline|&m T| for short. The description column provides a concise explanation of the function's semantics. We plot the complexity in a more readable format in the table, where $|l|$ represents the length or count of \lstinline|Cons| constructors of $l$ and $|t|$ represents the count of \lstinline|Node| constructors of the tree $t$. The concrete coefficients are inferred by \rarust{} according to annotation $\kwd{tick}(\delta)$ in examples. In our evaluation, we set those different concrete numbers of $\delta$ for two purposes: (i) we roughly add one $\kwd{tick}(\delta)$ around one statement to account for the number of operations by the statement, and (ii) we can use different numbers to test multiple times to check if our implementation is correct.
%
We will explain each group in detail in the rest of this section.

% \textbf{Mutable Borrows and Nested Borrows:} 
To show that \rarust{} can handle mutable borrows, we construct cases \lstinline|end_m| and \lstinline|end_c|. They are recursively to find the mutable borrow of the nil of a list $l$. For example, in ML syntax, the nil of the list \verb|1::2::3::4::[]| is \verb|[]|. \lstinline|end_m| returns the borrow, while  \lstinline|end_c| storing it in the parameter \lstinline|o:&m&m List|. The returned mutable borrow of \lstinline|end_m| works as a closure function with type \lstinline|List->List|, therefore it is non-trivial for resource analysis. We use cases \lstinline|append| and \lstinline|concat| to show the resource correctness as well as the compositionality of the analysis.

%\textbf{Recursive Functions and Loop Statements:} 
Rust programmers are able to write code with loop statements. We construct cases \lstinline|sum_rec|, \lstinline|sum_loop|, \lstinline|rev_rec| and \lstinline|rev_loop|. We focus on shared borrows in \lstinline|sum| and on mutable borrows in \lstinline|rev|. The suffix \lstinline|rec| means recursive function and \lstinline|loop| means loop statements \lstinline|while true { ... }|. The same analysis results reveal that both are supported by \rarust{}. 

% \textbf{Sharing and Prophesying:} 
We construct multiple calls of function for shared borrows and mutable borrows, to demonstrate sharing and prophesying. The suffix \lstinline|2| means twice in cases \lstinline|sum2|, \lstinline|rev2| and \lstinline|dup2|. The coefficients in the complexity of \lstinline|sum2| are exactly 2 times of \lstinline|sum|, testing the sharing for shared borrows. \lstinline|rev2| is similar but for the prophesying of mutable borrows. \lstinline|dup2| is made to point out the difference between sharing and prophesying. The function \lstinline|dup| mutates the length of list $l$, therefore the second call of \lstinline|dup| is with length $2|l|$, making the linear coefficient of \lstinline|dup2| be $33$, 3 times $11$.


% \textbf{Reborrows and Nested Borrows:} 
\rarust{} also supports reborrows and nested borrows. We construct cases \lstinline|reborrow_s|, \lstinline|reborrow_m|, \lstinline|nested_s_s|, \lstinline|nested_m_s| and \lstinline|nested_m_m|. The suffix \lstinline|s| denotes shared borrows, while \lstinline|m| denotes mutable borrows.

% \textbf{User-defined Data Types:} 
Besides \lstinline|List|, we construct simple examples for other safe user-defined data types like trees, tuples, records, and C-style lists. The sizes of trees are the counts of internal nodes \lstinline|Node|, instead of the intuitive measure, their depths. \lstinline|Tuple| and \lstinline|Record| are data types to compose \lstinline|List| and \lstinline|Tree|, with complexity as the linear composition of their fields', such as \lstinline|x.0| and \lstinline|x.t|. \rarust{} also supports mutually recursive data types like \lstinline|NList| and \lstinline|NListNode|; they are C-style lists, with the former as the nullable pointer, the latter as the internal node of lists.
\section{Discussion}
\label{sec:discussion}

In this section, we discuss some limitations of \rarust{} mentioned in \cref{section:introduction} and propose possible pathways towards overcoming them to improve the capability of \rarust{} in future work.

\paragraph{Unsafe code, interior mutability, vectors, reference counting, and cyclic data structures}
%
We focus on safe Rust programs because our design of \rarust{} relies on guarantees provided by Rust's borrow mechanisms, e.g., aliasing and mutation cannot happen simultaneously.
%
However, Rust programs cannot avoid unsafe code in general, because many standard libraries---including cells (\verb|Cell|), vectors (\verb|Vec|), and reference counting (\verb|Rc|)---rely on unsafe code to allow shared mutable states, e.g., interior mutability.
%
The unsafe code can operate C-style pointers in an unrestricted way and compromise Rust's memory safety; as a result, \rarust{}'s resource analysis cannot handle unsafe code.
%
This is actually a common limitation of advanced type systems and verification frameworks for Rust, including Flux~\cite{Flux}, Aeneas~\cite{Aeneas}, and Prusti~\cite{OOPSLA:AMP19}.
%
Nevertheless, there have been efforts to support unsafe code in formal reasoning about Rust programs.
%
RustBelt~\cite{RustBelt} pioneers a line of work on semantic typing and separation-logic-based verification of Rust programs with unsafe code. 
%
Verus~\cite{OOPSLA:LHC23} supports some unsafe features by providing specifications for unsafe memory operations to be memory safe and employing SMT solvers to check those specifications automatically.
%
However, it is unclear if one can integrate AARA type systems with those techniques.

One common method to support unsafe code is based on Rust's design philosophy: unsafe operations should be properly \emph{encapsulated} by safe APIs, and the developers of those unsafe operations take charge of ensuring the unsafe code does not break Rust's memory safety.
%
In terms of type systems, this amounts to assigning types to the safe APIs instead of inferring types from the unsafe code body.
%
Therefore, it would be possible for \rarust{} to analyze Rust programs with unsafe code, if all the unsafe code is encapsulated by resource-annotated safe APIs.
%
Rust's \verb|Cell| features interior mutability by providing operations for both getting and setting the content of a memory location.
%
At the API level, the \verb|Cell| type works similarly to references in an ML-like functional programming language, so it would be possible to adapt an AARA approach for supporting references~\cite{FSCD:LH17}.
%
For example, the resource-annotated APIs shown below can be used to manipulate cells storing lists:
%
\begin{align*}
    \texttt{new}: & \kwd{fn}(\texttt{l}: \kwd{list}(\alpha) ) \to \texttt{Cell<}\kwd{list}(\alpha)\texttt{>} | 0, \\
    \texttt{replace} : & \kwd{fn}(\texttt{self}: \&\texttt{Cell<}\kwd{list}(\alpha)\texttt{>}, \texttt{l}: \kwd{list}(\alpha) ) \to \kwd{list}(\alpha) | 0 ,
\end{align*}
in the sense that the potential type $\kwd{list}(\alpha)$ is an invariant for a cell type, and operations should maintain the invariant, e.g., \verb|replace| should store another list of the same type $\kwd{list}(\alpha)$.
%
Rust's \verb|Vec| also makes use of unsafe code to allow accessing uninitialized memory.
%
At the API level, we can treat vectors as abstract dynamic arrays, which fit nicely into the AARA framework because of their amortized complexity.
%
For example, we can declare the following APIs for integer vectors: 
\begin{align*}
    \texttt{new}: & \kwd{fn}() \to \texttt{Vec<}\kwd{i32},\alpha\texttt{>} | 0, \\
    \texttt{push} : & \kwd{fn}(\texttt{self}: \&\kwd{mut}~\texttt{Vec<}\kwd{i32},\alpha\texttt{>}, \texttt{n}: \kwd{i32} ) \to () | \alpha + 4,
\end{align*}
where the resource-annotated type $\texttt{Vec<}\kwd{i32},\alpha\texttt{>}$ denotes the potential function $\mathit{len} \cdot \alpha + (4 \cdot \mathit{len} - 2 \cdot \mathit{cap})$, with $\mathit{len}$ and $\mathit{cap}$ being the length and capacity of the vector, respectively. 
%
Intuitively, the potential function states that every vector element carries $\alpha$ units of potential and we need to store $(4 \cdot \mathit{len} - 2 \cdot \mathit{cap})$ units of extra potential for vector resizing, which would consume $2 \cdot \mathit{len}$ units of resource to extend the vector's capacity when the vector becomes full.
%

Rust's implementation of \verb|Rc| uses unsafe code, so we would annotate \verb|Rc| APIs with resource-annotated types. \verb|Rc| itself does not permit mutation, so we could model its behavior as if it is a shared reference: \verb|Rc::new()| should store potentials (e.g., \verb|Rc<list(4)>|) and \verb|Rc::clone()| should split potentials (e.g., splitting \verb|Rc<list(4)>| as \verb|Rc<list(1)>| and \verb|Rc<list(3)>|). We have not yet considered multithreading, and supporting \verb|Arc| would be interesting future work.

It is non-trivial to handle cyclic data structures. Rust provides \verb|Weak| pointers to accompany \verb|Rc| pointers. However, creating cyclic data structures usually requires using interior mutability (e.g., \verb|Cell| or \verb|RefCell|). The interaction between reference counting and interior mutability seems quite non-trivial. In the future, we may adapt \citet{ESOP:Atkey10}'s work on integrating AARA with separation logic.

\paragraph{Generic types, higher-order functions (closures), and trait objects}
%
Current \rarust{} does not support generic types like \verb|List<T>|. This is not a fundamental limitation, because we can always instantiate generic types. We will spare engineering efforts to support them in the future.
%

Our work currently only considers top-level functions, but Rust does support higher-order functions and closures to enable functional programming style. 
%
Fortunately, many AARA approaches support higher-order functions~\cite{AARA-HigherOrder,POPL:HDW17,ICFP:KWR20,ICFP:KH21}.
%
Conceptually,
it would be possible to adapt AARA's methodology of handling closures to \rarust{}
by extending the type system to deal with \emph{capturing} properly.
%
One simple extension is to enforce that closures cannot consume potentials stored in captured variables; in this way, it is sound to apply a closure multiple times.
%
It would be interesting future research to investigate how the interaction of borrow mechanisms (especially mutable borrows) and closures would affect AARA-style resource analysis.

Rust supports a form of dynamic dispatch through trait objects, in which the compiler knows an object's trait but not its actual type. One possible workaround is to annotate trait methods with resource annotations and require them not to change the resource type of \verb|Self|. In this way, even though we do not know an object's type, we know how calling its trait methods affects the resource-annotated context. Another possibility is to adapt \citet{AARA-OOP}'s work on integrating AARA with objective-oriented programming. 

\paragraph{Non-linear resource bounds}
%
Both our formalization and implementation of \rarust{} currently only consider linear resource bounds, which are too restrictive to analyze real-world programs.
%
Current \rarust{} can only support pattern matching of one single variable, without primitive support for pattern matching of tuple types, because tuple types usually introduce multivariate polynomial resource bounds like $\textit{first} \times \textit{second}$. 
%
This is not a fundamental limitation because it has been shown that AARA type systems can support polynomial bounds~\cite{AARA-Poly,AARA-Poly-Multivar}, exponential bounds~\cite{AARA-Exp}, logarithmic bounds~\cite{AARA-Log,CAV:LMZ21}, and value-dependent bounds~\cite{ICFP:KWR20,PLDI:KWP19}.
%
All of those approaches amount to devising proper type annotations that specify particular kinds of potential functions and developing constraint-based type-inference algorithms.
%
We plan to spare engineering efforts to extend our prototype implementation of \rarust{} to support various classes of resource bounds in the future.

% \todo{limitation also here, and some idea to solve limitation} 
% The limitation is linear and imprecision. (1) Prototype \rarust{} can only analyze linear bound of resource consumption, as $a + b\cdot n$. However, we can extend the bound to polynomials\cite{AARA-Poly} and \cite{AARA-Poly-Multivar}. (2) Recall the weak updates in \cref{sec:overview}. It will introduce inevitable imprecision. Our implementation introduce it when merging mutable borrows. Though inevitable, it can be delayed to actual writing on borrows, using similar techniques as \cite{CapTypes}. We leave extension as future works.
\section{Related Work} \label{sec:related}

In this section, we discuss related work that has not been covered in previous sections.

\paragraph{Static resource analysis}
%
AARA is not the only approach for type-based resource analysis.
%
For example, there are approaches based on sized types~\cite{phd:Vasconcelos08,ICFP:AL17}, dependent types~\cite{LICS:LG11,POPL:LP13,POPL:RGG21}, refinement types~\cite{POPL:HVH20,POPL:CBG17,ESOP:CGA15,OOPSLA:WWC17},
recurrence extraction~\cite{POPL:KML20,ICFP:DLR15}, logical frameworks~\cite{POPL:NSG22,POPL:GNS24}, and annotated types~\cite{POPL:CW00,POPL:Danielsson08}.
%
None of the aforementioned type systems considered supporting heap-manipulating programs with Rust's borrow mechanisms.
%
Besides type systems, there are also static resource-analysis techniques based on
defunctionalization~\cite{ICFP:ALM15}, recurrence relations~\cite{JAR:AAG11,TACAS:AFR15,APLAS:FH14,PLDI:BCK20,POPL:KBC19,PLDI:KBB17},
term rewriting~\cite{RTA:AM13,TACAS:BEG14,IJCAR:FNH16,JAR:NEG13}, and
abstract interpretation~\cite{SAS:ZSG11,LPAR:BHH10,CAV:SZV14,kn:DHW07,misc:fbinfer20,SAS:AG12}.
%
Some of the aforementioned techniques work on imperative programs (e.g., C programs) with arrays,
such as C4B~\cite{PLDI:CHS15}, SPEED~\cite{POPL:GMC09}, COSTA~\cite{JAR:AAG11}, ICRA~\cite{PLDI:KBB17}, etc., but none of them considered exploiting Rust's borrow mechanisms in the design.
%
It would be interesting future research direction to investigate how different static resource-analysis techniques can benefit from Rust's safety guarantees.

% Automatic Amortized Resource Analysis(AARA) first presented by \cite{AARA-Linear}, enriched type system  with resource annotations for a first-order functional language, to derive linear upper bounds on the heap-memory usage of list via linear programming. Derivative works support non-linear worst-case upper bound, like polynomial \cite{AARA-Poly}, multivariate polynomial \cite{AARA-Poly-Multivar}, exponential \cite{AARA-Exp} and logarithmic \cite {AARA-Log}. AARA can also support features like higher-order functions \cite{AARA-HigherOrder}, algebraic data types \cite{AARA-ADT} and general recursive data types \cite{AARA-GeneralRecursive}. Original AARA method does not support imperative mutation, whereas our work features AARA mainly with it and Rust borrow mechanism. Compared with AARA derivative works, our formalization is currently limited to single variate linear upper bound, however in our view, our extension is orthogonal to other features. Therefore it can be easy, as future works, to support complex recursive data structures and various upper bounds in Rust resource analyzer.

\paragraph{Graded type system}
Graded types \cite{Granule} introduce a graded modality for associating types with elements from a resource algebra. 
%
A graded type system can account for program variables' exact usages, security levels, and potentials (conceptually). 
%
Graded types seem to provide a more general mechanism than AARA types to reason about more general resources. 
%
On the other hand, our work focuses on how Rust's borrow mechanisms can aid resource analysis and chooses AARA as the starting point because AARA admits efficient type inference via linear programming. 

\paragraph{Program verification for Rust}
%
RustBelt~\cite{RustBelt} pioneers a line of work to use semantic typing and separation logic to verify Rust programs with both safe and unsafe code.
%
% Stacked Borrow \cite{StackedBorrow} presents an operational semantics for memory accesses in Rust, and defines an aliasing discipline to cooperate raw pointers and borrows when verifying. 
%
RustHorn~\cite{RustHorn} uses prophecy variables to model the future values of mutable borrows and proposes an automatic verification algorithm based on constrained Horn clauses.
%
% Another approach is to leverage Rust type system, restricting on safe Rust with borrow mechanism. 
Aeneas~\cite{Aeneas}---which inspires our development of RABC and supports our prototype implementation---uses LLBC to translate Rust programs into equivalent pure functional programs via symbolic execution.
%
Such pure functional programs can be ported into theorem provers such as Coq and F* to enable verification of functional correctness.
%
Prusti~\cite{OOPSLA:AMP19} also leverages Rust's advanced type system to devise a modular and automated verification approach.
%
\citet{master:Engel21} proposed a method to verify user-provided asymptotic resource bounds in Prusti.
%
In contrast, our work focuses on the automatic inference of concrete resource bounds via a type system.
%
% Refinement types based method has also been applied to verification, like Flux \cite{Flux} proposing a liquid type system and RefinedRust \cite{RefinedRust} exploiting prophecy variables. Our work follows type system based approach and focuses on analysis instead of verification. We admits safe restriction of Rust from Stacked Borrow, resembling Aeneas working on borrow calculus, but unnecessarily translating to functional program, straightforward analyzing resource consumption. We also adopt prophecy variables to model mutable borrows.

% \todo{MAY DELETE THIS merging? might similar to static analysis, when coming across loop}
\section{Conclusion}

Subgroup analysis is an important, yet under-utilized tool in data science.
Our results suggest that combining algorithm-generated, rule-based insights with human intuition and experimentation in an interactive workflow can help practitioners develop a thorough understanding of complex datasets.
By implementing these interactions in a lightweight notebook-based tool, we hope to lower the barrier for data scientists to try subgroup discovery and to curate unexpected, interesting subpopulations in their data.
Divisi is available as an open-source package so that data scientists and HCI researchers can build on this work, helping to make exploratory subgroup analysis more feasible for a wider range of contexts.

\section*{Data-Availability Statement}
The source code of the \rarust{} implementation and benchmarks referenced in \cref{sec:impl} are publicly available in the Zenodo \cite{RaRustArtifact}. The artifact contains necessary scripts and step-by-step guides to reproduce the experimental results.

\begin{acks}
We are grateful to Xuanyu Peng for the early discussion and investigation. We would like to thank the anonymous reviewers for their valuable feedback on our paper and the anonymous artifact reviewers for their suggestions for our artifact.
\end{acks}



\bibliographystyle{ACM-Reference-Format}
\bibliography{main,db}

\newpage
\appendix
\section{Judgements}
\centering
\judgement{Expression Evaluation}{$V\vDash e \rightsquigarrow v$}
    \begin{mathpar}
    \inferrule*[Right=\rulename{V-Ev-Int}]
    {~}
    {V\vDash \kwd{n}_\text{i32} \rightsquigarrow \kwd{n}_\text{i32}}
    \and
    \inferrule*[Right=\rulename{V-Ev-Op}]
    {V\vDash e_1 \rightsquigarrow \kwd{n}_1
    \\ V\vDash e_2 \rightsquigarrow \kwd{n}_2}
    {V\vDash e_1~\kwd{op}~e_2 \rightsquigarrow \kwd{n}_1~\kwd{op}~\kwd{n}_2}
    \\
    \inferrule*[Right=\rulename{V-Ev-True}]
    {~}
    {V\vDash \kwd{true} \rightsquigarrow \kwd{true}}
    \and
    \inferrule*[Right=\rulename{V-Ev-False}]
    {~}
    {V\vDash \kwd{false} \rightsquigarrow \kwd{false}}
    \\
    \inferrule*[Right=\rulename{V-Ev-Nil}]
    {~}
    {V\vDash \kwd{nil} \rightsquigarrow \kwd{nil}}
    \and
    \inferrule*[Right=\rulename{V-Ev-Box}]
    {V\vDash e \rightsquigarrow v}
    {V\vDash \kwd{box}(e) \rightsquigarrow \kwd{box}(v)}
    \\
    \inferrule*[Right=\rulename{V-Ev-Copy}]
    {V\vDash p \rightsquigarrow v}
    {V\vDash \kwd{copy}~p \rightsquigarrow v}
    \and
    \inferrule*[Right=\rulename{V-Ev-Move}]
    {V\vDash p \rightsquigarrow v}
    {V\vDash \kwd{move}~p \rightsquigarrow v}
    \and
    \inferrule*[Right=\rulename{V-Ev-Borrow}]
    {V\vDash p \rightsquigarrow v}
    {V\vDash \&^{\kwd{s}/\kwd{m}/\kwd{2}} p \rightsquigarrow \&(p, v)}
\end{mathpar}

\centering
\judgement{Statement Execution}{$V\vDash e \rightsquigarrow^\delta \Dashv V'$}
\begin{mathpar}
    \inferrule*[Right=\rulename{V-Ex-Ret}]
    {~}
    {V\vDash \kwd{return} \rightsquigarrow^0 \Dashv V}
    \and
    \inferrule*[Right=\rulename{V-Ex-Seq}]
    {V\vDash s_1\rightsquigarrow^{\delta_1}\Dashv V'
    \\ V'\vDash s_2\rightsquigarrow^{\delta_2}\Dashv V''}
    {V\vDash s_1; s_2\rightsquigarrow^{\delta_1+\delta_2}\Dashv V''}
    \\
    \inferrule*[Right=\rulename{V-Ex-Tick}]
    {~}
    {V\vDash \kwd{tick}(\delta)\rightsquigarrow^\delta \Dashv V}
    \and
    \inferrule*[Right=\rulename{V-Ex-Drop}]
    {~}
    {V\vDash \kwd{drop}~p\rightsquigarrow^0\Dashv V}
    \\
    \inferrule*[Right=\rulename{V-Ex-Assign}]
    {V\vDash e \rightsquigarrow v'
    \\ \VWt{V}{p}{v'}{V'}}
    {V\vDash p\from e\rightsquigarrow^0 \Dashv V'}
    \\
    \inferrule*[Right=\rulename{V-Ex-Cons}]
    {V\vDash e_1 \rightsquigarrow v_1
    \\ V\vDash e_2 \rightsquigarrow v_2
    \\ \VWt{V}{p}{\kwd{cons}(v_1, v_2)}{V'} }
    {V\vDash p\from \kwd{cons}(e_1, e_2)\rightsquigarrow^0 \Dashv V'}
    \\
    \inferrule*[Right=\rulename{V-Ex-IfT}]
    {V\vDash p\rightsquigarrow \kwd{true}
    \\ V\vDash s_1\rightsquigarrow^\delta \Dashv V'}
    {V\vDash \kwd{if}~ p ~\kwd{then}~ s_1 ~\kwd{else}~ s_2 ~\kwd{end} \rightsquigarrow^\delta \Dashv V'}
    \and
    \inferrule*[Right=\rulename{V-Ex-IfF}]
    {V\vDash p\rightsquigarrow \kwd{false}
    \\ V\vDash s_2\rightsquigarrow^\delta \Dashv V'}
    {V\vDash \kwd{if}~ p ~\kwd{then}~ s_1 ~\kwd{else}~ s_2 ~\kwd{end} \rightsquigarrow^\delta \Dashv V'}
    \\
    \inferrule*[Right=\rulename{V-Ex-Mat-Nil}]
    {V\vDash p\rightsquigarrow \kwd{nil}
    \\ V\vDash s_1\rightsquigarrow^\delta \Dashv V'}
    {V\vDash \kwd{match}~ p ~ \{\kwd{nil}\mapsto s_1, \kwd{cons}(x_\text{hd}, x_\text{tl})\mapsto s_2\} \rightsquigarrow^\delta \Dashv V'}

    \inferrule*[Right=\rulename{V-Ex-Mat-Cons}]
    {V\vDash p\rightsquigarrow \kwd{cons}(hd, tl)
    \\ \VWt{V}{p}{\bot}{V_1}
    \\ \VWt{V_1}{x_\text{hd}}{hd}{V_2}
    \\ \VWt{V_2}{x_\text{tl}}{tl}{V_\text{b}}
    \\\\ V_\text{b}\vDash s_2\rightsquigarrow^\delta \Dashv V'_\text{b}
    \\ V'_\text{b}\vDash x_\text{hd}\rightsquigarrow hd'
    \\ V'_\text{b}\vDash x_\text{tl}\rightsquigarrow tl'
    \\\\ \VWt{V'_\text{b}}{x_\text{hd}}{\bot}{V'_1}
    \\ \VWt{V'_1}{x_\text{tl}}{\bot}{V'_2}
    \\ \VWt{V'_2}{p}{\kwd{cons}(hd', tl')}{V'} 
    }
    {V\vDash \kwd{match}~ p ~ \{\kwd{nil}\mapsto s_1, \kwd{cons}(x_\text{hd}, x_\text{tl})\mapsto s_2\} \rightsquigarrow^\delta \Dashv V'}
    \\
    
    \inferrule*[Right=\rulename{V-Ex-App}]
    {\kwd{fn}~ f ~(\vec{x}_\text{param}:\vec{t}_\text{param}, \vec{x}_\text{local}:\vec{t}_\text{local}, x_\text{ret}:t_\text{ret}) \{~ s ~\}
    \\ V\vDash \vec{e}\rightsquigarrow \vec{v}
    \\\\ \VWt{V}{\vec{x}_\text{param}}{\vec{v}}{V_1}
    \\ \VWt{V_1}{\vec{x}_\text{local}}{\bot}{V_2}
    \\ \VWt{V_2}{x_\text{ret}}{\bot}{V_\text{b}}
    \\\\ V_\text{b}\vDash s\rightsquigarrow^\delta \Dashv V'_\text{b}
    \\ V'_\text{b}\vDash x_\text{ret} \rightsquigarrow v_\text{ret}
    \\\\ \VWt{V'_\text{b}}{\vec{x}_\text{param}}{\bot}{V'_1}
    \\ \VWt{V'_1}{\vec{x}_\text{local}}{\bot}{V'_2}
    \\ \VWt{V'_2}{x_\text{ret}}{\bot}{V'_3}
    \\ \VWt{V'_3}{p}{v_\text{ret}}{V'}
    } 
    {V\vDash p\from f(\vec{e})\rightsquigarrow^\delta \Dashv V'}
\end{mathpar}

\centering
\judgement{Enrich}{$\textit{enrich}~ t ~\textit{as}~ \tau$}
\begin{mathpar}
    \inferrule*[Right=\rulename{Enrich-Int}]
    {~}
    {\textit{enrich}~ \kwd{i32} ~\textit{as}~\kwd{i32}}
    \and
    \inferrule*[Right=\rulename{Enrich-Bool}]
    {~}
    {\textit{enrich}~ \kwd{bool} ~\textit{as}~ \kwd{bool}}
    \\

    \inferrule*[Right=\rulename{Enrich-List}]
    {\alpha~\text{fresh}}
    {\textit{enrich}~ \kwd{list} ~\textit{as}~  \kwd{list}(\alpha)}
    \and
    \inferrule*[Right=\rulename{Enrich-Box}]
    {\alpha~\text{fresh}}
    {\textit{enrich}~ \kwd{box}(\kwd{list}) ~\textit{as}~ \kwd{box}(\kwd{list}(\alpha))}
    \\
    
    \inferrule*[Right=\rulename{Enrich-Shared}]
    {\textit{enrich}~ t ~\textit{as}~ \tau}
    {\textit{enrich}~ \&^\kwd{s}(t) ~\textit{as}~ \&^\kwd{s}(\tau)}
    \\
    \inferrule*[Right=\rulename{Enrich-Mutable}]
    {\textit{enrich}~ t ~\textit{as}~ \tau_\text{c}
    \\ \textit{enrich}~ t ~\textit{as}~ \tau_\text{p}
    }
    {\textit{enrich}~ \&^\kwd{m}(t) ~\textit{as}~ \&^\kwd{m}(\tau_\text{c}, \tau_\text{p})}
\end{mathpar}

\centering
\judgement{Sharing}{$\textit{share}~ \tau ~\textit{as}~\tau_1, \tau_2$}
\begin{mathpar}
    \inferrule*[Right=\rulename{Share-Int}]
    {~}
    {\textit{share}~ \kwd{i32} ~\textit{as}~\kwd{i32}, \kwd{i32}}
    \and
    \inferrule*[Right=\rulename{Share-Bool}]
    {~}
    {\textit{share}~ \kwd{bool} ~\textit{as}~\kwd{bool}, \kwd{bool}}
    \\
    \inferrule*[Right=\rulename{Share-List}]
    {\alpha_1, \alpha_2 ~\text{fresh}
    \\\alpha = \alpha_1 + \alpha_2}
    {\textit{share}~ \kwd{list}(\alpha) ~\textit{as}~\kwd{list}(\alpha_1), \kwd{list}(\alpha_2)}
    \\
    \inferrule*[Right=\rulename{Share-Box}]
    {\textit{share}~ \tau ~\textit{as}~\tau_1, \tau_2}
    {\textit{share}~ \kwd{box}(\tau) ~\textit{as}~\kwd{box}(\tau_1), \kwd{box}(\tau_2)}
    \\
    \inferrule*[Right=\rulename{Share-Shared}]
    {\textit{share}~ \tau ~\textit{as}~\tau_1, \tau_2}
    {\textit{share}~ \&^\kwd{s}(\tau) ~\textit{as}~ \&^\kwd{s}(\tau_1), \&^\kwd{s}(\tau_2)}
\end{mathpar}

\centering
\judgement{Prophesying}{$\textit{prophesy}~ \tau_\text{c} ~\textit{as}~\tau_\text{p}$}
\begin{mathpar}
    \inferrule*[Right=\rulename{Prophesy-Int}]
    {~}
    {\textit{prophesy}~ \kwd{i32} ~\textit{as}~ \kwd{i32}}
    \and
    \inferrule*[Right=\rulename{Prophesy-Bool}]
    {~}
    {\textit{prophesy}~ \kwd{bool} ~\textit{as}~ \kwd{bool}}
    \\
    \inferrule*[Right=\rulename{Prophesy-List}]
    {\alpha_\text{p}~\text{fresh}}
    {\textit{prophesy}~ \kwd{list}(\alpha) ~\textit{as}~ \kwd{list}(\alpha_\text{p})}
    \\
    \inferrule*[Right=\rulename{Prophesy-Box}]
    {\textit{prophesy}~ \tau ~\textit{as}~ \tau_\text{p} }
    {\textit{prophesy}~ \kwd{box}(\tau) ~\textit{as}~ \kwd{box}(\tau_\text{p})}
    \\
    \inferrule*[Right=\rulename{Prophesy-Shared}]
    {\textit{prophesy}~ \tau ~\textit{as}~ \tau_\text{p}}
    {\textit{prophesy}~ \&^\kwd{s}(\tau) ~\textit{as}~ \&^\kwd{s}(\tau_\text{p})}
    \\
    \inferrule*[Right=\rulename{Prophesy-Mutable}]
    {\textit{prophesy}~ \tau_\text{c} ~\textit{as}~ \tau_\text{cp}
    \\ \textit{prophesy}~ \tau_\text{p} ~\textit{as}~ \tau_\text{pp}
    }
    {\textit{prophesy}~ \&^\kwd{m}(\tau_\text{c}, \tau_\text{p}) ~\textit{as}~ \&^\kwd{m}(\tau_\text{cp}, \tau_\text{pp})}
\end{mathpar}

\centering
\judgement{Meet/Join}{$\tau_1\cap\tau_2 / \tau_1\cup\tau_2$}
\begin{mathpar}    
    \inferrule*[Right=Meet-Int]
    {~}
    {\kwd{i32}\cap\kwd{i32}=\kwd{i32}}
    \and
    \inferrule*[Right=Join-Int]
    {~}
    {\kwd{i32}\cup\kwd{i32}=\kwd{i32}}
    \\
    \inferrule*[Right=Meet-Bool]
    {~}
    {\kwd{bool}\cap\kwd{bool}=\kwd{bool}}
    \and
    \inferrule*[Right=Join-Bool]
    {~}
    {\kwd{bool}\cup\kwd{bool}=\kwd{bool}}
    \\
    
    \inferrule*[Right=Meet-List]
    {\min(\alpha_1, \alpha_2)=\alpha}
    {\kwd{list}(\alpha_1)\cap\kwd{list}(\alpha_2)=\kwd{list}(\alpha)}
    \and
    \inferrule*[Right=Join-List]
    {\max(\alpha_1, \alpha_2)=\alpha}
    {\kwd{list}(\alpha_1)\cup\kwd{list}(\alpha_2)=\kwd{list}(\alpha)}
    \\
    
    \inferrule*[Right=Meet-Box]
    {\tau_1 \cap \tau_2=\tau}
    {\kwd{box}(\tau_1)\cap\kwd{box}(\tau_2)=\kwd{box}(\tau)}
    \and
    \inferrule*[Right=Join-Box]
    {\tau_1 \cup \tau_2=\tau}
    {\kwd{box}(\tau_1)\cup\kwd{box}(\tau_2)=\kwd{box}(\tau)}
    \\
    
    \inferrule*[Right=Meet-Shared]
    {\tau_1 \cap \tau_2=\tau}
    {\&^\kwd{s}(\tau_1)\cap\&^\kwd{s}(\tau_2)=\&^\kwd{s}(\tau)}
    \and
    \inferrule*[Right=Join-Shared]
    {\tau_1 \cup \tau_2=\tau}
    {\&^\kwd{s}(\tau_1)\cup\&^\kwd{s}(\tau_2)=\&^\kwd{s}(\tau)}
    \\

    \inferrule*[Right=Meet-Mutable]
    {\tau_{\text{c}, 1} \cap \tau_{\text{c}, 2}=\tau_\text{c}
    \\ \tau_{\text{p}, 1} \cup \tau_{\text{p}, 2}=\tau_\text{p}
    \\ \vdash \&^\kwd{m}(\tau_{\text{c}, 1}, \tau_{\text{p}, 1})
    \\ \vdash \&^\kwd{m}(\tau_{\text{c}, 2}, \tau_{\text{p}, 2})
    }
    {\&^\kwd{m}(\tau_{\text{c}, 1}, \tau_{\text{p}, 1})\cap\&^\kwd{m}(\tau_{\text{c}, 2}, \tau_{\text{p}, 2})=\&^\kwd{m}(\tau_\text{c}, \tau_\text{p})}
    \\
    \inferrule*[Right=Join-Mutable]
    {\tau_{\text{c}, 1} \cup \tau_{\text{c}, 2}=\tau_\text{c}
    \\ \tau_{\text{p}, 1} \cap \tau_{\text{p}, 2}=\tau_\text{p}
    \\ \vdash \&^\kwd{m}(\tau_{\text{c}, 1}, \tau_{\text{p}, 1})
    \\ \vdash \&^\kwd{m}(\tau_{\text{c}, 2}, \tau_{\text{p}, 2})
    }
    {\&^\kwd{m}(\tau_{\text{c}, 1}, \tau_{\text{p}, 1})\cup\&^\kwd{m}(\tau_{\text{c}, 2}, \tau_{\text{p}, 2})=\&^\kwd{m}(\tau_\text{c}, \tau_\text{p})}
\end{mathpar}

\centering
\judgement{Typing Expression}{$\Gamma\vdash e \hookrightarrow \tau\dashv\Gamma'$}
\begin{mathpar}
    \inferrule*[Right=\rulename{$\Gamma$-Ev-Int}]
    {~}
    {\Gamma\vdash \kwd{n}_\text{i32} \hookrightarrow \kwd{i32}\dashv\Gamma}
    \and
    \inferrule*[Right=\rulename{$\Gamma$-Ev-Op}]
    {\Gamma\vdash e_1\hookrightarrow \kwd{i32}\dashv\Gamma
    \\ \Gamma\vdash e_2\hookrightarrow \kwd{i32}\dashv\Gamma}
    {\Gamma\vdash e_1~\kwd{op}~e_2 \hookrightarrow \kwd{i32}\dashv\Gamma}
    \\
    
    \inferrule*[Right=\rulename{$\Gamma$-Ev-True}]
    {~}
    {\Gamma\vdash \kwd{true} \hookrightarrow \kwd{bool}\dashv\Gamma}
    \and
    \inferrule*[Right=\rulename{$\Gamma$-Ev-False}]
    {~}
    {\Gamma\vdash \kwd{false} \hookrightarrow \kwd{bool}\dashv\Gamma}
    \\
    \inferrule*[Right=\rulename{$\Gamma$-Ev-Copy}]
    {\Gamma\vdash p \hookrightarrow \tau
    \\ \tau = \kwd{i32} ~\text{or}~ \tau = \kwd{bool} }
    {\Gamma\vdash \kwd{copy}~p \hookrightarrow \tau\dashv\Gamma}
    \and
    \inferrule*[Right=\rulename{$\Gamma$-Ev-Box}]
    {\Gamma\vdash e \hookrightarrow \tau\vdash\Gamma'}
    {\Gamma\vdash \kwd{box}(e) \hookrightarrow \kwd{box}(\tau)\vdash\Gamma'}
    \\

    \inferrule*[Right=\rulename{$\Gamma$-Ev-Nil}]
    {\alpha ~\text{fresh}}
    {\Gamma\vdash \kwd{nil} \hookrightarrow \kwd{list}(\alpha)\vdash\Gamma}
    \and
    \inferrule*[Right=\rulename{$\Gamma$-Ev-Move}]
    {\Gamma\vdash p \hookrightarrow \tau
    \\ \GWt{\Gamma}{p}{\bot}{\Gamma'}}
    {\Gamma\vdash \kwd{move}~p \hookrightarrow \tau\dashv\Gamma'}
    \\
    
    \inferrule*[Right=\rulename{$\Gamma$-Ev-Shared}]
    {\Gamma\vdash p \hookrightarrow \tau
    \\ \textit{share}~ \tau ~\textit{as}~\tau_1, \tau_2
    \\ \GWt{\Gamma}{p}{\tau_1}{\Gamma'}
    }
    {\Gamma\vdash \&^\kwd{s}~p \hookrightarrow \&^\kwd{s}(\tau_2)\dashv\Gamma'}
    \\
    
    \inferrule*[Right=\rulename{$\Gamma$-Ev-Mutable}]
    {\Gamma\vdash p \hookrightarrow \tau
    \\ \textit{prophesy}~ \tau ~\textit{as}~ \tau_\text{p} 
    \\ \GWt{\Gamma}{p}{\tau_\text{p}}{\Gamma'}
    }
    {\Gamma\vdash \&^\kwd{m}~p \hookrightarrow \&^\kwd{m}(\tau, \tau_\text{p})\dashv\Gamma'}
\end{mathpar}

\centering
\judgement{Typing Statements}{$\Gamma\vdash s \hookrightarrow^\delta \dashv\Gamma'$}
\begin{mathpar}
    \inferrule*[Right=\rulename{$\Gamma$-Ex-Ret}]
    {~}
    {\Gamma\vdash \kwd{return} \hookrightarrow^0\vdash\Gamma}
    \and
    \inferrule*[Right=\rulename{$\Gamma$-Ex-Seq}]
    {\Gamma_1\vdash s_1\hookrightarrow^{\delta_1}\dashv\Gamma_2
    \\ \Gamma_2\vdash s_2\hookrightarrow^{\delta_2}\dashv\Gamma_3}
    {\Gamma_1\vdash s_1; s_2\hookrightarrow^{\delta_1+\delta_2}\dashv\Gamma_3}
    \\
    
    \inferrule*[Right=\rulename{$\Gamma$-Ex-Tick}]
    {~}
    {\Gamma\vdash\kwd{tick}(\delta)\hookrightarrow^\delta\vdash\Gamma}
    \and
    \inferrule*[Right=\rulename{$\Gamma$-Ex-Drop}]
    {\Gamma\vdash p\hookrightarrow \tau
    \\ \vdash \tau
    \\ \GWt{\Gamma}{p}{\bot}{\Gamma'}
    }
    {\Gamma\vdash \kwd{drop}~p \hookrightarrow^0\dashv \Gamma'}
    \\

    \inferrule*[Right=\rulename{$\Gamma$-Ex-Assign}]
    {\Gamma\vdash e\hookrightarrow \tau'\dashv\Gamma_1
    \\ \Gamma_1\vdash p \hookrightarrow \tau
    \\ \vdash \tau
    \\ \GWt{\Gamma_1}{p}{\tau'}{\Gamma'}}
    {\Gamma\vdash p\from e \hookrightarrow^0\dashv\Gamma'}
    \\

    \inferrule*[Right=\rulename{$\Gamma$-Ex-Cons}]
    {\Gamma\vdash e_1\hookrightarrow \kwd{i32} \dashv \Gamma_1
    \\ \Gamma_1\vdash e_2\hookrightarrow \kwd{box}(\kwd{list}(\alpha'))\dashv\Gamma_2
    \\ \GWt{\Gamma_2}{p}{\kwd{list}(\alpha')}{\Gamma'}}
    {\Gamma\vdash p\from \kwd{cons}(e_1, e_2)\hookrightarrow^{\alpha'}\dashv\Gamma'}
    \\

    \inferrule*[Right=\rulename{$\Gamma$-Ex-If}]
    {\Gamma\vdash p\hookrightarrow \kwd{bool}
    \\ \Gamma\vdash s_1\hookrightarrow^{\delta_1}\dashv\Gamma_1
    \\ \Gamma\vdash s_2\hookrightarrow^{\delta_2}\dashv\Gamma_2
    \\ \max(\delta_1, \delta_2)=\delta
    \\ \Gamma_1\sqcap\Gamma_2=\Gamma' }
    {\Gamma\vdash \kwd{if}~ p ~\kwd{then}~ s_1 ~\kwd{else}~ s_2 ~\kwd{end} \hookrightarrow^\delta \dashv\Gamma'}
    \\
    \inferrule*[Right=\rulename{$\Gamma$-Ex-Mat}]
    {\Gamma\vdash p\hookrightarrow \kwd{list}(\alpha)
    \\ \Gamma\vdash s_1\hookrightarrow^{\delta_1}\dashv\Gamma_1
    \\\\ \GWt{\Gamma}{p}{\bot}{\Gamma_{\text{b}, 1}}
    \\ \GWt{\Gamma_{\text{b}, 1}}{x_\text{hd}}{\kwd{i32}}{\Gamma_{\text{b}, 2}}
    \\ \GWt{\Gamma_{\text{b}, 2}}{x_\text{tl}}{\kwd{box}(\kwd{list}(\alpha))}{\Gamma_\text{b}}
    \\ \Gamma_\text{b}\vdash s_2\hookrightarrow^{\delta_2}\dashv\Gamma'_\text{b}
    \\\\ \Gamma'_\text{b}\vdash x_\text{tl}\hookrightarrow \kwd{list}(\beta)
    \\ \GWt{\Gamma'_\text{b}}{x_\text{hd}}{\bot}{\Gamma'_{\text{b}, 1}}
    \\ \GWt{\Gamma'_{\text{b}, 1}}{x_\text{tl}}{\bot}{\Gamma'_{\text{b}, 2}}
    \\ \GWt{\Gamma'_{\text{b}, 2}}{p}{\kwd{list}(\beta)}{\Gamma_2}
    \\\\ \max(\delta_1, \delta_2-(\alpha-\beta))=\delta
    \\ \Gamma_1\sqcap\Gamma_2=\Gamma'}
    {\Gamma\vdash \kwd{match}~ p ~ \{\kwd{nil}\mapsto s_1, \kwd{cons}(x_\text{hd}, x_\text{tl})\mapsto s_2\} \hookrightarrow^\delta \dashv\Gamma'}
    \\

    \inferrule*[Right=\rulename{$\Gamma$-Ex-App}]
    {\text{fn}~ f ~(\vec{x}_\text{param}:\vec{t}_\text{param}, \vec{x}_\text{local}:\vec{t}_\text{local}, x_\text{ret}:t_\text{ret}) \{~ s ~\}
    \\\\ \vdash f \Leftarrow (\Gamma_f, \delta_f)
    \\ \Gamma_f\vdash x_\text{ret} \hookrightarrow \tau_\text{ret}, (\forall x_i\in\vec{x}_\text{param}, i=1, ..., n) \Gamma_f \vdash x_i \hookrightarrow \tau_{\text{param}, i}
    \\\\ \Gamma_0=\Gamma, (\forall e_i\in \vec{e}, i=1, ..., n) \Gamma_{i-1}\vdash e_i\hookrightarrow\tau_{\text{arg}, i}\dashv\Gamma_i
    \\ (\forall i=1,..,n)~ \tau_{\text{param}, i} = \tau_{\text{arg}, i}
    \\ \Gamma_n \vdash p \hookrightarrow \tau
    \\ \vdash \tau
    \\ \GWt{\Gamma_n}{p}{\tau_\text{ret}}{\Gamma'}
    }
    {\Gamma\vdash p\from f(\vec{e})\hookrightarrow^{\delta_f}\dashv\Gamma'}
\end{mathpar}

\centering
\judgement{Function Analysis}{$\vdash f \Leftarrow (\Gamma_f, \delta_f)$}
\begin{mathpar}
    \inferrule
    {\text{fn}~ f ~(\vec{x}_\text{param}:\vec{t}_\text{param}, \vec{x}_\text{local}:\vec{t}_\text{local}, x_\text{ret}:t_\text{ret}) \{~ s ~\}
    \\  \vdash f \Rightarrow (\Gamma_f, \delta_f)
    \\ \Gamma_f\vdash s\hookrightarrow^\delta\dashv\Gamma'_f
    \\\\ \forall x \in \textbf{dom}(\Gamma'_f), \vdash \Gamma'_f(x)
    \\ \Gamma'_f \vdash x_\text{ret} \hookrightarrow \tau'_\text{ret}
    \\ \Gamma_f \vdash x_\text{ret} \hookrightarrow \tau_\text{ret}
    \\ \tau'_\text{ret} = \tau_\text{ret}
    \\ \delta = \delta_f}
    {\vdash f \Leftarrow (\Gamma_f, \delta_f)}
\end{mathpar}
    

\centering
\judgement{Potential Function}{$\phi(v:\tau)$}
\begin{mathpar}
    \inferrule*[Right=$\phi$-Bot]
    {~}
    {\phi(\_:\bot)=0}
    \and
    \inferrule*[Right=$\phi$-Int]
    {~}
    {\phi(\kwd{n}_\text{i32}:\kwd{i32})=0}
    \\
    \inferrule*[Right=$\phi$-True]
    {~}
    {\phi(\kwd{true}:\kwd{bool})=0}
    \and
    \inferrule*[Right=$\phi$-False]
    {~}
    {\phi(\kwd{false}:\kwd{bool})=0}
    \\
    
    \inferrule*[Right=$\phi$-Nil]
    {~}
    {\phi(\kwd{nil}:\kwd{list}(\alpha))=0}
    \\
    \inferrule*[Right=$\phi$-Cons]
    {~}
    {\phi(\kwd{cons}(\kwd{n}_\text{i32}, \kwd{box}(lv)):\kwd{list}(\alpha))=\alpha+\phi(\kwd{box}(lv):\kwd{box}(\kwd{list}(\alpha)))}
    \\
    \inferrule*[Right=$\phi$-Box]
    {~}
    {\phi(\kwd{box}(lv):\kwd{box}(\kwd{list}(\alpha)))=\phi(lv:\kwd{list}(\alpha))}
    \\
    \inferrule*[Right=$\phi$-Shared]
    {~}
    {\phi(\&(\_, v):\&^\kwd{s}(\tau))=\phi(v:\tau)}
    \\
    \inferrule*[Right=$\phi$-Mutable]
    {~}
    {\phi(\&(\_, v):\&^\kwd{m}(\tau_\text{c}, \tau_\text{p}))=\phi(v:\tau_\text{c})-\phi(v:\tau_\text{p})}
\end{mathpar}
\newpage
\section{Proof of Soundness}
\label{sec:proof}

\begin{lemma}
    Potential is non-negative and keeps subtyping:
    \begin{mathpar}
    \inferrule
    {  \vdash \tau_1
    \\ \vdash \tau_2
    \\ \tau_1 \preceq \tau_2
    }
    { 0 \leq \phi(v:\tau_1) \leq \phi(v:\tau_2)}
    \end{mathpar}
\end{lemma}
\begin{proof}
    First define the size of rich types structural inductively:
    \begin{align*}
    \textbf{size} &: \textbf{RichType} \to \NN \\
    &|~ \bot ~|~ \kwd{i32} ~|~ \kwd{bool} ~|~ \kwd{list}(\_) ~|~ \kwd{box}(\kwd{list}(\_))\mapsto 0 \\
    &|~ \&^\kwd{s}(\tau) \mapsto \textbf{size}(\tau) + 1 \\
    &|~ \&^\kwd{m}(\tau_\text{c}, \tau_\text{p}) \mapsto \textbf{size}(\tau_\text{c}) + \textbf{size}(\tau_\text{p}) + 1
    \end{align*}
    Prove by induction on $\textbf{size}(\tau_1) + \textbf{size}(\tau_2)$, and case on $\tau_1\preceq\tau_2$:
    \begin{enumerate}
        \item {\rulename{S-Bot} $\tau_1=\bot$: \textit{exfalso} due to no derivation for $\vdash \bot$; }
        \item {\rulename{S-Int}, \rulename{S-Bool} : $0 = \phi(v:\tau_1) = \phi(v:\tau_2)$ by definition;}
        \item {\rulename{S-List}, \rulename{S-Box} : $\phi(v:\tau_1) = \alpha_1\cdot n \leq \alpha_2\cdot n$, where $\alpha_1 \leq \alpha_2$ from $\tau_1\preceq \tau_2$ and $n$ is length of $v$;}
        \item {\rulename{S-Shared} : $\tau_1=\&^\kwd{s}(\tau'_1), \tau_2=\&^\kwd{s}(\tau'_2)$ by induction hypothesis on $\tau'_1, \tau'_2$, because $\textbf{size}(\tau'_1)+\textbf{size}(\tau'_2) < \textbf{size}(\tau_1) + \textbf{size}(\tau_2) = \textbf{size}(\tau'_1) + \textbf{size}(\tau'_2) + 2$, $\tau'_1\preceq\tau'_2$ from $\tau_1\preceq\tau_2$ and $\vdash \tau'_1, \vdash \tau'_2$ from $\vdash \tau_1, \vdash \tau_2$;}
        \item {\rulename{S-Mutable} : $\tau_1=\&^\kwd{m}(\tau_{\text{c}, 1}, \tau_{\text{p}, 1}), \tau_2=\&^\kwd{m}(\tau_{\text{c}, 2}, \tau_{\text{p}, 2})$, $v=\&(\_, v')$
        \begin{enumerate}
            \item {$0\leq\phi(v:\tau_1) = \phi(v':\tau_{\text{c}, 1}) - \phi(v':\tau_{\text{p}, 1})$ by induction hypothesis on $\tau_{\text{p}, 1}, \tau_{\text{c}, 1}$, because $\textbf{size}(\tau_{\text{p}, 1})+\textbf{size}(\tau_{\text{c}, 1}) < \textbf{size}(\tau_1)+\textbf{size}(\tau_2)$ and $\tau_{\text{p}, 1}\preceq \tau_{\text{c}, 1}, \vdash \tau_{\text{c}, 1}, \vdash \tau_{\text{p}, 1}$ from $\vdash \tau_1$;}
            \item {$\phi(v:\tau_1)\leq\phi(v:\tau_2)$, i.e. $\phi(v:\tau_{\text{c}, 1})-\phi(v:\tau_{\text{p}, 1})\leq \phi(v:\tau_{\text{c}, 2})-\phi(v:\tau_{\text{p}, 2})$ if and only if $\phi(v:\tau_{\text{c}, 1})\leq \phi(v:\tau_{\text{c}, 2}), \phi(v:\tau_{\text{p}, 2})\leq\phi(v:\tau_{\text{p}, 1})$. The goal can be proved by induction hypothesis because size reducing, subtyping and well-formedness inherited:
            \begin{itemize}
                \item {let $\text{x} = \text{c}$ or $\text{p}$, then $\textbf{size}(\tau_{\text{x}, 1}) + \textbf{size}(\tau_{\text{x}, 2}) < \textbf{size}(\tau_1) + \textbf{size}(\tau_2)$,\\
                the latter $= \textbf{size}(\tau_{\text{c}, 1}) + \textbf{size}(\tau_{\text{p}, 1}) + \textbf{size}(\tau_{\text{c}, 2}) + \textbf{size}(\tau_{\text{p}, 2}) + 2$; }
                \item {$\tau_{\text{c}, 1}\preceq\tau_{\text{c}, 2}, \tau_{\text{p}, 2}\preceq \tau_{\text{p}, 1}$ from $\tau_1\preceq\tau_2$;}
                \item {$\vdash \tau_{\text{c}, 1}, \vdash \tau_{\text{c}, 2}, \vdash \tau_{\text{p}, 1}, \vdash \tau_{\text{p}, 2}$ from $\vdash \tau_1, \vdash \tau_2$.}
            \end{itemize}
            }
        \end{enumerate}
        }
    \end{enumerate}
\end{proof}

\begin{lemma}[Update]
If store or context is written with new value or new type, the difference of potential over store or context is equal to that over new value or new type.
\begin{enumerate}
    \item {If $V\vDash p\rightsquigarrow v$, $\Gamma\vdash p\hookrightarrow \tau$ and $\VWt{V}{p}{v'}{V'}$,
    then $\Phi(V':\Gamma) - \Phi(V:\Gamma)=\Phi(v':\tau)-\Phi(v:\tau)$;}
    \item {If $V\vDash p\rightsquigarrow v$, $\Gamma\vdash p\hookrightarrow \tau$ and $\GWt{\Gamma}{p}{\tau'}{\Gamma'}$, 
    then $\Phi(V:\Gamma')-\Phi(V:\Gamma)=\Phi(v:\tau')-\Phi(v:\tau)$.}
\end{enumerate}
\end{lemma}
\begin{proof}
We first prove the update lemma on $V$, and then on $\Gamma$. \\
By induction on $\VWt{V}{p}{v'}{V'}$:
\begin{enumerate}
    \item {\rulename{V-Wt-Var} : We know from premise that $p = x$, $\forall y \neq x, V'(y) = V(y)$, $V'(x) = v', V(x) = v, \Gamma(x) = \tau$, then we reach
    \begin{align*}
        & \Phi(V':\Gamma)-\Phi(V:\Gamma) \\
        & = \left[\phi(V'(x):\Gamma(x)) + \sum_{y\neq x}\phi(V'(y):\Gamma(y))\right] - \left[\phi(V(x):\Gamma(x)) + \sum_{y\neq x}\phi(V(y):\Gamma(y))\right] \\
        & = \phi(v':\tau) -\phi(v:\tau)
    \end{align*}
    }
    \item {\rulename{V-Wt-Box} : We know from premise that $p = * p_1, V\vDash p_1\rightsquigarrow\kwd{box}(v), \VWt{V}{p_1}{\kwd{box}(v')}{V'}, \Gamma\vdash p_1 \hookrightarrow \kwd{box}(\tau)$, then we reach from hypothesis 
    \begin{align*}
        & \Phi(V':\Gamma)-\Phi(V:\Gamma) \\
        & = \phi(\kwd{box}(v'):\kwd{box}(\tau))-\phi(\kwd{box}(v):\kwd{box}(\tau)) \\
        & = \phi(v':\tau) - \phi(v:\tau)
    \end{align*} 
    }
    \item {\rulename{V-Wt-Borrow} : We know from premise that $p = * p_1, V\vDash p_1\rightsquigarrow\&(q, v), \VWt{V}{q}{v'}{V'}, \VWt{V'}{p_1}{\&(q, v')}{V''}$, and $p_1, q$ are separate and will not form a circle. Also we know $\Gamma\vdash p_1 \hookrightarrow \&^\kwd{m}(\tau, \tau_\text{p})$ \textbf{because only mutable borrows can be updated with values}, $\Gamma\vdash q\hookrightarrow \tau_\text{p}$ because $\Gamma$ is well formed by borrow checker. Therefore by induction hypothesis, we reach 
    \begin{align*}
    &\Phi(V'':\Gamma)-\Phi(V:\Gamma) \\
    &=\Phi(V'':\Gamma)-\Phi(V':\Gamma)+\Phi(V':\Gamma)-\Phi(V:\Gamma) \\
    &= \phi(v':\tau_\text{p})-\phi(v:\tau_\text{p}) + \phi(\&(q, v'):\&^\kwd{m}(\tau, \tau_\text{p}))-\phi(\&(q, v):\&^\kwd{m}(\tau, \tau_\text{p})) \\
    &= \phi(v':\tau_\text{p})-\phi(v:\tau_\text{p}) + [ \phi(v':\tau)-\phi(v':\tau_\text{p})]-[\phi(v:\tau)-\phi(v:\tau_\text{p})] \\
    &= \phi(v':\tau)-\phi(v:\tau)
    \end{align*}
    }
\end{enumerate}
By induction on $\GWt{\Gamma}{p}{\tau'}{\Gamma'}$:
\begin{enumerate}
    \item {\rulename{$\Gamma$-Wt-Var} : We know from premise that $p = x$, $\forall y \neq x, \Gamma'(y) = \Gamma(y)$, $\Gamma'(x) = \tau', \Gamma(x) = \tau, V(x) = v$, then we reach 
    \begin{align*}
        & \Phi(V:\Gamma')-\Phi(V:\Gamma) \\
        & = [\phi(V(x):\Gamma'(x)) + \sum_{y\neq x}\phi(V(y):\Gamma'(y))] - [\phi(V(x):\Gamma(x)) + \sum_{y\neq x}\phi(V(y):\Gamma(y))]\\
        & = \phi(v:\tau') -\phi(v:\tau)
    \end{align*}
    }
    \item {\rulename{$\Gamma$-Wt-Box} : We know from premise that $p = * p_1$, $\Gamma\vDash p_1\rightsquigarrow\kwd{box}(\tau)$, $\GWt{\Gamma}{p_1}{\kwd{box}(\tau')}{\Gamma'}$, and $V\vDash p_1 \rightsquigarrow \kwd{box}(v)$, then we reach from hypothesis 
    \begin{align*}
        & \Phi(V:\Gamma')-\Phi(V:\Gamma) \\
        & = \phi(\kwd{box}(v):\kwd{box}(\tau'))-\phi(\kwd{box}(v):\kwd{box}(\tau)) \\
        & = \phi(v:\tau') - \phi(v:\tau)
    \end{align*}
    }
    \item {\rulename{$\Gamma$-Wt-Shared} : We know from premise that $p = * p_1$, $\Gamma\vdash p_1 \hookrightarrow \&^\kwd{s}(\tau)$, $\GWt{\Gamma}{p_1}{\&^\kwd{s}(\tau')}{\Gamma'}$, and $V\vDash p_1 \rightsquigarrow \&(\_, v)$, then we reach from hypothesis 
    \begin{align*}
        & \Phi(V:\Gamma')-\Phi(V:\Gamma) \\
        & = \phi(\&(\_, v): \&^\kwd{s}(\tau')) - \phi(\&(\_, v): \&^\kwd{s}(\tau)) \\
        & = \phi(v:\tau') - \phi(v:\tau)
    \end{align*}
    }
    \item {\rulename{$\Gamma$-Wt-Mutable} : We know from premise that $p = * p_1$, $\Gamma\vdash p_1 \hookrightarrow \&^\kwd{m}(\tau, \tau_\text{p})$, $ \GWt{\Gamma}{p_1}{\&^\kwd{m}(\tau', \tau_\text{p})}{\Gamma'}$, and $V\vDash p_1 \rightsquigarrow \&(\_, v)$, then we reach from hypothesis 
    \begin{align*}
        & \Phi(V:\Gamma')-\Phi(V:\Gamma) \\
        & = \phi(\&(\_, v): \&^\kwd{m}(\tau', \tau_\text{p})) - \phi(\&(\_, v): \&^\kwd{m}(\tau, \tau_\text{p})) \\
        & = [\phi(v:\tau')-\phi(v:\tau_\text{p}] - [\phi(v:\tau)-\phi(v:\tau_\text{p})] \\
        & = \phi(v:\tau') - \phi(v:\tau)
    \end{align*}
    }
\end{enumerate}
\end{proof}

\newpage
\begin{lemma}[Evaluation]
If $V\vDash e\rightsquigarrow v, \Gamma\vdash e\hookrightarrow \tau\dashv\Gamma'$, then $\Phi(V:\Gamma')-\Phi(V:\Gamma)= -\phi(v:\tau)$.
\end{lemma}
\begin{proof}
By induction on $e$:
\begin{enumerate}
    \item {$e=\kwd{n}_\text{i32}, e_1 ~\kwd{op}~ e_2, \kwd{true}, \kwd{false}, \kwd{copy}~ p, \kwd{nil}$ : $\Gamma'=\Gamma$ and $\phi(v:\tau) = 0$;}
    \item {$e=\kwd{move}~p$ : We know that $V\vDash p\rightsquigarrow v$, $\Gamma\vdash p\hookrightarrow \tau$ and $\GWt{\Gamma}{p}{\bot}{\Gamma'}$, hence $\Phi(V:\Gamma')-\Phi(V:\Gamma)=\Phi(v:\bot)-\Phi(v:\tau)=-\Phi(v:\tau)$;} 
    \item {$e=\kwd{box}(e_1)$: simply by induction, $\Phi(V:\Gamma')-\Phi(V:\Gamma)=\Phi(v_1:\tau_1)=\Phi(\kwd{box}(v_1):\kwd{box}(\tau_1))$;}
    \item {$e=\&^\kwd{s}~p$: We assert that for all $\textit{share}~\tau~\textit{as}~\tau_1, \tau_2$, $\phi(v:\tau)=\phi(v:\tau_1)+\phi(v:\tau_2)$, then $\Phi(V:\Gamma')-\Phi(V:\Gamma)=\phi(v:\tau)-\phi(v:\tau_1)=\phi(v:\tau_2)=\phi(\&(p, v):\&^\kwd{s}(\tau_2))$, where the assertion can be proved by simple induction on sharing;}
    \item {$e=\&^\kwd{m}~p$: $V\vDash p\rightsquigarrow v, \Gamma\vdash p\hookrightarrow \tau, \GWt{\Gamma}{p}{\tau_\text{p}}{\Gamma'}$, with $\textit{prophecy}~ \tau ~\textit{as}~ \tau_\text{p}$, then $\Phi(V:\Gamma')-\Phi(V:\Gamma)=\phi(v:\tau_\text{p})-\phi(v:\tau)= - \phi(\&(p, v):\&^\text{m}(\tau, \tau_\text{p}))$.}
\end{enumerate}
\end{proof}

\begin{theorem}[Soundness]
If $V\vDash s \rightsquigarrow^{\delta_V} \Dashv V', \Gamma\vdash s \hookrightarrow^{\delta_\Gamma}\dashv\Gamma'$, then $\Phi(V':\Gamma')-\Phi(V:\Gamma)+\delta_V\leq\delta_\Gamma$.
\end{theorem}
\begin{proof}
By induction on $V\vDash s \rightsquigarrow^{\delta_V} \Dashv V'$:
\begin{enumerate}
    \item { \rulename{V-Ex-Ret} $s=\kwd{return}$ : $\delta_V=\delta_\Gamma=0, V'=V, \Gamma'=\Gamma$;}
    \item { \rulename{V-Ex-Seq} $s=s_1; s_2$ : By induction, we have $V\vDash s_1\rightsquigarrow^{\delta_{V, 1}}\Dashv V_1\vDash s_2\rightsquigarrow^{\delta_{V, 2}}\Dashv V'$, $\Gamma\vdash s_1\hookrightarrow^{\delta_{\Gamma, 1}}\dashv \Gamma_1\vdash s_2\hookrightarrow^{\delta_{\Gamma, 1}}\dashv \Gamma'$ and $\Phi(V_1:\Gamma_1)-\Phi(V:\Gamma)+\delta_{V, 1}\leq\delta_{\Gamma, 1}, \Phi(V':\Gamma')-\Phi(V_1:\Gamma_1)+\delta_{V, 2}\leq\delta_{\Gamma, 2}$. It is obvious that $\Phi(V:\Gamma)-\Phi(V':\Gamma')+\delta_{\Gamma, 1}+\delta_{\Gamma, 2}\geq\delta_{V, 1}+\delta_{V, 2}$;}
    \item {\rulename{V-Ex-Tick} $s=\kwd{tick}(\delta)$ : $\delta_V=\delta_\Gamma=\delta, V'=V, \Gamma'=\Gamma$;}
    \item {\rulename{V-Ex-Drop} $s=\kwd{drop}~p$ : $V=V', \delta_V=\delta_\Gamma=0$, we only need to prove that $\Phi(V:\Gamma')-\Phi(V:\Gamma')\leq 0$, but we know that $\GWt{\Gamma}{p}{\bot}{\Gamma'}$, then $\Phi(V:\Gamma')-\Phi(V:\Gamma)=\phi(v:\bot)-\phi(v:\tau)=-\phi(v:\tau)\leq0$, with the help of $\vdash \tau$ from premise of $\Gamma \vdash s \hookrightarrow^{\delta_\Gamma}\dashv \Gamma'$;}
    
    \item {\rulename{V-Ex-Assign} $s=p\from e$ : $\delta_V=\delta_\Gamma=0$, and we have $\Gamma\vdash e\hookrightarrow\tau'\dashv\Gamma_1$, $\Gamma_1\vdash p\hookrightarrow\tau$, $\GWt{\Gamma_1}{p}{\tau'}{\Gamma'}$, $V\vDash e\rightsquigarrow v'$, $V\vDash p\rightsquigarrow v$ and $\VWt{V}{p}{v'}{V'}$. With lemmas, we can imply that $\Phi(V:\Gamma_1)-\Phi(V:\Gamma)=-\phi(v':\tau')$, $\Phi(V:\Gamma')-\Phi(V:\Gamma_1)=\phi(v:\tau')-\phi(v:\tau)$ and $\Phi(V':\Gamma')-\Phi(V:\Gamma')=\phi(v':\tau')-\phi(v:\tau')$. Therefore $\Phi(V':\Gamma')-\Phi(V:\Gamma)=-\phi(v:\tau)\leq0$, with the help of $\vdash \tau$ from premise of $\Gamma \vdash s \hookrightarrow^{\delta_\Gamma}\dashv \Gamma'$; }
    
    \item{\rulename{V-Ex-Cons} $s=p\from \kwd{cons}(e_1; e_2)$ : $\delta_V=0, \delta_\Gamma=\alpha'$, and we have $\Gamma\vdash e_1\hookrightarrow\kwd{i32}$, $\Gamma\vdash e_2\hookrightarrow\kwd{box}(\kwd{list}(\alpha'))\dashv\Gamma_1$, $\Gamma_1\vdash p\hookrightarrow\kwd{list}(\alpha)$, $\GWt{\Gamma_1}{p}{\kwd{list}(\alpha')}{\Gamma'}$, $V\vDash p\rightsquigarrow v$, $V\vDash e_1\rightsquigarrow \kwd{n}_\text{i32}$, $V\vDash e_2\rightsquigarrow \kwd{box}(lv)$ and $\VWt{V}{p}{\kwd{cons}(n, \kwd{box}(lv))}{V'}$. With lemmas, we can imply that $\Phi(V:\Gamma_2)-\Phi(V:\Gamma)=-\phi(\kwd{box}(lv):\kwd{box}(\kwd{list}(\alpha)))$, $\Phi(V:\Gamma')-\Phi(V:\Gamma_2)=\phi(v:\kwd{list}(\alpha'))-\phi(v:\kwd{list}(\alpha))$, $\Phi(V':\Gamma')-\Phi(V:\Gamma')=\phi(\kwd{cons}(\kwd{n}_\text{i32}, \kwd{box}(lv)):\kwd{list}(\alpha'))-\phi(v:\kwd{list}(\alpha'))$. Therefore, we have $\Phi(V':\Gamma')-\Phi(V:\Gamma)=\alpha-\phi(v:\kwd{list}(\alpha'))\leq\alpha=\delta_\Gamma$, with the help of $\vdash \kwd{list}(\alpha')$;}
    
    \item {\rulename{V-Ex-IfT/F}$s=\kwd{if}~p~\kwd{then}~s_1~\kwd{else}~s_2~\kwd{end}$ : If $V\vDash p\rightsquigarrow\kwd{true}$, then we have $V\vDash s_1\rightsquigarrow^{\delta_{V, 1}}\Dashv V'$, $\Gamma\vdash s_1\hookrightarrow^{\delta_{\Gamma, 1}}\dashv\Gamma_1$ and $\Phi(V':\Gamma_1)-\Phi(V:\Gamma)+{\delta_{V, 1}}\leq{\delta_{\Gamma, 1}}$. $\Gamma_1\sqcap\Gamma_2=\Gamma'$ indicates $\Phi(V':\Gamma')\leq\Phi(V':\Gamma_1)$. From premise of statics, $\delta_{\Gamma, 1}\leq\delta_\Gamma$, hence $\Phi(V':\Gamma')-\Phi(V:\Gamma)+\delta_{V, 1}\leq\delta_\Gamma$. If $V\vDash p\rightsquigarrow\kwd{false}$, it is similar to prove;}
    \item {\rulename{V-Ex-Mat-Nil/Cons}$s=\kwd{match}~p~\{\kwd{nil}\mapsto s_1, \kwd{cons}(x_\text{hd}, x_\text{tl})\mapsto s_2\}$ : If $V\vDash p\rightsquigarrow\kwd{nil}$, it is similar to $\kwd{if}$ statement. We now turn to the possibility $V\vDash p\rightsquigarrow\kwd{cons}(n, \kwd{box}(lv))$. In such a case, we can obviously get that $\Phi(V_\text{b}:\Gamma_\text{b})-\Phi(V:\Gamma)=-\alpha$. From $V_\text{b}\vDash s_2\rightsquigarrow^{\delta_V}\Dashv V'_\text{b}, \Gamma_\text{b}\vdash s_2\hookrightarrow^{\delta_{\Gamma, 2}}\dashv \Gamma'_\text{b}$, we have that $\Phi(V'_\text{b}:\Gamma'_\text{b})-\Phi(V_\text{b}:\Gamma_\text{b})+\delta_V\leq\delta_{\Gamma, 2}$. Also, $\Phi(V':\Gamma_2)-\Phi(V'_\text{b}:\Gamma'_\text{b})=\beta$. Sum up inequalities, we have $\Phi(V':\Gamma_2)-\Phi(V:\Gamma)+\delta_V\leq\delta_{\Gamma, 2}-\alpha+\beta$. Similarly with $\Gamma'=\Gamma_1\sqcap\Gamma_2$ and $\delta_{\Gamma, 2}-\alpha+\beta\leq\delta_\Gamma$, we final reach that $\Phi(V':\Gamma')-\Phi(V:\Gamma)+\delta_V\leq\delta_\Gamma$; }
    \item {\rulename{V-Ex-App} $s=p\from f(\vec{e})$ : Assume $\text{fn}~ f ~(\vec{x}_\text{param}:\vec{t}_\text{param}, \vec{x}_\text{local}:\vec{t}_\text{local}, x_\text{ret}:t_\text{ret}) \{~ s_f ~\}, V\vDash p\rightsquigarrow v$, $\Gamma_n\vdash p\hookrightarrow \tau, \vdash \tau$, $V\vDash p\from f(\vec{e})\rightsquigarrow^{\delta_V}\Dashv V'$, and $\Gamma\vdash p\from f(\vec{e})\hookrightarrow^{\delta_\Gamma}\dashv\Gamma'$.
    
    To use induction hypothesis on $V_\text{b}\vDash s_f\rightsquigarrow^{\delta_V} V'_\text{b}$ from premise of dynamics, we need statics $\Gamma_\text{b}\vdash s_f\hookrightarrow^{\delta_\text{b}}\dashv \Gamma'_\text{b}$, where $\GWt{\Gamma_n}{\vec{x}_\text{param}}{\vec{\tau}_\text{arg}}{\Gamma_{\text{b}, 1}}$, $\GWt{\Gamma_{\text{b}, 1}}{\vec{x}_\text{local}}{\vec{\tau}_\text{local}}{\Gamma_{\text{b}, 2}}$ and $\GWt{{\Gamma_{\text{b}, 2}}}{x_\text{ret}}{\tau_\text{ret}}{\Gamma_\text{b}}$. However, it is not found but $\Gamma_f\vdash s_f \hookrightarrow^\delta \dashv\Gamma'_f$ appears in premise of $\vdash f \Leftarrow (\Gamma_f, \delta_f)$. Via context extension and weakening rules, we can reach our goal by proving $\Gamma_f \sqsubseteq \Gamma_\text{b}$. It is true because $\forall i=1, ..., n, \tau_{\text{param}, i} = \tau_{\text{arg}, i}$.
    
    \begin{enumerate}
        \item {$\Phi(V:\Gamma_n)-\Phi(V:\Gamma) = -\sum_i\phi(v_i:\tau_{\text{arg}, i})$;}
        \item {$\Phi(V_\text{b}:\Gamma_\text{b}) - \Phi(V:\Gamma_n) = \sum_i\phi(v_i:\tau_{\text{arg}, i})$; }

        \item {$\Phi(V'_\text{b}:\Gamma'_\text{b}) - \Phi(V_\text{b}:\Gamma_\text{b}) + \delta_V \leq \delta_\text{b}$; }
        
        \item {$\Phi(V':\Gamma')-\Phi(V'_\text{b}:\Gamma'_\text{b}) = -\sum_{x\in\vec{x}_\text{param} \text{or} x\in\vec{x}_\text{local}}
        \phi(v_x:\tau_x)-\phi(v:\tau) \leq 0$, \\
        considering that it does not affect resource to move $v_\text{ret}:\tau'_\text{ret}$ from $x_\text{ret}$ to $p$, \\
        while the erasure at parameters, local variables and $p$ affects; }

        \item {$\delta_\text{b} = \delta$ by weakening rules;}
        \item {$\delta = \delta_f$ from premise of $\vdash f \Leftarrow (\Gamma_f, \delta_f)$; }
        \item {$\delta_f = \delta_\Gamma$ from statics.}
    \end{enumerate}
    Sum up inequalities above, we reach $\Phi(V':\Gamma')-\Phi(V:\Gamma)+\delta_V\leq\delta_\Gamma$.
    }
\end{enumerate}
\end{proof}




\end{document}
\endinput
%%
%% End of file `sample-manuscript.tex'.
