\begin{abstract}

Rust has become a popular system programming language that strikes a balance between memory safety and performance.
%
Rust's type system ensures the safety of low-level memory controls; however, a well-typed Rust program is not guaranteed to enjoy high performance.
%
This article studies static analysis for resource consumption of Rust programs, aiming at understanding the performance of Rust programs.
%
Although there have been tons of studies on static resource analysis, exploiting Rust's memory safety---especially the borrow mechanisms and their properties---to aid resource-bound analysis, remains unexplored. 

This article presents \rarust{}, a type-based linear resource-bound analysis for well-typed Rust programs.
%
\rarust{} follows the methodology of automatic amortized resource analysis (AARA) to build a resource-aware type system.
%
To support Rust's borrow mechanisms, including shared and mutable borrows, \rarust{} introduces shared and novel prophecy potentials to reason about borrows compositionally.
%
To prove the soundness of \rarust{}, this article proposes Resource-Aware Borrow Calculus (RABC) as a variant of recently proposed Low-Level Borrow Calculus (LLBC).
%
The experimental evaluation of a prototype implementation of \rarust{} demonstrates that \rarust{} is capable of inferring symbolic linear resource bounds for Rust programs featuring shared and mutable borrows, reborrows, heap-allocated data structures, loops, and recursion. 

    % Resource consumption, such as time, memory and energy, is one critical property of programs. Besides functional correctness, we expect that programs can terminate in time, allocate moderate memory, or consume other kinds of resource within a proper bound. Automatic Amortized Resource Analysis (AARA) is usually applied to functional languages via integrating the potential method of amortized analysis in a type system, which collects and solves linear constraints over bound. However, resource analysis is still non-trivial for Rust's borrow mechanism, because Rust's borrow checker is complex itself and analyzer needs to track mutable borrows to return potential back.
    
    % We present a variant of borrow calculus named Resource Aware Borrow Calculus (RABC) and its novel type system to show resource consumption of programming languages with borrow mechanism. We extend the AARA approach with borrow types and track potential mutation via prophecy variables. With properties guaranteed by Rust's borrow checker, we prove the soundness theorem of our type system, ensuring the resource consumption in dynamics is bounded by types. We implement an automatic type inference algorithm and evaluate a suite of Rust programs featuring mutable borrows, revealing our system's ability to analyze resource consumption of Rust programs.

    
% \todo{CSS concepts and key words need to update}
\end{abstract}

%%
%% The code below is generated by the tool at http://dl.acm.org/ccs.cfm.
%% Please copy and paste the code instead of the example below.
%%
\begin{CCSXML}
<ccs2012>
   <concept>
       <concept_id>10003752.10010124.10010138.10010143</concept_id>
       <concept_desc>Theory of computation~Program analysis</concept_desc>
       <concept_significance>500</concept_significance>
       </concept>
   <concept>
       <concept_id>10003752.10010124.10010131.10010134</concept_id>
       <concept_desc>Theory of computation~Operational semantics</concept_desc>
       <concept_significance>500</concept_significance>
       </concept>
 </ccs2012>
\end{CCSXML}

% \ccsdesc[500]{Theory of computation~Program analysis}
% \ccsdesc[500]{Theory of computation~Operational semantics}
%%
%% Keywords. The author(s) should pick words that accurately describe
%% the work being presented. Separate the keywords with commas.
% \keywords{potential method, resource aware type system, borrow mechanism, prophecy variables}
