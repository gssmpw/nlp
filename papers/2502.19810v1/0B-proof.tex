\section{Proof of Soundness}
\label{sec:proof}

\begin{lemma}
    Potential is non-negative and keeps subtyping:
    \begin{mathpar}
    \inferrule
    {  \vdash \tau_1
    \\ \vdash \tau_2
    \\ \tau_1 \preceq \tau_2
    }
    { 0 \leq \phi(v:\tau_1) \leq \phi(v:\tau_2)}
    \end{mathpar}
\end{lemma}
\begin{proof}
    First define the size of rich types structural inductively:
    \begin{align*}
    \textbf{size} &: \textbf{RichType} \to \NN \\
    &|~ \bot ~|~ \kwd{i32} ~|~ \kwd{bool} ~|~ \kwd{list}(\_) ~|~ \kwd{box}(\kwd{list}(\_))\mapsto 0 \\
    &|~ \&^\kwd{s}(\tau) \mapsto \textbf{size}(\tau) + 1 \\
    &|~ \&^\kwd{m}(\tau_\text{c}, \tau_\text{p}) \mapsto \textbf{size}(\tau_\text{c}) + \textbf{size}(\tau_\text{p}) + 1
    \end{align*}
    Prove by induction on $\textbf{size}(\tau_1) + \textbf{size}(\tau_2)$, and case on $\tau_1\preceq\tau_2$:
    \begin{enumerate}
        \item {\rulename{S-Bot} $\tau_1=\bot$: \textit{exfalso} due to no derivation for $\vdash \bot$; }
        \item {\rulename{S-Int}, \rulename{S-Bool} : $0 = \phi(v:\tau_1) = \phi(v:\tau_2)$ by definition;}
        \item {\rulename{S-List}, \rulename{S-Box} : $\phi(v:\tau_1) = \alpha_1\cdot n \leq \alpha_2\cdot n$, where $\alpha_1 \leq \alpha_2$ from $\tau_1\preceq \tau_2$ and $n$ is length of $v$;}
        \item {\rulename{S-Shared} : $\tau_1=\&^\kwd{s}(\tau'_1), \tau_2=\&^\kwd{s}(\tau'_2)$ by induction hypothesis on $\tau'_1, \tau'_2$, because $\textbf{size}(\tau'_1)+\textbf{size}(\tau'_2) < \textbf{size}(\tau_1) + \textbf{size}(\tau_2) = \textbf{size}(\tau'_1) + \textbf{size}(\tau'_2) + 2$, $\tau'_1\preceq\tau'_2$ from $\tau_1\preceq\tau_2$ and $\vdash \tau'_1, \vdash \tau'_2$ from $\vdash \tau_1, \vdash \tau_2$;}
        \item {\rulename{S-Mutable} : $\tau_1=\&^\kwd{m}(\tau_{\text{c}, 1}, \tau_{\text{p}, 1}), \tau_2=\&^\kwd{m}(\tau_{\text{c}, 2}, \tau_{\text{p}, 2})$, $v=\&(\_, v')$
        \begin{enumerate}
            \item {$0\leq\phi(v:\tau_1) = \phi(v':\tau_{\text{c}, 1}) - \phi(v':\tau_{\text{p}, 1})$ by induction hypothesis on $\tau_{\text{p}, 1}, \tau_{\text{c}, 1}$, because $\textbf{size}(\tau_{\text{p}, 1})+\textbf{size}(\tau_{\text{c}, 1}) < \textbf{size}(\tau_1)+\textbf{size}(\tau_2)$ and $\tau_{\text{p}, 1}\preceq \tau_{\text{c}, 1}, \vdash \tau_{\text{c}, 1}, \vdash \tau_{\text{p}, 1}$ from $\vdash \tau_1$;}
            \item {$\phi(v:\tau_1)\leq\phi(v:\tau_2)$, i.e. $\phi(v:\tau_{\text{c}, 1})-\phi(v:\tau_{\text{p}, 1})\leq \phi(v:\tau_{\text{c}, 2})-\phi(v:\tau_{\text{p}, 2})$ if and only if $\phi(v:\tau_{\text{c}, 1})\leq \phi(v:\tau_{\text{c}, 2}), \phi(v:\tau_{\text{p}, 2})\leq\phi(v:\tau_{\text{p}, 1})$. The goal can be proved by induction hypothesis because size reducing, subtyping and well-formedness inherited:
            \begin{itemize}
                \item {let $\text{x} = \text{c}$ or $\text{p}$, then $\textbf{size}(\tau_{\text{x}, 1}) + \textbf{size}(\tau_{\text{x}, 2}) < \textbf{size}(\tau_1) + \textbf{size}(\tau_2)$,\\
                the latter $= \textbf{size}(\tau_{\text{c}, 1}) + \textbf{size}(\tau_{\text{p}, 1}) + \textbf{size}(\tau_{\text{c}, 2}) + \textbf{size}(\tau_{\text{p}, 2}) + 2$; }
                \item {$\tau_{\text{c}, 1}\preceq\tau_{\text{c}, 2}, \tau_{\text{p}, 2}\preceq \tau_{\text{p}, 1}$ from $\tau_1\preceq\tau_2$;}
                \item {$\vdash \tau_{\text{c}, 1}, \vdash \tau_{\text{c}, 2}, \vdash \tau_{\text{p}, 1}, \vdash \tau_{\text{p}, 2}$ from $\vdash \tau_1, \vdash \tau_2$.}
            \end{itemize}
            }
        \end{enumerate}
        }
    \end{enumerate}
\end{proof}

\begin{lemma}[Update]
If store or context is written with new value or new type, the difference of potential over store or context is equal to that over new value or new type.
\begin{enumerate}
    \item {If $V\vDash p\rightsquigarrow v$, $\Gamma\vdash p\hookrightarrow \tau$ and $\VWt{V}{p}{v'}{V'}$,
    then $\Phi(V':\Gamma) - \Phi(V:\Gamma)=\Phi(v':\tau)-\Phi(v:\tau)$;}
    \item {If $V\vDash p\rightsquigarrow v$, $\Gamma\vdash p\hookrightarrow \tau$ and $\GWt{\Gamma}{p}{\tau'}{\Gamma'}$, 
    then $\Phi(V:\Gamma')-\Phi(V:\Gamma)=\Phi(v:\tau')-\Phi(v:\tau)$.}
\end{enumerate}
\end{lemma}
\begin{proof}
We first prove the update lemma on $V$, and then on $\Gamma$. \\
By induction on $\VWt{V}{p}{v'}{V'}$:
\begin{enumerate}
    \item {\rulename{V-Wt-Var} : We know from premise that $p = x$, $\forall y \neq x, V'(y) = V(y)$, $V'(x) = v', V(x) = v, \Gamma(x) = \tau$, then we reach
    \begin{align*}
        & \Phi(V':\Gamma)-\Phi(V:\Gamma) \\
        & = \left[\phi(V'(x):\Gamma(x)) + \sum_{y\neq x}\phi(V'(y):\Gamma(y))\right] - \left[\phi(V(x):\Gamma(x)) + \sum_{y\neq x}\phi(V(y):\Gamma(y))\right] \\
        & = \phi(v':\tau) -\phi(v:\tau)
    \end{align*}
    }
    \item {\rulename{V-Wt-Box} : We know from premise that $p = * p_1, V\vDash p_1\rightsquigarrow\kwd{box}(v), \VWt{V}{p_1}{\kwd{box}(v')}{V'}, \Gamma\vdash p_1 \hookrightarrow \kwd{box}(\tau)$, then we reach from hypothesis 
    \begin{align*}
        & \Phi(V':\Gamma)-\Phi(V:\Gamma) \\
        & = \phi(\kwd{box}(v'):\kwd{box}(\tau))-\phi(\kwd{box}(v):\kwd{box}(\tau)) \\
        & = \phi(v':\tau) - \phi(v:\tau)
    \end{align*} 
    }
    \item {\rulename{V-Wt-Borrow} : We know from premise that $p = * p_1, V\vDash p_1\rightsquigarrow\&(q, v), \VWt{V}{q}{v'}{V'}, \VWt{V'}{p_1}{\&(q, v')}{V''}$, and $p_1, q$ are separate and will not form a circle. Also we know $\Gamma\vdash p_1 \hookrightarrow \&^\kwd{m}(\tau, \tau_\text{p})$ \textbf{because only mutable borrows can be updated with values}, $\Gamma\vdash q\hookrightarrow \tau_\text{p}$ because $\Gamma$ is well formed by borrow checker. Therefore by induction hypothesis, we reach 
    \begin{align*}
    &\Phi(V'':\Gamma)-\Phi(V:\Gamma) \\
    &=\Phi(V'':\Gamma)-\Phi(V':\Gamma)+\Phi(V':\Gamma)-\Phi(V:\Gamma) \\
    &= \phi(v':\tau_\text{p})-\phi(v:\tau_\text{p}) + \phi(\&(q, v'):\&^\kwd{m}(\tau, \tau_\text{p}))-\phi(\&(q, v):\&^\kwd{m}(\tau, \tau_\text{p})) \\
    &= \phi(v':\tau_\text{p})-\phi(v:\tau_\text{p}) + [ \phi(v':\tau)-\phi(v':\tau_\text{p})]-[\phi(v:\tau)-\phi(v:\tau_\text{p})] \\
    &= \phi(v':\tau)-\phi(v:\tau)
    \end{align*}
    }
\end{enumerate}
By induction on $\GWt{\Gamma}{p}{\tau'}{\Gamma'}$:
\begin{enumerate}
    \item {\rulename{$\Gamma$-Wt-Var} : We know from premise that $p = x$, $\forall y \neq x, \Gamma'(y) = \Gamma(y)$, $\Gamma'(x) = \tau', \Gamma(x) = \tau, V(x) = v$, then we reach 
    \begin{align*}
        & \Phi(V:\Gamma')-\Phi(V:\Gamma) \\
        & = [\phi(V(x):\Gamma'(x)) + \sum_{y\neq x}\phi(V(y):\Gamma'(y))] - [\phi(V(x):\Gamma(x)) + \sum_{y\neq x}\phi(V(y):\Gamma(y))]\\
        & = \phi(v:\tau') -\phi(v:\tau)
    \end{align*}
    }
    \item {\rulename{$\Gamma$-Wt-Box} : We know from premise that $p = * p_1$, $\Gamma\vDash p_1\rightsquigarrow\kwd{box}(\tau)$, $\GWt{\Gamma}{p_1}{\kwd{box}(\tau')}{\Gamma'}$, and $V\vDash p_1 \rightsquigarrow \kwd{box}(v)$, then we reach from hypothesis 
    \begin{align*}
        & \Phi(V:\Gamma')-\Phi(V:\Gamma) \\
        & = \phi(\kwd{box}(v):\kwd{box}(\tau'))-\phi(\kwd{box}(v):\kwd{box}(\tau)) \\
        & = \phi(v:\tau') - \phi(v:\tau)
    \end{align*}
    }
    \item {\rulename{$\Gamma$-Wt-Shared} : We know from premise that $p = * p_1$, $\Gamma\vdash p_1 \hookrightarrow \&^\kwd{s}(\tau)$, $\GWt{\Gamma}{p_1}{\&^\kwd{s}(\tau')}{\Gamma'}$, and $V\vDash p_1 \rightsquigarrow \&(\_, v)$, then we reach from hypothesis 
    \begin{align*}
        & \Phi(V:\Gamma')-\Phi(V:\Gamma) \\
        & = \phi(\&(\_, v): \&^\kwd{s}(\tau')) - \phi(\&(\_, v): \&^\kwd{s}(\tau)) \\
        & = \phi(v:\tau') - \phi(v:\tau)
    \end{align*}
    }
    \item {\rulename{$\Gamma$-Wt-Mutable} : We know from premise that $p = * p_1$, $\Gamma\vdash p_1 \hookrightarrow \&^\kwd{m}(\tau, \tau_\text{p})$, $ \GWt{\Gamma}{p_1}{\&^\kwd{m}(\tau', \tau_\text{p})}{\Gamma'}$, and $V\vDash p_1 \rightsquigarrow \&(\_, v)$, then we reach from hypothesis 
    \begin{align*}
        & \Phi(V:\Gamma')-\Phi(V:\Gamma) \\
        & = \phi(\&(\_, v): \&^\kwd{m}(\tau', \tau_\text{p})) - \phi(\&(\_, v): \&^\kwd{m}(\tau, \tau_\text{p})) \\
        & = [\phi(v:\tau')-\phi(v:\tau_\text{p}] - [\phi(v:\tau)-\phi(v:\tau_\text{p})] \\
        & = \phi(v:\tau') - \phi(v:\tau)
    \end{align*}
    }
\end{enumerate}
\end{proof}

\newpage
\begin{lemma}[Evaluation]
If $V\vDash e\rightsquigarrow v, \Gamma\vdash e\hookrightarrow \tau\dashv\Gamma'$, then $\Phi(V:\Gamma')-\Phi(V:\Gamma)= -\phi(v:\tau)$.
\end{lemma}
\begin{proof}
By induction on $e$:
\begin{enumerate}
    \item {$e=\kwd{n}_\text{i32}, e_1 ~\kwd{op}~ e_2, \kwd{true}, \kwd{false}, \kwd{copy}~ p, \kwd{nil}$ : $\Gamma'=\Gamma$ and $\phi(v:\tau) = 0$;}
    \item {$e=\kwd{move}~p$ : We know that $V\vDash p\rightsquigarrow v$, $\Gamma\vdash p\hookrightarrow \tau$ and $\GWt{\Gamma}{p}{\bot}{\Gamma'}$, hence $\Phi(V:\Gamma')-\Phi(V:\Gamma)=\Phi(v:\bot)-\Phi(v:\tau)=-\Phi(v:\tau)$;} 
    \item {$e=\kwd{box}(e_1)$: simply by induction, $\Phi(V:\Gamma')-\Phi(V:\Gamma)=\Phi(v_1:\tau_1)=\Phi(\kwd{box}(v_1):\kwd{box}(\tau_1))$;}
    \item {$e=\&^\kwd{s}~p$: We assert that for all $\textit{share}~\tau~\textit{as}~\tau_1, \tau_2$, $\phi(v:\tau)=\phi(v:\tau_1)+\phi(v:\tau_2)$, then $\Phi(V:\Gamma')-\Phi(V:\Gamma)=\phi(v:\tau)-\phi(v:\tau_1)=\phi(v:\tau_2)=\phi(\&(p, v):\&^\kwd{s}(\tau_2))$, where the assertion can be proved by simple induction on sharing;}
    \item {$e=\&^\kwd{m}~p$: $V\vDash p\rightsquigarrow v, \Gamma\vdash p\hookrightarrow \tau, \GWt{\Gamma}{p}{\tau_\text{p}}{\Gamma'}$, with $\textit{prophecy}~ \tau ~\textit{as}~ \tau_\text{p}$, then $\Phi(V:\Gamma')-\Phi(V:\Gamma)=\phi(v:\tau_\text{p})-\phi(v:\tau)= - \phi(\&(p, v):\&^\text{m}(\tau, \tau_\text{p}))$.}
\end{enumerate}
\end{proof}

\begin{theorem}[Soundness]
If $V\vDash s \rightsquigarrow^{\delta_V} \Dashv V', \Gamma\vdash s \hookrightarrow^{\delta_\Gamma}\dashv\Gamma'$, then $\Phi(V':\Gamma')-\Phi(V:\Gamma)+\delta_V\leq\delta_\Gamma$.
\end{theorem}
\begin{proof}
By induction on $V\vDash s \rightsquigarrow^{\delta_V} \Dashv V'$:
\begin{enumerate}
    \item { \rulename{V-Ex-Ret} $s=\kwd{return}$ : $\delta_V=\delta_\Gamma=0, V'=V, \Gamma'=\Gamma$;}
    \item { \rulename{V-Ex-Seq} $s=s_1; s_2$ : By induction, we have $V\vDash s_1\rightsquigarrow^{\delta_{V, 1}}\Dashv V_1\vDash s_2\rightsquigarrow^{\delta_{V, 2}}\Dashv V'$, $\Gamma\vdash s_1\hookrightarrow^{\delta_{\Gamma, 1}}\dashv \Gamma_1\vdash s_2\hookrightarrow^{\delta_{\Gamma, 1}}\dashv \Gamma'$ and $\Phi(V_1:\Gamma_1)-\Phi(V:\Gamma)+\delta_{V, 1}\leq\delta_{\Gamma, 1}, \Phi(V':\Gamma')-\Phi(V_1:\Gamma_1)+\delta_{V, 2}\leq\delta_{\Gamma, 2}$. It is obvious that $\Phi(V:\Gamma)-\Phi(V':\Gamma')+\delta_{\Gamma, 1}+\delta_{\Gamma, 2}\geq\delta_{V, 1}+\delta_{V, 2}$;}
    \item {\rulename{V-Ex-Tick} $s=\kwd{tick}(\delta)$ : $\delta_V=\delta_\Gamma=\delta, V'=V, \Gamma'=\Gamma$;}
    \item {\rulename{V-Ex-Drop} $s=\kwd{drop}~p$ : $V=V', \delta_V=\delta_\Gamma=0$, we only need to prove that $\Phi(V:\Gamma')-\Phi(V:\Gamma')\leq 0$, but we know that $\GWt{\Gamma}{p}{\bot}{\Gamma'}$, then $\Phi(V:\Gamma')-\Phi(V:\Gamma)=\phi(v:\bot)-\phi(v:\tau)=-\phi(v:\tau)\leq0$, with the help of $\vdash \tau$ from premise of $\Gamma \vdash s \hookrightarrow^{\delta_\Gamma}\dashv \Gamma'$;}
    
    \item {\rulename{V-Ex-Assign} $s=p\from e$ : $\delta_V=\delta_\Gamma=0$, and we have $\Gamma\vdash e\hookrightarrow\tau'\dashv\Gamma_1$, $\Gamma_1\vdash p\hookrightarrow\tau$, $\GWt{\Gamma_1}{p}{\tau'}{\Gamma'}$, $V\vDash e\rightsquigarrow v'$, $V\vDash p\rightsquigarrow v$ and $\VWt{V}{p}{v'}{V'}$. With lemmas, we can imply that $\Phi(V:\Gamma_1)-\Phi(V:\Gamma)=-\phi(v':\tau')$, $\Phi(V:\Gamma')-\Phi(V:\Gamma_1)=\phi(v:\tau')-\phi(v:\tau)$ and $\Phi(V':\Gamma')-\Phi(V:\Gamma')=\phi(v':\tau')-\phi(v:\tau')$. Therefore $\Phi(V':\Gamma')-\Phi(V:\Gamma)=-\phi(v:\tau)\leq0$, with the help of $\vdash \tau$ from premise of $\Gamma \vdash s \hookrightarrow^{\delta_\Gamma}\dashv \Gamma'$; }
    
    \item{\rulename{V-Ex-Cons} $s=p\from \kwd{cons}(e_1; e_2)$ : $\delta_V=0, \delta_\Gamma=\alpha'$, and we have $\Gamma\vdash e_1\hookrightarrow\kwd{i32}$, $\Gamma\vdash e_2\hookrightarrow\kwd{box}(\kwd{list}(\alpha'))\dashv\Gamma_1$, $\Gamma_1\vdash p\hookrightarrow\kwd{list}(\alpha)$, $\GWt{\Gamma_1}{p}{\kwd{list}(\alpha')}{\Gamma'}$, $V\vDash p\rightsquigarrow v$, $V\vDash e_1\rightsquigarrow \kwd{n}_\text{i32}$, $V\vDash e_2\rightsquigarrow \kwd{box}(lv)$ and $\VWt{V}{p}{\kwd{cons}(n, \kwd{box}(lv))}{V'}$. With lemmas, we can imply that $\Phi(V:\Gamma_2)-\Phi(V:\Gamma)=-\phi(\kwd{box}(lv):\kwd{box}(\kwd{list}(\alpha)))$, $\Phi(V:\Gamma')-\Phi(V:\Gamma_2)=\phi(v:\kwd{list}(\alpha'))-\phi(v:\kwd{list}(\alpha))$, $\Phi(V':\Gamma')-\Phi(V:\Gamma')=\phi(\kwd{cons}(\kwd{n}_\text{i32}, \kwd{box}(lv)):\kwd{list}(\alpha'))-\phi(v:\kwd{list}(\alpha'))$. Therefore, we have $\Phi(V':\Gamma')-\Phi(V:\Gamma)=\alpha-\phi(v:\kwd{list}(\alpha'))\leq\alpha=\delta_\Gamma$, with the help of $\vdash \kwd{list}(\alpha')$;}
    
    \item {\rulename{V-Ex-IfT/F}$s=\kwd{if}~p~\kwd{then}~s_1~\kwd{else}~s_2~\kwd{end}$ : If $V\vDash p\rightsquigarrow\kwd{true}$, then we have $V\vDash s_1\rightsquigarrow^{\delta_{V, 1}}\Dashv V'$, $\Gamma\vdash s_1\hookrightarrow^{\delta_{\Gamma, 1}}\dashv\Gamma_1$ and $\Phi(V':\Gamma_1)-\Phi(V:\Gamma)+{\delta_{V, 1}}\leq{\delta_{\Gamma, 1}}$. $\Gamma_1\sqcap\Gamma_2=\Gamma'$ indicates $\Phi(V':\Gamma')\leq\Phi(V':\Gamma_1)$. From premise of statics, $\delta_{\Gamma, 1}\leq\delta_\Gamma$, hence $\Phi(V':\Gamma')-\Phi(V:\Gamma)+\delta_{V, 1}\leq\delta_\Gamma$. If $V\vDash p\rightsquigarrow\kwd{false}$, it is similar to prove;}
    \item {\rulename{V-Ex-Mat-Nil/Cons}$s=\kwd{match}~p~\{\kwd{nil}\mapsto s_1, \kwd{cons}(x_\text{hd}, x_\text{tl})\mapsto s_2\}$ : If $V\vDash p\rightsquigarrow\kwd{nil}$, it is similar to $\kwd{if}$ statement. We now turn to the possibility $V\vDash p\rightsquigarrow\kwd{cons}(n, \kwd{box}(lv))$. In such a case, we can obviously get that $\Phi(V_\text{b}:\Gamma_\text{b})-\Phi(V:\Gamma)=-\alpha$. From $V_\text{b}\vDash s_2\rightsquigarrow^{\delta_V}\Dashv V'_\text{b}, \Gamma_\text{b}\vdash s_2\hookrightarrow^{\delta_{\Gamma, 2}}\dashv \Gamma'_\text{b}$, we have that $\Phi(V'_\text{b}:\Gamma'_\text{b})-\Phi(V_\text{b}:\Gamma_\text{b})+\delta_V\leq\delta_{\Gamma, 2}$. Also, $\Phi(V':\Gamma_2)-\Phi(V'_\text{b}:\Gamma'_\text{b})=\beta$. Sum up inequalities, we have $\Phi(V':\Gamma_2)-\Phi(V:\Gamma)+\delta_V\leq\delta_{\Gamma, 2}-\alpha+\beta$. Similarly with $\Gamma'=\Gamma_1\sqcap\Gamma_2$ and $\delta_{\Gamma, 2}-\alpha+\beta\leq\delta_\Gamma$, we final reach that $\Phi(V':\Gamma')-\Phi(V:\Gamma)+\delta_V\leq\delta_\Gamma$; }
    \item {\rulename{V-Ex-App} $s=p\from f(\vec{e})$ : Assume $\text{fn}~ f ~(\vec{x}_\text{param}:\vec{t}_\text{param}, \vec{x}_\text{local}:\vec{t}_\text{local}, x_\text{ret}:t_\text{ret}) \{~ s_f ~\}, V\vDash p\rightsquigarrow v$, $\Gamma_n\vdash p\hookrightarrow \tau, \vdash \tau$, $V\vDash p\from f(\vec{e})\rightsquigarrow^{\delta_V}\Dashv V'$, and $\Gamma\vdash p\from f(\vec{e})\hookrightarrow^{\delta_\Gamma}\dashv\Gamma'$.
    
    To use induction hypothesis on $V_\text{b}\vDash s_f\rightsquigarrow^{\delta_V} V'_\text{b}$ from premise of dynamics, we need statics $\Gamma_\text{b}\vdash s_f\hookrightarrow^{\delta_\text{b}}\dashv \Gamma'_\text{b}$, where $\GWt{\Gamma_n}{\vec{x}_\text{param}}{\vec{\tau}_\text{arg}}{\Gamma_{\text{b}, 1}}$, $\GWt{\Gamma_{\text{b}, 1}}{\vec{x}_\text{local}}{\vec{\tau}_\text{local}}{\Gamma_{\text{b}, 2}}$ and $\GWt{{\Gamma_{\text{b}, 2}}}{x_\text{ret}}{\tau_\text{ret}}{\Gamma_\text{b}}$. However, it is not found but $\Gamma_f\vdash s_f \hookrightarrow^\delta \dashv\Gamma'_f$ appears in premise of $\vdash f \Leftarrow (\Gamma_f, \delta_f)$. Via context extension and weakening rules, we can reach our goal by proving $\Gamma_f \sqsubseteq \Gamma_\text{b}$. It is true because $\forall i=1, ..., n, \tau_{\text{param}, i} = \tau_{\text{arg}, i}$.
    
    \begin{enumerate}
        \item {$\Phi(V:\Gamma_n)-\Phi(V:\Gamma) = -\sum_i\phi(v_i:\tau_{\text{arg}, i})$;}
        \item {$\Phi(V_\text{b}:\Gamma_\text{b}) - \Phi(V:\Gamma_n) = \sum_i\phi(v_i:\tau_{\text{arg}, i})$; }

        \item {$\Phi(V'_\text{b}:\Gamma'_\text{b}) - \Phi(V_\text{b}:\Gamma_\text{b}) + \delta_V \leq \delta_\text{b}$; }
        
        \item {$\Phi(V':\Gamma')-\Phi(V'_\text{b}:\Gamma'_\text{b}) = -\sum_{x\in\vec{x}_\text{param} \text{or} x\in\vec{x}_\text{local}}
        \phi(v_x:\tau_x)-\phi(v:\tau) \leq 0$, \\
        considering that it does not affect resource to move $v_\text{ret}:\tau'_\text{ret}$ from $x_\text{ret}$ to $p$, \\
        while the erasure at parameters, local variables and $p$ affects; }

        \item {$\delta_\text{b} = \delta$ by weakening rules;}
        \item {$\delta = \delta_f$ from premise of $\vdash f \Leftarrow (\Gamma_f, \delta_f)$; }
        \item {$\delta_f = \delta_\Gamma$ from statics.}
    \end{enumerate}
    Sum up inequalities above, we reach $\Phi(V':\Gamma')-\Phi(V:\Gamma)+\delta_V\leq\delta_\Gamma$.
    }
\end{enumerate}
\end{proof}