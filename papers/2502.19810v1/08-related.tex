\section{Related Work} \label{sec:related}

In this section, we discuss related work that has not been covered in previous sections.

\paragraph{Static resource analysis}
%
AARA is not the only approach for type-based resource analysis.
%
For example, there are approaches based on sized types~\cite{phd:Vasconcelos08,ICFP:AL17}, dependent types~\cite{LICS:LG11,POPL:LP13,POPL:RGG21}, refinement types~\cite{POPL:HVH20,POPL:CBG17,ESOP:CGA15,OOPSLA:WWC17},
recurrence extraction~\cite{POPL:KML20,ICFP:DLR15}, logical frameworks~\cite{POPL:NSG22,POPL:GNS24}, and annotated types~\cite{POPL:CW00,POPL:Danielsson08}.
%
None of the aforementioned type systems considered supporting heap-manipulating programs with Rust's borrow mechanisms.
%
Besides type systems, there are also static resource-analysis techniques based on
defunctionalization~\cite{ICFP:ALM15}, recurrence relations~\cite{JAR:AAG11,TACAS:AFR15,APLAS:FH14,PLDI:BCK20,POPL:KBC19,PLDI:KBB17},
term rewriting~\cite{RTA:AM13,TACAS:BEG14,IJCAR:FNH16,JAR:NEG13}, and
abstract interpretation~\cite{SAS:ZSG11,LPAR:BHH10,CAV:SZV14,kn:DHW07,misc:fbinfer20,SAS:AG12}.
%
Some of the aforementioned techniques work on imperative programs (e.g., C programs) with arrays,
such as C4B~\cite{PLDI:CHS15}, SPEED~\cite{POPL:GMC09}, COSTA~\cite{JAR:AAG11}, ICRA~\cite{PLDI:KBB17}, etc., but none of them considered exploiting Rust's borrow mechanisms in the design.
%
It would be interesting future research direction to investigate how different static resource-analysis techniques can benefit from Rust's safety guarantees.

% Automatic Amortized Resource Analysis(AARA) first presented by \cite{AARA-Linear}, enriched type system  with resource annotations for a first-order functional language, to derive linear upper bounds on the heap-memory usage of list via linear programming. Derivative works support non-linear worst-case upper bound, like polynomial \cite{AARA-Poly}, multivariate polynomial \cite{AARA-Poly-Multivar}, exponential \cite{AARA-Exp} and logarithmic \cite {AARA-Log}. AARA can also support features like higher-order functions \cite{AARA-HigherOrder}, algebraic data types \cite{AARA-ADT} and general recursive data types \cite{AARA-GeneralRecursive}. Original AARA method does not support imperative mutation, whereas our work features AARA mainly with it and Rust borrow mechanism. Compared with AARA derivative works, our formalization is currently limited to single variate linear upper bound, however in our view, our extension is orthogonal to other features. Therefore it can be easy, as future works, to support complex recursive data structures and various upper bounds in Rust resource analyzer.

\paragraph{Graded type system}
Graded types \cite{Granule} introduce a graded modality for associating types with elements from a resource algebra. 
%
A graded type system can account for program variables' exact usages, security levels, and potentials (conceptually). 
%
Graded types seem to provide a more general mechanism than AARA types to reason about more general resources. 
%
On the other hand, our work focuses on how Rust's borrow mechanisms can aid resource analysis and chooses AARA as the starting point because AARA admits efficient type inference via linear programming. 

\paragraph{Program verification for Rust}
%
RustBelt~\cite{RustBelt} pioneers a line of work to use semantic typing and separation logic to verify Rust programs with both safe and unsafe code.
%
% Stacked Borrow \cite{StackedBorrow} presents an operational semantics for memory accesses in Rust, and defines an aliasing discipline to cooperate raw pointers and borrows when verifying. 
%
RustHorn~\cite{RustHorn} uses prophecy variables to model the future values of mutable borrows and proposes an automatic verification algorithm based on constrained Horn clauses.
%
% Another approach is to leverage Rust type system, restricting on safe Rust with borrow mechanism. 
Aeneas~\cite{Aeneas}---which inspires our development of RABC and supports our prototype implementation---uses LLBC to translate Rust programs into equivalent pure functional programs via symbolic execution.
%
Such pure functional programs can be ported into theorem provers such as Coq and F* to enable verification of functional correctness.
%
Prusti~\cite{OOPSLA:AMP19} also leverages Rust's advanced type system to devise a modular and automated verification approach.
%
\citet{master:Engel21} proposed a method to verify user-provided asymptotic resource bounds in Prusti.
%
In contrast, our work focuses on the automatic inference of concrete resource bounds via a type system.
%
% Refinement types based method has also been applied to verification, like Flux \cite{Flux} proposing a liquid type system and RefinedRust \cite{RefinedRust} exploiting prophecy variables. Our work follows type system based approach and focuses on analysis instead of verification. We admits safe restriction of Rust from Stacked Borrow, resembling Aeneas working on borrow calculus, but unnecessarily translating to functional program, straightforward analyzing resource consumption. We also adopt prophecy variables to model mutable borrows.

% \todo{MAY DELETE THIS merging? might similar to static analysis, when coming across loop}