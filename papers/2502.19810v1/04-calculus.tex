\section{Resource Aware Borrow Calculus}
\label{sec:calculus}

This section introduces Resource-Aware Borrow Calculus (RABC) and resource-aware dynamic semantics.
%
RABC is a resource-aware variant of Low-Level Borrow Calculus (LLBC)~\cite{Aeneas}, which is based on Rust's MIR but keeps high-level information such as structured control flow and a structured memory model.
%
RABC includes essential features such as mutation, borrow mechanisms, integer lists with explicit boxing, and recursive top-level functions.\footnote{We include only integer lists as heap-allocated data structures and exclude loops in RABC for ease of presentation. Our implementation supports user-defined inductive data types using structs and enums, as well as \texttt{while true} loops with break and continue.}
%
RABC also includes $\kwd{tick}(\cdot)$ statements to annotate resource consumption.
%
\cref{sec:syntax} presents the syntax and discusses properties of a well-borrowed RABC program, guaranteed by Rust's borrow checker.
%
\cref{sec:semantics} formalizes the resource-aware dynamic semantics of RABC, which captures the resource consumption during the execution of an RABC program.

\subsection{Syntax}
\label{sec:syntax}

\cref{fig:syntax} summarizes the syntax of RABC. For the convenience of formalization, we distinguish between expressions and statements.
%
We then describe each syntactical construction separately. 

\begin{figure}[t]
\small
    \begin{align*}
    \textbf{Type}~ t &::= \\
        \tag{atom} &|~ \kwd{i32} ~|~ \kwd{bool} \\
        \tag{list} &|~ \kwd{list} ~|~ \kwd{box}~\kwd{list}\\
        \tag{borrow} &|~ \&^\kwd{s}~t ~|~ \&^\kwd{m}~t \\
    \textbf{Place}~ p &::= \\
        \tag{variable} &|~ x \\
        \tag{dereference} &|~ * p \\
    \textbf{Expression}~ e &::= \\
        &|~ \kwd{n}_\text{i32} ~|~ \kwd{true} ~|~ \kwd{false} ~|~ \kwd{nil} ~|~ \kwd{box}(e) \\
        \tag{integer} &|~ e_1 ~\kwd{op}~ e_2 \\
        \tag{scalar copy} &|~ \kwd{copy}~ p \\
        \tag{borrow} &|~ \&^\kwd{s}~ p ~|~ \&^\kwd{m}~ p \\
        \tag{move ownership} &|~ \kwd{move}~ p \\
    \textbf{Statement}~ s &::= \\
        \tag{resource cost} &|~ \kwd{tick}(\delta) \\
        \tag{return} &|~ \kwd{return} \\
        \tag{sequence} &|~ s_1; s_2 \\
        \tag{drop} &|~ \kwd{drop}~ p\\
        \tag{if bool} &|~ \kwd{if}~ p ~\kwd{then}~ s_1 ~\kwd{else}~ s_2 ~\kwd{end}\\
        \tag{match list} &|~ \kwd{match}~ p ~ \{\kwd{nil}\mapsto s_1, \kwd{cons}(x_\text{hd}, x_\text{tl})\mapsto s_2\} \\
        \tag{assignment} &|~ p \from e \\
        \tag{list constructor} &|~ p \from \kwd{cons}(e_1, e_2) \\
        \tag{function call} &|~ p \from f(\vec{e}) \\
    \textbf{Toplevel}~ tl &::= \\
        \tag{sequence} &|~ tl_1 ~ tl_2 \\
        \tag{function} &|~ \kwd{fn}~ f ~(\vec{x}_\text{param}:\vec{t}_\text{param}, \vec{x}_\text{local}:\vec{t}_\text{local}, x_\text{ret}:t_\text{ret}) \{~ s ~\}
    \end{align*}
    \caption{Syntax}
    \label{fig:syntax}
\end{figure}


\textbf{Types} are simple, without resource annotations; we will present the types with annotations in \cref{sec:inference}. Integer $\kwd{i32}$ and Boolean $\kwd{bool}$ are atom types. Lists $\kwd{list}$ and the box type of lists $\kwd{box}~\kwd{list}$ are types for functional lists defined in Rust. The box type is required for a list's tail, which is usually heap-allocated. The reference type is for borrows with different modes: $\&^\kwd{s}~ t$ is for the shared borrow, and $\&^\kwd{m}~ t$ is for the mutable borrow. These borrow modes are notions from Rust explained as follows: mutation is forbidden on shared borrows and only allowed on mutable borrows.

\textbf{Places} are memory locations storing values, including program variables $x, y, \ldots$ and dereferences $* p$ of borrows or boxes stored in $p$. We will soon show their role in the dynamic semantics in \cref{sec:semantics}. 

\textbf{Expressions} represent resource-free evaluation. Integer literals $\kwd{n}_\text{i32}$ and Boolean literals $\kwd{true}$, $\kwd{false}$ are atom values. The $\kwd{nil}$ constructor stands for empty lists. The boxing expression $\kwd{box}(e)$ allocates memory in a heap to store the value of $e$, resembling \lstinline|Box::new(e)| of Rust. Arithmetic expressions $e_1 ~\kwd{op}~ e_2$ operate on integer-valued operands with the binary operator $\kwd{op}$. For an atom value stored in a place $p$, we use $\kwd{copy}~p$ to make a copy of it.
% The scalar copies are for the smaller data like integers and Boolean values, while the borrows are usually used for those larger data structures like lists.
Given a place $p$, we can borrow from it with different modes: $\&^\kwd{s}~ p$ creates a shared borrow, and $\&^\kwd{m}~ p$ creates a mutable borrow.
%
For a borrow stored in a place $p$, we can use $\kwd{move}~ p$ to move ownership out from the original place $p$. 

\textbf{Statements} represent resource-aware evaluation. The statement $\kwd{tick}(\delta)$ with $\delta \in \ZZ$ is the explicit annotation for consuming $\delta$ units of resource.
%
Statements $\kwd{return}$ and $s_1; s_2$ usually form a function body such as $s_1; s_2; \ldots, s_n; \kwd{return}$. In RABC, we introduce $\kwd{drop}~ p$ to drop the borrow stored in $p$ explicitly. The conditional and pattern-match statements perform case analysis on Boolean values and lists, respectively. Note that we use place $p$ instead of the expression $e$ to indicate Boolean values and lists because we only need to peak the value instead of copying Boolean values or moving ownership of lists. The assignment has three variants: one for assigning atom values and borrows, one for constructing lists, and another for function applications; the latter two variants do not reside in expressions because they need to be resource-aware as they incur resource consumption.

\textbf{Toplevels} define a sequence of (possibly recursive) top-level functions like $\kwd{fn} ~f_1, \ldots, \kwd{fn}~ f_n$. Each function contains one statement as its body and variables with corresponding type declarations, including function parameters $\vec{x}_\text{param}:\vec{t}_\text{param}$, local variables $\vec{x}_\text{local}:\vec{t}_\text{local}$ used in the function body, and a distinguished variable $x_\text{ret} : t_\text{ret}$ for the returned value. The $\vec{\bullet}$ notation represents vectors.

\subsection{Resource Aware Dynamic Semantics}
\label{sec:semantics}

\begin{figure}[t]
\small
    \begin{align*}
    \tag{undefined} \textbf{Value}~ v &::= \bot \\
    \tag{atoms} &|~ \kwd{n}_\text{i32} ~|~ \kwd{true} ~|~ \kwd{false} \\
    \tag{list} &|~  lv \\
    \tag{box} &|~ \kwd{box}(lv) \\
    \tag{borrow} &|~ \&(p, v)\\
    \textbf{List Value}~ lv &::= \kwd{nil} ~|~ \kwd{cons}(\kwd{n}_\text{i32}, \kwd{box}(lv))
    \end{align*}
    \caption{Value}
    \label{fig:dyn-value}
\end{figure}

\begin{figure}[t]
\small
    \judgement{Store Reading}{$V\vDash p \rightsquigarrow v$}
    \begin{mathpar}
    \inferrule*[Right=\rulename{V-Rd-Var}]
    {V(x)=v}
    {V\vDash x \rightsquigarrow v}
    \and
    \inferrule*[Right=\rulename{V-Rd-Box}]
    {V\vDash p \rightsquigarrow \kwd{box}(v)}
    {V\vDash * p\rightsquigarrow v}
    \and
    \inferrule*[Right=\rulename{V-Rd-Borrow}]
    {V\vDash p \rightsquigarrow \&(\_, v)}
    {V\vDash *p \rightsquigarrow v}
    \end{mathpar}

    \judgement{Store Writing}{$\VWt{V}{p}{v}{V'}$}
    \begin{mathpar}
    \inferrule[V-Wt-Var]
    {\forall y\not=x, V'(y)=V(y) 
    \\ V'(x) = v}
    {\VWt{V}{x}{v}{V'}}
    \and
    \inferrule[V-Wt-Box]
    {V\vDash p\rightsquigarrow \kwd{box}(\_)
    \\ \VWt{V}{p}{\kwd{box}(v)}{V'}}
    {\VWt{V}{* p}{v}{V'}}
    \and
    \inferrule*[Right=\rulename{V-Wt-Borrow}]
    {V\vDash p\rightsquigarrow \&(q, \_)
    \\ \VWt{V}{q}{v}{V'}
    \\ \VWt{V'}{p}{\&(q, v)}{V''}}
    {\VWt{V}{*p}{v}{V''}}
    \end{mathpar}

    \caption{Store Reading and Writing}
    \label{fig:dyn-rw}
\end{figure}

\begin{figure}[t]
\small
\judgement{Expression Evaluation (Selected)}{$V\vDash e \rightsquigarrow v$}
    \begin{mathpar}
    \inferrule*[Right=\rulename{V-Ev-Borrow}]
    {V\vDash p \rightsquigarrow v}
    {V\vDash \&^{\kwd{s}/\kwd{m}} p \rightsquigarrow \&(p, v)}
    \end{mathpar}

\judgement{Statement Execution (Selected)}{$V\vDash e \rightsquigarrow^\delta \Dashv V'$}
    \begin{mathpar}
    \inferrule*[Right=\rulename{V-Ex-Tick}]
    {~}
    {V\vDash \kwd{tick}(\delta)\rightsquigarrow^\delta \Dashv V}
    \and
    \inferrule*[Right=\rulename{V-Ex-Drop}]
    {~}
    {V\vDash \kwd{drop}~p\rightsquigarrow^0\Dashv V}
    \\
    \inferrule*[Right=\rulename{V-Ex-Cons}]
    {V\vDash e_1 \rightsquigarrow v_1
    \\ V\vDash e_2 \rightsquigarrow v_2
    \\ \VWt{V}{p}{\kwd{cons}(v_1, v_2)}{V'} }
    {V\vDash p\from \kwd{cons}(e_1, e_2)\rightsquigarrow^0 \Dashv V'}
    \\
    \inferrule*[Right=\rulename{V-Ex-IfT}]
    {V\vDash p\rightsquigarrow \kwd{true}
    \\ V\vDash s_1\rightsquigarrow^\delta \Dashv V'}
    {V\vDash \kwd{if}~ p ~\kwd{then}~ s_1 ~\kwd{else}~ s_2 ~\kwd{end} \rightsquigarrow^\delta \Dashv V'}
    \and
    \inferrule*[Right=\rulename{V-Ex-IfF}]
    {V\vDash p\rightsquigarrow \kwd{false}
    \\ V\vDash s_2\rightsquigarrow^\delta \Dashv V'}
    {V\vDash \kwd{if}~ p ~\kwd{then}~ s_1 ~\kwd{else}~ s_2 ~\kwd{end} \rightsquigarrow^\delta \Dashv V'}
    \\
    \inferrule*[Right=\rulename{V-Ex-Mat-Nil}]
    {V\vDash p\rightsquigarrow \kwd{nil}
    \\ V\vDash s_1\rightsquigarrow^\delta \Dashv V'}
    {V\vDash \kwd{match}~ p ~ \{\kwd{nil}\mapsto s_1, \kwd{cons}(x_\text{hd}, x_\text{tl})\mapsto s_2\} \rightsquigarrow^\delta \Dashv V'}

    \inferrule*[Right=\rulename{V-Ex-Mat-Cons}]
    {V\vDash p\rightsquigarrow \kwd{cons}(hd, tl)
    \\ \VWt{V}{p}{\bot}{V_1}
    \\ \VWt{V_1}{x_\text{hd}}{hd}{V_2}
    \\ \VWt{V_2}{x_\text{tl}}{tl}{V_\text{b}}
    \\\\ V_\text{b}\vDash s_2\rightsquigarrow^\delta \Dashv V'_\text{b}
    \\ V'_\text{b}\vDash x_\text{hd}\rightsquigarrow hd'
    \\ V'_\text{b}\vDash x_\text{tl}\rightsquigarrow tl'
    \\\\ \VWt{V'_\text{b}}{x_\text{hd}}{\bot}{V'_1}
    \\ \VWt{V'_1}{x_\text{tl}}{\bot}{V'_2}
    \\ \VWt{V'_2}{p}{\kwd{cons}(hd', tl')}{V'} 
    }
    {V\vDash \kwd{match}~ p ~ \{\kwd{nil}\mapsto s_1, \kwd{cons}(x_\text{hd}, x_\text{tl})\mapsto s_2\} \rightsquigarrow^\delta \Dashv V'}
    \\
    
    \inferrule*[Right=\rulename{V-Ex-App}]
    {\kwd{fn}~ f ~(\vec{x}_\text{param}:\vec{t}_\text{param}, \vec{x}_\text{local}:\vec{t}_\text{local}, x_\text{ret}:t_\text{ret}) \{~ s ~\}
    \\ V\vDash \vec{e}\rightsquigarrow \vec{v}
    \\\\ \VWt{V}{\vec{x}_\text{param}}{\vec{v}}{V_1}
    \\ \VWt{V_1}{\vec{x}_\text{local}}{\bot}{V_2}
    \\ \VWt{V_2}{x_\text{ret}}{\bot}{V_\text{b}}
    \\\\ V_\text{b}\vDash s\rightsquigarrow^\delta \Dashv V'_\text{b}
    \\ V'_\text{b}\vDash x_\text{ret} \rightsquigarrow v_\text{ret}
    \\\\ \VWt{V'_\text{b}}{\vec{x}_\text{param}}{\bot}{V'_1}
    \\ \VWt{V'_1}{\vec{x}_\text{local}}{\bot}{V'_2}
    \\ \VWt{V'_2}{x_\text{ret}}{\bot}{V'_3}
    \\ \VWt{V'_3}{p}{v_\text{ret}}{V'}
    } 
    {V\vDash p\from f(\vec{e})\rightsquigarrow^\delta \Dashv V'}
    \end{mathpar}
    \caption{Resource Aware Dynamic Semantics}
    \label{fig:dyn-eval-exec}
\end{figure}


\cref{fig:dyn-value}, \cref{fig:dyn-rw}, and \cref{fig:dyn-eval-exec} define a  resource-aware big-step dynamic semantics for RABC.
%
\cref{fig:dyn-value} defines values of RABC, including atom values, list values, box values, borrow values, and a distinguished undefined value $\bot$. 
%
Note that borrow values take the form $\&(p,v)$, denoting a value $v$ borrowed from a place $p$, but do not record the borrow mode ($\&^\kwd{s}$ or $\&^\kwd{m}$).
%
The design is reasonable because we work on well-borrowed programs; thus, we do not need to track the borrow modes during runtime.

A \emph{store} is a mapping $V : \mathbf{Variable}\to\mathbf{Value}$, where unused variables can be mapped to $\bot$.
%
\cref{fig:dyn-rw} formalizes reading from and writing to a store.
%
Judgement $V \vDash p \rightsquigarrow v$ means that under a store $V$, the place $p$ records a value $v$.
%
Judgement $\VWt{V}{p}{v}{V'}$ means that starting from a store $V$, writing a value $v$ to the place $p$ yields a new store $V'$.
%
Note that the rule \rulename{V-Wt-Borrow} may not terminate in general for heap-manipulating languages like C.
%
In our setting, we exploit Rust's borrow mechanisms that ensure that one cannot construct cyclic reference relations using borrows.

% We can read or write on a variable $x$. Due to $\mathbf{Place}$ syntax, we extend it to judgement $V\vDash p \rightsquigarrow v$ and $\VWt{V}{p}{v}{V'}$, the former reading and the latter writing, as \cref{fig:dyn-rw}. All rules are nearly trivial except that rule \rulename{V-Wt-Borrow} immediately updates value in original place $q$, as $\VWt{V}{q}{v}{V'}$. The termination and safety of rule \rulename{V-Wt-Borrow} is guaranteed by the Rust borrow checker again.

\cref{fig:dyn-eval-exec} presents selected evaluation rules for expressions and statements.
%
Judgement $V\vDash e \rightsquigarrow v$ indicates that under a store $V$, the expression $e$ evaluates to the value $v$. Recall that expressions denote resource-free computation, so we do not record resource information.
%
% The rule \rulename{V-Ev-Copy} is restricted to scalar values, and the rule \rulename{V-Ev-Move} is restricted to borrows.
%
The rule \rulename{V-Ev-Borrow} reflects the design that the runtime does not need to track borrow modes for well-borrowed programs.
%
Judgement $V \vDash s \rightsquigarrow^\delta \Dashv V'$ means that starting from a store $V$, the statement $s$ executes with $\delta$ units of resource consumption and ends in the store $V'$.
%
The rule \rulename{V-Ex-Tick} introduces $\delta$ unit of resource consumption; this is the only rule to incur actual resource uses.
%
The rule \rulename{V-Ex-Drop} does nothing, i.e., it does need to put the value back to the borrowed place, because we immediately update values when writing through borrows, as indicated by the rule \rulename{V-Wt-Borrow} in \cref{fig:dyn-rw}.
%
Also, because of such immediate updates, it is necessary to make sure that original places and variables should be passed to function applications; the subscript $\textbf{b}$ of the store $V_\text{b}$ stands for \textbf{b}inding in the rule \rulename{V-Ex-App}.
%
% Though global uniqueness of variables is a requirement in dynamic semantics, this feature does not add complexity to the type system, which will be elaborated in the subsequent section. This is because dynamic semantics serves merely as a resource-aware semantic instrument in the formalization to prove the soundness of our type system; it is not executed in practice.


