% Three useful guides to the organization of the intro section
%
% Organization A (Simon Peyton-Jones)
%  o Describe the problem
%  o State your contributions
%  o Organization of the paper [if not able to cover point by point
%    in contributions]
%
% Organization B (Nickolai Zeldovich)
%  o Elevator story -- high-level statement of the problem
%   [Resource consumption of Rust programs is important.]
%  o The problem in more technical terms
%   [Devise a type system that automatically infers resource bounds of Rust programs.]
%  o Conventional wisdom: sketch of previous techniques and their
%    shortcomings
%   [1. type systems for resource analysis: AARA, sized types, etc.]
%   [2. type systems for Rust: Flux, etc.]
%    ... cite
%  o Describe the solution to the problem as a black box,
%    so that it is clear what our solution offers over previous work
%   [We propose a new approach of AARA to handle borrow mechanism.]
%  o Technical challenges to obtaining a solution (e.g., 1 paragraph
%    for each)
%   [1. Rust type system (borrow checker) is already very complex, and we want our resource analysis to be orthogonal to borrow checking.] {We adapt the methodology of Aeneas, extend LLBC to a resource-aware version, assuming borrow correctness, ...}
%   [2. Although AARA supports sharing can can easily support shared borrows, mutable borrows are challenging because potential return.] {Prophecy, previous work uses it for logical constraints of mutable borrows. We propose a prophecy mechanism to track potentials.}
%   [3. Strong/weak updates.] {Lattice?}
%  o State how we solve the challenges (at most a few paragraphs)
%  o Experimental highlights
%   [We collect xxx Rust programs.]
%  o Contributions
%   [1. A new AARA type system for Rust resource analysis]
%   [2. Formalization and soundness proof]
%   [3. An implementation of autoamtic type inference and evaluation]
%  o Organization of the paper
%   [Limitation: Although our type system supports xxx, but only limited to linear bounds, no unsafe code, no vectors, no higher-order, ...]
%
% Organization C (Derek Dreyer)
%  o Context: Set the stage, motivate the general topic
%  o Gap: Explain your specific problem and why existing work does not
%    adequately solve it
%  o Innovation: State what you've done that is new, and explain how it
%    helps fill the gap

\section{Introduction} \label{section:introduction}

%  o Elevator story -- high-level statement of the problem
%   [Resource consumption of Rust programs is important.]
%  o The problem in more technical terms
%   [Devise a type system that automatically infers resource bounds of Rust programs.]

Rust~\cite{sigada:MK14} has become a popular programming language that
aims to bridge the gap between fine-grained memory control and high-level memory safety,
especially in the context of system programming.
%
Rust introduces a complex type system that features linear types and borrow mechanisms,
enabling system engineers to write high-performance memory-manipulating code without
needing garbage collection or compromising memory safety.
%
The popularity of Rust indicates that system engineers still care about \emph{resource
consumption}---such as time, memory, and energy usage---of their programs, in order
to achieve high or predictable performance.
%
However, using Rust does \emph{not} directly mean achieving high performance;
in fact, there exists a book~\cite{misc:RPB20} that collects
techniques to improve the performance of Rust programs, including general and
Rust-specific techniques. Additionally, predicting the performance of a Rust program is often difficult to do manually.
%
Therefore, understanding the resource consumption---as a general topic in software
engineering---continues to be a crucial issue in Rust programming.

% Resource consumption is one critical property of Rust programs. A resource is an abstract concept of time and memory consumed during the execution of a program. It is expected to execute programs with a proper resource bound, not only for programmers but also for compilers and optimizers. Rust is a programming language aiming at high performance, especially for low-level programs. Therefore, it is significant to analyze the resource consumption of Rust programs. For real-world programs, it is necessary to present resource consumption in symbolic bounds with coefficients instead of in big-O notations. To automatically infer resource consumption of Rust programs, we devise a resource aware type system for Rust, and fuse resource information with type inference.

%  o Conventional wisdom: sketch of previous techniques and their
%    shortcomings
%   [1. type systems for resource analysis: AARA, sized types, etc.]
%   [2. type systems for Rust: Flux, etc.]
%    ... cite

In this paper, we study how to \emph{automatically} and \emph{statically} reason
about the resource consumption of Rust programs via a type system.
%
In particular, we want to understand how
the reasoning process could benefit from Rust's advanced type system.
%
Recent years have seen many studies on (i) static analysis for resource consumption,
and (ii) static analysis of Rust programs.
%
We highlight some mostly-related \emph{type-based} approaches below; other related work will be 
discussed in \cref{sec:related}.

% Analyzing properties of program via type system has been extensively studied. 

Automatic amortized resource analysis (AARA)---initially proposed by~\citet{AARA-Linear}---has
been a state-of-the-art framework for statically inferring symbolic resource bounds
of various kinds of programs~\cite{JMSCS:HJ22}.
%
The inferred bounds are \emph{symbolic} in the sense that they are interpreted as functions
of a program's input and output.
%
Despite the fact that most AARA approaches are focused on analyzing pure functional programs,
there are multiple exceptions related to memory control:
\citet{ESOP:Atkey10} studied the integration of AARA and separation logic,
\citet{PLDI:CHS15} developed C4B, which can analyze C programs with arrays,
and \citet{FSCD:LH17} proposed a method to analyze arrays and references in ML programs.
%
However, it is unclear whether any of those approaches can leverage Rust's type system---especially the borrow mechanisms---during the analysis.

% \cite{AARA-Linear}) is an approach to analyze ML-like functional programming languages, extending types with potential, deriving linear constraints of resource during type checking, obtaining upper bound by linear programming. Sized Type \cite{SizedTypes} \todo{cite, maybe more?} focuses mainly on termination of proof as program, indexing (dependent) types with sizes, add size-reducing condition as premise of type checking.

On the other hand, some recent studies build advanced type systems upon Rust's 
type system.
%
Flux~\cite{Flux} is a liquid-style refinement type system for Rust, which aims to
automatically verify the functional correctness of safe Rust programs.
%
Flux allows programmers to annotate types (including reference types) with decidable
logical predicates and then exploits Rust's borrow mechanisms to separate functional
verification from low-level memory reasoning.
%
RefinedRust~\cite{RefinedRust} is another refinement type system for Rust, which aims 
to carry out foundational reasoning and produce Coq proofs for the functional
correctness of both safe and unsafe Rust programs in a semi-automated way.
%
However, those approaches are focused on functional correctness rather than resource
consumption of Rust programs.

%   o Describe the solution to the problem as a black box,
%    so that it is clear what our solution offers over previous work
%   [We propose a new approach of AARA to handle borrow mechanism.]

In this paper, we propose \rarust{}, a new AARA approach that leverages Rust's 
borrow mechanisms and a prototype implementation that automatically infers linear 
resource bounds for safe Rust programs.
%
Rust's type system enforces \emph{ownership} principles around references:
one can \emph{borrow} a reference with ownership and later return it back,
and the type system checks the soundness of such borrows via a notion of \emph{lifetimes}.
%
Meanwhile, AARA enforces principles of \emph{amortized analysis}~\cite{JADM:Tarjan85}
around data structures: type annotations specify \emph{potential functions},
such that a value can \emph{own} some potentials to pay for resource consumption
and one can \emph{transfer} potentials among values,
and the type system checks the soundness of the potential-based reasoning.
%
Intuitively, our approach integrates the ownership principles of memory locations
and potentials.
%
Below we sketch three main technical challenges to obtaining such an integration and our solutions.

% Rust's borrow mechanism restricts the usage of borrows, or called pointers, and makes a strict distinction between shared borrows and mutable ones. Shared borrows are immutable and shared, whereas mutable borrows are mutable and exclusive. Therefore Rust's borrow mechanism provides a good property of borrows for resource analysis. Our work enrich types of shared and mutable borrows with resource potentials, and reflect the distinction between them from Rust type system to the world of resource analysis. The reflection finally handles Rust's borrow mechanism and analyze the resource consumption.

%  o Technical challenges to obtaining a solution (e.g., 1 paragraph
%    for each)
%   [1. Rust type system (borrow checker) is already very complex, and we want our resource analysis to be orthogonal to borrow checking.] {We adapt the methodology of Aeneas, extend LLBC to a resource-aware version, assuming borrow correctness, ...}
%   [2. Although AARA supports sharing can can easily support shared borrows, mutable borrows are challenging because potential return.] {Prophecy, previous work uses it for logical constraints of mutable borrows. We propose a prophecy mechanism to track potentials.}
%   [3. Strong/weak updates.] {Lattice?}

Firstly, Rust's type system is rather complex already, and its borrow mechanisms are being
updated actively, so we desire the resource analysis to be as orthogonal to Rust's own
borrow checking as possible.
%
Rust now features shared borrows, mutable borrows, reborrows,
and non-lexical lifetimes that allow complex uses of borrows.
%
Modifying Rust's type checker (especially the borrow checker) would be arduous and 
thus unacceptable.
%
To tackle this challenge, we propose a lightweight design that performs resource analysis
on \emph{well-borrowed} and \emph{well-typed} Rust programs.
%
\rarust{} relies on Rust's borrow checker to ensure the analyzed program to be well-borrowed
and then exploits Rust's borrow mechanism to associate data structures and references with
potentials.
%
We adapt the recently proposed Low-Level Borrow Calculus (LLBC)~\cite{Aeneas},
which is based on Rust's MIR, formulate a resource-aware version called Resource-Aware Borrow Calculus (RABC), and devise an AARA type system on RABC.

% The Rust borrow checker in the compiler has been already very complex for compiler developers, therefore we desire our resource analysis to be orthogonal to borrow checking. We then adapt the methodology of Aeneas\cite{Aeneas}, assume the correctness and properties of borrows guaranteed by Rust borrow checker, extend its Low Level Borrow Calculus(LLBC) to a resource-aware version. Our resource aware type system is designed to be subsequent to borrow checking in the compiling pipeline. This designation makes full use of existing code of Rust compiler.

Secondly, Rust's borrow mechanisms, especially mutable borrows, pose a non-trivial problem for AARA in tracking potentials along borrowing.
%
For example, one can mutably borrow a value $v$ from a location $\ell$ with $p(v)$ units 
of potentials, update the original value to $v'$, consume or store some potentials, and end the 
borrow to return $v'$ back to the location $\ell$ with $p'(v')$ units of potentials.
%
Note that the pre- and post-potential functions $p$ and $p'$ can be different; however,
in a typical AARA type system, types are \emph{invariant}, i.e., once a location $\ell$
has a resource-annotated type that specifies a potential function, the type would not change.
%
To tackle this challenge, we propose \emph{prophecy potentials}, which are inspired from
prophecy variables~\cite{ProphecyInSepLogic} and RefinedRust's borrow names~\cite{RefinedRust}.
%
Intuitively, prophecy variables provide a compositional reasoning mechanism for \emph{future}
states, as they stand for the value, returned to a location after a mutable borrow ends.
%
When a mutable borrow of a location $\ell$ with the potential function $p$ starts, \rarust{} generates a prophecy potential function $q$ for $\ell$ and tracks both $p$ and $q$ in
the mutable-borrow reference type.
%
When a mutable borrow with potential $p'$ and prophecy $q$ ends, \rarust{} enforces $p'$ to denote at least the same amount of potentials as $q$ denotes.

% Although AARA approach can easily support shared borrows of Rust, mutable borrows are challenging for resource analysis, because mutable borrows should return potential back to original places when borrows end. Prophecy variables\cite{ProphecyInSepLogic} are used to represent logical constraints of states in the future. RustHorn \cite{RustHorn} uses prophecy to capture future states of mutable borrows. We propose a prophecy mechanism to track resource potentials. We use prophecy types to represent the potential when the mutable borrow terminates. When mutable borrows are emit, we will fill original places with prophecy types, and move original types to mutable borrows. the  When borrows end, we only need to satisfy the linear constraint that the prophecy type are equal to current type of borrows. The constraint returns potential for borrows.

Thirdly, similarly to other static analyses of memory-manipulating programs, we desire
the resource analysis to deal with \emph{aliasing} soundly and precisely.
%
Rust's borrow checker already ensures some good properties about aliasing, e.g.,
aliasing and mutation cannot happen simultaneously, so it seems sound to always perform
precise \emph{strong updates} for mutation.
%
However, the type system may face a situation where a mutable reference points to
multiple locations, e.g., a conditional expression whose two branches borrow from two
difference locations.
%
In this case, the type system should not perform strong updates to the locations,
especially when the program stores some potentials via the mutable reference: it is unsound to increment both locations with the potentials.
%
To tackle this challenge, we propose a \emph{lattice} of resource-annotated types (including reference types) based on a subtyping relation.
%
\rarust{} then uses the lattice to compute the greatest lower bound of types for
a location when different control-flow paths join at a program location.

% An inevitable challenge is the aliasing problem. Some might realize that Rust borrow mechanism is made to avoid aliasing in runtime, because mutable borrow is exclusive to be unique. However, in the view of type system, it is not the case, due to branching statements like $\kwd{if}$. Without concrete knowledge about boolean expression in $\kwd{if}$, type system can only conservatively consider that each branch is possible, therefore different mutable borrows may point to the same place. Aliasing problem is critical for mutable borrows, because increased potential can only return to the exact place in runtime, otherwise the resource will increase many times. The solution is the lattice algebra and subtyping relation over resource aware types. The type system will consider the greatest lower bound of types in different branches as the type after branching statement. 


%  o Experimental highlights
%   [We collect xxx Rust programs.]

To summarize, we propose \rarust{}, which consists of a resource-aware calculus RABC for well-borrowed Rust programs and an AARA type system with novel prophecy potentials.
%
We formally prove the soundness of the type system with respect to RABC
and
implement a prototype of \textsc{RaRust} for inferring linear bounds for safe Rust programs.
%
We collect a suite of Rust programs to evaluate the effectiveness of the prototype, including non-trivial manipulations of mutable lists and trees.
%
Evaluation results show that the prototype successfully infers symbolic resource bounds. 

\paragraph{Contributions}
%
The paper makes the following three contributions:
\begin{itemize}
    \item {We propose \rarust{}, which aims at automatic amortized resource analysis for Rust programs. \rarust{} features a lightweight design that works on well-borrowed programs, novel prophecy potentials to handle mutable borrows, and a type lattice to deal with aliasing.}
    \item {We present a formulation of \rarust{}, including Resource-Aware Borrow Calculus (RABC) and an AARA type system. We also formally prove the soundness of the type system.}
    \item {We implement a prototype of \rarust{}, evaluate it on a suite of non-trivial Rust programs, and show that it can infer symbolic resource bounds effectively.}
\end{itemize}

%   [Limitation: Although our type system supports xxx, but only limited to linear bounds, no unsafe code, no vectors, no higher-order, ...]

\paragraph{Limitations}
%
Despite the contributions,
it is worth noting that \rarust{} currently still has some limitations:
(i) \rarust{} does not support unsafe code, which is a key feature of Rust,
(ii) \rarust{} does not support generic types, higher-order functions (i.e., closures), or trait objects,
(iii) \rarust{} does not support non-linear resource bounds.
%
We will discuss those limitations and sketch pathways for overcoming
them in \cref{sec:discussion}.

% There are some limitations of our formalization and implementation, but they are orthogonal to our core idea on Rust's borrow mechanism. Our works focuses solely on linear bounds, without support of polynomial bounds and other bounds, which can be done as AARA does \cite{AARA-Poly}\cite{AARA-Exp}. Our implementation \textsc{RaRust} also does not support vectors in standard libraries, higher order function and closures. We believe that theses built-in data structures and types can be extended to our formalization and implementation.

\paragraph{Organization}
%
The paper is organized as follows.
%
\cref{sec:background} briefly reviews Rust's borrow mechanisms and automatic amortized resource analysis (AARA).
%
\cref{sec:overview} uses a few running examples to explain the design and key ideas of \rarust{}.
%
\cref{sec:calculus} formulates Resource-Aware Borrow Calculus (RABC) and the resource-aware dynamic semantics.
%
\cref{sec:inference} presents an AARA type system for Rust's borrow mechanisms and a type-inference algorithm.
%
\cref{sec:soundness} sketches the soundness proof of the AARA type system with respect to RABC.
%
\cref{sec:impl} describes the prototype implementation and experimental evaluation.
%
\cref{sec:discussion,sec:related} discuss limitations and related work, respectively. 
%
\cref{sec:conclusion} concludes.

% \todo{may DELETE organization paragraph}
% The rest of the paper is organized as follows : In Section \ref{sec:background}, we give a simple introduction to Rust's borrow mechanism and Automatic Amortized Resource Analysis. In Section \ref{sec:overview}, by presenting running examples, we briefly explain types with potentials, especially mutable borrow types utilizing prophecy variables, and reveal how to tame borrows. In Section \ref{sec:calculus}, we define syntax and big step resource aware dynamic semantics. Section \ref{sec:inference} presents our type system based on linear programming. Section \ref{sec:soundness} give out the definition of potential functions and a proof sketch of soundness theorem. Section \ref{sec:impl} introduces the prototype implementation of our calculus and the evaluation on a suite of Rust programs. Finally we discuss related works in Section \ref{sec:related}, conclude our contribution and future works in Section \ref{sec:conclusion}.