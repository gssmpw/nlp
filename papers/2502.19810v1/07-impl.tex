\section{Experimental Evaluation} \label{sec:impl}

In this section, we present an experimental evaluation of \rarust{}.
%
\cref{sec:proto} describes our prototype implementation of \rarust{}.
%
\cref{sec:eval} presents the evaluation results of \rarust{} on a suite of benchmarks.

\subsection{Implementation}
\label{sec:proto}

% In this section, we will describe the work flow, and supported features of implementation. 

We have implemented a prototype linear resource analyzer, \rarust{}. It takes raw Rust programs (within only $\kwd{tick}$ annotation) as inputs, and prints functions' signatures with resource annotation as output.
%
\rarust{} analyzes the whole program regardless of whether there are annotations or not. We currently use explicit manually annotated $\kwd{tick}$ as the cost model, but it is straightforward to support parametric cost models, which assign a cost to each kind of statements, as prior AARA systems (e.g., \textsc{RaML}~\cite{RaML}) do.
(1) \rarust{} first obtains the typed IR of the borrow calculus with explicit drops via Charon \cite{Aeneas} as a plugin of Rust compiler. Rust compiler guarantees that the compiled IR is well-checked, and \rarust{} will utilize properties straightforwardly. 
(2) Towards IR, \rarust{} analyzes strongly connected components of function call graph and topologically sort components, generating precedence of type checking. 
(3) \rarust{} enriches function signatures with fresh linear variables as resource annotation and assigns each component a linear programming context to record linear constraints. 
(4) As stated in \cref{sec:inference:infer}, \rarust{} type-checks functions' bodies, generating linear constraints.
% For function call, when the function is in the same component, \rarust{} will only use the signature; when the function is in previous component, \rarust{} will clone previous linear programming context, add to current context, and use the cloned signature.
(5) \rarust{} finally solves these constraints, with its heuristic linear objective, by invoking linear programming solvers. We can utilize different solvers due to the separation of (4) and (5), and we select CoinCbc \cite{CoinCbc} for demonstration.

Our implementation supports some features, based on core calculus formalized in \cref{sec:calculus} and some practical extensions. We list as follows:
\begin{enumerate}
    \item{
    \rarust{} supports reborrows and nested borrows like \lstinline|& & T|, \lstinline|&mut & T| and \lstinline|&mut &mut T|.
    }
    \item {
    \rarust{} supports user-defined data types instead of only built-in lists, as \cref{tab:user-defined} shows. 
    We annotate potential for each constructor, \lstinline|Nil|, \lstinline|Cons|, \lstinline|Leaf|, \lstinline|Node|, \lstinline|(_, _)|, \lstinline|Record(_, _)|, \lstinline|NListNode| and \lstinline|NList|. It is worth noting that \lstinline|NListNode| and \lstinline|NList| are mutually recursive.
    }
    \item {
    \rarust{} supports looping statements including $\kwd{while true}(s), \kwd{break}(i), \kwd{continue}(i)$, where $i$ represents the $i$-th loop outward from $\kwd{break}$ or $\kwd{continue}$. \rarust{} supports this directly without transforming loops into recursive functions.
    }
\end{enumerate}


\begin{DIFnomarkup}
\begin{table}[t]
\centering
\caption{User-defined Data Types} \label{tab:user-defined}
\footnotesize
\begin{tabular}{|l|l|}
\hline
\begin{lstlisting}[language=Rust, style=colouredRust]
pub enum List {
    Nil,
    Cons(i32, Box<List>),
}
\end{lstlisting} 
&
\begin{lstlisting}[language=Rust, style=colouredRust]
pub enum Tree {
    Leaf,
    Node(i32, Box<Tree>, Box<Tree>),
}
\end{lstlisting} 
\\
\hline
\begin{lstlisting}[language=Rust, style=colouredRust]
pub type Tuple = (List, Tree);
\end{lstlisting}
&
\begin{lstlisting}[language=Rust, style=colouredRust]
pub struct Record{pub l:List, pub t:Tree}
\end{lstlisting}
\\
\hline
\begin{lstlisting}[language=Rust, style=colouredRust]
pub struct NListNode {
    pub value : i32,
    pub next : NList
}
\end{lstlisting} 
&
\begin{lstlisting}[language=Rust, style=colouredRust]
pub enum NList {
    None,
    Some(Box<NListNode>)
}
\end{lstlisting} 
\\
\hline
\end{tabular}
\end{table}
\end{DIFnomarkup}

\subsection{Evaluation}
\label{sec:eval}

\begin{DIFnomarkup}
\begin{table}[t]
    \centering
    \caption{Benchmarks}
    \label{tab:eval}
    \small
    \scalebox{0.67}{
    \begin{tabular}{|c|c|c|c|c|c|}
    \hline
    case & type & description & complexity & lines of code & size of constraints  \\
    \hline
    \multicolumn{6}{|c|}{\textbf{feature(s): mutable borrows}} \\ \hdashline
    end\_m & {\lstinline|fn(l:&m List)->&m List|} & find the mutable borrow of the {\lstinline|Nil|} of a list & $1+3|l|$ & 14 & 71 \\
    end\_c & {\lstinline|fn(l:&m List,o:&m&m List)|} & c style end\_m to write to {\lstinline|o|} & $2+3|l|$ & 13 & 96 \\ 
    append & {\lstinline|fn(l:&m List,x:i32)|} & append $x$ to $l$ by {\lstinline|end_m|} & $5+3|l|$ & 5 & 89 \\
    concat & {\lstinline|fn(l1:&m List,l2:List)|} & $l'_1 = l_1@l_2$ by {\lstinline|end_c|} & $6+3|l_1|$ & 8 & 144 \\
    \hline
    \multicolumn{6}{|c|}{\textbf{feature(s): shared borrows, mutable borrows, recursive functions, and loop statements}} \\ \hdashline
    sum\_rec & {\lstinline|fn(l:& List)->i32|} & sum up integers, recursively & $1 + 6|l|$ & 14 & 19 \\
    sum\_loop & {\lstinline|fn(l:& List)->i32|} & sum up integers, via loops & $1 + 6|l|$ & 24 & 54 \\
    
    rev\_rec & {\lstinline|fn(l:&m List, r: List)->List|} & reverse $l$ to head of $r$ & $1 + 9|l|$ & 15 & 57 \\
    
    rev\_loop & {\lstinline|fn(l:&m List)|} & reverse $l$ mutably via loops & $1 + 9|l|$ & 21 & 164 \\
    % remove rich
    % \hline
    % sum & \lstinline|fn(l:& List)->i32| & sum up integers in list & $1 + 6|l|$ \\
    % rich & \lstinline|fn(l:&m List)| & iterate $l$ with $-1, -6$ as $\kwd{tick}$ & $-6|l|$ \\
    % rich\_sum & \lstinline|fn(l:&m List)->i32| & \lstinline|rich(l)|, and then \lstinline|sum(l)| & $1$ \\
    \hline
    \multicolumn{6}{|c|}{\textbf{feature(s): function calls}} \\ \hdashline
    sum2 & {\lstinline|fn(l:& List)|} & {\lstinline|sum_rec(l);sum_loop(l)|} & $2 + 12|l|$ & 4 & 82 \\
    rev & {\lstinline|fn(l:&m List)|} & reverse $l$ mutably via {\lstinline|rev_rec|} & $4 + 9|l|$ & 5 & 71 \\
    rev2 & {\lstinline|fn(l:&m List)|} & apply {\lstinline|rev|} to $l$ twice & $8 + 18|l|$ & 4 & 170 \\
    dup & {\lstinline|fn(l: List)->List|} & duplicate each element in $l$ & $1+11|l|$ & 16 & 37 \\
    dup2 & {\lstinline|fn(l: List)->List|} & apply {\lstinline|dup|} to $l$ twice & $2+33|l|$ & 4 & 81 \\
    \hline
    \multicolumn{6}{|c|}{\textbf{feature(s): reborrow and nested borrows}} \\ \hdashline
    reborrow\_s & {\lstinline|fn(l:& List)|} & reborrow $l$ as $ll$, {\lstinline|sum2(ll);sum2(l)|}& $4+24|l|$ & 6 & 176 \\
    reborrow\_m & {\lstinline|fn(l:&mut List)|} & reborrow $l$ as $ll$, {\lstinline|rev2(ll);rev2(l)|}& $16+36|l|$ & 6 & 364 \\
    nested\_s\_s & {\lstinline|fn(l:& & List)|} & {\lstinline|sum2(*l);|} & $2+12|l|$ & 4 & 88 \\
    nested\_m\_s & {\lstinline|fn(l:&m & List)|} & {\lstinline|sum2(*l);|} & $2+12|l|$ & 4 & 90 \\
    nested\_m\_m & {\lstinline|fn(l:&m &m List)|} & {\lstinline|rev2(*l);|} & $8+18|l|$ & 4 & 188 \\
    \hline
    \multicolumn{6}{|c|}{\textbf{feature(s): user-defined datatypes}} \\ \hdashline
    min & {\lstinline|fn(t:&Tree,d:i32)->i32|} & find min in $t$, $d$ as default & $1+5|t|$ & 14 & 19 \\
    max & {\lstinline|fn(t:&Tree,d:i32)->i32|} & find max in $t$, $d$ as default & $1+5|t|$ & 14 & 19 \\
    find & {\lstinline|fn(t:&Tree,x:i32)->bool|} & find whether $x$ is in $t$ & $1+8|t|$ & 29 & 31 \\
    add & {\lstinline|fn(t:&m Tree,x:i32)|} & add up $x$ to each element of $t$ & $1+10|t|$ & 15 & 53\\
    tuple & {\lstinline|fn(x:&m Tuple)|} & {\lstinline|rev2(x.0);min(x.1, 0);|} & $9+18|x.0|+5|x.1|$ & 4 & 216 \\
    record & {\lstinline|fn(x:&m Record)|} & {\lstinline|rev2(x.l);min(x.t, 0);|} & $9+18|x.l|+5|x.t|$ & 4 & 216 \\
    sum\_rec\_nlist & {\lstinline|fn(l:&NList) -> i32|} & sum up {\lstinline|NList|} as \lstinline|sum_rec| & $1+5|l|$ & 16 & 26 \\
    rev\_rec\_nlist & {\lstinline|fn(l:&mut NList,r:NList)->NList| } & reverse {\lstinline|NList|} as \lstinline|rev_rec| & $1+7|l|$ & 21 & 89 \\
    \hline
    \end{tabular}
    }
\end{table}
\end{DIFnomarkup}

We used the prototype implementation of \rarust{} to perform an experimental evaluation of resource analysis on typical examples concerning Rust's borrow mechanism. We adapted several pure functional examples from \textsc{RaML} to their Rust counterparts employing borrows; by using borrows we can do in-place updates in Rust. Some examples were deliberately crafted to demonstrate the prototype’s capability to support the aforementioned features. Due to the linear limitation, we select those examples with linear complexity. The experiments were performed on a laptop with 2.20 GHz Intel Core i9-13900HX as CPU and WSL 2.3.24.0/Ubuntu 22.04.3 LTS as operation system. The compiling of the benchmarks was done in 0.15s and the analysis was done in 0.3s.

\cref{tab:eval} shows our experimental results. We manually checked the analysis results on the benchmarks and confirmed that the derived bounds are correct (but not tight for some benchmarks such as \lstinline|min| and \lstinline|max|). We encode some cases with the derived bounds in \textsc{Flux}~\cite{Flux} to check that the derived bounds are correct. Our encoding introduces a global counter to accumulate resource consumption. The encoding is shipped into the artifact and it includes cases \lstinline|append|, \lstinline|concat|, \lstinline|sum_rec|, \lstinline|rev_rec|, \lstinline|sum2|, \lstinline|rev|, \lstinline|rev2|, \lstinline|dup|, \lstinline|dup2|, \lstinline|min|, \lstinline|max|, \lstinline|find|, \lstinline|add|, \lstinline|tuple|, \lstinline|record|, \lstinline|sum_rec_nlist|, and \lstinline|rev_rec_nlist|.

\cref{tab:eval} gives out 5 groups of test cases, and each group exercises some features. For each analyzed function as a case, \cref{tab:eval} first declares the function type in a simplified syntax of Rust, writing \lstinline|&mut T| as \lstinline|&m T| for short. The description column provides a concise explanation of the function's semantics. We plot the complexity in a more readable format in the table, where $|l|$ represents the length or count of \lstinline|Cons| constructors of $l$ and $|t|$ represents the count of \lstinline|Node| constructors of the tree $t$. The concrete coefficients are inferred by \rarust{} according to annotation $\kwd{tick}(\delta)$ in examples. In our evaluation, we set those different concrete numbers of $\delta$ for two purposes: (i) we roughly add one $\kwd{tick}(\delta)$ around one statement to account for the number of operations by the statement, and (ii) we can use different numbers to test multiple times to check if our implementation is correct.
%
We will explain each group in detail in the rest of this section.

% \textbf{Mutable Borrows and Nested Borrows:} 
To show that \rarust{} can handle mutable borrows, we construct cases \lstinline|end_m| and \lstinline|end_c|. They are recursively to find the mutable borrow of the nil of a list $l$. For example, in ML syntax, the nil of the list \verb|1::2::3::4::[]| is \verb|[]|. \lstinline|end_m| returns the borrow, while  \lstinline|end_c| storing it in the parameter \lstinline|o:&m&m List|. The returned mutable borrow of \lstinline|end_m| works as a closure function with type \lstinline|List->List|, therefore it is non-trivial for resource analysis. We use cases \lstinline|append| and \lstinline|concat| to show the resource correctness as well as the compositionality of the analysis.

%\textbf{Recursive Functions and Loop Statements:} 
Rust programmers are able to write code with loop statements. We construct cases \lstinline|sum_rec|, \lstinline|sum_loop|, \lstinline|rev_rec| and \lstinline|rev_loop|. We focus on shared borrows in \lstinline|sum| and on mutable borrows in \lstinline|rev|. The suffix \lstinline|rec| means recursive function and \lstinline|loop| means loop statements \lstinline|while true { ... }|. The same analysis results reveal that both are supported by \rarust{}. 

% \textbf{Sharing and Prophesying:} 
We construct multiple calls of function for shared borrows and mutable borrows, to demonstrate sharing and prophesying. The suffix \lstinline|2| means twice in cases \lstinline|sum2|, \lstinline|rev2| and \lstinline|dup2|. The coefficients in the complexity of \lstinline|sum2| are exactly 2 times of \lstinline|sum|, testing the sharing for shared borrows. \lstinline|rev2| is similar but for the prophesying of mutable borrows. \lstinline|dup2| is made to point out the difference between sharing and prophesying. The function \lstinline|dup| mutates the length of list $l$, therefore the second call of \lstinline|dup| is with length $2|l|$, making the linear coefficient of \lstinline|dup2| be $33$, 3 times $11$.


% \textbf{Reborrows and Nested Borrows:} 
\rarust{} also supports reborrows and nested borrows. We construct cases \lstinline|reborrow_s|, \lstinline|reborrow_m|, \lstinline|nested_s_s|, \lstinline|nested_m_s| and \lstinline|nested_m_m|. The suffix \lstinline|s| denotes shared borrows, while \lstinline|m| denotes mutable borrows.

% \textbf{User-defined Data Types:} 
Besides \lstinline|List|, we construct simple examples for other safe user-defined data types like trees, tuples, records, and C-style lists. The sizes of trees are the counts of internal nodes \lstinline|Node|, instead of the intuitive measure, their depths. \lstinline|Tuple| and \lstinline|Record| are data types to compose \lstinline|List| and \lstinline|Tree|, with complexity as the linear composition of their fields', such as \lstinline|x.0| and \lstinline|x.t|. \rarust{} also supports mutually recursive data types like \lstinline|NList| and \lstinline|NListNode|; they are C-style lists, with the former as the nullable pointer, the latter as the internal node of lists.