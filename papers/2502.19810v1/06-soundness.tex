\section{Soundness} \label{sec:soundness}

In this section, we define potential functions indicated by the resource-enriched types from \cref{sec:inference} in \cref{sec:pot-funcs} and then sketch the soundness proof of the resource-aware type system in \cref{sec:proof-sketch}. We include the detailed proofs in the \cref{sec:proof}.

\subsection{Potential Functions}
\label{sec:pot-funcs}
 
% We always assume that $v$ is well typed with $\tau$, stated as $\vdash v : \tau$, when we notate $\phi(v:\tau)$. We omit the definition in literal, as well as well typing over store and context as $\vdash V : \Gamma$, both guaranteed by Rust type system.

% \begin{definition}
%    $ \Phi(V:\Gamma) = \sum_{x\in\textbf{dom}(\Gamma)} \phi(V(x):\Gamma(x)) $ .
% \end{definition}
\begin{figure}[t]
\small
    \judgement{Potential Functions}{$\Phi(V:\Gamma) = \sum_{x\in\textbf{dom}(\Gamma)} \phi(V(x):\Gamma(x))$} \\
    \judgement{Potential Functions (Selected)}{$\phi(v:\tau)$}
    \begin{mathpar}
    \inferrule[$\phi$-Nil]
    {~}
    {\phi(\kwd{nil}:\kwd{list}(\alpha))=0}
    \and
    \inferrule[$\phi$-Cons]
    {~}
    {\phi(\kwd{cons}(\kwd{n}_\text{i32}, \kwd{box}(lv)):\kwd{list}(\alpha))=\alpha+\phi(\kwd{box}(lv):\kwd{box}(\kwd{list}(\alpha)))}
    \\
    \inferrule[$\phi$-Shared]
    {~}
    {\phi(\&(\_, v):\&^\kwd{s}(\tau))=\phi(v:\tau)}
    \and
    \inferrule[$\phi$-Mutable]
    {~}
    {\phi(\&(\_, v):\&^\kwd{m}(\tau_\text{c}, \tau_\text{p}))=\phi(v:\tau_\text{c})-\phi(v:\tau_\text{p})}
    \end{mathpar}
    \caption{Potential Functions}
    \label{fig:sound-potential}
\end{figure}

\cref{fig:sound-potential} defines potential function $\phi(v:\tau)$ and $\Phi(V:\Gamma)$.
% 
\rulename{$\phi$-Nil} and \rulename{$\phi$-Cons} define the potential of a list $l : \kwd{list}(\alpha)$ with length $n$ to be $\alpha \cdot n$.
%
\rulename{$\phi$-Shared} defines the potential of shared borrows to be the potential of borrowed values and borrowed rich types. 
%
\rulename{$\phi$-Mutable} defines the potential of mutable borrows to be the difference between current and prophecy potential. It is worth noting that when the program incurs a mutable borrow, the potential of context will not change. The dropping condition $\tau_\text{p} \preceq \tau_\text{c}$ in $\vdash \&^\kwd{m}(\tau_\text{c}, \tau_\text{p})$, ensures the potential is non-negative. We, therefore have the following lemma about potential:

\begin{lemma}
    Potential is non-negative and keeps subtyping:
    \begin{mathpar}
    \inferrule
    {  \vdash \tau_1
    \\ \vdash \tau_2
    \\ \tau_1 \preceq \tau_2
    }
    { 0 \leq \phi(v:\tau_1) \leq \phi(v:\tau_2)}
    \end{mathpar}
\end{lemma}
\begin{proof}
    First define the size of rich types structural inductively $\mathbf{size} : \mathbf{RichType} \to \NN$, \ 
    and prove by induction on $\mathbf{size}(\tau_1) + \mathbf{size}(\tau_2)$. The well-formedness will be used to prove potential of mutable borrows is non-negative.
\end{proof}
\begin{corollary}
    Potential is non-negative $\dfrac{\vdash \tau}{0\leq \phi(v:\tau)}$ due to the derivable rule \rulename{S-Refl} $\dfrac{~}{\tau\preceq\tau}$.
\end{corollary}

The subtyping relation and well-formedness can be extended to typing context, which can be conceptualized as record types. Lemmas about the lattice operation and potential can be proved with definition and simple induction.

\begin{definition}
    Subcontext $\Gamma_1 \subseteq \Gamma_2$ if and only if $\forall x \in \textbf{dom}(\Gamma_1), x \in \textbf{dom}(\Gamma_2), \Gamma_1(x)\preceq \Gamma_2(x)$. \\
    Context well formed $\vdash \Gamma$ if and only if $\forall x \in \textbf{dom}(\Gamma), \vdash \Gamma(x)$.
\end{definition}
\begin{lemma}[]
    For any store $V$ and any context $\Gamma_1, \Gamma_2$, 
    \begin{enumerate}
        \item {$\Gamma_1\sqcap\Gamma_2 \subseteq \Gamma_1$ and $\Gamma_1\sqcap\Gamma_2 \subseteq \Gamma_2$; }
        \item {if $\vdash \Gamma_1, \vdash \Gamma_2$, then $\vdash \Gamma_1\sqcap\Gamma_2$; }
        \item {if $\vdash \Gamma_1, \vdash \Gamma_2, \Gamma_1 \subseteq \Gamma_2$, then $0\leq \Phi(V:\Gamma_1) \leq \Phi(V:\Gamma_2)$. }
    \end{enumerate}
\end{lemma}


\subsection{Soundness Theorem}
\label{sec:proof-sketch}

With potential, we are able to formulate soundness theorem, which states that resource consumption of dynamics, together with potential difference, is bounded by type system.
\begin{theorem}[Soundness]
Our type system is sound towards resource aware dynamic semantics: \\
If $V\vDash s \rightsquigarrow^{\delta_V} \Dashv V'$ and $\Gamma \vdash s \hookrightarrow^{\delta_\Gamma} \dashv \Gamma'$, then $\Phi(V':\Gamma') - \Phi(V:\Gamma)+\delta_V \leq \delta_\Gamma$.
\end{theorem}
\begin{proof}
By induction on $V\vDash s \rightsquigarrow^{\delta_V} \Dashv V'$, with the help of following lemmas.
\end{proof}

% Lemmas are listed as follows, and proof is simply done by induction with borrow properties.

\begin{lemma}[Update]
If store or context is written with new value or new type, the difference of potential over store or context is equal to that over new value or new type.
\begin{enumerate}
    \item {If $V\vDash p\rightsquigarrow v$, $\Gamma\vdash p\hookrightarrow \tau$ and $\VWt{V}{p}{v'}{V'}$,
    then $\Phi(V':\Gamma) - \Phi(V:\Gamma)=\Phi(v':\tau)-\Phi(v:\tau)$;}
    \item {If $V\vDash p\rightsquigarrow v$, $\Gamma\vdash p\hookrightarrow \tau$ and $\GWt{\Gamma}{p}{\tau'}{\Gamma'}$, 
    then $\Phi(V:\Gamma')-\Phi(V:\Gamma)=\Phi(v:\tau')-\Phi(v:\tau)$.}
\end{enumerate}
\end{lemma}
\begin{proof}
By induction on $\VWt{V}{p}{v'}{V'}$ and $\GWt{\Gamma}{p}{\tau'}{\Gamma'}$.
\end{proof}
\begin{lemma}[Evaluation]
If $V\vDash e\rightsquigarrow v, \Gamma\vdash e\hookrightarrow \tau\dashv\Gamma'$, then $\Phi(V:\Gamma')-\Phi(V:\Gamma)= -\phi(v:\tau)$.
\end{lemma}
\begin{proof}
By induction on $e$.
\end{proof}

Towards soundness proof for statements, especially for the rule \rulename{$\Gamma$-Ex-App}, we will need one series of lemmas about context extension. Though similar, context extension is different from subcontext. Context extension necessitates strict type equality, whereas subcontext only demands subtyping. The weakening rules are intuitive about context extension, as they merely involve adding variables and types that the program will not utilize. 

\begin{definition}
    Extension $\Gamma_1\sqsubseteq\Gamma_2$ if and only if $\forall x \in\textbf{dom}(\Gamma_1), x\in\textbf{dom}(\Gamma_2), \Gamma_1(x) = \Gamma_2(x)$.
\end{definition}
\begin{lemma}[Weakening]
In following rules, $\dfrac{A ~ B}{C ~ D}$ means if $A$ and $B$ then $C$ and $D$.
\begin{mathpar}
    \inferrule[$\Gamma$-Rd-Weaken]
    {\Gamma_1 \sqsubseteq \Gamma_2 
    \\ \Gamma_1 \vdash p \hookrightarrow \tau_1
    }
    {\Gamma_2 \vdash p \hookrightarrow \tau_2
    \\ \tau_1 = \tau_2
    }
    \and
    \inferrule[$\Gamma$-Wt-Weaken]
    {\Gamma_1 \sqsubseteq \Gamma_2
    \\ \GWt{\Gamma_1}{p}{\tau_1}{\Gamma'_1}
    \\ \tau_1 = \tau_2
    }
    { \GWt{\Gamma_2}{p}{\tau_2}{\Gamma'_2}
    \\ \Gamma'_1 \sqsubseteq \Gamma'_2 
    }
    \\
    \inferrule[$\Gamma$-Ev-Weaken]
    {\Gamma_1 \sqsubseteq \Gamma_2
    \\ \Gamma_1 \vdash e \hookrightarrow \tau_1 \dashv \Gamma'_1 
    }
    {\Gamma_2 \vdash e \hookrightarrow \tau_2 \dashv \Gamma'_2
    \\ \tau_1 = \tau_2 
    \\ \Gamma'_1 \sqsubseteq \Gamma'_2
    }
    \and
    \inferrule[$\Gamma$-Ex-Weaken]
    {\Gamma_1 \sqsubseteq \Gamma_2
    \\ \Gamma_1 \vdash s \hookrightarrow^{\delta_1} \dashv \Gamma'_1
    }
    { \Gamma_2 \vdash s \hookrightarrow^{\delta_2} \dashv \Gamma'_2
    \\ \delta_1 = \delta_2
    \\ \Gamma'_1 \sqsubseteq \Gamma'_2
    }
\end{mathpar}
\end{lemma}
\begin{proof}
    By induction on $\Gamma_1\vdash p\hookrightarrow \tau_1$, $\GWt{\Gamma_1}{p}{\tau_1}{\Gamma'_1}$, $\Gamma_1\vdash e\hookrightarrow\tau_1\dashv\Gamma'_1$, $\Gamma_1\vdash s\hookrightarrow^{\delta_1}\dashv\Gamma'_1$.
\end{proof}

