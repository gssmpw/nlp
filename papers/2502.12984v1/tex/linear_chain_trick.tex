\section{Approximate system of ordinary differential equations}\label{sec:lct}
In this section, we use the Erlang mixture approximation presented in Section~\ref{sec:approx} to approximate the multivariate kernel in the DDEs presented in Section~\ref{sec:system}, and we use the LCT to transform the resulting approximate system to ODEs. Furthermore, we describe the steady state equations for the approximate system of ODEs together with a stability criterion. Finally, we present a comparison of the stability criteria for the approximate and original system.

First, we approximate the system of DDEs~\eqref{eq:system:x}--\eqref{eq:system:delay} by
%
\begin{subequations}\label{eq:lct:approximate:system:DDE}
	\begin{align}
		\dot{\hat x}(t) &= f(\hat x(t), \hat z(t)), \\
		\hat z(t) &= \int\limits_{-\infty}^t \hat \alpha(t - s) \odot \hat r(s) \incr s, \\
		\hat r(t) &= h(\hat x(t)),
	\end{align}
\end{subequations}
%
where each element of the kernel $\hat \alpha: \Rnn^{n_z} \rightarrow \Rnn^{n_z}$ is an Erlang mixture kernel:
%
\begin{align}\label{eq:lct:approximate:kernel:DDE}
	\hat \alpha_i(t) &= \sum_{m=0}^{M_i} c_{m, i} \ell_{m, i}(t), & i &= 1, \ldots, n_z.
\end{align}
%
The subscript $i$ on the Erlang kernels, $\ell_{m, i}: \Rnn \rightarrow \Rnn$, indicates that it depends on the $i$'th rate parameter, $a_i \in \Rp$. Furthermore, $\hat x: \R \rightarrow \R^{n_x}$ and $\hat z, \hat r: \R \rightarrow \R^{n_z}$ are approximations of $x$, $z$, and $r$, respectively.%, and in Section~\ref{sec:algo}, we discuss how to determine the coefficients, $c_{m, i} \in [0, 1]$ for $m = 0, \ldots, M_i$ and $i = 1, \ldots, n_z$, and the orders, $M_i \in \Nnn$, and rate parameters, $a_i$, for $i = 1, \ldots, n_z$.

\begin{theorem}\label{thm:linear:chain:trick}
	The system of DDEs~\eqref{eq:lct:approximate:system:DDE} with the kernel~\eqref{eq:lct:approximate:kernel:DDE} is equivalent to the system of ODEs
	%
	\begin{subequations}\label{eq:lct:approximate:system:ODE}
		\begin{align}
			\label{eq:system:approximate:x}
			\dot{\hat x}(t) &= f(\hat x(t), \hat z(t)), \\
			%
			\label{eq:system:approximate:Z}
			\dot{\hat Z}(t) &= A \hat Z(t) + B \hat r(t), \\
			%
			\label{eq:system:approximate:z}
			\hat z(t) &= C \hat Z(t), \\
			%
			\hat r(t) &= h(\hat x(t)),
		\end{align}
	\end{subequations}
	%
	where $\hat Z: \R \rightarrow \R^{M + n_z}$ contains auxiliary memory states, and $M = \sum_{i=1}^{n_z} M_i$. Furthermore, the matrices $A\in \R^{M + n_z \times M + n_z}$, $B \in \Rnn^{M + n_z \times n_z}$, $C \in [0, 1]^{n_z \times M + n_z}$ are block-diagonal:
	%
	\begin{subequations}
		\begin{align}
			A &= \blkdiag\left(A^{(1)}, A^{(2)}, \ldots, A^{(n_z)}\right), \\
			B &= \blkdiag\left(b^{(1)}, b^{(2)}, \ldots, b^{(n_z)}\right), \\
			C &= \blkdiag\left(c^{(1)}, c^{(2)}, \ldots, c^{(n_z)}\right).
		\end{align}
	\end{subequations}
	%
	The block-diagonal elements, $A^{(i)} \in \R^{(M_i+1) \times (M_i+1)}$, $b^{(i)} \in \Rnn^{M_i+1 \times 1}$, and $c^{(i)} \in [0, 1]^{1 \times M_i+1}$, are given by
	%
	\begin{subequations}
		\begin{align}
			A^{(i)} &= a_i (L^{(i)} - I^{(i)}), &
			b^{(i)} &= a_i e_1^{(i)}, &
			c^{(i)} &=
			\begin{bmatrix}
				c_{0, i} & c_{1, i} & \cdots & c_{M_{i}, i}
			\end{bmatrix}, & i &= 1, \ldots, n_z,
		\end{align}
	\end{subequations}
	%
	where $I^{(i)}, L^{(i)} \in \R^{M_i+1 \times M_i+1}$ are an identity matrix and a lower shift matrix, respectively, i.e., $L^{(i)}$ contains ones in the first subdiagonal and zeros elsewhere. Furthermore, the first element of $e_1^{(i)} \in \R^{M_i+1 \times 1}$ is one and the others are zero.
\end{theorem}
%
The transformation of the system of DDEs~\eqref{eq:lct:approximate:system:DDE} to the system of ODEs~\eqref{eq:lct:approximate:system:ODE} is standard~\cite{MacDonald:1978, Ponosov:etal:2004}. Therefore, we postpone the proof of Theorem~\ref{thm:linear:chain:trick} to Appendix~\ref{sec:deriv}.
%
\begin{theorem}\label{thm:lct:approximate:system:ODE:steady:state}
	A steady state, $\bar x \in \R^{n_x}$, of the approximate system of ODEs~\eqref{eq:lct:approximate:system:ODE} satisfies
	%
	\begin{align}\label{eq:lct:steady:state}
		0 &= f(\bar x, \bar z), & \bar z &= \bar r = h(\bar x),
	\end{align}
	%
	i.e., it satisfies the same equations as a steady state of the original system~\eqref{eq:system:x}--\eqref{eq:system:delay}.
\end{theorem}
%
\begin{proof}
	This result follows directly from the equivalence of the system of ODEs~\eqref{eq:lct:approximate:system:ODE} to the system of DDEs~\eqref{eq:lct:approximate:system:DDE} and Corollary~\ref{thm:steady:state}. However, for completeness and to demonstrate some of the properties of the matrices $A$, $B$, and $C$, we prove it using direct computation.
	%
	In steady state,
	%
	\begin{align}
		0 &= f(\bar x, \bar z), & 0 &= A \bar Z + B \bar r.
	\end{align}
	%
	Consequently,
	%
	\begin{align}
		\bar Z &= -A^{-1} B \bar r, & \bar z &= C \bar Z = -C A^{-1} B \bar r.
	\end{align}
	%
	Note that each block-diagonal element of $A$ is a lower bidiagonal matrix with nonzero elements in the diagonal and the first subdiagonal. Therefore, $A$ is invertible. Specifically,
	%
	\begin{subequations}\label{eq:lct:ode:steady:state:proof}
		\begin{align}
			\label{eq:lct:ode:steady:state:proof:ainv}
			A^{-1} &= \blkdiag\left(\left(A^{(1)}\right)^{-1}, \ldots, \left(A^{(n_z)}\right)^{-1}\right), \\
			%
			\label{eq:lct:ode:steady:state:proof:cainvb}
			C A^{-1} B &= \diag\left(c^{(1)} \left(A^{(1)}\right)^{-1} b^{(1)}, \ldots, c^{(n_z)} \left(A^{(n_z)}\right)^{-1} b^{(n_z)}\right).
		\end{align}
	\end{subequations}
	%
	The inverse of $A^{(i)}$ is a multiple of a lower triangular matrix of ones and the diagonal elements of~\eqref{eq:lct:ode:steady:state:proof:cainvb} can be derived analytically:
	%
	\begin{align}
		\left(A^{(i)}\right)^{-1} &= \frac{-1}{a}
		\begin{bmatrix}
			1 \\
			1 & 1 \\
			\vdots & & \ddots \\
			1 & 1 & \cdots & 1
		\end{bmatrix}, &
		%
		c^{(i)} \left(A^{(i)}\right)^{-1} b^{(i)} &= -\sum_{m=0}^{M_i} c_{m, i} = -1.
	\end{align}
	%
	Consequently, $-C A^{-1} B = I$ is an identity matrix, and
	%
	\begin{align}
		\bar z &= \bar r = h(\bar x),
	\end{align}
	%
	which concludes the proof.
\end{proof}
%
\begin{theorem}\label{thm:lct:stability}
	The system~\eqref{eq:lct:approximate:system:ODE} is locally asymptotically stable around a steady state, $\bar x$, that satisfies~\eqref{eq:lct:steady:state} if $\real \lambda < 0$ for all eigenvalues $\lambda \in \C$ of the Jacobian matrix
	%
	\begin{align}\label{eq:lct:ode:jacobian}
		J &=
		\begin{bmatrix}
			F & G C \\
			B H & A
		\end{bmatrix},
	\end{align}
	%
	where $F$, $G$, and $H$ are the Jacobian matrices given by~\eqref{eq:jacobians}. Furthermore, the eigenvalues can be partitioned into two subsets. One of the subsets will contain $M_i$ eigenvalues equal to $-a_i$ for $i = 1, \ldots, n_z$, and the other will contain eigenvalues that satisfy the reduced characteristic equation
	%
	\begin{align}\label{eq:lct:stability:characteristic:equation}
		\det(F - \lambda I + G Q(\lambda) H) &= 0, &
		Q_{ij}(\lambda) &=
		\begin{cases}
			\sum\limits_{m=0}^{M_i} c_{m, i} \left(\frac{a_i}{a_i + \lambda}\right)^{m+1}, & i = j, \\
			0, & i \neq j,
		\end{cases}
	\end{align}
	%
	where $Q: \C \rightarrow \C^{n_z \times n_z}$.
\end{theorem}
%
\begin{proof}
	An eigenvalue, $\lambda$, is a root of the characteristic polynomial
	%
	\begin{align}\label{eq:lct:stability:proof:determinant}
		\det(J - \lambda I)
		&=
		\det
		\begin{bmatrix}
			F - \lambda I & GC \\
			BH & A - \lambda I
		\end{bmatrix} \\
		&=
		\det\left(F - \lambda I - GC (A - \lambda I)^{-1} BH\right) \det(A - \lambda I), \nonumber
	\end{align}
	%
	where we have used Schur's determinant formula (see, e.g., \cite{Zhang:2005}). Next, we exploit the block-diagonal structure of $A$:
	%
	\begin{align}
		A - \lambda I &= \blkdiag\left(A^{(1)} - \lambda I^{(1)}, \ldots, A^{(n_z)} - \lambda I^{(n_z)}\right).
	\end{align}
	%
	Consequently,
	%
	\begin{align}
		\det(A - \lambda I) &= \prod_{i=1}^{n_z} \det (A^{(i)} - \lambda I^{(i)}) = \prod_{i=1}^{n_z} (-1)^{M_i} (a_i + \lambda)^{M_i},
	\end{align}
	%
	and $\lambda = -a_i$ is an eigenvalue with algebraic multiplicity $M_i$ for $i = 1, \ldots, n_z$. As the rate parameters, $a_i$, are positive, $A - \lambda I$ is invertible. Next, we consider the diagonal matrix
	%
	\begin{align}\label{eq:lct:stability:proof:camliinvb}
		C (A - \lambda I)^{-1} B &= \diag\left(c^{(1)} \left(A^{(1)} - \lambda I^{(1)}\right)^{-1} b^{(1)}, \ldots, c^{(n_z)} \left(A^{(n_z)} - \lambda I^{(n_z)}\right)^{-1} b^{(n_z)}\right).
	\end{align}
	%
	First, we rewrite the matrix
	%
	\begin{align}
		A^{(i)} - \lambda I^{(i)} &= a_i \left(L^{(i)} - I^{(i)}\right) - \lambda I^{(i)} = -(a_i + \lambda)\left(\frac{-a_i}{a_i + \lambda} L^{(i)} + I^{(i)}\right).
	\end{align}
	%
	Consequently, the inverse is
	%
	\begin{align}
		\left(A^{(i)} - \lambda I^{(i)}\right)^{-1} &= \frac{-1}{a_i + \lambda}
		\begin{bmatrix}
			1 \\
			\frac{a_i}{a_i + \lambda} & 1 \\
			\vdots & & \ddots \\
			\left(\frac{a_i}{a_i + \lambda}\right)^{M_i} & \left(\frac{a_i}{a_i + \lambda}\right)^{M_i-1} & \cdots & 1
		\end{bmatrix},
	\end{align}
	%
	as can be verified by direct calculation, and the diagonal elements of the matrix in~\eqref{eq:lct:stability:proof:camliinvb} are
	%
	\begin{align}
		c^{(i)} \left(A^{(i)} - \lambda I^{(i)}\right)^{-1} b^{(i)} &= -\sum_{m=0}^{M_i} c_{m, i} \left(\frac{a_i}{a_i + \lambda}\right)^{m+1} = -Q_{ii}(\lambda).
	\end{align}
	%
	Finally, we substitute into the left determinant on the right-hand side of~\eqref{eq:lct:stability:proof:determinant} and obtain
	%
	\begin{align}
		\det(F - \lambda I + G Q(\lambda) H) &= 0.
	\end{align}
\end{proof}
%
\begin{proof}[Alternative proof]
	For completeness, we demonstrate that the result can also be obtained from the equivalence between~\eqref{eq:lct:approximate:system:DDE} and~\eqref{eq:lct:approximate:system:ODE} and the stability criterion in Lemma~\ref{thm:stability}. First, we write out the following integral for an Erlang kernel of order $m$ with rate parameter $a_i$:
	%
	\begin{align}
		\int_0^\infty e^{-\lambda s} \ell_{m, i}(s) \incr s &= b_{m, i} \int_0^\infty s^m e^{-(a_i + \lambda) s} \incr s = b_{m, i} \frac{m!}{(a_i + \lambda)^{m+1}} = \left(\frac{a_i}{a_i + \lambda}\right)^{m+1}.
	\end{align}
	%
	This follows from the fact that the integral of an Erlang kernel of order $m$ with rate parameter $a_i + \lambda$ is one. Next, we use this result to write out the following integral involving an Erlang mixture kernel of order $M_i$ with rate parameter $a_i$:
	%
	\begin{align}
		\int_0^\infty e^{-\lambda s} \hat \alpha_i(s) \incr s &= \sum_{m=0}^{M_i} c_{m, i} \int_0^\infty e^{-\lambda s} \ell_{m, i}(s) \incr s = \sum_{m=0}^{M_i} c_{m, i} \left(\frac{a_i}{a_i + \lambda}\right)^{m+1} = Q_{ii}(\lambda).
	\end{align}
	%
	Consequently,
	%
	\begin{align}
		G \int_0^\infty e^{-\lambda s} \diag \hat \alpha(s) \incr s H
		&= G \diag \left\{\int_0^\infty e^{-\lambda s} \hat \alpha_i(s) \incr s\right\}_{i=1}^{n_z} H = G Q(\lambda) H,
	\end{align}
	%
	and the characteristic equation~\eqref{eq:characteristic:equation} becomes
	%
	\begin{align}
		\det(F - \lambda I + G Q(\lambda) H) &= 0.
	\end{align}
\end{proof}

Next, we analyze the relation between the stability criterion from Theorem~\ref{thm:lct:stability} for the approximate system of ODEs and the one from Theorem~\ref{thm:stability} for the original system of DDEs. The analysis is based on the piecewise constant approximation of decaying exponentials described in Lemma~\ref{thm:exponential:approximation} and illustrated in Figure~\ref{fig:exponential:approximation}.
%
\begin{figure}
	\centering
	\subfloat{\includegraphics[width=\textwidth]{./fig/ExponentialFunctionApproximation-eps-converted-to}}
	\caption{Left: The piecewise constant approximation~\eqref{eq:exponential:approximation} from Lemma~\ref{thm:exponential:approximation} of a decaying exponential function, $e^{-\lambda s}$ for $\lambda = 1/2$ and different values of $a$. Right: The absolute error of the approximation.}
	\label{fig:exponential:approximation}
\end{figure}
%
\begin{lemma}\label{thm:exponential:approximation}
	The approximation
	%
	\begin{align}\label{eq:exponential:approximation}
		e^{-\lambda s} &\approx \left(\frac{a}{a + \lambda}\right)^{m+1}, &
		s &\in [s_m, s_{m+1}), & s_m &= m \Delta s, & \Delta s &= 1/a,
	\end{align}
	%
	converges as $a \rightarrow \infty$ for all $s, \lambda \in \Rnn$.
\end{lemma}
%
\begin{proof}
	First, we note that
	%
	\begin{align}
		\frac{a}{a + \lambda} &= 1 - \frac{\lambda}{a + \lambda}, &
		\left(1 - \frac{\lambda}{a + \lambda}\right)^{m+1} &\approx e^{-(m+1) \frac{\lambda}{a + \lambda}},
	\end{align}
	%
	where the approximation is obtained by truncating the sum in~\eqref{eq:exponential} after the two first terms. It is valid for small arguments, and $\lambda/(a + \lambda) \rightarrow 0$ as $a\rightarrow \infty$. Furthermore, $m+1 = s_{m+1}/\Delta s \approx as$ such that
	%
	\begin{align}
		e^{-(m+1) \frac{\lambda}{a + \lambda}} &\approx e^{-\frac{a}{a + \lambda} \lambda s} \rightarrow e^{-\lambda s}
	\end{align}
	%
	as $a \rightarrow \infty$. For $\lambda = 0$, the approximation is exact for all nonzero values of $a$.
\end{proof}
%
\begin{theorem}
	For infinite-order Erlang mixture approximations of the elements of the kernel, $\alpha$, as described in Theorem~\ref{thm:erlang:mixture:approximation}, the reduced characteristic function in~\eqref{eq:lct:stability:characteristic:equation} converges to the characteristic function in~\eqref{eq:characteristic:equation} as $a_i \rightarrow \infty$ for $i = 1, \ldots, n_z$.
\end{theorem}
%
\begin{proof}
	First, we approximate the exponential function in~\eqref{eq:characteristic:equation} by
	%
	\begin{align}\label{eq:exponential:approximation:stability:equivalence}
		e^{-\lambda s} &\approx \left(\frac{a_i}{a_i + \lambda}\right)^{m+1}, &
		s &\in [s_m, s_{m+1}), & s_m &= m \Delta s, & \Delta s &= 1/a_i,
	\end{align}
	%
	for the $i$'th kernel.
	%
	Next, we consider the integral in~\eqref{eq:characteristic:equation} for the $i$'th element of $\alpha$ and substitute the approximation:
	%
	\begin{align}
		\int_0 e^{-\lambda s} \alpha_i(s) \incr s
		&\approx \sum_{m=0}^\infty \left(\frac{a_i}{a_i + \lambda}\right)^{m+1} \int_{s_m}^{s_{m+1}} \alpha_i(s) \incr s = \sum_{m=0}^\infty c_{m, i} \left(\frac{a_i}{a_i + \lambda}\right)^{m+1} = Q_{ii}(\lambda).
	\end{align}
	%
	The coefficients were defined in~\eqref{eq:erlang:mixture:approximation:coefficients} in Theorem~\ref{thm:erlang:mixture:approximation}. Consequently, as the approximation of the exponential function in~\eqref{eq:exponential:approximation:stability:equivalence} converges as $a_i \rightarrow \infty$ for $i = 1, \ldots, n_z$,
	%
	\begin{align}
		\det(F - \lambda I + G Q(\lambda) H) \rightarrow \det\left(F - \lambda I + G \int_0^\infty e^{-\lambda s} \diag \alpha(s) \incr s H\right),
	\end{align}
	as well, which concludes the proof.
	% Next, the stability criterion in Theorem~\ref{thm:lct:stability} is obtained by truncating the sum to the first $M_i + 1$ terms. This corresponds to approximating $e^{-\lambda s}$ by zero for $s \geq s_{M_i+1} = (M_i + 1) \Delta s = (M_i + 1)/a_i$.
\end{proof}
%
\begin{corollary}
	If $\lambda = 0$ is an eigenvalue of the Jacobian matrix $J$ in~\eqref{eq:lct:ode:jacobian}, it is also a solution to the characteristic equation~\eqref{eq:characteristic:equation}.
\end{corollary}
%
\begin{proof}
	As the rate parameters, $a_i$, are positive for $i = 1, \ldots, n_z$, an eigenvalue of zero must be a root of the reduced characteristic equation in~\eqref{eq:lct:stability:characteristic:equation}, i.e., it is not an eigenvalue of $A$. Furthermore,
	%
	\begin{align}
		Q_{ii}(0) &= \sum_{m=0}^{M_i} c_{m, i} = 1,
	\end{align}
	%
	i.e., $Q(0)$ is an identity matrix.
	%
	Consequently, for an eigenvalue of zero,
	%
	\begin{align}
		\det(F + G H) = 0.
	\end{align}
	%
	Similarly, if $\lambda = 0$, the characteristic equation~\eqref{eq:characteristic:equation} becomes
	%
	\begin{align}
%		\det\left(F + \int_0^\infty \tilde \alpha(s) \incr s\right) &=
		\det\left(F + G \int_0^\infty \diag \alpha(s) \incr s H\right) &= \det(F + G H) = 0,
	\end{align}
	%
	which, as we have just shown, is satisfied.
\end{proof}
