\section{Derivation of the linear chain trick}\label{sec:deriv}
%
\begin{proof}[Proof of Theorem~\ref{thm:linear:chain:trick}]
	First, we introduce the auxiliary memory state $\hat z_{m, i}: \R \rightarrow \R$ given by
	%
	\begin{align}\label{eq:approximation:revisited:z}
		\hat z_{m, i}(t) &= \int\limits_{-\infty}^t \ell_{m, i}(t - s) \hat r_i(s) \incr s, & m &= 0, \ldots, M_i, & i &= 1, \ldots, n_z.
	\end{align}
	%
	Next, we rewrite the expression for the memory state,
	%
	\begin{align}\label{eq:approximation:z}
		\hat z_i(t) &= \int\limits_{-\infty}^t \hat \alpha_i(t - s) \hat r_i(s) \incr s = \sum_{m=0}^{M_i} c_{m, i} \int_{-\infty}^t \ell_{m, i}(t - s) \hat r_i(s) \incr s = \sum_{m=0}^{M_i} c_{m, i} \hat z_{m, i}(t),
	\end{align}
	%
	for $i = 1, \ldots, n_z$. The time derivatives of the Erlang kernels are
	%
	\begin{subequations}\label{eq:erlang:pdf:derivative}
		\begin{align}
			\dot{\ell}_{0, i}(t) &= -a_i \ell_{0, i}(t), &&& i &= 1, \ldots, n_z, \\
			%
			\dot{\ell}_{m, i}(t)
			&= a_i (\ell_{m-1, i}(t) - \ell_{m, i}(t)), & m &= 1, \ldots, M_i, & i &= 1, \ldots, n_z,
		\end{align}
	\end{subequations}
	%
	and the normalization factor satisfies the recursion
	%
	\begin{align}
		b_{m, i} &= \frac{a_i^{m+1}}{m!} = \frac{a_i}{m} \frac{a_i^m}{(m-1)!} = \frac{a_i}{m} b_{m-1, i}, & m &= 1, \ldots, M_i, & i &= 1, \ldots, n_z.
	\end{align}
	%
	In order to derive differential equations for the auxiliary memory states, we use Leibniz' integral rule~\cite[Thm.~3, Chap.~8]{Protter:Morrey:1985} to differentiate~\eqref{eq:approximation:revisited:z}:
	%
	\begin{align}
		\dot{\hat z}_{m, i}(t)
		&= \ell_{m, i}(0) \hat r_i(t) + \int\limits_{-\infty}^t \dot{\ell}_{m, i}(t - s) \hat r_i(s) \incr s, & m &= 0, \ldots, M_i, & i &= 1, \ldots, n_z.
	\end{align}
	%
	Next, we use the time derivatives of the Erlang kernels and the fact that
	%
	\begin{align}
		\ell_{m, i}(0) &=
		\begin{cases}
			a_i, & \text{for}~m = 0, \\
			0,   & \text{for}~m = 1, \ldots, M_i,
		\end{cases} & i &= 1, \ldots, n_z,
	\end{align}
	%
	to obtain the differential equations
	%
	\begin{subequations}
		\begin{align}
			\dot{\hat z}_{0, i}(t)
			&= a_i (\hat r_i(t) - \hat z_{0, i}(t)), \\
			%
			\dot{\hat z}_{m, i}(t)
			&= a_i (\hat z_{m-1, i}(t) - \hat z_{m, i}(t)), \quad m = 1, \ldots, M_i.
		\end{align}
	\end{subequations}
	%
	Finally, by introducing the vector of all memory states,
	%
	\begin{align}
		\hat Z &=
		\begin{bmatrix}
			\hat z_{0, 0} \\
			\hat z_{1, 0} \\
			\vdots \\
			\hat z_{M_{n_z}, n_z}
		\end{bmatrix},
	\end{align}
	%
	the system of DDEs~\eqref{eq:lct:approximate:system:DDE} can be transformed to the system of ODEs~\eqref{eq:lct:approximate:system:ODE}.
\end{proof}
%
\begin{corollary}\label{rem:constant:initial:function}
	When the approximate system of ODEs~\eqref{eq:lct:approximate:system:ODE} is in the steady state $\bar x \in \R^{n_x}$, the memory states are
	%
	\begin{align}
		\bar z_i &= \bar z_{m, i} = \bar r_i = h_i(\bar x), & m &= 0, \ldots, M_i, & i &= 1, \ldots, n_z.
	\end{align}
	%
	Similarly, if the initial state function, $x_0: \R \rightarrow \R^{n_x}$, is constant, the initial memory states are given by
	%
	\begin{align}
		\hat z_i(t_0) &= \hat z_{m, i}(t_0) = \hat r_i(t_0) = h_i(x_0), & m &= 0, \ldots, M_i, & i &= 1, \ldots, n_z.
	\end{align}
\end{corollary}
%
\begin{proof}
	Both results follow directly from the fact that the integral of an Erlang kernel is one and that the coefficients sum to one. First, we consider the memory states in steady state:
	%
	\begin{align}\label{eq:identification:initial:conditions:zm0}
		\bar z_{m, i}
		&= \bar r_i\int\limits_{-\infty}^{t} \ell_{m, i}(t - s) \incr s
		 = \bar r_i, & m &= 0, \ldots, M_i, & i &= 1, \ldots, n_z.
	\end{align}
	%
	This holds for all $t \in \R$. Furthermore,
	%
	\begin{align}
		\bar z_i &= \sum_{m=0}^{M_i} c_{m, i} \bar z_{m, i} = \bar r_i \sum_{m=0}^{M_i} c_{m, i} = \bar r_i, & i &= 1, \ldots, n_z.
	\end{align}
	%
	The proof is analogous for the case where the initial state function is constant.
\end{proof}
