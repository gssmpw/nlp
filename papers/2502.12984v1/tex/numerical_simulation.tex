\section{Numerical simulation}\label{sec:numerical:simulation}
In this appendix, we present two numerical methods for simulating DDEs with distributed time delays, i.e., for approximating the solution to initial value problems in the form~\eqref{eq:system}--\eqref{eq:system:delay}. The first is intended for non-stiff systems, and it uses Euler's explicit method to discretize the differential equations and a left rectangle rule to discretize the integral in the convolution in~\eqref{eq:system:z}. In contrast, the second is intended for stiff systems, and it uses Euler's implicit method and a right rectangle rule to discretize the differential equations and the integral, respectively. In both cases, we truncate the integral:
%
\begin{align}\label{eq:numerical:simulation:system:z}
	z(t) &= \int\limits_{-\infty}^t \alpha(t - s) \odot r(s) \incr s \approx \int\limits_{t - \Delta t_h}^t \alpha(t - s) \odot r(s) \incr s.
\end{align}
%
The memory horizon, $\Delta t_h \in \Rp$, can, e.g., be determined with the bisection approach described in Section~\ref{sec:algo:domain}.

\subsection{Non-stiff systems}\label{sec:numerical:simulation:non:stiff}
For non-stiff systems, we obtain the approximate system
%
\begin{subequations}\label{eq:numerical:simulation:non:stiff}
	\begin{align}
		\label{eq:numerical:simulation:non:stiff:x}
		x_{n+1} &= x_n + f(x_n, z_n) \Delta t, \\
		%
		\label{eq:numerical:simulation:non:stiff:z}
		z_{n+1} &= \sum_{j=1}^{N_h} \alpha(j \Delta t) \odot r_{n-j+1} \Delta t, \\
		%
		\label{eq:numerical:simulation:non:stiff:r}
		r_{n+1} &= h(x_{n+1}),
	\end{align}
\end{subequations}
%
where $t_n = t_0 + n \Delta t \in \R$ for some time step size, $\Delta t \in \Rp$. We assume that the memory horizon is a multiple, $N_h \in \N$, of the time step size, $\Delta t$. For $t_n \leq t_0$, $x_n \in \R^{n_x}$ and $z_n, r_n \in \R^{n_z}$ are equal to $x(t_n)$, $z(t_n)$, and $r(t_n)$, and for $t_n > t_0$, they are approximations. This is an \emph{explicit} approach, i.e., for given values of $x_n$, $z_n$, and $r_n$, $z_{n+1}$ can be computed using~\eqref{eq:numerical:simulation:non:stiff:z} and $x_{n+1}$ can be evaluated with~\eqref{eq:numerical:simulation:non:stiff:x}. Finally, $r_{n+1}$ can be evaluated with~\eqref{eq:numerical:simulation:non:stiff:r}, and the process is repeated for the subsequent time steps.

\subsection{Stiff systems}\label{sec:numerical:simulation:stiff}
For stiff systems, the approximate system is
%
\begin{subequations}\label{eq:numerical:simulation:stiff}
	\begin{align}
		\label{eq:numerical:simulation:stiff:x}
		x_{n+1} &= x_n + f(x_{n+1}, z_{n+1}) \Delta t, \\
		%
		\label{eq:numerical:simulation:stiff:z}
		z_{n+1} &= \sum_{j=0}^{N_h-1} \alpha(j \Delta t) \odot r_{n-j+1} \Delta t, \\
		%
		\label{eq:numerical:simulation:stiff:r}
		r_{n+1} &= h(x_{n+1}),
	\end{align}
\end{subequations}
%
which is a set of algebraic equations that, in general, must be solved numerically. In this work, we solve the residual equation
%
\begin{align}\label{eq:numerical:simulation:stiff:residual}
	R_n(x_{n+1}; x_n) &= x_{n+1} - x_n - f(x_{n+1}, z_{n+1}) \Delta t = 0,
\end{align}
%
for each time step sequentially. Here, $R_n: \R^{n_x} \times \R^{n_x} \rightarrow \R^{n_x}$ is a residual function, and we consider $z_{n+1}$ and $r_{n+1}$ to be functions of $x_{n+1}$.
%
Many numerical methods for solving nonlinear equations can utilize the Jacobian of the involved residual function. The Jacobian of the residual function in~\eqref{eq:numerical:simulation:stiff:residual} is given by
%
\begin{align}
	\pdiff{R_n}{x_{n+1}}(x_{n+1}; x_n) &= I - \Bigg(\pdiff{f}{x}(x_{n+1}, z_{n+1}) + \pdiff{f}{z}(x_{n+1}, z_{n+1}) \pdiff{z_{n+1}}{x_{n+1}}\Bigg) \Delta t,
\end{align}
%
where $I \in \R^{n_x \times n_x}$ is an identity matrix and the Jacobians of the memory state and the delayed variables are
%
\begin{subequations}
	\begin{align}
		\pdiff{z_{n+1}}{x_{n+1}} &= \left(\alpha(0) e^T\right) \odot \pdiff{r_{n+1}}{x_{n+1}} \Delta t, \\
		\pdiff{r_{n+1}}{x_{n+1}} &= \pdiff{h}{x}(x_{n+1}).
	\end{align}
\end{subequations}
%
Here, $e \in \R^{n_z}$ is a column vector of ones.
