\section{Introduction}\label{sec:intro}
Many industrial and natural processes exhibit time delays~\cite{Kolmanovskii:Myshkis:1992}, e.g., due to advection (flow in a pipe or river), diffusion (mixing), or feedback mechanisms (natural or man-made), and as time delays can have a significant impact on the dynamics and stability of a process~\cite{Niculescu:Gu:2004}, it is important to account for them when developing and analyzing mathematical models. Such models typically involve differential equations, and the time delays can either be absolute (discrete) or distributed (continuous)~\cite{Smith:2011}. In the former case, the right-hand side function depends on the state at specific points in the past, and in the latter, it depends on a weighted integral of all past states, i.e., a convolution. The weight function is called the kernel or memory function. Although discrete time delays are more commonly used, they typically arise from a simplification of the underlying phenomena. For instance, assuming plug-flow in a pipe leads to an absolute time delay whereas a nonuniform flow velocity (e.g., Hagen-Poiseuille flow) leads to a distributed delay~\cite{Ritschel:2024}. In this work, we focus on the latter.

In recent decades, there has been a significant interest in distributed time delays, and they have been used to model a variety of processes, e.g., in pharmacokinetics and pharmacodynamics (PK/PD)~\cite{Hu:etal:2018}, neuroscience~\cite{Darabsah:etal:2024}, side-effects of chemotherapy~\cite{Krzyzanski:etal:2018}, the Mackey-Glass system~\cite{Nevermann:Gros:2023, Zhang:Xiao:2016}, which can describe respiratory and hematopoietic diseases, population models in biology~\cite{Cassidy:etal:2019}, and pollution in fisheries~\cite{Bergland:etal:2022}. Mechanical~\cite{Aleksandrov:etal:2023} and economic~\cite{Guerrini:etal:2020} processes have also been considered, and we refer to the book by Kolmanovskii and Myshkis~\cite{Kolmanovskii:Myshkis:1992} for more examples.
%
Furthermore, many theoretical results have been developed specifically for DDEs with distributed time delays. Rahman et al.~\cite{Rahman:etal:2015} studied the stability of networked systems with distributed delays, Yuan and Belair~\cite{Yuan:Belair:2011} present general stability and bifurcation results, Bazighifan et al.~\cite{Bazighifan:etal:2019} analyze oscillations in higher-order DDEs with distributed delays, Cassidy~\cite{Cassidy:2021} demonstrate the equivalence between cyclic ordinary differential equations (ODEs) and a scalar DDE with a distributed delay, and Aleksandrov et al.~\cite{Aleksandrov:etal:2024} study the stability of systems with state-dependent kernels.

Despite the many applications, there exists far less theory and fewer numerical methods and software for DDEs with distributed time delays than for ODEs. Even for DDEs with absolute delays, there exist many numerical methods~\cite{Bellen:2000, Polyanin:etal:2023} and significant amounts of off-the-shelf software, e.g., for simulation~\cite{Shampine:Thompson:2001}. However, some modeling software does contain functionality for distributed time delays, e.g., NONMEM~\cite{Yan:etal:2021} and Phoenix~\cite{Hu:etal:2018}, and customized numerical methods have also been proposed. Typically, they 1)~approximate the integral in the convolution using a quadrature rule, e.g., a trapezoidal~\cite{Huang:Vandewalle:2004} or Gaussian~\cite{Torkamani:etal:2013} rule, and 2)~discretize the differential equations using a one-step method, e.g., a Runge-Kutta method~\cite{Zhang:Xiao:2016, Eremin:2019}, or a linear multistep method~\cite{Wang:etal:2018}. Methods based on splines have also been proposed~\cite{Zhou:2016}. However, compared to similar methods for ODEs, approximating the convolution is requiring in terms of computations and memory, and it is not straightforward to choose the time step size adaptively.

Consequently, it is desirable to identify approaches that can directly take advantage of existing theory, methods, and software for ODEs, and for a specific class of kernels, it is possible to transform DDEs with distributed delays to ODEs. Specifically, the transformation is called the \emph{linear chain trick} (LCT)~\cite{MacDonald:1978, Ponosov:etal:2004, Smith:2011}, and it is applicable when the kernel is given by the probability density function of an Erlang distribution (referred to as an Erlang kernel). The LCT has been widely used, and Hurtado and Kirosingh~\cite{Hurtado:Kirosingh:2019} and Hurtado and Richards~\cite{Hurtado:Richards:2020, Hurtado:Richards:2021} generalized it to transform a wider range of stochastic mean field models to ODEs. Additionally, both Cassidy et al.~\cite{Cassidy:etal:2022} and Krzyzanski~\cite{Krzyzanski:2019} have proposed to simplify the simulation and analysis of specific DDEs by approximating the involved gamma kernels by a hypoexponential kernel and a truncated binomial series, respectively.
%
However, to the best of the author's knowledge, there are no general-purpose LCTs or approximations for broader classes of kernels.

Therefore, in this work, we present a general-purpose Erlang mixture approximation for continuous and bounded kernels, which allows us to use the LCT to approximate a large class of DDEs with distributed time delays by ODEs. We prove that the approximation converges, and we compare the stability criteria for the approximate ODEs and the original system. Furthermore, we propose a least-squares approach for determining the Erlang mixture approximation coefficients and rate parameter, and we use a numerical example to demonstrate that such an optimal approximation is more accurate than an approximation which uses theoretical coefficients from the proof. Finally, using two numerical examples, we demonstrate that the proposed approach can be used to approximately simulate and analyze the solution of DDEs with distributed time delays.

The remainder of the paper is structured as follows. In Section~\ref{sec:notation} and~\ref{sec:system}, we present the notation used and the system considered in this paper, respectively. Next, in Section~\ref{sec:approx}, we present the mixed Erlang approximation and prove that it converges, and in Section~\ref{sec:lct}, we describe how to use the LCT to transform the approximate system of DDEs to a set of ODEs. In Section~\ref{sec:algo}, we describe the least-squares approach, and in Section~\ref{sec:ex:conv} and~\ref{sec:ex}, we present numerical examples. Finally, conclusions are presented in Section~\ref{sec:conclusions}.
