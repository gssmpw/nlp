\section{Numerical test of the accuracy}\label{sec:ex:conv}
In this section, we construct a DDE in the form~\eqref{eq:system:x}--\eqref{eq:system:delay} with a known solution and use it to investigate the accuracy of the Erlang mixture approximation and the numerical solution of the approximate system of ODEs~\eqref{eq:lct:approximate:system:ODE}. Furthermore, we assess the convergence rate of the two numerical methods for solving initial value problems in the form~\eqref{eq:system}--\eqref{eq:system:delay} described in Appendix~\ref{sec:numerical:simulation}.

We consider a modified logistic differential equation with a forcing term in the form
%
\begin{align}\label{eq:logistic:differential:equation}
	\dot x(t) &= \sigma x(t) \left(1 - \frac{z(t)}{\kappa}\right) + Q(t), &
	z(t) &= \int\limits_{-\infty}^{t} \alpha(t - s) x(s)\,\mathrm ds.
\end{align}
%
Here, $x: \R \rightarrow \Rnn$ is a population density, $z: \R \rightarrow \Rnn$ is the memory state, $\sigma \in \Rp$ is the logistic growth rate, $\kappa \in \Rp$ is a carrying capacity, and $Q: \R \rightarrow \R$ is the forcing term.
%
We choose the solution, $x^*: \R \rightarrow \Rnn$, and the kernel, $\alpha: \Rnn \rightarrow \Rnn$, as
%
\begin{align}\label{eq:verification:true:solution}
	x^*(t) &= 1 + e^{-\left(\frac{t}{\gamma}\right)^2}, &
	\alpha(t) &= \frac{2}{\sqrt{\pi}} e^{-t^2}, &
	\beta (t) &= \erf(t),
\end{align}
%
where $\gamma \in \Rp$ is a dilation parameter, $\beta: \Rnn \rightarrow [0, 1]$ is derived from~\eqref{eq:cumulative:distribution:function}, and $\erf: \R \rightarrow [0, 1]$ is the error function.
%
Consequently, the true memory state, $z^*: \R \rightarrow \Rnn$, is
%
\begin{align}
	z^*(t) &= 1 + \frac{\gamma}{\sqrt{\gamma^2 + 1}} e^{-\frac{t^2}{\gamma^2 + 1}} \left(1 + \erf\left(\frac{t}{\gamma \sqrt{\gamma^2 + 1}}\right)\right),
\end{align}
%
and the forcing term is chosen such that $x^*$ in~\eqref{eq:verification:true:solution} is the solution:
%
\begin{align}
	Q(t)
	&= \dot x^*(t) - \sigma x^*(t) \left(1 - \frac{z^*(t)}{\kappa}\right).
\end{align}
%
This is referred to as the \emph{method of manufactured solutions}~\cite{Roache:2002}.
%
\begin{remark}
	We choose $x^*$ such that it is bounded away from zero because the system becomes unstable if the numerical approximation of the solution becomes negative. This can happen when the true solution is close to zero, in which case the numerical approximation may diverge.
\end{remark}

Figure~\ref{fig:verification} shows the approximation errors for the state and the kernel:
%
\begin{align}\label{eq:verification:error}
	E_x &= \sum_{n=0}^{K_x-1} (\hat x(t_{n+1}) - x^*(t_{n+1}))^2 \Delta t, &
	E_\alpha &= \sum_{k=0}^{K_\alpha-1} (\hat \alpha(t_k) - \alpha(t_k))^2 \Delta t.
\end{align}
%
Here, $\hat x$ and $\hat \alpha$ denote the approximations. Note that the kernel error, $E_\alpha$, is analogous to the objective function in~\eqref{eq:algo:nlp:obj} but may contain a different number of terms. We have used the model parameter values $\sigma = 4$~mo$^{-1}$ and $\kappa = 1$~mo and the dilation parameter $\gamma = 10$~mo, and the initial and final times are $t_0 = 0$~mo and $t_f = 24$~mo. When we determine the Erlang mixture approximation of a given order, $M$, we use $N = 10^2$ points in the objective function in~\eqref{eq:algo:nlp:obj}. Furthermore, we use a threshold of $\epsilon = 10^{-14}$ and a convergence tolerance of $10^{-15}$ to determine the approximation interval length, $t_h$. Finally, we solve the NLP~\eqref{eq:algo:nlp} using \matlab{}'s \fmincon{} with an optimality tolerance of $10^{-10}$ and a step size tolerance below machine precision, and we use double precision in the involved kernel evaluations. We also use the identified value of $t_h$ in the reference approach described in Section~\ref{sec:algo:theoretical:approximation}.
%
We use \matlab{}'s \odeff{} with absolute and relative tolerances of $10^{-12}$ to simulate the approximate system of ODEs~\eqref{eq:lct:approximate:system:ODE}, and we use $K_x = 24\cdot 10^3$ quadrature points to evaluate the state error in~\eqref{eq:verification:error}, which corresponds to a time step size of $\Delta t = 10^{-3}$~mo. Furthermore, we use $K_\alpha = 10^6$ quadrature points to evaluate the kernel error.
%
For the results obtained with the numerical methods described in Appendix~\ref{sec:numerical:simulation}, we use a memory horizon of $\Delta t_h = t_f - t_0 = 24$~mo and the above time step size, and for the method for stiff DDEs, we use a function tolerance of $10^{-12}$ and an optimality tolerance of $10^{-8}$ when we solve the involved residual equations with \matlab{}'s \fsolve{}.

The results demonstrate that for this example, the Erlang mixture approximation obtained with the proposed least-squares approach is several orders of magnitude more accurate than the reference approach based on the theoretical expressions for the coefficients. In both cases, there is a high correlation between the error of the kernel and the error of the approximate state trajectory. The kernel and state error is higher for $M = 32$ than $16$ because the objective function in the NLP~\eqref{eq:algo:nlp} uses a coarser resolution in the rectangle rule than the error in~\eqref{eq:verification:error}. Finally, the state error decreases quadratically with the time step size for the two numerical methods described in Appendix~\ref{sec:numerical:simulation} because the involved integral have been discretized with first-order methods and the error is squared in~\eqref{eq:verification:error}.
%
\begin{figure*}
	\subfloat{
		\includegraphics[width=0.48\textwidth]{./fig/ConvergenceLCT-eps-converted-to}~
		\includegraphics[width=0.48\textwidth]{./fig/ConvergenceEuler-eps-converted-to}
	} \\
	%
	\subfloat{
		\includegraphics[width=0.48\textwidth]{./fig/ConvergenceKernel-eps-converted-to}~
		\includegraphics[width=0.48\textwidth]{./fig/ConvergenceKernelLCT-eps-converted-to}
	}
	\caption{The state (top left) and kernel (bottom left) errors for Erlang mixture approximations of different orders, $M$, obtained with the proposed least-squares approach and the reference approach based on the theoretical expressions for the coefficients. The bottom right shows the state and kernel errors against each other, and the top right plot shows the state error obtained with the numerical approaches for non-stiff and stiff DDEs for different time step sizes, $\Delta t$. The black solid line is proportional to $\Delta t^2$.}
	\label{fig:verification}
\end{figure*}
