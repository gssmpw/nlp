\section{Numerical examples}\label{sec:ex}
In this section, we present two numerical examples that demonstrate how Erlang mixture approximations can be used to approximately analyze and simulate DDEs with distributed time delays. In the first example, we use the approximate system of ODEs~\eqref{eq:lct:approximate:system:ODE} to perform a numerical bifurcation analysis of the modified logistic differential equation that was also considered in Section~\ref{sec:ex:conv}, and in the second example, we present a Monte Carlo simulation of a molten salt nuclear fission process, which is nonlinear, multivariate, and stiff. We use high orders, $M$, and strict tolerances in the numerical methods to mitigate any adverse effects on the results. However, in many practical contexts, lower orders and less restrictive tolerances can be used with limited effect on the accuracy (see also Section~\ref{sec:ex:conv}).

\subsection{Modified logistic equation}\label{sec:ex:logistic}
We consider the modified logistic differential equation~\eqref{eq:logistic:differential:equation} where the forcing term is $Q(t) = 0$ and the kernel is
%
\begin{align}\label{eq:logistic:equation:kernel}
	\alpha(t) &= \gamma_1 F(t; \mu_1, \sigma_1) + \gamma_2 F(t; \mu_2, \sigma_2).
\end{align}
%
Here, $\gamma_1, \gamma_2 \in [0, 1]$ are weights, $\mu_1, \mu_2 \in \R$ are location parameters, $\sigma_1, \sigma_2 \in \Rp$ are scale parameters, and $F: \Rnn \times \R \times \Rp \rightarrow \Rnn$ is given by the probability density function of a folded normal distribution (i.e., it is a folded normal kernel):
%
\begin{align}\label{eq:folded:normal:pdf}
	F(t; \mu_i, \sigma_i) &= \frac{\exp\left(-\frac{1}{2}\left(\frac{t - \mu_i}{\sigma_i}\right)^2\right) + \exp\left(-\frac{1}{2}\left(\frac{t + \mu_i}{\sigma_i}\right)^2\right)}{\sqrt{2 \pi}\, \sigma_i}, & i &= 1, 2.
\end{align}
%
Furthermore, we derive $\beta$ in~\eqref{eq:cumulative:distribution:function} analytically.

We use the parameter values in Table~\ref{tab:logistic:parameters}, and the initial state is $x(t) = 0.9$ for $t \leq t_0$. We use the same parameters as in Section~\ref{sec:ex:conv} to identify Erlang mixture kernels of order $M = 100$ in the bifurcation analysis with respect to the logistic growth rate, $\sigma$, and $M = 350$ in the analysis with respect to the kernel parameter $\mu_2$. In Figure~\ref{fig:logistic:equation:bifurcation:analysis}, the first row shows the eigenvalues of the Jacobian of the right-hand side function in the approximate system of ODEs~\eqref{eq:lct:approximate:system:ODE}. For completeness, we consider a very large range of values for $\sigma$, but for numerical reasons, we have not been able to investigate larger values of $\mu_2$. The second row shows the largest real part of the eigenvalues and when the parameters become sufficiently large, the steady state $\bar x = \kappa$ becomes unstable. The two bottom rows show simulations obtained with the numerical method for non-stiff DDEs for the parameter values indicated in the second row (the colors match). We use a memory horizon of $\Delta t_h = t_f - t_0 = 24$~mo and $10,000$ time steps. In both cases, the stability analysis based on the approximate set of ODEs~\eqref{eq:lct:approximate:system:ODE} accurately predicts when the steady state is stable, marginally stable (where the largest real part of the eigenvalues is equal to zero), and unstable. For large values of $\sigma$, a limit cycle appears whereas for large values of $\mu_2$, the population density oscillates unstably.
%
Figure~\ref{fig:logistic:equation:identified:kernels} shows the identified kernels, the corresponding error, and the coefficients for different values of $\mu_2$. There is a similarity between the coefficients and the shape of the approximated kernel, e.g., in terms of bimodality. Furthermore, as $\mu_2$ increases, more sharp spikes occur in the coefficients.
%
\begin{table}[t]
	\centering
	\caption{Values of the parameters in the modified logistic equation. For $\sigma$ and $\mu_2$, only the nominal values are shown.}
	\label{tab:logistic:parameters}
	\footnotesize
	\begin{tabular}{cccccccccc}
		\toprule
		\multicolumn{10}{c}{Model, kernel, and simulation parameters} \\
		\midrule
		$\sigma$~[1/mo] & $\kappa$~[--] & $\gamma_1$~[--] & $\gamma_2$~[--] & $\mu_1$~[mo] & $\mu_2$~[mo] & $\sigma_1$~[mo] & $\sigma_2$~[mo] & $t_0$ [mo] & $t_f$ [mo] \\
		4 & 1 & 0.5 & 0.5 & 0.35 & 0.45 & 0.06 & 0.12 & 0 & 24 \\
		\bottomrule
	\end{tabular}
\end{table}
%
\begin{figure}
	\centering
	\subfloat{
		\includegraphics[width=0.48\textwidth]{./fig/LogisticEquationBifurcationAnalysisParameter-eps-converted-to}~
		\includegraphics[width=0.48\textwidth]{./fig/LogisticEquationBifurcationAnalysisKernel-eps-converted-to}
	}
	\caption{Bifurcation analysis with respect to the model parameter $\sigma$ (left column) and the kernel parameter $\mu_2$ (right column) for the modified logistic equation. First row: Eigenvalues. Second row: The largest real part of the eigenvalues. Third and bottom row: Simulations for selected parameter values (obtained with the numerical method described in Appendix~\ref{sec:numerical:simulation:non:stiff}).}
	\label{fig:logistic:equation:bifurcation:analysis}
\end{figure}
%
\begin{figure}
	\centering
	\subfloat{
		\includegraphics[width=0.485\textwidth]{./fig/LogisticEquationKernelParameter-eps-converted-to}~
		\includegraphics[width=0.48\textwidth]{./fig/LogisticEquationKernelsKernel-eps-converted-to}
	}
	\caption{Top row: The true kernels and the corresponding Erlang mixture approximations for $M = 100$ (left) and $M = 350$ (right) obtained in connection with the bifurcation analysis shown in Figure~\ref{fig:logistic:equation:bifurcation:analysis}. Middle row: The absolute errors of the kernel approximations. Bottom row: The coefficients in the Erlang mixture approximations. The colors are consistent across each column.}
	\label{fig:logistic:equation:identified:kernels}
\end{figure}

\subsection{Nuclear fission}\label{sec:ex:nuclear:fission}
Next, we consider a point reactor kinetics model of a molten salt nuclear fission reactor~\cite{Duderstadt:Hamilton:1976, Wooten:Powers:2018} where the molten salt is circulated through a heat exchanger outside of the reactor core. The model describes 1)~the concentrations of $N_g = 6 \in \N$ neutron precursor groups, $C_i: \R \rightarrow \Rnn$ for $i = 1, \ldots, N_g$, which emit delayed neutrons, 2)~the concentration of neutrons, $C_n: \R \rightarrow \Rnn$ for $n = N_g + 1 \in \N$, and 3)~the reactivity, $\rho: \R \rightarrow \R$, which represents the relative number of neutrons created per fission event:
%
\begin{subequations}
	\begin{align}
		\dot C_i(t) &= (C_{i, in}(t) - C_i(t)) D + R_i(t), &
		\dot C_n(t) &= R_n(t), &
		\dot \rho(t) &= -\kappa H C_n(t).
	\end{align}
\end{subequations}
%
The dilution rate, $D \in \Rnn$, is the ratio between the volumetric inlet and outlet flow rate and the reactor core volume, $\kappa \in \Rnn$ is a proportionality constant, and $H \in \Rnn$ is the ratio between the power production proportionality constant and the heat capacity of the reactor core. Furthermore, the production rate $R: \R \rightarrow \R^n$ is defined in terms of the stochiometric matrix $S \in \R^{N_r \times n}$ where $N_r = n \in \N$ is the number of reactions and $r: \R \rightarrow \Rnn^{N_r}$ is a vector of reaction rates:
%
\begin{align}
	R(t) &= S^T(t) r(t), &
	S(t) &=
	\begin{bmatrix}
		-1      &         &             & 1               \\
		&  \ddots &             & \vdots          \\
		&         & -1          & 1               \\
		\beta_1 & \cdots  & \beta_{N_g} & \rho(t) - \beta
	\end{bmatrix}, &
	%
	r(t) &=
	\begin{bmatrix}
		\lambda_1 C_1(t) \\ \vdots \\ \lambda_{N_g} C_{N_g}(t) \\ C_n(t)/\Lambda
	\end{bmatrix}.
\end{align}
%
Here, $\lambda_i, \beta_i, \Lambda \in \Rnn$ for $i = 1, \ldots, N_g$ are decay constants, delayed neutron fractions (i.e., the relative number of neutrons that are generated from the decay of neutron precursors), and the mean neutron generation time, respectively, and $\beta \in \Rnn$ is the sum of $\beta_i$ for $i = 1, \ldots, N_g$.
%
The inlet concentration, $C_{i, in}: \R \rightarrow \Rnn$, is the memory state given by
%
\begin{align}\label{eq:nuclear:fission:delay}
	C_{i, in}(t) &= \int\limits_{-\infty}^t \alpha_i(t - s) C_i(s)\,\mathrm ds, & i &= 1, \ldots, N_g.
\end{align}
%
The kernels are described in Appendix~\ref{sec:nuclear:fission:kernel} and represent 1)~a nonuniform velocity profile in the external circulation of the molten salt (and neutron precursors) and 2)~the (exponential) decay of the neutron precursors during the external circulation.

Table~\ref{tab:nuclear:fission:parameters} shows the parameter values, which (except for $D$, $\mu_1$, and $\sigma_1$) are taken from~\cite{Leite:etal:2016}.
%
\begin{table}[t]
	\centering
	\caption{Values of the parameters in the molten salt reactor model. For $\kappa$, only the nominal value is shown.}
	\label{tab:nuclear:fission:parameters}
	\footnotesize
	\begin{tabular}{cccccc}
		\toprule
		\multicolumn{6}{c}{Decay constants~[1/s]} \\
		\midrule
		$\lambda_1$ & $\lambda_2$ & $\lambda_3$ & $\lambda_4$ & $\lambda_5$ & $\lambda_6$ \\
		0.0124 & 0.0305 & 0.1110 & 0.3010 & 1.1300 & 3.0000 \\
		\midrule
		\multicolumn{6}{c}{Delayed neutron fractions~[--]} \\
		\midrule
		$\beta_1$ & $\beta_2$ & $\beta_3$ & $\beta_4$ & $\beta_5$ & $\beta_6$ \\
		0.00021 & 0.00141 & 0.00127 & 0.00255 & 0.00074 & 0.00027 \\
		\midrule
		\multicolumn{6}{c}{Other model parameters} \\
		\midrule
		$\Lambda$~[s] & $\kappa$~[1/K] & $H$~[K\,cm$^3$/s] & $D$~[1/s] & $\mu_1$~[s] & $\sigma_1$~[s]\\% & $\tau_\ell$~[s] \\
		$5\cdot 10^{-5}$ & $3\cdot 10^{-4}$ & 0.05 & 2 & 2 & 0.1 \\
		\bottomrule
	\end{tabular}
\end{table}
%
We use $N = 1000$ points and an optimality tolerance of $10^{-10}$ in \fmincon{} to identify Erlang mixture kernel approximations of order $M = 1000$. Furthermore, we use $\epsilon = 10^{-13}$ and a tolerance of $10^{-14}$ in the bisection algorithm, we use Matlab's \integral{} with absolute and relative tolerances of $10^{-15}$ to approximate the integral in~\eqref{eq:cumulative:distribution:function}, and we use long double precision when evaluating the Erlang and Erlang mixture kernels.
%
Figure~\ref{fig:nuclear:fission:monte:carlo:simulation:parameters} shows a Monte Carlo simulation for 1000 samples of $\kappa$ drawn from a normal distribution with mean $3\cdot 10^{-4}$ and standard deviation $7.5 \cdot 10^{-5}$. For brevity of the presentation, we omit the plots of the neutron precursor group concentrations. The initial states are $C_i(t) = 1$~kmol/cm$^3$ for $i = 1, \ldots, n$ and $\rho(t) = 1.1 \beta$ for $t \leq t_0$, and we use an absolute and relative tolerance of $10^{-8}$ in \odeofs{} to simulate the approximate system of ODEs~\eqref{eq:lct:approximate:system:ODE}. Furthermore, we supply the analytical Jacobian of the right-hand side function, and we represent it as a sparse matrix.
%
The figure also shows the relative difference between the simulations obtained 1)~as described above and 2)~with the numerical method for stiff DDEs described in Appendix~\ref{sec:numerical:simulation:stiff} for the mean value of $\kappa$. The latter method uses a time step size of $6.25\cdot 10^{-5}$~s, and the relative difference, $E_{r, i}: \R \rightarrow \Rnn$, in the $i$'th state variable and $n$'th time step is given by
%
\begin{align}\label{eq:relative:error}
	E_{r, i}(t_n) &= \frac{|\hat x_i(t_n) - x_i(t_n)|}{1 + |x_i(t_n)|}, & i &= 1, \ldots, n_x,
\end{align}
%
where $\hat x_i$ is the approximate solution to the ODEs~\eqref{eq:lct:approximate:system:ODE} and $x_i$ is obtained with the numerical method for stiff DDEs.
%
For each point in time, the 95\% confidence intervals show the intervals from the 2.5 to the 97.5 percentiles and the span shows the interval of the minimum and maximum value of the states. For the neutron concentration, the confidence interval is almost symmetric (on the logarithmic axis) whereas the maximum is significantly higher above the mean than the minimum is below it. For the reactivity, the confidence interval and the span are hardly visible, and the simulation for the mean value of $\kappa$ is indistinguishable from the pointwise mean.
%
The maximum relative difference for all eight state variables is $2.14\cdot 10^{-3}$. There are three main sources of error that affect this difference: 1)~The kernel approximation error, 2)~the error from using \odeofs{} to simulate the approximate set of ODEs, and 3)~the error of the numerical method described in Appendix~\ref{sec:numerical:simulation:stiff}. If we use twice as large time steps in the latter, the maximum relative error is $4.28\cdot 10^{-3}$, i.e., twice as large (results not shown). Therefore, we believe that the maximum relative difference would be further reduced by using smaller time step sizes, i.e., that the main source of error is that of the numerical method for stiff DDEs and not the Erlang mixture kernel approximation. However, due to large memory requirements, it has not been possible to further reduce the time step size.
%
Finally, Figure~\ref{fig:nuclear:fission:identified:kernels} shows the Erlang mixture approximations of the kernels and the corresponding errors. It also shows the sum of folded normal kernels in~\eqref{eq:nuclear:fission:kernels} for the values of $\mu_1$ and $\sigma_1$ shown in Table~\ref{tab:nuclear:fission:parameters} and the coefficients in the Erlang mixture approximations. For some kernels, sharp spikes occur in the coefficients.

\begin{figure}
	\centering
	\subfloat{
		\includegraphics[width=0.48\textwidth]{./fig/NuclearFissionMonteCarloSimulationParameters_1-eps-converted-to}~
		\includegraphics[width=0.48\textwidth]{./fig/NuclearFissionMonteCarloSimulationParameters_2-eps-converted-to}
	}
	\caption{Monte Carlo simulation of the molten salt reactor model. Top row: The neutron concentration (left) and the reactivity (right). For each point in time, the span shows the interval of the minimum and maximum state, and the 95\% confidence interval spans the 2.5 and 97.5 percentiles. The mean is computed pointwise, and the sample is the simulation corresponding to the mean value of $\kappa$. Bottom row: The pointwise relative error~\eqref{eq:relative:error} between simulating the approximate ODEs~\eqref{eq:lct:approximate:system:ODE} using \odeofs{} and simulating the original DDEs using the numerical method for stiff DDEs described in Appendix~\ref{sec:numerical:simulation:stiff}.}
	\label{fig:nuclear:fission:monte:carlo:simulation:parameters}
\end{figure}

\begin{figure}
	\subfloat{
		\includegraphics[width=0.475\textwidth]{./fig/NuclearFissionIdentifiedKernelsParameters_1-eps-converted-to}~
		\includegraphics[width=0.485\textwidth]{./fig/NuclearFissionIdentifiedKernelsParameters_2-eps-converted-to}
	}
	\caption{Left column: The Erlang mixture approximations of the kernels in~\eqref{eq:nuclear:fission:kernels} used in the Monte Carlo simulation shown in Figure~\ref{fig:nuclear:fission:monte:carlo:simulation:parameters} (top) and the corresponding approximation errors (bottom). Right column: The normalized sum of folded normal kernels in the expression~\eqref{eq:nuclear:fission:kernels} for the kernels (top), and the coefficients in the Erlang mixture kernel approximations. The colors are consistent across all but the top right figure.}
	\label{fig:nuclear:fission:identified:kernels}
\end{figure}
