\section{System}\label{sec:system}
We consider systems in the form
%
\begin{subequations}\label{eq:system}
	\begin{align}
		\label{eq:system:x0}
		x(t) &= x_0(t), & t &\in (-\infty, t_0], \\
		%
		\label{eq:system:x}
		\dot x(t) &= f(x(t), z(t)), & t &\in [t_0, t_f],
	\end{align}
\end{subequations}
%
where $t \in \R$ is time, $t_0, t_f \in \R$ are the initial and final time, $x: \R \rightarrow \R^{n_x}$ is the state, and $x_0: \R \rightarrow \R^{n_x}$ is the initial state function. Furthermore, $f: \R^{n_x} \times \R^{n_z} \rightarrow \R^{n_x}$ is the right-hand side function, and the memory state, $z: \R \rightarrow \R^{n_z}$, is given by the convolution
%
\begin{subequations}\label{eq:system:delay}
	\begin{align}
		\label{eq:system:z}
		z(t) &= \int\limits_{-\infty}^t \alpha(t - s) \odot r(s) \incr s, \\
		%
		\label{eq:system:r}
		r(t) &= h(x(t)),
	\end{align}
\end{subequations}
%
where $r: \R \rightarrow \R^{n_z}$ is the delayed variable, and each element of $\alpha: \Rnn \rightarrow \Rnn^{n_z}$ is a \emph{regular} kernel (see Definition~\ref{def:regular:kernel}). Furthermore, $h: \R^{n_x} \rightarrow \R^{n_z}$ is the memory function. We assume that $f$ and $h$ are differentiable in their arguments, and we refer to the paper by Ponosov et al.~\cite[Thm.~1]{Ponosov:etal:2004} for more details on the existence and uniqueness of solutions to the initial value problem~\eqref{eq:system}--\eqref{eq:system:delay}. See also the book by Hale and Lunel~\cite{Hale:Lunel:1993}.
%
\begin{definition}\label{def:regular:kernel}
	A scalar-valued kernel, $\alpha: \Rnn \rightarrow \Rnn$, is \emph{regular} if it satisfies the following properties.
	%
	\begin{enumerate}
		\item It is non-negative and bounded, i.e., $0 \leq \alpha(t) \leq K$ for all $t \in \Rnn$ and for some finite $K \in \Rp$.
		%
		\item It is continuous, i.e., for all $\epsilon \in \Rp$ and $t \in \Rnn$, there exists a $\delta \in \Rp$ such that $|\alpha(s) - \alpha(t)| < \epsilon$ for all $s \in \Rnn$ satisfying $|s - t| < \delta$.
		%
		\item It is normalized such that
	\end{enumerate}
	%
	\begin{align}\label{eq:kernel:normalization}
		\int\limits_0^\infty \alpha(t) \incr t &= 1.
	\end{align}
\end{definition}
%
For a given system of DDEs with distributed time delays, each element of $\alpha$ may not satisfy~\eqref{eq:kernel:normalization}. However, as they are assumed to be nonzero and non-negative, it is straightforward to normalize them. Next, we present a few well-known corollaries about the steady states of~\eqref{eq:system:x}--\eqref{eq:system:delay} and their stability.
%
\begin{corollary}\label{thm:steady:state}
	A state $\bar x \in \R^{n_x}$ is a steady state of the system~\eqref{eq:system:x}--\eqref{eq:system:delay} if
	%
	\begin{align}\label{eq:steady:state}
		0 &= f(\bar x, \bar z), &
		\bar z &= \bar r = h(\bar x).
	\end{align}
\end{corollary}

\begin{proof}
	In steady state, $x(t) = \bar x$ for all $t$. Consequently, $r(t) = \bar r = h(\bar x)$ and
	%
	\begin{align}
		z(t)
		&= \int_{-\infty}^t \alpha(t - s) \odot \bar r \incr s
		 = \int_{-\infty}^t \alpha(t - s) \incr s \odot \bar r
		 = \bar r,
	\end{align}
	%
	for all $t$, where we have used the property~\eqref{eq:kernel:normalization} of each element of $\alpha$.
\end{proof}
%
\begin{corollary}\label{thm:stability}
	The system~\eqref{eq:system:x}--\eqref{eq:system:delay} is locally asymptotically stable around a steady state, $\bar x$, satisfying~\eqref{eq:steady:state} if $\real \lambda < 0$ for all $\lambda \in \C$ that satisfy the characteristic equation
	%
	\begin{align}\label{eq:characteristic:equation}
		\det\left(F - \lambda I + G \int_0^\infty e^{-\lambda s} \diag \alpha(s) \incr s H\right) = 0,
	\end{align}
	%
	where $I \in \R^{n_x \times n_x}$ is an identity matrix.
	%
	The matrices $F \in \R^{n_x \times n_x}$, $G \in \R^{n_x \times n_z}$, and $H \in \R^{n_z \times n_x}$ are the Jacobians of the right-hand side function and the delay function evaluated in the steady state:
	%
	\begin{align}\label{eq:jacobians}
		F &= \pdiff{f}{x}(\bar x, \bar z), &
		G &= \pdiff{f}{z}(\bar x, \bar z), &
		H &= \pdiff{h}{x}(\bar x).
	\end{align}
	%
\end{corollary}

\begin{proof}
	The linearized system corresponding to~\eqref{eq:system:x}--\eqref{eq:system:delay} describes the evolution of the deviation variable $X: \R \rightarrow \R^{n_x}$:
	%
	\begin{align}\label{eq:linearized:system}
		\dot X(t) &= F X(t) + G \int_{-\infty}^t \alpha(t - s) \odot H X(s) \incr s, &
		X(t) &= x(t) - \bar x.
	\end{align}
	%
	See, e.g., \cite{Cushing:1975, Cushing:1977, Miller:1972} for proofs of the condition~\eqref{eq:characteristic:equation} for asymptotic stability of the linearized system in~\eqref{eq:linearized:system}.
\end{proof}
