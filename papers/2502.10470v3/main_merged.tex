% !TEX program = pdflatex

\documentclass[journal]{IEEEtran}
\usepackage{lineno}
\modulolinenumbers[5]

%%% color some references
\usepackage{xpatch}
\makeatletter

\makeatother
\usepackage{bm}
\usepackage{array}
\usepackage{graphicx}
\usepackage{amsmath,amssymb,amsthm}
\usepackage{siunitx}
\usepackage{algpseudocode}
\usepackage{algorithmicx}
\usepackage{algorithm}
\usepackage{booktabs}
\usepackage{color}
\usepackage{changepage}
\usepackage{xr}
\usepackage{xr-hyper}
%\usepackage{geometry}%页面设置
\usepackage{graphicx}%图片设置
%\usepackage{subfig}%多个子图
\usepackage{subfigure}
\usepackage{caption}%注释设置
\usepackage{multirow}
\usepackage{float}
\usepackage{soul}
\usepackage[hidelinks]{hyperref}
\usepackage[numbers,sort&compress]{natbib}
\usepackage{bigstrut} %表格大竖线
\usepackage[table]{xcolor} %表格单元格颜色
\usepackage{enumitem} %enumerate标签样式\usepackage{listings}
\usepackage{listings} %listing代码
\usepackage[resetlabels]{multibib}
\newcites{supp}{Supplement References}



\definecolor{dkgreen}{rgb}{0,0.5,0}
\definecolor{gray}{rgb}{0.5,0.5,0.5}
\definecolor{mauve}{rgb}{0.58,0,0.82}
\definecolor{dkblue}{rgb}{0,0,0.6}

 % English theorem environment
 \newtheorem{theorem}{Theorem}
 \newtheorem{lemma}[theorem]{Lemma}
 \newtheorem{proposition}[theorem]{Proposition}
 \newtheorem*{corollary}{Corollary of Theorems 1 and 2}
 \newtheorem{definition}{Definition}
 \newtheorem{remark}{Remark}
 \newtheorem{example}{Example}
 \newenvironment{solution}{\begin{proof}[Solution]}{\end{proof}}

\renewcommand{\algorithmicrequire}{\textbf{Input:}}
\renewcommand{\algorithmicensure}{\textbf{Output:}}

\AtBeginDocument{%
 \abovedisplayskip=5pt plus 4pt minus 2pt
 \abovedisplayshortskip=5pt plus 4pt minus 4pt
 \belowdisplayskip=5pt plus 4pt minus 2pt
 \belowdisplayshortskip=5pt plus 4pt minus 4pt
}

\ifCLASSINFOpdf
\else
\fi

\hyphenation{op-tical net-works semi-conduc-tor}

\bibliographystyle{IEEEtran}

\begin{document}
\captionsetup{font={footnotesize}}
\captionsetup[table]{labelformat=simple, labelsep=newline, textfont=sc, justification=centering}
% paper title
% Titles are generally capitalized except for words such as a, an, and, as,
% at, but, by, for, in, nor, of, on, or, the, to and up, which are usually
% not capitalized unless they are the first or last word of the title.
% Linebreaks \\ can be used within to get better formatting as desired.
% Do not put math or special symbols in the title.
\title{MetaDE: Evolving Differential Evolution by Differential Evolution}
%
%
% author names and IEEE memberships
% note positions of commas and nonbreaking spaces ( ~ ) LaTeX will not break
% a structure at a ~ so this keeps an author's name from being broken across
% two lines.
% use \thanks{} to gain access to the first footnote area
% a separate \thanks must be used for each paragraph as LaTeX2e's \thanks
% was not built to handle multiple paragraphs
%

\author{Minyang Chen, Chenchen Feng,
        and Ran Cheng

        \thanks{
        Minyang Chen was with the Department of Computer Science and Engineering, Southern University of Science and Technology, Shenzhen 518055, China. E-mail: cmy1223605455@gmail.com. }
        \thanks{
        Chenchen Feng is with the Department of Computer Science and Engineering, Southern University of Science and Technology, Shenzhen 518055, China. E-mail: chenchenfengcn@gmail.com. 
        }
        \thanks{
       Ran Cheng is with the Department of Data Science and Artificial Intelligence, and the Department of Computing, The Hong Kong Polytechnic University, Hong Kong SAR, China. E-mail: ranchengcn@gmail.com. (\emph{Corresponding author: Ran Cheng})
        }
        }% <-this % stops a space


% The paper headers
\markboth{Bare Demo of IEEEtran.cls for IEEE Journals}%
{Shell \MakeLowercase{\textit{et al.}}: Bare Demo of IEEEtran.cls for IEEE Journals}
% The only time the second header will appear is for the odd numbered pages
% after the title page when using the twoside option.

% *** Note that you probably will NOT want to include the author's ***
% *** name in the headers of peer review papers.                   ***
% You can use \ifCLASSOPTIONpeerreview for conditional compilation here if
% you desire.


% If you want to put a publisher's ID mark on the page you can do it like
% this:
%\IEEEpubid{0000--0000/00\$00.00~\copyright~2015 IEEE}
% Remember, if you use this you must call \IEEEpubidadjcol in the second
% column for its text to clear the IEEEpubid mark.



% use for special paper notices
%\IEEEspecialpapernotice{(Invited Paper)}

% make the title area
\maketitle

% As a general rule, do not put math, special symbols or citations
% in the abstract or keywords.
\begin{abstract}
As a cornerstone in the Evolutionary Computation (EC) domain, Differential Evolution (DE) is known for its simplicity and effectiveness in handling challenging black-box optimization problems.
While the advantages of DE are well-recognized, achieving peak performance heavily depends on its hyperparameters such as the mutation factor, crossover probability, and the selection of specific DE strategies.
Traditional approaches to this hyperparameter dilemma have leaned towards parameter tuning or adaptive mechanisms.
However, identifying the optimal settings tailored for specific problems remains a persistent challenge.
In response, we introduce MetaDE, an approach that evolves DE's intrinsic hyperparameters and strategies using DE itself at a meta-level.
A pivotal aspect of MetaDE is a specialized parameterization technique, which endows it with the capability to dynamically modify DE's parameters and strategies throughout the evolutionary process.
To augment computational efficiency, MetaDE incorporates a design that leverages parallel processing through a GPU-accelerated computing framework.
Within such a framework, DE is not just a solver but also an optimizer for its own configurations, thus streamlining the process of hyperparameter optimization and problem-solving into a cohesive and automated workflow.
Extensive evaluations on the CEC2022 benchmark suite demonstrate MetaDE's promising performance.
Moreover, when applied to robot control via evolutionary reinforcement learning, MetaDE also demonstrates promising performance.
The source code of MetaDE is publicly accessible at: \url{https://github.com/EMI-Group/metade}.
\end{abstract}



% Note that keywords are not normally used for peerreview papers.
\begin{IEEEkeywords}
Differential Evolution, Meta Evolutionary Algorithm, GPU Computing
\end{IEEEkeywords}


% For peer review papers, you can put extra information on the cover
% page as needed:
% \ifCLASSOPTIONpeerreview
% \begin{center} \bfseries EDICS Category: 3-BBND \end{center}
% \fi
%
% For peerreview papers, this IEEEtran command inserts a page break and
% creates the second title. It will be ignored for other modes.
\IEEEpeerreviewmaketitle



\section{Introduction}
\IEEEPARstart{T}{he} Differential Evolution (DE) \cite{DE1996,DEcontest1996,DEusage1996,DE1997} algorithm, introduced by Storn and Price in 1995, has emerged as a cornerstone in the realm of evolutionary computation (EC) for its prowess in addressing complex optimization problems across diverse domains of science and engineering.
DE's comparative advantage over other evolutionary algorithms is evident in its streamlined design, robust performance, and ease of implementation.
Notably, with just three primary control parameters, i.e., scaling factor, crossover rate, and population size, DE operates efficiently.
This minimalistic design, paired with a lower algorithmic complexity, positions DE as an ideal candidate for large-scale optimization problems.
Its influential role in the optimization community is further cemented by its extensive research attention and successful applications over the past decades \cite{DEsurvey2011, DEsurvey2016, DEpapersurvey2020}, with DE and its derivatives often securing top positions in the IEEE Congress on Evolutionary Computation (CEC) competitions.



Despite the well recognized performance, DE is not without limitations.
Particularly, some studies indicate that DE's optimization process may stagnate if it fails to generate offspring solutions superior to their parents \cite{DEStagnation, neriDEsurvey}.
To avert this stagnation, selecting an appropriate parameter configuration to enhance DE's search capabilities becomes crucial.

However, the No Free Lunch (NFL) theorem \cite{NFL} suggests that a universally optimal parameter configuration is unattainable.
For example, while a higher mutation factor may aid in escaping local optima, a lower crossover probability might be preferable for problems with separability characteristics.

To address the intricate challenge of parameter configuration in DE, researchers often gravitate towards two predominant strategies: \emph{parameter control} and \emph{parameter tuning} \cite{param1999,paramTun2012,paramTun2020}.
Parameter control is a dynamic approach wherein the algorithm's parameters are adjusted on-the-fly during its execution.
This adaptability allows the algorithm to respond to the evolving characteristics of the problem landscape, enhancing its chance of finding optimal or near-optimal solutions.
Notably, DE has incorporated this strategy in several of its variants.
For instance, jDE \cite{jDE2006} adjusts the mutation factor and crossover rate during the run, while SaDE \cite{SaDE2008} dynamically chooses a mutation strategy based on its past success rates. Similarly, JaDE \cite{JADE2009} and CoDE \cite{CoDE2011} employ adaptive mechanisms to modify control parameters and mutation strategies, respectively.

In contrast, parameter tuning is a more static methodology, wherein the optimal configuration is established prior to the algorithm's initiation.
It aims to discover a parameter set that consistently demonstrates robust performance across various runs and problem instances.
Despite its potential for reliable outcomes, parameter tuning is known for its computational intensity, often necessitating dedicated optimization efforts or experimental designs to identify the optimal parameters, which may explain its limited exploration in the field.
Viewed as an optimization challenge, parameter tuning is also referred to as meta-optimization \cite{metaEAPhD2010}.
This perspective gave rise to \emph{MetaEA}, which optimizes the parameters of an EA using another EA.

Despite MetaEA's methodological elegance and simplicity, it confronts the significant challenge of depending on extensive function evaluations.
Fortunately, the inherent parallelism within MetaEA, across both meta-level and base-level populations, renders it particularly amenable to parallel computing environments.
However, a notable disparity exists between methodological innovations and the availability of advanced computational infrastructures, thus limiting MetaEA's potential due to the lack of advanced hardware accelerations such as GPUs.
To bridge this gap, we introduce the \emph{MetaDE} approach, which embodies the MetaEA paradigm by employing DE in a meta-level to guide the evolution of a specially tailored Parameterized Differential Evolution (PDE).

Designed with adaptability in mind, PDE can flexibly adjust its parameters and strategies, paving the way for a wide range of DE configurations.
As PDE interacts with the optimization problem at hand, the meta-level DE observes and refines PDE's settings to better align with the problem's characteristics.
Amplifying the efficiency of this nested optimization approach, MetaDE is integrated with a GPU-accelerated EC framework, thus weaving together parameter refinement and direct problem-solving into a seamless end-to-end approach to black-box optimization.
In summary, our main contributions are as follows.

\begin{itemize}
\item \textbf{Parameterized Differential Evolution:}
We have introduced Parameterized Differential Evolution (PDE), a variant of DE with augmented parameterization.
Unlike traditional DE algorithms that come with fixed mutation and crossover strategies, PDE’s architecture offers users the flexibility to adjust these parameters and strategies to fit the problem at hand.
This design not only allows for the creation of diverse DE configurations tailored for specific challenges but also ensures efficient computation.
To achieve this, all core operations of PDE, including mutation, crossover, and evaluation, have been optimized for parallel execution to harness advancement of GPU acceleration.

\item \textbf{MetaDE:}
Building on the MetaEA paradigm, we have designed the MetaDE approach.
Specifically, MetaDE employs a meta-level DE as an \texttt{evolver} to iteratively refine PDE's hyperparameters, which is guided by performance feedback from multiple PDE instances acting as \texttt{executors}.
This continuous optimization ensures PDE's configurations remain aligned with evolving problem landscapes.
Moreover, we have incorporated several specialized methods to further enhance the performance of MetaDE.



\item \textbf{GPU-accelerated Implementation:}
Breaking away from the limitations of conventional parameter tuning, we integrate MetaDE with a GPU-accelerated computing framework
 -- EvoX~\cite{evox}, which enhances MetaDE's computational prowess for facilitating swifter evaluations and algorithmic refinements.
With this specialized implementation, MetaDE provides an efficient and automated end-to-end approach to black-box optimization.
\end{itemize}



The subsequent sections are organized as follows. Section \ref{section_Preliminary} presents some preliminary knowledge for this work.
Section \ref{section_The proposed metade} elucidates the intricacies of the proposed approach, including PDE and the MetaDE.
Section \ref{section_Experimental_study} showcases the experimental results.
Finally, Section \ref{section Conclusion} wraps up the discourse and points towards avenues for future work.

\section{Preliminaries}\label{section_Preliminary}

\subsection{DE and its Parameter Adaption}\label{subsection_DE and Parameter adaption}
\subsubsection{Overview of DE}
As a typical EC algorithm, DE's essence lies in its differential mutation mechanism that drives the evolution of a population.
The operational cycle of DE unfolds iteratively, with each iteration embodying specific phases, as elaborated in Algorithm \ref{Alg_DE}:
\begin{enumerate}[label=\arabic*.]
  \item \textbf{Initialization} (Line 1):
  The algorithm initializes a set of potential solutions.
  Each of these solutions, representing vectors of decision variables, is randomly generated within the search space boundaries.
  \item \textbf{Mutation} (Lines 6-7):
  Each solution undergoes mutation to produce a mutant vector.
  This mutation process involves combinations of different individuals to form the mutant vectors.
  \item \textbf{Crossover} (Line 8):
  The crossover operation interchanges components between mutants and the original solutions to generate a trial vector.
  \item \textbf{Selection} (Lines 10-12): The trial vector competes against the original solution based on fitness, with the better solution progressing to the next generation.
\end{enumerate}


DE progresses through cycles of mutation, crossover, and selection, persisting until it encounters a termination criterion.
This could manifest as either reaching a predefined number of generations or achieving a target fitness threshold.
The algorithm's adaptability allows for the spawning of myriad DE variants by merely tweaking its mutation and crossover operations.
Specifically, DE variants follow a unified naming convention: \texttt{DE/x/y/z}, where \texttt{x} identifies the base vector used for mutation, \texttt{y} quantifies the number of difference involved, and \texttt{z} typifies the crossover method employed.
For example, the DE variant as presented in Algorithm \ref{Alg_DE} is named as \texttt{DE/rand/1/bin}.

{\linespread{1.1}
\begin{algorithm}
\small
\caption{DE}\label{Alg_DE}
\begin{algorithmic}[1]
  \Require {$D$, $NP$, $F$, $CR$, $G_{max}$}
  \State Initialize population $\mathbf{X} = \{\mathbf{x}_1, \mathbf{x}_2, \dots, \mathbf{x}_{\scalebox{0.5}{$\textit{NP}$}}\}$
  \State Evaluate the fitness of each individual in the population
  \State $g = 0$
  \While{$g \leq G_{max}$}
    \For{$i = 1$ to $NP$}
      \State Randomly select $\mathbf{x}_{r_1}$, $\mathbf{x}_{r_2}$, and $\mathbf{x}_{r_3}$ from $\mathbf{X}$,
      \Statex \qquad \quad such that $r_1 \neq r_2 \neq r_3 \neq i$
      \State Compute the mutant vector: $\mathbf{v}_i = \mathbf{x}_{r_1} + F \cdot (\mathbf{x}_{r_2} - \mathbf{x}_{r_3})$
      \State Perform crossover for each variable between $\mathbf{x}_i$ and $\mathbf{v}_i$:
      \begin{align*}
        \qquad \quad u_{i,j}=\begin{cases}
          v_{i,j},\ \text{if } \text{rand}(0, 1) \leq CR \text{ or } j = \text{randint}(1, D) \\
          x_{i,j},\ \text{otherwise}
        \end{cases}
      \end{align*}
      \State Evaluate the fitness of $\mathbf{u}_i$
      \If{$\textrm{f}(\mathbf{u}_i) \leq \textrm{f}(\mathbf{x}_i)$}
        \State Replace $\mathbf{x}_i$ with $\mathbf{u}_i$ in the population
      \EndIf
    \EndFor
    \State $g = g + 1$
  \EndWhile
  \State\Return the best fitness
\end{algorithmic}
\end{algorithm}
{\linespread{1}

\subsubsection{Parameter Modulation in DE}
DE employs a unique mutation mechanism, which adapts to the problem's natural scaling.
By adjusting the mutation step's size and orientation to the objective function landscape, DE embraces the \emph{contour matching principle} \cite{DEbook2006}, which promotes basin-to-basin transfer for enhancing the convergence of the algorithm.

At the core of DE's mutation is the scaling factor \( F \).
This factor not only determines the mutation's intensity but also governs its trajectory and ability to bypass local optima.
Commonly, \( F \) is set within the $[0.5, 1]$ interval, with a starting point often at 0.5.
While values outside the $[0.4, 1]$ range can sometimes yield good results, an \( F \) greater than 1 tends to slow convergence.
Conversely, values up to 1 generally promise swifter and more stable outcomes \cite{EPSDE2011}.
Nonetheless, to deter settling at suboptimal solutions too early, \( F \) should be adequately elevated.

Parallel to mutation, DE incorporates a uniform crossover operator, which is often labeled as discrete recombination or binomial crossover in the GA lexicon.
The crossover constant \( CR \) also plays a pivotal role, which determines the proportion of decision variables to be exchanged during the generation of offspring.
A low value for \( CR \) ensures only a small portion of decision variables are modified per iteration, thus leading to axis-aligned search steps.
As \( CR \) increases closer to 1, offspring tend to increasingly reflect their mutant parent, thereby curbing the generation of orthogonal search steps \cite{DEsurvey2011}.

For classical DE configurations, such as \texttt{DE/rand/1/bin}, rotational invariance is achieved only when \( CR \) is maxed out at 1.
Here, the crossover becomes wholly vector-driven, and offspring effectively mirror their mutants.
However, the optimal \( CR \) is intrinsically problem-dependent.
Empirical studies recommend a \( CR \) setting within the $[0, 0.2]$ range for problems characterized by separable decision variables.
Conversely, for problems with non-separable decision variables, a \( CR \) in the proximity of $[0.9, 1]$ is more effective \cite{DEsurvey2011}.

The adaptability of DE is evident in its wide spectrum of variants, each distinct in its mutation and crossover strategies with delicate parameter modulations.
In the following, we will detail seven mutation strategies and three crossover strategies, all of which are widely-recognized in state-of-the-art DE variants.
Here, the subscript notation in \( \textbf{x} \) specifies the individual selection technique.
For instance, \( \textbf{x}_r \) and \( \textbf{x}_{best} \) correspond to randomly selected and best-performing individuals respectively, whereas
\( \textbf{x}_i \) represents the currently evaluated individual.

\textbf{Mutation Strategies}:

\begin{enumerate}[label=\arabic*.]
  \item \texttt{DE/rand/1}:
        \begin{eqnarray}\label{equ_mutation rand}
        \begin{aligned}
        \mathbf{v}_i=\mathbf{x}_{r_1}+F \cdot\left(\mathbf{x}_{r_2}-\mathbf{x}_{r_3}\right).
        \end{aligned}
        \end{eqnarray}

  \item \texttt{DE/best/1}:
        \begin{eqnarray}\label{equ_mutation best}
        \begin{aligned}
        \mathbf{v}_i=\mathbf{x}_{\text {best }}+F \cdot\left(\mathbf{x}_{r_1}-\mathbf{x}_{r_2}\right).
        \end{aligned}
        \end{eqnarray}

  \item \texttt{DE/rand/2}:
        \begin{eqnarray}\label{equ_mutation rand2}
        \begin{aligned}
        \mathbf{v}_i & =\mathbf{x}_{r_1}+F \cdot\left(\mathbf{x}_{r_2}-\mathbf{x}_{r_3}\right)+F \cdot\left(\mathbf{x}_{r_4}-\mathbf{x}_{r_5}\right).
        \end{aligned}
        \end{eqnarray}

  \item \texttt{DE/best/2}:
        \begin{eqnarray}\label{equ_mutation best2}
        \begin{aligned}
        \mathbf{v}_i & =\mathbf{x}_{\text {best}}+F \cdot\left(\mathbf{x}_{r_1}-\mathbf{x}_{r_2}\right)+F \cdot\left(\mathbf{x}_{r_3}-\mathbf{x}_{r_4}\right).
        \end{aligned}
        \end{eqnarray}

  \item \texttt{DE/current-to-best/1}:
        \begin{eqnarray}\label{equ_mutation current2best}
        \begin{aligned}
        \mathbf{v}_i =\mathbf{x}_i+F \cdot\left(\mathbf{x}_{\text {best }}-\mathbf{x}_i\right)+F \cdot\left(\mathbf{x}_{r_1}-\mathbf{x}_{r_2}\right).
        \end{aligned}
        \end{eqnarray}
    The above five classical mutation strategies, introduced by Storn and Price \cite{DEbook2006}, cater to various problem landscapes.
    For instance, the `rand' variants help maintain population diversity, while strategies using two differences typically produce more diverse offspring than those relying on a single difference.

  \item \texttt{DE/current-to-pbest/1}:
        \begin{eqnarray}\label{equ_mutation current2pbest}
        \begin{aligned}
        \mathbf{v}_i =\mathbf{x}_i+F \cdot\left(\mathbf{x}_{\text {pbest}}-\mathbf{x}_i\right)+F \cdot\left(\mathbf{x}_{r_1}-\mathbf{x}_{r_2}\right).
        \end{aligned}
        \end{eqnarray}
    This strategy originates from JaDE \cite{JADE2009}. $\mathbf{x}_{\text{pbest}}$ is randomly selected from the top \emph{p}\% of individuals in the population (typically the top 10\%) to strike a balance between exploration and exploitation.


  \item \texttt{DE/current-to-rand/1}:
        \begin{eqnarray}\label{equ_mutation current2rand}
        \begin{aligned}
        &\mathbf{u}_i =\mathbf{x}_i+K_i\cdot\left(\mathbf{x}_{r_1}-\mathbf{x}_i\right)+F \cdot\left(\mathbf{x}_{r_2}-\mathbf{x}_{r_3}\right).
        \end{aligned}
        \end{eqnarray}
    Here, \(K_i\) is a random number from \(U(0,1)\).
    This strategy, originally proposed in \cite{DEintro1999}, emphasizes rotational invariance. By bypassing the crossover phase, it directly yields the trial vector \(\mathbf{u}_i\). Thus, it is ideal for addressing non-separable rotation challenges and has been a cornerstone for multiple adaptive DE variations.

  \end{enumerate}

\textbf{Crossover Strategies:}

\begin{enumerate}[label=\arabic*.]
  \item Binomial Crossover:
    \begin{eqnarray}\label{equ_cross bin}
    \begin{aligned}
    u_{i, j}= \begin{cases}v_{i, j}, & \text { if } r \leq C R \text { or } j=j_{\mathrm{rand}} \\ x_{i, j}, & \text {otherwise},\end{cases}
    \end{aligned}
    \end{eqnarray}
    where \(j_{\mathrm{rand}}\) is a random integer between 1 and \( D \). This strategy is a cornerstone in DE.

  \item Exponential Crossover:
  \begin{eqnarray}\label{equ_cross exp}
    \begin{aligned}
    \small
    u_{i, j}= \begin{cases}v_{i, j}{ } & \text { if } j=\langle n\rangle_d,\langle n+1\rangle_d,...,\langle n+L-1\rangle_d \\ x_{i, j} & \text {otherwise},\end{cases}
    \end{aligned}
    \end{eqnarray}
    where \(\langle \rangle_d\) is a modulo operation with \(D\) and \(L\) representing the crossover length, following a censored geometric distribution with a limit of \(D\) and probability of \(CR\).
    By focusing on consecutive variables, this strategy excels in handling problems with contiguous variable dependencies.

  \item Arithmetic Recombination:
  \begin{eqnarray}\label{equ_cross arith}
    \begin{aligned}
    \mathbf{u}_i=\mathbf{x}_i + K_i\cdot(\mathbf{v}_i - \mathbf{x}_i),
    \end{aligned}
    \end{eqnarray}
    where \(K_i\) is a random value from \(U(0,1)\).
    Exhibiting rotational invariance, this strategy, when combined with the \texttt{DE/rand/1} mutation, results in the \texttt{DE/current-to-rand/1} strategy \cite{DEsurvey2011}, as described by:
    \begin{eqnarray}\label{equ_rand_1_arith}
    \begin{aligned}
    \mathbf{u}_i&=\mathbf{x}_i + K_i\cdot(\mathbf{v}_i - \mathbf{x}_i)\\
    &=\mathbf{x}_i + K_i(\mathbf{x}_{r_1}+F\cdot(\mathbf{x}_{r_2}-\mathbf{x}_{r_3}) - \mathbf{x}_i)\\
    &=\mathbf{x}_i + K_i(\mathbf{x}_{r_1} - \mathbf{x}_i)+ K_i\cdot F(\mathbf{x}_{r_2}-\mathbf{x}_{r_3}),
    \end{aligned}
    \end{eqnarray}
    which is equivalent to Eq. (\ref{equ_mutation current2rand}).

\end{enumerate}


\subsubsection{Adaptive DE}
The development of parameter adaption in DE has witnessed significant advancements over time, from initial endeavors in parameter adaptation to recent sophisticated methods that merge multiple strategies.
This subsection traces the chronological advancements, emphasizing the pivotal contributions and their respective impacts on adaptive DE.

The earliest phase in DE's adaption centered on the modification of the crossover rate \( CR \) .
Pioneering algorithms such as SPDE \cite{SPDE2003} incorporated \( CR \) within the parameter set of individuals, enabling its simultaneous evolution with the decision variables of the problem to be solved.
This strategy was further refined by SDE \cite{SDE2005}, which assigned \( CR \) for each individual based on a normal distribution. Subsequent research efforts shifted focus to the scaling factor \( F \).
In this context, DETVSF \cite{DETVSF2005} dynamically adjusted \( F \), fostering exploration during the algorithm's nascent stages and pivoting to exploitation in later iterations.
Building on this, FaDE \cite{FaDE2005} employed fuzzy logic controllers to optimize mutation and crossover parameters.

The DESAP \cite{DESAP2006} algorithm marked a significant paradigm shift by introducing self-adapting populations and encapsulating control parameters within individuals.
Successive contributions like jDE \cite{jDE2006}, SaDE \cite{SaDE2008}, and JaDE \cite{JADE2009} accentuated the significance of parameter encoding, integrated innovative mutation strategies, and emphasized archiving optimization trajectories using external repositories.
Further, EPSDE \cite{EPSDE2011} and CoDE \cite{CoDE2011} enhanced the offspring generation process, amalgamating multiple strategies with randomized parameters.

The contemporary landscape of adaptive DE is characterized by complex methodologies and refined strategies.
Algorithms such as SHADE \cite{SHADE2013} and LSHADE \cite{LSHADE2014} championed the utilization of success-history mechanisms and dynamic population size modifications.
Notable developments like ADE \cite{ADE2014} introduced a biphasic parameter adaptation mechanism.
The domain further expanded with algorithms like LSHADE-RSP \cite{LSHADE-RSP2018}, IMODE \cite{IMODE2020}, and LADE \cite{LADE2023}, emphasizing mechanisms such as selective pressure, the integration of multiple DE variants, and the automation of the learning process.

Undoubtedly, the adaptive DE domain has witnessed transformative growth, with each phase of its evolution contributing to its current sophistication.
However, despite these advancements, many adaptive strategies remain empirical and hinge on manual designs, while their effectiveness is not universally guaranteed.

\subsection{Distributed DE}
The integration of distributed (i.e., multi-population) strategies also significantly enhances the efficacy of DE. 
Leading this advancement, Weber \textit{et al.} conducted extensive research on scale factor interactions and mechanisms within a distributed DE framework \cite{weber2010study, weber2011study, weber2013study}, followed by ongoing developments along the pathway \cite{DEpapersurvey2020}.
For example, some works such as EDEV \cite{EDEV2018}, MPEDE \cite{MPEDE2015} and IMPEDE \cite{IMPEDE} adopted multi-population frameworks to ensemble various DE variants/operators,
while the other works  such as DDE-AMS \cite{DDE-AMS} and DDE-ARA \cite{DDE-ARA} employed multiple populations for adaptive resource allocations.

Despite the achievements, current implementations of distributed DE often focus predominantly on algorithmic improvements, while overlooking potential enhancements from advanced hardware accelerations such as GPU computing. 
Besides, the design of these distributed strategies often features intricate and rigid configurations that lack proper flexibility.



\subsection{MetaEAs}\label{subsection_Meta-EA}

Generally, the term \emph{meta} refers to a higher-level abstraction of an underlying concept, often characterized by its \emph{recursive} nature.
In the context of EC, inception of the Meta Evolutionary Algorithms (MetaEAs) can be traced back to the pioneering works of Mercer and Sampson \cite{metaplan1978} in the late 1970s.
Under the initiative termed \emph{meta-plan}, their pioneering efforts aimed at enhancing EA performance by optimizing its parameters through another EA.
Although sharing similarities with hyperheuristics \cite{Hyperheu2013,Hyperheu2020,NeriHyperspam}, a major difference distinguishes MetaEAs: while hyperheuristics often delve into selecting and fine-tuning a set of predefined algorithms, MetaEAs concentrate on the paradigm of refining the parameters of EAs by EAs.
Notably, MetaEAs are also akin to ensemble of algorithms, such as EDEV \cite{EDEV2018} and CoDE \cite{CoDE2011}, which amalgamate diverse algorithms to ascertain the most efficacious among them.


Advancing the meta-plan concept, MetaGA \cite{MetaGA1986} emerged as a significant milestone.
Here, a genetic algorithm (GA) was deployed to fine-tune six intrinsic control parameters, namely: population size, crossover rate, mutation rate, generation gap, scaling window, and selection strategy.
The efficacy of this approach was gauged using dual metrics: online and offline performance.

The evolution of the concept continued with MetaEP \cite{metaEP1991}, which offers a meta-level evolutionary programming (EP) that could concurrently evolve optimal parameter settings.
Another pivotal contribution was the Parameter Relevance Estimation and Value Calibration (REVAC) \cite{REVAC2007}, which served as a meta estimation of distribution algorithm (MetaEDA).
Utilizing a GA at its core, REVAC iteratively discerned promising parameter value distributions within the configuration space.

Innovations in the domain persisted with the Gender-based GA (GGA) \cite{GGA2009}, inspired by natural gender differentiation, and other notable methods like MetaCMAES \cite{metaCMAES2012}.
As articulated in the PhD thesis by Pedersen \cite{metaEAPhD2010}, a profound insight into MetaEA revealed that while contemporary optimizers endowed with adaptive behavioral parameters offered advantages, they were often eclipsed by streamlined optimizers under appropriate parameter tuning.
This thesis, which embraced DE as one of its optimization tools, employed the Local Unimodal Sampling (LUS) heuristic for tuning parameters such as \( NP \), \( F \), and \( CR \).

Culminating the discourse, the work in \cite{metaEAdistributed} demonstrated the scalability of MetaEAs by harnessing it within a large-scale distributed computing environment.
With the ($\mu$, $\lambda$)$-$ES steering the meta-level tuning, base-level algorithms like GA, ES, and DE were adeptly optimized.
For DE, parameters optimized encompassed \( NP \), mutation operator, \( F \), \( CR \), and \( PF \) (parameter for the \emph{either-or} strategy), enhancing MetaEAs' prowess in addressing intricate, large-scale optimization problems.
Recently, the MetaEA paradigm has also been employed for automated design of ensemble DE \cite{EDE}.

The field of MetaEAs has shown steady progress since its inception in the 1970s.
However, despite the achievements, the landscape of MetaEAs research still confronts certain limitations.
Notably, the research, while promising, has predominantly remained confined to smaller-scale implementations.
The anticipated leap to large-scale experiments, especially those that might benefit from GPU acceleration, remains largely uncharted. This underscores an imperative need for more extensive empirical validations and the exploration of contemporary computational resources to fully realize the potential of MetaEAs.

\subsection{GPU-accelerated EC Framework}\label{subsection_Meta-optimization}
To capitalize on the advancements of modern computing infrastructures, we have seamlessly integrated our proposed MetaDE with EvoX~\cite{evox}, a distributed GPU-accelerated computing framework for scalable EC.
This integration ensures that MetaDE enables efficient execution and optimization for large-scale evaluations.

The EvoX framework provides several distinctive features.
Primarily, it is designed for optimal performance across diverse distributed systems and is tailored to manage large-scale challenges.
Its user-friendly functional programming model simplifies the EC algorithm development process, reducing inherent complexities.
The framework cohesively integrates data streams and functional elements into a comprehensive workflow, underpinned by a sophisticated hierarchical state management system.
Moreover, EvoX features a rich library of EC algorithms, proficient in addressing a wide array of tasks, from black-box optimization to advanced areas such as deep neuroevolution and evolutionary reinforcement learning.


\section{Proposed Approach}\label{section_The proposed metade}
The foundational premise of MetaDE is to utilize a core DE algorithm to evolve an ensemble of parameterized DE variants.
Within this framework, the core DE, which tunes the parameters, is termed the \texttt{evolver}.
In contrast, each parameterized DE variant, which optimizes the problem at hand, is termed the \texttt{executor}.
This section commences by augmenting the parameterization of DE in a more general manner, such that DE is made \emph{evolvable} by spawning various DE variants by modulating the parameters.
Then, this section details the integration of proposed MetaDE within a meta-framework, together with a brief introduction to the GPU-accelerated implementation.

\subsection{Augmented Parameterization of DE}\label{section_PDE}
To make DE evolvable, this subsection introduces the Parameterized Differential Evolution (PDE), an extension of the standard DE designed to augment its flexibility through the parameterization of mutation and crossover strategies.
While PDE retains the foundational principles of standard DE, its distinctiveness lies in its capability to generate a multitude of strategies by modulating the parameters.

In the standard DE framework, tunability is constrained to the adjustments of the \( F \) and \( CR \) parameters, and strategies are bound by predefined rules.
To augment this limited flexibility, PDE introduces a more granular parameterization.
Building upon DE's notation \texttt{DE/x/y/z}, PDE encompasses six parameters: \( F \) (scale factor), \( CR \) (crossover rate), \( bl \) (base vector left), \( br \) (base vector right), \( dn \) (difference number), and \( cs \) (crossover scheme).
The combined roles of \( bl \) and \( br \) determine the base vector, leading to a strategy notation for PDE expressed as \texttt{DE/bl-to-br/dn/cs}.

Each parameter's nuances and the array of strategy combinations they enable are elaborated upon in the subsequent sections.
Fig. \ref{Figure_structure} provides a visual representation of these parameters, delineated within dashed boxes.


\begin{figure*}[!h]
\centering
\includegraphics[scale=0.6]{fix_structure.pdf}
\caption{
Parameter delineation of PDE and their respective domains.
PDE comprehensively parameterizes DE, endorsing unrestricted parameter and strategy modifications.
In this schema, \( F \) and \( CR \) are continuous parameters, whereas others are categorical.
The dashed-line boxes exhibit their specific value ranges.
The mutation function is derived from the base vector left, base vector right, and difference number parameters.
}
\label{Figure_structure}
\end{figure*}

\subsubsection{Scale Factor \( F \) and Crossover Rate \( CR \)}
The scale factor \( F \) and crossover rate \( CR \) serve as pivotal parameters in DE, both represented as real numbers.

The \( F \) parameter regulates the differential variation among population entities.
Elevated values induce exploratory search behaviors, while lower values encourage more exploitation.
Although Storn and Price originally identified [0, 2] as an effective domain for \( F \) \cite{DE1997}, contemporary DE variants deem \( F \leq 1 \) as more judicious \cite{JADE2009,CoDE2011,EPSDE2011,SHADE2013,LSHADE2014}.
Consequently, PDE constrains \( F \) within [0,1].

On the other hand, \( CR \) controls the recombination extent during crossover.
Higher \( CR \) make it more likely to try out new gene combinations from changes, while lower values keep the genes more stable and similar to the original.
As a rate (or probability), \( CR \) is confined to the [0, 1] interval.

\subsubsection{Augmented Parameterization of Mutation}
The distinctiveness of PDE lies in its ability to generate a multitude of mutation strategies by modulating parameters $bl, br$, and $dn$.
Parameters $bl$ (base vector left) and $br$ (base vector right) select one of four possible vectors:
\begin{itemize}
  \item  \texttt{rand}: A randomly chosen individual from the population.
  \item  \texttt{best}: The best individual in the population.
  \item  \texttt{pbest}: A random selection from the top \( p\% \) of individuals.
  \item  \texttt{current}: The present parent individual.
\end{itemize}
Parameter \( dn \), controls the number of differences in mutation, assumes values in the set \{1, 2, 3, 4\}.
Specifically, each difference \( \Delta \) captures the difference between two unique and randomly selected individuals from the population.
Therefore, the mutation formulation \texttt{DE/bl-to-br/dn} is:
\begin{align}\label{equ_PDE_mutation}
\mathbf{v} = \mathbf{x}_{bl} + F \cdot (\mathbf{x}_{br} - \mathbf{x}_{bl}) + F \cdot (\Delta_1 + ... +\Delta_{dn}).
\end{align}
In the implementation of PDE, \( bl \) and \( br \) are encoded as: 1: \texttt{rand}, 2: \texttt{best},  3: \texttt{pbest}, and   4: \texttt{current}.
In addition, if both \( bl \) and \( br \) assume identical values, the term $F \cdot (\mathbf{x}_{br} - \mathbf{x}_{bl})$ will disappear. The base vector takes the value of $\mathbf{x}_{bl}$ directly and the mutation strategy then becomes non-directional (e.g., \texttt{DE/rand/1}).

\subsubsection{Augmented Parameterization of Crossover}
Parameter \( cs \) defines the crossover strategies in PDE, comprising those elaborated in Section~\ref{subsection_DE and Parameter adaption}: binomial crossover, exponential crossover, and arithmetic recombination.
Specifically, \( cs \) is encoded as:   1: \texttt{bin}, 2: \texttt{exp}, and 3: \texttt{arith}, representing binomial crossover, exponential crossover, and arithmetic recombination respectively.

As a result, the augmented parameterizations for mutation and crossover, denoted as \texttt{DE/bl-to-br/dn/cs}, will give rise to a spectrum of 192 distinctive strategies in total.
This breadth allows for the encapsulation of mainstream strategies detailed in Section~\ref{subsection_DE and Parameter adaption}, as illustrated in Table \ref{tab_param encoding}.
When synergized with \( F \) and \( CR \), this culminates in a comprehensive parameter configuration landscape.


% Table generated by Excel2LaTeX from sheet 'Sheet1'
\begin{table}[htbp]
  \centering
  \caption{The encoding of typical DE variants by the proposed PDE.}
         \renewcommand{\arraystretch}{1.1}
 \renewcommand{\tabcolsep}{10pt}
    \begin{tabular}{ccccc}
    \toprule
    strategy & $bl$  & $br$  & $dn$  & $cs$ \\
    \midrule
    \texttt{DE/rand/1/bin} & 1     & 1     & 1     & 1 \\
    \midrule
    \texttt{DE/best/1/bin} & 2     & 2     & 1     & 1 \\
    \midrule
    \texttt{DE/current-to-best/1/bin} & 4     & 2     & 1     & 1 \\
    \midrule
    \texttt{DE/rand/2/bin} & 1     & 1     & 2     & 1 \\
    \midrule
    \texttt{DE/best/2/bin} & 2     & 2     & 2     & 1 \\
    \midrule
    \texttt{DE/current-to-pbest/1/bin} & 4     & 3     & 1     & 1 \\
    \midrule
    \texttt{DE/current-to-rand/1*} & 1     & 1     & 1     & 3 \\
    \bottomrule
    \end{tabular}%
  \label{tab_param encoding}%
    \vspace{0.5em}

      \footnotesize
    \textsuperscript{*} As per Eq. (\ref{equ_rand_1_arith}), \texttt{DE/current-to-rand/1} is equivalent to \texttt{DE/rand/1/arith}.\\
\end{table}%

Algorithm \ref{Alg_PDE} details PDE's procedure.
Its distinct attribute is Line 3, where the mutation function is shaped by \( bl \), \( br \), and \( dn \).
Line 4 designates the crossover function based on \( cs \).
The evolutionary phase commences at Line 6.
Notably, PDE's concurrent mutation and crossover operations (Lines 7-10) are {tensorized} to facilitate parallel offspring generation, which deviate from the conventional sequential operations in standard DE.
With such a tailored procedure, the computational efficiency can be substantially improved.


\begin{algorithm}
\small
\caption{Parameterized DE (PDE)}\label{Alg_PDE}
\begin{algorithmic}[1]
  \Require {$D$, $NP$, $G_{max}$, $F$, $CR$, $bl$ (base vector left), $br$ (base vector right), $dn$ (difference number), $cs$ (crossover scheme)}
  \State Initialize population $\mathbf{X} = \{\mathbf{x}_1, \mathbf{x}_2, \dots, \mathbf{x}_{\scalebox{0.5}{$\textit{NP}$}}\}$
  \State Evaluate the fitness of each individual in the population
  \State Generate mutation function $M(\mathbf{X})$ according to $bl$, $br$, $dn$:
  \Statex $\mathbf{v} = \mathbf{x}_{bl} + F \cdot (\mathbf{x}_{br} - \mathbf{x}_{bl}) + F \cdot (\Delta_1 + ... +\Delta_{dn})$
  \State According to $cs$, choose a crossover function $C(\mathbf{V}, \mathbf{X})$ in (\ref{equ_cross bin}-\ref{equ_cross arith})
  \State $g = 0$
  \While{$g \leq G_{max}$}
  \State Generate $NP$ mutant vectors: $\mathbf{V} = M(\mathbf{X})$
  \State Perform crossover for all mutant vectors: $\mathbf{U} = C(\mathbf{V}, \mathbf{X})$
  \State Evaluate the fitness of $\mathbf{U}$
  \State Make selection between $\mathbf{U}$ and $\mathbf{X}$
  \State $g = g + 1$
  \EndWhile
  \State\Return the best fitness
\end{algorithmic}
\end{algorithm}


\subsection{Architecture of MetaDE}
Atop the proposed PDE, this subsection further introduces the architecture of MetaDE.
The main target of MetaDE is to evolve the parameters of PDE through an external DE, empowering PDE to identify optimal parameters tailored to the target problem.
% Every computational phase of MetaDE benefits from extensive parallelization with GPU acceleration integration.

\begin{figure}[t]
\centering
\includegraphics[scale=0.9]{MetaDE.pdf}
\caption{
Architecture of MetaDE.
Within this architecture, a conventional DE algorithm operates as an \texttt{evolver}, where its individual $\mathbf{x}_i$ represents a distinct parameter configuration $\mathbf{\theta}_i$.
These configurations are relayed to PDE  to instantiate diverse DE variants as the \texttt{executors}.
Each \texttt{executor} then evolves its distinct population and returns the best fitness $y^*$ as identified, which is subsequently set as the fitness of $\mathbf{x}_i$.
}
\label{Figure_MetaDE}
\end{figure}

As illustrated in Fig. \ref{Figure_MetaDE}, MetaDE is structured with a two-tiered optimization architecture.
The upper tier, termed the \texttt{evolver}, leverages DE to evolve the parameters of PDE. In contrast, the lower tier consists of a collection of \texttt{executors} that each run the parameterized PDE instance to optimize the objective function.
Every individual in the \texttt{evolver}, represented as $\mathbf{x}_i$, is decoded into a parameter configuration $\bm{\theta}_i$ with six elements: $F$, $CR$, $bl$, $br$, $dn$, and $cs$.
For the evaluation of each individual, the configuration $\bm{\theta}_i$ is directed to its respective \texttt{executor} $\textrm{PDE}_i$ for objective function optimization.
The final fitness $y^*$ as identified by each \texttt{executor}, is subsequently set as the fitness of the corresponding $\mathbf{x}_i$ individual.

The architecture of MetaDE is streamlined for simplicity.
Building upon this architecture, MetaDE integrates two tailored components: the \emph{one-shot evaluation method} and the \emph{power-up strategy}.
These components further enhance the adaptability and efficiency of the \texttt{executors}, thereby elevating the overall performance of MetaDE.

\subsection{One-shot Evaluation Method}
Within the context of an \texttt{executor} driven by DE itself, the inherent stochastic nature can lead to variability in the optimal fitness values returned.
Historically, several evaluation techniques, such as repeated evaluation~\cite{metaCMAES2012}, F-racing~\cite{FRacing}, and intensification~\cite{intens}, have been put forth to tackle this inconsistency.
Yet, these often come at the cost of an exorbitant number of functional evaluations (FEs).
To address this issue, we introduce the one-shot evaluation method.

Specifically, the method mandates each \texttt{executor} to undertake a singular, comprehensive independent run, subsequently returning its best-found solution.
A distinguishing aspect of this method is the consistent allocation of the same initial random seed to every \texttt{executor}.
As the algorithm progresses, this uniform seed ensures that the PDE fine-tunes its parameters in a consistent manner, thereby identifying optimal parameters tailored to the given seed environment.
Essentially, this strategy embeds the seed as an integral facet of the problem domain.

\subsection{Power-up Strategy}
During the independent runs of an \texttt{executor}, the allocation of FEs plays a pivotal role in determining both the quality of solutions and computational efficiency. Allocating an excessive number of FEs indiscriminately can lead to undue computational resource consumption without necessarily improving solution quality.
To address this issue, we propose the power-up strategy.

The essence of this strategy is dynamic FE allocation: while earlier iterations receive a moderate number of FEs to ensure resource efficiency, a more generous allocation (fivefold) is reserved for the terminal iteration within the evolutionary process.
This strategy ensures that the \texttt{executor} has the resources for a thorough and comprehensive evaluation during its most crucial phase -- the final generation of the \texttt{evolver}.


\subsection{Implementation}
As outlined in Algorithm~\ref{Alg_MetaDE}, MetaDE draws its simple algorithmic workflow from conventional DE. MetaDE adopts \texttt{DE/rand/1/bin} as the \texttt{evolver}.
The initialization phase (Line 1) spawns the MetaDE population within the parameter boundaries \( [\mathbf{lb}, \mathbf{ub}] \).
During the evaluation phase (Lines 6-11), each individual is decoded into a parameter blueprint and directed to an independent PDE instance (\texttt{executor}) for problem resolution.
Running for a predetermined iteration count \( G' \), each \texttt{executor} subsequently reports the best fitness.
Notably, Line 10 encapsulates the essence of the power-up strategy: for MetaDE's concluding iteration (\( g == G_{max} \)), the evaluation quota is amplified to \( 5 \times G' \) for the \texttt{executors}.


\begin{algorithm}
\small
\caption{MetaDE}
\label{Alg_MetaDE}
\begin{algorithmic}[1]
  \Require {$D$, $NP$, $G_{max}$, $\mathbf{lb}$ (lower boundaries of PDE's parameters), $\mathbf{ub}$ (upper boundaries of PDE's parameters), $NP'$ (population size of PDE), $G'$ (max generations of PDE)}
  \State Initialize population $\mathbf{X} = \{\mathbf{x}_1, \mathbf{x}_2, \dots, \mathbf{x}_{\scalebox{0.5}{$\textit{NP}$}}\}$ between $[\mathbf{lb}, \mathbf{ub}]$
  \State Initialize the fitness of $\mathbf{X}$: $\mathbf{y}=\mathbf{inf}$
  \State $g = 0$
  \While{$g \leq G_{max}$}
  \State Generate trial vectors $\mathbf{U}$ by mutation and crossover
  \Statex \quad \ /* The mutation and crossover scheme used is rand/1/bin\ */
  \State Decode each trial vector $\mathbf{u}$ into parameters:
  \Statex \quad \ $F=\mathbf{u}[1], CR=\mathbf{u}[2], bl=\textrm{floor}(\mathbf{u}[3]), br=\textrm{floor}(\mathbf{u}[4])$,
  \Statex \quad \ $dn=\textrm{floor}(\mathbf{u}[5]), cs=\textrm{floor}(\mathbf{u}[6])$
  %\Statex /*Evaluate each $\mathbf{u}$ by running PDE for certain generations*/
  \If {$g < G_{max}$}
    \State $\mathbf{y} = \textrm{PDE}(D, NP', G', F, CR, bl, br, dn, cs)$
  \Else
    \State $\mathbf{y} = \textrm{PDE}(D, NP', 5 * G', F, CR, bl, br, dn, cs)$
  \EndIf
  \State Make selection between $\mathbf{U}$ and $\mathbf{X}$
  \State $g = g + 1$
  \EndWhile
  \State\Return the best individual and fitness
\end{algorithmic}
\end{algorithm}


Evidently, the algorithmic design of MetaDE provides an automated end-to-end approach to black-box optimization.
However, the computational demands of MetaDE, particularly in terms of FEs, cannot be understated.
In historical computational contexts, such intensive demands might have posed significant impediments.
Fortunately, contemporary advancements in computational infrastructures, coupled with the ubiquity of high-performance computational apparatuses such as GPUs, have substantially alleviated such a challenge.


Hence, we leverage the GPU-accelerated framework of EvoX~\cite{evox} for the implementation of MetaDE.
Thanks to the inherently parallel nature of MetaDE, computational tasks can be judiciously delegated to GPUs to engender optimized runtime performance.
Specifically, the parallelism in MetaDE manifests in three distinct facets:
\begin{itemize}
  \item \textbf{Parallel Initialization and Execution:}
 The multiple \texttt{executors} are instantiated and operated concurrently, each tailored by a unique parameter configuration derived from the MetaDE ensemble.
  This simultaneous operation enables comprehensive exploration across varied parameter landscapes.

  \item \textbf{Parallel Offspring Generation:}
  Both the \texttt{evolver} and \texttt{executors} adhere to parallel strategies for offspring inception.
  By synchronizing and coordinating mutations and crossover operations in their respective populations, MetaDE is able to rapidly produce offspring, thereby accelerating the evolutionary process.

  \item \textbf{Parallel Fitness Evaluations:}
  Each \texttt{executor} conducts fitness evaluations concurrently across its member individuals.
  Given the substantial number of the individuals within the populations of the \texttt{executors}, this parallel strategy significantly enhances the overall efficiency of MetaDE.
\end{itemize}


\lstset{
  language=Python,
  aboveskip=3mm,
  belowskip=3mm,
  showstringspaces=false,
  columns=flexible,
  basicstyle={\footnotesize\ttfamily},
  numbers=left,
  numberstyle=\tiny\color{gray},
  xleftmargin=2em,
  keywordstyle=\bfseries\color{dkgreen},
  commentstyle=\color{gray}\itshape,
  stringstyle=\color{mauve},
  breaklines=true,
  breakatwhitespace=true,
  tabsize=3,
  emph={[1]evox,BatchExecutor,MetaProblem},          % Emphasize numpy
  emphstyle={[1]\bfseries\color{dkblue}},  % Set the style for emphasized words
  emph={[2]__init__, min, evaluate, reproduce, init},
  emphstyle={[2]\color{dkblue}},
  emph={[3]self,super},          % Emphasize numpy
  emphstyle={[3]\color{dkgreen}},
  morekeywords={from,import},  % Add more keywords
  captionpos=b,             % Caption position
      frame=lines,
  framesep=2mm,
}
\
\begin{lstlisting}[caption={Demonstrative implementation of MetaDE leveraging the computational workflow of EvoX. The implementation is distinctly divided into four pivotal components: Workflow Initialization, Meta Problem Transformation, Computing Workflow Creation, and Execution.}, label={lst:python_example}, float=!t]
from evox import algorithms, problems, ...

### Initialization ###
evolver = algorithm.DE()  # specify evolver
executor = algorithm.PDE()  # specify executor
problem = ...  # specify optimization problem

### Meta Problem Transformation ###
class MetaProblem(Problem):
    def __init__(self, batch_executor, ... ):
        # vectorize fitness evaluations
        self.batch_evaluate = vectorize(vectorize(problem.evaluate))

    def evaluate(self, state, ...):
        ...
        # run executors
        while ...:
            ...
            batch_fits, ... = self.batch_evaluate(...)
        # return fitness
        return min(min(batch_fits))

### Computing Workflow Creation ###
batch_executor = create_batch_executor(...)
meta_problem = MetaProblem(batch_executor, ...)
workflow = workflow.UniWorkflow(
            algorithm = evolver,
            pop_transform = decoder,
            problem = meta_problem,
        )

### Execution ###
while ...:
    state = workflow.step(state)
\end{lstlisting}

MetaDE adheres rigorously to the functional programming paradigm, capitalizing on automatic vectorization for parallel execution. Core algorithmic components, including crossover, mutation, and evaluation, are constructed using pure functions. Subsequently, the entire program is mapped to a GPU-based computation graph, ushering in accelerated processing. EvoX's adept state management ensures a seamless transfer of the algorithm's prevailing state, encompassing aspects like population, fitness, hyperparameters, and auxiliary data.

Listing \ref{lst:python_example} elucidates a representative implementation of MetaDE underpinned by the EvoX framework, which is meticulously segmented into four salient phases:

\begin{itemize}
    \item \textbf{Initialization}: Herein, primary entities like the \texttt{evolver} (employing the traditional DE) and the \texttt{executor} (utilizing the proposed PDE) are instantiated. Concurrently, the target optimization problem is defined.

    \item \textbf{Meta Problem Transformation}: Within this phase, the original optimization problem is transformed to align with the meta framework. This metamorphosis is realized via the \texttt{\textbf{MetaProblem}} class, where the evaluation function undergoes vectorization, priming it for efficient batch assessments and facilitating concurrent evaluations of manifold configurations.

    \item \textbf{Computing Workflow Creation}: Post transformation, the workflow is architected to seamlessly amalgamate the initialized components. The \texttt{batch\_executor} is crafted for batched operations of DE variants, and the \texttt{\textbf{MetaProblem}} is instantiated therewith. The holistic workflow, embodied by the \texttt{UniWorkflow} class, is then constructed, weaving together the \texttt{evolver}, the transformed problem, and a (\texttt{decoder}) which transforms the \texttt{evolver}'s population into specific hyperparameters for instantiating  the DE variant of each \texttt{executor}.

    \item \textbf{Execution}: Having established the groundwork, MetaDE's execution phase is triggered, autonomously driving the computing workflow across distributed GPUs.
    This workflow is traversed iteratively, culminating once a predefined termination criterion is met.
\end{itemize}

\section{Experimental Study}\label{section_Experimental_study}
In this section, we conduct detailed experimental assessments of MetaDE's capabilities.
First, we comprehensively benchmark MetaDE against several representative DE variants and CEC2022 top algorithms to gauge its relative performance on the CEC2022 benchmark suite \cite{CEC2022SO}.
Then, we investigate the optimal DE variants obtained by MetaDE in the benchmark experiment.
Finally, we apply MetaDE to robot control tasks.
All experiments were conducted on a system equipped with an Intel Core i9-10900X CPU and  an NVIDIA RTX 3090 GPU.
For GPU acceleration, all the algorithms and test functions were implemented within EvoX \cite{evox}.

\subsection{Benchmarks against Representative DE Variants}\label{section_Comparison with Classic DE Variants}

\subsubsection{Experimental Setup}
The CEC2022 benchmark suite for single-objective black-box optimization was utilized for this study.
This suite includes basic ($F_1-F_5$), hybrid ($F_6-F_8$), and composition functions ($F_9-F_{12}$), catering to various optimization characteristics such as unimodality/multimodality and separability/non-separability.

For benchmark comparisons, we selected seven representative DE variants: DE (\texttt{rand/1/bin}) \cite{DE1997}, SaDE \cite{SaDE2008}, JaDE \cite{JADE2009}, CoDE \cite{CoDE2011}, SHADE \cite{SHADE2013}, LSHADE-RSP \cite{LSHADE-RSP2018}, and EDEV \cite{EDEV2018}, which encapsulate a spectrum of mutation, crossover, and adaptation strategies. All algorithms were reimplemented using EvoX, with each capable of running in parallel, including the concurrent evaluation and reproduction.

Their respective descriptions are as follows:
\begin{itemize}
  \item \texttt{DE/rand/1/bin} is a foundational DE variant, which leverages a random mutation strategy coupled with binomial crossover.
  \item SaDE maintains an archive for tracking successful strategies and \( CR \) values and exhibits adaptability in strategy selection and parameter adjustments throughout the optimization process.
  \item JaDE relies on the \texttt{current-to-pbest} mutation strategy and dynamically adjusts its \( F \) and \( CR \) parameters during the optimization trajectory.
  \item CoDE infuses generational diversity by composing three disparate strategies, each complemented with randomized parameters, for offspring generation.
  \item SHADE employs the current-to-pbest mutation strategy and integrates a success-history mechanism to fine-tune its \( F \) and \( CR \)  parameters adaptively.
  \item LSHADE-RSP, as one of the most competitive DE variants, employs delicate strategies such as linear population size reduction and ranking-based mutation.
  \item EDEV adopts a distributed framework that ensembles three classic DE variants: JaDE, CoDE, and EPSDE.
\end{itemize}

The population size for all comparative algorithms was uniformly set to 100, except for experiments involving large populations. The other parameters for these algorithms were adopted as per their default settings described in their respective publications.

In our MetaDE configuration, on one hand, the \texttt{evolver} had a population size of 100 and adopted the vanilla \texttt{rand/1/bin} strategy with $F=0.5$ and $CR=0.9$;
On the other hand, each \texttt{executor} maintained a population of 100, iterating 1000 times for all the problems.
For simplicity, any result exceeding the precision of $10^{-8}$ was truncated to 0.
All statistical results were obtained via 31 independent runs\footnote{Full results, including the statistical results applying Wilcoxon rank-sum tests with a a significance level of 0.05, can be found in the Supplementary Document.}.




\begin{figure}[!t]
\centering
\includegraphics[scale=0.3]{D10.pdf}
\caption{Convergence curves on 10D problems in CEC2022 benchmark suite. The peer DE variants are set with population size of 100.}
\label{Figure_convergence_10D}
\end{figure}

\begin{figure}[!t]
\centering
\includegraphics[scale=0.3]{D20.pdf}
\caption{Convergence curves on 20D problems in CEC2022 benchmark suite. The peer DE variants are set with population size of 100.}
\label{Figure_convergence_20D}
\end{figure}


\subsubsection{Performance under Equal Wall-clock Time}\label{sec:expereiment_time}
In this part, we set equal  wall-clock time (\SI{60}{\second}) as the termination condition for running each test. 
{
This approach aligns with the practical constraints of modern GPU computing, where execution time serves as a more meaningful and comparable measure of performance across algorithms. Since all algorithms in our experiments are implemented with GPU parallelism, this setup ensures fairness by standardizing the computational resources and focusing on efficiency within the same time budget.
}

As shown in Figs. \ref{Figure_convergence_10D} and \ref{Figure_convergence_20D}, we selected five challenging problems, specifically $F_2$, $F_4$, $F_6$, $F_9$, and $F_{10}$, to demonstrate the convergence profiles.
Notably, MetaDE's convergence curve is observably more favorable, consistently registering lower errors than its counterparts across the majority of the problems.
Particularly, on $F_2$, $F_4$, $F_9$, and $F_{10}$, MetaDE exhibits resilience against local optima entrapment and subsequent convergence stagnation.
This is attributed to MetaDE's capability to identify optimal algorithm settings tailored for diverse problems, rather than merely tweaking parameters based on isolated segments of the optimization trajectory, as is the case with some DE variants.
An intriguing characteristic of MetaDE's convergence, evident in functions like $F_9$ (refer to Fig. \ref{Figure_convergence_10D}), is its pronounced performance surge in the optimization's terminal phase.
This enhancement can be linked to MetaDE's power-up strategy of allocating bonus computational resources in its final phase (as per Line 10 of Algorithm \ref{Alg_MetaDE}).


\begin{figure}[!h]
\centering
\includegraphics[scale=0.3]{FEs.pdf}
\caption{The number of FEs achieved by each algorithm within \SI{60}{\second}. The results are averaged on all 10D and 20D problems in the CEC2022 benchmark suite.}
\label{fig:maxFEs}
\end{figure}

{
Furthermore, to assess the concurrency of the algorithms, the number of FEs achieved by each algorithm within 60 seconds is shown in Table~\ref{tab:FEs} and Fig~\ref{fig:maxFEs}.
}
The results indicate that MetaDE achieves approximately $10^9$ FEs within 60 seconds, while the other algorithms manage to attain only around $10^7$ FEs in the same time frame.
The results demonstrate the high concurrency of MetaDE, which is particularly favorable in GPU computing.

\begin{table}[htbp]
  \centering
  
  \caption{{The number of FEs achieved by each algorithm within \SI{60}{\second}.}}
 
\scriptsize                   %设置字体大小
\renewcommand{\arraystretch}{1}
\renewcommand{\tabcolsep}{2.5pt}   %pt越大字越小
\resizebox{\linewidth}{!}{
% Table generated by Excel2LaTeX from sheet 'Experiment1 60S'
\begin{tabular}{cccccccccc}
\toprule
Dim   & Func  & MetaDE & DE    & SaDE  & JaDE  & CoDE  & SHADE &LSHADE-RSP&EDEV\\
\midrule
\multirow{12}[2]{*}{10D} & $F_{1}$ & \textbf{1.85E+09} & 4.28E+06 & 1.96E+06 & 2.55E+06 & 1.09E+07 & 2.24E+06&2.57E+06&2.79E+06 \\
      & $F_{2}$ & \textbf{1.84E+09} & 4.19E+06 & 1.89E+06 & 2.42E+06 & 1.11E+07 & 2.18E+06&2.58E+06&2.82E+06 \\
      & $F_{3}$ & \textbf{1.50E+09} & 4.00E+06 & 1.89E+06 & 2.50E+06 & 1.11E+07 & 2.10E+06& 2.46E+06&2.68E+06\\
      & $F_{4}$ & \textbf{1.84E+09} & 4.11E+06 & 2.02E+06 & 2.61E+06 & 1.11E+07 & 2.15E+06&2.60E+06&2.88E+06 \\
      & $F_{5}$ & \textbf{1.83E+09} & 4.13E+06 & 2.00E+06 & 2.60E+06 & 1.15E+07 & 2.17E+06& 2.53E+06&2.96E+06\\
      & $F_{6}$ & \textbf{1.84E+09} & 4.31E+06 & 1.96E+06 & 2.65E+06 & 1.07E+07 & 2.14E+06&2.55E+06&2.95E+06\\
      & $F_{7}$ & \textbf{1.74E+09} & 3.35E+06 & 1.90E+06 & 2.55E+06 & 9.96E+06 & 2.14E+06& 2.41E+06&2.87E+06\\
      & $F_{8}$ & \textbf{1.72E+09} & 3.34E+06 & 1.83E+06 & 2.53E+06 & 9.60E+06 & 2.17E+06&2.34E+06&2.74E+06 \\
      & $F_{9}$ & \textbf{1.78E+09} & 3.35E+06 & 1.84E+06 & 2.52E+06 & 9.69E+06 & 2.18E+06&2.44E+06 &2.82E+06\\
      & $F_{10}$ & \textbf{1.44E+09} & 3.32E+06 & 1.83E+06 & 2.46E+06 & 9.00E+06 & 2.12E+06& 2.30E+06&2.70E+06\\
      & $F_{11}$ & \textbf{1.46E+09} & 3.55E+06 & 1.88E+06 & 2.34E+06 & 9.66E+06 & 2.13E+06&2.34E+06& 2.63E+06\\
      & $F_{12}$ & \textbf{1.43E+09} & 3.46E+06 & 1.83E+06 & 2.41E+06 & 9.51E+06 & 2.09E+06&2.32E+06 &2.67E+06\\
\midrule
\midrule
\multirow{12}[2]{*}{20D} & $F_{1}$ & \textbf{1.66E+09} & 4.32E+06 & 1.92E+06 & 2.46E+06 & 1.13E+07 & 2.21E+06&2.68E+06&2.80E+06 \\
      & $F_{2}$ & \textbf{1.66E+09} & 3.91E+06 & 1.88E+06 & 2.37E+06 & 1.17E+07 & 2.09E+06& 2.66E+06&2.74E+06\\
      & $F_{3}$ & \textbf{1.18E+09} & 3.76E+06 & 1.77E+06 & 2.33E+06 & 9.75E+06 & 1.95E+06&2.62E+06& 2.64E+06\\
      & $F_{4}$ & \textbf{1.65E+09} & 3.50E+06 & 1.87E+06 & 2.28E+06 & 1.07E+07 & 2.00E+06&2.63E+06 &2.80E+06\\
      & $F_{5}$ & \textbf{1.64E+09} & 3.57E+06 & 1.86E+06 & 2.32E+06 & 1.07E+07 & 2.05E+06& 2.55E+06&2.74E+06\\
      & $F_{6}$ & \textbf{1.64E+09} & 3.89E+06 & 1.90E+06 & 2.34E+06 & 1.16E+07 & 2.09E+06& 2.62E+06&2.80E+06\\
      & $F_{7}$ & \textbf{1.45E+09} & 4.15E+06 & 1.84E+06 & 2.36E+06 & 1.01E+07 & 1.99E+06&2.32E+06& 2.80E+06\\
      & $F_{8}$ & \textbf{1.44E+09} & 3.42E+06 & 1.82E+06 & 2.31E+06 & 9.51E+06 & 2.12E+06&2.18E+06&2.30E+06 \\
      & $F_{9}$ & \textbf{1.57E+09} & 3.30E+06 & 1.77E+06 & 2.33E+06 & 9.48E+06 & 2.03E+06&2.59E+06&2.75E+06 \\
      & $F_{10}$ & \textbf{9.80E+08} & 3.67E+06 & 1.82E+06 & 2.43E+06 & 1.01E+07 & 2.08E+06& 2.09E+06&2.35E+06\\
      & $F_{11}$ & \textbf{1.00E+09} & 3.51E+06 & 1.95E+06 & 2.41E+06 & 9.81E+06 & 2.07E+06&2.17E+06& 2.37E+06\\
      & $F_{12}$ & \textbf{9.90E+08} & 3.44E+06 & 1.85E+06 & 2.37E+06 & 9.51E+06 & 2.07E+06& 2.17E+06&2.39E+06\\
\bottomrule
\end{tabular}%
}
  \label{tab:FEs}%
\end{table}%

\subsubsection{Performance under Equal FEs}\label{sec:expereiment_FEs}

% Table generated by Excel2LaTeX from sheet 'Sheet1'
\begin{table*}[htbp]
  %\centering
\caption{Comparisons between MetaDE and other DE variants under equal FEs. 
The mean and standard deviation (in parentheses) of the results over multiple runs are displayed in pairs. 
Results with the best mean values are highlighted.}
  \resizebox{\linewidth}{!}{
  \renewcommand{\arraystretch}{1.2}
 \renewcommand{\tabcolsep}{2pt}
% Table generated by Excel2LaTeX from sheet 'Exp3 same FEs'
\begin{tabular}{cccccccccc}
\toprule
\multicolumn{2}{c}{Func} & MetaDE & DE    & SaDE  & JaDE  & CoDE  & SHADE & LSHADE-RSP & EVDE \\
\midrule
\multirow{3}[2]{*}{10D} & $F_{2}$ & \textbf{0.00E+00 (0.00E+00)} & 4.52E+00 (2.36E+00)$-$ & 6.85E+00 (3.52E+00)$-$ & 6.31E+00 (3.05E+00)$-$ & 5.78E+00 (2.37E+00)$-$ & 4.33E+00 (3.81E+00)$-$ & 2.35E+00 (3.44E+00)$-$ & 5.86E+00 (2.99E+00)$-$ \\
      & $F_{6}$ & \textbf{5.50E-04 (3.96E-04)} & 1.13E-01 (7.83E-02)$-$ & 3.54E+01 (1.11E+02)$-$ & 2.02E+00 (3.34E+00)$-$ & 6.96E-03 (5.99E-03)$-$ & 9.27E-01 (1.31E+00)$-$ & 3.10E-02 (4.83E-02)$-$ & 1.46E+00 (2.76E+00)$-$ \\
      & $F_{10}$ & \textbf{0.00E+00 (0.00E+00)} & 1.00E+02 (4.40E-02)$-$ & 1.00E+02 (6.10E-02)$-$ & 1.21E+02 (4.35E+01)$-$ & 1.00E+02 (6.88E-02)$-$ & 1.29E+02 (4.67E+01)$-$ & 1.09E+02 (2.87E+01)$-$ & 1.10E+02 (3.05E+01)$-$ \\
\midrule
\midrule
\multirow{3}[2]{*}{20D} & $F_{2}$ & \textbf{1.26E-02 (3.74E-02)} & 4.72E+01 (2.09E+00)$-$ & 4.76E+01 (2.02E+00)$-$ & 1.34E+01 (2.19E+01)$-$ & 4.91E+01 (1.70E-06)$-$ & 4.91E+01 (3.40E-06)$-$ & 4.84E+01 (1.62E+00)$-$ & 4.47E+01 (1.41E+01)$-$ \\
      & $F_{6}$ & \textbf{1.16E-01 (2.79E-02)} & 7.28E-01 (5.22E-01)$-$ & 3.20E+01 (1.61E+01)$-$ & 4.90E+01 (3.31E+01)$-$ & 2.26E+01 (1.80E+01)$-$ & 5.67E+01 (3.90E+01)$-$ & 1.15E+01 (8.38E+00)$-$ & 2.92E+03 (5.82E+03)$-$ \\
      & $F_{10}$ & \textbf{0.00E+00 (0.00E+00)} & 1.07E+02 (2.07E+01)$-$ & 1.00E+02 (2.72E-02)$-$ & 1.01E+02 (3.64E-02)$-$ & 1.00E+02 (3.71E-02)$-$ & 1.43E+02 (5.64E+01)$-$ & 1.21E+02 (4.52E+01)$-$ & 1.14E+02 (4.65E+01)$-$ \\
\midrule
\multicolumn{2}{c}{$+$ / $\approx$ / $-$} & --    & 0/0/6 & 0/0/6 & 0/0/6 & 0/0/6 & 0/0/6 & 0/0/6 & 0/0/6 \\
\bottomrule
\end{tabular}%

}
\label{tab:sameFEs}%

\footnotesize
\textsuperscript{*} The Wilcoxon rank-sum tests (with a significance level of 0.05) were conducted between MetaDE and each algorithm individually.
The final row displays the number of problems where the corresponding algorithm performs statistically better ($+$),  similar ($\thickapprox$), or worse ($-$) compared to MetaDE.
\end{table*}%



In the preceding part, the performance benchmarking of MetaDE with other algorithms was anchored to equal wall-clock durations.
However, to ensure a comprehensive assessment, it is imperative to evaluate their performances under equivalent FEs.
In this part, we run each algorithm using the FEs achieved by MetaDE in \SI{60}{\second} (i.e., $1.84\times10^9/1.66\times10^9$, $1.84\times10^9/1.64\times10^9$, and $1.44\times10^9/9.8\times10^8$) on $F_2$, $F_6$, and $F_{10}$ for 10D/20D cases.
These selected functions collectively epitomize the basic, hybrid, and composition challenges within the CEC2022 benchmark suite.

As summarized in Table \ref{tab:sameFEs}, MetaDE consistently demonstrates the best performance, even when other algorithms are endowed with comparable FEs.
The reason can be traced to the inherent stagnation tendencies of other algorithms: after a certain point, additional FEs may not contribute to performance improvements.
This behavioral pattern is also lucidly captured in the convergence curves as presented in Figs. \ref{Figure_convergence_10D} and \ref{Figure_convergence_20D}.

Another noteworthy observation is the extended computation time required for a singular run of the comparison algorithms under these enhanced FEs, often extending to several hours or even transcending a day (e.g., running a single run of DE can take up to seven hours).
This elongated computational span can largely be attributed to their low concurrency, which struggles to benefit the parallelism of GPU computing.


\subsubsection{Performance with Large Populations}\label{sec:expereiment_large_pop}
Since a large population size could potentially increase the concurrency of fitness evaluations, for rigorousness, we further investigate the performance of the algorithms with large populations.

Specifically, MetaDE adopted the same population size setting as in previous experiments (i.e., 100 for both \texttt{evolver} and \texttt{executor}), while the population size of the other DE variants was increased to 1,000. 
This adjustment significantly enhances the concurrency of the other DE variants when utilizing GPU accelerations, thereby preventing insufficient convergence.


As evidenced in Figs. \ref{Figure_convergence_10D_NP10k}-\ref{Figure_convergence_20D_NP10k}, MetaDE still outperforms the other DE variants across all problems.
However, the performances of the other DE variants did not show significant improvements, which can be attributed to two factors.
First, since the conventional DE variants were not tailored for large populations, simply enlarging the populations may not help.
Second, since the sorting and archiving operations in some DE variants (e.g., SaDE) suffer from high computational complexities related to the population size, enlarging the populations brings additional computation overheads, thus limiting their performances under fixed wall-clock time.

By contrast, the large population in MetaDE is delicately organized in a \emph{hierarchical} manner: the \texttt{executor} maintains a population of moderate size, with each individual initializing an \texttt{executor} with a normal population.
This strategy not only capitalizes on the small-population advantage of conventional DE, but also benefits the concurrency brought by large populations.

\begin{figure}[!t]
\centering
\includegraphics[scale=0.3]{D10NP1000.pdf}
\caption{Convergence curves on 10D problems in CEC2022 benchmark suite. The peer DE variants are set with population size of 1,000.}
\label{Figure_convergence_10D_NP10k}
\end{figure}

\begin{figure}[!t]
\centering
\includegraphics[scale=0.3]{D20NP1000.pdf}
\caption{Convergence curves on 20D problems in CEC2022 benchmark suite. The peer DE variants are set with population size of 1,000.}
\label{Figure_convergence_20D_NP10k}
\end{figure}


{
\subsection{Comparisons with Top Algorithms in CEC2022 Competition}\label{section_Comparison with Top Algorithms of CEC Competition}

% Table generated by Excel2LaTeX from sheet 'Sheet1'
\begin{table*}[htbp]
  \centering
  
  \caption{{Comparisons between MetaDE and the top 4 algorithms from CEC2022 Competition (10D). 
The mean and standard deviation (in parentheses) of the results over multiple runs are displayed in pairs. 
Results with the best mean values are highlighted. }
  }
\footnotesize
% Table generated by Excel2LaTeX from sheet 'Exp 7 vs CECtop'
\begin{tabular}{cccccc}
\toprule
Func  & MetaDE & EA4eig & NL-SHADE-LBC & NL-SHADE-RSP & S-LSHADE-DP \\
\midrule
$F_{1}$ & \textbf{0.00E+00 (0.00E+00)} & \boldmath{}\textbf{0.00E+00 (0.00E+00)$\approx$}\unboldmath{} & \boldmath{}\textbf{0.00E+00 (0.00E+00)$\approx$}\unboldmath{} & \boldmath{}\textbf{0.00E+00 (0.00E+00)$\approx$}\unboldmath{} & \boldmath{}\textbf{0.00E+00 (0.00E+00)$\approx$}\unboldmath{} \\
$F_{2}$ & \textbf{0.00E+00 (0.00E+00)} & 7.97E-01 (1.78E+00)$-$ & 7.97E-01 (1.78E+00)$-$ & \boldmath{}\textbf{0.00E+00 (0.00E+00)$\approx$}\unboldmath{} & \boldmath{}\textbf{0.00E+00 (0.00E+00)$\approx$}\unboldmath{} \\
$F_{3}$ & \textbf{0.00E+00 (0.00E+00)} & \boldmath{}\textbf{0.00E+00 (0.00E+00)$\approx$}\unboldmath{} & \boldmath{}\textbf{0.00E+00 (0.00E+00)$\approx$}\unboldmath{} & \boldmath{}\textbf{0.00E+00 (0.00E+00)$\approx$}\unboldmath{} & \boldmath{}\textbf{0.00E+00 (0.00E+00)$\approx$}\unboldmath{} \\
$F_{4}$ & \textbf{0.00E+00 (0.00E+00)} & 9.95E-01 (1.22E+00)$-$ & 1.99E-01 (4.45E-01)$-$ & 2.98E+00 (1.15E+00)$-$ & \boldmath{}\textbf{0.00E+00 (0.00E+00)$\approx$}\unboldmath{} \\
$F_{5}$ & \textbf{0.00E+00 (0.00E+00)} & \boldmath{}\textbf{0.00E+00 (0.00E+00)$\approx$}\unboldmath{} & \boldmath{}\textbf{0.00E+00 (0.00E+00)$\approx$}\unboldmath{} & \boldmath{}\textbf{0.00E+00 (0.00E+00)$\approx$}\unboldmath{} & \boldmath{}\textbf{0.00E+00 (0.00E+00)$\approx$}\unboldmath{} \\
$F_{6}$ & 5.50E-04 (3.96E-04) & 7.53E-04 (5.52E-04)$\approx$ & 8.93E-02 (1.18E-01)$-$ & 4.37E-02 (5.41E-02)$-$ & \textbf{5.84E-05 (4.73E-05)$+$} \\
$F_{7}$ & \textbf{0.00E+00 (0.00E+00)} & \boldmath{}\textbf{0.00E+00 (0.00E+00)$\approx$}\unboldmath{} & \boldmath{}\textbf{0.00E+00 (0.00E+00)$\approx$}\unboldmath{} & \boldmath{}\textbf{0.00E+00 (0.00E+00)$\approx$}\unboldmath{} & \boldmath{}\textbf{0.00E+00 (0.00E+00)$\approx$}\unboldmath{} \\
$F_{8}$ & 5.52E-03 (4.41E-03) & 1.01E-04 (1.66E-04)$+$ & 3.96E-04 (4.23E-04)$+$ & 3.13E-01 (3.60E-01)$-$ & \textbf{1.26E-05 (1.56E-05)$+$} \\
$F_{9}$ & \textbf{3.36E+00 (1.77E+01)} & 1.86E+02 (0.00E+00)$-$ & 2.29E+02 (3.18E-14)$-$ & 8.03E+01 (1.08E+02)$-$ & 2.23E+02 (1.31E+01)$-$ \\
$F_{10}$ & \textbf{0.00E+00 (0.00E+00)} & 1.00E+02 (0.00E+00)$-$ & 1.00E+02 (0.00E+00)$-$ & 1.56E-02 (3.12E-02)$-$ & \boldmath{}\textbf{0.00E+00 (0.00E+00)$\approx$}\unboldmath{} \\
$F_{11}$ & \textbf{0.00E+00 (0.00E+00)} & \boldmath{}\textbf{0.00E+00 (0.00E+00)$\approx$}\unboldmath{} & \boldmath{}\textbf{0.00E+00 (0.00E+00)$\approx$}\unboldmath{} & \boldmath{}\textbf{0.00E+00 (0.00E+00)$\approx$}\unboldmath{} & \boldmath{}\textbf{0.00E+00 (0.00E+00)$\approx$}\unboldmath{} \\
$F_{12}$ & \textbf{1.39E+02 (4.63E+01)} & 1.48E+02 (5.98E+00)$-$ & 1.65E+02 (0.00E+00)$-$ & 1.62E+02 (2.15E+00)$-$ & 1.59E+02 (0.00E+00)$-$ \\
\midrule
$+$ / $\approx$ / $-$ & --    & 1/6/5 & 1/5/6 & 0/6/6 & 2/8/2 \\
\bottomrule
\end{tabular}%

\footnotesize
\textsuperscript{*} The Wilcoxon rank-sum tests (with a significance level of 0.05) were conducted between MetaDE and each algorithm individually.
The final row displays the number of problems where the corresponding algorithm performs statistically better ($+$),  similar ($\thickapprox$), or worse ($-$) compared to MetaDE.\\


\label{tab:vsCECTop 10D}%
\end{table*}%

% Table generated by Excel2LaTeX from sheet 'Sheet1'
\begin{table*}[htbp]
  \centering
  
  \caption{{Comparisons between MetaDE and the top 4 algorithms from CEC2022 Competition (20D). 
The mean and standard deviation (in parentheses) of the results over multiple runs are displayed in pairs. 
Results with the best mean values are highlighted.
  }
  }
  %\resizebox{\linewidth}{!}{
  %       \renewcommand{\arraystretch}{1}
 %\renewcommand{\tabcolsep}{3pt}
% Table generated by Excel2LaTeX from sheet 'Sheet1'
\footnotesize
% Table generated by Excel2LaTeX from sheet 'Exp 7 vs CECtop'
\begin{tabular}{cccccc}
\toprule
Func  & MetaDE & EA4eig & NL-SHADE-LBC & NL-SHADE-RSP & S-LSHADE-DP \\
\midrule
$F_{1}$ & \textbf{0.00E+00 (0.00E+00)} & \boldmath{}\textbf{0.00E+00 (0.00E+00)$\approx$}\unboldmath{} & \boldmath{}\textbf{0.00E+00 (0.00E+00)$\approx$}\unboldmath{} & \boldmath{}\textbf{0.00E+00 (0.00E+00)$\approx$}\unboldmath{} & \boldmath{}\textbf{0.00E+00 (0.00E+00)$\approx$}\unboldmath{} \\
$F_{2}$ & 3.83E-04 (2.10E-03) & \textbf{0.00E+00 (0.00E+00)$+$} & 4.91E+01 (0.00E+00)$-$ & \textbf{0.00E+00 (0.00E+00)$+$} & \textbf{0.00E+00 (0.00E+00)$+$} \\
$F_{3}$ & \textbf{0.00E+00 (0.00E+00)} & \boldmath{}\textbf{0.00E+00 (0.00E+00)$\approx$}\unboldmath{} & \boldmath{}\textbf{0.00E+00 (0.00E+00)$\approx$}\unboldmath{} & \boldmath{}\textbf{0.00E+00 (0.00E+00)$\approx$}\unboldmath{} & \boldmath{}\textbf{0.00E+00 (0.00E+00)$\approx$}\unboldmath{} \\
$F_{4}$ & 1.96E+00 (7.76E-01) & 7.36E+00 (2.06E+00)$-$ & \boldmath{}\textbf{1.59E+00 (5.45E-01)$\approx$}\unboldmath{} & 1.07E+02 (1.54E+02)$-$ & 3.20E+00 (1.94E+00)$-$ \\
$F_{5}$ & \textbf{0.00E+00 (0.00E+00)} & \boldmath{}\textbf{0.00E+00 (0.00E+00)$\approx$}\unboldmath{} & \boldmath{}\textbf{0.00E+00 (0.00E+00)$\approx$}\unboldmath{} & 2.27E-01 (4.54E-01)$-$ & \boldmath{}\textbf{0.00E+00 (0.00E+00)$\approx$}\unboldmath{} \\
$F_{6}$ & \textbf{1.38E-01 (5.56E-02)} & 2.54E-01 (4.28E-01)$-$ & 3.06E-01 (2.01E-01)$-$ & 2.08E-01 (9.78E-02)$-$ & 5.02E-01 (5.34E-01)$-$ \\
$F_{7}$ & 8.42E-02 (1.01E-01) & 1.37E+00 (1.10E+00)$-$ & \boldmath{}\textbf{6.24E-02 (1.40E-01)$\approx$}\unboldmath{} & 1.28E+00 (1.95E+00)$-$ & 9.83E-01 (8.12E-01)$-$ \\
$F_{8}$ & 2.66E+00 (3.83E+00) & 2.02E+01 (1.28E-01)$-$ & \textbf{1.01E-01 (1.41E-01)$+$} & 1.99E+01 (4.97E-01)$-$ & 2.30E-01 (1.82E-01)$+$ \\
$F_{9}$ & \textbf{1.32E+02 (3.43E+01)} & 1.65E+02 (0.00E+00)$-$ & 1.81E+02 (0.00E+00)$-$ & 1.81E+02 (0.00E+00)$-$ & 1.81E+02 (0.00E+00)$-$ \\
$F_{10}$ & \textbf{0.00E+00 (0.00E+00)} & 1.23E+02 (5.12E+01)$-$ & 1.00E+02 (9.27E-03)$-$ & \boldmath{}\textbf{0.00E+00 (0.00E+00)$\approx$}\unboldmath{} & \boldmath{}\textbf{0.00E+00 (0.00E+00)$\approx$}\unboldmath{} \\
$F_{11}$ & 1.74E-03 (7.97E-03) & 3.20E+02 (4.47E+01)$-$ & 3.00E+02 (0.00E+00)$-$ & \textbf{0.00E+00 (0.00E+00)$+$} & \textbf{0.00E+00 (0.00E+00)$+$} \\
$F_{12}$ & 2.29E+02 (9.70E-01) & \textbf{2.00E+02 (2.04E-04)$+$} & 2.37E+02 (3.17E+00)$-$ & 2.34E+02 (1.46E+00)$-$ & 2.34E+02 (4.51E+00)$-$ \\
\midrule
$+$ / $\approx$ / $-$ & --    & 2/3/7 & 1/5/6 & 2/3/7 & 3/4/5 \\
\bottomrule
\end{tabular}%

\footnotesize
\textsuperscript{*} The Wilcoxon rank-sum tests (with a significance level of 0.05) were conducted between MetaDE and each algorithm individually.
The final row displays the number of problems where the corresponding algorithm performs statistically better ($+$),  similar ($\thickapprox$), or worse ($-$) compared to MetaDE.\\

\label{tab:vsCECTop 20D}%
\end{table*}%



To further assess the performance of MetaDE, we compare it with the top 4 algorithms from the CEC2022 Competition on Single Objective Bound Constrained Numerical Optimization\footnote{\url{https://github.com/P-N-Suganthan/2022-SO-BO}}.
For each algorithm, we set equal FEs as achieved by MetaDE within 60 seconds (refer to Table~\ref{tab:FEs} for details).

The top 4 algorithms from the CEC2022 Competition are {EA4eig}~\cite{EA4eig}, {NL-SHADE-LBC}~\cite{NL-SHADE-LBC}, {NL-SHADE-RSP-MID}~\cite{NL-SHADE-RSP}, and {S-LSHADE-DP}~\cite{S_LSHADE_DP}:
\begin{itemize}
  \item {EA4eig} combines the strengths of four evolutionary algorithms (CMA-ES, CoBiDE, an adaptive variant of jSO, and IDE) using Eigen crossover.
  \item {NL-SHADE-LBC} is a dynamic DE variant that integrates linear bias changes for parameter adaptation, repeated point generation to handle boundary constraints, non-linear population size reduction, and a selective pressure mechanism.
  \item {NL-SHADE-RSP-MID} is an advanced version of NL-SHADE-RSP, which estimates the optimum using the population midpoint, incorporates a restart mechanism, and improves boundary constraint handling.
  \item {S-LSHADE-DP} focuses on maintaining population diversity through dynamic perturbation, adjusting noise intensity to enhance exploration.
\end{itemize}





The experimental results are summarized in Tables \ref{tab:vsCECTop 10D} and \ref{tab:vsCECTop 20D}.
On 10D problems, MetaDE outperforms EA4eig, NL-SHADE-LBC, and NL-SHADE-RSP, while achieving comparable performance to S-LSHADE-DP. 
On 20D problems, MetaDE consistently outperforms the four algorithms.
An additional noteworthy observation is that S-LSHADE-DP exhibits promising performance under a large number of FEs.
}

\subsection{Investigation of Optimal DE Variants}\label{section_Optimal Parameter Analysis}


\begin{table}[h]
  \centering
  \caption{Optimal DE variants obtained by MetaDE on each problem of the CEC2022 benchmark suite. FDC and RIE are two fitness landscape characteristics that measure the difficulty and ruggedness of the problem.}
% Table generated by Excel2LaTeX from sheet 'Exp4 param'
\resizebox{\columnwidth}{!}{
\begin{tabular}{cccccccc}
\toprule
\multicolumn{2}{c}{Problem} & F     & CR    & \multicolumn{2}{c}{Strategy} & FDC & RIE \\
\midrule
\multirow{4}[2]{*}{10D} & $F_{6}$ & 0.70  & 0.99  & \multicolumn{2}{c}{\texttt{rand-to-pbest/1/arith}} & 0.61  & 0.81  \\
      & $F_{8}$ & 0.51  & 0.44  & \multicolumn{2}{c}{\texttt{pbest-to-best/1/bin}} & 0.27  & 0.62  \\
      & $F_{9}$ & 0.02  & 0.03  & \multicolumn{2}{c}{\texttt{current/2/bin}} & 0.08  & 0.82  \\
      & $F_{12}$ & 0.16  & 0.00  & \multicolumn{2}{c}{\texttt{current-to-best/4/bin}} & -0.15  & 0.78  \\
\midrule
\multirow{6}[2]{*}{20D} & $F_{4}$ & 0.13  & 0.71  & \multicolumn{2}{c}{\texttt{rand-to-best/3/bin}} & 0.90  & 0.79  \\
      & $F_{6}$ & 0.67  & 0.99  & \multicolumn{2}{c}{\texttt{pbest-to-rand/1/bin}} & 0.48  & 0.80  \\
      & $F_{7}$ & 0.27  & 0.93  & \multicolumn{2}{c}{\texttt{rand/2/bin}} & 0.26  & 0.78  \\
      & $F_{8}$ & 0.65  & 0.00  & \multicolumn{2}{c}{\texttt{pbest/1/exp}} & 0.12  & 0.40  \\
      & $F_{9}$ & 0.06  & 0.00  & \multicolumn{2}{c}{\texttt{current/2/bin}} & -0.17  & 0.84  \\
      & $F_{12}$ & 0.33  & 0.44  & \multicolumn{2}{c}{\texttt{rand-to-best/2/bin}} & -0.16  & 0.85  \\
\bottomrule
\end{tabular}%
}
\label{tab:optimal_param}
\end{table}%


This part provides an in-depth examination of the optimal DE variants obtained by MetaDE in Section~\ref{sec:expereiment_time}, as summarized in Table \ref{tab:optimal_param}.
The optimal parameters correspond to the best individual in the final population of MetaDE.
The table only displays the optimal parameters for the ten listed problems, as the remaining problems are relatively simpler, with numerous DE variants capable of locating the optimal solutions of the problems. Furthermore, the optimal parameters presented in the table represent the best results of MetaDE derived from the finest run out of 31 independent trials.



All the problems in the table are characterized by both multimodality and non-separability.
Additionally, to further depict the characteristics of the problems' fitness landscapes, we computed both the fitness distance correlation (FDC) \cite{FDC} and the ruggedness of information entropy (RIE) \cite{RIE}; the former measures the complexity (difficulty) of the problems, while the latter characterizes the ruggedness of the landscape.

Analyzing the obtained data, it is evident that no single set of parameters or strategies consistently excels across all problems.
Parameters such as \(F\) and \(CR\) exhibit variability across problems without adhering to a specific trend.
Similarly, the selection of base vectors ($bl$ and $br$) does not show a uniform preference either.
Regarding the fitness landscape characteristics of each problem, the selection of parameters exhibits distinct patterns.
The FDC indicates problem complexity; with simpler problems (higher FDC), such as 10-dimensional $F_6$, $F_8$ and 20-dimensional $F_4$, $F_6$, $F_7$, a larger \(CR\) value is favored. Conversely, smaller \(CR\) values are chosen for problems with lower FDC. A larger \(CR\) tends to facilitate convergence, whereas a \(CR\) close to 0 leads to offspring that change incrementally, dimension by dimension. However, the other characteristic, RIE, does not seem to have a clear association with parameter choices.
The optimal strategies for identical problems across different dimensions exhibit closeness, with $F_8$, $F_9$, and $F_{12}$ demonstrating notably parallel strategies between their 10D and 20D problems.
In terms of crossover strategies ($cs$), it seems to have a preference for binomial crossover. This aligns with the traditional DE configurations.

These observations align with the No Free Lunch (NFL) theorem \cite{NFL}, thus underscoring the importance of distinct optimization strategies tailored for diverse problems.
Conventionally, the optimization strategies have oscillated between seeking a generalist set of parameters for broad applicability and a specialist set tailored for specific problems. However, the dynamic nature of optimization problems, where even minute changes like a different random seed can pivot the problem's dynamics, highlights the challenges of a generalist approach.
In contrast, MetaDE provides a simple yet effective approach, showing promising generality and adaptability.


\subsection{Application to Robot Control}\label{sec:expereiment_brax}
In this experiment, we demonstrate the extended application of MetaDE to robot control.
Specifically, we adopted the evolutionary reinforcement learning paradigm~\cite{ERL} as illustrated in Fig.~\ref{Figure_EvoRL}.
The experiment was conducted on Brax \cite{brax} for robotics simulations with GPU acceleration.

\begin{figure}[!h]
\centering
\includegraphics[scale=0.38]{EvoRL_Workflow.pdf}
\caption{Illustration of robot control via evolutionary reinforcement learning. The evolutionary algorithm optimizes the parameters of a population of candidate policy models for controlling the robotics behaviors. The simulation environment returns rewards achieved by the candidate policy models to the evolutionary algorithm as fitness values.}
\label{Figure_EvoRL}
\end{figure}

This experiment involved three robot control tasks: ``swimmer'', ``hopper'', and ``reacher''.
As summarized in Table \ref{tab:Neural network structures}, we adopted similar policy models for these three tasks, each consisting of a multilayer perceptron (MLP) with three fully connected layers, but with different input and output dimensions.
Consequently, the three policy models comprise 1410, 1539, and 1506 parameters for optimization respectively, where the optimization objective is to achieve maximum reward of each task.
MetaDE, vanilla DE \cite{DE1996}, SHADE~\cite{SHADE2013}, LSHADE-RSP~\cite{LSHADE-RSP2018}, EDEV~\cite{EDEV2018}, CSO \cite{CSO}\footnote{The competitive swarm optimizer (CSO) is a tailored PSO variant for large-scale optimization.}, and CMA-ES~\cite{CMAES} were applied as the optimizer respectively.

%The policy models  were initialized with identical random parameters.
The iteration count for PDE within MetaDE was set to 50, while other algorithms maintained a population size of 100.
Each algorithm was run independently 15 times.
Considering the time-intensive nature of the robotics simulations, we set 60 minutes as the termination condition for each run.


\begin{table}[htbp]
\centering
\caption{Neural network structure of the policy model for each robot control task}
\label{tab:Neural network structures}
\resizebox{\columnwidth}{!}{%
% Table generated by Excel2LaTeX from sheet 'Exp6 brax'
\begin{tabular}{cccccc}
\toprule
\textbf{Task} & \textbf{D} & \textbf{Input} & \textbf{Hidden Layers} &   \textbf{Output}    & \textbf{Overview of objectives} \\
\midrule
Hopper & 1539  & 11    & 32$\times$32 & 3     & balance and jump \\
Swimmer & 1410  & 8     & 32$\times$32 & 2     & maximizing movement \\
Reacher & 1506  & 11    & 32$\times$32 & 2     & precise reaching \\
\bottomrule
\end{tabular}%
}
\end{table}

\begin{figure}[!h]
\centering
\includegraphics[scale=0.3]{brax_all_convergence.pdf}
\caption{The reward curves achieved by MetaDE and peer evolutionary algorithms when applied to each robot control task. }
\label{Figure_brax_all_convergence}
\end{figure}

\begin{figure}[!h]
\centering
\includegraphics[width=\linewidth]{brax_distribution.pdf}
\caption{The fitness distribution of MetaDE's initial population when applied to each robot control task.}
\label{Figure_brax_distribution}
\end{figure}

As shown in Fig. \ref{Figure_brax_all_convergence}, it is evident that MetaDE achieves the best performance in the Swimmer tasks, while slightly outperformed by CMA-ES and CSO in the Hopper and Reacher task.
An interesting observation from the reward curves is that MetaDE almost reaches optimality nearly at the first generation and does not show further significant improvements thereafter.
To elucidate this phenomenon, Fig.~\ref{Figure_brax_distribution} provides the fitness distribution of MetaDE's initial population, indicating that MetaDE harbored several individuals with considerably high fitness from the initial generation.
In other words, MetaDE was able to generate high-performance DE variants for these problems even by random sampling.
This can be attributed to the unique nature of neural network optimization.
As widely acknowledged, the neural network optimization typically features numerous plateaus in the fitness landscape, thus making it relatively easy to find one of the local optima.
MetaDE provides unbiased sampling of parameter settings for generating diverse DE variants.
Even without further evolution, some of the randomly sampled DE variants are very likely to reach the plateaus.
In contrast, the other algorithms are specially tailored with biases; in such large-scale optimization scenarios, the biases can be further amplified, thus making them ineffective.




\section{Conclusion}\label{section Conclusion}
In this paper, we introduced MetaDE, a method that leverages the strengths of DE not only to address optimization tasks but also to adapt and refine its own strategies. This meta-evolutionary approach demonstrates how DE can autonomously evolve its parameter configurations and strategies. 
Our experiments demonstrate that MetaDE has robust performance across various benchmarks, as well as the application in robot control through evolutionary reinforcement learning. 
Nevertheless, the study also emphasizes the complexity of finding universally optimal parameter configurations. The intricate balance between generalization and specialization remains a challenge, and MetaDE has shed light on further research into self-adapting algorithms. 
We anticipate that the insights gained from this work will inspire the development of more advanced meta-evolutionary approaches, pushing the boundaries of evolutionary optimization in even more complex and dynamic environments.





\footnotesize

% \bibliography{manuscript_references}
% !TEX program = pdflatex

\documentclass[journal]{IEEEtran}
\usepackage{lineno}
\modulolinenumbers[5]

%%% color some references
\usepackage{xpatch}
\makeatletter

\makeatother
\usepackage{bm}
\usepackage{array}
\usepackage{graphicx}
\usepackage{amsmath,amssymb,amsthm}
\usepackage{siunitx}
\usepackage{algpseudocode}
\usepackage{algorithmicx}
\usepackage{algorithm}
\usepackage{booktabs}
\usepackage{color}
\usepackage{changepage}
\usepackage{xr}
\usepackage{xr-hyper}
%\usepackage{geometry}%页面设置
\usepackage{graphicx}%图片设置
%\usepackage{subfig}%多个子图
\usepackage{subfigure}
\usepackage{caption}%注释设置
\usepackage{multirow}
\usepackage{float}
\usepackage{soul}
\usepackage[hidelinks]{hyperref}
\usepackage[numbers,sort&compress]{natbib}
\usepackage{bigstrut} %表格大竖线
\usepackage[table]{xcolor} %表格单元格颜色
\usepackage{enumitem} %enumerate标签样式\usepackage{listings}
\usepackage{listings} %listing代码
\usepackage[resetlabels]{multibib}
\newcites{supp}{Supplement References}



\definecolor{dkgreen}{rgb}{0,0.5,0}
\definecolor{gray}{rgb}{0.5,0.5,0.5}
\definecolor{mauve}{rgb}{0.58,0,0.82}
\definecolor{dkblue}{rgb}{0,0,0.6}

 % English theorem environment
 \newtheorem{theorem}{Theorem}
 \newtheorem{lemma}[theorem]{Lemma}
 \newtheorem{proposition}[theorem]{Proposition}
 \newtheorem*{corollary}{Corollary of Theorems 1 and 2}
 \newtheorem{definition}{Definition}
 \newtheorem{remark}{Remark}
 \newtheorem{example}{Example}
 \newenvironment{solution}{\begin{proof}[Solution]}{\end{proof}}

\renewcommand{\algorithmicrequire}{\textbf{Input:}}
\renewcommand{\algorithmicensure}{\textbf{Output:}}

\AtBeginDocument{%
 \abovedisplayskip=5pt plus 4pt minus 2pt
 \abovedisplayshortskip=5pt plus 4pt minus 4pt
 \belowdisplayskip=5pt plus 4pt minus 2pt
 \belowdisplayshortskip=5pt plus 4pt minus 4pt
}

\ifCLASSINFOpdf
\else
\fi

\hyphenation{op-tical net-works semi-conduc-tor}

\bibliographystyle{IEEEtran}

\begin{document}
\captionsetup{font={footnotesize}}
\captionsetup[table]{labelformat=simple, labelsep=newline, textfont=sc, justification=centering}
% paper title
% Titles are generally capitalized except for words such as a, an, and, as,
% at, but, by, for, in, nor, of, on, or, the, to and up, which are usually
% not capitalized unless they are the first or last word of the title.
% Linebreaks \\ can be used within to get better formatting as desired.
% Do not put math or special symbols in the title.
\title{MetaDE: Evolving Differential Evolution by Differential Evolution}
%
%
% author names and IEEE memberships
% note positions of commas and nonbreaking spaces ( ~ ) LaTeX will not break
% a structure at a ~ so this keeps an author's name from being broken across
% two lines.
% use \thanks{} to gain access to the first footnote area
% a separate \thanks must be used for each paragraph as LaTeX2e's \thanks
% was not built to handle multiple paragraphs
%

\author{Minyang Chen, Chenchen Feng,
        and Ran Cheng

        \thanks{
        Minyang Chen was with the Department of Computer Science and Engineering, Southern University of Science and Technology, Shenzhen 518055, China. E-mail: cmy1223605455@gmail.com. }
        \thanks{
        Chenchen Feng is with the Department of Computer Science and Engineering, Southern University of Science and Technology, Shenzhen 518055, China. E-mail: chenchenfengcn@gmail.com. 
        }
        \thanks{
       Ran Cheng is with the Department of Data Science and Artificial Intelligence, and the Department of Computing, The Hong Kong Polytechnic University, Hong Kong SAR, China. E-mail: ranchengcn@gmail.com. (\emph{Corresponding author: Ran Cheng})
        }
        }% <-this % stops a space


% The paper headers
\markboth{Bare Demo of IEEEtran.cls for IEEE Journals}%
{Shell \MakeLowercase{\textit{et al.}}: Bare Demo of IEEEtran.cls for IEEE Journals}
% The only time the second header will appear is for the odd numbered pages
% after the title page when using the twoside option.

% *** Note that you probably will NOT want to include the author's ***
% *** name in the headers of peer review papers.                   ***
% You can use \ifCLASSOPTIONpeerreview for conditional compilation here if
% you desire.


% If you want to put a publisher's ID mark on the page you can do it like
% this:
%\IEEEpubid{0000--0000/00\$00.00~\copyright~2015 IEEE}
% Remember, if you use this you must call \IEEEpubidadjcol in the second
% column for its text to clear the IEEEpubid mark.



% use for special paper notices
%\IEEEspecialpapernotice{(Invited Paper)}

% make the title area
\maketitle

% As a general rule, do not put math, special symbols or citations
% in the abstract or keywords.
\begin{abstract}
As a cornerstone in the Evolutionary Computation (EC) domain, Differential Evolution (DE) is known for its simplicity and effectiveness in handling challenging black-box optimization problems.
While the advantages of DE are well-recognized, achieving peak performance heavily depends on its hyperparameters such as the mutation factor, crossover probability, and the selection of specific DE strategies.
Traditional approaches to this hyperparameter dilemma have leaned towards parameter tuning or adaptive mechanisms.
However, identifying the optimal settings tailored for specific problems remains a persistent challenge.
In response, we introduce MetaDE, an approach that evolves DE's intrinsic hyperparameters and strategies using DE itself at a meta-level.
A pivotal aspect of MetaDE is a specialized parameterization technique, which endows it with the capability to dynamically modify DE's parameters and strategies throughout the evolutionary process.
To augment computational efficiency, MetaDE incorporates a design that leverages parallel processing through a GPU-accelerated computing framework.
Within such a framework, DE is not just a solver but also an optimizer for its own configurations, thus streamlining the process of hyperparameter optimization and problem-solving into a cohesive and automated workflow.
Extensive evaluations on the CEC2022 benchmark suite demonstrate MetaDE's promising performance.
Moreover, when applied to robot control via evolutionary reinforcement learning, MetaDE also demonstrates promising performance.
The source code of MetaDE is publicly accessible at: \url{https://github.com/EMI-Group/metade}.
\end{abstract}



% Note that keywords are not normally used for peerreview papers.
\begin{IEEEkeywords}
Differential Evolution, Meta Evolutionary Algorithm, GPU Computing
\end{IEEEkeywords}


% For peer review papers, you can put extra information on the cover
% page as needed:
% \ifCLASSOPTIONpeerreview
% \begin{center} \bfseries EDICS Category: 3-BBND \end{center}
% \fi
%
% For peerreview papers, this IEEEtran command inserts a page break and
% creates the second title. It will be ignored for other modes.
\IEEEpeerreviewmaketitle



\section{Introduction}
\IEEEPARstart{T}{he} Differential Evolution (DE) \cite{DE1996,DEcontest1996,DEusage1996,DE1997} algorithm, introduced by Storn and Price in 1995, has emerged as a cornerstone in the realm of evolutionary computation (EC) for its prowess in addressing complex optimization problems across diverse domains of science and engineering.
DE's comparative advantage over other evolutionary algorithms is evident in its streamlined design, robust performance, and ease of implementation.
Notably, with just three primary control parameters, i.e., scaling factor, crossover rate, and population size, DE operates efficiently.
This minimalistic design, paired with a lower algorithmic complexity, positions DE as an ideal candidate for large-scale optimization problems.
Its influential role in the optimization community is further cemented by its extensive research attention and successful applications over the past decades \cite{DEsurvey2011, DEsurvey2016, DEpapersurvey2020}, with DE and its derivatives often securing top positions in the IEEE Congress on Evolutionary Computation (CEC) competitions.



Despite the well recognized performance, DE is not without limitations.
Particularly, some studies indicate that DE's optimization process may stagnate if it fails to generate offspring solutions superior to their parents \cite{DEStagnation, neriDEsurvey}.
To avert this stagnation, selecting an appropriate parameter configuration to enhance DE's search capabilities becomes crucial.

However, the No Free Lunch (NFL) theorem \cite{NFL} suggests that a universally optimal parameter configuration is unattainable.
For example, while a higher mutation factor may aid in escaping local optima, a lower crossover probability might be preferable for problems with separability characteristics.

To address the intricate challenge of parameter configuration in DE, researchers often gravitate towards two predominant strategies: \emph{parameter control} and \emph{parameter tuning} \cite{param1999,paramTun2012,paramTun2020}.
Parameter control is a dynamic approach wherein the algorithm's parameters are adjusted on-the-fly during its execution.
This adaptability allows the algorithm to respond to the evolving characteristics of the problem landscape, enhancing its chance of finding optimal or near-optimal solutions.
Notably, DE has incorporated this strategy in several of its variants.
For instance, jDE \cite{jDE2006} adjusts the mutation factor and crossover rate during the run, while SaDE \cite{SaDE2008} dynamically chooses a mutation strategy based on its past success rates. Similarly, JaDE \cite{JADE2009} and CoDE \cite{CoDE2011} employ adaptive mechanisms to modify control parameters and mutation strategies, respectively.

In contrast, parameter tuning is a more static methodology, wherein the optimal configuration is established prior to the algorithm's initiation.
It aims to discover a parameter set that consistently demonstrates robust performance across various runs and problem instances.
Despite its potential for reliable outcomes, parameter tuning is known for its computational intensity, often necessitating dedicated optimization efforts or experimental designs to identify the optimal parameters, which may explain its limited exploration in the field.
Viewed as an optimization challenge, parameter tuning is also referred to as meta-optimization \cite{metaEAPhD2010}.
This perspective gave rise to \emph{MetaEA}, which optimizes the parameters of an EA using another EA.

Despite MetaEA's methodological elegance and simplicity, it confronts the significant challenge of depending on extensive function evaluations.
Fortunately, the inherent parallelism within MetaEA, across both meta-level and base-level populations, renders it particularly amenable to parallel computing environments.
However, a notable disparity exists between methodological innovations and the availability of advanced computational infrastructures, thus limiting MetaEA's potential due to the lack of advanced hardware accelerations such as GPUs.
To bridge this gap, we introduce the \emph{MetaDE} approach, which embodies the MetaEA paradigm by employing DE in a meta-level to guide the evolution of a specially tailored Parameterized Differential Evolution (PDE).

Designed with adaptability in mind, PDE can flexibly adjust its parameters and strategies, paving the way for a wide range of DE configurations.
As PDE interacts with the optimization problem at hand, the meta-level DE observes and refines PDE's settings to better align with the problem's characteristics.
Amplifying the efficiency of this nested optimization approach, MetaDE is integrated with a GPU-accelerated EC framework, thus weaving together parameter refinement and direct problem-solving into a seamless end-to-end approach to black-box optimization.
In summary, our main contributions are as follows.

\begin{itemize}
\item \textbf{Parameterized Differential Evolution:}
We have introduced Parameterized Differential Evolution (PDE), a variant of DE with augmented parameterization.
Unlike traditional DE algorithms that come with fixed mutation and crossover strategies, PDE’s architecture offers users the flexibility to adjust these parameters and strategies to fit the problem at hand.
This design not only allows for the creation of diverse DE configurations tailored for specific challenges but also ensures efficient computation.
To achieve this, all core operations of PDE, including mutation, crossover, and evaluation, have been optimized for parallel execution to harness advancement of GPU acceleration.

\item \textbf{MetaDE:}
Building on the MetaEA paradigm, we have designed the MetaDE approach.
Specifically, MetaDE employs a meta-level DE as an \texttt{evolver} to iteratively refine PDE's hyperparameters, which is guided by performance feedback from multiple PDE instances acting as \texttt{executors}.
This continuous optimization ensures PDE's configurations remain aligned with evolving problem landscapes.
Moreover, we have incorporated several specialized methods to further enhance the performance of MetaDE.



\item \textbf{GPU-accelerated Implementation:}
Breaking away from the limitations of conventional parameter tuning, we integrate MetaDE with a GPU-accelerated computing framework
 -- EvoX~\cite{evox}, which enhances MetaDE's computational prowess for facilitating swifter evaluations and algorithmic refinements.
With this specialized implementation, MetaDE provides an efficient and automated end-to-end approach to black-box optimization.
\end{itemize}



The subsequent sections are organized as follows. Section \ref{section_Preliminary} presents some preliminary knowledge for this work.
Section \ref{section_The proposed metade} elucidates the intricacies of the proposed approach, including PDE and the MetaDE.
Section \ref{section_Experimental_study} showcases the experimental results.
Finally, Section \ref{section Conclusion} wraps up the discourse and points towards avenues for future work.

\section{Preliminaries}\label{section_Preliminary}

\subsection{DE and its Parameter Adaption}\label{subsection_DE and Parameter adaption}
\subsubsection{Overview of DE}
As a typical EC algorithm, DE's essence lies in its differential mutation mechanism that drives the evolution of a population.
The operational cycle of DE unfolds iteratively, with each iteration embodying specific phases, as elaborated in Algorithm \ref{Alg_DE}:
\begin{enumerate}[label=\arabic*.]
  \item \textbf{Initialization} (Line 1):
  The algorithm initializes a set of potential solutions.
  Each of these solutions, representing vectors of decision variables, is randomly generated within the search space boundaries.
  \item \textbf{Mutation} (Lines 6-7):
  Each solution undergoes mutation to produce a mutant vector.
  This mutation process involves combinations of different individuals to form the mutant vectors.
  \item \textbf{Crossover} (Line 8):
  The crossover operation interchanges components between mutants and the original solutions to generate a trial vector.
  \item \textbf{Selection} (Lines 10-12): The trial vector competes against the original solution based on fitness, with the better solution progressing to the next generation.
\end{enumerate}


DE progresses through cycles of mutation, crossover, and selection, persisting until it encounters a termination criterion.
This could manifest as either reaching a predefined number of generations or achieving a target fitness threshold.
The algorithm's adaptability allows for the spawning of myriad DE variants by merely tweaking its mutation and crossover operations.
Specifically, DE variants follow a unified naming convention: \texttt{DE/x/y/z}, where \texttt{x} identifies the base vector used for mutation, \texttt{y} quantifies the number of difference involved, and \texttt{z} typifies the crossover method employed.
For example, the DE variant as presented in Algorithm \ref{Alg_DE} is named as \texttt{DE/rand/1/bin}.

{\linespread{1.1}
\begin{algorithm}
\small
\caption{DE}\label{Alg_DE}
\begin{algorithmic}[1]
  \Require {$D$, $NP$, $F$, $CR$, $G_{max}$}
  \State Initialize population $\mathbf{X} = \{\mathbf{x}_1, \mathbf{x}_2, \dots, \mathbf{x}_{\scalebox{0.5}{$\textit{NP}$}}\}$
  \State Evaluate the fitness of each individual in the population
  \State $g = 0$
  \While{$g \leq G_{max}$}
    \For{$i = 1$ to $NP$}
      \State Randomly select $\mathbf{x}_{r_1}$, $\mathbf{x}_{r_2}$, and $\mathbf{x}_{r_3}$ from $\mathbf{X}$,
      \Statex \qquad \quad such that $r_1 \neq r_2 \neq r_3 \neq i$
      \State Compute the mutant vector: $\mathbf{v}_i = \mathbf{x}_{r_1} + F \cdot (\mathbf{x}_{r_2} - \mathbf{x}_{r_3})$
      \State Perform crossover for each variable between $\mathbf{x}_i$ and $\mathbf{v}_i$:
      \begin{align*}
        \qquad \quad u_{i,j}=\begin{cases}
          v_{i,j},\ \text{if } \text{rand}(0, 1) \leq CR \text{ or } j = \text{randint}(1, D) \\
          x_{i,j},\ \text{otherwise}
        \end{cases}
      \end{align*}
      \State Evaluate the fitness of $\mathbf{u}_i$
      \If{$\textrm{f}(\mathbf{u}_i) \leq \textrm{f}(\mathbf{x}_i)$}
        \State Replace $\mathbf{x}_i$ with $\mathbf{u}_i$ in the population
      \EndIf
    \EndFor
    \State $g = g + 1$
  \EndWhile
  \State\Return the best fitness
\end{algorithmic}
\end{algorithm}
{\linespread{1}

\subsubsection{Parameter Modulation in DE}
DE employs a unique mutation mechanism, which adapts to the problem's natural scaling.
By adjusting the mutation step's size and orientation to the objective function landscape, DE embraces the \emph{contour matching principle} \cite{DEbook2006}, which promotes basin-to-basin transfer for enhancing the convergence of the algorithm.

At the core of DE's mutation is the scaling factor \( F \).
This factor not only determines the mutation's intensity but also governs its trajectory and ability to bypass local optima.
Commonly, \( F \) is set within the $[0.5, 1]$ interval, with a starting point often at 0.5.
While values outside the $[0.4, 1]$ range can sometimes yield good results, an \( F \) greater than 1 tends to slow convergence.
Conversely, values up to 1 generally promise swifter and more stable outcomes \cite{EPSDE2011}.
Nonetheless, to deter settling at suboptimal solutions too early, \( F \) should be adequately elevated.

Parallel to mutation, DE incorporates a uniform crossover operator, which is often labeled as discrete recombination or binomial crossover in the GA lexicon.
The crossover constant \( CR \) also plays a pivotal role, which determines the proportion of decision variables to be exchanged during the generation of offspring.
A low value for \( CR \) ensures only a small portion of decision variables are modified per iteration, thus leading to axis-aligned search steps.
As \( CR \) increases closer to 1, offspring tend to increasingly reflect their mutant parent, thereby curbing the generation of orthogonal search steps \cite{DEsurvey2011}.

For classical DE configurations, such as \texttt{DE/rand/1/bin}, rotational invariance is achieved only when \( CR \) is maxed out at 1.
Here, the crossover becomes wholly vector-driven, and offspring effectively mirror their mutants.
However, the optimal \( CR \) is intrinsically problem-dependent.
Empirical studies recommend a \( CR \) setting within the $[0, 0.2]$ range for problems characterized by separable decision variables.
Conversely, for problems with non-separable decision variables, a \( CR \) in the proximity of $[0.9, 1]$ is more effective \cite{DEsurvey2011}.

The adaptability of DE is evident in its wide spectrum of variants, each distinct in its mutation and crossover strategies with delicate parameter modulations.
In the following, we will detail seven mutation strategies and three crossover strategies, all of which are widely-recognized in state-of-the-art DE variants.
Here, the subscript notation in \( \textbf{x} \) specifies the individual selection technique.
For instance, \( \textbf{x}_r \) and \( \textbf{x}_{best} \) correspond to randomly selected and best-performing individuals respectively, whereas
\( \textbf{x}_i \) represents the currently evaluated individual.

\textbf{Mutation Strategies}:

\begin{enumerate}[label=\arabic*.]
  \item \texttt{DE/rand/1}:
        \begin{eqnarray}\label{equ_mutation rand}
        \begin{aligned}
        \mathbf{v}_i=\mathbf{x}_{r_1}+F \cdot\left(\mathbf{x}_{r_2}-\mathbf{x}_{r_3}\right).
        \end{aligned}
        \end{eqnarray}

  \item \texttt{DE/best/1}:
        \begin{eqnarray}\label{equ_mutation best}
        \begin{aligned}
        \mathbf{v}_i=\mathbf{x}_{\text {best }}+F \cdot\left(\mathbf{x}_{r_1}-\mathbf{x}_{r_2}\right).
        \end{aligned}
        \end{eqnarray}

  \item \texttt{DE/rand/2}:
        \begin{eqnarray}\label{equ_mutation rand2}
        \begin{aligned}
        \mathbf{v}_i & =\mathbf{x}_{r_1}+F \cdot\left(\mathbf{x}_{r_2}-\mathbf{x}_{r_3}\right)+F \cdot\left(\mathbf{x}_{r_4}-\mathbf{x}_{r_5}\right).
        \end{aligned}
        \end{eqnarray}

  \item \texttt{DE/best/2}:
        \begin{eqnarray}\label{equ_mutation best2}
        \begin{aligned}
        \mathbf{v}_i & =\mathbf{x}_{\text {best}}+F \cdot\left(\mathbf{x}_{r_1}-\mathbf{x}_{r_2}\right)+F \cdot\left(\mathbf{x}_{r_3}-\mathbf{x}_{r_4}\right).
        \end{aligned}
        \end{eqnarray}

  \item \texttt{DE/current-to-best/1}:
        \begin{eqnarray}\label{equ_mutation current2best}
        \begin{aligned}
        \mathbf{v}_i =\mathbf{x}_i+F \cdot\left(\mathbf{x}_{\text {best }}-\mathbf{x}_i\right)+F \cdot\left(\mathbf{x}_{r_1}-\mathbf{x}_{r_2}\right).
        \end{aligned}
        \end{eqnarray}
    The above five classical mutation strategies, introduced by Storn and Price \cite{DEbook2006}, cater to various problem landscapes.
    For instance, the `rand' variants help maintain population diversity, while strategies using two differences typically produce more diverse offspring than those relying on a single difference.

  \item \texttt{DE/current-to-pbest/1}:
        \begin{eqnarray}\label{equ_mutation current2pbest}
        \begin{aligned}
        \mathbf{v}_i =\mathbf{x}_i+F \cdot\left(\mathbf{x}_{\text {pbest}}-\mathbf{x}_i\right)+F \cdot\left(\mathbf{x}_{r_1}-\mathbf{x}_{r_2}\right).
        \end{aligned}
        \end{eqnarray}
    This strategy originates from JaDE \cite{JADE2009}. $\mathbf{x}_{\text{pbest}}$ is randomly selected from the top \emph{p}\% of individuals in the population (typically the top 10\%) to strike a balance between exploration and exploitation.


  \item \texttt{DE/current-to-rand/1}:
        \begin{eqnarray}\label{equ_mutation current2rand}
        \begin{aligned}
        &\mathbf{u}_i =\mathbf{x}_i+K_i\cdot\left(\mathbf{x}_{r_1}-\mathbf{x}_i\right)+F \cdot\left(\mathbf{x}_{r_2}-\mathbf{x}_{r_3}\right).
        \end{aligned}
        \end{eqnarray}
    Here, \(K_i\) is a random number from \(U(0,1)\).
    This strategy, originally proposed in \cite{DEintro1999}, emphasizes rotational invariance. By bypassing the crossover phase, it directly yields the trial vector \(\mathbf{u}_i\). Thus, it is ideal for addressing non-separable rotation challenges and has been a cornerstone for multiple adaptive DE variations.

  \end{enumerate}

\textbf{Crossover Strategies:}

\begin{enumerate}[label=\arabic*.]
  \item Binomial Crossover:
    \begin{eqnarray}\label{equ_cross bin}
    \begin{aligned}
    u_{i, j}= \begin{cases}v_{i, j}, & \text { if } r \leq C R \text { or } j=j_{\mathrm{rand}} \\ x_{i, j}, & \text {otherwise},\end{cases}
    \end{aligned}
    \end{eqnarray}
    where \(j_{\mathrm{rand}}\) is a random integer between 1 and \( D \). This strategy is a cornerstone in DE.

  \item Exponential Crossover:
  \begin{eqnarray}\label{equ_cross exp}
    \begin{aligned}
    \small
    u_{i, j}= \begin{cases}v_{i, j}{ } & \text { if } j=\langle n\rangle_d,\langle n+1\rangle_d,...,\langle n+L-1\rangle_d \\ x_{i, j} & \text {otherwise},\end{cases}
    \end{aligned}
    \end{eqnarray}
    where \(\langle \rangle_d\) is a modulo operation with \(D\) and \(L\) representing the crossover length, following a censored geometric distribution with a limit of \(D\) and probability of \(CR\).
    By focusing on consecutive variables, this strategy excels in handling problems with contiguous variable dependencies.

  \item Arithmetic Recombination:
  \begin{eqnarray}\label{equ_cross arith}
    \begin{aligned}
    \mathbf{u}_i=\mathbf{x}_i + K_i\cdot(\mathbf{v}_i - \mathbf{x}_i),
    \end{aligned}
    \end{eqnarray}
    where \(K_i\) is a random value from \(U(0,1)\).
    Exhibiting rotational invariance, this strategy, when combined with the \texttt{DE/rand/1} mutation, results in the \texttt{DE/current-to-rand/1} strategy \cite{DEsurvey2011}, as described by:
    \begin{eqnarray}\label{equ_rand_1_arith}
    \begin{aligned}
    \mathbf{u}_i&=\mathbf{x}_i + K_i\cdot(\mathbf{v}_i - \mathbf{x}_i)\\
    &=\mathbf{x}_i + K_i(\mathbf{x}_{r_1}+F\cdot(\mathbf{x}_{r_2}-\mathbf{x}_{r_3}) - \mathbf{x}_i)\\
    &=\mathbf{x}_i + K_i(\mathbf{x}_{r_1} - \mathbf{x}_i)+ K_i\cdot F(\mathbf{x}_{r_2}-\mathbf{x}_{r_3}),
    \end{aligned}
    \end{eqnarray}
    which is equivalent to Eq. (\ref{equ_mutation current2rand}).

\end{enumerate}


\subsubsection{Adaptive DE}
The development of parameter adaption in DE has witnessed significant advancements over time, from initial endeavors in parameter adaptation to recent sophisticated methods that merge multiple strategies.
This subsection traces the chronological advancements, emphasizing the pivotal contributions and their respective impacts on adaptive DE.

The earliest phase in DE's adaption centered on the modification of the crossover rate \( CR \) .
Pioneering algorithms such as SPDE \cite{SPDE2003} incorporated \( CR \) within the parameter set of individuals, enabling its simultaneous evolution with the decision variables of the problem to be solved.
This strategy was further refined by SDE \cite{SDE2005}, which assigned \( CR \) for each individual based on a normal distribution. Subsequent research efforts shifted focus to the scaling factor \( F \).
In this context, DETVSF \cite{DETVSF2005} dynamically adjusted \( F \), fostering exploration during the algorithm's nascent stages and pivoting to exploitation in later iterations.
Building on this, FaDE \cite{FaDE2005} employed fuzzy logic controllers to optimize mutation and crossover parameters.

The DESAP \cite{DESAP2006} algorithm marked a significant paradigm shift by introducing self-adapting populations and encapsulating control parameters within individuals.
Successive contributions like jDE \cite{jDE2006}, SaDE \cite{SaDE2008}, and JaDE \cite{JADE2009} accentuated the significance of parameter encoding, integrated innovative mutation strategies, and emphasized archiving optimization trajectories using external repositories.
Further, EPSDE \cite{EPSDE2011} and CoDE \cite{CoDE2011} enhanced the offspring generation process, amalgamating multiple strategies with randomized parameters.

The contemporary landscape of adaptive DE is characterized by complex methodologies and refined strategies.
Algorithms such as SHADE \cite{SHADE2013} and LSHADE \cite{LSHADE2014} championed the utilization of success-history mechanisms and dynamic population size modifications.
Notable developments like ADE \cite{ADE2014} introduced a biphasic parameter adaptation mechanism.
The domain further expanded with algorithms like LSHADE-RSP \cite{LSHADE-RSP2018}, IMODE \cite{IMODE2020}, and LADE \cite{LADE2023}, emphasizing mechanisms such as selective pressure, the integration of multiple DE variants, and the automation of the learning process.

Undoubtedly, the adaptive DE domain has witnessed transformative growth, with each phase of its evolution contributing to its current sophistication.
However, despite these advancements, many adaptive strategies remain empirical and hinge on manual designs, while their effectiveness is not universally guaranteed.

\subsection{Distributed DE}
The integration of distributed (i.e., multi-population) strategies also significantly enhances the efficacy of DE. 
Leading this advancement, Weber \textit{et al.} conducted extensive research on scale factor interactions and mechanisms within a distributed DE framework \cite{weber2010study, weber2011study, weber2013study}, followed by ongoing developments along the pathway \cite{DEpapersurvey2020}.
For example, some works such as EDEV \cite{EDEV2018}, MPEDE \cite{MPEDE2015} and IMPEDE \cite{IMPEDE} adopted multi-population frameworks to ensemble various DE variants/operators,
while the other works  such as DDE-AMS \cite{DDE-AMS} and DDE-ARA \cite{DDE-ARA} employed multiple populations for adaptive resource allocations.

Despite the achievements, current implementations of distributed DE often focus predominantly on algorithmic improvements, while overlooking potential enhancements from advanced hardware accelerations such as GPU computing. 
Besides, the design of these distributed strategies often features intricate and rigid configurations that lack proper flexibility.



\subsection{MetaEAs}\label{subsection_Meta-EA}

Generally, the term \emph{meta} refers to a higher-level abstraction of an underlying concept, often characterized by its \emph{recursive} nature.
In the context of EC, inception of the Meta Evolutionary Algorithms (MetaEAs) can be traced back to the pioneering works of Mercer and Sampson \cite{metaplan1978} in the late 1970s.
Under the initiative termed \emph{meta-plan}, their pioneering efforts aimed at enhancing EA performance by optimizing its parameters through another EA.
Although sharing similarities with hyperheuristics \cite{Hyperheu2013,Hyperheu2020,NeriHyperspam}, a major difference distinguishes MetaEAs: while hyperheuristics often delve into selecting and fine-tuning a set of predefined algorithms, MetaEAs concentrate on the paradigm of refining the parameters of EAs by EAs.
Notably, MetaEAs are also akin to ensemble of algorithms, such as EDEV \cite{EDEV2018} and CoDE \cite{CoDE2011}, which amalgamate diverse algorithms to ascertain the most efficacious among them.


Advancing the meta-plan concept, MetaGA \cite{MetaGA1986} emerged as a significant milestone.
Here, a genetic algorithm (GA) was deployed to fine-tune six intrinsic control parameters, namely: population size, crossover rate, mutation rate, generation gap, scaling window, and selection strategy.
The efficacy of this approach was gauged using dual metrics: online and offline performance.

The evolution of the concept continued with MetaEP \cite{metaEP1991}, which offers a meta-level evolutionary programming (EP) that could concurrently evolve optimal parameter settings.
Another pivotal contribution was the Parameter Relevance Estimation and Value Calibration (REVAC) \cite{REVAC2007}, which served as a meta estimation of distribution algorithm (MetaEDA).
Utilizing a GA at its core, REVAC iteratively discerned promising parameter value distributions within the configuration space.

Innovations in the domain persisted with the Gender-based GA (GGA) \cite{GGA2009}, inspired by natural gender differentiation, and other notable methods like MetaCMAES \cite{metaCMAES2012}.
As articulated in the PhD thesis by Pedersen \cite{metaEAPhD2010}, a profound insight into MetaEA revealed that while contemporary optimizers endowed with adaptive behavioral parameters offered advantages, they were often eclipsed by streamlined optimizers under appropriate parameter tuning.
This thesis, which embraced DE as one of its optimization tools, employed the Local Unimodal Sampling (LUS) heuristic for tuning parameters such as \( NP \), \( F \), and \( CR \).

Culminating the discourse, the work in \cite{metaEAdistributed} demonstrated the scalability of MetaEAs by harnessing it within a large-scale distributed computing environment.
With the ($\mu$, $\lambda$)$-$ES steering the meta-level tuning, base-level algorithms like GA, ES, and DE were adeptly optimized.
For DE, parameters optimized encompassed \( NP \), mutation operator, \( F \), \( CR \), and \( PF \) (parameter for the \emph{either-or} strategy), enhancing MetaEAs' prowess in addressing intricate, large-scale optimization problems.
Recently, the MetaEA paradigm has also been employed for automated design of ensemble DE \cite{EDE}.

The field of MetaEAs has shown steady progress since its inception in the 1970s.
However, despite the achievements, the landscape of MetaEAs research still confronts certain limitations.
Notably, the research, while promising, has predominantly remained confined to smaller-scale implementations.
The anticipated leap to large-scale experiments, especially those that might benefit from GPU acceleration, remains largely uncharted. This underscores an imperative need for more extensive empirical validations and the exploration of contemporary computational resources to fully realize the potential of MetaEAs.

\subsection{GPU-accelerated EC Framework}\label{subsection_Meta-optimization}
To capitalize on the advancements of modern computing infrastructures, we have seamlessly integrated our proposed MetaDE with EvoX~\cite{evox}, a distributed GPU-accelerated computing framework for scalable EC.
This integration ensures that MetaDE enables efficient execution and optimization for large-scale evaluations.

The EvoX framework provides several distinctive features.
Primarily, it is designed for optimal performance across diverse distributed systems and is tailored to manage large-scale challenges.
Its user-friendly functional programming model simplifies the EC algorithm development process, reducing inherent complexities.
The framework cohesively integrates data streams and functional elements into a comprehensive workflow, underpinned by a sophisticated hierarchical state management system.
Moreover, EvoX features a rich library of EC algorithms, proficient in addressing a wide array of tasks, from black-box optimization to advanced areas such as deep neuroevolution and evolutionary reinforcement learning.


\section{Proposed Approach}\label{section_The proposed metade}
The foundational premise of MetaDE is to utilize a core DE algorithm to evolve an ensemble of parameterized DE variants.
Within this framework, the core DE, which tunes the parameters, is termed the \texttt{evolver}.
In contrast, each parameterized DE variant, which optimizes the problem at hand, is termed the \texttt{executor}.
This section commences by augmenting the parameterization of DE in a more general manner, such that DE is made \emph{evolvable} by spawning various DE variants by modulating the parameters.
Then, this section details the integration of proposed MetaDE within a meta-framework, together with a brief introduction to the GPU-accelerated implementation.

\subsection{Augmented Parameterization of DE}\label{section_PDE}
To make DE evolvable, this subsection introduces the Parameterized Differential Evolution (PDE), an extension of the standard DE designed to augment its flexibility through the parameterization of mutation and crossover strategies.
While PDE retains the foundational principles of standard DE, its distinctiveness lies in its capability to generate a multitude of strategies by modulating the parameters.

In the standard DE framework, tunability is constrained to the adjustments of the \( F \) and \( CR \) parameters, and strategies are bound by predefined rules.
To augment this limited flexibility, PDE introduces a more granular parameterization.
Building upon DE's notation \texttt{DE/x/y/z}, PDE encompasses six parameters: \( F \) (scale factor), \( CR \) (crossover rate), \( bl \) (base vector left), \( br \) (base vector right), \( dn \) (difference number), and \( cs \) (crossover scheme).
The combined roles of \( bl \) and \( br \) determine the base vector, leading to a strategy notation for PDE expressed as \texttt{DE/bl-to-br/dn/cs}.

Each parameter's nuances and the array of strategy combinations they enable are elaborated upon in the subsequent sections.
Fig. \ref{Figure_structure} provides a visual representation of these parameters, delineated within dashed boxes.


\begin{figure*}[!h]
\centering
\includegraphics[scale=0.6]{fix_structure.pdf}
\caption{
Parameter delineation of PDE and their respective domains.
PDE comprehensively parameterizes DE, endorsing unrestricted parameter and strategy modifications.
In this schema, \( F \) and \( CR \) are continuous parameters, whereas others are categorical.
The dashed-line boxes exhibit their specific value ranges.
The mutation function is derived from the base vector left, base vector right, and difference number parameters.
}
\label{Figure_structure}
\end{figure*}

\subsubsection{Scale Factor \( F \) and Crossover Rate \( CR \)}
The scale factor \( F \) and crossover rate \( CR \) serve as pivotal parameters in DE, both represented as real numbers.

The \( F \) parameter regulates the differential variation among population entities.
Elevated values induce exploratory search behaviors, while lower values encourage more exploitation.
Although Storn and Price originally identified [0, 2] as an effective domain for \( F \) \cite{DE1997}, contemporary DE variants deem \( F \leq 1 \) as more judicious \cite{JADE2009,CoDE2011,EPSDE2011,SHADE2013,LSHADE2014}.
Consequently, PDE constrains \( F \) within [0,1].

On the other hand, \( CR \) controls the recombination extent during crossover.
Higher \( CR \) make it more likely to try out new gene combinations from changes, while lower values keep the genes more stable and similar to the original.
As a rate (or probability), \( CR \) is confined to the [0, 1] interval.

\subsubsection{Augmented Parameterization of Mutation}
The distinctiveness of PDE lies in its ability to generate a multitude of mutation strategies by modulating parameters $bl, br$, and $dn$.
Parameters $bl$ (base vector left) and $br$ (base vector right) select one of four possible vectors:
\begin{itemize}
  \item  \texttt{rand}: A randomly chosen individual from the population.
  \item  \texttt{best}: The best individual in the population.
  \item  \texttt{pbest}: A random selection from the top \( p\% \) of individuals.
  \item  \texttt{current}: The present parent individual.
\end{itemize}
Parameter \( dn \), controls the number of differences in mutation, assumes values in the set \{1, 2, 3, 4\}.
Specifically, each difference \( \Delta \) captures the difference between two unique and randomly selected individuals from the population.
Therefore, the mutation formulation \texttt{DE/bl-to-br/dn} is:
\begin{align}\label{equ_PDE_mutation}
\mathbf{v} = \mathbf{x}_{bl} + F \cdot (\mathbf{x}_{br} - \mathbf{x}_{bl}) + F \cdot (\Delta_1 + ... +\Delta_{dn}).
\end{align}
In the implementation of PDE, \( bl \) and \( br \) are encoded as: 1: \texttt{rand}, 2: \texttt{best},  3: \texttt{pbest}, and   4: \texttt{current}.
In addition, if both \( bl \) and \( br \) assume identical values, the term $F \cdot (\mathbf{x}_{br} - \mathbf{x}_{bl})$ will disappear. The base vector takes the value of $\mathbf{x}_{bl}$ directly and the mutation strategy then becomes non-directional (e.g., \texttt{DE/rand/1}).

\subsubsection{Augmented Parameterization of Crossover}
Parameter \( cs \) defines the crossover strategies in PDE, comprising those elaborated in Section~\ref{subsection_DE and Parameter adaption}: binomial crossover, exponential crossover, and arithmetic recombination.
Specifically, \( cs \) is encoded as:   1: \texttt{bin}, 2: \texttt{exp}, and 3: \texttt{arith}, representing binomial crossover, exponential crossover, and arithmetic recombination respectively.

As a result, the augmented parameterizations for mutation and crossover, denoted as \texttt{DE/bl-to-br/dn/cs}, will give rise to a spectrum of 192 distinctive strategies in total.
This breadth allows for the encapsulation of mainstream strategies detailed in Section~\ref{subsection_DE and Parameter adaption}, as illustrated in Table \ref{tab_param encoding}.
When synergized with \( F \) and \( CR \), this culminates in a comprehensive parameter configuration landscape.


% Table generated by Excel2LaTeX from sheet 'Sheet1'
\begin{table}[htbp]
  \centering
  \caption{The encoding of typical DE variants by the proposed PDE.}
         \renewcommand{\arraystretch}{1.1}
 \renewcommand{\tabcolsep}{10pt}
    \begin{tabular}{ccccc}
    \toprule
    strategy & $bl$  & $br$  & $dn$  & $cs$ \\
    \midrule
    \texttt{DE/rand/1/bin} & 1     & 1     & 1     & 1 \\
    \midrule
    \texttt{DE/best/1/bin} & 2     & 2     & 1     & 1 \\
    \midrule
    \texttt{DE/current-to-best/1/bin} & 4     & 2     & 1     & 1 \\
    \midrule
    \texttt{DE/rand/2/bin} & 1     & 1     & 2     & 1 \\
    \midrule
    \texttt{DE/best/2/bin} & 2     & 2     & 2     & 1 \\
    \midrule
    \texttt{DE/current-to-pbest/1/bin} & 4     & 3     & 1     & 1 \\
    \midrule
    \texttt{DE/current-to-rand/1*} & 1     & 1     & 1     & 3 \\
    \bottomrule
    \end{tabular}%
  \label{tab_param encoding}%
    \vspace{0.5em}

      \footnotesize
    \textsuperscript{*} As per Eq. (\ref{equ_rand_1_arith}), \texttt{DE/current-to-rand/1} is equivalent to \texttt{DE/rand/1/arith}.\\
\end{table}%

Algorithm \ref{Alg_PDE} details PDE's procedure.
Its distinct attribute is Line 3, where the mutation function is shaped by \( bl \), \( br \), and \( dn \).
Line 4 designates the crossover function based on \( cs \).
The evolutionary phase commences at Line 6.
Notably, PDE's concurrent mutation and crossover operations (Lines 7-10) are {tensorized} to facilitate parallel offspring generation, which deviate from the conventional sequential operations in standard DE.
With such a tailored procedure, the computational efficiency can be substantially improved.


\begin{algorithm}
\small
\caption{Parameterized DE (PDE)}\label{Alg_PDE}
\begin{algorithmic}[1]
  \Require {$D$, $NP$, $G_{max}$, $F$, $CR$, $bl$ (base vector left), $br$ (base vector right), $dn$ (difference number), $cs$ (crossover scheme)}
  \State Initialize population $\mathbf{X} = \{\mathbf{x}_1, \mathbf{x}_2, \dots, \mathbf{x}_{\scalebox{0.5}{$\textit{NP}$}}\}$
  \State Evaluate the fitness of each individual in the population
  \State Generate mutation function $M(\mathbf{X})$ according to $bl$, $br$, $dn$:
  \Statex $\mathbf{v} = \mathbf{x}_{bl} + F \cdot (\mathbf{x}_{br} - \mathbf{x}_{bl}) + F \cdot (\Delta_1 + ... +\Delta_{dn})$
  \State According to $cs$, choose a crossover function $C(\mathbf{V}, \mathbf{X})$ in (\ref{equ_cross bin}-\ref{equ_cross arith})
  \State $g = 0$
  \While{$g \leq G_{max}$}
  \State Generate $NP$ mutant vectors: $\mathbf{V} = M(\mathbf{X})$
  \State Perform crossover for all mutant vectors: $\mathbf{U} = C(\mathbf{V}, \mathbf{X})$
  \State Evaluate the fitness of $\mathbf{U}$
  \State Make selection between $\mathbf{U}$ and $\mathbf{X}$
  \State $g = g + 1$
  \EndWhile
  \State\Return the best fitness
\end{algorithmic}
\end{algorithm}


\subsection{Architecture of MetaDE}
Atop the proposed PDE, this subsection further introduces the architecture of MetaDE.
The main target of MetaDE is to evolve the parameters of PDE through an external DE, empowering PDE to identify optimal parameters tailored to the target problem.
% Every computational phase of MetaDE benefits from extensive parallelization with GPU acceleration integration.

\begin{figure}[t]
\centering
\includegraphics[scale=0.9]{MetaDE.pdf}
\caption{
Architecture of MetaDE.
Within this architecture, a conventional DE algorithm operates as an \texttt{evolver}, where its individual $\mathbf{x}_i$ represents a distinct parameter configuration $\mathbf{\theta}_i$.
These configurations are relayed to PDE  to instantiate diverse DE variants as the \texttt{executors}.
Each \texttt{executor} then evolves its distinct population and returns the best fitness $y^*$ as identified, which is subsequently set as the fitness of $\mathbf{x}_i$.
}
\label{Figure_MetaDE}
\end{figure}

As illustrated in Fig. \ref{Figure_MetaDE}, MetaDE is structured with a two-tiered optimization architecture.
The upper tier, termed the \texttt{evolver}, leverages DE to evolve the parameters of PDE. In contrast, the lower tier consists of a collection of \texttt{executors} that each run the parameterized PDE instance to optimize the objective function.
Every individual in the \texttt{evolver}, represented as $\mathbf{x}_i$, is decoded into a parameter configuration $\bm{\theta}_i$ with six elements: $F$, $CR$, $bl$, $br$, $dn$, and $cs$.
For the evaluation of each individual, the configuration $\bm{\theta}_i$ is directed to its respective \texttt{executor} $\textrm{PDE}_i$ for objective function optimization.
The final fitness $y^*$ as identified by each \texttt{executor}, is subsequently set as the fitness of the corresponding $\mathbf{x}_i$ individual.

The architecture of MetaDE is streamlined for simplicity.
Building upon this architecture, MetaDE integrates two tailored components: the \emph{one-shot evaluation method} and the \emph{power-up strategy}.
These components further enhance the adaptability and efficiency of the \texttt{executors}, thereby elevating the overall performance of MetaDE.

\subsection{One-shot Evaluation Method}
Within the context of an \texttt{executor} driven by DE itself, the inherent stochastic nature can lead to variability in the optimal fitness values returned.
Historically, several evaluation techniques, such as repeated evaluation~\cite{metaCMAES2012}, F-racing~\cite{FRacing}, and intensification~\cite{intens}, have been put forth to tackle this inconsistency.
Yet, these often come at the cost of an exorbitant number of functional evaluations (FEs).
To address this issue, we introduce the one-shot evaluation method.

Specifically, the method mandates each \texttt{executor} to undertake a singular, comprehensive independent run, subsequently returning its best-found solution.
A distinguishing aspect of this method is the consistent allocation of the same initial random seed to every \texttt{executor}.
As the algorithm progresses, this uniform seed ensures that the PDE fine-tunes its parameters in a consistent manner, thereby identifying optimal parameters tailored to the given seed environment.
Essentially, this strategy embeds the seed as an integral facet of the problem domain.

\subsection{Power-up Strategy}
During the independent runs of an \texttt{executor}, the allocation of FEs plays a pivotal role in determining both the quality of solutions and computational efficiency. Allocating an excessive number of FEs indiscriminately can lead to undue computational resource consumption without necessarily improving solution quality.
To address this issue, we propose the power-up strategy.

The essence of this strategy is dynamic FE allocation: while earlier iterations receive a moderate number of FEs to ensure resource efficiency, a more generous allocation (fivefold) is reserved for the terminal iteration within the evolutionary process.
This strategy ensures that the \texttt{executor} has the resources for a thorough and comprehensive evaluation during its most crucial phase -- the final generation of the \texttt{evolver}.


\subsection{Implementation}
As outlined in Algorithm~\ref{Alg_MetaDE}, MetaDE draws its simple algorithmic workflow from conventional DE. MetaDE adopts \texttt{DE/rand/1/bin} as the \texttt{evolver}.
The initialization phase (Line 1) spawns the MetaDE population within the parameter boundaries \( [\mathbf{lb}, \mathbf{ub}] \).
During the evaluation phase (Lines 6-11), each individual is decoded into a parameter blueprint and directed to an independent PDE instance (\texttt{executor}) for problem resolution.
Running for a predetermined iteration count \( G' \), each \texttt{executor} subsequently reports the best fitness.
Notably, Line 10 encapsulates the essence of the power-up strategy: for MetaDE's concluding iteration (\( g == G_{max} \)), the evaluation quota is amplified to \( 5 \times G' \) for the \texttt{executors}.


\begin{algorithm}
\small
\caption{MetaDE}
\label{Alg_MetaDE}
\begin{algorithmic}[1]
  \Require {$D$, $NP$, $G_{max}$, $\mathbf{lb}$ (lower boundaries of PDE's parameters), $\mathbf{ub}$ (upper boundaries of PDE's parameters), $NP'$ (population size of PDE), $G'$ (max generations of PDE)}
  \State Initialize population $\mathbf{X} = \{\mathbf{x}_1, \mathbf{x}_2, \dots, \mathbf{x}_{\scalebox{0.5}{$\textit{NP}$}}\}$ between $[\mathbf{lb}, \mathbf{ub}]$
  \State Initialize the fitness of $\mathbf{X}$: $\mathbf{y}=\mathbf{inf}$
  \State $g = 0$
  \While{$g \leq G_{max}$}
  \State Generate trial vectors $\mathbf{U}$ by mutation and crossover
  \Statex \quad \ /* The mutation and crossover scheme used is rand/1/bin\ */
  \State Decode each trial vector $\mathbf{u}$ into parameters:
  \Statex \quad \ $F=\mathbf{u}[1], CR=\mathbf{u}[2], bl=\textrm{floor}(\mathbf{u}[3]), br=\textrm{floor}(\mathbf{u}[4])$,
  \Statex \quad \ $dn=\textrm{floor}(\mathbf{u}[5]), cs=\textrm{floor}(\mathbf{u}[6])$
  %\Statex /*Evaluate each $\mathbf{u}$ by running PDE for certain generations*/
  \If {$g < G_{max}$}
    \State $\mathbf{y} = \textrm{PDE}(D, NP', G', F, CR, bl, br, dn, cs)$
  \Else
    \State $\mathbf{y} = \textrm{PDE}(D, NP', 5 * G', F, CR, bl, br, dn, cs)$
  \EndIf
  \State Make selection between $\mathbf{U}$ and $\mathbf{X}$
  \State $g = g + 1$
  \EndWhile
  \State\Return the best individual and fitness
\end{algorithmic}
\end{algorithm}


Evidently, the algorithmic design of MetaDE provides an automated end-to-end approach to black-box optimization.
However, the computational demands of MetaDE, particularly in terms of FEs, cannot be understated.
In historical computational contexts, such intensive demands might have posed significant impediments.
Fortunately, contemporary advancements in computational infrastructures, coupled with the ubiquity of high-performance computational apparatuses such as GPUs, have substantially alleviated such a challenge.


Hence, we leverage the GPU-accelerated framework of EvoX~\cite{evox} for the implementation of MetaDE.
Thanks to the inherently parallel nature of MetaDE, computational tasks can be judiciously delegated to GPUs to engender optimized runtime performance.
Specifically, the parallelism in MetaDE manifests in three distinct facets:
\begin{itemize}
  \item \textbf{Parallel Initialization and Execution:}
 The multiple \texttt{executors} are instantiated and operated concurrently, each tailored by a unique parameter configuration derived from the MetaDE ensemble.
  This simultaneous operation enables comprehensive exploration across varied parameter landscapes.

  \item \textbf{Parallel Offspring Generation:}
  Both the \texttt{evolver} and \texttt{executors} adhere to parallel strategies for offspring inception.
  By synchronizing and coordinating mutations and crossover operations in their respective populations, MetaDE is able to rapidly produce offspring, thereby accelerating the evolutionary process.

  \item \textbf{Parallel Fitness Evaluations:}
  Each \texttt{executor} conducts fitness evaluations concurrently across its member individuals.
  Given the substantial number of the individuals within the populations of the \texttt{executors}, this parallel strategy significantly enhances the overall efficiency of MetaDE.
\end{itemize}


\lstset{
  language=Python,
  aboveskip=3mm,
  belowskip=3mm,
  showstringspaces=false,
  columns=flexible,
  basicstyle={\footnotesize\ttfamily},
  numbers=left,
  numberstyle=\tiny\color{gray},
  xleftmargin=2em,
  keywordstyle=\bfseries\color{dkgreen},
  commentstyle=\color{gray}\itshape,
  stringstyle=\color{mauve},
  breaklines=true,
  breakatwhitespace=true,
  tabsize=3,
  emph={[1]evox,BatchExecutor,MetaProblem},          % Emphasize numpy
  emphstyle={[1]\bfseries\color{dkblue}},  % Set the style for emphasized words
  emph={[2]__init__, min, evaluate, reproduce, init},
  emphstyle={[2]\color{dkblue}},
  emph={[3]self,super},          % Emphasize numpy
  emphstyle={[3]\color{dkgreen}},
  morekeywords={from,import},  % Add more keywords
  captionpos=b,             % Caption position
      frame=lines,
  framesep=2mm,
}
\
\begin{lstlisting}[caption={Demonstrative implementation of MetaDE leveraging the computational workflow of EvoX. The implementation is distinctly divided into four pivotal components: Workflow Initialization, Meta Problem Transformation, Computing Workflow Creation, and Execution.}, label={lst:python_example}, float=!t]
from evox import algorithms, problems, ...

### Initialization ###
evolver = algorithm.DE()  # specify evolver
executor = algorithm.PDE()  # specify executor
problem = ...  # specify optimization problem

### Meta Problem Transformation ###
class MetaProblem(Problem):
    def __init__(self, batch_executor, ... ):
        # vectorize fitness evaluations
        self.batch_evaluate = vectorize(vectorize(problem.evaluate))

    def evaluate(self, state, ...):
        ...
        # run executors
        while ...:
            ...
            batch_fits, ... = self.batch_evaluate(...)
        # return fitness
        return min(min(batch_fits))

### Computing Workflow Creation ###
batch_executor = create_batch_executor(...)
meta_problem = MetaProblem(batch_executor, ...)
workflow = workflow.UniWorkflow(
            algorithm = evolver,
            pop_transform = decoder,
            problem = meta_problem,
        )

### Execution ###
while ...:
    state = workflow.step(state)
\end{lstlisting}

MetaDE adheres rigorously to the functional programming paradigm, capitalizing on automatic vectorization for parallel execution. Core algorithmic components, including crossover, mutation, and evaluation, are constructed using pure functions. Subsequently, the entire program is mapped to a GPU-based computation graph, ushering in accelerated processing. EvoX's adept state management ensures a seamless transfer of the algorithm's prevailing state, encompassing aspects like population, fitness, hyperparameters, and auxiliary data.

Listing \ref{lst:python_example} elucidates a representative implementation of MetaDE underpinned by the EvoX framework, which is meticulously segmented into four salient phases:

\begin{itemize}
    \item \textbf{Initialization}: Herein, primary entities like the \texttt{evolver} (employing the traditional DE) and the \texttt{executor} (utilizing the proposed PDE) are instantiated. Concurrently, the target optimization problem is defined.

    \item \textbf{Meta Problem Transformation}: Within this phase, the original optimization problem is transformed to align with the meta framework. This metamorphosis is realized via the \texttt{\textbf{MetaProblem}} class, where the evaluation function undergoes vectorization, priming it for efficient batch assessments and facilitating concurrent evaluations of manifold configurations.

    \item \textbf{Computing Workflow Creation}: Post transformation, the workflow is architected to seamlessly amalgamate the initialized components. The \texttt{batch\_executor} is crafted for batched operations of DE variants, and the \texttt{\textbf{MetaProblem}} is instantiated therewith. The holistic workflow, embodied by the \texttt{UniWorkflow} class, is then constructed, weaving together the \texttt{evolver}, the transformed problem, and a (\texttt{decoder}) which transforms the \texttt{evolver}'s population into specific hyperparameters for instantiating  the DE variant of each \texttt{executor}.

    \item \textbf{Execution}: Having established the groundwork, MetaDE's execution phase is triggered, autonomously driving the computing workflow across distributed GPUs.
    This workflow is traversed iteratively, culminating once a predefined termination criterion is met.
\end{itemize}

\section{Experimental Study}\label{section_Experimental_study}
In this section, we conduct detailed experimental assessments of MetaDE's capabilities.
First, we comprehensively benchmark MetaDE against several representative DE variants and CEC2022 top algorithms to gauge its relative performance on the CEC2022 benchmark suite \cite{CEC2022SO}.
Then, we investigate the optimal DE variants obtained by MetaDE in the benchmark experiment.
Finally, we apply MetaDE to robot control tasks.
All experiments were conducted on a system equipped with an Intel Core i9-10900X CPU and  an NVIDIA RTX 3090 GPU.
For GPU acceleration, all the algorithms and test functions were implemented within EvoX \cite{evox}.

\subsection{Benchmarks against Representative DE Variants}\label{section_Comparison with Classic DE Variants}

\subsubsection{Experimental Setup}
The CEC2022 benchmark suite for single-objective black-box optimization was utilized for this study.
This suite includes basic ($F_1-F_5$), hybrid ($F_6-F_8$), and composition functions ($F_9-F_{12}$), catering to various optimization characteristics such as unimodality/multimodality and separability/non-separability.

For benchmark comparisons, we selected seven representative DE variants: DE (\texttt{rand/1/bin}) \cite{DE1997}, SaDE \cite{SaDE2008}, JaDE \cite{JADE2009}, CoDE \cite{CoDE2011}, SHADE \cite{SHADE2013}, LSHADE-RSP \cite{LSHADE-RSP2018}, and EDEV \cite{EDEV2018}, which encapsulate a spectrum of mutation, crossover, and adaptation strategies. All algorithms were reimplemented using EvoX, with each capable of running in parallel, including the concurrent evaluation and reproduction.

Their respective descriptions are as follows:
\begin{itemize}
  \item \texttt{DE/rand/1/bin} is a foundational DE variant, which leverages a random mutation strategy coupled with binomial crossover.
  \item SaDE maintains an archive for tracking successful strategies and \( CR \) values and exhibits adaptability in strategy selection and parameter adjustments throughout the optimization process.
  \item JaDE relies on the \texttt{current-to-pbest} mutation strategy and dynamically adjusts its \( F \) and \( CR \) parameters during the optimization trajectory.
  \item CoDE infuses generational diversity by composing three disparate strategies, each complemented with randomized parameters, for offspring generation.
  \item SHADE employs the current-to-pbest mutation strategy and integrates a success-history mechanism to fine-tune its \( F \) and \( CR \)  parameters adaptively.
  \item LSHADE-RSP, as one of the most competitive DE variants, employs delicate strategies such as linear population size reduction and ranking-based mutation.
  \item EDEV adopts a distributed framework that ensembles three classic DE variants: JaDE, CoDE, and EPSDE.
\end{itemize}

The population size for all comparative algorithms was uniformly set to 100, except for experiments involving large populations. The other parameters for these algorithms were adopted as per their default settings described in their respective publications.

In our MetaDE configuration, on one hand, the \texttt{evolver} had a population size of 100 and adopted the vanilla \texttt{rand/1/bin} strategy with $F=0.5$ and $CR=0.9$;
On the other hand, each \texttt{executor} maintained a population of 100, iterating 1000 times for all the problems.
For simplicity, any result exceeding the precision of $10^{-8}$ was truncated to 0.
All statistical results were obtained via 31 independent runs\footnote{Full results, including the statistical results applying Wilcoxon rank-sum tests with a a significance level of 0.05, can be found in the Supplementary Document.}.




\begin{figure}[!t]
\centering
\includegraphics[scale=0.3]{D10.pdf}
\caption{Convergence curves on 10D problems in CEC2022 benchmark suite. The peer DE variants are set with population size of 100.}
\label{Figure_convergence_10D}
\end{figure}

\begin{figure}[!t]
\centering
\includegraphics[scale=0.3]{D20.pdf}
\caption{Convergence curves on 20D problems in CEC2022 benchmark suite. The peer DE variants are set with population size of 100.}
\label{Figure_convergence_20D}
\end{figure}


\subsubsection{Performance under Equal Wall-clock Time}\label{sec:expereiment_time}
In this part, we set equal  wall-clock time (\SI{60}{\second}) as the termination condition for running each test. 
{
This approach aligns with the practical constraints of modern GPU computing, where execution time serves as a more meaningful and comparable measure of performance across algorithms. Since all algorithms in our experiments are implemented with GPU parallelism, this setup ensures fairness by standardizing the computational resources and focusing on efficiency within the same time budget.
}

As shown in Figs. \ref{Figure_convergence_10D} and \ref{Figure_convergence_20D}, we selected five challenging problems, specifically $F_2$, $F_4$, $F_6$, $F_9$, and $F_{10}$, to demonstrate the convergence profiles.
Notably, MetaDE's convergence curve is observably more favorable, consistently registering lower errors than its counterparts across the majority of the problems.
Particularly, on $F_2$, $F_4$, $F_9$, and $F_{10}$, MetaDE exhibits resilience against local optima entrapment and subsequent convergence stagnation.
This is attributed to MetaDE's capability to identify optimal algorithm settings tailored for diverse problems, rather than merely tweaking parameters based on isolated segments of the optimization trajectory, as is the case with some DE variants.
An intriguing characteristic of MetaDE's convergence, evident in functions like $F_9$ (refer to Fig. \ref{Figure_convergence_10D}), is its pronounced performance surge in the optimization's terminal phase.
This enhancement can be linked to MetaDE's power-up strategy of allocating bonus computational resources in its final phase (as per Line 10 of Algorithm \ref{Alg_MetaDE}).


\begin{figure}[!h]
\centering
\includegraphics[scale=0.3]{FEs.pdf}
\caption{The number of FEs achieved by each algorithm within \SI{60}{\second}. The results are averaged on all 10D and 20D problems in the CEC2022 benchmark suite.}
\label{fig:maxFEs}
\end{figure}

{
Furthermore, to assess the concurrency of the algorithms, the number of FEs achieved by each algorithm within 60 seconds is shown in Table~\ref{tab:FEs} and Fig~\ref{fig:maxFEs}.
}
The results indicate that MetaDE achieves approximately $10^9$ FEs within 60 seconds, while the other algorithms manage to attain only around $10^7$ FEs in the same time frame.
The results demonstrate the high concurrency of MetaDE, which is particularly favorable in GPU computing.

\begin{table}[htbp]
  \centering
  
  \caption{{The number of FEs achieved by each algorithm within \SI{60}{\second}.}}
 
\scriptsize                   %设置字体大小
\renewcommand{\arraystretch}{1}
\renewcommand{\tabcolsep}{2.5pt}   %pt越大字越小
\resizebox{\linewidth}{!}{
% Table generated by Excel2LaTeX from sheet 'Experiment1 60S'
\begin{tabular}{cccccccccc}
\toprule
Dim   & Func  & MetaDE & DE    & SaDE  & JaDE  & CoDE  & SHADE &LSHADE-RSP&EDEV\\
\midrule
\multirow{12}[2]{*}{10D} & $F_{1}$ & \textbf{1.85E+09} & 4.28E+06 & 1.96E+06 & 2.55E+06 & 1.09E+07 & 2.24E+06&2.57E+06&2.79E+06 \\
      & $F_{2}$ & \textbf{1.84E+09} & 4.19E+06 & 1.89E+06 & 2.42E+06 & 1.11E+07 & 2.18E+06&2.58E+06&2.82E+06 \\
      & $F_{3}$ & \textbf{1.50E+09} & 4.00E+06 & 1.89E+06 & 2.50E+06 & 1.11E+07 & 2.10E+06& 2.46E+06&2.68E+06\\
      & $F_{4}$ & \textbf{1.84E+09} & 4.11E+06 & 2.02E+06 & 2.61E+06 & 1.11E+07 & 2.15E+06&2.60E+06&2.88E+06 \\
      & $F_{5}$ & \textbf{1.83E+09} & 4.13E+06 & 2.00E+06 & 2.60E+06 & 1.15E+07 & 2.17E+06& 2.53E+06&2.96E+06\\
      & $F_{6}$ & \textbf{1.84E+09} & 4.31E+06 & 1.96E+06 & 2.65E+06 & 1.07E+07 & 2.14E+06&2.55E+06&2.95E+06\\
      & $F_{7}$ & \textbf{1.74E+09} & 3.35E+06 & 1.90E+06 & 2.55E+06 & 9.96E+06 & 2.14E+06& 2.41E+06&2.87E+06\\
      & $F_{8}$ & \textbf{1.72E+09} & 3.34E+06 & 1.83E+06 & 2.53E+06 & 9.60E+06 & 2.17E+06&2.34E+06&2.74E+06 \\
      & $F_{9}$ & \textbf{1.78E+09} & 3.35E+06 & 1.84E+06 & 2.52E+06 & 9.69E+06 & 2.18E+06&2.44E+06 &2.82E+06\\
      & $F_{10}$ & \textbf{1.44E+09} & 3.32E+06 & 1.83E+06 & 2.46E+06 & 9.00E+06 & 2.12E+06& 2.30E+06&2.70E+06\\
      & $F_{11}$ & \textbf{1.46E+09} & 3.55E+06 & 1.88E+06 & 2.34E+06 & 9.66E+06 & 2.13E+06&2.34E+06& 2.63E+06\\
      & $F_{12}$ & \textbf{1.43E+09} & 3.46E+06 & 1.83E+06 & 2.41E+06 & 9.51E+06 & 2.09E+06&2.32E+06 &2.67E+06\\
\midrule
\midrule
\multirow{12}[2]{*}{20D} & $F_{1}$ & \textbf{1.66E+09} & 4.32E+06 & 1.92E+06 & 2.46E+06 & 1.13E+07 & 2.21E+06&2.68E+06&2.80E+06 \\
      & $F_{2}$ & \textbf{1.66E+09} & 3.91E+06 & 1.88E+06 & 2.37E+06 & 1.17E+07 & 2.09E+06& 2.66E+06&2.74E+06\\
      & $F_{3}$ & \textbf{1.18E+09} & 3.76E+06 & 1.77E+06 & 2.33E+06 & 9.75E+06 & 1.95E+06&2.62E+06& 2.64E+06\\
      & $F_{4}$ & \textbf{1.65E+09} & 3.50E+06 & 1.87E+06 & 2.28E+06 & 1.07E+07 & 2.00E+06&2.63E+06 &2.80E+06\\
      & $F_{5}$ & \textbf{1.64E+09} & 3.57E+06 & 1.86E+06 & 2.32E+06 & 1.07E+07 & 2.05E+06& 2.55E+06&2.74E+06\\
      & $F_{6}$ & \textbf{1.64E+09} & 3.89E+06 & 1.90E+06 & 2.34E+06 & 1.16E+07 & 2.09E+06& 2.62E+06&2.80E+06\\
      & $F_{7}$ & \textbf{1.45E+09} & 4.15E+06 & 1.84E+06 & 2.36E+06 & 1.01E+07 & 1.99E+06&2.32E+06& 2.80E+06\\
      & $F_{8}$ & \textbf{1.44E+09} & 3.42E+06 & 1.82E+06 & 2.31E+06 & 9.51E+06 & 2.12E+06&2.18E+06&2.30E+06 \\
      & $F_{9}$ & \textbf{1.57E+09} & 3.30E+06 & 1.77E+06 & 2.33E+06 & 9.48E+06 & 2.03E+06&2.59E+06&2.75E+06 \\
      & $F_{10}$ & \textbf{9.80E+08} & 3.67E+06 & 1.82E+06 & 2.43E+06 & 1.01E+07 & 2.08E+06& 2.09E+06&2.35E+06\\
      & $F_{11}$ & \textbf{1.00E+09} & 3.51E+06 & 1.95E+06 & 2.41E+06 & 9.81E+06 & 2.07E+06&2.17E+06& 2.37E+06\\
      & $F_{12}$ & \textbf{9.90E+08} & 3.44E+06 & 1.85E+06 & 2.37E+06 & 9.51E+06 & 2.07E+06& 2.17E+06&2.39E+06\\
\bottomrule
\end{tabular}%
}
  \label{tab:FEs}%
\end{table}%

\subsubsection{Performance under Equal FEs}\label{sec:expereiment_FEs}

% Table generated by Excel2LaTeX from sheet 'Sheet1'
\begin{table*}[htbp]
  %\centering
\caption{Comparisons between MetaDE and other DE variants under equal FEs. 
The mean and standard deviation (in parentheses) of the results over multiple runs are displayed in pairs. 
Results with the best mean values are highlighted.}
  \resizebox{\linewidth}{!}{
  \renewcommand{\arraystretch}{1.2}
 \renewcommand{\tabcolsep}{2pt}
% Table generated by Excel2LaTeX from sheet 'Exp3 same FEs'
\begin{tabular}{cccccccccc}
\toprule
\multicolumn{2}{c}{Func} & MetaDE & DE    & SaDE  & JaDE  & CoDE  & SHADE & LSHADE-RSP & EVDE \\
\midrule
\multirow{3}[2]{*}{10D} & $F_{2}$ & \textbf{0.00E+00 (0.00E+00)} & 4.52E+00 (2.36E+00)$-$ & 6.85E+00 (3.52E+00)$-$ & 6.31E+00 (3.05E+00)$-$ & 5.78E+00 (2.37E+00)$-$ & 4.33E+00 (3.81E+00)$-$ & 2.35E+00 (3.44E+00)$-$ & 5.86E+00 (2.99E+00)$-$ \\
      & $F_{6}$ & \textbf{5.50E-04 (3.96E-04)} & 1.13E-01 (7.83E-02)$-$ & 3.54E+01 (1.11E+02)$-$ & 2.02E+00 (3.34E+00)$-$ & 6.96E-03 (5.99E-03)$-$ & 9.27E-01 (1.31E+00)$-$ & 3.10E-02 (4.83E-02)$-$ & 1.46E+00 (2.76E+00)$-$ \\
      & $F_{10}$ & \textbf{0.00E+00 (0.00E+00)} & 1.00E+02 (4.40E-02)$-$ & 1.00E+02 (6.10E-02)$-$ & 1.21E+02 (4.35E+01)$-$ & 1.00E+02 (6.88E-02)$-$ & 1.29E+02 (4.67E+01)$-$ & 1.09E+02 (2.87E+01)$-$ & 1.10E+02 (3.05E+01)$-$ \\
\midrule
\midrule
\multirow{3}[2]{*}{20D} & $F_{2}$ & \textbf{1.26E-02 (3.74E-02)} & 4.72E+01 (2.09E+00)$-$ & 4.76E+01 (2.02E+00)$-$ & 1.34E+01 (2.19E+01)$-$ & 4.91E+01 (1.70E-06)$-$ & 4.91E+01 (3.40E-06)$-$ & 4.84E+01 (1.62E+00)$-$ & 4.47E+01 (1.41E+01)$-$ \\
      & $F_{6}$ & \textbf{1.16E-01 (2.79E-02)} & 7.28E-01 (5.22E-01)$-$ & 3.20E+01 (1.61E+01)$-$ & 4.90E+01 (3.31E+01)$-$ & 2.26E+01 (1.80E+01)$-$ & 5.67E+01 (3.90E+01)$-$ & 1.15E+01 (8.38E+00)$-$ & 2.92E+03 (5.82E+03)$-$ \\
      & $F_{10}$ & \textbf{0.00E+00 (0.00E+00)} & 1.07E+02 (2.07E+01)$-$ & 1.00E+02 (2.72E-02)$-$ & 1.01E+02 (3.64E-02)$-$ & 1.00E+02 (3.71E-02)$-$ & 1.43E+02 (5.64E+01)$-$ & 1.21E+02 (4.52E+01)$-$ & 1.14E+02 (4.65E+01)$-$ \\
\midrule
\multicolumn{2}{c}{$+$ / $\approx$ / $-$} & --    & 0/0/6 & 0/0/6 & 0/0/6 & 0/0/6 & 0/0/6 & 0/0/6 & 0/0/6 \\
\bottomrule
\end{tabular}%

}
\label{tab:sameFEs}%

\footnotesize
\textsuperscript{*} The Wilcoxon rank-sum tests (with a significance level of 0.05) were conducted between MetaDE and each algorithm individually.
The final row displays the number of problems where the corresponding algorithm performs statistically better ($+$),  similar ($\thickapprox$), or worse ($-$) compared to MetaDE.
\end{table*}%



In the preceding part, the performance benchmarking of MetaDE with other algorithms was anchored to equal wall-clock durations.
However, to ensure a comprehensive assessment, it is imperative to evaluate their performances under equivalent FEs.
In this part, we run each algorithm using the FEs achieved by MetaDE in \SI{60}{\second} (i.e., $1.84\times10^9/1.66\times10^9$, $1.84\times10^9/1.64\times10^9$, and $1.44\times10^9/9.8\times10^8$) on $F_2$, $F_6$, and $F_{10}$ for 10D/20D cases.
These selected functions collectively epitomize the basic, hybrid, and composition challenges within the CEC2022 benchmark suite.

As summarized in Table \ref{tab:sameFEs}, MetaDE consistently demonstrates the best performance, even when other algorithms are endowed with comparable FEs.
The reason can be traced to the inherent stagnation tendencies of other algorithms: after a certain point, additional FEs may not contribute to performance improvements.
This behavioral pattern is also lucidly captured in the convergence curves as presented in Figs. \ref{Figure_convergence_10D} and \ref{Figure_convergence_20D}.

Another noteworthy observation is the extended computation time required for a singular run of the comparison algorithms under these enhanced FEs, often extending to several hours or even transcending a day (e.g., running a single run of DE can take up to seven hours).
This elongated computational span can largely be attributed to their low concurrency, which struggles to benefit the parallelism of GPU computing.


\subsubsection{Performance with Large Populations}\label{sec:expereiment_large_pop}
Since a large population size could potentially increase the concurrency of fitness evaluations, for rigorousness, we further investigate the performance of the algorithms with large populations.

Specifically, MetaDE adopted the same population size setting as in previous experiments (i.e., 100 for both \texttt{evolver} and \texttt{executor}), while the population size of the other DE variants was increased to 1,000. 
This adjustment significantly enhances the concurrency of the other DE variants when utilizing GPU accelerations, thereby preventing insufficient convergence.


As evidenced in Figs. \ref{Figure_convergence_10D_NP10k}-\ref{Figure_convergence_20D_NP10k}, MetaDE still outperforms the other DE variants across all problems.
However, the performances of the other DE variants did not show significant improvements, which can be attributed to two factors.
First, since the conventional DE variants were not tailored for large populations, simply enlarging the populations may not help.
Second, since the sorting and archiving operations in some DE variants (e.g., SaDE) suffer from high computational complexities related to the population size, enlarging the populations brings additional computation overheads, thus limiting their performances under fixed wall-clock time.

By contrast, the large population in MetaDE is delicately organized in a \emph{hierarchical} manner: the \texttt{executor} maintains a population of moderate size, with each individual initializing an \texttt{executor} with a normal population.
This strategy not only capitalizes on the small-population advantage of conventional DE, but also benefits the concurrency brought by large populations.

\begin{figure}[!t]
\centering
\includegraphics[scale=0.3]{D10NP1000.pdf}
\caption{Convergence curves on 10D problems in CEC2022 benchmark suite. The peer DE variants are set with population size of 1,000.}
\label{Figure_convergence_10D_NP10k}
\end{figure}

\begin{figure}[!t]
\centering
\includegraphics[scale=0.3]{D20NP1000.pdf}
\caption{Convergence curves on 20D problems in CEC2022 benchmark suite. The peer DE variants are set with population size of 1,000.}
\label{Figure_convergence_20D_NP10k}
\end{figure}


{
\subsection{Comparisons with Top Algorithms in CEC2022 Competition}\label{section_Comparison with Top Algorithms of CEC Competition}

% Table generated by Excel2LaTeX from sheet 'Sheet1'
\begin{table*}[htbp]
  \centering
  
  \caption{{Comparisons between MetaDE and the top 4 algorithms from CEC2022 Competition (10D). 
The mean and standard deviation (in parentheses) of the results over multiple runs are displayed in pairs. 
Results with the best mean values are highlighted. }
  }
\footnotesize
% Table generated by Excel2LaTeX from sheet 'Exp 7 vs CECtop'
\begin{tabular}{cccccc}
\toprule
Func  & MetaDE & EA4eig & NL-SHADE-LBC & NL-SHADE-RSP & S-LSHADE-DP \\
\midrule
$F_{1}$ & \textbf{0.00E+00 (0.00E+00)} & \boldmath{}\textbf{0.00E+00 (0.00E+00)$\approx$}\unboldmath{} & \boldmath{}\textbf{0.00E+00 (0.00E+00)$\approx$}\unboldmath{} & \boldmath{}\textbf{0.00E+00 (0.00E+00)$\approx$}\unboldmath{} & \boldmath{}\textbf{0.00E+00 (0.00E+00)$\approx$}\unboldmath{} \\
$F_{2}$ & \textbf{0.00E+00 (0.00E+00)} & 7.97E-01 (1.78E+00)$-$ & 7.97E-01 (1.78E+00)$-$ & \boldmath{}\textbf{0.00E+00 (0.00E+00)$\approx$}\unboldmath{} & \boldmath{}\textbf{0.00E+00 (0.00E+00)$\approx$}\unboldmath{} \\
$F_{3}$ & \textbf{0.00E+00 (0.00E+00)} & \boldmath{}\textbf{0.00E+00 (0.00E+00)$\approx$}\unboldmath{} & \boldmath{}\textbf{0.00E+00 (0.00E+00)$\approx$}\unboldmath{} & \boldmath{}\textbf{0.00E+00 (0.00E+00)$\approx$}\unboldmath{} & \boldmath{}\textbf{0.00E+00 (0.00E+00)$\approx$}\unboldmath{} \\
$F_{4}$ & \textbf{0.00E+00 (0.00E+00)} & 9.95E-01 (1.22E+00)$-$ & 1.99E-01 (4.45E-01)$-$ & 2.98E+00 (1.15E+00)$-$ & \boldmath{}\textbf{0.00E+00 (0.00E+00)$\approx$}\unboldmath{} \\
$F_{5}$ & \textbf{0.00E+00 (0.00E+00)} & \boldmath{}\textbf{0.00E+00 (0.00E+00)$\approx$}\unboldmath{} & \boldmath{}\textbf{0.00E+00 (0.00E+00)$\approx$}\unboldmath{} & \boldmath{}\textbf{0.00E+00 (0.00E+00)$\approx$}\unboldmath{} & \boldmath{}\textbf{0.00E+00 (0.00E+00)$\approx$}\unboldmath{} \\
$F_{6}$ & 5.50E-04 (3.96E-04) & 7.53E-04 (5.52E-04)$\approx$ & 8.93E-02 (1.18E-01)$-$ & 4.37E-02 (5.41E-02)$-$ & \textbf{5.84E-05 (4.73E-05)$+$} \\
$F_{7}$ & \textbf{0.00E+00 (0.00E+00)} & \boldmath{}\textbf{0.00E+00 (0.00E+00)$\approx$}\unboldmath{} & \boldmath{}\textbf{0.00E+00 (0.00E+00)$\approx$}\unboldmath{} & \boldmath{}\textbf{0.00E+00 (0.00E+00)$\approx$}\unboldmath{} & \boldmath{}\textbf{0.00E+00 (0.00E+00)$\approx$}\unboldmath{} \\
$F_{8}$ & 5.52E-03 (4.41E-03) & 1.01E-04 (1.66E-04)$+$ & 3.96E-04 (4.23E-04)$+$ & 3.13E-01 (3.60E-01)$-$ & \textbf{1.26E-05 (1.56E-05)$+$} \\
$F_{9}$ & \textbf{3.36E+00 (1.77E+01)} & 1.86E+02 (0.00E+00)$-$ & 2.29E+02 (3.18E-14)$-$ & 8.03E+01 (1.08E+02)$-$ & 2.23E+02 (1.31E+01)$-$ \\
$F_{10}$ & \textbf{0.00E+00 (0.00E+00)} & 1.00E+02 (0.00E+00)$-$ & 1.00E+02 (0.00E+00)$-$ & 1.56E-02 (3.12E-02)$-$ & \boldmath{}\textbf{0.00E+00 (0.00E+00)$\approx$}\unboldmath{} \\
$F_{11}$ & \textbf{0.00E+00 (0.00E+00)} & \boldmath{}\textbf{0.00E+00 (0.00E+00)$\approx$}\unboldmath{} & \boldmath{}\textbf{0.00E+00 (0.00E+00)$\approx$}\unboldmath{} & \boldmath{}\textbf{0.00E+00 (0.00E+00)$\approx$}\unboldmath{} & \boldmath{}\textbf{0.00E+00 (0.00E+00)$\approx$}\unboldmath{} \\
$F_{12}$ & \textbf{1.39E+02 (4.63E+01)} & 1.48E+02 (5.98E+00)$-$ & 1.65E+02 (0.00E+00)$-$ & 1.62E+02 (2.15E+00)$-$ & 1.59E+02 (0.00E+00)$-$ \\
\midrule
$+$ / $\approx$ / $-$ & --    & 1/6/5 & 1/5/6 & 0/6/6 & 2/8/2 \\
\bottomrule
\end{tabular}%

\footnotesize
\textsuperscript{*} The Wilcoxon rank-sum tests (with a significance level of 0.05) were conducted between MetaDE and each algorithm individually.
The final row displays the number of problems where the corresponding algorithm performs statistically better ($+$),  similar ($\thickapprox$), or worse ($-$) compared to MetaDE.\\


\label{tab:vsCECTop 10D}%
\end{table*}%

% Table generated by Excel2LaTeX from sheet 'Sheet1'
\begin{table*}[htbp]
  \centering
  
  \caption{{Comparisons between MetaDE and the top 4 algorithms from CEC2022 Competition (20D). 
The mean and standard deviation (in parentheses) of the results over multiple runs are displayed in pairs. 
Results with the best mean values are highlighted.
  }
  }
  %\resizebox{\linewidth}{!}{
  %       \renewcommand{\arraystretch}{1}
 %\renewcommand{\tabcolsep}{3pt}
% Table generated by Excel2LaTeX from sheet 'Sheet1'
\footnotesize
% Table generated by Excel2LaTeX from sheet 'Exp 7 vs CECtop'
\begin{tabular}{cccccc}
\toprule
Func  & MetaDE & EA4eig & NL-SHADE-LBC & NL-SHADE-RSP & S-LSHADE-DP \\
\midrule
$F_{1}$ & \textbf{0.00E+00 (0.00E+00)} & \boldmath{}\textbf{0.00E+00 (0.00E+00)$\approx$}\unboldmath{} & \boldmath{}\textbf{0.00E+00 (0.00E+00)$\approx$}\unboldmath{} & \boldmath{}\textbf{0.00E+00 (0.00E+00)$\approx$}\unboldmath{} & \boldmath{}\textbf{0.00E+00 (0.00E+00)$\approx$}\unboldmath{} \\
$F_{2}$ & 3.83E-04 (2.10E-03) & \textbf{0.00E+00 (0.00E+00)$+$} & 4.91E+01 (0.00E+00)$-$ & \textbf{0.00E+00 (0.00E+00)$+$} & \textbf{0.00E+00 (0.00E+00)$+$} \\
$F_{3}$ & \textbf{0.00E+00 (0.00E+00)} & \boldmath{}\textbf{0.00E+00 (0.00E+00)$\approx$}\unboldmath{} & \boldmath{}\textbf{0.00E+00 (0.00E+00)$\approx$}\unboldmath{} & \boldmath{}\textbf{0.00E+00 (0.00E+00)$\approx$}\unboldmath{} & \boldmath{}\textbf{0.00E+00 (0.00E+00)$\approx$}\unboldmath{} \\
$F_{4}$ & 1.96E+00 (7.76E-01) & 7.36E+00 (2.06E+00)$-$ & \boldmath{}\textbf{1.59E+00 (5.45E-01)$\approx$}\unboldmath{} & 1.07E+02 (1.54E+02)$-$ & 3.20E+00 (1.94E+00)$-$ \\
$F_{5}$ & \textbf{0.00E+00 (0.00E+00)} & \boldmath{}\textbf{0.00E+00 (0.00E+00)$\approx$}\unboldmath{} & \boldmath{}\textbf{0.00E+00 (0.00E+00)$\approx$}\unboldmath{} & 2.27E-01 (4.54E-01)$-$ & \boldmath{}\textbf{0.00E+00 (0.00E+00)$\approx$}\unboldmath{} \\
$F_{6}$ & \textbf{1.38E-01 (5.56E-02)} & 2.54E-01 (4.28E-01)$-$ & 3.06E-01 (2.01E-01)$-$ & 2.08E-01 (9.78E-02)$-$ & 5.02E-01 (5.34E-01)$-$ \\
$F_{7}$ & 8.42E-02 (1.01E-01) & 1.37E+00 (1.10E+00)$-$ & \boldmath{}\textbf{6.24E-02 (1.40E-01)$\approx$}\unboldmath{} & 1.28E+00 (1.95E+00)$-$ & 9.83E-01 (8.12E-01)$-$ \\
$F_{8}$ & 2.66E+00 (3.83E+00) & 2.02E+01 (1.28E-01)$-$ & \textbf{1.01E-01 (1.41E-01)$+$} & 1.99E+01 (4.97E-01)$-$ & 2.30E-01 (1.82E-01)$+$ \\
$F_{9}$ & \textbf{1.32E+02 (3.43E+01)} & 1.65E+02 (0.00E+00)$-$ & 1.81E+02 (0.00E+00)$-$ & 1.81E+02 (0.00E+00)$-$ & 1.81E+02 (0.00E+00)$-$ \\
$F_{10}$ & \textbf{0.00E+00 (0.00E+00)} & 1.23E+02 (5.12E+01)$-$ & 1.00E+02 (9.27E-03)$-$ & \boldmath{}\textbf{0.00E+00 (0.00E+00)$\approx$}\unboldmath{} & \boldmath{}\textbf{0.00E+00 (0.00E+00)$\approx$}\unboldmath{} \\
$F_{11}$ & 1.74E-03 (7.97E-03) & 3.20E+02 (4.47E+01)$-$ & 3.00E+02 (0.00E+00)$-$ & \textbf{0.00E+00 (0.00E+00)$+$} & \textbf{0.00E+00 (0.00E+00)$+$} \\
$F_{12}$ & 2.29E+02 (9.70E-01) & \textbf{2.00E+02 (2.04E-04)$+$} & 2.37E+02 (3.17E+00)$-$ & 2.34E+02 (1.46E+00)$-$ & 2.34E+02 (4.51E+00)$-$ \\
\midrule
$+$ / $\approx$ / $-$ & --    & 2/3/7 & 1/5/6 & 2/3/7 & 3/4/5 \\
\bottomrule
\end{tabular}%

\footnotesize
\textsuperscript{*} The Wilcoxon rank-sum tests (with a significance level of 0.05) were conducted between MetaDE and each algorithm individually.
The final row displays the number of problems where the corresponding algorithm performs statistically better ($+$),  similar ($\thickapprox$), or worse ($-$) compared to MetaDE.\\

\label{tab:vsCECTop 20D}%
\end{table*}%



To further assess the performance of MetaDE, we compare it with the top 4 algorithms from the CEC2022 Competition on Single Objective Bound Constrained Numerical Optimization\footnote{\url{https://github.com/P-N-Suganthan/2022-SO-BO}}.
For each algorithm, we set equal FEs as achieved by MetaDE within 60 seconds (refer to Table~\ref{tab:FEs} for details).

The top 4 algorithms from the CEC2022 Competition are {EA4eig}~\cite{EA4eig}, {NL-SHADE-LBC}~\cite{NL-SHADE-LBC}, {NL-SHADE-RSP-MID}~\cite{NL-SHADE-RSP}, and {S-LSHADE-DP}~\cite{S_LSHADE_DP}:
\begin{itemize}
  \item {EA4eig} combines the strengths of four evolutionary algorithms (CMA-ES, CoBiDE, an adaptive variant of jSO, and IDE) using Eigen crossover.
  \item {NL-SHADE-LBC} is a dynamic DE variant that integrates linear bias changes for parameter adaptation, repeated point generation to handle boundary constraints, non-linear population size reduction, and a selective pressure mechanism.
  \item {NL-SHADE-RSP-MID} is an advanced version of NL-SHADE-RSP, which estimates the optimum using the population midpoint, incorporates a restart mechanism, and improves boundary constraint handling.
  \item {S-LSHADE-DP} focuses on maintaining population diversity through dynamic perturbation, adjusting noise intensity to enhance exploration.
\end{itemize}





The experimental results are summarized in Tables \ref{tab:vsCECTop 10D} and \ref{tab:vsCECTop 20D}.
On 10D problems, MetaDE outperforms EA4eig, NL-SHADE-LBC, and NL-SHADE-RSP, while achieving comparable performance to S-LSHADE-DP. 
On 20D problems, MetaDE consistently outperforms the four algorithms.
An additional noteworthy observation is that S-LSHADE-DP exhibits promising performance under a large number of FEs.
}

\subsection{Investigation of Optimal DE Variants}\label{section_Optimal Parameter Analysis}


\begin{table}[h]
  \centering
  \caption{Optimal DE variants obtained by MetaDE on each problem of the CEC2022 benchmark suite. FDC and RIE are two fitness landscape characteristics that measure the difficulty and ruggedness of the problem.}
% Table generated by Excel2LaTeX from sheet 'Exp4 param'
\resizebox{\columnwidth}{!}{
\begin{tabular}{cccccccc}
\toprule
\multicolumn{2}{c}{Problem} & F     & CR    & \multicolumn{2}{c}{Strategy} & FDC & RIE \\
\midrule
\multirow{4}[2]{*}{10D} & $F_{6}$ & 0.70  & 0.99  & \multicolumn{2}{c}{\texttt{rand-to-pbest/1/arith}} & 0.61  & 0.81  \\
      & $F_{8}$ & 0.51  & 0.44  & \multicolumn{2}{c}{\texttt{pbest-to-best/1/bin}} & 0.27  & 0.62  \\
      & $F_{9}$ & 0.02  & 0.03  & \multicolumn{2}{c}{\texttt{current/2/bin}} & 0.08  & 0.82  \\
      & $F_{12}$ & 0.16  & 0.00  & \multicolumn{2}{c}{\texttt{current-to-best/4/bin}} & -0.15  & 0.78  \\
\midrule
\multirow{6}[2]{*}{20D} & $F_{4}$ & 0.13  & 0.71  & \multicolumn{2}{c}{\texttt{rand-to-best/3/bin}} & 0.90  & 0.79  \\
      & $F_{6}$ & 0.67  & 0.99  & \multicolumn{2}{c}{\texttt{pbest-to-rand/1/bin}} & 0.48  & 0.80  \\
      & $F_{7}$ & 0.27  & 0.93  & \multicolumn{2}{c}{\texttt{rand/2/bin}} & 0.26  & 0.78  \\
      & $F_{8}$ & 0.65  & 0.00  & \multicolumn{2}{c}{\texttt{pbest/1/exp}} & 0.12  & 0.40  \\
      & $F_{9}$ & 0.06  & 0.00  & \multicolumn{2}{c}{\texttt{current/2/bin}} & -0.17  & 0.84  \\
      & $F_{12}$ & 0.33  & 0.44  & \multicolumn{2}{c}{\texttt{rand-to-best/2/bin}} & -0.16  & 0.85  \\
\bottomrule
\end{tabular}%
}
\label{tab:optimal_param}
\end{table}%


This part provides an in-depth examination of the optimal DE variants obtained by MetaDE in Section~\ref{sec:expereiment_time}, as summarized in Table \ref{tab:optimal_param}.
The optimal parameters correspond to the best individual in the final population of MetaDE.
The table only displays the optimal parameters for the ten listed problems, as the remaining problems are relatively simpler, with numerous DE variants capable of locating the optimal solutions of the problems. Furthermore, the optimal parameters presented in the table represent the best results of MetaDE derived from the finest run out of 31 independent trials.



All the problems in the table are characterized by both multimodality and non-separability.
Additionally, to further depict the characteristics of the problems' fitness landscapes, we computed both the fitness distance correlation (FDC) \cite{FDC} and the ruggedness of information entropy (RIE) \cite{RIE}; the former measures the complexity (difficulty) of the problems, while the latter characterizes the ruggedness of the landscape.

Analyzing the obtained data, it is evident that no single set of parameters or strategies consistently excels across all problems.
Parameters such as \(F\) and \(CR\) exhibit variability across problems without adhering to a specific trend.
Similarly, the selection of base vectors ($bl$ and $br$) does not show a uniform preference either.
Regarding the fitness landscape characteristics of each problem, the selection of parameters exhibits distinct patterns.
The FDC indicates problem complexity; with simpler problems (higher FDC), such as 10-dimensional $F_6$, $F_8$ and 20-dimensional $F_4$, $F_6$, $F_7$, a larger \(CR\) value is favored. Conversely, smaller \(CR\) values are chosen for problems with lower FDC. A larger \(CR\) tends to facilitate convergence, whereas a \(CR\) close to 0 leads to offspring that change incrementally, dimension by dimension. However, the other characteristic, RIE, does not seem to have a clear association with parameter choices.
The optimal strategies for identical problems across different dimensions exhibit closeness, with $F_8$, $F_9$, and $F_{12}$ demonstrating notably parallel strategies between their 10D and 20D problems.
In terms of crossover strategies ($cs$), it seems to have a preference for binomial crossover. This aligns with the traditional DE configurations.

These observations align with the No Free Lunch (NFL) theorem \cite{NFL}, thus underscoring the importance of distinct optimization strategies tailored for diverse problems.
Conventionally, the optimization strategies have oscillated between seeking a generalist set of parameters for broad applicability and a specialist set tailored for specific problems. However, the dynamic nature of optimization problems, where even minute changes like a different random seed can pivot the problem's dynamics, highlights the challenges of a generalist approach.
In contrast, MetaDE provides a simple yet effective approach, showing promising generality and adaptability.


\subsection{Application to Robot Control}\label{sec:expereiment_brax}
In this experiment, we demonstrate the extended application of MetaDE to robot control.
Specifically, we adopted the evolutionary reinforcement learning paradigm~\cite{ERL} as illustrated in Fig.~\ref{Figure_EvoRL}.
The experiment was conducted on Brax \cite{brax} for robotics simulations with GPU acceleration.

\begin{figure}[!h]
\centering
\includegraphics[scale=0.38]{EvoRL_Workflow.pdf}
\caption{Illustration of robot control via evolutionary reinforcement learning. The evolutionary algorithm optimizes the parameters of a population of candidate policy models for controlling the robotics behaviors. The simulation environment returns rewards achieved by the candidate policy models to the evolutionary algorithm as fitness values.}
\label{Figure_EvoRL}
\end{figure}

This experiment involved three robot control tasks: ``swimmer'', ``hopper'', and ``reacher''.
As summarized in Table \ref{tab:Neural network structures}, we adopted similar policy models for these three tasks, each consisting of a multilayer perceptron (MLP) with three fully connected layers, but with different input and output dimensions.
Consequently, the three policy models comprise 1410, 1539, and 1506 parameters for optimization respectively, where the optimization objective is to achieve maximum reward of each task.
MetaDE, vanilla DE \cite{DE1996}, SHADE~\cite{SHADE2013}, LSHADE-RSP~\cite{LSHADE-RSP2018}, EDEV~\cite{EDEV2018}, CSO \cite{CSO}\footnote{The competitive swarm optimizer (CSO) is a tailored PSO variant for large-scale optimization.}, and CMA-ES~\cite{CMAES} were applied as the optimizer respectively.

%The policy models  were initialized with identical random parameters.
The iteration count for PDE within MetaDE was set to 50, while other algorithms maintained a population size of 100.
Each algorithm was run independently 15 times.
Considering the time-intensive nature of the robotics simulations, we set 60 minutes as the termination condition for each run.


\begin{table}[htbp]
\centering
\caption{Neural network structure of the policy model for each robot control task}
\label{tab:Neural network structures}
\resizebox{\columnwidth}{!}{%
% Table generated by Excel2LaTeX from sheet 'Exp6 brax'
\begin{tabular}{cccccc}
\toprule
\textbf{Task} & \textbf{D} & \textbf{Input} & \textbf{Hidden Layers} &   \textbf{Output}    & \textbf{Overview of objectives} \\
\midrule
Hopper & 1539  & 11    & 32$\times$32 & 3     & balance and jump \\
Swimmer & 1410  & 8     & 32$\times$32 & 2     & maximizing movement \\
Reacher & 1506  & 11    & 32$\times$32 & 2     & precise reaching \\
\bottomrule
\end{tabular}%
}
\end{table}

\begin{figure}[!h]
\centering
\includegraphics[scale=0.3]{brax_all_convergence.pdf}
\caption{The reward curves achieved by MetaDE and peer evolutionary algorithms when applied to each robot control task. }
\label{Figure_brax_all_convergence}
\end{figure}

\begin{figure}[!h]
\centering
\includegraphics[width=\linewidth]{brax_distribution.pdf}
\caption{The fitness distribution of MetaDE's initial population when applied to each robot control task.}
\label{Figure_brax_distribution}
\end{figure}

As shown in Fig. \ref{Figure_brax_all_convergence}, it is evident that MetaDE achieves the best performance in the Swimmer tasks, while slightly outperformed by CMA-ES and CSO in the Hopper and Reacher task.
An interesting observation from the reward curves is that MetaDE almost reaches optimality nearly at the first generation and does not show further significant improvements thereafter.
To elucidate this phenomenon, Fig.~\ref{Figure_brax_distribution} provides the fitness distribution of MetaDE's initial population, indicating that MetaDE harbored several individuals with considerably high fitness from the initial generation.
In other words, MetaDE was able to generate high-performance DE variants for these problems even by random sampling.
This can be attributed to the unique nature of neural network optimization.
As widely acknowledged, the neural network optimization typically features numerous plateaus in the fitness landscape, thus making it relatively easy to find one of the local optima.
MetaDE provides unbiased sampling of parameter settings for generating diverse DE variants.
Even without further evolution, some of the randomly sampled DE variants are very likely to reach the plateaus.
In contrast, the other algorithms are specially tailored with biases; in such large-scale optimization scenarios, the biases can be further amplified, thus making them ineffective.




\section{Conclusion}\label{section Conclusion}
In this paper, we introduced MetaDE, a method that leverages the strengths of DE not only to address optimization tasks but also to adapt and refine its own strategies. This meta-evolutionary approach demonstrates how DE can autonomously evolve its parameter configurations and strategies. 
Our experiments demonstrate that MetaDE has robust performance across various benchmarks, as well as the application in robot control through evolutionary reinforcement learning. 
Nevertheless, the study also emphasizes the complexity of finding universally optimal parameter configurations. The intricate balance between generalization and specialization remains a challenge, and MetaDE has shed light on further research into self-adapting algorithms. 
We anticipate that the insights gained from this work will inspire the development of more advanced meta-evolutionary approaches, pushing the boundaries of evolutionary optimization in even more complex and dynamic environments.





\footnotesize

% \bibliography{manuscript_references}
% !TEX program = pdflatex

\documentclass[journal]{IEEEtran}
\usepackage{lineno}
\modulolinenumbers[5]

%%% color some references
\usepackage{xpatch}
\makeatletter

\makeatother
\usepackage{bm}
\usepackage{array}
\usepackage{graphicx}
\usepackage{amsmath,amssymb,amsthm}
\usepackage{siunitx}
\usepackage{algpseudocode}
\usepackage{algorithmicx}
\usepackage{algorithm}
\usepackage{booktabs}
\usepackage{color}
\usepackage{changepage}
\usepackage{xr}
\usepackage{xr-hyper}
%\usepackage{geometry}%页面设置
\usepackage{graphicx}%图片设置
%\usepackage{subfig}%多个子图
\usepackage{subfigure}
\usepackage{caption}%注释设置
\usepackage{multirow}
\usepackage{float}
\usepackage{soul}
\usepackage[hidelinks]{hyperref}
\usepackage[numbers,sort&compress]{natbib}
\usepackage{bigstrut} %表格大竖线
\usepackage[table]{xcolor} %表格单元格颜色
\usepackage{enumitem} %enumerate标签样式\usepackage{listings}
\usepackage{listings} %listing代码
\usepackage[resetlabels]{multibib}
\newcites{supp}{Supplement References}



\definecolor{dkgreen}{rgb}{0,0.5,0}
\definecolor{gray}{rgb}{0.5,0.5,0.5}
\definecolor{mauve}{rgb}{0.58,0,0.82}
\definecolor{dkblue}{rgb}{0,0,0.6}

 % English theorem environment
 \newtheorem{theorem}{Theorem}
 \newtheorem{lemma}[theorem]{Lemma}
 \newtheorem{proposition}[theorem]{Proposition}
 \newtheorem*{corollary}{Corollary of Theorems 1 and 2}
 \newtheorem{definition}{Definition}
 \newtheorem{remark}{Remark}
 \newtheorem{example}{Example}
 \newenvironment{solution}{\begin{proof}[Solution]}{\end{proof}}

\renewcommand{\algorithmicrequire}{\textbf{Input:}}
\renewcommand{\algorithmicensure}{\textbf{Output:}}

\AtBeginDocument{%
 \abovedisplayskip=5pt plus 4pt minus 2pt
 \abovedisplayshortskip=5pt plus 4pt minus 4pt
 \belowdisplayskip=5pt plus 4pt minus 2pt
 \belowdisplayshortskip=5pt plus 4pt minus 4pt
}

\ifCLASSINFOpdf
\else
\fi

\hyphenation{op-tical net-works semi-conduc-tor}

\bibliographystyle{IEEEtran}

\begin{document}
\captionsetup{font={footnotesize}}
\captionsetup[table]{labelformat=simple, labelsep=newline, textfont=sc, justification=centering}
% paper title
% Titles are generally capitalized except for words such as a, an, and, as,
% at, but, by, for, in, nor, of, on, or, the, to and up, which are usually
% not capitalized unless they are the first or last word of the title.
% Linebreaks \\ can be used within to get better formatting as desired.
% Do not put math or special symbols in the title.
\title{MetaDE: Evolving Differential Evolution by Differential Evolution}
%
%
% author names and IEEE memberships
% note positions of commas and nonbreaking spaces ( ~ ) LaTeX will not break
% a structure at a ~ so this keeps an author's name from being broken across
% two lines.
% use \thanks{} to gain access to the first footnote area
% a separate \thanks must be used for each paragraph as LaTeX2e's \thanks
% was not built to handle multiple paragraphs
%

\author{Minyang Chen, Chenchen Feng,
        and Ran Cheng

        \thanks{
        Minyang Chen was with the Department of Computer Science and Engineering, Southern University of Science and Technology, Shenzhen 518055, China. E-mail: cmy1223605455@gmail.com. }
        \thanks{
        Chenchen Feng is with the Department of Computer Science and Engineering, Southern University of Science and Technology, Shenzhen 518055, China. E-mail: chenchenfengcn@gmail.com. 
        }
        \thanks{
       Ran Cheng is with the Department of Data Science and Artificial Intelligence, and the Department of Computing, The Hong Kong Polytechnic University, Hong Kong SAR, China. E-mail: ranchengcn@gmail.com. (\emph{Corresponding author: Ran Cheng})
        }
        }% <-this % stops a space


% The paper headers
\markboth{Bare Demo of IEEEtran.cls for IEEE Journals}%
{Shell \MakeLowercase{\textit{et al.}}: Bare Demo of IEEEtran.cls for IEEE Journals}
% The only time the second header will appear is for the odd numbered pages
% after the title page when using the twoside option.

% *** Note that you probably will NOT want to include the author's ***
% *** name in the headers of peer review papers.                   ***
% You can use \ifCLASSOPTIONpeerreview for conditional compilation here if
% you desire.


% If you want to put a publisher's ID mark on the page you can do it like
% this:
%\IEEEpubid{0000--0000/00\$00.00~\copyright~2015 IEEE}
% Remember, if you use this you must call \IEEEpubidadjcol in the second
% column for its text to clear the IEEEpubid mark.



% use for special paper notices
%\IEEEspecialpapernotice{(Invited Paper)}

% make the title area
\maketitle

% As a general rule, do not put math, special symbols or citations
% in the abstract or keywords.
\begin{abstract}
As a cornerstone in the Evolutionary Computation (EC) domain, Differential Evolution (DE) is known for its simplicity and effectiveness in handling challenging black-box optimization problems.
While the advantages of DE are well-recognized, achieving peak performance heavily depends on its hyperparameters such as the mutation factor, crossover probability, and the selection of specific DE strategies.
Traditional approaches to this hyperparameter dilemma have leaned towards parameter tuning or adaptive mechanisms.
However, identifying the optimal settings tailored for specific problems remains a persistent challenge.
In response, we introduce MetaDE, an approach that evolves DE's intrinsic hyperparameters and strategies using DE itself at a meta-level.
A pivotal aspect of MetaDE is a specialized parameterization technique, which endows it with the capability to dynamically modify DE's parameters and strategies throughout the evolutionary process.
To augment computational efficiency, MetaDE incorporates a design that leverages parallel processing through a GPU-accelerated computing framework.
Within such a framework, DE is not just a solver but also an optimizer for its own configurations, thus streamlining the process of hyperparameter optimization and problem-solving into a cohesive and automated workflow.
Extensive evaluations on the CEC2022 benchmark suite demonstrate MetaDE's promising performance.
Moreover, when applied to robot control via evolutionary reinforcement learning, MetaDE also demonstrates promising performance.
The source code of MetaDE is publicly accessible at: \url{https://github.com/EMI-Group/metade}.
\end{abstract}



% Note that keywords are not normally used for peerreview papers.
\begin{IEEEkeywords}
Differential Evolution, Meta Evolutionary Algorithm, GPU Computing
\end{IEEEkeywords}


% For peer review papers, you can put extra information on the cover
% page as needed:
% \ifCLASSOPTIONpeerreview
% \begin{center} \bfseries EDICS Category: 3-BBND \end{center}
% \fi
%
% For peerreview papers, this IEEEtran command inserts a page break and
% creates the second title. It will be ignored for other modes.
\IEEEpeerreviewmaketitle



\section{Introduction}
\IEEEPARstart{T}{he} Differential Evolution (DE) \cite{DE1996,DEcontest1996,DEusage1996,DE1997} algorithm, introduced by Storn and Price in 1995, has emerged as a cornerstone in the realm of evolutionary computation (EC) for its prowess in addressing complex optimization problems across diverse domains of science and engineering.
DE's comparative advantage over other evolutionary algorithms is evident in its streamlined design, robust performance, and ease of implementation.
Notably, with just three primary control parameters, i.e., scaling factor, crossover rate, and population size, DE operates efficiently.
This minimalistic design, paired with a lower algorithmic complexity, positions DE as an ideal candidate for large-scale optimization problems.
Its influential role in the optimization community is further cemented by its extensive research attention and successful applications over the past decades \cite{DEsurvey2011, DEsurvey2016, DEpapersurvey2020}, with DE and its derivatives often securing top positions in the IEEE Congress on Evolutionary Computation (CEC) competitions.



Despite the well recognized performance, DE is not without limitations.
Particularly, some studies indicate that DE's optimization process may stagnate if it fails to generate offspring solutions superior to their parents \cite{DEStagnation, neriDEsurvey}.
To avert this stagnation, selecting an appropriate parameter configuration to enhance DE's search capabilities becomes crucial.

However, the No Free Lunch (NFL) theorem \cite{NFL} suggests that a universally optimal parameter configuration is unattainable.
For example, while a higher mutation factor may aid in escaping local optima, a lower crossover probability might be preferable for problems with separability characteristics.

To address the intricate challenge of parameter configuration in DE, researchers often gravitate towards two predominant strategies: \emph{parameter control} and \emph{parameter tuning} \cite{param1999,paramTun2012,paramTun2020}.
Parameter control is a dynamic approach wherein the algorithm's parameters are adjusted on-the-fly during its execution.
This adaptability allows the algorithm to respond to the evolving characteristics of the problem landscape, enhancing its chance of finding optimal or near-optimal solutions.
Notably, DE has incorporated this strategy in several of its variants.
For instance, jDE \cite{jDE2006} adjusts the mutation factor and crossover rate during the run, while SaDE \cite{SaDE2008} dynamically chooses a mutation strategy based on its past success rates. Similarly, JaDE \cite{JADE2009} and CoDE \cite{CoDE2011} employ adaptive mechanisms to modify control parameters and mutation strategies, respectively.

In contrast, parameter tuning is a more static methodology, wherein the optimal configuration is established prior to the algorithm's initiation.
It aims to discover a parameter set that consistently demonstrates robust performance across various runs and problem instances.
Despite its potential for reliable outcomes, parameter tuning is known for its computational intensity, often necessitating dedicated optimization efforts or experimental designs to identify the optimal parameters, which may explain its limited exploration in the field.
Viewed as an optimization challenge, parameter tuning is also referred to as meta-optimization \cite{metaEAPhD2010}.
This perspective gave rise to \emph{MetaEA}, which optimizes the parameters of an EA using another EA.

Despite MetaEA's methodological elegance and simplicity, it confronts the significant challenge of depending on extensive function evaluations.
Fortunately, the inherent parallelism within MetaEA, across both meta-level and base-level populations, renders it particularly amenable to parallel computing environments.
However, a notable disparity exists between methodological innovations and the availability of advanced computational infrastructures, thus limiting MetaEA's potential due to the lack of advanced hardware accelerations such as GPUs.
To bridge this gap, we introduce the \emph{MetaDE} approach, which embodies the MetaEA paradigm by employing DE in a meta-level to guide the evolution of a specially tailored Parameterized Differential Evolution (PDE).

Designed with adaptability in mind, PDE can flexibly adjust its parameters and strategies, paving the way for a wide range of DE configurations.
As PDE interacts with the optimization problem at hand, the meta-level DE observes and refines PDE's settings to better align with the problem's characteristics.
Amplifying the efficiency of this nested optimization approach, MetaDE is integrated with a GPU-accelerated EC framework, thus weaving together parameter refinement and direct problem-solving into a seamless end-to-end approach to black-box optimization.
In summary, our main contributions are as follows.

\begin{itemize}
\item \textbf{Parameterized Differential Evolution:}
We have introduced Parameterized Differential Evolution (PDE), a variant of DE with augmented parameterization.
Unlike traditional DE algorithms that come with fixed mutation and crossover strategies, PDE’s architecture offers users the flexibility to adjust these parameters and strategies to fit the problem at hand.
This design not only allows for the creation of diverse DE configurations tailored for specific challenges but also ensures efficient computation.
To achieve this, all core operations of PDE, including mutation, crossover, and evaluation, have been optimized for parallel execution to harness advancement of GPU acceleration.

\item \textbf{MetaDE:}
Building on the MetaEA paradigm, we have designed the MetaDE approach.
Specifically, MetaDE employs a meta-level DE as an \texttt{evolver} to iteratively refine PDE's hyperparameters, which is guided by performance feedback from multiple PDE instances acting as \texttt{executors}.
This continuous optimization ensures PDE's configurations remain aligned with evolving problem landscapes.
Moreover, we have incorporated several specialized methods to further enhance the performance of MetaDE.



\item \textbf{GPU-accelerated Implementation:}
Breaking away from the limitations of conventional parameter tuning, we integrate MetaDE with a GPU-accelerated computing framework
 -- EvoX~\cite{evox}, which enhances MetaDE's computational prowess for facilitating swifter evaluations and algorithmic refinements.
With this specialized implementation, MetaDE provides an efficient and automated end-to-end approach to black-box optimization.
\end{itemize}



The subsequent sections are organized as follows. Section \ref{section_Preliminary} presents some preliminary knowledge for this work.
Section \ref{section_The proposed metade} elucidates the intricacies of the proposed approach, including PDE and the MetaDE.
Section \ref{section_Experimental_study} showcases the experimental results.
Finally, Section \ref{section Conclusion} wraps up the discourse and points towards avenues for future work.

\section{Preliminaries}\label{section_Preliminary}

\subsection{DE and its Parameter Adaption}\label{subsection_DE and Parameter adaption}
\subsubsection{Overview of DE}
As a typical EC algorithm, DE's essence lies in its differential mutation mechanism that drives the evolution of a population.
The operational cycle of DE unfolds iteratively, with each iteration embodying specific phases, as elaborated in Algorithm \ref{Alg_DE}:
\begin{enumerate}[label=\arabic*.]
  \item \textbf{Initialization} (Line 1):
  The algorithm initializes a set of potential solutions.
  Each of these solutions, representing vectors of decision variables, is randomly generated within the search space boundaries.
  \item \textbf{Mutation} (Lines 6-7):
  Each solution undergoes mutation to produce a mutant vector.
  This mutation process involves combinations of different individuals to form the mutant vectors.
  \item \textbf{Crossover} (Line 8):
  The crossover operation interchanges components between mutants and the original solutions to generate a trial vector.
  \item \textbf{Selection} (Lines 10-12): The trial vector competes against the original solution based on fitness, with the better solution progressing to the next generation.
\end{enumerate}


DE progresses through cycles of mutation, crossover, and selection, persisting until it encounters a termination criterion.
This could manifest as either reaching a predefined number of generations or achieving a target fitness threshold.
The algorithm's adaptability allows for the spawning of myriad DE variants by merely tweaking its mutation and crossover operations.
Specifically, DE variants follow a unified naming convention: \texttt{DE/x/y/z}, where \texttt{x} identifies the base vector used for mutation, \texttt{y} quantifies the number of difference involved, and \texttt{z} typifies the crossover method employed.
For example, the DE variant as presented in Algorithm \ref{Alg_DE} is named as \texttt{DE/rand/1/bin}.

{\linespread{1.1}
\begin{algorithm}
\small
\caption{DE}\label{Alg_DE}
\begin{algorithmic}[1]
  \Require {$D$, $NP$, $F$, $CR$, $G_{max}$}
  \State Initialize population $\mathbf{X} = \{\mathbf{x}_1, \mathbf{x}_2, \dots, \mathbf{x}_{\scalebox{0.5}{$\textit{NP}$}}\}$
  \State Evaluate the fitness of each individual in the population
  \State $g = 0$
  \While{$g \leq G_{max}$}
    \For{$i = 1$ to $NP$}
      \State Randomly select $\mathbf{x}_{r_1}$, $\mathbf{x}_{r_2}$, and $\mathbf{x}_{r_3}$ from $\mathbf{X}$,
      \Statex \qquad \quad such that $r_1 \neq r_2 \neq r_3 \neq i$
      \State Compute the mutant vector: $\mathbf{v}_i = \mathbf{x}_{r_1} + F \cdot (\mathbf{x}_{r_2} - \mathbf{x}_{r_3})$
      \State Perform crossover for each variable between $\mathbf{x}_i$ and $\mathbf{v}_i$:
      \begin{align*}
        \qquad \quad u_{i,j}=\begin{cases}
          v_{i,j},\ \text{if } \text{rand}(0, 1) \leq CR \text{ or } j = \text{randint}(1, D) \\
          x_{i,j},\ \text{otherwise}
        \end{cases}
      \end{align*}
      \State Evaluate the fitness of $\mathbf{u}_i$
      \If{$\textrm{f}(\mathbf{u}_i) \leq \textrm{f}(\mathbf{x}_i)$}
        \State Replace $\mathbf{x}_i$ with $\mathbf{u}_i$ in the population
      \EndIf
    \EndFor
    \State $g = g + 1$
  \EndWhile
  \State\Return the best fitness
\end{algorithmic}
\end{algorithm}
{\linespread{1}

\subsubsection{Parameter Modulation in DE}
DE employs a unique mutation mechanism, which adapts to the problem's natural scaling.
By adjusting the mutation step's size and orientation to the objective function landscape, DE embraces the \emph{contour matching principle} \cite{DEbook2006}, which promotes basin-to-basin transfer for enhancing the convergence of the algorithm.

At the core of DE's mutation is the scaling factor \( F \).
This factor not only determines the mutation's intensity but also governs its trajectory and ability to bypass local optima.
Commonly, \( F \) is set within the $[0.5, 1]$ interval, with a starting point often at 0.5.
While values outside the $[0.4, 1]$ range can sometimes yield good results, an \( F \) greater than 1 tends to slow convergence.
Conversely, values up to 1 generally promise swifter and more stable outcomes \cite{EPSDE2011}.
Nonetheless, to deter settling at suboptimal solutions too early, \( F \) should be adequately elevated.

Parallel to mutation, DE incorporates a uniform crossover operator, which is often labeled as discrete recombination or binomial crossover in the GA lexicon.
The crossover constant \( CR \) also plays a pivotal role, which determines the proportion of decision variables to be exchanged during the generation of offspring.
A low value for \( CR \) ensures only a small portion of decision variables are modified per iteration, thus leading to axis-aligned search steps.
As \( CR \) increases closer to 1, offspring tend to increasingly reflect their mutant parent, thereby curbing the generation of orthogonal search steps \cite{DEsurvey2011}.

For classical DE configurations, such as \texttt{DE/rand/1/bin}, rotational invariance is achieved only when \( CR \) is maxed out at 1.
Here, the crossover becomes wholly vector-driven, and offspring effectively mirror their mutants.
However, the optimal \( CR \) is intrinsically problem-dependent.
Empirical studies recommend a \( CR \) setting within the $[0, 0.2]$ range for problems characterized by separable decision variables.
Conversely, for problems with non-separable decision variables, a \( CR \) in the proximity of $[0.9, 1]$ is more effective \cite{DEsurvey2011}.

The adaptability of DE is evident in its wide spectrum of variants, each distinct in its mutation and crossover strategies with delicate parameter modulations.
In the following, we will detail seven mutation strategies and three crossover strategies, all of which are widely-recognized in state-of-the-art DE variants.
Here, the subscript notation in \( \textbf{x} \) specifies the individual selection technique.
For instance, \( \textbf{x}_r \) and \( \textbf{x}_{best} \) correspond to randomly selected and best-performing individuals respectively, whereas
\( \textbf{x}_i \) represents the currently evaluated individual.

\textbf{Mutation Strategies}:

\begin{enumerate}[label=\arabic*.]
  \item \texttt{DE/rand/1}:
        \begin{eqnarray}\label{equ_mutation rand}
        \begin{aligned}
        \mathbf{v}_i=\mathbf{x}_{r_1}+F \cdot\left(\mathbf{x}_{r_2}-\mathbf{x}_{r_3}\right).
        \end{aligned}
        \end{eqnarray}

  \item \texttt{DE/best/1}:
        \begin{eqnarray}\label{equ_mutation best}
        \begin{aligned}
        \mathbf{v}_i=\mathbf{x}_{\text {best }}+F \cdot\left(\mathbf{x}_{r_1}-\mathbf{x}_{r_2}\right).
        \end{aligned}
        \end{eqnarray}

  \item \texttt{DE/rand/2}:
        \begin{eqnarray}\label{equ_mutation rand2}
        \begin{aligned}
        \mathbf{v}_i & =\mathbf{x}_{r_1}+F \cdot\left(\mathbf{x}_{r_2}-\mathbf{x}_{r_3}\right)+F \cdot\left(\mathbf{x}_{r_4}-\mathbf{x}_{r_5}\right).
        \end{aligned}
        \end{eqnarray}

  \item \texttt{DE/best/2}:
        \begin{eqnarray}\label{equ_mutation best2}
        \begin{aligned}
        \mathbf{v}_i & =\mathbf{x}_{\text {best}}+F \cdot\left(\mathbf{x}_{r_1}-\mathbf{x}_{r_2}\right)+F \cdot\left(\mathbf{x}_{r_3}-\mathbf{x}_{r_4}\right).
        \end{aligned}
        \end{eqnarray}

  \item \texttt{DE/current-to-best/1}:
        \begin{eqnarray}\label{equ_mutation current2best}
        \begin{aligned}
        \mathbf{v}_i =\mathbf{x}_i+F \cdot\left(\mathbf{x}_{\text {best }}-\mathbf{x}_i\right)+F \cdot\left(\mathbf{x}_{r_1}-\mathbf{x}_{r_2}\right).
        \end{aligned}
        \end{eqnarray}
    The above five classical mutation strategies, introduced by Storn and Price \cite{DEbook2006}, cater to various problem landscapes.
    For instance, the `rand' variants help maintain population diversity, while strategies using two differences typically produce more diverse offspring than those relying on a single difference.

  \item \texttt{DE/current-to-pbest/1}:
        \begin{eqnarray}\label{equ_mutation current2pbest}
        \begin{aligned}
        \mathbf{v}_i =\mathbf{x}_i+F \cdot\left(\mathbf{x}_{\text {pbest}}-\mathbf{x}_i\right)+F \cdot\left(\mathbf{x}_{r_1}-\mathbf{x}_{r_2}\right).
        \end{aligned}
        \end{eqnarray}
    This strategy originates from JaDE \cite{JADE2009}. $\mathbf{x}_{\text{pbest}}$ is randomly selected from the top \emph{p}\% of individuals in the population (typically the top 10\%) to strike a balance between exploration and exploitation.


  \item \texttt{DE/current-to-rand/1}:
        \begin{eqnarray}\label{equ_mutation current2rand}
        \begin{aligned}
        &\mathbf{u}_i =\mathbf{x}_i+K_i\cdot\left(\mathbf{x}_{r_1}-\mathbf{x}_i\right)+F \cdot\left(\mathbf{x}_{r_2}-\mathbf{x}_{r_3}\right).
        \end{aligned}
        \end{eqnarray}
    Here, \(K_i\) is a random number from \(U(0,1)\).
    This strategy, originally proposed in \cite{DEintro1999}, emphasizes rotational invariance. By bypassing the crossover phase, it directly yields the trial vector \(\mathbf{u}_i\). Thus, it is ideal for addressing non-separable rotation challenges and has been a cornerstone for multiple adaptive DE variations.

  \end{enumerate}

\textbf{Crossover Strategies:}

\begin{enumerate}[label=\arabic*.]
  \item Binomial Crossover:
    \begin{eqnarray}\label{equ_cross bin}
    \begin{aligned}
    u_{i, j}= \begin{cases}v_{i, j}, & \text { if } r \leq C R \text { or } j=j_{\mathrm{rand}} \\ x_{i, j}, & \text {otherwise},\end{cases}
    \end{aligned}
    \end{eqnarray}
    where \(j_{\mathrm{rand}}\) is a random integer between 1 and \( D \). This strategy is a cornerstone in DE.

  \item Exponential Crossover:
  \begin{eqnarray}\label{equ_cross exp}
    \begin{aligned}
    \small
    u_{i, j}= \begin{cases}v_{i, j}{ } & \text { if } j=\langle n\rangle_d,\langle n+1\rangle_d,...,\langle n+L-1\rangle_d \\ x_{i, j} & \text {otherwise},\end{cases}
    \end{aligned}
    \end{eqnarray}
    where \(\langle \rangle_d\) is a modulo operation with \(D\) and \(L\) representing the crossover length, following a censored geometric distribution with a limit of \(D\) and probability of \(CR\).
    By focusing on consecutive variables, this strategy excels in handling problems with contiguous variable dependencies.

  \item Arithmetic Recombination:
  \begin{eqnarray}\label{equ_cross arith}
    \begin{aligned}
    \mathbf{u}_i=\mathbf{x}_i + K_i\cdot(\mathbf{v}_i - \mathbf{x}_i),
    \end{aligned}
    \end{eqnarray}
    where \(K_i\) is a random value from \(U(0,1)\).
    Exhibiting rotational invariance, this strategy, when combined with the \texttt{DE/rand/1} mutation, results in the \texttt{DE/current-to-rand/1} strategy \cite{DEsurvey2011}, as described by:
    \begin{eqnarray}\label{equ_rand_1_arith}
    \begin{aligned}
    \mathbf{u}_i&=\mathbf{x}_i + K_i\cdot(\mathbf{v}_i - \mathbf{x}_i)\\
    &=\mathbf{x}_i + K_i(\mathbf{x}_{r_1}+F\cdot(\mathbf{x}_{r_2}-\mathbf{x}_{r_3}) - \mathbf{x}_i)\\
    &=\mathbf{x}_i + K_i(\mathbf{x}_{r_1} - \mathbf{x}_i)+ K_i\cdot F(\mathbf{x}_{r_2}-\mathbf{x}_{r_3}),
    \end{aligned}
    \end{eqnarray}
    which is equivalent to Eq. (\ref{equ_mutation current2rand}).

\end{enumerate}


\subsubsection{Adaptive DE}
The development of parameter adaption in DE has witnessed significant advancements over time, from initial endeavors in parameter adaptation to recent sophisticated methods that merge multiple strategies.
This subsection traces the chronological advancements, emphasizing the pivotal contributions and their respective impacts on adaptive DE.

The earliest phase in DE's adaption centered on the modification of the crossover rate \( CR \) .
Pioneering algorithms such as SPDE \cite{SPDE2003} incorporated \( CR \) within the parameter set of individuals, enabling its simultaneous evolution with the decision variables of the problem to be solved.
This strategy was further refined by SDE \cite{SDE2005}, which assigned \( CR \) for each individual based on a normal distribution. Subsequent research efforts shifted focus to the scaling factor \( F \).
In this context, DETVSF \cite{DETVSF2005} dynamically adjusted \( F \), fostering exploration during the algorithm's nascent stages and pivoting to exploitation in later iterations.
Building on this, FaDE \cite{FaDE2005} employed fuzzy logic controllers to optimize mutation and crossover parameters.

The DESAP \cite{DESAP2006} algorithm marked a significant paradigm shift by introducing self-adapting populations and encapsulating control parameters within individuals.
Successive contributions like jDE \cite{jDE2006}, SaDE \cite{SaDE2008}, and JaDE \cite{JADE2009} accentuated the significance of parameter encoding, integrated innovative mutation strategies, and emphasized archiving optimization trajectories using external repositories.
Further, EPSDE \cite{EPSDE2011} and CoDE \cite{CoDE2011} enhanced the offspring generation process, amalgamating multiple strategies with randomized parameters.

The contemporary landscape of adaptive DE is characterized by complex methodologies and refined strategies.
Algorithms such as SHADE \cite{SHADE2013} and LSHADE \cite{LSHADE2014} championed the utilization of success-history mechanisms and dynamic population size modifications.
Notable developments like ADE \cite{ADE2014} introduced a biphasic parameter adaptation mechanism.
The domain further expanded with algorithms like LSHADE-RSP \cite{LSHADE-RSP2018}, IMODE \cite{IMODE2020}, and LADE \cite{LADE2023}, emphasizing mechanisms such as selective pressure, the integration of multiple DE variants, and the automation of the learning process.

Undoubtedly, the adaptive DE domain has witnessed transformative growth, with each phase of its evolution contributing to its current sophistication.
However, despite these advancements, many adaptive strategies remain empirical and hinge on manual designs, while their effectiveness is not universally guaranteed.

\subsection{Distributed DE}
The integration of distributed (i.e., multi-population) strategies also significantly enhances the efficacy of DE. 
Leading this advancement, Weber \textit{et al.} conducted extensive research on scale factor interactions and mechanisms within a distributed DE framework \cite{weber2010study, weber2011study, weber2013study}, followed by ongoing developments along the pathway \cite{DEpapersurvey2020}.
For example, some works such as EDEV \cite{EDEV2018}, MPEDE \cite{MPEDE2015} and IMPEDE \cite{IMPEDE} adopted multi-population frameworks to ensemble various DE variants/operators,
while the other works  such as DDE-AMS \cite{DDE-AMS} and DDE-ARA \cite{DDE-ARA} employed multiple populations for adaptive resource allocations.

Despite the achievements, current implementations of distributed DE often focus predominantly on algorithmic improvements, while overlooking potential enhancements from advanced hardware accelerations such as GPU computing. 
Besides, the design of these distributed strategies often features intricate and rigid configurations that lack proper flexibility.



\subsection{MetaEAs}\label{subsection_Meta-EA}

Generally, the term \emph{meta} refers to a higher-level abstraction of an underlying concept, often characterized by its \emph{recursive} nature.
In the context of EC, inception of the Meta Evolutionary Algorithms (MetaEAs) can be traced back to the pioneering works of Mercer and Sampson \cite{metaplan1978} in the late 1970s.
Under the initiative termed \emph{meta-plan}, their pioneering efforts aimed at enhancing EA performance by optimizing its parameters through another EA.
Although sharing similarities with hyperheuristics \cite{Hyperheu2013,Hyperheu2020,NeriHyperspam}, a major difference distinguishes MetaEAs: while hyperheuristics often delve into selecting and fine-tuning a set of predefined algorithms, MetaEAs concentrate on the paradigm of refining the parameters of EAs by EAs.
Notably, MetaEAs are also akin to ensemble of algorithms, such as EDEV \cite{EDEV2018} and CoDE \cite{CoDE2011}, which amalgamate diverse algorithms to ascertain the most efficacious among them.


Advancing the meta-plan concept, MetaGA \cite{MetaGA1986} emerged as a significant milestone.
Here, a genetic algorithm (GA) was deployed to fine-tune six intrinsic control parameters, namely: population size, crossover rate, mutation rate, generation gap, scaling window, and selection strategy.
The efficacy of this approach was gauged using dual metrics: online and offline performance.

The evolution of the concept continued with MetaEP \cite{metaEP1991}, which offers a meta-level evolutionary programming (EP) that could concurrently evolve optimal parameter settings.
Another pivotal contribution was the Parameter Relevance Estimation and Value Calibration (REVAC) \cite{REVAC2007}, which served as a meta estimation of distribution algorithm (MetaEDA).
Utilizing a GA at its core, REVAC iteratively discerned promising parameter value distributions within the configuration space.

Innovations in the domain persisted with the Gender-based GA (GGA) \cite{GGA2009}, inspired by natural gender differentiation, and other notable methods like MetaCMAES \cite{metaCMAES2012}.
As articulated in the PhD thesis by Pedersen \cite{metaEAPhD2010}, a profound insight into MetaEA revealed that while contemporary optimizers endowed with adaptive behavioral parameters offered advantages, they were often eclipsed by streamlined optimizers under appropriate parameter tuning.
This thesis, which embraced DE as one of its optimization tools, employed the Local Unimodal Sampling (LUS) heuristic for tuning parameters such as \( NP \), \( F \), and \( CR \).

Culminating the discourse, the work in \cite{metaEAdistributed} demonstrated the scalability of MetaEAs by harnessing it within a large-scale distributed computing environment.
With the ($\mu$, $\lambda$)$-$ES steering the meta-level tuning, base-level algorithms like GA, ES, and DE were adeptly optimized.
For DE, parameters optimized encompassed \( NP \), mutation operator, \( F \), \( CR \), and \( PF \) (parameter for the \emph{either-or} strategy), enhancing MetaEAs' prowess in addressing intricate, large-scale optimization problems.
Recently, the MetaEA paradigm has also been employed for automated design of ensemble DE \cite{EDE}.

The field of MetaEAs has shown steady progress since its inception in the 1970s.
However, despite the achievements, the landscape of MetaEAs research still confronts certain limitations.
Notably, the research, while promising, has predominantly remained confined to smaller-scale implementations.
The anticipated leap to large-scale experiments, especially those that might benefit from GPU acceleration, remains largely uncharted. This underscores an imperative need for more extensive empirical validations and the exploration of contemporary computational resources to fully realize the potential of MetaEAs.

\subsection{GPU-accelerated EC Framework}\label{subsection_Meta-optimization}
To capitalize on the advancements of modern computing infrastructures, we have seamlessly integrated our proposed MetaDE with EvoX~\cite{evox}, a distributed GPU-accelerated computing framework for scalable EC.
This integration ensures that MetaDE enables efficient execution and optimization for large-scale evaluations.

The EvoX framework provides several distinctive features.
Primarily, it is designed for optimal performance across diverse distributed systems and is tailored to manage large-scale challenges.
Its user-friendly functional programming model simplifies the EC algorithm development process, reducing inherent complexities.
The framework cohesively integrates data streams and functional elements into a comprehensive workflow, underpinned by a sophisticated hierarchical state management system.
Moreover, EvoX features a rich library of EC algorithms, proficient in addressing a wide array of tasks, from black-box optimization to advanced areas such as deep neuroevolution and evolutionary reinforcement learning.


\section{Proposed Approach}\label{section_The proposed metade}
The foundational premise of MetaDE is to utilize a core DE algorithm to evolve an ensemble of parameterized DE variants.
Within this framework, the core DE, which tunes the parameters, is termed the \texttt{evolver}.
In contrast, each parameterized DE variant, which optimizes the problem at hand, is termed the \texttt{executor}.
This section commences by augmenting the parameterization of DE in a more general manner, such that DE is made \emph{evolvable} by spawning various DE variants by modulating the parameters.
Then, this section details the integration of proposed MetaDE within a meta-framework, together with a brief introduction to the GPU-accelerated implementation.

\subsection{Augmented Parameterization of DE}\label{section_PDE}
To make DE evolvable, this subsection introduces the Parameterized Differential Evolution (PDE), an extension of the standard DE designed to augment its flexibility through the parameterization of mutation and crossover strategies.
While PDE retains the foundational principles of standard DE, its distinctiveness lies in its capability to generate a multitude of strategies by modulating the parameters.

In the standard DE framework, tunability is constrained to the adjustments of the \( F \) and \( CR \) parameters, and strategies are bound by predefined rules.
To augment this limited flexibility, PDE introduces a more granular parameterization.
Building upon DE's notation \texttt{DE/x/y/z}, PDE encompasses six parameters: \( F \) (scale factor), \( CR \) (crossover rate), \( bl \) (base vector left), \( br \) (base vector right), \( dn \) (difference number), and \( cs \) (crossover scheme).
The combined roles of \( bl \) and \( br \) determine the base vector, leading to a strategy notation for PDE expressed as \texttt{DE/bl-to-br/dn/cs}.

Each parameter's nuances and the array of strategy combinations they enable are elaborated upon in the subsequent sections.
Fig. \ref{Figure_structure} provides a visual representation of these parameters, delineated within dashed boxes.


\begin{figure*}[!h]
\centering
\includegraphics[scale=0.6]{fix_structure.pdf}
\caption{
Parameter delineation of PDE and their respective domains.
PDE comprehensively parameterizes DE, endorsing unrestricted parameter and strategy modifications.
In this schema, \( F \) and \( CR \) are continuous parameters, whereas others are categorical.
The dashed-line boxes exhibit their specific value ranges.
The mutation function is derived from the base vector left, base vector right, and difference number parameters.
}
\label{Figure_structure}
\end{figure*}

\subsubsection{Scale Factor \( F \) and Crossover Rate \( CR \)}
The scale factor \( F \) and crossover rate \( CR \) serve as pivotal parameters in DE, both represented as real numbers.

The \( F \) parameter regulates the differential variation among population entities.
Elevated values induce exploratory search behaviors, while lower values encourage more exploitation.
Although Storn and Price originally identified [0, 2] as an effective domain for \( F \) \cite{DE1997}, contemporary DE variants deem \( F \leq 1 \) as more judicious \cite{JADE2009,CoDE2011,EPSDE2011,SHADE2013,LSHADE2014}.
Consequently, PDE constrains \( F \) within [0,1].

On the other hand, \( CR \) controls the recombination extent during crossover.
Higher \( CR \) make it more likely to try out new gene combinations from changes, while lower values keep the genes more stable and similar to the original.
As a rate (or probability), \( CR \) is confined to the [0, 1] interval.

\subsubsection{Augmented Parameterization of Mutation}
The distinctiveness of PDE lies in its ability to generate a multitude of mutation strategies by modulating parameters $bl, br$, and $dn$.
Parameters $bl$ (base vector left) and $br$ (base vector right) select one of four possible vectors:
\begin{itemize}
  \item  \texttt{rand}: A randomly chosen individual from the population.
  \item  \texttt{best}: The best individual in the population.
  \item  \texttt{pbest}: A random selection from the top \( p\% \) of individuals.
  \item  \texttt{current}: The present parent individual.
\end{itemize}
Parameter \( dn \), controls the number of differences in mutation, assumes values in the set \{1, 2, 3, 4\}.
Specifically, each difference \( \Delta \) captures the difference between two unique and randomly selected individuals from the population.
Therefore, the mutation formulation \texttt{DE/bl-to-br/dn} is:
\begin{align}\label{equ_PDE_mutation}
\mathbf{v} = \mathbf{x}_{bl} + F \cdot (\mathbf{x}_{br} - \mathbf{x}_{bl}) + F \cdot (\Delta_1 + ... +\Delta_{dn}).
\end{align}
In the implementation of PDE, \( bl \) and \( br \) are encoded as: 1: \texttt{rand}, 2: \texttt{best},  3: \texttt{pbest}, and   4: \texttt{current}.
In addition, if both \( bl \) and \( br \) assume identical values, the term $F \cdot (\mathbf{x}_{br} - \mathbf{x}_{bl})$ will disappear. The base vector takes the value of $\mathbf{x}_{bl}$ directly and the mutation strategy then becomes non-directional (e.g., \texttt{DE/rand/1}).

\subsubsection{Augmented Parameterization of Crossover}
Parameter \( cs \) defines the crossover strategies in PDE, comprising those elaborated in Section~\ref{subsection_DE and Parameter adaption}: binomial crossover, exponential crossover, and arithmetic recombination.
Specifically, \( cs \) is encoded as:   1: \texttt{bin}, 2: \texttt{exp}, and 3: \texttt{arith}, representing binomial crossover, exponential crossover, and arithmetic recombination respectively.

As a result, the augmented parameterizations for mutation and crossover, denoted as \texttt{DE/bl-to-br/dn/cs}, will give rise to a spectrum of 192 distinctive strategies in total.
This breadth allows for the encapsulation of mainstream strategies detailed in Section~\ref{subsection_DE and Parameter adaption}, as illustrated in Table \ref{tab_param encoding}.
When synergized with \( F \) and \( CR \), this culminates in a comprehensive parameter configuration landscape.


% Table generated by Excel2LaTeX from sheet 'Sheet1'
\begin{table}[htbp]
  \centering
  \caption{The encoding of typical DE variants by the proposed PDE.}
         \renewcommand{\arraystretch}{1.1}
 \renewcommand{\tabcolsep}{10pt}
    \begin{tabular}{ccccc}
    \toprule
    strategy & $bl$  & $br$  & $dn$  & $cs$ \\
    \midrule
    \texttt{DE/rand/1/bin} & 1     & 1     & 1     & 1 \\
    \midrule
    \texttt{DE/best/1/bin} & 2     & 2     & 1     & 1 \\
    \midrule
    \texttt{DE/current-to-best/1/bin} & 4     & 2     & 1     & 1 \\
    \midrule
    \texttt{DE/rand/2/bin} & 1     & 1     & 2     & 1 \\
    \midrule
    \texttt{DE/best/2/bin} & 2     & 2     & 2     & 1 \\
    \midrule
    \texttt{DE/current-to-pbest/1/bin} & 4     & 3     & 1     & 1 \\
    \midrule
    \texttt{DE/current-to-rand/1*} & 1     & 1     & 1     & 3 \\
    \bottomrule
    \end{tabular}%
  \label{tab_param encoding}%
    \vspace{0.5em}

      \footnotesize
    \textsuperscript{*} As per Eq. (\ref{equ_rand_1_arith}), \texttt{DE/current-to-rand/1} is equivalent to \texttt{DE/rand/1/arith}.\\
\end{table}%

Algorithm \ref{Alg_PDE} details PDE's procedure.
Its distinct attribute is Line 3, where the mutation function is shaped by \( bl \), \( br \), and \( dn \).
Line 4 designates the crossover function based on \( cs \).
The evolutionary phase commences at Line 6.
Notably, PDE's concurrent mutation and crossover operations (Lines 7-10) are {tensorized} to facilitate parallel offspring generation, which deviate from the conventional sequential operations in standard DE.
With such a tailored procedure, the computational efficiency can be substantially improved.


\begin{algorithm}
\small
\caption{Parameterized DE (PDE)}\label{Alg_PDE}
\begin{algorithmic}[1]
  \Require {$D$, $NP$, $G_{max}$, $F$, $CR$, $bl$ (base vector left), $br$ (base vector right), $dn$ (difference number), $cs$ (crossover scheme)}
  \State Initialize population $\mathbf{X} = \{\mathbf{x}_1, \mathbf{x}_2, \dots, \mathbf{x}_{\scalebox{0.5}{$\textit{NP}$}}\}$
  \State Evaluate the fitness of each individual in the population
  \State Generate mutation function $M(\mathbf{X})$ according to $bl$, $br$, $dn$:
  \Statex $\mathbf{v} = \mathbf{x}_{bl} + F \cdot (\mathbf{x}_{br} - \mathbf{x}_{bl}) + F \cdot (\Delta_1 + ... +\Delta_{dn})$
  \State According to $cs$, choose a crossover function $C(\mathbf{V}, \mathbf{X})$ in (\ref{equ_cross bin}-\ref{equ_cross arith})
  \State $g = 0$
  \While{$g \leq G_{max}$}
  \State Generate $NP$ mutant vectors: $\mathbf{V} = M(\mathbf{X})$
  \State Perform crossover for all mutant vectors: $\mathbf{U} = C(\mathbf{V}, \mathbf{X})$
  \State Evaluate the fitness of $\mathbf{U}$
  \State Make selection between $\mathbf{U}$ and $\mathbf{X}$
  \State $g = g + 1$
  \EndWhile
  \State\Return the best fitness
\end{algorithmic}
\end{algorithm}


\subsection{Architecture of MetaDE}
Atop the proposed PDE, this subsection further introduces the architecture of MetaDE.
The main target of MetaDE is to evolve the parameters of PDE through an external DE, empowering PDE to identify optimal parameters tailored to the target problem.
% Every computational phase of MetaDE benefits from extensive parallelization with GPU acceleration integration.

\begin{figure}[t]
\centering
\includegraphics[scale=0.9]{MetaDE.pdf}
\caption{
Architecture of MetaDE.
Within this architecture, a conventional DE algorithm operates as an \texttt{evolver}, where its individual $\mathbf{x}_i$ represents a distinct parameter configuration $\mathbf{\theta}_i$.
These configurations are relayed to PDE  to instantiate diverse DE variants as the \texttt{executors}.
Each \texttt{executor} then evolves its distinct population and returns the best fitness $y^*$ as identified, which is subsequently set as the fitness of $\mathbf{x}_i$.
}
\label{Figure_MetaDE}
\end{figure}

As illustrated in Fig. \ref{Figure_MetaDE}, MetaDE is structured with a two-tiered optimization architecture.
The upper tier, termed the \texttt{evolver}, leverages DE to evolve the parameters of PDE. In contrast, the lower tier consists of a collection of \texttt{executors} that each run the parameterized PDE instance to optimize the objective function.
Every individual in the \texttt{evolver}, represented as $\mathbf{x}_i$, is decoded into a parameter configuration $\bm{\theta}_i$ with six elements: $F$, $CR$, $bl$, $br$, $dn$, and $cs$.
For the evaluation of each individual, the configuration $\bm{\theta}_i$ is directed to its respective \texttt{executor} $\textrm{PDE}_i$ for objective function optimization.
The final fitness $y^*$ as identified by each \texttt{executor}, is subsequently set as the fitness of the corresponding $\mathbf{x}_i$ individual.

The architecture of MetaDE is streamlined for simplicity.
Building upon this architecture, MetaDE integrates two tailored components: the \emph{one-shot evaluation method} and the \emph{power-up strategy}.
These components further enhance the adaptability and efficiency of the \texttt{executors}, thereby elevating the overall performance of MetaDE.

\subsection{One-shot Evaluation Method}
Within the context of an \texttt{executor} driven by DE itself, the inherent stochastic nature can lead to variability in the optimal fitness values returned.
Historically, several evaluation techniques, such as repeated evaluation~\cite{metaCMAES2012}, F-racing~\cite{FRacing}, and intensification~\cite{intens}, have been put forth to tackle this inconsistency.
Yet, these often come at the cost of an exorbitant number of functional evaluations (FEs).
To address this issue, we introduce the one-shot evaluation method.

Specifically, the method mandates each \texttt{executor} to undertake a singular, comprehensive independent run, subsequently returning its best-found solution.
A distinguishing aspect of this method is the consistent allocation of the same initial random seed to every \texttt{executor}.
As the algorithm progresses, this uniform seed ensures that the PDE fine-tunes its parameters in a consistent manner, thereby identifying optimal parameters tailored to the given seed environment.
Essentially, this strategy embeds the seed as an integral facet of the problem domain.

\subsection{Power-up Strategy}
During the independent runs of an \texttt{executor}, the allocation of FEs plays a pivotal role in determining both the quality of solutions and computational efficiency. Allocating an excessive number of FEs indiscriminately can lead to undue computational resource consumption without necessarily improving solution quality.
To address this issue, we propose the power-up strategy.

The essence of this strategy is dynamic FE allocation: while earlier iterations receive a moderate number of FEs to ensure resource efficiency, a more generous allocation (fivefold) is reserved for the terminal iteration within the evolutionary process.
This strategy ensures that the \texttt{executor} has the resources for a thorough and comprehensive evaluation during its most crucial phase -- the final generation of the \texttt{evolver}.


\subsection{Implementation}
As outlined in Algorithm~\ref{Alg_MetaDE}, MetaDE draws its simple algorithmic workflow from conventional DE. MetaDE adopts \texttt{DE/rand/1/bin} as the \texttt{evolver}.
The initialization phase (Line 1) spawns the MetaDE population within the parameter boundaries \( [\mathbf{lb}, \mathbf{ub}] \).
During the evaluation phase (Lines 6-11), each individual is decoded into a parameter blueprint and directed to an independent PDE instance (\texttt{executor}) for problem resolution.
Running for a predetermined iteration count \( G' \), each \texttt{executor} subsequently reports the best fitness.
Notably, Line 10 encapsulates the essence of the power-up strategy: for MetaDE's concluding iteration (\( g == G_{max} \)), the evaluation quota is amplified to \( 5 \times G' \) for the \texttt{executors}.


\begin{algorithm}
\small
\caption{MetaDE}
\label{Alg_MetaDE}
\begin{algorithmic}[1]
  \Require {$D$, $NP$, $G_{max}$, $\mathbf{lb}$ (lower boundaries of PDE's parameters), $\mathbf{ub}$ (upper boundaries of PDE's parameters), $NP'$ (population size of PDE), $G'$ (max generations of PDE)}
  \State Initialize population $\mathbf{X} = \{\mathbf{x}_1, \mathbf{x}_2, \dots, \mathbf{x}_{\scalebox{0.5}{$\textit{NP}$}}\}$ between $[\mathbf{lb}, \mathbf{ub}]$
  \State Initialize the fitness of $\mathbf{X}$: $\mathbf{y}=\mathbf{inf}$
  \State $g = 0$
  \While{$g \leq G_{max}$}
  \State Generate trial vectors $\mathbf{U}$ by mutation and crossover
  \Statex \quad \ /* The mutation and crossover scheme used is rand/1/bin\ */
  \State Decode each trial vector $\mathbf{u}$ into parameters:
  \Statex \quad \ $F=\mathbf{u}[1], CR=\mathbf{u}[2], bl=\textrm{floor}(\mathbf{u}[3]), br=\textrm{floor}(\mathbf{u}[4])$,
  \Statex \quad \ $dn=\textrm{floor}(\mathbf{u}[5]), cs=\textrm{floor}(\mathbf{u}[6])$
  %\Statex /*Evaluate each $\mathbf{u}$ by running PDE for certain generations*/
  \If {$g < G_{max}$}
    \State $\mathbf{y} = \textrm{PDE}(D, NP', G', F, CR, bl, br, dn, cs)$
  \Else
    \State $\mathbf{y} = \textrm{PDE}(D, NP', 5 * G', F, CR, bl, br, dn, cs)$
  \EndIf
  \State Make selection between $\mathbf{U}$ and $\mathbf{X}$
  \State $g = g + 1$
  \EndWhile
  \State\Return the best individual and fitness
\end{algorithmic}
\end{algorithm}


Evidently, the algorithmic design of MetaDE provides an automated end-to-end approach to black-box optimization.
However, the computational demands of MetaDE, particularly in terms of FEs, cannot be understated.
In historical computational contexts, such intensive demands might have posed significant impediments.
Fortunately, contemporary advancements in computational infrastructures, coupled with the ubiquity of high-performance computational apparatuses such as GPUs, have substantially alleviated such a challenge.


Hence, we leverage the GPU-accelerated framework of EvoX~\cite{evox} for the implementation of MetaDE.
Thanks to the inherently parallel nature of MetaDE, computational tasks can be judiciously delegated to GPUs to engender optimized runtime performance.
Specifically, the parallelism in MetaDE manifests in three distinct facets:
\begin{itemize}
  \item \textbf{Parallel Initialization and Execution:}
 The multiple \texttt{executors} are instantiated and operated concurrently, each tailored by a unique parameter configuration derived from the MetaDE ensemble.
  This simultaneous operation enables comprehensive exploration across varied parameter landscapes.

  \item \textbf{Parallel Offspring Generation:}
  Both the \texttt{evolver} and \texttt{executors} adhere to parallel strategies for offspring inception.
  By synchronizing and coordinating mutations and crossover operations in their respective populations, MetaDE is able to rapidly produce offspring, thereby accelerating the evolutionary process.

  \item \textbf{Parallel Fitness Evaluations:}
  Each \texttt{executor} conducts fitness evaluations concurrently across its member individuals.
  Given the substantial number of the individuals within the populations of the \texttt{executors}, this parallel strategy significantly enhances the overall efficiency of MetaDE.
\end{itemize}


\lstset{
  language=Python,
  aboveskip=3mm,
  belowskip=3mm,
  showstringspaces=false,
  columns=flexible,
  basicstyle={\footnotesize\ttfamily},
  numbers=left,
  numberstyle=\tiny\color{gray},
  xleftmargin=2em,
  keywordstyle=\bfseries\color{dkgreen},
  commentstyle=\color{gray}\itshape,
  stringstyle=\color{mauve},
  breaklines=true,
  breakatwhitespace=true,
  tabsize=3,
  emph={[1]evox,BatchExecutor,MetaProblem},          % Emphasize numpy
  emphstyle={[1]\bfseries\color{dkblue}},  % Set the style for emphasized words
  emph={[2]__init__, min, evaluate, reproduce, init},
  emphstyle={[2]\color{dkblue}},
  emph={[3]self,super},          % Emphasize numpy
  emphstyle={[3]\color{dkgreen}},
  morekeywords={from,import},  % Add more keywords
  captionpos=b,             % Caption position
      frame=lines,
  framesep=2mm,
}
\
\begin{lstlisting}[caption={Demonstrative implementation of MetaDE leveraging the computational workflow of EvoX. The implementation is distinctly divided into four pivotal components: Workflow Initialization, Meta Problem Transformation, Computing Workflow Creation, and Execution.}, label={lst:python_example}, float=!t]
from evox import algorithms, problems, ...

### Initialization ###
evolver = algorithm.DE()  # specify evolver
executor = algorithm.PDE()  # specify executor
problem = ...  # specify optimization problem

### Meta Problem Transformation ###
class MetaProblem(Problem):
    def __init__(self, batch_executor, ... ):
        # vectorize fitness evaluations
        self.batch_evaluate = vectorize(vectorize(problem.evaluate))

    def evaluate(self, state, ...):
        ...
        # run executors
        while ...:
            ...
            batch_fits, ... = self.batch_evaluate(...)
        # return fitness
        return min(min(batch_fits))

### Computing Workflow Creation ###
batch_executor = create_batch_executor(...)
meta_problem = MetaProblem(batch_executor, ...)
workflow = workflow.UniWorkflow(
            algorithm = evolver,
            pop_transform = decoder,
            problem = meta_problem,
        )

### Execution ###
while ...:
    state = workflow.step(state)
\end{lstlisting}

MetaDE adheres rigorously to the functional programming paradigm, capitalizing on automatic vectorization for parallel execution. Core algorithmic components, including crossover, mutation, and evaluation, are constructed using pure functions. Subsequently, the entire program is mapped to a GPU-based computation graph, ushering in accelerated processing. EvoX's adept state management ensures a seamless transfer of the algorithm's prevailing state, encompassing aspects like population, fitness, hyperparameters, and auxiliary data.

Listing \ref{lst:python_example} elucidates a representative implementation of MetaDE underpinned by the EvoX framework, which is meticulously segmented into four salient phases:

\begin{itemize}
    \item \textbf{Initialization}: Herein, primary entities like the \texttt{evolver} (employing the traditional DE) and the \texttt{executor} (utilizing the proposed PDE) are instantiated. Concurrently, the target optimization problem is defined.

    \item \textbf{Meta Problem Transformation}: Within this phase, the original optimization problem is transformed to align with the meta framework. This metamorphosis is realized via the \texttt{\textbf{MetaProblem}} class, where the evaluation function undergoes vectorization, priming it for efficient batch assessments and facilitating concurrent evaluations of manifold configurations.

    \item \textbf{Computing Workflow Creation}: Post transformation, the workflow is architected to seamlessly amalgamate the initialized components. The \texttt{batch\_executor} is crafted for batched operations of DE variants, and the \texttt{\textbf{MetaProblem}} is instantiated therewith. The holistic workflow, embodied by the \texttt{UniWorkflow} class, is then constructed, weaving together the \texttt{evolver}, the transformed problem, and a (\texttt{decoder}) which transforms the \texttt{evolver}'s population into specific hyperparameters for instantiating  the DE variant of each \texttt{executor}.

    \item \textbf{Execution}: Having established the groundwork, MetaDE's execution phase is triggered, autonomously driving the computing workflow across distributed GPUs.
    This workflow is traversed iteratively, culminating once a predefined termination criterion is met.
\end{itemize}

\section{Experimental Study}\label{section_Experimental_study}
In this section, we conduct detailed experimental assessments of MetaDE's capabilities.
First, we comprehensively benchmark MetaDE against several representative DE variants and CEC2022 top algorithms to gauge its relative performance on the CEC2022 benchmark suite \cite{CEC2022SO}.
Then, we investigate the optimal DE variants obtained by MetaDE in the benchmark experiment.
Finally, we apply MetaDE to robot control tasks.
All experiments were conducted on a system equipped with an Intel Core i9-10900X CPU and  an NVIDIA RTX 3090 GPU.
For GPU acceleration, all the algorithms and test functions were implemented within EvoX \cite{evox}.

\subsection{Benchmarks against Representative DE Variants}\label{section_Comparison with Classic DE Variants}

\subsubsection{Experimental Setup}
The CEC2022 benchmark suite for single-objective black-box optimization was utilized for this study.
This suite includes basic ($F_1-F_5$), hybrid ($F_6-F_8$), and composition functions ($F_9-F_{12}$), catering to various optimization characteristics such as unimodality/multimodality and separability/non-separability.

For benchmark comparisons, we selected seven representative DE variants: DE (\texttt{rand/1/bin}) \cite{DE1997}, SaDE \cite{SaDE2008}, JaDE \cite{JADE2009}, CoDE \cite{CoDE2011}, SHADE \cite{SHADE2013}, LSHADE-RSP \cite{LSHADE-RSP2018}, and EDEV \cite{EDEV2018}, which encapsulate a spectrum of mutation, crossover, and adaptation strategies. All algorithms were reimplemented using EvoX, with each capable of running in parallel, including the concurrent evaluation and reproduction.

Their respective descriptions are as follows:
\begin{itemize}
  \item \texttt{DE/rand/1/bin} is a foundational DE variant, which leverages a random mutation strategy coupled with binomial crossover.
  \item SaDE maintains an archive for tracking successful strategies and \( CR \) values and exhibits adaptability in strategy selection and parameter adjustments throughout the optimization process.
  \item JaDE relies on the \texttt{current-to-pbest} mutation strategy and dynamically adjusts its \( F \) and \( CR \) parameters during the optimization trajectory.
  \item CoDE infuses generational diversity by composing three disparate strategies, each complemented with randomized parameters, for offspring generation.
  \item SHADE employs the current-to-pbest mutation strategy and integrates a success-history mechanism to fine-tune its \( F \) and \( CR \)  parameters adaptively.
  \item LSHADE-RSP, as one of the most competitive DE variants, employs delicate strategies such as linear population size reduction and ranking-based mutation.
  \item EDEV adopts a distributed framework that ensembles three classic DE variants: JaDE, CoDE, and EPSDE.
\end{itemize}

The population size for all comparative algorithms was uniformly set to 100, except for experiments involving large populations. The other parameters for these algorithms were adopted as per their default settings described in their respective publications.

In our MetaDE configuration, on one hand, the \texttt{evolver} had a population size of 100 and adopted the vanilla \texttt{rand/1/bin} strategy with $F=0.5$ and $CR=0.9$;
On the other hand, each \texttt{executor} maintained a population of 100, iterating 1000 times for all the problems.
For simplicity, any result exceeding the precision of $10^{-8}$ was truncated to 0.
All statistical results were obtained via 31 independent runs\footnote{Full results, including the statistical results applying Wilcoxon rank-sum tests with a a significance level of 0.05, can be found in the Supplementary Document.}.




\begin{figure}[!t]
\centering
\includegraphics[scale=0.3]{D10.pdf}
\caption{Convergence curves on 10D problems in CEC2022 benchmark suite. The peer DE variants are set with population size of 100.}
\label{Figure_convergence_10D}
\end{figure}

\begin{figure}[!t]
\centering
\includegraphics[scale=0.3]{D20.pdf}
\caption{Convergence curves on 20D problems in CEC2022 benchmark suite. The peer DE variants are set with population size of 100.}
\label{Figure_convergence_20D}
\end{figure}


\subsubsection{Performance under Equal Wall-clock Time}\label{sec:expereiment_time}
In this part, we set equal  wall-clock time (\SI{60}{\second}) as the termination condition for running each test. 
{
This approach aligns with the practical constraints of modern GPU computing, where execution time serves as a more meaningful and comparable measure of performance across algorithms. Since all algorithms in our experiments are implemented with GPU parallelism, this setup ensures fairness by standardizing the computational resources and focusing on efficiency within the same time budget.
}

As shown in Figs. \ref{Figure_convergence_10D} and \ref{Figure_convergence_20D}, we selected five challenging problems, specifically $F_2$, $F_4$, $F_6$, $F_9$, and $F_{10}$, to demonstrate the convergence profiles.
Notably, MetaDE's convergence curve is observably more favorable, consistently registering lower errors than its counterparts across the majority of the problems.
Particularly, on $F_2$, $F_4$, $F_9$, and $F_{10}$, MetaDE exhibits resilience against local optima entrapment and subsequent convergence stagnation.
This is attributed to MetaDE's capability to identify optimal algorithm settings tailored for diverse problems, rather than merely tweaking parameters based on isolated segments of the optimization trajectory, as is the case with some DE variants.
An intriguing characteristic of MetaDE's convergence, evident in functions like $F_9$ (refer to Fig. \ref{Figure_convergence_10D}), is its pronounced performance surge in the optimization's terminal phase.
This enhancement can be linked to MetaDE's power-up strategy of allocating bonus computational resources in its final phase (as per Line 10 of Algorithm \ref{Alg_MetaDE}).


\begin{figure}[!h]
\centering
\includegraphics[scale=0.3]{FEs.pdf}
\caption{The number of FEs achieved by each algorithm within \SI{60}{\second}. The results are averaged on all 10D and 20D problems in the CEC2022 benchmark suite.}
\label{fig:maxFEs}
\end{figure}

{
Furthermore, to assess the concurrency of the algorithms, the number of FEs achieved by each algorithm within 60 seconds is shown in Table~\ref{tab:FEs} and Fig~\ref{fig:maxFEs}.
}
The results indicate that MetaDE achieves approximately $10^9$ FEs within 60 seconds, while the other algorithms manage to attain only around $10^7$ FEs in the same time frame.
The results demonstrate the high concurrency of MetaDE, which is particularly favorable in GPU computing.

\begin{table}[htbp]
  \centering
  
  \caption{{The number of FEs achieved by each algorithm within \SI{60}{\second}.}}
 
\scriptsize                   %设置字体大小
\renewcommand{\arraystretch}{1}
\renewcommand{\tabcolsep}{2.5pt}   %pt越大字越小
\resizebox{\linewidth}{!}{
% Table generated by Excel2LaTeX from sheet 'Experiment1 60S'
\begin{tabular}{cccccccccc}
\toprule
Dim   & Func  & MetaDE & DE    & SaDE  & JaDE  & CoDE  & SHADE &LSHADE-RSP&EDEV\\
\midrule
\multirow{12}[2]{*}{10D} & $F_{1}$ & \textbf{1.85E+09} & 4.28E+06 & 1.96E+06 & 2.55E+06 & 1.09E+07 & 2.24E+06&2.57E+06&2.79E+06 \\
      & $F_{2}$ & \textbf{1.84E+09} & 4.19E+06 & 1.89E+06 & 2.42E+06 & 1.11E+07 & 2.18E+06&2.58E+06&2.82E+06 \\
      & $F_{3}$ & \textbf{1.50E+09} & 4.00E+06 & 1.89E+06 & 2.50E+06 & 1.11E+07 & 2.10E+06& 2.46E+06&2.68E+06\\
      & $F_{4}$ & \textbf{1.84E+09} & 4.11E+06 & 2.02E+06 & 2.61E+06 & 1.11E+07 & 2.15E+06&2.60E+06&2.88E+06 \\
      & $F_{5}$ & \textbf{1.83E+09} & 4.13E+06 & 2.00E+06 & 2.60E+06 & 1.15E+07 & 2.17E+06& 2.53E+06&2.96E+06\\
      & $F_{6}$ & \textbf{1.84E+09} & 4.31E+06 & 1.96E+06 & 2.65E+06 & 1.07E+07 & 2.14E+06&2.55E+06&2.95E+06\\
      & $F_{7}$ & \textbf{1.74E+09} & 3.35E+06 & 1.90E+06 & 2.55E+06 & 9.96E+06 & 2.14E+06& 2.41E+06&2.87E+06\\
      & $F_{8}$ & \textbf{1.72E+09} & 3.34E+06 & 1.83E+06 & 2.53E+06 & 9.60E+06 & 2.17E+06&2.34E+06&2.74E+06 \\
      & $F_{9}$ & \textbf{1.78E+09} & 3.35E+06 & 1.84E+06 & 2.52E+06 & 9.69E+06 & 2.18E+06&2.44E+06 &2.82E+06\\
      & $F_{10}$ & \textbf{1.44E+09} & 3.32E+06 & 1.83E+06 & 2.46E+06 & 9.00E+06 & 2.12E+06& 2.30E+06&2.70E+06\\
      & $F_{11}$ & \textbf{1.46E+09} & 3.55E+06 & 1.88E+06 & 2.34E+06 & 9.66E+06 & 2.13E+06&2.34E+06& 2.63E+06\\
      & $F_{12}$ & \textbf{1.43E+09} & 3.46E+06 & 1.83E+06 & 2.41E+06 & 9.51E+06 & 2.09E+06&2.32E+06 &2.67E+06\\
\midrule
\midrule
\multirow{12}[2]{*}{20D} & $F_{1}$ & \textbf{1.66E+09} & 4.32E+06 & 1.92E+06 & 2.46E+06 & 1.13E+07 & 2.21E+06&2.68E+06&2.80E+06 \\
      & $F_{2}$ & \textbf{1.66E+09} & 3.91E+06 & 1.88E+06 & 2.37E+06 & 1.17E+07 & 2.09E+06& 2.66E+06&2.74E+06\\
      & $F_{3}$ & \textbf{1.18E+09} & 3.76E+06 & 1.77E+06 & 2.33E+06 & 9.75E+06 & 1.95E+06&2.62E+06& 2.64E+06\\
      & $F_{4}$ & \textbf{1.65E+09} & 3.50E+06 & 1.87E+06 & 2.28E+06 & 1.07E+07 & 2.00E+06&2.63E+06 &2.80E+06\\
      & $F_{5}$ & \textbf{1.64E+09} & 3.57E+06 & 1.86E+06 & 2.32E+06 & 1.07E+07 & 2.05E+06& 2.55E+06&2.74E+06\\
      & $F_{6}$ & \textbf{1.64E+09} & 3.89E+06 & 1.90E+06 & 2.34E+06 & 1.16E+07 & 2.09E+06& 2.62E+06&2.80E+06\\
      & $F_{7}$ & \textbf{1.45E+09} & 4.15E+06 & 1.84E+06 & 2.36E+06 & 1.01E+07 & 1.99E+06&2.32E+06& 2.80E+06\\
      & $F_{8}$ & \textbf{1.44E+09} & 3.42E+06 & 1.82E+06 & 2.31E+06 & 9.51E+06 & 2.12E+06&2.18E+06&2.30E+06 \\
      & $F_{9}$ & \textbf{1.57E+09} & 3.30E+06 & 1.77E+06 & 2.33E+06 & 9.48E+06 & 2.03E+06&2.59E+06&2.75E+06 \\
      & $F_{10}$ & \textbf{9.80E+08} & 3.67E+06 & 1.82E+06 & 2.43E+06 & 1.01E+07 & 2.08E+06& 2.09E+06&2.35E+06\\
      & $F_{11}$ & \textbf{1.00E+09} & 3.51E+06 & 1.95E+06 & 2.41E+06 & 9.81E+06 & 2.07E+06&2.17E+06& 2.37E+06\\
      & $F_{12}$ & \textbf{9.90E+08} & 3.44E+06 & 1.85E+06 & 2.37E+06 & 9.51E+06 & 2.07E+06& 2.17E+06&2.39E+06\\
\bottomrule
\end{tabular}%
}
  \label{tab:FEs}%
\end{table}%

\subsubsection{Performance under Equal FEs}\label{sec:expereiment_FEs}

% Table generated by Excel2LaTeX from sheet 'Sheet1'
\begin{table*}[htbp]
  %\centering
\caption{Comparisons between MetaDE and other DE variants under equal FEs. 
The mean and standard deviation (in parentheses) of the results over multiple runs are displayed in pairs. 
Results with the best mean values are highlighted.}
  \resizebox{\linewidth}{!}{
  \renewcommand{\arraystretch}{1.2}
 \renewcommand{\tabcolsep}{2pt}
% Table generated by Excel2LaTeX from sheet 'Exp3 same FEs'
\begin{tabular}{cccccccccc}
\toprule
\multicolumn{2}{c}{Func} & MetaDE & DE    & SaDE  & JaDE  & CoDE  & SHADE & LSHADE-RSP & EVDE \\
\midrule
\multirow{3}[2]{*}{10D} & $F_{2}$ & \textbf{0.00E+00 (0.00E+00)} & 4.52E+00 (2.36E+00)$-$ & 6.85E+00 (3.52E+00)$-$ & 6.31E+00 (3.05E+00)$-$ & 5.78E+00 (2.37E+00)$-$ & 4.33E+00 (3.81E+00)$-$ & 2.35E+00 (3.44E+00)$-$ & 5.86E+00 (2.99E+00)$-$ \\
      & $F_{6}$ & \textbf{5.50E-04 (3.96E-04)} & 1.13E-01 (7.83E-02)$-$ & 3.54E+01 (1.11E+02)$-$ & 2.02E+00 (3.34E+00)$-$ & 6.96E-03 (5.99E-03)$-$ & 9.27E-01 (1.31E+00)$-$ & 3.10E-02 (4.83E-02)$-$ & 1.46E+00 (2.76E+00)$-$ \\
      & $F_{10}$ & \textbf{0.00E+00 (0.00E+00)} & 1.00E+02 (4.40E-02)$-$ & 1.00E+02 (6.10E-02)$-$ & 1.21E+02 (4.35E+01)$-$ & 1.00E+02 (6.88E-02)$-$ & 1.29E+02 (4.67E+01)$-$ & 1.09E+02 (2.87E+01)$-$ & 1.10E+02 (3.05E+01)$-$ \\
\midrule
\midrule
\multirow{3}[2]{*}{20D} & $F_{2}$ & \textbf{1.26E-02 (3.74E-02)} & 4.72E+01 (2.09E+00)$-$ & 4.76E+01 (2.02E+00)$-$ & 1.34E+01 (2.19E+01)$-$ & 4.91E+01 (1.70E-06)$-$ & 4.91E+01 (3.40E-06)$-$ & 4.84E+01 (1.62E+00)$-$ & 4.47E+01 (1.41E+01)$-$ \\
      & $F_{6}$ & \textbf{1.16E-01 (2.79E-02)} & 7.28E-01 (5.22E-01)$-$ & 3.20E+01 (1.61E+01)$-$ & 4.90E+01 (3.31E+01)$-$ & 2.26E+01 (1.80E+01)$-$ & 5.67E+01 (3.90E+01)$-$ & 1.15E+01 (8.38E+00)$-$ & 2.92E+03 (5.82E+03)$-$ \\
      & $F_{10}$ & \textbf{0.00E+00 (0.00E+00)} & 1.07E+02 (2.07E+01)$-$ & 1.00E+02 (2.72E-02)$-$ & 1.01E+02 (3.64E-02)$-$ & 1.00E+02 (3.71E-02)$-$ & 1.43E+02 (5.64E+01)$-$ & 1.21E+02 (4.52E+01)$-$ & 1.14E+02 (4.65E+01)$-$ \\
\midrule
\multicolumn{2}{c}{$+$ / $\approx$ / $-$} & --    & 0/0/6 & 0/0/6 & 0/0/6 & 0/0/6 & 0/0/6 & 0/0/6 & 0/0/6 \\
\bottomrule
\end{tabular}%

}
\label{tab:sameFEs}%

\footnotesize
\textsuperscript{*} The Wilcoxon rank-sum tests (with a significance level of 0.05) were conducted between MetaDE and each algorithm individually.
The final row displays the number of problems where the corresponding algorithm performs statistically better ($+$),  similar ($\thickapprox$), or worse ($-$) compared to MetaDE.
\end{table*}%



In the preceding part, the performance benchmarking of MetaDE with other algorithms was anchored to equal wall-clock durations.
However, to ensure a comprehensive assessment, it is imperative to evaluate their performances under equivalent FEs.
In this part, we run each algorithm using the FEs achieved by MetaDE in \SI{60}{\second} (i.e., $1.84\times10^9/1.66\times10^9$, $1.84\times10^9/1.64\times10^9$, and $1.44\times10^9/9.8\times10^8$) on $F_2$, $F_6$, and $F_{10}$ for 10D/20D cases.
These selected functions collectively epitomize the basic, hybrid, and composition challenges within the CEC2022 benchmark suite.

As summarized in Table \ref{tab:sameFEs}, MetaDE consistently demonstrates the best performance, even when other algorithms are endowed with comparable FEs.
The reason can be traced to the inherent stagnation tendencies of other algorithms: after a certain point, additional FEs may not contribute to performance improvements.
This behavioral pattern is also lucidly captured in the convergence curves as presented in Figs. \ref{Figure_convergence_10D} and \ref{Figure_convergence_20D}.

Another noteworthy observation is the extended computation time required for a singular run of the comparison algorithms under these enhanced FEs, often extending to several hours or even transcending a day (e.g., running a single run of DE can take up to seven hours).
This elongated computational span can largely be attributed to their low concurrency, which struggles to benefit the parallelism of GPU computing.


\subsubsection{Performance with Large Populations}\label{sec:expereiment_large_pop}
Since a large population size could potentially increase the concurrency of fitness evaluations, for rigorousness, we further investigate the performance of the algorithms with large populations.

Specifically, MetaDE adopted the same population size setting as in previous experiments (i.e., 100 for both \texttt{evolver} and \texttt{executor}), while the population size of the other DE variants was increased to 1,000. 
This adjustment significantly enhances the concurrency of the other DE variants when utilizing GPU accelerations, thereby preventing insufficient convergence.


As evidenced in Figs. \ref{Figure_convergence_10D_NP10k}-\ref{Figure_convergence_20D_NP10k}, MetaDE still outperforms the other DE variants across all problems.
However, the performances of the other DE variants did not show significant improvements, which can be attributed to two factors.
First, since the conventional DE variants were not tailored for large populations, simply enlarging the populations may not help.
Second, since the sorting and archiving operations in some DE variants (e.g., SaDE) suffer from high computational complexities related to the population size, enlarging the populations brings additional computation overheads, thus limiting their performances under fixed wall-clock time.

By contrast, the large population in MetaDE is delicately organized in a \emph{hierarchical} manner: the \texttt{executor} maintains a population of moderate size, with each individual initializing an \texttt{executor} with a normal population.
This strategy not only capitalizes on the small-population advantage of conventional DE, but also benefits the concurrency brought by large populations.

\begin{figure}[!t]
\centering
\includegraphics[scale=0.3]{D10NP1000.pdf}
\caption{Convergence curves on 10D problems in CEC2022 benchmark suite. The peer DE variants are set with population size of 1,000.}
\label{Figure_convergence_10D_NP10k}
\end{figure}

\begin{figure}[!t]
\centering
\includegraphics[scale=0.3]{D20NP1000.pdf}
\caption{Convergence curves on 20D problems in CEC2022 benchmark suite. The peer DE variants are set with population size of 1,000.}
\label{Figure_convergence_20D_NP10k}
\end{figure}


{
\subsection{Comparisons with Top Algorithms in CEC2022 Competition}\label{section_Comparison with Top Algorithms of CEC Competition}

% Table generated by Excel2LaTeX from sheet 'Sheet1'
\begin{table*}[htbp]
  \centering
  
  \caption{{Comparisons between MetaDE and the top 4 algorithms from CEC2022 Competition (10D). 
The mean and standard deviation (in parentheses) of the results over multiple runs are displayed in pairs. 
Results with the best mean values are highlighted. }
  }
\footnotesize
% Table generated by Excel2LaTeX from sheet 'Exp 7 vs CECtop'
\begin{tabular}{cccccc}
\toprule
Func  & MetaDE & EA4eig & NL-SHADE-LBC & NL-SHADE-RSP & S-LSHADE-DP \\
\midrule
$F_{1}$ & \textbf{0.00E+00 (0.00E+00)} & \boldmath{}\textbf{0.00E+00 (0.00E+00)$\approx$}\unboldmath{} & \boldmath{}\textbf{0.00E+00 (0.00E+00)$\approx$}\unboldmath{} & \boldmath{}\textbf{0.00E+00 (0.00E+00)$\approx$}\unboldmath{} & \boldmath{}\textbf{0.00E+00 (0.00E+00)$\approx$}\unboldmath{} \\
$F_{2}$ & \textbf{0.00E+00 (0.00E+00)} & 7.97E-01 (1.78E+00)$-$ & 7.97E-01 (1.78E+00)$-$ & \boldmath{}\textbf{0.00E+00 (0.00E+00)$\approx$}\unboldmath{} & \boldmath{}\textbf{0.00E+00 (0.00E+00)$\approx$}\unboldmath{} \\
$F_{3}$ & \textbf{0.00E+00 (0.00E+00)} & \boldmath{}\textbf{0.00E+00 (0.00E+00)$\approx$}\unboldmath{} & \boldmath{}\textbf{0.00E+00 (0.00E+00)$\approx$}\unboldmath{} & \boldmath{}\textbf{0.00E+00 (0.00E+00)$\approx$}\unboldmath{} & \boldmath{}\textbf{0.00E+00 (0.00E+00)$\approx$}\unboldmath{} \\
$F_{4}$ & \textbf{0.00E+00 (0.00E+00)} & 9.95E-01 (1.22E+00)$-$ & 1.99E-01 (4.45E-01)$-$ & 2.98E+00 (1.15E+00)$-$ & \boldmath{}\textbf{0.00E+00 (0.00E+00)$\approx$}\unboldmath{} \\
$F_{5}$ & \textbf{0.00E+00 (0.00E+00)} & \boldmath{}\textbf{0.00E+00 (0.00E+00)$\approx$}\unboldmath{} & \boldmath{}\textbf{0.00E+00 (0.00E+00)$\approx$}\unboldmath{} & \boldmath{}\textbf{0.00E+00 (0.00E+00)$\approx$}\unboldmath{} & \boldmath{}\textbf{0.00E+00 (0.00E+00)$\approx$}\unboldmath{} \\
$F_{6}$ & 5.50E-04 (3.96E-04) & 7.53E-04 (5.52E-04)$\approx$ & 8.93E-02 (1.18E-01)$-$ & 4.37E-02 (5.41E-02)$-$ & \textbf{5.84E-05 (4.73E-05)$+$} \\
$F_{7}$ & \textbf{0.00E+00 (0.00E+00)} & \boldmath{}\textbf{0.00E+00 (0.00E+00)$\approx$}\unboldmath{} & \boldmath{}\textbf{0.00E+00 (0.00E+00)$\approx$}\unboldmath{} & \boldmath{}\textbf{0.00E+00 (0.00E+00)$\approx$}\unboldmath{} & \boldmath{}\textbf{0.00E+00 (0.00E+00)$\approx$}\unboldmath{} \\
$F_{8}$ & 5.52E-03 (4.41E-03) & 1.01E-04 (1.66E-04)$+$ & 3.96E-04 (4.23E-04)$+$ & 3.13E-01 (3.60E-01)$-$ & \textbf{1.26E-05 (1.56E-05)$+$} \\
$F_{9}$ & \textbf{3.36E+00 (1.77E+01)} & 1.86E+02 (0.00E+00)$-$ & 2.29E+02 (3.18E-14)$-$ & 8.03E+01 (1.08E+02)$-$ & 2.23E+02 (1.31E+01)$-$ \\
$F_{10}$ & \textbf{0.00E+00 (0.00E+00)} & 1.00E+02 (0.00E+00)$-$ & 1.00E+02 (0.00E+00)$-$ & 1.56E-02 (3.12E-02)$-$ & \boldmath{}\textbf{0.00E+00 (0.00E+00)$\approx$}\unboldmath{} \\
$F_{11}$ & \textbf{0.00E+00 (0.00E+00)} & \boldmath{}\textbf{0.00E+00 (0.00E+00)$\approx$}\unboldmath{} & \boldmath{}\textbf{0.00E+00 (0.00E+00)$\approx$}\unboldmath{} & \boldmath{}\textbf{0.00E+00 (0.00E+00)$\approx$}\unboldmath{} & \boldmath{}\textbf{0.00E+00 (0.00E+00)$\approx$}\unboldmath{} \\
$F_{12}$ & \textbf{1.39E+02 (4.63E+01)} & 1.48E+02 (5.98E+00)$-$ & 1.65E+02 (0.00E+00)$-$ & 1.62E+02 (2.15E+00)$-$ & 1.59E+02 (0.00E+00)$-$ \\
\midrule
$+$ / $\approx$ / $-$ & --    & 1/6/5 & 1/5/6 & 0/6/6 & 2/8/2 \\
\bottomrule
\end{tabular}%

\footnotesize
\textsuperscript{*} The Wilcoxon rank-sum tests (with a significance level of 0.05) were conducted between MetaDE and each algorithm individually.
The final row displays the number of problems where the corresponding algorithm performs statistically better ($+$),  similar ($\thickapprox$), or worse ($-$) compared to MetaDE.\\


\label{tab:vsCECTop 10D}%
\end{table*}%

% Table generated by Excel2LaTeX from sheet 'Sheet1'
\begin{table*}[htbp]
  \centering
  
  \caption{{Comparisons between MetaDE and the top 4 algorithms from CEC2022 Competition (20D). 
The mean and standard deviation (in parentheses) of the results over multiple runs are displayed in pairs. 
Results with the best mean values are highlighted.
  }
  }
  %\resizebox{\linewidth}{!}{
  %       \renewcommand{\arraystretch}{1}
 %\renewcommand{\tabcolsep}{3pt}
% Table generated by Excel2LaTeX from sheet 'Sheet1'
\footnotesize
% Table generated by Excel2LaTeX from sheet 'Exp 7 vs CECtop'
\begin{tabular}{cccccc}
\toprule
Func  & MetaDE & EA4eig & NL-SHADE-LBC & NL-SHADE-RSP & S-LSHADE-DP \\
\midrule
$F_{1}$ & \textbf{0.00E+00 (0.00E+00)} & \boldmath{}\textbf{0.00E+00 (0.00E+00)$\approx$}\unboldmath{} & \boldmath{}\textbf{0.00E+00 (0.00E+00)$\approx$}\unboldmath{} & \boldmath{}\textbf{0.00E+00 (0.00E+00)$\approx$}\unboldmath{} & \boldmath{}\textbf{0.00E+00 (0.00E+00)$\approx$}\unboldmath{} \\
$F_{2}$ & 3.83E-04 (2.10E-03) & \textbf{0.00E+00 (0.00E+00)$+$} & 4.91E+01 (0.00E+00)$-$ & \textbf{0.00E+00 (0.00E+00)$+$} & \textbf{0.00E+00 (0.00E+00)$+$} \\
$F_{3}$ & \textbf{0.00E+00 (0.00E+00)} & \boldmath{}\textbf{0.00E+00 (0.00E+00)$\approx$}\unboldmath{} & \boldmath{}\textbf{0.00E+00 (0.00E+00)$\approx$}\unboldmath{} & \boldmath{}\textbf{0.00E+00 (0.00E+00)$\approx$}\unboldmath{} & \boldmath{}\textbf{0.00E+00 (0.00E+00)$\approx$}\unboldmath{} \\
$F_{4}$ & 1.96E+00 (7.76E-01) & 7.36E+00 (2.06E+00)$-$ & \boldmath{}\textbf{1.59E+00 (5.45E-01)$\approx$}\unboldmath{} & 1.07E+02 (1.54E+02)$-$ & 3.20E+00 (1.94E+00)$-$ \\
$F_{5}$ & \textbf{0.00E+00 (0.00E+00)} & \boldmath{}\textbf{0.00E+00 (0.00E+00)$\approx$}\unboldmath{} & \boldmath{}\textbf{0.00E+00 (0.00E+00)$\approx$}\unboldmath{} & 2.27E-01 (4.54E-01)$-$ & \boldmath{}\textbf{0.00E+00 (0.00E+00)$\approx$}\unboldmath{} \\
$F_{6}$ & \textbf{1.38E-01 (5.56E-02)} & 2.54E-01 (4.28E-01)$-$ & 3.06E-01 (2.01E-01)$-$ & 2.08E-01 (9.78E-02)$-$ & 5.02E-01 (5.34E-01)$-$ \\
$F_{7}$ & 8.42E-02 (1.01E-01) & 1.37E+00 (1.10E+00)$-$ & \boldmath{}\textbf{6.24E-02 (1.40E-01)$\approx$}\unboldmath{} & 1.28E+00 (1.95E+00)$-$ & 9.83E-01 (8.12E-01)$-$ \\
$F_{8}$ & 2.66E+00 (3.83E+00) & 2.02E+01 (1.28E-01)$-$ & \textbf{1.01E-01 (1.41E-01)$+$} & 1.99E+01 (4.97E-01)$-$ & 2.30E-01 (1.82E-01)$+$ \\
$F_{9}$ & \textbf{1.32E+02 (3.43E+01)} & 1.65E+02 (0.00E+00)$-$ & 1.81E+02 (0.00E+00)$-$ & 1.81E+02 (0.00E+00)$-$ & 1.81E+02 (0.00E+00)$-$ \\
$F_{10}$ & \textbf{0.00E+00 (0.00E+00)} & 1.23E+02 (5.12E+01)$-$ & 1.00E+02 (9.27E-03)$-$ & \boldmath{}\textbf{0.00E+00 (0.00E+00)$\approx$}\unboldmath{} & \boldmath{}\textbf{0.00E+00 (0.00E+00)$\approx$}\unboldmath{} \\
$F_{11}$ & 1.74E-03 (7.97E-03) & 3.20E+02 (4.47E+01)$-$ & 3.00E+02 (0.00E+00)$-$ & \textbf{0.00E+00 (0.00E+00)$+$} & \textbf{0.00E+00 (0.00E+00)$+$} \\
$F_{12}$ & 2.29E+02 (9.70E-01) & \textbf{2.00E+02 (2.04E-04)$+$} & 2.37E+02 (3.17E+00)$-$ & 2.34E+02 (1.46E+00)$-$ & 2.34E+02 (4.51E+00)$-$ \\
\midrule
$+$ / $\approx$ / $-$ & --    & 2/3/7 & 1/5/6 & 2/3/7 & 3/4/5 \\
\bottomrule
\end{tabular}%

\footnotesize
\textsuperscript{*} The Wilcoxon rank-sum tests (with a significance level of 0.05) were conducted between MetaDE and each algorithm individually.
The final row displays the number of problems where the corresponding algorithm performs statistically better ($+$),  similar ($\thickapprox$), or worse ($-$) compared to MetaDE.\\

\label{tab:vsCECTop 20D}%
\end{table*}%



To further assess the performance of MetaDE, we compare it with the top 4 algorithms from the CEC2022 Competition on Single Objective Bound Constrained Numerical Optimization\footnote{\url{https://github.com/P-N-Suganthan/2022-SO-BO}}.
For each algorithm, we set equal FEs as achieved by MetaDE within 60 seconds (refer to Table~\ref{tab:FEs} for details).

The top 4 algorithms from the CEC2022 Competition are {EA4eig}~\cite{EA4eig}, {NL-SHADE-LBC}~\cite{NL-SHADE-LBC}, {NL-SHADE-RSP-MID}~\cite{NL-SHADE-RSP}, and {S-LSHADE-DP}~\cite{S_LSHADE_DP}:
\begin{itemize}
  \item {EA4eig} combines the strengths of four evolutionary algorithms (CMA-ES, CoBiDE, an adaptive variant of jSO, and IDE) using Eigen crossover.
  \item {NL-SHADE-LBC} is a dynamic DE variant that integrates linear bias changes for parameter adaptation, repeated point generation to handle boundary constraints, non-linear population size reduction, and a selective pressure mechanism.
  \item {NL-SHADE-RSP-MID} is an advanced version of NL-SHADE-RSP, which estimates the optimum using the population midpoint, incorporates a restart mechanism, and improves boundary constraint handling.
  \item {S-LSHADE-DP} focuses on maintaining population diversity through dynamic perturbation, adjusting noise intensity to enhance exploration.
\end{itemize}





The experimental results are summarized in Tables \ref{tab:vsCECTop 10D} and \ref{tab:vsCECTop 20D}.
On 10D problems, MetaDE outperforms EA4eig, NL-SHADE-LBC, and NL-SHADE-RSP, while achieving comparable performance to S-LSHADE-DP. 
On 20D problems, MetaDE consistently outperforms the four algorithms.
An additional noteworthy observation is that S-LSHADE-DP exhibits promising performance under a large number of FEs.
}

\subsection{Investigation of Optimal DE Variants}\label{section_Optimal Parameter Analysis}


\begin{table}[h]
  \centering
  \caption{Optimal DE variants obtained by MetaDE on each problem of the CEC2022 benchmark suite. FDC and RIE are two fitness landscape characteristics that measure the difficulty and ruggedness of the problem.}
% Table generated by Excel2LaTeX from sheet 'Exp4 param'
\resizebox{\columnwidth}{!}{
\begin{tabular}{cccccccc}
\toprule
\multicolumn{2}{c}{Problem} & F     & CR    & \multicolumn{2}{c}{Strategy} & FDC & RIE \\
\midrule
\multirow{4}[2]{*}{10D} & $F_{6}$ & 0.70  & 0.99  & \multicolumn{2}{c}{\texttt{rand-to-pbest/1/arith}} & 0.61  & 0.81  \\
      & $F_{8}$ & 0.51  & 0.44  & \multicolumn{2}{c}{\texttt{pbest-to-best/1/bin}} & 0.27  & 0.62  \\
      & $F_{9}$ & 0.02  & 0.03  & \multicolumn{2}{c}{\texttt{current/2/bin}} & 0.08  & 0.82  \\
      & $F_{12}$ & 0.16  & 0.00  & \multicolumn{2}{c}{\texttt{current-to-best/4/bin}} & -0.15  & 0.78  \\
\midrule
\multirow{6}[2]{*}{20D} & $F_{4}$ & 0.13  & 0.71  & \multicolumn{2}{c}{\texttt{rand-to-best/3/bin}} & 0.90  & 0.79  \\
      & $F_{6}$ & 0.67  & 0.99  & \multicolumn{2}{c}{\texttt{pbest-to-rand/1/bin}} & 0.48  & 0.80  \\
      & $F_{7}$ & 0.27  & 0.93  & \multicolumn{2}{c}{\texttt{rand/2/bin}} & 0.26  & 0.78  \\
      & $F_{8}$ & 0.65  & 0.00  & \multicolumn{2}{c}{\texttt{pbest/1/exp}} & 0.12  & 0.40  \\
      & $F_{9}$ & 0.06  & 0.00  & \multicolumn{2}{c}{\texttt{current/2/bin}} & -0.17  & 0.84  \\
      & $F_{12}$ & 0.33  & 0.44  & \multicolumn{2}{c}{\texttt{rand-to-best/2/bin}} & -0.16  & 0.85  \\
\bottomrule
\end{tabular}%
}
\label{tab:optimal_param}
\end{table}%


This part provides an in-depth examination of the optimal DE variants obtained by MetaDE in Section~\ref{sec:expereiment_time}, as summarized in Table \ref{tab:optimal_param}.
The optimal parameters correspond to the best individual in the final population of MetaDE.
The table only displays the optimal parameters for the ten listed problems, as the remaining problems are relatively simpler, with numerous DE variants capable of locating the optimal solutions of the problems. Furthermore, the optimal parameters presented in the table represent the best results of MetaDE derived from the finest run out of 31 independent trials.



All the problems in the table are characterized by both multimodality and non-separability.
Additionally, to further depict the characteristics of the problems' fitness landscapes, we computed both the fitness distance correlation (FDC) \cite{FDC} and the ruggedness of information entropy (RIE) \cite{RIE}; the former measures the complexity (difficulty) of the problems, while the latter characterizes the ruggedness of the landscape.

Analyzing the obtained data, it is evident that no single set of parameters or strategies consistently excels across all problems.
Parameters such as \(F\) and \(CR\) exhibit variability across problems without adhering to a specific trend.
Similarly, the selection of base vectors ($bl$ and $br$) does not show a uniform preference either.
Regarding the fitness landscape characteristics of each problem, the selection of parameters exhibits distinct patterns.
The FDC indicates problem complexity; with simpler problems (higher FDC), such as 10-dimensional $F_6$, $F_8$ and 20-dimensional $F_4$, $F_6$, $F_7$, a larger \(CR\) value is favored. Conversely, smaller \(CR\) values are chosen for problems with lower FDC. A larger \(CR\) tends to facilitate convergence, whereas a \(CR\) close to 0 leads to offspring that change incrementally, dimension by dimension. However, the other characteristic, RIE, does not seem to have a clear association with parameter choices.
The optimal strategies for identical problems across different dimensions exhibit closeness, with $F_8$, $F_9$, and $F_{12}$ demonstrating notably parallel strategies between their 10D and 20D problems.
In terms of crossover strategies ($cs$), it seems to have a preference for binomial crossover. This aligns with the traditional DE configurations.

These observations align with the No Free Lunch (NFL) theorem \cite{NFL}, thus underscoring the importance of distinct optimization strategies tailored for diverse problems.
Conventionally, the optimization strategies have oscillated between seeking a generalist set of parameters for broad applicability and a specialist set tailored for specific problems. However, the dynamic nature of optimization problems, where even minute changes like a different random seed can pivot the problem's dynamics, highlights the challenges of a generalist approach.
In contrast, MetaDE provides a simple yet effective approach, showing promising generality and adaptability.


\subsection{Application to Robot Control}\label{sec:expereiment_brax}
In this experiment, we demonstrate the extended application of MetaDE to robot control.
Specifically, we adopted the evolutionary reinforcement learning paradigm~\cite{ERL} as illustrated in Fig.~\ref{Figure_EvoRL}.
The experiment was conducted on Brax \cite{brax} for robotics simulations with GPU acceleration.

\begin{figure}[!h]
\centering
\includegraphics[scale=0.38]{EvoRL_Workflow.pdf}
\caption{Illustration of robot control via evolutionary reinforcement learning. The evolutionary algorithm optimizes the parameters of a population of candidate policy models for controlling the robotics behaviors. The simulation environment returns rewards achieved by the candidate policy models to the evolutionary algorithm as fitness values.}
\label{Figure_EvoRL}
\end{figure}

This experiment involved three robot control tasks: ``swimmer'', ``hopper'', and ``reacher''.
As summarized in Table \ref{tab:Neural network structures}, we adopted similar policy models for these three tasks, each consisting of a multilayer perceptron (MLP) with three fully connected layers, but with different input and output dimensions.
Consequently, the three policy models comprise 1410, 1539, and 1506 parameters for optimization respectively, where the optimization objective is to achieve maximum reward of each task.
MetaDE, vanilla DE \cite{DE1996}, SHADE~\cite{SHADE2013}, LSHADE-RSP~\cite{LSHADE-RSP2018}, EDEV~\cite{EDEV2018}, CSO \cite{CSO}\footnote{The competitive swarm optimizer (CSO) is a tailored PSO variant for large-scale optimization.}, and CMA-ES~\cite{CMAES} were applied as the optimizer respectively.

%The policy models  were initialized with identical random parameters.
The iteration count for PDE within MetaDE was set to 50, while other algorithms maintained a population size of 100.
Each algorithm was run independently 15 times.
Considering the time-intensive nature of the robotics simulations, we set 60 minutes as the termination condition for each run.


\begin{table}[htbp]
\centering
\caption{Neural network structure of the policy model for each robot control task}
\label{tab:Neural network structures}
\resizebox{\columnwidth}{!}{%
% Table generated by Excel2LaTeX from sheet 'Exp6 brax'
\begin{tabular}{cccccc}
\toprule
\textbf{Task} & \textbf{D} & \textbf{Input} & \textbf{Hidden Layers} &   \textbf{Output}    & \textbf{Overview of objectives} \\
\midrule
Hopper & 1539  & 11    & 32$\times$32 & 3     & balance and jump \\
Swimmer & 1410  & 8     & 32$\times$32 & 2     & maximizing movement \\
Reacher & 1506  & 11    & 32$\times$32 & 2     & precise reaching \\
\bottomrule
\end{tabular}%
}
\end{table}

\begin{figure}[!h]
\centering
\includegraphics[scale=0.3]{brax_all_convergence.pdf}
\caption{The reward curves achieved by MetaDE and peer evolutionary algorithms when applied to each robot control task. }
\label{Figure_brax_all_convergence}
\end{figure}

\begin{figure}[!h]
\centering
\includegraphics[width=\linewidth]{brax_distribution.pdf}
\caption{The fitness distribution of MetaDE's initial population when applied to each robot control task.}
\label{Figure_brax_distribution}
\end{figure}

As shown in Fig. \ref{Figure_brax_all_convergence}, it is evident that MetaDE achieves the best performance in the Swimmer tasks, while slightly outperformed by CMA-ES and CSO in the Hopper and Reacher task.
An interesting observation from the reward curves is that MetaDE almost reaches optimality nearly at the first generation and does not show further significant improvements thereafter.
To elucidate this phenomenon, Fig.~\ref{Figure_brax_distribution} provides the fitness distribution of MetaDE's initial population, indicating that MetaDE harbored several individuals with considerably high fitness from the initial generation.
In other words, MetaDE was able to generate high-performance DE variants for these problems even by random sampling.
This can be attributed to the unique nature of neural network optimization.
As widely acknowledged, the neural network optimization typically features numerous plateaus in the fitness landscape, thus making it relatively easy to find one of the local optima.
MetaDE provides unbiased sampling of parameter settings for generating diverse DE variants.
Even without further evolution, some of the randomly sampled DE variants are very likely to reach the plateaus.
In contrast, the other algorithms are specially tailored with biases; in such large-scale optimization scenarios, the biases can be further amplified, thus making them ineffective.




\section{Conclusion}\label{section Conclusion}
In this paper, we introduced MetaDE, a method that leverages the strengths of DE not only to address optimization tasks but also to adapt and refine its own strategies. This meta-evolutionary approach demonstrates how DE can autonomously evolve its parameter configurations and strategies. 
Our experiments demonstrate that MetaDE has robust performance across various benchmarks, as well as the application in robot control through evolutionary reinforcement learning. 
Nevertheless, the study also emphasizes the complexity of finding universally optimal parameter configurations. The intricate balance between generalization and specialization remains a challenge, and MetaDE has shed light on further research into self-adapting algorithms. 
We anticipate that the insights gained from this work will inspire the development of more advanced meta-evolutionary approaches, pushing the boundaries of evolutionary optimization in even more complex and dynamic environments.





\footnotesize

% \bibliography{manuscript_references}
% !TEX program = pdflatex

\documentclass[journal]{IEEEtran}
\usepackage{lineno}
\modulolinenumbers[5]

%%% color some references
\usepackage{xpatch}
\makeatletter

\makeatother
\usepackage{bm}
\usepackage{array}
\usepackage{graphicx}
\usepackage{amsmath,amssymb,amsthm}
\usepackage{siunitx}
\usepackage{algpseudocode}
\usepackage{algorithmicx}
\usepackage{algorithm}
\usepackage{booktabs}
\usepackage{color}
\usepackage{changepage}
\usepackage{xr}
\usepackage{xr-hyper}
%\usepackage{geometry}%页面设置
\usepackage{graphicx}%图片设置
%\usepackage{subfig}%多个子图
\usepackage{subfigure}
\usepackage{caption}%注释设置
\usepackage{multirow}
\usepackage{float}
\usepackage{soul}
\usepackage[hidelinks]{hyperref}
\usepackage[numbers,sort&compress]{natbib}
\usepackage{bigstrut} %表格大竖线
\usepackage[table]{xcolor} %表格单元格颜色
\usepackage{enumitem} %enumerate标签样式\usepackage{listings}
\usepackage{listings} %listing代码
\usepackage[resetlabels]{multibib}
\newcites{supp}{Supplement References}



\definecolor{dkgreen}{rgb}{0,0.5,0}
\definecolor{gray}{rgb}{0.5,0.5,0.5}
\definecolor{mauve}{rgb}{0.58,0,0.82}
\definecolor{dkblue}{rgb}{0,0,0.6}

 % English theorem environment
 \newtheorem{theorem}{Theorem}
 \newtheorem{lemma}[theorem]{Lemma}
 \newtheorem{proposition}[theorem]{Proposition}
 \newtheorem*{corollary}{Corollary of Theorems 1 and 2}
 \newtheorem{definition}{Definition}
 \newtheorem{remark}{Remark}
 \newtheorem{example}{Example}
 \newenvironment{solution}{\begin{proof}[Solution]}{\end{proof}}

\renewcommand{\algorithmicrequire}{\textbf{Input:}}
\renewcommand{\algorithmicensure}{\textbf{Output:}}

\AtBeginDocument{%
 \abovedisplayskip=5pt plus 4pt minus 2pt
 \abovedisplayshortskip=5pt plus 4pt minus 4pt
 \belowdisplayskip=5pt plus 4pt minus 2pt
 \belowdisplayshortskip=5pt plus 4pt minus 4pt
}

\ifCLASSINFOpdf
\else
\fi

\hyphenation{op-tical net-works semi-conduc-tor}

\bibliographystyle{IEEEtran}

\begin{document}
\captionsetup{font={footnotesize}}
\captionsetup[table]{labelformat=simple, labelsep=newline, textfont=sc, justification=centering}
% paper title
% Titles are generally capitalized except for words such as a, an, and, as,
% at, but, by, for, in, nor, of, on, or, the, to and up, which are usually
% not capitalized unless they are the first or last word of the title.
% Linebreaks \\ can be used within to get better formatting as desired.
% Do not put math or special symbols in the title.
\title{MetaDE: Evolving Differential Evolution by Differential Evolution}
%
%
% author names and IEEE memberships
% note positions of commas and nonbreaking spaces ( ~ ) LaTeX will not break
% a structure at a ~ so this keeps an author's name from being broken across
% two lines.
% use \thanks{} to gain access to the first footnote area
% a separate \thanks must be used for each paragraph as LaTeX2e's \thanks
% was not built to handle multiple paragraphs
%

\author{Minyang Chen, Chenchen Feng,
        and Ran Cheng

        \thanks{
        Minyang Chen was with the Department of Computer Science and Engineering, Southern University of Science and Technology, Shenzhen 518055, China. E-mail: cmy1223605455@gmail.com. }
        \thanks{
        Chenchen Feng is with the Department of Computer Science and Engineering, Southern University of Science and Technology, Shenzhen 518055, China. E-mail: chenchenfengcn@gmail.com. 
        }
        \thanks{
       Ran Cheng is with the Department of Data Science and Artificial Intelligence, and the Department of Computing, The Hong Kong Polytechnic University, Hong Kong SAR, China. E-mail: ranchengcn@gmail.com. (\emph{Corresponding author: Ran Cheng})
        }
        }% <-this % stops a space


% The paper headers
\markboth{Bare Demo of IEEEtran.cls for IEEE Journals}%
{Shell \MakeLowercase{\textit{et al.}}: Bare Demo of IEEEtran.cls for IEEE Journals}
% The only time the second header will appear is for the odd numbered pages
% after the title page when using the twoside option.

% *** Note that you probably will NOT want to include the author's ***
% *** name in the headers of peer review papers.                   ***
% You can use \ifCLASSOPTIONpeerreview for conditional compilation here if
% you desire.


% If you want to put a publisher's ID mark on the page you can do it like
% this:
%\IEEEpubid{0000--0000/00\$00.00~\copyright~2015 IEEE}
% Remember, if you use this you must call \IEEEpubidadjcol in the second
% column for its text to clear the IEEEpubid mark.



% use for special paper notices
%\IEEEspecialpapernotice{(Invited Paper)}

% make the title area
\maketitle

% As a general rule, do not put math, special symbols or citations
% in the abstract or keywords.
\begin{abstract}
As a cornerstone in the Evolutionary Computation (EC) domain, Differential Evolution (DE) is known for its simplicity and effectiveness in handling challenging black-box optimization problems.
While the advantages of DE are well-recognized, achieving peak performance heavily depends on its hyperparameters such as the mutation factor, crossover probability, and the selection of specific DE strategies.
Traditional approaches to this hyperparameter dilemma have leaned towards parameter tuning or adaptive mechanisms.
However, identifying the optimal settings tailored for specific problems remains a persistent challenge.
In response, we introduce MetaDE, an approach that evolves DE's intrinsic hyperparameters and strategies using DE itself at a meta-level.
A pivotal aspect of MetaDE is a specialized parameterization technique, which endows it with the capability to dynamically modify DE's parameters and strategies throughout the evolutionary process.
To augment computational efficiency, MetaDE incorporates a design that leverages parallel processing through a GPU-accelerated computing framework.
Within such a framework, DE is not just a solver but also an optimizer for its own configurations, thus streamlining the process of hyperparameter optimization and problem-solving into a cohesive and automated workflow.
Extensive evaluations on the CEC2022 benchmark suite demonstrate MetaDE's promising performance.
Moreover, when applied to robot control via evolutionary reinforcement learning, MetaDE also demonstrates promising performance.
The source code of MetaDE is publicly accessible at: \url{https://github.com/EMI-Group/metade}.
\end{abstract}



% Note that keywords are not normally used for peerreview papers.
\begin{IEEEkeywords}
Differential Evolution, Meta Evolutionary Algorithm, GPU Computing
\end{IEEEkeywords}


% For peer review papers, you can put extra information on the cover
% page as needed:
% \ifCLASSOPTIONpeerreview
% \begin{center} \bfseries EDICS Category: 3-BBND \end{center}
% \fi
%
% For peerreview papers, this IEEEtran command inserts a page break and
% creates the second title. It will be ignored for other modes.
\IEEEpeerreviewmaketitle



\section{Introduction}
\IEEEPARstart{T}{he} Differential Evolution (DE) \cite{DE1996,DEcontest1996,DEusage1996,DE1997} algorithm, introduced by Storn and Price in 1995, has emerged as a cornerstone in the realm of evolutionary computation (EC) for its prowess in addressing complex optimization problems across diverse domains of science and engineering.
DE's comparative advantage over other evolutionary algorithms is evident in its streamlined design, robust performance, and ease of implementation.
Notably, with just three primary control parameters, i.e., scaling factor, crossover rate, and population size, DE operates efficiently.
This minimalistic design, paired with a lower algorithmic complexity, positions DE as an ideal candidate for large-scale optimization problems.
Its influential role in the optimization community is further cemented by its extensive research attention and successful applications over the past decades \cite{DEsurvey2011, DEsurvey2016, DEpapersurvey2020}, with DE and its derivatives often securing top positions in the IEEE Congress on Evolutionary Computation (CEC) competitions.



Despite the well recognized performance, DE is not without limitations.
Particularly, some studies indicate that DE's optimization process may stagnate if it fails to generate offspring solutions superior to their parents \cite{DEStagnation, neriDEsurvey}.
To avert this stagnation, selecting an appropriate parameter configuration to enhance DE's search capabilities becomes crucial.

However, the No Free Lunch (NFL) theorem \cite{NFL} suggests that a universally optimal parameter configuration is unattainable.
For example, while a higher mutation factor may aid in escaping local optima, a lower crossover probability might be preferable for problems with separability characteristics.

To address the intricate challenge of parameter configuration in DE, researchers often gravitate towards two predominant strategies: \emph{parameter control} and \emph{parameter tuning} \cite{param1999,paramTun2012,paramTun2020}.
Parameter control is a dynamic approach wherein the algorithm's parameters are adjusted on-the-fly during its execution.
This adaptability allows the algorithm to respond to the evolving characteristics of the problem landscape, enhancing its chance of finding optimal or near-optimal solutions.
Notably, DE has incorporated this strategy in several of its variants.
For instance, jDE \cite{jDE2006} adjusts the mutation factor and crossover rate during the run, while SaDE \cite{SaDE2008} dynamically chooses a mutation strategy based on its past success rates. Similarly, JaDE \cite{JADE2009} and CoDE \cite{CoDE2011} employ adaptive mechanisms to modify control parameters and mutation strategies, respectively.

In contrast, parameter tuning is a more static methodology, wherein the optimal configuration is established prior to the algorithm's initiation.
It aims to discover a parameter set that consistently demonstrates robust performance across various runs and problem instances.
Despite its potential for reliable outcomes, parameter tuning is known for its computational intensity, often necessitating dedicated optimization efforts or experimental designs to identify the optimal parameters, which may explain its limited exploration in the field.
Viewed as an optimization challenge, parameter tuning is also referred to as meta-optimization \cite{metaEAPhD2010}.
This perspective gave rise to \emph{MetaEA}, which optimizes the parameters of an EA using another EA.

Despite MetaEA's methodological elegance and simplicity, it confronts the significant challenge of depending on extensive function evaluations.
Fortunately, the inherent parallelism within MetaEA, across both meta-level and base-level populations, renders it particularly amenable to parallel computing environments.
However, a notable disparity exists between methodological innovations and the availability of advanced computational infrastructures, thus limiting MetaEA's potential due to the lack of advanced hardware accelerations such as GPUs.
To bridge this gap, we introduce the \emph{MetaDE} approach, which embodies the MetaEA paradigm by employing DE in a meta-level to guide the evolution of a specially tailored Parameterized Differential Evolution (PDE).

Designed with adaptability in mind, PDE can flexibly adjust its parameters and strategies, paving the way for a wide range of DE configurations.
As PDE interacts with the optimization problem at hand, the meta-level DE observes and refines PDE's settings to better align with the problem's characteristics.
Amplifying the efficiency of this nested optimization approach, MetaDE is integrated with a GPU-accelerated EC framework, thus weaving together parameter refinement and direct problem-solving into a seamless end-to-end approach to black-box optimization.
In summary, our main contributions are as follows.

\begin{itemize}
\item \textbf{Parameterized Differential Evolution:}
We have introduced Parameterized Differential Evolution (PDE), a variant of DE with augmented parameterization.
Unlike traditional DE algorithms that come with fixed mutation and crossover strategies, PDE’s architecture offers users the flexibility to adjust these parameters and strategies to fit the problem at hand.
This design not only allows for the creation of diverse DE configurations tailored for specific challenges but also ensures efficient computation.
To achieve this, all core operations of PDE, including mutation, crossover, and evaluation, have been optimized for parallel execution to harness advancement of GPU acceleration.

\item \textbf{MetaDE:}
Building on the MetaEA paradigm, we have designed the MetaDE approach.
Specifically, MetaDE employs a meta-level DE as an \texttt{evolver} to iteratively refine PDE's hyperparameters, which is guided by performance feedback from multiple PDE instances acting as \texttt{executors}.
This continuous optimization ensures PDE's configurations remain aligned with evolving problem landscapes.
Moreover, we have incorporated several specialized methods to further enhance the performance of MetaDE.



\item \textbf{GPU-accelerated Implementation:}
Breaking away from the limitations of conventional parameter tuning, we integrate MetaDE with a GPU-accelerated computing framework
 -- EvoX~\cite{evox}, which enhances MetaDE's computational prowess for facilitating swifter evaluations and algorithmic refinements.
With this specialized implementation, MetaDE provides an efficient and automated end-to-end approach to black-box optimization.
\end{itemize}



The subsequent sections are organized as follows. Section \ref{section_Preliminary} presents some preliminary knowledge for this work.
Section \ref{section_The proposed metade} elucidates the intricacies of the proposed approach, including PDE and the MetaDE.
Section \ref{section_Experimental_study} showcases the experimental results.
Finally, Section \ref{section Conclusion} wraps up the discourse and points towards avenues for future work.

\section{Preliminaries}\label{section_Preliminary}

\subsection{DE and its Parameter Adaption}\label{subsection_DE and Parameter adaption}
\subsubsection{Overview of DE}
As a typical EC algorithm, DE's essence lies in its differential mutation mechanism that drives the evolution of a population.
The operational cycle of DE unfolds iteratively, with each iteration embodying specific phases, as elaborated in Algorithm \ref{Alg_DE}:
\begin{enumerate}[label=\arabic*.]
  \item \textbf{Initialization} (Line 1):
  The algorithm initializes a set of potential solutions.
  Each of these solutions, representing vectors of decision variables, is randomly generated within the search space boundaries.
  \item \textbf{Mutation} (Lines 6-7):
  Each solution undergoes mutation to produce a mutant vector.
  This mutation process involves combinations of different individuals to form the mutant vectors.
  \item \textbf{Crossover} (Line 8):
  The crossover operation interchanges components between mutants and the original solutions to generate a trial vector.
  \item \textbf{Selection} (Lines 10-12): The trial vector competes against the original solution based on fitness, with the better solution progressing to the next generation.
\end{enumerate}


DE progresses through cycles of mutation, crossover, and selection, persisting until it encounters a termination criterion.
This could manifest as either reaching a predefined number of generations or achieving a target fitness threshold.
The algorithm's adaptability allows for the spawning of myriad DE variants by merely tweaking its mutation and crossover operations.
Specifically, DE variants follow a unified naming convention: \texttt{DE/x/y/z}, where \texttt{x} identifies the base vector used for mutation, \texttt{y} quantifies the number of difference involved, and \texttt{z} typifies the crossover method employed.
For example, the DE variant as presented in Algorithm \ref{Alg_DE} is named as \texttt{DE/rand/1/bin}.

{\linespread{1.1}
\begin{algorithm}
\small
\caption{DE}\label{Alg_DE}
\begin{algorithmic}[1]
  \Require {$D$, $NP$, $F$, $CR$, $G_{max}$}
  \State Initialize population $\mathbf{X} = \{\mathbf{x}_1, \mathbf{x}_2, \dots, \mathbf{x}_{\scalebox{0.5}{$\textit{NP}$}}\}$
  \State Evaluate the fitness of each individual in the population
  \State $g = 0$
  \While{$g \leq G_{max}$}
    \For{$i = 1$ to $NP$}
      \State Randomly select $\mathbf{x}_{r_1}$, $\mathbf{x}_{r_2}$, and $\mathbf{x}_{r_3}$ from $\mathbf{X}$,
      \Statex \qquad \quad such that $r_1 \neq r_2 \neq r_3 \neq i$
      \State Compute the mutant vector: $\mathbf{v}_i = \mathbf{x}_{r_1} + F \cdot (\mathbf{x}_{r_2} - \mathbf{x}_{r_3})$
      \State Perform crossover for each variable between $\mathbf{x}_i$ and $\mathbf{v}_i$:
      \begin{align*}
        \qquad \quad u_{i,j}=\begin{cases}
          v_{i,j},\ \text{if } \text{rand}(0, 1) \leq CR \text{ or } j = \text{randint}(1, D) \\
          x_{i,j},\ \text{otherwise}
        \end{cases}
      \end{align*}
      \State Evaluate the fitness of $\mathbf{u}_i$
      \If{$\textrm{f}(\mathbf{u}_i) \leq \textrm{f}(\mathbf{x}_i)$}
        \State Replace $\mathbf{x}_i$ with $\mathbf{u}_i$ in the population
      \EndIf
    \EndFor
    \State $g = g + 1$
  \EndWhile
  \State\Return the best fitness
\end{algorithmic}
\end{algorithm}
{\linespread{1}

\subsubsection{Parameter Modulation in DE}
DE employs a unique mutation mechanism, which adapts to the problem's natural scaling.
By adjusting the mutation step's size and orientation to the objective function landscape, DE embraces the \emph{contour matching principle} \cite{DEbook2006}, which promotes basin-to-basin transfer for enhancing the convergence of the algorithm.

At the core of DE's mutation is the scaling factor \( F \).
This factor not only determines the mutation's intensity but also governs its trajectory and ability to bypass local optima.
Commonly, \( F \) is set within the $[0.5, 1]$ interval, with a starting point often at 0.5.
While values outside the $[0.4, 1]$ range can sometimes yield good results, an \( F \) greater than 1 tends to slow convergence.
Conversely, values up to 1 generally promise swifter and more stable outcomes \cite{EPSDE2011}.
Nonetheless, to deter settling at suboptimal solutions too early, \( F \) should be adequately elevated.

Parallel to mutation, DE incorporates a uniform crossover operator, which is often labeled as discrete recombination or binomial crossover in the GA lexicon.
The crossover constant \( CR \) also plays a pivotal role, which determines the proportion of decision variables to be exchanged during the generation of offspring.
A low value for \( CR \) ensures only a small portion of decision variables are modified per iteration, thus leading to axis-aligned search steps.
As \( CR \) increases closer to 1, offspring tend to increasingly reflect their mutant parent, thereby curbing the generation of orthogonal search steps \cite{DEsurvey2011}.

For classical DE configurations, such as \texttt{DE/rand/1/bin}, rotational invariance is achieved only when \( CR \) is maxed out at 1.
Here, the crossover becomes wholly vector-driven, and offspring effectively mirror their mutants.
However, the optimal \( CR \) is intrinsically problem-dependent.
Empirical studies recommend a \( CR \) setting within the $[0, 0.2]$ range for problems characterized by separable decision variables.
Conversely, for problems with non-separable decision variables, a \( CR \) in the proximity of $[0.9, 1]$ is more effective \cite{DEsurvey2011}.

The adaptability of DE is evident in its wide spectrum of variants, each distinct in its mutation and crossover strategies with delicate parameter modulations.
In the following, we will detail seven mutation strategies and three crossover strategies, all of which are widely-recognized in state-of-the-art DE variants.
Here, the subscript notation in \( \textbf{x} \) specifies the individual selection technique.
For instance, \( \textbf{x}_r \) and \( \textbf{x}_{best} \) correspond to randomly selected and best-performing individuals respectively, whereas
\( \textbf{x}_i \) represents the currently evaluated individual.

\textbf{Mutation Strategies}:

\begin{enumerate}[label=\arabic*.]
  \item \texttt{DE/rand/1}:
        \begin{eqnarray}\label{equ_mutation rand}
        \begin{aligned}
        \mathbf{v}_i=\mathbf{x}_{r_1}+F \cdot\left(\mathbf{x}_{r_2}-\mathbf{x}_{r_3}\right).
        \end{aligned}
        \end{eqnarray}

  \item \texttt{DE/best/1}:
        \begin{eqnarray}\label{equ_mutation best}
        \begin{aligned}
        \mathbf{v}_i=\mathbf{x}_{\text {best }}+F \cdot\left(\mathbf{x}_{r_1}-\mathbf{x}_{r_2}\right).
        \end{aligned}
        \end{eqnarray}

  \item \texttt{DE/rand/2}:
        \begin{eqnarray}\label{equ_mutation rand2}
        \begin{aligned}
        \mathbf{v}_i & =\mathbf{x}_{r_1}+F \cdot\left(\mathbf{x}_{r_2}-\mathbf{x}_{r_3}\right)+F \cdot\left(\mathbf{x}_{r_4}-\mathbf{x}_{r_5}\right).
        \end{aligned}
        \end{eqnarray}

  \item \texttt{DE/best/2}:
        \begin{eqnarray}\label{equ_mutation best2}
        \begin{aligned}
        \mathbf{v}_i & =\mathbf{x}_{\text {best}}+F \cdot\left(\mathbf{x}_{r_1}-\mathbf{x}_{r_2}\right)+F \cdot\left(\mathbf{x}_{r_3}-\mathbf{x}_{r_4}\right).
        \end{aligned}
        \end{eqnarray}

  \item \texttt{DE/current-to-best/1}:
        \begin{eqnarray}\label{equ_mutation current2best}
        \begin{aligned}
        \mathbf{v}_i =\mathbf{x}_i+F \cdot\left(\mathbf{x}_{\text {best }}-\mathbf{x}_i\right)+F \cdot\left(\mathbf{x}_{r_1}-\mathbf{x}_{r_2}\right).
        \end{aligned}
        \end{eqnarray}
    The above five classical mutation strategies, introduced by Storn and Price \cite{DEbook2006}, cater to various problem landscapes.
    For instance, the `rand' variants help maintain population diversity, while strategies using two differences typically produce more diverse offspring than those relying on a single difference.

  \item \texttt{DE/current-to-pbest/1}:
        \begin{eqnarray}\label{equ_mutation current2pbest}
        \begin{aligned}
        \mathbf{v}_i =\mathbf{x}_i+F \cdot\left(\mathbf{x}_{\text {pbest}}-\mathbf{x}_i\right)+F \cdot\left(\mathbf{x}_{r_1}-\mathbf{x}_{r_2}\right).
        \end{aligned}
        \end{eqnarray}
    This strategy originates from JaDE \cite{JADE2009}. $\mathbf{x}_{\text{pbest}}$ is randomly selected from the top \emph{p}\% of individuals in the population (typically the top 10\%) to strike a balance between exploration and exploitation.


  \item \texttt{DE/current-to-rand/1}:
        \begin{eqnarray}\label{equ_mutation current2rand}
        \begin{aligned}
        &\mathbf{u}_i =\mathbf{x}_i+K_i\cdot\left(\mathbf{x}_{r_1}-\mathbf{x}_i\right)+F \cdot\left(\mathbf{x}_{r_2}-\mathbf{x}_{r_3}\right).
        \end{aligned}
        \end{eqnarray}
    Here, \(K_i\) is a random number from \(U(0,1)\).
    This strategy, originally proposed in \cite{DEintro1999}, emphasizes rotational invariance. By bypassing the crossover phase, it directly yields the trial vector \(\mathbf{u}_i\). Thus, it is ideal for addressing non-separable rotation challenges and has been a cornerstone for multiple adaptive DE variations.

  \end{enumerate}

\textbf{Crossover Strategies:}

\begin{enumerate}[label=\arabic*.]
  \item Binomial Crossover:
    \begin{eqnarray}\label{equ_cross bin}
    \begin{aligned}
    u_{i, j}= \begin{cases}v_{i, j}, & \text { if } r \leq C R \text { or } j=j_{\mathrm{rand}} \\ x_{i, j}, & \text {otherwise},\end{cases}
    \end{aligned}
    \end{eqnarray}
    where \(j_{\mathrm{rand}}\) is a random integer between 1 and \( D \). This strategy is a cornerstone in DE.

  \item Exponential Crossover:
  \begin{eqnarray}\label{equ_cross exp}
    \begin{aligned}
    \small
    u_{i, j}= \begin{cases}v_{i, j}{ } & \text { if } j=\langle n\rangle_d,\langle n+1\rangle_d,...,\langle n+L-1\rangle_d \\ x_{i, j} & \text {otherwise},\end{cases}
    \end{aligned}
    \end{eqnarray}
    where \(\langle \rangle_d\) is a modulo operation with \(D\) and \(L\) representing the crossover length, following a censored geometric distribution with a limit of \(D\) and probability of \(CR\).
    By focusing on consecutive variables, this strategy excels in handling problems with contiguous variable dependencies.

  \item Arithmetic Recombination:
  \begin{eqnarray}\label{equ_cross arith}
    \begin{aligned}
    \mathbf{u}_i=\mathbf{x}_i + K_i\cdot(\mathbf{v}_i - \mathbf{x}_i),
    \end{aligned}
    \end{eqnarray}
    where \(K_i\) is a random value from \(U(0,1)\).
    Exhibiting rotational invariance, this strategy, when combined with the \texttt{DE/rand/1} mutation, results in the \texttt{DE/current-to-rand/1} strategy \cite{DEsurvey2011}, as described by:
    \begin{eqnarray}\label{equ_rand_1_arith}
    \begin{aligned}
    \mathbf{u}_i&=\mathbf{x}_i + K_i\cdot(\mathbf{v}_i - \mathbf{x}_i)\\
    &=\mathbf{x}_i + K_i(\mathbf{x}_{r_1}+F\cdot(\mathbf{x}_{r_2}-\mathbf{x}_{r_3}) - \mathbf{x}_i)\\
    &=\mathbf{x}_i + K_i(\mathbf{x}_{r_1} - \mathbf{x}_i)+ K_i\cdot F(\mathbf{x}_{r_2}-\mathbf{x}_{r_3}),
    \end{aligned}
    \end{eqnarray}
    which is equivalent to Eq. (\ref{equ_mutation current2rand}).

\end{enumerate}


\subsubsection{Adaptive DE}
The development of parameter adaption in DE has witnessed significant advancements over time, from initial endeavors in parameter adaptation to recent sophisticated methods that merge multiple strategies.
This subsection traces the chronological advancements, emphasizing the pivotal contributions and their respective impacts on adaptive DE.

The earliest phase in DE's adaption centered on the modification of the crossover rate \( CR \) .
Pioneering algorithms such as SPDE \cite{SPDE2003} incorporated \( CR \) within the parameter set of individuals, enabling its simultaneous evolution with the decision variables of the problem to be solved.
This strategy was further refined by SDE \cite{SDE2005}, which assigned \( CR \) for each individual based on a normal distribution. Subsequent research efforts shifted focus to the scaling factor \( F \).
In this context, DETVSF \cite{DETVSF2005} dynamically adjusted \( F \), fostering exploration during the algorithm's nascent stages and pivoting to exploitation in later iterations.
Building on this, FaDE \cite{FaDE2005} employed fuzzy logic controllers to optimize mutation and crossover parameters.

The DESAP \cite{DESAP2006} algorithm marked a significant paradigm shift by introducing self-adapting populations and encapsulating control parameters within individuals.
Successive contributions like jDE \cite{jDE2006}, SaDE \cite{SaDE2008}, and JaDE \cite{JADE2009} accentuated the significance of parameter encoding, integrated innovative mutation strategies, and emphasized archiving optimization trajectories using external repositories.
Further, EPSDE \cite{EPSDE2011} and CoDE \cite{CoDE2011} enhanced the offspring generation process, amalgamating multiple strategies with randomized parameters.

The contemporary landscape of adaptive DE is characterized by complex methodologies and refined strategies.
Algorithms such as SHADE \cite{SHADE2013} and LSHADE \cite{LSHADE2014} championed the utilization of success-history mechanisms and dynamic population size modifications.
Notable developments like ADE \cite{ADE2014} introduced a biphasic parameter adaptation mechanism.
The domain further expanded with algorithms like LSHADE-RSP \cite{LSHADE-RSP2018}, IMODE \cite{IMODE2020}, and LADE \cite{LADE2023}, emphasizing mechanisms such as selective pressure, the integration of multiple DE variants, and the automation of the learning process.

Undoubtedly, the adaptive DE domain has witnessed transformative growth, with each phase of its evolution contributing to its current sophistication.
However, despite these advancements, many adaptive strategies remain empirical and hinge on manual designs, while their effectiveness is not universally guaranteed.

\subsection{Distributed DE}
The integration of distributed (i.e., multi-population) strategies also significantly enhances the efficacy of DE. 
Leading this advancement, Weber \textit{et al.} conducted extensive research on scale factor interactions and mechanisms within a distributed DE framework \cite{weber2010study, weber2011study, weber2013study}, followed by ongoing developments along the pathway \cite{DEpapersurvey2020}.
For example, some works such as EDEV \cite{EDEV2018}, MPEDE \cite{MPEDE2015} and IMPEDE \cite{IMPEDE} adopted multi-population frameworks to ensemble various DE variants/operators,
while the other works  such as DDE-AMS \cite{DDE-AMS} and DDE-ARA \cite{DDE-ARA} employed multiple populations for adaptive resource allocations.

Despite the achievements, current implementations of distributed DE often focus predominantly on algorithmic improvements, while overlooking potential enhancements from advanced hardware accelerations such as GPU computing. 
Besides, the design of these distributed strategies often features intricate and rigid configurations that lack proper flexibility.



\subsection{MetaEAs}\label{subsection_Meta-EA}

Generally, the term \emph{meta} refers to a higher-level abstraction of an underlying concept, often characterized by its \emph{recursive} nature.
In the context of EC, inception of the Meta Evolutionary Algorithms (MetaEAs) can be traced back to the pioneering works of Mercer and Sampson \cite{metaplan1978} in the late 1970s.
Under the initiative termed \emph{meta-plan}, their pioneering efforts aimed at enhancing EA performance by optimizing its parameters through another EA.
Although sharing similarities with hyperheuristics \cite{Hyperheu2013,Hyperheu2020,NeriHyperspam}, a major difference distinguishes MetaEAs: while hyperheuristics often delve into selecting and fine-tuning a set of predefined algorithms, MetaEAs concentrate on the paradigm of refining the parameters of EAs by EAs.
Notably, MetaEAs are also akin to ensemble of algorithms, such as EDEV \cite{EDEV2018} and CoDE \cite{CoDE2011}, which amalgamate diverse algorithms to ascertain the most efficacious among them.


Advancing the meta-plan concept, MetaGA \cite{MetaGA1986} emerged as a significant milestone.
Here, a genetic algorithm (GA) was deployed to fine-tune six intrinsic control parameters, namely: population size, crossover rate, mutation rate, generation gap, scaling window, and selection strategy.
The efficacy of this approach was gauged using dual metrics: online and offline performance.

The evolution of the concept continued with MetaEP \cite{metaEP1991}, which offers a meta-level evolutionary programming (EP) that could concurrently evolve optimal parameter settings.
Another pivotal contribution was the Parameter Relevance Estimation and Value Calibration (REVAC) \cite{REVAC2007}, which served as a meta estimation of distribution algorithm (MetaEDA).
Utilizing a GA at its core, REVAC iteratively discerned promising parameter value distributions within the configuration space.

Innovations in the domain persisted with the Gender-based GA (GGA) \cite{GGA2009}, inspired by natural gender differentiation, and other notable methods like MetaCMAES \cite{metaCMAES2012}.
As articulated in the PhD thesis by Pedersen \cite{metaEAPhD2010}, a profound insight into MetaEA revealed that while contemporary optimizers endowed with adaptive behavioral parameters offered advantages, they were often eclipsed by streamlined optimizers under appropriate parameter tuning.
This thesis, which embraced DE as one of its optimization tools, employed the Local Unimodal Sampling (LUS) heuristic for tuning parameters such as \( NP \), \( F \), and \( CR \).

Culminating the discourse, the work in \cite{metaEAdistributed} demonstrated the scalability of MetaEAs by harnessing it within a large-scale distributed computing environment.
With the ($\mu$, $\lambda$)$-$ES steering the meta-level tuning, base-level algorithms like GA, ES, and DE were adeptly optimized.
For DE, parameters optimized encompassed \( NP \), mutation operator, \( F \), \( CR \), and \( PF \) (parameter for the \emph{either-or} strategy), enhancing MetaEAs' prowess in addressing intricate, large-scale optimization problems.
Recently, the MetaEA paradigm has also been employed for automated design of ensemble DE \cite{EDE}.

The field of MetaEAs has shown steady progress since its inception in the 1970s.
However, despite the achievements, the landscape of MetaEAs research still confronts certain limitations.
Notably, the research, while promising, has predominantly remained confined to smaller-scale implementations.
The anticipated leap to large-scale experiments, especially those that might benefit from GPU acceleration, remains largely uncharted. This underscores an imperative need for more extensive empirical validations and the exploration of contemporary computational resources to fully realize the potential of MetaEAs.

\subsection{GPU-accelerated EC Framework}\label{subsection_Meta-optimization}
To capitalize on the advancements of modern computing infrastructures, we have seamlessly integrated our proposed MetaDE with EvoX~\cite{evox}, a distributed GPU-accelerated computing framework for scalable EC.
This integration ensures that MetaDE enables efficient execution and optimization for large-scale evaluations.

The EvoX framework provides several distinctive features.
Primarily, it is designed for optimal performance across diverse distributed systems and is tailored to manage large-scale challenges.
Its user-friendly functional programming model simplifies the EC algorithm development process, reducing inherent complexities.
The framework cohesively integrates data streams and functional elements into a comprehensive workflow, underpinned by a sophisticated hierarchical state management system.
Moreover, EvoX features a rich library of EC algorithms, proficient in addressing a wide array of tasks, from black-box optimization to advanced areas such as deep neuroevolution and evolutionary reinforcement learning.


\section{Proposed Approach}\label{section_The proposed metade}
The foundational premise of MetaDE is to utilize a core DE algorithm to evolve an ensemble of parameterized DE variants.
Within this framework, the core DE, which tunes the parameters, is termed the \texttt{evolver}.
In contrast, each parameterized DE variant, which optimizes the problem at hand, is termed the \texttt{executor}.
This section commences by augmenting the parameterization of DE in a more general manner, such that DE is made \emph{evolvable} by spawning various DE variants by modulating the parameters.
Then, this section details the integration of proposed MetaDE within a meta-framework, together with a brief introduction to the GPU-accelerated implementation.

\subsection{Augmented Parameterization of DE}\label{section_PDE}
To make DE evolvable, this subsection introduces the Parameterized Differential Evolution (PDE), an extension of the standard DE designed to augment its flexibility through the parameterization of mutation and crossover strategies.
While PDE retains the foundational principles of standard DE, its distinctiveness lies in its capability to generate a multitude of strategies by modulating the parameters.

In the standard DE framework, tunability is constrained to the adjustments of the \( F \) and \( CR \) parameters, and strategies are bound by predefined rules.
To augment this limited flexibility, PDE introduces a more granular parameterization.
Building upon DE's notation \texttt{DE/x/y/z}, PDE encompasses six parameters: \( F \) (scale factor), \( CR \) (crossover rate), \( bl \) (base vector left), \( br \) (base vector right), \( dn \) (difference number), and \( cs \) (crossover scheme).
The combined roles of \( bl \) and \( br \) determine the base vector, leading to a strategy notation for PDE expressed as \texttt{DE/bl-to-br/dn/cs}.

Each parameter's nuances and the array of strategy combinations they enable are elaborated upon in the subsequent sections.
Fig. \ref{Figure_structure} provides a visual representation of these parameters, delineated within dashed boxes.


\begin{figure*}[!h]
\centering
\includegraphics[scale=0.6]{fix_structure.pdf}
\caption{
Parameter delineation of PDE and their respective domains.
PDE comprehensively parameterizes DE, endorsing unrestricted parameter and strategy modifications.
In this schema, \( F \) and \( CR \) are continuous parameters, whereas others are categorical.
The dashed-line boxes exhibit their specific value ranges.
The mutation function is derived from the base vector left, base vector right, and difference number parameters.
}
\label{Figure_structure}
\end{figure*}

\subsubsection{Scale Factor \( F \) and Crossover Rate \( CR \)}
The scale factor \( F \) and crossover rate \( CR \) serve as pivotal parameters in DE, both represented as real numbers.

The \( F \) parameter regulates the differential variation among population entities.
Elevated values induce exploratory search behaviors, while lower values encourage more exploitation.
Although Storn and Price originally identified [0, 2] as an effective domain for \( F \) \cite{DE1997}, contemporary DE variants deem \( F \leq 1 \) as more judicious \cite{JADE2009,CoDE2011,EPSDE2011,SHADE2013,LSHADE2014}.
Consequently, PDE constrains \( F \) within [0,1].

On the other hand, \( CR \) controls the recombination extent during crossover.
Higher \( CR \) make it more likely to try out new gene combinations from changes, while lower values keep the genes more stable and similar to the original.
As a rate (or probability), \( CR \) is confined to the [0, 1] interval.

\subsubsection{Augmented Parameterization of Mutation}
The distinctiveness of PDE lies in its ability to generate a multitude of mutation strategies by modulating parameters $bl, br$, and $dn$.
Parameters $bl$ (base vector left) and $br$ (base vector right) select one of four possible vectors:
\begin{itemize}
  \item  \texttt{rand}: A randomly chosen individual from the population.
  \item  \texttt{best}: The best individual in the population.
  \item  \texttt{pbest}: A random selection from the top \( p\% \) of individuals.
  \item  \texttt{current}: The present parent individual.
\end{itemize}
Parameter \( dn \), controls the number of differences in mutation, assumes values in the set \{1, 2, 3, 4\}.
Specifically, each difference \( \Delta \) captures the difference between two unique and randomly selected individuals from the population.
Therefore, the mutation formulation \texttt{DE/bl-to-br/dn} is:
\begin{align}\label{equ_PDE_mutation}
\mathbf{v} = \mathbf{x}_{bl} + F \cdot (\mathbf{x}_{br} - \mathbf{x}_{bl}) + F \cdot (\Delta_1 + ... +\Delta_{dn}).
\end{align}
In the implementation of PDE, \( bl \) and \( br \) are encoded as: 1: \texttt{rand}, 2: \texttt{best},  3: \texttt{pbest}, and   4: \texttt{current}.
In addition, if both \( bl \) and \( br \) assume identical values, the term $F \cdot (\mathbf{x}_{br} - \mathbf{x}_{bl})$ will disappear. The base vector takes the value of $\mathbf{x}_{bl}$ directly and the mutation strategy then becomes non-directional (e.g., \texttt{DE/rand/1}).

\subsubsection{Augmented Parameterization of Crossover}
Parameter \( cs \) defines the crossover strategies in PDE, comprising those elaborated in Section~\ref{subsection_DE and Parameter adaption}: binomial crossover, exponential crossover, and arithmetic recombination.
Specifically, \( cs \) is encoded as:   1: \texttt{bin}, 2: \texttt{exp}, and 3: \texttt{arith}, representing binomial crossover, exponential crossover, and arithmetic recombination respectively.

As a result, the augmented parameterizations for mutation and crossover, denoted as \texttt{DE/bl-to-br/dn/cs}, will give rise to a spectrum of 192 distinctive strategies in total.
This breadth allows for the encapsulation of mainstream strategies detailed in Section~\ref{subsection_DE and Parameter adaption}, as illustrated in Table \ref{tab_param encoding}.
When synergized with \( F \) and \( CR \), this culminates in a comprehensive parameter configuration landscape.


% Table generated by Excel2LaTeX from sheet 'Sheet1'
\begin{table}[htbp]
  \centering
  \caption{The encoding of typical DE variants by the proposed PDE.}
         \renewcommand{\arraystretch}{1.1}
 \renewcommand{\tabcolsep}{10pt}
    \begin{tabular}{ccccc}
    \toprule
    strategy & $bl$  & $br$  & $dn$  & $cs$ \\
    \midrule
    \texttt{DE/rand/1/bin} & 1     & 1     & 1     & 1 \\
    \midrule
    \texttt{DE/best/1/bin} & 2     & 2     & 1     & 1 \\
    \midrule
    \texttt{DE/current-to-best/1/bin} & 4     & 2     & 1     & 1 \\
    \midrule
    \texttt{DE/rand/2/bin} & 1     & 1     & 2     & 1 \\
    \midrule
    \texttt{DE/best/2/bin} & 2     & 2     & 2     & 1 \\
    \midrule
    \texttt{DE/current-to-pbest/1/bin} & 4     & 3     & 1     & 1 \\
    \midrule
    \texttt{DE/current-to-rand/1*} & 1     & 1     & 1     & 3 \\
    \bottomrule
    \end{tabular}%
  \label{tab_param encoding}%
    \vspace{0.5em}

      \footnotesize
    \textsuperscript{*} As per Eq. (\ref{equ_rand_1_arith}), \texttt{DE/current-to-rand/1} is equivalent to \texttt{DE/rand/1/arith}.\\
\end{table}%

Algorithm \ref{Alg_PDE} details PDE's procedure.
Its distinct attribute is Line 3, where the mutation function is shaped by \( bl \), \( br \), and \( dn \).
Line 4 designates the crossover function based on \( cs \).
The evolutionary phase commences at Line 6.
Notably, PDE's concurrent mutation and crossover operations (Lines 7-10) are {tensorized} to facilitate parallel offspring generation, which deviate from the conventional sequential operations in standard DE.
With such a tailored procedure, the computational efficiency can be substantially improved.


\begin{algorithm}
\small
\caption{Parameterized DE (PDE)}\label{Alg_PDE}
\begin{algorithmic}[1]
  \Require {$D$, $NP$, $G_{max}$, $F$, $CR$, $bl$ (base vector left), $br$ (base vector right), $dn$ (difference number), $cs$ (crossover scheme)}
  \State Initialize population $\mathbf{X} = \{\mathbf{x}_1, \mathbf{x}_2, \dots, \mathbf{x}_{\scalebox{0.5}{$\textit{NP}$}}\}$
  \State Evaluate the fitness of each individual in the population
  \State Generate mutation function $M(\mathbf{X})$ according to $bl$, $br$, $dn$:
  \Statex $\mathbf{v} = \mathbf{x}_{bl} + F \cdot (\mathbf{x}_{br} - \mathbf{x}_{bl}) + F \cdot (\Delta_1 + ... +\Delta_{dn})$
  \State According to $cs$, choose a crossover function $C(\mathbf{V}, \mathbf{X})$ in (\ref{equ_cross bin}-\ref{equ_cross arith})
  \State $g = 0$
  \While{$g \leq G_{max}$}
  \State Generate $NP$ mutant vectors: $\mathbf{V} = M(\mathbf{X})$
  \State Perform crossover for all mutant vectors: $\mathbf{U} = C(\mathbf{V}, \mathbf{X})$
  \State Evaluate the fitness of $\mathbf{U}$
  \State Make selection between $\mathbf{U}$ and $\mathbf{X}$
  \State $g = g + 1$
  \EndWhile
  \State\Return the best fitness
\end{algorithmic}
\end{algorithm}


\subsection{Architecture of MetaDE}
Atop the proposed PDE, this subsection further introduces the architecture of MetaDE.
The main target of MetaDE is to evolve the parameters of PDE through an external DE, empowering PDE to identify optimal parameters tailored to the target problem.
% Every computational phase of MetaDE benefits from extensive parallelization with GPU acceleration integration.

\begin{figure}[t]
\centering
\includegraphics[scale=0.9]{MetaDE.pdf}
\caption{
Architecture of MetaDE.
Within this architecture, a conventional DE algorithm operates as an \texttt{evolver}, where its individual $\mathbf{x}_i$ represents a distinct parameter configuration $\mathbf{\theta}_i$.
These configurations are relayed to PDE  to instantiate diverse DE variants as the \texttt{executors}.
Each \texttt{executor} then evolves its distinct population and returns the best fitness $y^*$ as identified, which is subsequently set as the fitness of $\mathbf{x}_i$.
}
\label{Figure_MetaDE}
\end{figure}

As illustrated in Fig. \ref{Figure_MetaDE}, MetaDE is structured with a two-tiered optimization architecture.
The upper tier, termed the \texttt{evolver}, leverages DE to evolve the parameters of PDE. In contrast, the lower tier consists of a collection of \texttt{executors} that each run the parameterized PDE instance to optimize the objective function.
Every individual in the \texttt{evolver}, represented as $\mathbf{x}_i$, is decoded into a parameter configuration $\bm{\theta}_i$ with six elements: $F$, $CR$, $bl$, $br$, $dn$, and $cs$.
For the evaluation of each individual, the configuration $\bm{\theta}_i$ is directed to its respective \texttt{executor} $\textrm{PDE}_i$ for objective function optimization.
The final fitness $y^*$ as identified by each \texttt{executor}, is subsequently set as the fitness of the corresponding $\mathbf{x}_i$ individual.

The architecture of MetaDE is streamlined for simplicity.
Building upon this architecture, MetaDE integrates two tailored components: the \emph{one-shot evaluation method} and the \emph{power-up strategy}.
These components further enhance the adaptability and efficiency of the \texttt{executors}, thereby elevating the overall performance of MetaDE.

\subsection{One-shot Evaluation Method}
Within the context of an \texttt{executor} driven by DE itself, the inherent stochastic nature can lead to variability in the optimal fitness values returned.
Historically, several evaluation techniques, such as repeated evaluation~\cite{metaCMAES2012}, F-racing~\cite{FRacing}, and intensification~\cite{intens}, have been put forth to tackle this inconsistency.
Yet, these often come at the cost of an exorbitant number of functional evaluations (FEs).
To address this issue, we introduce the one-shot evaluation method.

Specifically, the method mandates each \texttt{executor} to undertake a singular, comprehensive independent run, subsequently returning its best-found solution.
A distinguishing aspect of this method is the consistent allocation of the same initial random seed to every \texttt{executor}.
As the algorithm progresses, this uniform seed ensures that the PDE fine-tunes its parameters in a consistent manner, thereby identifying optimal parameters tailored to the given seed environment.
Essentially, this strategy embeds the seed as an integral facet of the problem domain.

\subsection{Power-up Strategy}
During the independent runs of an \texttt{executor}, the allocation of FEs plays a pivotal role in determining both the quality of solutions and computational efficiency. Allocating an excessive number of FEs indiscriminately can lead to undue computational resource consumption without necessarily improving solution quality.
To address this issue, we propose the power-up strategy.

The essence of this strategy is dynamic FE allocation: while earlier iterations receive a moderate number of FEs to ensure resource efficiency, a more generous allocation (fivefold) is reserved for the terminal iteration within the evolutionary process.
This strategy ensures that the \texttt{executor} has the resources for a thorough and comprehensive evaluation during its most crucial phase -- the final generation of the \texttt{evolver}.


\subsection{Implementation}
As outlined in Algorithm~\ref{Alg_MetaDE}, MetaDE draws its simple algorithmic workflow from conventional DE. MetaDE adopts \texttt{DE/rand/1/bin} as the \texttt{evolver}.
The initialization phase (Line 1) spawns the MetaDE population within the parameter boundaries \( [\mathbf{lb}, \mathbf{ub}] \).
During the evaluation phase (Lines 6-11), each individual is decoded into a parameter blueprint and directed to an independent PDE instance (\texttt{executor}) for problem resolution.
Running for a predetermined iteration count \( G' \), each \texttt{executor} subsequently reports the best fitness.
Notably, Line 10 encapsulates the essence of the power-up strategy: for MetaDE's concluding iteration (\( g == G_{max} \)), the evaluation quota is amplified to \( 5 \times G' \) for the \texttt{executors}.


\begin{algorithm}
\small
\caption{MetaDE}
\label{Alg_MetaDE}
\begin{algorithmic}[1]
  \Require {$D$, $NP$, $G_{max}$, $\mathbf{lb}$ (lower boundaries of PDE's parameters), $\mathbf{ub}$ (upper boundaries of PDE's parameters), $NP'$ (population size of PDE), $G'$ (max generations of PDE)}
  \State Initialize population $\mathbf{X} = \{\mathbf{x}_1, \mathbf{x}_2, \dots, \mathbf{x}_{\scalebox{0.5}{$\textit{NP}$}}\}$ between $[\mathbf{lb}, \mathbf{ub}]$
  \State Initialize the fitness of $\mathbf{X}$: $\mathbf{y}=\mathbf{inf}$
  \State $g = 0$
  \While{$g \leq G_{max}$}
  \State Generate trial vectors $\mathbf{U}$ by mutation and crossover
  \Statex \quad \ /* The mutation and crossover scheme used is rand/1/bin\ */
  \State Decode each trial vector $\mathbf{u}$ into parameters:
  \Statex \quad \ $F=\mathbf{u}[1], CR=\mathbf{u}[2], bl=\textrm{floor}(\mathbf{u}[3]), br=\textrm{floor}(\mathbf{u}[4])$,
  \Statex \quad \ $dn=\textrm{floor}(\mathbf{u}[5]), cs=\textrm{floor}(\mathbf{u}[6])$
  %\Statex /*Evaluate each $\mathbf{u}$ by running PDE for certain generations*/
  \If {$g < G_{max}$}
    \State $\mathbf{y} = \textrm{PDE}(D, NP', G', F, CR, bl, br, dn, cs)$
  \Else
    \State $\mathbf{y} = \textrm{PDE}(D, NP', 5 * G', F, CR, bl, br, dn, cs)$
  \EndIf
  \State Make selection between $\mathbf{U}$ and $\mathbf{X}$
  \State $g = g + 1$
  \EndWhile
  \State\Return the best individual and fitness
\end{algorithmic}
\end{algorithm}


Evidently, the algorithmic design of MetaDE provides an automated end-to-end approach to black-box optimization.
However, the computational demands of MetaDE, particularly in terms of FEs, cannot be understated.
In historical computational contexts, such intensive demands might have posed significant impediments.
Fortunately, contemporary advancements in computational infrastructures, coupled with the ubiquity of high-performance computational apparatuses such as GPUs, have substantially alleviated such a challenge.


Hence, we leverage the GPU-accelerated framework of EvoX~\cite{evox} for the implementation of MetaDE.
Thanks to the inherently parallel nature of MetaDE, computational tasks can be judiciously delegated to GPUs to engender optimized runtime performance.
Specifically, the parallelism in MetaDE manifests in three distinct facets:
\begin{itemize}
  \item \textbf{Parallel Initialization and Execution:}
 The multiple \texttt{executors} are instantiated and operated concurrently, each tailored by a unique parameter configuration derived from the MetaDE ensemble.
  This simultaneous operation enables comprehensive exploration across varied parameter landscapes.

  \item \textbf{Parallel Offspring Generation:}
  Both the \texttt{evolver} and \texttt{executors} adhere to parallel strategies for offspring inception.
  By synchronizing and coordinating mutations and crossover operations in their respective populations, MetaDE is able to rapidly produce offspring, thereby accelerating the evolutionary process.

  \item \textbf{Parallel Fitness Evaluations:}
  Each \texttt{executor} conducts fitness evaluations concurrently across its member individuals.
  Given the substantial number of the individuals within the populations of the \texttt{executors}, this parallel strategy significantly enhances the overall efficiency of MetaDE.
\end{itemize}


\lstset{
  language=Python,
  aboveskip=3mm,
  belowskip=3mm,
  showstringspaces=false,
  columns=flexible,
  basicstyle={\footnotesize\ttfamily},
  numbers=left,
  numberstyle=\tiny\color{gray},
  xleftmargin=2em,
  keywordstyle=\bfseries\color{dkgreen},
  commentstyle=\color{gray}\itshape,
  stringstyle=\color{mauve},
  breaklines=true,
  breakatwhitespace=true,
  tabsize=3,
  emph={[1]evox,BatchExecutor,MetaProblem},          % Emphasize numpy
  emphstyle={[1]\bfseries\color{dkblue}},  % Set the style for emphasized words
  emph={[2]__init__, min, evaluate, reproduce, init},
  emphstyle={[2]\color{dkblue}},
  emph={[3]self,super},          % Emphasize numpy
  emphstyle={[3]\color{dkgreen}},
  morekeywords={from,import},  % Add more keywords
  captionpos=b,             % Caption position
      frame=lines,
  framesep=2mm,
}
\
\begin{lstlisting}[caption={Demonstrative implementation of MetaDE leveraging the computational workflow of EvoX. The implementation is distinctly divided into four pivotal components: Workflow Initialization, Meta Problem Transformation, Computing Workflow Creation, and Execution.}, label={lst:python_example}, float=!t]
from evox import algorithms, problems, ...

### Initialization ###
evolver = algorithm.DE()  # specify evolver
executor = algorithm.PDE()  # specify executor
problem = ...  # specify optimization problem

### Meta Problem Transformation ###
class MetaProblem(Problem):
    def __init__(self, batch_executor, ... ):
        # vectorize fitness evaluations
        self.batch_evaluate = vectorize(vectorize(problem.evaluate))

    def evaluate(self, state, ...):
        ...
        # run executors
        while ...:
            ...
            batch_fits, ... = self.batch_evaluate(...)
        # return fitness
        return min(min(batch_fits))

### Computing Workflow Creation ###
batch_executor = create_batch_executor(...)
meta_problem = MetaProblem(batch_executor, ...)
workflow = workflow.UniWorkflow(
            algorithm = evolver,
            pop_transform = decoder,
            problem = meta_problem,
        )

### Execution ###
while ...:
    state = workflow.step(state)
\end{lstlisting}

MetaDE adheres rigorously to the functional programming paradigm, capitalizing on automatic vectorization for parallel execution. Core algorithmic components, including crossover, mutation, and evaluation, are constructed using pure functions. Subsequently, the entire program is mapped to a GPU-based computation graph, ushering in accelerated processing. EvoX's adept state management ensures a seamless transfer of the algorithm's prevailing state, encompassing aspects like population, fitness, hyperparameters, and auxiliary data.

Listing \ref{lst:python_example} elucidates a representative implementation of MetaDE underpinned by the EvoX framework, which is meticulously segmented into four salient phases:

\begin{itemize}
    \item \textbf{Initialization}: Herein, primary entities like the \texttt{evolver} (employing the traditional DE) and the \texttt{executor} (utilizing the proposed PDE) are instantiated. Concurrently, the target optimization problem is defined.

    \item \textbf{Meta Problem Transformation}: Within this phase, the original optimization problem is transformed to align with the meta framework. This metamorphosis is realized via the \texttt{\textbf{MetaProblem}} class, where the evaluation function undergoes vectorization, priming it for efficient batch assessments and facilitating concurrent evaluations of manifold configurations.

    \item \textbf{Computing Workflow Creation}: Post transformation, the workflow is architected to seamlessly amalgamate the initialized components. The \texttt{batch\_executor} is crafted for batched operations of DE variants, and the \texttt{\textbf{MetaProblem}} is instantiated therewith. The holistic workflow, embodied by the \texttt{UniWorkflow} class, is then constructed, weaving together the \texttt{evolver}, the transformed problem, and a (\texttt{decoder}) which transforms the \texttt{evolver}'s population into specific hyperparameters for instantiating  the DE variant of each \texttt{executor}.

    \item \textbf{Execution}: Having established the groundwork, MetaDE's execution phase is triggered, autonomously driving the computing workflow across distributed GPUs.
    This workflow is traversed iteratively, culminating once a predefined termination criterion is met.
\end{itemize}

\section{Experimental Study}\label{section_Experimental_study}
In this section, we conduct detailed experimental assessments of MetaDE's capabilities.
First, we comprehensively benchmark MetaDE against several representative DE variants and CEC2022 top algorithms to gauge its relative performance on the CEC2022 benchmark suite \cite{CEC2022SO}.
Then, we investigate the optimal DE variants obtained by MetaDE in the benchmark experiment.
Finally, we apply MetaDE to robot control tasks.
All experiments were conducted on a system equipped with an Intel Core i9-10900X CPU and  an NVIDIA RTX 3090 GPU.
For GPU acceleration, all the algorithms and test functions were implemented within EvoX \cite{evox}.

\subsection{Benchmarks against Representative DE Variants}\label{section_Comparison with Classic DE Variants}

\subsubsection{Experimental Setup}
The CEC2022 benchmark suite for single-objective black-box optimization was utilized for this study.
This suite includes basic ($F_1-F_5$), hybrid ($F_6-F_8$), and composition functions ($F_9-F_{12}$), catering to various optimization characteristics such as unimodality/multimodality and separability/non-separability.

For benchmark comparisons, we selected seven representative DE variants: DE (\texttt{rand/1/bin}) \cite{DE1997}, SaDE \cite{SaDE2008}, JaDE \cite{JADE2009}, CoDE \cite{CoDE2011}, SHADE \cite{SHADE2013}, LSHADE-RSP \cite{LSHADE-RSP2018}, and EDEV \cite{EDEV2018}, which encapsulate a spectrum of mutation, crossover, and adaptation strategies. All algorithms were reimplemented using EvoX, with each capable of running in parallel, including the concurrent evaluation and reproduction.

Their respective descriptions are as follows:
\begin{itemize}
  \item \texttt{DE/rand/1/bin} is a foundational DE variant, which leverages a random mutation strategy coupled with binomial crossover.
  \item SaDE maintains an archive for tracking successful strategies and \( CR \) values and exhibits adaptability in strategy selection and parameter adjustments throughout the optimization process.
  \item JaDE relies on the \texttt{current-to-pbest} mutation strategy and dynamically adjusts its \( F \) and \( CR \) parameters during the optimization trajectory.
  \item CoDE infuses generational diversity by composing three disparate strategies, each complemented with randomized parameters, for offspring generation.
  \item SHADE employs the current-to-pbest mutation strategy and integrates a success-history mechanism to fine-tune its \( F \) and \( CR \)  parameters adaptively.
  \item LSHADE-RSP, as one of the most competitive DE variants, employs delicate strategies such as linear population size reduction and ranking-based mutation.
  \item EDEV adopts a distributed framework that ensembles three classic DE variants: JaDE, CoDE, and EPSDE.
\end{itemize}

The population size for all comparative algorithms was uniformly set to 100, except for experiments involving large populations. The other parameters for these algorithms were adopted as per their default settings described in their respective publications.

In our MetaDE configuration, on one hand, the \texttt{evolver} had a population size of 100 and adopted the vanilla \texttt{rand/1/bin} strategy with $F=0.5$ and $CR=0.9$;
On the other hand, each \texttt{executor} maintained a population of 100, iterating 1000 times for all the problems.
For simplicity, any result exceeding the precision of $10^{-8}$ was truncated to 0.
All statistical results were obtained via 31 independent runs\footnote{Full results, including the statistical results applying Wilcoxon rank-sum tests with a a significance level of 0.05, can be found in the Supplementary Document.}.




\begin{figure}[!t]
\centering
\includegraphics[scale=0.3]{D10.pdf}
\caption{Convergence curves on 10D problems in CEC2022 benchmark suite. The peer DE variants are set with population size of 100.}
\label{Figure_convergence_10D}
\end{figure}

\begin{figure}[!t]
\centering
\includegraphics[scale=0.3]{D20.pdf}
\caption{Convergence curves on 20D problems in CEC2022 benchmark suite. The peer DE variants are set with population size of 100.}
\label{Figure_convergence_20D}
\end{figure}


\subsubsection{Performance under Equal Wall-clock Time}\label{sec:expereiment_time}
In this part, we set equal  wall-clock time (\SI{60}{\second}) as the termination condition for running each test. 
{
This approach aligns with the practical constraints of modern GPU computing, where execution time serves as a more meaningful and comparable measure of performance across algorithms. Since all algorithms in our experiments are implemented with GPU parallelism, this setup ensures fairness by standardizing the computational resources and focusing on efficiency within the same time budget.
}

As shown in Figs. \ref{Figure_convergence_10D} and \ref{Figure_convergence_20D}, we selected five challenging problems, specifically $F_2$, $F_4$, $F_6$, $F_9$, and $F_{10}$, to demonstrate the convergence profiles.
Notably, MetaDE's convergence curve is observably more favorable, consistently registering lower errors than its counterparts across the majority of the problems.
Particularly, on $F_2$, $F_4$, $F_9$, and $F_{10}$, MetaDE exhibits resilience against local optima entrapment and subsequent convergence stagnation.
This is attributed to MetaDE's capability to identify optimal algorithm settings tailored for diverse problems, rather than merely tweaking parameters based on isolated segments of the optimization trajectory, as is the case with some DE variants.
An intriguing characteristic of MetaDE's convergence, evident in functions like $F_9$ (refer to Fig. \ref{Figure_convergence_10D}), is its pronounced performance surge in the optimization's terminal phase.
This enhancement can be linked to MetaDE's power-up strategy of allocating bonus computational resources in its final phase (as per Line 10 of Algorithm \ref{Alg_MetaDE}).


\begin{figure}[!h]
\centering
\includegraphics[scale=0.3]{FEs.pdf}
\caption{The number of FEs achieved by each algorithm within \SI{60}{\second}. The results are averaged on all 10D and 20D problems in the CEC2022 benchmark suite.}
\label{fig:maxFEs}
\end{figure}

{
Furthermore, to assess the concurrency of the algorithms, the number of FEs achieved by each algorithm within 60 seconds is shown in Table~\ref{tab:FEs} and Fig~\ref{fig:maxFEs}.
}
The results indicate that MetaDE achieves approximately $10^9$ FEs within 60 seconds, while the other algorithms manage to attain only around $10^7$ FEs in the same time frame.
The results demonstrate the high concurrency of MetaDE, which is particularly favorable in GPU computing.

\begin{table}[htbp]
  \centering
  
  \caption{{The number of FEs achieved by each algorithm within \SI{60}{\second}.}}
 
\scriptsize                   %设置字体大小
\renewcommand{\arraystretch}{1}
\renewcommand{\tabcolsep}{2.5pt}   %pt越大字越小
\resizebox{\linewidth}{!}{
% Table generated by Excel2LaTeX from sheet 'Experiment1 60S'
\begin{tabular}{cccccccccc}
\toprule
Dim   & Func  & MetaDE & DE    & SaDE  & JaDE  & CoDE  & SHADE &LSHADE-RSP&EDEV\\
\midrule
\multirow{12}[2]{*}{10D} & $F_{1}$ & \textbf{1.85E+09} & 4.28E+06 & 1.96E+06 & 2.55E+06 & 1.09E+07 & 2.24E+06&2.57E+06&2.79E+06 \\
      & $F_{2}$ & \textbf{1.84E+09} & 4.19E+06 & 1.89E+06 & 2.42E+06 & 1.11E+07 & 2.18E+06&2.58E+06&2.82E+06 \\
      & $F_{3}$ & \textbf{1.50E+09} & 4.00E+06 & 1.89E+06 & 2.50E+06 & 1.11E+07 & 2.10E+06& 2.46E+06&2.68E+06\\
      & $F_{4}$ & \textbf{1.84E+09} & 4.11E+06 & 2.02E+06 & 2.61E+06 & 1.11E+07 & 2.15E+06&2.60E+06&2.88E+06 \\
      & $F_{5}$ & \textbf{1.83E+09} & 4.13E+06 & 2.00E+06 & 2.60E+06 & 1.15E+07 & 2.17E+06& 2.53E+06&2.96E+06\\
      & $F_{6}$ & \textbf{1.84E+09} & 4.31E+06 & 1.96E+06 & 2.65E+06 & 1.07E+07 & 2.14E+06&2.55E+06&2.95E+06\\
      & $F_{7}$ & \textbf{1.74E+09} & 3.35E+06 & 1.90E+06 & 2.55E+06 & 9.96E+06 & 2.14E+06& 2.41E+06&2.87E+06\\
      & $F_{8}$ & \textbf{1.72E+09} & 3.34E+06 & 1.83E+06 & 2.53E+06 & 9.60E+06 & 2.17E+06&2.34E+06&2.74E+06 \\
      & $F_{9}$ & \textbf{1.78E+09} & 3.35E+06 & 1.84E+06 & 2.52E+06 & 9.69E+06 & 2.18E+06&2.44E+06 &2.82E+06\\
      & $F_{10}$ & \textbf{1.44E+09} & 3.32E+06 & 1.83E+06 & 2.46E+06 & 9.00E+06 & 2.12E+06& 2.30E+06&2.70E+06\\
      & $F_{11}$ & \textbf{1.46E+09} & 3.55E+06 & 1.88E+06 & 2.34E+06 & 9.66E+06 & 2.13E+06&2.34E+06& 2.63E+06\\
      & $F_{12}$ & \textbf{1.43E+09} & 3.46E+06 & 1.83E+06 & 2.41E+06 & 9.51E+06 & 2.09E+06&2.32E+06 &2.67E+06\\
\midrule
\midrule
\multirow{12}[2]{*}{20D} & $F_{1}$ & \textbf{1.66E+09} & 4.32E+06 & 1.92E+06 & 2.46E+06 & 1.13E+07 & 2.21E+06&2.68E+06&2.80E+06 \\
      & $F_{2}$ & \textbf{1.66E+09} & 3.91E+06 & 1.88E+06 & 2.37E+06 & 1.17E+07 & 2.09E+06& 2.66E+06&2.74E+06\\
      & $F_{3}$ & \textbf{1.18E+09} & 3.76E+06 & 1.77E+06 & 2.33E+06 & 9.75E+06 & 1.95E+06&2.62E+06& 2.64E+06\\
      & $F_{4}$ & \textbf{1.65E+09} & 3.50E+06 & 1.87E+06 & 2.28E+06 & 1.07E+07 & 2.00E+06&2.63E+06 &2.80E+06\\
      & $F_{5}$ & \textbf{1.64E+09} & 3.57E+06 & 1.86E+06 & 2.32E+06 & 1.07E+07 & 2.05E+06& 2.55E+06&2.74E+06\\
      & $F_{6}$ & \textbf{1.64E+09} & 3.89E+06 & 1.90E+06 & 2.34E+06 & 1.16E+07 & 2.09E+06& 2.62E+06&2.80E+06\\
      & $F_{7}$ & \textbf{1.45E+09} & 4.15E+06 & 1.84E+06 & 2.36E+06 & 1.01E+07 & 1.99E+06&2.32E+06& 2.80E+06\\
      & $F_{8}$ & \textbf{1.44E+09} & 3.42E+06 & 1.82E+06 & 2.31E+06 & 9.51E+06 & 2.12E+06&2.18E+06&2.30E+06 \\
      & $F_{9}$ & \textbf{1.57E+09} & 3.30E+06 & 1.77E+06 & 2.33E+06 & 9.48E+06 & 2.03E+06&2.59E+06&2.75E+06 \\
      & $F_{10}$ & \textbf{9.80E+08} & 3.67E+06 & 1.82E+06 & 2.43E+06 & 1.01E+07 & 2.08E+06& 2.09E+06&2.35E+06\\
      & $F_{11}$ & \textbf{1.00E+09} & 3.51E+06 & 1.95E+06 & 2.41E+06 & 9.81E+06 & 2.07E+06&2.17E+06& 2.37E+06\\
      & $F_{12}$ & \textbf{9.90E+08} & 3.44E+06 & 1.85E+06 & 2.37E+06 & 9.51E+06 & 2.07E+06& 2.17E+06&2.39E+06\\
\bottomrule
\end{tabular}%
}
  \label{tab:FEs}%
\end{table}%

\subsubsection{Performance under Equal FEs}\label{sec:expereiment_FEs}

% Table generated by Excel2LaTeX from sheet 'Sheet1'
\begin{table*}[htbp]
  %\centering
\caption{Comparisons between MetaDE and other DE variants under equal FEs. 
The mean and standard deviation (in parentheses) of the results over multiple runs are displayed in pairs. 
Results with the best mean values are highlighted.}
  \resizebox{\linewidth}{!}{
  \renewcommand{\arraystretch}{1.2}
 \renewcommand{\tabcolsep}{2pt}
% Table generated by Excel2LaTeX from sheet 'Exp3 same FEs'
\begin{tabular}{cccccccccc}
\toprule
\multicolumn{2}{c}{Func} & MetaDE & DE    & SaDE  & JaDE  & CoDE  & SHADE & LSHADE-RSP & EVDE \\
\midrule
\multirow{3}[2]{*}{10D} & $F_{2}$ & \textbf{0.00E+00 (0.00E+00)} & 4.52E+00 (2.36E+00)$-$ & 6.85E+00 (3.52E+00)$-$ & 6.31E+00 (3.05E+00)$-$ & 5.78E+00 (2.37E+00)$-$ & 4.33E+00 (3.81E+00)$-$ & 2.35E+00 (3.44E+00)$-$ & 5.86E+00 (2.99E+00)$-$ \\
      & $F_{6}$ & \textbf{5.50E-04 (3.96E-04)} & 1.13E-01 (7.83E-02)$-$ & 3.54E+01 (1.11E+02)$-$ & 2.02E+00 (3.34E+00)$-$ & 6.96E-03 (5.99E-03)$-$ & 9.27E-01 (1.31E+00)$-$ & 3.10E-02 (4.83E-02)$-$ & 1.46E+00 (2.76E+00)$-$ \\
      & $F_{10}$ & \textbf{0.00E+00 (0.00E+00)} & 1.00E+02 (4.40E-02)$-$ & 1.00E+02 (6.10E-02)$-$ & 1.21E+02 (4.35E+01)$-$ & 1.00E+02 (6.88E-02)$-$ & 1.29E+02 (4.67E+01)$-$ & 1.09E+02 (2.87E+01)$-$ & 1.10E+02 (3.05E+01)$-$ \\
\midrule
\midrule
\multirow{3}[2]{*}{20D} & $F_{2}$ & \textbf{1.26E-02 (3.74E-02)} & 4.72E+01 (2.09E+00)$-$ & 4.76E+01 (2.02E+00)$-$ & 1.34E+01 (2.19E+01)$-$ & 4.91E+01 (1.70E-06)$-$ & 4.91E+01 (3.40E-06)$-$ & 4.84E+01 (1.62E+00)$-$ & 4.47E+01 (1.41E+01)$-$ \\
      & $F_{6}$ & \textbf{1.16E-01 (2.79E-02)} & 7.28E-01 (5.22E-01)$-$ & 3.20E+01 (1.61E+01)$-$ & 4.90E+01 (3.31E+01)$-$ & 2.26E+01 (1.80E+01)$-$ & 5.67E+01 (3.90E+01)$-$ & 1.15E+01 (8.38E+00)$-$ & 2.92E+03 (5.82E+03)$-$ \\
      & $F_{10}$ & \textbf{0.00E+00 (0.00E+00)} & 1.07E+02 (2.07E+01)$-$ & 1.00E+02 (2.72E-02)$-$ & 1.01E+02 (3.64E-02)$-$ & 1.00E+02 (3.71E-02)$-$ & 1.43E+02 (5.64E+01)$-$ & 1.21E+02 (4.52E+01)$-$ & 1.14E+02 (4.65E+01)$-$ \\
\midrule
\multicolumn{2}{c}{$+$ / $\approx$ / $-$} & --    & 0/0/6 & 0/0/6 & 0/0/6 & 0/0/6 & 0/0/6 & 0/0/6 & 0/0/6 \\
\bottomrule
\end{tabular}%

}
\label{tab:sameFEs}%

\footnotesize
\textsuperscript{*} The Wilcoxon rank-sum tests (with a significance level of 0.05) were conducted between MetaDE and each algorithm individually.
The final row displays the number of problems where the corresponding algorithm performs statistically better ($+$),  similar ($\thickapprox$), or worse ($-$) compared to MetaDE.
\end{table*}%



In the preceding part, the performance benchmarking of MetaDE with other algorithms was anchored to equal wall-clock durations.
However, to ensure a comprehensive assessment, it is imperative to evaluate their performances under equivalent FEs.
In this part, we run each algorithm using the FEs achieved by MetaDE in \SI{60}{\second} (i.e., $1.84\times10^9/1.66\times10^9$, $1.84\times10^9/1.64\times10^9$, and $1.44\times10^9/9.8\times10^8$) on $F_2$, $F_6$, and $F_{10}$ for 10D/20D cases.
These selected functions collectively epitomize the basic, hybrid, and composition challenges within the CEC2022 benchmark suite.

As summarized in Table \ref{tab:sameFEs}, MetaDE consistently demonstrates the best performance, even when other algorithms are endowed with comparable FEs.
The reason can be traced to the inherent stagnation tendencies of other algorithms: after a certain point, additional FEs may not contribute to performance improvements.
This behavioral pattern is also lucidly captured in the convergence curves as presented in Figs. \ref{Figure_convergence_10D} and \ref{Figure_convergence_20D}.

Another noteworthy observation is the extended computation time required for a singular run of the comparison algorithms under these enhanced FEs, often extending to several hours or even transcending a day (e.g., running a single run of DE can take up to seven hours).
This elongated computational span can largely be attributed to their low concurrency, which struggles to benefit the parallelism of GPU computing.


\subsubsection{Performance with Large Populations}\label{sec:expereiment_large_pop}
Since a large population size could potentially increase the concurrency of fitness evaluations, for rigorousness, we further investigate the performance of the algorithms with large populations.

Specifically, MetaDE adopted the same population size setting as in previous experiments (i.e., 100 for both \texttt{evolver} and \texttt{executor}), while the population size of the other DE variants was increased to 1,000. 
This adjustment significantly enhances the concurrency of the other DE variants when utilizing GPU accelerations, thereby preventing insufficient convergence.


As evidenced in Figs. \ref{Figure_convergence_10D_NP10k}-\ref{Figure_convergence_20D_NP10k}, MetaDE still outperforms the other DE variants across all problems.
However, the performances of the other DE variants did not show significant improvements, which can be attributed to two factors.
First, since the conventional DE variants were not tailored for large populations, simply enlarging the populations may not help.
Second, since the sorting and archiving operations in some DE variants (e.g., SaDE) suffer from high computational complexities related to the population size, enlarging the populations brings additional computation overheads, thus limiting their performances under fixed wall-clock time.

By contrast, the large population in MetaDE is delicately organized in a \emph{hierarchical} manner: the \texttt{executor} maintains a population of moderate size, with each individual initializing an \texttt{executor} with a normal population.
This strategy not only capitalizes on the small-population advantage of conventional DE, but also benefits the concurrency brought by large populations.

\begin{figure}[!t]
\centering
\includegraphics[scale=0.3]{D10NP1000.pdf}
\caption{Convergence curves on 10D problems in CEC2022 benchmark suite. The peer DE variants are set with population size of 1,000.}
\label{Figure_convergence_10D_NP10k}
\end{figure}

\begin{figure}[!t]
\centering
\includegraphics[scale=0.3]{D20NP1000.pdf}
\caption{Convergence curves on 20D problems in CEC2022 benchmark suite. The peer DE variants are set with population size of 1,000.}
\label{Figure_convergence_20D_NP10k}
\end{figure}


{
\subsection{Comparisons with Top Algorithms in CEC2022 Competition}\label{section_Comparison with Top Algorithms of CEC Competition}

% Table generated by Excel2LaTeX from sheet 'Sheet1'
\begin{table*}[htbp]
  \centering
  
  \caption{{Comparisons between MetaDE and the top 4 algorithms from CEC2022 Competition (10D). 
The mean and standard deviation (in parentheses) of the results over multiple runs are displayed in pairs. 
Results with the best mean values are highlighted. }
  }
\footnotesize
% Table generated by Excel2LaTeX from sheet 'Exp 7 vs CECtop'
\begin{tabular}{cccccc}
\toprule
Func  & MetaDE & EA4eig & NL-SHADE-LBC & NL-SHADE-RSP & S-LSHADE-DP \\
\midrule
$F_{1}$ & \textbf{0.00E+00 (0.00E+00)} & \boldmath{}\textbf{0.00E+00 (0.00E+00)$\approx$}\unboldmath{} & \boldmath{}\textbf{0.00E+00 (0.00E+00)$\approx$}\unboldmath{} & \boldmath{}\textbf{0.00E+00 (0.00E+00)$\approx$}\unboldmath{} & \boldmath{}\textbf{0.00E+00 (0.00E+00)$\approx$}\unboldmath{} \\
$F_{2}$ & \textbf{0.00E+00 (0.00E+00)} & 7.97E-01 (1.78E+00)$-$ & 7.97E-01 (1.78E+00)$-$ & \boldmath{}\textbf{0.00E+00 (0.00E+00)$\approx$}\unboldmath{} & \boldmath{}\textbf{0.00E+00 (0.00E+00)$\approx$}\unboldmath{} \\
$F_{3}$ & \textbf{0.00E+00 (0.00E+00)} & \boldmath{}\textbf{0.00E+00 (0.00E+00)$\approx$}\unboldmath{} & \boldmath{}\textbf{0.00E+00 (0.00E+00)$\approx$}\unboldmath{} & \boldmath{}\textbf{0.00E+00 (0.00E+00)$\approx$}\unboldmath{} & \boldmath{}\textbf{0.00E+00 (0.00E+00)$\approx$}\unboldmath{} \\
$F_{4}$ & \textbf{0.00E+00 (0.00E+00)} & 9.95E-01 (1.22E+00)$-$ & 1.99E-01 (4.45E-01)$-$ & 2.98E+00 (1.15E+00)$-$ & \boldmath{}\textbf{0.00E+00 (0.00E+00)$\approx$}\unboldmath{} \\
$F_{5}$ & \textbf{0.00E+00 (0.00E+00)} & \boldmath{}\textbf{0.00E+00 (0.00E+00)$\approx$}\unboldmath{} & \boldmath{}\textbf{0.00E+00 (0.00E+00)$\approx$}\unboldmath{} & \boldmath{}\textbf{0.00E+00 (0.00E+00)$\approx$}\unboldmath{} & \boldmath{}\textbf{0.00E+00 (0.00E+00)$\approx$}\unboldmath{} \\
$F_{6}$ & 5.50E-04 (3.96E-04) & 7.53E-04 (5.52E-04)$\approx$ & 8.93E-02 (1.18E-01)$-$ & 4.37E-02 (5.41E-02)$-$ & \textbf{5.84E-05 (4.73E-05)$+$} \\
$F_{7}$ & \textbf{0.00E+00 (0.00E+00)} & \boldmath{}\textbf{0.00E+00 (0.00E+00)$\approx$}\unboldmath{} & \boldmath{}\textbf{0.00E+00 (0.00E+00)$\approx$}\unboldmath{} & \boldmath{}\textbf{0.00E+00 (0.00E+00)$\approx$}\unboldmath{} & \boldmath{}\textbf{0.00E+00 (0.00E+00)$\approx$}\unboldmath{} \\
$F_{8}$ & 5.52E-03 (4.41E-03) & 1.01E-04 (1.66E-04)$+$ & 3.96E-04 (4.23E-04)$+$ & 3.13E-01 (3.60E-01)$-$ & \textbf{1.26E-05 (1.56E-05)$+$} \\
$F_{9}$ & \textbf{3.36E+00 (1.77E+01)} & 1.86E+02 (0.00E+00)$-$ & 2.29E+02 (3.18E-14)$-$ & 8.03E+01 (1.08E+02)$-$ & 2.23E+02 (1.31E+01)$-$ \\
$F_{10}$ & \textbf{0.00E+00 (0.00E+00)} & 1.00E+02 (0.00E+00)$-$ & 1.00E+02 (0.00E+00)$-$ & 1.56E-02 (3.12E-02)$-$ & \boldmath{}\textbf{0.00E+00 (0.00E+00)$\approx$}\unboldmath{} \\
$F_{11}$ & \textbf{0.00E+00 (0.00E+00)} & \boldmath{}\textbf{0.00E+00 (0.00E+00)$\approx$}\unboldmath{} & \boldmath{}\textbf{0.00E+00 (0.00E+00)$\approx$}\unboldmath{} & \boldmath{}\textbf{0.00E+00 (0.00E+00)$\approx$}\unboldmath{} & \boldmath{}\textbf{0.00E+00 (0.00E+00)$\approx$}\unboldmath{} \\
$F_{12}$ & \textbf{1.39E+02 (4.63E+01)} & 1.48E+02 (5.98E+00)$-$ & 1.65E+02 (0.00E+00)$-$ & 1.62E+02 (2.15E+00)$-$ & 1.59E+02 (0.00E+00)$-$ \\
\midrule
$+$ / $\approx$ / $-$ & --    & 1/6/5 & 1/5/6 & 0/6/6 & 2/8/2 \\
\bottomrule
\end{tabular}%

\footnotesize
\textsuperscript{*} The Wilcoxon rank-sum tests (with a significance level of 0.05) were conducted between MetaDE and each algorithm individually.
The final row displays the number of problems where the corresponding algorithm performs statistically better ($+$),  similar ($\thickapprox$), or worse ($-$) compared to MetaDE.\\


\label{tab:vsCECTop 10D}%
\end{table*}%

% Table generated by Excel2LaTeX from sheet 'Sheet1'
\begin{table*}[htbp]
  \centering
  
  \caption{{Comparisons between MetaDE and the top 4 algorithms from CEC2022 Competition (20D). 
The mean and standard deviation (in parentheses) of the results over multiple runs are displayed in pairs. 
Results with the best mean values are highlighted.
  }
  }
  %\resizebox{\linewidth}{!}{
  %       \renewcommand{\arraystretch}{1}
 %\renewcommand{\tabcolsep}{3pt}
% Table generated by Excel2LaTeX from sheet 'Sheet1'
\footnotesize
% Table generated by Excel2LaTeX from sheet 'Exp 7 vs CECtop'
\begin{tabular}{cccccc}
\toprule
Func  & MetaDE & EA4eig & NL-SHADE-LBC & NL-SHADE-RSP & S-LSHADE-DP \\
\midrule
$F_{1}$ & \textbf{0.00E+00 (0.00E+00)} & \boldmath{}\textbf{0.00E+00 (0.00E+00)$\approx$}\unboldmath{} & \boldmath{}\textbf{0.00E+00 (0.00E+00)$\approx$}\unboldmath{} & \boldmath{}\textbf{0.00E+00 (0.00E+00)$\approx$}\unboldmath{} & \boldmath{}\textbf{0.00E+00 (0.00E+00)$\approx$}\unboldmath{} \\
$F_{2}$ & 3.83E-04 (2.10E-03) & \textbf{0.00E+00 (0.00E+00)$+$} & 4.91E+01 (0.00E+00)$-$ & \textbf{0.00E+00 (0.00E+00)$+$} & \textbf{0.00E+00 (0.00E+00)$+$} \\
$F_{3}$ & \textbf{0.00E+00 (0.00E+00)} & \boldmath{}\textbf{0.00E+00 (0.00E+00)$\approx$}\unboldmath{} & \boldmath{}\textbf{0.00E+00 (0.00E+00)$\approx$}\unboldmath{} & \boldmath{}\textbf{0.00E+00 (0.00E+00)$\approx$}\unboldmath{} & \boldmath{}\textbf{0.00E+00 (0.00E+00)$\approx$}\unboldmath{} \\
$F_{4}$ & 1.96E+00 (7.76E-01) & 7.36E+00 (2.06E+00)$-$ & \boldmath{}\textbf{1.59E+00 (5.45E-01)$\approx$}\unboldmath{} & 1.07E+02 (1.54E+02)$-$ & 3.20E+00 (1.94E+00)$-$ \\
$F_{5}$ & \textbf{0.00E+00 (0.00E+00)} & \boldmath{}\textbf{0.00E+00 (0.00E+00)$\approx$}\unboldmath{} & \boldmath{}\textbf{0.00E+00 (0.00E+00)$\approx$}\unboldmath{} & 2.27E-01 (4.54E-01)$-$ & \boldmath{}\textbf{0.00E+00 (0.00E+00)$\approx$}\unboldmath{} \\
$F_{6}$ & \textbf{1.38E-01 (5.56E-02)} & 2.54E-01 (4.28E-01)$-$ & 3.06E-01 (2.01E-01)$-$ & 2.08E-01 (9.78E-02)$-$ & 5.02E-01 (5.34E-01)$-$ \\
$F_{7}$ & 8.42E-02 (1.01E-01) & 1.37E+00 (1.10E+00)$-$ & \boldmath{}\textbf{6.24E-02 (1.40E-01)$\approx$}\unboldmath{} & 1.28E+00 (1.95E+00)$-$ & 9.83E-01 (8.12E-01)$-$ \\
$F_{8}$ & 2.66E+00 (3.83E+00) & 2.02E+01 (1.28E-01)$-$ & \textbf{1.01E-01 (1.41E-01)$+$} & 1.99E+01 (4.97E-01)$-$ & 2.30E-01 (1.82E-01)$+$ \\
$F_{9}$ & \textbf{1.32E+02 (3.43E+01)} & 1.65E+02 (0.00E+00)$-$ & 1.81E+02 (0.00E+00)$-$ & 1.81E+02 (0.00E+00)$-$ & 1.81E+02 (0.00E+00)$-$ \\
$F_{10}$ & \textbf{0.00E+00 (0.00E+00)} & 1.23E+02 (5.12E+01)$-$ & 1.00E+02 (9.27E-03)$-$ & \boldmath{}\textbf{0.00E+00 (0.00E+00)$\approx$}\unboldmath{} & \boldmath{}\textbf{0.00E+00 (0.00E+00)$\approx$}\unboldmath{} \\
$F_{11}$ & 1.74E-03 (7.97E-03) & 3.20E+02 (4.47E+01)$-$ & 3.00E+02 (0.00E+00)$-$ & \textbf{0.00E+00 (0.00E+00)$+$} & \textbf{0.00E+00 (0.00E+00)$+$} \\
$F_{12}$ & 2.29E+02 (9.70E-01) & \textbf{2.00E+02 (2.04E-04)$+$} & 2.37E+02 (3.17E+00)$-$ & 2.34E+02 (1.46E+00)$-$ & 2.34E+02 (4.51E+00)$-$ \\
\midrule
$+$ / $\approx$ / $-$ & --    & 2/3/7 & 1/5/6 & 2/3/7 & 3/4/5 \\
\bottomrule
\end{tabular}%

\footnotesize
\textsuperscript{*} The Wilcoxon rank-sum tests (with a significance level of 0.05) were conducted between MetaDE and each algorithm individually.
The final row displays the number of problems where the corresponding algorithm performs statistically better ($+$),  similar ($\thickapprox$), or worse ($-$) compared to MetaDE.\\

\label{tab:vsCECTop 20D}%
\end{table*}%



To further assess the performance of MetaDE, we compare it with the top 4 algorithms from the CEC2022 Competition on Single Objective Bound Constrained Numerical Optimization\footnote{\url{https://github.com/P-N-Suganthan/2022-SO-BO}}.
For each algorithm, we set equal FEs as achieved by MetaDE within 60 seconds (refer to Table~\ref{tab:FEs} for details).

The top 4 algorithms from the CEC2022 Competition are {EA4eig}~\cite{EA4eig}, {NL-SHADE-LBC}~\cite{NL-SHADE-LBC}, {NL-SHADE-RSP-MID}~\cite{NL-SHADE-RSP}, and {S-LSHADE-DP}~\cite{S_LSHADE_DP}:
\begin{itemize}
  \item {EA4eig} combines the strengths of four evolutionary algorithms (CMA-ES, CoBiDE, an adaptive variant of jSO, and IDE) using Eigen crossover.
  \item {NL-SHADE-LBC} is a dynamic DE variant that integrates linear bias changes for parameter adaptation, repeated point generation to handle boundary constraints, non-linear population size reduction, and a selective pressure mechanism.
  \item {NL-SHADE-RSP-MID} is an advanced version of NL-SHADE-RSP, which estimates the optimum using the population midpoint, incorporates a restart mechanism, and improves boundary constraint handling.
  \item {S-LSHADE-DP} focuses on maintaining population diversity through dynamic perturbation, adjusting noise intensity to enhance exploration.
\end{itemize}





The experimental results are summarized in Tables \ref{tab:vsCECTop 10D} and \ref{tab:vsCECTop 20D}.
On 10D problems, MetaDE outperforms EA4eig, NL-SHADE-LBC, and NL-SHADE-RSP, while achieving comparable performance to S-LSHADE-DP. 
On 20D problems, MetaDE consistently outperforms the four algorithms.
An additional noteworthy observation is that S-LSHADE-DP exhibits promising performance under a large number of FEs.
}

\subsection{Investigation of Optimal DE Variants}\label{section_Optimal Parameter Analysis}


\begin{table}[h]
  \centering
  \caption{Optimal DE variants obtained by MetaDE on each problem of the CEC2022 benchmark suite. FDC and RIE are two fitness landscape characteristics that measure the difficulty and ruggedness of the problem.}
% Table generated by Excel2LaTeX from sheet 'Exp4 param'
\resizebox{\columnwidth}{!}{
\begin{tabular}{cccccccc}
\toprule
\multicolumn{2}{c}{Problem} & F     & CR    & \multicolumn{2}{c}{Strategy} & FDC & RIE \\
\midrule
\multirow{4}[2]{*}{10D} & $F_{6}$ & 0.70  & 0.99  & \multicolumn{2}{c}{\texttt{rand-to-pbest/1/arith}} & 0.61  & 0.81  \\
      & $F_{8}$ & 0.51  & 0.44  & \multicolumn{2}{c}{\texttt{pbest-to-best/1/bin}} & 0.27  & 0.62  \\
      & $F_{9}$ & 0.02  & 0.03  & \multicolumn{2}{c}{\texttt{current/2/bin}} & 0.08  & 0.82  \\
      & $F_{12}$ & 0.16  & 0.00  & \multicolumn{2}{c}{\texttt{current-to-best/4/bin}} & -0.15  & 0.78  \\
\midrule
\multirow{6}[2]{*}{20D} & $F_{4}$ & 0.13  & 0.71  & \multicolumn{2}{c}{\texttt{rand-to-best/3/bin}} & 0.90  & 0.79  \\
      & $F_{6}$ & 0.67  & 0.99  & \multicolumn{2}{c}{\texttt{pbest-to-rand/1/bin}} & 0.48  & 0.80  \\
      & $F_{7}$ & 0.27  & 0.93  & \multicolumn{2}{c}{\texttt{rand/2/bin}} & 0.26  & 0.78  \\
      & $F_{8}$ & 0.65  & 0.00  & \multicolumn{2}{c}{\texttt{pbest/1/exp}} & 0.12  & 0.40  \\
      & $F_{9}$ & 0.06  & 0.00  & \multicolumn{2}{c}{\texttt{current/2/bin}} & -0.17  & 0.84  \\
      & $F_{12}$ & 0.33  & 0.44  & \multicolumn{2}{c}{\texttt{rand-to-best/2/bin}} & -0.16  & 0.85  \\
\bottomrule
\end{tabular}%
}
\label{tab:optimal_param}
\end{table}%


This part provides an in-depth examination of the optimal DE variants obtained by MetaDE in Section~\ref{sec:expereiment_time}, as summarized in Table \ref{tab:optimal_param}.
The optimal parameters correspond to the best individual in the final population of MetaDE.
The table only displays the optimal parameters for the ten listed problems, as the remaining problems are relatively simpler, with numerous DE variants capable of locating the optimal solutions of the problems. Furthermore, the optimal parameters presented in the table represent the best results of MetaDE derived from the finest run out of 31 independent trials.



All the problems in the table are characterized by both multimodality and non-separability.
Additionally, to further depict the characteristics of the problems' fitness landscapes, we computed both the fitness distance correlation (FDC) \cite{FDC} and the ruggedness of information entropy (RIE) \cite{RIE}; the former measures the complexity (difficulty) of the problems, while the latter characterizes the ruggedness of the landscape.

Analyzing the obtained data, it is evident that no single set of parameters or strategies consistently excels across all problems.
Parameters such as \(F\) and \(CR\) exhibit variability across problems without adhering to a specific trend.
Similarly, the selection of base vectors ($bl$ and $br$) does not show a uniform preference either.
Regarding the fitness landscape characteristics of each problem, the selection of parameters exhibits distinct patterns.
The FDC indicates problem complexity; with simpler problems (higher FDC), such as 10-dimensional $F_6$, $F_8$ and 20-dimensional $F_4$, $F_6$, $F_7$, a larger \(CR\) value is favored. Conversely, smaller \(CR\) values are chosen for problems with lower FDC. A larger \(CR\) tends to facilitate convergence, whereas a \(CR\) close to 0 leads to offspring that change incrementally, dimension by dimension. However, the other characteristic, RIE, does not seem to have a clear association with parameter choices.
The optimal strategies for identical problems across different dimensions exhibit closeness, with $F_8$, $F_9$, and $F_{12}$ demonstrating notably parallel strategies between their 10D and 20D problems.
In terms of crossover strategies ($cs$), it seems to have a preference for binomial crossover. This aligns with the traditional DE configurations.

These observations align with the No Free Lunch (NFL) theorem \cite{NFL}, thus underscoring the importance of distinct optimization strategies tailored for diverse problems.
Conventionally, the optimization strategies have oscillated between seeking a generalist set of parameters for broad applicability and a specialist set tailored for specific problems. However, the dynamic nature of optimization problems, where even minute changes like a different random seed can pivot the problem's dynamics, highlights the challenges of a generalist approach.
In contrast, MetaDE provides a simple yet effective approach, showing promising generality and adaptability.


\subsection{Application to Robot Control}\label{sec:expereiment_brax}
In this experiment, we demonstrate the extended application of MetaDE to robot control.
Specifically, we adopted the evolutionary reinforcement learning paradigm~\cite{ERL} as illustrated in Fig.~\ref{Figure_EvoRL}.
The experiment was conducted on Brax \cite{brax} for robotics simulations with GPU acceleration.

\begin{figure}[!h]
\centering
\includegraphics[scale=0.38]{EvoRL_Workflow.pdf}
\caption{Illustration of robot control via evolutionary reinforcement learning. The evolutionary algorithm optimizes the parameters of a population of candidate policy models for controlling the robotics behaviors. The simulation environment returns rewards achieved by the candidate policy models to the evolutionary algorithm as fitness values.}
\label{Figure_EvoRL}
\end{figure}

This experiment involved three robot control tasks: ``swimmer'', ``hopper'', and ``reacher''.
As summarized in Table \ref{tab:Neural network structures}, we adopted similar policy models for these three tasks, each consisting of a multilayer perceptron (MLP) with three fully connected layers, but with different input and output dimensions.
Consequently, the three policy models comprise 1410, 1539, and 1506 parameters for optimization respectively, where the optimization objective is to achieve maximum reward of each task.
MetaDE, vanilla DE \cite{DE1996}, SHADE~\cite{SHADE2013}, LSHADE-RSP~\cite{LSHADE-RSP2018}, EDEV~\cite{EDEV2018}, CSO \cite{CSO}\footnote{The competitive swarm optimizer (CSO) is a tailored PSO variant for large-scale optimization.}, and CMA-ES~\cite{CMAES} were applied as the optimizer respectively.

%The policy models  were initialized with identical random parameters.
The iteration count for PDE within MetaDE was set to 50, while other algorithms maintained a population size of 100.
Each algorithm was run independently 15 times.
Considering the time-intensive nature of the robotics simulations, we set 60 minutes as the termination condition for each run.


\begin{table}[htbp]
\centering
\caption{Neural network structure of the policy model for each robot control task}
\label{tab:Neural network structures}
\resizebox{\columnwidth}{!}{%
% Table generated by Excel2LaTeX from sheet 'Exp6 brax'
\begin{tabular}{cccccc}
\toprule
\textbf{Task} & \textbf{D} & \textbf{Input} & \textbf{Hidden Layers} &   \textbf{Output}    & \textbf{Overview of objectives} \\
\midrule
Hopper & 1539  & 11    & 32$\times$32 & 3     & balance and jump \\
Swimmer & 1410  & 8     & 32$\times$32 & 2     & maximizing movement \\
Reacher & 1506  & 11    & 32$\times$32 & 2     & precise reaching \\
\bottomrule
\end{tabular}%
}
\end{table}

\begin{figure}[!h]
\centering
\includegraphics[scale=0.3]{brax_all_convergence.pdf}
\caption{The reward curves achieved by MetaDE and peer evolutionary algorithms when applied to each robot control task. }
\label{Figure_brax_all_convergence}
\end{figure}

\begin{figure}[!h]
\centering
\includegraphics[width=\linewidth]{brax_distribution.pdf}
\caption{The fitness distribution of MetaDE's initial population when applied to each robot control task.}
\label{Figure_brax_distribution}
\end{figure}

As shown in Fig. \ref{Figure_brax_all_convergence}, it is evident that MetaDE achieves the best performance in the Swimmer tasks, while slightly outperformed by CMA-ES and CSO in the Hopper and Reacher task.
An interesting observation from the reward curves is that MetaDE almost reaches optimality nearly at the first generation and does not show further significant improvements thereafter.
To elucidate this phenomenon, Fig.~\ref{Figure_brax_distribution} provides the fitness distribution of MetaDE's initial population, indicating that MetaDE harbored several individuals with considerably high fitness from the initial generation.
In other words, MetaDE was able to generate high-performance DE variants for these problems even by random sampling.
This can be attributed to the unique nature of neural network optimization.
As widely acknowledged, the neural network optimization typically features numerous plateaus in the fitness landscape, thus making it relatively easy to find one of the local optima.
MetaDE provides unbiased sampling of parameter settings for generating diverse DE variants.
Even without further evolution, some of the randomly sampled DE variants are very likely to reach the plateaus.
In contrast, the other algorithms are specially tailored with biases; in such large-scale optimization scenarios, the biases can be further amplified, thus making them ineffective.




\section{Conclusion}\label{section Conclusion}
In this paper, we introduced MetaDE, a method that leverages the strengths of DE not only to address optimization tasks but also to adapt and refine its own strategies. This meta-evolutionary approach demonstrates how DE can autonomously evolve its parameter configurations and strategies. 
Our experiments demonstrate that MetaDE has robust performance across various benchmarks, as well as the application in robot control through evolutionary reinforcement learning. 
Nevertheless, the study also emphasizes the complexity of finding universally optimal parameter configurations. The intricate balance between generalization and specialization remains a challenge, and MetaDE has shed light on further research into self-adapting algorithms. 
We anticipate that the insights gained from this work will inspire the development of more advanced meta-evolutionary approaches, pushing the boundaries of evolutionary optimization in even more complex and dynamic environments.





\footnotesize

% \bibliography{manuscript_references}
\input{main_merged.bbl}



















\newpage

\renewcommand\thealgorithm{S.\arabic{algorithm}}
\renewcommand\thetable{S.\arabic{table}}
\renewcommand\thefigure{S.\arabic{figure}}
\renewcommand\thesection{S.\roman{section}}
\renewcommand\theequation{S.\arabic{equation}}

\title{Supplementary Document for ``MetaDE: Evolving Differential Evolution by Differential Evolution"}
%
%
\author{Minyang Chen, Chenchen Feng,
        and Ran Cheng
        \thanks{
        Minyang Chen was with the Department of Computer Science and Engineering, Southern University of Science and Technology, Shenzhen 518055, China. E-mail: cmy1223605455@gmail.com. }
        \thanks{
        Chenchen Feng is with the Department of Computer Science and Engineering, Southern University of Science and Technology, Shenzhen 518055, China. E-mail: chenchenfengcn@gmail.com. 
        }
        \thanks{
       Ran Cheng is with the Department of Data Science and Artificial Intelligence, and the Department of Computing, The Hong Kong Polytechnic University, Hong Kong SAR, China. E-mail: ranchengcn@gmail.com. (\emph{Corresponding author: Ran Cheng})
        }
        }

\onecolumn{}

% The paper headers
\markboth{Journal of \LaTeX\ Class Files,~Vol.~0, No.~0, 0~0}%
{Shell \MakeLowercase{\textit{et al.}}: Bare Demo of IEEEtran.cls for IEEE Journals}
% The only time the second header will appear is for the odd numbered pages
% after the title page when using the twoside option.

% *** Note that you probably will NOT want to include the author's ***
% *** name in the headers of peer review papers.                   ***
% You can use \ifCLASSOPTIONpeerreview for conditional compilation here if
% you desire.


% If you want to put a publisher's ID mark on the page you can do it like
% this:
%\IEEEpubid{0000--0000/00\$00.00~\copyright~2015 IEEE}
% Remember, if you use this you must call \IEEEpubidadjcol in the second
% column for its text to clear the IEEEpubid mark.

 

% use for special paper notices
%\IEEEspecialpapernotice{(Invited Paper)}




% make the title area
\maketitle

% As a general rule, do not put math, special symbols or citations
% in the abstract or keywords.

% Note that keywords are not normally used for peerreview papers.

% For peer review papers, you can put extra information on the cover
% page as needed:
% \ifCLASSOPTIONpeerreview
% \begin{center} \bfseries EDICS Category: 3-BBND \end{center}
% \fi
%
% For peerreview papers, this IEEEtran command inserts a page break and
% creates the second title. It will be ignored for other modes.
\IEEEpeerreviewmaketitle



% \clearpage
% \begin{titlepage}
% \centering
% {\LARGE Supplementary Document for ``MetaDE: Evolving Differential Evolution by Differential Evolution"}\\[1.5cm]
% {\large Minyang Chen, Chenchen Feng, and Ran Cheng}\\[1cm]
% {\small
% Minyang Chen was with the Department of Computer Science and Engineering, Southern University of Science and Technology, Shenzhen 518055, China. E-mail: cmy1223605455@gmail.com.\\[0.2cm]
% Chenchen Feng is with the Department of Computer Science and Engineering, Southern University of Science and Technology, Shenzhen 518055, China. E-mail: chenchenfengcn@gmail.com.\\[0.2cm]
% Ran Cheng is with the Department of Data Science and Artificial Intelligence, and the Department of Computing, The Hong Kong Polytechnic University, Hong Kong SAR, China. E-mail: ranchengcn@gmail.com. (Corresponding author: Ran Cheng)
% }\\[2cm]
% \end{titlepage}


\clearpage
\begin{center}
  {\Huge Supplementary Document for ``MetaDE: Evolving \\[0.3em]
  Differential Evolution by Differential Evolution"}\\[2em]
  {\Large Minyang Chen, Chenchen Feng, and Ran Cheng}\\[6em]
\end{center}







\section{Supplementary Experimental data}\label{section:FEs}

\subsection{Supplementary Figures}\label{section:FEs}

\begin{figure*}[htpb]
\centering
\includegraphics[scale=0.3]{su_D10_all.pdf}
\caption{Convergence curves on 10D problems in CEC2022 benchmark suite. The peer DE variants are set with population size of 100.}
\label{Figure_convergence_10D_supp}
\end{figure*}

\vfill 
{\small
\noindent

Minyang Chen was with the Department of Computer Science and Engineering, Southern University of Science and Technology, Shenzhen 518055, China. E-mail: cmy1223605455@gmail.com.

Chenchen Feng is with the Department of Computer Science and Engineering, Southern University of Science and Technology, Shenzhen 518055, China. E-mail: chenchenfengcn@gmail.com.

Ran Cheng is with the Department of Data Science and Artificial Intelligence, and the Department of Computing, The Hong Kong Polytechnic University, Hong Kong SAR, China. E-mail: ranchengcn@gmail.com. \textit{(Corresponding author: Ran Cheng)}
}


\begin{figure*}[htpb]
\centering
\includegraphics[scale=0.3]{su_D20_all.pdf}
\caption{Convergence curves on 20D problems in CEC2022 benchmark suite. The peer DE variants are set with population size of 100.}
\label{Figure_convergence_20D_supp}
\end{figure*}


\begin{figure*}[htpb]
\centering
\includegraphics[scale=0.3]{su_D10NP1000_all.pdf}
\caption{Convergence curves on 10D problems in CEC2022 benchmark suite. The peer DE variants are set with population size of 1,000.}
\label{Figure_convergence_10D_NP10k_supp}
\end{figure*}


\begin{figure*}[htpb]
\centering
\includegraphics[scale=0.3]{su_D20NP1000_all.pdf}
\caption{Convergence curves on 20D problems in CEC2022 benchmark suite. The peer DE variants are set with population size of 1,000.}
\label{Figure_convergence_20D_NP10k_supp}
\end{figure*}

\clearpage

\subsection{Detailed Experimental Results}\label{section:FEs_supp}

% Tables \ref{tab:vsClass10D} and \ref{tab:vsClass20D} present the detailed results of MetaDE compared to other algorithms on 10-dimensional and 20-dimensional problems from CEC2022 within a 60-second time frame. The convergence curves for all problems are shown in Figs. \ref{Figure_convergence_10D} and \ref{Figure_convergence_20D}.

% Tables \ref{tab:NP10000 10D} and \ref{tab:NP10000 20D} display the detailed results for MetaDE versus comparison algorithms with equal concurrency (population size = 10,000) on 10-dimensional and 20-dimensional problems in CEC2022, all within a span of 60 seconds. The convergence curves for all problems are shown in Figs. \ref{Figure_convergence_10D_NP10k} and \ref{Figure_convergence_20D_NP10k}.



% Table generated by Excel2LaTeX from sheet 'Sheet1'
\begin{table}[htbp]
  \centering
  \caption{
  Detailed results on 10D problems in CEC2022 benchmark suite. The peer DE variants are set with population size of 100.
The mean and standard deviation (in parentheses) of the results over multiple runs are displayed in pairs. 
Results with the best mean values are highlighted.
  }
  \resizebox{\textwidth}{!}{
   \renewcommand{\arraystretch}{1.2}
% Table generated by Excel2LaTeX from sheet 'Experiment1 60S'
\begin{tabular}{ccccccccc}
\toprule
Func  & MetaDE & DE    & SaDE  & JaDE  & CoDE  & SHADE & LSHADE-RSP & EDEV \\
\midrule
$F_{1}$ & \textbf{0.00E+00 (0.00E+00)} & \boldmath{}\textbf{0.00E+00 (0.00E+00)$\approx$}\unboldmath{} & \boldmath{}\textbf{0.00E+00 (0.00E+00)$\approx$}\unboldmath{} & \boldmath{}\textbf{0.00E+00 (0.00E+00)$\approx$}\unboldmath{} & \boldmath{}\textbf{0.00E+00 (0.00E+00)$\approx$}\unboldmath{} & \boldmath{}\textbf{0.00E+00 (0.00E+00)$\approx$}\unboldmath{} & \boldmath{}\textbf{0.00E+00 (0.00E+00)$\approx$}\unboldmath{} & \boldmath{}\textbf{0.00E+00 (0.00E+00)$\approx$}\unboldmath{} \\
$F_{2}$ & \textbf{0.00E+00 (0.00E+00)} & 6.05E+00 (2.43E+00)$-$ & 4.86E+00 (4.29E+00)$-$ & 4.89E+00 (3.74E+00)$-$ & 4.71E+00 (2.55E+00)$-$ & 5.47E+00 (3.62E+00)$-$ & 2.38E+00 (2.58E+00)$-$ & 6.11E+00 (2.87E+00)$-$ \\
$F_{3}$ & \textbf{0.00E+00 (0.00E+00)} & \boldmath{}\textbf{0.00E+00 (0.00E+00)$\approx$}\unboldmath{} & \boldmath{}\textbf{0.00E+00 (0.00E+00)$\approx$}\unboldmath{} & \boldmath{}\textbf{0.00E+00 (0.00E+00)$\approx$}\unboldmath{} & \boldmath{}\textbf{0.00E+00 (0.00E+00)$\approx$}\unboldmath{} & \boldmath{}\textbf{0.00E+00 (0.00E+00)$\approx$}\unboldmath{} & \boldmath{}\textbf{0.00E+00 (0.00E+00)$\approx$}\unboldmath{} & \boldmath{}\textbf{0.00E+00 (0.00E+00)$\approx$}\unboldmath{} \\
$F_{4}$ & \textbf{0.00E+00 (0.00E+00)} & 6.90E+00 (4.01E+00)$-$ & 1.03E+00 (8.93E-01)$-$ & 2.31E+01 (1.19E+01)$-$ & 8.34E-01 (7.62E-01)$-$ & 3.05E+00 (9.43E-01)$-$ & 2.12E+00 (6.56E-01)$-$ & 6.52E+00 (4.51E+00)$-$ \\
$F_{5}$ & \textbf{0.00E+00 (0.00E+00)} & \boldmath{}\textbf{0.00E+00 (0.00E+00)$\approx$}\unboldmath{} & \boldmath{}\textbf{0.00E+00 (0.00E+00)$\approx$}\unboldmath{} & \boldmath{}\textbf{0.00E+00 (0.00E+00)$\approx$}\unboldmath{} & \boldmath{}\textbf{0.00E+00 (0.00E+00)$\approx$}\unboldmath{} & \boldmath{}\textbf{0.00E+00 (0.00E+00)$\approx$}\unboldmath{} & \boldmath{}\textbf{0.00E+00 (0.00E+00)$\approx$}\unboldmath{} & \boldmath{}\textbf{0.00E+00 (0.00E+00)$\approx$}\unboldmath{} \\
$F_{6}$ & \textbf{5.50E-04 (3.96E-04)} & 1.11E-01 (8.92E-02)$-$ & 6.04E+01 (2.25E+02)$-$ & 1.60E+00 (2.48E+00)$-$ & 9.08E-03 (1.17E-02)$-$ & 1.33E+00 (2.22E+00)$-$ & 3.84E-02 (5.60E-02)$-$ & 9.06E-01 (1.76E+00)$-$ \\
$F_{7}$ & \textbf{0.00E+00 (0.00E+00)} & 5.18E-02 (1.59E-01)$-$ & 2.15E-02 (1.39E-02)$-$ & \boldmath{}\textbf{0.00E+00 (0.00E+00)$\approx$}\unboldmath{} & 1.92E-03 (7.32E-03)$-$ & 6.59E-03 (1.41E-02)$-$ & 1.06E+02 (1.42E-02)$-$ & 9.54E+00 (9.86E+00)$-$ \\
$F_{8}$ & \textbf{5.52E-03 (4.41E-03)} & 1.42E-01 (2.45E-01)$-$ & 5.15E-02 (2.29E-02)$-$ & 1.74E+01 (4.79E+00)$-$ & 6.02E-03 (1.02E-02)$-$ & 2.29E+00 (5.89E+00)$-$ & 2.19E+00 (5.89E+00)$-$ & 7.06E+00 (9.36E+00)$-$ \\
$F_{9}$ & \textbf{3.36E+00 (1.77E+01)} & 2.29E+02 (7.53E-06)$-$ & 2.29E+02 (6.38E-06)$-$ & 2.29E+02 (6.38E-06)$-$ & 2.29E+02 (7.17E-06)$-$ & 2.29E+02 (7.43E-06)$-$ & 2.29E+02 (8.19E-05)$-$ & 2.29E+02 (1.01E-05)$-$ \\
$F_{10}$ & \textbf{0.00E+00 (0.00E+00)} & 1.00E+02 (5.18E-02)$-$ & 1.03E+02 (1.83E+01)$-$ & 1.04E+02 (1.91E+01)$-$ & 1.00E+02 (6.83E-02)$-$ & 1.10E+02 (3.09E+01)$-$ & 1.03E+02 (1.77E+01)$-$ & 1.07E+02 (2.62E+01)$-$ \\
$F_{11}$ & \textbf{0.00E+00 (0.00E+00)} & \boldmath{}\textbf{0.00E+00 (0.00E+00)$\approx$}\unboldmath{} & 2.42E+01 (5.52E+01)$-$ & \boldmath{}\textbf{0.00E+00 (0.00E+00)$\approx$}\unboldmath{} & \boldmath{}\textbf{0.00E+00 (0.00E+00)$\approx$}\unboldmath{} & \boldmath{}\textbf{0.00E+00 (0.00E+00)$\approx$}\unboldmath{} & \boldmath{}\textbf{0.00E+00 (0.00E+00)$\approx$}\unboldmath{} & 4.84E+00 (2.65E+01)$-$ \\
$F_{12}$ & \textbf{1.39E+02 (4.63E+01)} & 1.62E+02 (1.04E+00)$-$ & 1.63E+02 (1.57E+00)$-$ & 1.62E+02 (2.22E+00)$-$ & 1.59E+02 (1.14E+00)$-$ & 1.63E+02 (1.25E+00)$-$ & 1.64E+02 (1.38E+00)$-$ & 1.62E+02 (1.71E+00)$-$ \\
\midrule
$+$ / $\approx$ / $-$ & --    & 0/4/8 & 0/3/9 & 0/5/7 & 0/5/7 & 0/4/8 & 0/4/8 & 0/3/9 \\
\bottomrule
\end{tabular}%
}
\footnotesize
\textsuperscript{*} The Wilcoxon rank-sum tests (with a significance level of 0.05) were conducted between MetaDE and each individually.
The final row displays the number of problems where the corresponding algorithm performs statistically better ($+$),  similar ($\thickapprox$), or worse ($-$) compared to MetaDE.\\
\label{tab:vsClass10D_supp}%

\end{table}%


% Table generated by Excel2LaTeX from sheet 'Sheet1'
\begin{table}[htbp]
  \centering
  \caption{Detailed results on 20D problems in CEC2022 benchmark suite. The peer DE variants are set with population size of 100.
  The mean and standard deviation (in parentheses) of the results over multiple runs are displayed in pairs. 
Results with the best mean values are highlighted.
  }
  {
  \resizebox{\textwidth}{!}{
   \renewcommand{\arraystretch}{1.2}
% Table generated by Excel2LaTeX from sheet 'Experiment1 60S'
\begin{tabular}{ccccccccc}
\toprule
Func  & MetaDE & DE    & SaDE  & JaDE  & CoDE  & SHADE & LSHADE-RSP & EDEV \\
\midrule
$F_{1}$ & \multicolumn{1}{l}{\textbf{0.00E+00 (0.00E+00)}} & \multicolumn{1}{l}{\boldmath{}\textbf{0.00E+00 (0.00E+00)$\approx$}\unboldmath{}} & \multicolumn{1}{l}{\boldmath{}\textbf{0.00E+00 (0.00E+00)$\approx$}\unboldmath{}} & \multicolumn{1}{l}{\boldmath{}\textbf{0.00E+00 (0.00E+00)$\approx$}\unboldmath{}} & \multicolumn{1}{l}{\boldmath{}\textbf{0.00E+00 (0.00E+00)$\approx$}\unboldmath{}} & \multicolumn{1}{l}{\boldmath{}\textbf{0.00E+00 (0.00E+00)$\approx$}\unboldmath{}} & \multicolumn{1}{l}{\boldmath{}\textbf{0.00E+00 (0.00E+00)$\approx$}\unboldmath{}} & \multicolumn{1}{l}{\boldmath{}\textbf{0.00E+00 (0.00E+00)$\approx$}\unboldmath{}} \\
$F_{2}$ & \multicolumn{1}{l}{\textbf{1.26E-02 (3.74E-02)}} & \multicolumn{1}{l}{4.69E+01 (2.09E+00)$-$} & \multicolumn{1}{l}{3.50E+01 (2.21E+01)$-$} & \multicolumn{1}{l}{4.75E+01 (8.67E+00)$-$} & \multicolumn{1}{l}{4.58E+01 (1.20E+01)$-$} & \multicolumn{1}{l}{4.75E+01 (8.67E+00)$-$} & \multicolumn{1}{l}{4.30E+01 (1.71E+01)$-$} & \multicolumn{1}{l}{4.13E+01 (1.78E+01)$-$} \\
$F_{3}$ & \multicolumn{1}{l}{\textbf{0.00E+00 (0.00E+00)}} & \multicolumn{1}{l}{\boldmath{}\textbf{0.00E+00 (0.00E+00)$\approx$}\unboldmath{}} & \multicolumn{1}{l}{\boldmath{}\textbf{0.00E+00 (0.00E+00)$\approx$}\unboldmath{}} & \multicolumn{1}{l}{\boldmath{}\textbf{0.00E+00 (0.00E+00)$\approx$}\unboldmath{}} & \multicolumn{1}{l}{\boldmath{}\textbf{0.00E+00 (0.00E+00)$\approx$}\unboldmath{}} & \multicolumn{1}{l}{1.03E-08 (5.64E-08)$\approx$} & \multicolumn{1}{l}{\boldmath{}\textbf{0.00E+00 (0.00E+00)$\approx$}\unboldmath{}} & \multicolumn{1}{l}{6.91E-05 (3.31E-04)$-$} \\
$F_{4}$ & \multicolumn{1}{l}{\textbf{2.02E+00 (8.56E-01)}} & \multicolumn{1}{l}{1.95E+01 (8.17E+00)$-$} & \multicolumn{1}{l}{7.42E+00 (2.14E+00)$-$} & \multicolumn{1}{l}{7.21E+01 (3.45E+01)$-$} & \multicolumn{1}{l}{1.11E+01 (2.22E+00)$-$} & \multicolumn{1}{l}{1.27E+01 (2.73E+00)$-$} & \multicolumn{1}{l}{9.05E+00 (1.44E+00)$-$} & \multicolumn{1}{l}{2.37E+01 (1.37E+01)$-$} \\
$F_{5}$ & \multicolumn{1}{l}{\textbf{0.00E+00 (0.00E+00)}} & \multicolumn{1}{l}{\boldmath{}\textbf{0.00E+00 (0.00E+00)$\approx$}\unboldmath{}} & \multicolumn{1}{l}{7.24E-01 (1.22E+00)$-$} & \multicolumn{1}{l}{\boldmath{}\textbf{0.00E+00 (0.00E+00)$\approx$}\unboldmath{}} & \multicolumn{1}{l}{\boldmath{}\textbf{0.00E+00 (0.00E+00)$\approx$}\unboldmath{}} & \multicolumn{1}{l}{\boldmath{}\textbf{0.00E+00 (0.00E+00)$\approx$}\unboldmath{}} & \multicolumn{1}{l}{\boldmath{}\textbf{0.00E+00 (0.00E+00)$\approx$}\unboldmath{}} & \multicolumn{1}{l}{1.78E-01 (3.12E-01)$-$} \\
$F_{6}$ & \multicolumn{1}{l}{\textbf{1.16E-01 (2.79E-02)}} & \multicolumn{1}{l}{4.99E-01 (4.11E-01)$-$} & \multicolumn{1}{l}{3.13E+01 (1.54E+01)$-$} & \multicolumn{1}{l}{5.34E+01 (3.33E+01)$-$} & \multicolumn{1}{l}{1.89E+01 (1.83E+01)$-$} & \multicolumn{1}{l}{5.06E+01 (3.16E+01)$-$} & \multicolumn{1}{l}{1.25E+01 (1.00E+01)$-$} & \multicolumn{1}{l}{4.91E+03 (6.53E+03)$-$} \\
$F_{7}$ & \multicolumn{1}{l}{\textbf{5.17E-02 (6.21E-02)}} & \multicolumn{1}{l}{4.23E+00 (7.90E+00)$-$} & \multicolumn{1}{l}{1.07E+01 (5.20E+00)$-$} & \multicolumn{1}{l}{2.98E+00 (3.61E+00)$-$} & \multicolumn{1}{l}{1.16E+00 (1.26E+00)$-$} & \multicolumn{1}{l}{7.77E+00 (6.65E+00)$-$} & \multicolumn{1}{l}{1.42E+01 (8.96E+00)$-$} & \multicolumn{1}{l}{2.29E+01 (8.80E+00)$-$} \\
$F_{8}$ & \multicolumn{1}{l}{\textbf{7.19E-01 (1.02E+00)}} & \multicolumn{1}{l}{8.24E+00 (1.00E+01)$-$} & \multicolumn{1}{l}{2.10E+01 (7.26E-01)$-$} & \multicolumn{1}{l}{2.64E+01 (9.73E-01)$-$} & \multicolumn{1}{l}{1.38E+01 (8.98E+00)$-$} & \multicolumn{1}{l}{2.02E+01 (8.29E-01)$-$} & \multicolumn{1}{l}{1.96E+01 (3.80E+00)$-$} & \multicolumn{1}{l}{2.08E+01 (3.81E-01)$-$} \\
$F_{9}$ & \multicolumn{1}{l}{\textbf{1.07E+02 (1.98E+01)}} & \multicolumn{1}{l}{1.81E+02 (9.39E-06)$-$} & \multicolumn{1}{l}{1.81E+02 (3.75E-06)$-$} & \multicolumn{1}{l}{1.81E+02 (9.02E-06)$-$} & \multicolumn{1}{l}{1.81E+02 (8.41E-06)$-$} & \multicolumn{1}{l}{1.81E+02 (1.04E-05)$-$} & \multicolumn{1}{l}{1.81E+02 (2.18E-05)$-$} & \multicolumn{1}{l}{1.81E+02 (5.43E-04)$-$} \\
$F_{10}$ & \multicolumn{1}{l}{\textbf{0.00E+00 (0.00E+00)}} & \multicolumn{1}{l}{1.13E+02 (3.52E+01)$-$} & \multicolumn{1}{l}{1.00E+02 (3.03E-02)$-$} & \multicolumn{1}{l}{1.13E+02 (3.76E+01)$-$} & \multicolumn{1}{l}{1.00E+02 (3.55E-02)$-$} & \multicolumn{1}{l}{1.12E+02 (3.47E+01)$-$} & \multicolumn{1}{l}{1.11E+02 (3.42E+01)$-$} & \multicolumn{1}{l}{1.07E+02 (3.80E+01)$-$} \\
$F_{11}$ & \multicolumn{1}{l}{\textbf{7.28E-05 (3.06E-04)}} & \multicolumn{1}{l}{3.39E+02 (4.87E+01)$-$} & \multicolumn{1}{l}{3.06E+02 (2.46E+01)$-$} & \multicolumn{1}{l}{3.19E+02 (3.95E+01)$-$} & \multicolumn{1}{l}{3.39E+02 (4.87E+01)$-$} & \multicolumn{1}{l}{3.16E+02 (3.68E+01)$-$} & \multicolumn{1}{l}{3.39E+02 (4.87E+01)$-$} & \multicolumn{1}{l}{3.19E+02 (3.95E+01)$-$} \\
$F_{12}$ & \multicolumn{1}{l}{\textbf{2.29E+02 (6.08E-01)}} & \multicolumn{1}{l}{2.37E+02 (3.11E+00)$-$} & \multicolumn{1}{l}{2.41E+02 (5.26E+00)$-$} & \multicolumn{1}{l}{2.37E+02 (5.10E+00)$-$} & \multicolumn{1}{l}{2.34E+02 (2.68E+00)$-$} & \multicolumn{1}{l}{2.39E+02 (4.46E+00)$-$} & \multicolumn{1}{l}{2.44E+02 (1.63E+01)$-$} & \multicolumn{1}{l}{2.42E+02 (8.38E+00)$-$} \\
\midrule
$+$ / $\approx$ / $-$ & --    & 0/3/9 & 0/2/10 & 0/3/9 & 0/3/9 & 0/3/9 & 0/3/9 & 0/1/11 \\
\bottomrule
\end{tabular}%
    }
\footnotesize
\textsuperscript{*} The Wilcoxon rank-sum tests (with a significance level of 0.05) were conducted between MetaDE and each individually.
The final row displays the number of problems where the corresponding algorithm performs statistically better ($+$),  similar ($\thickapprox$), or worse ($-$) compared to MetaDE.\\
\label{tab:vsClass20D_supp}%
}
\end{table}%

\clearpage

% Table generated by Excel2LaTeX from sheet 'Sheet1'
\begin{table}[htbp]
  \centering
  \caption{Detailed results on 10D problems in CEC2022 benchmark suite. The peer DE variants are set with population size of 1,000. 
The mean and standard deviation (in parentheses) of the results over multiple runs are displayed in pairs. 
Results with the best mean values are highlighted.
  }
  {
  \resizebox{\textwidth}{!}{
   \renewcommand{\arraystretch}{1.2}
% Table generated by Excel2LaTeX from sheet 'Exp2 NP1000'
\begin{tabular}{ccccccccc}
\toprule
Func  & MetaDE & DE    & SaDE  & JaDE  & CoDE  & SHADE & LSHADE-RSP & EDEV \\
\midrule
$F_{1}$ & \multicolumn{1}{l}{\textbf{0.00E+00 (0.00E+00)}} & \multicolumn{1}{l}{\boldmath{}\textbf{0.00E+00 (0.00E+00)$\approx$}\unboldmath{}} & \multicolumn{1}{l}{\boldmath{}\textbf{0.00E+00 (0.00E+00)$\approx$}\unboldmath{}} & \multicolumn{1}{l}{\boldmath{}\textbf{0.00E+00 (0.00E+00)$\approx$}\unboldmath{}} & \multicolumn{1}{l}{\boldmath{}\textbf{0.00E+00 (0.00E+00)$\approx$}\unboldmath{}} & \multicolumn{1}{l}{\boldmath{}\textbf{0.00E+00 (0.00E+00)$\approx$}\unboldmath{}} & \multicolumn{1}{l}{\boldmath{}\textbf{0.00E+00 (0.00E+00)$\approx$}\unboldmath{}} & \multicolumn{1}{l}{\boldmath{}\textbf{0.00E+00 (0.00E+00)$\approx$}\unboldmath{}} \\
$F_{2}$ & \multicolumn{1}{l}{\textbf{0.00E+00 (0.00E+00)}} & \multicolumn{1}{l}{3.60E+00 (1.20E+00)$-$} & \multicolumn{1}{l}{6.87E+00 (3.63E+00)$-$} & \multicolumn{1}{l}{8.15E+00 (2.11E+00)$-$} & \multicolumn{1}{l}{2.44E+00 (1.97E+00)$-$} & \multicolumn{1}{l}{8.02E+00 (2.47E+00)$-$} & \multicolumn{1}{l}{1.12E+00 (1.79E+00)$-$} & \multicolumn{1}{l}{6.27E+00 (2.93E+00)$-$} \\
$F_{3}$ & \multicolumn{1}{l}{\textbf{0.00E+00 (0.00E+00)}} & \multicolumn{1}{l}{\boldmath{}\textbf{0.00E+00 (0.00E+00)$\approx$}\unboldmath{}} & \multicolumn{1}{l}{\boldmath{}\textbf{0.00E+00 (0.00E+00)$\approx$}\unboldmath{}} & \multicolumn{1}{l}{\boldmath{}\textbf{0.00E+00 (0.00E+00)$\approx$}\unboldmath{}} & \multicolumn{1}{l}{\boldmath{}\textbf{0.00E+00 (0.00E+00)$\approx$}\unboldmath{}} & \multicolumn{1}{l}{\boldmath{}\textbf{0.00E+00 (0.00E+00)$\approx$}\unboldmath{}} & \multicolumn{1}{l}{\boldmath{}\textbf{0.00E+00 (0.00E+00)$\approx$}\unboldmath{}} & \multicolumn{1}{l}{\boldmath{}\textbf{0.00E+00 (0.00E+00)$\approx$}\unboldmath{}} \\
$F_{4}$ & \multicolumn{1}{l}{\textbf{0.00E+00 (0.00E+00)}} & \multicolumn{1}{l}{1.50E+01 (2.59E+00)$-$} & \multicolumn{1}{l}{4.75E-01 (5.04E-01)$-$} & \multicolumn{1}{l}{2.20E+00 (5.30E-01)$-$} & \multicolumn{1}{l}{\boldmath{}\textbf{0.00E+00 (0.00E+00)$\approx$}\unboldmath{}} & \multicolumn{1}{l}{\boldmath{}\textbf{0.00E+00 (0.00E+00)$\approx$}\unboldmath{}} & \multicolumn{1}{l}{9.63E-01 (6.54E-01)$-$} & \multicolumn{1}{l}{4.91E+00 (5.24E+00)$-$} \\
$F_{5}$ & \multicolumn{1}{l}{\textbf{0.00E+00 (0.00E+00)}} & \multicolumn{1}{l}{\boldmath{}\textbf{0.00E+00 (0.00E+00)$\approx$}\unboldmath{}} & \multicolumn{1}{l}{\boldmath{}\textbf{0.00E+00 (0.00E+00)$\approx$}\unboldmath{}} & \multicolumn{1}{l}{\boldmath{}\textbf{0.00E+00 (0.00E+00)$\approx$}\unboldmath{}} & \multicolumn{1}{l}{\boldmath{}\textbf{0.00E+00 (0.00E+00)$\approx$}\unboldmath{}} & \multicolumn{1}{l}{\boldmath{}\textbf{0.00E+00 (0.00E+00)$\approx$}\unboldmath{}} & \multicolumn{1}{l}{\boldmath{}\textbf{0.00E+00 (0.00E+00)$\approx$}\unboldmath{}} & \multicolumn{1}{l}{\boldmath{}\textbf{0.00E+00 (0.00E+00)$\approx$}\unboldmath{}} \\
$F_{6}$ & \multicolumn{1}{l}{\textbf{5.50E-04 (3.96E-04)}} & \multicolumn{1}{l}{1.49E-02 (4.74E-03)$-$} & \multicolumn{1}{l}{1.95E+00 (1.54E+00)$-$} & \multicolumn{1}{l}{1.01E-01 (6.20E-02)$-$} & \multicolumn{1}{l}{7.69E-04 (4.31E-04)$\approx$} & \multicolumn{1}{l}{4.62E-03 (8.17E-03)$-$} & \multicolumn{1}{l}{1.89E-03 (7.18E-04)$-$} & \multicolumn{1}{l}{4.37E-02 (6.48E-02)$-$} \\
$F_{7}$ & \multicolumn{1}{l}{\textbf{0.00E+00 (0.00E+00)}} & \multicolumn{1}{l}{9.60E-04 (5.35E-03)$-$} & \multicolumn{1}{l}{1.98E-02 (8.61E-03)$-$} & \multicolumn{1}{l}{\boldmath{}\textbf{0.00E+00 (0.00E+00)$\approx$}\unboldmath{}} & \multicolumn{1}{l}{\boldmath{}\textbf{0.00E+00 (0.00E+00)$\approx$}\unboldmath{}} & \multicolumn{1}{l}{\boldmath{}\textbf{0.00E+00 (0.00E+00)$\approx$}\unboldmath{}} & \multicolumn{1}{l}{\boldmath{}\textbf{0.00E+00 (0.00E+00)$\approx$}\unboldmath{}} & \multicolumn{1}{l}{1.35E+00 (4.98E+00)$-$} \\
$F_{8}$ & \multicolumn{1}{l}{5.52E-03 (4.41E-03)} & \multicolumn{1}{l}{1.19E-01 (2.50E-02)$-$} & \multicolumn{1}{l}{1.13E+00 (4.44E-01)$-$} & \multicolumn{1}{l}{1.03E-01 (2.62E-02)$-$} & \multicolumn{1}{l}{\boldmath{}\textbf{0.00E+00 (0.00E+00)$\approx$}\unboldmath{}} & \multicolumn{1}{l}{8.28E-02 (2.66E-02)$-$} & \multicolumn{1}{l}{9.13E-02 (1.11E-01)$-$} & \multicolumn{1}{l}{1.03E+00 (3.58E+00)$-$} \\
$F_{9}$ & \multicolumn{1}{l}{\textbf{3.36E+00 (1.77E+01)}} & \multicolumn{1}{l}{2.29E+02 (7.85E-06)$-$} & \multicolumn{1}{l}{2.29E+02 (8.58E-06)$-$} & \multicolumn{1}{l}{2.29E+02 (8.28E-06)$-$} & \multicolumn{1}{l}{2.29E+02 (8.67E-14)$-$} & \multicolumn{1}{l}{2.29E+02 (2.74E-06)$-$} & \multicolumn{1}{l}{2.29E+02 (9.39E-06)$-$} & \multicolumn{1}{l}{2.29E+02 (1.17E-05)$-$} \\
$F_{10}$ & \multicolumn{1}{l}{\textbf{0.00E+00 (0.00E+00)}} & \multicolumn{1}{l}{1.00E+02 (1.70E-02)$-$} & \multicolumn{1}{l}{1.00E+02 (3.93E-02)$-$} & \multicolumn{1}{l}{1.00E+02 (2.49E-02)$-$} & \multicolumn{1}{l}{1.00E+02 (1.01E-02)$-$} & \multicolumn{1}{l}{1.00E+02 (1.83E-02)$-$} & \multicolumn{1}{l}{1.00E+02 (8.05E-04)$-$} & \multicolumn{1}{l}{1.00E+02 (3.93E-02)$-$} \\
$F_{11}$ & \multicolumn{1}{l}{\textbf{0.00E+00 (0.00E+00)}} & \multicolumn{1}{l}{\boldmath{}\textbf{0.00E+00 (0.00E+00)$\approx$}\unboldmath{}} & \multicolumn{1}{l}{\boldmath{}\textbf{0.00E+00 (0.00E+00)$\approx$}\unboldmath{}} & \multicolumn{1}{l}{\boldmath{}\textbf{0.00E+00 (0.00E+00)$\approx$}\unboldmath{}} & \multicolumn{1}{l}{3.65E-07 (2.03E-06)$-$} & \multicolumn{1}{l}{\boldmath{}\textbf{0.00E+00 (0.00E+00)$\approx$}\unboldmath{}} & \multicolumn{1}{l}{\boldmath{}\textbf{0.00E+00 (0.00E+00)$\approx$}\unboldmath{}} & \multicolumn{1}{l}{\boldmath{}\textbf{0.00E+00 (0.00E+00)$\approx$}\unboldmath{}} \\
$F_{12}$ & \multicolumn{1}{l}{\textbf{1.39E+02 (4.63E+01)}} & \multicolumn{1}{l}{1.60E+02 (9.79E-01)$-$} & \multicolumn{1}{l}{1.60E+02 (1.56E+00)$-$} & \multicolumn{1}{l}{1.59E+02 (1.28E+00)$-$} & \multicolumn{1}{l}{1.59E+02 (8.67E-14)$-$} & \multicolumn{1}{l}{1.61E+02 (1.69E+00)$-$} & \multicolumn{1}{l}{1.63E+02 (7.62E-01)$-$} & \multicolumn{1}{l}{1.60E+02 (1.18E+00)$-$} \\
\midrule
$+$ / $\approx$ / $-$ & --    & 0/4/8 & 0/4/8 & 0/5/7 & 0/7/5 & 0/6/6 & 0/5/7 & 0/4/8 \\
\bottomrule
\end{tabular}%
}
\footnotesize
\textsuperscript{*} The Wilcoxon rank-sum tests (with a significance level of 0.05) were conducted between MetaDE and each individually.
The final row displays the number of problems where the corresponding algorithm performs statistically better ($+$),  similar ($\thickapprox$), or worse ($-$) compared to MetaDE.\\
\label{tab:NP10000 10D_supp}%
}
\end{table}%


% Table generated by Excel2LaTeX from sheet 'Sheet1'
\begin{table}[htbp]
  \centering
  \caption{Detailed results on 20D problems in CEC2022 benchmark suite. The peer DE variants are set with population size of 1,000. 
The mean and standard deviation (in parentheses) of the results over multiple runs are displayed in pairs. 
Results with the best mean values are highlighted.
  }
  {
  \resizebox{\textwidth}{!}{
   \renewcommand{\arraystretch}{1.2}
% Table generated by Excel2LaTeX from sheet 'Exp2 NP1000'
\begin{tabular}{ccccccccc}
\toprule
Func  & MetaDE & DE    & SaDE  & JaDE  & CoDE  & SHADE & LSHADE-RSP & EDEV \\
\midrule
$F_{1}$ & \multicolumn{1}{l}{\textbf{0.00E+00 (0.00E+00)}} & \multicolumn{1}{l}{\boldmath{}\textbf{0.00E+00 (0.00E+00)$\approx$}\unboldmath{}} & \multicolumn{1}{l}{\boldmath{}\textbf{0.00E+00 (0.00E+00)$\approx$}\unboldmath{}} & \multicolumn{1}{l}{\boldmath{}\textbf{0.00E+00 (0.00E+00)$\approx$}\unboldmath{}} & \multicolumn{1}{l}{\boldmath{}\textbf{0.00E+00 (0.00E+00)$\approx$}\unboldmath{}} & \multicolumn{1}{l}{\boldmath{}\textbf{0.00E+00 (0.00E+00)$\approx$}\unboldmath{}} & \multicolumn{1}{l}{\boldmath{}\textbf{0.00E+00 (0.00E+00)$\approx$}\unboldmath{}} & \multicolumn{1}{l}{\boldmath{}\textbf{0.00E+00 (0.00E+00)$\approx$}\unboldmath{}} \\
$F_{2}$ & \multicolumn{1}{l}{\textbf{1.26E-02 (3.74E-02)}} & \multicolumn{1}{l}{4.49E+01 (0.00E+00)$-$} & \multicolumn{1}{l}{4.91E+01 (7.67E-06)$-$} & \multicolumn{1}{l}{4.91E+01 (0.00E+00)$-$} & \multicolumn{1}{l}{4.91E+01 (0.00E+00)$-$} & \multicolumn{1}{l}{4.91E+01 (0.00E+00)$-$} & \multicolumn{1}{l}{4.52E+01 (1.05E+00)$-$} & \multicolumn{1}{l}{4.89E+01 (1.05E+00)$-$} \\
$F_{3}$ & \multicolumn{1}{l}{\textbf{0.00E+00 (0.00E+00)}} & \multicolumn{1}{l}{\boldmath{}\textbf{0.00E+00 (0.00E+00)$\approx$}\unboldmath{}} & \multicolumn{1}{l}{\boldmath{}\textbf{0.00E+00 (0.00E+00)$\approx$}\unboldmath{}} & \multicolumn{1}{l}{\boldmath{}\textbf{0.00E+00 (0.00E+00)$\approx$}\unboldmath{}} & \multicolumn{1}{l}{\boldmath{}\textbf{0.00E+00 (0.00E+00)$\approx$}\unboldmath{}} & \multicolumn{1}{l}{\boldmath{}\textbf{0.00E+00 (0.00E+00)$\approx$}\unboldmath{}} & \multicolumn{1}{l}{\boldmath{}\textbf{0.00E+00 (0.00E+00)$\approx$}\unboldmath{}} & \multicolumn{1}{l}{\boldmath{}\textbf{0.00E+00 (0.00E+00)$\approx$}\unboldmath{}} \\
$F_{4}$ & \multicolumn{1}{l}{\textbf{2.02E+00 (8.56E-01)}} & \multicolumn{1}{l}{8.33E+01 (5.85E+00)$-$} & \multicolumn{1}{l}{6.09E+00 (1.02E+00)$-$} & \multicolumn{1}{l}{1.07E+01 (1.95E+00)$-$} & \multicolumn{1}{l}{8.02E+00 (1.08E+00)$-$} & \multicolumn{1}{l}{7.39E+00 (1.17E+00)$-$} & \multicolumn{1}{l}{4.82E+00 (8.64E-01)$-$} & \multicolumn{1}{l}{1.81E+01 (1.52E+01)$-$} \\
$F_{5}$ & \multicolumn{1}{l}{\textbf{0.00E+00 (0.00E+00)}} & \multicolumn{1}{l}{\boldmath{}\textbf{0.00E+00 (0.00E+00)$\approx$}\unboldmath{}} & \multicolumn{1}{l}{6.16E-02 (2.71E-01)$-$} & \multicolumn{1}{l}{\boldmath{}\textbf{0.00E+00 (0.00E+00)$\approx$}\unboldmath{}} & \multicolumn{1}{l}{\boldmath{}\textbf{0.00E+00 (0.00E+00)$\approx$}\unboldmath{}} & \multicolumn{1}{l}{\boldmath{}\textbf{0.00E+00 (0.00E+00)$\approx$}\unboldmath{}} & \multicolumn{1}{l}{\boldmath{}\textbf{0.00E+00 (0.00E+00)$\approx$}\unboldmath{}} & \multicolumn{1}{l}{\boldmath{}\textbf{0.00E+00 (0.00E+00)$\approx$}\unboldmath{}} \\
$F_{6}$ & \multicolumn{1}{l}{1.16E-01 (2.79E-02)} & \multicolumn{1}{l}{8.86E+00 (1.34E+00)$-$} & \multicolumn{1}{l}{1.72E+02 (5.24E+02)$-$} & \multicolumn{1}{l}{1.80E+00 (7.49E-01)$-$} & \multicolumn{1}{l}{2.07E-01 (5.04E-02)$-$} & \multicolumn{1}{l}{1.35E-01 (5.23E-02)$\approx$} & \multicolumn{1}{l}{\textbf{7.35E-02 (3.33E-02)$+$}} & \multicolumn{1}{l}{2.70E+00 (1.13E+01)$-$} \\
$F_{7}$ & \multicolumn{1}{l}{5.17E-02 (6.21E-02)} & \multicolumn{1}{l}{3.28E+01 (1.82E+00)$-$} & \multicolumn{1}{l}{1.57E+01 (3.45E+00)$-$} & \multicolumn{1}{l}{1.31E+01 (2.39E+00)$-$} & \multicolumn{1}{l}{\textbf{2.02E-03 (1.44E-03)$+$}} & \multicolumn{1}{l}{9.48E+00 (1.76E+00)$-$} & \multicolumn{1}{l}{2.92E+00 (9.57E-01)$-$} & \multicolumn{1}{l}{1.68E+01 (1.04E+01)$-$} \\
$F_{8}$ & \multicolumn{1}{l}{\textbf{7.19E-01 (1.02E+00)}} & \multicolumn{1}{l}{2.41E+01 (2.25E+00)$-$} & \multicolumn{1}{l}{2.11E+01 (1.76E+00)$-$} & \multicolumn{1}{l}{2.15E+01 (1.27E+00)$-$} & \multicolumn{1}{l}{1.12E+01 (1.98E+00)$-$} & \multicolumn{1}{l}{1.99E+01 (2.81E+00)$-$} & \multicolumn{1}{l}{8.53E+00 (4.47E+00)$-$} & \multicolumn{1}{l}{1.88E+01 (5.20E+00)$-$} \\
$F_{9}$ & \multicolumn{1}{l}{\textbf{1.07E+02 (1.98E+01)}} & \multicolumn{1}{l}{1.81E+02 (2.68E-05)$-$} & \multicolumn{1}{l}{1.81E+02 (2.75E-05)$-$} & \multicolumn{1}{l}{1.81E+02 (1.08E-05)$-$} & \multicolumn{1}{l}{1.81E+02 (7.25E-06)$-$} & \multicolumn{1}{l}{1.81E+02 (9.14E-06)$-$} & \multicolumn{1}{l}{1.81E+02 (1.24E-05)$-$} & \multicolumn{1}{l}{1.81E+02 (1.01E-04)$-$} \\
$F_{10}$ & \multicolumn{1}{l}{\textbf{0.00E+00 (0.00E+00)}} & \multicolumn{1}{l}{1.00E+02 (3.21E-02)$-$} & \multicolumn{1}{l}{1.00E+02 (2.58E-02)$-$} & \multicolumn{1}{l}{1.00E+02 (3.51E-02)$-$} & \multicolumn{1}{l}{1.00E+02 (1.80E-02)$-$} & \multicolumn{1}{l}{1.00E+02 (1.80E-02)$-$} & \multicolumn{1}{l}{1.00E+02 (1.21E-02)$-$} & \multicolumn{1}{l}{1.00E+02 (4.13E-02)$-$} \\
$F_{11}$ & \multicolumn{1}{l}{\textbf{7.28E-05 (3.06E-04)}} & \multicolumn{1}{l}{3.97E+02 (1.80E+01)$-$} & \multicolumn{1}{l}{3.26E+02 (4.45E+01)$-$} & \multicolumn{1}{l}{3.16E+02 (3.74E+01)$-$} & \multicolumn{1}{l}{3.81E+02 (4.02E+01)$-$} & \multicolumn{1}{l}{3.16E+02 (3.74E+01)$-$} & \multicolumn{1}{l}{3.77E+02 (4.25E+01)$-$} & \multicolumn{1}{l}{3.19E+02 (9.80E+01)$-$} \\
$F_{12}$ & \multicolumn{1}{l}{\textbf{2.29E+02 (6.08E-01)}} & \multicolumn{1}{l}{2.35E+02 (2.45E+00)$-$} & \multicolumn{1}{l}{2.34E+02 (2.55E+00)$-$} & \multicolumn{1}{l}{2.30E+02 (1.99E+00)$-$} & \multicolumn{1}{l}{2.30E+02 (1.40E+00)$-$} & \multicolumn{1}{l}{2.32E+02 (1.09E+00)$-$} & \multicolumn{1}{l}{2.33E+02 (2.27E+00)$-$} & \multicolumn{1}{l}{2.38E+02 (3.81E+00)$-$} \\
\midrule
$+$ / $\approx$ / $-$ & --    & 0/3/9 & 0/2/10 & 0/3/9 & 1/3/8 & 0/4/8 & 1/3/8 & 0/3/9 \\
\bottomrule
\end{tabular}%
}
\footnotesize
\textsuperscript{*} The Wilcoxon rank-sum tests (with a significance level of 0.05) were conducted between MetaDE and each individually.
The final row displays the number of problems where the corresponding algorithm performs statistically better ($+$),  similar ($\thickapprox$), or worse ($-$) compared to MetaDE.\\
\label{tab:NP10000 20D_supp}%
}
\end{table}%


% \subsection{Supplementary results}\label{section:FEs}
 % This experiment leveraged parallel GPU computation and constrained the runtime: 30 seconds for the 10-dimensional problems and 60 seconds for the 20-dimensional problems. 
 
% We recorded the number of FEs each algorithm achieved within 60s, which are presented in Table \ref{tab:FEs}. 
 
%  In addition, when the population size of the comparative DE variants was increased to 10,000 (same level of concurrency as MetaDE), the FEs achieved by all algorithms are displayed in Table \ref{tab:FEs NP10000}.

% The results show that MetaDE achieved considerably more FEs within a given time than the other algorithms. This demonstrates that MetaDE has a high degree of parallelism, making it particularly well-suited for GPU computing.



% Table generated by Excel2LaTeX from sheet 'Sheet1'
\begin{table}[htbp]
  \centering
  \caption{The number of FEs achieved by each algorithm within \SI{60}{\second}. The peer DE variants are set with population size of 1,000.}
  {
           \renewcommand{\arraystretch}{1}
 \renewcommand{\tabcolsep}{10pt}
% Table generated by Excel2LaTeX from sheet 'Exp2 NP10000'
\begin{tabular}{cccccccccc}
\toprule
Dim   & Func  & MetaDE & DE    & SaDE  & JaDE  & CoDE  & SHADE &LSHADE-RSP&EDEV\\
\midrule
\multirow{12}[2]{*}{10D} & $F_{1}$ & \textbf{1.85E+09} & 3.91E+07 & 4.38E+06 & 6.26E+06 & 7.67E+07 & 4.62E+06&1.92E+07&1.91E+07 \\
      & $F_{2}$ & \textbf{1.84E+09} & 4.05E+07 &4.35E+06 & 6.20E+06 & 7.61E+07 & 4.63E+06&1.89E+07&1.96E+07 \\
      & $F_{3}$ & \textbf{1.50E+09} & 3.63E+07 & 4.31E+06 &6.18E+06& 6.87E+07& 4.54E+06&1.84E+07 &1.89E+07\\
      & $F_{4}$ & \textbf{1.84E+09} &3.77E+07 & 4.27E+06 & 6.16E+06 &7.42E+07 & 4.64E+06&1.89E+07&1.98E+07 \\
      & $F_{5}$ & \textbf{1.83E+09} & 3.85E+07 &4.33E+06 & 6.16E+06 & 7.50E+07& 4.64E+06&1.85E+07& 1.99E+07\\
      & $F_{6}$ & \textbf{1.84E+09} &3.71E+07 & 4.30E+06 & 6.01E+06& 7.38E+07 & 4.61E+06& 1.86E+07&1.99E+07\\
      & $F_{7}$ & \textbf{1.74E+09} &3.71E+07 & 4.15E+06 & 5.84E+06 & 7.05E+07& 4.37E+06&1.77E+07&1.91E+07\\
      & $F_{8}$ & \textbf{1.72E+09} & 3.69E+07 & 4.02E+06 & 5.75E+06 & 6.82E+07 & 4.35E+06& 1.71E+07&1.93E+07\\
      & $F_{9}$ & \textbf{1.78E+09} &3.81E+07 &4.35E+06 &6.10E+06 & 6.84E+07 & 4.56E+06&1.84E+07 &1.92E+07\\
      & $F_{10}$ & \textbf{1.44E+09} & 3.50E+07&4.12E+06 &  5.77E+06 & 6.94E+07 & 4.41E+06&1.76E+07& 1.83E+07\\
      & $F_{11}$ & \textbf{1.46E+09} & 3.43E+07& 4.11E+06 & 5.87E+06 & 6.89E+07 & 4.36E+06 &1.69E+07&1.83E+07\\
      & $F_{12}$ & \textbf{1.43E+09} &3.37E+07 & 3.97E+06 & 5.81E+06 & 6.56E+07 & 4.32E+06 &1.72E+07&1.92E+07\\
\midrule
\midrule
\multirow{12}[2]{*}{20D} & $F_{1}$ & \textbf{1.66E+09} & 3.93E+07 &4.29E+06 & 5.94E+06 & 7.20E+07 & 4.62E+06&1.87E+07&1.94E+07 \\
      & $F_{2}$ & \textbf{1.66E+09} & 3.89E+07 &4.17E+06 & 5.94E+06 & 7.33E+07 & 4.61E+06&1.91E+07& 1.95E+07\\
      & $F_{3}$ & \textbf{1.18E+09} & 3.38E+07 & 4.16E+06& 5.93E+06 &  6.77E+07& 4.49E+06& 1.72E+07&1.85E+07\\
      & $F_{4}$ & \textbf{1.65E+09} &3.56E+07 & 4.27E+06 & 5.85E+06 & 7.05E+07& 4.63E+06& 1.90E+07&1.94E+07\\
      & $F_{5}$ & \textbf{1.64E+09} & 3.88E+07 & 4.20E+06 & 6.04E+06& 7.16E+07& 4.62E+06&1.86E+07&1.95E+07 \\
      & $F_{6}$ & \textbf{1.64E+09} & 3.73E+07 & 4.25E+06 & 5.84E+06 & 7.10E+07 & 4.61E+06&1.86E+07& 1.92E+07\\
      & $F_{7}$ & \textbf{1.45E+09} & 3.13E+07 & 3.87E+06 & 5.27E+06 & 6.40E+07 & 4.26E+06& 1.72E+07&1.79E+07\\
      & $F_{8}$ & \textbf{1.44E+09} & 3.08E+07 & 3.53E+06& 5.29E+06 & 5.98E+07& 4.09E+06&1.69E+07&1.70E+07 \\
      & $F_{9}$ & \textbf{1.57E+09} & 3.62E+07 &4.05E+06 & 6.15E+06 &  6.91E+07& 4.54E+06&1.84E+07& 1.89E+07\\
      & $F_{10}$ & \textbf{9.80E+08} &2.94E+07 & 3.74E+06 & 5.28E+06& 5.69E+07& 4.12E+06& 1.54E+07& 1.63E+07\\
      & $F_{11}$ & \textbf{1.00E+09} & 2.37E+07 &3.76E+06 & 5.38E+06&5.86E+07& 4.04E+06&1.59E+07& 1.64E+07 \\
      & $F_{12}$ & \textbf{9.90E+08} & 2.33E+07 & 3.76E+06 &5.39E+06&5.47E+07& 4.09E+06& 1.57E+07& 1.60E+07\\
\bottomrule
\end{tabular}%
}
  \label{tab:FEs NP10000_supp}%
\end{table}%

\clearpage

\section{Supplementary information for the application of robotics control}\label{section_brax_supp}

\subsection{Illustrations of the robotics control tasks}\label{section_brax_image_supp}

% Figs. \ref{Figure_swimmer}-\ref{Figure_hopper}, illustrate the three robotics control tasks from the Brax \cite{brax} reinforcement learning library: swimmer, reacher, and hopper.

% \begin{enumerate}
%     \item Swimmer: A serpentine agent that must coordinate joint movements to navigate through a fluid environment, aiming for efficient propulsion and forward movement.
%     \item Reacher: A robotic arm environment where the goal is to control joint torques to reach a target point with precision, testing the fine control of the learning algorithm.
%     \item Hopper: A one-legged robot that must learn to balance and hop forward as far and as fast as possible, providing a benchmark for locomotion and stability in dynamic environments.
% \end{enumerate}

\begin{figure}[!h]
\centering
\includegraphics[width=6cm, height=3cm]{su_swimmer.png}
\caption{The swimmer task in Brax. It is designed to simulate a multi-jointed creature navigating through a fluid medium.}
\label{Figure_swimmer_supp}
\end{figure}


\begin{figure}[!h]
\centering
\includegraphics[width=6cm, height=3cm]{su_hopper.png}
\caption{The hopper task in Brax. It resembles a one-legged robotic creature with the objective to hop forward smoothly and quickly.}
\label{Figure_hopper_supp}
\end{figure}

\begin{figure}[!h]
\centering
\includegraphics[width=6cm, height=3cm]{su_reacher.png}
\caption{The reacher task in Brax. It simulates a robotic arm tasked with reaching a target location.}
\label{Figure_reacher_supp}
\end{figure}


\subsection{Supplementary results}\label{section_brax_results_supp}
% Table \ref{tab:comparative-reward-analysis} displays the results of the neuroevolution experimrnt for 60 minutes.

\begin{table}[htbp]
\centering
\caption{Rewards achieved by MetaDE and peer evolutionary algorithms on the robotics tasks. 
The mean and standard deviation (in parentheses) of the results over multiple runs are displayed in pairs. 
Results with the best mean values are highlighted.
}
\label{tab:comparative-reward-analysis_supp}
{%
 \renewcommand{\arraystretch}{1}
\renewcommand{\tabcolsep}{3pt}
% Table generated by Excel2LaTeX from sheet 'Exp6 brax'
\begin{tabular}{cccccccc}
\toprule
Task & MetaDE  & \textbf{CSO } & \textbf{CMAES } & \textbf{SHADE } & \textbf{DE}&\textbf{LSHADE-RSP}&\textbf{EDEV} \\
\midrule
Swimmer  & \textbf{ 190.85 (2.39) } &  183.45 (1.15)  &  186.07 (2.68)  & 185.90 (3.31) &  182.15 (2.54) &186.36 (1.21)&145.52 (41.22)\\
%\midrule
Hopper   &  1187.53 (122.72) & 1330.45 (325.55) &  \textbf{1389.72 (474.41)}  &  1022.25 (122.94)  &  871.06 (147.12)&1102.23 (100.33)&457.22 (80.16) \\
%\midrule
Reacher  &  -21.05 (5.05)  &  -504.07 (112.18)  & \textbf{ -3.76 (1.05) } &  -343.18 (105.65)  &  -522.99 (142.78)    &    -342.74 (36.63)  &  -493.46 (173.67)\\
\bottomrule
\end{tabular}%

}
\end{table}

\clearpage
\section{More Evolvers}\label{subsection Evolver Comparison_supp}
We selected \texttt{DE/rand/1/bin} as the evolver for its simplicity and adaptability, hypothesizing its capability to self-evolve. Nonetheless, it is also interesting to evaluate the performance implications of utilizing other EAs as evolvers.

In this experiment, we maintained the identical framework and parameter settings for MetaDE as utilized in the primary experiments. 
The distinction lies in the comparative evaluation of the effectiveness of DE, PSO, Natural Evolution Strategies (NES) \citesupp{NES}, and Competitive Swarm Optimization (CSO) \citesupp{CSO} as evolvers. 
For PSO, parameters were set to the recommended values of $w=0.729$ and $c_1=c_2=1.49$ \citesupp{PSOParamSetting}. 
It is important to note that CSO and NES do not require parameter tuning.

As shown in Tables \ref{tab:diffTunner 10D}-\ref{tab:diffTunner 20D} and Figs. \ref{Figure_evolver_10D}-\ref{Figure_evolver_20D}, the experimental outcomes suggest minimal performance disparities among the EAs when employed as evolvers. 
Specifically, DE demonstrates a marginal advantage in 10-dimensional problems, whereas the performance is comparably uniform across all evolvers for 20-dimensional problems. 
These results imply that the selection of an evolver within the MetaDE framework is relatively flexible. 
Such findings highlight the adaptability and flexibility of MetaDE, illustrating that its efficiency is not significantly influenced by the particular choice of evolver.


% Table generated by Excel2LaTeX from sheet 'Sheet1'
\begin{table}[htbp]
  \centering
  \caption{
  Performance of MetaDE with different evolvers on 10D problems in CEC2022 benchmark suite. 
The mean and standard deviation (in parentheses) of the results over multiple runs are displayed in pairs. 
Results with the best mean values are highlighted.
  }
  {
         \renewcommand{\arraystretch}{1}
 \renewcommand{\tabcolsep}{10pt}
% Table generated by Excel2LaTeX from sheet 'Exp5 evolver'
\begin{tabular}{ccccc}
\toprule
Func  & DE    & PSO   & NES   & CSO \\
\midrule
$F_{1}$ & \textbf{0.00E+00 (0.00E+00)} & \boldmath{}\textbf{0.00E+00 (0.00E+00)$\approx$}\unboldmath{} & \boldmath{}\textbf{0.00E+00 (0.00E+00)$\approx$}\unboldmath{} & \boldmath{}\textbf{0.00E+00 (0.00E+00)$\approx$}\unboldmath{} \\
$F_{2}$ & \textbf{0.00E+00 (0.00E+00)} & \boldmath{}\textbf{0.00E+00 (0.00E+00)$\approx$}\unboldmath{} & 1.60E-06 (2.82E-06)$-$ & \boldmath{}\textbf{0.00E+00 (0.00E+00)$\approx$}\unboldmath{} \\
$F_{3}$ & \textbf{0.00E+00 (0.00E+00)} & \boldmath{}\textbf{0.00E+00 (0.00E+00)$\approx$}\unboldmath{} & \boldmath{}\textbf{0.00E+00 (0.00E+00)$\approx$}\unboldmath{} & \boldmath{}\textbf{0.00E+00 (0.00E+00)$\approx$}\unboldmath{} \\
$F_{4}$ & \textbf{0.00E+00 (0.00E+00)} & 1.41E-03 (7.70E-03)$\approx$ & 3.23E-02 (1.76E-01)$\approx$ & \boldmath{}\textbf{0.00E+00 (0.00E+00)$\approx$}\unboldmath{} \\
$F_{5}$ & \textbf{0.00E+00 (0.00E+00)} & \boldmath{}\textbf{0.00E+00 (0.00E+00)$\approx$}\unboldmath{} & \boldmath{}\textbf{0.00E+00 (0.00E+00)$\approx$}\unboldmath{} & \boldmath{}\textbf{0.00E+00 (0.00E+00)$\approx$}\unboldmath{} \\
$F_{6}$ & \textbf{5.50E-04 (3.96E-04)} & 1.45E-03 (1.61E-03)$\approx$ & 1.66E-03 (1.79E-03)$-$ & 1.37E-03 (9.80E-04)$-$ \\
$F_{7}$ & \textbf{0.00E+00 (0.00E+00)} & \boldmath{}\textbf{0.00E+00 (0.00E+00)$\approx$}\unboldmath{} & \boldmath{}\textbf{0.00E+00 (0.00E+00)$\approx$}\unboldmath{} & \boldmath{}\textbf{0.00E+00 (0.00E+00)$\approx$}\unboldmath{} \\
$F_{8}$ & 5.52E-03 (4.41E-03) & \textbf{1.77E-06 (1.48E-06)$+$} & 1.83E-02 (1.09E-02)$-$ & 1.57E-03 (2.26E-03)$+$ \\
$F_{9}$ & \textbf{3.36E+00 (1.77E+01)} & 1.40E+01 (4.63E+01)$-$ & 2.27E+01 (4.18E+01)$\approx$ & 1.75E+01 (4.86E+01)$-$ \\
$F_{10}$ & \textbf{0.00E+00 (0.00E+00)} & \boldmath{}\textbf{0.00E+00 (0.00E+00)$\approx$}\unboldmath{} & \boldmath{}\textbf{0.00E+00 (0.00E+00)$\approx$}\unboldmath{} & 7.38E-02 (3.99E-01)$\approx$ \\
$F_{11}$ & \textbf{0.00E+00 (0.00E+00)} & \boldmath{}\textbf{0.00E+00 (0.00E+00)$\approx$}\unboldmath{} & \boldmath{}\textbf{0.00E+00 (0.00E+00)$\approx$}\unboldmath{} & \boldmath{}\textbf{0.00E+00 (0.00E+00)$\approx$}\unboldmath{} \\
$F_{12}$ & \textbf{1.39E+02 (4.63E+01)} & 1.48E+02 (3.39E+01)$\approx$ & 1.59E+02 (6.82E-06)$-$ & 1.54E+02 (1.78E+01)$\approx$ \\
\midrule
$+$ / $\approx$ / $-$ & --    & 1/10/1 & 4/8/0 & 1/9/2 \\
\bottomrule
\end{tabular}%
}

\footnotesize
\textsuperscript{*} The Wilcoxon rank-sum tests (with a significance level of 0.05) were conducted between MetaDE and each individually.
The final row displays the number of problems where the corresponding evlover performs statistically better ($+$),  similar ($\thickapprox$), or worse ($-$) compared to DE.\\
\label{tab:diffTunner 10D_supp}%
\end{table}%


% Table generated by Excel2LaTeX from sheet 'Sheet1'
\begin{table}[htbp]
  \centering
  \caption{
  Performance of MetaDE with different evolvers on 20D problems in CEC2022 benchmark suite.
The mean and standard deviation (in parentheses) of the results over multiple runs are displayed in pairs. 
Results with the best mean values are highlighted.
  }
  {
        \renewcommand{\arraystretch}{1}
 \renewcommand{\tabcolsep}{10pt}
% Table generated by Excel2LaTeX from sheet 'Exp5 evolver'
\begin{tabular}{ccccc}
\toprule
Func  & DE    & PSO   & NES   & CSO \\
\midrule
$F_{1}$ & \multicolumn{1}{l}{\textbf{0.00E+00 (0.00E+00)}} & \multicolumn{1}{l}{\boldmath{}\textbf{0.00E+00 (0.00E+00)$\approx$}\unboldmath{}} & \multicolumn{1}{l}{\boldmath{}\textbf{0.00E+00 (0.00E+00)$\approx$}\unboldmath{}} & \multicolumn{1}{l}{\boldmath{}\textbf{0.00E+00 (0.00E+00)$\approx$}\unboldmath{}} \\
$F_{2}$ & \multicolumn{1}{l}{1.26E-02 (3.74E-02)} & \multicolumn{1}{l}{4.28E-02 (1.25E-01)$\approx$} & \multicolumn{1}{l}{1.11E+00 (1.30E+00)$-$} & \multicolumn{1}{l}{\textbf{0.00E+00 (0.00E+00)$+$}} \\
$F_{3}$ & \multicolumn{1}{l}{\textbf{0.00E+00 (0.00E+00)}} & \multicolumn{1}{l}{\boldmath{}\textbf{0.00E+00 (0.00E+00)$\approx$}\unboldmath{}} & \multicolumn{1}{l}{\boldmath{}\textbf{0.00E+00 (0.00E+00)$\approx$}\unboldmath{}} & \multicolumn{1}{l}{\boldmath{}\textbf{0.00E+00 (0.00E+00)$\approx$}\unboldmath{}} \\
$F_{4}$ & \multicolumn{1}{l}{\textbf{2.02E+00 (8.56E-01)}} & \multicolumn{1}{l}{2.51E+00 (1.19E+00)$-$} & \multicolumn{1}{l}{5.53E+00 (1.55E+00)$-$} & \multicolumn{1}{l}{2.85E+00 (1.48E+00)$\approx$} \\
$F_{5}$ & \multicolumn{1}{l}{\textbf{0.00E+00 (0.00E+00)}} & \multicolumn{1}{l}{\boldmath{}\textbf{0.00E+00 (0.00E+00)$\approx$}\unboldmath{}} & \multicolumn{1}{l}{\boldmath{}\textbf{0.00E+00 (0.00E+00)$\approx$}\unboldmath{}} & \multicolumn{1}{l}{\boldmath{}\textbf{0.00E+00 (0.00E+00)$\approx$}\unboldmath{}} \\
$F_{6}$ & \multicolumn{1}{l}{1.16E-01 (2.79E-02)} & \multicolumn{1}{l}{1.81E-01 (2.42E-01)$-$} & \multicolumn{1}{l}{9.32E-02 (2.65E-02)$+$} & \multicolumn{1}{l}{\boldmath{}\textbf{1.10E-01 (2.85E-02)$\approx$}\unboldmath{}} \\
$F_{7}$ & \multicolumn{1}{l}{5.17E-02 (6.21E-02)} & \multicolumn{1}{l}{1.61E-01 (3.84E-01)$\approx$} & \multicolumn{1}{l}{8.13E-01 (6.01E-01)$-$} & \multicolumn{1}{l}{\textbf{6.37E-02 (2.03E-01)$+$}} \\
$F_{8}$ & \multicolumn{1}{l}{\textbf{7.19E-01 (1.02E+00)}} & \multicolumn{1}{l}{2.31E+00 (4.16E+00)$\approx$} & \multicolumn{1}{l}{1.82E+01 (3.82E+00)$-$} & \multicolumn{1}{l}{2.19E+00 (4.05E+00)$\approx$} \\
$F_{9}$ & \multicolumn{1}{l}{\textbf{1.07E+02 (1.98E+01)}} & \multicolumn{1}{l}{1.34E+02 (3.72E+01)$-$} & \multicolumn{1}{l}{1.81E+02 (3.75E-06)$-$} & \multicolumn{1}{l}{1.75E+02 (1.81E+01)$-$} \\
$F_{10}$ & \multicolumn{1}{l}{\textbf{0.00E+00 (0.00E+00)}} & \multicolumn{1}{l}{\boldmath{}\textbf{0.00E+00 (0.00E+00)$\approx$}\unboldmath{}} & \multicolumn{1}{l}{\boldmath{}\textbf{0.00E+00 (0.00E+00)$\approx$}\unboldmath{}} & \multicolumn{1}{l}{7.86E-01 (2.39E+00)$-$} \\
$F_{11}$ & \multicolumn{1}{l}{7.28E-05 (3.06E-04)} & \multicolumn{1}{l}{1.01E-02 (3.90E-02)$\approx$} & \multicolumn{1}{l}{8.79E-03 (1.52E-02)$-$} & \multicolumn{1}{l}{\textbf{1.45E-05 (0.00E+00)$-$}} \\
$F_{12}$ & \multicolumn{1}{l}{\textbf{2.29E+02 (6.08E-01)}} & \multicolumn{1}{l}{2.29E+02 (1.17E+00)$\approx$} & \multicolumn{1}{l}{2.31E+02 (8.77E-01)$-$} & \multicolumn{1}{l}{2.30E+02 (9.60E-01)$-$} \\
\midrule
$+$ / $\approx$ / $-$ & --    & 0/9/3 & 1/4/7 & 2/6/4 \\
\bottomrule
\end{tabular}%
}

\footnotesize
\textsuperscript{*} The Wilcoxon rank-sum tests (with a significance level of 0.05) were conducted between MetaDE and each individually.
The final row displays the number of problems where the corresponding evlover performs statistically better ($+$),  similar ($\thickapprox$), or worse ($-$) compared to DE.\\
\label{tab:diffTunner 20D_supp}%
\end{table}%

\clearpage

\begin{figure*}[htpb]
\centering
\includegraphics[scale=0.29]{su_D10_all_evolver.pdf}
\caption{Convergence curves with different evolvers on 10D problems in CEC2022 benchmark suite.}
\label{Figure_evolver_10D_supp}
\end{figure*}


\begin{figure*}[htpb]
\centering
\includegraphics[scale=0.29]{su_D20_all_evolver.pdf}
\caption{Convergence curves with different evolvers on 20D problems in CEC2022 benchmark suite.}
\label{Figure_evolver_20D_supp}
\end{figure*}


\clearpage
% \bibliography{Supplement_references}

% \bibliographystylesupp{IEEEtran}
% \bibliographysupp{Supplement_references}
\input{supp.bbl}







\end{document}






















\newpage

\renewcommand\thealgorithm{S.\arabic{algorithm}}
\renewcommand\thetable{S.\arabic{table}}
\renewcommand\thefigure{S.\arabic{figure}}
\renewcommand\thesection{S.\roman{section}}
\renewcommand\theequation{S.\arabic{equation}}

\title{Supplementary Document for ``MetaDE: Evolving Differential Evolution by Differential Evolution"}
%
%
\author{Minyang Chen, Chenchen Feng,
        and Ran Cheng
        \thanks{
        Minyang Chen was with the Department of Computer Science and Engineering, Southern University of Science and Technology, Shenzhen 518055, China. E-mail: cmy1223605455@gmail.com. }
        \thanks{
        Chenchen Feng is with the Department of Computer Science and Engineering, Southern University of Science and Technology, Shenzhen 518055, China. E-mail: chenchenfengcn@gmail.com. 
        }
        \thanks{
       Ran Cheng is with the Department of Data Science and Artificial Intelligence, and the Department of Computing, The Hong Kong Polytechnic University, Hong Kong SAR, China. E-mail: ranchengcn@gmail.com. (\emph{Corresponding author: Ran Cheng})
        }
        }

\onecolumn{}

% The paper headers
\markboth{Journal of \LaTeX\ Class Files,~Vol.~0, No.~0, 0~0}%
{Shell \MakeLowercase{\textit{et al.}}: Bare Demo of IEEEtran.cls for IEEE Journals}
% The only time the second header will appear is for the odd numbered pages
% after the title page when using the twoside option.

% *** Note that you probably will NOT want to include the author's ***
% *** name in the headers of peer review papers.                   ***
% You can use \ifCLASSOPTIONpeerreview for conditional compilation here if
% you desire.


% If you want to put a publisher's ID mark on the page you can do it like
% this:
%\IEEEpubid{0000--0000/00\$00.00~\copyright~2015 IEEE}
% Remember, if you use this you must call \IEEEpubidadjcol in the second
% column for its text to clear the IEEEpubid mark.

 

% use for special paper notices
%\IEEEspecialpapernotice{(Invited Paper)}




% make the title area
\maketitle

% As a general rule, do not put math, special symbols or citations
% in the abstract or keywords.

% Note that keywords are not normally used for peerreview papers.

% For peer review papers, you can put extra information on the cover
% page as needed:
% \ifCLASSOPTIONpeerreview
% \begin{center} \bfseries EDICS Category: 3-BBND \end{center}
% \fi
%
% For peerreview papers, this IEEEtran command inserts a page break and
% creates the second title. It will be ignored for other modes.
\IEEEpeerreviewmaketitle



% \clearpage
% \begin{titlepage}
% \centering
% {\LARGE Supplementary Document for ``MetaDE: Evolving Differential Evolution by Differential Evolution"}\\[1.5cm]
% {\large Minyang Chen, Chenchen Feng, and Ran Cheng}\\[1cm]
% {\small
% Minyang Chen was with the Department of Computer Science and Engineering, Southern University of Science and Technology, Shenzhen 518055, China. E-mail: cmy1223605455@gmail.com.\\[0.2cm]
% Chenchen Feng is with the Department of Computer Science and Engineering, Southern University of Science and Technology, Shenzhen 518055, China. E-mail: chenchenfengcn@gmail.com.\\[0.2cm]
% Ran Cheng is with the Department of Data Science and Artificial Intelligence, and the Department of Computing, The Hong Kong Polytechnic University, Hong Kong SAR, China. E-mail: ranchengcn@gmail.com. (Corresponding author: Ran Cheng)
% }\\[2cm]
% \end{titlepage}


\clearpage
\begin{center}
  {\Huge Supplementary Document for ``MetaDE: Evolving \\[0.3em]
  Differential Evolution by Differential Evolution"}\\[2em]
  {\Large Minyang Chen, Chenchen Feng, and Ran Cheng}\\[6em]
\end{center}







\section{Supplementary Experimental data}\label{section:FEs}

\subsection{Supplementary Figures}\label{section:FEs}

\begin{figure*}[htpb]
\centering
\includegraphics[scale=0.3]{su_D10_all.pdf}
\caption{Convergence curves on 10D problems in CEC2022 benchmark suite. The peer DE variants are set with population size of 100.}
\label{Figure_convergence_10D_supp}
\end{figure*}

\vfill 
{\small
\noindent

Minyang Chen was with the Department of Computer Science and Engineering, Southern University of Science and Technology, Shenzhen 518055, China. E-mail: cmy1223605455@gmail.com.

Chenchen Feng is with the Department of Computer Science and Engineering, Southern University of Science and Technology, Shenzhen 518055, China. E-mail: chenchenfengcn@gmail.com.

Ran Cheng is with the Department of Data Science and Artificial Intelligence, and the Department of Computing, The Hong Kong Polytechnic University, Hong Kong SAR, China. E-mail: ranchengcn@gmail.com. \textit{(Corresponding author: Ran Cheng)}
}


\begin{figure*}[htpb]
\centering
\includegraphics[scale=0.3]{su_D20_all.pdf}
\caption{Convergence curves on 20D problems in CEC2022 benchmark suite. The peer DE variants are set with population size of 100.}
\label{Figure_convergence_20D_supp}
\end{figure*}


\begin{figure*}[htpb]
\centering
\includegraphics[scale=0.3]{su_D10NP1000_all.pdf}
\caption{Convergence curves on 10D problems in CEC2022 benchmark suite. The peer DE variants are set with population size of 1,000.}
\label{Figure_convergence_10D_NP10k_supp}
\end{figure*}


\begin{figure*}[htpb]
\centering
\includegraphics[scale=0.3]{su_D20NP1000_all.pdf}
\caption{Convergence curves on 20D problems in CEC2022 benchmark suite. The peer DE variants are set with population size of 1,000.}
\label{Figure_convergence_20D_NP10k_supp}
\end{figure*}

\clearpage

\subsection{Detailed Experimental Results}\label{section:FEs_supp}

% Tables \ref{tab:vsClass10D} and \ref{tab:vsClass20D} present the detailed results of MetaDE compared to other algorithms on 10-dimensional and 20-dimensional problems from CEC2022 within a 60-second time frame. The convergence curves for all problems are shown in Figs. \ref{Figure_convergence_10D} and \ref{Figure_convergence_20D}.

% Tables \ref{tab:NP10000 10D} and \ref{tab:NP10000 20D} display the detailed results for MetaDE versus comparison algorithms with equal concurrency (population size = 10,000) on 10-dimensional and 20-dimensional problems in CEC2022, all within a span of 60 seconds. The convergence curves for all problems are shown in Figs. \ref{Figure_convergence_10D_NP10k} and \ref{Figure_convergence_20D_NP10k}.



% Table generated by Excel2LaTeX from sheet 'Sheet1'
\begin{table}[htbp]
  \centering
  \caption{
  Detailed results on 10D problems in CEC2022 benchmark suite. The peer DE variants are set with population size of 100.
The mean and standard deviation (in parentheses) of the results over multiple runs are displayed in pairs. 
Results with the best mean values are highlighted.
  }
  \resizebox{\textwidth}{!}{
   \renewcommand{\arraystretch}{1.2}
% Table generated by Excel2LaTeX from sheet 'Experiment1 60S'
\begin{tabular}{ccccccccc}
\toprule
Func  & MetaDE & DE    & SaDE  & JaDE  & CoDE  & SHADE & LSHADE-RSP & EDEV \\
\midrule
$F_{1}$ & \textbf{0.00E+00 (0.00E+00)} & \boldmath{}\textbf{0.00E+00 (0.00E+00)$\approx$}\unboldmath{} & \boldmath{}\textbf{0.00E+00 (0.00E+00)$\approx$}\unboldmath{} & \boldmath{}\textbf{0.00E+00 (0.00E+00)$\approx$}\unboldmath{} & \boldmath{}\textbf{0.00E+00 (0.00E+00)$\approx$}\unboldmath{} & \boldmath{}\textbf{0.00E+00 (0.00E+00)$\approx$}\unboldmath{} & \boldmath{}\textbf{0.00E+00 (0.00E+00)$\approx$}\unboldmath{} & \boldmath{}\textbf{0.00E+00 (0.00E+00)$\approx$}\unboldmath{} \\
$F_{2}$ & \textbf{0.00E+00 (0.00E+00)} & 6.05E+00 (2.43E+00)$-$ & 4.86E+00 (4.29E+00)$-$ & 4.89E+00 (3.74E+00)$-$ & 4.71E+00 (2.55E+00)$-$ & 5.47E+00 (3.62E+00)$-$ & 2.38E+00 (2.58E+00)$-$ & 6.11E+00 (2.87E+00)$-$ \\
$F_{3}$ & \textbf{0.00E+00 (0.00E+00)} & \boldmath{}\textbf{0.00E+00 (0.00E+00)$\approx$}\unboldmath{} & \boldmath{}\textbf{0.00E+00 (0.00E+00)$\approx$}\unboldmath{} & \boldmath{}\textbf{0.00E+00 (0.00E+00)$\approx$}\unboldmath{} & \boldmath{}\textbf{0.00E+00 (0.00E+00)$\approx$}\unboldmath{} & \boldmath{}\textbf{0.00E+00 (0.00E+00)$\approx$}\unboldmath{} & \boldmath{}\textbf{0.00E+00 (0.00E+00)$\approx$}\unboldmath{} & \boldmath{}\textbf{0.00E+00 (0.00E+00)$\approx$}\unboldmath{} \\
$F_{4}$ & \textbf{0.00E+00 (0.00E+00)} & 6.90E+00 (4.01E+00)$-$ & 1.03E+00 (8.93E-01)$-$ & 2.31E+01 (1.19E+01)$-$ & 8.34E-01 (7.62E-01)$-$ & 3.05E+00 (9.43E-01)$-$ & 2.12E+00 (6.56E-01)$-$ & 6.52E+00 (4.51E+00)$-$ \\
$F_{5}$ & \textbf{0.00E+00 (0.00E+00)} & \boldmath{}\textbf{0.00E+00 (0.00E+00)$\approx$}\unboldmath{} & \boldmath{}\textbf{0.00E+00 (0.00E+00)$\approx$}\unboldmath{} & \boldmath{}\textbf{0.00E+00 (0.00E+00)$\approx$}\unboldmath{} & \boldmath{}\textbf{0.00E+00 (0.00E+00)$\approx$}\unboldmath{} & \boldmath{}\textbf{0.00E+00 (0.00E+00)$\approx$}\unboldmath{} & \boldmath{}\textbf{0.00E+00 (0.00E+00)$\approx$}\unboldmath{} & \boldmath{}\textbf{0.00E+00 (0.00E+00)$\approx$}\unboldmath{} \\
$F_{6}$ & \textbf{5.50E-04 (3.96E-04)} & 1.11E-01 (8.92E-02)$-$ & 6.04E+01 (2.25E+02)$-$ & 1.60E+00 (2.48E+00)$-$ & 9.08E-03 (1.17E-02)$-$ & 1.33E+00 (2.22E+00)$-$ & 3.84E-02 (5.60E-02)$-$ & 9.06E-01 (1.76E+00)$-$ \\
$F_{7}$ & \textbf{0.00E+00 (0.00E+00)} & 5.18E-02 (1.59E-01)$-$ & 2.15E-02 (1.39E-02)$-$ & \boldmath{}\textbf{0.00E+00 (0.00E+00)$\approx$}\unboldmath{} & 1.92E-03 (7.32E-03)$-$ & 6.59E-03 (1.41E-02)$-$ & 1.06E+02 (1.42E-02)$-$ & 9.54E+00 (9.86E+00)$-$ \\
$F_{8}$ & \textbf{5.52E-03 (4.41E-03)} & 1.42E-01 (2.45E-01)$-$ & 5.15E-02 (2.29E-02)$-$ & 1.74E+01 (4.79E+00)$-$ & 6.02E-03 (1.02E-02)$-$ & 2.29E+00 (5.89E+00)$-$ & 2.19E+00 (5.89E+00)$-$ & 7.06E+00 (9.36E+00)$-$ \\
$F_{9}$ & \textbf{3.36E+00 (1.77E+01)} & 2.29E+02 (7.53E-06)$-$ & 2.29E+02 (6.38E-06)$-$ & 2.29E+02 (6.38E-06)$-$ & 2.29E+02 (7.17E-06)$-$ & 2.29E+02 (7.43E-06)$-$ & 2.29E+02 (8.19E-05)$-$ & 2.29E+02 (1.01E-05)$-$ \\
$F_{10}$ & \textbf{0.00E+00 (0.00E+00)} & 1.00E+02 (5.18E-02)$-$ & 1.03E+02 (1.83E+01)$-$ & 1.04E+02 (1.91E+01)$-$ & 1.00E+02 (6.83E-02)$-$ & 1.10E+02 (3.09E+01)$-$ & 1.03E+02 (1.77E+01)$-$ & 1.07E+02 (2.62E+01)$-$ \\
$F_{11}$ & \textbf{0.00E+00 (0.00E+00)} & \boldmath{}\textbf{0.00E+00 (0.00E+00)$\approx$}\unboldmath{} & 2.42E+01 (5.52E+01)$-$ & \boldmath{}\textbf{0.00E+00 (0.00E+00)$\approx$}\unboldmath{} & \boldmath{}\textbf{0.00E+00 (0.00E+00)$\approx$}\unboldmath{} & \boldmath{}\textbf{0.00E+00 (0.00E+00)$\approx$}\unboldmath{} & \boldmath{}\textbf{0.00E+00 (0.00E+00)$\approx$}\unboldmath{} & 4.84E+00 (2.65E+01)$-$ \\
$F_{12}$ & \textbf{1.39E+02 (4.63E+01)} & 1.62E+02 (1.04E+00)$-$ & 1.63E+02 (1.57E+00)$-$ & 1.62E+02 (2.22E+00)$-$ & 1.59E+02 (1.14E+00)$-$ & 1.63E+02 (1.25E+00)$-$ & 1.64E+02 (1.38E+00)$-$ & 1.62E+02 (1.71E+00)$-$ \\
\midrule
$+$ / $\approx$ / $-$ & --    & 0/4/8 & 0/3/9 & 0/5/7 & 0/5/7 & 0/4/8 & 0/4/8 & 0/3/9 \\
\bottomrule
\end{tabular}%
}
\footnotesize
\textsuperscript{*} The Wilcoxon rank-sum tests (with a significance level of 0.05) were conducted between MetaDE and each individually.
The final row displays the number of problems where the corresponding algorithm performs statistically better ($+$),  similar ($\thickapprox$), or worse ($-$) compared to MetaDE.\\
\label{tab:vsClass10D_supp}%

\end{table}%


% Table generated by Excel2LaTeX from sheet 'Sheet1'
\begin{table}[htbp]
  \centering
  \caption{Detailed results on 20D problems in CEC2022 benchmark suite. The peer DE variants are set with population size of 100.
  The mean and standard deviation (in parentheses) of the results over multiple runs are displayed in pairs. 
Results with the best mean values are highlighted.
  }
  {
  \resizebox{\textwidth}{!}{
   \renewcommand{\arraystretch}{1.2}
% Table generated by Excel2LaTeX from sheet 'Experiment1 60S'
\begin{tabular}{ccccccccc}
\toprule
Func  & MetaDE & DE    & SaDE  & JaDE  & CoDE  & SHADE & LSHADE-RSP & EDEV \\
\midrule
$F_{1}$ & \multicolumn{1}{l}{\textbf{0.00E+00 (0.00E+00)}} & \multicolumn{1}{l}{\boldmath{}\textbf{0.00E+00 (0.00E+00)$\approx$}\unboldmath{}} & \multicolumn{1}{l}{\boldmath{}\textbf{0.00E+00 (0.00E+00)$\approx$}\unboldmath{}} & \multicolumn{1}{l}{\boldmath{}\textbf{0.00E+00 (0.00E+00)$\approx$}\unboldmath{}} & \multicolumn{1}{l}{\boldmath{}\textbf{0.00E+00 (0.00E+00)$\approx$}\unboldmath{}} & \multicolumn{1}{l}{\boldmath{}\textbf{0.00E+00 (0.00E+00)$\approx$}\unboldmath{}} & \multicolumn{1}{l}{\boldmath{}\textbf{0.00E+00 (0.00E+00)$\approx$}\unboldmath{}} & \multicolumn{1}{l}{\boldmath{}\textbf{0.00E+00 (0.00E+00)$\approx$}\unboldmath{}} \\
$F_{2}$ & \multicolumn{1}{l}{\textbf{1.26E-02 (3.74E-02)}} & \multicolumn{1}{l}{4.69E+01 (2.09E+00)$-$} & \multicolumn{1}{l}{3.50E+01 (2.21E+01)$-$} & \multicolumn{1}{l}{4.75E+01 (8.67E+00)$-$} & \multicolumn{1}{l}{4.58E+01 (1.20E+01)$-$} & \multicolumn{1}{l}{4.75E+01 (8.67E+00)$-$} & \multicolumn{1}{l}{4.30E+01 (1.71E+01)$-$} & \multicolumn{1}{l}{4.13E+01 (1.78E+01)$-$} \\
$F_{3}$ & \multicolumn{1}{l}{\textbf{0.00E+00 (0.00E+00)}} & \multicolumn{1}{l}{\boldmath{}\textbf{0.00E+00 (0.00E+00)$\approx$}\unboldmath{}} & \multicolumn{1}{l}{\boldmath{}\textbf{0.00E+00 (0.00E+00)$\approx$}\unboldmath{}} & \multicolumn{1}{l}{\boldmath{}\textbf{0.00E+00 (0.00E+00)$\approx$}\unboldmath{}} & \multicolumn{1}{l}{\boldmath{}\textbf{0.00E+00 (0.00E+00)$\approx$}\unboldmath{}} & \multicolumn{1}{l}{1.03E-08 (5.64E-08)$\approx$} & \multicolumn{1}{l}{\boldmath{}\textbf{0.00E+00 (0.00E+00)$\approx$}\unboldmath{}} & \multicolumn{1}{l}{6.91E-05 (3.31E-04)$-$} \\
$F_{4}$ & \multicolumn{1}{l}{\textbf{2.02E+00 (8.56E-01)}} & \multicolumn{1}{l}{1.95E+01 (8.17E+00)$-$} & \multicolumn{1}{l}{7.42E+00 (2.14E+00)$-$} & \multicolumn{1}{l}{7.21E+01 (3.45E+01)$-$} & \multicolumn{1}{l}{1.11E+01 (2.22E+00)$-$} & \multicolumn{1}{l}{1.27E+01 (2.73E+00)$-$} & \multicolumn{1}{l}{9.05E+00 (1.44E+00)$-$} & \multicolumn{1}{l}{2.37E+01 (1.37E+01)$-$} \\
$F_{5}$ & \multicolumn{1}{l}{\textbf{0.00E+00 (0.00E+00)}} & \multicolumn{1}{l}{\boldmath{}\textbf{0.00E+00 (0.00E+00)$\approx$}\unboldmath{}} & \multicolumn{1}{l}{7.24E-01 (1.22E+00)$-$} & \multicolumn{1}{l}{\boldmath{}\textbf{0.00E+00 (0.00E+00)$\approx$}\unboldmath{}} & \multicolumn{1}{l}{\boldmath{}\textbf{0.00E+00 (0.00E+00)$\approx$}\unboldmath{}} & \multicolumn{1}{l}{\boldmath{}\textbf{0.00E+00 (0.00E+00)$\approx$}\unboldmath{}} & \multicolumn{1}{l}{\boldmath{}\textbf{0.00E+00 (0.00E+00)$\approx$}\unboldmath{}} & \multicolumn{1}{l}{1.78E-01 (3.12E-01)$-$} \\
$F_{6}$ & \multicolumn{1}{l}{\textbf{1.16E-01 (2.79E-02)}} & \multicolumn{1}{l}{4.99E-01 (4.11E-01)$-$} & \multicolumn{1}{l}{3.13E+01 (1.54E+01)$-$} & \multicolumn{1}{l}{5.34E+01 (3.33E+01)$-$} & \multicolumn{1}{l}{1.89E+01 (1.83E+01)$-$} & \multicolumn{1}{l}{5.06E+01 (3.16E+01)$-$} & \multicolumn{1}{l}{1.25E+01 (1.00E+01)$-$} & \multicolumn{1}{l}{4.91E+03 (6.53E+03)$-$} \\
$F_{7}$ & \multicolumn{1}{l}{\textbf{5.17E-02 (6.21E-02)}} & \multicolumn{1}{l}{4.23E+00 (7.90E+00)$-$} & \multicolumn{1}{l}{1.07E+01 (5.20E+00)$-$} & \multicolumn{1}{l}{2.98E+00 (3.61E+00)$-$} & \multicolumn{1}{l}{1.16E+00 (1.26E+00)$-$} & \multicolumn{1}{l}{7.77E+00 (6.65E+00)$-$} & \multicolumn{1}{l}{1.42E+01 (8.96E+00)$-$} & \multicolumn{1}{l}{2.29E+01 (8.80E+00)$-$} \\
$F_{8}$ & \multicolumn{1}{l}{\textbf{7.19E-01 (1.02E+00)}} & \multicolumn{1}{l}{8.24E+00 (1.00E+01)$-$} & \multicolumn{1}{l}{2.10E+01 (7.26E-01)$-$} & \multicolumn{1}{l}{2.64E+01 (9.73E-01)$-$} & \multicolumn{1}{l}{1.38E+01 (8.98E+00)$-$} & \multicolumn{1}{l}{2.02E+01 (8.29E-01)$-$} & \multicolumn{1}{l}{1.96E+01 (3.80E+00)$-$} & \multicolumn{1}{l}{2.08E+01 (3.81E-01)$-$} \\
$F_{9}$ & \multicolumn{1}{l}{\textbf{1.07E+02 (1.98E+01)}} & \multicolumn{1}{l}{1.81E+02 (9.39E-06)$-$} & \multicolumn{1}{l}{1.81E+02 (3.75E-06)$-$} & \multicolumn{1}{l}{1.81E+02 (9.02E-06)$-$} & \multicolumn{1}{l}{1.81E+02 (8.41E-06)$-$} & \multicolumn{1}{l}{1.81E+02 (1.04E-05)$-$} & \multicolumn{1}{l}{1.81E+02 (2.18E-05)$-$} & \multicolumn{1}{l}{1.81E+02 (5.43E-04)$-$} \\
$F_{10}$ & \multicolumn{1}{l}{\textbf{0.00E+00 (0.00E+00)}} & \multicolumn{1}{l}{1.13E+02 (3.52E+01)$-$} & \multicolumn{1}{l}{1.00E+02 (3.03E-02)$-$} & \multicolumn{1}{l}{1.13E+02 (3.76E+01)$-$} & \multicolumn{1}{l}{1.00E+02 (3.55E-02)$-$} & \multicolumn{1}{l}{1.12E+02 (3.47E+01)$-$} & \multicolumn{1}{l}{1.11E+02 (3.42E+01)$-$} & \multicolumn{1}{l}{1.07E+02 (3.80E+01)$-$} \\
$F_{11}$ & \multicolumn{1}{l}{\textbf{7.28E-05 (3.06E-04)}} & \multicolumn{1}{l}{3.39E+02 (4.87E+01)$-$} & \multicolumn{1}{l}{3.06E+02 (2.46E+01)$-$} & \multicolumn{1}{l}{3.19E+02 (3.95E+01)$-$} & \multicolumn{1}{l}{3.39E+02 (4.87E+01)$-$} & \multicolumn{1}{l}{3.16E+02 (3.68E+01)$-$} & \multicolumn{1}{l}{3.39E+02 (4.87E+01)$-$} & \multicolumn{1}{l}{3.19E+02 (3.95E+01)$-$} \\
$F_{12}$ & \multicolumn{1}{l}{\textbf{2.29E+02 (6.08E-01)}} & \multicolumn{1}{l}{2.37E+02 (3.11E+00)$-$} & \multicolumn{1}{l}{2.41E+02 (5.26E+00)$-$} & \multicolumn{1}{l}{2.37E+02 (5.10E+00)$-$} & \multicolumn{1}{l}{2.34E+02 (2.68E+00)$-$} & \multicolumn{1}{l}{2.39E+02 (4.46E+00)$-$} & \multicolumn{1}{l}{2.44E+02 (1.63E+01)$-$} & \multicolumn{1}{l}{2.42E+02 (8.38E+00)$-$} \\
\midrule
$+$ / $\approx$ / $-$ & --    & 0/3/9 & 0/2/10 & 0/3/9 & 0/3/9 & 0/3/9 & 0/3/9 & 0/1/11 \\
\bottomrule
\end{tabular}%
    }
\footnotesize
\textsuperscript{*} The Wilcoxon rank-sum tests (with a significance level of 0.05) were conducted between MetaDE and each individually.
The final row displays the number of problems where the corresponding algorithm performs statistically better ($+$),  similar ($\thickapprox$), or worse ($-$) compared to MetaDE.\\
\label{tab:vsClass20D_supp}%
}
\end{table}%

\clearpage

% Table generated by Excel2LaTeX from sheet 'Sheet1'
\begin{table}[htbp]
  \centering
  \caption{Detailed results on 10D problems in CEC2022 benchmark suite. The peer DE variants are set with population size of 1,000. 
The mean and standard deviation (in parentheses) of the results over multiple runs are displayed in pairs. 
Results with the best mean values are highlighted.
  }
  {
  \resizebox{\textwidth}{!}{
   \renewcommand{\arraystretch}{1.2}
% Table generated by Excel2LaTeX from sheet 'Exp2 NP1000'
\begin{tabular}{ccccccccc}
\toprule
Func  & MetaDE & DE    & SaDE  & JaDE  & CoDE  & SHADE & LSHADE-RSP & EDEV \\
\midrule
$F_{1}$ & \multicolumn{1}{l}{\textbf{0.00E+00 (0.00E+00)}} & \multicolumn{1}{l}{\boldmath{}\textbf{0.00E+00 (0.00E+00)$\approx$}\unboldmath{}} & \multicolumn{1}{l}{\boldmath{}\textbf{0.00E+00 (0.00E+00)$\approx$}\unboldmath{}} & \multicolumn{1}{l}{\boldmath{}\textbf{0.00E+00 (0.00E+00)$\approx$}\unboldmath{}} & \multicolumn{1}{l}{\boldmath{}\textbf{0.00E+00 (0.00E+00)$\approx$}\unboldmath{}} & \multicolumn{1}{l}{\boldmath{}\textbf{0.00E+00 (0.00E+00)$\approx$}\unboldmath{}} & \multicolumn{1}{l}{\boldmath{}\textbf{0.00E+00 (0.00E+00)$\approx$}\unboldmath{}} & \multicolumn{1}{l}{\boldmath{}\textbf{0.00E+00 (0.00E+00)$\approx$}\unboldmath{}} \\
$F_{2}$ & \multicolumn{1}{l}{\textbf{0.00E+00 (0.00E+00)}} & \multicolumn{1}{l}{3.60E+00 (1.20E+00)$-$} & \multicolumn{1}{l}{6.87E+00 (3.63E+00)$-$} & \multicolumn{1}{l}{8.15E+00 (2.11E+00)$-$} & \multicolumn{1}{l}{2.44E+00 (1.97E+00)$-$} & \multicolumn{1}{l}{8.02E+00 (2.47E+00)$-$} & \multicolumn{1}{l}{1.12E+00 (1.79E+00)$-$} & \multicolumn{1}{l}{6.27E+00 (2.93E+00)$-$} \\
$F_{3}$ & \multicolumn{1}{l}{\textbf{0.00E+00 (0.00E+00)}} & \multicolumn{1}{l}{\boldmath{}\textbf{0.00E+00 (0.00E+00)$\approx$}\unboldmath{}} & \multicolumn{1}{l}{\boldmath{}\textbf{0.00E+00 (0.00E+00)$\approx$}\unboldmath{}} & \multicolumn{1}{l}{\boldmath{}\textbf{0.00E+00 (0.00E+00)$\approx$}\unboldmath{}} & \multicolumn{1}{l}{\boldmath{}\textbf{0.00E+00 (0.00E+00)$\approx$}\unboldmath{}} & \multicolumn{1}{l}{\boldmath{}\textbf{0.00E+00 (0.00E+00)$\approx$}\unboldmath{}} & \multicolumn{1}{l}{\boldmath{}\textbf{0.00E+00 (0.00E+00)$\approx$}\unboldmath{}} & \multicolumn{1}{l}{\boldmath{}\textbf{0.00E+00 (0.00E+00)$\approx$}\unboldmath{}} \\
$F_{4}$ & \multicolumn{1}{l}{\textbf{0.00E+00 (0.00E+00)}} & \multicolumn{1}{l}{1.50E+01 (2.59E+00)$-$} & \multicolumn{1}{l}{4.75E-01 (5.04E-01)$-$} & \multicolumn{1}{l}{2.20E+00 (5.30E-01)$-$} & \multicolumn{1}{l}{\boldmath{}\textbf{0.00E+00 (0.00E+00)$\approx$}\unboldmath{}} & \multicolumn{1}{l}{\boldmath{}\textbf{0.00E+00 (0.00E+00)$\approx$}\unboldmath{}} & \multicolumn{1}{l}{9.63E-01 (6.54E-01)$-$} & \multicolumn{1}{l}{4.91E+00 (5.24E+00)$-$} \\
$F_{5}$ & \multicolumn{1}{l}{\textbf{0.00E+00 (0.00E+00)}} & \multicolumn{1}{l}{\boldmath{}\textbf{0.00E+00 (0.00E+00)$\approx$}\unboldmath{}} & \multicolumn{1}{l}{\boldmath{}\textbf{0.00E+00 (0.00E+00)$\approx$}\unboldmath{}} & \multicolumn{1}{l}{\boldmath{}\textbf{0.00E+00 (0.00E+00)$\approx$}\unboldmath{}} & \multicolumn{1}{l}{\boldmath{}\textbf{0.00E+00 (0.00E+00)$\approx$}\unboldmath{}} & \multicolumn{1}{l}{\boldmath{}\textbf{0.00E+00 (0.00E+00)$\approx$}\unboldmath{}} & \multicolumn{1}{l}{\boldmath{}\textbf{0.00E+00 (0.00E+00)$\approx$}\unboldmath{}} & \multicolumn{1}{l}{\boldmath{}\textbf{0.00E+00 (0.00E+00)$\approx$}\unboldmath{}} \\
$F_{6}$ & \multicolumn{1}{l}{\textbf{5.50E-04 (3.96E-04)}} & \multicolumn{1}{l}{1.49E-02 (4.74E-03)$-$} & \multicolumn{1}{l}{1.95E+00 (1.54E+00)$-$} & \multicolumn{1}{l}{1.01E-01 (6.20E-02)$-$} & \multicolumn{1}{l}{7.69E-04 (4.31E-04)$\approx$} & \multicolumn{1}{l}{4.62E-03 (8.17E-03)$-$} & \multicolumn{1}{l}{1.89E-03 (7.18E-04)$-$} & \multicolumn{1}{l}{4.37E-02 (6.48E-02)$-$} \\
$F_{7}$ & \multicolumn{1}{l}{\textbf{0.00E+00 (0.00E+00)}} & \multicolumn{1}{l}{9.60E-04 (5.35E-03)$-$} & \multicolumn{1}{l}{1.98E-02 (8.61E-03)$-$} & \multicolumn{1}{l}{\boldmath{}\textbf{0.00E+00 (0.00E+00)$\approx$}\unboldmath{}} & \multicolumn{1}{l}{\boldmath{}\textbf{0.00E+00 (0.00E+00)$\approx$}\unboldmath{}} & \multicolumn{1}{l}{\boldmath{}\textbf{0.00E+00 (0.00E+00)$\approx$}\unboldmath{}} & \multicolumn{1}{l}{\boldmath{}\textbf{0.00E+00 (0.00E+00)$\approx$}\unboldmath{}} & \multicolumn{1}{l}{1.35E+00 (4.98E+00)$-$} \\
$F_{8}$ & \multicolumn{1}{l}{5.52E-03 (4.41E-03)} & \multicolumn{1}{l}{1.19E-01 (2.50E-02)$-$} & \multicolumn{1}{l}{1.13E+00 (4.44E-01)$-$} & \multicolumn{1}{l}{1.03E-01 (2.62E-02)$-$} & \multicolumn{1}{l}{\boldmath{}\textbf{0.00E+00 (0.00E+00)$\approx$}\unboldmath{}} & \multicolumn{1}{l}{8.28E-02 (2.66E-02)$-$} & \multicolumn{1}{l}{9.13E-02 (1.11E-01)$-$} & \multicolumn{1}{l}{1.03E+00 (3.58E+00)$-$} \\
$F_{9}$ & \multicolumn{1}{l}{\textbf{3.36E+00 (1.77E+01)}} & \multicolumn{1}{l}{2.29E+02 (7.85E-06)$-$} & \multicolumn{1}{l}{2.29E+02 (8.58E-06)$-$} & \multicolumn{1}{l}{2.29E+02 (8.28E-06)$-$} & \multicolumn{1}{l}{2.29E+02 (8.67E-14)$-$} & \multicolumn{1}{l}{2.29E+02 (2.74E-06)$-$} & \multicolumn{1}{l}{2.29E+02 (9.39E-06)$-$} & \multicolumn{1}{l}{2.29E+02 (1.17E-05)$-$} \\
$F_{10}$ & \multicolumn{1}{l}{\textbf{0.00E+00 (0.00E+00)}} & \multicolumn{1}{l}{1.00E+02 (1.70E-02)$-$} & \multicolumn{1}{l}{1.00E+02 (3.93E-02)$-$} & \multicolumn{1}{l}{1.00E+02 (2.49E-02)$-$} & \multicolumn{1}{l}{1.00E+02 (1.01E-02)$-$} & \multicolumn{1}{l}{1.00E+02 (1.83E-02)$-$} & \multicolumn{1}{l}{1.00E+02 (8.05E-04)$-$} & \multicolumn{1}{l}{1.00E+02 (3.93E-02)$-$} \\
$F_{11}$ & \multicolumn{1}{l}{\textbf{0.00E+00 (0.00E+00)}} & \multicolumn{1}{l}{\boldmath{}\textbf{0.00E+00 (0.00E+00)$\approx$}\unboldmath{}} & \multicolumn{1}{l}{\boldmath{}\textbf{0.00E+00 (0.00E+00)$\approx$}\unboldmath{}} & \multicolumn{1}{l}{\boldmath{}\textbf{0.00E+00 (0.00E+00)$\approx$}\unboldmath{}} & \multicolumn{1}{l}{3.65E-07 (2.03E-06)$-$} & \multicolumn{1}{l}{\boldmath{}\textbf{0.00E+00 (0.00E+00)$\approx$}\unboldmath{}} & \multicolumn{1}{l}{\boldmath{}\textbf{0.00E+00 (0.00E+00)$\approx$}\unboldmath{}} & \multicolumn{1}{l}{\boldmath{}\textbf{0.00E+00 (0.00E+00)$\approx$}\unboldmath{}} \\
$F_{12}$ & \multicolumn{1}{l}{\textbf{1.39E+02 (4.63E+01)}} & \multicolumn{1}{l}{1.60E+02 (9.79E-01)$-$} & \multicolumn{1}{l}{1.60E+02 (1.56E+00)$-$} & \multicolumn{1}{l}{1.59E+02 (1.28E+00)$-$} & \multicolumn{1}{l}{1.59E+02 (8.67E-14)$-$} & \multicolumn{1}{l}{1.61E+02 (1.69E+00)$-$} & \multicolumn{1}{l}{1.63E+02 (7.62E-01)$-$} & \multicolumn{1}{l}{1.60E+02 (1.18E+00)$-$} \\
\midrule
$+$ / $\approx$ / $-$ & --    & 0/4/8 & 0/4/8 & 0/5/7 & 0/7/5 & 0/6/6 & 0/5/7 & 0/4/8 \\
\bottomrule
\end{tabular}%
}
\footnotesize
\textsuperscript{*} The Wilcoxon rank-sum tests (with a significance level of 0.05) were conducted between MetaDE and each individually.
The final row displays the number of problems where the corresponding algorithm performs statistically better ($+$),  similar ($\thickapprox$), or worse ($-$) compared to MetaDE.\\
\label{tab:NP10000 10D_supp}%
}
\end{table}%


% Table generated by Excel2LaTeX from sheet 'Sheet1'
\begin{table}[htbp]
  \centering
  \caption{Detailed results on 20D problems in CEC2022 benchmark suite. The peer DE variants are set with population size of 1,000. 
The mean and standard deviation (in parentheses) of the results over multiple runs are displayed in pairs. 
Results with the best mean values are highlighted.
  }
  {
  \resizebox{\textwidth}{!}{
   \renewcommand{\arraystretch}{1.2}
% Table generated by Excel2LaTeX from sheet 'Exp2 NP1000'
\begin{tabular}{ccccccccc}
\toprule
Func  & MetaDE & DE    & SaDE  & JaDE  & CoDE  & SHADE & LSHADE-RSP & EDEV \\
\midrule
$F_{1}$ & \multicolumn{1}{l}{\textbf{0.00E+00 (0.00E+00)}} & \multicolumn{1}{l}{\boldmath{}\textbf{0.00E+00 (0.00E+00)$\approx$}\unboldmath{}} & \multicolumn{1}{l}{\boldmath{}\textbf{0.00E+00 (0.00E+00)$\approx$}\unboldmath{}} & \multicolumn{1}{l}{\boldmath{}\textbf{0.00E+00 (0.00E+00)$\approx$}\unboldmath{}} & \multicolumn{1}{l}{\boldmath{}\textbf{0.00E+00 (0.00E+00)$\approx$}\unboldmath{}} & \multicolumn{1}{l}{\boldmath{}\textbf{0.00E+00 (0.00E+00)$\approx$}\unboldmath{}} & \multicolumn{1}{l}{\boldmath{}\textbf{0.00E+00 (0.00E+00)$\approx$}\unboldmath{}} & \multicolumn{1}{l}{\boldmath{}\textbf{0.00E+00 (0.00E+00)$\approx$}\unboldmath{}} \\
$F_{2}$ & \multicolumn{1}{l}{\textbf{1.26E-02 (3.74E-02)}} & \multicolumn{1}{l}{4.49E+01 (0.00E+00)$-$} & \multicolumn{1}{l}{4.91E+01 (7.67E-06)$-$} & \multicolumn{1}{l}{4.91E+01 (0.00E+00)$-$} & \multicolumn{1}{l}{4.91E+01 (0.00E+00)$-$} & \multicolumn{1}{l}{4.91E+01 (0.00E+00)$-$} & \multicolumn{1}{l}{4.52E+01 (1.05E+00)$-$} & \multicolumn{1}{l}{4.89E+01 (1.05E+00)$-$} \\
$F_{3}$ & \multicolumn{1}{l}{\textbf{0.00E+00 (0.00E+00)}} & \multicolumn{1}{l}{\boldmath{}\textbf{0.00E+00 (0.00E+00)$\approx$}\unboldmath{}} & \multicolumn{1}{l}{\boldmath{}\textbf{0.00E+00 (0.00E+00)$\approx$}\unboldmath{}} & \multicolumn{1}{l}{\boldmath{}\textbf{0.00E+00 (0.00E+00)$\approx$}\unboldmath{}} & \multicolumn{1}{l}{\boldmath{}\textbf{0.00E+00 (0.00E+00)$\approx$}\unboldmath{}} & \multicolumn{1}{l}{\boldmath{}\textbf{0.00E+00 (0.00E+00)$\approx$}\unboldmath{}} & \multicolumn{1}{l}{\boldmath{}\textbf{0.00E+00 (0.00E+00)$\approx$}\unboldmath{}} & \multicolumn{1}{l}{\boldmath{}\textbf{0.00E+00 (0.00E+00)$\approx$}\unboldmath{}} \\
$F_{4}$ & \multicolumn{1}{l}{\textbf{2.02E+00 (8.56E-01)}} & \multicolumn{1}{l}{8.33E+01 (5.85E+00)$-$} & \multicolumn{1}{l}{6.09E+00 (1.02E+00)$-$} & \multicolumn{1}{l}{1.07E+01 (1.95E+00)$-$} & \multicolumn{1}{l}{8.02E+00 (1.08E+00)$-$} & \multicolumn{1}{l}{7.39E+00 (1.17E+00)$-$} & \multicolumn{1}{l}{4.82E+00 (8.64E-01)$-$} & \multicolumn{1}{l}{1.81E+01 (1.52E+01)$-$} \\
$F_{5}$ & \multicolumn{1}{l}{\textbf{0.00E+00 (0.00E+00)}} & \multicolumn{1}{l}{\boldmath{}\textbf{0.00E+00 (0.00E+00)$\approx$}\unboldmath{}} & \multicolumn{1}{l}{6.16E-02 (2.71E-01)$-$} & \multicolumn{1}{l}{\boldmath{}\textbf{0.00E+00 (0.00E+00)$\approx$}\unboldmath{}} & \multicolumn{1}{l}{\boldmath{}\textbf{0.00E+00 (0.00E+00)$\approx$}\unboldmath{}} & \multicolumn{1}{l}{\boldmath{}\textbf{0.00E+00 (0.00E+00)$\approx$}\unboldmath{}} & \multicolumn{1}{l}{\boldmath{}\textbf{0.00E+00 (0.00E+00)$\approx$}\unboldmath{}} & \multicolumn{1}{l}{\boldmath{}\textbf{0.00E+00 (0.00E+00)$\approx$}\unboldmath{}} \\
$F_{6}$ & \multicolumn{1}{l}{1.16E-01 (2.79E-02)} & \multicolumn{1}{l}{8.86E+00 (1.34E+00)$-$} & \multicolumn{1}{l}{1.72E+02 (5.24E+02)$-$} & \multicolumn{1}{l}{1.80E+00 (7.49E-01)$-$} & \multicolumn{1}{l}{2.07E-01 (5.04E-02)$-$} & \multicolumn{1}{l}{1.35E-01 (5.23E-02)$\approx$} & \multicolumn{1}{l}{\textbf{7.35E-02 (3.33E-02)$+$}} & \multicolumn{1}{l}{2.70E+00 (1.13E+01)$-$} \\
$F_{7}$ & \multicolumn{1}{l}{5.17E-02 (6.21E-02)} & \multicolumn{1}{l}{3.28E+01 (1.82E+00)$-$} & \multicolumn{1}{l}{1.57E+01 (3.45E+00)$-$} & \multicolumn{1}{l}{1.31E+01 (2.39E+00)$-$} & \multicolumn{1}{l}{\textbf{2.02E-03 (1.44E-03)$+$}} & \multicolumn{1}{l}{9.48E+00 (1.76E+00)$-$} & \multicolumn{1}{l}{2.92E+00 (9.57E-01)$-$} & \multicolumn{1}{l}{1.68E+01 (1.04E+01)$-$} \\
$F_{8}$ & \multicolumn{1}{l}{\textbf{7.19E-01 (1.02E+00)}} & \multicolumn{1}{l}{2.41E+01 (2.25E+00)$-$} & \multicolumn{1}{l}{2.11E+01 (1.76E+00)$-$} & \multicolumn{1}{l}{2.15E+01 (1.27E+00)$-$} & \multicolumn{1}{l}{1.12E+01 (1.98E+00)$-$} & \multicolumn{1}{l}{1.99E+01 (2.81E+00)$-$} & \multicolumn{1}{l}{8.53E+00 (4.47E+00)$-$} & \multicolumn{1}{l}{1.88E+01 (5.20E+00)$-$} \\
$F_{9}$ & \multicolumn{1}{l}{\textbf{1.07E+02 (1.98E+01)}} & \multicolumn{1}{l}{1.81E+02 (2.68E-05)$-$} & \multicolumn{1}{l}{1.81E+02 (2.75E-05)$-$} & \multicolumn{1}{l}{1.81E+02 (1.08E-05)$-$} & \multicolumn{1}{l}{1.81E+02 (7.25E-06)$-$} & \multicolumn{1}{l}{1.81E+02 (9.14E-06)$-$} & \multicolumn{1}{l}{1.81E+02 (1.24E-05)$-$} & \multicolumn{1}{l}{1.81E+02 (1.01E-04)$-$} \\
$F_{10}$ & \multicolumn{1}{l}{\textbf{0.00E+00 (0.00E+00)}} & \multicolumn{1}{l}{1.00E+02 (3.21E-02)$-$} & \multicolumn{1}{l}{1.00E+02 (2.58E-02)$-$} & \multicolumn{1}{l}{1.00E+02 (3.51E-02)$-$} & \multicolumn{1}{l}{1.00E+02 (1.80E-02)$-$} & \multicolumn{1}{l}{1.00E+02 (1.80E-02)$-$} & \multicolumn{1}{l}{1.00E+02 (1.21E-02)$-$} & \multicolumn{1}{l}{1.00E+02 (4.13E-02)$-$} \\
$F_{11}$ & \multicolumn{1}{l}{\textbf{7.28E-05 (3.06E-04)}} & \multicolumn{1}{l}{3.97E+02 (1.80E+01)$-$} & \multicolumn{1}{l}{3.26E+02 (4.45E+01)$-$} & \multicolumn{1}{l}{3.16E+02 (3.74E+01)$-$} & \multicolumn{1}{l}{3.81E+02 (4.02E+01)$-$} & \multicolumn{1}{l}{3.16E+02 (3.74E+01)$-$} & \multicolumn{1}{l}{3.77E+02 (4.25E+01)$-$} & \multicolumn{1}{l}{3.19E+02 (9.80E+01)$-$} \\
$F_{12}$ & \multicolumn{1}{l}{\textbf{2.29E+02 (6.08E-01)}} & \multicolumn{1}{l}{2.35E+02 (2.45E+00)$-$} & \multicolumn{1}{l}{2.34E+02 (2.55E+00)$-$} & \multicolumn{1}{l}{2.30E+02 (1.99E+00)$-$} & \multicolumn{1}{l}{2.30E+02 (1.40E+00)$-$} & \multicolumn{1}{l}{2.32E+02 (1.09E+00)$-$} & \multicolumn{1}{l}{2.33E+02 (2.27E+00)$-$} & \multicolumn{1}{l}{2.38E+02 (3.81E+00)$-$} \\
\midrule
$+$ / $\approx$ / $-$ & --    & 0/3/9 & 0/2/10 & 0/3/9 & 1/3/8 & 0/4/8 & 1/3/8 & 0/3/9 \\
\bottomrule
\end{tabular}%
}
\footnotesize
\textsuperscript{*} The Wilcoxon rank-sum tests (with a significance level of 0.05) were conducted between MetaDE and each individually.
The final row displays the number of problems where the corresponding algorithm performs statistically better ($+$),  similar ($\thickapprox$), or worse ($-$) compared to MetaDE.\\
\label{tab:NP10000 20D_supp}%
}
\end{table}%


% \subsection{Supplementary results}\label{section:FEs}
 % This experiment leveraged parallel GPU computation and constrained the runtime: 30 seconds for the 10-dimensional problems and 60 seconds for the 20-dimensional problems. 
 
% We recorded the number of FEs each algorithm achieved within 60s, which are presented in Table \ref{tab:FEs}. 
 
%  In addition, when the population size of the comparative DE variants was increased to 10,000 (same level of concurrency as MetaDE), the FEs achieved by all algorithms are displayed in Table \ref{tab:FEs NP10000}.

% The results show that MetaDE achieved considerably more FEs within a given time than the other algorithms. This demonstrates that MetaDE has a high degree of parallelism, making it particularly well-suited for GPU computing.



% Table generated by Excel2LaTeX from sheet 'Sheet1'
\begin{table}[htbp]
  \centering
  \caption{The number of FEs achieved by each algorithm within \SI{60}{\second}. The peer DE variants are set with population size of 1,000.}
  {
           \renewcommand{\arraystretch}{1}
 \renewcommand{\tabcolsep}{10pt}
% Table generated by Excel2LaTeX from sheet 'Exp2 NP10000'
\begin{tabular}{cccccccccc}
\toprule
Dim   & Func  & MetaDE & DE    & SaDE  & JaDE  & CoDE  & SHADE &LSHADE-RSP&EDEV\\
\midrule
\multirow{12}[2]{*}{10D} & $F_{1}$ & \textbf{1.85E+09} & 3.91E+07 & 4.38E+06 & 6.26E+06 & 7.67E+07 & 4.62E+06&1.92E+07&1.91E+07 \\
      & $F_{2}$ & \textbf{1.84E+09} & 4.05E+07 &4.35E+06 & 6.20E+06 & 7.61E+07 & 4.63E+06&1.89E+07&1.96E+07 \\
      & $F_{3}$ & \textbf{1.50E+09} & 3.63E+07 & 4.31E+06 &6.18E+06& 6.87E+07& 4.54E+06&1.84E+07 &1.89E+07\\
      & $F_{4}$ & \textbf{1.84E+09} &3.77E+07 & 4.27E+06 & 6.16E+06 &7.42E+07 & 4.64E+06&1.89E+07&1.98E+07 \\
      & $F_{5}$ & \textbf{1.83E+09} & 3.85E+07 &4.33E+06 & 6.16E+06 & 7.50E+07& 4.64E+06&1.85E+07& 1.99E+07\\
      & $F_{6}$ & \textbf{1.84E+09} &3.71E+07 & 4.30E+06 & 6.01E+06& 7.38E+07 & 4.61E+06& 1.86E+07&1.99E+07\\
      & $F_{7}$ & \textbf{1.74E+09} &3.71E+07 & 4.15E+06 & 5.84E+06 & 7.05E+07& 4.37E+06&1.77E+07&1.91E+07\\
      & $F_{8}$ & \textbf{1.72E+09} & 3.69E+07 & 4.02E+06 & 5.75E+06 & 6.82E+07 & 4.35E+06& 1.71E+07&1.93E+07\\
      & $F_{9}$ & \textbf{1.78E+09} &3.81E+07 &4.35E+06 &6.10E+06 & 6.84E+07 & 4.56E+06&1.84E+07 &1.92E+07\\
      & $F_{10}$ & \textbf{1.44E+09} & 3.50E+07&4.12E+06 &  5.77E+06 & 6.94E+07 & 4.41E+06&1.76E+07& 1.83E+07\\
      & $F_{11}$ & \textbf{1.46E+09} & 3.43E+07& 4.11E+06 & 5.87E+06 & 6.89E+07 & 4.36E+06 &1.69E+07&1.83E+07\\
      & $F_{12}$ & \textbf{1.43E+09} &3.37E+07 & 3.97E+06 & 5.81E+06 & 6.56E+07 & 4.32E+06 &1.72E+07&1.92E+07\\
\midrule
\midrule
\multirow{12}[2]{*}{20D} & $F_{1}$ & \textbf{1.66E+09} & 3.93E+07 &4.29E+06 & 5.94E+06 & 7.20E+07 & 4.62E+06&1.87E+07&1.94E+07 \\
      & $F_{2}$ & \textbf{1.66E+09} & 3.89E+07 &4.17E+06 & 5.94E+06 & 7.33E+07 & 4.61E+06&1.91E+07& 1.95E+07\\
      & $F_{3}$ & \textbf{1.18E+09} & 3.38E+07 & 4.16E+06& 5.93E+06 &  6.77E+07& 4.49E+06& 1.72E+07&1.85E+07\\
      & $F_{4}$ & \textbf{1.65E+09} &3.56E+07 & 4.27E+06 & 5.85E+06 & 7.05E+07& 4.63E+06& 1.90E+07&1.94E+07\\
      & $F_{5}$ & \textbf{1.64E+09} & 3.88E+07 & 4.20E+06 & 6.04E+06& 7.16E+07& 4.62E+06&1.86E+07&1.95E+07 \\
      & $F_{6}$ & \textbf{1.64E+09} & 3.73E+07 & 4.25E+06 & 5.84E+06 & 7.10E+07 & 4.61E+06&1.86E+07& 1.92E+07\\
      & $F_{7}$ & \textbf{1.45E+09} & 3.13E+07 & 3.87E+06 & 5.27E+06 & 6.40E+07 & 4.26E+06& 1.72E+07&1.79E+07\\
      & $F_{8}$ & \textbf{1.44E+09} & 3.08E+07 & 3.53E+06& 5.29E+06 & 5.98E+07& 4.09E+06&1.69E+07&1.70E+07 \\
      & $F_{9}$ & \textbf{1.57E+09} & 3.62E+07 &4.05E+06 & 6.15E+06 &  6.91E+07& 4.54E+06&1.84E+07& 1.89E+07\\
      & $F_{10}$ & \textbf{9.80E+08} &2.94E+07 & 3.74E+06 & 5.28E+06& 5.69E+07& 4.12E+06& 1.54E+07& 1.63E+07\\
      & $F_{11}$ & \textbf{1.00E+09} & 2.37E+07 &3.76E+06 & 5.38E+06&5.86E+07& 4.04E+06&1.59E+07& 1.64E+07 \\
      & $F_{12}$ & \textbf{9.90E+08} & 2.33E+07 & 3.76E+06 &5.39E+06&5.47E+07& 4.09E+06& 1.57E+07& 1.60E+07\\
\bottomrule
\end{tabular}%
}
  \label{tab:FEs NP10000_supp}%
\end{table}%

\clearpage

\section{Supplementary information for the application of robotics control}\label{section_brax_supp}

\subsection{Illustrations of the robotics control tasks}\label{section_brax_image_supp}

% Figs. \ref{Figure_swimmer}-\ref{Figure_hopper}, illustrate the three robotics control tasks from the Brax \cite{brax} reinforcement learning library: swimmer, reacher, and hopper.

% \begin{enumerate}
%     \item Swimmer: A serpentine agent that must coordinate joint movements to navigate through a fluid environment, aiming for efficient propulsion and forward movement.
%     \item Reacher: A robotic arm environment where the goal is to control joint torques to reach a target point with precision, testing the fine control of the learning algorithm.
%     \item Hopper: A one-legged robot that must learn to balance and hop forward as far and as fast as possible, providing a benchmark for locomotion and stability in dynamic environments.
% \end{enumerate}

\begin{figure}[!h]
\centering
\includegraphics[width=6cm, height=3cm]{su_swimmer.png}
\caption{The swimmer task in Brax. It is designed to simulate a multi-jointed creature navigating through a fluid medium.}
\label{Figure_swimmer_supp}
\end{figure}


\begin{figure}[!h]
\centering
\includegraphics[width=6cm, height=3cm]{su_hopper.png}
\caption{The hopper task in Brax. It resembles a one-legged robotic creature with the objective to hop forward smoothly and quickly.}
\label{Figure_hopper_supp}
\end{figure}

\begin{figure}[!h]
\centering
\includegraphics[width=6cm, height=3cm]{su_reacher.png}
\caption{The reacher task in Brax. It simulates a robotic arm tasked with reaching a target location.}
\label{Figure_reacher_supp}
\end{figure}


\subsection{Supplementary results}\label{section_brax_results_supp}
% Table \ref{tab:comparative-reward-analysis} displays the results of the neuroevolution experimrnt for 60 minutes.

\begin{table}[htbp]
\centering
\caption{Rewards achieved by MetaDE and peer evolutionary algorithms on the robotics tasks. 
The mean and standard deviation (in parentheses) of the results over multiple runs are displayed in pairs. 
Results with the best mean values are highlighted.
}
\label{tab:comparative-reward-analysis_supp}
{%
 \renewcommand{\arraystretch}{1}
\renewcommand{\tabcolsep}{3pt}
% Table generated by Excel2LaTeX from sheet 'Exp6 brax'
\begin{tabular}{cccccccc}
\toprule
Task & MetaDE  & \textbf{CSO } & \textbf{CMAES } & \textbf{SHADE } & \textbf{DE}&\textbf{LSHADE-RSP}&\textbf{EDEV} \\
\midrule
Swimmer  & \textbf{ 190.85 (2.39) } &  183.45 (1.15)  &  186.07 (2.68)  & 185.90 (3.31) &  182.15 (2.54) &186.36 (1.21)&145.52 (41.22)\\
%\midrule
Hopper   &  1187.53 (122.72) & 1330.45 (325.55) &  \textbf{1389.72 (474.41)}  &  1022.25 (122.94)  &  871.06 (147.12)&1102.23 (100.33)&457.22 (80.16) \\
%\midrule
Reacher  &  -21.05 (5.05)  &  -504.07 (112.18)  & \textbf{ -3.76 (1.05) } &  -343.18 (105.65)  &  -522.99 (142.78)    &    -342.74 (36.63)  &  -493.46 (173.67)\\
\bottomrule
\end{tabular}%

}
\end{table}

\clearpage
\section{More Evolvers}\label{subsection Evolver Comparison_supp}
We selected \texttt{DE/rand/1/bin} as the evolver for its simplicity and adaptability, hypothesizing its capability to self-evolve. Nonetheless, it is also interesting to evaluate the performance implications of utilizing other EAs as evolvers.

In this experiment, we maintained the identical framework and parameter settings for MetaDE as utilized in the primary experiments. 
The distinction lies in the comparative evaluation of the effectiveness of DE, PSO, Natural Evolution Strategies (NES) \citesupp{NES}, and Competitive Swarm Optimization (CSO) \citesupp{CSO} as evolvers. 
For PSO, parameters were set to the recommended values of $w=0.729$ and $c_1=c_2=1.49$ \citesupp{PSOParamSetting}. 
It is important to note that CSO and NES do not require parameter tuning.

As shown in Tables \ref{tab:diffTunner 10D}-\ref{tab:diffTunner 20D} and Figs. \ref{Figure_evolver_10D}-\ref{Figure_evolver_20D}, the experimental outcomes suggest minimal performance disparities among the EAs when employed as evolvers. 
Specifically, DE demonstrates a marginal advantage in 10-dimensional problems, whereas the performance is comparably uniform across all evolvers for 20-dimensional problems. 
These results imply that the selection of an evolver within the MetaDE framework is relatively flexible. 
Such findings highlight the adaptability and flexibility of MetaDE, illustrating that its efficiency is not significantly influenced by the particular choice of evolver.


% Table generated by Excel2LaTeX from sheet 'Sheet1'
\begin{table}[htbp]
  \centering
  \caption{
  Performance of MetaDE with different evolvers on 10D problems in CEC2022 benchmark suite. 
The mean and standard deviation (in parentheses) of the results over multiple runs are displayed in pairs. 
Results with the best mean values are highlighted.
  }
  {
         \renewcommand{\arraystretch}{1}
 \renewcommand{\tabcolsep}{10pt}
% Table generated by Excel2LaTeX from sheet 'Exp5 evolver'
\begin{tabular}{ccccc}
\toprule
Func  & DE    & PSO   & NES   & CSO \\
\midrule
$F_{1}$ & \textbf{0.00E+00 (0.00E+00)} & \boldmath{}\textbf{0.00E+00 (0.00E+00)$\approx$}\unboldmath{} & \boldmath{}\textbf{0.00E+00 (0.00E+00)$\approx$}\unboldmath{} & \boldmath{}\textbf{0.00E+00 (0.00E+00)$\approx$}\unboldmath{} \\
$F_{2}$ & \textbf{0.00E+00 (0.00E+00)} & \boldmath{}\textbf{0.00E+00 (0.00E+00)$\approx$}\unboldmath{} & 1.60E-06 (2.82E-06)$-$ & \boldmath{}\textbf{0.00E+00 (0.00E+00)$\approx$}\unboldmath{} \\
$F_{3}$ & \textbf{0.00E+00 (0.00E+00)} & \boldmath{}\textbf{0.00E+00 (0.00E+00)$\approx$}\unboldmath{} & \boldmath{}\textbf{0.00E+00 (0.00E+00)$\approx$}\unboldmath{} & \boldmath{}\textbf{0.00E+00 (0.00E+00)$\approx$}\unboldmath{} \\
$F_{4}$ & \textbf{0.00E+00 (0.00E+00)} & 1.41E-03 (7.70E-03)$\approx$ & 3.23E-02 (1.76E-01)$\approx$ & \boldmath{}\textbf{0.00E+00 (0.00E+00)$\approx$}\unboldmath{} \\
$F_{5}$ & \textbf{0.00E+00 (0.00E+00)} & \boldmath{}\textbf{0.00E+00 (0.00E+00)$\approx$}\unboldmath{} & \boldmath{}\textbf{0.00E+00 (0.00E+00)$\approx$}\unboldmath{} & \boldmath{}\textbf{0.00E+00 (0.00E+00)$\approx$}\unboldmath{} \\
$F_{6}$ & \textbf{5.50E-04 (3.96E-04)} & 1.45E-03 (1.61E-03)$\approx$ & 1.66E-03 (1.79E-03)$-$ & 1.37E-03 (9.80E-04)$-$ \\
$F_{7}$ & \textbf{0.00E+00 (0.00E+00)} & \boldmath{}\textbf{0.00E+00 (0.00E+00)$\approx$}\unboldmath{} & \boldmath{}\textbf{0.00E+00 (0.00E+00)$\approx$}\unboldmath{} & \boldmath{}\textbf{0.00E+00 (0.00E+00)$\approx$}\unboldmath{} \\
$F_{8}$ & 5.52E-03 (4.41E-03) & \textbf{1.77E-06 (1.48E-06)$+$} & 1.83E-02 (1.09E-02)$-$ & 1.57E-03 (2.26E-03)$+$ \\
$F_{9}$ & \textbf{3.36E+00 (1.77E+01)} & 1.40E+01 (4.63E+01)$-$ & 2.27E+01 (4.18E+01)$\approx$ & 1.75E+01 (4.86E+01)$-$ \\
$F_{10}$ & \textbf{0.00E+00 (0.00E+00)} & \boldmath{}\textbf{0.00E+00 (0.00E+00)$\approx$}\unboldmath{} & \boldmath{}\textbf{0.00E+00 (0.00E+00)$\approx$}\unboldmath{} & 7.38E-02 (3.99E-01)$\approx$ \\
$F_{11}$ & \textbf{0.00E+00 (0.00E+00)} & \boldmath{}\textbf{0.00E+00 (0.00E+00)$\approx$}\unboldmath{} & \boldmath{}\textbf{0.00E+00 (0.00E+00)$\approx$}\unboldmath{} & \boldmath{}\textbf{0.00E+00 (0.00E+00)$\approx$}\unboldmath{} \\
$F_{12}$ & \textbf{1.39E+02 (4.63E+01)} & 1.48E+02 (3.39E+01)$\approx$ & 1.59E+02 (6.82E-06)$-$ & 1.54E+02 (1.78E+01)$\approx$ \\
\midrule
$+$ / $\approx$ / $-$ & --    & 1/10/1 & 4/8/0 & 1/9/2 \\
\bottomrule
\end{tabular}%
}

\footnotesize
\textsuperscript{*} The Wilcoxon rank-sum tests (with a significance level of 0.05) were conducted between MetaDE and each individually.
The final row displays the number of problems where the corresponding evlover performs statistically better ($+$),  similar ($\thickapprox$), or worse ($-$) compared to DE.\\
\label{tab:diffTunner 10D_supp}%
\end{table}%


% Table generated by Excel2LaTeX from sheet 'Sheet1'
\begin{table}[htbp]
  \centering
  \caption{
  Performance of MetaDE with different evolvers on 20D problems in CEC2022 benchmark suite.
The mean and standard deviation (in parentheses) of the results over multiple runs are displayed in pairs. 
Results with the best mean values are highlighted.
  }
  {
        \renewcommand{\arraystretch}{1}
 \renewcommand{\tabcolsep}{10pt}
% Table generated by Excel2LaTeX from sheet 'Exp5 evolver'
\begin{tabular}{ccccc}
\toprule
Func  & DE    & PSO   & NES   & CSO \\
\midrule
$F_{1}$ & \multicolumn{1}{l}{\textbf{0.00E+00 (0.00E+00)}} & \multicolumn{1}{l}{\boldmath{}\textbf{0.00E+00 (0.00E+00)$\approx$}\unboldmath{}} & \multicolumn{1}{l}{\boldmath{}\textbf{0.00E+00 (0.00E+00)$\approx$}\unboldmath{}} & \multicolumn{1}{l}{\boldmath{}\textbf{0.00E+00 (0.00E+00)$\approx$}\unboldmath{}} \\
$F_{2}$ & \multicolumn{1}{l}{1.26E-02 (3.74E-02)} & \multicolumn{1}{l}{4.28E-02 (1.25E-01)$\approx$} & \multicolumn{1}{l}{1.11E+00 (1.30E+00)$-$} & \multicolumn{1}{l}{\textbf{0.00E+00 (0.00E+00)$+$}} \\
$F_{3}$ & \multicolumn{1}{l}{\textbf{0.00E+00 (0.00E+00)}} & \multicolumn{1}{l}{\boldmath{}\textbf{0.00E+00 (0.00E+00)$\approx$}\unboldmath{}} & \multicolumn{1}{l}{\boldmath{}\textbf{0.00E+00 (0.00E+00)$\approx$}\unboldmath{}} & \multicolumn{1}{l}{\boldmath{}\textbf{0.00E+00 (0.00E+00)$\approx$}\unboldmath{}} \\
$F_{4}$ & \multicolumn{1}{l}{\textbf{2.02E+00 (8.56E-01)}} & \multicolumn{1}{l}{2.51E+00 (1.19E+00)$-$} & \multicolumn{1}{l}{5.53E+00 (1.55E+00)$-$} & \multicolumn{1}{l}{2.85E+00 (1.48E+00)$\approx$} \\
$F_{5}$ & \multicolumn{1}{l}{\textbf{0.00E+00 (0.00E+00)}} & \multicolumn{1}{l}{\boldmath{}\textbf{0.00E+00 (0.00E+00)$\approx$}\unboldmath{}} & \multicolumn{1}{l}{\boldmath{}\textbf{0.00E+00 (0.00E+00)$\approx$}\unboldmath{}} & \multicolumn{1}{l}{\boldmath{}\textbf{0.00E+00 (0.00E+00)$\approx$}\unboldmath{}} \\
$F_{6}$ & \multicolumn{1}{l}{1.16E-01 (2.79E-02)} & \multicolumn{1}{l}{1.81E-01 (2.42E-01)$-$} & \multicolumn{1}{l}{9.32E-02 (2.65E-02)$+$} & \multicolumn{1}{l}{\boldmath{}\textbf{1.10E-01 (2.85E-02)$\approx$}\unboldmath{}} \\
$F_{7}$ & \multicolumn{1}{l}{5.17E-02 (6.21E-02)} & \multicolumn{1}{l}{1.61E-01 (3.84E-01)$\approx$} & \multicolumn{1}{l}{8.13E-01 (6.01E-01)$-$} & \multicolumn{1}{l}{\textbf{6.37E-02 (2.03E-01)$+$}} \\
$F_{8}$ & \multicolumn{1}{l}{\textbf{7.19E-01 (1.02E+00)}} & \multicolumn{1}{l}{2.31E+00 (4.16E+00)$\approx$} & \multicolumn{1}{l}{1.82E+01 (3.82E+00)$-$} & \multicolumn{1}{l}{2.19E+00 (4.05E+00)$\approx$} \\
$F_{9}$ & \multicolumn{1}{l}{\textbf{1.07E+02 (1.98E+01)}} & \multicolumn{1}{l}{1.34E+02 (3.72E+01)$-$} & \multicolumn{1}{l}{1.81E+02 (3.75E-06)$-$} & \multicolumn{1}{l}{1.75E+02 (1.81E+01)$-$} \\
$F_{10}$ & \multicolumn{1}{l}{\textbf{0.00E+00 (0.00E+00)}} & \multicolumn{1}{l}{\boldmath{}\textbf{0.00E+00 (0.00E+00)$\approx$}\unboldmath{}} & \multicolumn{1}{l}{\boldmath{}\textbf{0.00E+00 (0.00E+00)$\approx$}\unboldmath{}} & \multicolumn{1}{l}{7.86E-01 (2.39E+00)$-$} \\
$F_{11}$ & \multicolumn{1}{l}{7.28E-05 (3.06E-04)} & \multicolumn{1}{l}{1.01E-02 (3.90E-02)$\approx$} & \multicolumn{1}{l}{8.79E-03 (1.52E-02)$-$} & \multicolumn{1}{l}{\textbf{1.45E-05 (0.00E+00)$-$}} \\
$F_{12}$ & \multicolumn{1}{l}{\textbf{2.29E+02 (6.08E-01)}} & \multicolumn{1}{l}{2.29E+02 (1.17E+00)$\approx$} & \multicolumn{1}{l}{2.31E+02 (8.77E-01)$-$} & \multicolumn{1}{l}{2.30E+02 (9.60E-01)$-$} \\
\midrule
$+$ / $\approx$ / $-$ & --    & 0/9/3 & 1/4/7 & 2/6/4 \\
\bottomrule
\end{tabular}%
}

\footnotesize
\textsuperscript{*} The Wilcoxon rank-sum tests (with a significance level of 0.05) were conducted between MetaDE and each individually.
The final row displays the number of problems where the corresponding evlover performs statistically better ($+$),  similar ($\thickapprox$), or worse ($-$) compared to DE.\\
\label{tab:diffTunner 20D_supp}%
\end{table}%

\clearpage

\begin{figure*}[htpb]
\centering
\includegraphics[scale=0.29]{su_D10_all_evolver.pdf}
\caption{Convergence curves with different evolvers on 10D problems in CEC2022 benchmark suite.}
\label{Figure_evolver_10D_supp}
\end{figure*}


\begin{figure*}[htpb]
\centering
\includegraphics[scale=0.29]{su_D20_all_evolver.pdf}
\caption{Convergence curves with different evolvers on 20D problems in CEC2022 benchmark suite.}
\label{Figure_evolver_20D_supp}
\end{figure*}


\clearpage
% \bibliography{Supplement_references}

% \bibliographystylesupp{IEEEtran}
% \bibliographysupp{Supplement_references}

%\title{Generating 3D \hl{Small} Binding Molecules Using Shape-Conditioned Diffusion Models with Guidance}
%\date{\vspace{-5ex}}

%\author{
%	Ziqi Chen\textsuperscript{\rm 1}, 
%	Bo Peng\textsuperscript{\rm 1}, 
%	Tianhua Zhai\textsuperscript{\rm 2},
%	Xia Ning\textsuperscript{\rm 1,3,4 \Letter}
%}
%\newcommand{\Address}{
%	\textsuperscript{\rm 1}Computer Science and Engineering, The Ohio Sate University, Columbus, OH 43210.
%	\textsuperscript{\rm 2}Perelman School of Medicine, University of Pennsylvania, Philadelphia, PA 19104.
%	\textsuperscript{\rm 3}Translational Data Analytics Institute, The Ohio Sate University, Columbus, OH 43210.
%	\textsuperscript{\rm 4}Biomedical Informatics, The Ohio Sate University, Columbus, OH 43210.
%	\textsuperscript{\Letter}ning.104@osu.edu
%}

%\newcommand\affiliation[1]{%
%	\begingroup
%	\renewcommand\thefootnote{}\footnote{#1}%
%	\addtocounter{footnote}{-1}%
%	\endgroup
%}



\setcounter{secnumdepth}{2} %May be changed to 1 or 2 if section numbers are desired.

\setcounter{section}{0}
\renewcommand{\thesection}{S\arabic{section}}

\setcounter{table}{0}
\renewcommand{\thetable}{S\arabic{table}}

\setcounter{figure}{0}
\renewcommand{\thefigure}{S\arabic{figure}}

\setcounter{algorithm}{0}
\renewcommand{\thealgorithm}{S\arabic{algorithm}}

\setcounter{equation}{0}
\renewcommand{\theequation}{S\arabic{equation}}


\begin{center}
	\begin{minipage}{0.95\linewidth}
		\centering
		\LARGE 
	Generating 3D Binding Molecules Using Shape-Conditioned Diffusion Models with Guidance (Supplementary Information)
	\end{minipage}
\end{center}
\vspace{10pt}

%%%%%%%%%%%%%%%%%%%%%%%%%%%%%%%%%%%%%%%%%%%%%
\section{Parameters for Reproducibility}
\label{supp:experiments:parameters}
%%%%%%%%%%%%%%%%%%%%%%%%%%%%%%%%%%%%%%%%%%%%%

We implemented both \SE and \methoddiff using Python-3.7.16, PyTorch-1.11.0, PyTorch-scatter-2.0.9, Numpy-1.21.5, Scikit-learn-1.0.2.
%
We trained the models using a Tesla V100 GPU with 32GB memory and a CPU with 80GB memory on Red Hat Enterprise 7.7.
%
%We released the code, data, and the trained model at Google Drive~\footnote{\url{https://drive.google.com/drive/folders/146cpjuwenKGTd6Zh4sYBy-Wv6BMfGwe4?usp=sharing}} (will release to the public on github once the manuscript is accepted).

%===================================================================
\subsection{Parameters of \SE}
%===================================================================


In \SE, we tuned the dimension of all the hidden layers including VN-DGCNN layers
(Eq.~\ref{eqn:shape_embed}), MLP layers (Eq.~\ref{eqn:se:decoder}) and
VN-In layer (Eq.~\ref{eqn:se:decoder}), and the dimension $d_p$ of generated shape latent embeddings $\shapehiddenmat$ with the grid-search algorithm in the 
parameter space presented in Table~\ref{tbl:hyper_se}.
%
We determined the optimal hyper-parameters according to the mean squared errors of the predictions of signed distances for 1,000 validation molecules that are selected as described in Section ``Data'' 
in the main manuscript.
%
The optimal dimension of all the hidden layers is 256, and the optimal dimension $d_p$ of shape latent embedding \shapehiddenmat is 128.
%
The optimal number of points $|\pc|$ in the point cloud \pc is 512.
%
We sampled 1,024 query points in $\mathcal{Z}$ for each molecule shape.
%
We constructed graphs from point clouds, which are employed to learn $\shapehiddenmat$ with VN-DGCNN layer (Eq.~\ref{eqn:shape_embed}), using the $k$-nearest neighbors based on Euclidean distance with $k=20$.
%
We set the number of VN-DGCNN layers as 4.
%
We set the number of MLP layers in the decoder (Eq.~\ref{eqn:se:decoder}) as 2.
%
We set the number of VN-In layers as 1.

%
We optimized the \SE model via Adam~\cite{adam} with its parameters (0.950, 0.999), %betas (0.95, 0.999), 
learning rate 0.001, and batch size 16.
%
We evaluated the validation loss every 2,000 training steps.
%
We scheduled to decay the learning rate with a factor of 0.6 and a minimum learning rate of 1e-6 if 
the validation loss does not decrease in 5 consecutive evaluations.
%
The optimal \SE model has 28.3K learnable parameters. 
%
We trained the \SE model %for at most 80 hours 
with $\sim$156,000 training steps.
%
The training took 80 hours with our GPUs.
%
The trained \SE model achieved the minimum validation loss at 152,000 steps.


\input{tables/hyper_para_se}
%
\input{tables/hyper_para_diff}


%===================================================================
\subsection{Parameters of \methoddiff}
%===================================================================

Table~\ref{tbl:hyper_diff} presents the parameters used to train \methoddiff.
%
In \methoddiff, we set the hidden dimensions of all the MLP layers and the scalar hidden layers in GVPs (Eq.~\ref{eqn:pred:gvp} and Eq.~\ref{eqn:mess:gvp}) as 128. %, including all the MLP layers in \methoddiff and the scalar dimension of GVP layers in Eq.~\ref{eqn:pred:gvp} and Eq.~\ref{eqn:mess:gvp}. %, and MLP layer (Eq.~\ref{eqn:diff:graph:atompred}) as 128.
%
We set the dimensions of all the vector hidden layers in GVPs as 32.
%
We set the number of layers $L$ in \molpred as 8.
%and the number of layers in graph neural networks as 8.
%
Both two GVP modules in Eq.~\ref{eqn:pred:gvp} and Eq.~\ref{eqn:mess:gvp} consist of three GVP layers. %, which consisa GVP modset the number of layer of GVP modules %is a multi-head attention layer ($\text{MHA}^{\mathtt{x}}$ or $\text{MHA}^{\mathtt{h}}$) with 16 heads.
% 
We set the number of VN-MLP layers in Eq.~\ref{eqn:shaper} as 1 and the number of MLP layers as 2 for all the involved MLP functions.
%

We constructed graphs from atoms in molecules, which are employed to learn the scalar embeddings and vector embeddings for atoms %predict atom coordinates and features  
(Eq.~\ref{eqn:geometric_embedding} and \ref{eqn:attention}), using the $N$-nearest neighbors based on Euclidean distance with $N=8$. 
%
We used $K=15$ atom features in total, indicating the atom types and its aromaticity.
%
These atom features include 10 non-aromatic atoms (i.e., ``H'', ``C'', ``N'', ``O'', ``F'', ``P'', ``S'', ``Cl'', ``Br'', ``I''), 
and 5 aromatic atoms (i.e., ``C'', ``N'', ``O'', ``P'', ``S'').
%
We set the number of diffusion steps $T$ as 1,000.
%
We set the weight $\xi$ of atom type loss (Eq.~\ref{eqn:loss}) as $100$ to balance the values of atom type loss and atom coordinate loss.
%
We set the threshold $\delta$ (Eq.~\ref{eqn:diff:obj:pos}) as 10.
%
The parameters $\beta_t^{\mathtt{x}}$ and $\beta_t^{\mathtt{v}}$ of variance scheduling in the forward diffusion process of \methoddiff are discussed in 
Supplementary Section~\ref{supp:forward:variance}.
%
%Please note that as in \squid, we did not perform extensive hyperparameter optimization for \methoddiff.
%
Following \squid, we did not perform extensive hyperparameter tunning for \methoddiff given that the used 
hyperparameters have enabled good performance.

%
We optimized the \methoddiff model via Adam~\cite{adam} with its parameters (0.950, 0.999), learning rate 0.001, and batch size 32.
%
We evaluated the validation loss every 2,000 training steps.
%
We scheduled to decay the learning rate with a factor of 0.6 and a minimum learning rate of 1e-5 if 
the validation loss does not decrease in 10 consecutive evaluations.
%
The \methoddiff model has 7.8M learnable parameters. 
%
We trained the \methoddiff model %for at most 60 hours 
with $\sim$770,000 training steps.
%
The training took 70 hours with our GPUs.
%
The trained \methoddiff achieved the minimum validation loss at 758,000 steps.

During inference, %the sampling, 
following Adams and Coley~\cite{adams2023equivariant}, we set the variance $\phi$ 
of atom-centered Gaussians as 0.049, which is used to build a set of points for shape guidance in Section ``\method with Shape Guidance'' 
in the main manuscript.
%
We determined the number of atoms in the generated molecule using the atom number distribution of training molecules that have surface shape sizes similar to the condition molecule.
%
The optimal distance threshold $\gamma$ is 0.2, and the optimal stop step $S$ for shape guidance is 300.
%
With shape guidance, each time we updated the atom position (Eq.~\ref{eqn:shape_guidance}), we randomly sampled the weight $\sigma$ from $[0.2, 0.8]$. %\bo{(XXX)}.
%
Moreover, when using pocket guidance as mentioned in Section ``\method with Pocket Guidance'' in the main manuscript, each time we updated the atom position (Eq.~\ref{eqn:pocket_guidance}), we randomly sampled the weight $\epsilon$ from $[0, 0.5]$. 
%
For each condition molecule, it took around 40 seconds on average to generate 50 molecule candidates with our GPUs.



%%%%%%%%%%%%%%%%%%%%%%%%%%%%%%%%%%%%%%%%%%%%%%
\section{Performance of \decompdiff with Protein Pocket Prior}
\label{supp:app:decompdiff}
%%%%%%%%%%%%%%%%%%%%%%%%%%%%%%%%%%%%%%%%%%%%%%

In this section, we demonstrate that \decompdiff with protein pocket prior, referred to as \decompdiffbeta, shows very limited performance in generating drug-like and synthesizable molecules compared to all the other methods, including \methodwithpguide and \methodwithsandpguide.
%
We evaluate the performance of \decompdiffbeta in terms of binding affinities, drug-likeness, and diversity.
%
We compare \decompdiffbeta with \methodwithpguide and \methodwithsandpguide and report the results in Table~\ref{tbl:comparison_results_decompdiff}.
%
Note that the results of \methodwithpguide and \methodwithsandpguide here are consistent with those in Table~\ref{tbl:overall_results_docking2} in the main manuscript.
%
As shown in Table~\ref{tbl:comparison_results_decompdiff}, while \decompdiffbeta achieves high binding affinities in Vina M and Vina D, it substantially underperforms \methodwithpguide and \methodwithsandpguide in QED and SA.
%
Particularly, \decompdiffbeta shows a QED score of 0.36, while \methodwithpguide substantially outperforms \decompdiffbeta in QED (0.77) with 113.9\% improvement.
%
\decompdiffbeta also substantially underperforms \methodwithpguide in terms of SA scores (0.55 vs 0.76).
%
These results demonstrate the limited capacity of \decompdiffbeta in generating drug-like and synthesizable molecules.
%
As a result, the generated molecules from \decompdiffbeta can have considerably lower utility compared to other methods.
%
Considering these limitations of \decompdiffbeta, we exclude it from the baselines for comparison.

\input{tables/decompdiff_method_compare}

%===================================================================
\section{{Additional Experimental Results on SMG}}
\label{supp:app:results}
%===================================================================

%-------------------------------------------------------------------------------------------------------------------------------------
\subsection{Comparison on Shape and Graph Similarity}
\label{supp:app:results:overall_shape}
%-------------------------------------------------------------------------------------------------------------------------------------

%\ziqi{Outline for this section:
%	\begin{itemize}
%		\item \method can consistently generate molecules with novel structures (low graph similarity) and similar shapes (high shape similarity), such that these molecules have comparable binding capacity with the condition molecules, and potentially better properties as will be shown in Table~\ref{tbl:overall_results_quality_10}.
%	\end{itemize}
%}

\input{tables/overall_results_sims}
%\label{tbl:overall_sim}


{We evaluate the shape similarity \shapesim and graph similarity \graphsim of molecules generated from}
%Table~\ref{tbl:overall_sim} presents the comparison of shape-conditioned molecule generation among 
\dataset, \squid, \method and \methodwithsguide under different graph similarity constraints  ($\delta_g$=1.0, 0.7, 0.5, 0.3). 
%
%During the evaluation, for each molecule in the test set, all the methods are employed to generate or identify 50 molecules with similar shapes.
%
We calculate evaluation metrics using all the generated molecules satisfying the graph similarity constraints.
%
Particularly, when $\delta_g$=1.0, we do not filter out any molecules based on the constraints and directly calculate metrics on all the generated molecules.
%
When $\delta_g$=0.7, 0.5 or 0.3, we consider only generated molecules with similarities lower than $\delta_g$.
%
Based on \shapesim and \graphsim as described in Section ``Evaluation Metrics'' in the main manuscript,
we calculate the following metrics using the subset of molecules with \graphsim lower than $\delta_g$, from a set of 50 generated molecules for each test molecule and report the average of  these metrics across all test molecules:
%
(1) \avgshapesim\ measures the average \shapesim across each subset of generated molecules with $\graphsim$ lower than $\delta_g$; %per test molecule, with the overall average calculated across all test molecules; }%the 50 generated molecules for each test molecule, averaged across all test molecules;
(2) \avggraphsim\ calculates the average \graphsim for each set; %, with these means averaged across all test molecules}; %} 50 molecules, %\bo{@Ziqi rephrase}, with results averaged on the test set;\ziqi{with the average computed over the test set; }
(3) \maxshapesim\ determines the maximum \shapesim within each set; %, with these maxima averaged across all test molecules; }%\hl{among 50 molecules}, averaged across all test molecules;
(4) \maxgraphsim\ measures the \graphsim of the molecule with maximum \shapesim in each set. %, averaged across all test molecules; }%\hl{among 50 molecules}, averaged across all test molecules;

%
As shown in Table~\ref{tbl:overall_sim}, \method and \methodwithsguide demonstrate outstanding performance in terms of the average shape similarities (\avgshapesim) and the average graph similarities (\avggraphsim) among generated molecules.
%
%\ziqi{
%Table~\ref{tbl:overall} also shows that \method and \methodwithsguide consistently outperform all the baseline methods in average shape similarities (\avgshapesim) and only slightly underperform 
%the best baseline \dataset in average graph similarities (\avggraphsim).
%}
%
Specifically, when $\delta_g$=0.3, \methodwithsguide achieves a substantial 10.5\% improvement in \avgshapesim\ over the best baseline \dataset. 
%
In terms of \avggraphsim, \methodwithsguide also achieves highly comparable performance with \dataset (0.217 vs 0.211, in \avggraphsim, lower values indicate better performance).
%
%This trend remains consistent across various $\delta_g$ values.
This trend remains consistent when applying various similarity constraints (i.e., $\delta_g$) as shown in Table~\ref{tbl:overall_sim}.


Similarly, \method and \methodwithsguide demonstrate superior performance in terms of the average maximum shape similarity across generated molecules for all test molecules (\maxshapesim), as well as the average graph similarity of the molecules with the maximum shape similarities (\maxgraphsim). %maximum shape similarities of generated molecules (\maxshapesim) and the average graph similarities of molecules with the maximum shape similarities (\maxgraphsim). %\bo{\maxgraphsim is misleading... how about $\text{avgMSim}_\text{g}$}
%
%\bo{
%in terms of the maximum shape similarities (\maxshapesim) and the maximum graph similarities (\maxgraphsim) among all the generated molecules.
%@Ziqi are the metrics maximum values or the average of maximum values?
%}
%
Specifically, at \maxshapesim, Table~\ref{tbl:overall_sim} shows that \methodwithsguide outperforms the best baseline \squid ($\lambda$=0.3) when $\delta_g$=0.3, 0.5, and 0.7, and only underperforms
it by 0.7\% when $\delta$=1.0.
%
We also note that the molecules generated by {\methodwithsguide} with the maximum shape similarities have substantially lower graph similarities ({\maxgraphsim}) compared to those generated by {\squid} ({$\lambda$}=0.3).
%\hl{We also note that the molecules with the maximum shape similarities generated by {\methodwithsguide} are with significantly lower graph similarities ({\maxgraphsim}) than those generated by {\squid} ({$\lambda$}=0.3).}
%
%\bo{@Ziqi please rephrase the language}
%
%\bo{
%@Ziqi the conclusion is not obvious. You may want to remind the meaning of \maxshapesim and \maxgraphsim here, and based on what performance you say this.
%}
%
%\bo{\st{This also underscores the ability of {\methodwithsguide} in generating molecules with similar shapes to condition molecules and novel graph structures.}}
%
As evidenced by these results, \methodwithsguide features strong capacities of generating molecules with similar shapes yet novel graph structures compared to the condition molecule, facilitating the discovery of promising drug candidates.
%

\begin{comment}
\ziqi{replace \#n\% with the percentage of novel molecules that do not exist in the dataset and update the discussion accordingly}
%\ziqi{
Table~\ref{tbl:overall_sim} also presents \bo{\#n\%}, the percentage of molecules generated by each method %\st{(\#n\%)} 
with graph similarities lower than the constraint $\delta_g$. 
%
%\bo{
%Table~\ref{tbl:overall_sim} also presents \#n\%, the percentage of generated molecules with graph similarities lower than the constraint $\delta_g$, of different methods. 
%}
%
As shown in Table~\ref{tbl:overall_sim},  when a restricted constraint (i.e., $\delta_g$=0.3) is applied, \method and \methodwithsguide could still generate a sufficient number of molecules satisfying the constraint.
%
Particularly, when $\delta_g$=0.3, \method outperforms \squid with $\lambda$=0.3 by XXX and \squid with $\lambda$=1.0 by XXX.
% achieve the second and the third in \#n\% and only underperform the best baseline \dataset.
%
This demonstrates the ability of \method in generating molecules with novel structures. 
%
When $\delta_g$=0.5, 0.7 and 1.0, both methods generate over 99.0\% of molecules satisfying the similarity constraint $\delta_g$.
%
%Note that \dataset is guaranteed to identify at least 50 molecules satisfying the $\delta_g$ by searching within a training dataset of diverse molecules.
%
Note that \dataset is a search algorithm that always first identifies the molecules satisfying $\delta_g$ and then selects the top-50 molecules of the highest shape similarities among them. 
%
Due to the diverse molecules in %\hl{the subset} \bo{@Ziqi why do you want to stress subset?} of 
the training set, \dataset can always identify at least 50 molecules under different $\delta_g$ and thus achieve 100\% in \#n\%.
%
%\bo{
%Note that \dataset is a search algorithm that always generate molecules XXX
%@Ziqi
%We need to discuss here. For \dataset, \#n\% in this table does not look aligned with that in Fig 1 if the highlighted defination is correct...
%}
%
%Thus, \dataset achieves 100.0\% in \#n\% under different $\delta_g$.
%
It is also worth noting that when $\delta_g$=1.0, \#n\% reflects the validity among all the generated molecules. 
%
As shown in Table~\ref{tbl:overall_sim}, \method and \methodwithsguide are able to generate 99.3\% and 99.2\% valid molecules.
%
This demonstrates their ability to effectively capture the underlying chemical rules in a purely data-driven manner without relying on any prior knowledge (e.g., fragments) as \squid does.
%
%\bo{
%@Ziqi I feel this metric is redundant with the avg graph similarity when constraint is 1.0. Generally, if the avg similarity is small. You have more mols satisfying the requirement right?
%}
\end{comment}

Table~\ref{tbl:overall_sim} also shows that by incorporating shape guidance, \methodwithsguide
%\bo{
%@Ziqi where does this come from...
%}
substantially outperforms \method in both \avgshapesim and \maxshapesim, while maintaining comparable graph similarities (i.e., \avggraphsim\ and \maxgraphsim).
%
Particularly, when $\delta_g$=0.3, \methodwithsguide 
establishes a considerable improvement of 6.9\% and 4.9\%
%\bo{\st{achieves 6.9\% and 4.9\% improvements}} 
over \method in \avgshapesim and \maxshapesim, respectively. 
%
%\hl{In the meanwhile}, 
%\bo{@Ziqi it is not the right word...}
Meanwhile, \methodwithsguide achieves the same \avggraphsim with \method and only slightly underperforms \method in \maxgraphsim (0.223 vs 0.220).
%\bo{
%XXX also achieves XXX
%}
%it maintains the same \avggraphsim\ with \method and only slightly underperforms \method in \maxgraphsim (0.223 vs 0.220).
%
%Compared with \method, \methodwithsguide consistently generates molecules with higher shape similarities while maintaining comparable graph similarities.
%
%\bo{
%@Ziqi you may want to highlight the utility of "generating molecules with higher shape similarities while maintaining comparable graph similarities" in real drug discovery applications.
%
%
%\bo{
%@Ziqi You did not present the details of method yet...
%}
%
%\methodwithsguide leverages additional shape guidance to push the predicted atoms to the shape of condition molecules \bo{and XXX (@Ziqi boosts the shape similarities XXX)} , as will be discussed in Section ``\method with Shape Guidance'' later.
%
The superior performance of \methodwithsguide suggests that the incorporation of shape guidance effectively boosts the shape similarities of generated molecules without compromising graph similarities.
%
%This capability could be crucial in drug discovery, 
%\bo{@Ziqi it is a strong statement. Need citations here}, 
%as it enables the discovery of drug candidates that are both more potentially effective due to the improved shape similarities and novel induced by low graph similarities.
%as it could enable the identification of candidates with similar binding patterns %with the condition molecule (i.e., high shape similarities) 
%(i.e., high shape similarities) and graph structures distinct from the condition molecules (i.e., low graph similarities).
%\bo{\st{and enjoys novel structures (i.e., low graph similarities) with potentially better properties. } \ziqi{change enjoys}}
%\bo{
%and enjoys potentially better properties (i.e., low graph similarities). \ziqi{this looks weird to me... need to discuss}
%}
%\st{potentially better properties (i.e., low graph similarities).}}

%-------------------------------------------------------------------------------------------------------------------------------------
\subsection{Comparison on Validity and Novelty}
\label{supp:app:results:valid_novel}
%-------------------------------------------------------------------------------------------------------------------------------------

We evaluate the ability of \method and \squid to generate molecules with valid and novel 2D molecular graphs.
%
We calculate the percentages of the valid and novel molecules among all the generated molecules.
%
As shown in Table~\ref{tbl:validity_novelty}, both \method and \methodwithsguide outperform \squid with $\lambda$=0.3 and $\lambda$=1.0 in generating novel molecules.
%
Particularly, almost all valid molecules generated by \method and \methodwithsguide are novel (99.8\% and 99.9\% at \#n\%), while the best baseline \squid with $\lambda$=0.3 achieves 98.4\% in novelty.
%
In terms of the percentage of valid and novel molecules among all the generated ones (\#v\&n\%), \method and \methodwithsguide again outperform \squid with $\lambda$=0.3 and $\lambda$=1.0.
%
We also note that at \#v\%,  \method (99.1\%) and \methodwithsguide (99.2\%) slightly underperform \squid with $\lambda$=0.3 and $\lambda$=1.0 (100.0\%) in generating valid molecules.
%
\squid guarantees the validity of generated molecules by incorporating valence rules into the generation process and ensuring it to avoid fragments that violate these rules.
%
Conversely, \method and \methodwithsguide use a purely data-driven approach to learn the generation of valid molecules.
%
These results suggest that, even without integrating valence rules, \method and \methodwithsguide can still achieve a remarkably high percentage of valid and novel generated molecules.

\input{tables/validity_novelty}

%-------------------------------------------------------------------------------------------------------------------------------------
\subsection{Additional Quality Comparison between Desirable Molecules Generated by \method and \squid}
\label{supp:app:results:quality_desirable}
%-------------------------------------------------------------------------------------------------------------------------------------

\input{tables/overall_results_quality0.5}
%\label{tbl:overall_quality05}

\input{tables/overall_results_quality0.7}
%\label{tbl:overall_quality07}

\input{tables/overall_results_quality1.0}
%\label{tbl:overall_quality10}

Similar to Table~\ref{tbl:overall_results_quality_desired} in the main manuscript, we present the performance comparison on the quality of desirable molecules generated by different methods under different graph similarity constraints $\delta_g$=0.5, 0.7 and 1.0, as detailed in Table~\ref{tbl:overall_results_quality_05}, Table~\ref{tbl:overall_results_quality_07}, and Table~\ref{tbl:overall_results_quality_10}, respectively.
%
Overall, these tables show that under varying graph similarity constraints, \method and \methodwithsguide can always generate desirable molecules with comparable quality to baselines in terms of stability, 3D structures, and 2D structures.
%
These results demonstrate the strong effectiveness of \method and \methodwithsguide in generating high-quality desirable molecules with stable and realistic structures in both 2D and 3D.
%
This enables the high utility of \method and \methodwithsguide in discovering promising drug candidates.


\begin{comment}
The results across these tables demonstrate similar observations with those under $\delta_g$=0.3 in Table~\ref{tbl:overall_results_quality_desired}.
%
For stability, when $\delta_g$=0.5, 0.7 or 1.0, \method and \methodwithsguide achieve comparable performance or fall slightly behind \squid ($\lambda$=0.3) and \squid ($\lambda$=1.0) in atom stability and molecule stability.
%
For example, when $\delta_g$=0.5, as shown in Table~\ref{tbl:overall_results_quality_05}, \method achieves similar performance with the best baseline \squid ($\lambda$=0.3) in atom stability (0.992 for \method vs 0.996 for \squid with $\lambda$=0.3).
%
\method underperforms \squid ($\lambda$=0.3) in terms of molecule stability.
%
For 3D structures, \method and \methodwithsguide also consistently generate molecules with more realistic 3D structures compared to \squid.
%
Particularly, \methodwithsguide achieves the best performance in RMSD and JS of bond lengths across $\delta_g$=0.5, 0.7 and 1.0.
%
In JS of dihedral angles, \method achieves the best performance among all the methods.
%
\method and \methodwithsguide underperform \squid in JS of bond angles, primarily because \squid constrains the bond angles in the generated molecules.
%
For 2D structures, \method and \methodwithsguide again achieve the best performance 
\end{comment}

%===================================================================
\section{Additional Experimental Results on PMG}
\label{supp:app:results_PMG}
%===================================================================

%\label{tbl:comparison_results_decompdiff}


%-------------------------------------------------------------------------------------------------------------------------------------
%\subsection{{Additional Comparison for PMG}}
%\label{supp:app:results:docking}
%-------------------------------------------------------------------------------------------------------------------------------------

In this section, we present the results of \methodwithpguide and \methodwithsandpguide when generating 100 molecules. 
%
Please note that both \methodwithpguide and \methodwithsandpguide show remarkable efficiency over the PMG baselines.
%
\methodwithpguide and \methodwithsandpguide generate 100 molecules in 48 and 58 seconds on average, respectively, while the most efficient baseline \targetdiff requires 1,252 seconds.
%
We report the performance of \methodwithpguide and \methodwithsandpguide against state-of-the-art PMG baselines in Table~\ref{tbl:overall_results_docking_100}.


%
According to Table~\ref{tbl:overall_results_docking_100}, \methodwithpguide and \methodwithsandpguide achieve comparable performance with the PMG baselines in generating molecules with high binding affinities.
%
Particularly, in terms of Vina S, \methodwithsandpguide achieves very comparable performance (-4.56 kcal/mol) to the third-best baseline \decompdiff (-4.58 kcal/mol) in average Vina S; it also achieves the third-best performance (-4.82 kcal/mol) among all the methods and slightly underperforms the second-best baseline \AR (-4.99 kcal/mol) in median Vina S
%
\methodwithsandpguide also achieves very close average Vina M (-5.53 kcal/mol) with the third-best baseline \AR (-5.59 kcal/mol) and the third-best performance (-5.47 kcal/mol) in median Vina M.
%
Notably, for Vina D, \methodwithpguide and \methodwithsandpguide achieve the second and third performance among all the methods.
%
In terms of the average percentage of generated molecules with Vina D higher than those of known ligands (i.e., HA), \methodwithpguide (58.52\%) and \methodwithsandpguide (58.28\%) outperform the best baseline \targetdiff (57.57\%).
%
These results signify the high utility of \methodwithpguide and \methodwithsandpguide in generating molecules that effectively bind with protein targets and have better binding affinities than known ligands.

In addition to binding affinities, \methodwithpguide and \methodwithsandpguide also demonstrate similar performance compared to the baselines in metrics related to drug-likeness and diversity.
%
For drug-likeness, both \methodwithpguide and \methodwithsandpguide achieve the best (0.67) and the second-best (0.66) QED scores.
%
They also achieve the third and fourth performance in SA scores.
%
In terms of the diversity among generated molecules,  \methodwithpguide and \methodwithsandpguide slightly underperform the baselines, possibly due to the design that generates molecules with similar shapes to the ligands.
%
These results highlight the strong ability of \methodwithpguide and \methodwithsandpguide in efficiently generating effective binding molecules with favorable drug-likeness and diversity.
%
This ability enables them to potentially serve as promising tools to facilitate effective and efficient drug development.

\input{tables/overall_results_docking4}
%\label{tbl:overall_results_docking_100}

%-------------------------------------------------------------------------------------------------------------------------------------
%\subsection{{Comparison of Pocket Guidance}}
%\label{supp:app:results:docking}
%-------------------------------------------------------------------------------------------------------------------------------------


\begin{comment}
%-------------------------------------------------------------------------------------------------------------------------------------
\subsection{\ziqi{Simiarity Comparison for Pocket-based Molecule Generation}}
%-------------------------------------------------------------------------------------------------------------------------------------


\input{tables/docking_results_similarity}
%\label{tbl:docking_results_similarity}

\bo{@Ziqi you may want to check my edits for the discussion in Table 1 first.
%
If the pocket if known, do you still care about the shape similarity in real applications?
}

\ziqi{Table~\ref{tbl:docking_results_similarity} presents the overall comparison on similarity-based metrics between \methodwithpguide, \methodwithsandpguide and other baselines under different graph similarity constraints  ($\delta_g$=1.0, 0.7, 0.5, 0.3), similar to Table~\ref{tbl:overall}. 
%
As shown in Table~\ref{tbl:docking_results_similarity}, regarding desirable molecules,  \methodwithsandpguide consistently outperforms all the baseline methods in the likelihood of generating desirable molecules (i.e., $\#d\%$).
%
For example, when $\delta_g$=1.0, at $\#d\%$, \methodwithsandpguide (45.2\%) demonstrates significant improvement of $21.2\%$ compared to the best baseline \decompdiff (37.3\%).
%
In terms of $\diversity_d$, \methodwithpguide and \methodwithsandpguide also achieve the second and the third best performance. 
%
Note that the best baseline \targetdiff in $\diversity_d$ achieves the least percentage of desirable molecules (7.1\%), substantially lower than \methodwithpguide and \methodwithsandpguide.
%
This makes its diversity among desirable molecules incomparable with other methods. 
%
When $\delta_g$=0.7, 0.5, and 0.3, \methodwithsandpguide also establishes a significant improvement of 24.3\%, 27.8\%, and 31.1\% compared to the best baseline method \decompdiff.
%
It is also worth noting that the state-of-the-art baseline \decompdiff underperforms \methodwithpguide and \methodwithsandpguide in binding affinities as shown in Table~\ref{tbl:overall_results_docking}, even though it outperforms \methodwithpguide in \#d\%.
%
\methodwithpguide and \methodwithsandpguide also achieve the second and the third best performance in $\diversity_d$ at $\delta_g$=0.7, 0.5, and 0.3. 
%
The superior performance of \methodwithpguide and \methodwithsandpguide in $\#d\%$ at small $\delta_g$ indicates their strong capacity in generating desirable molecules of novel graph structures, thereby facilitating the discovery of novel drug candidates.
%
}

\ziqi{Apart from the desirable molecules, \methodwithpguide and \methodwithsandpguide also demonstrate outstanding performance in terms of the average shape similarities (\avgshapesim) and the average graph similarities (\avggraphsim).
%
Specifically, when $\delta_g$=1.0, \methodwithsandpguide achieves a significant 2.5\% improvement in \avgshapesim\ over the best baseline \decompdiff. 
%
In terms of \avggraphsim, \methodwithsandpguide also achieves higher performance than the baseline \decompdiff of the highest \avgshapesim (0.265 vs 0.282).
%
Please note that all the baseline methods except \decompdiff achieve substantially lower performance in \avgshapesim than \methodwithpguide and \methodwithsandpguide, even though these methods achieve higher \avggraphsim values.
%
This trend remains consistent when applying various similarity constraints (i.e., $\delta_g$) as shown in Table~\ref{tbl:overall_results_docking}.
}

\ziqi{Similarly, \methodwithpguide and \methodwithsandpguide also achieve superior performance in \maxshapesim and \maxgraphsim.
%
Specifically, when $\delta_g$=1.0, for \maxshapesim, \methodwithsandpguide achieves highly comparable performance in \maxshapesim\ compared to the best baseline \decompdiff (0.876 vs 0.878).
%
We also note that \methodwithsandpguide achieves lower \maxgraphsim\ than the \decompdiff with 23.0\% difference. 
%
When $\delta_g$ gets smaller from 0.7 to 0.3, \methodwithsandpguide maintains a high \maxshapesim value around 0.876, while the best baseline \decompdiff has \maxshapesim decreased from 0.878 to 0.854.
%
This demonstrates the superior ability of \methodwithsandpguide in generating molecules with similar shapes and novel structures.
%
}

\ziqi{
In terms of \#n\%, when $\delta_g$=1.0, the percentage of molecules with \graphsim below $\delta_g$ can be interpreted as the percentage of valid molecules among all the generated molecules. 
%
As shown in Table~\ref{tbl:docking_results_similarity}, \methodwithpguide and \methodwithsandpguide are able to generate 98.1\% and 97.8\% of valid molecules, slightly below the best baseline \pockettwomol (98.3\%). 
%
When $\delta_g$=0.7, 0.5, or 0.3, all the methods, including \methodwithpguide and \methodwithsandpguide, can consistently find a sufficient number of novel molecules that meet the graph similarity constraints.
%
The only exception is \decompdiff, which substantially underperforms all the other methods in \#n\%.
}
\end{comment}

%%%%%%%%%%%%%%%%%%%%%%%%%%%%%%%%%%%%%%%%%%%%%
\section{Properties of Molecules in Case Studies for Targets}
\label{supp:app:results:properties}
%%%%%%%%%%%%%%%%%%%%%%%%%%%%%%%%%%%%%%%%%%%%%

%-------------------------------------------------------------------------------------------------------------------------------------
\subsection{Drug Properties of Generated Molecules}
\label{supp:app:results:properties:drug}
%-------------------------------------------------------------------------------------------------------------------------------------

Table~\ref{tbl:drug_property} presents the drug properties of three generated molecules: NL-001, NL-002, and NL-003.
%
As shown in Table~\ref{tbl:drug_property}, each of these molecules has a favorable profile, making them promising drug candidates. 
%
{As discussed in Section ``Case Studies for Targets'' in the main manuscript, all three molecules have high binding affinities in terms of Vina S, Vina M and Vina D, and favorable QED and SA values.
%
In addition, all of them meet the Lipinski's rule of five criteria~\cite{Lipinski1997}.}
%
In terms of physicochemical properties, all these properties of NL-001, NL-002 and NL-003, including number of rotatable bonds, molecule weight, LogP value, number of hydrogen bond doners and acceptors, and molecule charges, fall within the desired range of drug molecules. 
%
This indicates that these molecules could potentially have good solubility and membrane permeability, essential qualities for effective drug absorption.

These generated molecules also demonstrate promising safety profiles based on the predictions from ICM~\cite{Neves2012}.
%
In terms of drug-induced liver injury prediction scores, all three molecules have low scores (0.188 to 0.376), indicating a minimal risk of hepatotoxicity. 
%
NL-001 and NL-002 fall under `Ambiguous/Less concern' for liver injury, while NL-003 is categorized under 'No concern' for liver injury. 
%
Moreover, all these molecules have low toxicity scores (0.000 to 0.236). 
%
NL-002 and NL-003 do not have any known toxicity-inducing functional groups. 
%
NL-001 and NL-003 are also predicted not to include any known bad groups that lead to inappropriate features.
%
These attributes highlight the potential of NL-001, NL-002, and NL-003 as promising treatments for cancers and Alzheimer’s disease.

%\input{tables/binding_generated_mols}
%\label{tbl:binding_drug_mols}

\input{tables/drug_property_generated_mols}
%\label{tbl:drug_property}

%-------------------------------------------------------------------------------------------------------------------------------------
\subsection{Comparison on ADMET Profiles between Generated Molecules and Approved Drugs}
\label{supp:app:results:properties:admet}
%-------------------------------------------------------------------------------------------------------------------------------------

\paragraph{Generated Molecules for CDK6}
%
Table~\ref{tbl:admet_cdk6} presents the comparison on ADMET profiles between two generated molecules for CDK6 and the approved CDK6 inhibitors, including Abemaciclib~\cite{Patnaik2016}, Palbociclib~\cite{Lu2015}, and Ribociclib~\cite{Tripathy2017}.
%
As shown in Table~\ref{tbl:admet_cdk6}, the generated molecules, NL-001 and NL-002, exhibit comparable ADMET profiles with those of the approved CDK6 inhibitors. 
%
Importantly, both molecules demonstrate good potential in most crucial properties, including Ames mutagenesis, favorable oral toxicity, carcinogenicity, estrogen receptor binding, high intestinal absorption and favorable oral bioavailability.
%
Although the generated molecules are predicted as positive in hepatotoxicity and mitochondrial toxicity, all the approved drugs are also predicted as positive in these two toxicity.
%
This result suggests that these issues might stem from the limited prediction accuracy rather than being specific to our generated molecules.
%
Notably, NL-001 displays a potentially better plasma protein binding score compared to other molecules, which may improve its distribution within the body. 
%
Overall, these results indicate that NL-001 and NL-002 could be promising candidates for further drug development.


\input{tables/admet_property_cdk6}
%\label{tbl:admet_cdk6}

\paragraph{Generated Molecule for NEP}
%
Table~\ref{tbl:admet_nep} presents the comparison on ADMET profiles between a generated molecule for NEP targeting Alzheimer's disease and three approved drugs, Donepezil, Galantamine, and Rivastigmine, for Alzheimer's disease~\cite{Hansen2008}.
%
Overall, NL-003 exhibits a comparable ADMET profile with the three approved drugs.  
%
Notably, same as other approved drugs, NL-003 is predicted to be able to penetrate the blood brain barrier, a crucial property for Alzheimer's disease.
%  
In addition, it demonstrates a promising safety profile in terms of Ames mutagenesis, favorable oral toxicity, carcinogenicity, estrogen receptor binding, high intestinal absorption, nephrotoxicity and so on.
%
These results suggest that NL-003 could be promising candidates for the drug development of Alzheimer's disease.

\input{tables/admet_property_nep}
%\label{tbl:admet_nep}

\clearpage
%%%%%%%%%%%%%%%%%%%%%%%%%%%%%%%%%%%%%%%%%%%%%
\section{Algorithms}
\label{supp:algorithms}
%%%%%%%%%%%%%%%%%%%%%%%%%%%%%%%%%%%%%%%%%%%%%

Algorithm~\ref{alg:shapemol} describes the molecule generation process of \method.
%
Given a known ligand \molx, \method generates a novel molecule \moly that has a similar shape to \molx and thus potentially similar binding activity.
%
\method can also take the protein pocket \pocket as input and adjust the atoms of generated molecules for optimal fit and improved binding affinities.
%
Specifically, \method first calculates the shape embedding \shapehiddenmat for \molx using the shape encoder \SEE described in Algorithm~\ref{alg:see_shaperep}.
%
Based on \shapehiddenmat, \method then generates a novel molecule with a similar shape to \molx using the diffusion-based generative model \methoddiff as in Algorithm~\ref{alg:diffgen}.
%
During generation, \method can use shape guidance to directly modify the shape of \moly to closely resemble the shape of \molx.
%
When the protein pocket \pocket is available, \method can also use pocket guidance to ensure that \moly is specifically tailored to closely fit within \pocket.

\input{algorithms/shapemol}
%\label{alg:shapemol}

\input{algorithms/shaperep}
%\label{alg:see_shaperep}

\input{algorithms/diffgen}
%\label{alg:diffgen}

%\input{algorithms/train_SE}
%\label{alg:train_se}

%\input{algorithms/train_diff}
%\label{alg:train_diff}

%---------------------------------------------------------------------------------------------------------------------
\section{{Equivariance and Invariance}}
\label{supp:ei}
%---------------------------------------------------------------------------------------------------------------------

%.................................................................................................
\subsection{Equivariance}
\label{supp:ei:equivariance}
%.................................................................................................

{Equivariance refers to the property of a function $f(\pos)$ %\bo{is it the property of the function or embedding (x)?} 
that any translation and rotation transformation from the special Euclidean group SE(3)~\cite{Atz2021} applied to a geometric object
$\pos\in\mathbb{R}^3$ is mirrored in the output of $f(\pos)$, accordingly.
%
This property ensures $f(\pos)$ to learn a consistent representation of an object's geometric information, regardless of its orientation or location in 3D space.
%
%As a result, it provides $f(\pos)$ better generalization capabilities~\cite{Jonas20a}.
%
Formally, given any translation transformation $\mathbf{t}\in\mathbb{R}^3$ and rotation transformation $\mathbf{R}\in\mathbb{R}^{3\times3}$ ($\mathbf{R}^{\mathsf{T}}\mathbf{R}=\mathbb{I}$), %\xia{change the font types for $^{\mathsf{T}}$ and $\mathbb{I}$ in the entire manuscript}), 
$f(\pos)$ is equivariant with respect to these transformations %$g$ (\bo{where is $g$...})
if it satisfies
\begin{equation}
f(\mathbf{R}\pos+\mathbf{t}) = \mathbf{R}f(\pos) + \mathbf{t}. %\ \text{where}\ \hiddenpos = f(\pos).
\end{equation}
%
%where $\hiddenpos=f(\pos)$ is the output of $\pos$. 
%
In \method, both \SE and \methoddiff are developed to guarantee equivariance in capturing the geometric features of objects regardless of any translation or rotation transformations, as will be detailed in the following sections.
}

%.................................................................................................
\subsection{Invariance}
\label{supp:ei:invariance}
%.................................................................................................

%In contrast to equivariance, 
Invariance refers to the property of a function that its output {$f(\pos)$} remains constant under any translation and rotation transformations of the input $\pos$. %a geometric object's feature $\pos$.
%
This property enables $f(\pos)$ to accurately capture %a geometric object's 
the inherent features (e.g., atom features for 3D molecules) that are invariant of its orientation or position in 3D space.
%
Formally, $f(\pos)$ is invariant under any translation $\mathbf{t}$ and  rotation $\mathbf{R}$ if it satisfies
%
\begin{equation}
f(\mathbf{R}\pos+\mathbf{t}) = f(\pos).
\end{equation}
%
In \method, both \SE and \methoddiff capture the inherent features of objects in an invariant way, regardless of any translation or rotation transformations, as will be detailed in the following sections.

%%%%%%%%%%%%%%%%%%%%%%%%%%%%%%%%%%%%%%%%%%%%%
\section{Point Cloud Construction}
\label{supp:point_clouds}
%%%%%%%%%%%%%%%%%%%%%%%%%%%%%%%%%%%%%%%%%%%%%

In \method, we represented molecular surface shapes using point clouds (\pc).
%
$\pc$
serves as input to \SE, from which we derive shape latent embeddings.
%
To generate $\pc$, %\bo{\st{create this}}, \bo{generate $\pc$}
we initially generated a molecular surface mesh using the algorithm from the Open Drug Discovery Toolkit~\cite{Wjcikowski2015oddt}.
%
Following this, we uniformly sampled points on the mesh surface with probability proportional to the face area, %\xia{how to uniformly?}, ensuring the sampling is done proportionally to the face area, with
using the algorithm from PyTorch3D~\cite{ravi2020pytorch3d}.
%
This point cloud $\pc$ is then centralized by setting the center of its points to zero.
%
%

%%%%%%%%%%%%%%%%%%%%%%%%%%%%%%%%%%%%%%%%%%%%%
\section{Query Point Sampling}
\label{supp:training:shapeemb}
%%%%%%%%%%%%%%%%%%%%%%%%%%%%%%%%%%%%%%%%%%%%%

As described in Section ``Shape Decoder (\SED)'', the signed distances of query points $z_q$ to molecule surface shape $\pc$ are used to optimize \SE.
%
In this section, we present how to sample these points $z_q$ in 3D space.
%
Particularly, we first determined the bounding box around the molecular surface shape, using the maximum and minimum \mbox{($x$, $y$, $z$)-axis} coordinates for points in our point cloud \pc,
denoted as $(x_\text{min}, y_\text{min}, z_\text{min})$ and $(x_\text{max}, y_\text{max}, z_\text{max})$.
%
We extended this box slightly by defining its corners as \mbox{$(x_\text{min}-1, y_\text{min}-1, z_\text{min}-1)$} and \mbox{$(x_\text{max}+1, y_\text{max}+1, z_\text{max}+1)$}.
%
For sampling $|\mathcal{Z}|$ query points, we wanted an even distribution of points inside and outside the molecule surface shape.
%
%\ziqi{Typically, within this bounding box, molecules occupy only a small portion of volume, which makes it more likely to sample
%points outside the molecule surface shape.}
%
When a bounding box is defined around the molecule surface shape, there could be a lot of empty spaces within the box that the molecule does not occupy due to 
its complex and irregular shape.
%
This could lead to that fewer points within the molecule surface shape could be sampled within the box.
%
Therefore, we started by randomly sampling $3k$ points within our bounding box to ensure that there are sufficient points within the surface.
%
We then determined whether each point lies within the molecular surface, using an algorithm from Trimesh~\footnote{https://trimsh.org/} based on the molecule surface mesh.
%
If there are $n_w$ points found within the surface, we selected $n=\min(n_w, k/2)$ points from these points, 
and randomly choose the remaining 
%\bo{what do you mean by remaining? If all the 3k sampled points are inside the surface, you get no points left.} 
$k-n$ points 
from those outside the surface.
%
For each query point, we determined its signed distance to the molecule surface by its closest distance to points in \pc with a sign indicating whether it is inside the surface.

%%%%%%%%%%%%%%%%%%%%%%%%%%%%%%%%%%%%%%%%%%%%%
\section{Forward Diffusion (\diffnoise)}
\label{supp:forward}
%%%%%%%%%%%%%%%%%%%%%%%%%%%%%%%%%%%%%%%%%%%%%

%===================================================================
\subsection{{Forward Process}}
\label{supp:forward:forward}
%===================================================================

Formally, for atom positions, the probability of $\pos_t$ sampled given $\pos_{t-1}$, denoted as $q(\pos_t|\pos_{t-1})$, is defined as follows,
%\xia{revise the representation, should be $\beta^x_t$ -- note the space} as follows,
%
\begin{equation}
q(\pos_t|\pos_{t-1}) = \mathcal{N}(\pos_t|\sqrt{1-\beta^{\mathtt{x}}_t}\pos_{t-1}, \beta^{\mathtt{x}}_t\mathbb{I}), 
\label{eqn:noiseposinter}
\end{equation}
%
%\xia{should be a comma after the equation. you also missed it. }
%\st{in which} 
where %\hl{$\pos_0$ denotes the original atom position;} \xia{no $\pos_0$ in the equation...}
%$\mathbf{I}$ denotes the identity matrix;
$\mathcal{N}(\cdot)$ is a Gaussian distribution of $\pos_t$ with mean $\sqrt{1-\beta_t^{\mathtt{x}}}\pos_{t-1}$ and covariance $\beta_t^{\mathtt{x}}\mathbf{I}$.
%\xia{what is $\mathcal{N}$? what is $q$? you abused $q$. need to be crystal clear... }
%\bo{Should be $\sim$ not $=$ in the equation}
%
Following Hoogeboom \etal~\cite{hoogeboom2021catdiff}, 
%the forward process for the discrete atom feature $\atomfeat_t\in\mathbb{R}^K$ adds 
%categorical noise into $\atomfeat_{t-1}$ according to a variance schedule $\beta_t^v\in (0, 1)$. %as follows, %\hl{$\betav_t\in (0, 1)$} as follows,
%\xia{presentation...check across the entire manuscript... } as follows,
%
%\ziqi{Formally, 
for atom features, the probability of $\atomfeat_t$ across $K$ classes given $\atomfeat_{t-1}$ is defined as follows,
%
\begin{equation}
q(\atomfeat_t|\atomfeat_{t-1}) = \mathcal{C}(\atomfeat_t|(1-\beta^{\mathtt{v}}_t) \atomfeat_{t-1}+\beta^{\mathtt{v}}_t\mathbf{1}/K),
\label{eqn:noisetypeinter}
\end{equation}
%
where %\hl{$\atomfeat_0$ denotes the original atom positions}; 
$\mathcal{C}$ is a categorical distribution of $\atomfeat_t$ derived from the %by 
noising $\atomfeat_{t-1}$ with a uniform noise $\beta^{\mathtt{v}}_t\mathbf{1}/K$ across $K$ classes.
%adding an uniform noise $\beta^v_t$ to $\atomfeat_{t-1}$ across K classes.
%\xia{there is always a comma or period after the equations. Equations are part of a sentence. you always missed it. }
%\xia{what is $\mathcal{C}$? what does $q$ mean? it is abused. }

Since the above distributions form Markov chains, %} \xia{grammar!}, 
the probability of any $\pos_t$ or $\atomfeat_t$ can be derived from $\pos_0$ or $\atomfeat_0$:
%samples $\mol_0$ as follows,
%
\begin{eqnarray}
%\begin{aligned}
& q(\pos_t|\pos_{0}) & = \mathcal{N}(\pos_t|\sqrt{\cumalpha^{\mathtt{x}}_t}\pos_0, (1-\cumalpha^{\mathtt{x}}_t)\mathbb{I}), \label{eqn:noisepos}\\
& q(\atomfeat_t|\atomfeat_0)  & = \mathcal{C}(\atomfeat_t|\cumalpha^{\mathtt{v}}_t\atomfeat_0 + (1-\cumalpha^{\mathtt{v}}_t)\mathbf{1}/K), \label{eqn:noisetype}\\
& \text{where }\cumalpha^{\mathtt{u}}_t & = \prod\nolimits_{\tau=1}^{t}\alpha^{\mathtt{u}}_\tau, \ \alpha^{\mathtt{u}}_\tau=1 - \beta^{\mathtt{u}}_\tau, \ {\mathtt{u}}={\mathtt{x}} \text{ or } {\mathtt{v}}.\;\;\;\label{eqn:noiseschedule}
%\end{aligned}
\label{eqn:pos_prior}
\end{eqnarray}
%\xia{always punctuations after equations!!! also use ``eqnarray" instead of ``equation" + ``aligned" for multiple equations, each
%with a separate reference numbering...}
%\st{in which}, 
%where \ziqi{$\cumalpha^u_t = \prod_{\tau=1}^{t}\alpha^u_\tau$ and $\alpha^u_\tau=1 - \beta^u_\tau$ ($u$=$x$ or $v$)}.
%\xia{no such notations in the above equations; also subscript $s$ is abused with shape};
%$K$ is the number of categories for atom features.
%
%The details about noise schedules $\beta^x_t$ and $\beta^v_t$ are available in Supplementary Section \ref{XXX}. \ziqi{add trend}
%
Note that $\bar{\alpha}^{\mathtt{u}}_t$ ($\mathtt{u}={\mathtt{x}}\text{ or }{\mathtt{v}}$)
%($u$=$x$ or $v$) 
is monotonically decreasing from 1 to 0 over $t=[1,T]$. %\xia{=???}. 
%
As $t\rightarrow 1$, $\cumalpha^{\mathtt{x}}_t$ and $\cumalpha^{\mathtt{v}}_t$ are close to 1, leading to that $\pos_t$ or $\atomfeat_t$ approximates 
%the original data 
$\pos_0$ or $\atomfeat_0$.
%
Conversely, as  $t\rightarrow T$, $\cumalpha^{\mathtt{x}}_t$ and $\cumalpha^{\mathtt{v}}_t$ are close to 0,
leading to that $q(\pos_T|\pos_{0})$ %\st{$\rightarrow \mathcal{N}(\mathbf{0}, \mathbf{I})$} 
resembles  {$\mathcal{N}(\mathbf{0}, \mathbb{I})$} 
and $q(\atomfeat_T|\atomfeat_0)$ %\st{$\rightarrow \mathcal{C}(\mathbf{I}/K)$} 
resembles {$\mathcal{C}(\mathbf{1}/K)$}.

Using Bayes theorem, the ground-truth Normal posterior of atom positions $p(\pos_{t-1}|\pos_t, \pos_0)$ can be calculated in a
closed form~\cite{ho2020ddpm} as below,
%
\begin{eqnarray}
& p(\pos_{t-1}|\pos_t, \pos_0) = \mathcal{N}(\pos_{t-1}|\mu(\pos_t, \pos_0), \tilde{\beta}^\mathtt{x}_t\mathbb{I}), \label{eqn:gt_pos_posterior_1}\\
&\!\!\!\!\!\!\!\!\!\!\!\mu(\pos_t, \pos_0)\!=\!\frac{\sqrt{\bar{\alpha}^{\mathtt{x}}_{t-1}}\beta^{\mathtt{x}}_t}{1-\bar{\alpha}^{\mathtt{x}}_t}\pos_0\!+\!\frac{\sqrt{\alpha^{\mathtt{x}}_t}(1-\bar{\alpha}^{\mathtt{x}}_{t-1})}{1-\bar{\alpha}^{\mathtt{x}}_t}\pos_t, 
\tilde{\beta}^\mathtt{x}_t\!=\!\frac{1-\bar{\alpha}^{\mathtt{x}}_{t-1}}{1-\bar{\alpha}^{\mathtt{x}}_{t}}\beta^{\mathtt{x}}_t.\;\;\;
\end{eqnarray}
%
%\xia{Ziqi, please double check the above two equations!}
Similarly, the ground-truth categorical posterior of atom features $p(\atomfeat_{t-1}|\atomfeat_{t}, \atomfeat_0)$ can be calculated~\cite{hoogeboom2021catdiff} as below,
%
\begin{eqnarray}
& p(\atomfeat_{t-1}|\atomfeat_{t}, \atomfeat_0) = \mathcal{C}(\atomfeat_{t-1}|\mathbf{c}(\atomfeat_t, \atomfeat_0)), \label{eqn:gt_atomfeat_posterior_1}\\
& \mathbf{c}(\atomfeat_t, \atomfeat_0) = \tilde{\mathbf{c}}/{\sum_{k=1}^K \tilde{c}_k}, \label{eqn:gt_atomfeat_posterior_2} \\
& \tilde{\mathbf{c}} = [\alpha^{\mathtt{v}}_t\atomfeat_t + \frac{1 - \alpha^{\mathtt{v}}_t}{K}]\odot[\bar{\alpha}^{\mathtt{v}}_{t-1}\atomfeat_{0}+\frac{1-\bar{\alpha}^{\mathtt{v}}_{t-1}}{K}], 
\label{eqn:gt_atomfeat_posterior_3}
%\label{eqn:atomfeat_posterior}
\end{eqnarray}
%
%\xia{Ziqi: please double check the above equations!}
%
where $\tilde{c}_k$ denotes the likelihood of $k$-th class across $K$ classes in $\tilde{\mathbf{c}}$; 
$\odot$ denotes the element-wise product operation;
$\tilde{\mathbf{c}}$ is calculated using $\atomfeat_t$ and $\atomfeat_{0}$ and normalized into $\mathbf{c}(\atomfeat_t, \atomfeat_0)$ so as to represent
probabilities. %\xia{is this correct? is $\tilde{c}_k$ always greater than 0?}
%\xia{how is it calculated?}.
%\ziqi{the likelihood distribution $\tilde{c}$ is calculated by $p(\atomfeat_t|\atomfeat_{t-1})p(\atomfeat_{t-1}|\atomfeat_0)$, according to 
%Equation~\ref{eqn:noisetypeinter} and \ref{eqn:noisetype}.
%\xia{need to write the key idea of the above calculation...}
%
The proof of the above equations is available in Supplementary Section~\ref{supp:forward:proof}.

%===================================================================
\subsection{Variance Scheduling in \diffnoise}
\label{supp:forward:variance}
%===================================================================

Following Guan \etal~\cite{guan2023targetdiff}, we used a sigmoid $\beta$ schedule for the variance schedule $\beta_t^{\mathtt{x}}$ of atom coordinates as below,

\begin{equation}
\beta_t^{\mathtt{x}} = \text{sigmoid}(w_1(2 t / T - 1)) (w_2 - w_3) + w_3
\end{equation}
in which $w_i$($i$=1,2, or 3) are hyperparameters; $T$ is the maximum step.
%
We set $w_1=6$, $w_2=1.e-7$ and $w_3=0.01$.
%
For atom types, we used a cosine $\beta$ schedule~\cite{nichol2021} for $\beta_t^{\mathtt{v}}$ as below,

\begin{equation}
\begin{aligned}
& \bar{\alpha}_t^{\mathtt{v}} = \frac{f(t)}{f(0)}, f(t) = \cos(\frac{t/T+s}{1+s} \cdot \frac{\pi}{2})^2\\
& \beta_t^{\mathtt{v}} = 1 - \alpha_t^{\mathtt{v}} = 1 - \frac{\bar{\alpha}_t^{\mathtt{v}} }{\bar{\alpha}_{t-1}^{\mathtt{v}} }
\end{aligned}
\end{equation}
in which $s$ is a hyperparameter and set as 0.01.

As shown in Section ``Forward Diffusion Process'', the values of $\beta_t^{\mathtt{x}}$ and $\beta_t^{\mathtt{v}}$ should be 
sufficiently small to ensure the smoothness of forward diffusion process. In the meanwhile, their corresponding $\bar{\alpha}_t$
values should decrease from 1 to 0 over $t=[1,T]$.
%
Figure~\ref{fig:schedule} shows the values of $\beta_t$ and $\bar{\alpha}_t$ for atom coordinates and atom types with our hyperparameters.
%
Please note that the value of $\beta_{t}^{\mathtt{x}}$ is less than 0.1 for 990 out of 1,000 steps. %\bo{\st{, though it increases when $t$ is close to 1,000}}.
%
This guarantees the smoothness of the forward diffusion process.
%\bo{add $\beta_t^{\mathtt{x}}$ and $\beta_t^{\mathtt{v}}$ in the legend of the figure...}
%\bo{$\beta_t^{\mathtt{v}}$ does not look small when $t$ is close to 1000...}

\begin{figure}
	\begin{subfigure}[t]{.45\linewidth}
		\centering
		\includegraphics[width=.7\linewidth]{figures/var_schedule_beta.pdf}
	\end{subfigure}
	%
	\hfill
	\begin{subfigure}[t]{.45\linewidth}
		\centering
		\includegraphics[width=.7\linewidth]{figures/var_schedule_alpha.pdf}
	\end{subfigure}
	\caption{Schedule}
	\label{fig:schedule}
\end{figure}

%===================================================================
\subsection{Derivation of Forward Diffusion Process}
\label{supp:forward:proof}
%===================================================================

In \method, a Gaussian noise and a categorical noise are added to continuous atom position and discrete atom features, respectively.
%
Here, we briefly describe the derivation of posterior equations (i.e., Eq.~\ref{eqn:gt_pos_posterior_1}, and   \ref{eqn:gt_atomfeat_posterior_1}) for atom positions and atom types in our work.
%
We refer readers to Ho \etal~\cite{ho2020ddpm} and Kong \etal~\cite{kong2021diffwave} %\bo{add XXX~\etal here...} \cite{ho2020ddpm,kong2021diffwave} 	
for a detailed description of diffusion process for continuous variables and Hoogeboom \etal~\cite{hoogeboom2021catdiff} for
%\bo{add XXX~\etal here...} \cite{hoogeboom2021catdiff} for
the description of diffusion process for discrete variables.

For continuous atom positions, as shown in Kong \etal~\cite{kong2021diffwave}, according to Bayes theorem, given $q(\pos_t|\pos_{t-1})$ defined in Eq.~\ref{eqn:noiseposinter} and 
$q(\pos_t|\pos_0)$ defined in Eq.~\ref{eqn:noisepos}, the posterior $q(\pos_{t-1}|\pos_{t}, \pos_0)$ is derived as below (superscript $\mathtt{x}$ is omitted for brevity),

\begin{equation}
\begin{aligned}
& q(\pos_{t-1}|\pos_{t}, \pos_0)  = \frac{q(\pos_t|\pos_{t-1}, \pos_0)q(\pos_{t-1}|\pos_0)}{q(\pos_t|\pos_0)} \\
& =  \frac{\mathcal{N}(\pos_t|\sqrt{1-\beta_t}\pos_{t-1}, \beta_{t}\mathbf{I}) \mathcal{N}(\pos_{t-1}|\sqrt{\bar{\alpha}_{t-1}}\pos_{0}, (1-\bar{\alpha}_{t-1})\mathbf{I}) }{ \mathcal{N}(\pos_{t}|\sqrt{\bar{\alpha}_t}\pos_{0}, (1-\bar{\alpha}_t)\mathbf{I})}\\
& =  (2\pi{\beta_t})^{-\frac{3}{2}} (2\pi{(1-\bar{\alpha}_{t-1})})^{-\frac{3}{2}} (2\pi(1-\bar{\alpha}_t))^{\frac{3}{2}} \times \exp( \\
& -\frac{\|\pos_t - \sqrt{\alpha}_t\pos_{t-1}\|^2}{2\beta_t} -\frac{\|\pos_{t-1} - \sqrt{\bar{\alpha}}_{t-1}\pos_{0} \|^2}{2(1-\bar{\alpha}_{t-1})} \\
& + \frac{\|\pos_t - \sqrt{\bar{\alpha}_t}\pos_0\|^2}{2(1-\bar{\alpha}_t)}) \\
& = (2\pi\tilde{\beta}_t)^{-\frac{3}{2}} \exp(-\frac{1}{2\tilde{\beta}_t}\|\pos_{t-1}-\frac{\sqrt{\bar{\alpha}_{t-1}}\beta_t}{1-\bar{\alpha}_t}\pos_0 \\
& - \frac{\sqrt{\alpha_t}(1-\bar{\alpha}_{t-1})}{1-\bar{\alpha}_t}\pos_{t}\|^2) \\
& \text{where}\ \tilde{\beta}_t = \frac{1-\bar{\alpha}_{t-1}}{1-\bar{\alpha}_t}\beta_t.
\end{aligned}
\end{equation}
%\bo{marked part does not look right to me.}
%\bo{How to you derive from the second equation to the third one?}

Therefore, the posterior of atom positions is derived as below,

\begin{equation}
q(\pos_{t-1}|\pos_{t}, \pos_0)\!\!=\!\!\mathcal{N}(\pos_{t-1}|\frac{\sqrt{\bar{\alpha}_{t-1}}\beta_t}{1-\bar{\alpha}_t}\pos_0 + \frac{\sqrt{\alpha_t}(1-\bar{\alpha}_{t-1})}{1-\bar{\alpha}_t}\pos_{t}, \tilde{\beta}_t\mathbf{I}).
\end{equation}

For discrete atom features, as shown in Hoogeboom \etal~\cite{hoogeboom2021catdiff} and Guan \etal~\cite{guan2023targetdiff},
according to Bayes theorem, the posterior $q(\atomfeat_{t-1}|\atomfeat_{t}, \atomfeat_0)$ is derived as below (supperscript $\mathtt{v}$ is omitted for brevity),

\begin{equation}
\begin{aligned}
& q(\atomfeat_{t-1}|\atomfeat_{t}, \atomfeat_0) =  \frac{q(\atomfeat_t|\atomfeat_{t-1}, \atomfeat_0)q(\atomfeat_{t-1}|\atomfeat_0)}{\sum_{\scriptsize{\atomfeat}_{t-1}}q(\atomfeat_t|\atomfeat_{t-1}, \atomfeat_0)q(\atomfeat_{t-1}|\atomfeat_0)} \\
%& = \frac{\mathcal{C}(\atomfeat_t|(1-\beta_t)\atomfeat_{t-1} + \beta_t\frac{\mathbf{1}}{K}) \mathcal{C}(\atomfeat_{t-1}|\bar{\alpha}_{t-1}\atomfeat_0+(1-\bar{\alpha}_{t-1})\frac{\mathbf{1}}{K})} \\
\end{aligned}
\end{equation}

For $q(\atomfeat_t|\atomfeat_{t-1}, \atomfeat_0)$, we have % $\atomfeat_t=\atomfeat_{t-1}$ with probability $1-\beta_t+\beta_t / K$, and $\atomfeat_t \neq \atomfeat_{t-1}$
%with probability $\beta_t / K$.
%
%Therefore, this function can be symmetric, that is, 
%
\begin{equation}
\begin{aligned}
q(\atomfeat_t|\atomfeat_{t-1}, \atomfeat_0) & = \mathcal{C}(\atomfeat_t|(1-\beta_t)\atomfeat_{t-1} + \beta_t/{K})\\
& = \begin{cases}
1-\beta_t+\beta_t/K,\!&\text{when}\ \atomfeat_{t} = \atomfeat_{t-1},\\
\beta_t / K,\! &\text{when}\ \atomfeat_{t} \neq \atomfeat_{t-1},
\end{cases}\\
& = \mathcal{C}(\atomfeat_{t-1}|(1-\beta_t)\atomfeat_{t} + \beta_t/{K}).
\end{aligned}
%\mathcal{C}(\atomfeat_{t-1}|(1-\beta_{t})\atomfeat_{t} + \beta_t\frac{\mathbf{1}}{K}).
\end{equation}
%
Therefore, we have
%\bo{why it can be symmetric}
%
\begin{equation}
\begin{aligned}
& q(\atomfeat_t|\atomfeat_{t-1}, \atomfeat_0)q(\atomfeat_{t-1}|\atomfeat_0) \\
& = \mathcal{C}(\atomfeat_{t-1}|(1-\beta_t)\atomfeat_{t} + \beta_t\frac{\mathbf{1}}{K}) \mathcal{C}(\atomfeat_{t-1}|\bar{\alpha}_{t-1}\atomfeat_0+(1-\bar{\alpha}_{t-1})\frac{\mathbf{1}}{K}) \\
& = [\alpha_t\atomfeat_t + \frac{1 - \alpha_t}{K}]\odot[\bar{\alpha}_{t-1}\atomfeat_{0}+\frac{1-\bar{\alpha}_{t-1}}{K}].
\end{aligned}
\end{equation}
%
%\bo{what is $\tilde{\mathbf{c}}$...}
Therefore, with $q(\atomfeat_t|\atomfeat_{t-1}, \atomfeat_0)q(\atomfeat_{t-1}|\atomfeat_0) = \tilde{\mathbf{c}}$, the posterior is as below,

\begin{equation}
q(\atomfeat_{t-1}|\atomfeat_{t}, \atomfeat_0) = \mathcal{C}(\atomfeat_{t-1}|\mathbf{c}(\atomfeat_t, \atomfeat_0)) = \frac{\tilde{\mathbf{c}}}{\sum_{k}^K\tilde{c}_k}.
\end{equation}

%%%%%%%%%%%%%%%%%%%%%%%%%%%%%%%%%%%%%%%%%%%%%
\section{{Backward Generative Process} (\diffgenerative)}
\label{supp:backward}
%%%%%%%%%%%%%%%%%%%%%%%%%%%%%%%%%%%%%%%%%%%%%

Following Ho \etal~\cite{ho2020ddpm}, with $\tilde{\pos}_{0,t}$, the probability of $\pos_{t-1}$ denoised from $\pos_t$, denoted as $p(\pos_{t-1}|\pos_t)$,
can be estimated %\hl{parameterized} \xia{???} 
by the approximated posterior $p_{\boldsymbol{\Theta}}(\pos_{t-1}|\pos_t, \tilde{\pos}_{0,t})$ as below,
%
\begin{equation}
\begin{aligned}
p(\pos_{t-1}|\pos_t) & \approx p_{\boldsymbol{\Theta}}(\pos_{t-1}|\pos_t, \tilde{\pos}_{0,t}) \\
& = \mathcal{N}(\pos_{t-1}|\mu_{\boldsymbol{\Theta}}(\pos_t, \tilde{\pos}_{0,t}),\tilde{\beta}_t^{\mathtt{x}}\mathbb{I}),
\end{aligned}
\label{eqn:aprox_pos_posterior}
\end{equation}
%
where ${\boldsymbol{\Theta}}$ is the learnable parameter; $\mu_{\boldsymbol{\Theta}}(\pos_t, \tilde{\pos}_{0,t})$ is an estimate %estimation
of $\mu(\pos_t, \pos_{0})$ by replacing $\pos_0$ with its estimate $\tilde{\pos}_{0,t}$ 
in Equation~{\ref{eqn:gt_pos_posterior_1}}.
%
Similarly, with $\tilde{\atomfeat}_{0,t}$, the probability of $\atomfeat_{t-1}$ denoised from $\atomfeat_t$, denoted as $p(\atomfeat_{t-1}|\atomfeat_t)$, 
can be estimated %\hl{parameterized} 
by the approximated posterior $p_{\boldsymbol{\Theta}}(\atomfeat_{t-1}|\atomfeat_t, \tilde{\atomfeat}_{0,t})$ as below,
%
\begin{equation}
\begin{aligned}
p(\atomfeat_{t-1}|\atomfeat_t)\approx p_{\boldsymbol{\Theta}}(\atomfeat_{t-1}|\atomfeat_{t}, \tilde{\atomfeat}_{0,t}) 
=\mathcal{C}(\atomfeat_{t-1}|\mathbf{c}_{\boldsymbol{\Theta}}(\atomfeat_t, \tilde{\atomfeat}_{0,t})),\!\!\!\!
\end{aligned}
\label{eqn:aprox_atomfeat_posterior}
\end{equation}
%
where $\mathbf{c}_{\boldsymbol{\Theta}}(\atomfeat_t, \tilde{\atomfeat}_{0,t})$ is an estimate of $\mathbf{c}(\atomfeat_t, \atomfeat_0)$
by replacing $\atomfeat_0$  
with its estimate $\tilde{\atomfeat}_{0,t}$ in Equation~\ref{eqn:gt_atomfeat_posterior_1}.



%===================================================================
\section{\method Loss Function Derivation}
\label{supp:training:loss}
%===================================================================

In this section, we demonstrate that a step weight $w_t^{\mathtt{x}}$ based on the signal-to-noise ratio $\lambda_t$ should be 
included into the atom position loss (Eq.~\ref{eqn:diff:obj:pos}).
%
In the diffusion process for continuous variables, the optimization problem is defined 
as below~\cite{ho2020ddpm},
%
\begin{equation*}
\begin{aligned}
& \arg\min_{\boldsymbol{\Theta}}KL(q(\pos_{t-1}|\pos_t, \pos_0)|p_{\boldsymbol{\Theta}}(\pos_{t-1}|\pos_t, \tilde{\pos}_{0,t})) \\
& = \arg\min_{\boldsymbol{\Theta}} \frac{\bar{\alpha}_{t-1}(1-\alpha_t)}{2(1-\bar{\alpha}_{t-1})(1-\bar{\alpha}_{t})}\|\tilde{\pos}_{0, t}-\pos_0\|^2 \\
& = \arg\min_{\boldsymbol{\Theta}} \frac{1-\alpha_t}{2(1-\bar{\alpha}_{t-1})\alpha_{t}} \|\tilde{\boldsymbol{\epsilon}}_{0,t}-\boldsymbol{\epsilon}_0\|^2,
\end{aligned}
\end{equation*}
where $\boldsymbol{\epsilon}_0 = \frac{\pos_t - \sqrt{\bar{\alpha}_t}\pos_0}{\sqrt{1-\bar{\alpha}_t}}$ is the ground-truth noise variable sampled from $\mathcal{N}(\mathbf{0}, \mathbf{1})$ and is used to sample $\pos_t$ from $\mathcal{N}(\pos_t|\sqrt{\cumalpha_t}\pos_0, (1-\cumalpha_t)\mathbf{I})$ in Eq.~\ref{eqn:noisetype};
$\tilde{\boldsymbol{\epsilon}}_0 = \frac{\pos_t - \sqrt{\bar{\alpha}_t}\tilde{\pos}_{0, t}}{\sqrt{1-\bar{\alpha}_t}}$ is the predicted noise variable. 

%A simplified training objective is proposed by Ho \etal~\cite{ho2020ddpm} as below,
Ho \etal~\cite{ho2020ddpm} further simplified the above objective as below and
demonstrated that the simplified one can achieve better performance:
%
\begin{equation}
\begin{aligned}
& \arg\min_{\boldsymbol{\Theta}} \|\tilde{\boldsymbol{\epsilon}}_{0,t}-\boldsymbol{\epsilon}_0\|^2 \\
& = \arg\min_{\boldsymbol{\Theta}} \frac{\bar{\alpha}_t}{1-\bar{\alpha}_t}\|\tilde{\pos}_{0,t}-\pos_0\|^2,
\end{aligned}
\label{eqn:supp:losspos}
\end{equation}
%
where $\lambda_t=\frac{\bar{\alpha}_t}{1-\bar{\alpha}_t}$ is the signal-to-noise ratio.
%
While previous work~\cite{guan2023targetdiff} applies uniform step weights across
different steps, we demonstrate that a step weight should be included into the atom position loss according to Eq.~\ref{eqn:supp:losspos}.
%
However, the value of $\lambda_t$ could be very large when $\bar{\alpha}_t$ is close to 1 as $t$ approaches 1.
%
Therefore, we clip the value of $\lambda_t$ with threshold $\delta$ in Eq.~\ref{eqn:diff:obj:pos}.










\end{document}






















\newpage

\renewcommand\thealgorithm{S.\arabic{algorithm}}
\renewcommand\thetable{S.\arabic{table}}
\renewcommand\thefigure{S.\arabic{figure}}
\renewcommand\thesection{S.\roman{section}}
\renewcommand\theequation{S.\arabic{equation}}

\title{Supplementary Document for ``MetaDE: Evolving Differential Evolution by Differential Evolution"}
%
%
\author{Minyang Chen, Chenchen Feng,
        and Ran Cheng
        \thanks{
        Minyang Chen was with the Department of Computer Science and Engineering, Southern University of Science and Technology, Shenzhen 518055, China. E-mail: cmy1223605455@gmail.com. }
        \thanks{
        Chenchen Feng is with the Department of Computer Science and Engineering, Southern University of Science and Technology, Shenzhen 518055, China. E-mail: chenchenfengcn@gmail.com. 
        }
        \thanks{
       Ran Cheng is with the Department of Data Science and Artificial Intelligence, and the Department of Computing, The Hong Kong Polytechnic University, Hong Kong SAR, China. E-mail: ranchengcn@gmail.com. (\emph{Corresponding author: Ran Cheng})
        }
        }

\onecolumn{}

% The paper headers
\markboth{Journal of \LaTeX\ Class Files,~Vol.~0, No.~0, 0~0}%
{Shell \MakeLowercase{\textit{et al.}}: Bare Demo of IEEEtran.cls for IEEE Journals}
% The only time the second header will appear is for the odd numbered pages
% after the title page when using the twoside option.

% *** Note that you probably will NOT want to include the author's ***
% *** name in the headers of peer review papers.                   ***
% You can use \ifCLASSOPTIONpeerreview for conditional compilation here if
% you desire.


% If you want to put a publisher's ID mark on the page you can do it like
% this:
%\IEEEpubid{0000--0000/00\$00.00~\copyright~2015 IEEE}
% Remember, if you use this you must call \IEEEpubidadjcol in the second
% column for its text to clear the IEEEpubid mark.

 

% use for special paper notices
%\IEEEspecialpapernotice{(Invited Paper)}




% make the title area
\maketitle

% As a general rule, do not put math, special symbols or citations
% in the abstract or keywords.

% Note that keywords are not normally used for peerreview papers.

% For peer review papers, you can put extra information on the cover
% page as needed:
% \ifCLASSOPTIONpeerreview
% \begin{center} \bfseries EDICS Category: 3-BBND \end{center}
% \fi
%
% For peerreview papers, this IEEEtran command inserts a page break and
% creates the second title. It will be ignored for other modes.
\IEEEpeerreviewmaketitle



% \clearpage
% \begin{titlepage}
% \centering
% {\LARGE Supplementary Document for ``MetaDE: Evolving Differential Evolution by Differential Evolution"}\\[1.5cm]
% {\large Minyang Chen, Chenchen Feng, and Ran Cheng}\\[1cm]
% {\small
% Minyang Chen was with the Department of Computer Science and Engineering, Southern University of Science and Technology, Shenzhen 518055, China. E-mail: cmy1223605455@gmail.com.\\[0.2cm]
% Chenchen Feng is with the Department of Computer Science and Engineering, Southern University of Science and Technology, Shenzhen 518055, China. E-mail: chenchenfengcn@gmail.com.\\[0.2cm]
% Ran Cheng is with the Department of Data Science and Artificial Intelligence, and the Department of Computing, The Hong Kong Polytechnic University, Hong Kong SAR, China. E-mail: ranchengcn@gmail.com. (Corresponding author: Ran Cheng)
% }\\[2cm]
% \end{titlepage}


\clearpage
\begin{center}
  {\Huge Supplementary Document for ``MetaDE: Evolving \\[0.3em]
  Differential Evolution by Differential Evolution"}\\[2em]
  {\Large Minyang Chen, Chenchen Feng, and Ran Cheng}\\[6em]
\end{center}







\section{Supplementary Experimental data}\label{section:FEs}

\subsection{Supplementary Figures}\label{section:FEs}

\begin{figure*}[htpb]
\centering
\includegraphics[scale=0.3]{su_D10_all.pdf}
\caption{Convergence curves on 10D problems in CEC2022 benchmark suite. The peer DE variants are set with population size of 100.}
\label{Figure_convergence_10D_supp}
\end{figure*}

\vfill 
{\small
\noindent

Minyang Chen was with the Department of Computer Science and Engineering, Southern University of Science and Technology, Shenzhen 518055, China. E-mail: cmy1223605455@gmail.com.

Chenchen Feng is with the Department of Computer Science and Engineering, Southern University of Science and Technology, Shenzhen 518055, China. E-mail: chenchenfengcn@gmail.com.

Ran Cheng is with the Department of Data Science and Artificial Intelligence, and the Department of Computing, The Hong Kong Polytechnic University, Hong Kong SAR, China. E-mail: ranchengcn@gmail.com. \textit{(Corresponding author: Ran Cheng)}
}


\begin{figure*}[htpb]
\centering
\includegraphics[scale=0.3]{su_D20_all.pdf}
\caption{Convergence curves on 20D problems in CEC2022 benchmark suite. The peer DE variants are set with population size of 100.}
\label{Figure_convergence_20D_supp}
\end{figure*}


\begin{figure*}[htpb]
\centering
\includegraphics[scale=0.3]{su_D10NP1000_all.pdf}
\caption{Convergence curves on 10D problems in CEC2022 benchmark suite. The peer DE variants are set with population size of 1,000.}
\label{Figure_convergence_10D_NP10k_supp}
\end{figure*}


\begin{figure*}[htpb]
\centering
\includegraphics[scale=0.3]{su_D20NP1000_all.pdf}
\caption{Convergence curves on 20D problems in CEC2022 benchmark suite. The peer DE variants are set with population size of 1,000.}
\label{Figure_convergence_20D_NP10k_supp}
\end{figure*}

\clearpage

\subsection{Detailed Experimental Results}\label{section:FEs_supp}

% Tables \ref{tab:vsClass10D} and \ref{tab:vsClass20D} present the detailed results of MetaDE compared to other algorithms on 10-dimensional and 20-dimensional problems from CEC2022 within a 60-second time frame. The convergence curves for all problems are shown in Figs. \ref{Figure_convergence_10D} and \ref{Figure_convergence_20D}.

% Tables \ref{tab:NP10000 10D} and \ref{tab:NP10000 20D} display the detailed results for MetaDE versus comparison algorithms with equal concurrency (population size = 10,000) on 10-dimensional and 20-dimensional problems in CEC2022, all within a span of 60 seconds. The convergence curves for all problems are shown in Figs. \ref{Figure_convergence_10D_NP10k} and \ref{Figure_convergence_20D_NP10k}.



% Table generated by Excel2LaTeX from sheet 'Sheet1'
\begin{table}[htbp]
  \centering
  \caption{
  Detailed results on 10D problems in CEC2022 benchmark suite. The peer DE variants are set with population size of 100.
The mean and standard deviation (in parentheses) of the results over multiple runs are displayed in pairs. 
Results with the best mean values are highlighted.
  }
  \resizebox{\textwidth}{!}{
   \renewcommand{\arraystretch}{1.2}
% Table generated by Excel2LaTeX from sheet 'Experiment1 60S'
\begin{tabular}{ccccccccc}
\toprule
Func  & MetaDE & DE    & SaDE  & JaDE  & CoDE  & SHADE & LSHADE-RSP & EDEV \\
\midrule
$F_{1}$ & \textbf{0.00E+00 (0.00E+00)} & \boldmath{}\textbf{0.00E+00 (0.00E+00)$\approx$}\unboldmath{} & \boldmath{}\textbf{0.00E+00 (0.00E+00)$\approx$}\unboldmath{} & \boldmath{}\textbf{0.00E+00 (0.00E+00)$\approx$}\unboldmath{} & \boldmath{}\textbf{0.00E+00 (0.00E+00)$\approx$}\unboldmath{} & \boldmath{}\textbf{0.00E+00 (0.00E+00)$\approx$}\unboldmath{} & \boldmath{}\textbf{0.00E+00 (0.00E+00)$\approx$}\unboldmath{} & \boldmath{}\textbf{0.00E+00 (0.00E+00)$\approx$}\unboldmath{} \\
$F_{2}$ & \textbf{0.00E+00 (0.00E+00)} & 6.05E+00 (2.43E+00)$-$ & 4.86E+00 (4.29E+00)$-$ & 4.89E+00 (3.74E+00)$-$ & 4.71E+00 (2.55E+00)$-$ & 5.47E+00 (3.62E+00)$-$ & 2.38E+00 (2.58E+00)$-$ & 6.11E+00 (2.87E+00)$-$ \\
$F_{3}$ & \textbf{0.00E+00 (0.00E+00)} & \boldmath{}\textbf{0.00E+00 (0.00E+00)$\approx$}\unboldmath{} & \boldmath{}\textbf{0.00E+00 (0.00E+00)$\approx$}\unboldmath{} & \boldmath{}\textbf{0.00E+00 (0.00E+00)$\approx$}\unboldmath{} & \boldmath{}\textbf{0.00E+00 (0.00E+00)$\approx$}\unboldmath{} & \boldmath{}\textbf{0.00E+00 (0.00E+00)$\approx$}\unboldmath{} & \boldmath{}\textbf{0.00E+00 (0.00E+00)$\approx$}\unboldmath{} & \boldmath{}\textbf{0.00E+00 (0.00E+00)$\approx$}\unboldmath{} \\
$F_{4}$ & \textbf{0.00E+00 (0.00E+00)} & 6.90E+00 (4.01E+00)$-$ & 1.03E+00 (8.93E-01)$-$ & 2.31E+01 (1.19E+01)$-$ & 8.34E-01 (7.62E-01)$-$ & 3.05E+00 (9.43E-01)$-$ & 2.12E+00 (6.56E-01)$-$ & 6.52E+00 (4.51E+00)$-$ \\
$F_{5}$ & \textbf{0.00E+00 (0.00E+00)} & \boldmath{}\textbf{0.00E+00 (0.00E+00)$\approx$}\unboldmath{} & \boldmath{}\textbf{0.00E+00 (0.00E+00)$\approx$}\unboldmath{} & \boldmath{}\textbf{0.00E+00 (0.00E+00)$\approx$}\unboldmath{} & \boldmath{}\textbf{0.00E+00 (0.00E+00)$\approx$}\unboldmath{} & \boldmath{}\textbf{0.00E+00 (0.00E+00)$\approx$}\unboldmath{} & \boldmath{}\textbf{0.00E+00 (0.00E+00)$\approx$}\unboldmath{} & \boldmath{}\textbf{0.00E+00 (0.00E+00)$\approx$}\unboldmath{} \\
$F_{6}$ & \textbf{5.50E-04 (3.96E-04)} & 1.11E-01 (8.92E-02)$-$ & 6.04E+01 (2.25E+02)$-$ & 1.60E+00 (2.48E+00)$-$ & 9.08E-03 (1.17E-02)$-$ & 1.33E+00 (2.22E+00)$-$ & 3.84E-02 (5.60E-02)$-$ & 9.06E-01 (1.76E+00)$-$ \\
$F_{7}$ & \textbf{0.00E+00 (0.00E+00)} & 5.18E-02 (1.59E-01)$-$ & 2.15E-02 (1.39E-02)$-$ & \boldmath{}\textbf{0.00E+00 (0.00E+00)$\approx$}\unboldmath{} & 1.92E-03 (7.32E-03)$-$ & 6.59E-03 (1.41E-02)$-$ & 1.06E+02 (1.42E-02)$-$ & 9.54E+00 (9.86E+00)$-$ \\
$F_{8}$ & \textbf{5.52E-03 (4.41E-03)} & 1.42E-01 (2.45E-01)$-$ & 5.15E-02 (2.29E-02)$-$ & 1.74E+01 (4.79E+00)$-$ & 6.02E-03 (1.02E-02)$-$ & 2.29E+00 (5.89E+00)$-$ & 2.19E+00 (5.89E+00)$-$ & 7.06E+00 (9.36E+00)$-$ \\
$F_{9}$ & \textbf{3.36E+00 (1.77E+01)} & 2.29E+02 (7.53E-06)$-$ & 2.29E+02 (6.38E-06)$-$ & 2.29E+02 (6.38E-06)$-$ & 2.29E+02 (7.17E-06)$-$ & 2.29E+02 (7.43E-06)$-$ & 2.29E+02 (8.19E-05)$-$ & 2.29E+02 (1.01E-05)$-$ \\
$F_{10}$ & \textbf{0.00E+00 (0.00E+00)} & 1.00E+02 (5.18E-02)$-$ & 1.03E+02 (1.83E+01)$-$ & 1.04E+02 (1.91E+01)$-$ & 1.00E+02 (6.83E-02)$-$ & 1.10E+02 (3.09E+01)$-$ & 1.03E+02 (1.77E+01)$-$ & 1.07E+02 (2.62E+01)$-$ \\
$F_{11}$ & \textbf{0.00E+00 (0.00E+00)} & \boldmath{}\textbf{0.00E+00 (0.00E+00)$\approx$}\unboldmath{} & 2.42E+01 (5.52E+01)$-$ & \boldmath{}\textbf{0.00E+00 (0.00E+00)$\approx$}\unboldmath{} & \boldmath{}\textbf{0.00E+00 (0.00E+00)$\approx$}\unboldmath{} & \boldmath{}\textbf{0.00E+00 (0.00E+00)$\approx$}\unboldmath{} & \boldmath{}\textbf{0.00E+00 (0.00E+00)$\approx$}\unboldmath{} & 4.84E+00 (2.65E+01)$-$ \\
$F_{12}$ & \textbf{1.39E+02 (4.63E+01)} & 1.62E+02 (1.04E+00)$-$ & 1.63E+02 (1.57E+00)$-$ & 1.62E+02 (2.22E+00)$-$ & 1.59E+02 (1.14E+00)$-$ & 1.63E+02 (1.25E+00)$-$ & 1.64E+02 (1.38E+00)$-$ & 1.62E+02 (1.71E+00)$-$ \\
\midrule
$+$ / $\approx$ / $-$ & --    & 0/4/8 & 0/3/9 & 0/5/7 & 0/5/7 & 0/4/8 & 0/4/8 & 0/3/9 \\
\bottomrule
\end{tabular}%
}
\footnotesize
\textsuperscript{*} The Wilcoxon rank-sum tests (with a significance level of 0.05) were conducted between MetaDE and each individually.
The final row displays the number of problems where the corresponding algorithm performs statistically better ($+$),  similar ($\thickapprox$), or worse ($-$) compared to MetaDE.\\
\label{tab:vsClass10D_supp}%

\end{table}%


% Table generated by Excel2LaTeX from sheet 'Sheet1'
\begin{table}[htbp]
  \centering
  \caption{Detailed results on 20D problems in CEC2022 benchmark suite. The peer DE variants are set with population size of 100.
  The mean and standard deviation (in parentheses) of the results over multiple runs are displayed in pairs. 
Results with the best mean values are highlighted.
  }
  {
  \resizebox{\textwidth}{!}{
   \renewcommand{\arraystretch}{1.2}
% Table generated by Excel2LaTeX from sheet 'Experiment1 60S'
\begin{tabular}{ccccccccc}
\toprule
Func  & MetaDE & DE    & SaDE  & JaDE  & CoDE  & SHADE & LSHADE-RSP & EDEV \\
\midrule
$F_{1}$ & \multicolumn{1}{l}{\textbf{0.00E+00 (0.00E+00)}} & \multicolumn{1}{l}{\boldmath{}\textbf{0.00E+00 (0.00E+00)$\approx$}\unboldmath{}} & \multicolumn{1}{l}{\boldmath{}\textbf{0.00E+00 (0.00E+00)$\approx$}\unboldmath{}} & \multicolumn{1}{l}{\boldmath{}\textbf{0.00E+00 (0.00E+00)$\approx$}\unboldmath{}} & \multicolumn{1}{l}{\boldmath{}\textbf{0.00E+00 (0.00E+00)$\approx$}\unboldmath{}} & \multicolumn{1}{l}{\boldmath{}\textbf{0.00E+00 (0.00E+00)$\approx$}\unboldmath{}} & \multicolumn{1}{l}{\boldmath{}\textbf{0.00E+00 (0.00E+00)$\approx$}\unboldmath{}} & \multicolumn{1}{l}{\boldmath{}\textbf{0.00E+00 (0.00E+00)$\approx$}\unboldmath{}} \\
$F_{2}$ & \multicolumn{1}{l}{\textbf{1.26E-02 (3.74E-02)}} & \multicolumn{1}{l}{4.69E+01 (2.09E+00)$-$} & \multicolumn{1}{l}{3.50E+01 (2.21E+01)$-$} & \multicolumn{1}{l}{4.75E+01 (8.67E+00)$-$} & \multicolumn{1}{l}{4.58E+01 (1.20E+01)$-$} & \multicolumn{1}{l}{4.75E+01 (8.67E+00)$-$} & \multicolumn{1}{l}{4.30E+01 (1.71E+01)$-$} & \multicolumn{1}{l}{4.13E+01 (1.78E+01)$-$} \\
$F_{3}$ & \multicolumn{1}{l}{\textbf{0.00E+00 (0.00E+00)}} & \multicolumn{1}{l}{\boldmath{}\textbf{0.00E+00 (0.00E+00)$\approx$}\unboldmath{}} & \multicolumn{1}{l}{\boldmath{}\textbf{0.00E+00 (0.00E+00)$\approx$}\unboldmath{}} & \multicolumn{1}{l}{\boldmath{}\textbf{0.00E+00 (0.00E+00)$\approx$}\unboldmath{}} & \multicolumn{1}{l}{\boldmath{}\textbf{0.00E+00 (0.00E+00)$\approx$}\unboldmath{}} & \multicolumn{1}{l}{1.03E-08 (5.64E-08)$\approx$} & \multicolumn{1}{l}{\boldmath{}\textbf{0.00E+00 (0.00E+00)$\approx$}\unboldmath{}} & \multicolumn{1}{l}{6.91E-05 (3.31E-04)$-$} \\
$F_{4}$ & \multicolumn{1}{l}{\textbf{2.02E+00 (8.56E-01)}} & \multicolumn{1}{l}{1.95E+01 (8.17E+00)$-$} & \multicolumn{1}{l}{7.42E+00 (2.14E+00)$-$} & \multicolumn{1}{l}{7.21E+01 (3.45E+01)$-$} & \multicolumn{1}{l}{1.11E+01 (2.22E+00)$-$} & \multicolumn{1}{l}{1.27E+01 (2.73E+00)$-$} & \multicolumn{1}{l}{9.05E+00 (1.44E+00)$-$} & \multicolumn{1}{l}{2.37E+01 (1.37E+01)$-$} \\
$F_{5}$ & \multicolumn{1}{l}{\textbf{0.00E+00 (0.00E+00)}} & \multicolumn{1}{l}{\boldmath{}\textbf{0.00E+00 (0.00E+00)$\approx$}\unboldmath{}} & \multicolumn{1}{l}{7.24E-01 (1.22E+00)$-$} & \multicolumn{1}{l}{\boldmath{}\textbf{0.00E+00 (0.00E+00)$\approx$}\unboldmath{}} & \multicolumn{1}{l}{\boldmath{}\textbf{0.00E+00 (0.00E+00)$\approx$}\unboldmath{}} & \multicolumn{1}{l}{\boldmath{}\textbf{0.00E+00 (0.00E+00)$\approx$}\unboldmath{}} & \multicolumn{1}{l}{\boldmath{}\textbf{0.00E+00 (0.00E+00)$\approx$}\unboldmath{}} & \multicolumn{1}{l}{1.78E-01 (3.12E-01)$-$} \\
$F_{6}$ & \multicolumn{1}{l}{\textbf{1.16E-01 (2.79E-02)}} & \multicolumn{1}{l}{4.99E-01 (4.11E-01)$-$} & \multicolumn{1}{l}{3.13E+01 (1.54E+01)$-$} & \multicolumn{1}{l}{5.34E+01 (3.33E+01)$-$} & \multicolumn{1}{l}{1.89E+01 (1.83E+01)$-$} & \multicolumn{1}{l}{5.06E+01 (3.16E+01)$-$} & \multicolumn{1}{l}{1.25E+01 (1.00E+01)$-$} & \multicolumn{1}{l}{4.91E+03 (6.53E+03)$-$} \\
$F_{7}$ & \multicolumn{1}{l}{\textbf{5.17E-02 (6.21E-02)}} & \multicolumn{1}{l}{4.23E+00 (7.90E+00)$-$} & \multicolumn{1}{l}{1.07E+01 (5.20E+00)$-$} & \multicolumn{1}{l}{2.98E+00 (3.61E+00)$-$} & \multicolumn{1}{l}{1.16E+00 (1.26E+00)$-$} & \multicolumn{1}{l}{7.77E+00 (6.65E+00)$-$} & \multicolumn{1}{l}{1.42E+01 (8.96E+00)$-$} & \multicolumn{1}{l}{2.29E+01 (8.80E+00)$-$} \\
$F_{8}$ & \multicolumn{1}{l}{\textbf{7.19E-01 (1.02E+00)}} & \multicolumn{1}{l}{8.24E+00 (1.00E+01)$-$} & \multicolumn{1}{l}{2.10E+01 (7.26E-01)$-$} & \multicolumn{1}{l}{2.64E+01 (9.73E-01)$-$} & \multicolumn{1}{l}{1.38E+01 (8.98E+00)$-$} & \multicolumn{1}{l}{2.02E+01 (8.29E-01)$-$} & \multicolumn{1}{l}{1.96E+01 (3.80E+00)$-$} & \multicolumn{1}{l}{2.08E+01 (3.81E-01)$-$} \\
$F_{9}$ & \multicolumn{1}{l}{\textbf{1.07E+02 (1.98E+01)}} & \multicolumn{1}{l}{1.81E+02 (9.39E-06)$-$} & \multicolumn{1}{l}{1.81E+02 (3.75E-06)$-$} & \multicolumn{1}{l}{1.81E+02 (9.02E-06)$-$} & \multicolumn{1}{l}{1.81E+02 (8.41E-06)$-$} & \multicolumn{1}{l}{1.81E+02 (1.04E-05)$-$} & \multicolumn{1}{l}{1.81E+02 (2.18E-05)$-$} & \multicolumn{1}{l}{1.81E+02 (5.43E-04)$-$} \\
$F_{10}$ & \multicolumn{1}{l}{\textbf{0.00E+00 (0.00E+00)}} & \multicolumn{1}{l}{1.13E+02 (3.52E+01)$-$} & \multicolumn{1}{l}{1.00E+02 (3.03E-02)$-$} & \multicolumn{1}{l}{1.13E+02 (3.76E+01)$-$} & \multicolumn{1}{l}{1.00E+02 (3.55E-02)$-$} & \multicolumn{1}{l}{1.12E+02 (3.47E+01)$-$} & \multicolumn{1}{l}{1.11E+02 (3.42E+01)$-$} & \multicolumn{1}{l}{1.07E+02 (3.80E+01)$-$} \\
$F_{11}$ & \multicolumn{1}{l}{\textbf{7.28E-05 (3.06E-04)}} & \multicolumn{1}{l}{3.39E+02 (4.87E+01)$-$} & \multicolumn{1}{l}{3.06E+02 (2.46E+01)$-$} & \multicolumn{1}{l}{3.19E+02 (3.95E+01)$-$} & \multicolumn{1}{l}{3.39E+02 (4.87E+01)$-$} & \multicolumn{1}{l}{3.16E+02 (3.68E+01)$-$} & \multicolumn{1}{l}{3.39E+02 (4.87E+01)$-$} & \multicolumn{1}{l}{3.19E+02 (3.95E+01)$-$} \\
$F_{12}$ & \multicolumn{1}{l}{\textbf{2.29E+02 (6.08E-01)}} & \multicolumn{1}{l}{2.37E+02 (3.11E+00)$-$} & \multicolumn{1}{l}{2.41E+02 (5.26E+00)$-$} & \multicolumn{1}{l}{2.37E+02 (5.10E+00)$-$} & \multicolumn{1}{l}{2.34E+02 (2.68E+00)$-$} & \multicolumn{1}{l}{2.39E+02 (4.46E+00)$-$} & \multicolumn{1}{l}{2.44E+02 (1.63E+01)$-$} & \multicolumn{1}{l}{2.42E+02 (8.38E+00)$-$} \\
\midrule
$+$ / $\approx$ / $-$ & --    & 0/3/9 & 0/2/10 & 0/3/9 & 0/3/9 & 0/3/9 & 0/3/9 & 0/1/11 \\
\bottomrule
\end{tabular}%
    }
\footnotesize
\textsuperscript{*} The Wilcoxon rank-sum tests (with a significance level of 0.05) were conducted between MetaDE and each individually.
The final row displays the number of problems where the corresponding algorithm performs statistically better ($+$),  similar ($\thickapprox$), or worse ($-$) compared to MetaDE.\\
\label{tab:vsClass20D_supp}%
}
\end{table}%

\clearpage

% Table generated by Excel2LaTeX from sheet 'Sheet1'
\begin{table}[htbp]
  \centering
  \caption{Detailed results on 10D problems in CEC2022 benchmark suite. The peer DE variants are set with population size of 1,000. 
The mean and standard deviation (in parentheses) of the results over multiple runs are displayed in pairs. 
Results with the best mean values are highlighted.
  }
  {
  \resizebox{\textwidth}{!}{
   \renewcommand{\arraystretch}{1.2}
% Table generated by Excel2LaTeX from sheet 'Exp2 NP1000'
\begin{tabular}{ccccccccc}
\toprule
Func  & MetaDE & DE    & SaDE  & JaDE  & CoDE  & SHADE & LSHADE-RSP & EDEV \\
\midrule
$F_{1}$ & \multicolumn{1}{l}{\textbf{0.00E+00 (0.00E+00)}} & \multicolumn{1}{l}{\boldmath{}\textbf{0.00E+00 (0.00E+00)$\approx$}\unboldmath{}} & \multicolumn{1}{l}{\boldmath{}\textbf{0.00E+00 (0.00E+00)$\approx$}\unboldmath{}} & \multicolumn{1}{l}{\boldmath{}\textbf{0.00E+00 (0.00E+00)$\approx$}\unboldmath{}} & \multicolumn{1}{l}{\boldmath{}\textbf{0.00E+00 (0.00E+00)$\approx$}\unboldmath{}} & \multicolumn{1}{l}{\boldmath{}\textbf{0.00E+00 (0.00E+00)$\approx$}\unboldmath{}} & \multicolumn{1}{l}{\boldmath{}\textbf{0.00E+00 (0.00E+00)$\approx$}\unboldmath{}} & \multicolumn{1}{l}{\boldmath{}\textbf{0.00E+00 (0.00E+00)$\approx$}\unboldmath{}} \\
$F_{2}$ & \multicolumn{1}{l}{\textbf{0.00E+00 (0.00E+00)}} & \multicolumn{1}{l}{3.60E+00 (1.20E+00)$-$} & \multicolumn{1}{l}{6.87E+00 (3.63E+00)$-$} & \multicolumn{1}{l}{8.15E+00 (2.11E+00)$-$} & \multicolumn{1}{l}{2.44E+00 (1.97E+00)$-$} & \multicolumn{1}{l}{8.02E+00 (2.47E+00)$-$} & \multicolumn{1}{l}{1.12E+00 (1.79E+00)$-$} & \multicolumn{1}{l}{6.27E+00 (2.93E+00)$-$} \\
$F_{3}$ & \multicolumn{1}{l}{\textbf{0.00E+00 (0.00E+00)}} & \multicolumn{1}{l}{\boldmath{}\textbf{0.00E+00 (0.00E+00)$\approx$}\unboldmath{}} & \multicolumn{1}{l}{\boldmath{}\textbf{0.00E+00 (0.00E+00)$\approx$}\unboldmath{}} & \multicolumn{1}{l}{\boldmath{}\textbf{0.00E+00 (0.00E+00)$\approx$}\unboldmath{}} & \multicolumn{1}{l}{\boldmath{}\textbf{0.00E+00 (0.00E+00)$\approx$}\unboldmath{}} & \multicolumn{1}{l}{\boldmath{}\textbf{0.00E+00 (0.00E+00)$\approx$}\unboldmath{}} & \multicolumn{1}{l}{\boldmath{}\textbf{0.00E+00 (0.00E+00)$\approx$}\unboldmath{}} & \multicolumn{1}{l}{\boldmath{}\textbf{0.00E+00 (0.00E+00)$\approx$}\unboldmath{}} \\
$F_{4}$ & \multicolumn{1}{l}{\textbf{0.00E+00 (0.00E+00)}} & \multicolumn{1}{l}{1.50E+01 (2.59E+00)$-$} & \multicolumn{1}{l}{4.75E-01 (5.04E-01)$-$} & \multicolumn{1}{l}{2.20E+00 (5.30E-01)$-$} & \multicolumn{1}{l}{\boldmath{}\textbf{0.00E+00 (0.00E+00)$\approx$}\unboldmath{}} & \multicolumn{1}{l}{\boldmath{}\textbf{0.00E+00 (0.00E+00)$\approx$}\unboldmath{}} & \multicolumn{1}{l}{9.63E-01 (6.54E-01)$-$} & \multicolumn{1}{l}{4.91E+00 (5.24E+00)$-$} \\
$F_{5}$ & \multicolumn{1}{l}{\textbf{0.00E+00 (0.00E+00)}} & \multicolumn{1}{l}{\boldmath{}\textbf{0.00E+00 (0.00E+00)$\approx$}\unboldmath{}} & \multicolumn{1}{l}{\boldmath{}\textbf{0.00E+00 (0.00E+00)$\approx$}\unboldmath{}} & \multicolumn{1}{l}{\boldmath{}\textbf{0.00E+00 (0.00E+00)$\approx$}\unboldmath{}} & \multicolumn{1}{l}{\boldmath{}\textbf{0.00E+00 (0.00E+00)$\approx$}\unboldmath{}} & \multicolumn{1}{l}{\boldmath{}\textbf{0.00E+00 (0.00E+00)$\approx$}\unboldmath{}} & \multicolumn{1}{l}{\boldmath{}\textbf{0.00E+00 (0.00E+00)$\approx$}\unboldmath{}} & \multicolumn{1}{l}{\boldmath{}\textbf{0.00E+00 (0.00E+00)$\approx$}\unboldmath{}} \\
$F_{6}$ & \multicolumn{1}{l}{\textbf{5.50E-04 (3.96E-04)}} & \multicolumn{1}{l}{1.49E-02 (4.74E-03)$-$} & \multicolumn{1}{l}{1.95E+00 (1.54E+00)$-$} & \multicolumn{1}{l}{1.01E-01 (6.20E-02)$-$} & \multicolumn{1}{l}{7.69E-04 (4.31E-04)$\approx$} & \multicolumn{1}{l}{4.62E-03 (8.17E-03)$-$} & \multicolumn{1}{l}{1.89E-03 (7.18E-04)$-$} & \multicolumn{1}{l}{4.37E-02 (6.48E-02)$-$} \\
$F_{7}$ & \multicolumn{1}{l}{\textbf{0.00E+00 (0.00E+00)}} & \multicolumn{1}{l}{9.60E-04 (5.35E-03)$-$} & \multicolumn{1}{l}{1.98E-02 (8.61E-03)$-$} & \multicolumn{1}{l}{\boldmath{}\textbf{0.00E+00 (0.00E+00)$\approx$}\unboldmath{}} & \multicolumn{1}{l}{\boldmath{}\textbf{0.00E+00 (0.00E+00)$\approx$}\unboldmath{}} & \multicolumn{1}{l}{\boldmath{}\textbf{0.00E+00 (0.00E+00)$\approx$}\unboldmath{}} & \multicolumn{1}{l}{\boldmath{}\textbf{0.00E+00 (0.00E+00)$\approx$}\unboldmath{}} & \multicolumn{1}{l}{1.35E+00 (4.98E+00)$-$} \\
$F_{8}$ & \multicolumn{1}{l}{5.52E-03 (4.41E-03)} & \multicolumn{1}{l}{1.19E-01 (2.50E-02)$-$} & \multicolumn{1}{l}{1.13E+00 (4.44E-01)$-$} & \multicolumn{1}{l}{1.03E-01 (2.62E-02)$-$} & \multicolumn{1}{l}{\boldmath{}\textbf{0.00E+00 (0.00E+00)$\approx$}\unboldmath{}} & \multicolumn{1}{l}{8.28E-02 (2.66E-02)$-$} & \multicolumn{1}{l}{9.13E-02 (1.11E-01)$-$} & \multicolumn{1}{l}{1.03E+00 (3.58E+00)$-$} \\
$F_{9}$ & \multicolumn{1}{l}{\textbf{3.36E+00 (1.77E+01)}} & \multicolumn{1}{l}{2.29E+02 (7.85E-06)$-$} & \multicolumn{1}{l}{2.29E+02 (8.58E-06)$-$} & \multicolumn{1}{l}{2.29E+02 (8.28E-06)$-$} & \multicolumn{1}{l}{2.29E+02 (8.67E-14)$-$} & \multicolumn{1}{l}{2.29E+02 (2.74E-06)$-$} & \multicolumn{1}{l}{2.29E+02 (9.39E-06)$-$} & \multicolumn{1}{l}{2.29E+02 (1.17E-05)$-$} \\
$F_{10}$ & \multicolumn{1}{l}{\textbf{0.00E+00 (0.00E+00)}} & \multicolumn{1}{l}{1.00E+02 (1.70E-02)$-$} & \multicolumn{1}{l}{1.00E+02 (3.93E-02)$-$} & \multicolumn{1}{l}{1.00E+02 (2.49E-02)$-$} & \multicolumn{1}{l}{1.00E+02 (1.01E-02)$-$} & \multicolumn{1}{l}{1.00E+02 (1.83E-02)$-$} & \multicolumn{1}{l}{1.00E+02 (8.05E-04)$-$} & \multicolumn{1}{l}{1.00E+02 (3.93E-02)$-$} \\
$F_{11}$ & \multicolumn{1}{l}{\textbf{0.00E+00 (0.00E+00)}} & \multicolumn{1}{l}{\boldmath{}\textbf{0.00E+00 (0.00E+00)$\approx$}\unboldmath{}} & \multicolumn{1}{l}{\boldmath{}\textbf{0.00E+00 (0.00E+00)$\approx$}\unboldmath{}} & \multicolumn{1}{l}{\boldmath{}\textbf{0.00E+00 (0.00E+00)$\approx$}\unboldmath{}} & \multicolumn{1}{l}{3.65E-07 (2.03E-06)$-$} & \multicolumn{1}{l}{\boldmath{}\textbf{0.00E+00 (0.00E+00)$\approx$}\unboldmath{}} & \multicolumn{1}{l}{\boldmath{}\textbf{0.00E+00 (0.00E+00)$\approx$}\unboldmath{}} & \multicolumn{1}{l}{\boldmath{}\textbf{0.00E+00 (0.00E+00)$\approx$}\unboldmath{}} \\
$F_{12}$ & \multicolumn{1}{l}{\textbf{1.39E+02 (4.63E+01)}} & \multicolumn{1}{l}{1.60E+02 (9.79E-01)$-$} & \multicolumn{1}{l}{1.60E+02 (1.56E+00)$-$} & \multicolumn{1}{l}{1.59E+02 (1.28E+00)$-$} & \multicolumn{1}{l}{1.59E+02 (8.67E-14)$-$} & \multicolumn{1}{l}{1.61E+02 (1.69E+00)$-$} & \multicolumn{1}{l}{1.63E+02 (7.62E-01)$-$} & \multicolumn{1}{l}{1.60E+02 (1.18E+00)$-$} \\
\midrule
$+$ / $\approx$ / $-$ & --    & 0/4/8 & 0/4/8 & 0/5/7 & 0/7/5 & 0/6/6 & 0/5/7 & 0/4/8 \\
\bottomrule
\end{tabular}%
}
\footnotesize
\textsuperscript{*} The Wilcoxon rank-sum tests (with a significance level of 0.05) were conducted between MetaDE and each individually.
The final row displays the number of problems where the corresponding algorithm performs statistically better ($+$),  similar ($\thickapprox$), or worse ($-$) compared to MetaDE.\\
\label{tab:NP10000 10D_supp}%
}
\end{table}%


% Table generated by Excel2LaTeX from sheet 'Sheet1'
\begin{table}[htbp]
  \centering
  \caption{Detailed results on 20D problems in CEC2022 benchmark suite. The peer DE variants are set with population size of 1,000. 
The mean and standard deviation (in parentheses) of the results over multiple runs are displayed in pairs. 
Results with the best mean values are highlighted.
  }
  {
  \resizebox{\textwidth}{!}{
   \renewcommand{\arraystretch}{1.2}
% Table generated by Excel2LaTeX from sheet 'Exp2 NP1000'
\begin{tabular}{ccccccccc}
\toprule
Func  & MetaDE & DE    & SaDE  & JaDE  & CoDE  & SHADE & LSHADE-RSP & EDEV \\
\midrule
$F_{1}$ & \multicolumn{1}{l}{\textbf{0.00E+00 (0.00E+00)}} & \multicolumn{1}{l}{\boldmath{}\textbf{0.00E+00 (0.00E+00)$\approx$}\unboldmath{}} & \multicolumn{1}{l}{\boldmath{}\textbf{0.00E+00 (0.00E+00)$\approx$}\unboldmath{}} & \multicolumn{1}{l}{\boldmath{}\textbf{0.00E+00 (0.00E+00)$\approx$}\unboldmath{}} & \multicolumn{1}{l}{\boldmath{}\textbf{0.00E+00 (0.00E+00)$\approx$}\unboldmath{}} & \multicolumn{1}{l}{\boldmath{}\textbf{0.00E+00 (0.00E+00)$\approx$}\unboldmath{}} & \multicolumn{1}{l}{\boldmath{}\textbf{0.00E+00 (0.00E+00)$\approx$}\unboldmath{}} & \multicolumn{1}{l}{\boldmath{}\textbf{0.00E+00 (0.00E+00)$\approx$}\unboldmath{}} \\
$F_{2}$ & \multicolumn{1}{l}{\textbf{1.26E-02 (3.74E-02)}} & \multicolumn{1}{l}{4.49E+01 (0.00E+00)$-$} & \multicolumn{1}{l}{4.91E+01 (7.67E-06)$-$} & \multicolumn{1}{l}{4.91E+01 (0.00E+00)$-$} & \multicolumn{1}{l}{4.91E+01 (0.00E+00)$-$} & \multicolumn{1}{l}{4.91E+01 (0.00E+00)$-$} & \multicolumn{1}{l}{4.52E+01 (1.05E+00)$-$} & \multicolumn{1}{l}{4.89E+01 (1.05E+00)$-$} \\
$F_{3}$ & \multicolumn{1}{l}{\textbf{0.00E+00 (0.00E+00)}} & \multicolumn{1}{l}{\boldmath{}\textbf{0.00E+00 (0.00E+00)$\approx$}\unboldmath{}} & \multicolumn{1}{l}{\boldmath{}\textbf{0.00E+00 (0.00E+00)$\approx$}\unboldmath{}} & \multicolumn{1}{l}{\boldmath{}\textbf{0.00E+00 (0.00E+00)$\approx$}\unboldmath{}} & \multicolumn{1}{l}{\boldmath{}\textbf{0.00E+00 (0.00E+00)$\approx$}\unboldmath{}} & \multicolumn{1}{l}{\boldmath{}\textbf{0.00E+00 (0.00E+00)$\approx$}\unboldmath{}} & \multicolumn{1}{l}{\boldmath{}\textbf{0.00E+00 (0.00E+00)$\approx$}\unboldmath{}} & \multicolumn{1}{l}{\boldmath{}\textbf{0.00E+00 (0.00E+00)$\approx$}\unboldmath{}} \\
$F_{4}$ & \multicolumn{1}{l}{\textbf{2.02E+00 (8.56E-01)}} & \multicolumn{1}{l}{8.33E+01 (5.85E+00)$-$} & \multicolumn{1}{l}{6.09E+00 (1.02E+00)$-$} & \multicolumn{1}{l}{1.07E+01 (1.95E+00)$-$} & \multicolumn{1}{l}{8.02E+00 (1.08E+00)$-$} & \multicolumn{1}{l}{7.39E+00 (1.17E+00)$-$} & \multicolumn{1}{l}{4.82E+00 (8.64E-01)$-$} & \multicolumn{1}{l}{1.81E+01 (1.52E+01)$-$} \\
$F_{5}$ & \multicolumn{1}{l}{\textbf{0.00E+00 (0.00E+00)}} & \multicolumn{1}{l}{\boldmath{}\textbf{0.00E+00 (0.00E+00)$\approx$}\unboldmath{}} & \multicolumn{1}{l}{6.16E-02 (2.71E-01)$-$} & \multicolumn{1}{l}{\boldmath{}\textbf{0.00E+00 (0.00E+00)$\approx$}\unboldmath{}} & \multicolumn{1}{l}{\boldmath{}\textbf{0.00E+00 (0.00E+00)$\approx$}\unboldmath{}} & \multicolumn{1}{l}{\boldmath{}\textbf{0.00E+00 (0.00E+00)$\approx$}\unboldmath{}} & \multicolumn{1}{l}{\boldmath{}\textbf{0.00E+00 (0.00E+00)$\approx$}\unboldmath{}} & \multicolumn{1}{l}{\boldmath{}\textbf{0.00E+00 (0.00E+00)$\approx$}\unboldmath{}} \\
$F_{6}$ & \multicolumn{1}{l}{1.16E-01 (2.79E-02)} & \multicolumn{1}{l}{8.86E+00 (1.34E+00)$-$} & \multicolumn{1}{l}{1.72E+02 (5.24E+02)$-$} & \multicolumn{1}{l}{1.80E+00 (7.49E-01)$-$} & \multicolumn{1}{l}{2.07E-01 (5.04E-02)$-$} & \multicolumn{1}{l}{1.35E-01 (5.23E-02)$\approx$} & \multicolumn{1}{l}{\textbf{7.35E-02 (3.33E-02)$+$}} & \multicolumn{1}{l}{2.70E+00 (1.13E+01)$-$} \\
$F_{7}$ & \multicolumn{1}{l}{5.17E-02 (6.21E-02)} & \multicolumn{1}{l}{3.28E+01 (1.82E+00)$-$} & \multicolumn{1}{l}{1.57E+01 (3.45E+00)$-$} & \multicolumn{1}{l}{1.31E+01 (2.39E+00)$-$} & \multicolumn{1}{l}{\textbf{2.02E-03 (1.44E-03)$+$}} & \multicolumn{1}{l}{9.48E+00 (1.76E+00)$-$} & \multicolumn{1}{l}{2.92E+00 (9.57E-01)$-$} & \multicolumn{1}{l}{1.68E+01 (1.04E+01)$-$} \\
$F_{8}$ & \multicolumn{1}{l}{\textbf{7.19E-01 (1.02E+00)}} & \multicolumn{1}{l}{2.41E+01 (2.25E+00)$-$} & \multicolumn{1}{l}{2.11E+01 (1.76E+00)$-$} & \multicolumn{1}{l}{2.15E+01 (1.27E+00)$-$} & \multicolumn{1}{l}{1.12E+01 (1.98E+00)$-$} & \multicolumn{1}{l}{1.99E+01 (2.81E+00)$-$} & \multicolumn{1}{l}{8.53E+00 (4.47E+00)$-$} & \multicolumn{1}{l}{1.88E+01 (5.20E+00)$-$} \\
$F_{9}$ & \multicolumn{1}{l}{\textbf{1.07E+02 (1.98E+01)}} & \multicolumn{1}{l}{1.81E+02 (2.68E-05)$-$} & \multicolumn{1}{l}{1.81E+02 (2.75E-05)$-$} & \multicolumn{1}{l}{1.81E+02 (1.08E-05)$-$} & \multicolumn{1}{l}{1.81E+02 (7.25E-06)$-$} & \multicolumn{1}{l}{1.81E+02 (9.14E-06)$-$} & \multicolumn{1}{l}{1.81E+02 (1.24E-05)$-$} & \multicolumn{1}{l}{1.81E+02 (1.01E-04)$-$} \\
$F_{10}$ & \multicolumn{1}{l}{\textbf{0.00E+00 (0.00E+00)}} & \multicolumn{1}{l}{1.00E+02 (3.21E-02)$-$} & \multicolumn{1}{l}{1.00E+02 (2.58E-02)$-$} & \multicolumn{1}{l}{1.00E+02 (3.51E-02)$-$} & \multicolumn{1}{l}{1.00E+02 (1.80E-02)$-$} & \multicolumn{1}{l}{1.00E+02 (1.80E-02)$-$} & \multicolumn{1}{l}{1.00E+02 (1.21E-02)$-$} & \multicolumn{1}{l}{1.00E+02 (4.13E-02)$-$} \\
$F_{11}$ & \multicolumn{1}{l}{\textbf{7.28E-05 (3.06E-04)}} & \multicolumn{1}{l}{3.97E+02 (1.80E+01)$-$} & \multicolumn{1}{l}{3.26E+02 (4.45E+01)$-$} & \multicolumn{1}{l}{3.16E+02 (3.74E+01)$-$} & \multicolumn{1}{l}{3.81E+02 (4.02E+01)$-$} & \multicolumn{1}{l}{3.16E+02 (3.74E+01)$-$} & \multicolumn{1}{l}{3.77E+02 (4.25E+01)$-$} & \multicolumn{1}{l}{3.19E+02 (9.80E+01)$-$} \\
$F_{12}$ & \multicolumn{1}{l}{\textbf{2.29E+02 (6.08E-01)}} & \multicolumn{1}{l}{2.35E+02 (2.45E+00)$-$} & \multicolumn{1}{l}{2.34E+02 (2.55E+00)$-$} & \multicolumn{1}{l}{2.30E+02 (1.99E+00)$-$} & \multicolumn{1}{l}{2.30E+02 (1.40E+00)$-$} & \multicolumn{1}{l}{2.32E+02 (1.09E+00)$-$} & \multicolumn{1}{l}{2.33E+02 (2.27E+00)$-$} & \multicolumn{1}{l}{2.38E+02 (3.81E+00)$-$} \\
\midrule
$+$ / $\approx$ / $-$ & --    & 0/3/9 & 0/2/10 & 0/3/9 & 1/3/8 & 0/4/8 & 1/3/8 & 0/3/9 \\
\bottomrule
\end{tabular}%
}
\footnotesize
\textsuperscript{*} The Wilcoxon rank-sum tests (with a significance level of 0.05) were conducted between MetaDE and each individually.
The final row displays the number of problems where the corresponding algorithm performs statistically better ($+$),  similar ($\thickapprox$), or worse ($-$) compared to MetaDE.\\
\label{tab:NP10000 20D_supp}%
}
\end{table}%


% \subsection{Supplementary results}\label{section:FEs}
 % This experiment leveraged parallel GPU computation and constrained the runtime: 30 seconds for the 10-dimensional problems and 60 seconds for the 20-dimensional problems. 
 
% We recorded the number of FEs each algorithm achieved within 60s, which are presented in Table \ref{tab:FEs}. 
 
%  In addition, when the population size of the comparative DE variants was increased to 10,000 (same level of concurrency as MetaDE), the FEs achieved by all algorithms are displayed in Table \ref{tab:FEs NP10000}.

% The results show that MetaDE achieved considerably more FEs within a given time than the other algorithms. This demonstrates that MetaDE has a high degree of parallelism, making it particularly well-suited for GPU computing.



% Table generated by Excel2LaTeX from sheet 'Sheet1'
\begin{table}[htbp]
  \centering
  \caption{The number of FEs achieved by each algorithm within \SI{60}{\second}. The peer DE variants are set with population size of 1,000.}
  {
           \renewcommand{\arraystretch}{1}
 \renewcommand{\tabcolsep}{10pt}
% Table generated by Excel2LaTeX from sheet 'Exp2 NP10000'
\begin{tabular}{cccccccccc}
\toprule
Dim   & Func  & MetaDE & DE    & SaDE  & JaDE  & CoDE  & SHADE &LSHADE-RSP&EDEV\\
\midrule
\multirow{12}[2]{*}{10D} & $F_{1}$ & \textbf{1.85E+09} & 3.91E+07 & 4.38E+06 & 6.26E+06 & 7.67E+07 & 4.62E+06&1.92E+07&1.91E+07 \\
      & $F_{2}$ & \textbf{1.84E+09} & 4.05E+07 &4.35E+06 & 6.20E+06 & 7.61E+07 & 4.63E+06&1.89E+07&1.96E+07 \\
      & $F_{3}$ & \textbf{1.50E+09} & 3.63E+07 & 4.31E+06 &6.18E+06& 6.87E+07& 4.54E+06&1.84E+07 &1.89E+07\\
      & $F_{4}$ & \textbf{1.84E+09} &3.77E+07 & 4.27E+06 & 6.16E+06 &7.42E+07 & 4.64E+06&1.89E+07&1.98E+07 \\
      & $F_{5}$ & \textbf{1.83E+09} & 3.85E+07 &4.33E+06 & 6.16E+06 & 7.50E+07& 4.64E+06&1.85E+07& 1.99E+07\\
      & $F_{6}$ & \textbf{1.84E+09} &3.71E+07 & 4.30E+06 & 6.01E+06& 7.38E+07 & 4.61E+06& 1.86E+07&1.99E+07\\
      & $F_{7}$ & \textbf{1.74E+09} &3.71E+07 & 4.15E+06 & 5.84E+06 & 7.05E+07& 4.37E+06&1.77E+07&1.91E+07\\
      & $F_{8}$ & \textbf{1.72E+09} & 3.69E+07 & 4.02E+06 & 5.75E+06 & 6.82E+07 & 4.35E+06& 1.71E+07&1.93E+07\\
      & $F_{9}$ & \textbf{1.78E+09} &3.81E+07 &4.35E+06 &6.10E+06 & 6.84E+07 & 4.56E+06&1.84E+07 &1.92E+07\\
      & $F_{10}$ & \textbf{1.44E+09} & 3.50E+07&4.12E+06 &  5.77E+06 & 6.94E+07 & 4.41E+06&1.76E+07& 1.83E+07\\
      & $F_{11}$ & \textbf{1.46E+09} & 3.43E+07& 4.11E+06 & 5.87E+06 & 6.89E+07 & 4.36E+06 &1.69E+07&1.83E+07\\
      & $F_{12}$ & \textbf{1.43E+09} &3.37E+07 & 3.97E+06 & 5.81E+06 & 6.56E+07 & 4.32E+06 &1.72E+07&1.92E+07\\
\midrule
\midrule
\multirow{12}[2]{*}{20D} & $F_{1}$ & \textbf{1.66E+09} & 3.93E+07 &4.29E+06 & 5.94E+06 & 7.20E+07 & 4.62E+06&1.87E+07&1.94E+07 \\
      & $F_{2}$ & \textbf{1.66E+09} & 3.89E+07 &4.17E+06 & 5.94E+06 & 7.33E+07 & 4.61E+06&1.91E+07& 1.95E+07\\
      & $F_{3}$ & \textbf{1.18E+09} & 3.38E+07 & 4.16E+06& 5.93E+06 &  6.77E+07& 4.49E+06& 1.72E+07&1.85E+07\\
      & $F_{4}$ & \textbf{1.65E+09} &3.56E+07 & 4.27E+06 & 5.85E+06 & 7.05E+07& 4.63E+06& 1.90E+07&1.94E+07\\
      & $F_{5}$ & \textbf{1.64E+09} & 3.88E+07 & 4.20E+06 & 6.04E+06& 7.16E+07& 4.62E+06&1.86E+07&1.95E+07 \\
      & $F_{6}$ & \textbf{1.64E+09} & 3.73E+07 & 4.25E+06 & 5.84E+06 & 7.10E+07 & 4.61E+06&1.86E+07& 1.92E+07\\
      & $F_{7}$ & \textbf{1.45E+09} & 3.13E+07 & 3.87E+06 & 5.27E+06 & 6.40E+07 & 4.26E+06& 1.72E+07&1.79E+07\\
      & $F_{8}$ & \textbf{1.44E+09} & 3.08E+07 & 3.53E+06& 5.29E+06 & 5.98E+07& 4.09E+06&1.69E+07&1.70E+07 \\
      & $F_{9}$ & \textbf{1.57E+09} & 3.62E+07 &4.05E+06 & 6.15E+06 &  6.91E+07& 4.54E+06&1.84E+07& 1.89E+07\\
      & $F_{10}$ & \textbf{9.80E+08} &2.94E+07 & 3.74E+06 & 5.28E+06& 5.69E+07& 4.12E+06& 1.54E+07& 1.63E+07\\
      & $F_{11}$ & \textbf{1.00E+09} & 2.37E+07 &3.76E+06 & 5.38E+06&5.86E+07& 4.04E+06&1.59E+07& 1.64E+07 \\
      & $F_{12}$ & \textbf{9.90E+08} & 2.33E+07 & 3.76E+06 &5.39E+06&5.47E+07& 4.09E+06& 1.57E+07& 1.60E+07\\
\bottomrule
\end{tabular}%
}
  \label{tab:FEs NP10000_supp}%
\end{table}%

\clearpage

\section{Supplementary information for the application of robotics control}\label{section_brax_supp}

\subsection{Illustrations of the robotics control tasks}\label{section_brax_image_supp}

% Figs. \ref{Figure_swimmer}-\ref{Figure_hopper}, illustrate the three robotics control tasks from the Brax \cite{brax} reinforcement learning library: swimmer, reacher, and hopper.

% \begin{enumerate}
%     \item Swimmer: A serpentine agent that must coordinate joint movements to navigate through a fluid environment, aiming for efficient propulsion and forward movement.
%     \item Reacher: A robotic arm environment where the goal is to control joint torques to reach a target point with precision, testing the fine control of the learning algorithm.
%     \item Hopper: A one-legged robot that must learn to balance and hop forward as far and as fast as possible, providing a benchmark for locomotion and stability in dynamic environments.
% \end{enumerate}

\begin{figure}[!h]
\centering
\includegraphics[width=6cm, height=3cm]{su_swimmer.png}
\caption{The swimmer task in Brax. It is designed to simulate a multi-jointed creature navigating through a fluid medium.}
\label{Figure_swimmer_supp}
\end{figure}


\begin{figure}[!h]
\centering
\includegraphics[width=6cm, height=3cm]{su_hopper.png}
\caption{The hopper task in Brax. It resembles a one-legged robotic creature with the objective to hop forward smoothly and quickly.}
\label{Figure_hopper_supp}
\end{figure}

\begin{figure}[!h]
\centering
\includegraphics[width=6cm, height=3cm]{su_reacher.png}
\caption{The reacher task in Brax. It simulates a robotic arm tasked with reaching a target location.}
\label{Figure_reacher_supp}
\end{figure}


\subsection{Supplementary results}\label{section_brax_results_supp}
% Table \ref{tab:comparative-reward-analysis} displays the results of the neuroevolution experimrnt for 60 minutes.

\begin{table}[htbp]
\centering
\caption{Rewards achieved by MetaDE and peer evolutionary algorithms on the robotics tasks. 
The mean and standard deviation (in parentheses) of the results over multiple runs are displayed in pairs. 
Results with the best mean values are highlighted.
}
\label{tab:comparative-reward-analysis_supp}
{%
 \renewcommand{\arraystretch}{1}
\renewcommand{\tabcolsep}{3pt}
% Table generated by Excel2LaTeX from sheet 'Exp6 brax'
\begin{tabular}{cccccccc}
\toprule
Task & MetaDE  & \textbf{CSO } & \textbf{CMAES } & \textbf{SHADE } & \textbf{DE}&\textbf{LSHADE-RSP}&\textbf{EDEV} \\
\midrule
Swimmer  & \textbf{ 190.85 (2.39) } &  183.45 (1.15)  &  186.07 (2.68)  & 185.90 (3.31) &  182.15 (2.54) &186.36 (1.21)&145.52 (41.22)\\
%\midrule
Hopper   &  1187.53 (122.72) & 1330.45 (325.55) &  \textbf{1389.72 (474.41)}  &  1022.25 (122.94)  &  871.06 (147.12)&1102.23 (100.33)&457.22 (80.16) \\
%\midrule
Reacher  &  -21.05 (5.05)  &  -504.07 (112.18)  & \textbf{ -3.76 (1.05) } &  -343.18 (105.65)  &  -522.99 (142.78)    &    -342.74 (36.63)  &  -493.46 (173.67)\\
\bottomrule
\end{tabular}%

}
\end{table}

\clearpage
\section{More Evolvers}\label{subsection Evolver Comparison_supp}
We selected \texttt{DE/rand/1/bin} as the evolver for its simplicity and adaptability, hypothesizing its capability to self-evolve. Nonetheless, it is also interesting to evaluate the performance implications of utilizing other EAs as evolvers.

In this experiment, we maintained the identical framework and parameter settings for MetaDE as utilized in the primary experiments. 
The distinction lies in the comparative evaluation of the effectiveness of DE, PSO, Natural Evolution Strategies (NES) \citesupp{NES}, and Competitive Swarm Optimization (CSO) \citesupp{CSO} as evolvers. 
For PSO, parameters were set to the recommended values of $w=0.729$ and $c_1=c_2=1.49$ \citesupp{PSOParamSetting}. 
It is important to note that CSO and NES do not require parameter tuning.

As shown in Tables \ref{tab:diffTunner 10D}-\ref{tab:diffTunner 20D} and Figs. \ref{Figure_evolver_10D}-\ref{Figure_evolver_20D}, the experimental outcomes suggest minimal performance disparities among the EAs when employed as evolvers. 
Specifically, DE demonstrates a marginal advantage in 10-dimensional problems, whereas the performance is comparably uniform across all evolvers for 20-dimensional problems. 
These results imply that the selection of an evolver within the MetaDE framework is relatively flexible. 
Such findings highlight the adaptability and flexibility of MetaDE, illustrating that its efficiency is not significantly influenced by the particular choice of evolver.


% Table generated by Excel2LaTeX from sheet 'Sheet1'
\begin{table}[htbp]
  \centering
  \caption{
  Performance of MetaDE with different evolvers on 10D problems in CEC2022 benchmark suite. 
The mean and standard deviation (in parentheses) of the results over multiple runs are displayed in pairs. 
Results with the best mean values are highlighted.
  }
  {
         \renewcommand{\arraystretch}{1}
 \renewcommand{\tabcolsep}{10pt}
% Table generated by Excel2LaTeX from sheet 'Exp5 evolver'
\begin{tabular}{ccccc}
\toprule
Func  & DE    & PSO   & NES   & CSO \\
\midrule
$F_{1}$ & \textbf{0.00E+00 (0.00E+00)} & \boldmath{}\textbf{0.00E+00 (0.00E+00)$\approx$}\unboldmath{} & \boldmath{}\textbf{0.00E+00 (0.00E+00)$\approx$}\unboldmath{} & \boldmath{}\textbf{0.00E+00 (0.00E+00)$\approx$}\unboldmath{} \\
$F_{2}$ & \textbf{0.00E+00 (0.00E+00)} & \boldmath{}\textbf{0.00E+00 (0.00E+00)$\approx$}\unboldmath{} & 1.60E-06 (2.82E-06)$-$ & \boldmath{}\textbf{0.00E+00 (0.00E+00)$\approx$}\unboldmath{} \\
$F_{3}$ & \textbf{0.00E+00 (0.00E+00)} & \boldmath{}\textbf{0.00E+00 (0.00E+00)$\approx$}\unboldmath{} & \boldmath{}\textbf{0.00E+00 (0.00E+00)$\approx$}\unboldmath{} & \boldmath{}\textbf{0.00E+00 (0.00E+00)$\approx$}\unboldmath{} \\
$F_{4}$ & \textbf{0.00E+00 (0.00E+00)} & 1.41E-03 (7.70E-03)$\approx$ & 3.23E-02 (1.76E-01)$\approx$ & \boldmath{}\textbf{0.00E+00 (0.00E+00)$\approx$}\unboldmath{} \\
$F_{5}$ & \textbf{0.00E+00 (0.00E+00)} & \boldmath{}\textbf{0.00E+00 (0.00E+00)$\approx$}\unboldmath{} & \boldmath{}\textbf{0.00E+00 (0.00E+00)$\approx$}\unboldmath{} & \boldmath{}\textbf{0.00E+00 (0.00E+00)$\approx$}\unboldmath{} \\
$F_{6}$ & \textbf{5.50E-04 (3.96E-04)} & 1.45E-03 (1.61E-03)$\approx$ & 1.66E-03 (1.79E-03)$-$ & 1.37E-03 (9.80E-04)$-$ \\
$F_{7}$ & \textbf{0.00E+00 (0.00E+00)} & \boldmath{}\textbf{0.00E+00 (0.00E+00)$\approx$}\unboldmath{} & \boldmath{}\textbf{0.00E+00 (0.00E+00)$\approx$}\unboldmath{} & \boldmath{}\textbf{0.00E+00 (0.00E+00)$\approx$}\unboldmath{} \\
$F_{8}$ & 5.52E-03 (4.41E-03) & \textbf{1.77E-06 (1.48E-06)$+$} & 1.83E-02 (1.09E-02)$-$ & 1.57E-03 (2.26E-03)$+$ \\
$F_{9}$ & \textbf{3.36E+00 (1.77E+01)} & 1.40E+01 (4.63E+01)$-$ & 2.27E+01 (4.18E+01)$\approx$ & 1.75E+01 (4.86E+01)$-$ \\
$F_{10}$ & \textbf{0.00E+00 (0.00E+00)} & \boldmath{}\textbf{0.00E+00 (0.00E+00)$\approx$}\unboldmath{} & \boldmath{}\textbf{0.00E+00 (0.00E+00)$\approx$}\unboldmath{} & 7.38E-02 (3.99E-01)$\approx$ \\
$F_{11}$ & \textbf{0.00E+00 (0.00E+00)} & \boldmath{}\textbf{0.00E+00 (0.00E+00)$\approx$}\unboldmath{} & \boldmath{}\textbf{0.00E+00 (0.00E+00)$\approx$}\unboldmath{} & \boldmath{}\textbf{0.00E+00 (0.00E+00)$\approx$}\unboldmath{} \\
$F_{12}$ & \textbf{1.39E+02 (4.63E+01)} & 1.48E+02 (3.39E+01)$\approx$ & 1.59E+02 (6.82E-06)$-$ & 1.54E+02 (1.78E+01)$\approx$ \\
\midrule
$+$ / $\approx$ / $-$ & --    & 1/10/1 & 4/8/0 & 1/9/2 \\
\bottomrule
\end{tabular}%
}

\footnotesize
\textsuperscript{*} The Wilcoxon rank-sum tests (with a significance level of 0.05) were conducted between MetaDE and each individually.
The final row displays the number of problems where the corresponding evlover performs statistically better ($+$),  similar ($\thickapprox$), or worse ($-$) compared to DE.\\
\label{tab:diffTunner 10D_supp}%
\end{table}%


% Table generated by Excel2LaTeX from sheet 'Sheet1'
\begin{table}[htbp]
  \centering
  \caption{
  Performance of MetaDE with different evolvers on 20D problems in CEC2022 benchmark suite.
The mean and standard deviation (in parentheses) of the results over multiple runs are displayed in pairs. 
Results with the best mean values are highlighted.
  }
  {
        \renewcommand{\arraystretch}{1}
 \renewcommand{\tabcolsep}{10pt}
% Table generated by Excel2LaTeX from sheet 'Exp5 evolver'
\begin{tabular}{ccccc}
\toprule
Func  & DE    & PSO   & NES   & CSO \\
\midrule
$F_{1}$ & \multicolumn{1}{l}{\textbf{0.00E+00 (0.00E+00)}} & \multicolumn{1}{l}{\boldmath{}\textbf{0.00E+00 (0.00E+00)$\approx$}\unboldmath{}} & \multicolumn{1}{l}{\boldmath{}\textbf{0.00E+00 (0.00E+00)$\approx$}\unboldmath{}} & \multicolumn{1}{l}{\boldmath{}\textbf{0.00E+00 (0.00E+00)$\approx$}\unboldmath{}} \\
$F_{2}$ & \multicolumn{1}{l}{1.26E-02 (3.74E-02)} & \multicolumn{1}{l}{4.28E-02 (1.25E-01)$\approx$} & \multicolumn{1}{l}{1.11E+00 (1.30E+00)$-$} & \multicolumn{1}{l}{\textbf{0.00E+00 (0.00E+00)$+$}} \\
$F_{3}$ & \multicolumn{1}{l}{\textbf{0.00E+00 (0.00E+00)}} & \multicolumn{1}{l}{\boldmath{}\textbf{0.00E+00 (0.00E+00)$\approx$}\unboldmath{}} & \multicolumn{1}{l}{\boldmath{}\textbf{0.00E+00 (0.00E+00)$\approx$}\unboldmath{}} & \multicolumn{1}{l}{\boldmath{}\textbf{0.00E+00 (0.00E+00)$\approx$}\unboldmath{}} \\
$F_{4}$ & \multicolumn{1}{l}{\textbf{2.02E+00 (8.56E-01)}} & \multicolumn{1}{l}{2.51E+00 (1.19E+00)$-$} & \multicolumn{1}{l}{5.53E+00 (1.55E+00)$-$} & \multicolumn{1}{l}{2.85E+00 (1.48E+00)$\approx$} \\
$F_{5}$ & \multicolumn{1}{l}{\textbf{0.00E+00 (0.00E+00)}} & \multicolumn{1}{l}{\boldmath{}\textbf{0.00E+00 (0.00E+00)$\approx$}\unboldmath{}} & \multicolumn{1}{l}{\boldmath{}\textbf{0.00E+00 (0.00E+00)$\approx$}\unboldmath{}} & \multicolumn{1}{l}{\boldmath{}\textbf{0.00E+00 (0.00E+00)$\approx$}\unboldmath{}} \\
$F_{6}$ & \multicolumn{1}{l}{1.16E-01 (2.79E-02)} & \multicolumn{1}{l}{1.81E-01 (2.42E-01)$-$} & \multicolumn{1}{l}{9.32E-02 (2.65E-02)$+$} & \multicolumn{1}{l}{\boldmath{}\textbf{1.10E-01 (2.85E-02)$\approx$}\unboldmath{}} \\
$F_{7}$ & \multicolumn{1}{l}{5.17E-02 (6.21E-02)} & \multicolumn{1}{l}{1.61E-01 (3.84E-01)$\approx$} & \multicolumn{1}{l}{8.13E-01 (6.01E-01)$-$} & \multicolumn{1}{l}{\textbf{6.37E-02 (2.03E-01)$+$}} \\
$F_{8}$ & \multicolumn{1}{l}{\textbf{7.19E-01 (1.02E+00)}} & \multicolumn{1}{l}{2.31E+00 (4.16E+00)$\approx$} & \multicolumn{1}{l}{1.82E+01 (3.82E+00)$-$} & \multicolumn{1}{l}{2.19E+00 (4.05E+00)$\approx$} \\
$F_{9}$ & \multicolumn{1}{l}{\textbf{1.07E+02 (1.98E+01)}} & \multicolumn{1}{l}{1.34E+02 (3.72E+01)$-$} & \multicolumn{1}{l}{1.81E+02 (3.75E-06)$-$} & \multicolumn{1}{l}{1.75E+02 (1.81E+01)$-$} \\
$F_{10}$ & \multicolumn{1}{l}{\textbf{0.00E+00 (0.00E+00)}} & \multicolumn{1}{l}{\boldmath{}\textbf{0.00E+00 (0.00E+00)$\approx$}\unboldmath{}} & \multicolumn{1}{l}{\boldmath{}\textbf{0.00E+00 (0.00E+00)$\approx$}\unboldmath{}} & \multicolumn{1}{l}{7.86E-01 (2.39E+00)$-$} \\
$F_{11}$ & \multicolumn{1}{l}{7.28E-05 (3.06E-04)} & \multicolumn{1}{l}{1.01E-02 (3.90E-02)$\approx$} & \multicolumn{1}{l}{8.79E-03 (1.52E-02)$-$} & \multicolumn{1}{l}{\textbf{1.45E-05 (0.00E+00)$-$}} \\
$F_{12}$ & \multicolumn{1}{l}{\textbf{2.29E+02 (6.08E-01)}} & \multicolumn{1}{l}{2.29E+02 (1.17E+00)$\approx$} & \multicolumn{1}{l}{2.31E+02 (8.77E-01)$-$} & \multicolumn{1}{l}{2.30E+02 (9.60E-01)$-$} \\
\midrule
$+$ / $\approx$ / $-$ & --    & 0/9/3 & 1/4/7 & 2/6/4 \\
\bottomrule
\end{tabular}%
}

\footnotesize
\textsuperscript{*} The Wilcoxon rank-sum tests (with a significance level of 0.05) were conducted between MetaDE and each individually.
The final row displays the number of problems where the corresponding evlover performs statistically better ($+$),  similar ($\thickapprox$), or worse ($-$) compared to DE.\\
\label{tab:diffTunner 20D_supp}%
\end{table}%

\clearpage

\begin{figure*}[htpb]
\centering
\includegraphics[scale=0.29]{su_D10_all_evolver.pdf}
\caption{Convergence curves with different evolvers on 10D problems in CEC2022 benchmark suite.}
\label{Figure_evolver_10D_supp}
\end{figure*}


\begin{figure*}[htpb]
\centering
\includegraphics[scale=0.29]{su_D20_all_evolver.pdf}
\caption{Convergence curves with different evolvers on 20D problems in CEC2022 benchmark suite.}
\label{Figure_evolver_20D_supp}
\end{figure*}


\clearpage
% \bibliography{Supplement_references}

% \bibliographystylesupp{IEEEtran}
% \bibliographysupp{Supplement_references}

%\title{Generating 3D \hl{Small} Binding Molecules Using Shape-Conditioned Diffusion Models with Guidance}
%\date{\vspace{-5ex}}

%\author{
%	Ziqi Chen\textsuperscript{\rm 1}, 
%	Bo Peng\textsuperscript{\rm 1}, 
%	Tianhua Zhai\textsuperscript{\rm 2},
%	Xia Ning\textsuperscript{\rm 1,3,4 \Letter}
%}
%\newcommand{\Address}{
%	\textsuperscript{\rm 1}Computer Science and Engineering, The Ohio Sate University, Columbus, OH 43210.
%	\textsuperscript{\rm 2}Perelman School of Medicine, University of Pennsylvania, Philadelphia, PA 19104.
%	\textsuperscript{\rm 3}Translational Data Analytics Institute, The Ohio Sate University, Columbus, OH 43210.
%	\textsuperscript{\rm 4}Biomedical Informatics, The Ohio Sate University, Columbus, OH 43210.
%	\textsuperscript{\Letter}ning.104@osu.edu
%}

%\newcommand\affiliation[1]{%
%	\begingroup
%	\renewcommand\thefootnote{}\footnote{#1}%
%	\addtocounter{footnote}{-1}%
%	\endgroup
%}



\setcounter{secnumdepth}{2} %May be changed to 1 or 2 if section numbers are desired.

\setcounter{section}{0}
\renewcommand{\thesection}{S\arabic{section}}

\setcounter{table}{0}
\renewcommand{\thetable}{S\arabic{table}}

\setcounter{figure}{0}
\renewcommand{\thefigure}{S\arabic{figure}}

\setcounter{algorithm}{0}
\renewcommand{\thealgorithm}{S\arabic{algorithm}}

\setcounter{equation}{0}
\renewcommand{\theequation}{S\arabic{equation}}


\begin{center}
	\begin{minipage}{0.95\linewidth}
		\centering
		\LARGE 
	Generating 3D Binding Molecules Using Shape-Conditioned Diffusion Models with Guidance (Supplementary Information)
	\end{minipage}
\end{center}
\vspace{10pt}

%%%%%%%%%%%%%%%%%%%%%%%%%%%%%%%%%%%%%%%%%%%%%
\section{Parameters for Reproducibility}
\label{supp:experiments:parameters}
%%%%%%%%%%%%%%%%%%%%%%%%%%%%%%%%%%%%%%%%%%%%%

We implemented both \SE and \methoddiff using Python-3.7.16, PyTorch-1.11.0, PyTorch-scatter-2.0.9, Numpy-1.21.5, Scikit-learn-1.0.2.
%
We trained the models using a Tesla V100 GPU with 32GB memory and a CPU with 80GB memory on Red Hat Enterprise 7.7.
%
%We released the code, data, and the trained model at Google Drive~\footnote{\url{https://drive.google.com/drive/folders/146cpjuwenKGTd6Zh4sYBy-Wv6BMfGwe4?usp=sharing}} (will release to the public on github once the manuscript is accepted).

%===================================================================
\subsection{Parameters of \SE}
%===================================================================


In \SE, we tuned the dimension of all the hidden layers including VN-DGCNN layers
(Eq.~\ref{eqn:shape_embed}), MLP layers (Eq.~\ref{eqn:se:decoder}) and
VN-In layer (Eq.~\ref{eqn:se:decoder}), and the dimension $d_p$ of generated shape latent embeddings $\shapehiddenmat$ with the grid-search algorithm in the 
parameter space presented in Table~\ref{tbl:hyper_se}.
%
We determined the optimal hyper-parameters according to the mean squared errors of the predictions of signed distances for 1,000 validation molecules that are selected as described in Section ``Data'' 
in the main manuscript.
%
The optimal dimension of all the hidden layers is 256, and the optimal dimension $d_p$ of shape latent embedding \shapehiddenmat is 128.
%
The optimal number of points $|\pc|$ in the point cloud \pc is 512.
%
We sampled 1,024 query points in $\mathcal{Z}$ for each molecule shape.
%
We constructed graphs from point clouds, which are employed to learn $\shapehiddenmat$ with VN-DGCNN layer (Eq.~\ref{eqn:shape_embed}), using the $k$-nearest neighbors based on Euclidean distance with $k=20$.
%
We set the number of VN-DGCNN layers as 4.
%
We set the number of MLP layers in the decoder (Eq.~\ref{eqn:se:decoder}) as 2.
%
We set the number of VN-In layers as 1.

%
We optimized the \SE model via Adam~\cite{adam} with its parameters (0.950, 0.999), %betas (0.95, 0.999), 
learning rate 0.001, and batch size 16.
%
We evaluated the validation loss every 2,000 training steps.
%
We scheduled to decay the learning rate with a factor of 0.6 and a minimum learning rate of 1e-6 if 
the validation loss does not decrease in 5 consecutive evaluations.
%
The optimal \SE model has 28.3K learnable parameters. 
%
We trained the \SE model %for at most 80 hours 
with $\sim$156,000 training steps.
%
The training took 80 hours with our GPUs.
%
The trained \SE model achieved the minimum validation loss at 152,000 steps.


\begin{table*}[!h]
  \centering
      \caption{{Hyper-Parameter Space for \SE Optimization}}
  \label{tbl:hyper_se}
  \begin{threeparttable}
 \begin{scriptsize}
      \begin{tabular}{
%	@{\hspace{2pt}}l@{\hspace{2pt}}
	@{\hspace{2pt}}l@{\hspace{5pt}} 
	@{\hspace{2pt}}r@{\hspace{2pt}}         
	}
        \toprule
        %Notation &
          Hyper-parameters &  Space\\
        \midrule
        %$t_a$    & 
         %hidden layer dimension         & \{16, 32, 64, 128\} \\
         %atom/node embedding dimension &  \{16, 32, 64, 128\} \\
         %$\latent^{\add}$/$\latent^{\delete}$ dimension        & \{8, 16, 32, 64\} \\
         hidden layer dimension            & \{128, 256\}\\
         dimension $d_p$ of \shapehiddenmat        &  \{64, 128\} \\
         \#points in \pc        & \{512, 1,024\} \\
         \#query points in $\mathcal{Z}$                & 1,024 \\%1024 \\%\bo{\{1024\}}\\
         \#nearest neighbors              & 20          \\
         \#VN-DGCNN layers (Eq~\ref{eqn:shape_embed})               & 4            \\
         \#MLP layers in Eq~\ref{eqn:se:decoder} & 4           \\
        \bottomrule
      \end{tabular}
%  	\begin{tablenotes}[normal,flushleft]
%  		\begin{footnotesize}
%  	
%  	\item In this table, hidden dimension represents the dimension of hidden layers and 
%  	atom/node embeddings; latent dimension represents the dimension of latent embedding \latent.
%  	\par
%  \end{footnotesize}
%  
%\end{tablenotes}
%      \begin{tablenotes}
%      \item 
%      \par
%      \end{tablenotes}
\end{scriptsize}
  \end{threeparttable}
\end{table*}

%
\begin{table*}[!h]
  \centering
      \caption{{Hyper-Parameter Space for \methoddiff Optimization}}
  \label{tbl:hyper_diff}
  \begin{threeparttable}
 \begin{scriptsize}
      \begin{tabular}{
%	@{\hspace{2pt}}l@{\hspace{2pt}}
	@{\hspace{2pt}}l@{\hspace{5pt}} 
	@{\hspace{2pt}}r@{\hspace{2pt}}         
	}
        \toprule
        %Notation &
          Hyper-parameters &  Space\\
        \midrule
        %$t_a$    & 
         %hidden layer dimension         & \{16, 32, 64, 128\} \\
         %atom/node embedding dimension &  \{16, 32, 64, 128\} \\
         %$\latent^{\add}$/$\latent^{\delete}$ dimension        & \{8, 16, 32, 64\} \\
         scalar hidden layer dimension         & 128 \\
         vector hidden layer dimension         & 32 \\
         weight of atom type loss $\xi$ (Eq.~\ref{eqn:loss})  & 100           \\
         threshold of step weight $\delta$ (Eq.~\ref{eqn:diff:obj:pos}) & 10 \\
         \#atom features $K$                   & 15 \\
         \#layers $L$ in \molpred             & 8 \\
         %\# \eqgnn/\invgnn layers     &  8 \\
         %\# heads {$n_h$} in $\text{MHA}^{\mathtt{x}}/\text{MHA}^{\mathtt{v}}$                               & 16 \\
         \#nearest neighbors {$N$}  (Eq.~\ref{eqn:geometric_embedding} and \ref{eqn:attention})            & 8          \\
         {\#diffusion steps $T$}                  & 1,000 \\
        \bottomrule
      \end{tabular}
%  	\begin{tablenotes}[normal,flushleft]
%  		\begin{footnotesize}
%  	
%  	\item In this table, hidden dimension represents the dimension of hidden layers and 
%  	atom/node embeddings; latent dimension represents the dimension of latent embedding \latent.
%  	\par
%  \end{footnotesize}
%  
%\end{tablenotes}
%      \begin{tablenotes}
%      \item 
%      \par
%      \end{tablenotes}
\end{scriptsize}
  \end{threeparttable}

\end{table*}


%===================================================================
\subsection{Parameters of \methoddiff}
%===================================================================

Table~\ref{tbl:hyper_diff} presents the parameters used to train \methoddiff.
%
In \methoddiff, we set the hidden dimensions of all the MLP layers and the scalar hidden layers in GVPs (Eq.~\ref{eqn:pred:gvp} and Eq.~\ref{eqn:mess:gvp}) as 128. %, including all the MLP layers in \methoddiff and the scalar dimension of GVP layers in Eq.~\ref{eqn:pred:gvp} and Eq.~\ref{eqn:mess:gvp}. %, and MLP layer (Eq.~\ref{eqn:diff:graph:atompred}) as 128.
%
We set the dimensions of all the vector hidden layers in GVPs as 32.
%
We set the number of layers $L$ in \molpred as 8.
%and the number of layers in graph neural networks as 8.
%
Both two GVP modules in Eq.~\ref{eqn:pred:gvp} and Eq.~\ref{eqn:mess:gvp} consist of three GVP layers. %, which consisa GVP modset the number of layer of GVP modules %is a multi-head attention layer ($\text{MHA}^{\mathtt{x}}$ or $\text{MHA}^{\mathtt{h}}$) with 16 heads.
% 
We set the number of VN-MLP layers in Eq.~\ref{eqn:shaper} as 1 and the number of MLP layers as 2 for all the involved MLP functions.
%

We constructed graphs from atoms in molecules, which are employed to learn the scalar embeddings and vector embeddings for atoms %predict atom coordinates and features  
(Eq.~\ref{eqn:geometric_embedding} and \ref{eqn:attention}), using the $N$-nearest neighbors based on Euclidean distance with $N=8$. 
%
We used $K=15$ atom features in total, indicating the atom types and its aromaticity.
%
These atom features include 10 non-aromatic atoms (i.e., ``H'', ``C'', ``N'', ``O'', ``F'', ``P'', ``S'', ``Cl'', ``Br'', ``I''), 
and 5 aromatic atoms (i.e., ``C'', ``N'', ``O'', ``P'', ``S'').
%
We set the number of diffusion steps $T$ as 1,000.
%
We set the weight $\xi$ of atom type loss (Eq.~\ref{eqn:loss}) as $100$ to balance the values of atom type loss and atom coordinate loss.
%
We set the threshold $\delta$ (Eq.~\ref{eqn:diff:obj:pos}) as 10.
%
The parameters $\beta_t^{\mathtt{x}}$ and $\beta_t^{\mathtt{v}}$ of variance scheduling in the forward diffusion process of \methoddiff are discussed in 
Supplementary Section~\ref{supp:forward:variance}.
%
%Please note that as in \squid, we did not perform extensive hyperparameter optimization for \methoddiff.
%
Following \squid, we did not perform extensive hyperparameter tunning for \methoddiff given that the used 
hyperparameters have enabled good performance.

%
We optimized the \methoddiff model via Adam~\cite{adam} with its parameters (0.950, 0.999), learning rate 0.001, and batch size 32.
%
We evaluated the validation loss every 2,000 training steps.
%
We scheduled to decay the learning rate with a factor of 0.6 and a minimum learning rate of 1e-5 if 
the validation loss does not decrease in 10 consecutive evaluations.
%
The \methoddiff model has 7.8M learnable parameters. 
%
We trained the \methoddiff model %for at most 60 hours 
with $\sim$770,000 training steps.
%
The training took 70 hours with our GPUs.
%
The trained \methoddiff achieved the minimum validation loss at 758,000 steps.

During inference, %the sampling, 
following Adams and Coley~\cite{adams2023equivariant}, we set the variance $\phi$ 
of atom-centered Gaussians as 0.049, which is used to build a set of points for shape guidance in Section ``\method with Shape Guidance'' 
in the main manuscript.
%
We determined the number of atoms in the generated molecule using the atom number distribution of training molecules that have surface shape sizes similar to the condition molecule.
%
The optimal distance threshold $\gamma$ is 0.2, and the optimal stop step $S$ for shape guidance is 300.
%
With shape guidance, each time we updated the atom position (Eq.~\ref{eqn:shape_guidance}), we randomly sampled the weight $\sigma$ from $[0.2, 0.8]$. %\bo{(XXX)}.
%
Moreover, when using pocket guidance as mentioned in Section ``\method with Pocket Guidance'' in the main manuscript, each time we updated the atom position (Eq.~\ref{eqn:pocket_guidance}), we randomly sampled the weight $\epsilon$ from $[0, 0.5]$. 
%
For each condition molecule, it took around 40 seconds on average to generate 50 molecule candidates with our GPUs.



%%%%%%%%%%%%%%%%%%%%%%%%%%%%%%%%%%%%%%%%%%%%%%
\section{Performance of \decompdiff with Protein Pocket Prior}
\label{supp:app:decompdiff}
%%%%%%%%%%%%%%%%%%%%%%%%%%%%%%%%%%%%%%%%%%%%%%

In this section, we demonstrate that \decompdiff with protein pocket prior, referred to as \decompdiffbeta, shows very limited performance in generating drug-like and synthesizable molecules compared to all the other methods, including \methodwithpguide and \methodwithsandpguide.
%
We evaluate the performance of \decompdiffbeta in terms of binding affinities, drug-likeness, and diversity.
%
We compare \decompdiffbeta with \methodwithpguide and \methodwithsandpguide and report the results in Table~\ref{tbl:comparison_results_decompdiff}.
%
Note that the results of \methodwithpguide and \methodwithsandpguide here are consistent with those in Table~\ref{tbl:overall_results_docking2} in the main manuscript.
%
As shown in Table~\ref{tbl:comparison_results_decompdiff}, while \decompdiffbeta achieves high binding affinities in Vina M and Vina D, it substantially underperforms \methodwithpguide and \methodwithsandpguide in QED and SA.
%
Particularly, \decompdiffbeta shows a QED score of 0.36, while \methodwithpguide substantially outperforms \decompdiffbeta in QED (0.77) with 113.9\% improvement.
%
\decompdiffbeta also substantially underperforms \methodwithpguide in terms of SA scores (0.55 vs 0.76).
%
These results demonstrate the limited capacity of \decompdiffbeta in generating drug-like and synthesizable molecules.
%
As a result, the generated molecules from \decompdiffbeta can have considerably lower utility compared to other methods.
%
Considering these limitations of \decompdiffbeta, we exclude it from the baselines for comparison.

\begin{table*}[!h]
	\centering
		\caption{Comparison on PMG among \methodwithpguide, \methodwithsandpguide and \decompdiffbeta}
	\label{tbl:comparison_results_decompdiff}
\begin{threeparttable}
	\begin{scriptsize}
	\begin{tabular}{
		@{\hspace{2pt}}l@{\hspace{2pt}}
		%
		%@{\hspace{2pt}}l@{\hspace{2pt}}
		%
		@{\hspace{2pt}}r@{\hspace{2pt}}
		@{\hspace{2pt}}r@{\hspace{2pt}}
		%
		@{\hspace{6pt}}r@{\hspace{6pt}}
		%
		@{\hspace{2pt}}r@{\hspace{2pt}}
		@{\hspace{2pt}}r@{\hspace{2pt}}
		%
		@{\hspace{5pt}}r@{\hspace{5pt}}
		%
		@{\hspace{2pt}}r@{\hspace{2pt}}
		@{\hspace{2pt}}r@{\hspace{2pt}}
		%
		@{\hspace{5pt}}r@{\hspace{5pt}}
		%
		@{\hspace{2pt}}r@{\hspace{2pt}}
	         @{\hspace{2pt}}r@{\hspace{2pt}}
		%
		@{\hspace{5pt}}r@{\hspace{5pt}}
		%
		@{\hspace{2pt}}r@{\hspace{2pt}}
		@{\hspace{2pt}}r@{\hspace{2pt}}
		%
		@{\hspace{5pt}}r@{\hspace{5pt}}
		%
		@{\hspace{2pt}}r@{\hspace{2pt}}
		@{\hspace{2pt}}r@{\hspace{2pt}}
		%
		@{\hspace{5pt}}r@{\hspace{5pt}}
		%
		@{\hspace{2pt}}r@{\hspace{2pt}}
		@{\hspace{2pt}}r@{\hspace{2pt}}
		%
		@{\hspace{5pt}}r@{\hspace{5pt}}
		%
		@{\hspace{2pt}}r@{\hspace{2pt}}
		%@{\hspace{2pt}}r@{\hspace{2pt}}
		%@{\hspace{2pt}}r@{\hspace{2pt}}
		}
		\toprule
		\multirow{2}{*}{method} & \multicolumn{2}{c}{Vina S$\downarrow$} & & \multicolumn{2}{c}{Vina M$\downarrow$} & & \multicolumn{2}{c}{Vina D$\downarrow$} & & \multicolumn{2}{c}{{HA\%$\uparrow$}}  & & \multicolumn{2}{c}{QED$\uparrow$} & & \multicolumn{2}{c}{SA$\uparrow$} & & \multicolumn{2}{c}{Div$\uparrow$} & %& \multirow{2}{*}{SR\%$\uparrow$} & 
		& \multirow{2}{*}{time$\downarrow$} \\
	    \cmidrule{2-3}\cmidrule{5-6} \cmidrule{8-9} \cmidrule{11-12} \cmidrule{14-15} \cmidrule{17-18} \cmidrule{20-21}
		& Avg. & Med. &  & Avg. & Med. &  & Avg. & Med. & & Avg. & Med.  & & Avg. & Med.  & & Avg. & Med.  & & Avg. & Med.  & & \\ %& & \\
		%\multirow{2}{*}{method} & \multirow{2}{*}{\#c\%} &  \multirow{2}{*}{\#u\%} &  \multirow{2}{*}{QED} & \multicolumn{3}{c}{$\nmax=50$} & & \multicolumn{2}{c}{$\nmax=1$}\\
		%\cmidrule(r){5-7} \cmidrule(r){8-10} 
		%& & & & \avgshapesim(std) & \avggraphsim(std  &  \diversity(std  & & \avgshapesim(std) & \avggraphsim(std \\
		\midrule
		%Reference                          & -5.32 & -5.66 & & -5.78 & -5.76 & & -6.63 & -6.67 & & - & - & & 0.53 & 0.49 & & 0.77 & 0.77 & & - & - & %& 23.1 & & - \\
		%\midrule
		%\multirow{4}{*}{PM} 
		%& \AR & -5.06 & -4.99 & &  -5.59 & -5.29 & &  -6.16 & -6.05 & &  37.69 & 31.00 & &  0.50 & 0.49 & &  0.66 & 0.65 & & - & - & %& 7.0 & 
		%& 7,789 \\
		%& \pockettwomol   & -4.50 & -4.21 & &  -5.70 & -5.27 & &  -6.43 & -6.25 & &  48.00 & 51.00 & &  0.58 & 0.58 & &  \textbf{0.77} & \textbf{0.78} & &  0.69 & 0.71 &  %& 24.9 & 
		%& 2,544 \\
		%& \targetdiff     & -4.88 & \underline{-5.82} & &  -6.20 & \underline{-6.36} & &  \textbf{-7.37} & \underline{-7.51} & &  57.57 & 58.27 & &  0.50 & 0.51 & &  0.60 & 0.59 & &  0.72 & 0.71 & % & 10.4 & 
		%& 1,252 \\
		 \decompdiffbeta             & -4.72 & -4.86 & & \textbf{-6.84} & \textbf{-6.91} & & \textbf{-8.85} & \textbf{-8.90} & &  {72.16} & {72.16} & &  0.36 & 0.36 & &  0.55 & 0.55 & & 0.59 & 0.59 & & 3,549 \\ 
		%-4.76 & -6.18 & &  \textbf{-6.86} & \textbf{-7.52} & &  \textbf{-8.85} & \textbf{-8.96} & &  \textbf{72.7} & \textbf{89.8} & &  0.36 & 0.34 & &  0.55 & 0.57 & & 0.59 & 0.59 & & 15.4 \\
		%& \decompdiffref  & -4.58 & -4.77 & &  -5.47 & -5.51 & &  -6.43 & -6.56 & &  47.76 & 48.66 & &  0.56 & 0.56 & &  0.70 & 0.69  & &  0.72 & 0.72 &  %& 15.2 & 
		%& 1,859 \\
		%\midrule
		%\multirow{2}{*}{PC}
		\methodwithpguide       &  \underline{-5.53} & \underline{-5.64} & & {-6.37} & -6.33 & &  \underline{-7.19} & \underline{-7.52} & &  \underline{78.75} & \textbf{94.00} & &  \textbf{0.77} & \textbf{0.80} & &  \textbf{0.76} & \textbf{0.76} & & 0.63 & 0.66 & & 462 \\
		\methodwithsandpguide   & \textbf{-5.81} & \textbf{-5.96} & &  \underline{-6.50} & \underline{-6.58} & & -7.16 & {-7.51} & &  \textbf{79.92} & \underline{93.00} & &  \underline{0.76} & \underline{0.79} & &  \underline{0.75} & \underline{0.74} & & 0.64 & 0.66 & & 561\\
		\bottomrule
	\end{tabular}%
	\begin{tablenotes}[normal,flushleft]
		\begin{footnotesize}
	\item 
\!\!Columns represent: {``Vina S'': the binding affinities between the initially generated poses of molecules and the protein pockets; 
		``Vina M'': the binding affinities between the poses after local structure minimization and the protein pockets;
		``Vina D'': the binding affinities between the poses determined by AutoDock Vina~\cite{Eberhardt2021} and the protein targets;
		``QED'': the drug-likeness score;
		``SA'': the synthesizability score;
		``Div'': the diversity among generated molecules;
		``time'': the time cost to generate molecules.}
		
		\par
		\par
		\end{footnotesize}
	\end{tablenotes}
	\end{scriptsize}
\end{threeparttable}
  \vspace{-10pt}    
\end{table*}



%===================================================================
\section{{Additional Experimental Results on SMG}}
\label{supp:app:results}
%===================================================================

%-------------------------------------------------------------------------------------------------------------------------------------
\subsection{Comparison on Shape and Graph Similarity}
\label{supp:app:results:overall_shape}
%-------------------------------------------------------------------------------------------------------------------------------------

%\ziqi{Outline for this section:
%	\begin{itemize}
%		\item \method can consistently generate molecules with novel structures (low graph similarity) and similar shapes (high shape similarity), such that these molecules have comparable binding capacity with the condition molecules, and potentially better properties as will be shown in Table~\ref{tbl:overall_results_quality_10}.
%	\end{itemize}
%}

\begin{table*}[!h]
	\centering
		\caption{Similarity Comparison on SMG}
	\label{tbl:overall_sim}
\begin{threeparttable}
	\begin{scriptsize}
	\begin{tabular}{
		@{\hspace{0pt}}l@{\hspace{8pt}}
		%
		@{\hspace{8pt}}l@{\hspace{8pt}}
		%
		@{\hspace{8pt}}c@{\hspace{8pt}}
		@{\hspace{8pt}}c@{\hspace{8pt}}
		%
	    	@{\hspace{0pt}}c@{\hspace{0pt}}
		%
		@{\hspace{8pt}}c@{\hspace{8pt}}
		@{\hspace{8pt}}c@{\hspace{8pt}}
		%
		%@{\hspace{8pt}}r@{\hspace{8pt}}
		}
		\toprule
		$\delta_g$  & method          & \avgshapesim$\uparrow$(std) & \avggraphsim$\downarrow$(std) & & \maxshapesim$\uparrow$(std) & \maxgraphsim$\downarrow$(std)       \\ %& \#n\%$\uparrow$  \\ 
		\midrule
		%\multirow{5}{0.079\linewidth}%{\hspace{0pt}0.1} & \dataset   & 0.0             & 0.628(0.139)          & 0.567(0.068)          & 0.078(0.010)          &  & 0.588(0.086)          & 0.081(0.013)          & 4.7              \\
		%&  \squid($\lambda$=0.3) & 0.0             & 0.320(0.000)          & 0.420(0.163)          & \textbf{0.056}(0.032) &  & 0.461(0.170)          & \textbf{0.065}(0.033) & 1.4              \\
		%& \squid($\lambda$=1.0) & 0.0             & 0.414(0.177)          & 0.483(0.184)          & \underline{0.064}(0.030)  &  & 0.531(0.182)          & \underline{0.073}(0.029)  & 2.4              \\
		%& \method               & \underline{1.6}     & \textbf{0.857}(0.034) & \underline{0.773}(0.045)  & 0.086(0.011)          &  & \underline{0.791}(0.053)  & 0.087(0.012)          & \underline{5.1}      \\
		%& \methodwithsguide      & \textbf{3.7}    & \underline{0.833}(0.062)  & \textbf{0.812}(0.037) & 0.088(0.009)          &  & \textbf{0.835}(0.047) & 0.089(0.010)          & \textbf{6.2}     \\ 
		%\cmidrule{2-10}
		%& improv\% & - & 36.5 & 43.2 & -53.6 &  & 42.0 & -33.8 & 31.9  \\
		%\midrule
		\multirow{6}{0.059\linewidth}{\hspace{0pt}0.3} & \dataset             & 0.745(0.037)          & \textbf{0.211}(0.026) &  & 0.815(0.039)          & \textbf{0.215}(0.047)      \\ %    & \textbf{100.0}   \\
			& \squid($\lambda$=0.3) & 0.709(0.076)          & 0.237(0.033)          &  & 0.841(0.070)          & 0.253(0.038)        \\ %  & 45.5             \\
		    & \squid($\lambda$=1.0) & 0.695(0.064)          & \underline{0.216}(0.034)  &  & 0.841(0.056)          & 0.231(0.047)        \\ %  & 84.3             \\
			& \method               & \underline{0.770}(0.039)  & 0.217(0.031)          &  & \underline{0.858}(0.038)  & \underline{0.220}(0.046)  \\ %& \underline{87.1}     \\
			& \methodwithsguide     & \textbf{0.823}(0.029) & 0.217(0.032)          &  & \textbf{0.900}(0.028) & 0.223(0.048)  \\ % & 86.0             \\ 
		%\cmidrule{2-7}
		%& improv\% & 10.5 & -2.8 &  & 7.0 & -2.3  \\ % & %-12.9  \\
		\midrule
		\multirow{6}{0.059\linewidth}{\hspace{0pt}0.5} & \dataset & 0.750(0.037)          & \textbf{0.225}(0.037) &  & 0.819(0.039)          & \textbf{0.236}(0.070)          \\ %& \textbf{100.0}   \\
			& \squid($\lambda$=0.3)  & 0.728(0.072)          & 0.301(0.054)          &  & \underline{0.888}(0.061)  & 0.355(0.088)          \\ %& 85.9             \\
			& \squid($\lambda$=1.0)  & 0.699(0.063)          & 0.233(0.043)          &  & 0.850(0.057)          & 0.263(0.080)          \\ %& \underline{99.5}     \\
			& \method               & \underline{0.771}(0.039)  & \underline{0.229}(0.043)  &  & 0.862(0.036)          & \textbf{0.236}(0.065) \\ %& 99.2             \\
			& \methodwithsguide    & \textbf{0.824}(0.029) & \underline{0.229}(0.044)  &  & \textbf{0.903}(0.027) & \underline{0.242}(0.069)  \\ %& 99.0             \\ 
		%\cmidrule{2-7}
		%& improv\% & 9.9 & -1.8 &  & 1.7 & 0.0 \\ %& -0.8  \\
		\midrule
		\multirow{6}{0.059\linewidth}{\hspace{0pt}0.7} 
		& \dataset &  0.750(0.037) & \textbf{0.226}(0.038) & & 0.819(0.039) & \underline{0.240}(0.081) \\ %& \textbf{100.0} \\
		%& \dataset & 12.3            & 0.736(0.076)          & 0.768(0.037)          & \textbf{0.228}(0.042) &  & 0.819(0.039)          & \underline{0.242}(0.085)  & \textbf{100.0}   \\
			& \squid($\lambda$=0.3) &  0.735(0.074)          & 0.328(0.070)          &  & \underline{0.900}(0.062)  & 0.435(0.143)          \\ %& 95.4             \\
			& \squid($\lambda$=1.0) &  0.699(0.064)          & 0.234(0.045)          &  & 0.851(0.057)          & 0.268(0.090)          \\ %& \underline{99.9}     \\
			& \method               &  \underline{0.771}(0.039)  & \underline{0.229}(0.043)  &  & 0.862(0.036)          & \textbf{0.237}(0.066) \\ %& 99.3             \\
			& \methodwithsguide     &  \textbf{0.824}(0.029) & 0.230(0.045)          &  & \textbf{0.903}(0.027) & 0.244(0.074)          \\ %& 99.2             \\ 
		%\cmidrule{2-7}
		%& improv\% & 9.9 & -1.3 &  & 0.3 & 1.3 \\%& -0.7  \\
		\midrule
		\multirow{6}{0.059\linewidth}{\hspace{0pt}1.0} 
		& \dataset & 0.750(0.037)          & \textbf{0.226}(0.038) &  & 0.819(0.039)          & \underline{0.242}(0.085)  \\%& \textbf{100.0}  \\
		& \squid($\lambda$=0.3) & 0.740(0.076)          & 0.349(0.088)          &  & \textbf{0.909}(0.065) & 0.547(0.245)       \\ %   & \textbf{100.0}  \\
		& \squid($\lambda$=1.0) & 0.699(0.064)          & 0.235(0.045)          &  & 0.851(0.057)          & 0.271(0.097)          \\ %& \textbf{100.0}   \\
		& \method               & \underline{0.771}(0.039)  & \underline{0.229}(0.043)  &  & 0.862(0.036)          & \textbf{0.237}(0.066) \\ %& \underline{99.3}  \\
		& \methodwithsguide      & \textbf{0.824}(0.029) & 0.230(0.045)          &  & \underline{0.903}(0.027)  & 0.244(0.076)          \\ %& 99.2            \\
		%\cmidrule{2-7}
		%& improv\% &  9.9               & -1.3              &  & -0.7              & -2.1           \\ %       & -0.7 \\
		\bottomrule
	\end{tabular}%
	\begin{tablenotes}[normal,flushleft]
		\begin{footnotesize}
	\item 
\!\!Columns represent: ``$\delta_g$'': the graph similarity constraint; 
%``\#d\%'': the percentage of molecules that satisfy the graph similarity constraint and are with high \shapesim ($\shapesim>=0.8$);
%``\diversity'': the diversity among the generated molecules;
``\avgshapesim/\avggraphsim'': the average of shape or graph similarities between the condition molecules and generated molecules with $\graphsim<=\delta_g$;
``\maxshapesim'': the maximum of shape similarities between the condition molecules and generated molecules with $\graphsim<=\delta_g$;
``\maxgraphsim'': the graph similarities between the condition molecules and the molecules with the maximum shape similarities and $\graphsim<=\delta_g$;
%``\#n\%'': the percentage of molecules that satisfy the graph similarity constraint ($\graphsim<=\delta_g$).
%
``$\uparrow$'' represents higher values are better, and ``$\downarrow$'' represents lower values are better.
%
 Best values are in \textbf{bold}, and second-best values are \underline{underlined}. 
\par
		\par
		\end{footnotesize}
	\end{tablenotes}
\end{scriptsize}
\end{threeparttable}
  \vspace{-10pt}    
\end{table*}
%\label{tbl:overall_sim}


{We evaluate the shape similarity \shapesim and graph similarity \graphsim of molecules generated from}
%Table~\ref{tbl:overall_sim} presents the comparison of shape-conditioned molecule generation among 
\dataset, \squid, \method and \methodwithsguide under different graph similarity constraints  ($\delta_g$=1.0, 0.7, 0.5, 0.3). 
%
%During the evaluation, for each molecule in the test set, all the methods are employed to generate or identify 50 molecules with similar shapes.
%
We calculate evaluation metrics using all the generated molecules satisfying the graph similarity constraints.
%
Particularly, when $\delta_g$=1.0, we do not filter out any molecules based on the constraints and directly calculate metrics on all the generated molecules.
%
When $\delta_g$=0.7, 0.5 or 0.3, we consider only generated molecules with similarities lower than $\delta_g$.
%
Based on \shapesim and \graphsim as described in Section ``Evaluation Metrics'' in the main manuscript,
we calculate the following metrics using the subset of molecules with \graphsim lower than $\delta_g$, from a set of 50 generated molecules for each test molecule and report the average of  these metrics across all test molecules:
%
(1) \avgshapesim\ measures the average \shapesim across each subset of generated molecules with $\graphsim$ lower than $\delta_g$; %per test molecule, with the overall average calculated across all test molecules; }%the 50 generated molecules for each test molecule, averaged across all test molecules;
(2) \avggraphsim\ calculates the average \graphsim for each set; %, with these means averaged across all test molecules}; %} 50 molecules, %\bo{@Ziqi rephrase}, with results averaged on the test set;\ziqi{with the average computed over the test set; }
(3) \maxshapesim\ determines the maximum \shapesim within each set; %, with these maxima averaged across all test molecules; }%\hl{among 50 molecules}, averaged across all test molecules;
(4) \maxgraphsim\ measures the \graphsim of the molecule with maximum \shapesim in each set. %, averaged across all test molecules; }%\hl{among 50 molecules}, averaged across all test molecules;

%
As shown in Table~\ref{tbl:overall_sim}, \method and \methodwithsguide demonstrate outstanding performance in terms of the average shape similarities (\avgshapesim) and the average graph similarities (\avggraphsim) among generated molecules.
%
%\ziqi{
%Table~\ref{tbl:overall} also shows that \method and \methodwithsguide consistently outperform all the baseline methods in average shape similarities (\avgshapesim) and only slightly underperform 
%the best baseline \dataset in average graph similarities (\avggraphsim).
%}
%
Specifically, when $\delta_g$=0.3, \methodwithsguide achieves a substantial 10.5\% improvement in \avgshapesim\ over the best baseline \dataset. 
%
In terms of \avggraphsim, \methodwithsguide also achieves highly comparable performance with \dataset (0.217 vs 0.211, in \avggraphsim, lower values indicate better performance).
%
%This trend remains consistent across various $\delta_g$ values.
This trend remains consistent when applying various similarity constraints (i.e., $\delta_g$) as shown in Table~\ref{tbl:overall_sim}.


Similarly, \method and \methodwithsguide demonstrate superior performance in terms of the average maximum shape similarity across generated molecules for all test molecules (\maxshapesim), as well as the average graph similarity of the molecules with the maximum shape similarities (\maxgraphsim). %maximum shape similarities of generated molecules (\maxshapesim) and the average graph similarities of molecules with the maximum shape similarities (\maxgraphsim). %\bo{\maxgraphsim is misleading... how about $\text{avgMSim}_\text{g}$}
%
%\bo{
%in terms of the maximum shape similarities (\maxshapesim) and the maximum graph similarities (\maxgraphsim) among all the generated molecules.
%@Ziqi are the metrics maximum values or the average of maximum values?
%}
%
Specifically, at \maxshapesim, Table~\ref{tbl:overall_sim} shows that \methodwithsguide outperforms the best baseline \squid ($\lambda$=0.3) when $\delta_g$=0.3, 0.5, and 0.7, and only underperforms
it by 0.7\% when $\delta$=1.0.
%
We also note that the molecules generated by {\methodwithsguide} with the maximum shape similarities have substantially lower graph similarities ({\maxgraphsim}) compared to those generated by {\squid} ({$\lambda$}=0.3).
%\hl{We also note that the molecules with the maximum shape similarities generated by {\methodwithsguide} are with significantly lower graph similarities ({\maxgraphsim}) than those generated by {\squid} ({$\lambda$}=0.3).}
%
%\bo{@Ziqi please rephrase the language}
%
%\bo{
%@Ziqi the conclusion is not obvious. You may want to remind the meaning of \maxshapesim and \maxgraphsim here, and based on what performance you say this.
%}
%
%\bo{\st{This also underscores the ability of {\methodwithsguide} in generating molecules with similar shapes to condition molecules and novel graph structures.}}
%
As evidenced by these results, \methodwithsguide features strong capacities of generating molecules with similar shapes yet novel graph structures compared to the condition molecule, facilitating the discovery of promising drug candidates.
%

\begin{comment}
\ziqi{replace \#n\% with the percentage of novel molecules that do not exist in the dataset and update the discussion accordingly}
%\ziqi{
Table~\ref{tbl:overall_sim} also presents \bo{\#n\%}, the percentage of molecules generated by each method %\st{(\#n\%)} 
with graph similarities lower than the constraint $\delta_g$. 
%
%\bo{
%Table~\ref{tbl:overall_sim} also presents \#n\%, the percentage of generated molecules with graph similarities lower than the constraint $\delta_g$, of different methods. 
%}
%
As shown in Table~\ref{tbl:overall_sim},  when a restricted constraint (i.e., $\delta_g$=0.3) is applied, \method and \methodwithsguide could still generate a sufficient number of molecules satisfying the constraint.
%
Particularly, when $\delta_g$=0.3, \method outperforms \squid with $\lambda$=0.3 by XXX and \squid with $\lambda$=1.0 by XXX.
% achieve the second and the third in \#n\% and only underperform the best baseline \dataset.
%
This demonstrates the ability of \method in generating molecules with novel structures. 
%
When $\delta_g$=0.5, 0.7 and 1.0, both methods generate over 99.0\% of molecules satisfying the similarity constraint $\delta_g$.
%
%Note that \dataset is guaranteed to identify at least 50 molecules satisfying the $\delta_g$ by searching within a training dataset of diverse molecules.
%
Note that \dataset is a search algorithm that always first identifies the molecules satisfying $\delta_g$ and then selects the top-50 molecules of the highest shape similarities among them. 
%
Due to the diverse molecules in %\hl{the subset} \bo{@Ziqi why do you want to stress subset?} of 
the training set, \dataset can always identify at least 50 molecules under different $\delta_g$ and thus achieve 100\% in \#n\%.
%
%\bo{
%Note that \dataset is a search algorithm that always generate molecules XXX
%@Ziqi
%We need to discuss here. For \dataset, \#n\% in this table does not look aligned with that in Fig 1 if the highlighted defination is correct...
%}
%
%Thus, \dataset achieves 100.0\% in \#n\% under different $\delta_g$.
%
It is also worth noting that when $\delta_g$=1.0, \#n\% reflects the validity among all the generated molecules. 
%
As shown in Table~\ref{tbl:overall_sim}, \method and \methodwithsguide are able to generate 99.3\% and 99.2\% valid molecules.
%
This demonstrates their ability to effectively capture the underlying chemical rules in a purely data-driven manner without relying on any prior knowledge (e.g., fragments) as \squid does.
%
%\bo{
%@Ziqi I feel this metric is redundant with the avg graph similarity when constraint is 1.0. Generally, if the avg similarity is small. You have more mols satisfying the requirement right?
%}
\end{comment}

Table~\ref{tbl:overall_sim} also shows that by incorporating shape guidance, \methodwithsguide
%\bo{
%@Ziqi where does this come from...
%}
substantially outperforms \method in both \avgshapesim and \maxshapesim, while maintaining comparable graph similarities (i.e., \avggraphsim\ and \maxgraphsim).
%
Particularly, when $\delta_g$=0.3, \methodwithsguide 
establishes a considerable improvement of 6.9\% and 4.9\%
%\bo{\st{achieves 6.9\% and 4.9\% improvements}} 
over \method in \avgshapesim and \maxshapesim, respectively. 
%
%\hl{In the meanwhile}, 
%\bo{@Ziqi it is not the right word...}
Meanwhile, \methodwithsguide achieves the same \avggraphsim with \method and only slightly underperforms \method in \maxgraphsim (0.223 vs 0.220).
%\bo{
%XXX also achieves XXX
%}
%it maintains the same \avggraphsim\ with \method and only slightly underperforms \method in \maxgraphsim (0.223 vs 0.220).
%
%Compared with \method, \methodwithsguide consistently generates molecules with higher shape similarities while maintaining comparable graph similarities.
%
%\bo{
%@Ziqi you may want to highlight the utility of "generating molecules with higher shape similarities while maintaining comparable graph similarities" in real drug discovery applications.
%
%
%\bo{
%@Ziqi You did not present the details of method yet...
%}
%
%\methodwithsguide leverages additional shape guidance to push the predicted atoms to the shape of condition molecules \bo{and XXX (@Ziqi boosts the shape similarities XXX)} , as will be discussed in Section ``\method with Shape Guidance'' later.
%
The superior performance of \methodwithsguide suggests that the incorporation of shape guidance effectively boosts the shape similarities of generated molecules without compromising graph similarities.
%
%This capability could be crucial in drug discovery, 
%\bo{@Ziqi it is a strong statement. Need citations here}, 
%as it enables the discovery of drug candidates that are both more potentially effective due to the improved shape similarities and novel induced by low graph similarities.
%as it could enable the identification of candidates with similar binding patterns %with the condition molecule (i.e., high shape similarities) 
%(i.e., high shape similarities) and graph structures distinct from the condition molecules (i.e., low graph similarities).
%\bo{\st{and enjoys novel structures (i.e., low graph similarities) with potentially better properties. } \ziqi{change enjoys}}
%\bo{
%and enjoys potentially better properties (i.e., low graph similarities). \ziqi{this looks weird to me... need to discuss}
%}
%\st{potentially better properties (i.e., low graph similarities).}}

%-------------------------------------------------------------------------------------------------------------------------------------
\subsection{Comparison on Validity and Novelty}
\label{supp:app:results:valid_novel}
%-------------------------------------------------------------------------------------------------------------------------------------

We evaluate the ability of \method and \squid to generate molecules with valid and novel 2D molecular graphs.
%
We calculate the percentages of the valid and novel molecules among all the generated molecules.
%
As shown in Table~\ref{tbl:validity_novelty}, both \method and \methodwithsguide outperform \squid with $\lambda$=0.3 and $\lambda$=1.0 in generating novel molecules.
%
Particularly, almost all valid molecules generated by \method and \methodwithsguide are novel (99.8\% and 99.9\% at \#n\%), while the best baseline \squid with $\lambda$=0.3 achieves 98.4\% in novelty.
%
In terms of the percentage of valid and novel molecules among all the generated ones (\#v\&n\%), \method and \methodwithsguide again outperform \squid with $\lambda$=0.3 and $\lambda$=1.0.
%
We also note that at \#v\%,  \method (99.1\%) and \methodwithsguide (99.2\%) slightly underperform \squid with $\lambda$=0.3 and $\lambda$=1.0 (100.0\%) in generating valid molecules.
%
\squid guarantees the validity of generated molecules by incorporating valence rules into the generation process and ensuring it to avoid fragments that violate these rules.
%
Conversely, \method and \methodwithsguide use a purely data-driven approach to learn the generation of valid molecules.
%
These results suggest that, even without integrating valence rules, \method and \methodwithsguide can still achieve a remarkably high percentage of valid and novel generated molecules.

\begin{table*}
	\centering
		\caption{Comparison on Validity and Novelty between \method and \squid}
	\label{tbl:validity_novelty}
	\begin{scriptsize}
\begin{threeparttable}
%	\setlength\tabcolsep{0pt}
	\begin{tabular}{
		@{\hspace{3pt}}l@{\hspace{10pt}}
		%
		@{\hspace{10pt}}r@{\hspace{10pt}}
		%
		@{\hspace{10pt}}r@{\hspace{10pt}}
		%
		@{\hspace{10pt}}r@{\hspace{3pt}}
		}
		\toprule
		method & \#v\% & \#n\% & \#v\&n\% \\
		\midrule
		\squid ($\lambda$=0.3) & \textbf{100.0} & 96.7 & 96.7 \\
		\squid ($\lambda$=1.0) & \textbf{100.0} & 98.4 & 98.4 \\
		\method & 99.1 & 99.8 & 98.9 \\
		\methodwithsguide & 99.2 & \textbf{99.9} & \textbf{99.1} \\
		\bottomrule
	\end{tabular}%
	%
	\begin{tablenotes}[normal,flushleft]
		\begin{footnotesize}
	\item 
\!\!Columns represent: ``\#v\%'': the percentage of generated molecules that are valid;
		``\#n\%'': the percentage of valid molecules that are novel;
		``\#v\&n\%'': the percentage of generated molecules that are valid and novel.
		Best values are in \textbf{bold}. 
		\par
		\end{footnotesize}
	\end{tablenotes}
\end{threeparttable}
\end{scriptsize}
\end{table*}


%-------------------------------------------------------------------------------------------------------------------------------------
\subsection{Additional Quality Comparison between Desirable Molecules Generated by \method and \squid}
\label{supp:app:results:quality_desirable}
%-------------------------------------------------------------------------------------------------------------------------------------

\begin{table*}[!h]
	\centering
		\caption{Comparison on Quality of Generated Desirable Molecules between \method and \squid ($\delta_g$=0.5)}
	\label{tbl:overall_results_quality_05}
	\begin{scriptsize}
\begin{threeparttable}
	\begin{tabular}{
		@{\hspace{0pt}}l@{\hspace{16pt}}
		@{\hspace{0pt}}l@{\hspace{2pt}}
		%
		@{\hspace{6pt}}c@{\hspace{6pt}}
		%
		%@{\hspace{3pt}}c@{\hspace{3pt}}
		@{\hspace{3pt}}c@{\hspace{3pt}}
		@{\hspace{3pt}}c@{\hspace{3pt}}
		@{\hspace{3pt}}c@{\hspace{3pt}}
		@{\hspace{3pt}}c@{\hspace{3pt}}
		%
		%
		}
		\toprule
		group & metric & 
        %& \dataset 
        & \squid ($\lambda$=0.3) & \squid ($\lambda$=1.0)  &  \method & \methodwithsguide  \\
		%\multirow{2}{*}{method} & \multirow{2}{*}{\#c\%} &  \multirow{2}{*}{\#u\%} &  \multirow{2}{*}{QED} & \multicolumn{3}{c}{$\nmax=50$} & & \multicolumn{2}{c}{$\nmax=1$}\\
		%\cmidrule(r){5-7} \cmidrule(r){8-10} 
		%& & & & \avgshapesim(std) & \avggraphsim(std  &  \diversity(std  & & \avgshapesim(std) & \avggraphsim(std \\
		\midrule
		\multirow{2}{*}{stability}
		& atom stability ($\uparrow$) & 
        %& 0.990 
        & \textbf{0.996} & 0.995 & 0.992 & 0.989     \\
		& mol stability ($\uparrow$) & 
        %& 0.875 
        & \textbf{0.948} & 0.947 & 0.886 & 0.839    \\
		%\midrule
		%\multirow{3}{*}{Drug-likeness} 
		%& QED ($\uparrow$) & 
        %& \textbf{0.805} 
        %& 0.766 & 0.760 & 0.755 & 0.751    \\
	%	& SA ($\uparrow$) & 
        %& \textbf{0.874} 
        %& 0.814 & 0.813 & 0.699 & 0.692    \\
	%	& Lipinski ($\uparrow$) & 
        %& \textbf{4.999} 
        %& 4.979 & 4.980 & 4.967 & 4.975    \\
		\midrule
		\multirow{4}{*}{3D structures} 
		& RMSD ($\downarrow$) & 
        %& \textbf{0.419} 
        & 0.907 & 0.906 & 0.897 & \textbf{0.881}    \\
		& JS. bond lengths ($\downarrow$) & 
        %& \textbf{0.286} 
        & 0.457 & 0.477 & 0.436 & \textbf{0.428}    \\
		& JS. bond angles ($\downarrow$) & 
        %& \textbf{0.078} 
        & 0.269 & 0.289 & \textbf{0.186} & 0.200    \\
		& JS. dihedral angles ($\downarrow$) & 
        %& \textbf{0.151} 
        & 0.199 & 0.209 & \textbf{0.168} & 0.170    \\
		\midrule
		\multirow{5}{*}{2D structures} 
		& JS. \#bonds per atoms ($\downarrow$) & 
        %& 0.325 
        & 0.291 & 0.331 & \textbf{0.176} & 0.181    \\
		& JS. basic bond types ($\downarrow$) & 
        %& \textbf{0.055} 
        & \textbf{0.071} & 0.083 & 0.181 & 0.191    \\
		%& JS. freq. bond types ($\downarrow$) & 
        %& \textbf{0.089} 
        %& 0.123 & 0.130 & 0.245 & 0.254    \\
		%& JS. freq. bond pairs ($\downarrow$) & 
        %& \textbf{0.078} 
        %& 0.085 & 0.089 & 0.209 & 0.221    \\
		%& JS. freq. bond triplets ($\downarrow$) & 
        %& \textbf{0.089} 
        %& 0.097 & 0.114 & 0.211 & 0.223    \\
		%\midrule
		%\multirow{3}{*}{Rings} 
		& JS. \#rings ($\downarrow$) & 
        %& 0.142 
        & 0.280 & 0.330 & \textbf{0.043} & 0.049    \\
		& JS. \#n-sized rings ($\downarrow$) & 
        %& \textbf{0.055} 
        & \textbf{0.077} & 0.091 & 0.099 & 0.112    \\
		& \#Intersecting rings ($\uparrow$) & 
        %& \textbf{6} 
        & \textbf{6} & 5 & 4 & 5    \\
		%\method (+bt)            & 100.0 & 98.0 & 100.0 & 0.742 & 0.772 (0.040) & 0.211 (0.033) & & 0.862 (0.036) & 0.211 (0.033) & 0.743 (0.043) \\
		%\methodwithguide (+bt)    & 99.8 & 98.0 & 100.0 & 0.736 & 0.814 (0.031) & 0.193 (0.042) & & 0.895 (0.029) & 0.193 (0.042) & 0.745 (0.045) \\
		%
		\bottomrule
	\end{tabular}%
	\begin{tablenotes}[normal,flushleft]
		\begin{footnotesize}
	\item 
\!\!Rows represent:  {``atom stability'': the proportion of stable atoms that have the correct valency; 
		``molecule stability'': the proportion of generated molecules with all atoms stable;
		%``QED'': the drug-likeness score;
		%``SA'': the synthesizability score;
		%``Lipinski'': the Lipinski 
		``RMSD'': the root mean square deviation (RMSD) between the generated 3D structures of molecules and their optimal conformations; % identified via energy minimization;
		``JS. bond lengths/bond angles/dihedral angles'': the Jensen-Shannon (JS) divergences of bond lengths, bond angles and dihedral angles;
		``JS. \#bonds per atom/basic bond types/\#rings/\#n-sized rings'': the JS divergences of bond counts per atom, basic bond types, counts of all rings, and counts of n-sized rings;
		%``JS. \#rings/\#n-sized rings'': the JS divergences of the total counts of rings and the counts of n-sized rings;
		``\#Intersecting rings'': the number of rings observed in the top-10 frequent rings of both generated and real molecules. } \par
		\par
		\end{footnotesize}
	\end{tablenotes}
\end{threeparttable}
\end{scriptsize}
\end{table*}

%\label{tbl:overall_quality05}

\begin{table*}[!h]
	\centering
		\caption{Comparison on Quality of Generated Desirable Molecules between \method and \squid ($\delta_g$=0.7)}
	\label{tbl:overall_results_quality_07}
	\begin{scriptsize}
\begin{threeparttable}
	\begin{tabular}{
		@{\hspace{0pt}}l@{\hspace{14pt}}
		@{\hspace{0pt}}l@{\hspace{2pt}}
		%
		@{\hspace{4pt}}c@{\hspace{4pt}}
		%
		%@{\hspace{3pt}}c@{\hspace{3pt}}
		@{\hspace{3pt}}c@{\hspace{3pt}}
		@{\hspace{3pt}}c@{\hspace{3pt}}
		@{\hspace{3pt}}c@{\hspace{3pt}}
		@{\hspace{3pt}}c@{\hspace{3pt}}
		%
		%
		}
		\toprule
		group & metric & 
        %& \dataset 
        & \squid ($\lambda$=0.3) & \squid ($\lambda$=1.0)  &  \method & \methodwithsguide  \\
		%\multirow{2}{*}{method} & \multirow{2}{*}{\#c\%} &  \multirow{2}{*}{\#u\%} &  \multirow{2}{*}{QED} & \multicolumn{3}{c}{$\nmax=50$} & & \multicolumn{2}{c}{$\nmax=1$}\\
		%\cmidrule(r){5-7} \cmidrule(r){8-10} 
		%& & & & \avgshapesim(std) & \avggraphsim(std  &  \diversity(std  & & \avgshapesim(std) & \avggraphsim(std \\
		\midrule
		\multirow{2}{*}{stability} 
		& atom stability ($\uparrow$) & 
        %&  0.990 
        & \textbf{0.995} & 0.995 & 0.992 & 0.988 \\
		& molecule stability ($\uparrow$) & 
        %& 0.876 
        & 0.944 & \textbf{0.947} & 0.885 & 0.839 \\
		\midrule
		%\multirow{3}{*}{Drug-likeness} 
		%& QED ($\uparrow$) & 
        %& \textbf{0.805} 
        %& 0.766 & 0.760 & 0.755 & 0.751    \\
	%	& SA ($\uparrow$) & 
        %& \textbf{0.874} 
        %& 0.814 & 0.813 & 0.699 & 0.692    \\
	%	& Lipinski ($\uparrow$) & 
        %& \textbf{4.999} 
        %& 4.979 & 4.980 & 4.967 & 4.975    \\
	%	\midrule
		\multirow{4}{*}{3D structures} 
		& RMSD ($\downarrow$) & 
        %& \textbf{0.420} 
        & 0.897 & 0.906 & 0.897 & \textbf{0.881}    \\
		& JS. bond lengths ($\downarrow$) & 
        %& \textbf{0.286} 
        & 0.457 & 0.477 & 0.436 & \textbf{0.428}    \\
		& JS. bond angles ($\downarrow$) & 
        %& \textbf{0.078} 
        & 0.269 & 0.289 & \textbf{0.186} & 0.200    \\
		& JS. dihedral angles ($\downarrow$) & 
        %& \textbf{0.151} 
        & 0.199 & 0.209 & \textbf{0.168} & 0.170    \\
		\midrule
		\multirow{5}{*}{2D structures} 
		& JS. \#bonds per atoms ($\downarrow$) & 
        %& 0.325 
        & 0.285 & 0.329 & \textbf{0.176} & 0.181    \\
		& JS. basic bond types ($\downarrow$) & 
        %& \textbf{0.055} 
        & \textbf{0.067} & 0.083 & 0.181 & 0.191    \\
	%	& JS. freq. bond types ($\downarrow$) & 
        %& \textbf{0.089} 
        %& 0.123 & 0.130 & 0.245 & 0.254    \\
	%	& JS. freq. bond pairs ($\downarrow$) & 
        %& \textbf{0.078} 
        %& 0.085 & 0.089 & 0.209 & 0.221    \\
	%	& JS. freq. bond triplets ($\downarrow$) & 
        %& \textbf{0.089} 
        %& 0.097 & 0.114 & 0.211 & 0.223    \\
	%	\midrule
	%	\multirow{3}{*}{Rings} 
		& JS. \#rings ($\downarrow$) & 
        %& 0.143 
        & 0.273 & 0.328 & \textbf{0.043} & 0.049    \\
		& JS. \#n-sized rings ($\downarrow$) & 
        %& \textbf{0.055} 
        & \textbf{0.076} & 0.091 & 0.099 & 0.112    \\
		& \#Intersecting rings ($\uparrow$) & 
        %& \textbf{6} 
        & \textbf{6} & 5 & 4 & 5    \\
		%\method (+bt)            & 100.0 & 98.0 & 100.0 & 0.742 & 0.772 (0.040) & 0.211 (0.033) & & 0.862 (0.036) & 0.211 (0.033) & 0.743 (0.043) \\
		%\methodwithguide (+bt)    & 99.8 & 98.0 & 100.0 & 0.736 & 0.814 (0.031) & 0.193 (0.042) & & 0.895 (0.029) & 0.193 (0.042) & 0.745 (0.045) \\
		%
		\bottomrule
	\end{tabular}%
	\begin{tablenotes}[normal,flushleft]
		\begin{footnotesize}
	\item 
\!\!Rows represent:  {``atom stability'': the proportion of stable atoms that have the correct valency; 
		``molecule stability'': the proportion of generated molecules with all atoms stable;
		%``QED'': the drug-likeness score;
		%``SA'': the synthesizability score;
		%``Lipinski'': the Lipinski 
		``RMSD'': the root mean square deviation (RMSD) between the generated 3D structures of molecules and their optimal conformations; % identified via energy minimization;
		``JS. bond lengths/bond angles/dihedral angles'': the Jensen-Shannon (JS) divergences of bond lengths, bond angles and dihedral angles;
		``JS. \#bonds per atom/basic bond types/\#rings/\#n-sized rings'': the JS divergences of bond counts per atom, basic bond types, counts of all rings, and counts of n-sized rings;
		%``JS. \#rings/\#n-sized rings'': the JS divergences of the total counts of rings and the counts of n-sized rings;
		``\#Intersecting rings'': the number of rings observed in the top-10 frequent rings of both generated and real molecules. } \par
		\par
		\end{footnotesize}
	\end{tablenotes}
\end{threeparttable}
\end{scriptsize}
\end{table*}

%\label{tbl:overall_quality07}

\begin{table*}[!h]
	\centering
		\caption{Comparison on Quality of Generated Desirable Molecules between \method and \squid ($\delta_g$=1.0)}
	\label{tbl:overall_results_quality_10}
	\begin{scriptsize}
\begin{threeparttable}
	\begin{tabular}{
		@{\hspace{0pt}}l@{\hspace{14pt}}
		@{\hspace{0pt}}l@{\hspace{2pt}}
		%
		@{\hspace{4pt}}c@{\hspace{4pt}}
		%
		%@{\hspace{3pt}}c@{\hspace{3pt}}
		@{\hspace{3pt}}c@{\hspace{3pt}}
		@{\hspace{3pt}}c@{\hspace{3pt}}
		@{\hspace{3pt}}c@{\hspace{3pt}}
		@{\hspace{3pt}}c@{\hspace{3pt}}
		%
		%
		}
		\toprule
		group & metric & 
        %& \dataset 
        & \squid ($\lambda$=0.3) & \squid ($\lambda$=1.0)  &  \method & \methodwithsguide \\
		%\multirow{2}{*}{method} & \multirow{2}{*}{\#c\%} &  \multirow{2}{*}{\#u\%} &  \multirow{2}{*}{QED} & \multicolumn{3}{c}{$\nmax=50$} & & \multicolumn{2}{c}{$\nmax=1$}\\
		%\cmidrule(r){5-7} \cmidrule(r){8-10} 
		%& & & & \avgshapesim(std) & \avggraphsim(std  &  \diversity(std  & & \avgshapesim(std) & \avggraphsim(std \\
		\midrule
		\multirow{2}{*}{stability}
		& atom stability ($\uparrow$) & 
        %& 0.990 
        & \textbf{0.995} & \textbf{0.995} & 0.992 & 0.988     \\
		& mol stability ($\uparrow$) & 
        %& 0.876 
        & 0.942 & \textbf{0.947} & 0.885 & 0.839    \\
		\midrule
	%	\multirow{3}{*}{Drug-likeness} 
	%	& QED ($\uparrow$) & 
        %& \textbf{0.805} 
        %& \textbf{0.766} & 0.760 & 0.755 & 0.751    \\
	%	& SA ($\uparrow$) & 
        %& \textbf{0.874} 
        %& \textbf{0.813} & \textbf{0.813} & 0.699 & 0.692    \\
	%	& Lipinski ($\uparrow$) & 
        %& \textbf{4.999} 
        %& 4.979 & \textbf{4.980} & 4.967 & 4.975    \\
	%	\midrule
		\multirow{4}{*}{3D structures} 
		& RMSD ($\downarrow$) & 
        %& \textbf{0.420} 
        & 0.898 & 0.906 & 0.897 & \textbf{0.881}    \\
		& JS. bond lengths ($\downarrow$) & 
        %& \textbf{0.286} 
        & 0.457 & 0.477 & 0.436 & \textbf{0.428}    \\
		& JS. bond angles ($\downarrow$) & 
        %& \textbf{0.078} 
        & 0.269 & 0.289 & \textbf{0.186} & 0.200   \\
		& JS. dihedral angles ($\downarrow$) & 
        %& \textbf{0.151} 
        & 0.199 & 0.209 & \textbf{0.168} & 0.170    \\
		\midrule
		\multirow{5}{*}{2D structures} 
		& JS. \#bonds per atoms ($\downarrow$) & 
        %& 0.325 
        & 0.280 & 0.330 & \textbf{0.176} & 0.181    \\
		& JS. basic bond types ($\downarrow$) & 
        %& \textbf{0.055} 
        & \textbf{0.066} & 0.083 & 0.181 & 0.191   \\
	%	& JS. freq. bond types ($\downarrow$) & 
        %& \textbf{0.089} 
        %& \textbf{0.123} & 0.130 & 0.245 & 0.254    \\
	%	& JS. freq. bond pairs ($\downarrow$) & 
        %& \textbf{0.078} 
        %& \textbf{0.085} & 0.089 & 0.209 & 0.221    \\
	%	& JS. freq. bond triplets ($\downarrow$) & 
        %& \textbf{0.089} 
        %& \textbf{0.097} & 0.114 & 0.211 & 0.223    \\
		%\midrule
		%\multirow{3}{*}{Rings} 
		& JS. \#rings ($\downarrow$) & 
        %& 0.143 
        & 0.269 & 0.328 & \textbf{0.043} & 0.049    \\
		& JS. \#n-sized rings ($\downarrow$) & 
        %& \textbf{0.055} 
        & \textbf{0.075} & 0.091 & 0.099 & 0.112    \\
		& \#Intersecting rings ($\uparrow$) & 
        %& \textbf{6} 
        & \textbf{6} & 5 & 4 & 5    \\
		%\method (+bt)            & 100.0 & 98.0 & 100.0 & 0.742 & 0.772 (0.040) & 0.211 (0.033) & & 0.862 (0.036) & 0.211 (0.033) & 0.743 (0.043) \\
		%\methodwithguide (+bt)    & 99.8 & 98.0 & 100.0 & 0.736 & 0.814 (0.031) & 0.193 (0.042) & & 0.895 (0.029) & 0.193 (0.042) & 0.745 (0.045) \\
		%
		\bottomrule
	\end{tabular}%
	\begin{tablenotes}[normal,flushleft]
		\begin{footnotesize}
	\item 
\!\!Rows represent:  {``atom stability'': the proportion of stable atoms that have the correct valency; 
		``molecule stability'': the proportion of generated molecules with all atoms stable;
		%``QED'': the drug-likeness score;
		%``SA'': the synthesizability score;
		%``Lipinski'': the Lipinski 
		``RMSD'': the root mean square deviation (RMSD) between the generated 3D structures of molecules and their optimal conformations; % identified via energy minimization;
		``JS. bond lengths/bond angles/dihedral angles'': the Jensen-Shannon (JS) divergences of bond lengths, bond angles and dihedral angles;
		``JS. \#bonds per atom/basic bond types/\#rings/\#n-sized rings'': the JS divergences of bond counts per atom, basic bond types, counts of all rings, and counts of n-sized rings;
		%``JS. \#rings/\#n-sized rings'': the JS divergences of the total counts of rings and the counts of n-sized rings;
		``\#Intersecting rings'': the number of rings observed in the top-10 frequent rings of both generated and real molecules. } \par
		\par
		\end{footnotesize}
	\end{tablenotes}
\end{threeparttable}
\end{scriptsize}
\end{table*}

%\label{tbl:overall_quality10}

Similar to Table~\ref{tbl:overall_results_quality_desired} in the main manuscript, we present the performance comparison on the quality of desirable molecules generated by different methods under different graph similarity constraints $\delta_g$=0.5, 0.7 and 1.0, as detailed in Table~\ref{tbl:overall_results_quality_05}, Table~\ref{tbl:overall_results_quality_07}, and Table~\ref{tbl:overall_results_quality_10}, respectively.
%
Overall, these tables show that under varying graph similarity constraints, \method and \methodwithsguide can always generate desirable molecules with comparable quality to baselines in terms of stability, 3D structures, and 2D structures.
%
These results demonstrate the strong effectiveness of \method and \methodwithsguide in generating high-quality desirable molecules with stable and realistic structures in both 2D and 3D.
%
This enables the high utility of \method and \methodwithsguide in discovering promising drug candidates.


\begin{comment}
The results across these tables demonstrate similar observations with those under $\delta_g$=0.3 in Table~\ref{tbl:overall_results_quality_desired}.
%
For stability, when $\delta_g$=0.5, 0.7 or 1.0, \method and \methodwithsguide achieve comparable performance or fall slightly behind \squid ($\lambda$=0.3) and \squid ($\lambda$=1.0) in atom stability and molecule stability.
%
For example, when $\delta_g$=0.5, as shown in Table~\ref{tbl:overall_results_quality_05}, \method achieves similar performance with the best baseline \squid ($\lambda$=0.3) in atom stability (0.992 for \method vs 0.996 for \squid with $\lambda$=0.3).
%
\method underperforms \squid ($\lambda$=0.3) in terms of molecule stability.
%
For 3D structures, \method and \methodwithsguide also consistently generate molecules with more realistic 3D structures compared to \squid.
%
Particularly, \methodwithsguide achieves the best performance in RMSD and JS of bond lengths across $\delta_g$=0.5, 0.7 and 1.0.
%
In JS of dihedral angles, \method achieves the best performance among all the methods.
%
\method and \methodwithsguide underperform \squid in JS of bond angles, primarily because \squid constrains the bond angles in the generated molecules.
%
For 2D structures, \method and \methodwithsguide again achieve the best performance 
\end{comment}

%===================================================================
\section{Additional Experimental Results on PMG}
\label{supp:app:results_PMG}
%===================================================================

%\label{tbl:comparison_results_decompdiff}


%-------------------------------------------------------------------------------------------------------------------------------------
%\subsection{{Additional Comparison for PMG}}
%\label{supp:app:results:docking}
%-------------------------------------------------------------------------------------------------------------------------------------

In this section, we present the results of \methodwithpguide and \methodwithsandpguide when generating 100 molecules. 
%
Please note that both \methodwithpguide and \methodwithsandpguide show remarkable efficiency over the PMG baselines.
%
\methodwithpguide and \methodwithsandpguide generate 100 molecules in 48 and 58 seconds on average, respectively, while the most efficient baseline \targetdiff requires 1,252 seconds.
%
We report the performance of \methodwithpguide and \methodwithsandpguide against state-of-the-art PMG baselines in Table~\ref{tbl:overall_results_docking_100}.


%
According to Table~\ref{tbl:overall_results_docking_100}, \methodwithpguide and \methodwithsandpguide achieve comparable performance with the PMG baselines in generating molecules with high binding affinities.
%
Particularly, in terms of Vina S, \methodwithsandpguide achieves very comparable performance (-4.56 kcal/mol) to the third-best baseline \decompdiff (-4.58 kcal/mol) in average Vina S; it also achieves the third-best performance (-4.82 kcal/mol) among all the methods and slightly underperforms the second-best baseline \AR (-4.99 kcal/mol) in median Vina S
%
\methodwithsandpguide also achieves very close average Vina M (-5.53 kcal/mol) with the third-best baseline \AR (-5.59 kcal/mol) and the third-best performance (-5.47 kcal/mol) in median Vina M.
%
Notably, for Vina D, \methodwithpguide and \methodwithsandpguide achieve the second and third performance among all the methods.
%
In terms of the average percentage of generated molecules with Vina D higher than those of known ligands (i.e., HA), \methodwithpguide (58.52\%) and \methodwithsandpguide (58.28\%) outperform the best baseline \targetdiff (57.57\%).
%
These results signify the high utility of \methodwithpguide and \methodwithsandpguide in generating molecules that effectively bind with protein targets and have better binding affinities than known ligands.

In addition to binding affinities, \methodwithpguide and \methodwithsandpguide also demonstrate similar performance compared to the baselines in metrics related to drug-likeness and diversity.
%
For drug-likeness, both \methodwithpguide and \methodwithsandpguide achieve the best (0.67) and the second-best (0.66) QED scores.
%
They also achieve the third and fourth performance in SA scores.
%
In terms of the diversity among generated molecules,  \methodwithpguide and \methodwithsandpguide slightly underperform the baselines, possibly due to the design that generates molecules with similar shapes to the ligands.
%
These results highlight the strong ability of \methodwithpguide and \methodwithsandpguide in efficiently generating effective binding molecules with favorable drug-likeness and diversity.
%
This ability enables them to potentially serve as promising tools to facilitate effective and efficient drug development.

\begin{table*}[!h]
	\centering
		\caption{Additional Comparison on PMG When All Methods Generate 100 Molecules}
	\label{tbl:overall_results_docking_100}
\begin{threeparttable}
	\begin{scriptsize}
	\begin{tabular}{
		@{\hspace{2pt}}l@{\hspace{2pt}}
		%
		@{\hspace{2pt}}r@{\hspace{2pt}}
		%
		@{\hspace{2pt}}r@{\hspace{2pt}}
		@{\hspace{2pt}}r@{\hspace{2pt}}
		%
		@{\hspace{6pt}}r@{\hspace{6pt}}
		%
		@{\hspace{2pt}}r@{\hspace{2pt}}
		@{\hspace{2pt}}r@{\hspace{2pt}}
		%
		@{\hspace{5pt}}r@{\hspace{5pt}}
		%
		@{\hspace{2pt}}r@{\hspace{2pt}}
		@{\hspace{2pt}}r@{\hspace{2pt}}
		%
		@{\hspace{5pt}}r@{\hspace{5pt}}
		%
		@{\hspace{2pt}}r@{\hspace{2pt}}
	         @{\hspace{2pt}}r@{\hspace{2pt}}
		%
		@{\hspace{5pt}}r@{\hspace{5pt}}
		%
		@{\hspace{2pt}}r@{\hspace{2pt}}
		@{\hspace{2pt}}r@{\hspace{2pt}}
		%
		@{\hspace{5pt}}r@{\hspace{5pt}}
		%
		@{\hspace{2pt}}r@{\hspace{2pt}}
		@{\hspace{2pt}}r@{\hspace{2pt}}
		%
		@{\hspace{5pt}}r@{\hspace{5pt}}
		%
		@{\hspace{2pt}}r@{\hspace{2pt}}
		@{\hspace{2pt}}r@{\hspace{2pt}}
		%
		@{\hspace{5pt}}r@{\hspace{5pt}}
		%
		@{\hspace{2pt}}r@{\hspace{2pt}}
		%@{\hspace{2pt}}r@{\hspace{2pt}}
		%@{\hspace{2pt}}r@{\hspace{2pt}}
		}
		\toprule
		\multirow{2}{*}{method} & \multicolumn{2}{c}{Vina S$\downarrow$} & & \multicolumn{2}{c}{Vina M$\downarrow$} & & \multicolumn{2}{c}{Vina D$\downarrow$} & & \multicolumn{2}{c}{{HA\%$\uparrow$}}  & & \multicolumn{2}{c}{QED$\uparrow$} & & \multicolumn{2}{c}{SA$\uparrow$} & & \multicolumn{2}{c}{Div$\uparrow$} & %& \multirow{2}{*}{SR\%$\uparrow$} & 
		& \multirow{2}{*}{time$\downarrow$} \\
	    \cmidrule{2-3}\cmidrule{5-6} \cmidrule{8-9} \cmidrule{11-12} \cmidrule{14-15} \cmidrule{17-18} \cmidrule{20-21}
		 & Avg. & Med. &  & Avg. & Med. &  & Avg. & Med. & & Avg. & Med.  & & Avg. & Med.  & & Avg. & Med.  & & Avg. & Med.  & & \\ %& & \\
		%\multirow{2}{*}{method} & \multirow{2}{*}{\#c\%} &  \multirow{2}{*}{\#u\%} &  \multirow{2}{*}{QED} & \multicolumn{3}{c}{$\nmax=50$} & & \multicolumn{2}{c}{$\nmax=1$}\\
		%\cmidrule(r){5-7} \cmidrule(r){8-10} 
		%& & & & \avgshapesim(std) & \avggraphsim(std  &  \diversity(std  & & \avgshapesim(std) & \avggraphsim(std \\
		\midrule
		Reference                          & -5.32 & -5.66 & & -5.78 & -5.76 & & -6.63 & -6.67 & & - & - & & 0.53 & 0.49 & & 0.77 & 0.77 & & - & - & %& 23.1 & 
		& - \\
		\midrule
		\AR & \textbf{-5.06} & -4.99 & &  -5.59 & -5.29 & &  -6.16 & -6.05 & &  37.69 & 31.00 & &  0.50 & 0.49 & &  0.66 & 0.65 & & 0.70 & 0.70 & %& 7.0 & 
		& 7,789 \\
		\pockettwomol   & -4.50 & -4.21 & &  -5.70 & -5.27 & &  -6.43 & -6.25 & &  48.00 & 51.00 & &  0.58 & 0.58 & &  \textbf{0.77} & \textbf{0.78} & &  0.69 & 0.71 &  %& 24.9 & 
		& 2,150 \\
		\targetdiff     & -4.88 & \textbf{-5.82} & &  \textbf{-6.20} & \textbf{-6.36} & &  \textbf{-7.37} & \textbf{-7.51} & &  57.57 & 58.27 & &  0.50 & 0.51 & &  0.60 & 0.59 & &  \textbf{0.72} & 0.71 & % & 10.4 & 
		& 1,252 \\
		%& \decompdiffbeta                    & 63.03 & %-4.72 & -4.86 & & \textbf{-6.84} & \textbf{-6.91} & & \textbf{-8.85} & \textbf{-8.90} & &  \textbf{72.16} & \textbf{72.16} & &  0.36 & 0.36 & &  0.55 & 0.55 & & 0.59 & 0.59 & & 14.9 \\ 
		%-4.76 & -6.18 & &  \textbf{-6.86} & \textbf{-7.52} & &  \textbf{-8.85} & \textbf{-8.96} & &  \textbf{72.7} & \textbf{89.8} & &  0.36 & 0.34 & &  0.55 & 0.57 & & 0.59 & 0.59 & & 15.4 \\
		\decompdiffref  & -4.58 & -4.77 & &  -5.47 & -5.51 & &  -6.43 & -6.56 & &  47.76 & 48.66 & &  0.56 & 0.56 & &  0.70 & 0.69  & &  \textbf{0.72} & \textbf{0.72} &  %& 15.2 & 
		& 1,859 \\
		\midrule
		%\method & 14.04 & 9.74 & &  -2.80 & -3.87 & &  -6.32 & -6.41 & &  42.37 & 40.40 & &  0.70 & 0.71 & &  0.73 & 0.72 & & 0.71 & 0.74 & & 42 \\
		%\methodwithsguide & 1.04 & -0.33 & &  -4.23 & -4.39 & &  -6.31 & -6.46 & &  46.18 & 44.00 & &  0.69 & 0.71 & &  0.72 & 0.71 & & 0.70 & 0.73 & 53 \\
		\methodwithpguide      & -4.15 & -4.59 & &  -5.41 & -5.34 & &  -6.49 & -6.74 & &  \textbf{58.52} & 59.00 & &  \textbf{0.67} & \textbf{0.69} & &  0.68 & 0.68 & & 0.67 & 0.70 & %& 28.0 & 
		& 48 \\
		\methodwithsandpguide  & -4.56 & -4.82 & &  -5.53 & -5.47 & &  -6.60 & -6.78 & &  58.28 & \textbf{60.00} & &  0.66 & 0.68 & &  0.67 & 0.66 & & 0.68 & 0.71 &
		& 58 \\
		\bottomrule
	\end{tabular}%
	\begin{tablenotes}[normal,flushleft]
		\begin{footnotesize}
	\item 
\!\!Columns represent: {``Vina S'': the binding affinities between the initially generated poses of molecules and the protein pockets; 
		``Vina M'': the binding affinities between the poses after local structure minimization and the protein pockets;
		``Vina D'': the binding affinities between the poses determined by AutoDock Vina~\cite{Eberhardt2021} and the protein pockets;
		``HA'': the percentage of generated molecules with Vina D higher than those of condition molecules;
		``QED'': the drug-likeness score;
		``SA'': the synthesizability score;
		``Div'': the diversity among generated molecules;
		``time'': the time cost to generate molecules.}
		\par
		\par
		\end{footnotesize}
	\end{tablenotes}
	\end{scriptsize}
\end{threeparttable}
\end{table*}


%\label{tbl:overall_results_docking_100}

%-------------------------------------------------------------------------------------------------------------------------------------
%\subsection{{Comparison of Pocket Guidance}}
%\label{supp:app:results:docking}
%-------------------------------------------------------------------------------------------------------------------------------------


\begin{comment}
%-------------------------------------------------------------------------------------------------------------------------------------
\subsection{\ziqi{Simiarity Comparison for Pocket-based Molecule Generation}}
%-------------------------------------------------------------------------------------------------------------------------------------


\begin{table*}[t!]
	\centering
	\caption{{Overall Comparison on Similarity for Pocket-based Molecule Generation}}
	\label{tbl:docking_results_similarity}
	\begin{small}
		\begin{threeparttable}
			\begin{tabular}{
					@{\hspace{0pt}}l@{\hspace{5pt}}
					%
					@{\hspace{3pt}}l@{\hspace{3pt}}
					%
					@{\hspace{3pt}}r@{\hspace{8pt}}
					@{\hspace{3pt}}c@{\hspace{3pt}}
					%
					@{\hspace{3pt}}c@{\hspace{3pt}}
					@{\hspace{3pt}}c@{\hspace{3pt}}
					%
					@{\hspace{0pt}}c@{\hspace{0pt}}
					%
					@{\hspace{3pt}}c@{\hspace{3pt}}
					@{\hspace{3pt}}c@{\hspace{3pt}}
					%
					@{\hspace{3pt}}r@{\hspace{3pt}}
				}
				\toprule
				$\delta_g$  & method          & \#d\%$\uparrow$ & $\diversity_d$$\uparrow$(std) & \avgshapesim$\uparrow$(std) & \avggraphsim$\downarrow$(std) & & \maxshapesim$\uparrow$(std) & \maxgraphsim$\downarrow$(std)       & \#n\%$\uparrow$  \\ 
				\midrule
				%\multirow{6}{0.059\linewidth}{\hspace{0pt}0.1} 
				%& \AR   & 4.4 & 0.781(0.076) & 0.511(0.197) & \textbf{0.056}(0.020) & & 0.619(0.222) & 0.074(0.024) & 21.4  \\
				%& \pockettwomol & 6.6 & 0.795(0.099) & 0.519(0.216) & 0.063(0.020) & & 0.608(0.236) & 0.076(0.022) & \textbf{24.1}  \\
				%& \targetdiff & 2.0 & 0.872(0.041) & 0.619(0.110) & 0.068(0.018) & & 0.721(0.146) & 0.075(0.023) & 17.7  \\
				%& \decompdiffbeta & 0.0 & - & 0.374(0.138) & 0.059(0.031) & & 0.414(0.141) & \textbf{0.058}(0.032) & 9.8  \\
				%& \decompdiffref & 8.5 & 0.805(0.096) & 0.810(0.070) & 0.076(0.018) & & 0.861(0.085) & 0.076(0.020) & 11.3  \\
				%& \methodwithpguide   &  9.9 & \textbf{0.876}(0.041) & 0.795(0.058) & 0.073(0.015) & & 0.869(0.073) & 0.076(0.020) & 17.7  \\
				%& \methodwithsandpguide & \textbf{11.9} & 0.872(0.036) & \textbf{0.813}(0.052) & 0.075(0.014) & & \textbf{0.874}(0.069) & 0.080(0.014) & 17.0  \\
				%\cmidrule{2-10}
				%& improv\% & 40.4$^*$ & 8.8$^*$ & 0.4 & -30.4$^*$ &  & 1.6 & -30.0$^*$ & -26.3$^*$  \\
				%\midrule
				\multirow{7}{0.059\linewidth}{\hspace{0pt}1.0} 
				& \AR & 14.6 & 0.681(0.163) & 0.644(0.119) & 0.236(0.123) & & 0.780(0.110) & 0.284(0.177) & 95.8  \\
				& \pockettwomol & 18.6 & 0.711(0.152) & 0.654(0.131) &   \textbf{0.217}(0.129) & & 0.778(0.121) &   \textbf{0.243}(0.137) &  \textbf{98.3}  \\
				& \targetdiff & 7.1 &  \textbf{0.785}(0.085) & 0.622(0.083) & 0.238(0.122) & & 0.790(0.102) & 0.274(0.158) & 90.4  \\
				%& \decompdiffbeta & 0.1 & 0.589(0.030) & 0.494(0.124) & 0.263(0.143) & & 0.567(0.143) & 0.275(0.162) & 67.7  \\
				& \decompdiffref & 37.3 & 0.721(0.108) & 0.770(0.087) & 0.282(0.130) & & \textbf{0.878}(0.059) & 0.343(0.174) & 83.7  \\
				& \methodwithpguide   &  27.4 & 0.757(0.134) & 0.747(0.078) & 0.265(0.165) & & 0.841(0.081) & 0.272(0.168) & 98.1  \\
				& \methodwithsandpguide &\textbf{45.2} & 0.724(0.142) &   \textbf{0.789}(0.063) & 0.265(0.162) & & 0.876(0.062) & 0.264(0.159) & 97.8  \\
				\cmidrule{2-10}
				& Improv\%  & 21.2$^*$ & -3.6 & 2.5$^*$ & -21.7$^*$ &  & -0.1 & -8.4$^*$ & -0.2  \\
				\midrule
				\multirow{7}{0.059\linewidth}{\hspace{0pt}0.7} 
				& \AR   & 14.5 & 0.692(0.151) & 0.644(0.119) & 0.233(0.116) & & 0.779(0.110) & 0.266(0.140) & 94.9  \\
				& \pockettwomol & 18.6 & 0.711(0.152) & 0.654(0.131) & \textbf{0.217}(0.129) & & 0.778(0.121) & \textbf{0.243}(0.137) & \textbf{98.2}  \\
				& \targetdiff & 7.1 & \textbf{0.786}(0.084) & 0.622(0.083) & 0.238(0.121) & & 0.790(0.101) & 0.270(0.151) & 90.3  \\
				%& \decompdiffbeta & 0.1 & 0.589(0.030) & 0.494(0.124) & 0.263(0.142) & &0.567(0.143) & 0.273(0.156) & 67.6  \\
				& \decompdiffref & 36.2 & 0.721(0.113) & 0.770(0.086) & 0.273(0.123) & & \textbf{0.876}(0.059) & 0.325(0.139) & 82.3  \\
				& \methodwithpguide   &  27.4 & 0.757(0.134) & 0.746(0.078) & 0.263(0.160) & & 0.841(0.081) & 0.271(0.164) & 96.8  \\
				& \methodwithsandpguide      & \textbf{45.0} & 0.732(0.129) & \textbf{0.789}(0.063) & 0.262(0.157) & & \textbf{0.876}(0.063) & 0.262(0.153) & 96.2  \\
				\cmidrule{2-10}
				& Improv\%  & 24.3$^*$ & -3.6 & 2.5$^*$ & -20.8$^*$ &  & 0.0 & -7.6$^*$ & -1.5  \\
				\midrule
				\multirow{7}{0.059\linewidth}{\hspace{0pt}0.5} 
				& \AR   & 14.1 & 0.687(0.160) & 0.639(0.124) & 0.218(0.097) & & 0.778(0.110) & 0.260(0.130) & 89.8  \\
				& \pockettwomol & 18.5 & 0.711(0.152) & 0.649(0.134) & \textbf{0.209}(0.114) & & 0.777(0.121) & \textbf{0.240}(0.131) & \textbf{93.2}  \\
				& \targetdiff & 7.1 & \textbf{0.786}(0.084) & 0.621(0.083) & 0.230(0.111) & & 0.788(0.105) & 0.254(0.127) & 86.5  \\
				%&\decompdiffbeta & 0.1 & 0.595(0.025) & 0.494(0.124) & 0.254(0.129) & & 0.565(0.142) & 0.259(0.138) & 63.9  \\
				& \decompdiffref & 34.7 & 0.730(0.105) & 0.769(0.086) & 0.261(0.109) & & 0.874(0.080) & 0.301(0.117) & 77.3   \\
				& \methodwithpguide  &  27.2 & 0.765(0.123) & 0.749(0.075) & 0.245(0.135) & & 0.840(0.082) & 0.252(0.137) & 88.6  \\
				& \methodwithsandpguide & \textbf{44.3} & 0.738(0.122) & \textbf{0.791}(0.059) & 0.247(0.132) &  & \textbf{0.875}(0.065) & 0.249(0.130) & 88.8  \\
				\cmidrule{2-10}
				& Improv\%   & 27.8$^*$ & -2.7 & 2.9$^*$ & -17.6$^*$ &  & 0.2 & -3.4 & -4.7$^*$  \\
				\midrule
				\multirow{7}{0.059\linewidth}{\hspace{0pt}0.3} 
				& \AR   & 12.2 & 0.704(0.146) & 0.614(0.146) & 0.164(0.059) & & 0.751(0.138) & 0.206(0.059) & 66.4  \\
				& \pockettwomol & 17.1 & 0.731(0.129) & 0.617(0.163) & \textbf{0.155}(0.056) & & 0.740(0.159) & \textbf{0.190}(0.076) & \textbf{71.0}  \\
				& \targetdiff & 6.2 & \textbf{0.809}(0.061) & 0.619(0.087) & 0.181(0.068) & & 0.768(0.119) & 0.196(0.076) & 61.7  \\				
                %& \decompdiffbeta & 0.0 & - & 0.489(0.124) & 0.195(0.080) & & 0.547(0.139) & 0.203(0.087) & 42.0  \\
				& \decompdiffref & 27.7 & 0.775(0.081) & 0.767(0.086) & 0.202(0.062) & & 0.854(0.093) & 0.216(0.068) & 52.6  \\
				& \methodwithpguide   &  24.4 & 0.805(0.084) & 0.763(0.066) & 0.180(0.074) & & 0.847(0.080) & \textbf{0.190}(0.059) & 61.4  \\
				& \methodwithsandpguide & \textbf{36.3} & 0.789(0.081) & \textbf{0.800}(0.056) & 0.181(0.071) & &\textbf{0.878}(0.067) & \textbf{0.190}(0.078) & 61.8  \\
				\cmidrule{2-10}
				& improv\% & 31.1$^*$ & 3.9$^*$ & 4.3$^*$ & -16.5$^*$ &  & 2.8$^*$ & 0.0 & -12.9$^*$  \\
				\bottomrule
			\end{tabular}%
			\begin{tablenotes}[normal,flushleft]
				\begin{footnotesize}
					\item 
					\!\!Columns represent: \ziqi{``$\delta_g$'': the graph similarity constraint; ``\#n\%'': the percentage of molecules that satisfy the graph similarity constraint ($\graphsim<=\delta_g$);
						``\#d\%'': the percentage of molecules that satisfy the graph similarity constraint and are with high \shapesim ($\shapesim>=0.8$);
						``\avgshapesim/\avggraphsim'': the average of shape or graph similarities between the condition molecules and generated molecules with $\graphsim<=\delta_g$;
						``\maxshapesim'': the maximum of shape similarities between the condition molecules and generated molecules with $\graphsim<=\delta_g$;
						``\maxgraphsim'': the graph similarities between the condition molecules and the molecules with the maximum shape similarities and $\graphsim<=\delta_g$;
						``\diversity'': the diversity among the generated molecules.
						%
						``$\uparrow$'' represents higher values are better, and ``$\downarrow$'' represents lower values are better.
						%
						Best values are in \textbf{bold}, and second-best values are \underline{underlined}. 
					} 
					%\todo{double-check the significance value}
					\par
					\par
				\end{footnotesize}
			\end{tablenotes}
		\end{threeparttable}
	\end{small}
	\vspace{-10pt}    
\end{table*}
%\label{tbl:docking_results_similarity}

\bo{@Ziqi you may want to check my edits for the discussion in Table 1 first.
%
If the pocket if known, do you still care about the shape similarity in real applications?
}

\ziqi{Table~\ref{tbl:docking_results_similarity} presents the overall comparison on similarity-based metrics between \methodwithpguide, \methodwithsandpguide and other baselines under different graph similarity constraints  ($\delta_g$=1.0, 0.7, 0.5, 0.3), similar to Table~\ref{tbl:overall}. 
%
As shown in Table~\ref{tbl:docking_results_similarity}, regarding desirable molecules,  \methodwithsandpguide consistently outperforms all the baseline methods in the likelihood of generating desirable molecules (i.e., $\#d\%$).
%
For example, when $\delta_g$=1.0, at $\#d\%$, \methodwithsandpguide (45.2\%) demonstrates significant improvement of $21.2\%$ compared to the best baseline \decompdiff (37.3\%).
%
In terms of $\diversity_d$, \methodwithpguide and \methodwithsandpguide also achieve the second and the third best performance. 
%
Note that the best baseline \targetdiff in $\diversity_d$ achieves the least percentage of desirable molecules (7.1\%), substantially lower than \methodwithpguide and \methodwithsandpguide.
%
This makes its diversity among desirable molecules incomparable with other methods. 
%
When $\delta_g$=0.7, 0.5, and 0.3, \methodwithsandpguide also establishes a significant improvement of 24.3\%, 27.8\%, and 31.1\% compared to the best baseline method \decompdiff.
%
It is also worth noting that the state-of-the-art baseline \decompdiff underperforms \methodwithpguide and \methodwithsandpguide in binding affinities as shown in Table~\ref{tbl:overall_results_docking}, even though it outperforms \methodwithpguide in \#d\%.
%
\methodwithpguide and \methodwithsandpguide also achieve the second and the third best performance in $\diversity_d$ at $\delta_g$=0.7, 0.5, and 0.3. 
%
The superior performance of \methodwithpguide and \methodwithsandpguide in $\#d\%$ at small $\delta_g$ indicates their strong capacity in generating desirable molecules of novel graph structures, thereby facilitating the discovery of novel drug candidates.
%
}

\ziqi{Apart from the desirable molecules, \methodwithpguide and \methodwithsandpguide also demonstrate outstanding performance in terms of the average shape similarities (\avgshapesim) and the average graph similarities (\avggraphsim).
%
Specifically, when $\delta_g$=1.0, \methodwithsandpguide achieves a significant 2.5\% improvement in \avgshapesim\ over the best baseline \decompdiff. 
%
In terms of \avggraphsim, \methodwithsandpguide also achieves higher performance than the baseline \decompdiff of the highest \avgshapesim (0.265 vs 0.282).
%
Please note that all the baseline methods except \decompdiff achieve substantially lower performance in \avgshapesim than \methodwithpguide and \methodwithsandpguide, even though these methods achieve higher \avggraphsim values.
%
This trend remains consistent when applying various similarity constraints (i.e., $\delta_g$) as shown in Table~\ref{tbl:overall_results_docking}.
}

\ziqi{Similarly, \methodwithpguide and \methodwithsandpguide also achieve superior performance in \maxshapesim and \maxgraphsim.
%
Specifically, when $\delta_g$=1.0, for \maxshapesim, \methodwithsandpguide achieves highly comparable performance in \maxshapesim\ compared to the best baseline \decompdiff (0.876 vs 0.878).
%
We also note that \methodwithsandpguide achieves lower \maxgraphsim\ than the \decompdiff with 23.0\% difference. 
%
When $\delta_g$ gets smaller from 0.7 to 0.3, \methodwithsandpguide maintains a high \maxshapesim value around 0.876, while the best baseline \decompdiff has \maxshapesim decreased from 0.878 to 0.854.
%
This demonstrates the superior ability of \methodwithsandpguide in generating molecules with similar shapes and novel structures.
%
}

\ziqi{
In terms of \#n\%, when $\delta_g$=1.0, the percentage of molecules with \graphsim below $\delta_g$ can be interpreted as the percentage of valid molecules among all the generated molecules. 
%
As shown in Table~\ref{tbl:docking_results_similarity}, \methodwithpguide and \methodwithsandpguide are able to generate 98.1\% and 97.8\% of valid molecules, slightly below the best baseline \pockettwomol (98.3\%). 
%
When $\delta_g$=0.7, 0.5, or 0.3, all the methods, including \methodwithpguide and \methodwithsandpguide, can consistently find a sufficient number of novel molecules that meet the graph similarity constraints.
%
The only exception is \decompdiff, which substantially underperforms all the other methods in \#n\%.
}
\end{comment}

%%%%%%%%%%%%%%%%%%%%%%%%%%%%%%%%%%%%%%%%%%%%%
\section{Properties of Molecules in Case Studies for Targets}
\label{supp:app:results:properties}
%%%%%%%%%%%%%%%%%%%%%%%%%%%%%%%%%%%%%%%%%%%%%

%-------------------------------------------------------------------------------------------------------------------------------------
\subsection{Drug Properties of Generated Molecules}
\label{supp:app:results:properties:drug}
%-------------------------------------------------------------------------------------------------------------------------------------

Table~\ref{tbl:drug_property} presents the drug properties of three generated molecules: NL-001, NL-002, and NL-003.
%
As shown in Table~\ref{tbl:drug_property}, each of these molecules has a favorable profile, making them promising drug candidates. 
%
{As discussed in Section ``Case Studies for Targets'' in the main manuscript, all three molecules have high binding affinities in terms of Vina S, Vina M and Vina D, and favorable QED and SA values.
%
In addition, all of them meet the Lipinski's rule of five criteria~\cite{Lipinski1997}.}
%
In terms of physicochemical properties, all these properties of NL-001, NL-002 and NL-003, including number of rotatable bonds, molecule weight, LogP value, number of hydrogen bond doners and acceptors, and molecule charges, fall within the desired range of drug molecules. 
%
This indicates that these molecules could potentially have good solubility and membrane permeability, essential qualities for effective drug absorption.

These generated molecules also demonstrate promising safety profiles based on the predictions from ICM~\cite{Neves2012}.
%
In terms of drug-induced liver injury prediction scores, all three molecules have low scores (0.188 to 0.376), indicating a minimal risk of hepatotoxicity. 
%
NL-001 and NL-002 fall under `Ambiguous/Less concern' for liver injury, while NL-003 is categorized under 'No concern' for liver injury. 
%
Moreover, all these molecules have low toxicity scores (0.000 to 0.236). 
%
NL-002 and NL-003 do not have any known toxicity-inducing functional groups. 
%
NL-001 and NL-003 are also predicted not to include any known bad groups that lead to inappropriate features.
%
These attributes highlight the potential of NL-001, NL-002, and NL-003 as promising treatments for cancers and Alzheimer’s disease.

%\begin{table*}
	\centering
		\caption{Drug Properties of Generated Molecules}
	\label{tbl:binding_drug_mols}
	\begin{scriptsize}
\begin{threeparttable}
	\begin{tabular}{
		@{\hspace{6pt}}r@{\hspace{6pt}}
		@{\hspace{6pt}}r@{\hspace{6pt}}
		@{\hspace{6pt}}r@{\hspace{6pt}}
		@{\hspace{6pt}}r@{\hspace{6pt}}
		@{\hspace{6pt}}r@{\hspace{6pt}}
		@{\hspace{6pt}}r@{\hspace{6pt}}
		@{\hspace{6pt}}r@{\hspace{6pt}}
		@{\hspace{6pt}}r@{\hspace{6pt}}
		@{\hspace{6pt}}r@{\hspace{6pt}}
		%
		}
		\toprule
Target & Molecule & Vina S & Vina M & Vina D & QED   & SA   & Logp  & Lipinski \\
\midrule
\multirow{3}{*}{CDK6} & NL-001 & -6.817      & -7.251    & -8.319     & 0.834 & 0.72 & 1.313 & 5        \\
& NL-002 & -6.970       & -7.605    & -8.986     & 0.851 & 0.74 & 3.196 & 5        \\
\cmidrule{2-9}
& 4AU & 0.736       & -5.939    & -7.592     & 0.773 & 0.79 & 2.104 & 5        \\
\midrule
\multirow{2}{*}{NEP} & NL-003 & -11.953     & -12.165   & -12.308    & 0.772 & 0.57 & 2.944 & 5        \\
\cmidrule{2-9}
& BIR & -9.399      & -9.505    & -9.561     & 0.463 & 0.73 & 2.677 & 5        \\
		\bottomrule
	\end{tabular}%
	\begin{tablenotes}[normal,flushleft]
		\begin{footnotesize}
	\item Columns represent: {``Target'': the names of protein targets;
		``Molecule'': the names of generated molecules and known ligands;
		``Vina S'': the binding affinities between the initially generated poses of molecules and the protein pockets; 
		``Vina M'': the binding affinities between the poses after local structure minimization and the protein pockets;
		``Vina D'': the binding affinities between the poses determined by AutoDock Vina~\cite{Eberhardt2021} and the protein pockets;
		``HA'': the percentage of generated molecules with Vina D higher than those of condition molecules;
		``QED'': the drug-likeness score;
		``SA'': the synthesizability score;
		``Div'': the diversity among generated molecules;
		``time'': the time cost to generate molecules.}
\!\! \par
		\par
		\end{footnotesize}
	\end{tablenotes}
\end{threeparttable}
\end{scriptsize}
  \vspace{-10pt}    
\end{table*}

%\label{tbl:binding_drug_mols}

\begin{table*}
	\centering
		\caption{Drug Properties of Generated Molecules}
	\label{tbl:drug_property}
	\begin{scriptsize}
\begin{threeparttable}
	\begin{tabular}{
		@{\hspace{0pt}}p{0.23\linewidth}@{\hspace{5pt}}
		%
		@{\hspace{1pt}}r@{\hspace{2pt}}
		@{\hspace{2pt}}r@{\hspace{6pt}}
		@{\hspace{6pt}}r@{\hspace{6pt}}
		%
		}
		\toprule
		Property Name & NL-001 & NL-002 & NL-003 \\
		\midrule
Vina S & -6.817 &  -6.970 & -11.953 \\
Vina M & -7.251 & -7.605 & -12.165 \\
Vina D & -8.319 & -8.986 & -12.308 \\
QED    & 0.834  & 0.851  & 0.772 \\
SA       & 0.72    & 0.74    & 0.57    \\
Lipinski & 5 & 5 & 5 \\
%bbbScore          & 3.386                                                                                        & 4.240                                                                                        & 3.892      \\
%drugLikeness      & -0.081                                                                                       & -0.442                                                                                       & -0.325     \\
%molLogP1          & 1.698                                                                                        & 2.685                                                                                        & 2.382      \\
\#rotatable bonds          & 3                                                                                        & 2                                                                                        & 2      \\
molecule weight         & 267.112                                                                                      & 270.117                                                                                      & 390.206    \\
molecule LogP           & 1.698                                                                                        & 2.685                                                                                        & 2.382     \\
\#hydrogen bond doners           & 1                                                                                        & 1                                                                                        & 2      \\
\#hydrogen bond acceptors           & 5                                                                                       & 3                                                                                        & 5      \\
\#molecule charges   & 1                                                                                        & 0                                                                                        & 0      \\
drug-induced liver injury predScore    & 0.227                                                                                        & 0.376                                                                                        & 0.188      \\
drug-induced liver injury predConcern  & Ambiguous/Less concern                                                                       & Ambiguous/Less concern                                                                       & No concern \\
drug-induced liver injury predLabel    & Warnings/Precautions/Adverse reactions & Warnings/Precautions/Adverse reactions & No match   \\
drug-induced liver injury predSeverity & 2                                                                                        & 3                                                                                        & 2      \\
%molSynth1         & 0.254                                                                                        & 0.220                                                                                        & 0.201      \\
%toxicity class         & 0.480                                                                                        & 0.480                                                                                        & 0.450      \\
toxicity names         & hydrazone                                                                                    &   -                                                                                           &   -         \\
toxicity score         & 0.236                                                                                        & 0.000                                                                                        & 0.000      \\
bad groups         & -                                                                                             & Tetrahydroisoquinoline:   allergies                                                          &   -         \\
%MolCovalent       &                                                                                              &                                                                                              &            \\
%MolProdrug        &                                                                                              &                                                                                              &            \\
		\bottomrule
	\end{tabular}%
	\begin{tablenotes}[normal,flushleft]
		\begin{footnotesize}
	\item ``-'': no results found by algorithms
\!\! \par
		\par
		\end{footnotesize}
	\end{tablenotes}
\end{threeparttable}
\end{scriptsize}
  \vspace{-10pt}    
\end{table*}

%\label{tbl:drug_property}

%-------------------------------------------------------------------------------------------------------------------------------------
\subsection{Comparison on ADMET Profiles between Generated Molecules and Approved Drugs}
\label{supp:app:results:properties:admet}
%-------------------------------------------------------------------------------------------------------------------------------------

\paragraph{Generated Molecules for CDK6}
%
Table~\ref{tbl:admet_cdk6} presents the comparison on ADMET profiles between two generated molecules for CDK6 and the approved CDK6 inhibitors, including Abemaciclib~\cite{Patnaik2016}, Palbociclib~\cite{Lu2015}, and Ribociclib~\cite{Tripathy2017}.
%
As shown in Table~\ref{tbl:admet_cdk6}, the generated molecules, NL-001 and NL-002, exhibit comparable ADMET profiles with those of the approved CDK6 inhibitors. 
%
Importantly, both molecules demonstrate good potential in most crucial properties, including Ames mutagenesis, favorable oral toxicity, carcinogenicity, estrogen receptor binding, high intestinal absorption and favorable oral bioavailability.
%
Although the generated molecules are predicted as positive in hepatotoxicity and mitochondrial toxicity, all the approved drugs are also predicted as positive in these two toxicity.
%
This result suggests that these issues might stem from the limited prediction accuracy rather than being specific to our generated molecules.
%
Notably, NL-001 displays a potentially better plasma protein binding score compared to other molecules, which may improve its distribution within the body. 
%
Overall, these results indicate that NL-001 and NL-002 could be promising candidates for further drug development.


\begin{table*}
	\centering
		\caption{Comparison on ADMET Profiles among Generated Molecules and Approved Drugs Targeting CDK6}
	\label{tbl:admet_cdk6}
	\begin{scriptsize}
\begin{threeparttable}
	\begin{tabular}{
		%@{\hspace{0pt}}p{0.23\linewidth}@{\hspace{5pt}}
		%
		@{\hspace{6pt}}l@{\hspace{5pt}}
		@{\hspace{6pt}}r@{\hspace{6pt}}
		@{\hspace{6pt}}r@{\hspace{6pt}}
		@{\hspace{6pt}}r@{\hspace{6pt}}
		@{\hspace{6pt}}r@{\hspace{6pt}}
		@{\hspace{6pt}}r@{\hspace{6pt}}
		%
		%
		@{\hspace{6pt}}r@{\hspace{6pt}}
		%@{\hspace{6pt}}r@{\hspace{6pt}}
		%
		}
		\toprule
		\multirow{2}{*}{Property name} & \multicolumn{2}{c}{Generated molecules} & & \multicolumn{3}{c}{FDA-approved drugs} \\
		\cmidrule{2-3}\cmidrule{5-7}
		 & NL--001 & NL--002 & & Abemaciclib & Palbociclib & Ribociclib \\
		\midrule
\rowcolor[HTML]{D2EAD9}Ames   mutagenesis                             & --   &  --  & & + &  --  & --  \\
\rowcolor[HTML]{D2EAD9}Acute oral toxicity (c)                           & III & III & &  III          & III          & III         \\
Androgen receptor binding                         & +                          & +            &              & +            & +            & +             \\
Aromatase binding                                 & +                          & +            &              & +            & +            & +            \\
Avian toxicity                                    & --                          & --          &                & --            & --            & --            \\
Blood brain barrier                               & +                          & +            &              & +            & +            & +            \\
BRCP inhibitior                                   & --                          & --          &                & --            & --            & --            \\
Biodegradation                                    & --                          & --          &                & --            & --            & --           \\
BSEP inhibitior            & +                          & +            &              & +            & +            & +        \\
Caco-2                                            & +                          & +            &              & --            & --            & --            \\
\rowcolor[HTML]{D2EAD9}Carcinogenicity (binary)                          & --                          & --             &             & --            & --            & --          \\
\rowcolor[HTML]{D2EAD9}Carcinogenicity (trinary)                         & Non-required               & Non-required   &            & Non-required & Non-required & Non-required  \\
Crustacea aquatic toxicity & --                          & --            &              & --            & --            & --            \\
 CYP1A2 inhibition                                 & +                          & +            &              & --            & --            & +             \\
CYP2C19 inhibition                                & --                          & +            &              & +            & --            & +            \\
CYP2C8 inhibition                                 & --                          & --           &               & +            & +            & +            \\
CYP2C9 inhibition                                 & --                          & --           &               & --            & --            & +             \\
CYP2C9 substrate                                  & --                          & --           &               & --            & --            & --            \\
CYP2D6 inhibition                                 & --                          & --           &               & --            & --            & --            \\
CYP2D6 substrate                                  & --                          & --           &               & --            & --            & --            \\
CYP3A4 inhibition                                 & --                          & +            &              & --            & --            & --            \\
CYP3A4 substrate                                  & +                          & --            &              & +            & +            & +            \\
\rowcolor[HTML]{D2EAD9}CYP inhibitory promiscuity                        & +                          & +                    &      & +            & --            & +            \\
Eye corrosion                                     & --                          & --           &               & --            & --            & --            \\
Eye irritation                                    & --                          & --           &               & --            & --            & --             \\
\rowcolor[HTML]{D8E7FF}Estrogen receptor binding                         & +                          & +                    &      & +            & +            & +            \\
Fish aquatic toxicity                             & --                          & +            &              & +            & --            & --            \\
Glucocorticoid receptor   binding                 & +                          & +             &             & +            & +            & +            \\
Honey bee toxicity                                & --                          & --           &               & --            & --            & --            \\
\rowcolor[HTML]{D2EAD9}Hepatotoxicity                                    & +                          & +            &              & +            & +            & +             \\
Human ether-a-go-go-related gene inhibition     & +                          & +               &           & +            & --            & --           \\
\rowcolor[HTML]{D8E7FF}Human intestinal absorption                       & +                          & +             &             & +            & +            & +    \\
\rowcolor[HTML]{D8E7FF}Human oral bioavailability                        & +                          & +              &            & +            & +            & +     \\
\rowcolor[HTML]{D2EAD9}MATE1 inhibitior                                  & --                          & --              &            & --            & --            & --    \\
\rowcolor[HTML]{D2EAD9}Mitochondrial toxicity                            & +                          & +                &          & +            & +            & +    \\
Micronuclear                                      & +                          & +                          & +            & +            & +           \\
\rowcolor[HTML]{D2EAD9}Nephrotoxicity                                    & --                          & --             &             & --            & --            & --             \\
Acute oral toxicity                               & 2.325                      & 1.874    &     & 1.870        & 3.072        & 3.138        \\
\rowcolor[HTML]{D8E7FF}OATP1B1 inhibitior                                & +                          & +              &            & +            & +            & +             \\
\rowcolor[HTML]{D8E7FF}OATP1B3 inhibitior                                & +                          & +              &            & +            & +            & +             \\
\rowcolor[HTML]{D2EAD9}OATP2B1 inhibitior                                & --                          & --             &             & --            & --            & --             \\
OCT1 inhibitior                                   & --                          & --        &                  & +            & --            & +             \\
OCT2 inhibitior                                   & --                          & --        &                  & --            & --            & +             \\
P-glycoprotein inhibitior                         & --                          & --        &                  & +            & +            & +     \\
P-glycoprotein substrate                          & --                          & --        &                  & +            & +            & +     \\
PPAR gamma                                        & +                          & +          &                & +            & +            & +      \\
\rowcolor[HTML]{D8E7FF}Plasma protein binding                            & 0.359        & 0.745     &    & 0.865        & 0.872        & 0.636       \\
Reproductive toxicity                             & +                          & +          &                & +            & +            & +           \\
Respiratory toxicity                              & +                          & +          &                & +            & +            & +         \\
Skin corrosion                                    & --                          & --        &                  & --            & --            & --           \\
Skin irritation                                   & --                          & --        &                  & --            & --            & --         \\
Skin sensitisation                                & --                          & --        &                  & --            & --            & --          \\
Subcellular localzation                           & Mitochondria               & Mitochondria  &             & Lysosomes    & Mitochondria & Mitochondria \\
Tetrahymena pyriformis                            & 0.398                      & 0.903         &             & 1.033        & 1.958        & 1.606         \\
Thyroid receptor binding                          & +                          & +             &             & +            & +            & +           \\
UGT catelyzed                                     & --                          & --           &               & --            & --            & --           \\
\rowcolor[HTML]{D8E7FF}Water solubility                                  & -3.050                     & -3.078              &       & -3.942       & -3.288       & -2.673     \\
		\bottomrule
	\end{tabular}%
	\begin{tablenotes}[normal,flushleft]
		\begin{footnotesize}
	\item Blue cells highlight crucial properties where a negative outcome (``--'') is desired; for acute oral toxicity (c), a higher category (e.g., ``III'') is desired; and for carcinogenicity (trinary), ``Non-required'' is desired.
	%
	Green cells highlight crucial properties where a positive result (``+'') is desired; for plasma protein binding, a lower value is desired; and for water solubility, values higher than -4 are desired~\cite{logs}.
\!\! \par
		\par
		\end{footnotesize}
	\end{tablenotes}
\end{threeparttable}
\end{scriptsize}
  \vspace{--10pt}    
\end{table*}

%\label{tbl:admet_cdk6}

\paragraph{Generated Molecule for NEP}
%
Table~\ref{tbl:admet_nep} presents the comparison on ADMET profiles between a generated molecule for NEP targeting Alzheimer's disease and three approved drugs, Donepezil, Galantamine, and Rivastigmine, for Alzheimer's disease~\cite{Hansen2008}.
%
Overall, NL-003 exhibits a comparable ADMET profile with the three approved drugs.  
%
Notably, same as other approved drugs, NL-003 is predicted to be able to penetrate the blood brain barrier, a crucial property for Alzheimer's disease.
%  
In addition, it demonstrates a promising safety profile in terms of Ames mutagenesis, favorable oral toxicity, carcinogenicity, estrogen receptor binding, high intestinal absorption, nephrotoxicity and so on.
%
These results suggest that NL-003 could be promising candidates for the drug development of Alzheimer's disease.

\begin{table*}
	\centering
		\caption{Comparison on ADMET Profiles among Generated Molecule Targeting NEP and Approved Drugs for Alzhimer's Disease}
	\label{tbl:admet_nep}
	\begin{scriptsize}
\begin{threeparttable}
	\begin{tabular}{
		@{\hspace{6pt}}l@{\hspace{5pt}}
		%
		@{\hspace{6pt}}r@{\hspace{6pt}}
		@{\hspace{6pt}}r@{\hspace{6pt}}
		@{\hspace{6pt}}r@{\hspace{6pt}}
		@{\hspace{6pt}}r@{\hspace{6pt}}
		@{\hspace{6pt}}r@{\hspace{6pt}}
		%
		%
		%@{\hspace{6pt}}r@{\hspace{6pt}}
		%
		}
		\toprule
		\multirow{2}{*}{Property name} & Generated molecule & & \multicolumn{3}{c}{FDA-approved drugs} \\
\cmidrule{2-2}\cmidrule{4-6}
			& NL--003 & & Donepezil	& Galantamine & Rivastigmine \\
		\midrule
\rowcolor[HTML]{D2EAD9} 
Ames   mutagenesis                            & --                      &              & --                                    & --                                 & --                     \\
\rowcolor[HTML]{D2EAD9}Acute oral toxicity (c)                       & III           &                       & III                                  & III                               & II                      \\
Androgen receptor binding                     & +      &      & +            & --         & --         \\
Aromatase binding                             & --     &       & +            & --         & --        \\
Avian toxicity                                & --     &                               & --                                    & --                                 & --                        \\
\rowcolor[HTML]{D8E7FF} 
Blood brain barrier                           & +      &                              & +                                    & +                                 & +                        \\
BRCP inhibitior                               & --     &       & --            & --         & --         \\
Biodegradation                                & --     &                               & --                                    & --                                 & --                        \\
BSEP inhibitior                               & +      &      & +            & --         & --         \\
Caco-2                                        & +      &      & +            & +         & +         \\
\rowcolor[HTML]{D2EAD9} 
Carcinogenicity (binary)                      & --     &                               & --                                    & --                                 & --                        \\
\rowcolor[HTML]{D2EAD9} 
Carcinogenicity (trinary)                     & Non-required    &                     & Non-required                         & Non-required                      & Non-required             \\
Crustacea aquatic toxicity                    & +               &                     & +                                    & +                                 & --                        \\
CYP1A2 inhibition                             & +               &                     & +                                    & --                                 & --                        \\
CYP2C19 inhibition                            & +               &                     & --                                    & --                                 & --                        \\
CYP2C8 inhibition                             & +               &                     & --                                    & --                                 & --                        \\
CYP2C9 inhibition                             & --              &                      & --                                    & --                                 & --                        \\
CYP2C9 substrate                              & --              &                      & --                                    & --                                 & --                        \\
CYP2D6 inhibition                             & --              &                      & +                                    & --                                 & --                        \\
CYP2D6 substrate                              & --              &                      & +                                    & +                                 & +                        \\
CYP3A4 inhibition                             & --              &                      & --                                    & --                                 & --                        \\
CYP3A4 substrate                              & +               &                     & +                                    & +                                 & --                        \\
\rowcolor[HTML]{D2EAD9} 
CYP inhibitory promiscuity                    & +               &                     & +                                    & --                                 & --                        \\
Eye corrosion                                 & --     &       & --            & --         & --         \\
Eye irritation                                & --     &       & --            & --         & --         \\
Estrogen receptor binding                     & +      &      & +            & --         & --         \\
Fish aquatic toxicity                         & --     &                               & +                                    & +                                 & +                        \\
Glucocorticoid receptor binding             & --      &      & +            & --         & --         \\
Honey bee toxicity                            & --    &                                & --                                    & --                                 & --                        \\
\rowcolor[HTML]{D2EAD9} 
Hepatotoxicity                                & +     &                               & +                                    & --                                 & --                        \\
Human ether-a-go-go-related gene inhibition & +       &     & +            & --         & --         \\
\rowcolor[HTML]{D8E7FF} 
Human intestinal absorption                   & +     &                               & +                                    & +                                 & +                        \\
\rowcolor[HTML]{D8E7FF} 
Human oral bioavailability                    & --    &                                & +                                    & +                                 & +                        \\
\rowcolor[HTML]{D2EAD9} 
MATE1 inhibitior                              & --    &                                & --                                    & --                                 & --                        \\
\rowcolor[HTML]{D2EAD9} 
Mitochondrial toxicity                        & +     &                               & +                                    & +                                 & +                        \\
Micronuclear                                  & +     &       & --            & --         & +         \\
\rowcolor[HTML]{D2EAD9} 
Nephrotoxicity                                & --    &                                & --                                    & --                                 & --                        \\
Acute oral toxicity                           & 2.704  &      & 2.098        & 2.767     & 2.726     \\
\rowcolor[HTML]{D8E7FF} 
OATP1B1 inhibitior                            & +      &                              & +                                    & +                                 & +                        \\
\rowcolor[HTML]{D8E7FF} 
OATP1B3 inhibitior                            & +      &                              & +                                    & +                                 & +                        \\
\rowcolor[HTML]{D2EAD9} 
OATP2B1 inhibitior                            & --     &                               & --                                    & --                                 & --                        \\
OCT1 inhibitior                               & +      &      & +            & --         & --         \\
OCT2 inhibitior                               & --     &       & +            & --         & --         \\
P-glycoprotein inhibitior                     & +      &      & +            & --         & --         \\
\rowcolor[HTML]{D8E7FF} 
P-glycoprotein substrate                      & +      &                              & +                                    & +                                 & --                        \\
PPAR gamma                                    & +      &      & --            & --         & --         \\
\rowcolor[HTML]{D8E7FF} 
Plasma protein binding                        & 0.227   &                             & 0.883                                & 0.230                             & 0.606                    \\
Reproductive toxicity                         & +       &     & +            & +         & +         \\
Respiratory toxicity                          & +       &     & +            & +         & +         \\
Skin corrosion                                & --      &      & --            & --         & --         \\
Skin irritation                               & --      &      & --            & --         & --         \\
Skin sensitisation                            & --      &      & --            & --         & --         \\
Subcellular localzation                       & Mitochondria & &Mitochondria & Lysosomes & Mitochondria  \\
Tetrahymena pyriformis                        & 0.053           &                     & 0.979                                & 0.563                             & 0.702                        \\
Thyroid receptor binding                      & +       &     & +            & +         & --             \\
UGT catelyzed                                 & --      &      & --            & +         & --             \\
\rowcolor[HTML]{D8E7FF} 
Water solubility                              & -3.586   &                            & -2.425                               & -2.530                            & -3.062                       \\
		\bottomrule
	\end{tabular}%
	\begin{tablenotes}[normal,flushleft]
		\begin{footnotesize}
	\item Blue cells highlight crucial properties where a negative outcome (``--'') is desired; for acute oral toxicity (c), a higher category (e.g., ``III'') is desired; and for carcinogenicity (trinary), ``Non-required'' is desired.
	%
	Green cells highlight crucial properties where a positive result (``+'') is desired; for plasma protein binding, a lower value is desired; and for water solubility, values higher than -4 are desired~\cite{logs}.
\!\! \par
		\par
		\end{footnotesize}
	\end{tablenotes}
\end{threeparttable}
\end{scriptsize}
  \vspace{--10pt}    
\end{table*}

%\label{tbl:admet_nep}

\clearpage
%%%%%%%%%%%%%%%%%%%%%%%%%%%%%%%%%%%%%%%%%%%%%
\section{Algorithms}
\label{supp:algorithms}
%%%%%%%%%%%%%%%%%%%%%%%%%%%%%%%%%%%%%%%%%%%%%

Algorithm~\ref{alg:shapemol} describes the molecule generation process of \method.
%
Given a known ligand \molx, \method generates a novel molecule \moly that has a similar shape to \molx and thus potentially similar binding activity.
%
\method can also take the protein pocket \pocket as input and adjust the atoms of generated molecules for optimal fit and improved binding affinities.
%
Specifically, \method first calculates the shape embedding \shapehiddenmat for \molx using the shape encoder \SEE described in Algorithm~\ref{alg:see_shaperep}.
%
Based on \shapehiddenmat, \method then generates a novel molecule with a similar shape to \molx using the diffusion-based generative model \methoddiff as in Algorithm~\ref{alg:diffgen}.
%
During generation, \method can use shape guidance to directly modify the shape of \moly to closely resemble the shape of \molx.
%
When the protein pocket \pocket is available, \method can also use pocket guidance to ensure that \moly is specifically tailored to closely fit within \pocket.

\begin{algorithm}[!h]
    \caption{\method}
    \label{alg:shapemol}
         %\hspace*{\algorithmicindent} 
	\textbf{Required Input: $\molx$} \\
 	%\hspace*{\algorithmicindent} 
	\textbf{Optional Input: $\pocket$} 
    \begin{algorithmic}[1]
        \FullLineComment{calculate a shape embedding with Algorithm~\ref{alg:see_shaperep}}
        \State $\shapehiddenmat$, $\pc$ = $\SEE(\molx)$
        \FullLineComment{generate a molecule conditioned on the shape embedding with Algorithm~\ref{alg:diffgen}}
         \If{\pocket is not available}
        \State $\moly = \diffgenerative(\shapehiddenmat, \molx)$
        \Else
        \State $\moly = \diffgenerative(\shapehiddenmat, \molx, \pocket)$
        \EndIf
        \State \Return \moly
    \end{algorithmic}
\end{algorithm}
%\label{alg:shapemol}

\begin{algorithm}[!h]
    \caption{\SEE for shape embedding calculation}
    \label{alg:see_shaperep}
    \textbf{Required Input: $\molx$}
    \begin{algorithmic}[1]
        %\Require $\molx$
        \FullLineComment{sample a point cloud over the molecule surface shape}
        \State $\pc$ = $\text{samplePointCloud}(\molx)$
        \FullLineComment{encode the point cloud into a latent embedding (Equation~\ref{eqn:shape_embed})}
        \State $\shapehiddenmat = \SEE(\pc)$
        \FullLineComment{move the center of \pc to zero}
        \State $\pc = \pc - \text{center}(\pc)$
        \State \Return \shapehiddenmat, \pc
    \end{algorithmic}
\end{algorithm}
%\label{alg:see_shaperep}

\begin{algorithm}[!h]
    \caption{\diffgenerative for molecule generation}
    \label{alg:diffgen}
    	\textbf{Required Input: $\molx$, \shapehiddenmat} \\
 	%\hspace*{\algorithmicindent} 
	\textbf{Optional Input: $\pocket$} 
    \begin{algorithmic}[1]
        \FullLineComment{sample the number of atoms in the generated molecule}
        \State $n = \text{sampleAtomNum}(\molx)$
        \FullLineComment{sample initial positions and types of $n$ atoms}
        \State $\{\pos_T\}^n = \mathcal{N}(0, I)$
        \State $\{\atomfeat_T\}^n = C(K, \frac{1}{K})$
        \FullLineComment{generate a molecule by denoising $\{(\pos_T, \atomfeat_T)\}^n$ to $\{(\pos_0, \atomfeat_0)\}^n$}
        \For{$t = T$ to $1$}
            \IndentLineComment{predict the molecule without noise using the shape-conditioned molecule prediction module \molpred}{1.5}
            \State $(\tilde{\pos}_{0,t}, \tilde{\atomfeat}_{0,t}) = \molpred(\pos_t, \atomfeat_t, \shapehiddenmat)$
            \If{use shape guidance and $t > s$}
                \State $\tilde{\pos}_{0,t} = \shapeguide(\tilde{\pos}_{0,t}, \molx)$
                %\State $\tilde{\pos}_{0,t} = \pos^*_{0,t}$
            \EndIf
            \IndentLineComment{sample $(\pos_{t-1}, \atomfeat_{t-1})$ from $(\pos_t, \atomfeat_t)$ and $(\tilde{\pos}_{0,t}, \tilde{\atomfeat}_{0,t})$}{1.5}
            \State $\pos_{t-1} = P(\pos_{t-1}|\pos_t, \tilde{\pos}_{o,t})$
            \State $\atomfeat_{t-1} = P(\atomfeat_{t-1}|\atomfeat_t, \tilde{\atomfeat}_{o,t})$
            \If{use pocket guidance and $\pocket$ is available}
                \State $\pos_{t-1} = \pocketguide(\pos_{t-1}, \pocket)$
                %\State $\pos_{t-1} = \pos_{t-1}^*$
            \EndIf  
        \EndFor
        \State \Return $\moly = (\pos_0, \atomfeat_0)$
    \end{algorithmic}
\end{algorithm}
%\label{alg:diffgen}

%\input{algorithms/train_SE}
%\label{alg:train_se}

%\begin{algorithm}[!h]
    \caption{Training Procedure of \methoddiff}
    \label{alg:diffgen}
    \begin{algorithmic}[1]
        \Require $\shapehiddenmat, \molx, \pocket$
        \FullLineComment{sample the number of atoms in the generated molecule}
    \end{algorithmic}
\end{algorithm}
%\label{alg:train_diff}

%---------------------------------------------------------------------------------------------------------------------
\section{{Equivariance and Invariance}}
\label{supp:ei}
%---------------------------------------------------------------------------------------------------------------------

%.................................................................................................
\subsection{Equivariance}
\label{supp:ei:equivariance}
%.................................................................................................

{Equivariance refers to the property of a function $f(\pos)$ %\bo{is it the property of the function or embedding (x)?} 
that any translation and rotation transformation from the special Euclidean group SE(3)~\cite{Atz2021} applied to a geometric object
$\pos\in\mathbb{R}^3$ is mirrored in the output of $f(\pos)$, accordingly.
%
This property ensures $f(\pos)$ to learn a consistent representation of an object's geometric information, regardless of its orientation or location in 3D space.
%
%As a result, it provides $f(\pos)$ better generalization capabilities~\cite{Jonas20a}.
%
Formally, given any translation transformation $\mathbf{t}\in\mathbb{R}^3$ and rotation transformation $\mathbf{R}\in\mathbb{R}^{3\times3}$ ($\mathbf{R}^{\mathsf{T}}\mathbf{R}=\mathbb{I}$), %\xia{change the font types for $^{\mathsf{T}}$ and $\mathbb{I}$ in the entire manuscript}), 
$f(\pos)$ is equivariant with respect to these transformations %$g$ (\bo{where is $g$...})
if it satisfies
\begin{equation}
f(\mathbf{R}\pos+\mathbf{t}) = \mathbf{R}f(\pos) + \mathbf{t}. %\ \text{where}\ \hiddenpos = f(\pos).
\end{equation}
%
%where $\hiddenpos=f(\pos)$ is the output of $\pos$. 
%
In \method, both \SE and \methoddiff are developed to guarantee equivariance in capturing the geometric features of objects regardless of any translation or rotation transformations, as will be detailed in the following sections.
}

%.................................................................................................
\subsection{Invariance}
\label{supp:ei:invariance}
%.................................................................................................

%In contrast to equivariance, 
Invariance refers to the property of a function that its output {$f(\pos)$} remains constant under any translation and rotation transformations of the input $\pos$. %a geometric object's feature $\pos$.
%
This property enables $f(\pos)$ to accurately capture %a geometric object's 
the inherent features (e.g., atom features for 3D molecules) that are invariant of its orientation or position in 3D space.
%
Formally, $f(\pos)$ is invariant under any translation $\mathbf{t}$ and  rotation $\mathbf{R}$ if it satisfies
%
\begin{equation}
f(\mathbf{R}\pos+\mathbf{t}) = f(\pos).
\end{equation}
%
In \method, both \SE and \methoddiff capture the inherent features of objects in an invariant way, regardless of any translation or rotation transformations, as will be detailed in the following sections.

%%%%%%%%%%%%%%%%%%%%%%%%%%%%%%%%%%%%%%%%%%%%%
\section{Point Cloud Construction}
\label{supp:point_clouds}
%%%%%%%%%%%%%%%%%%%%%%%%%%%%%%%%%%%%%%%%%%%%%

In \method, we represented molecular surface shapes using point clouds (\pc).
%
$\pc$
serves as input to \SE, from which we derive shape latent embeddings.
%
To generate $\pc$, %\bo{\st{create this}}, \bo{generate $\pc$}
we initially generated a molecular surface mesh using the algorithm from the Open Drug Discovery Toolkit~\cite{Wjcikowski2015oddt}.
%
Following this, we uniformly sampled points on the mesh surface with probability proportional to the face area, %\xia{how to uniformly?}, ensuring the sampling is done proportionally to the face area, with
using the algorithm from PyTorch3D~\cite{ravi2020pytorch3d}.
%
This point cloud $\pc$ is then centralized by setting the center of its points to zero.
%
%

%%%%%%%%%%%%%%%%%%%%%%%%%%%%%%%%%%%%%%%%%%%%%
\section{Query Point Sampling}
\label{supp:training:shapeemb}
%%%%%%%%%%%%%%%%%%%%%%%%%%%%%%%%%%%%%%%%%%%%%

As described in Section ``Shape Decoder (\SED)'', the signed distances of query points $z_q$ to molecule surface shape $\pc$ are used to optimize \SE.
%
In this section, we present how to sample these points $z_q$ in 3D space.
%
Particularly, we first determined the bounding box around the molecular surface shape, using the maximum and minimum \mbox{($x$, $y$, $z$)-axis} coordinates for points in our point cloud \pc,
denoted as $(x_\text{min}, y_\text{min}, z_\text{min})$ and $(x_\text{max}, y_\text{max}, z_\text{max})$.
%
We extended this box slightly by defining its corners as \mbox{$(x_\text{min}-1, y_\text{min}-1, z_\text{min}-1)$} and \mbox{$(x_\text{max}+1, y_\text{max}+1, z_\text{max}+1)$}.
%
For sampling $|\mathcal{Z}|$ query points, we wanted an even distribution of points inside and outside the molecule surface shape.
%
%\ziqi{Typically, within this bounding box, molecules occupy only a small portion of volume, which makes it more likely to sample
%points outside the molecule surface shape.}
%
When a bounding box is defined around the molecule surface shape, there could be a lot of empty spaces within the box that the molecule does not occupy due to 
its complex and irregular shape.
%
This could lead to that fewer points within the molecule surface shape could be sampled within the box.
%
Therefore, we started by randomly sampling $3k$ points within our bounding box to ensure that there are sufficient points within the surface.
%
We then determined whether each point lies within the molecular surface, using an algorithm from Trimesh~\footnote{https://trimsh.org/} based on the molecule surface mesh.
%
If there are $n_w$ points found within the surface, we selected $n=\min(n_w, k/2)$ points from these points, 
and randomly choose the remaining 
%\bo{what do you mean by remaining? If all the 3k sampled points are inside the surface, you get no points left.} 
$k-n$ points 
from those outside the surface.
%
For each query point, we determined its signed distance to the molecule surface by its closest distance to points in \pc with a sign indicating whether it is inside the surface.

%%%%%%%%%%%%%%%%%%%%%%%%%%%%%%%%%%%%%%%%%%%%%
\section{Forward Diffusion (\diffnoise)}
\label{supp:forward}
%%%%%%%%%%%%%%%%%%%%%%%%%%%%%%%%%%%%%%%%%%%%%

%===================================================================
\subsection{{Forward Process}}
\label{supp:forward:forward}
%===================================================================

Formally, for atom positions, the probability of $\pos_t$ sampled given $\pos_{t-1}$, denoted as $q(\pos_t|\pos_{t-1})$, is defined as follows,
%\xia{revise the representation, should be $\beta^x_t$ -- note the space} as follows,
%
\begin{equation}
q(\pos_t|\pos_{t-1}) = \mathcal{N}(\pos_t|\sqrt{1-\beta^{\mathtt{x}}_t}\pos_{t-1}, \beta^{\mathtt{x}}_t\mathbb{I}), 
\label{eqn:noiseposinter}
\end{equation}
%
%\xia{should be a comma after the equation. you also missed it. }
%\st{in which} 
where %\hl{$\pos_0$ denotes the original atom position;} \xia{no $\pos_0$ in the equation...}
%$\mathbf{I}$ denotes the identity matrix;
$\mathcal{N}(\cdot)$ is a Gaussian distribution of $\pos_t$ with mean $\sqrt{1-\beta_t^{\mathtt{x}}}\pos_{t-1}$ and covariance $\beta_t^{\mathtt{x}}\mathbf{I}$.
%\xia{what is $\mathcal{N}$? what is $q$? you abused $q$. need to be crystal clear... }
%\bo{Should be $\sim$ not $=$ in the equation}
%
Following Hoogeboom \etal~\cite{hoogeboom2021catdiff}, 
%the forward process for the discrete atom feature $\atomfeat_t\in\mathbb{R}^K$ adds 
%categorical noise into $\atomfeat_{t-1}$ according to a variance schedule $\beta_t^v\in (0, 1)$. %as follows, %\hl{$\betav_t\in (0, 1)$} as follows,
%\xia{presentation...check across the entire manuscript... } as follows,
%
%\ziqi{Formally, 
for atom features, the probability of $\atomfeat_t$ across $K$ classes given $\atomfeat_{t-1}$ is defined as follows,
%
\begin{equation}
q(\atomfeat_t|\atomfeat_{t-1}) = \mathcal{C}(\atomfeat_t|(1-\beta^{\mathtt{v}}_t) \atomfeat_{t-1}+\beta^{\mathtt{v}}_t\mathbf{1}/K),
\label{eqn:noisetypeinter}
\end{equation}
%
where %\hl{$\atomfeat_0$ denotes the original atom positions}; 
$\mathcal{C}$ is a categorical distribution of $\atomfeat_t$ derived from the %by 
noising $\atomfeat_{t-1}$ with a uniform noise $\beta^{\mathtt{v}}_t\mathbf{1}/K$ across $K$ classes.
%adding an uniform noise $\beta^v_t$ to $\atomfeat_{t-1}$ across K classes.
%\xia{there is always a comma or period after the equations. Equations are part of a sentence. you always missed it. }
%\xia{what is $\mathcal{C}$? what does $q$ mean? it is abused. }

Since the above distributions form Markov chains, %} \xia{grammar!}, 
the probability of any $\pos_t$ or $\atomfeat_t$ can be derived from $\pos_0$ or $\atomfeat_0$:
%samples $\mol_0$ as follows,
%
\begin{eqnarray}
%\begin{aligned}
& q(\pos_t|\pos_{0}) & = \mathcal{N}(\pos_t|\sqrt{\cumalpha^{\mathtt{x}}_t}\pos_0, (1-\cumalpha^{\mathtt{x}}_t)\mathbb{I}), \label{eqn:noisepos}\\
& q(\atomfeat_t|\atomfeat_0)  & = \mathcal{C}(\atomfeat_t|\cumalpha^{\mathtt{v}}_t\atomfeat_0 + (1-\cumalpha^{\mathtt{v}}_t)\mathbf{1}/K), \label{eqn:noisetype}\\
& \text{where }\cumalpha^{\mathtt{u}}_t & = \prod\nolimits_{\tau=1}^{t}\alpha^{\mathtt{u}}_\tau, \ \alpha^{\mathtt{u}}_\tau=1 - \beta^{\mathtt{u}}_\tau, \ {\mathtt{u}}={\mathtt{x}} \text{ or } {\mathtt{v}}.\;\;\;\label{eqn:noiseschedule}
%\end{aligned}
\label{eqn:pos_prior}
\end{eqnarray}
%\xia{always punctuations after equations!!! also use ``eqnarray" instead of ``equation" + ``aligned" for multiple equations, each
%with a separate reference numbering...}
%\st{in which}, 
%where \ziqi{$\cumalpha^u_t = \prod_{\tau=1}^{t}\alpha^u_\tau$ and $\alpha^u_\tau=1 - \beta^u_\tau$ ($u$=$x$ or $v$)}.
%\xia{no such notations in the above equations; also subscript $s$ is abused with shape};
%$K$ is the number of categories for atom features.
%
%The details about noise schedules $\beta^x_t$ and $\beta^v_t$ are available in Supplementary Section \ref{XXX}. \ziqi{add trend}
%
Note that $\bar{\alpha}^{\mathtt{u}}_t$ ($\mathtt{u}={\mathtt{x}}\text{ or }{\mathtt{v}}$)
%($u$=$x$ or $v$) 
is monotonically decreasing from 1 to 0 over $t=[1,T]$. %\xia{=???}. 
%
As $t\rightarrow 1$, $\cumalpha^{\mathtt{x}}_t$ and $\cumalpha^{\mathtt{v}}_t$ are close to 1, leading to that $\pos_t$ or $\atomfeat_t$ approximates 
%the original data 
$\pos_0$ or $\atomfeat_0$.
%
Conversely, as  $t\rightarrow T$, $\cumalpha^{\mathtt{x}}_t$ and $\cumalpha^{\mathtt{v}}_t$ are close to 0,
leading to that $q(\pos_T|\pos_{0})$ %\st{$\rightarrow \mathcal{N}(\mathbf{0}, \mathbf{I})$} 
resembles  {$\mathcal{N}(\mathbf{0}, \mathbb{I})$} 
and $q(\atomfeat_T|\atomfeat_0)$ %\st{$\rightarrow \mathcal{C}(\mathbf{I}/K)$} 
resembles {$\mathcal{C}(\mathbf{1}/K)$}.

Using Bayes theorem, the ground-truth Normal posterior of atom positions $p(\pos_{t-1}|\pos_t, \pos_0)$ can be calculated in a
closed form~\cite{ho2020ddpm} as below,
%
\begin{eqnarray}
& p(\pos_{t-1}|\pos_t, \pos_0) = \mathcal{N}(\pos_{t-1}|\mu(\pos_t, \pos_0), \tilde{\beta}^\mathtt{x}_t\mathbb{I}), \label{eqn:gt_pos_posterior_1}\\
&\!\!\!\!\!\!\!\!\!\!\!\mu(\pos_t, \pos_0)\!=\!\frac{\sqrt{\bar{\alpha}^{\mathtt{x}}_{t-1}}\beta^{\mathtt{x}}_t}{1-\bar{\alpha}^{\mathtt{x}}_t}\pos_0\!+\!\frac{\sqrt{\alpha^{\mathtt{x}}_t}(1-\bar{\alpha}^{\mathtt{x}}_{t-1})}{1-\bar{\alpha}^{\mathtt{x}}_t}\pos_t, 
\tilde{\beta}^\mathtt{x}_t\!=\!\frac{1-\bar{\alpha}^{\mathtt{x}}_{t-1}}{1-\bar{\alpha}^{\mathtt{x}}_{t}}\beta^{\mathtt{x}}_t.\;\;\;
\end{eqnarray}
%
%\xia{Ziqi, please double check the above two equations!}
Similarly, the ground-truth categorical posterior of atom features $p(\atomfeat_{t-1}|\atomfeat_{t}, \atomfeat_0)$ can be calculated~\cite{hoogeboom2021catdiff} as below,
%
\begin{eqnarray}
& p(\atomfeat_{t-1}|\atomfeat_{t}, \atomfeat_0) = \mathcal{C}(\atomfeat_{t-1}|\mathbf{c}(\atomfeat_t, \atomfeat_0)), \label{eqn:gt_atomfeat_posterior_1}\\
& \mathbf{c}(\atomfeat_t, \atomfeat_0) = \tilde{\mathbf{c}}/{\sum_{k=1}^K \tilde{c}_k}, \label{eqn:gt_atomfeat_posterior_2} \\
& \tilde{\mathbf{c}} = [\alpha^{\mathtt{v}}_t\atomfeat_t + \frac{1 - \alpha^{\mathtt{v}}_t}{K}]\odot[\bar{\alpha}^{\mathtt{v}}_{t-1}\atomfeat_{0}+\frac{1-\bar{\alpha}^{\mathtt{v}}_{t-1}}{K}], 
\label{eqn:gt_atomfeat_posterior_3}
%\label{eqn:atomfeat_posterior}
\end{eqnarray}
%
%\xia{Ziqi: please double check the above equations!}
%
where $\tilde{c}_k$ denotes the likelihood of $k$-th class across $K$ classes in $\tilde{\mathbf{c}}$; 
$\odot$ denotes the element-wise product operation;
$\tilde{\mathbf{c}}$ is calculated using $\atomfeat_t$ and $\atomfeat_{0}$ and normalized into $\mathbf{c}(\atomfeat_t, \atomfeat_0)$ so as to represent
probabilities. %\xia{is this correct? is $\tilde{c}_k$ always greater than 0?}
%\xia{how is it calculated?}.
%\ziqi{the likelihood distribution $\tilde{c}$ is calculated by $p(\atomfeat_t|\atomfeat_{t-1})p(\atomfeat_{t-1}|\atomfeat_0)$, according to 
%Equation~\ref{eqn:noisetypeinter} and \ref{eqn:noisetype}.
%\xia{need to write the key idea of the above calculation...}
%
The proof of the above equations is available in Supplementary Section~\ref{supp:forward:proof}.

%===================================================================
\subsection{Variance Scheduling in \diffnoise}
\label{supp:forward:variance}
%===================================================================

Following Guan \etal~\cite{guan2023targetdiff}, we used a sigmoid $\beta$ schedule for the variance schedule $\beta_t^{\mathtt{x}}$ of atom coordinates as below,

\begin{equation}
\beta_t^{\mathtt{x}} = \text{sigmoid}(w_1(2 t / T - 1)) (w_2 - w_3) + w_3
\end{equation}
in which $w_i$($i$=1,2, or 3) are hyperparameters; $T$ is the maximum step.
%
We set $w_1=6$, $w_2=1.e-7$ and $w_3=0.01$.
%
For atom types, we used a cosine $\beta$ schedule~\cite{nichol2021} for $\beta_t^{\mathtt{v}}$ as below,

\begin{equation}
\begin{aligned}
& \bar{\alpha}_t^{\mathtt{v}} = \frac{f(t)}{f(0)}, f(t) = \cos(\frac{t/T+s}{1+s} \cdot \frac{\pi}{2})^2\\
& \beta_t^{\mathtt{v}} = 1 - \alpha_t^{\mathtt{v}} = 1 - \frac{\bar{\alpha}_t^{\mathtt{v}} }{\bar{\alpha}_{t-1}^{\mathtt{v}} }
\end{aligned}
\end{equation}
in which $s$ is a hyperparameter and set as 0.01.

As shown in Section ``Forward Diffusion Process'', the values of $\beta_t^{\mathtt{x}}$ and $\beta_t^{\mathtt{v}}$ should be 
sufficiently small to ensure the smoothness of forward diffusion process. In the meanwhile, their corresponding $\bar{\alpha}_t$
values should decrease from 1 to 0 over $t=[1,T]$.
%
Figure~\ref{fig:schedule} shows the values of $\beta_t$ and $\bar{\alpha}_t$ for atom coordinates and atom types with our hyperparameters.
%
Please note that the value of $\beta_{t}^{\mathtt{x}}$ is less than 0.1 for 990 out of 1,000 steps. %\bo{\st{, though it increases when $t$ is close to 1,000}}.
%
This guarantees the smoothness of the forward diffusion process.
%\bo{add $\beta_t^{\mathtt{x}}$ and $\beta_t^{\mathtt{v}}$ in the legend of the figure...}
%\bo{$\beta_t^{\mathtt{v}}$ does not look small when $t$ is close to 1000...}

\begin{figure}
	\begin{subfigure}[t]{.45\linewidth}
		\centering
		\includegraphics[width=.7\linewidth]{figures/var_schedule_beta.pdf}
	\end{subfigure}
	%
	\hfill
	\begin{subfigure}[t]{.45\linewidth}
		\centering
		\includegraphics[width=.7\linewidth]{figures/var_schedule_alpha.pdf}
	\end{subfigure}
	\caption{Schedule}
	\label{fig:schedule}
\end{figure}

%===================================================================
\subsection{Derivation of Forward Diffusion Process}
\label{supp:forward:proof}
%===================================================================

In \method, a Gaussian noise and a categorical noise are added to continuous atom position and discrete atom features, respectively.
%
Here, we briefly describe the derivation of posterior equations (i.e., Eq.~\ref{eqn:gt_pos_posterior_1}, and   \ref{eqn:gt_atomfeat_posterior_1}) for atom positions and atom types in our work.
%
We refer readers to Ho \etal~\cite{ho2020ddpm} and Kong \etal~\cite{kong2021diffwave} %\bo{add XXX~\etal here...} \cite{ho2020ddpm,kong2021diffwave} 	
for a detailed description of diffusion process for continuous variables and Hoogeboom \etal~\cite{hoogeboom2021catdiff} for
%\bo{add XXX~\etal here...} \cite{hoogeboom2021catdiff} for
the description of diffusion process for discrete variables.

For continuous atom positions, as shown in Kong \etal~\cite{kong2021diffwave}, according to Bayes theorem, given $q(\pos_t|\pos_{t-1})$ defined in Eq.~\ref{eqn:noiseposinter} and 
$q(\pos_t|\pos_0)$ defined in Eq.~\ref{eqn:noisepos}, the posterior $q(\pos_{t-1}|\pos_{t}, \pos_0)$ is derived as below (superscript $\mathtt{x}$ is omitted for brevity),

\begin{equation}
\begin{aligned}
& q(\pos_{t-1}|\pos_{t}, \pos_0)  = \frac{q(\pos_t|\pos_{t-1}, \pos_0)q(\pos_{t-1}|\pos_0)}{q(\pos_t|\pos_0)} \\
& =  \frac{\mathcal{N}(\pos_t|\sqrt{1-\beta_t}\pos_{t-1}, \beta_{t}\mathbf{I}) \mathcal{N}(\pos_{t-1}|\sqrt{\bar{\alpha}_{t-1}}\pos_{0}, (1-\bar{\alpha}_{t-1})\mathbf{I}) }{ \mathcal{N}(\pos_{t}|\sqrt{\bar{\alpha}_t}\pos_{0}, (1-\bar{\alpha}_t)\mathbf{I})}\\
& =  (2\pi{\beta_t})^{-\frac{3}{2}} (2\pi{(1-\bar{\alpha}_{t-1})})^{-\frac{3}{2}} (2\pi(1-\bar{\alpha}_t))^{\frac{3}{2}} \times \exp( \\
& -\frac{\|\pos_t - \sqrt{\alpha}_t\pos_{t-1}\|^2}{2\beta_t} -\frac{\|\pos_{t-1} - \sqrt{\bar{\alpha}}_{t-1}\pos_{0} \|^2}{2(1-\bar{\alpha}_{t-1})} \\
& + \frac{\|\pos_t - \sqrt{\bar{\alpha}_t}\pos_0\|^2}{2(1-\bar{\alpha}_t)}) \\
& = (2\pi\tilde{\beta}_t)^{-\frac{3}{2}} \exp(-\frac{1}{2\tilde{\beta}_t}\|\pos_{t-1}-\frac{\sqrt{\bar{\alpha}_{t-1}}\beta_t}{1-\bar{\alpha}_t}\pos_0 \\
& - \frac{\sqrt{\alpha_t}(1-\bar{\alpha}_{t-1})}{1-\bar{\alpha}_t}\pos_{t}\|^2) \\
& \text{where}\ \tilde{\beta}_t = \frac{1-\bar{\alpha}_{t-1}}{1-\bar{\alpha}_t}\beta_t.
\end{aligned}
\end{equation}
%\bo{marked part does not look right to me.}
%\bo{How to you derive from the second equation to the third one?}

Therefore, the posterior of atom positions is derived as below,

\begin{equation}
q(\pos_{t-1}|\pos_{t}, \pos_0)\!\!=\!\!\mathcal{N}(\pos_{t-1}|\frac{\sqrt{\bar{\alpha}_{t-1}}\beta_t}{1-\bar{\alpha}_t}\pos_0 + \frac{\sqrt{\alpha_t}(1-\bar{\alpha}_{t-1})}{1-\bar{\alpha}_t}\pos_{t}, \tilde{\beta}_t\mathbf{I}).
\end{equation}

For discrete atom features, as shown in Hoogeboom \etal~\cite{hoogeboom2021catdiff} and Guan \etal~\cite{guan2023targetdiff},
according to Bayes theorem, the posterior $q(\atomfeat_{t-1}|\atomfeat_{t}, \atomfeat_0)$ is derived as below (supperscript $\mathtt{v}$ is omitted for brevity),

\begin{equation}
\begin{aligned}
& q(\atomfeat_{t-1}|\atomfeat_{t}, \atomfeat_0) =  \frac{q(\atomfeat_t|\atomfeat_{t-1}, \atomfeat_0)q(\atomfeat_{t-1}|\atomfeat_0)}{\sum_{\scriptsize{\atomfeat}_{t-1}}q(\atomfeat_t|\atomfeat_{t-1}, \atomfeat_0)q(\atomfeat_{t-1}|\atomfeat_0)} \\
%& = \frac{\mathcal{C}(\atomfeat_t|(1-\beta_t)\atomfeat_{t-1} + \beta_t\frac{\mathbf{1}}{K}) \mathcal{C}(\atomfeat_{t-1}|\bar{\alpha}_{t-1}\atomfeat_0+(1-\bar{\alpha}_{t-1})\frac{\mathbf{1}}{K})} \\
\end{aligned}
\end{equation}

For $q(\atomfeat_t|\atomfeat_{t-1}, \atomfeat_0)$, we have % $\atomfeat_t=\atomfeat_{t-1}$ with probability $1-\beta_t+\beta_t / K$, and $\atomfeat_t \neq \atomfeat_{t-1}$
%with probability $\beta_t / K$.
%
%Therefore, this function can be symmetric, that is, 
%
\begin{equation}
\begin{aligned}
q(\atomfeat_t|\atomfeat_{t-1}, \atomfeat_0) & = \mathcal{C}(\atomfeat_t|(1-\beta_t)\atomfeat_{t-1} + \beta_t/{K})\\
& = \begin{cases}
1-\beta_t+\beta_t/K,\!&\text{when}\ \atomfeat_{t} = \atomfeat_{t-1},\\
\beta_t / K,\! &\text{when}\ \atomfeat_{t} \neq \atomfeat_{t-1},
\end{cases}\\
& = \mathcal{C}(\atomfeat_{t-1}|(1-\beta_t)\atomfeat_{t} + \beta_t/{K}).
\end{aligned}
%\mathcal{C}(\atomfeat_{t-1}|(1-\beta_{t})\atomfeat_{t} + \beta_t\frac{\mathbf{1}}{K}).
\end{equation}
%
Therefore, we have
%\bo{why it can be symmetric}
%
\begin{equation}
\begin{aligned}
& q(\atomfeat_t|\atomfeat_{t-1}, \atomfeat_0)q(\atomfeat_{t-1}|\atomfeat_0) \\
& = \mathcal{C}(\atomfeat_{t-1}|(1-\beta_t)\atomfeat_{t} + \beta_t\frac{\mathbf{1}}{K}) \mathcal{C}(\atomfeat_{t-1}|\bar{\alpha}_{t-1}\atomfeat_0+(1-\bar{\alpha}_{t-1})\frac{\mathbf{1}}{K}) \\
& = [\alpha_t\atomfeat_t + \frac{1 - \alpha_t}{K}]\odot[\bar{\alpha}_{t-1}\atomfeat_{0}+\frac{1-\bar{\alpha}_{t-1}}{K}].
\end{aligned}
\end{equation}
%
%\bo{what is $\tilde{\mathbf{c}}$...}
Therefore, with $q(\atomfeat_t|\atomfeat_{t-1}, \atomfeat_0)q(\atomfeat_{t-1}|\atomfeat_0) = \tilde{\mathbf{c}}$, the posterior is as below,

\begin{equation}
q(\atomfeat_{t-1}|\atomfeat_{t}, \atomfeat_0) = \mathcal{C}(\atomfeat_{t-1}|\mathbf{c}(\atomfeat_t, \atomfeat_0)) = \frac{\tilde{\mathbf{c}}}{\sum_{k}^K\tilde{c}_k}.
\end{equation}

%%%%%%%%%%%%%%%%%%%%%%%%%%%%%%%%%%%%%%%%%%%%%
\section{{Backward Generative Process} (\diffgenerative)}
\label{supp:backward}
%%%%%%%%%%%%%%%%%%%%%%%%%%%%%%%%%%%%%%%%%%%%%

Following Ho \etal~\cite{ho2020ddpm}, with $\tilde{\pos}_{0,t}$, the probability of $\pos_{t-1}$ denoised from $\pos_t$, denoted as $p(\pos_{t-1}|\pos_t)$,
can be estimated %\hl{parameterized} \xia{???} 
by the approximated posterior $p_{\boldsymbol{\Theta}}(\pos_{t-1}|\pos_t, \tilde{\pos}_{0,t})$ as below,
%
\begin{equation}
\begin{aligned}
p(\pos_{t-1}|\pos_t) & \approx p_{\boldsymbol{\Theta}}(\pos_{t-1}|\pos_t, \tilde{\pos}_{0,t}) \\
& = \mathcal{N}(\pos_{t-1}|\mu_{\boldsymbol{\Theta}}(\pos_t, \tilde{\pos}_{0,t}),\tilde{\beta}_t^{\mathtt{x}}\mathbb{I}),
\end{aligned}
\label{eqn:aprox_pos_posterior}
\end{equation}
%
where ${\boldsymbol{\Theta}}$ is the learnable parameter; $\mu_{\boldsymbol{\Theta}}(\pos_t, \tilde{\pos}_{0,t})$ is an estimate %estimation
of $\mu(\pos_t, \pos_{0})$ by replacing $\pos_0$ with its estimate $\tilde{\pos}_{0,t}$ 
in Equation~{\ref{eqn:gt_pos_posterior_1}}.
%
Similarly, with $\tilde{\atomfeat}_{0,t}$, the probability of $\atomfeat_{t-1}$ denoised from $\atomfeat_t$, denoted as $p(\atomfeat_{t-1}|\atomfeat_t)$, 
can be estimated %\hl{parameterized} 
by the approximated posterior $p_{\boldsymbol{\Theta}}(\atomfeat_{t-1}|\atomfeat_t, \tilde{\atomfeat}_{0,t})$ as below,
%
\begin{equation}
\begin{aligned}
p(\atomfeat_{t-1}|\atomfeat_t)\approx p_{\boldsymbol{\Theta}}(\atomfeat_{t-1}|\atomfeat_{t}, \tilde{\atomfeat}_{0,t}) 
=\mathcal{C}(\atomfeat_{t-1}|\mathbf{c}_{\boldsymbol{\Theta}}(\atomfeat_t, \tilde{\atomfeat}_{0,t})),\!\!\!\!
\end{aligned}
\label{eqn:aprox_atomfeat_posterior}
\end{equation}
%
where $\mathbf{c}_{\boldsymbol{\Theta}}(\atomfeat_t, \tilde{\atomfeat}_{0,t})$ is an estimate of $\mathbf{c}(\atomfeat_t, \atomfeat_0)$
by replacing $\atomfeat_0$  
with its estimate $\tilde{\atomfeat}_{0,t}$ in Equation~\ref{eqn:gt_atomfeat_posterior_1}.



%===================================================================
\section{\method Loss Function Derivation}
\label{supp:training:loss}
%===================================================================

In this section, we demonstrate that a step weight $w_t^{\mathtt{x}}$ based on the signal-to-noise ratio $\lambda_t$ should be 
included into the atom position loss (Eq.~\ref{eqn:diff:obj:pos}).
%
In the diffusion process for continuous variables, the optimization problem is defined 
as below~\cite{ho2020ddpm},
%
\begin{equation*}
\begin{aligned}
& \arg\min_{\boldsymbol{\Theta}}KL(q(\pos_{t-1}|\pos_t, \pos_0)|p_{\boldsymbol{\Theta}}(\pos_{t-1}|\pos_t, \tilde{\pos}_{0,t})) \\
& = \arg\min_{\boldsymbol{\Theta}} \frac{\bar{\alpha}_{t-1}(1-\alpha_t)}{2(1-\bar{\alpha}_{t-1})(1-\bar{\alpha}_{t})}\|\tilde{\pos}_{0, t}-\pos_0\|^2 \\
& = \arg\min_{\boldsymbol{\Theta}} \frac{1-\alpha_t}{2(1-\bar{\alpha}_{t-1})\alpha_{t}} \|\tilde{\boldsymbol{\epsilon}}_{0,t}-\boldsymbol{\epsilon}_0\|^2,
\end{aligned}
\end{equation*}
where $\boldsymbol{\epsilon}_0 = \frac{\pos_t - \sqrt{\bar{\alpha}_t}\pos_0}{\sqrt{1-\bar{\alpha}_t}}$ is the ground-truth noise variable sampled from $\mathcal{N}(\mathbf{0}, \mathbf{1})$ and is used to sample $\pos_t$ from $\mathcal{N}(\pos_t|\sqrt{\cumalpha_t}\pos_0, (1-\cumalpha_t)\mathbf{I})$ in Eq.~\ref{eqn:noisetype};
$\tilde{\boldsymbol{\epsilon}}_0 = \frac{\pos_t - \sqrt{\bar{\alpha}_t}\tilde{\pos}_{0, t}}{\sqrt{1-\bar{\alpha}_t}}$ is the predicted noise variable. 

%A simplified training objective is proposed by Ho \etal~\cite{ho2020ddpm} as below,
Ho \etal~\cite{ho2020ddpm} further simplified the above objective as below and
demonstrated that the simplified one can achieve better performance:
%
\begin{equation}
\begin{aligned}
& \arg\min_{\boldsymbol{\Theta}} \|\tilde{\boldsymbol{\epsilon}}_{0,t}-\boldsymbol{\epsilon}_0\|^2 \\
& = \arg\min_{\boldsymbol{\Theta}} \frac{\bar{\alpha}_t}{1-\bar{\alpha}_t}\|\tilde{\pos}_{0,t}-\pos_0\|^2,
\end{aligned}
\label{eqn:supp:losspos}
\end{equation}
%
where $\lambda_t=\frac{\bar{\alpha}_t}{1-\bar{\alpha}_t}$ is the signal-to-noise ratio.
%
While previous work~\cite{guan2023targetdiff} applies uniform step weights across
different steps, we demonstrate that a step weight should be included into the atom position loss according to Eq.~\ref{eqn:supp:losspos}.
%
However, the value of $\lambda_t$ could be very large when $\bar{\alpha}_t$ is close to 1 as $t$ approaches 1.
%
Therefore, we clip the value of $\lambda_t$ with threshold $\delta$ in Eq.~\ref{eqn:diff:obj:pos}.










\end{document}






















\newpage

\renewcommand\thealgorithm{S.\arabic{algorithm}}
\renewcommand\thetable{S.\arabic{table}}
\renewcommand\thefigure{S.\arabic{figure}}
\renewcommand\thesection{S.\roman{section}}
\renewcommand\theequation{S.\arabic{equation}}

\title{Supplementary Document for ``MetaDE: Evolving Differential Evolution by Differential Evolution"}
%
%
\author{Minyang Chen, Chenchen Feng,
        and Ran Cheng
        \thanks{
        Minyang Chen was with the Department of Computer Science and Engineering, Southern University of Science and Technology, Shenzhen 518055, China. E-mail: cmy1223605455@gmail.com. }
        \thanks{
        Chenchen Feng is with the Department of Computer Science and Engineering, Southern University of Science and Technology, Shenzhen 518055, China. E-mail: chenchenfengcn@gmail.com. 
        }
        \thanks{
       Ran Cheng is with the Department of Data Science and Artificial Intelligence, and the Department of Computing, The Hong Kong Polytechnic University, Hong Kong SAR, China. E-mail: ranchengcn@gmail.com. (\emph{Corresponding author: Ran Cheng})
        }
        }

\onecolumn{}

% The paper headers
\markboth{Journal of \LaTeX\ Class Files,~Vol.~0, No.~0, 0~0}%
{Shell \MakeLowercase{\textit{et al.}}: Bare Demo of IEEEtran.cls for IEEE Journals}
% The only time the second header will appear is for the odd numbered pages
% after the title page when using the twoside option.

% *** Note that you probably will NOT want to include the author's ***
% *** name in the headers of peer review papers.                   ***
% You can use \ifCLASSOPTIONpeerreview for conditional compilation here if
% you desire.


% If you want to put a publisher's ID mark on the page you can do it like
% this:
%\IEEEpubid{0000--0000/00\$00.00~\copyright~2015 IEEE}
% Remember, if you use this you must call \IEEEpubidadjcol in the second
% column for its text to clear the IEEEpubid mark.

 

% use for special paper notices
%\IEEEspecialpapernotice{(Invited Paper)}




% make the title area
\maketitle

% As a general rule, do not put math, special symbols or citations
% in the abstract or keywords.

% Note that keywords are not normally used for peerreview papers.

% For peer review papers, you can put extra information on the cover
% page as needed:
% \ifCLASSOPTIONpeerreview
% \begin{center} \bfseries EDICS Category: 3-BBND \end{center}
% \fi
%
% For peerreview papers, this IEEEtran command inserts a page break and
% creates the second title. It will be ignored for other modes.
\IEEEpeerreviewmaketitle



% \clearpage
% \begin{titlepage}
% \centering
% {\LARGE Supplementary Document for ``MetaDE: Evolving Differential Evolution by Differential Evolution"}\\[1.5cm]
% {\large Minyang Chen, Chenchen Feng, and Ran Cheng}\\[1cm]
% {\small
% Minyang Chen was with the Department of Computer Science and Engineering, Southern University of Science and Technology, Shenzhen 518055, China. E-mail: cmy1223605455@gmail.com.\\[0.2cm]
% Chenchen Feng is with the Department of Computer Science and Engineering, Southern University of Science and Technology, Shenzhen 518055, China. E-mail: chenchenfengcn@gmail.com.\\[0.2cm]
% Ran Cheng is with the Department of Data Science and Artificial Intelligence, and the Department of Computing, The Hong Kong Polytechnic University, Hong Kong SAR, China. E-mail: ranchengcn@gmail.com. (Corresponding author: Ran Cheng)
% }\\[2cm]
% \end{titlepage}


\clearpage
\begin{center}
  {\Huge Supplementary Document for ``MetaDE: Evolving \\[0.3em]
  Differential Evolution by Differential Evolution"}\\[2em]
  {\Large Minyang Chen, Chenchen Feng, and Ran Cheng}\\[6em]
\end{center}







\section{Supplementary Experimental data}\label{section:FEs}

\subsection{Supplementary Figures}\label{section:FEs}

\begin{figure*}[htpb]
\centering
\includegraphics[scale=0.3]{su_D10_all.pdf}
\caption{Convergence curves on 10D problems in CEC2022 benchmark suite. The peer DE variants are set with population size of 100.}
\label{Figure_convergence_10D_supp}
\end{figure*}

\vfill 
{\small
\noindent

Minyang Chen was with the Department of Computer Science and Engineering, Southern University of Science and Technology, Shenzhen 518055, China. E-mail: cmy1223605455@gmail.com.

Chenchen Feng is with the Department of Computer Science and Engineering, Southern University of Science and Technology, Shenzhen 518055, China. E-mail: chenchenfengcn@gmail.com.

Ran Cheng is with the Department of Data Science and Artificial Intelligence, and the Department of Computing, The Hong Kong Polytechnic University, Hong Kong SAR, China. E-mail: ranchengcn@gmail.com. \textit{(Corresponding author: Ran Cheng)}
}


\begin{figure*}[htpb]
\centering
\includegraphics[scale=0.3]{su_D20_all.pdf}
\caption{Convergence curves on 20D problems in CEC2022 benchmark suite. The peer DE variants are set with population size of 100.}
\label{Figure_convergence_20D_supp}
\end{figure*}


\begin{figure*}[htpb]
\centering
\includegraphics[scale=0.3]{su_D10NP1000_all.pdf}
\caption{Convergence curves on 10D problems in CEC2022 benchmark suite. The peer DE variants are set with population size of 1,000.}
\label{Figure_convergence_10D_NP10k_supp}
\end{figure*}


\begin{figure*}[htpb]
\centering
\includegraphics[scale=0.3]{su_D20NP1000_all.pdf}
\caption{Convergence curves on 20D problems in CEC2022 benchmark suite. The peer DE variants are set with population size of 1,000.}
\label{Figure_convergence_20D_NP10k_supp}
\end{figure*}

\clearpage

\subsection{Detailed Experimental Results}\label{section:FEs_supp}

% Tables \ref{tab:vsClass10D} and \ref{tab:vsClass20D} present the detailed results of MetaDE compared to other algorithms on 10-dimensional and 20-dimensional problems from CEC2022 within a 60-second time frame. The convergence curves for all problems are shown in Figs. \ref{Figure_convergence_10D} and \ref{Figure_convergence_20D}.

% Tables \ref{tab:NP10000 10D} and \ref{tab:NP10000 20D} display the detailed results for MetaDE versus comparison algorithms with equal concurrency (population size = 10,000) on 10-dimensional and 20-dimensional problems in CEC2022, all within a span of 60 seconds. The convergence curves for all problems are shown in Figs. \ref{Figure_convergence_10D_NP10k} and \ref{Figure_convergence_20D_NP10k}.



% Table generated by Excel2LaTeX from sheet 'Sheet1'
\begin{table}[htbp]
  \centering
  \caption{
  Detailed results on 10D problems in CEC2022 benchmark suite. The peer DE variants are set with population size of 100.
The mean and standard deviation (in parentheses) of the results over multiple runs are displayed in pairs. 
Results with the best mean values are highlighted.
  }
  \resizebox{\textwidth}{!}{
   \renewcommand{\arraystretch}{1.2}
% Table generated by Excel2LaTeX from sheet 'Experiment1 60S'
\begin{tabular}{ccccccccc}
\toprule
Func  & MetaDE & DE    & SaDE  & JaDE  & CoDE  & SHADE & LSHADE-RSP & EDEV \\
\midrule
$F_{1}$ & \textbf{0.00E+00 (0.00E+00)} & \boldmath{}\textbf{0.00E+00 (0.00E+00)$\approx$}\unboldmath{} & \boldmath{}\textbf{0.00E+00 (0.00E+00)$\approx$}\unboldmath{} & \boldmath{}\textbf{0.00E+00 (0.00E+00)$\approx$}\unboldmath{} & \boldmath{}\textbf{0.00E+00 (0.00E+00)$\approx$}\unboldmath{} & \boldmath{}\textbf{0.00E+00 (0.00E+00)$\approx$}\unboldmath{} & \boldmath{}\textbf{0.00E+00 (0.00E+00)$\approx$}\unboldmath{} & \boldmath{}\textbf{0.00E+00 (0.00E+00)$\approx$}\unboldmath{} \\
$F_{2}$ & \textbf{0.00E+00 (0.00E+00)} & 6.05E+00 (2.43E+00)$-$ & 4.86E+00 (4.29E+00)$-$ & 4.89E+00 (3.74E+00)$-$ & 4.71E+00 (2.55E+00)$-$ & 5.47E+00 (3.62E+00)$-$ & 2.38E+00 (2.58E+00)$-$ & 6.11E+00 (2.87E+00)$-$ \\
$F_{3}$ & \textbf{0.00E+00 (0.00E+00)} & \boldmath{}\textbf{0.00E+00 (0.00E+00)$\approx$}\unboldmath{} & \boldmath{}\textbf{0.00E+00 (0.00E+00)$\approx$}\unboldmath{} & \boldmath{}\textbf{0.00E+00 (0.00E+00)$\approx$}\unboldmath{} & \boldmath{}\textbf{0.00E+00 (0.00E+00)$\approx$}\unboldmath{} & \boldmath{}\textbf{0.00E+00 (0.00E+00)$\approx$}\unboldmath{} & \boldmath{}\textbf{0.00E+00 (0.00E+00)$\approx$}\unboldmath{} & \boldmath{}\textbf{0.00E+00 (0.00E+00)$\approx$}\unboldmath{} \\
$F_{4}$ & \textbf{0.00E+00 (0.00E+00)} & 6.90E+00 (4.01E+00)$-$ & 1.03E+00 (8.93E-01)$-$ & 2.31E+01 (1.19E+01)$-$ & 8.34E-01 (7.62E-01)$-$ & 3.05E+00 (9.43E-01)$-$ & 2.12E+00 (6.56E-01)$-$ & 6.52E+00 (4.51E+00)$-$ \\
$F_{5}$ & \textbf{0.00E+00 (0.00E+00)} & \boldmath{}\textbf{0.00E+00 (0.00E+00)$\approx$}\unboldmath{} & \boldmath{}\textbf{0.00E+00 (0.00E+00)$\approx$}\unboldmath{} & \boldmath{}\textbf{0.00E+00 (0.00E+00)$\approx$}\unboldmath{} & \boldmath{}\textbf{0.00E+00 (0.00E+00)$\approx$}\unboldmath{} & \boldmath{}\textbf{0.00E+00 (0.00E+00)$\approx$}\unboldmath{} & \boldmath{}\textbf{0.00E+00 (0.00E+00)$\approx$}\unboldmath{} & \boldmath{}\textbf{0.00E+00 (0.00E+00)$\approx$}\unboldmath{} \\
$F_{6}$ & \textbf{5.50E-04 (3.96E-04)} & 1.11E-01 (8.92E-02)$-$ & 6.04E+01 (2.25E+02)$-$ & 1.60E+00 (2.48E+00)$-$ & 9.08E-03 (1.17E-02)$-$ & 1.33E+00 (2.22E+00)$-$ & 3.84E-02 (5.60E-02)$-$ & 9.06E-01 (1.76E+00)$-$ \\
$F_{7}$ & \textbf{0.00E+00 (0.00E+00)} & 5.18E-02 (1.59E-01)$-$ & 2.15E-02 (1.39E-02)$-$ & \boldmath{}\textbf{0.00E+00 (0.00E+00)$\approx$}\unboldmath{} & 1.92E-03 (7.32E-03)$-$ & 6.59E-03 (1.41E-02)$-$ & 1.06E+02 (1.42E-02)$-$ & 9.54E+00 (9.86E+00)$-$ \\
$F_{8}$ & \textbf{5.52E-03 (4.41E-03)} & 1.42E-01 (2.45E-01)$-$ & 5.15E-02 (2.29E-02)$-$ & 1.74E+01 (4.79E+00)$-$ & 6.02E-03 (1.02E-02)$-$ & 2.29E+00 (5.89E+00)$-$ & 2.19E+00 (5.89E+00)$-$ & 7.06E+00 (9.36E+00)$-$ \\
$F_{9}$ & \textbf{3.36E+00 (1.77E+01)} & 2.29E+02 (7.53E-06)$-$ & 2.29E+02 (6.38E-06)$-$ & 2.29E+02 (6.38E-06)$-$ & 2.29E+02 (7.17E-06)$-$ & 2.29E+02 (7.43E-06)$-$ & 2.29E+02 (8.19E-05)$-$ & 2.29E+02 (1.01E-05)$-$ \\
$F_{10}$ & \textbf{0.00E+00 (0.00E+00)} & 1.00E+02 (5.18E-02)$-$ & 1.03E+02 (1.83E+01)$-$ & 1.04E+02 (1.91E+01)$-$ & 1.00E+02 (6.83E-02)$-$ & 1.10E+02 (3.09E+01)$-$ & 1.03E+02 (1.77E+01)$-$ & 1.07E+02 (2.62E+01)$-$ \\
$F_{11}$ & \textbf{0.00E+00 (0.00E+00)} & \boldmath{}\textbf{0.00E+00 (0.00E+00)$\approx$}\unboldmath{} & 2.42E+01 (5.52E+01)$-$ & \boldmath{}\textbf{0.00E+00 (0.00E+00)$\approx$}\unboldmath{} & \boldmath{}\textbf{0.00E+00 (0.00E+00)$\approx$}\unboldmath{} & \boldmath{}\textbf{0.00E+00 (0.00E+00)$\approx$}\unboldmath{} & \boldmath{}\textbf{0.00E+00 (0.00E+00)$\approx$}\unboldmath{} & 4.84E+00 (2.65E+01)$-$ \\
$F_{12}$ & \textbf{1.39E+02 (4.63E+01)} & 1.62E+02 (1.04E+00)$-$ & 1.63E+02 (1.57E+00)$-$ & 1.62E+02 (2.22E+00)$-$ & 1.59E+02 (1.14E+00)$-$ & 1.63E+02 (1.25E+00)$-$ & 1.64E+02 (1.38E+00)$-$ & 1.62E+02 (1.71E+00)$-$ \\
\midrule
$+$ / $\approx$ / $-$ & --    & 0/4/8 & 0/3/9 & 0/5/7 & 0/5/7 & 0/4/8 & 0/4/8 & 0/3/9 \\
\bottomrule
\end{tabular}%
}
\footnotesize
\textsuperscript{*} The Wilcoxon rank-sum tests (with a significance level of 0.05) were conducted between MetaDE and each individually.
The final row displays the number of problems where the corresponding algorithm performs statistically better ($+$),  similar ($\thickapprox$), or worse ($-$) compared to MetaDE.\\
\label{tab:vsClass10D_supp}%

\end{table}%


% Table generated by Excel2LaTeX from sheet 'Sheet1'
\begin{table}[htbp]
  \centering
  \caption{Detailed results on 20D problems in CEC2022 benchmark suite. The peer DE variants are set with population size of 100.
  The mean and standard deviation (in parentheses) of the results over multiple runs are displayed in pairs. 
Results with the best mean values are highlighted.
  }
  {
  \resizebox{\textwidth}{!}{
   \renewcommand{\arraystretch}{1.2}
% Table generated by Excel2LaTeX from sheet 'Experiment1 60S'
\begin{tabular}{ccccccccc}
\toprule
Func  & MetaDE & DE    & SaDE  & JaDE  & CoDE  & SHADE & LSHADE-RSP & EDEV \\
\midrule
$F_{1}$ & \multicolumn{1}{l}{\textbf{0.00E+00 (0.00E+00)}} & \multicolumn{1}{l}{\boldmath{}\textbf{0.00E+00 (0.00E+00)$\approx$}\unboldmath{}} & \multicolumn{1}{l}{\boldmath{}\textbf{0.00E+00 (0.00E+00)$\approx$}\unboldmath{}} & \multicolumn{1}{l}{\boldmath{}\textbf{0.00E+00 (0.00E+00)$\approx$}\unboldmath{}} & \multicolumn{1}{l}{\boldmath{}\textbf{0.00E+00 (0.00E+00)$\approx$}\unboldmath{}} & \multicolumn{1}{l}{\boldmath{}\textbf{0.00E+00 (0.00E+00)$\approx$}\unboldmath{}} & \multicolumn{1}{l}{\boldmath{}\textbf{0.00E+00 (0.00E+00)$\approx$}\unboldmath{}} & \multicolumn{1}{l}{\boldmath{}\textbf{0.00E+00 (0.00E+00)$\approx$}\unboldmath{}} \\
$F_{2}$ & \multicolumn{1}{l}{\textbf{1.26E-02 (3.74E-02)}} & \multicolumn{1}{l}{4.69E+01 (2.09E+00)$-$} & \multicolumn{1}{l}{3.50E+01 (2.21E+01)$-$} & \multicolumn{1}{l}{4.75E+01 (8.67E+00)$-$} & \multicolumn{1}{l}{4.58E+01 (1.20E+01)$-$} & \multicolumn{1}{l}{4.75E+01 (8.67E+00)$-$} & \multicolumn{1}{l}{4.30E+01 (1.71E+01)$-$} & \multicolumn{1}{l}{4.13E+01 (1.78E+01)$-$} \\
$F_{3}$ & \multicolumn{1}{l}{\textbf{0.00E+00 (0.00E+00)}} & \multicolumn{1}{l}{\boldmath{}\textbf{0.00E+00 (0.00E+00)$\approx$}\unboldmath{}} & \multicolumn{1}{l}{\boldmath{}\textbf{0.00E+00 (0.00E+00)$\approx$}\unboldmath{}} & \multicolumn{1}{l}{\boldmath{}\textbf{0.00E+00 (0.00E+00)$\approx$}\unboldmath{}} & \multicolumn{1}{l}{\boldmath{}\textbf{0.00E+00 (0.00E+00)$\approx$}\unboldmath{}} & \multicolumn{1}{l}{1.03E-08 (5.64E-08)$\approx$} & \multicolumn{1}{l}{\boldmath{}\textbf{0.00E+00 (0.00E+00)$\approx$}\unboldmath{}} & \multicolumn{1}{l}{6.91E-05 (3.31E-04)$-$} \\
$F_{4}$ & \multicolumn{1}{l}{\textbf{2.02E+00 (8.56E-01)}} & \multicolumn{1}{l}{1.95E+01 (8.17E+00)$-$} & \multicolumn{1}{l}{7.42E+00 (2.14E+00)$-$} & \multicolumn{1}{l}{7.21E+01 (3.45E+01)$-$} & \multicolumn{1}{l}{1.11E+01 (2.22E+00)$-$} & \multicolumn{1}{l}{1.27E+01 (2.73E+00)$-$} & \multicolumn{1}{l}{9.05E+00 (1.44E+00)$-$} & \multicolumn{1}{l}{2.37E+01 (1.37E+01)$-$} \\
$F_{5}$ & \multicolumn{1}{l}{\textbf{0.00E+00 (0.00E+00)}} & \multicolumn{1}{l}{\boldmath{}\textbf{0.00E+00 (0.00E+00)$\approx$}\unboldmath{}} & \multicolumn{1}{l}{7.24E-01 (1.22E+00)$-$} & \multicolumn{1}{l}{\boldmath{}\textbf{0.00E+00 (0.00E+00)$\approx$}\unboldmath{}} & \multicolumn{1}{l}{\boldmath{}\textbf{0.00E+00 (0.00E+00)$\approx$}\unboldmath{}} & \multicolumn{1}{l}{\boldmath{}\textbf{0.00E+00 (0.00E+00)$\approx$}\unboldmath{}} & \multicolumn{1}{l}{\boldmath{}\textbf{0.00E+00 (0.00E+00)$\approx$}\unboldmath{}} & \multicolumn{1}{l}{1.78E-01 (3.12E-01)$-$} \\
$F_{6}$ & \multicolumn{1}{l}{\textbf{1.16E-01 (2.79E-02)}} & \multicolumn{1}{l}{4.99E-01 (4.11E-01)$-$} & \multicolumn{1}{l}{3.13E+01 (1.54E+01)$-$} & \multicolumn{1}{l}{5.34E+01 (3.33E+01)$-$} & \multicolumn{1}{l}{1.89E+01 (1.83E+01)$-$} & \multicolumn{1}{l}{5.06E+01 (3.16E+01)$-$} & \multicolumn{1}{l}{1.25E+01 (1.00E+01)$-$} & \multicolumn{1}{l}{4.91E+03 (6.53E+03)$-$} \\
$F_{7}$ & \multicolumn{1}{l}{\textbf{5.17E-02 (6.21E-02)}} & \multicolumn{1}{l}{4.23E+00 (7.90E+00)$-$} & \multicolumn{1}{l}{1.07E+01 (5.20E+00)$-$} & \multicolumn{1}{l}{2.98E+00 (3.61E+00)$-$} & \multicolumn{1}{l}{1.16E+00 (1.26E+00)$-$} & \multicolumn{1}{l}{7.77E+00 (6.65E+00)$-$} & \multicolumn{1}{l}{1.42E+01 (8.96E+00)$-$} & \multicolumn{1}{l}{2.29E+01 (8.80E+00)$-$} \\
$F_{8}$ & \multicolumn{1}{l}{\textbf{7.19E-01 (1.02E+00)}} & \multicolumn{1}{l}{8.24E+00 (1.00E+01)$-$} & \multicolumn{1}{l}{2.10E+01 (7.26E-01)$-$} & \multicolumn{1}{l}{2.64E+01 (9.73E-01)$-$} & \multicolumn{1}{l}{1.38E+01 (8.98E+00)$-$} & \multicolumn{1}{l}{2.02E+01 (8.29E-01)$-$} & \multicolumn{1}{l}{1.96E+01 (3.80E+00)$-$} & \multicolumn{1}{l}{2.08E+01 (3.81E-01)$-$} \\
$F_{9}$ & \multicolumn{1}{l}{\textbf{1.07E+02 (1.98E+01)}} & \multicolumn{1}{l}{1.81E+02 (9.39E-06)$-$} & \multicolumn{1}{l}{1.81E+02 (3.75E-06)$-$} & \multicolumn{1}{l}{1.81E+02 (9.02E-06)$-$} & \multicolumn{1}{l}{1.81E+02 (8.41E-06)$-$} & \multicolumn{1}{l}{1.81E+02 (1.04E-05)$-$} & \multicolumn{1}{l}{1.81E+02 (2.18E-05)$-$} & \multicolumn{1}{l}{1.81E+02 (5.43E-04)$-$} \\
$F_{10}$ & \multicolumn{1}{l}{\textbf{0.00E+00 (0.00E+00)}} & \multicolumn{1}{l}{1.13E+02 (3.52E+01)$-$} & \multicolumn{1}{l}{1.00E+02 (3.03E-02)$-$} & \multicolumn{1}{l}{1.13E+02 (3.76E+01)$-$} & \multicolumn{1}{l}{1.00E+02 (3.55E-02)$-$} & \multicolumn{1}{l}{1.12E+02 (3.47E+01)$-$} & \multicolumn{1}{l}{1.11E+02 (3.42E+01)$-$} & \multicolumn{1}{l}{1.07E+02 (3.80E+01)$-$} \\
$F_{11}$ & \multicolumn{1}{l}{\textbf{7.28E-05 (3.06E-04)}} & \multicolumn{1}{l}{3.39E+02 (4.87E+01)$-$} & \multicolumn{1}{l}{3.06E+02 (2.46E+01)$-$} & \multicolumn{1}{l}{3.19E+02 (3.95E+01)$-$} & \multicolumn{1}{l}{3.39E+02 (4.87E+01)$-$} & \multicolumn{1}{l}{3.16E+02 (3.68E+01)$-$} & \multicolumn{1}{l}{3.39E+02 (4.87E+01)$-$} & \multicolumn{1}{l}{3.19E+02 (3.95E+01)$-$} \\
$F_{12}$ & \multicolumn{1}{l}{\textbf{2.29E+02 (6.08E-01)}} & \multicolumn{1}{l}{2.37E+02 (3.11E+00)$-$} & \multicolumn{1}{l}{2.41E+02 (5.26E+00)$-$} & \multicolumn{1}{l}{2.37E+02 (5.10E+00)$-$} & \multicolumn{1}{l}{2.34E+02 (2.68E+00)$-$} & \multicolumn{1}{l}{2.39E+02 (4.46E+00)$-$} & \multicolumn{1}{l}{2.44E+02 (1.63E+01)$-$} & \multicolumn{1}{l}{2.42E+02 (8.38E+00)$-$} \\
\midrule
$+$ / $\approx$ / $-$ & --    & 0/3/9 & 0/2/10 & 0/3/9 & 0/3/9 & 0/3/9 & 0/3/9 & 0/1/11 \\
\bottomrule
\end{tabular}%
    }
\footnotesize
\textsuperscript{*} The Wilcoxon rank-sum tests (with a significance level of 0.05) were conducted between MetaDE and each individually.
The final row displays the number of problems where the corresponding algorithm performs statistically better ($+$),  similar ($\thickapprox$), or worse ($-$) compared to MetaDE.\\
\label{tab:vsClass20D_supp}%
}
\end{table}%

\clearpage

% Table generated by Excel2LaTeX from sheet 'Sheet1'
\begin{table}[htbp]
  \centering
  \caption{Detailed results on 10D problems in CEC2022 benchmark suite. The peer DE variants are set with population size of 1,000. 
The mean and standard deviation (in parentheses) of the results over multiple runs are displayed in pairs. 
Results with the best mean values are highlighted.
  }
  {
  \resizebox{\textwidth}{!}{
   \renewcommand{\arraystretch}{1.2}
% Table generated by Excel2LaTeX from sheet 'Exp2 NP1000'
\begin{tabular}{ccccccccc}
\toprule
Func  & MetaDE & DE    & SaDE  & JaDE  & CoDE  & SHADE & LSHADE-RSP & EDEV \\
\midrule
$F_{1}$ & \multicolumn{1}{l}{\textbf{0.00E+00 (0.00E+00)}} & \multicolumn{1}{l}{\boldmath{}\textbf{0.00E+00 (0.00E+00)$\approx$}\unboldmath{}} & \multicolumn{1}{l}{\boldmath{}\textbf{0.00E+00 (0.00E+00)$\approx$}\unboldmath{}} & \multicolumn{1}{l}{\boldmath{}\textbf{0.00E+00 (0.00E+00)$\approx$}\unboldmath{}} & \multicolumn{1}{l}{\boldmath{}\textbf{0.00E+00 (0.00E+00)$\approx$}\unboldmath{}} & \multicolumn{1}{l}{\boldmath{}\textbf{0.00E+00 (0.00E+00)$\approx$}\unboldmath{}} & \multicolumn{1}{l}{\boldmath{}\textbf{0.00E+00 (0.00E+00)$\approx$}\unboldmath{}} & \multicolumn{1}{l}{\boldmath{}\textbf{0.00E+00 (0.00E+00)$\approx$}\unboldmath{}} \\
$F_{2}$ & \multicolumn{1}{l}{\textbf{0.00E+00 (0.00E+00)}} & \multicolumn{1}{l}{3.60E+00 (1.20E+00)$-$} & \multicolumn{1}{l}{6.87E+00 (3.63E+00)$-$} & \multicolumn{1}{l}{8.15E+00 (2.11E+00)$-$} & \multicolumn{1}{l}{2.44E+00 (1.97E+00)$-$} & \multicolumn{1}{l}{8.02E+00 (2.47E+00)$-$} & \multicolumn{1}{l}{1.12E+00 (1.79E+00)$-$} & \multicolumn{1}{l}{6.27E+00 (2.93E+00)$-$} \\
$F_{3}$ & \multicolumn{1}{l}{\textbf{0.00E+00 (0.00E+00)}} & \multicolumn{1}{l}{\boldmath{}\textbf{0.00E+00 (0.00E+00)$\approx$}\unboldmath{}} & \multicolumn{1}{l}{\boldmath{}\textbf{0.00E+00 (0.00E+00)$\approx$}\unboldmath{}} & \multicolumn{1}{l}{\boldmath{}\textbf{0.00E+00 (0.00E+00)$\approx$}\unboldmath{}} & \multicolumn{1}{l}{\boldmath{}\textbf{0.00E+00 (0.00E+00)$\approx$}\unboldmath{}} & \multicolumn{1}{l}{\boldmath{}\textbf{0.00E+00 (0.00E+00)$\approx$}\unboldmath{}} & \multicolumn{1}{l}{\boldmath{}\textbf{0.00E+00 (0.00E+00)$\approx$}\unboldmath{}} & \multicolumn{1}{l}{\boldmath{}\textbf{0.00E+00 (0.00E+00)$\approx$}\unboldmath{}} \\
$F_{4}$ & \multicolumn{1}{l}{\textbf{0.00E+00 (0.00E+00)}} & \multicolumn{1}{l}{1.50E+01 (2.59E+00)$-$} & \multicolumn{1}{l}{4.75E-01 (5.04E-01)$-$} & \multicolumn{1}{l}{2.20E+00 (5.30E-01)$-$} & \multicolumn{1}{l}{\boldmath{}\textbf{0.00E+00 (0.00E+00)$\approx$}\unboldmath{}} & \multicolumn{1}{l}{\boldmath{}\textbf{0.00E+00 (0.00E+00)$\approx$}\unboldmath{}} & \multicolumn{1}{l}{9.63E-01 (6.54E-01)$-$} & \multicolumn{1}{l}{4.91E+00 (5.24E+00)$-$} \\
$F_{5}$ & \multicolumn{1}{l}{\textbf{0.00E+00 (0.00E+00)}} & \multicolumn{1}{l}{\boldmath{}\textbf{0.00E+00 (0.00E+00)$\approx$}\unboldmath{}} & \multicolumn{1}{l}{\boldmath{}\textbf{0.00E+00 (0.00E+00)$\approx$}\unboldmath{}} & \multicolumn{1}{l}{\boldmath{}\textbf{0.00E+00 (0.00E+00)$\approx$}\unboldmath{}} & \multicolumn{1}{l}{\boldmath{}\textbf{0.00E+00 (0.00E+00)$\approx$}\unboldmath{}} & \multicolumn{1}{l}{\boldmath{}\textbf{0.00E+00 (0.00E+00)$\approx$}\unboldmath{}} & \multicolumn{1}{l}{\boldmath{}\textbf{0.00E+00 (0.00E+00)$\approx$}\unboldmath{}} & \multicolumn{1}{l}{\boldmath{}\textbf{0.00E+00 (0.00E+00)$\approx$}\unboldmath{}} \\
$F_{6}$ & \multicolumn{1}{l}{\textbf{5.50E-04 (3.96E-04)}} & \multicolumn{1}{l}{1.49E-02 (4.74E-03)$-$} & \multicolumn{1}{l}{1.95E+00 (1.54E+00)$-$} & \multicolumn{1}{l}{1.01E-01 (6.20E-02)$-$} & \multicolumn{1}{l}{7.69E-04 (4.31E-04)$\approx$} & \multicolumn{1}{l}{4.62E-03 (8.17E-03)$-$} & \multicolumn{1}{l}{1.89E-03 (7.18E-04)$-$} & \multicolumn{1}{l}{4.37E-02 (6.48E-02)$-$} \\
$F_{7}$ & \multicolumn{1}{l}{\textbf{0.00E+00 (0.00E+00)}} & \multicolumn{1}{l}{9.60E-04 (5.35E-03)$-$} & \multicolumn{1}{l}{1.98E-02 (8.61E-03)$-$} & \multicolumn{1}{l}{\boldmath{}\textbf{0.00E+00 (0.00E+00)$\approx$}\unboldmath{}} & \multicolumn{1}{l}{\boldmath{}\textbf{0.00E+00 (0.00E+00)$\approx$}\unboldmath{}} & \multicolumn{1}{l}{\boldmath{}\textbf{0.00E+00 (0.00E+00)$\approx$}\unboldmath{}} & \multicolumn{1}{l}{\boldmath{}\textbf{0.00E+00 (0.00E+00)$\approx$}\unboldmath{}} & \multicolumn{1}{l}{1.35E+00 (4.98E+00)$-$} \\
$F_{8}$ & \multicolumn{1}{l}{5.52E-03 (4.41E-03)} & \multicolumn{1}{l}{1.19E-01 (2.50E-02)$-$} & \multicolumn{1}{l}{1.13E+00 (4.44E-01)$-$} & \multicolumn{1}{l}{1.03E-01 (2.62E-02)$-$} & \multicolumn{1}{l}{\boldmath{}\textbf{0.00E+00 (0.00E+00)$\approx$}\unboldmath{}} & \multicolumn{1}{l}{8.28E-02 (2.66E-02)$-$} & \multicolumn{1}{l}{9.13E-02 (1.11E-01)$-$} & \multicolumn{1}{l}{1.03E+00 (3.58E+00)$-$} \\
$F_{9}$ & \multicolumn{1}{l}{\textbf{3.36E+00 (1.77E+01)}} & \multicolumn{1}{l}{2.29E+02 (7.85E-06)$-$} & \multicolumn{1}{l}{2.29E+02 (8.58E-06)$-$} & \multicolumn{1}{l}{2.29E+02 (8.28E-06)$-$} & \multicolumn{1}{l}{2.29E+02 (8.67E-14)$-$} & \multicolumn{1}{l}{2.29E+02 (2.74E-06)$-$} & \multicolumn{1}{l}{2.29E+02 (9.39E-06)$-$} & \multicolumn{1}{l}{2.29E+02 (1.17E-05)$-$} \\
$F_{10}$ & \multicolumn{1}{l}{\textbf{0.00E+00 (0.00E+00)}} & \multicolumn{1}{l}{1.00E+02 (1.70E-02)$-$} & \multicolumn{1}{l}{1.00E+02 (3.93E-02)$-$} & \multicolumn{1}{l}{1.00E+02 (2.49E-02)$-$} & \multicolumn{1}{l}{1.00E+02 (1.01E-02)$-$} & \multicolumn{1}{l}{1.00E+02 (1.83E-02)$-$} & \multicolumn{1}{l}{1.00E+02 (8.05E-04)$-$} & \multicolumn{1}{l}{1.00E+02 (3.93E-02)$-$} \\
$F_{11}$ & \multicolumn{1}{l}{\textbf{0.00E+00 (0.00E+00)}} & \multicolumn{1}{l}{\boldmath{}\textbf{0.00E+00 (0.00E+00)$\approx$}\unboldmath{}} & \multicolumn{1}{l}{\boldmath{}\textbf{0.00E+00 (0.00E+00)$\approx$}\unboldmath{}} & \multicolumn{1}{l}{\boldmath{}\textbf{0.00E+00 (0.00E+00)$\approx$}\unboldmath{}} & \multicolumn{1}{l}{3.65E-07 (2.03E-06)$-$} & \multicolumn{1}{l}{\boldmath{}\textbf{0.00E+00 (0.00E+00)$\approx$}\unboldmath{}} & \multicolumn{1}{l}{\boldmath{}\textbf{0.00E+00 (0.00E+00)$\approx$}\unboldmath{}} & \multicolumn{1}{l}{\boldmath{}\textbf{0.00E+00 (0.00E+00)$\approx$}\unboldmath{}} \\
$F_{12}$ & \multicolumn{1}{l}{\textbf{1.39E+02 (4.63E+01)}} & \multicolumn{1}{l}{1.60E+02 (9.79E-01)$-$} & \multicolumn{1}{l}{1.60E+02 (1.56E+00)$-$} & \multicolumn{1}{l}{1.59E+02 (1.28E+00)$-$} & \multicolumn{1}{l}{1.59E+02 (8.67E-14)$-$} & \multicolumn{1}{l}{1.61E+02 (1.69E+00)$-$} & \multicolumn{1}{l}{1.63E+02 (7.62E-01)$-$} & \multicolumn{1}{l}{1.60E+02 (1.18E+00)$-$} \\
\midrule
$+$ / $\approx$ / $-$ & --    & 0/4/8 & 0/4/8 & 0/5/7 & 0/7/5 & 0/6/6 & 0/5/7 & 0/4/8 \\
\bottomrule
\end{tabular}%
}
\footnotesize
\textsuperscript{*} The Wilcoxon rank-sum tests (with a significance level of 0.05) were conducted between MetaDE and each individually.
The final row displays the number of problems where the corresponding algorithm performs statistically better ($+$),  similar ($\thickapprox$), or worse ($-$) compared to MetaDE.\\
\label{tab:NP10000 10D_supp}%
}
\end{table}%


% Table generated by Excel2LaTeX from sheet 'Sheet1'
\begin{table}[htbp]
  \centering
  \caption{Detailed results on 20D problems in CEC2022 benchmark suite. The peer DE variants are set with population size of 1,000. 
The mean and standard deviation (in parentheses) of the results over multiple runs are displayed in pairs. 
Results with the best mean values are highlighted.
  }
  {
  \resizebox{\textwidth}{!}{
   \renewcommand{\arraystretch}{1.2}
% Table generated by Excel2LaTeX from sheet 'Exp2 NP1000'
\begin{tabular}{ccccccccc}
\toprule
Func  & MetaDE & DE    & SaDE  & JaDE  & CoDE  & SHADE & LSHADE-RSP & EDEV \\
\midrule
$F_{1}$ & \multicolumn{1}{l}{\textbf{0.00E+00 (0.00E+00)}} & \multicolumn{1}{l}{\boldmath{}\textbf{0.00E+00 (0.00E+00)$\approx$}\unboldmath{}} & \multicolumn{1}{l}{\boldmath{}\textbf{0.00E+00 (0.00E+00)$\approx$}\unboldmath{}} & \multicolumn{1}{l}{\boldmath{}\textbf{0.00E+00 (0.00E+00)$\approx$}\unboldmath{}} & \multicolumn{1}{l}{\boldmath{}\textbf{0.00E+00 (0.00E+00)$\approx$}\unboldmath{}} & \multicolumn{1}{l}{\boldmath{}\textbf{0.00E+00 (0.00E+00)$\approx$}\unboldmath{}} & \multicolumn{1}{l}{\boldmath{}\textbf{0.00E+00 (0.00E+00)$\approx$}\unboldmath{}} & \multicolumn{1}{l}{\boldmath{}\textbf{0.00E+00 (0.00E+00)$\approx$}\unboldmath{}} \\
$F_{2}$ & \multicolumn{1}{l}{\textbf{1.26E-02 (3.74E-02)}} & \multicolumn{1}{l}{4.49E+01 (0.00E+00)$-$} & \multicolumn{1}{l}{4.91E+01 (7.67E-06)$-$} & \multicolumn{1}{l}{4.91E+01 (0.00E+00)$-$} & \multicolumn{1}{l}{4.91E+01 (0.00E+00)$-$} & \multicolumn{1}{l}{4.91E+01 (0.00E+00)$-$} & \multicolumn{1}{l}{4.52E+01 (1.05E+00)$-$} & \multicolumn{1}{l}{4.89E+01 (1.05E+00)$-$} \\
$F_{3}$ & \multicolumn{1}{l}{\textbf{0.00E+00 (0.00E+00)}} & \multicolumn{1}{l}{\boldmath{}\textbf{0.00E+00 (0.00E+00)$\approx$}\unboldmath{}} & \multicolumn{1}{l}{\boldmath{}\textbf{0.00E+00 (0.00E+00)$\approx$}\unboldmath{}} & \multicolumn{1}{l}{\boldmath{}\textbf{0.00E+00 (0.00E+00)$\approx$}\unboldmath{}} & \multicolumn{1}{l}{\boldmath{}\textbf{0.00E+00 (0.00E+00)$\approx$}\unboldmath{}} & \multicolumn{1}{l}{\boldmath{}\textbf{0.00E+00 (0.00E+00)$\approx$}\unboldmath{}} & \multicolumn{1}{l}{\boldmath{}\textbf{0.00E+00 (0.00E+00)$\approx$}\unboldmath{}} & \multicolumn{1}{l}{\boldmath{}\textbf{0.00E+00 (0.00E+00)$\approx$}\unboldmath{}} \\
$F_{4}$ & \multicolumn{1}{l}{\textbf{2.02E+00 (8.56E-01)}} & \multicolumn{1}{l}{8.33E+01 (5.85E+00)$-$} & \multicolumn{1}{l}{6.09E+00 (1.02E+00)$-$} & \multicolumn{1}{l}{1.07E+01 (1.95E+00)$-$} & \multicolumn{1}{l}{8.02E+00 (1.08E+00)$-$} & \multicolumn{1}{l}{7.39E+00 (1.17E+00)$-$} & \multicolumn{1}{l}{4.82E+00 (8.64E-01)$-$} & \multicolumn{1}{l}{1.81E+01 (1.52E+01)$-$} \\
$F_{5}$ & \multicolumn{1}{l}{\textbf{0.00E+00 (0.00E+00)}} & \multicolumn{1}{l}{\boldmath{}\textbf{0.00E+00 (0.00E+00)$\approx$}\unboldmath{}} & \multicolumn{1}{l}{6.16E-02 (2.71E-01)$-$} & \multicolumn{1}{l}{\boldmath{}\textbf{0.00E+00 (0.00E+00)$\approx$}\unboldmath{}} & \multicolumn{1}{l}{\boldmath{}\textbf{0.00E+00 (0.00E+00)$\approx$}\unboldmath{}} & \multicolumn{1}{l}{\boldmath{}\textbf{0.00E+00 (0.00E+00)$\approx$}\unboldmath{}} & \multicolumn{1}{l}{\boldmath{}\textbf{0.00E+00 (0.00E+00)$\approx$}\unboldmath{}} & \multicolumn{1}{l}{\boldmath{}\textbf{0.00E+00 (0.00E+00)$\approx$}\unboldmath{}} \\
$F_{6}$ & \multicolumn{1}{l}{1.16E-01 (2.79E-02)} & \multicolumn{1}{l}{8.86E+00 (1.34E+00)$-$} & \multicolumn{1}{l}{1.72E+02 (5.24E+02)$-$} & \multicolumn{1}{l}{1.80E+00 (7.49E-01)$-$} & \multicolumn{1}{l}{2.07E-01 (5.04E-02)$-$} & \multicolumn{1}{l}{1.35E-01 (5.23E-02)$\approx$} & \multicolumn{1}{l}{\textbf{7.35E-02 (3.33E-02)$+$}} & \multicolumn{1}{l}{2.70E+00 (1.13E+01)$-$} \\
$F_{7}$ & \multicolumn{1}{l}{5.17E-02 (6.21E-02)} & \multicolumn{1}{l}{3.28E+01 (1.82E+00)$-$} & \multicolumn{1}{l}{1.57E+01 (3.45E+00)$-$} & \multicolumn{1}{l}{1.31E+01 (2.39E+00)$-$} & \multicolumn{1}{l}{\textbf{2.02E-03 (1.44E-03)$+$}} & \multicolumn{1}{l}{9.48E+00 (1.76E+00)$-$} & \multicolumn{1}{l}{2.92E+00 (9.57E-01)$-$} & \multicolumn{1}{l}{1.68E+01 (1.04E+01)$-$} \\
$F_{8}$ & \multicolumn{1}{l}{\textbf{7.19E-01 (1.02E+00)}} & \multicolumn{1}{l}{2.41E+01 (2.25E+00)$-$} & \multicolumn{1}{l}{2.11E+01 (1.76E+00)$-$} & \multicolumn{1}{l}{2.15E+01 (1.27E+00)$-$} & \multicolumn{1}{l}{1.12E+01 (1.98E+00)$-$} & \multicolumn{1}{l}{1.99E+01 (2.81E+00)$-$} & \multicolumn{1}{l}{8.53E+00 (4.47E+00)$-$} & \multicolumn{1}{l}{1.88E+01 (5.20E+00)$-$} \\
$F_{9}$ & \multicolumn{1}{l}{\textbf{1.07E+02 (1.98E+01)}} & \multicolumn{1}{l}{1.81E+02 (2.68E-05)$-$} & \multicolumn{1}{l}{1.81E+02 (2.75E-05)$-$} & \multicolumn{1}{l}{1.81E+02 (1.08E-05)$-$} & \multicolumn{1}{l}{1.81E+02 (7.25E-06)$-$} & \multicolumn{1}{l}{1.81E+02 (9.14E-06)$-$} & \multicolumn{1}{l}{1.81E+02 (1.24E-05)$-$} & \multicolumn{1}{l}{1.81E+02 (1.01E-04)$-$} \\
$F_{10}$ & \multicolumn{1}{l}{\textbf{0.00E+00 (0.00E+00)}} & \multicolumn{1}{l}{1.00E+02 (3.21E-02)$-$} & \multicolumn{1}{l}{1.00E+02 (2.58E-02)$-$} & \multicolumn{1}{l}{1.00E+02 (3.51E-02)$-$} & \multicolumn{1}{l}{1.00E+02 (1.80E-02)$-$} & \multicolumn{1}{l}{1.00E+02 (1.80E-02)$-$} & \multicolumn{1}{l}{1.00E+02 (1.21E-02)$-$} & \multicolumn{1}{l}{1.00E+02 (4.13E-02)$-$} \\
$F_{11}$ & \multicolumn{1}{l}{\textbf{7.28E-05 (3.06E-04)}} & \multicolumn{1}{l}{3.97E+02 (1.80E+01)$-$} & \multicolumn{1}{l}{3.26E+02 (4.45E+01)$-$} & \multicolumn{1}{l}{3.16E+02 (3.74E+01)$-$} & \multicolumn{1}{l}{3.81E+02 (4.02E+01)$-$} & \multicolumn{1}{l}{3.16E+02 (3.74E+01)$-$} & \multicolumn{1}{l}{3.77E+02 (4.25E+01)$-$} & \multicolumn{1}{l}{3.19E+02 (9.80E+01)$-$} \\
$F_{12}$ & \multicolumn{1}{l}{\textbf{2.29E+02 (6.08E-01)}} & \multicolumn{1}{l}{2.35E+02 (2.45E+00)$-$} & \multicolumn{1}{l}{2.34E+02 (2.55E+00)$-$} & \multicolumn{1}{l}{2.30E+02 (1.99E+00)$-$} & \multicolumn{1}{l}{2.30E+02 (1.40E+00)$-$} & \multicolumn{1}{l}{2.32E+02 (1.09E+00)$-$} & \multicolumn{1}{l}{2.33E+02 (2.27E+00)$-$} & \multicolumn{1}{l}{2.38E+02 (3.81E+00)$-$} \\
\midrule
$+$ / $\approx$ / $-$ & --    & 0/3/9 & 0/2/10 & 0/3/9 & 1/3/8 & 0/4/8 & 1/3/8 & 0/3/9 \\
\bottomrule
\end{tabular}%
}
\footnotesize
\textsuperscript{*} The Wilcoxon rank-sum tests (with a significance level of 0.05) were conducted between MetaDE and each individually.
The final row displays the number of problems where the corresponding algorithm performs statistically better ($+$),  similar ($\thickapprox$), or worse ($-$) compared to MetaDE.\\
\label{tab:NP10000 20D_supp}%
}
\end{table}%


% \subsection{Supplementary results}\label{section:FEs}
 % This experiment leveraged parallel GPU computation and constrained the runtime: 30 seconds for the 10-dimensional problems and 60 seconds for the 20-dimensional problems. 
 
% We recorded the number of FEs each algorithm achieved within 60s, which are presented in Table \ref{tab:FEs}. 
 
%  In addition, when the population size of the comparative DE variants was increased to 10,000 (same level of concurrency as MetaDE), the FEs achieved by all algorithms are displayed in Table \ref{tab:FEs NP10000}.

% The results show that MetaDE achieved considerably more FEs within a given time than the other algorithms. This demonstrates that MetaDE has a high degree of parallelism, making it particularly well-suited for GPU computing.



% Table generated by Excel2LaTeX from sheet 'Sheet1'
\begin{table}[htbp]
  \centering
  \caption{The number of FEs achieved by each algorithm within \SI{60}{\second}. The peer DE variants are set with population size of 1,000.}
  {
           \renewcommand{\arraystretch}{1}
 \renewcommand{\tabcolsep}{10pt}
% Table generated by Excel2LaTeX from sheet 'Exp2 NP10000'
\begin{tabular}{cccccccccc}
\toprule
Dim   & Func  & MetaDE & DE    & SaDE  & JaDE  & CoDE  & SHADE &LSHADE-RSP&EDEV\\
\midrule
\multirow{12}[2]{*}{10D} & $F_{1}$ & \textbf{1.85E+09} & 3.91E+07 & 4.38E+06 & 6.26E+06 & 7.67E+07 & 4.62E+06&1.92E+07&1.91E+07 \\
      & $F_{2}$ & \textbf{1.84E+09} & 4.05E+07 &4.35E+06 & 6.20E+06 & 7.61E+07 & 4.63E+06&1.89E+07&1.96E+07 \\
      & $F_{3}$ & \textbf{1.50E+09} & 3.63E+07 & 4.31E+06 &6.18E+06& 6.87E+07& 4.54E+06&1.84E+07 &1.89E+07\\
      & $F_{4}$ & \textbf{1.84E+09} &3.77E+07 & 4.27E+06 & 6.16E+06 &7.42E+07 & 4.64E+06&1.89E+07&1.98E+07 \\
      & $F_{5}$ & \textbf{1.83E+09} & 3.85E+07 &4.33E+06 & 6.16E+06 & 7.50E+07& 4.64E+06&1.85E+07& 1.99E+07\\
      & $F_{6}$ & \textbf{1.84E+09} &3.71E+07 & 4.30E+06 & 6.01E+06& 7.38E+07 & 4.61E+06& 1.86E+07&1.99E+07\\
      & $F_{7}$ & \textbf{1.74E+09} &3.71E+07 & 4.15E+06 & 5.84E+06 & 7.05E+07& 4.37E+06&1.77E+07&1.91E+07\\
      & $F_{8}$ & \textbf{1.72E+09} & 3.69E+07 & 4.02E+06 & 5.75E+06 & 6.82E+07 & 4.35E+06& 1.71E+07&1.93E+07\\
      & $F_{9}$ & \textbf{1.78E+09} &3.81E+07 &4.35E+06 &6.10E+06 & 6.84E+07 & 4.56E+06&1.84E+07 &1.92E+07\\
      & $F_{10}$ & \textbf{1.44E+09} & 3.50E+07&4.12E+06 &  5.77E+06 & 6.94E+07 & 4.41E+06&1.76E+07& 1.83E+07\\
      & $F_{11}$ & \textbf{1.46E+09} & 3.43E+07& 4.11E+06 & 5.87E+06 & 6.89E+07 & 4.36E+06 &1.69E+07&1.83E+07\\
      & $F_{12}$ & \textbf{1.43E+09} &3.37E+07 & 3.97E+06 & 5.81E+06 & 6.56E+07 & 4.32E+06 &1.72E+07&1.92E+07\\
\midrule
\midrule
\multirow{12}[2]{*}{20D} & $F_{1}$ & \textbf{1.66E+09} & 3.93E+07 &4.29E+06 & 5.94E+06 & 7.20E+07 & 4.62E+06&1.87E+07&1.94E+07 \\
      & $F_{2}$ & \textbf{1.66E+09} & 3.89E+07 &4.17E+06 & 5.94E+06 & 7.33E+07 & 4.61E+06&1.91E+07& 1.95E+07\\
      & $F_{3}$ & \textbf{1.18E+09} & 3.38E+07 & 4.16E+06& 5.93E+06 &  6.77E+07& 4.49E+06& 1.72E+07&1.85E+07\\
      & $F_{4}$ & \textbf{1.65E+09} &3.56E+07 & 4.27E+06 & 5.85E+06 & 7.05E+07& 4.63E+06& 1.90E+07&1.94E+07\\
      & $F_{5}$ & \textbf{1.64E+09} & 3.88E+07 & 4.20E+06 & 6.04E+06& 7.16E+07& 4.62E+06&1.86E+07&1.95E+07 \\
      & $F_{6}$ & \textbf{1.64E+09} & 3.73E+07 & 4.25E+06 & 5.84E+06 & 7.10E+07 & 4.61E+06&1.86E+07& 1.92E+07\\
      & $F_{7}$ & \textbf{1.45E+09} & 3.13E+07 & 3.87E+06 & 5.27E+06 & 6.40E+07 & 4.26E+06& 1.72E+07&1.79E+07\\
      & $F_{8}$ & \textbf{1.44E+09} & 3.08E+07 & 3.53E+06& 5.29E+06 & 5.98E+07& 4.09E+06&1.69E+07&1.70E+07 \\
      & $F_{9}$ & \textbf{1.57E+09} & 3.62E+07 &4.05E+06 & 6.15E+06 &  6.91E+07& 4.54E+06&1.84E+07& 1.89E+07\\
      & $F_{10}$ & \textbf{9.80E+08} &2.94E+07 & 3.74E+06 & 5.28E+06& 5.69E+07& 4.12E+06& 1.54E+07& 1.63E+07\\
      & $F_{11}$ & \textbf{1.00E+09} & 2.37E+07 &3.76E+06 & 5.38E+06&5.86E+07& 4.04E+06&1.59E+07& 1.64E+07 \\
      & $F_{12}$ & \textbf{9.90E+08} & 2.33E+07 & 3.76E+06 &5.39E+06&5.47E+07& 4.09E+06& 1.57E+07& 1.60E+07\\
\bottomrule
\end{tabular}%
}
  \label{tab:FEs NP10000_supp}%
\end{table}%

\clearpage

\section{Supplementary information for the application of robotics control}\label{section_brax_supp}

\subsection{Illustrations of the robotics control tasks}\label{section_brax_image_supp}

% Figs. \ref{Figure_swimmer}-\ref{Figure_hopper}, illustrate the three robotics control tasks from the Brax \cite{brax} reinforcement learning library: swimmer, reacher, and hopper.

% \begin{enumerate}
%     \item Swimmer: A serpentine agent that must coordinate joint movements to navigate through a fluid environment, aiming for efficient propulsion and forward movement.
%     \item Reacher: A robotic arm environment where the goal is to control joint torques to reach a target point with precision, testing the fine control of the learning algorithm.
%     \item Hopper: A one-legged robot that must learn to balance and hop forward as far and as fast as possible, providing a benchmark for locomotion and stability in dynamic environments.
% \end{enumerate}

\begin{figure}[!h]
\centering
\includegraphics[width=6cm, height=3cm]{su_swimmer.png}
\caption{The swimmer task in Brax. It is designed to simulate a multi-jointed creature navigating through a fluid medium.}
\label{Figure_swimmer_supp}
\end{figure}


\begin{figure}[!h]
\centering
\includegraphics[width=6cm, height=3cm]{su_hopper.png}
\caption{The hopper task in Brax. It resembles a one-legged robotic creature with the objective to hop forward smoothly and quickly.}
\label{Figure_hopper_supp}
\end{figure}

\begin{figure}[!h]
\centering
\includegraphics[width=6cm, height=3cm]{su_reacher.png}
\caption{The reacher task in Brax. It simulates a robotic arm tasked with reaching a target location.}
\label{Figure_reacher_supp}
\end{figure}


\subsection{Supplementary results}\label{section_brax_results_supp}
% Table \ref{tab:comparative-reward-analysis} displays the results of the neuroevolution experimrnt for 60 minutes.

\begin{table}[htbp]
\centering
\caption{Rewards achieved by MetaDE and peer evolutionary algorithms on the robotics tasks. 
The mean and standard deviation (in parentheses) of the results over multiple runs are displayed in pairs. 
Results with the best mean values are highlighted.
}
\label{tab:comparative-reward-analysis_supp}
{%
 \renewcommand{\arraystretch}{1}
\renewcommand{\tabcolsep}{3pt}
% Table generated by Excel2LaTeX from sheet 'Exp6 brax'
\begin{tabular}{cccccccc}
\toprule
Task & MetaDE  & \textbf{CSO } & \textbf{CMAES } & \textbf{SHADE } & \textbf{DE}&\textbf{LSHADE-RSP}&\textbf{EDEV} \\
\midrule
Swimmer  & \textbf{ 190.85 (2.39) } &  183.45 (1.15)  &  186.07 (2.68)  & 185.90 (3.31) &  182.15 (2.54) &186.36 (1.21)&145.52 (41.22)\\
%\midrule
Hopper   &  1187.53 (122.72) & 1330.45 (325.55) &  \textbf{1389.72 (474.41)}  &  1022.25 (122.94)  &  871.06 (147.12)&1102.23 (100.33)&457.22 (80.16) \\
%\midrule
Reacher  &  -21.05 (5.05)  &  -504.07 (112.18)  & \textbf{ -3.76 (1.05) } &  -343.18 (105.65)  &  -522.99 (142.78)    &    -342.74 (36.63)  &  -493.46 (173.67)\\
\bottomrule
\end{tabular}%

}
\end{table}

\clearpage
\section{More Evolvers}\label{subsection Evolver Comparison_supp}
We selected \texttt{DE/rand/1/bin} as the evolver for its simplicity and adaptability, hypothesizing its capability to self-evolve. Nonetheless, it is also interesting to evaluate the performance implications of utilizing other EAs as evolvers.

In this experiment, we maintained the identical framework and parameter settings for MetaDE as utilized in the primary experiments. 
The distinction lies in the comparative evaluation of the effectiveness of DE, PSO, Natural Evolution Strategies (NES) \citesupp{NES}, and Competitive Swarm Optimization (CSO) \citesupp{CSO} as evolvers. 
For PSO, parameters were set to the recommended values of $w=0.729$ and $c_1=c_2=1.49$ \citesupp{PSOParamSetting}. 
It is important to note that CSO and NES do not require parameter tuning.

As shown in Tables \ref{tab:diffTunner 10D}-\ref{tab:diffTunner 20D} and Figs. \ref{Figure_evolver_10D}-\ref{Figure_evolver_20D}, the experimental outcomes suggest minimal performance disparities among the EAs when employed as evolvers. 
Specifically, DE demonstrates a marginal advantage in 10-dimensional problems, whereas the performance is comparably uniform across all evolvers for 20-dimensional problems. 
These results imply that the selection of an evolver within the MetaDE framework is relatively flexible. 
Such findings highlight the adaptability and flexibility of MetaDE, illustrating that its efficiency is not significantly influenced by the particular choice of evolver.


% Table generated by Excel2LaTeX from sheet 'Sheet1'
\begin{table}[htbp]
  \centering
  \caption{
  Performance of MetaDE with different evolvers on 10D problems in CEC2022 benchmark suite. 
The mean and standard deviation (in parentheses) of the results over multiple runs are displayed in pairs. 
Results with the best mean values are highlighted.
  }
  {
         \renewcommand{\arraystretch}{1}
 \renewcommand{\tabcolsep}{10pt}
% Table generated by Excel2LaTeX from sheet 'Exp5 evolver'
\begin{tabular}{ccccc}
\toprule
Func  & DE    & PSO   & NES   & CSO \\
\midrule
$F_{1}$ & \textbf{0.00E+00 (0.00E+00)} & \boldmath{}\textbf{0.00E+00 (0.00E+00)$\approx$}\unboldmath{} & \boldmath{}\textbf{0.00E+00 (0.00E+00)$\approx$}\unboldmath{} & \boldmath{}\textbf{0.00E+00 (0.00E+00)$\approx$}\unboldmath{} \\
$F_{2}$ & \textbf{0.00E+00 (0.00E+00)} & \boldmath{}\textbf{0.00E+00 (0.00E+00)$\approx$}\unboldmath{} & 1.60E-06 (2.82E-06)$-$ & \boldmath{}\textbf{0.00E+00 (0.00E+00)$\approx$}\unboldmath{} \\
$F_{3}$ & \textbf{0.00E+00 (0.00E+00)} & \boldmath{}\textbf{0.00E+00 (0.00E+00)$\approx$}\unboldmath{} & \boldmath{}\textbf{0.00E+00 (0.00E+00)$\approx$}\unboldmath{} & \boldmath{}\textbf{0.00E+00 (0.00E+00)$\approx$}\unboldmath{} \\
$F_{4}$ & \textbf{0.00E+00 (0.00E+00)} & 1.41E-03 (7.70E-03)$\approx$ & 3.23E-02 (1.76E-01)$\approx$ & \boldmath{}\textbf{0.00E+00 (0.00E+00)$\approx$}\unboldmath{} \\
$F_{5}$ & \textbf{0.00E+00 (0.00E+00)} & \boldmath{}\textbf{0.00E+00 (0.00E+00)$\approx$}\unboldmath{} & \boldmath{}\textbf{0.00E+00 (0.00E+00)$\approx$}\unboldmath{} & \boldmath{}\textbf{0.00E+00 (0.00E+00)$\approx$}\unboldmath{} \\
$F_{6}$ & \textbf{5.50E-04 (3.96E-04)} & 1.45E-03 (1.61E-03)$\approx$ & 1.66E-03 (1.79E-03)$-$ & 1.37E-03 (9.80E-04)$-$ \\
$F_{7}$ & \textbf{0.00E+00 (0.00E+00)} & \boldmath{}\textbf{0.00E+00 (0.00E+00)$\approx$}\unboldmath{} & \boldmath{}\textbf{0.00E+00 (0.00E+00)$\approx$}\unboldmath{} & \boldmath{}\textbf{0.00E+00 (0.00E+00)$\approx$}\unboldmath{} \\
$F_{8}$ & 5.52E-03 (4.41E-03) & \textbf{1.77E-06 (1.48E-06)$+$} & 1.83E-02 (1.09E-02)$-$ & 1.57E-03 (2.26E-03)$+$ \\
$F_{9}$ & \textbf{3.36E+00 (1.77E+01)} & 1.40E+01 (4.63E+01)$-$ & 2.27E+01 (4.18E+01)$\approx$ & 1.75E+01 (4.86E+01)$-$ \\
$F_{10}$ & \textbf{0.00E+00 (0.00E+00)} & \boldmath{}\textbf{0.00E+00 (0.00E+00)$\approx$}\unboldmath{} & \boldmath{}\textbf{0.00E+00 (0.00E+00)$\approx$}\unboldmath{} & 7.38E-02 (3.99E-01)$\approx$ \\
$F_{11}$ & \textbf{0.00E+00 (0.00E+00)} & \boldmath{}\textbf{0.00E+00 (0.00E+00)$\approx$}\unboldmath{} & \boldmath{}\textbf{0.00E+00 (0.00E+00)$\approx$}\unboldmath{} & \boldmath{}\textbf{0.00E+00 (0.00E+00)$\approx$}\unboldmath{} \\
$F_{12}$ & \textbf{1.39E+02 (4.63E+01)} & 1.48E+02 (3.39E+01)$\approx$ & 1.59E+02 (6.82E-06)$-$ & 1.54E+02 (1.78E+01)$\approx$ \\
\midrule
$+$ / $\approx$ / $-$ & --    & 1/10/1 & 4/8/0 & 1/9/2 \\
\bottomrule
\end{tabular}%
}

\footnotesize
\textsuperscript{*} The Wilcoxon rank-sum tests (with a significance level of 0.05) were conducted between MetaDE and each individually.
The final row displays the number of problems where the corresponding evlover performs statistically better ($+$),  similar ($\thickapprox$), or worse ($-$) compared to DE.\\
\label{tab:diffTunner 10D_supp}%
\end{table}%


% Table generated by Excel2LaTeX from sheet 'Sheet1'
\begin{table}[htbp]
  \centering
  \caption{
  Performance of MetaDE with different evolvers on 20D problems in CEC2022 benchmark suite.
The mean and standard deviation (in parentheses) of the results over multiple runs are displayed in pairs. 
Results with the best mean values are highlighted.
  }
  {
        \renewcommand{\arraystretch}{1}
 \renewcommand{\tabcolsep}{10pt}
% Table generated by Excel2LaTeX from sheet 'Exp5 evolver'
\begin{tabular}{ccccc}
\toprule
Func  & DE    & PSO   & NES   & CSO \\
\midrule
$F_{1}$ & \multicolumn{1}{l}{\textbf{0.00E+00 (0.00E+00)}} & \multicolumn{1}{l}{\boldmath{}\textbf{0.00E+00 (0.00E+00)$\approx$}\unboldmath{}} & \multicolumn{1}{l}{\boldmath{}\textbf{0.00E+00 (0.00E+00)$\approx$}\unboldmath{}} & \multicolumn{1}{l}{\boldmath{}\textbf{0.00E+00 (0.00E+00)$\approx$}\unboldmath{}} \\
$F_{2}$ & \multicolumn{1}{l}{1.26E-02 (3.74E-02)} & \multicolumn{1}{l}{4.28E-02 (1.25E-01)$\approx$} & \multicolumn{1}{l}{1.11E+00 (1.30E+00)$-$} & \multicolumn{1}{l}{\textbf{0.00E+00 (0.00E+00)$+$}} \\
$F_{3}$ & \multicolumn{1}{l}{\textbf{0.00E+00 (0.00E+00)}} & \multicolumn{1}{l}{\boldmath{}\textbf{0.00E+00 (0.00E+00)$\approx$}\unboldmath{}} & \multicolumn{1}{l}{\boldmath{}\textbf{0.00E+00 (0.00E+00)$\approx$}\unboldmath{}} & \multicolumn{1}{l}{\boldmath{}\textbf{0.00E+00 (0.00E+00)$\approx$}\unboldmath{}} \\
$F_{4}$ & \multicolumn{1}{l}{\textbf{2.02E+00 (8.56E-01)}} & \multicolumn{1}{l}{2.51E+00 (1.19E+00)$-$} & \multicolumn{1}{l}{5.53E+00 (1.55E+00)$-$} & \multicolumn{1}{l}{2.85E+00 (1.48E+00)$\approx$} \\
$F_{5}$ & \multicolumn{1}{l}{\textbf{0.00E+00 (0.00E+00)}} & \multicolumn{1}{l}{\boldmath{}\textbf{0.00E+00 (0.00E+00)$\approx$}\unboldmath{}} & \multicolumn{1}{l}{\boldmath{}\textbf{0.00E+00 (0.00E+00)$\approx$}\unboldmath{}} & \multicolumn{1}{l}{\boldmath{}\textbf{0.00E+00 (0.00E+00)$\approx$}\unboldmath{}} \\
$F_{6}$ & \multicolumn{1}{l}{1.16E-01 (2.79E-02)} & \multicolumn{1}{l}{1.81E-01 (2.42E-01)$-$} & \multicolumn{1}{l}{9.32E-02 (2.65E-02)$+$} & \multicolumn{1}{l}{\boldmath{}\textbf{1.10E-01 (2.85E-02)$\approx$}\unboldmath{}} \\
$F_{7}$ & \multicolumn{1}{l}{5.17E-02 (6.21E-02)} & \multicolumn{1}{l}{1.61E-01 (3.84E-01)$\approx$} & \multicolumn{1}{l}{8.13E-01 (6.01E-01)$-$} & \multicolumn{1}{l}{\textbf{6.37E-02 (2.03E-01)$+$}} \\
$F_{8}$ & \multicolumn{1}{l}{\textbf{7.19E-01 (1.02E+00)}} & \multicolumn{1}{l}{2.31E+00 (4.16E+00)$\approx$} & \multicolumn{1}{l}{1.82E+01 (3.82E+00)$-$} & \multicolumn{1}{l}{2.19E+00 (4.05E+00)$\approx$} \\
$F_{9}$ & \multicolumn{1}{l}{\textbf{1.07E+02 (1.98E+01)}} & \multicolumn{1}{l}{1.34E+02 (3.72E+01)$-$} & \multicolumn{1}{l}{1.81E+02 (3.75E-06)$-$} & \multicolumn{1}{l}{1.75E+02 (1.81E+01)$-$} \\
$F_{10}$ & \multicolumn{1}{l}{\textbf{0.00E+00 (0.00E+00)}} & \multicolumn{1}{l}{\boldmath{}\textbf{0.00E+00 (0.00E+00)$\approx$}\unboldmath{}} & \multicolumn{1}{l}{\boldmath{}\textbf{0.00E+00 (0.00E+00)$\approx$}\unboldmath{}} & \multicolumn{1}{l}{7.86E-01 (2.39E+00)$-$} \\
$F_{11}$ & \multicolumn{1}{l}{7.28E-05 (3.06E-04)} & \multicolumn{1}{l}{1.01E-02 (3.90E-02)$\approx$} & \multicolumn{1}{l}{8.79E-03 (1.52E-02)$-$} & \multicolumn{1}{l}{\textbf{1.45E-05 (0.00E+00)$-$}} \\
$F_{12}$ & \multicolumn{1}{l}{\textbf{2.29E+02 (6.08E-01)}} & \multicolumn{1}{l}{2.29E+02 (1.17E+00)$\approx$} & \multicolumn{1}{l}{2.31E+02 (8.77E-01)$-$} & \multicolumn{1}{l}{2.30E+02 (9.60E-01)$-$} \\
\midrule
$+$ / $\approx$ / $-$ & --    & 0/9/3 & 1/4/7 & 2/6/4 \\
\bottomrule
\end{tabular}%
}

\footnotesize
\textsuperscript{*} The Wilcoxon rank-sum tests (with a significance level of 0.05) were conducted between MetaDE and each individually.
The final row displays the number of problems where the corresponding evlover performs statistically better ($+$),  similar ($\thickapprox$), or worse ($-$) compared to DE.\\
\label{tab:diffTunner 20D_supp}%
\end{table}%

\clearpage

\begin{figure*}[htpb]
\centering
\includegraphics[scale=0.29]{su_D10_all_evolver.pdf}
\caption{Convergence curves with different evolvers on 10D problems in CEC2022 benchmark suite.}
\label{Figure_evolver_10D_supp}
\end{figure*}


\begin{figure*}[htpb]
\centering
\includegraphics[scale=0.29]{su_D20_all_evolver.pdf}
\caption{Convergence curves with different evolvers on 20D problems in CEC2022 benchmark suite.}
\label{Figure_evolver_20D_supp}
\end{figure*}


\clearpage
% \bibliography{Supplement_references}

% \bibliographystylesupp{IEEEtran}
% \bibliographysupp{Supplement_references}

%\title{Generating 3D \hl{Small} Binding Molecules Using Shape-Conditioned Diffusion Models with Guidance}
%\date{\vspace{-5ex}}

%\author{
%	Ziqi Chen\textsuperscript{\rm 1}, 
%	Bo Peng\textsuperscript{\rm 1}, 
%	Tianhua Zhai\textsuperscript{\rm 2},
%	Xia Ning\textsuperscript{\rm 1,3,4 \Letter}
%}
%\newcommand{\Address}{
%	\textsuperscript{\rm 1}Computer Science and Engineering, The Ohio Sate University, Columbus, OH 43210.
%	\textsuperscript{\rm 2}Perelman School of Medicine, University of Pennsylvania, Philadelphia, PA 19104.
%	\textsuperscript{\rm 3}Translational Data Analytics Institute, The Ohio Sate University, Columbus, OH 43210.
%	\textsuperscript{\rm 4}Biomedical Informatics, The Ohio Sate University, Columbus, OH 43210.
%	\textsuperscript{\Letter}ning.104@osu.edu
%}

%\newcommand\affiliation[1]{%
%	\begingroup
%	\renewcommand\thefootnote{}\footnote{#1}%
%	\addtocounter{footnote}{-1}%
%	\endgroup
%}



\setcounter{secnumdepth}{2} %May be changed to 1 or 2 if section numbers are desired.

\setcounter{section}{0}
\renewcommand{\thesection}{S\arabic{section}}

\setcounter{table}{0}
\renewcommand{\thetable}{S\arabic{table}}

\setcounter{figure}{0}
\renewcommand{\thefigure}{S\arabic{figure}}

\setcounter{algorithm}{0}
\renewcommand{\thealgorithm}{S\arabic{algorithm}}

\setcounter{equation}{0}
\renewcommand{\theequation}{S\arabic{equation}}


\begin{center}
	\begin{minipage}{0.95\linewidth}
		\centering
		\LARGE 
	Generating 3D Binding Molecules Using Shape-Conditioned Diffusion Models with Guidance (Supplementary Information)
	\end{minipage}
\end{center}
\vspace{10pt}

%%%%%%%%%%%%%%%%%%%%%%%%%%%%%%%%%%%%%%%%%%%%%
\section{Parameters for Reproducibility}
\label{supp:experiments:parameters}
%%%%%%%%%%%%%%%%%%%%%%%%%%%%%%%%%%%%%%%%%%%%%

We implemented both \SE and \methoddiff using Python-3.7.16, PyTorch-1.11.0, PyTorch-scatter-2.0.9, Numpy-1.21.5, Scikit-learn-1.0.2.
%
We trained the models using a Tesla V100 GPU with 32GB memory and a CPU with 80GB memory on Red Hat Enterprise 7.7.
%
%We released the code, data, and the trained model at Google Drive~\footnote{\url{https://drive.google.com/drive/folders/146cpjuwenKGTd6Zh4sYBy-Wv6BMfGwe4?usp=sharing}} (will release to the public on github once the manuscript is accepted).

%===================================================================
\subsection{Parameters of \SE}
%===================================================================


In \SE, we tuned the dimension of all the hidden layers including VN-DGCNN layers
(Eq.~\ref{eqn:shape_embed}), MLP layers (Eq.~\ref{eqn:se:decoder}) and
VN-In layer (Eq.~\ref{eqn:se:decoder}), and the dimension $d_p$ of generated shape latent embeddings $\shapehiddenmat$ with the grid-search algorithm in the 
parameter space presented in Table~\ref{tbl:hyper_se}.
%
We determined the optimal hyper-parameters according to the mean squared errors of the predictions of signed distances for 1,000 validation molecules that are selected as described in Section ``Data'' 
in the main manuscript.
%
The optimal dimension of all the hidden layers is 256, and the optimal dimension $d_p$ of shape latent embedding \shapehiddenmat is 128.
%
The optimal number of points $|\pc|$ in the point cloud \pc is 512.
%
We sampled 1,024 query points in $\mathcal{Z}$ for each molecule shape.
%
We constructed graphs from point clouds, which are employed to learn $\shapehiddenmat$ with VN-DGCNN layer (Eq.~\ref{eqn:shape_embed}), using the $k$-nearest neighbors based on Euclidean distance with $k=20$.
%
We set the number of VN-DGCNN layers as 4.
%
We set the number of MLP layers in the decoder (Eq.~\ref{eqn:se:decoder}) as 2.
%
We set the number of VN-In layers as 1.

%
We optimized the \SE model via Adam~\cite{adam} with its parameters (0.950, 0.999), %betas (0.95, 0.999), 
learning rate 0.001, and batch size 16.
%
We evaluated the validation loss every 2,000 training steps.
%
We scheduled to decay the learning rate with a factor of 0.6 and a minimum learning rate of 1e-6 if 
the validation loss does not decrease in 5 consecutive evaluations.
%
The optimal \SE model has 28.3K learnable parameters. 
%
We trained the \SE model %for at most 80 hours 
with $\sim$156,000 training steps.
%
The training took 80 hours with our GPUs.
%
The trained \SE model achieved the minimum validation loss at 152,000 steps.


\begin{table*}[!h]
  \centering
      \caption{{Hyper-Parameter Space for \SE Optimization}}
  \label{tbl:hyper_se}
  \begin{threeparttable}
 \begin{scriptsize}
      \begin{tabular}{
%	@{\hspace{2pt}}l@{\hspace{2pt}}
	@{\hspace{2pt}}l@{\hspace{5pt}} 
	@{\hspace{2pt}}r@{\hspace{2pt}}         
	}
        \toprule
        %Notation &
          Hyper-parameters &  Space\\
        \midrule
        %$t_a$    & 
         %hidden layer dimension         & \{16, 32, 64, 128\} \\
         %atom/node embedding dimension &  \{16, 32, 64, 128\} \\
         %$\latent^{\add}$/$\latent^{\delete}$ dimension        & \{8, 16, 32, 64\} \\
         hidden layer dimension            & \{128, 256\}\\
         dimension $d_p$ of \shapehiddenmat        &  \{64, 128\} \\
         \#points in \pc        & \{512, 1,024\} \\
         \#query points in $\mathcal{Z}$                & 1,024 \\%1024 \\%\bo{\{1024\}}\\
         \#nearest neighbors              & 20          \\
         \#VN-DGCNN layers (Eq~\ref{eqn:shape_embed})               & 4            \\
         \#MLP layers in Eq~\ref{eqn:se:decoder} & 4           \\
        \bottomrule
      \end{tabular}
%  	\begin{tablenotes}[normal,flushleft]
%  		\begin{footnotesize}
%  	
%  	\item In this table, hidden dimension represents the dimension of hidden layers and 
%  	atom/node embeddings; latent dimension represents the dimension of latent embedding \latent.
%  	\par
%  \end{footnotesize}
%  
%\end{tablenotes}
%      \begin{tablenotes}
%      \item 
%      \par
%      \end{tablenotes}
\end{scriptsize}
  \end{threeparttable}
\end{table*}

%
\begin{table*}[!h]
  \centering
      \caption{{Hyper-Parameter Space for \methoddiff Optimization}}
  \label{tbl:hyper_diff}
  \begin{threeparttable}
 \begin{scriptsize}
      \begin{tabular}{
%	@{\hspace{2pt}}l@{\hspace{2pt}}
	@{\hspace{2pt}}l@{\hspace{5pt}} 
	@{\hspace{2pt}}r@{\hspace{2pt}}         
	}
        \toprule
        %Notation &
          Hyper-parameters &  Space\\
        \midrule
        %$t_a$    & 
         %hidden layer dimension         & \{16, 32, 64, 128\} \\
         %atom/node embedding dimension &  \{16, 32, 64, 128\} \\
         %$\latent^{\add}$/$\latent^{\delete}$ dimension        & \{8, 16, 32, 64\} \\
         scalar hidden layer dimension         & 128 \\
         vector hidden layer dimension         & 32 \\
         weight of atom type loss $\xi$ (Eq.~\ref{eqn:loss})  & 100           \\
         threshold of step weight $\delta$ (Eq.~\ref{eqn:diff:obj:pos}) & 10 \\
         \#atom features $K$                   & 15 \\
         \#layers $L$ in \molpred             & 8 \\
         %\# \eqgnn/\invgnn layers     &  8 \\
         %\# heads {$n_h$} in $\text{MHA}^{\mathtt{x}}/\text{MHA}^{\mathtt{v}}$                               & 16 \\
         \#nearest neighbors {$N$}  (Eq.~\ref{eqn:geometric_embedding} and \ref{eqn:attention})            & 8          \\
         {\#diffusion steps $T$}                  & 1,000 \\
        \bottomrule
      \end{tabular}
%  	\begin{tablenotes}[normal,flushleft]
%  		\begin{footnotesize}
%  	
%  	\item In this table, hidden dimension represents the dimension of hidden layers and 
%  	atom/node embeddings; latent dimension represents the dimension of latent embedding \latent.
%  	\par
%  \end{footnotesize}
%  
%\end{tablenotes}
%      \begin{tablenotes}
%      \item 
%      \par
%      \end{tablenotes}
\end{scriptsize}
  \end{threeparttable}

\end{table*}


%===================================================================
\subsection{Parameters of \methoddiff}
%===================================================================

Table~\ref{tbl:hyper_diff} presents the parameters used to train \methoddiff.
%
In \methoddiff, we set the hidden dimensions of all the MLP layers and the scalar hidden layers in GVPs (Eq.~\ref{eqn:pred:gvp} and Eq.~\ref{eqn:mess:gvp}) as 128. %, including all the MLP layers in \methoddiff and the scalar dimension of GVP layers in Eq.~\ref{eqn:pred:gvp} and Eq.~\ref{eqn:mess:gvp}. %, and MLP layer (Eq.~\ref{eqn:diff:graph:atompred}) as 128.
%
We set the dimensions of all the vector hidden layers in GVPs as 32.
%
We set the number of layers $L$ in \molpred as 8.
%and the number of layers in graph neural networks as 8.
%
Both two GVP modules in Eq.~\ref{eqn:pred:gvp} and Eq.~\ref{eqn:mess:gvp} consist of three GVP layers. %, which consisa GVP modset the number of layer of GVP modules %is a multi-head attention layer ($\text{MHA}^{\mathtt{x}}$ or $\text{MHA}^{\mathtt{h}}$) with 16 heads.
% 
We set the number of VN-MLP layers in Eq.~\ref{eqn:shaper} as 1 and the number of MLP layers as 2 for all the involved MLP functions.
%

We constructed graphs from atoms in molecules, which are employed to learn the scalar embeddings and vector embeddings for atoms %predict atom coordinates and features  
(Eq.~\ref{eqn:geometric_embedding} and \ref{eqn:attention}), using the $N$-nearest neighbors based on Euclidean distance with $N=8$. 
%
We used $K=15$ atom features in total, indicating the atom types and its aromaticity.
%
These atom features include 10 non-aromatic atoms (i.e., ``H'', ``C'', ``N'', ``O'', ``F'', ``P'', ``S'', ``Cl'', ``Br'', ``I''), 
and 5 aromatic atoms (i.e., ``C'', ``N'', ``O'', ``P'', ``S'').
%
We set the number of diffusion steps $T$ as 1,000.
%
We set the weight $\xi$ of atom type loss (Eq.~\ref{eqn:loss}) as $100$ to balance the values of atom type loss and atom coordinate loss.
%
We set the threshold $\delta$ (Eq.~\ref{eqn:diff:obj:pos}) as 10.
%
The parameters $\beta_t^{\mathtt{x}}$ and $\beta_t^{\mathtt{v}}$ of variance scheduling in the forward diffusion process of \methoddiff are discussed in 
Supplementary Section~\ref{supp:forward:variance}.
%
%Please note that as in \squid, we did not perform extensive hyperparameter optimization for \methoddiff.
%
Following \squid, we did not perform extensive hyperparameter tunning for \methoddiff given that the used 
hyperparameters have enabled good performance.

%
We optimized the \methoddiff model via Adam~\cite{adam} with its parameters (0.950, 0.999), learning rate 0.001, and batch size 32.
%
We evaluated the validation loss every 2,000 training steps.
%
We scheduled to decay the learning rate with a factor of 0.6 and a minimum learning rate of 1e-5 if 
the validation loss does not decrease in 10 consecutive evaluations.
%
The \methoddiff model has 7.8M learnable parameters. 
%
We trained the \methoddiff model %for at most 60 hours 
with $\sim$770,000 training steps.
%
The training took 70 hours with our GPUs.
%
The trained \methoddiff achieved the minimum validation loss at 758,000 steps.

During inference, %the sampling, 
following Adams and Coley~\cite{adams2023equivariant}, we set the variance $\phi$ 
of atom-centered Gaussians as 0.049, which is used to build a set of points for shape guidance in Section ``\method with Shape Guidance'' 
in the main manuscript.
%
We determined the number of atoms in the generated molecule using the atom number distribution of training molecules that have surface shape sizes similar to the condition molecule.
%
The optimal distance threshold $\gamma$ is 0.2, and the optimal stop step $S$ for shape guidance is 300.
%
With shape guidance, each time we updated the atom position (Eq.~\ref{eqn:shape_guidance}), we randomly sampled the weight $\sigma$ from $[0.2, 0.8]$. %\bo{(XXX)}.
%
Moreover, when using pocket guidance as mentioned in Section ``\method with Pocket Guidance'' in the main manuscript, each time we updated the atom position (Eq.~\ref{eqn:pocket_guidance}), we randomly sampled the weight $\epsilon$ from $[0, 0.5]$. 
%
For each condition molecule, it took around 40 seconds on average to generate 50 molecule candidates with our GPUs.



%%%%%%%%%%%%%%%%%%%%%%%%%%%%%%%%%%%%%%%%%%%%%%
\section{Performance of \decompdiff with Protein Pocket Prior}
\label{supp:app:decompdiff}
%%%%%%%%%%%%%%%%%%%%%%%%%%%%%%%%%%%%%%%%%%%%%%

In this section, we demonstrate that \decompdiff with protein pocket prior, referred to as \decompdiffbeta, shows very limited performance in generating drug-like and synthesizable molecules compared to all the other methods, including \methodwithpguide and \methodwithsandpguide.
%
We evaluate the performance of \decompdiffbeta in terms of binding affinities, drug-likeness, and diversity.
%
We compare \decompdiffbeta with \methodwithpguide and \methodwithsandpguide and report the results in Table~\ref{tbl:comparison_results_decompdiff}.
%
Note that the results of \methodwithpguide and \methodwithsandpguide here are consistent with those in Table~\ref{tbl:overall_results_docking2} in the main manuscript.
%
As shown in Table~\ref{tbl:comparison_results_decompdiff}, while \decompdiffbeta achieves high binding affinities in Vina M and Vina D, it substantially underperforms \methodwithpguide and \methodwithsandpguide in QED and SA.
%
Particularly, \decompdiffbeta shows a QED score of 0.36, while \methodwithpguide substantially outperforms \decompdiffbeta in QED (0.77) with 113.9\% improvement.
%
\decompdiffbeta also substantially underperforms \methodwithpguide in terms of SA scores (0.55 vs 0.76).
%
These results demonstrate the limited capacity of \decompdiffbeta in generating drug-like and synthesizable molecules.
%
As a result, the generated molecules from \decompdiffbeta can have considerably lower utility compared to other methods.
%
Considering these limitations of \decompdiffbeta, we exclude it from the baselines for comparison.

\begin{table*}[!h]
	\centering
		\caption{Comparison on PMG among \methodwithpguide, \methodwithsandpguide and \decompdiffbeta}
	\label{tbl:comparison_results_decompdiff}
\begin{threeparttable}
	\begin{scriptsize}
	\begin{tabular}{
		@{\hspace{2pt}}l@{\hspace{2pt}}
		%
		%@{\hspace{2pt}}l@{\hspace{2pt}}
		%
		@{\hspace{2pt}}r@{\hspace{2pt}}
		@{\hspace{2pt}}r@{\hspace{2pt}}
		%
		@{\hspace{6pt}}r@{\hspace{6pt}}
		%
		@{\hspace{2pt}}r@{\hspace{2pt}}
		@{\hspace{2pt}}r@{\hspace{2pt}}
		%
		@{\hspace{5pt}}r@{\hspace{5pt}}
		%
		@{\hspace{2pt}}r@{\hspace{2pt}}
		@{\hspace{2pt}}r@{\hspace{2pt}}
		%
		@{\hspace{5pt}}r@{\hspace{5pt}}
		%
		@{\hspace{2pt}}r@{\hspace{2pt}}
	         @{\hspace{2pt}}r@{\hspace{2pt}}
		%
		@{\hspace{5pt}}r@{\hspace{5pt}}
		%
		@{\hspace{2pt}}r@{\hspace{2pt}}
		@{\hspace{2pt}}r@{\hspace{2pt}}
		%
		@{\hspace{5pt}}r@{\hspace{5pt}}
		%
		@{\hspace{2pt}}r@{\hspace{2pt}}
		@{\hspace{2pt}}r@{\hspace{2pt}}
		%
		@{\hspace{5pt}}r@{\hspace{5pt}}
		%
		@{\hspace{2pt}}r@{\hspace{2pt}}
		@{\hspace{2pt}}r@{\hspace{2pt}}
		%
		@{\hspace{5pt}}r@{\hspace{5pt}}
		%
		@{\hspace{2pt}}r@{\hspace{2pt}}
		%@{\hspace{2pt}}r@{\hspace{2pt}}
		%@{\hspace{2pt}}r@{\hspace{2pt}}
		}
		\toprule
		\multirow{2}{*}{method} & \multicolumn{2}{c}{Vina S$\downarrow$} & & \multicolumn{2}{c}{Vina M$\downarrow$} & & \multicolumn{2}{c}{Vina D$\downarrow$} & & \multicolumn{2}{c}{{HA\%$\uparrow$}}  & & \multicolumn{2}{c}{QED$\uparrow$} & & \multicolumn{2}{c}{SA$\uparrow$} & & \multicolumn{2}{c}{Div$\uparrow$} & %& \multirow{2}{*}{SR\%$\uparrow$} & 
		& \multirow{2}{*}{time$\downarrow$} \\
	    \cmidrule{2-3}\cmidrule{5-6} \cmidrule{8-9} \cmidrule{11-12} \cmidrule{14-15} \cmidrule{17-18} \cmidrule{20-21}
		& Avg. & Med. &  & Avg. & Med. &  & Avg. & Med. & & Avg. & Med.  & & Avg. & Med.  & & Avg. & Med.  & & Avg. & Med.  & & \\ %& & \\
		%\multirow{2}{*}{method} & \multirow{2}{*}{\#c\%} &  \multirow{2}{*}{\#u\%} &  \multirow{2}{*}{QED} & \multicolumn{3}{c}{$\nmax=50$} & & \multicolumn{2}{c}{$\nmax=1$}\\
		%\cmidrule(r){5-7} \cmidrule(r){8-10} 
		%& & & & \avgshapesim(std) & \avggraphsim(std  &  \diversity(std  & & \avgshapesim(std) & \avggraphsim(std \\
		\midrule
		%Reference                          & -5.32 & -5.66 & & -5.78 & -5.76 & & -6.63 & -6.67 & & - & - & & 0.53 & 0.49 & & 0.77 & 0.77 & & - & - & %& 23.1 & & - \\
		%\midrule
		%\multirow{4}{*}{PM} 
		%& \AR & -5.06 & -4.99 & &  -5.59 & -5.29 & &  -6.16 & -6.05 & &  37.69 & 31.00 & &  0.50 & 0.49 & &  0.66 & 0.65 & & - & - & %& 7.0 & 
		%& 7,789 \\
		%& \pockettwomol   & -4.50 & -4.21 & &  -5.70 & -5.27 & &  -6.43 & -6.25 & &  48.00 & 51.00 & &  0.58 & 0.58 & &  \textbf{0.77} & \textbf{0.78} & &  0.69 & 0.71 &  %& 24.9 & 
		%& 2,544 \\
		%& \targetdiff     & -4.88 & \underline{-5.82} & &  -6.20 & \underline{-6.36} & &  \textbf{-7.37} & \underline{-7.51} & &  57.57 & 58.27 & &  0.50 & 0.51 & &  0.60 & 0.59 & &  0.72 & 0.71 & % & 10.4 & 
		%& 1,252 \\
		 \decompdiffbeta             & -4.72 & -4.86 & & \textbf{-6.84} & \textbf{-6.91} & & \textbf{-8.85} & \textbf{-8.90} & &  {72.16} & {72.16} & &  0.36 & 0.36 & &  0.55 & 0.55 & & 0.59 & 0.59 & & 3,549 \\ 
		%-4.76 & -6.18 & &  \textbf{-6.86} & \textbf{-7.52} & &  \textbf{-8.85} & \textbf{-8.96} & &  \textbf{72.7} & \textbf{89.8} & &  0.36 & 0.34 & &  0.55 & 0.57 & & 0.59 & 0.59 & & 15.4 \\
		%& \decompdiffref  & -4.58 & -4.77 & &  -5.47 & -5.51 & &  -6.43 & -6.56 & &  47.76 & 48.66 & &  0.56 & 0.56 & &  0.70 & 0.69  & &  0.72 & 0.72 &  %& 15.2 & 
		%& 1,859 \\
		%\midrule
		%\multirow{2}{*}{PC}
		\methodwithpguide       &  \underline{-5.53} & \underline{-5.64} & & {-6.37} & -6.33 & &  \underline{-7.19} & \underline{-7.52} & &  \underline{78.75} & \textbf{94.00} & &  \textbf{0.77} & \textbf{0.80} & &  \textbf{0.76} & \textbf{0.76} & & 0.63 & 0.66 & & 462 \\
		\methodwithsandpguide   & \textbf{-5.81} & \textbf{-5.96} & &  \underline{-6.50} & \underline{-6.58} & & -7.16 & {-7.51} & &  \textbf{79.92} & \underline{93.00} & &  \underline{0.76} & \underline{0.79} & &  \underline{0.75} & \underline{0.74} & & 0.64 & 0.66 & & 561\\
		\bottomrule
	\end{tabular}%
	\begin{tablenotes}[normal,flushleft]
		\begin{footnotesize}
	\item 
\!\!Columns represent: {``Vina S'': the binding affinities between the initially generated poses of molecules and the protein pockets; 
		``Vina M'': the binding affinities between the poses after local structure minimization and the protein pockets;
		``Vina D'': the binding affinities between the poses determined by AutoDock Vina~\cite{Eberhardt2021} and the protein targets;
		``QED'': the drug-likeness score;
		``SA'': the synthesizability score;
		``Div'': the diversity among generated molecules;
		``time'': the time cost to generate molecules.}
		
		\par
		\par
		\end{footnotesize}
	\end{tablenotes}
	\end{scriptsize}
\end{threeparttable}
  \vspace{-10pt}    
\end{table*}



%===================================================================
\section{{Additional Experimental Results on SMG}}
\label{supp:app:results}
%===================================================================

%-------------------------------------------------------------------------------------------------------------------------------------
\subsection{Comparison on Shape and Graph Similarity}
\label{supp:app:results:overall_shape}
%-------------------------------------------------------------------------------------------------------------------------------------

%\ziqi{Outline for this section:
%	\begin{itemize}
%		\item \method can consistently generate molecules with novel structures (low graph similarity) and similar shapes (high shape similarity), such that these molecules have comparable binding capacity with the condition molecules, and potentially better properties as will be shown in Table~\ref{tbl:overall_results_quality_10}.
%	\end{itemize}
%}

\begin{table*}[!h]
	\centering
		\caption{Similarity Comparison on SMG}
	\label{tbl:overall_sim}
\begin{threeparttable}
	\begin{scriptsize}
	\begin{tabular}{
		@{\hspace{0pt}}l@{\hspace{8pt}}
		%
		@{\hspace{8pt}}l@{\hspace{8pt}}
		%
		@{\hspace{8pt}}c@{\hspace{8pt}}
		@{\hspace{8pt}}c@{\hspace{8pt}}
		%
	    	@{\hspace{0pt}}c@{\hspace{0pt}}
		%
		@{\hspace{8pt}}c@{\hspace{8pt}}
		@{\hspace{8pt}}c@{\hspace{8pt}}
		%
		%@{\hspace{8pt}}r@{\hspace{8pt}}
		}
		\toprule
		$\delta_g$  & method          & \avgshapesim$\uparrow$(std) & \avggraphsim$\downarrow$(std) & & \maxshapesim$\uparrow$(std) & \maxgraphsim$\downarrow$(std)       \\ %& \#n\%$\uparrow$  \\ 
		\midrule
		%\multirow{5}{0.079\linewidth}%{\hspace{0pt}0.1} & \dataset   & 0.0             & 0.628(0.139)          & 0.567(0.068)          & 0.078(0.010)          &  & 0.588(0.086)          & 0.081(0.013)          & 4.7              \\
		%&  \squid($\lambda$=0.3) & 0.0             & 0.320(0.000)          & 0.420(0.163)          & \textbf{0.056}(0.032) &  & 0.461(0.170)          & \textbf{0.065}(0.033) & 1.4              \\
		%& \squid($\lambda$=1.0) & 0.0             & 0.414(0.177)          & 0.483(0.184)          & \underline{0.064}(0.030)  &  & 0.531(0.182)          & \underline{0.073}(0.029)  & 2.4              \\
		%& \method               & \underline{1.6}     & \textbf{0.857}(0.034) & \underline{0.773}(0.045)  & 0.086(0.011)          &  & \underline{0.791}(0.053)  & 0.087(0.012)          & \underline{5.1}      \\
		%& \methodwithsguide      & \textbf{3.7}    & \underline{0.833}(0.062)  & \textbf{0.812}(0.037) & 0.088(0.009)          &  & \textbf{0.835}(0.047) & 0.089(0.010)          & \textbf{6.2}     \\ 
		%\cmidrule{2-10}
		%& improv\% & - & 36.5 & 43.2 & -53.6 &  & 42.0 & -33.8 & 31.9  \\
		%\midrule
		\multirow{6}{0.059\linewidth}{\hspace{0pt}0.3} & \dataset             & 0.745(0.037)          & \textbf{0.211}(0.026) &  & 0.815(0.039)          & \textbf{0.215}(0.047)      \\ %    & \textbf{100.0}   \\
			& \squid($\lambda$=0.3) & 0.709(0.076)          & 0.237(0.033)          &  & 0.841(0.070)          & 0.253(0.038)        \\ %  & 45.5             \\
		    & \squid($\lambda$=1.0) & 0.695(0.064)          & \underline{0.216}(0.034)  &  & 0.841(0.056)          & 0.231(0.047)        \\ %  & 84.3             \\
			& \method               & \underline{0.770}(0.039)  & 0.217(0.031)          &  & \underline{0.858}(0.038)  & \underline{0.220}(0.046)  \\ %& \underline{87.1}     \\
			& \methodwithsguide     & \textbf{0.823}(0.029) & 0.217(0.032)          &  & \textbf{0.900}(0.028) & 0.223(0.048)  \\ % & 86.0             \\ 
		%\cmidrule{2-7}
		%& improv\% & 10.5 & -2.8 &  & 7.0 & -2.3  \\ % & %-12.9  \\
		\midrule
		\multirow{6}{0.059\linewidth}{\hspace{0pt}0.5} & \dataset & 0.750(0.037)          & \textbf{0.225}(0.037) &  & 0.819(0.039)          & \textbf{0.236}(0.070)          \\ %& \textbf{100.0}   \\
			& \squid($\lambda$=0.3)  & 0.728(0.072)          & 0.301(0.054)          &  & \underline{0.888}(0.061)  & 0.355(0.088)          \\ %& 85.9             \\
			& \squid($\lambda$=1.0)  & 0.699(0.063)          & 0.233(0.043)          &  & 0.850(0.057)          & 0.263(0.080)          \\ %& \underline{99.5}     \\
			& \method               & \underline{0.771}(0.039)  & \underline{0.229}(0.043)  &  & 0.862(0.036)          & \textbf{0.236}(0.065) \\ %& 99.2             \\
			& \methodwithsguide    & \textbf{0.824}(0.029) & \underline{0.229}(0.044)  &  & \textbf{0.903}(0.027) & \underline{0.242}(0.069)  \\ %& 99.0             \\ 
		%\cmidrule{2-7}
		%& improv\% & 9.9 & -1.8 &  & 1.7 & 0.0 \\ %& -0.8  \\
		\midrule
		\multirow{6}{0.059\linewidth}{\hspace{0pt}0.7} 
		& \dataset &  0.750(0.037) & \textbf{0.226}(0.038) & & 0.819(0.039) & \underline{0.240}(0.081) \\ %& \textbf{100.0} \\
		%& \dataset & 12.3            & 0.736(0.076)          & 0.768(0.037)          & \textbf{0.228}(0.042) &  & 0.819(0.039)          & \underline{0.242}(0.085)  & \textbf{100.0}   \\
			& \squid($\lambda$=0.3) &  0.735(0.074)          & 0.328(0.070)          &  & \underline{0.900}(0.062)  & 0.435(0.143)          \\ %& 95.4             \\
			& \squid($\lambda$=1.0) &  0.699(0.064)          & 0.234(0.045)          &  & 0.851(0.057)          & 0.268(0.090)          \\ %& \underline{99.9}     \\
			& \method               &  \underline{0.771}(0.039)  & \underline{0.229}(0.043)  &  & 0.862(0.036)          & \textbf{0.237}(0.066) \\ %& 99.3             \\
			& \methodwithsguide     &  \textbf{0.824}(0.029) & 0.230(0.045)          &  & \textbf{0.903}(0.027) & 0.244(0.074)          \\ %& 99.2             \\ 
		%\cmidrule{2-7}
		%& improv\% & 9.9 & -1.3 &  & 0.3 & 1.3 \\%& -0.7  \\
		\midrule
		\multirow{6}{0.059\linewidth}{\hspace{0pt}1.0} 
		& \dataset & 0.750(0.037)          & \textbf{0.226}(0.038) &  & 0.819(0.039)          & \underline{0.242}(0.085)  \\%& \textbf{100.0}  \\
		& \squid($\lambda$=0.3) & 0.740(0.076)          & 0.349(0.088)          &  & \textbf{0.909}(0.065) & 0.547(0.245)       \\ %   & \textbf{100.0}  \\
		& \squid($\lambda$=1.0) & 0.699(0.064)          & 0.235(0.045)          &  & 0.851(0.057)          & 0.271(0.097)          \\ %& \textbf{100.0}   \\
		& \method               & \underline{0.771}(0.039)  & \underline{0.229}(0.043)  &  & 0.862(0.036)          & \textbf{0.237}(0.066) \\ %& \underline{99.3}  \\
		& \methodwithsguide      & \textbf{0.824}(0.029) & 0.230(0.045)          &  & \underline{0.903}(0.027)  & 0.244(0.076)          \\ %& 99.2            \\
		%\cmidrule{2-7}
		%& improv\% &  9.9               & -1.3              &  & -0.7              & -2.1           \\ %       & -0.7 \\
		\bottomrule
	\end{tabular}%
	\begin{tablenotes}[normal,flushleft]
		\begin{footnotesize}
	\item 
\!\!Columns represent: ``$\delta_g$'': the graph similarity constraint; 
%``\#d\%'': the percentage of molecules that satisfy the graph similarity constraint and are with high \shapesim ($\shapesim>=0.8$);
%``\diversity'': the diversity among the generated molecules;
``\avgshapesim/\avggraphsim'': the average of shape or graph similarities between the condition molecules and generated molecules with $\graphsim<=\delta_g$;
``\maxshapesim'': the maximum of shape similarities between the condition molecules and generated molecules with $\graphsim<=\delta_g$;
``\maxgraphsim'': the graph similarities between the condition molecules and the molecules with the maximum shape similarities and $\graphsim<=\delta_g$;
%``\#n\%'': the percentage of molecules that satisfy the graph similarity constraint ($\graphsim<=\delta_g$).
%
``$\uparrow$'' represents higher values are better, and ``$\downarrow$'' represents lower values are better.
%
 Best values are in \textbf{bold}, and second-best values are \underline{underlined}. 
\par
		\par
		\end{footnotesize}
	\end{tablenotes}
\end{scriptsize}
\end{threeparttable}
  \vspace{-10pt}    
\end{table*}
%\label{tbl:overall_sim}


{We evaluate the shape similarity \shapesim and graph similarity \graphsim of molecules generated from}
%Table~\ref{tbl:overall_sim} presents the comparison of shape-conditioned molecule generation among 
\dataset, \squid, \method and \methodwithsguide under different graph similarity constraints  ($\delta_g$=1.0, 0.7, 0.5, 0.3). 
%
%During the evaluation, for each molecule in the test set, all the methods are employed to generate or identify 50 molecules with similar shapes.
%
We calculate evaluation metrics using all the generated molecules satisfying the graph similarity constraints.
%
Particularly, when $\delta_g$=1.0, we do not filter out any molecules based on the constraints and directly calculate metrics on all the generated molecules.
%
When $\delta_g$=0.7, 0.5 or 0.3, we consider only generated molecules with similarities lower than $\delta_g$.
%
Based on \shapesim and \graphsim as described in Section ``Evaluation Metrics'' in the main manuscript,
we calculate the following metrics using the subset of molecules with \graphsim lower than $\delta_g$, from a set of 50 generated molecules for each test molecule and report the average of  these metrics across all test molecules:
%
(1) \avgshapesim\ measures the average \shapesim across each subset of generated molecules with $\graphsim$ lower than $\delta_g$; %per test molecule, with the overall average calculated across all test molecules; }%the 50 generated molecules for each test molecule, averaged across all test molecules;
(2) \avggraphsim\ calculates the average \graphsim for each set; %, with these means averaged across all test molecules}; %} 50 molecules, %\bo{@Ziqi rephrase}, with results averaged on the test set;\ziqi{with the average computed over the test set; }
(3) \maxshapesim\ determines the maximum \shapesim within each set; %, with these maxima averaged across all test molecules; }%\hl{among 50 molecules}, averaged across all test molecules;
(4) \maxgraphsim\ measures the \graphsim of the molecule with maximum \shapesim in each set. %, averaged across all test molecules; }%\hl{among 50 molecules}, averaged across all test molecules;

%
As shown in Table~\ref{tbl:overall_sim}, \method and \methodwithsguide demonstrate outstanding performance in terms of the average shape similarities (\avgshapesim) and the average graph similarities (\avggraphsim) among generated molecules.
%
%\ziqi{
%Table~\ref{tbl:overall} also shows that \method and \methodwithsguide consistently outperform all the baseline methods in average shape similarities (\avgshapesim) and only slightly underperform 
%the best baseline \dataset in average graph similarities (\avggraphsim).
%}
%
Specifically, when $\delta_g$=0.3, \methodwithsguide achieves a substantial 10.5\% improvement in \avgshapesim\ over the best baseline \dataset. 
%
In terms of \avggraphsim, \methodwithsguide also achieves highly comparable performance with \dataset (0.217 vs 0.211, in \avggraphsim, lower values indicate better performance).
%
%This trend remains consistent across various $\delta_g$ values.
This trend remains consistent when applying various similarity constraints (i.e., $\delta_g$) as shown in Table~\ref{tbl:overall_sim}.


Similarly, \method and \methodwithsguide demonstrate superior performance in terms of the average maximum shape similarity across generated molecules for all test molecules (\maxshapesim), as well as the average graph similarity of the molecules with the maximum shape similarities (\maxgraphsim). %maximum shape similarities of generated molecules (\maxshapesim) and the average graph similarities of molecules with the maximum shape similarities (\maxgraphsim). %\bo{\maxgraphsim is misleading... how about $\text{avgMSim}_\text{g}$}
%
%\bo{
%in terms of the maximum shape similarities (\maxshapesim) and the maximum graph similarities (\maxgraphsim) among all the generated molecules.
%@Ziqi are the metrics maximum values or the average of maximum values?
%}
%
Specifically, at \maxshapesim, Table~\ref{tbl:overall_sim} shows that \methodwithsguide outperforms the best baseline \squid ($\lambda$=0.3) when $\delta_g$=0.3, 0.5, and 0.7, and only underperforms
it by 0.7\% when $\delta$=1.0.
%
We also note that the molecules generated by {\methodwithsguide} with the maximum shape similarities have substantially lower graph similarities ({\maxgraphsim}) compared to those generated by {\squid} ({$\lambda$}=0.3).
%\hl{We also note that the molecules with the maximum shape similarities generated by {\methodwithsguide} are with significantly lower graph similarities ({\maxgraphsim}) than those generated by {\squid} ({$\lambda$}=0.3).}
%
%\bo{@Ziqi please rephrase the language}
%
%\bo{
%@Ziqi the conclusion is not obvious. You may want to remind the meaning of \maxshapesim and \maxgraphsim here, and based on what performance you say this.
%}
%
%\bo{\st{This also underscores the ability of {\methodwithsguide} in generating molecules with similar shapes to condition molecules and novel graph structures.}}
%
As evidenced by these results, \methodwithsguide features strong capacities of generating molecules with similar shapes yet novel graph structures compared to the condition molecule, facilitating the discovery of promising drug candidates.
%

\begin{comment}
\ziqi{replace \#n\% with the percentage of novel molecules that do not exist in the dataset and update the discussion accordingly}
%\ziqi{
Table~\ref{tbl:overall_sim} also presents \bo{\#n\%}, the percentage of molecules generated by each method %\st{(\#n\%)} 
with graph similarities lower than the constraint $\delta_g$. 
%
%\bo{
%Table~\ref{tbl:overall_sim} also presents \#n\%, the percentage of generated molecules with graph similarities lower than the constraint $\delta_g$, of different methods. 
%}
%
As shown in Table~\ref{tbl:overall_sim},  when a restricted constraint (i.e., $\delta_g$=0.3) is applied, \method and \methodwithsguide could still generate a sufficient number of molecules satisfying the constraint.
%
Particularly, when $\delta_g$=0.3, \method outperforms \squid with $\lambda$=0.3 by XXX and \squid with $\lambda$=1.0 by XXX.
% achieve the second and the third in \#n\% and only underperform the best baseline \dataset.
%
This demonstrates the ability of \method in generating molecules with novel structures. 
%
When $\delta_g$=0.5, 0.7 and 1.0, both methods generate over 99.0\% of molecules satisfying the similarity constraint $\delta_g$.
%
%Note that \dataset is guaranteed to identify at least 50 molecules satisfying the $\delta_g$ by searching within a training dataset of diverse molecules.
%
Note that \dataset is a search algorithm that always first identifies the molecules satisfying $\delta_g$ and then selects the top-50 molecules of the highest shape similarities among them. 
%
Due to the diverse molecules in %\hl{the subset} \bo{@Ziqi why do you want to stress subset?} of 
the training set, \dataset can always identify at least 50 molecules under different $\delta_g$ and thus achieve 100\% in \#n\%.
%
%\bo{
%Note that \dataset is a search algorithm that always generate molecules XXX
%@Ziqi
%We need to discuss here. For \dataset, \#n\% in this table does not look aligned with that in Fig 1 if the highlighted defination is correct...
%}
%
%Thus, \dataset achieves 100.0\% in \#n\% under different $\delta_g$.
%
It is also worth noting that when $\delta_g$=1.0, \#n\% reflects the validity among all the generated molecules. 
%
As shown in Table~\ref{tbl:overall_sim}, \method and \methodwithsguide are able to generate 99.3\% and 99.2\% valid molecules.
%
This demonstrates their ability to effectively capture the underlying chemical rules in a purely data-driven manner without relying on any prior knowledge (e.g., fragments) as \squid does.
%
%\bo{
%@Ziqi I feel this metric is redundant with the avg graph similarity when constraint is 1.0. Generally, if the avg similarity is small. You have more mols satisfying the requirement right?
%}
\end{comment}

Table~\ref{tbl:overall_sim} also shows that by incorporating shape guidance, \methodwithsguide
%\bo{
%@Ziqi where does this come from...
%}
substantially outperforms \method in both \avgshapesim and \maxshapesim, while maintaining comparable graph similarities (i.e., \avggraphsim\ and \maxgraphsim).
%
Particularly, when $\delta_g$=0.3, \methodwithsguide 
establishes a considerable improvement of 6.9\% and 4.9\%
%\bo{\st{achieves 6.9\% and 4.9\% improvements}} 
over \method in \avgshapesim and \maxshapesim, respectively. 
%
%\hl{In the meanwhile}, 
%\bo{@Ziqi it is not the right word...}
Meanwhile, \methodwithsguide achieves the same \avggraphsim with \method and only slightly underperforms \method in \maxgraphsim (0.223 vs 0.220).
%\bo{
%XXX also achieves XXX
%}
%it maintains the same \avggraphsim\ with \method and only slightly underperforms \method in \maxgraphsim (0.223 vs 0.220).
%
%Compared with \method, \methodwithsguide consistently generates molecules with higher shape similarities while maintaining comparable graph similarities.
%
%\bo{
%@Ziqi you may want to highlight the utility of "generating molecules with higher shape similarities while maintaining comparable graph similarities" in real drug discovery applications.
%
%
%\bo{
%@Ziqi You did not present the details of method yet...
%}
%
%\methodwithsguide leverages additional shape guidance to push the predicted atoms to the shape of condition molecules \bo{and XXX (@Ziqi boosts the shape similarities XXX)} , as will be discussed in Section ``\method with Shape Guidance'' later.
%
The superior performance of \methodwithsguide suggests that the incorporation of shape guidance effectively boosts the shape similarities of generated molecules without compromising graph similarities.
%
%This capability could be crucial in drug discovery, 
%\bo{@Ziqi it is a strong statement. Need citations here}, 
%as it enables the discovery of drug candidates that are both more potentially effective due to the improved shape similarities and novel induced by low graph similarities.
%as it could enable the identification of candidates with similar binding patterns %with the condition molecule (i.e., high shape similarities) 
%(i.e., high shape similarities) and graph structures distinct from the condition molecules (i.e., low graph similarities).
%\bo{\st{and enjoys novel structures (i.e., low graph similarities) with potentially better properties. } \ziqi{change enjoys}}
%\bo{
%and enjoys potentially better properties (i.e., low graph similarities). \ziqi{this looks weird to me... need to discuss}
%}
%\st{potentially better properties (i.e., low graph similarities).}}

%-------------------------------------------------------------------------------------------------------------------------------------
\subsection{Comparison on Validity and Novelty}
\label{supp:app:results:valid_novel}
%-------------------------------------------------------------------------------------------------------------------------------------

We evaluate the ability of \method and \squid to generate molecules with valid and novel 2D molecular graphs.
%
We calculate the percentages of the valid and novel molecules among all the generated molecules.
%
As shown in Table~\ref{tbl:validity_novelty}, both \method and \methodwithsguide outperform \squid with $\lambda$=0.3 and $\lambda$=1.0 in generating novel molecules.
%
Particularly, almost all valid molecules generated by \method and \methodwithsguide are novel (99.8\% and 99.9\% at \#n\%), while the best baseline \squid with $\lambda$=0.3 achieves 98.4\% in novelty.
%
In terms of the percentage of valid and novel molecules among all the generated ones (\#v\&n\%), \method and \methodwithsguide again outperform \squid with $\lambda$=0.3 and $\lambda$=1.0.
%
We also note that at \#v\%,  \method (99.1\%) and \methodwithsguide (99.2\%) slightly underperform \squid with $\lambda$=0.3 and $\lambda$=1.0 (100.0\%) in generating valid molecules.
%
\squid guarantees the validity of generated molecules by incorporating valence rules into the generation process and ensuring it to avoid fragments that violate these rules.
%
Conversely, \method and \methodwithsguide use a purely data-driven approach to learn the generation of valid molecules.
%
These results suggest that, even without integrating valence rules, \method and \methodwithsguide can still achieve a remarkably high percentage of valid and novel generated molecules.

\begin{table*}
	\centering
		\caption{Comparison on Validity and Novelty between \method and \squid}
	\label{tbl:validity_novelty}
	\begin{scriptsize}
\begin{threeparttable}
%	\setlength\tabcolsep{0pt}
	\begin{tabular}{
		@{\hspace{3pt}}l@{\hspace{10pt}}
		%
		@{\hspace{10pt}}r@{\hspace{10pt}}
		%
		@{\hspace{10pt}}r@{\hspace{10pt}}
		%
		@{\hspace{10pt}}r@{\hspace{3pt}}
		}
		\toprule
		method & \#v\% & \#n\% & \#v\&n\% \\
		\midrule
		\squid ($\lambda$=0.3) & \textbf{100.0} & 96.7 & 96.7 \\
		\squid ($\lambda$=1.0) & \textbf{100.0} & 98.4 & 98.4 \\
		\method & 99.1 & 99.8 & 98.9 \\
		\methodwithsguide & 99.2 & \textbf{99.9} & \textbf{99.1} \\
		\bottomrule
	\end{tabular}%
	%
	\begin{tablenotes}[normal,flushleft]
		\begin{footnotesize}
	\item 
\!\!Columns represent: ``\#v\%'': the percentage of generated molecules that are valid;
		``\#n\%'': the percentage of valid molecules that are novel;
		``\#v\&n\%'': the percentage of generated molecules that are valid and novel.
		Best values are in \textbf{bold}. 
		\par
		\end{footnotesize}
	\end{tablenotes}
\end{threeparttable}
\end{scriptsize}
\end{table*}


%-------------------------------------------------------------------------------------------------------------------------------------
\subsection{Additional Quality Comparison between Desirable Molecules Generated by \method and \squid}
\label{supp:app:results:quality_desirable}
%-------------------------------------------------------------------------------------------------------------------------------------

\begin{table*}[!h]
	\centering
		\caption{Comparison on Quality of Generated Desirable Molecules between \method and \squid ($\delta_g$=0.5)}
	\label{tbl:overall_results_quality_05}
	\begin{scriptsize}
\begin{threeparttable}
	\begin{tabular}{
		@{\hspace{0pt}}l@{\hspace{16pt}}
		@{\hspace{0pt}}l@{\hspace{2pt}}
		%
		@{\hspace{6pt}}c@{\hspace{6pt}}
		%
		%@{\hspace{3pt}}c@{\hspace{3pt}}
		@{\hspace{3pt}}c@{\hspace{3pt}}
		@{\hspace{3pt}}c@{\hspace{3pt}}
		@{\hspace{3pt}}c@{\hspace{3pt}}
		@{\hspace{3pt}}c@{\hspace{3pt}}
		%
		%
		}
		\toprule
		group & metric & 
        %& \dataset 
        & \squid ($\lambda$=0.3) & \squid ($\lambda$=1.0)  &  \method & \methodwithsguide  \\
		%\multirow{2}{*}{method} & \multirow{2}{*}{\#c\%} &  \multirow{2}{*}{\#u\%} &  \multirow{2}{*}{QED} & \multicolumn{3}{c}{$\nmax=50$} & & \multicolumn{2}{c}{$\nmax=1$}\\
		%\cmidrule(r){5-7} \cmidrule(r){8-10} 
		%& & & & \avgshapesim(std) & \avggraphsim(std  &  \diversity(std  & & \avgshapesim(std) & \avggraphsim(std \\
		\midrule
		\multirow{2}{*}{stability}
		& atom stability ($\uparrow$) & 
        %& 0.990 
        & \textbf{0.996} & 0.995 & 0.992 & 0.989     \\
		& mol stability ($\uparrow$) & 
        %& 0.875 
        & \textbf{0.948} & 0.947 & 0.886 & 0.839    \\
		%\midrule
		%\multirow{3}{*}{Drug-likeness} 
		%& QED ($\uparrow$) & 
        %& \textbf{0.805} 
        %& 0.766 & 0.760 & 0.755 & 0.751    \\
	%	& SA ($\uparrow$) & 
        %& \textbf{0.874} 
        %& 0.814 & 0.813 & 0.699 & 0.692    \\
	%	& Lipinski ($\uparrow$) & 
        %& \textbf{4.999} 
        %& 4.979 & 4.980 & 4.967 & 4.975    \\
		\midrule
		\multirow{4}{*}{3D structures} 
		& RMSD ($\downarrow$) & 
        %& \textbf{0.419} 
        & 0.907 & 0.906 & 0.897 & \textbf{0.881}    \\
		& JS. bond lengths ($\downarrow$) & 
        %& \textbf{0.286} 
        & 0.457 & 0.477 & 0.436 & \textbf{0.428}    \\
		& JS. bond angles ($\downarrow$) & 
        %& \textbf{0.078} 
        & 0.269 & 0.289 & \textbf{0.186} & 0.200    \\
		& JS. dihedral angles ($\downarrow$) & 
        %& \textbf{0.151} 
        & 0.199 & 0.209 & \textbf{0.168} & 0.170    \\
		\midrule
		\multirow{5}{*}{2D structures} 
		& JS. \#bonds per atoms ($\downarrow$) & 
        %& 0.325 
        & 0.291 & 0.331 & \textbf{0.176} & 0.181    \\
		& JS. basic bond types ($\downarrow$) & 
        %& \textbf{0.055} 
        & \textbf{0.071} & 0.083 & 0.181 & 0.191    \\
		%& JS. freq. bond types ($\downarrow$) & 
        %& \textbf{0.089} 
        %& 0.123 & 0.130 & 0.245 & 0.254    \\
		%& JS. freq. bond pairs ($\downarrow$) & 
        %& \textbf{0.078} 
        %& 0.085 & 0.089 & 0.209 & 0.221    \\
		%& JS. freq. bond triplets ($\downarrow$) & 
        %& \textbf{0.089} 
        %& 0.097 & 0.114 & 0.211 & 0.223    \\
		%\midrule
		%\multirow{3}{*}{Rings} 
		& JS. \#rings ($\downarrow$) & 
        %& 0.142 
        & 0.280 & 0.330 & \textbf{0.043} & 0.049    \\
		& JS. \#n-sized rings ($\downarrow$) & 
        %& \textbf{0.055} 
        & \textbf{0.077} & 0.091 & 0.099 & 0.112    \\
		& \#Intersecting rings ($\uparrow$) & 
        %& \textbf{6} 
        & \textbf{6} & 5 & 4 & 5    \\
		%\method (+bt)            & 100.0 & 98.0 & 100.0 & 0.742 & 0.772 (0.040) & 0.211 (0.033) & & 0.862 (0.036) & 0.211 (0.033) & 0.743 (0.043) \\
		%\methodwithguide (+bt)    & 99.8 & 98.0 & 100.0 & 0.736 & 0.814 (0.031) & 0.193 (0.042) & & 0.895 (0.029) & 0.193 (0.042) & 0.745 (0.045) \\
		%
		\bottomrule
	\end{tabular}%
	\begin{tablenotes}[normal,flushleft]
		\begin{footnotesize}
	\item 
\!\!Rows represent:  {``atom stability'': the proportion of stable atoms that have the correct valency; 
		``molecule stability'': the proportion of generated molecules with all atoms stable;
		%``QED'': the drug-likeness score;
		%``SA'': the synthesizability score;
		%``Lipinski'': the Lipinski 
		``RMSD'': the root mean square deviation (RMSD) between the generated 3D structures of molecules and their optimal conformations; % identified via energy minimization;
		``JS. bond lengths/bond angles/dihedral angles'': the Jensen-Shannon (JS) divergences of bond lengths, bond angles and dihedral angles;
		``JS. \#bonds per atom/basic bond types/\#rings/\#n-sized rings'': the JS divergences of bond counts per atom, basic bond types, counts of all rings, and counts of n-sized rings;
		%``JS. \#rings/\#n-sized rings'': the JS divergences of the total counts of rings and the counts of n-sized rings;
		``\#Intersecting rings'': the number of rings observed in the top-10 frequent rings of both generated and real molecules. } \par
		\par
		\end{footnotesize}
	\end{tablenotes}
\end{threeparttable}
\end{scriptsize}
\end{table*}

%\label{tbl:overall_quality05}

\begin{table*}[!h]
	\centering
		\caption{Comparison on Quality of Generated Desirable Molecules between \method and \squid ($\delta_g$=0.7)}
	\label{tbl:overall_results_quality_07}
	\begin{scriptsize}
\begin{threeparttable}
	\begin{tabular}{
		@{\hspace{0pt}}l@{\hspace{14pt}}
		@{\hspace{0pt}}l@{\hspace{2pt}}
		%
		@{\hspace{4pt}}c@{\hspace{4pt}}
		%
		%@{\hspace{3pt}}c@{\hspace{3pt}}
		@{\hspace{3pt}}c@{\hspace{3pt}}
		@{\hspace{3pt}}c@{\hspace{3pt}}
		@{\hspace{3pt}}c@{\hspace{3pt}}
		@{\hspace{3pt}}c@{\hspace{3pt}}
		%
		%
		}
		\toprule
		group & metric & 
        %& \dataset 
        & \squid ($\lambda$=0.3) & \squid ($\lambda$=1.0)  &  \method & \methodwithsguide  \\
		%\multirow{2}{*}{method} & \multirow{2}{*}{\#c\%} &  \multirow{2}{*}{\#u\%} &  \multirow{2}{*}{QED} & \multicolumn{3}{c}{$\nmax=50$} & & \multicolumn{2}{c}{$\nmax=1$}\\
		%\cmidrule(r){5-7} \cmidrule(r){8-10} 
		%& & & & \avgshapesim(std) & \avggraphsim(std  &  \diversity(std  & & \avgshapesim(std) & \avggraphsim(std \\
		\midrule
		\multirow{2}{*}{stability} 
		& atom stability ($\uparrow$) & 
        %&  0.990 
        & \textbf{0.995} & 0.995 & 0.992 & 0.988 \\
		& molecule stability ($\uparrow$) & 
        %& 0.876 
        & 0.944 & \textbf{0.947} & 0.885 & 0.839 \\
		\midrule
		%\multirow{3}{*}{Drug-likeness} 
		%& QED ($\uparrow$) & 
        %& \textbf{0.805} 
        %& 0.766 & 0.760 & 0.755 & 0.751    \\
	%	& SA ($\uparrow$) & 
        %& \textbf{0.874} 
        %& 0.814 & 0.813 & 0.699 & 0.692    \\
	%	& Lipinski ($\uparrow$) & 
        %& \textbf{4.999} 
        %& 4.979 & 4.980 & 4.967 & 4.975    \\
	%	\midrule
		\multirow{4}{*}{3D structures} 
		& RMSD ($\downarrow$) & 
        %& \textbf{0.420} 
        & 0.897 & 0.906 & 0.897 & \textbf{0.881}    \\
		& JS. bond lengths ($\downarrow$) & 
        %& \textbf{0.286} 
        & 0.457 & 0.477 & 0.436 & \textbf{0.428}    \\
		& JS. bond angles ($\downarrow$) & 
        %& \textbf{0.078} 
        & 0.269 & 0.289 & \textbf{0.186} & 0.200    \\
		& JS. dihedral angles ($\downarrow$) & 
        %& \textbf{0.151} 
        & 0.199 & 0.209 & \textbf{0.168} & 0.170    \\
		\midrule
		\multirow{5}{*}{2D structures} 
		& JS. \#bonds per atoms ($\downarrow$) & 
        %& 0.325 
        & 0.285 & 0.329 & \textbf{0.176} & 0.181    \\
		& JS. basic bond types ($\downarrow$) & 
        %& \textbf{0.055} 
        & \textbf{0.067} & 0.083 & 0.181 & 0.191    \\
	%	& JS. freq. bond types ($\downarrow$) & 
        %& \textbf{0.089} 
        %& 0.123 & 0.130 & 0.245 & 0.254    \\
	%	& JS. freq. bond pairs ($\downarrow$) & 
        %& \textbf{0.078} 
        %& 0.085 & 0.089 & 0.209 & 0.221    \\
	%	& JS. freq. bond triplets ($\downarrow$) & 
        %& \textbf{0.089} 
        %& 0.097 & 0.114 & 0.211 & 0.223    \\
	%	\midrule
	%	\multirow{3}{*}{Rings} 
		& JS. \#rings ($\downarrow$) & 
        %& 0.143 
        & 0.273 & 0.328 & \textbf{0.043} & 0.049    \\
		& JS. \#n-sized rings ($\downarrow$) & 
        %& \textbf{0.055} 
        & \textbf{0.076} & 0.091 & 0.099 & 0.112    \\
		& \#Intersecting rings ($\uparrow$) & 
        %& \textbf{6} 
        & \textbf{6} & 5 & 4 & 5    \\
		%\method (+bt)            & 100.0 & 98.0 & 100.0 & 0.742 & 0.772 (0.040) & 0.211 (0.033) & & 0.862 (0.036) & 0.211 (0.033) & 0.743 (0.043) \\
		%\methodwithguide (+bt)    & 99.8 & 98.0 & 100.0 & 0.736 & 0.814 (0.031) & 0.193 (0.042) & & 0.895 (0.029) & 0.193 (0.042) & 0.745 (0.045) \\
		%
		\bottomrule
	\end{tabular}%
	\begin{tablenotes}[normal,flushleft]
		\begin{footnotesize}
	\item 
\!\!Rows represent:  {``atom stability'': the proportion of stable atoms that have the correct valency; 
		``molecule stability'': the proportion of generated molecules with all atoms stable;
		%``QED'': the drug-likeness score;
		%``SA'': the synthesizability score;
		%``Lipinski'': the Lipinski 
		``RMSD'': the root mean square deviation (RMSD) between the generated 3D structures of molecules and their optimal conformations; % identified via energy minimization;
		``JS. bond lengths/bond angles/dihedral angles'': the Jensen-Shannon (JS) divergences of bond lengths, bond angles and dihedral angles;
		``JS. \#bonds per atom/basic bond types/\#rings/\#n-sized rings'': the JS divergences of bond counts per atom, basic bond types, counts of all rings, and counts of n-sized rings;
		%``JS. \#rings/\#n-sized rings'': the JS divergences of the total counts of rings and the counts of n-sized rings;
		``\#Intersecting rings'': the number of rings observed in the top-10 frequent rings of both generated and real molecules. } \par
		\par
		\end{footnotesize}
	\end{tablenotes}
\end{threeparttable}
\end{scriptsize}
\end{table*}

%\label{tbl:overall_quality07}

\begin{table*}[!h]
	\centering
		\caption{Comparison on Quality of Generated Desirable Molecules between \method and \squid ($\delta_g$=1.0)}
	\label{tbl:overall_results_quality_10}
	\begin{scriptsize}
\begin{threeparttable}
	\begin{tabular}{
		@{\hspace{0pt}}l@{\hspace{14pt}}
		@{\hspace{0pt}}l@{\hspace{2pt}}
		%
		@{\hspace{4pt}}c@{\hspace{4pt}}
		%
		%@{\hspace{3pt}}c@{\hspace{3pt}}
		@{\hspace{3pt}}c@{\hspace{3pt}}
		@{\hspace{3pt}}c@{\hspace{3pt}}
		@{\hspace{3pt}}c@{\hspace{3pt}}
		@{\hspace{3pt}}c@{\hspace{3pt}}
		%
		%
		}
		\toprule
		group & metric & 
        %& \dataset 
        & \squid ($\lambda$=0.3) & \squid ($\lambda$=1.0)  &  \method & \methodwithsguide \\
		%\multirow{2}{*}{method} & \multirow{2}{*}{\#c\%} &  \multirow{2}{*}{\#u\%} &  \multirow{2}{*}{QED} & \multicolumn{3}{c}{$\nmax=50$} & & \multicolumn{2}{c}{$\nmax=1$}\\
		%\cmidrule(r){5-7} \cmidrule(r){8-10} 
		%& & & & \avgshapesim(std) & \avggraphsim(std  &  \diversity(std  & & \avgshapesim(std) & \avggraphsim(std \\
		\midrule
		\multirow{2}{*}{stability}
		& atom stability ($\uparrow$) & 
        %& 0.990 
        & \textbf{0.995} & \textbf{0.995} & 0.992 & 0.988     \\
		& mol stability ($\uparrow$) & 
        %& 0.876 
        & 0.942 & \textbf{0.947} & 0.885 & 0.839    \\
		\midrule
	%	\multirow{3}{*}{Drug-likeness} 
	%	& QED ($\uparrow$) & 
        %& \textbf{0.805} 
        %& \textbf{0.766} & 0.760 & 0.755 & 0.751    \\
	%	& SA ($\uparrow$) & 
        %& \textbf{0.874} 
        %& \textbf{0.813} & \textbf{0.813} & 0.699 & 0.692    \\
	%	& Lipinski ($\uparrow$) & 
        %& \textbf{4.999} 
        %& 4.979 & \textbf{4.980} & 4.967 & 4.975    \\
	%	\midrule
		\multirow{4}{*}{3D structures} 
		& RMSD ($\downarrow$) & 
        %& \textbf{0.420} 
        & 0.898 & 0.906 & 0.897 & \textbf{0.881}    \\
		& JS. bond lengths ($\downarrow$) & 
        %& \textbf{0.286} 
        & 0.457 & 0.477 & 0.436 & \textbf{0.428}    \\
		& JS. bond angles ($\downarrow$) & 
        %& \textbf{0.078} 
        & 0.269 & 0.289 & \textbf{0.186} & 0.200   \\
		& JS. dihedral angles ($\downarrow$) & 
        %& \textbf{0.151} 
        & 0.199 & 0.209 & \textbf{0.168} & 0.170    \\
		\midrule
		\multirow{5}{*}{2D structures} 
		& JS. \#bonds per atoms ($\downarrow$) & 
        %& 0.325 
        & 0.280 & 0.330 & \textbf{0.176} & 0.181    \\
		& JS. basic bond types ($\downarrow$) & 
        %& \textbf{0.055} 
        & \textbf{0.066} & 0.083 & 0.181 & 0.191   \\
	%	& JS. freq. bond types ($\downarrow$) & 
        %& \textbf{0.089} 
        %& \textbf{0.123} & 0.130 & 0.245 & 0.254    \\
	%	& JS. freq. bond pairs ($\downarrow$) & 
        %& \textbf{0.078} 
        %& \textbf{0.085} & 0.089 & 0.209 & 0.221    \\
	%	& JS. freq. bond triplets ($\downarrow$) & 
        %& \textbf{0.089} 
        %& \textbf{0.097} & 0.114 & 0.211 & 0.223    \\
		%\midrule
		%\multirow{3}{*}{Rings} 
		& JS. \#rings ($\downarrow$) & 
        %& 0.143 
        & 0.269 & 0.328 & \textbf{0.043} & 0.049    \\
		& JS. \#n-sized rings ($\downarrow$) & 
        %& \textbf{0.055} 
        & \textbf{0.075} & 0.091 & 0.099 & 0.112    \\
		& \#Intersecting rings ($\uparrow$) & 
        %& \textbf{6} 
        & \textbf{6} & 5 & 4 & 5    \\
		%\method (+bt)            & 100.0 & 98.0 & 100.0 & 0.742 & 0.772 (0.040) & 0.211 (0.033) & & 0.862 (0.036) & 0.211 (0.033) & 0.743 (0.043) \\
		%\methodwithguide (+bt)    & 99.8 & 98.0 & 100.0 & 0.736 & 0.814 (0.031) & 0.193 (0.042) & & 0.895 (0.029) & 0.193 (0.042) & 0.745 (0.045) \\
		%
		\bottomrule
	\end{tabular}%
	\begin{tablenotes}[normal,flushleft]
		\begin{footnotesize}
	\item 
\!\!Rows represent:  {``atom stability'': the proportion of stable atoms that have the correct valency; 
		``molecule stability'': the proportion of generated molecules with all atoms stable;
		%``QED'': the drug-likeness score;
		%``SA'': the synthesizability score;
		%``Lipinski'': the Lipinski 
		``RMSD'': the root mean square deviation (RMSD) between the generated 3D structures of molecules and their optimal conformations; % identified via energy minimization;
		``JS. bond lengths/bond angles/dihedral angles'': the Jensen-Shannon (JS) divergences of bond lengths, bond angles and dihedral angles;
		``JS. \#bonds per atom/basic bond types/\#rings/\#n-sized rings'': the JS divergences of bond counts per atom, basic bond types, counts of all rings, and counts of n-sized rings;
		%``JS. \#rings/\#n-sized rings'': the JS divergences of the total counts of rings and the counts of n-sized rings;
		``\#Intersecting rings'': the number of rings observed in the top-10 frequent rings of both generated and real molecules. } \par
		\par
		\end{footnotesize}
	\end{tablenotes}
\end{threeparttable}
\end{scriptsize}
\end{table*}

%\label{tbl:overall_quality10}

Similar to Table~\ref{tbl:overall_results_quality_desired} in the main manuscript, we present the performance comparison on the quality of desirable molecules generated by different methods under different graph similarity constraints $\delta_g$=0.5, 0.7 and 1.0, as detailed in Table~\ref{tbl:overall_results_quality_05}, Table~\ref{tbl:overall_results_quality_07}, and Table~\ref{tbl:overall_results_quality_10}, respectively.
%
Overall, these tables show that under varying graph similarity constraints, \method and \methodwithsguide can always generate desirable molecules with comparable quality to baselines in terms of stability, 3D structures, and 2D structures.
%
These results demonstrate the strong effectiveness of \method and \methodwithsguide in generating high-quality desirable molecules with stable and realistic structures in both 2D and 3D.
%
This enables the high utility of \method and \methodwithsguide in discovering promising drug candidates.


\begin{comment}
The results across these tables demonstrate similar observations with those under $\delta_g$=0.3 in Table~\ref{tbl:overall_results_quality_desired}.
%
For stability, when $\delta_g$=0.5, 0.7 or 1.0, \method and \methodwithsguide achieve comparable performance or fall slightly behind \squid ($\lambda$=0.3) and \squid ($\lambda$=1.0) in atom stability and molecule stability.
%
For example, when $\delta_g$=0.5, as shown in Table~\ref{tbl:overall_results_quality_05}, \method achieves similar performance with the best baseline \squid ($\lambda$=0.3) in atom stability (0.992 for \method vs 0.996 for \squid with $\lambda$=0.3).
%
\method underperforms \squid ($\lambda$=0.3) in terms of molecule stability.
%
For 3D structures, \method and \methodwithsguide also consistently generate molecules with more realistic 3D structures compared to \squid.
%
Particularly, \methodwithsguide achieves the best performance in RMSD and JS of bond lengths across $\delta_g$=0.5, 0.7 and 1.0.
%
In JS of dihedral angles, \method achieves the best performance among all the methods.
%
\method and \methodwithsguide underperform \squid in JS of bond angles, primarily because \squid constrains the bond angles in the generated molecules.
%
For 2D structures, \method and \methodwithsguide again achieve the best performance 
\end{comment}

%===================================================================
\section{Additional Experimental Results on PMG}
\label{supp:app:results_PMG}
%===================================================================

%\label{tbl:comparison_results_decompdiff}


%-------------------------------------------------------------------------------------------------------------------------------------
%\subsection{{Additional Comparison for PMG}}
%\label{supp:app:results:docking}
%-------------------------------------------------------------------------------------------------------------------------------------

In this section, we present the results of \methodwithpguide and \methodwithsandpguide when generating 100 molecules. 
%
Please note that both \methodwithpguide and \methodwithsandpguide show remarkable efficiency over the PMG baselines.
%
\methodwithpguide and \methodwithsandpguide generate 100 molecules in 48 and 58 seconds on average, respectively, while the most efficient baseline \targetdiff requires 1,252 seconds.
%
We report the performance of \methodwithpguide and \methodwithsandpguide against state-of-the-art PMG baselines in Table~\ref{tbl:overall_results_docking_100}.


%
According to Table~\ref{tbl:overall_results_docking_100}, \methodwithpguide and \methodwithsandpguide achieve comparable performance with the PMG baselines in generating molecules with high binding affinities.
%
Particularly, in terms of Vina S, \methodwithsandpguide achieves very comparable performance (-4.56 kcal/mol) to the third-best baseline \decompdiff (-4.58 kcal/mol) in average Vina S; it also achieves the third-best performance (-4.82 kcal/mol) among all the methods and slightly underperforms the second-best baseline \AR (-4.99 kcal/mol) in median Vina S
%
\methodwithsandpguide also achieves very close average Vina M (-5.53 kcal/mol) with the third-best baseline \AR (-5.59 kcal/mol) and the third-best performance (-5.47 kcal/mol) in median Vina M.
%
Notably, for Vina D, \methodwithpguide and \methodwithsandpguide achieve the second and third performance among all the methods.
%
In terms of the average percentage of generated molecules with Vina D higher than those of known ligands (i.e., HA), \methodwithpguide (58.52\%) and \methodwithsandpguide (58.28\%) outperform the best baseline \targetdiff (57.57\%).
%
These results signify the high utility of \methodwithpguide and \methodwithsandpguide in generating molecules that effectively bind with protein targets and have better binding affinities than known ligands.

In addition to binding affinities, \methodwithpguide and \methodwithsandpguide also demonstrate similar performance compared to the baselines in metrics related to drug-likeness and diversity.
%
For drug-likeness, both \methodwithpguide and \methodwithsandpguide achieve the best (0.67) and the second-best (0.66) QED scores.
%
They also achieve the third and fourth performance in SA scores.
%
In terms of the diversity among generated molecules,  \methodwithpguide and \methodwithsandpguide slightly underperform the baselines, possibly due to the design that generates molecules with similar shapes to the ligands.
%
These results highlight the strong ability of \methodwithpguide and \methodwithsandpguide in efficiently generating effective binding molecules with favorable drug-likeness and diversity.
%
This ability enables them to potentially serve as promising tools to facilitate effective and efficient drug development.

\begin{table*}[!h]
	\centering
		\caption{Additional Comparison on PMG When All Methods Generate 100 Molecules}
	\label{tbl:overall_results_docking_100}
\begin{threeparttable}
	\begin{scriptsize}
	\begin{tabular}{
		@{\hspace{2pt}}l@{\hspace{2pt}}
		%
		@{\hspace{2pt}}r@{\hspace{2pt}}
		%
		@{\hspace{2pt}}r@{\hspace{2pt}}
		@{\hspace{2pt}}r@{\hspace{2pt}}
		%
		@{\hspace{6pt}}r@{\hspace{6pt}}
		%
		@{\hspace{2pt}}r@{\hspace{2pt}}
		@{\hspace{2pt}}r@{\hspace{2pt}}
		%
		@{\hspace{5pt}}r@{\hspace{5pt}}
		%
		@{\hspace{2pt}}r@{\hspace{2pt}}
		@{\hspace{2pt}}r@{\hspace{2pt}}
		%
		@{\hspace{5pt}}r@{\hspace{5pt}}
		%
		@{\hspace{2pt}}r@{\hspace{2pt}}
	         @{\hspace{2pt}}r@{\hspace{2pt}}
		%
		@{\hspace{5pt}}r@{\hspace{5pt}}
		%
		@{\hspace{2pt}}r@{\hspace{2pt}}
		@{\hspace{2pt}}r@{\hspace{2pt}}
		%
		@{\hspace{5pt}}r@{\hspace{5pt}}
		%
		@{\hspace{2pt}}r@{\hspace{2pt}}
		@{\hspace{2pt}}r@{\hspace{2pt}}
		%
		@{\hspace{5pt}}r@{\hspace{5pt}}
		%
		@{\hspace{2pt}}r@{\hspace{2pt}}
		@{\hspace{2pt}}r@{\hspace{2pt}}
		%
		@{\hspace{5pt}}r@{\hspace{5pt}}
		%
		@{\hspace{2pt}}r@{\hspace{2pt}}
		%@{\hspace{2pt}}r@{\hspace{2pt}}
		%@{\hspace{2pt}}r@{\hspace{2pt}}
		}
		\toprule
		\multirow{2}{*}{method} & \multicolumn{2}{c}{Vina S$\downarrow$} & & \multicolumn{2}{c}{Vina M$\downarrow$} & & \multicolumn{2}{c}{Vina D$\downarrow$} & & \multicolumn{2}{c}{{HA\%$\uparrow$}}  & & \multicolumn{2}{c}{QED$\uparrow$} & & \multicolumn{2}{c}{SA$\uparrow$} & & \multicolumn{2}{c}{Div$\uparrow$} & %& \multirow{2}{*}{SR\%$\uparrow$} & 
		& \multirow{2}{*}{time$\downarrow$} \\
	    \cmidrule{2-3}\cmidrule{5-6} \cmidrule{8-9} \cmidrule{11-12} \cmidrule{14-15} \cmidrule{17-18} \cmidrule{20-21}
		 & Avg. & Med. &  & Avg. & Med. &  & Avg. & Med. & & Avg. & Med.  & & Avg. & Med.  & & Avg. & Med.  & & Avg. & Med.  & & \\ %& & \\
		%\multirow{2}{*}{method} & \multirow{2}{*}{\#c\%} &  \multirow{2}{*}{\#u\%} &  \multirow{2}{*}{QED} & \multicolumn{3}{c}{$\nmax=50$} & & \multicolumn{2}{c}{$\nmax=1$}\\
		%\cmidrule(r){5-7} \cmidrule(r){8-10} 
		%& & & & \avgshapesim(std) & \avggraphsim(std  &  \diversity(std  & & \avgshapesim(std) & \avggraphsim(std \\
		\midrule
		Reference                          & -5.32 & -5.66 & & -5.78 & -5.76 & & -6.63 & -6.67 & & - & - & & 0.53 & 0.49 & & 0.77 & 0.77 & & - & - & %& 23.1 & 
		& - \\
		\midrule
		\AR & \textbf{-5.06} & -4.99 & &  -5.59 & -5.29 & &  -6.16 & -6.05 & &  37.69 & 31.00 & &  0.50 & 0.49 & &  0.66 & 0.65 & & 0.70 & 0.70 & %& 7.0 & 
		& 7,789 \\
		\pockettwomol   & -4.50 & -4.21 & &  -5.70 & -5.27 & &  -6.43 & -6.25 & &  48.00 & 51.00 & &  0.58 & 0.58 & &  \textbf{0.77} & \textbf{0.78} & &  0.69 & 0.71 &  %& 24.9 & 
		& 2,150 \\
		\targetdiff     & -4.88 & \textbf{-5.82} & &  \textbf{-6.20} & \textbf{-6.36} & &  \textbf{-7.37} & \textbf{-7.51} & &  57.57 & 58.27 & &  0.50 & 0.51 & &  0.60 & 0.59 & &  \textbf{0.72} & 0.71 & % & 10.4 & 
		& 1,252 \\
		%& \decompdiffbeta                    & 63.03 & %-4.72 & -4.86 & & \textbf{-6.84} & \textbf{-6.91} & & \textbf{-8.85} & \textbf{-8.90} & &  \textbf{72.16} & \textbf{72.16} & &  0.36 & 0.36 & &  0.55 & 0.55 & & 0.59 & 0.59 & & 14.9 \\ 
		%-4.76 & -6.18 & &  \textbf{-6.86} & \textbf{-7.52} & &  \textbf{-8.85} & \textbf{-8.96} & &  \textbf{72.7} & \textbf{89.8} & &  0.36 & 0.34 & &  0.55 & 0.57 & & 0.59 & 0.59 & & 15.4 \\
		\decompdiffref  & -4.58 & -4.77 & &  -5.47 & -5.51 & &  -6.43 & -6.56 & &  47.76 & 48.66 & &  0.56 & 0.56 & &  0.70 & 0.69  & &  \textbf{0.72} & \textbf{0.72} &  %& 15.2 & 
		& 1,859 \\
		\midrule
		%\method & 14.04 & 9.74 & &  -2.80 & -3.87 & &  -6.32 & -6.41 & &  42.37 & 40.40 & &  0.70 & 0.71 & &  0.73 & 0.72 & & 0.71 & 0.74 & & 42 \\
		%\methodwithsguide & 1.04 & -0.33 & &  -4.23 & -4.39 & &  -6.31 & -6.46 & &  46.18 & 44.00 & &  0.69 & 0.71 & &  0.72 & 0.71 & & 0.70 & 0.73 & 53 \\
		\methodwithpguide      & -4.15 & -4.59 & &  -5.41 & -5.34 & &  -6.49 & -6.74 & &  \textbf{58.52} & 59.00 & &  \textbf{0.67} & \textbf{0.69} & &  0.68 & 0.68 & & 0.67 & 0.70 & %& 28.0 & 
		& 48 \\
		\methodwithsandpguide  & -4.56 & -4.82 & &  -5.53 & -5.47 & &  -6.60 & -6.78 & &  58.28 & \textbf{60.00} & &  0.66 & 0.68 & &  0.67 & 0.66 & & 0.68 & 0.71 &
		& 58 \\
		\bottomrule
	\end{tabular}%
	\begin{tablenotes}[normal,flushleft]
		\begin{footnotesize}
	\item 
\!\!Columns represent: {``Vina S'': the binding affinities between the initially generated poses of molecules and the protein pockets; 
		``Vina M'': the binding affinities between the poses after local structure minimization and the protein pockets;
		``Vina D'': the binding affinities between the poses determined by AutoDock Vina~\cite{Eberhardt2021} and the protein pockets;
		``HA'': the percentage of generated molecules with Vina D higher than those of condition molecules;
		``QED'': the drug-likeness score;
		``SA'': the synthesizability score;
		``Div'': the diversity among generated molecules;
		``time'': the time cost to generate molecules.}
		\par
		\par
		\end{footnotesize}
	\end{tablenotes}
	\end{scriptsize}
\end{threeparttable}
\end{table*}


%\label{tbl:overall_results_docking_100}

%-------------------------------------------------------------------------------------------------------------------------------------
%\subsection{{Comparison of Pocket Guidance}}
%\label{supp:app:results:docking}
%-------------------------------------------------------------------------------------------------------------------------------------


\begin{comment}
%-------------------------------------------------------------------------------------------------------------------------------------
\subsection{\ziqi{Simiarity Comparison for Pocket-based Molecule Generation}}
%-------------------------------------------------------------------------------------------------------------------------------------


\begin{table*}[t!]
	\centering
	\caption{{Overall Comparison on Similarity for Pocket-based Molecule Generation}}
	\label{tbl:docking_results_similarity}
	\begin{small}
		\begin{threeparttable}
			\begin{tabular}{
					@{\hspace{0pt}}l@{\hspace{5pt}}
					%
					@{\hspace{3pt}}l@{\hspace{3pt}}
					%
					@{\hspace{3pt}}r@{\hspace{8pt}}
					@{\hspace{3pt}}c@{\hspace{3pt}}
					%
					@{\hspace{3pt}}c@{\hspace{3pt}}
					@{\hspace{3pt}}c@{\hspace{3pt}}
					%
					@{\hspace{0pt}}c@{\hspace{0pt}}
					%
					@{\hspace{3pt}}c@{\hspace{3pt}}
					@{\hspace{3pt}}c@{\hspace{3pt}}
					%
					@{\hspace{3pt}}r@{\hspace{3pt}}
				}
				\toprule
				$\delta_g$  & method          & \#d\%$\uparrow$ & $\diversity_d$$\uparrow$(std) & \avgshapesim$\uparrow$(std) & \avggraphsim$\downarrow$(std) & & \maxshapesim$\uparrow$(std) & \maxgraphsim$\downarrow$(std)       & \#n\%$\uparrow$  \\ 
				\midrule
				%\multirow{6}{0.059\linewidth}{\hspace{0pt}0.1} 
				%& \AR   & 4.4 & 0.781(0.076) & 0.511(0.197) & \textbf{0.056}(0.020) & & 0.619(0.222) & 0.074(0.024) & 21.4  \\
				%& \pockettwomol & 6.6 & 0.795(0.099) & 0.519(0.216) & 0.063(0.020) & & 0.608(0.236) & 0.076(0.022) & \textbf{24.1}  \\
				%& \targetdiff & 2.0 & 0.872(0.041) & 0.619(0.110) & 0.068(0.018) & & 0.721(0.146) & 0.075(0.023) & 17.7  \\
				%& \decompdiffbeta & 0.0 & - & 0.374(0.138) & 0.059(0.031) & & 0.414(0.141) & \textbf{0.058}(0.032) & 9.8  \\
				%& \decompdiffref & 8.5 & 0.805(0.096) & 0.810(0.070) & 0.076(0.018) & & 0.861(0.085) & 0.076(0.020) & 11.3  \\
				%& \methodwithpguide   &  9.9 & \textbf{0.876}(0.041) & 0.795(0.058) & 0.073(0.015) & & 0.869(0.073) & 0.076(0.020) & 17.7  \\
				%& \methodwithsandpguide & \textbf{11.9} & 0.872(0.036) & \textbf{0.813}(0.052) & 0.075(0.014) & & \textbf{0.874}(0.069) & 0.080(0.014) & 17.0  \\
				%\cmidrule{2-10}
				%& improv\% & 40.4$^*$ & 8.8$^*$ & 0.4 & -30.4$^*$ &  & 1.6 & -30.0$^*$ & -26.3$^*$  \\
				%\midrule
				\multirow{7}{0.059\linewidth}{\hspace{0pt}1.0} 
				& \AR & 14.6 & 0.681(0.163) & 0.644(0.119) & 0.236(0.123) & & 0.780(0.110) & 0.284(0.177) & 95.8  \\
				& \pockettwomol & 18.6 & 0.711(0.152) & 0.654(0.131) &   \textbf{0.217}(0.129) & & 0.778(0.121) &   \textbf{0.243}(0.137) &  \textbf{98.3}  \\
				& \targetdiff & 7.1 &  \textbf{0.785}(0.085) & 0.622(0.083) & 0.238(0.122) & & 0.790(0.102) & 0.274(0.158) & 90.4  \\
				%& \decompdiffbeta & 0.1 & 0.589(0.030) & 0.494(0.124) & 0.263(0.143) & & 0.567(0.143) & 0.275(0.162) & 67.7  \\
				& \decompdiffref & 37.3 & 0.721(0.108) & 0.770(0.087) & 0.282(0.130) & & \textbf{0.878}(0.059) & 0.343(0.174) & 83.7  \\
				& \methodwithpguide   &  27.4 & 0.757(0.134) & 0.747(0.078) & 0.265(0.165) & & 0.841(0.081) & 0.272(0.168) & 98.1  \\
				& \methodwithsandpguide &\textbf{45.2} & 0.724(0.142) &   \textbf{0.789}(0.063) & 0.265(0.162) & & 0.876(0.062) & 0.264(0.159) & 97.8  \\
				\cmidrule{2-10}
				& Improv\%  & 21.2$^*$ & -3.6 & 2.5$^*$ & -21.7$^*$ &  & -0.1 & -8.4$^*$ & -0.2  \\
				\midrule
				\multirow{7}{0.059\linewidth}{\hspace{0pt}0.7} 
				& \AR   & 14.5 & 0.692(0.151) & 0.644(0.119) & 0.233(0.116) & & 0.779(0.110) & 0.266(0.140) & 94.9  \\
				& \pockettwomol & 18.6 & 0.711(0.152) & 0.654(0.131) & \textbf{0.217}(0.129) & & 0.778(0.121) & \textbf{0.243}(0.137) & \textbf{98.2}  \\
				& \targetdiff & 7.1 & \textbf{0.786}(0.084) & 0.622(0.083) & 0.238(0.121) & & 0.790(0.101) & 0.270(0.151) & 90.3  \\
				%& \decompdiffbeta & 0.1 & 0.589(0.030) & 0.494(0.124) & 0.263(0.142) & &0.567(0.143) & 0.273(0.156) & 67.6  \\
				& \decompdiffref & 36.2 & 0.721(0.113) & 0.770(0.086) & 0.273(0.123) & & \textbf{0.876}(0.059) & 0.325(0.139) & 82.3  \\
				& \methodwithpguide   &  27.4 & 0.757(0.134) & 0.746(0.078) & 0.263(0.160) & & 0.841(0.081) & 0.271(0.164) & 96.8  \\
				& \methodwithsandpguide      & \textbf{45.0} & 0.732(0.129) & \textbf{0.789}(0.063) & 0.262(0.157) & & \textbf{0.876}(0.063) & 0.262(0.153) & 96.2  \\
				\cmidrule{2-10}
				& Improv\%  & 24.3$^*$ & -3.6 & 2.5$^*$ & -20.8$^*$ &  & 0.0 & -7.6$^*$ & -1.5  \\
				\midrule
				\multirow{7}{0.059\linewidth}{\hspace{0pt}0.5} 
				& \AR   & 14.1 & 0.687(0.160) & 0.639(0.124) & 0.218(0.097) & & 0.778(0.110) & 0.260(0.130) & 89.8  \\
				& \pockettwomol & 18.5 & 0.711(0.152) & 0.649(0.134) & \textbf{0.209}(0.114) & & 0.777(0.121) & \textbf{0.240}(0.131) & \textbf{93.2}  \\
				& \targetdiff & 7.1 & \textbf{0.786}(0.084) & 0.621(0.083) & 0.230(0.111) & & 0.788(0.105) & 0.254(0.127) & 86.5  \\
				%&\decompdiffbeta & 0.1 & 0.595(0.025) & 0.494(0.124) & 0.254(0.129) & & 0.565(0.142) & 0.259(0.138) & 63.9  \\
				& \decompdiffref & 34.7 & 0.730(0.105) & 0.769(0.086) & 0.261(0.109) & & 0.874(0.080) & 0.301(0.117) & 77.3   \\
				& \methodwithpguide  &  27.2 & 0.765(0.123) & 0.749(0.075) & 0.245(0.135) & & 0.840(0.082) & 0.252(0.137) & 88.6  \\
				& \methodwithsandpguide & \textbf{44.3} & 0.738(0.122) & \textbf{0.791}(0.059) & 0.247(0.132) &  & \textbf{0.875}(0.065) & 0.249(0.130) & 88.8  \\
				\cmidrule{2-10}
				& Improv\%   & 27.8$^*$ & -2.7 & 2.9$^*$ & -17.6$^*$ &  & 0.2 & -3.4 & -4.7$^*$  \\
				\midrule
				\multirow{7}{0.059\linewidth}{\hspace{0pt}0.3} 
				& \AR   & 12.2 & 0.704(0.146) & 0.614(0.146) & 0.164(0.059) & & 0.751(0.138) & 0.206(0.059) & 66.4  \\
				& \pockettwomol & 17.1 & 0.731(0.129) & 0.617(0.163) & \textbf{0.155}(0.056) & & 0.740(0.159) & \textbf{0.190}(0.076) & \textbf{71.0}  \\
				& \targetdiff & 6.2 & \textbf{0.809}(0.061) & 0.619(0.087) & 0.181(0.068) & & 0.768(0.119) & 0.196(0.076) & 61.7  \\				
                %& \decompdiffbeta & 0.0 & - & 0.489(0.124) & 0.195(0.080) & & 0.547(0.139) & 0.203(0.087) & 42.0  \\
				& \decompdiffref & 27.7 & 0.775(0.081) & 0.767(0.086) & 0.202(0.062) & & 0.854(0.093) & 0.216(0.068) & 52.6  \\
				& \methodwithpguide   &  24.4 & 0.805(0.084) & 0.763(0.066) & 0.180(0.074) & & 0.847(0.080) & \textbf{0.190}(0.059) & 61.4  \\
				& \methodwithsandpguide & \textbf{36.3} & 0.789(0.081) & \textbf{0.800}(0.056) & 0.181(0.071) & &\textbf{0.878}(0.067) & \textbf{0.190}(0.078) & 61.8  \\
				\cmidrule{2-10}
				& improv\% & 31.1$^*$ & 3.9$^*$ & 4.3$^*$ & -16.5$^*$ &  & 2.8$^*$ & 0.0 & -12.9$^*$  \\
				\bottomrule
			\end{tabular}%
			\begin{tablenotes}[normal,flushleft]
				\begin{footnotesize}
					\item 
					\!\!Columns represent: \ziqi{``$\delta_g$'': the graph similarity constraint; ``\#n\%'': the percentage of molecules that satisfy the graph similarity constraint ($\graphsim<=\delta_g$);
						``\#d\%'': the percentage of molecules that satisfy the graph similarity constraint and are with high \shapesim ($\shapesim>=0.8$);
						``\avgshapesim/\avggraphsim'': the average of shape or graph similarities between the condition molecules and generated molecules with $\graphsim<=\delta_g$;
						``\maxshapesim'': the maximum of shape similarities between the condition molecules and generated molecules with $\graphsim<=\delta_g$;
						``\maxgraphsim'': the graph similarities between the condition molecules and the molecules with the maximum shape similarities and $\graphsim<=\delta_g$;
						``\diversity'': the diversity among the generated molecules.
						%
						``$\uparrow$'' represents higher values are better, and ``$\downarrow$'' represents lower values are better.
						%
						Best values are in \textbf{bold}, and second-best values are \underline{underlined}. 
					} 
					%\todo{double-check the significance value}
					\par
					\par
				\end{footnotesize}
			\end{tablenotes}
		\end{threeparttable}
	\end{small}
	\vspace{-10pt}    
\end{table*}
%\label{tbl:docking_results_similarity}

\bo{@Ziqi you may want to check my edits for the discussion in Table 1 first.
%
If the pocket if known, do you still care about the shape similarity in real applications?
}

\ziqi{Table~\ref{tbl:docking_results_similarity} presents the overall comparison on similarity-based metrics between \methodwithpguide, \methodwithsandpguide and other baselines under different graph similarity constraints  ($\delta_g$=1.0, 0.7, 0.5, 0.3), similar to Table~\ref{tbl:overall}. 
%
As shown in Table~\ref{tbl:docking_results_similarity}, regarding desirable molecules,  \methodwithsandpguide consistently outperforms all the baseline methods in the likelihood of generating desirable molecules (i.e., $\#d\%$).
%
For example, when $\delta_g$=1.0, at $\#d\%$, \methodwithsandpguide (45.2\%) demonstrates significant improvement of $21.2\%$ compared to the best baseline \decompdiff (37.3\%).
%
In terms of $\diversity_d$, \methodwithpguide and \methodwithsandpguide also achieve the second and the third best performance. 
%
Note that the best baseline \targetdiff in $\diversity_d$ achieves the least percentage of desirable molecules (7.1\%), substantially lower than \methodwithpguide and \methodwithsandpguide.
%
This makes its diversity among desirable molecules incomparable with other methods. 
%
When $\delta_g$=0.7, 0.5, and 0.3, \methodwithsandpguide also establishes a significant improvement of 24.3\%, 27.8\%, and 31.1\% compared to the best baseline method \decompdiff.
%
It is also worth noting that the state-of-the-art baseline \decompdiff underperforms \methodwithpguide and \methodwithsandpguide in binding affinities as shown in Table~\ref{tbl:overall_results_docking}, even though it outperforms \methodwithpguide in \#d\%.
%
\methodwithpguide and \methodwithsandpguide also achieve the second and the third best performance in $\diversity_d$ at $\delta_g$=0.7, 0.5, and 0.3. 
%
The superior performance of \methodwithpguide and \methodwithsandpguide in $\#d\%$ at small $\delta_g$ indicates their strong capacity in generating desirable molecules of novel graph structures, thereby facilitating the discovery of novel drug candidates.
%
}

\ziqi{Apart from the desirable molecules, \methodwithpguide and \methodwithsandpguide also demonstrate outstanding performance in terms of the average shape similarities (\avgshapesim) and the average graph similarities (\avggraphsim).
%
Specifically, when $\delta_g$=1.0, \methodwithsandpguide achieves a significant 2.5\% improvement in \avgshapesim\ over the best baseline \decompdiff. 
%
In terms of \avggraphsim, \methodwithsandpguide also achieves higher performance than the baseline \decompdiff of the highest \avgshapesim (0.265 vs 0.282).
%
Please note that all the baseline methods except \decompdiff achieve substantially lower performance in \avgshapesim than \methodwithpguide and \methodwithsandpguide, even though these methods achieve higher \avggraphsim values.
%
This trend remains consistent when applying various similarity constraints (i.e., $\delta_g$) as shown in Table~\ref{tbl:overall_results_docking}.
}

\ziqi{Similarly, \methodwithpguide and \methodwithsandpguide also achieve superior performance in \maxshapesim and \maxgraphsim.
%
Specifically, when $\delta_g$=1.0, for \maxshapesim, \methodwithsandpguide achieves highly comparable performance in \maxshapesim\ compared to the best baseline \decompdiff (0.876 vs 0.878).
%
We also note that \methodwithsandpguide achieves lower \maxgraphsim\ than the \decompdiff with 23.0\% difference. 
%
When $\delta_g$ gets smaller from 0.7 to 0.3, \methodwithsandpguide maintains a high \maxshapesim value around 0.876, while the best baseline \decompdiff has \maxshapesim decreased from 0.878 to 0.854.
%
This demonstrates the superior ability of \methodwithsandpguide in generating molecules with similar shapes and novel structures.
%
}

\ziqi{
In terms of \#n\%, when $\delta_g$=1.0, the percentage of molecules with \graphsim below $\delta_g$ can be interpreted as the percentage of valid molecules among all the generated molecules. 
%
As shown in Table~\ref{tbl:docking_results_similarity}, \methodwithpguide and \methodwithsandpguide are able to generate 98.1\% and 97.8\% of valid molecules, slightly below the best baseline \pockettwomol (98.3\%). 
%
When $\delta_g$=0.7, 0.5, or 0.3, all the methods, including \methodwithpguide and \methodwithsandpguide, can consistently find a sufficient number of novel molecules that meet the graph similarity constraints.
%
The only exception is \decompdiff, which substantially underperforms all the other methods in \#n\%.
}
\end{comment}

%%%%%%%%%%%%%%%%%%%%%%%%%%%%%%%%%%%%%%%%%%%%%
\section{Properties of Molecules in Case Studies for Targets}
\label{supp:app:results:properties}
%%%%%%%%%%%%%%%%%%%%%%%%%%%%%%%%%%%%%%%%%%%%%

%-------------------------------------------------------------------------------------------------------------------------------------
\subsection{Drug Properties of Generated Molecules}
\label{supp:app:results:properties:drug}
%-------------------------------------------------------------------------------------------------------------------------------------

Table~\ref{tbl:drug_property} presents the drug properties of three generated molecules: NL-001, NL-002, and NL-003.
%
As shown in Table~\ref{tbl:drug_property}, each of these molecules has a favorable profile, making them promising drug candidates. 
%
{As discussed in Section ``Case Studies for Targets'' in the main manuscript, all three molecules have high binding affinities in terms of Vina S, Vina M and Vina D, and favorable QED and SA values.
%
In addition, all of them meet the Lipinski's rule of five criteria~\cite{Lipinski1997}.}
%
In terms of physicochemical properties, all these properties of NL-001, NL-002 and NL-003, including number of rotatable bonds, molecule weight, LogP value, number of hydrogen bond doners and acceptors, and molecule charges, fall within the desired range of drug molecules. 
%
This indicates that these molecules could potentially have good solubility and membrane permeability, essential qualities for effective drug absorption.

These generated molecules also demonstrate promising safety profiles based on the predictions from ICM~\cite{Neves2012}.
%
In terms of drug-induced liver injury prediction scores, all three molecules have low scores (0.188 to 0.376), indicating a minimal risk of hepatotoxicity. 
%
NL-001 and NL-002 fall under `Ambiguous/Less concern' for liver injury, while NL-003 is categorized under 'No concern' for liver injury. 
%
Moreover, all these molecules have low toxicity scores (0.000 to 0.236). 
%
NL-002 and NL-003 do not have any known toxicity-inducing functional groups. 
%
NL-001 and NL-003 are also predicted not to include any known bad groups that lead to inappropriate features.
%
These attributes highlight the potential of NL-001, NL-002, and NL-003 as promising treatments for cancers and Alzheimer’s disease.

%\begin{table*}
	\centering
		\caption{Drug Properties of Generated Molecules}
	\label{tbl:binding_drug_mols}
	\begin{scriptsize}
\begin{threeparttable}
	\begin{tabular}{
		@{\hspace{6pt}}r@{\hspace{6pt}}
		@{\hspace{6pt}}r@{\hspace{6pt}}
		@{\hspace{6pt}}r@{\hspace{6pt}}
		@{\hspace{6pt}}r@{\hspace{6pt}}
		@{\hspace{6pt}}r@{\hspace{6pt}}
		@{\hspace{6pt}}r@{\hspace{6pt}}
		@{\hspace{6pt}}r@{\hspace{6pt}}
		@{\hspace{6pt}}r@{\hspace{6pt}}
		@{\hspace{6pt}}r@{\hspace{6pt}}
		%
		}
		\toprule
Target & Molecule & Vina S & Vina M & Vina D & QED   & SA   & Logp  & Lipinski \\
\midrule
\multirow{3}{*}{CDK6} & NL-001 & -6.817      & -7.251    & -8.319     & 0.834 & 0.72 & 1.313 & 5        \\
& NL-002 & -6.970       & -7.605    & -8.986     & 0.851 & 0.74 & 3.196 & 5        \\
\cmidrule{2-9}
& 4AU & 0.736       & -5.939    & -7.592     & 0.773 & 0.79 & 2.104 & 5        \\
\midrule
\multirow{2}{*}{NEP} & NL-003 & -11.953     & -12.165   & -12.308    & 0.772 & 0.57 & 2.944 & 5        \\
\cmidrule{2-9}
& BIR & -9.399      & -9.505    & -9.561     & 0.463 & 0.73 & 2.677 & 5        \\
		\bottomrule
	\end{tabular}%
	\begin{tablenotes}[normal,flushleft]
		\begin{footnotesize}
	\item Columns represent: {``Target'': the names of protein targets;
		``Molecule'': the names of generated molecules and known ligands;
		``Vina S'': the binding affinities between the initially generated poses of molecules and the protein pockets; 
		``Vina M'': the binding affinities between the poses after local structure minimization and the protein pockets;
		``Vina D'': the binding affinities between the poses determined by AutoDock Vina~\cite{Eberhardt2021} and the protein pockets;
		``HA'': the percentage of generated molecules with Vina D higher than those of condition molecules;
		``QED'': the drug-likeness score;
		``SA'': the synthesizability score;
		``Div'': the diversity among generated molecules;
		``time'': the time cost to generate molecules.}
\!\! \par
		\par
		\end{footnotesize}
	\end{tablenotes}
\end{threeparttable}
\end{scriptsize}
  \vspace{-10pt}    
\end{table*}

%\label{tbl:binding_drug_mols}

\begin{table*}
	\centering
		\caption{Drug Properties of Generated Molecules}
	\label{tbl:drug_property}
	\begin{scriptsize}
\begin{threeparttable}
	\begin{tabular}{
		@{\hspace{0pt}}p{0.23\linewidth}@{\hspace{5pt}}
		%
		@{\hspace{1pt}}r@{\hspace{2pt}}
		@{\hspace{2pt}}r@{\hspace{6pt}}
		@{\hspace{6pt}}r@{\hspace{6pt}}
		%
		}
		\toprule
		Property Name & NL-001 & NL-002 & NL-003 \\
		\midrule
Vina S & -6.817 &  -6.970 & -11.953 \\
Vina M & -7.251 & -7.605 & -12.165 \\
Vina D & -8.319 & -8.986 & -12.308 \\
QED    & 0.834  & 0.851  & 0.772 \\
SA       & 0.72    & 0.74    & 0.57    \\
Lipinski & 5 & 5 & 5 \\
%bbbScore          & 3.386                                                                                        & 4.240                                                                                        & 3.892      \\
%drugLikeness      & -0.081                                                                                       & -0.442                                                                                       & -0.325     \\
%molLogP1          & 1.698                                                                                        & 2.685                                                                                        & 2.382      \\
\#rotatable bonds          & 3                                                                                        & 2                                                                                        & 2      \\
molecule weight         & 267.112                                                                                      & 270.117                                                                                      & 390.206    \\
molecule LogP           & 1.698                                                                                        & 2.685                                                                                        & 2.382     \\
\#hydrogen bond doners           & 1                                                                                        & 1                                                                                        & 2      \\
\#hydrogen bond acceptors           & 5                                                                                       & 3                                                                                        & 5      \\
\#molecule charges   & 1                                                                                        & 0                                                                                        & 0      \\
drug-induced liver injury predScore    & 0.227                                                                                        & 0.376                                                                                        & 0.188      \\
drug-induced liver injury predConcern  & Ambiguous/Less concern                                                                       & Ambiguous/Less concern                                                                       & No concern \\
drug-induced liver injury predLabel    & Warnings/Precautions/Adverse reactions & Warnings/Precautions/Adverse reactions & No match   \\
drug-induced liver injury predSeverity & 2                                                                                        & 3                                                                                        & 2      \\
%molSynth1         & 0.254                                                                                        & 0.220                                                                                        & 0.201      \\
%toxicity class         & 0.480                                                                                        & 0.480                                                                                        & 0.450      \\
toxicity names         & hydrazone                                                                                    &   -                                                                                           &   -         \\
toxicity score         & 0.236                                                                                        & 0.000                                                                                        & 0.000      \\
bad groups         & -                                                                                             & Tetrahydroisoquinoline:   allergies                                                          &   -         \\
%MolCovalent       &                                                                                              &                                                                                              &            \\
%MolProdrug        &                                                                                              &                                                                                              &            \\
		\bottomrule
	\end{tabular}%
	\begin{tablenotes}[normal,flushleft]
		\begin{footnotesize}
	\item ``-'': no results found by algorithms
\!\! \par
		\par
		\end{footnotesize}
	\end{tablenotes}
\end{threeparttable}
\end{scriptsize}
  \vspace{-10pt}    
\end{table*}

%\label{tbl:drug_property}

%-------------------------------------------------------------------------------------------------------------------------------------
\subsection{Comparison on ADMET Profiles between Generated Molecules and Approved Drugs}
\label{supp:app:results:properties:admet}
%-------------------------------------------------------------------------------------------------------------------------------------

\paragraph{Generated Molecules for CDK6}
%
Table~\ref{tbl:admet_cdk6} presents the comparison on ADMET profiles between two generated molecules for CDK6 and the approved CDK6 inhibitors, including Abemaciclib~\cite{Patnaik2016}, Palbociclib~\cite{Lu2015}, and Ribociclib~\cite{Tripathy2017}.
%
As shown in Table~\ref{tbl:admet_cdk6}, the generated molecules, NL-001 and NL-002, exhibit comparable ADMET profiles with those of the approved CDK6 inhibitors. 
%
Importantly, both molecules demonstrate good potential in most crucial properties, including Ames mutagenesis, favorable oral toxicity, carcinogenicity, estrogen receptor binding, high intestinal absorption and favorable oral bioavailability.
%
Although the generated molecules are predicted as positive in hepatotoxicity and mitochondrial toxicity, all the approved drugs are also predicted as positive in these two toxicity.
%
This result suggests that these issues might stem from the limited prediction accuracy rather than being specific to our generated molecules.
%
Notably, NL-001 displays a potentially better plasma protein binding score compared to other molecules, which may improve its distribution within the body. 
%
Overall, these results indicate that NL-001 and NL-002 could be promising candidates for further drug development.


\begin{table*}
	\centering
		\caption{Comparison on ADMET Profiles among Generated Molecules and Approved Drugs Targeting CDK6}
	\label{tbl:admet_cdk6}
	\begin{scriptsize}
\begin{threeparttable}
	\begin{tabular}{
		%@{\hspace{0pt}}p{0.23\linewidth}@{\hspace{5pt}}
		%
		@{\hspace{6pt}}l@{\hspace{5pt}}
		@{\hspace{6pt}}r@{\hspace{6pt}}
		@{\hspace{6pt}}r@{\hspace{6pt}}
		@{\hspace{6pt}}r@{\hspace{6pt}}
		@{\hspace{6pt}}r@{\hspace{6pt}}
		@{\hspace{6pt}}r@{\hspace{6pt}}
		%
		%
		@{\hspace{6pt}}r@{\hspace{6pt}}
		%@{\hspace{6pt}}r@{\hspace{6pt}}
		%
		}
		\toprule
		\multirow{2}{*}{Property name} & \multicolumn{2}{c}{Generated molecules} & & \multicolumn{3}{c}{FDA-approved drugs} \\
		\cmidrule{2-3}\cmidrule{5-7}
		 & NL--001 & NL--002 & & Abemaciclib & Palbociclib & Ribociclib \\
		\midrule
\rowcolor[HTML]{D2EAD9}Ames   mutagenesis                             & --   &  --  & & + &  --  & --  \\
\rowcolor[HTML]{D2EAD9}Acute oral toxicity (c)                           & III & III & &  III          & III          & III         \\
Androgen receptor binding                         & +                          & +            &              & +            & +            & +             \\
Aromatase binding                                 & +                          & +            &              & +            & +            & +            \\
Avian toxicity                                    & --                          & --          &                & --            & --            & --            \\
Blood brain barrier                               & +                          & +            &              & +            & +            & +            \\
BRCP inhibitior                                   & --                          & --          &                & --            & --            & --            \\
Biodegradation                                    & --                          & --          &                & --            & --            & --           \\
BSEP inhibitior            & +                          & +            &              & +            & +            & +        \\
Caco-2                                            & +                          & +            &              & --            & --            & --            \\
\rowcolor[HTML]{D2EAD9}Carcinogenicity (binary)                          & --                          & --             &             & --            & --            & --          \\
\rowcolor[HTML]{D2EAD9}Carcinogenicity (trinary)                         & Non-required               & Non-required   &            & Non-required & Non-required & Non-required  \\
Crustacea aquatic toxicity & --                          & --            &              & --            & --            & --            \\
 CYP1A2 inhibition                                 & +                          & +            &              & --            & --            & +             \\
CYP2C19 inhibition                                & --                          & +            &              & +            & --            & +            \\
CYP2C8 inhibition                                 & --                          & --           &               & +            & +            & +            \\
CYP2C9 inhibition                                 & --                          & --           &               & --            & --            & +             \\
CYP2C9 substrate                                  & --                          & --           &               & --            & --            & --            \\
CYP2D6 inhibition                                 & --                          & --           &               & --            & --            & --            \\
CYP2D6 substrate                                  & --                          & --           &               & --            & --            & --            \\
CYP3A4 inhibition                                 & --                          & +            &              & --            & --            & --            \\
CYP3A4 substrate                                  & +                          & --            &              & +            & +            & +            \\
\rowcolor[HTML]{D2EAD9}CYP inhibitory promiscuity                        & +                          & +                    &      & +            & --            & +            \\
Eye corrosion                                     & --                          & --           &               & --            & --            & --            \\
Eye irritation                                    & --                          & --           &               & --            & --            & --             \\
\rowcolor[HTML]{D8E7FF}Estrogen receptor binding                         & +                          & +                    &      & +            & +            & +            \\
Fish aquatic toxicity                             & --                          & +            &              & +            & --            & --            \\
Glucocorticoid receptor   binding                 & +                          & +             &             & +            & +            & +            \\
Honey bee toxicity                                & --                          & --           &               & --            & --            & --            \\
\rowcolor[HTML]{D2EAD9}Hepatotoxicity                                    & +                          & +            &              & +            & +            & +             \\
Human ether-a-go-go-related gene inhibition     & +                          & +               &           & +            & --            & --           \\
\rowcolor[HTML]{D8E7FF}Human intestinal absorption                       & +                          & +             &             & +            & +            & +    \\
\rowcolor[HTML]{D8E7FF}Human oral bioavailability                        & +                          & +              &            & +            & +            & +     \\
\rowcolor[HTML]{D2EAD9}MATE1 inhibitior                                  & --                          & --              &            & --            & --            & --    \\
\rowcolor[HTML]{D2EAD9}Mitochondrial toxicity                            & +                          & +                &          & +            & +            & +    \\
Micronuclear                                      & +                          & +                          & +            & +            & +           \\
\rowcolor[HTML]{D2EAD9}Nephrotoxicity                                    & --                          & --             &             & --            & --            & --             \\
Acute oral toxicity                               & 2.325                      & 1.874    &     & 1.870        & 3.072        & 3.138        \\
\rowcolor[HTML]{D8E7FF}OATP1B1 inhibitior                                & +                          & +              &            & +            & +            & +             \\
\rowcolor[HTML]{D8E7FF}OATP1B3 inhibitior                                & +                          & +              &            & +            & +            & +             \\
\rowcolor[HTML]{D2EAD9}OATP2B1 inhibitior                                & --                          & --             &             & --            & --            & --             \\
OCT1 inhibitior                                   & --                          & --        &                  & +            & --            & +             \\
OCT2 inhibitior                                   & --                          & --        &                  & --            & --            & +             \\
P-glycoprotein inhibitior                         & --                          & --        &                  & +            & +            & +     \\
P-glycoprotein substrate                          & --                          & --        &                  & +            & +            & +     \\
PPAR gamma                                        & +                          & +          &                & +            & +            & +      \\
\rowcolor[HTML]{D8E7FF}Plasma protein binding                            & 0.359        & 0.745     &    & 0.865        & 0.872        & 0.636       \\
Reproductive toxicity                             & +                          & +          &                & +            & +            & +           \\
Respiratory toxicity                              & +                          & +          &                & +            & +            & +         \\
Skin corrosion                                    & --                          & --        &                  & --            & --            & --           \\
Skin irritation                                   & --                          & --        &                  & --            & --            & --         \\
Skin sensitisation                                & --                          & --        &                  & --            & --            & --          \\
Subcellular localzation                           & Mitochondria               & Mitochondria  &             & Lysosomes    & Mitochondria & Mitochondria \\
Tetrahymena pyriformis                            & 0.398                      & 0.903         &             & 1.033        & 1.958        & 1.606         \\
Thyroid receptor binding                          & +                          & +             &             & +            & +            & +           \\
UGT catelyzed                                     & --                          & --           &               & --            & --            & --           \\
\rowcolor[HTML]{D8E7FF}Water solubility                                  & -3.050                     & -3.078              &       & -3.942       & -3.288       & -2.673     \\
		\bottomrule
	\end{tabular}%
	\begin{tablenotes}[normal,flushleft]
		\begin{footnotesize}
	\item Blue cells highlight crucial properties where a negative outcome (``--'') is desired; for acute oral toxicity (c), a higher category (e.g., ``III'') is desired; and for carcinogenicity (trinary), ``Non-required'' is desired.
	%
	Green cells highlight crucial properties where a positive result (``+'') is desired; for plasma protein binding, a lower value is desired; and for water solubility, values higher than -4 are desired~\cite{logs}.
\!\! \par
		\par
		\end{footnotesize}
	\end{tablenotes}
\end{threeparttable}
\end{scriptsize}
  \vspace{--10pt}    
\end{table*}

%\label{tbl:admet_cdk6}

\paragraph{Generated Molecule for NEP}
%
Table~\ref{tbl:admet_nep} presents the comparison on ADMET profiles between a generated molecule for NEP targeting Alzheimer's disease and three approved drugs, Donepezil, Galantamine, and Rivastigmine, for Alzheimer's disease~\cite{Hansen2008}.
%
Overall, NL-003 exhibits a comparable ADMET profile with the three approved drugs.  
%
Notably, same as other approved drugs, NL-003 is predicted to be able to penetrate the blood brain barrier, a crucial property for Alzheimer's disease.
%  
In addition, it demonstrates a promising safety profile in terms of Ames mutagenesis, favorable oral toxicity, carcinogenicity, estrogen receptor binding, high intestinal absorption, nephrotoxicity and so on.
%
These results suggest that NL-003 could be promising candidates for the drug development of Alzheimer's disease.

\begin{table*}
	\centering
		\caption{Comparison on ADMET Profiles among Generated Molecule Targeting NEP and Approved Drugs for Alzhimer's Disease}
	\label{tbl:admet_nep}
	\begin{scriptsize}
\begin{threeparttable}
	\begin{tabular}{
		@{\hspace{6pt}}l@{\hspace{5pt}}
		%
		@{\hspace{6pt}}r@{\hspace{6pt}}
		@{\hspace{6pt}}r@{\hspace{6pt}}
		@{\hspace{6pt}}r@{\hspace{6pt}}
		@{\hspace{6pt}}r@{\hspace{6pt}}
		@{\hspace{6pt}}r@{\hspace{6pt}}
		%
		%
		%@{\hspace{6pt}}r@{\hspace{6pt}}
		%
		}
		\toprule
		\multirow{2}{*}{Property name} & Generated molecule & & \multicolumn{3}{c}{FDA-approved drugs} \\
\cmidrule{2-2}\cmidrule{4-6}
			& NL--003 & & Donepezil	& Galantamine & Rivastigmine \\
		\midrule
\rowcolor[HTML]{D2EAD9} 
Ames   mutagenesis                            & --                      &              & --                                    & --                                 & --                     \\
\rowcolor[HTML]{D2EAD9}Acute oral toxicity (c)                       & III           &                       & III                                  & III                               & II                      \\
Androgen receptor binding                     & +      &      & +            & --         & --         \\
Aromatase binding                             & --     &       & +            & --         & --        \\
Avian toxicity                                & --     &                               & --                                    & --                                 & --                        \\
\rowcolor[HTML]{D8E7FF} 
Blood brain barrier                           & +      &                              & +                                    & +                                 & +                        \\
BRCP inhibitior                               & --     &       & --            & --         & --         \\
Biodegradation                                & --     &                               & --                                    & --                                 & --                        \\
BSEP inhibitior                               & +      &      & +            & --         & --         \\
Caco-2                                        & +      &      & +            & +         & +         \\
\rowcolor[HTML]{D2EAD9} 
Carcinogenicity (binary)                      & --     &                               & --                                    & --                                 & --                        \\
\rowcolor[HTML]{D2EAD9} 
Carcinogenicity (trinary)                     & Non-required    &                     & Non-required                         & Non-required                      & Non-required             \\
Crustacea aquatic toxicity                    & +               &                     & +                                    & +                                 & --                        \\
CYP1A2 inhibition                             & +               &                     & +                                    & --                                 & --                        \\
CYP2C19 inhibition                            & +               &                     & --                                    & --                                 & --                        \\
CYP2C8 inhibition                             & +               &                     & --                                    & --                                 & --                        \\
CYP2C9 inhibition                             & --              &                      & --                                    & --                                 & --                        \\
CYP2C9 substrate                              & --              &                      & --                                    & --                                 & --                        \\
CYP2D6 inhibition                             & --              &                      & +                                    & --                                 & --                        \\
CYP2D6 substrate                              & --              &                      & +                                    & +                                 & +                        \\
CYP3A4 inhibition                             & --              &                      & --                                    & --                                 & --                        \\
CYP3A4 substrate                              & +               &                     & +                                    & +                                 & --                        \\
\rowcolor[HTML]{D2EAD9} 
CYP inhibitory promiscuity                    & +               &                     & +                                    & --                                 & --                        \\
Eye corrosion                                 & --     &       & --            & --         & --         \\
Eye irritation                                & --     &       & --            & --         & --         \\
Estrogen receptor binding                     & +      &      & +            & --         & --         \\
Fish aquatic toxicity                         & --     &                               & +                                    & +                                 & +                        \\
Glucocorticoid receptor binding             & --      &      & +            & --         & --         \\
Honey bee toxicity                            & --    &                                & --                                    & --                                 & --                        \\
\rowcolor[HTML]{D2EAD9} 
Hepatotoxicity                                & +     &                               & +                                    & --                                 & --                        \\
Human ether-a-go-go-related gene inhibition & +       &     & +            & --         & --         \\
\rowcolor[HTML]{D8E7FF} 
Human intestinal absorption                   & +     &                               & +                                    & +                                 & +                        \\
\rowcolor[HTML]{D8E7FF} 
Human oral bioavailability                    & --    &                                & +                                    & +                                 & +                        \\
\rowcolor[HTML]{D2EAD9} 
MATE1 inhibitior                              & --    &                                & --                                    & --                                 & --                        \\
\rowcolor[HTML]{D2EAD9} 
Mitochondrial toxicity                        & +     &                               & +                                    & +                                 & +                        \\
Micronuclear                                  & +     &       & --            & --         & +         \\
\rowcolor[HTML]{D2EAD9} 
Nephrotoxicity                                & --    &                                & --                                    & --                                 & --                        \\
Acute oral toxicity                           & 2.704  &      & 2.098        & 2.767     & 2.726     \\
\rowcolor[HTML]{D8E7FF} 
OATP1B1 inhibitior                            & +      &                              & +                                    & +                                 & +                        \\
\rowcolor[HTML]{D8E7FF} 
OATP1B3 inhibitior                            & +      &                              & +                                    & +                                 & +                        \\
\rowcolor[HTML]{D2EAD9} 
OATP2B1 inhibitior                            & --     &                               & --                                    & --                                 & --                        \\
OCT1 inhibitior                               & +      &      & +            & --         & --         \\
OCT2 inhibitior                               & --     &       & +            & --         & --         \\
P-glycoprotein inhibitior                     & +      &      & +            & --         & --         \\
\rowcolor[HTML]{D8E7FF} 
P-glycoprotein substrate                      & +      &                              & +                                    & +                                 & --                        \\
PPAR gamma                                    & +      &      & --            & --         & --         \\
\rowcolor[HTML]{D8E7FF} 
Plasma protein binding                        & 0.227   &                             & 0.883                                & 0.230                             & 0.606                    \\
Reproductive toxicity                         & +       &     & +            & +         & +         \\
Respiratory toxicity                          & +       &     & +            & +         & +         \\
Skin corrosion                                & --      &      & --            & --         & --         \\
Skin irritation                               & --      &      & --            & --         & --         \\
Skin sensitisation                            & --      &      & --            & --         & --         \\
Subcellular localzation                       & Mitochondria & &Mitochondria & Lysosomes & Mitochondria  \\
Tetrahymena pyriformis                        & 0.053           &                     & 0.979                                & 0.563                             & 0.702                        \\
Thyroid receptor binding                      & +       &     & +            & +         & --             \\
UGT catelyzed                                 & --      &      & --            & +         & --             \\
\rowcolor[HTML]{D8E7FF} 
Water solubility                              & -3.586   &                            & -2.425                               & -2.530                            & -3.062                       \\
		\bottomrule
	\end{tabular}%
	\begin{tablenotes}[normal,flushleft]
		\begin{footnotesize}
	\item Blue cells highlight crucial properties where a negative outcome (``--'') is desired; for acute oral toxicity (c), a higher category (e.g., ``III'') is desired; and for carcinogenicity (trinary), ``Non-required'' is desired.
	%
	Green cells highlight crucial properties where a positive result (``+'') is desired; for plasma protein binding, a lower value is desired; and for water solubility, values higher than -4 are desired~\cite{logs}.
\!\! \par
		\par
		\end{footnotesize}
	\end{tablenotes}
\end{threeparttable}
\end{scriptsize}
  \vspace{--10pt}    
\end{table*}

%\label{tbl:admet_nep}

\clearpage
%%%%%%%%%%%%%%%%%%%%%%%%%%%%%%%%%%%%%%%%%%%%%
\section{Algorithms}
\label{supp:algorithms}
%%%%%%%%%%%%%%%%%%%%%%%%%%%%%%%%%%%%%%%%%%%%%

Algorithm~\ref{alg:shapemol} describes the molecule generation process of \method.
%
Given a known ligand \molx, \method generates a novel molecule \moly that has a similar shape to \molx and thus potentially similar binding activity.
%
\method can also take the protein pocket \pocket as input and adjust the atoms of generated molecules for optimal fit and improved binding affinities.
%
Specifically, \method first calculates the shape embedding \shapehiddenmat for \molx using the shape encoder \SEE described in Algorithm~\ref{alg:see_shaperep}.
%
Based on \shapehiddenmat, \method then generates a novel molecule with a similar shape to \molx using the diffusion-based generative model \methoddiff as in Algorithm~\ref{alg:diffgen}.
%
During generation, \method can use shape guidance to directly modify the shape of \moly to closely resemble the shape of \molx.
%
When the protein pocket \pocket is available, \method can also use pocket guidance to ensure that \moly is specifically tailored to closely fit within \pocket.

\begin{algorithm}[!h]
    \caption{\method}
    \label{alg:shapemol}
         %\hspace*{\algorithmicindent} 
	\textbf{Required Input: $\molx$} \\
 	%\hspace*{\algorithmicindent} 
	\textbf{Optional Input: $\pocket$} 
    \begin{algorithmic}[1]
        \FullLineComment{calculate a shape embedding with Algorithm~\ref{alg:see_shaperep}}
        \State $\shapehiddenmat$, $\pc$ = $\SEE(\molx)$
        \FullLineComment{generate a molecule conditioned on the shape embedding with Algorithm~\ref{alg:diffgen}}
         \If{\pocket is not available}
        \State $\moly = \diffgenerative(\shapehiddenmat, \molx)$
        \Else
        \State $\moly = \diffgenerative(\shapehiddenmat, \molx, \pocket)$
        \EndIf
        \State \Return \moly
    \end{algorithmic}
\end{algorithm}
%\label{alg:shapemol}

\begin{algorithm}[!h]
    \caption{\SEE for shape embedding calculation}
    \label{alg:see_shaperep}
    \textbf{Required Input: $\molx$}
    \begin{algorithmic}[1]
        %\Require $\molx$
        \FullLineComment{sample a point cloud over the molecule surface shape}
        \State $\pc$ = $\text{samplePointCloud}(\molx)$
        \FullLineComment{encode the point cloud into a latent embedding (Equation~\ref{eqn:shape_embed})}
        \State $\shapehiddenmat = \SEE(\pc)$
        \FullLineComment{move the center of \pc to zero}
        \State $\pc = \pc - \text{center}(\pc)$
        \State \Return \shapehiddenmat, \pc
    \end{algorithmic}
\end{algorithm}
%\label{alg:see_shaperep}

\begin{algorithm}[!h]
    \caption{\diffgenerative for molecule generation}
    \label{alg:diffgen}
    	\textbf{Required Input: $\molx$, \shapehiddenmat} \\
 	%\hspace*{\algorithmicindent} 
	\textbf{Optional Input: $\pocket$} 
    \begin{algorithmic}[1]
        \FullLineComment{sample the number of atoms in the generated molecule}
        \State $n = \text{sampleAtomNum}(\molx)$
        \FullLineComment{sample initial positions and types of $n$ atoms}
        \State $\{\pos_T\}^n = \mathcal{N}(0, I)$
        \State $\{\atomfeat_T\}^n = C(K, \frac{1}{K})$
        \FullLineComment{generate a molecule by denoising $\{(\pos_T, \atomfeat_T)\}^n$ to $\{(\pos_0, \atomfeat_0)\}^n$}
        \For{$t = T$ to $1$}
            \IndentLineComment{predict the molecule without noise using the shape-conditioned molecule prediction module \molpred}{1.5}
            \State $(\tilde{\pos}_{0,t}, \tilde{\atomfeat}_{0,t}) = \molpred(\pos_t, \atomfeat_t, \shapehiddenmat)$
            \If{use shape guidance and $t > s$}
                \State $\tilde{\pos}_{0,t} = \shapeguide(\tilde{\pos}_{0,t}, \molx)$
                %\State $\tilde{\pos}_{0,t} = \pos^*_{0,t}$
            \EndIf
            \IndentLineComment{sample $(\pos_{t-1}, \atomfeat_{t-1})$ from $(\pos_t, \atomfeat_t)$ and $(\tilde{\pos}_{0,t}, \tilde{\atomfeat}_{0,t})$}{1.5}
            \State $\pos_{t-1} = P(\pos_{t-1}|\pos_t, \tilde{\pos}_{o,t})$
            \State $\atomfeat_{t-1} = P(\atomfeat_{t-1}|\atomfeat_t, \tilde{\atomfeat}_{o,t})$
            \If{use pocket guidance and $\pocket$ is available}
                \State $\pos_{t-1} = \pocketguide(\pos_{t-1}, \pocket)$
                %\State $\pos_{t-1} = \pos_{t-1}^*$
            \EndIf  
        \EndFor
        \State \Return $\moly = (\pos_0, \atomfeat_0)$
    \end{algorithmic}
\end{algorithm}
%\label{alg:diffgen}

%\input{algorithms/train_SE}
%\label{alg:train_se}

%\begin{algorithm}[!h]
    \caption{Training Procedure of \methoddiff}
    \label{alg:diffgen}
    \begin{algorithmic}[1]
        \Require $\shapehiddenmat, \molx, \pocket$
        \FullLineComment{sample the number of atoms in the generated molecule}
    \end{algorithmic}
\end{algorithm}
%\label{alg:train_diff}

%---------------------------------------------------------------------------------------------------------------------
\section{{Equivariance and Invariance}}
\label{supp:ei}
%---------------------------------------------------------------------------------------------------------------------

%.................................................................................................
\subsection{Equivariance}
\label{supp:ei:equivariance}
%.................................................................................................

{Equivariance refers to the property of a function $f(\pos)$ %\bo{is it the property of the function or embedding (x)?} 
that any translation and rotation transformation from the special Euclidean group SE(3)~\cite{Atz2021} applied to a geometric object
$\pos\in\mathbb{R}^3$ is mirrored in the output of $f(\pos)$, accordingly.
%
This property ensures $f(\pos)$ to learn a consistent representation of an object's geometric information, regardless of its orientation or location in 3D space.
%
%As a result, it provides $f(\pos)$ better generalization capabilities~\cite{Jonas20a}.
%
Formally, given any translation transformation $\mathbf{t}\in\mathbb{R}^3$ and rotation transformation $\mathbf{R}\in\mathbb{R}^{3\times3}$ ($\mathbf{R}^{\mathsf{T}}\mathbf{R}=\mathbb{I}$), %\xia{change the font types for $^{\mathsf{T}}$ and $\mathbb{I}$ in the entire manuscript}), 
$f(\pos)$ is equivariant with respect to these transformations %$g$ (\bo{where is $g$...})
if it satisfies
\begin{equation}
f(\mathbf{R}\pos+\mathbf{t}) = \mathbf{R}f(\pos) + \mathbf{t}. %\ \text{where}\ \hiddenpos = f(\pos).
\end{equation}
%
%where $\hiddenpos=f(\pos)$ is the output of $\pos$. 
%
In \method, both \SE and \methoddiff are developed to guarantee equivariance in capturing the geometric features of objects regardless of any translation or rotation transformations, as will be detailed in the following sections.
}

%.................................................................................................
\subsection{Invariance}
\label{supp:ei:invariance}
%.................................................................................................

%In contrast to equivariance, 
Invariance refers to the property of a function that its output {$f(\pos)$} remains constant under any translation and rotation transformations of the input $\pos$. %a geometric object's feature $\pos$.
%
This property enables $f(\pos)$ to accurately capture %a geometric object's 
the inherent features (e.g., atom features for 3D molecules) that are invariant of its orientation or position in 3D space.
%
Formally, $f(\pos)$ is invariant under any translation $\mathbf{t}$ and  rotation $\mathbf{R}$ if it satisfies
%
\begin{equation}
f(\mathbf{R}\pos+\mathbf{t}) = f(\pos).
\end{equation}
%
In \method, both \SE and \methoddiff capture the inherent features of objects in an invariant way, regardless of any translation or rotation transformations, as will be detailed in the following sections.

%%%%%%%%%%%%%%%%%%%%%%%%%%%%%%%%%%%%%%%%%%%%%
\section{Point Cloud Construction}
\label{supp:point_clouds}
%%%%%%%%%%%%%%%%%%%%%%%%%%%%%%%%%%%%%%%%%%%%%

In \method, we represented molecular surface shapes using point clouds (\pc).
%
$\pc$
serves as input to \SE, from which we derive shape latent embeddings.
%
To generate $\pc$, %\bo{\st{create this}}, \bo{generate $\pc$}
we initially generated a molecular surface mesh using the algorithm from the Open Drug Discovery Toolkit~\cite{Wjcikowski2015oddt}.
%
Following this, we uniformly sampled points on the mesh surface with probability proportional to the face area, %\xia{how to uniformly?}, ensuring the sampling is done proportionally to the face area, with
using the algorithm from PyTorch3D~\cite{ravi2020pytorch3d}.
%
This point cloud $\pc$ is then centralized by setting the center of its points to zero.
%
%

%%%%%%%%%%%%%%%%%%%%%%%%%%%%%%%%%%%%%%%%%%%%%
\section{Query Point Sampling}
\label{supp:training:shapeemb}
%%%%%%%%%%%%%%%%%%%%%%%%%%%%%%%%%%%%%%%%%%%%%

As described in Section ``Shape Decoder (\SED)'', the signed distances of query points $z_q$ to molecule surface shape $\pc$ are used to optimize \SE.
%
In this section, we present how to sample these points $z_q$ in 3D space.
%
Particularly, we first determined the bounding box around the molecular surface shape, using the maximum and minimum \mbox{($x$, $y$, $z$)-axis} coordinates for points in our point cloud \pc,
denoted as $(x_\text{min}, y_\text{min}, z_\text{min})$ and $(x_\text{max}, y_\text{max}, z_\text{max})$.
%
We extended this box slightly by defining its corners as \mbox{$(x_\text{min}-1, y_\text{min}-1, z_\text{min}-1)$} and \mbox{$(x_\text{max}+1, y_\text{max}+1, z_\text{max}+1)$}.
%
For sampling $|\mathcal{Z}|$ query points, we wanted an even distribution of points inside and outside the molecule surface shape.
%
%\ziqi{Typically, within this bounding box, molecules occupy only a small portion of volume, which makes it more likely to sample
%points outside the molecule surface shape.}
%
When a bounding box is defined around the molecule surface shape, there could be a lot of empty spaces within the box that the molecule does not occupy due to 
its complex and irregular shape.
%
This could lead to that fewer points within the molecule surface shape could be sampled within the box.
%
Therefore, we started by randomly sampling $3k$ points within our bounding box to ensure that there are sufficient points within the surface.
%
We then determined whether each point lies within the molecular surface, using an algorithm from Trimesh~\footnote{https://trimsh.org/} based on the molecule surface mesh.
%
If there are $n_w$ points found within the surface, we selected $n=\min(n_w, k/2)$ points from these points, 
and randomly choose the remaining 
%\bo{what do you mean by remaining? If all the 3k sampled points are inside the surface, you get no points left.} 
$k-n$ points 
from those outside the surface.
%
For each query point, we determined its signed distance to the molecule surface by its closest distance to points in \pc with a sign indicating whether it is inside the surface.

%%%%%%%%%%%%%%%%%%%%%%%%%%%%%%%%%%%%%%%%%%%%%
\section{Forward Diffusion (\diffnoise)}
\label{supp:forward}
%%%%%%%%%%%%%%%%%%%%%%%%%%%%%%%%%%%%%%%%%%%%%

%===================================================================
\subsection{{Forward Process}}
\label{supp:forward:forward}
%===================================================================

Formally, for atom positions, the probability of $\pos_t$ sampled given $\pos_{t-1}$, denoted as $q(\pos_t|\pos_{t-1})$, is defined as follows,
%\xia{revise the representation, should be $\beta^x_t$ -- note the space} as follows,
%
\begin{equation}
q(\pos_t|\pos_{t-1}) = \mathcal{N}(\pos_t|\sqrt{1-\beta^{\mathtt{x}}_t}\pos_{t-1}, \beta^{\mathtt{x}}_t\mathbb{I}), 
\label{eqn:noiseposinter}
\end{equation}
%
%\xia{should be a comma after the equation. you also missed it. }
%\st{in which} 
where %\hl{$\pos_0$ denotes the original atom position;} \xia{no $\pos_0$ in the equation...}
%$\mathbf{I}$ denotes the identity matrix;
$\mathcal{N}(\cdot)$ is a Gaussian distribution of $\pos_t$ with mean $\sqrt{1-\beta_t^{\mathtt{x}}}\pos_{t-1}$ and covariance $\beta_t^{\mathtt{x}}\mathbf{I}$.
%\xia{what is $\mathcal{N}$? what is $q$? you abused $q$. need to be crystal clear... }
%\bo{Should be $\sim$ not $=$ in the equation}
%
Following Hoogeboom \etal~\cite{hoogeboom2021catdiff}, 
%the forward process for the discrete atom feature $\atomfeat_t\in\mathbb{R}^K$ adds 
%categorical noise into $\atomfeat_{t-1}$ according to a variance schedule $\beta_t^v\in (0, 1)$. %as follows, %\hl{$\betav_t\in (0, 1)$} as follows,
%\xia{presentation...check across the entire manuscript... } as follows,
%
%\ziqi{Formally, 
for atom features, the probability of $\atomfeat_t$ across $K$ classes given $\atomfeat_{t-1}$ is defined as follows,
%
\begin{equation}
q(\atomfeat_t|\atomfeat_{t-1}) = \mathcal{C}(\atomfeat_t|(1-\beta^{\mathtt{v}}_t) \atomfeat_{t-1}+\beta^{\mathtt{v}}_t\mathbf{1}/K),
\label{eqn:noisetypeinter}
\end{equation}
%
where %\hl{$\atomfeat_0$ denotes the original atom positions}; 
$\mathcal{C}$ is a categorical distribution of $\atomfeat_t$ derived from the %by 
noising $\atomfeat_{t-1}$ with a uniform noise $\beta^{\mathtt{v}}_t\mathbf{1}/K$ across $K$ classes.
%adding an uniform noise $\beta^v_t$ to $\atomfeat_{t-1}$ across K classes.
%\xia{there is always a comma or period after the equations. Equations are part of a sentence. you always missed it. }
%\xia{what is $\mathcal{C}$? what does $q$ mean? it is abused. }

Since the above distributions form Markov chains, %} \xia{grammar!}, 
the probability of any $\pos_t$ or $\atomfeat_t$ can be derived from $\pos_0$ or $\atomfeat_0$:
%samples $\mol_0$ as follows,
%
\begin{eqnarray}
%\begin{aligned}
& q(\pos_t|\pos_{0}) & = \mathcal{N}(\pos_t|\sqrt{\cumalpha^{\mathtt{x}}_t}\pos_0, (1-\cumalpha^{\mathtt{x}}_t)\mathbb{I}), \label{eqn:noisepos}\\
& q(\atomfeat_t|\atomfeat_0)  & = \mathcal{C}(\atomfeat_t|\cumalpha^{\mathtt{v}}_t\atomfeat_0 + (1-\cumalpha^{\mathtt{v}}_t)\mathbf{1}/K), \label{eqn:noisetype}\\
& \text{where }\cumalpha^{\mathtt{u}}_t & = \prod\nolimits_{\tau=1}^{t}\alpha^{\mathtt{u}}_\tau, \ \alpha^{\mathtt{u}}_\tau=1 - \beta^{\mathtt{u}}_\tau, \ {\mathtt{u}}={\mathtt{x}} \text{ or } {\mathtt{v}}.\;\;\;\label{eqn:noiseschedule}
%\end{aligned}
\label{eqn:pos_prior}
\end{eqnarray}
%\xia{always punctuations after equations!!! also use ``eqnarray" instead of ``equation" + ``aligned" for multiple equations, each
%with a separate reference numbering...}
%\st{in which}, 
%where \ziqi{$\cumalpha^u_t = \prod_{\tau=1}^{t}\alpha^u_\tau$ and $\alpha^u_\tau=1 - \beta^u_\tau$ ($u$=$x$ or $v$)}.
%\xia{no such notations in the above equations; also subscript $s$ is abused with shape};
%$K$ is the number of categories for atom features.
%
%The details about noise schedules $\beta^x_t$ and $\beta^v_t$ are available in Supplementary Section \ref{XXX}. \ziqi{add trend}
%
Note that $\bar{\alpha}^{\mathtt{u}}_t$ ($\mathtt{u}={\mathtt{x}}\text{ or }{\mathtt{v}}$)
%($u$=$x$ or $v$) 
is monotonically decreasing from 1 to 0 over $t=[1,T]$. %\xia{=???}. 
%
As $t\rightarrow 1$, $\cumalpha^{\mathtt{x}}_t$ and $\cumalpha^{\mathtt{v}}_t$ are close to 1, leading to that $\pos_t$ or $\atomfeat_t$ approximates 
%the original data 
$\pos_0$ or $\atomfeat_0$.
%
Conversely, as  $t\rightarrow T$, $\cumalpha^{\mathtt{x}}_t$ and $\cumalpha^{\mathtt{v}}_t$ are close to 0,
leading to that $q(\pos_T|\pos_{0})$ %\st{$\rightarrow \mathcal{N}(\mathbf{0}, \mathbf{I})$} 
resembles  {$\mathcal{N}(\mathbf{0}, \mathbb{I})$} 
and $q(\atomfeat_T|\atomfeat_0)$ %\st{$\rightarrow \mathcal{C}(\mathbf{I}/K)$} 
resembles {$\mathcal{C}(\mathbf{1}/K)$}.

Using Bayes theorem, the ground-truth Normal posterior of atom positions $p(\pos_{t-1}|\pos_t, \pos_0)$ can be calculated in a
closed form~\cite{ho2020ddpm} as below,
%
\begin{eqnarray}
& p(\pos_{t-1}|\pos_t, \pos_0) = \mathcal{N}(\pos_{t-1}|\mu(\pos_t, \pos_0), \tilde{\beta}^\mathtt{x}_t\mathbb{I}), \label{eqn:gt_pos_posterior_1}\\
&\!\!\!\!\!\!\!\!\!\!\!\mu(\pos_t, \pos_0)\!=\!\frac{\sqrt{\bar{\alpha}^{\mathtt{x}}_{t-1}}\beta^{\mathtt{x}}_t}{1-\bar{\alpha}^{\mathtt{x}}_t}\pos_0\!+\!\frac{\sqrt{\alpha^{\mathtt{x}}_t}(1-\bar{\alpha}^{\mathtt{x}}_{t-1})}{1-\bar{\alpha}^{\mathtt{x}}_t}\pos_t, 
\tilde{\beta}^\mathtt{x}_t\!=\!\frac{1-\bar{\alpha}^{\mathtt{x}}_{t-1}}{1-\bar{\alpha}^{\mathtt{x}}_{t}}\beta^{\mathtt{x}}_t.\;\;\;
\end{eqnarray}
%
%\xia{Ziqi, please double check the above two equations!}
Similarly, the ground-truth categorical posterior of atom features $p(\atomfeat_{t-1}|\atomfeat_{t}, \atomfeat_0)$ can be calculated~\cite{hoogeboom2021catdiff} as below,
%
\begin{eqnarray}
& p(\atomfeat_{t-1}|\atomfeat_{t}, \atomfeat_0) = \mathcal{C}(\atomfeat_{t-1}|\mathbf{c}(\atomfeat_t, \atomfeat_0)), \label{eqn:gt_atomfeat_posterior_1}\\
& \mathbf{c}(\atomfeat_t, \atomfeat_0) = \tilde{\mathbf{c}}/{\sum_{k=1}^K \tilde{c}_k}, \label{eqn:gt_atomfeat_posterior_2} \\
& \tilde{\mathbf{c}} = [\alpha^{\mathtt{v}}_t\atomfeat_t + \frac{1 - \alpha^{\mathtt{v}}_t}{K}]\odot[\bar{\alpha}^{\mathtt{v}}_{t-1}\atomfeat_{0}+\frac{1-\bar{\alpha}^{\mathtt{v}}_{t-1}}{K}], 
\label{eqn:gt_atomfeat_posterior_3}
%\label{eqn:atomfeat_posterior}
\end{eqnarray}
%
%\xia{Ziqi: please double check the above equations!}
%
where $\tilde{c}_k$ denotes the likelihood of $k$-th class across $K$ classes in $\tilde{\mathbf{c}}$; 
$\odot$ denotes the element-wise product operation;
$\tilde{\mathbf{c}}$ is calculated using $\atomfeat_t$ and $\atomfeat_{0}$ and normalized into $\mathbf{c}(\atomfeat_t, \atomfeat_0)$ so as to represent
probabilities. %\xia{is this correct? is $\tilde{c}_k$ always greater than 0?}
%\xia{how is it calculated?}.
%\ziqi{the likelihood distribution $\tilde{c}$ is calculated by $p(\atomfeat_t|\atomfeat_{t-1})p(\atomfeat_{t-1}|\atomfeat_0)$, according to 
%Equation~\ref{eqn:noisetypeinter} and \ref{eqn:noisetype}.
%\xia{need to write the key idea of the above calculation...}
%
The proof of the above equations is available in Supplementary Section~\ref{supp:forward:proof}.

%===================================================================
\subsection{Variance Scheduling in \diffnoise}
\label{supp:forward:variance}
%===================================================================

Following Guan \etal~\cite{guan2023targetdiff}, we used a sigmoid $\beta$ schedule for the variance schedule $\beta_t^{\mathtt{x}}$ of atom coordinates as below,

\begin{equation}
\beta_t^{\mathtt{x}} = \text{sigmoid}(w_1(2 t / T - 1)) (w_2 - w_3) + w_3
\end{equation}
in which $w_i$($i$=1,2, or 3) are hyperparameters; $T$ is the maximum step.
%
We set $w_1=6$, $w_2=1.e-7$ and $w_3=0.01$.
%
For atom types, we used a cosine $\beta$ schedule~\cite{nichol2021} for $\beta_t^{\mathtt{v}}$ as below,

\begin{equation}
\begin{aligned}
& \bar{\alpha}_t^{\mathtt{v}} = \frac{f(t)}{f(0)}, f(t) = \cos(\frac{t/T+s}{1+s} \cdot \frac{\pi}{2})^2\\
& \beta_t^{\mathtt{v}} = 1 - \alpha_t^{\mathtt{v}} = 1 - \frac{\bar{\alpha}_t^{\mathtt{v}} }{\bar{\alpha}_{t-1}^{\mathtt{v}} }
\end{aligned}
\end{equation}
in which $s$ is a hyperparameter and set as 0.01.

As shown in Section ``Forward Diffusion Process'', the values of $\beta_t^{\mathtt{x}}$ and $\beta_t^{\mathtt{v}}$ should be 
sufficiently small to ensure the smoothness of forward diffusion process. In the meanwhile, their corresponding $\bar{\alpha}_t$
values should decrease from 1 to 0 over $t=[1,T]$.
%
Figure~\ref{fig:schedule} shows the values of $\beta_t$ and $\bar{\alpha}_t$ for atom coordinates and atom types with our hyperparameters.
%
Please note that the value of $\beta_{t}^{\mathtt{x}}$ is less than 0.1 for 990 out of 1,000 steps. %\bo{\st{, though it increases when $t$ is close to 1,000}}.
%
This guarantees the smoothness of the forward diffusion process.
%\bo{add $\beta_t^{\mathtt{x}}$ and $\beta_t^{\mathtt{v}}$ in the legend of the figure...}
%\bo{$\beta_t^{\mathtt{v}}$ does not look small when $t$ is close to 1000...}

\begin{figure}
	\begin{subfigure}[t]{.45\linewidth}
		\centering
		\includegraphics[width=.7\linewidth]{figures/var_schedule_beta.pdf}
	\end{subfigure}
	%
	\hfill
	\begin{subfigure}[t]{.45\linewidth}
		\centering
		\includegraphics[width=.7\linewidth]{figures/var_schedule_alpha.pdf}
	\end{subfigure}
	\caption{Schedule}
	\label{fig:schedule}
\end{figure}

%===================================================================
\subsection{Derivation of Forward Diffusion Process}
\label{supp:forward:proof}
%===================================================================

In \method, a Gaussian noise and a categorical noise are added to continuous atom position and discrete atom features, respectively.
%
Here, we briefly describe the derivation of posterior equations (i.e., Eq.~\ref{eqn:gt_pos_posterior_1}, and   \ref{eqn:gt_atomfeat_posterior_1}) for atom positions and atom types in our work.
%
We refer readers to Ho \etal~\cite{ho2020ddpm} and Kong \etal~\cite{kong2021diffwave} %\bo{add XXX~\etal here...} \cite{ho2020ddpm,kong2021diffwave} 	
for a detailed description of diffusion process for continuous variables and Hoogeboom \etal~\cite{hoogeboom2021catdiff} for
%\bo{add XXX~\etal here...} \cite{hoogeboom2021catdiff} for
the description of diffusion process for discrete variables.

For continuous atom positions, as shown in Kong \etal~\cite{kong2021diffwave}, according to Bayes theorem, given $q(\pos_t|\pos_{t-1})$ defined in Eq.~\ref{eqn:noiseposinter} and 
$q(\pos_t|\pos_0)$ defined in Eq.~\ref{eqn:noisepos}, the posterior $q(\pos_{t-1}|\pos_{t}, \pos_0)$ is derived as below (superscript $\mathtt{x}$ is omitted for brevity),

\begin{equation}
\begin{aligned}
& q(\pos_{t-1}|\pos_{t}, \pos_0)  = \frac{q(\pos_t|\pos_{t-1}, \pos_0)q(\pos_{t-1}|\pos_0)}{q(\pos_t|\pos_0)} \\
& =  \frac{\mathcal{N}(\pos_t|\sqrt{1-\beta_t}\pos_{t-1}, \beta_{t}\mathbf{I}) \mathcal{N}(\pos_{t-1}|\sqrt{\bar{\alpha}_{t-1}}\pos_{0}, (1-\bar{\alpha}_{t-1})\mathbf{I}) }{ \mathcal{N}(\pos_{t}|\sqrt{\bar{\alpha}_t}\pos_{0}, (1-\bar{\alpha}_t)\mathbf{I})}\\
& =  (2\pi{\beta_t})^{-\frac{3}{2}} (2\pi{(1-\bar{\alpha}_{t-1})})^{-\frac{3}{2}} (2\pi(1-\bar{\alpha}_t))^{\frac{3}{2}} \times \exp( \\
& -\frac{\|\pos_t - \sqrt{\alpha}_t\pos_{t-1}\|^2}{2\beta_t} -\frac{\|\pos_{t-1} - \sqrt{\bar{\alpha}}_{t-1}\pos_{0} \|^2}{2(1-\bar{\alpha}_{t-1})} \\
& + \frac{\|\pos_t - \sqrt{\bar{\alpha}_t}\pos_0\|^2}{2(1-\bar{\alpha}_t)}) \\
& = (2\pi\tilde{\beta}_t)^{-\frac{3}{2}} \exp(-\frac{1}{2\tilde{\beta}_t}\|\pos_{t-1}-\frac{\sqrt{\bar{\alpha}_{t-1}}\beta_t}{1-\bar{\alpha}_t}\pos_0 \\
& - \frac{\sqrt{\alpha_t}(1-\bar{\alpha}_{t-1})}{1-\bar{\alpha}_t}\pos_{t}\|^2) \\
& \text{where}\ \tilde{\beta}_t = \frac{1-\bar{\alpha}_{t-1}}{1-\bar{\alpha}_t}\beta_t.
\end{aligned}
\end{equation}
%\bo{marked part does not look right to me.}
%\bo{How to you derive from the second equation to the third one?}

Therefore, the posterior of atom positions is derived as below,

\begin{equation}
q(\pos_{t-1}|\pos_{t}, \pos_0)\!\!=\!\!\mathcal{N}(\pos_{t-1}|\frac{\sqrt{\bar{\alpha}_{t-1}}\beta_t}{1-\bar{\alpha}_t}\pos_0 + \frac{\sqrt{\alpha_t}(1-\bar{\alpha}_{t-1})}{1-\bar{\alpha}_t}\pos_{t}, \tilde{\beta}_t\mathbf{I}).
\end{equation}

For discrete atom features, as shown in Hoogeboom \etal~\cite{hoogeboom2021catdiff} and Guan \etal~\cite{guan2023targetdiff},
according to Bayes theorem, the posterior $q(\atomfeat_{t-1}|\atomfeat_{t}, \atomfeat_0)$ is derived as below (supperscript $\mathtt{v}$ is omitted for brevity),

\begin{equation}
\begin{aligned}
& q(\atomfeat_{t-1}|\atomfeat_{t}, \atomfeat_0) =  \frac{q(\atomfeat_t|\atomfeat_{t-1}, \atomfeat_0)q(\atomfeat_{t-1}|\atomfeat_0)}{\sum_{\scriptsize{\atomfeat}_{t-1}}q(\atomfeat_t|\atomfeat_{t-1}, \atomfeat_0)q(\atomfeat_{t-1}|\atomfeat_0)} \\
%& = \frac{\mathcal{C}(\atomfeat_t|(1-\beta_t)\atomfeat_{t-1} + \beta_t\frac{\mathbf{1}}{K}) \mathcal{C}(\atomfeat_{t-1}|\bar{\alpha}_{t-1}\atomfeat_0+(1-\bar{\alpha}_{t-1})\frac{\mathbf{1}}{K})} \\
\end{aligned}
\end{equation}

For $q(\atomfeat_t|\atomfeat_{t-1}, \atomfeat_0)$, we have % $\atomfeat_t=\atomfeat_{t-1}$ with probability $1-\beta_t+\beta_t / K$, and $\atomfeat_t \neq \atomfeat_{t-1}$
%with probability $\beta_t / K$.
%
%Therefore, this function can be symmetric, that is, 
%
\begin{equation}
\begin{aligned}
q(\atomfeat_t|\atomfeat_{t-1}, \atomfeat_0) & = \mathcal{C}(\atomfeat_t|(1-\beta_t)\atomfeat_{t-1} + \beta_t/{K})\\
& = \begin{cases}
1-\beta_t+\beta_t/K,\!&\text{when}\ \atomfeat_{t} = \atomfeat_{t-1},\\
\beta_t / K,\! &\text{when}\ \atomfeat_{t} \neq \atomfeat_{t-1},
\end{cases}\\
& = \mathcal{C}(\atomfeat_{t-1}|(1-\beta_t)\atomfeat_{t} + \beta_t/{K}).
\end{aligned}
%\mathcal{C}(\atomfeat_{t-1}|(1-\beta_{t})\atomfeat_{t} + \beta_t\frac{\mathbf{1}}{K}).
\end{equation}
%
Therefore, we have
%\bo{why it can be symmetric}
%
\begin{equation}
\begin{aligned}
& q(\atomfeat_t|\atomfeat_{t-1}, \atomfeat_0)q(\atomfeat_{t-1}|\atomfeat_0) \\
& = \mathcal{C}(\atomfeat_{t-1}|(1-\beta_t)\atomfeat_{t} + \beta_t\frac{\mathbf{1}}{K}) \mathcal{C}(\atomfeat_{t-1}|\bar{\alpha}_{t-1}\atomfeat_0+(1-\bar{\alpha}_{t-1})\frac{\mathbf{1}}{K}) \\
& = [\alpha_t\atomfeat_t + \frac{1 - \alpha_t}{K}]\odot[\bar{\alpha}_{t-1}\atomfeat_{0}+\frac{1-\bar{\alpha}_{t-1}}{K}].
\end{aligned}
\end{equation}
%
%\bo{what is $\tilde{\mathbf{c}}$...}
Therefore, with $q(\atomfeat_t|\atomfeat_{t-1}, \atomfeat_0)q(\atomfeat_{t-1}|\atomfeat_0) = \tilde{\mathbf{c}}$, the posterior is as below,

\begin{equation}
q(\atomfeat_{t-1}|\atomfeat_{t}, \atomfeat_0) = \mathcal{C}(\atomfeat_{t-1}|\mathbf{c}(\atomfeat_t, \atomfeat_0)) = \frac{\tilde{\mathbf{c}}}{\sum_{k}^K\tilde{c}_k}.
\end{equation}

%%%%%%%%%%%%%%%%%%%%%%%%%%%%%%%%%%%%%%%%%%%%%
\section{{Backward Generative Process} (\diffgenerative)}
\label{supp:backward}
%%%%%%%%%%%%%%%%%%%%%%%%%%%%%%%%%%%%%%%%%%%%%

Following Ho \etal~\cite{ho2020ddpm}, with $\tilde{\pos}_{0,t}$, the probability of $\pos_{t-1}$ denoised from $\pos_t$, denoted as $p(\pos_{t-1}|\pos_t)$,
can be estimated %\hl{parameterized} \xia{???} 
by the approximated posterior $p_{\boldsymbol{\Theta}}(\pos_{t-1}|\pos_t, \tilde{\pos}_{0,t})$ as below,
%
\begin{equation}
\begin{aligned}
p(\pos_{t-1}|\pos_t) & \approx p_{\boldsymbol{\Theta}}(\pos_{t-1}|\pos_t, \tilde{\pos}_{0,t}) \\
& = \mathcal{N}(\pos_{t-1}|\mu_{\boldsymbol{\Theta}}(\pos_t, \tilde{\pos}_{0,t}),\tilde{\beta}_t^{\mathtt{x}}\mathbb{I}),
\end{aligned}
\label{eqn:aprox_pos_posterior}
\end{equation}
%
where ${\boldsymbol{\Theta}}$ is the learnable parameter; $\mu_{\boldsymbol{\Theta}}(\pos_t, \tilde{\pos}_{0,t})$ is an estimate %estimation
of $\mu(\pos_t, \pos_{0})$ by replacing $\pos_0$ with its estimate $\tilde{\pos}_{0,t}$ 
in Equation~{\ref{eqn:gt_pos_posterior_1}}.
%
Similarly, with $\tilde{\atomfeat}_{0,t}$, the probability of $\atomfeat_{t-1}$ denoised from $\atomfeat_t$, denoted as $p(\atomfeat_{t-1}|\atomfeat_t)$, 
can be estimated %\hl{parameterized} 
by the approximated posterior $p_{\boldsymbol{\Theta}}(\atomfeat_{t-1}|\atomfeat_t, \tilde{\atomfeat}_{0,t})$ as below,
%
\begin{equation}
\begin{aligned}
p(\atomfeat_{t-1}|\atomfeat_t)\approx p_{\boldsymbol{\Theta}}(\atomfeat_{t-1}|\atomfeat_{t}, \tilde{\atomfeat}_{0,t}) 
=\mathcal{C}(\atomfeat_{t-1}|\mathbf{c}_{\boldsymbol{\Theta}}(\atomfeat_t, \tilde{\atomfeat}_{0,t})),\!\!\!\!
\end{aligned}
\label{eqn:aprox_atomfeat_posterior}
\end{equation}
%
where $\mathbf{c}_{\boldsymbol{\Theta}}(\atomfeat_t, \tilde{\atomfeat}_{0,t})$ is an estimate of $\mathbf{c}(\atomfeat_t, \atomfeat_0)$
by replacing $\atomfeat_0$  
with its estimate $\tilde{\atomfeat}_{0,t}$ in Equation~\ref{eqn:gt_atomfeat_posterior_1}.



%===================================================================
\section{\method Loss Function Derivation}
\label{supp:training:loss}
%===================================================================

In this section, we demonstrate that a step weight $w_t^{\mathtt{x}}$ based on the signal-to-noise ratio $\lambda_t$ should be 
included into the atom position loss (Eq.~\ref{eqn:diff:obj:pos}).
%
In the diffusion process for continuous variables, the optimization problem is defined 
as below~\cite{ho2020ddpm},
%
\begin{equation*}
\begin{aligned}
& \arg\min_{\boldsymbol{\Theta}}KL(q(\pos_{t-1}|\pos_t, \pos_0)|p_{\boldsymbol{\Theta}}(\pos_{t-1}|\pos_t, \tilde{\pos}_{0,t})) \\
& = \arg\min_{\boldsymbol{\Theta}} \frac{\bar{\alpha}_{t-1}(1-\alpha_t)}{2(1-\bar{\alpha}_{t-1})(1-\bar{\alpha}_{t})}\|\tilde{\pos}_{0, t}-\pos_0\|^2 \\
& = \arg\min_{\boldsymbol{\Theta}} \frac{1-\alpha_t}{2(1-\bar{\alpha}_{t-1})\alpha_{t}} \|\tilde{\boldsymbol{\epsilon}}_{0,t}-\boldsymbol{\epsilon}_0\|^2,
\end{aligned}
\end{equation*}
where $\boldsymbol{\epsilon}_0 = \frac{\pos_t - \sqrt{\bar{\alpha}_t}\pos_0}{\sqrt{1-\bar{\alpha}_t}}$ is the ground-truth noise variable sampled from $\mathcal{N}(\mathbf{0}, \mathbf{1})$ and is used to sample $\pos_t$ from $\mathcal{N}(\pos_t|\sqrt{\cumalpha_t}\pos_0, (1-\cumalpha_t)\mathbf{I})$ in Eq.~\ref{eqn:noisetype};
$\tilde{\boldsymbol{\epsilon}}_0 = \frac{\pos_t - \sqrt{\bar{\alpha}_t}\tilde{\pos}_{0, t}}{\sqrt{1-\bar{\alpha}_t}}$ is the predicted noise variable. 

%A simplified training objective is proposed by Ho \etal~\cite{ho2020ddpm} as below,
Ho \etal~\cite{ho2020ddpm} further simplified the above objective as below and
demonstrated that the simplified one can achieve better performance:
%
\begin{equation}
\begin{aligned}
& \arg\min_{\boldsymbol{\Theta}} \|\tilde{\boldsymbol{\epsilon}}_{0,t}-\boldsymbol{\epsilon}_0\|^2 \\
& = \arg\min_{\boldsymbol{\Theta}} \frac{\bar{\alpha}_t}{1-\bar{\alpha}_t}\|\tilde{\pos}_{0,t}-\pos_0\|^2,
\end{aligned}
\label{eqn:supp:losspos}
\end{equation}
%
where $\lambda_t=\frac{\bar{\alpha}_t}{1-\bar{\alpha}_t}$ is the signal-to-noise ratio.
%
While previous work~\cite{guan2023targetdiff} applies uniform step weights across
different steps, we demonstrate that a step weight should be included into the atom position loss according to Eq.~\ref{eqn:supp:losspos}.
%
However, the value of $\lambda_t$ could be very large when $\bar{\alpha}_t$ is close to 1 as $t$ approaches 1.
%
Therefore, we clip the value of $\lambda_t$ with threshold $\delta$ in Eq.~\ref{eqn:diff:obj:pos}.










\end{document}


