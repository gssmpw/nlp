\section{Related Work}
\label{sec:related}

The Linux Foundation conducted an SBOM readiness survey in 2022 targeting 412 organizations of various industry types and sizes \cite{linux-foundation-readiness}. While 76\% of organizations work on SBOM adoption, the survey revealed concerns such as the industry's honest commitment to SBOM adoption, the lack of industry consensus on what an SBOM should contain, and the value to their customers by providing them with an SBOM being unclear. Xia et al. conducted an interview on the adoption status and challenges of SBOM among developers, highlighting the importance of incentives for SBOM generation, industry consensus on what to include in SBOM, mechanisms for selective sharing of SBOM content, SBOM content validation/verification, mature SBOM tools, and increasing awareness of SBOM \cite{sbom-study}. Stalnaker et al. investigated the challenges stakeholders face during SBOM creation and usage, emphasizing the importance of multi-dimensional SBOM specifications, enhanced SBOM tooling and build system support, SBOM content validation, and incentives for SBOM adoption \cite{bomsaway-stakeholders-study}.

While these studies provide insights into the current status and future challenges of SBOM adoption in the industry, they do not target ordinal software developers outside of companies or organizations. In contrast, Nocera et al. analyzed the SBOM adoption on open-source software (OSS) repositories on GitHub, the most popular version control platform for both personal and professional use \cite{sbom-study-github}. The results indicated a low adoption of SBOM in OSS repositories, with few repositories meeting the guidelines set by the National Telecommunications and Information Administration (NTIA) for how SBOM should be supplied, highlighting the importance of convenient SBOM generation tools that can be easily integrated into Continuous Integration and Continuous Delivery (CI/CD) pipelines and build-automation tools. However, this study provides quantitative insights and does not reveal the concrete challenges software developers face.