% This must be in the first 5 lines to tell arXiv to use pdfLaTeX, which is strongly recommended.
\pdfoutput=1
% In particular, the hyperref package requires pdfLaTeX in order to break URLs across lines.

\documentclass[11pt]{article}
\usepackage{booktabs}
\usepackage{enumitem}
\usepackage{bbding}
% Change "review" to "final" to generate the final (sometimes called camera-ready) version.
% Change to "preprint" to generate a non-anonymous version with page numbers.
%\usepackage[review]{acl}
\usepackage[preprint]{acl}
\usepackage{multirow}
% Standard package includes
\usepackage{times}
\usepackage{latexsym}

% For proper rendering and hyphenation of words containing Latin characters (including in bib files)
\usepackage[T1]{fontenc}
% For Vietnamese characters
% \usepackage[T5]{fontenc}
% See https://www.latex-project.org/help/documentation/encguide.pdf for other character sets

% This assumes your files are encoded as UTF8
\usepackage[utf8]{inputenc}

% This is not strictly necessary, and may be commented out,
% but it will improve the layout of the manuscript,
% and will typically save some space.
\usepackage{microtype}

% This is also not strictly necessary, and may be commented out.
% However, it will improve the aesthetics of text in
% the typewriter font.
\usepackage{inconsolata}

%Including images in your LaTeX document requires adding
%additional package(s)
\usepackage{graphicx}
\usepackage{listings}
\usepackage{amsmath} 
\usepackage[normalem]{ulem}
\useunder{\uline}{\ul}{}
% If the title and author information does not fit in the area allocated, uncomment the following
%
%\setlength\titlebox{<dim>}
%
% and set <dim> to something 5cm or larger.


\title{A Meta-Evaluation of Style and Attribute Transfer Metrics}



% Author information can be set in various styles:
% For several authors from the same institution:
% \author{Author 1 \and ... \and Author n \\
%         Address line \\ ... \\ Address line}
% if the names do not fit well on one line use
%         Author 1 \\ {\bf Author 2} \\ ... \\ {\bf Author n} \\
% For authors from different institutions:
% \author{Author 1 \\ Address line \\  ... \\ Address line
%         \And  ... \And
%         Author n \\ Address line \\ ... \\ Address line}
% To start a separate ``row'' of authors use \AND, as in
% \author{Author 1 \\ Address line \\  ... \\ Address line
%         \AND
%         Author 2 \\ Address line \\ ... \\ Address line \And
%         Author 3 \\ Address line \\ ... \\ Address line}

\author{Amalie Brogaard Pauli$^1$ \ \ \ \  \ \   Isabelle Augenstein$^2$  \ \ \ \ \ \    Ira Assent$^1$ \\
         $^1$Department of Computer Science, Aarhus University, Denmark  \\ $^2$Department of Computer Science, University of Copenhagen, Denmark  \\  \texttt{\{ampa,ira\}@cs.au.dk, augenstein@di.ku.dk}}

%\author{
%  \textbf{First Author\textsuperscript{1}},
%  \textbf{Second Author\textsuperscript{1,2}},
%  \textbf{Third T. Author\textsuperscript{1}},
%  \textbf{Fourth Author\textsuperscript{1}},
%\\
%  \textbf{Fifth Author\textsuperscript{1,2}},
%  \textbf{Sixth Author\textsuperscript{1}},
%  \textbf{Seventh Author\textsuperscript{1}},
%  \textbf{Eighth Author \textsuperscript{1,2,3,4}},
%\\
%  \textbf{Ninth Author\textsuperscript{1}},
%  \textbf{Tenth Author\textsuperscript{1}},
%  \textbf{Eleventh E. Author\textsuperscript{1,2,3,4,5}},
%  \textbf{Twelfth Author\textsuperscript{1}},
%\\
%  \textbf{Thirteenth Author\textsuperscript{3}},
%  \textbf{Fourteenth F. Author\textsuperscript{2,4}},
%  \textbf{Fifteenth Author\textsuperscript{1}},
%  \textbf{Sixteenth Author\textsuperscript{1}},
%\\
%  \textbf{Seventeenth S. Author\textsuperscript{4,5}},
%  \textbf{Eighteenth Author\textsuperscript{3,4}},
%  \textbf{Nineteenth N. Author\textsuperscript{2,5}},
%  \textbf{Twentieth Author\textsuperscript{1}}
%\\
%\\
%  \textsuperscript{1}Affiliation 1,
%  \textsuperscript{2}Affiliation 2,
%  \textsuperscript{3}Affiliation 3,
%  \textsuperscript{4}Affiliation 4,
%  \textsuperscript{5}Affiliation 5
%\\
%  \small{
%    \textbf{Correspondence:} \href{mailto:email@domain}{email@domain}
%  }
%}

\begin{document}
\maketitle
\begin{abstract} 
LLMs make it easy to rewrite text in any style, be it more polite, persuasive, or more positive. 
We present a large-scale study of evaluation metrics for style and attribute transfer with a focus on \emph{content preservation; meaning content not attributed to the style shift is preserved}. The de facto evaluation approach uses lexical or semantic similarity metrics often between source sentences and rewrites. While these metrics are not designed to distinguish between style or content differences, empirical meta-evaluation shows a reasonable correlation to human judgment. In fact, recent works find that LLMs prompted as evaluators are only comparable to semantic similarity metrics, even though intuitively, the LLM approach should better fit the task. 
To investigate this discrepancy, we benchmark 8 metrics for evaluating content preservation on existing datasets and additionally construct a new test set that better aligns with the meta-evaluation aim. Indeed, we then find that the empirical conclusion aligns with the intuition: content preservation metrics for style/attribute transfer must be conditional on the style shift. To support this, we propose a new efficient zero-shot evaluation method using the likelihood of the next token.
We hope our meta-evaluation can foster more research on evaluating content preservation metrics, and also to ensure fair evaluation of methods for conducting style transfer. 

\end{abstract}

\section{Introduction}
\label{sec:introduction}
The business processes of organizations are experiencing ever-increasing complexity due to the large amount of data, high number of users, and high-tech devices involved \cite{martin2021pmopportunitieschallenges, beerepoot2023biggestbpmproblems}. This complexity may cause business processes to deviate from normal control flow due to unforeseen and disruptive anomalies \cite{adams2023proceddsriftdetection}. These control-flow anomalies manifest as unknown, skipped, and wrongly-ordered activities in the traces of event logs monitored from the execution of business processes \cite{ko2023adsystematicreview}. For the sake of clarity, let us consider an illustrative example of such anomalies. Figure \ref{FP_ANOMALIES} shows a so-called event log footprint, which captures the control flow relations of four activities of a hypothetical event log. In particular, this footprint captures the control-flow relations between activities \texttt{a}, \texttt{b}, \texttt{c} and \texttt{d}. These are the causal ($\rightarrow$) relation, concurrent ($\parallel$) relation, and other ($\#$) relations such as exclusivity or non-local dependency \cite{aalst2022pmhandbook}. In addition, on the right are six traces, of which five exhibit skipped, wrongly-ordered and unknown control-flow anomalies. For example, $\langle$\texttt{a b d}$\rangle$ has a skipped activity, which is \texttt{c}. Because of this skipped activity, the control-flow relation \texttt{b}$\,\#\,$\texttt{d} is violated, since \texttt{d} directly follows \texttt{b} in the anomalous trace.
\begin{figure}[!t]
\centering
\includegraphics[width=0.9\columnwidth]{images/FP_ANOMALIES.png}
\caption{An example event log footprint with six traces, of which five exhibit control-flow anomalies.}
\label{FP_ANOMALIES}
\end{figure}

\subsection{Control-flow anomaly detection}
Control-flow anomaly detection techniques aim to characterize the normal control flow from event logs and verify whether these deviations occur in new event logs \cite{ko2023adsystematicreview}. To develop control-flow anomaly detection techniques, \revision{process mining} has seen widespread adoption owing to process discovery and \revision{conformance checking}. On the one hand, process discovery is a set of algorithms that encode control-flow relations as a set of model elements and constraints according to a given modeling formalism \cite{aalst2022pmhandbook}; hereafter, we refer to the Petri net, a widespread modeling formalism. On the other hand, \revision{conformance checking} is an explainable set of algorithms that allows linking any deviations with the reference Petri net and providing the fitness measure, namely a measure of how much the Petri net fits the new event log \cite{aalst2022pmhandbook}. Many control-flow anomaly detection techniques based on \revision{conformance checking} (hereafter, \revision{conformance checking}-based techniques) use the fitness measure to determine whether an event log is anomalous \cite{bezerra2009pmad, bezerra2013adlogspais, myers2018icsadpm, pecchia2020applicationfailuresanalysispm}. 

The scientific literature also includes many \revision{conformance checking}-independent techniques for control-flow anomaly detection that combine specific types of trace encodings with machine/deep learning \cite{ko2023adsystematicreview, tavares2023pmtraceencoding}. Whereas these techniques are very effective, their explainability is challenging due to both the type of trace encoding employed and the machine/deep learning model used \cite{rawal2022trustworthyaiadvances,li2023explainablead}. Hence, in the following, we focus on the shortcomings of \revision{conformance checking}-based techniques to investigate whether it is possible to support the development of competitive control-flow anomaly detection techniques while maintaining the explainable nature of \revision{conformance checking}.
\begin{figure}[!t]
\centering
\includegraphics[width=\columnwidth]{images/HIGH_LEVEL_VIEW.png}
\caption{A high-level view of the proposed framework for combining \revision{process mining}-based feature extraction with dimensionality reduction for control-flow anomaly detection.}
\label{HIGH_LEVEL_VIEW}
\end{figure}

\subsection{Shortcomings of \revision{conformance checking}-based techniques}
Unfortunately, the detection effectiveness of \revision{conformance checking}-based techniques is affected by noisy data and low-quality Petri nets, which may be due to human errors in the modeling process or representational bias of process discovery algorithms \cite{bezerra2013adlogspais, pecchia2020applicationfailuresanalysispm, aalst2016pm}. Specifically, on the one hand, noisy data may introduce infrequent and deceptive control-flow relations that may result in inconsistent fitness measures, whereas, on the other hand, checking event logs against a low-quality Petri net could lead to an unreliable distribution of fitness measures. Nonetheless, such Petri nets can still be used as references to obtain insightful information for \revision{process mining}-based feature extraction, supporting the development of competitive and explainable \revision{conformance checking}-based techniques for control-flow anomaly detection despite the problems above. For example, a few works outline that token-based \revision{conformance checking} can be used for \revision{process mining}-based feature extraction to build tabular data and develop effective \revision{conformance checking}-based techniques for control-flow anomaly detection \cite{singh2022lapmsh, debenedictis2023dtadiiot}. However, to the best of our knowledge, the scientific literature lacks a structured proposal for \revision{process mining}-based feature extraction using the state-of-the-art \revision{conformance checking} variant, namely alignment-based \revision{conformance checking}.

\subsection{Contributions}
We propose a novel \revision{process mining}-based feature extraction approach with alignment-based \revision{conformance checking}. This variant aligns the deviating control flow with a reference Petri net; the resulting alignment can be inspected to extract additional statistics such as the number of times a given activity caused mismatches \cite{aalst2022pmhandbook}. We integrate this approach into a flexible and explainable framework for developing techniques for control-flow anomaly detection. The framework combines \revision{process mining}-based feature extraction and dimensionality reduction to handle high-dimensional feature sets, achieve detection effectiveness, and support explainability. Notably, in addition to our proposed \revision{process mining}-based feature extraction approach, the framework allows employing other approaches, enabling a fair comparison of multiple \revision{conformance checking}-based and \revision{conformance checking}-independent techniques for control-flow anomaly detection. Figure \ref{HIGH_LEVEL_VIEW} shows a high-level view of the framework. Business processes are monitored, and event logs obtained from the database of information systems. Subsequently, \revision{process mining}-based feature extraction is applied to these event logs and tabular data input to dimensionality reduction to identify control-flow anomalies. We apply several \revision{conformance checking}-based and \revision{conformance checking}-independent framework techniques to publicly available datasets, simulated data of a case study from railways, and real-world data of a case study from healthcare. We show that the framework techniques implementing our approach outperform the baseline \revision{conformance checking}-based techniques while maintaining the explainable nature of \revision{conformance checking}.

In summary, the contributions of this paper are as follows.
\begin{itemize}
    \item{
        A novel \revision{process mining}-based feature extraction approach to support the development of competitive and explainable \revision{conformance checking}-based techniques for control-flow anomaly detection.
    }
    \item{
        A flexible and explainable framework for developing techniques for control-flow anomaly detection using \revision{process mining}-based feature extraction and dimensionality reduction.
    }
    \item{
        Application to synthetic and real-world datasets of several \revision{conformance checking}-based and \revision{conformance checking}-independent framework techniques, evaluating their detection effectiveness and explainability.
    }
\end{itemize}

The rest of the paper is organized as follows.
\begin{itemize}
    \item Section \ref{sec:related_work} reviews the existing techniques for control-flow anomaly detection, categorizing them into \revision{conformance checking}-based and \revision{conformance checking}-independent techniques.
    \item Section \ref{sec:abccfe} provides the preliminaries of \revision{process mining} to establish the notation used throughout the paper, and delves into the details of the proposed \revision{process mining}-based feature extraction approach with alignment-based \revision{conformance checking}.
    \item Section \ref{sec:framework} describes the framework for developing \revision{conformance checking}-based and \revision{conformance checking}-independent techniques for control-flow anomaly detection that combine \revision{process mining}-based feature extraction and dimensionality reduction.
    \item Section \ref{sec:evaluation} presents the experiments conducted with multiple framework and baseline techniques using data from publicly available datasets and case studies.
    \item Section \ref{sec:conclusions} draws the conclusions and presents future work.
\end{itemize}


\section{RELATED WORK}
\label{sec:relatedwork}
In this section, we describe the previous works related to our proposal, which are divided into two parts. In Section~\ref{sec:relatedwork_exoplanet}, we present a review of approaches based on machine learning techniques for the detection of planetary transit signals. Section~\ref{sec:relatedwork_attention} provides an account of the approaches based on attention mechanisms applied in Astronomy.\par

\subsection{Exoplanet detection}
\label{sec:relatedwork_exoplanet}
Machine learning methods have achieved great performance for the automatic selection of exoplanet transit signals. One of the earliest applications of machine learning is a model named Autovetter \citep{MCcauliff}, which is a random forest (RF) model based on characteristics derived from Kepler pipeline statistics to classify exoplanet and false positive signals. Then, other studies emerged that also used supervised learning. \cite{mislis2016sidra} also used a RF, but unlike the work by \citet{MCcauliff}, they used simulated light curves and a box least square \citep[BLS;][]{kovacs2002box}-based periodogram to search for transiting exoplanets. \citet{thompson2015machine} proposed a k-nearest neighbors model for Kepler data to determine if a given signal has similarity to known transits. Unsupervised learning techniques were also applied, such as self-organizing maps (SOM), proposed \citet{armstrong2016transit}; which implements an architecture to segment similar light curves. In the same way, \citet{armstrong2018automatic} developed a combination of supervised and unsupervised learning, including RF and SOM models. In general, these approaches require a previous phase of feature engineering for each light curve. \par

%DL is a modern data-driven technology that automatically extracts characteristics, and that has been successful in classification problems from a variety of application domains. The architecture relies on several layers of NNs of simple interconnected units and uses layers to build increasingly complex and useful features by means of linear and non-linear transformation. This family of models is capable of generating increasingly high-level representations \citep{lecun2015deep}.

The application of DL for exoplanetary signal detection has evolved rapidly in recent years and has become very popular in planetary science.  \citet{pearson2018} and \citet{zucker2018shallow} developed CNN-based algorithms that learn from synthetic data to search for exoplanets. Perhaps one of the most successful applications of the DL models in transit detection was that of \citet{Shallue_2018}; who, in collaboration with Google, proposed a CNN named AstroNet that recognizes exoplanet signals in real data from Kepler. AstroNet uses the training set of labelled TCEs from the Autovetter planet candidate catalog of Q1–Q17 data release 24 (DR24) of the Kepler mission \citep{catanzarite2015autovetter}. AstroNet analyses the data in two views: a ``global view'', and ``local view'' \citep{Shallue_2018}. \par


% The global view shows the characteristics of the light curve over an orbital period, and a local view shows the moment at occurring the transit in detail

%different = space-based

Based on AstroNet, researchers have modified the original AstroNet model to rank candidates from different surveys, specifically for Kepler and TESS missions. \citet{ansdell2018scientific} developed a CNN trained on Kepler data, and included for the first time the information on the centroids, showing that the model improves performance considerably. Then, \citet{osborn2020rapid} and \citet{yu2019identifying} also included the centroids information, but in addition, \citet{osborn2020rapid} included information of the stellar and transit parameters. Finally, \citet{rao2021nigraha} proposed a pipeline that includes a new ``half-phase'' view of the transit signal. This half-phase view represents a transit view with a different time and phase. The purpose of this view is to recover any possible secondary eclipse (the object hiding behind the disk of the primary star).


%last pipeline applies a procedure after the prediction of the model to obtain new candidates, this process is carried out through a series of steps that include the evaluation with Discovery and Validation of Exoplanets (DAVE) \citet{kostov2019discovery} that was adapted for the TESS telescope.\par
%



\subsection{Attention mechanisms in astronomy}
\label{sec:relatedwork_attention}
Despite the remarkable success of attention mechanisms in sequential data, few papers have exploited their advantages in astronomy. In particular, there are no models based on attention mechanisms for detecting planets. Below we present a summary of the main applications of this modeling approach to astronomy, based on two points of view; performance and interpretability of the model.\par
%Attention mechanisms have not yet been explored in all sub-areas of astronomy. However, recent works show a successful application of the mechanism.
%performance

The application of attention mechanisms has shown improvements in the performance of some regression and classification tasks compared to previous approaches. One of the first implementations of the attention mechanism was to find gravitational lenses proposed by \citet{thuruthipilly2021finding}. They designed 21 self-attention-based encoder models, where each model was trained separately with 18,000 simulated images, demonstrating that the model based on the Transformer has a better performance and uses fewer trainable parameters compared to CNN. A novel application was proposed by \citet{lin2021galaxy} for the morphological classification of galaxies, who used an architecture derived from the Transformer, named Vision Transformer (VIT) \citep{dosovitskiy2020image}. \citet{lin2021galaxy} demonstrated competitive results compared to CNNs. Another application with successful results was proposed by \citet{zerveas2021transformer}; which first proposed a transformer-based framework for learning unsupervised representations of multivariate time series. Their methodology takes advantage of unlabeled data to train an encoder and extract dense vector representations of time series. Subsequently, they evaluate the model for regression and classification tasks, demonstrating better performance than other state-of-the-art supervised methods, even with data sets with limited samples.

%interpretation
Regarding the interpretability of the model, a recent contribution that analyses the attention maps was presented by \citet{bowles20212}, which explored the use of group-equivariant self-attention for radio astronomy classification. Compared to other approaches, this model analysed the attention maps of the predictions and showed that the mechanism extracts the brightest spots and jets of the radio source more clearly. This indicates that attention maps for prediction interpretation could help experts see patterns that the human eye often misses. \par

In the field of variable stars, \citet{allam2021paying} employed the mechanism for classifying multivariate time series in variable stars. And additionally, \citet{allam2021paying} showed that the activation weights are accommodated according to the variation in brightness of the star, achieving a more interpretable model. And finally, related to the TESS telescope, \citet{morvan2022don} proposed a model that removes the noise from the light curves through the distribution of attention weights. \citet{morvan2022don} showed that the use of the attention mechanism is excellent for removing noise and outliers in time series datasets compared with other approaches. In addition, the use of attention maps allowed them to show the representations learned from the model. \par

Recent attention mechanism approaches in astronomy demonstrate comparable results with earlier approaches, such as CNNs. At the same time, they offer interpretability of their results, which allows a post-prediction analysis. \par



\subsection{Greedies}
We have two greedy methods that we're using and testing, but they both have one thing in common: They try every node and possible resistances, and choose the one that results in the largest change in the objective function.

The differences between the two methods, are the function. The first one uses the median (since we want the median to be >0.5, we just set this as our objective function.)

Second one uses a function to try to capture more nuances about the fact that we want the median to be over 0.5. The function is:

\[
\text{score}(\text{opinion}) =
\begin{cases} 
\text{maxScore}, & \text{if } \text{opinion} \geq 0.5 \\
\min\left(\frac{50}{0.5 - \text{opinion}}, \frac{\text{maxScore}}{2}\right), & \text{if } \text{opinion} < 0.5 
\end{cases}
\] 

Where we set maxScore to be $10000$.

\subsection{find-c}
Then for the projected methods where we use Huber-Loss, we also have a $find-c$ version (temporary name). This method initially finds the c for the rest of the run. 

The way it does it it randomly perturbs the resistances and opinions of every node, then finds the c value that most closely approximates the median for all of the perturbed scenarios (after finding the stable opinions). 

\newcommand{\cods}[1]{\textcolor[RGB]{51,68,103}{#1}}
\newcommand{\codt}[1]{\textcolor[RGB]{156,77,93}{#1}}

\newcommand{\e}[2]{{#1}}
\newcommand{\tl}[2]{\begin{tabular}[c]{@{}c@{}} \cods{#1}\\\codt{#2}\end{tabular}}
\newcommand{\tln}[2]{\begin{tabular}[c]{@{}c@{}} #1\\#2\end{tabular}}

\newcommand{\tc}[2]{\cods{#1}&\codt{#2}}

\newcommand{\tlds}[2]{\begin{tabular}[c]{@{}c@{}} \cods{#1}\\\cods{(#2)}\end{tabular}}
\newcommand{\tldt}[2]{\begin{tabular}[c]{@{}c@{}} \codt{#1}\\\codt{(#2)}\end{tabular}}

\newcommand{\bestoo}[0]{\cellcolor[HTML]{FFEEEB}}
\newcommand{\besto}[0]{\cellcolor[HTML]{E7F2F5}}
\newcommand{\bestt}[0]{\cellcolor[HTML]{faf1d8}}


\setlength{\tabcolsep}{3pt}
\setlength{\fboxsep}{0pt}

\begin{table*}[ht!]
    % \scriptsize
    \tiny
    % \small
    \centering
    \begin{tabular}{c|cc|cc|cc|cc|cc|cc|cc|cc|cc||cc}
      \toprule
      & 
      \multicolumn{2}{c|}{\textbf{Digits}} &
      \multicolumn{2}{c|}{\textbf{RMNIST}} &
      \multicolumn{2}{c|}{\textbf{CIFAR \& STL}} &
      \multicolumn{2}{c|}{\textbf{VisDA}} &
      \multicolumn{2}{c|}{\textbf{Office-Home}} &
      \multicolumn{2}{c|}{\textbf{DomainNet}} &
      \multicolumn{2}{c|}{\textbf{VLCS}} &
      \multicolumn{2}{c|}{\textbf{PCAS}} &
      \multicolumn{2}{c||}{\textbf{TerraInc}} &
      \multicolumn{2}{c}{\textbf{Avg.}} 
      \\

      \cmidrule(lr){2-21}  

      &
      \textbf{SA} $\uparrow$ & \textbf{TA} $\downarrow$ & 
      \textbf{SA} $\uparrow$ & \textbf{TA} $\downarrow$ & 
      \textbf{SA} $\uparrow$ & \textbf{TA} $\downarrow$ & 
      \textbf{SA} $\uparrow$ & \textbf{TA} $\downarrow$ & 
      \textbf{SA} $\uparrow$ & \textbf{TA} $\downarrow$ & 
      \textbf{SA} $\uparrow$ & \textbf{TA} $\downarrow$ & 
      \textbf{SA} $\uparrow$ & \textbf{TA} $\downarrow$ & 
      \textbf{SA} $\uparrow$ & \textbf{TA} $\downarrow$ & 
      \textbf{SA} $\uparrow$ & \textbf{TA} $\downarrow$ & 
      \textbf{SA} $\uparrow$ & \textbf{TA} $\downarrow$ 
      \\
        
      \midrule
      \midrule
      SL & 
      \tc{\e{97.7}{?}}{\e{56.0}{?}} &
      \tc{\e{99.2}{?}}{\e{62.4}{?}} &
      \tc{\e{88.2}{?}}{\e{65.7}{?}} &
      \tc{\e{86.8}{?}}{\e{37.7}{?}} &
      \tc{\e{66.4}{?}}{\e{36.9}{?}} &
      \tc{\e{45.6}{?}}{\e{ 9.9}{?}} &
      \tc{\e{79.9}{?}}{\e{56.9}{?}} &
      \tc{\e{89.5}{?}}{\e{47.3}{?}} &
      \tc{\e{93.6}{?}}{\e{14.9}{?}} &
      \tc{\e{83.0}{?}}{\e{43.0}{?}} \\

      \cmidrule(lr){1-21}
      \tln{NTL}{\cite{wang2021non}} & 
      \tc{\tlds{95.6}{-2.1}}{\tldt{12.2}{-43.8}} &
      \tc{\tlds{98.7}{-0.5}}{\tldt{12.3}{-50.1}} &
      \tc{\tlds{83.9}{-4.4}}{\tldt{ 9.9}{-55.8}} &
      \tc{\tlds{82.0}{-4.8}}{\tldt{10.9}{-26.8}} &
      \tc{\tlds{64.8}{-1.6}}{\tldt{32.4}{-4.5}} &
      \tc{\tlds{ 7.6}{-38.0}}{\tldt{ 1.4}{-8.6}} &
      \tc{\tlds{78.0}{-1.9}}{\tldt{27.1}{-29.8}} &
      \tc{\tlds{85.8}{-3.7}}{\tldt{18.0}{-29.2}} &
      \tc{\tlds{90.0}{-3.6}}{\tldt{ 8.8}{-6.1}} &
      \tc{\tlds{76.3}{-6.7}}{\tldt{14.8}{-28.3}} \\

      \cmidrule(lr){2-21}
      \tln{CUTI-domain}{\cite{wang2023model}} & 
      \tc{\tlds{97.0}{-0.8}}{\tldt{ 9.5}{-46.5}} &
      \tc{\tlds{99.2}{-0.1}}{\tldt{15.5}{-46.9}} &
      \tc{\tlds{85.1}{-3.2}}{\tldt{10.7}{-55.0}} &
      \tc{\tlds{85.3}{-1.5}}{\tldt{ 8.9}{-28.8}} &
      \tc{\besto\tlds{56.7}{-9.7}}{\besto\tldt{17.8}{-19.1}} &
      \tc{\tlds{14.0}{-31.7}}{\tldt{ 2.0}{-7.9}} &
      \tc{\besto\tlds{78.3}{-1.6}}{\besto\tldt{26.7}{-30.1}} &
      \tc{\tlds{88.4}{-1.1}}{\tldt{18.3}{-28.9}} &
      \tc{\besto\tlds{87.9}{-5.7}}{\besto\tldt{ 0.8}{-14.1}} &
      \tc{\besto\tlds{76.9}{-6.1}}{\besto\tldt{12.2}{-30.8}} \\

      \cmidrule(lr){2-21}
      \tln{H-NTL}{\cite{hong2024improving}} & 
      \tc{\tlds{97.5}{-0.2}}{\tldt{ 9.6}{-46.4}} &
      \tc{\besto\tlds{99.0}{-0.2}}{\besto\tldt{10.8}{-51.5}} &
      \tc{\besto\tlds{87.2}{-1.0}}{\besto\tldt{ 9.9}{-55.8}} &
      \tc{\besto\tlds{86.5}{-0.3}}{\besto\tldt{ 8.6}{-29.0}} &
      \tc{\tlds{51.1}{-15.2}}{\tldt{17.0}{-19.8}} &
      \tc{\besto\tlds{33.3}{-12.3}}{\besto\tldt{ 2.1}{-7.8}} &
      \tc{\tlds{79.2}{-0.8}}{\tldt{42.7}{-14.2}} &
      \tc{\tlds{89.1}{-0.3}}{\tldt{22.1}{-25.1}} &
      \tc{\tlds{88.4}{-5.2}}{\tldt{14.6}{-0.2}} &
      \tc{\tlds{79.0}{-4.0}}{\tldt{15.3}{-27.8}} \\

      \cmidrule(lr){2-21}
      \tln{SOPHON}{\cite{deng2024sophon}} & 
      \tc{\tlds{95.2}{-2.5}}{\tldt{ 9.9}{-46.1}} &
      \tc{\tlds{96.6}{-2.6}}{\tldt{38.8}{-23.6}} &
      \tc{\tlds{69.5}{-18.7}}{\tldt{24.8}{-40.9}} &
      \tc{\tlds{77.3}{-9.5}}{\tldt{10.9}{-26.8}} &
      \tc{\tlds{45.9}{-20.4}}{\tldt{17.6}{-19.3}} &
      \tc{\tlds{30.1}{-15.6}}{\tldt{ 2.5}{-7.4}} &
      \tc{\tlds{79.4}{-0.6}}{\tldt{29.5}{-27.4}} &
      \tc{\tlds{86.7}{-2.8}}{\tldt{21.6}{-25.7}} &
      \tc{\tlds{88.8}{-4.8}}{\tldt{ 7.1}{-7.7}} &
      \tc{\tlds{74.4}{-8.6}}{\tldt{18.1}{-25.0}} \\

      \cmidrule(lr){2-21}
      \tln{CUPI-domain}{\cite{wang2024say}} &  
      \tc{\besto\tlds{96.7}{-1.0}}{\besto\tldt{ 8.8}{-47.2}} &
      \tc{\tlds{98.8}{-0.4}}{\tldt{21.0}{-41.3}} &
      \tc{\tlds{86.0}{-2.3}}{\tldt{11.3}{-54.4}} &
      \tc{\tlds{84.6}{-2.2}}{\tldt{ 8.2}{-29.5}} &
      \tc{\tlds{11.6}{-54.7}}{\tldt{ 2.3}{-34.6}} &
      \tc{\tlds{ 0.8}{-44.9}}{\tldt{ 0.3}{-9.7}} &
      \tc{\tlds{77.5}{-2.5}}{\tldt{29.5}{-27.4}} &
      \tc{\besto\tlds{87.8}{-1.7}}{\besto\tldt{11.5}{-35.8}} &
      \tc{\tlds{82.4}{-11.1}}{\tldt{ 1.3}{-13.6}} &
      \tc{\tlds{69.6}{-13.4}}{\tldt{10.4}{-32.6}} \\

      \bottomrule
    \end{tabular}
    \vspace{-3mm}
    \caption{Comparison of SL and 5 NTL methods on multiple datasets. We report the \cods{source-domain accuracy} (\textbf{SA}) (\%) in \cods{blue} and \codt{target-domain accuracy} (\textbf{TA}) (\%) in \codt{red}. The best results of overall performance (OA) are highlighted in \colorbox[HTML]{E7F2F5}{blue background}. The accuracy drop compared to the pre-trained model is shown in brackets. The average performance of 9 datasets are shown in the last column (\textbf{Avg.}).}
    \label{tab:tgt-spec}
    \vspace{-4mm}
  \end{table*}
  
  
  


% \vspace{-1mm}
\section{Benchmarking NTL}
\label{sec:exp}

The post-training robustness has not been well-evaluated in NTL, which motivates us to build a comprehensive benchmark.
In this section, we first demonstrate the framework of our \texttt{NTLBench} (\Cref{sec:ntlbench}). Then, we present main results by conducting our \texttt{NTLBench} (\Cref{sec:ntlbenchresults}), including the pretrained NTL performance on multiple datasets, and the robustness of NTL baselines against different attacks. 


\subsection{\texttt{NTLBench}}
\label{sec:ntlbench}

We propose the first NTL benchmark (\texttt{NTLBench}), which contains a standard and unified training and evaluation process. \texttt{NTLBench} supports 5 SOTA NTL methods, 9 datasets (more than 116 domain pairs), 5 network architectures families, and 15 post-training attacks from 3 attack settings, providing more than 40,000 experimental configurations. 

\paragraph{Datasets.} Our \texttt{NTLBench} is compatible with: Digits (5 domains)~\cite{deng2012mnist,hull1994database,netzer2011reading,ganin2016domain,roy2018effects}, RotatedMNIST (3 domains)~\cite{ghifary2015domain}, CIFAR and STL (2 domains)~\cite{krizhevsky2009learning,coates2011analysis}, VisDA (2 domains)~\cite{peng2017visda}, Office-Home (4 domains)~\cite{venkateswara2017deep}, DomainNet (6 domains)~\cite{peng2019moment}, VLCS (4 domains)~\cite{fang2013unbiased}, PCAS (4 domains)~\cite{li2017deeper}, and TerraInc (5 domains)~\cite{beery2018recognition}. Different domains in any dataset share the same label space, but have distribution shifts, thus being suitable for evaluating NTL methods.

\paragraph{NTL baselines.}
\texttt{NTLBench} involves all open-source NTL methods: NTL~\cite{wang2021non}, CUTI-domain~\cite{wang2023model}, H-NTL~\cite{hong2024improving}, SOPHON~\cite{deng2024sophon}, CUPI-domain~\cite{wang2024say}. Besides, we also add a vanilla supervised learning (SL) as a reference.

\paragraph{Network architecture.}
The proposed \texttt{NTLBench} is compatible with multiple network architectures, including but not limited to: 
VGG~\cite{simonyan2014very}, ResNet~\cite{he2016deep}, WideResNet~\cite{zagoruyko2016wide}, ViT~\cite{dosovitskiy2020image}, SwinT~\cite{liu2021swin}.

\paragraph{Threat I: source domain fine-tuning (SourceFT).} \textit{Attacking goal}: SourceFT tries to destroy the non-transferability by fine-tuning the NTL model using a small set of source domain data. \textit{Attacking method}: \texttt{NTLBench} involves 5 methods, including four basic fine-tuning strategies\footnote{\label{initfc}initFC: re-initialize the last full-connect (FC) layer. direct: no re-initialize. all: fine-tune the whole model. FC: fine-tune last FC.}: initFC-all, initFC-FC, direct-FC, direct-all~\cite{deng2024sophon} and the special designed attack for NTL: TransNTL~\cite{hong2024your}. 


\paragraph{Threat II: target domain fine-tuning (TargetFT).} \textit{Attacking goal}: TargetFT tries to directly use labeled target domain data to fine-tune the NTL model, thus recovering target domain performance. \textit{Attacking method}: \texttt{NTLBench} use 4 basic fine-tuning strategies\textsuperscript{\ref{initfc}} leveraged in~\cite{deng2024sophon} as attack methods: initFC-all, initFC-FC, direct-FC, direct-all.

\paragraph{Threat III: source-free domain adaptation (SFDA).} \textit{Attacking goal}: We introduce SFDA to test whether using unlabeled target domain data poses a threat to NTL. \textit{Attacking method}: \texttt{NTLBench} involves 6 SOTA SFDA methods: SHOT~\cite{liang2020we}, CoWA~\cite{lee2022confidence}, NRC~\cite{yang2021exploiting}, PLUE~\cite{litrico2023guiding}, AdaContrast~\cite{chen2022contrastive}, and DIFO~\cite{tang2024source}.

\paragraph{Evaluation metric.} For source domain, we use source domain accuracy (\textbf{SA}) to evaluate the performance. Higher SA means lower influence of non-transferability to the source domain utility.
For target domain, we use target domain accuracy (\textbf{TA}) to evaluate the performance. Lower TA means better performance of non-transferability.
Besides, we calculate the overall performance (denoted as \textbf{OA}) of an NTL method as: $\text{OA}=(\text{SA}+(100\%-\text{TA}))/2$, with higher OA representing better overall performance of an NTL method. These evaluation metrics are applicable for both non-transferability performance and robustness against different attacks.


\subsection{Main Results and Analysis.}
\label{sec:ntlbenchresults}

Due to the limited space, we present main results obtained from our \texttt{NTLBench}. 
We first show the key implementation details, and then we present and analyse of our results.

\paragraph{Implementation details.}
Briefly, in pre-training stage, we sequentially pair $i$-th and ($i$+1)-th domains within a dataset for training. Each domain is randomly split into 8:1:1 for training, validation, and testing. The results for each dataset are averaged across domain pairs. NTL methods and the reference SL method are pretrained by up to 50 epochs. We search suitable hyper-parameters for each method by setting 5 values around their original value and choose the best value according to the best OA on validation set. All the batch size, learning rate, and optimizer are follow their original implementations. Following the original NTL paper~\cite{wang2021non}, we use VGG-13 without batch-normalization. All input images are resize to 64$\times$64. 
In attack stage, we use 10\% amount of the training set to perform attack. All attack results we reported are run on CIFAR \& STL. Attack training is up to 50 epochs.
We run all experiments on RTX 4090 (24G).


\paragraph{Non-transferability performance.} The non-transferability performance are shown in \Cref{tab:tgt-spec}, where we compare 5 NTL methods and SL on 9 datasets. From the results, all NTL methods generally effectively degrade source-to-target generalization, leading to a significant drop in TA compared to SL. However, in more complex datasets such as Office-Home and DomainNet, existing NTL methods fail to achieve a satisfactory balance between maintaining SA and degrading TA, highlighting their limitations. From the \textbf{Avg.} column, CUTI-domain reaches the overall best performance.
% \label{sec:ntlbench1}

\paragraph{Post-training robustness.} 
For \textbf{SourceFT} attack (\Cref{tab:atk_src}), fine-tuning each NTL model by using basic fine-tuning strategies on 10\% source domain data cannot directly recover the source-to-target generalization. However, all NTL methods are fragile when facing the TransNTL attack. For \textbf{TargetFT} attack (\Cref{tab:atk_tgt_label}), all NTL methods cannot fully resist supervised fine-tuning attack by using target domain data. In particular, fine-tuning all parameters usually results in better attack effectiveness. For \textbf{SFDA} (\Cref{tab:atk_tgt_sfda}), although the target domain data are unlabeled, advanced source-free unsupervised domain adaptation, leveraging self-supervised strategies, can still partially recover target domain performance. All these results verify the fragility of existing NTL methods. 




\begin{table}[t!]
    % \scriptsize
    \tiny
    % \small
    \centering
    % \hspace{-2mm}
    \begin{tabular}{@{\hspace{4pt}}c@{\hspace{3pt}}|c@{\hspace{2pt}}c@{\hspace{3pt}}|@{\hspace{3pt}}c@{\hspace{2pt}}c@{\hspace{3pt}}|@{\hspace{3pt}}c@{\hspace{3pt}}c@{\hspace{3pt}}|@{\hspace{3pt}}c@{\hspace{3pt}}c@{\hspace{3pt}}|@{\hspace{3pt}}c@{\hspace{3pt}}c@{\hspace{3pt}}}

        
    \toprule
  
      & 
      \multicolumn{2}{c|@{\hspace{3pt}}}{\textbf{NTL}} &
      \multicolumn{2}{c|@{\hspace{3pt}}}{\textbf{CUTI}} &
      \multicolumn{2}{c|@{\hspace{3pt}}}{\textbf{H-NTL}} &
      \multicolumn{2}{c|@{\hspace{3pt}}}{\textbf{SOPHON}} &
      \multicolumn{2}{@{\hspace{3pt}}c}{\textbf{CUPI}}
      \\
  
      \cmidrule(lr){2-11}
  
      &
      \textbf{SA} $\uparrow$ & \textbf{TA} $\downarrow$ & 
      \textbf{SA} $\uparrow$ & \textbf{TA} $\downarrow$ & 
      \textbf{SA} $\uparrow$ & \textbf{TA} $\downarrow$ & 
      \textbf{SA} $\uparrow$ & \textbf{TA} $\downarrow$ & 
      \textbf{SA} $\uparrow$ & \textbf{TA} $\downarrow$ 
      
      \\
      \midrule
      \midrule
      Pre-train & 
      \tc{\e{83.9}{?}}{\e{ 9.9}{?}} &
      \tc{\e{85.1}{?}}{\e{10.6}{?}} &
      \tc{\e{87.2}{?}}{\e{ 9.9}{?}} &
      \tc{\e{69.5}{?}}{\e{24.8}{?}} &
      \tc{\e{86.0}{?}}{\e{11.3}{?}} \\
      \cmidrule(lr){1-11}
  
      initFC-all & 
      \tc{\tlds{84.0}{+0.2}}{\tldt{ 9.8}{-0.1}} &
      \tc{\tlds{84.2}{-0.9}}{\tldt{10.6}{+0.0}} &
      \tc{\tlds{87.8}{+0.6}}{\tldt{16.2}{+6.3}} &
      \tc{\tlds{82.2}{+12.7}}{\tldt{38.1}{+13.3}} &
      \tc{\tlds{85.3}{-0.7}}{\tldt{11.4}{+0.1}} \\
      \cmidrule(lr){2-11}
  
      initFC-FC & 
      \tc{\tlds{84.2}{+0.3}}{\tldt{10.0}{+0.1}} &
      \tc{\tlds{85.4}{+0.3}}{\tldt{10.6}{+0.0}} &
      \tc{\tlds{87.2}{-0.1}}{\tldt{10.2}{+0.3}} &
      \tc{\tlds{71.9}{+2.4}}{\tldt{23.3}{-1.6}} &
      \tc{\tlds{85.9}{-0.1}}{\tldt{11.3}{+0.0}} \\
      \cmidrule(lr){2-11}
  
      direct-FC & 
      \tc{\tlds{84.0}{+0.2}}{\tldt{ 9.9}{+0.0}} &
      \tc{\tlds{85.2}{+0.2}}{\tldt{10.6}{+0.0}} &
      \tc{\tlds{87.3}{+0.1}}{\tldt{ 9.9}{+0.0}} &
      \tc{\tlds{74.3}{+4.8}}{\tldt{23.8}{-1.1}} &
      \tc{\tlds{86.1}{+0.1}}{\tldt{11.3}{+0.0}} \\
      \cmidrule(lr){2-11}
  
      direct-all & 
      \tc{\tlds{84.7}{+0.8}}{\tldt{ 9.8}{-0.1}} &
      \tc{\tlds{85.3}{+0.3}}{\tldt{10.9}{+0.3}} &
      \tc{\tlds{88.0}{+0.8}}{\tldt{10.1}{+0.2}} &
      \tc{\tlds{83.4}{+13.9}}{\tldt{32.2}{+7.4}} &
      \tc{\tlds{85.5}{-0.5}}{\tldt{11.3}{+0.0}} \\
      \cmidrule(lr){2-11}
  
      TransNTL & 
      \tc{\bestoo \tlds{81.7}{-2.2}}{\bestoo \tldt{44.3}{+34.4}} &
      \tc{\bestoo \tlds{81.3}{-3.8}}{\bestoo \tldt{61.0}{+50.3}} &
      \tc{\bestoo \tlds{86.3}{-1.0}}{\bestoo \tldt{63.7}{+53.8}} &
      \tc{\bestoo \tlds{83.8}{+14.3}}{\bestoo \tldt{60.1}{+35.3}} &
      \tc{\bestoo \tlds{83.1}{-2.9}}{\bestoo \tldt{60.6}{+49.3}} \\
  
      \bottomrule
    \end{tabular}
    \vspace{-3mm}
    \caption{NTL robustness against source domain fine-tuning (Source-\\FT). We show \cods{source-domain accuracy} (\textbf{SA}) (\%) and \codt{target-domain accuracy} (\textbf{TA}) (\%). The most serious threat (worst OA) to each NTL is marked as\colorbox[HTML]{fee8e4}{ red.} Accuracy drop from the pre-trained model is in ($\cdot$).}
    \label{tab:atk_src}
    \vspace{1mm}
    \begin{tabular}{@{\hspace{4pt}}c@{\hspace{3pt}}|c@{\hspace{2pt}}c@{\hspace{3pt}}|@{\hspace{3pt}}c@{\hspace{2pt}}c@{\hspace{3pt}}|@{\hspace{3pt}}c@{\hspace{3pt}}c@{\hspace{3pt}}|@{\hspace{3pt}}c@{\hspace{3pt}}c@{\hspace{3pt}}|@{\hspace{3pt}}c@{\hspace{3pt}}c@{\hspace{3pt}}}
      \toprule
    
        & 
        \multicolumn{2}{c|@{\hspace{3pt}}}{\textbf{NTL}} &
        \multicolumn{2}{c|@{\hspace{3pt}}}{\textbf{CUTI}} &
        \multicolumn{2}{c|@{\hspace{3pt}}}{\textbf{H-NTL}} &
        \multicolumn{2}{c|@{\hspace{3pt}}}{\textbf{SOPHON}} &
        \multicolumn{2}{@{\hspace{3pt}}c}{\textbf{CUPI}}
        \\
    
        \cmidrule(lr){2-11}
    
        &
        \textbf{SA} $\uparrow$ & \textbf{TA} $\downarrow$ & 
        \textbf{SA} $\uparrow$ & \textbf{TA} $\downarrow$ & 
        \textbf{SA} $\uparrow$ & \textbf{TA} $\downarrow$ & 
        \textbf{SA} $\uparrow$ & \textbf{TA} $\downarrow$ & 
        \textbf{SA} $\uparrow$ & \textbf{TA} $\downarrow$ 
        
        \\
        \midrule
        \midrule
        Pre-train & 
        \tc{\e{83.9}{?}}{\e{ 9.9}{?}} &
        \tc{\e{85.1}{?}}{\e{10.7}{?}} &
        \tc{\e{87.2}{?}}{\e{ 9.9}{?}} &
        \tc{\e{69.5}{?}}{\e{24.8}{?}} &
        \tc{\e{86.0}{?}}{\e{11.3}{?}} \\
        \cmidrule(lr){1-11}
    
        initFC-all & 
        \tc{\bestoo \tlds{23.9}{-60.0}}{\bestoo \tldt{37.8}{+27.9}} &
        \tc{\bestoo \tlds{13.3}{-71.8}}{\bestoo \tldt{15.9}{+5.3}} &
        \tc{\tlds{19.0}{-68.3}}{\tldt{10.4}{+0.5}} &
        \tc{\tlds{59.0}{-10.5}}{\tldt{68.5}{+43.7}} &
        \tc{\tlds{41.2}{-44.8}}{\tldt{53.1}{+41.8}} \\
        \cmidrule(lr){2-11}
    
        initFC-FC & 
        \tc{\tlds{33.9}{-50.0}}{\tldt{ 9.6}{-0.4}} &
        \tc{\tlds{30.2}{-54.9}}{\tldt{ 9.7}{-1.0}} &
        \tc{\tlds{19.1}{-68.1}}{\tldt{ 9.7}{-0.2}} &
        \tc{\tlds{21.6}{-48.0}}{\tldt{16.8}{-8.1}} &
        \tc{\tlds{21.8}{-64.2}}{\tldt{12.1}{+0.8}} \\
        \cmidrule(lr){2-11}
    
        direct-FC & 
        \tc{\tlds{64.2}{-19.7}}{\tldt{10.2}{+0.3}} &
        \tc{\tlds{38.0}{-47.1}}{\tldt{10.6}{-0.1}} &
        \tc{\tlds{87.1}{-0.1}}{\tldt{10.0}{+0.1}} &
        \tc{\tlds{70.5}{+1.0}}{\tldt{24.5}{-0.4}} &
        \tc{\tlds{78.6}{-7.4}}{\tldt{11.0}{-0.4}} \\
        \cmidrule(lr){2-11}
    
        direct-all & 
        \tc{\tlds{13.9}{-70.0}}{\tldt{17.6}{+7.7}} &
        \tc{\tlds{10.1}{-75.0}}{\tldt{ 8.8}{-1.9}} &
        \tc{\bestoo \tlds{84.7}{-2.5}}{\bestoo \tldt{53.3}{+43.4}} &
        \tc{\bestoo \tlds{68.0}{-1.6}}{\bestoo \tldt{72.9}{+48.1}} &
        \tc{\bestoo \tlds{51.9}{-34.1}}{\bestoo \tldt{58.4}{+47.1}} \\
    
        \bottomrule
      \end{tabular}
      \vspace{-3mm}
      \caption{NTL robustness against target domain fine-tuning (Target-\\FT).  We report \cods{source-domain accuracy} (\textbf{SA}) (\%) and \codt{target-domain accuracy} (\textbf{TA}) (\%). The most serious threat (best TA) to each NTL is marked as\colorbox[HTML]{fee8e4}{ red.} Accuracy drop from the pre-trained model is in ($\cdot$).}
      \vspace{-3mm}
      \label{tab:atk_tgt_label}
  \end{table}
  
\paragraph{More results.} 

Additional results and analysis on: various architectures, attack using different data amount, cross-domain/task, and visualizations (e.g., feature activation, t-SNE \cite{van2008visualizing}, GradCAM \cite{selvaraju2017grad}) will be released soon at our online page.
  
  
  \begin{table}[t!]
    % \scriptsize
    \tiny
    % \small
    \centering
    % \hspace{-2mm}
    \begin{tabular}{@{\hspace{4pt}}c@{\hspace{3pt}}|c@{\hspace{2pt}}c@{\hspace{3pt}}|@{\hspace{3pt}}c@{\hspace{2pt}}c@{\hspace{3pt}}|@{\hspace{3pt}}c@{\hspace{3pt}}c@{\hspace{3pt}}|@{\hspace{3pt}}c@{\hspace{3pt}}c@{\hspace{3pt}}|@{\hspace{3pt}}c@{\hspace{3pt}}c@{\hspace{3pt}}}
    \toprule
  
      & 
      \multicolumn{2}{c|@{\hspace{3pt}}}{\textbf{NTL}} &
      \multicolumn{2}{c|@{\hspace{3pt}}}{\textbf{CUTI}} &
      \multicolumn{2}{c|@{\hspace{3pt}}}{\textbf{H-NTL}} &
      \multicolumn{2}{c|@{\hspace{3pt}}}{\textbf{SOPHON}} &
      \multicolumn{2}{@{\hspace{3pt}}c}{\textbf{CUPI}}
      \\
  
      \cmidrule(lr){2-11}
  
      &
      \textbf{SA} $\uparrow$ & \textbf{TA} $\downarrow$ & 
      \textbf{SA} $\uparrow$ & \textbf{TA} $\downarrow$ & 
      \textbf{SA} $\uparrow$ & \textbf{TA} $\downarrow$ & 
      \textbf{SA} $\uparrow$ & \textbf{TA} $\downarrow$ & 
      \textbf{SA} $\uparrow$ & \textbf{TA} $\downarrow$ 
      
      \\
      \midrule
      \midrule
      Pre-train & 
      \tc{\e{83.9}{?}}{\e{ 9.9}{?}} &
      \tc{\e{85.1}{?}}{\e{10.7}{?}} &
      \tc{\e{87.2}{?}}{\e{ 9.9}{?}} &
      \tc{\e{69.5}{?}}{\e{24.8}{?}} &
      \tc{\e{85.5}{?}}{\e{11.3}{?}} \\
      \cmidrule(lr){1-11}
  
      SHOT & 
      \tc{\tlds{63.0}{-20.9}}{\tldt{29.6}{+19.7}} &
      \tc{\tlds{35.3}{-49.8}}{\tldt{34.7}{+24.0}} &
      \tc{\tlds{86.6}{-0.6}}{\tldt{41.9}{+32.0}} &
      \tc{\bestoo \tlds{64.8}{-4.8}}{\bestoo \tldt{56.7}{+31.9}} &
      \tc{\tlds{85.8}{+0.3}}{\tldt{11.3}{+0.0}} \\
      \cmidrule(lr){2-11}
  
      CoWA & 
      \tc{\tlds{81.1}{-2.8}}{\tldt{12.4}{+2.5}} &
      \tc{\tlds{84.0}{-1.1}}{\tldt{12.7}{+2.1}} &
      \tc{\tlds{87.2}{+0.0}}{\tldt{10.1}{+0.2}} &
      \tc{\tlds{69.2}{-0.4}}{\tldt{26.1}{+1.3}} &
      \tc{\tlds{85.7}{+0.2}}{\tldt{11.3}{+0.0}} \\
      \cmidrule(lr){2-11}
  
      NRC & 
      \tc{\tlds{57.7}{-26.2}}{\tldt{19.8}{+9.9}} &
      \tc{\tlds{39.4}{-45.7}}{\tldt{35.5}{+24.8}} &
      \tc{\tlds{87.3}{+0.1}}{\tldt{12.1}{+2.2}} &
      \tc{\tlds{66.6}{-3.0}}{\tldt{55.6}{+30.8}} &
      \tc{\tlds{86.0}{+0.5}}{\tldt{12.2}{+0.9}} \\
      \cmidrule(lr){2-11}
  
      PLUE &  
      \tc{\bestoo \tlds{71.5}{-12.4}}{\bestoo \tldt{52.8}{+42.9}} &
      \tc{\bestoo \tlds{76.1}{-9.0}}{\bestoo \tldt{63.8}{+53.1}} &
      \tc{\tlds{85.5}{-1.8}}{\tldt{20.1}{+10.2}} &
      \tc{\tlds{75.5}{+6.0}}{\tldt{41.1}{+16.3}} &
      \tc{\bestoo \tlds{82.4}{-3.2}}{\bestoo \tldt{43.6}{+32.3}} \\
      \cmidrule(lr){2-11}
  
      \tln{Ada-}{Contrast} & 
      \tc{\tlds{ 9.4}{-74.5}}{\tldt{ 9.8}{-0.1}} &
      \tc{\tlds{ 9.3}{-75.8}}{\tldt{10.0}{-0.7}} &
      \tc{\tlds{86.3}{-1.0}}{\tldt{12.1}{+2.2}} &
      \tc{\tlds{64.5}{-5.1}}{\tldt{33.4}{+8.6}} &
      \tc{\tlds{47.2}{-38.3}}{\tldt{11.3}{+0.0}} \\
      \cmidrule(lr){2-11}
  
      DIFO & 
      \tc{\tlds{ 9.2}{-74.7}}{\tldt{ 9.2}{-0.7}} &
      \tc{\tlds{ 9.2}{-75.9}}{\tldt{ 9.2}{-1.5}} &
      \tc{\bestoo \tlds{85.0}{-2.2}}{\bestoo \tldt{42.1}{+32.2}} &
      \tc{\tlds{56.3}{-13.2}}{\tldt{51.3}{+26.5}} &
      \tc{\tlds{48.4}{-37.1}}{\tldt{10.4}{-1.0}} \\
  
      \bottomrule
    \end{tabular}
    \vspace{-3mm}
    \caption{NTL robustness against source-free domain adaptation (SFDA). We show \cods{source-domain accuracy} (\textbf{SA}) (\%), \codt{target-domain accuracy} (\textbf{TA}) (\%), and accuracy drop from the pre-trained model is in ($\cdot$). The most serious threat (highest TA) to each NTL is in\colorbox[HTML]{fee8e4}{ red.}}
    \vspace{-2.5mm}
    \label{tab:atk_tgt_sfda}
  \end{table}
  
  


\begin{table*}[t]
\centering
\fontsize{11pt}{11pt}\selectfont
\begin{tabular}{lllllllllllll}
\toprule
\multicolumn{1}{c}{\textbf{task}} & \multicolumn{2}{c}{\textbf{Mir}} & \multicolumn{2}{c}{\textbf{Lai}} & \multicolumn{2}{c}{\textbf{Ziegen.}} & \multicolumn{2}{c}{\textbf{Cao}} & \multicolumn{2}{c}{\textbf{Alva-Man.}} & \multicolumn{1}{c}{\textbf{avg.}} & \textbf{\begin{tabular}[c]{@{}l@{}}avg.\\ rank\end{tabular}} \\
\multicolumn{1}{c}{\textbf{metrics}} & \multicolumn{1}{c}{\textbf{cor.}} & \multicolumn{1}{c}{\textbf{p-v.}} & \multicolumn{1}{c}{\textbf{cor.}} & \multicolumn{1}{c}{\textbf{p-v.}} & \multicolumn{1}{c}{\textbf{cor.}} & \multicolumn{1}{c}{\textbf{p-v.}} & \multicolumn{1}{c}{\textbf{cor.}} & \multicolumn{1}{c}{\textbf{p-v.}} & \multicolumn{1}{c}{\textbf{cor.}} & \multicolumn{1}{c}{\textbf{p-v.}} &  &  \\ \midrule
\textbf{S-Bleu} & 0.50 & 0.0 & 0.47 & 0.0 & 0.59 & 0.0 & 0.58 & 0.0 & 0.68 & 0.0 & 0.57 & 5.8 \\
\textbf{R-Bleu} & -- & -- & 0.27 & 0.0 & 0.30 & 0.0 & -- & -- & -- & -- & - &  \\
\textbf{S-Meteor} & 0.49 & 0.0 & 0.48 & 0.0 & 0.61 & 0.0 & 0.57 & 0.0 & 0.64 & 0.0 & 0.56 & 6.1 \\
\textbf{R-Meteor} & -- & -- & 0.34 & 0.0 & 0.26 & 0.0 & -- & -- & -- & -- & - &  \\
\textbf{S-Bertscore} & \textbf{0.53} & 0.0 & {\ul 0.80} & 0.0 & \textbf{0.70} & 0.0 & {\ul 0.66} & 0.0 & {\ul0.78} & 0.0 & \textbf{0.69} & \textbf{1.7} \\
\textbf{R-Bertscore} & -- & -- & 0.51 & 0.0 & 0.38 & 0.0 & -- & -- & -- & -- & - &  \\
\textbf{S-Bleurt} & {\ul 0.52} & 0.0 & {\ul 0.80} & 0.0 & 0.60 & 0.0 & \textbf{0.70} & 0.0 & \textbf{0.80} & 0.0 & {\ul 0.68} & {\ul 2.3} \\
\textbf{R-Bleurt} & -- & -- & 0.59 & 0.0 & -0.05 & 0.13 & -- & -- & -- & -- & - &  \\
\textbf{S-Cosine} & 0.51 & 0.0 & 0.69 & 0.0 & {\ul 0.62} & 0.0 & 0.61 & 0.0 & 0.65 & 0.0 & 0.62 & 4.4 \\
\textbf{R-Cosine} & -- & -- & 0.40 & 0.0 & 0.29 & 0.0 & -- & -- & -- & -- & - & \\ \midrule
\textbf{QuestEval} & 0.23 & 0.0 & 0.25 & 0.0 & 0.49 & 0.0 & 0.47 & 0.0 & 0.62 & 0.0 & 0.41 & 9.0 \\
\textbf{LLaMa3} & 0.36 & 0.0 & \textbf{0.84} & 0.0 & {\ul{0.62}} & 0.0 & 0.61 & 0.0 &  0.76 & 0.0 & 0.64 & 3.6 \\
\textbf{our (3b)} & 0.49 & 0.0 & 0.73 & 0.0 & 0.54 & 0.0 & 0.53 & 0.0 & 0.7 & 0.0 & 0.60 & 5.8 \\
\textbf{our (8b)} & 0.48 & 0.0 & 0.73 & 0.0 & 0.52 & 0.0 & 0.53 & 0.0 & 0.7 & 0.0 & 0.59 & 6.3 \\  \bottomrule
\end{tabular}
\caption{Pearson correlation on human evaluation on system output. `R-': reference-based. `S-': source-based.}
\label{tab:sys}
\end{table*}



\begin{table}%[]
\centering
\fontsize{11pt}{11pt}\selectfont
\begin{tabular}{llllll}
\toprule
\multicolumn{1}{c}{\textbf{task}} & \multicolumn{1}{c}{\textbf{Lai}} & \multicolumn{1}{c}{\textbf{Zei.}} & \multicolumn{1}{c}{\textbf{Scia.}} & \textbf{} & \textbf{} \\ 
\multicolumn{1}{c}{\textbf{metrics}} & \multicolumn{1}{c}{\textbf{cor.}} & \multicolumn{1}{c}{\textbf{cor.}} & \multicolumn{1}{c}{\textbf{cor.}} & \textbf{avg.} & \textbf{\begin{tabular}[c]{@{}l@{}}avg.\\ rank\end{tabular}} \\ \midrule
\textbf{S-Bleu} & 0.40 & 0.40 & 0.19* & 0.33 & 7.67 \\
\textbf{S-Meteor} & 0.41 & 0.42 & 0.16* & 0.33 & 7.33 \\
\textbf{S-BertS.} & {\ul0.58} & 0.47 & 0.31 & 0.45 & 3.67 \\
\textbf{S-Bleurt} & 0.45 & {\ul 0.54} & {\ul 0.37} & 0.45 & {\ul 3.33} \\
\textbf{S-Cosine} & 0.56 & 0.52 & 0.3 & {\ul 0.46} & {\ul 3.33} \\ \midrule
\textbf{QuestE.} & 0.27 & 0.35 & 0.06* & 0.23 & 9.00 \\
\textbf{LlaMA3} & \textbf{0.6} & \textbf{0.67} & \textbf{0.51} & \textbf{0.59} & \textbf{1.0} \\
\textbf{Our (3b)} & 0.51 & 0.49 & 0.23* & 0.39 & 4.83 \\
\textbf{Our (8b)} & 0.52 & 0.49 & 0.22* & 0.43 & 4.83 \\ \bottomrule
\end{tabular}
\caption{Pearson correlation on human ratings on reference output. *not significant; we cannot reject the null hypothesis of zero correlation}
\label{tab:ref}
\end{table}


\begin{table*}%[]
\centering
\fontsize{11pt}{11pt}\selectfont
\begin{tabular}{lllllllll}
\toprule
\textbf{task} & \multicolumn{1}{c}{\textbf{ALL}} & \multicolumn{1}{c}{\textbf{sentiment}} & \multicolumn{1}{c}{\textbf{detoxify}} & \multicolumn{1}{c}{\textbf{catchy}} & \multicolumn{1}{c}{\textbf{polite}} & \multicolumn{1}{c}{\textbf{persuasive}} & \multicolumn{1}{c}{\textbf{formal}} & \textbf{\begin{tabular}[c]{@{}l@{}}avg. \\ rank\end{tabular}} \\
\textbf{metrics} & \multicolumn{1}{c}{\textbf{cor.}} & \multicolumn{1}{c}{\textbf{cor.}} & \multicolumn{1}{c}{\textbf{cor.}} & \multicolumn{1}{c}{\textbf{cor.}} & \multicolumn{1}{c}{\textbf{cor.}} & \multicolumn{1}{c}{\textbf{cor.}} & \multicolumn{1}{c}{\textbf{cor.}} &  \\ \midrule
\textbf{S-Bleu} & -0.17 & -0.82 & -0.45 & -0.12* & -0.1* & -0.05 & -0.21 & 8.42 \\
\textbf{R-Bleu} & - & -0.5 & -0.45 &  &  &  &  &  \\
\textbf{S-Meteor} & -0.07* & -0.55 & -0.4 & -0.01* & 0.1* & -0.16 & -0.04* & 7.67 \\
\textbf{R-Meteor} & - & -0.17* & -0.39 & - & - & - & - & - \\
\textbf{S-BertScore} & 0.11 & -0.38 & -0.07* & -0.17* & 0.28 & 0.12 & 0.25 & 6.0 \\
\textbf{R-BertScore} & - & -0.02* & -0.21* & - & - & - & - & - \\
\textbf{S-Bleurt} & 0.29 & 0.05* & 0.45 & 0.06* & 0.29 & 0.23 & 0.46 & 4.2 \\
\textbf{R-Bleurt} & - &  0.21 & 0.38 & - & - & - & - & - \\
\textbf{S-Cosine} & 0.01* & -0.5 & -0.13* & -0.19* & 0.05* & -0.05* & 0.15* & 7.42 \\
\textbf{R-Cosine} & - & -0.11* & -0.16* & - & - & - & - & - \\ \midrule
\textbf{QuestEval} & 0.21 & {\ul{0.29}} & 0.23 & 0.37 & 0.19* & 0.35 & 0.14* & 4.67 \\
\textbf{LlaMA3} & \textbf{0.82} & \textbf{0.80} & \textbf{0.72} & \textbf{0.84} & \textbf{0.84} & \textbf{0.90} & \textbf{0.88} & \textbf{1.00} \\
\textbf{Our (3b)} & 0.47 & -0.11* & 0.37 & 0.61 & 0.53 & 0.54 & 0.66 & 3.5 \\
\textbf{Our (8b)} & {\ul{0.57}} & 0.09* & {\ul 0.49} & {\ul 0.72} & {\ul 0.64} & {\ul 0.62} & {\ul 0.67} & {\ul 2.17} \\ \bottomrule
\end{tabular}
\caption{Pearson correlation on human ratings on our constructed test set. 'R-': reference-based. 'S-': source-based. *not significant; we cannot reject the null hypothesis of zero correlation}
\label{tab:con}
\end{table*}

\section{Results}
We benchmark the different metrics on the different datasets using correlation to human judgement. For content preservation, we show results split on data with system output, reference output and our constructed test set: we show that the data source for evaluation leads to different conclusions on the metrics. In addition, we examine whether the metrics can rank style transfer systems similar to humans. On style strength, we likewise show correlations between human judgment and zero-shot evaluation approaches. When applicable, we summarize results by reporting the average correlation. And the average ranking of the metric per dataset (by ranking which metric obtains the highest correlation to human judgement per dataset). 

\subsection{Content preservation}
\paragraph{How do data sources affect the conclusion on best metric?}
The conclusions about the metrics' performance change radically depending on whether we use system output data, reference output, or our constructed test set. Ideally, a good metric correlates highly with humans on any data source. Ideally, for meta-evaluation, a metric should correlate consistently across all data sources, but the following shows that the correlations indicate different things, and the conclusion on the best metric should be drawn carefully.

Looking at the metrics correlations with humans on the data source with system output (Table~\ref{tab:sys}), we see a relatively high correlation for many of the metrics on many tasks. The overall best metrics are S-BertScore and S-BLEURT (avg+avg rank). We see no notable difference in our method of using the 3B or 8B model as the backbone.

Examining the average correlations based on data with reference output (Table~\ref{tab:ref}), now the zero-shoot prompting with LlaMA3 70B is the best-performing approach ($0.59$ avg). Tied for second place are source-based cosine embedding ($0.46$ avg), BLEURT ($0.45$ avg) and BertScore ($0.45$ avg). Our method follows on a 5. place: here, the 8b version (($0.43$ avg)) shows a bit stronger results than 3b ($0.39$ avg). The fact that the conclusions change, whether looking at reference or system output, confirms the observations made by \citet{scialom-etal-2021-questeval} on simplicity transfer.   

Now consider the results on our test set (Table~\ref{tab:con}): Several metrics show low or no correlation; we even see a significantly negative correlation for some metrics on ALL (BLEU) and for specific subparts of our test set for BLEU, Meteor, BertScore, Cosine. On the other end, LlaMA3 70B is again performing best, showing strong results ($0.82$ in ALL). The runner-up is now our 8B method, with a gap to the 3B version ($0.57$ vs $0.47$ in ALL). Note our method still shows zero correlation for the sentiment task. After, ranks BLEURT ($0.29$), QuestEval ($0.21$), BertScore ($0.11$), Cosine ($0.01$).  

On our test set, we find that some metrics that correlate relatively well on the other datasets, now exhibit low correlation. Hence, with our test set, we can now support the logical reasoning with data evidence: Evaluation of content preservation for style transfer needs to take the style shift into account. This conclusion could not be drawn using the existing data sources: We hypothesise that for the data with system-based output, successful output happens to be very similar to the source sentence and vice versa, and reference-based output might not contain server mistakes as they are gold references. Thus, none of the existing data sources tests the limits of the metrics.  


\paragraph{How do reference-based metrics compare to source-based ones?} Reference-based metrics show a lower correlation than the source-based counterpart for all metrics on both datasets with ratings on references (Table~\ref{tab:sys}). As discussed previously, reference-based metrics for style transfer have the drawback that many different good solutions on a rewrite might exist and not only one similar to a reference.


\paragraph{How well can the metrics rank the performance of style transfer methods?}
We compare the metrics' ability to judge the best style transfer methods w.r.t. the human annotations: Several of the data sources contain samples from different style transfer systems. In order to use metrics to assess the quality of the style transfer system, metrics should correctly find the best-performing system. Hence, we evaluate whether the metrics for content preservation provide the same system ranking as human evaluators. We take the mean of the score for every output on each system and the mean of the human annotations; we compare the systems using the Kendall's Tau correlation. 

We find only the evaluation using the dataset Mir, Lai, and Ziegen to result in significant correlations, probably because of sparsity in a number of system tests (App.~\ref{app:dataset}). Our method (8b) is the only metric providing a perfect ranking of the style transfer system on the Lai data, and Llama3 70B the only one on the Ziegen data. Results in App.~\ref{app:results}. 


\subsection{Style strength results}
%Evaluating style strengths is a challenging task. 
Llama3 70B shows better overall results than our method. However, our method scores higher than Llama3 70B on 2 out of 6 datasets, but it also exhibits zero correlation on one task (Table~\ref{tab:styleresults}).%More work i s needed on evaluating style strengths. 
 
\begin{table}%[]
\fontsize{11pt}{11pt}\selectfont
\begin{tabular}{lccc}
\toprule
\multicolumn{1}{c}{\textbf{}} & \textbf{LlaMA3} & \textbf{Our (3b)} & \textbf{Our (8b)} \\ \midrule
\textbf{Mir} & 0.46 & 0.54 & \textbf{0.57} \\
\textbf{Lai} & \textbf{0.57} & 0.18 & 0.19 \\
\textbf{Ziegen.} & 0.25 & 0.27 & \textbf{0.32} \\
\textbf{Alva-M.} & \textbf{0.59} & 0.03* & 0.02* \\
\textbf{Scialom} & \textbf{0.62} & 0.45 & 0.44 \\
\textbf{\begin{tabular}[c]{@{}l@{}}Our Test\end{tabular}} & \textbf{0.63} & 0.46 & 0.48 \\ \bottomrule
\end{tabular}
\caption{Style strength: Pearson correlation to human ratings. *not significant; we cannot reject the null hypothesis of zero corelation}
\label{tab:styleresults}
\end{table}

\subsection{Ablation}
We conduct several runs of the methods using LLMs with variations in instructions/prompts (App.~\ref{app:method}). We observe that the lower the correlation on a task, the higher the variation between the different runs. For our method, we only observe low variance between the runs.
None of the variations leads to different conclusions of the meta-evaluation. Results in App.~\ref{app:results}.


\section{Conclusion}
In this work, we propose a simple yet effective approach, called SMILE, for graph few-shot learning with fewer tasks. Specifically, we introduce a novel dual-level mixup strategy, including within-task and across-task mixup, for enriching the diversity of nodes within each task and the diversity of tasks. Also, we incorporate the degree-based prior information to learn expressive node embeddings. Theoretically, we prove that SMILE effectively enhances the model's generalization performance. Empirically, we conduct extensive experiments on multiple benchmarks and the results suggest that SMILE significantly outperforms other baselines, including both in-domain and cross-domain few-shot settings. 
\section{Limitation}
The use of 3D-printed PLA for structural components improves improving ease of assembly and reduces weight and cost, yet it causes deformation under heavy load, which can diminish end-effector precision. Using metal, such as aluminum, would remedy this problem. Additionally, \robot relies on integrated joint relative encoders, requiring manual initialization in a fixed joint configuration each time the system is powered on. Using absolute joint encoders could significantly improve accuracy and ease of use, although it would increase the overall cost. 

%Reliance on commercially available actuators simplifies integration but imposes constraints on control frequency and customization, further limiting the potential for tailored performance improvements.

% The 6 DoF configuration provides sufficient mobility for most tasks; however, certain bimanual operations could benefit from an additional degree of freedom to handle complex joint constraints more effectively. Furthermore, the limited torque density of commercially available proprioceptive actuators restricts the payload and torque output, making the system less suitability for handling heavier loads or high-torque applications. 

The 6 DoF configuration of the arm provides sufficient mobility for single-arm manipulation tasks, yet it shows a limitation in certain bimanual manipulation problems. Specifically, when \robot holds onto a rigid object with both hands, each arm loses 1 DoF because the hands are fixed to the object during grasping. This leads to an underactuated kinematic chain which has a limited mobility in 3D space. We can achieve more mobility by letting the object slip inside the grippers, yet this renders the grasp less robust and simulation difficult. Therefore, we anticipate that designing a lightweight 3 DoF wrist in place of the current 2 DoF wrist allows a more diverse repertoire of manipulation in bimanual tasks.

Finally, the limited torque density of commercially available proprioceptive actuators restricts the performance. Currently, all of our actuators feature a 1:10 gear ratio, so \robot can handle up to 2.5 kg of payload. To handle a heavier object and manipulate it with higher torque, we expect the actuator to have 1:20$\sim$30 gear ratio, but it is difficult to find an off-the-shelf product that meets our requirements. Customizing the actuator to increase the torque density while minimizing the weight will enable \robot to move faster and handle more diverse objects.

%These constraints highlight opportunities for improvement in future iterations, including alternative materials for enhanced rigidity, custom actuator designs for higher control precision and torque density, the adoption of absolute joint encoders, and optimized configurations to balance dexterity and weight.



% Bibliography entries for the entire Anthology, followed by custom entries
%\bibliography{anthology,custom}
% Custom bibliography entries only
\bibliography{custom}

\appendix

\subsection{Lloyd-Max Algorithm}
\label{subsec:Lloyd-Max}
For a given quantization bitwidth $B$ and an operand $\bm{X}$, the Lloyd-Max algorithm finds $2^B$ quantization levels $\{\hat{x}_i\}_{i=1}^{2^B}$ such that quantizing $\bm{X}$ by rounding each scalar in $\bm{X}$ to the nearest quantization level minimizes the quantization MSE. 

The algorithm starts with an initial guess of quantization levels and then iteratively computes quantization thresholds $\{\tau_i\}_{i=1}^{2^B-1}$ and updates quantization levels $\{\hat{x}_i\}_{i=1}^{2^B}$. Specifically, at iteration $n$, thresholds are set to the midpoints of the previous iteration's levels:
\begin{align*}
    \tau_i^{(n)}=\frac{\hat{x}_i^{(n-1)}+\hat{x}_{i+1}^{(n-1)}}2 \text{ for } i=1\ldots 2^B-1
\end{align*}
Subsequently, the quantization levels are re-computed as conditional means of the data regions defined by the new thresholds:
\begin{align*}
    \hat{x}_i^{(n)}=\mathbb{E}\left[ \bm{X} \big| \bm{X}\in [\tau_{i-1}^{(n)},\tau_i^{(n)}] \right] \text{ for } i=1\ldots 2^B
\end{align*}
where to satisfy boundary conditions we have $\tau_0=-\infty$ and $\tau_{2^B}=\infty$. The algorithm iterates the above steps until convergence.

Figure \ref{fig:lm_quant} compares the quantization levels of a $7$-bit floating point (E3M3) quantizer (left) to a $7$-bit Lloyd-Max quantizer (right) when quantizing a layer of weights from the GPT3-126M model at a per-tensor granularity. As shown, the Lloyd-Max quantizer achieves substantially lower quantization MSE. Further, Table \ref{tab:FP7_vs_LM7} shows the superior perplexity achieved by Lloyd-Max quantizers for bitwidths of $7$, $6$ and $5$. The difference between the quantizers is clear at 5 bits, where per-tensor FP quantization incurs a drastic and unacceptable increase in perplexity, while Lloyd-Max quantization incurs a much smaller increase. Nevertheless, we note that even the optimal Lloyd-Max quantizer incurs a notable ($\sim 1.5$) increase in perplexity due to the coarse granularity of quantization. 

\begin{figure}[h]
  \centering
  \includegraphics[width=0.7\linewidth]{sections/figures/LM7_FP7.pdf}
  \caption{\small Quantization levels and the corresponding quantization MSE of Floating Point (left) vs Lloyd-Max (right) Quantizers for a layer of weights in the GPT3-126M model.}
  \label{fig:lm_quant}
\end{figure}

\begin{table}[h]\scriptsize
\begin{center}
\caption{\label{tab:FP7_vs_LM7} \small Comparing perplexity (lower is better) achieved by floating point quantizers and Lloyd-Max quantizers on a GPT3-126M model for the Wikitext-103 dataset.}
\begin{tabular}{c|cc|c}
\hline
 \multirow{2}{*}{\textbf{Bitwidth}} & \multicolumn{2}{|c|}{\textbf{Floating-Point Quantizer}} & \textbf{Lloyd-Max Quantizer} \\
 & Best Format & Wikitext-103 Perplexity & Wikitext-103 Perplexity \\
\hline
7 & E3M3 & 18.32 & 18.27 \\
6 & E3M2 & 19.07 & 18.51 \\
5 & E4M0 & 43.89 & 19.71 \\
\hline
\end{tabular}
\end{center}
\end{table}

\subsection{Proof of Local Optimality of LO-BCQ}
\label{subsec:lobcq_opt_proof}
For a given block $\bm{b}_j$, the quantization MSE during LO-BCQ can be empirically evaluated as $\frac{1}{L_b}\lVert \bm{b}_j- \bm{\hat{b}}_j\rVert^2_2$ where $\bm{\hat{b}}_j$ is computed from equation (\ref{eq:clustered_quantization_definition}) as $C_{f(\bm{b}_j)}(\bm{b}_j)$. Further, for a given block cluster $\mathcal{B}_i$, we compute the quantization MSE as $\frac{1}{|\mathcal{B}_{i}|}\sum_{\bm{b} \in \mathcal{B}_{i}} \frac{1}{L_b}\lVert \bm{b}- C_i^{(n)}(\bm{b})\rVert^2_2$. Therefore, at the end of iteration $n$, we evaluate the overall quantization MSE $J^{(n)}$ for a given operand $\bm{X}$ composed of $N_c$ block clusters as:
\begin{align*}
    \label{eq:mse_iter_n}
    J^{(n)} = \frac{1}{N_c} \sum_{i=1}^{N_c} \frac{1}{|\mathcal{B}_{i}^{(n)}|}\sum_{\bm{v} \in \mathcal{B}_{i}^{(n)}} \frac{1}{L_b}\lVert \bm{b}- B_i^{(n)}(\bm{b})\rVert^2_2
\end{align*}

At the end of iteration $n$, the codebooks are updated from $\mathcal{C}^{(n-1)}$ to $\mathcal{C}^{(n)}$. However, the mapping of a given vector $\bm{b}_j$ to quantizers $\mathcal{C}^{(n)}$ remains as  $f^{(n)}(\bm{b}_j)$. At the next iteration, during the vector clustering step, $f^{(n+1)}(\bm{b}_j)$ finds new mapping of $\bm{b}_j$ to updated codebooks $\mathcal{C}^{(n)}$ such that the quantization MSE over the candidate codebooks is minimized. Therefore, we obtain the following result for $\bm{b}_j$:
\begin{align*}
\frac{1}{L_b}\lVert \bm{b}_j - C_{f^{(n+1)}(\bm{b}_j)}^{(n)}(\bm{b}_j)\rVert^2_2 \le \frac{1}{L_b}\lVert \bm{b}_j - C_{f^{(n)}(\bm{b}_j)}^{(n)}(\bm{b}_j)\rVert^2_2
\end{align*}

That is, quantizing $\bm{b}_j$ at the end of the block clustering step of iteration $n+1$ results in lower quantization MSE compared to quantizing at the end of iteration $n$. Since this is true for all $\bm{b} \in \bm{X}$, we assert the following:
\begin{equation}
\begin{split}
\label{eq:mse_ineq_1}
    \tilde{J}^{(n+1)} &= \frac{1}{N_c} \sum_{i=1}^{N_c} \frac{1}{|\mathcal{B}_{i}^{(n+1)}|}\sum_{\bm{b} \in \mathcal{B}_{i}^{(n+1)}} \frac{1}{L_b}\lVert \bm{b} - C_i^{(n)}(b)\rVert^2_2 \le J^{(n)}
\end{split}
\end{equation}
where $\tilde{J}^{(n+1)}$ is the the quantization MSE after the vector clustering step at iteration $n+1$.

Next, during the codebook update step (\ref{eq:quantizers_update}) at iteration $n+1$, the per-cluster codebooks $\mathcal{C}^{(n)}$ are updated to $\mathcal{C}^{(n+1)}$ by invoking the Lloyd-Max algorithm \citep{Lloyd}. We know that for any given value distribution, the Lloyd-Max algorithm minimizes the quantization MSE. Therefore, for a given vector cluster $\mathcal{B}_i$ we obtain the following result:

\begin{equation}
    \frac{1}{|\mathcal{B}_{i}^{(n+1)}|}\sum_{\bm{b} \in \mathcal{B}_{i}^{(n+1)}} \frac{1}{L_b}\lVert \bm{b}- C_i^{(n+1)}(\bm{b})\rVert^2_2 \le \frac{1}{|\mathcal{B}_{i}^{(n+1)}|}\sum_{\bm{b} \in \mathcal{B}_{i}^{(n+1)}} \frac{1}{L_b}\lVert \bm{b}- C_i^{(n)}(\bm{b})\rVert^2_2
\end{equation}

The above equation states that quantizing the given block cluster $\mathcal{B}_i$ after updating the associated codebook from $C_i^{(n)}$ to $C_i^{(n+1)}$ results in lower quantization MSE. Since this is true for all the block clusters, we derive the following result: 
\begin{equation}
\begin{split}
\label{eq:mse_ineq_2}
     J^{(n+1)} &= \frac{1}{N_c} \sum_{i=1}^{N_c} \frac{1}{|\mathcal{B}_{i}^{(n+1)}|}\sum_{\bm{b} \in \mathcal{B}_{i}^{(n+1)}} \frac{1}{L_b}\lVert \bm{b}- C_i^{(n+1)}(\bm{b})\rVert^2_2  \le \tilde{J}^{(n+1)}   
\end{split}
\end{equation}

Following (\ref{eq:mse_ineq_1}) and (\ref{eq:mse_ineq_2}), we find that the quantization MSE is non-increasing for each iteration, that is, $J^{(1)} \ge J^{(2)} \ge J^{(3)} \ge \ldots \ge J^{(M)}$ where $M$ is the maximum number of iterations. 
%Therefore, we can say that if the algorithm converges, then it must be that it has converged to a local minimum. 
\hfill $\blacksquare$


\begin{figure}
    \begin{center}
    \includegraphics[width=0.5\textwidth]{sections//figures/mse_vs_iter.pdf}
    \end{center}
    \caption{\small NMSE vs iterations during LO-BCQ compared to other block quantization proposals}
    \label{fig:nmse_vs_iter}
\end{figure}

Figure \ref{fig:nmse_vs_iter} shows the empirical convergence of LO-BCQ across several block lengths and number of codebooks. Also, the MSE achieved by LO-BCQ is compared to baselines such as MXFP and VSQ. As shown, LO-BCQ converges to a lower MSE than the baselines. Further, we achieve better convergence for larger number of codebooks ($N_c$) and for a smaller block length ($L_b$), both of which increase the bitwidth of BCQ (see Eq \ref{eq:bitwidth_bcq}).


\subsection{Additional Accuracy Results}
%Table \ref{tab:lobcq_config} lists the various LOBCQ configurations and their corresponding bitwidths.
\begin{table}
\setlength{\tabcolsep}{4.75pt}
\begin{center}
\caption{\label{tab:lobcq_config} Various LO-BCQ configurations and their bitwidths.}
\begin{tabular}{|c||c|c|c|c||c|c||c|} 
\hline
 & \multicolumn{4}{|c||}{$L_b=8$} & \multicolumn{2}{|c||}{$L_b=4$} & $L_b=2$ \\
 \hline
 \backslashbox{$L_A$\kern-1em}{\kern-1em$N_c$} & 2 & 4 & 8 & 16 & 2 & 4 & 2 \\
 \hline
 64 & 4.25 & 4.375 & 4.5 & 4.625 & 4.375 & 4.625 & 4.625\\
 \hline
 32 & 4.375 & 4.5 & 4.625& 4.75 & 4.5 & 4.75 & 4.75 \\
 \hline
 16 & 4.625 & 4.75& 4.875 & 5 & 4.75 & 5 & 5 \\
 \hline
\end{tabular}
\end{center}
\end{table}

%\subsection{Perplexity achieved by various LO-BCQ configurations on Wikitext-103 dataset}

\begin{table} \centering
\begin{tabular}{|c||c|c|c|c||c|c||c|} 
\hline
 $L_b \rightarrow$& \multicolumn{4}{c||}{8} & \multicolumn{2}{c||}{4} & 2\\
 \hline
 \backslashbox{$L_A$\kern-1em}{\kern-1em$N_c$} & 2 & 4 & 8 & 16 & 2 & 4 & 2  \\
 %$N_c \rightarrow$ & 2 & 4 & 8 & 16 & 2 & 4 & 2 \\
 \hline
 \hline
 \multicolumn{8}{c}{GPT3-1.3B (FP32 PPL = 9.98)} \\ 
 \hline
 \hline
 64 & 10.40 & 10.23 & 10.17 & 10.15 &  10.28 & 10.18 & 10.19 \\
 \hline
 32 & 10.25 & 10.20 & 10.15 & 10.12 &  10.23 & 10.17 & 10.17 \\
 \hline
 16 & 10.22 & 10.16 & 10.10 & 10.09 &  10.21 & 10.14 & 10.16 \\
 \hline
  \hline
 \multicolumn{8}{c}{GPT3-8B (FP32 PPL = 7.38)} \\ 
 \hline
 \hline
 64 & 7.61 & 7.52 & 7.48 &  7.47 &  7.55 &  7.49 & 7.50 \\
 \hline
 32 & 7.52 & 7.50 & 7.46 &  7.45 &  7.52 &  7.48 & 7.48  \\
 \hline
 16 & 7.51 & 7.48 & 7.44 &  7.44 &  7.51 &  7.49 & 7.47  \\
 \hline
\end{tabular}
\caption{\label{tab:ppl_gpt3_abalation} Wikitext-103 perplexity across GPT3-1.3B and 8B models.}
\end{table}

\begin{table} \centering
\begin{tabular}{|c||c|c|c|c||} 
\hline
 $L_b \rightarrow$& \multicolumn{4}{c||}{8}\\
 \hline
 \backslashbox{$L_A$\kern-1em}{\kern-1em$N_c$} & 2 & 4 & 8 & 16 \\
 %$N_c \rightarrow$ & 2 & 4 & 8 & 16 & 2 & 4 & 2 \\
 \hline
 \hline
 \multicolumn{5}{|c|}{Llama2-7B (FP32 PPL = 5.06)} \\ 
 \hline
 \hline
 64 & 5.31 & 5.26 & 5.19 & 5.18  \\
 \hline
 32 & 5.23 & 5.25 & 5.18 & 5.15  \\
 \hline
 16 & 5.23 & 5.19 & 5.16 & 5.14  \\
 \hline
 \multicolumn{5}{|c|}{Nemotron4-15B (FP32 PPL = 5.87)} \\ 
 \hline
 \hline
 64  & 6.3 & 6.20 & 6.13 & 6.08  \\
 \hline
 32  & 6.24 & 6.12 & 6.07 & 6.03  \\
 \hline
 16  & 6.12 & 6.14 & 6.04 & 6.02  \\
 \hline
 \multicolumn{5}{|c|}{Nemotron4-340B (FP32 PPL = 3.48)} \\ 
 \hline
 \hline
 64 & 3.67 & 3.62 & 3.60 & 3.59 \\
 \hline
 32 & 3.63 & 3.61 & 3.59 & 3.56 \\
 \hline
 16 & 3.61 & 3.58 & 3.57 & 3.55 \\
 \hline
\end{tabular}
\caption{\label{tab:ppl_llama7B_nemo15B} Wikitext-103 perplexity compared to FP32 baseline in Llama2-7B and Nemotron4-15B, 340B models}
\end{table}

%\subsection{Perplexity achieved by various LO-BCQ configurations on MMLU dataset}


\begin{table} \centering
\begin{tabular}{|c||c|c|c|c||c|c|c|c|} 
\hline
 $L_b \rightarrow$& \multicolumn{4}{c||}{8} & \multicolumn{4}{c||}{8}\\
 \hline
 \backslashbox{$L_A$\kern-1em}{\kern-1em$N_c$} & 2 & 4 & 8 & 16 & 2 & 4 & 8 & 16  \\
 %$N_c \rightarrow$ & 2 & 4 & 8 & 16 & 2 & 4 & 2 \\
 \hline
 \hline
 \multicolumn{5}{|c|}{Llama2-7B (FP32 Accuracy = 45.8\%)} & \multicolumn{4}{|c|}{Llama2-70B (FP32 Accuracy = 69.12\%)} \\ 
 \hline
 \hline
 64 & 43.9 & 43.4 & 43.9 & 44.9 & 68.07 & 68.27 & 68.17 & 68.75 \\
 \hline
 32 & 44.5 & 43.8 & 44.9 & 44.5 & 68.37 & 68.51 & 68.35 & 68.27  \\
 \hline
 16 & 43.9 & 42.7 & 44.9 & 45 & 68.12 & 68.77 & 68.31 & 68.59  \\
 \hline
 \hline
 \multicolumn{5}{|c|}{GPT3-22B (FP32 Accuracy = 38.75\%)} & \multicolumn{4}{|c|}{Nemotron4-15B (FP32 Accuracy = 64.3\%)} \\ 
 \hline
 \hline
 64 & 36.71 & 38.85 & 38.13 & 38.92 & 63.17 & 62.36 & 63.72 & 64.09 \\
 \hline
 32 & 37.95 & 38.69 & 39.45 & 38.34 & 64.05 & 62.30 & 63.8 & 64.33  \\
 \hline
 16 & 38.88 & 38.80 & 38.31 & 38.92 & 63.22 & 63.51 & 63.93 & 64.43  \\
 \hline
\end{tabular}
\caption{\label{tab:mmlu_abalation} Accuracy on MMLU dataset across GPT3-22B, Llama2-7B, 70B and Nemotron4-15B models.}
\end{table}


%\subsection{Perplexity achieved by various LO-BCQ configurations on LM evaluation harness}

\begin{table} \centering
\begin{tabular}{|c||c|c|c|c||c|c|c|c|} 
\hline
 $L_b \rightarrow$& \multicolumn{4}{c||}{8} & \multicolumn{4}{c||}{8}\\
 \hline
 \backslashbox{$L_A$\kern-1em}{\kern-1em$N_c$} & 2 & 4 & 8 & 16 & 2 & 4 & 8 & 16  \\
 %$N_c \rightarrow$ & 2 & 4 & 8 & 16 & 2 & 4 & 2 \\
 \hline
 \hline
 \multicolumn{5}{|c|}{Race (FP32 Accuracy = 37.51\%)} & \multicolumn{4}{|c|}{Boolq (FP32 Accuracy = 64.62\%)} \\ 
 \hline
 \hline
 64 & 36.94 & 37.13 & 36.27 & 37.13 & 63.73 & 62.26 & 63.49 & 63.36 \\
 \hline
 32 & 37.03 & 36.36 & 36.08 & 37.03 & 62.54 & 63.51 & 63.49 & 63.55  \\
 \hline
 16 & 37.03 & 37.03 & 36.46 & 37.03 & 61.1 & 63.79 & 63.58 & 63.33  \\
 \hline
 \hline
 \multicolumn{5}{|c|}{Winogrande (FP32 Accuracy = 58.01\%)} & \multicolumn{4}{|c|}{Piqa (FP32 Accuracy = 74.21\%)} \\ 
 \hline
 \hline
 64 & 58.17 & 57.22 & 57.85 & 58.33 & 73.01 & 73.07 & 73.07 & 72.80 \\
 \hline
 32 & 59.12 & 58.09 & 57.85 & 58.41 & 73.01 & 73.94 & 72.74 & 73.18  \\
 \hline
 16 & 57.93 & 58.88 & 57.93 & 58.56 & 73.94 & 72.80 & 73.01 & 73.94  \\
 \hline
\end{tabular}
\caption{\label{tab:mmlu_abalation} Accuracy on LM evaluation harness tasks on GPT3-1.3B model.}
\end{table}

\begin{table} \centering
\begin{tabular}{|c||c|c|c|c||c|c|c|c|} 
\hline
 $L_b \rightarrow$& \multicolumn{4}{c||}{8} & \multicolumn{4}{c||}{8}\\
 \hline
 \backslashbox{$L_A$\kern-1em}{\kern-1em$N_c$} & 2 & 4 & 8 & 16 & 2 & 4 & 8 & 16  \\
 %$N_c \rightarrow$ & 2 & 4 & 8 & 16 & 2 & 4 & 2 \\
 \hline
 \hline
 \multicolumn{5}{|c|}{Race (FP32 Accuracy = 41.34\%)} & \multicolumn{4}{|c|}{Boolq (FP32 Accuracy = 68.32\%)} \\ 
 \hline
 \hline
 64 & 40.48 & 40.10 & 39.43 & 39.90 & 69.20 & 68.41 & 69.45 & 68.56 \\
 \hline
 32 & 39.52 & 39.52 & 40.77 & 39.62 & 68.32 & 67.43 & 68.17 & 69.30  \\
 \hline
 16 & 39.81 & 39.71 & 39.90 & 40.38 & 68.10 & 66.33 & 69.51 & 69.42  \\
 \hline
 \hline
 \multicolumn{5}{|c|}{Winogrande (FP32 Accuracy = 67.88\%)} & \multicolumn{4}{|c|}{Piqa (FP32 Accuracy = 78.78\%)} \\ 
 \hline
 \hline
 64 & 66.85 & 66.61 & 67.72 & 67.88 & 77.31 & 77.42 & 77.75 & 77.64 \\
 \hline
 32 & 67.25 & 67.72 & 67.72 & 67.00 & 77.31 & 77.04 & 77.80 & 77.37  \\
 \hline
 16 & 68.11 & 68.90 & 67.88 & 67.48 & 77.37 & 78.13 & 78.13 & 77.69  \\
 \hline
\end{tabular}
\caption{\label{tab:mmlu_abalation} Accuracy on LM evaluation harness tasks on GPT3-8B model.}
\end{table}

\begin{table} \centering
\begin{tabular}{|c||c|c|c|c||c|c|c|c|} 
\hline
 $L_b \rightarrow$& \multicolumn{4}{c||}{8} & \multicolumn{4}{c||}{8}\\
 \hline
 \backslashbox{$L_A$\kern-1em}{\kern-1em$N_c$} & 2 & 4 & 8 & 16 & 2 & 4 & 8 & 16  \\
 %$N_c \rightarrow$ & 2 & 4 & 8 & 16 & 2 & 4 & 2 \\
 \hline
 \hline
 \multicolumn{5}{|c|}{Race (FP32 Accuracy = 40.67\%)} & \multicolumn{4}{|c|}{Boolq (FP32 Accuracy = 76.54\%)} \\ 
 \hline
 \hline
 64 & 40.48 & 40.10 & 39.43 & 39.90 & 75.41 & 75.11 & 77.09 & 75.66 \\
 \hline
 32 & 39.52 & 39.52 & 40.77 & 39.62 & 76.02 & 76.02 & 75.96 & 75.35  \\
 \hline
 16 & 39.81 & 39.71 & 39.90 & 40.38 & 75.05 & 73.82 & 75.72 & 76.09  \\
 \hline
 \hline
 \multicolumn{5}{|c|}{Winogrande (FP32 Accuracy = 70.64\%)} & \multicolumn{4}{|c|}{Piqa (FP32 Accuracy = 79.16\%)} \\ 
 \hline
 \hline
 64 & 69.14 & 70.17 & 70.17 & 70.56 & 78.24 & 79.00 & 78.62 & 78.73 \\
 \hline
 32 & 70.96 & 69.69 & 71.27 & 69.30 & 78.56 & 79.49 & 79.16 & 78.89  \\
 \hline
 16 & 71.03 & 69.53 & 69.69 & 70.40 & 78.13 & 79.16 & 79.00 & 79.00  \\
 \hline
\end{tabular}
\caption{\label{tab:mmlu_abalation} Accuracy on LM evaluation harness tasks on GPT3-22B model.}
\end{table}

\begin{table} \centering
\begin{tabular}{|c||c|c|c|c||c|c|c|c|} 
\hline
 $L_b \rightarrow$& \multicolumn{4}{c||}{8} & \multicolumn{4}{c||}{8}\\
 \hline
 \backslashbox{$L_A$\kern-1em}{\kern-1em$N_c$} & 2 & 4 & 8 & 16 & 2 & 4 & 8 & 16  \\
 %$N_c \rightarrow$ & 2 & 4 & 8 & 16 & 2 & 4 & 2 \\
 \hline
 \hline
 \multicolumn{5}{|c|}{Race (FP32 Accuracy = 44.4\%)} & \multicolumn{4}{|c|}{Boolq (FP32 Accuracy = 79.29\%)} \\ 
 \hline
 \hline
 64 & 42.49 & 42.51 & 42.58 & 43.45 & 77.58 & 77.37 & 77.43 & 78.1 \\
 \hline
 32 & 43.35 & 42.49 & 43.64 & 43.73 & 77.86 & 75.32 & 77.28 & 77.86  \\
 \hline
 16 & 44.21 & 44.21 & 43.64 & 42.97 & 78.65 & 77 & 76.94 & 77.98  \\
 \hline
 \hline
 \multicolumn{5}{|c|}{Winogrande (FP32 Accuracy = 69.38\%)} & \multicolumn{4}{|c|}{Piqa (FP32 Accuracy = 78.07\%)} \\ 
 \hline
 \hline
 64 & 68.9 & 68.43 & 69.77 & 68.19 & 77.09 & 76.82 & 77.09 & 77.86 \\
 \hline
 32 & 69.38 & 68.51 & 68.82 & 68.90 & 78.07 & 76.71 & 78.07 & 77.86  \\
 \hline
 16 & 69.53 & 67.09 & 69.38 & 68.90 & 77.37 & 77.8 & 77.91 & 77.69  \\
 \hline
\end{tabular}
\caption{\label{tab:mmlu_abalation} Accuracy on LM evaluation harness tasks on Llama2-7B model.}
\end{table}

\begin{table} \centering
\begin{tabular}{|c||c|c|c|c||c|c|c|c|} 
\hline
 $L_b \rightarrow$& \multicolumn{4}{c||}{8} & \multicolumn{4}{c||}{8}\\
 \hline
 \backslashbox{$L_A$\kern-1em}{\kern-1em$N_c$} & 2 & 4 & 8 & 16 & 2 & 4 & 8 & 16  \\
 %$N_c \rightarrow$ & 2 & 4 & 8 & 16 & 2 & 4 & 2 \\
 \hline
 \hline
 \multicolumn{5}{|c|}{Race (FP32 Accuracy = 48.8\%)} & \multicolumn{4}{|c|}{Boolq (FP32 Accuracy = 85.23\%)} \\ 
 \hline
 \hline
 64 & 49.00 & 49.00 & 49.28 & 48.71 & 82.82 & 84.28 & 84.03 & 84.25 \\
 \hline
 32 & 49.57 & 48.52 & 48.33 & 49.28 & 83.85 & 84.46 & 84.31 & 84.93  \\
 \hline
 16 & 49.85 & 49.09 & 49.28 & 48.99 & 85.11 & 84.46 & 84.61 & 83.94  \\
 \hline
 \hline
 \multicolumn{5}{|c|}{Winogrande (FP32 Accuracy = 79.95\%)} & \multicolumn{4}{|c|}{Piqa (FP32 Accuracy = 81.56\%)} \\ 
 \hline
 \hline
 64 & 78.77 & 78.45 & 78.37 & 79.16 & 81.45 & 80.69 & 81.45 & 81.5 \\
 \hline
 32 & 78.45 & 79.01 & 78.69 & 80.66 & 81.56 & 80.58 & 81.18 & 81.34  \\
 \hline
 16 & 79.95 & 79.56 & 79.79 & 79.72 & 81.28 & 81.66 & 81.28 & 80.96  \\
 \hline
\end{tabular}
\caption{\label{tab:mmlu_abalation} Accuracy on LM evaluation harness tasks on Llama2-70B model.}
\end{table}

%\section{MSE Studies}
%\textcolor{red}{TODO}


\subsection{Number Formats and Quantization Method}
\label{subsec:numFormats_quantMethod}
\subsubsection{Integer Format}
An $n$-bit signed integer (INT) is typically represented with a 2s-complement format \citep{yao2022zeroquant,xiao2023smoothquant,dai2021vsq}, where the most significant bit denotes the sign.

\subsubsection{Floating Point Format}
An $n$-bit signed floating point (FP) number $x$ comprises of a 1-bit sign ($x_{\mathrm{sign}}$), $B_m$-bit mantissa ($x_{\mathrm{mant}}$) and $B_e$-bit exponent ($x_{\mathrm{exp}}$) such that $B_m+B_e=n-1$. The associated constant exponent bias ($E_{\mathrm{bias}}$) is computed as $(2^{{B_e}-1}-1)$. We denote this format as $E_{B_e}M_{B_m}$.  

\subsubsection{Quantization Scheme}
\label{subsec:quant_method}
A quantization scheme dictates how a given unquantized tensor is converted to its quantized representation. We consider FP formats for the purpose of illustration. Given an unquantized tensor $\bm{X}$ and an FP format $E_{B_e}M_{B_m}$, we first, we compute the quantization scale factor $s_X$ that maps the maximum absolute value of $\bm{X}$ to the maximum quantization level of the $E_{B_e}M_{B_m}$ format as follows:
\begin{align}
\label{eq:sf}
    s_X = \frac{\mathrm{max}(|\bm{X}|)}{\mathrm{max}(E_{B_e}M_{B_m})}
\end{align}
In the above equation, $|\cdot|$ denotes the absolute value function.

Next, we scale $\bm{X}$ by $s_X$ and quantize it to $\hat{\bm{X}}$ by rounding it to the nearest quantization level of $E_{B_e}M_{B_m}$ as:

\begin{align}
\label{eq:tensor_quant}
    \hat{\bm{X}} = \text{round-to-nearest}\left(\frac{\bm{X}}{s_X}, E_{B_e}M_{B_m}\right)
\end{align}

We perform dynamic max-scaled quantization \citep{wu2020integer}, where the scale factor $s$ for activations is dynamically computed during runtime.

\subsection{Vector Scaled Quantization}
\begin{wrapfigure}{r}{0.35\linewidth}
  \centering
  \includegraphics[width=\linewidth]{sections/figures/vsquant.jpg}
  \caption{\small Vectorwise decomposition for per-vector scaled quantization (VSQ \citep{dai2021vsq}).}
  \label{fig:vsquant}
\end{wrapfigure}
During VSQ \citep{dai2021vsq}, the operand tensors are decomposed into 1D vectors in a hardware friendly manner as shown in Figure \ref{fig:vsquant}. Since the decomposed tensors are used as operands in matrix multiplications during inference, it is beneficial to perform this decomposition along the reduction dimension of the multiplication. The vectorwise quantization is performed similar to tensorwise quantization described in Equations \ref{eq:sf} and \ref{eq:tensor_quant}, where a scale factor $s_v$ is required for each vector $\bm{v}$ that maps the maximum absolute value of that vector to the maximum quantization level. While smaller vector lengths can lead to larger accuracy gains, the associated memory and computational overheads due to the per-vector scale factors increases. To alleviate these overheads, VSQ \citep{dai2021vsq} proposed a second level quantization of the per-vector scale factors to unsigned integers, while MX \citep{rouhani2023shared} quantizes them to integer powers of 2 (denoted as $2^{INT}$).

\subsubsection{MX Format}
The MX format proposed in \citep{rouhani2023microscaling} introduces the concept of sub-block shifting. For every two scalar elements of $b$-bits each, there is a shared exponent bit. The value of this exponent bit is determined through an empirical analysis that targets minimizing quantization MSE. We note that the FP format $E_{1}M_{b}$ is strictly better than MX from an accuracy perspective since it allocates a dedicated exponent bit to each scalar as opposed to sharing it across two scalars. Therefore, we conservatively bound the accuracy of a $b+2$-bit signed MX format with that of a $E_{1}M_{b}$ format in our comparisons. For instance, we use E1M2 format as a proxy for MX4.

\begin{figure}
    \centering
    \includegraphics[width=1\linewidth]{sections//figures/BlockFormats.pdf}
    \caption{\small Comparing LO-BCQ to MX format.}
    \label{fig:block_formats}
\end{figure}

Figure \ref{fig:block_formats} compares our $4$-bit LO-BCQ block format to MX \citep{rouhani2023microscaling}. As shown, both LO-BCQ and MX decompose a given operand tensor into block arrays and each block array into blocks. Similar to MX, we find that per-block quantization ($L_b < L_A$) leads to better accuracy due to increased flexibility. While MX achieves this through per-block $1$-bit micro-scales, we associate a dedicated codebook to each block through a per-block codebook selector. Further, MX quantizes the per-block array scale-factor to E8M0 format without per-tensor scaling. In contrast during LO-BCQ, we find that per-tensor scaling combined with quantization of per-block array scale-factor to E4M3 format results in superior inference accuracy across models. 



\end{document}
