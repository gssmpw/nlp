\section{Limitations}
% we do not cover all metrics, but show from different types of metrics
% we dont cover fluency as we consider it independt of the style transfer task
In evaluating style transfer, many different metrics are used for content preservation, which is why there is a need for standardization \cite{ostheimer-etal-2023-call}. We are limited in the number of metrics we test, but we do test the more widely used ones, as well as metrics of different types: lexical similarity, semantic similarity, fact-based, and LLMs conditioned on style shift.   

In our paper, we limit our scope to mainly content preservation and some aspects of style strength. Hence, we do not examine fluency; we leave this dimension about judging whether a text is natural and fluent, because it is not specific to the style transfer task. Evaluating whether a text is fluent does not, in our opinion, have to be conditioned on the source sentence or on the task.   

We posit, as others in the field have, that in evaluation of content preservation, the style shift must be taken into account. On the other hand, it is not necessarily wrong to consider that a more radical style shift also hurts content preservation. However, it should be weighted less than an actual shift in content unrelated to the style change.

\section*{Acknowledgements}
$\begin{array}{l}\includegraphics[width=1cm]{LOGO_ERC-FLAG_EU_.jpg} \end{array}$ This work was supported by the Danish Data Science Academy, which is funded by the Novo Nordisk Foundation (NNF21SA0069429) and VILLUM FONDEN (40516). 
It was further supported by the European Union (ERC, ExplainYourself, 101077481), and by the Pioneer Centre for AI, DNRF grant number P1.