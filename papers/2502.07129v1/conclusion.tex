% In this context, the proposed Fourier-enhanced Neural Networks for Systems Biology (SB-FNN) is an alternative approach inspired by PINN that can handle complex biological models more efficiently and accurately. SB-FNN has a customized design that incorporates three features for systems biology applications: an embedded Fourier neural network, an adaptive activation function, and a cyclic penalty function.

% The embedded Fourier neural network represents input data in the frequency domain using Fourier transforms. This enables the model to capture complex patterns and features that may not be easily detectable in the spatial domain. The adaptive activation function, which replaces the original GeLU, offers a promising avenue for improving the performance and interpretability of deep learning models for physical systems. By leveraging prior knowledge about the system and carefully selecting the appropriate set of activation functions, we can build models that are both more effective and adaptable to different systems biology models. The cyclic penalty function focuses on the oscillatory pattern of some systems biology models. It helps to incorporate the periodicity constraint into the loss function, further improving the accuracy of SB-FNN on models involving oscillatory patterns.

% One limitation of this study is the number of tested models. Some recent systems biology models involve dozens of differential equations, which require higher GPU computation resources. It presents a challenge and an opportunity for further research. We can also deal with the parameter estimation work in the future to estimate the best matching parameters for structure-specified models using given dynamic truth.

% In summary, SB-FNN offers a promising method for efficiently and accurately predicting the dynamics of complex systems biology models. It outperforms PINN on six biological models and incorporates three features that make it particularly well-suited to handling biological systems, especially for ubiquitous and vital oscillatory patterns like cellular systems. While further research is necessary to determine the full potential of SB-FNN, this method represents an exciting avenue for integrating machine learning and mathematical modeling to make new discoveries in biology and health.
In this context, the proposed Fourier-enhanced Neural Networks for systems biology (SB-FNN) is an alternative approach inspired by PINN that can handle complex biological models more efficiently and accurately. SB-FNN has an embedded Fourier neural network with a customized design that incorporates two features for systems biology applications: an adaptive activation function and a variance constraint.

The embedded Fourier neural network, which is a type of neural network that represents the input data in the frequency domain using Fourier transforms, enables the model to capture complex patterns and features that may not be easily detectable in the spatial domain. This capability is particularly valuable in systems biology, where many biological processes exhibit oscillatory patterns or other complex temporal dynamics. By representing the input data in the frequency domain, SB-FNN can effectively capture these dynamics, resulting in more accurate predictions.

The adaptive activation function for each Fourier layer, replacing the original GeLU, offers a promising avenue for improving the performance and interpretability of deep learning models for physical systems. By leveraging prior knowledge about the system and carefully selecting the appropriate set of activation functions, we can build models that are both more effective and adaptable to different systems biology models. Moreover, the use of adaptive activation functions makes it easier to interpret the model's behavior and provides a more intuitive understanding of the underlying biological processes.

The variance constraint focuses on the oscillatory pattern of some system biology models. It helps to incorporate the periodicity constraint into the loss function, further improving the accuracy of SB-FNN on the models involving oscillatory patterns. This is particularly important in systems biology, where many biological processes exhibit periodic behavior, such as the circadian rhythm or cell cycle.

One limitation of this study is that it does not perform well on predicting some stiff dynamics produced by differential equations, which are commonly seen in some Cell-cycle models. One potential solution is to apply some wavelets method which can represent PDE dynamics as a linear combination of wavelet basis functions. Except for applying new base level techniques, in the future we can also explore the inverse techniques based on SB-FNN to estimate the best mapping parameters for structure-specified models using given dynamic truth.

In summary, SB-FNN offers a promising method for efficiently and accurately predicting the dynamics of complex systems biology models. By incorporating the embedded Fourier neural network, adaptive activation functions, and variance constraint, SB-FNN is particularly well-suited to handling biological systems, especially those exhibiting ubiquitous and vital oscillatory patterns. While further research is necessary to determine the full potential of SB-FNN, this method represents an exciting avenue for integrating machine learning and mathematical modeling to make new discoveries in biology and health.