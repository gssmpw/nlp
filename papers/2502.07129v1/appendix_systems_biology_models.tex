\label{appendix:models}
Systems biology models are computational tools used to study complex biological systems. These models can help us understand the behavior of biological systems, predict their responses to various stimuli, and design interventions to alter their behavior. In this appendix, we will introduce some popular types of systems biology models.

% \section{Predator–Prey}
% \label{sup:pp}

% The predator-prey model describes the dynamics of two interacting species, predators and prey. The population change over time is governed by a pair of nonlinear differential equations:
% \begin{equation}
% \begin{cases}
% \dfrac{dU}{dt} = \alpha U - \beta UV \vspace{1ex} \\
% \dfrac{dV}{dt} = - \gamma V + \delta UV \vspace{1ex} 
% \end{cases}
% \label{eq:pp}
% \end{equation}
% In Eqs.~\ref{eq:pp}, $U$ and $V$ represent the population densities of prey and predators. Parameters $\alpha, \beta, \gamma, \delta$ are the rate constants for the birth of prey, killing because of predators, death of predators, and reproduction because of prey, respectively.

% \subsubsection{Lorenz Model}
% \label{sup:lorenz}
% The Lorenz system \cite{lorenz1962statistical} is a system of ordinary differential equations first studied by mathematician and meteorologist Edward Lorenz. The model is a system of three ordinary differential equations:

% \begin{equation}
% \begin{cases}
% \dfrac{dX}{dt} = \sigma\left(Y-X\right) \vspace{1ex} \\
% \dfrac{dY}{dt} = X\left(\rho-Z\right)-Y \vspace{1ex} \\
% \dfrac{dZ}{dt} = XY-\beta Z\vspace{1ex} 
% \end{cases}
% \label{eq:lorenz}
% \end{equation}

% In equation \ref{eq:lorenz}, $X$ is proportional to the rate of convection, $Y$ to the horizontal temperature variation, and $Z$ to the vertical temperature variation \cite{sparrow2012lorenz}. The constants $\sigma$, $\rho$, and $\beta$ are system parameters proportional to the Prandtl number, Rayleigh number, and certain physical dimensions of the layer itself.


\section{Repressilator Model: mRNA and Protein (Rep6)}
\label{appendix:rep6}
The repressilator model \cite{elowitz2000synthetic} describes a synthetic oscillatory system of transcriptional repressors. 
The transcription and translation reactions among three pairs of repressor mRNAs and proteins can be represented by the following kinetic equations: 
\begin{equation}
\begin{cases}
\dfrac{dM_{i}}{dt} = - M_{i}+\dfrac{\alpha}{1+P^{n}_{j}}+\alpha_{0}\quad\vspace{1ex} \\
\dfrac{dP_{i}}{dt} = - \beta\left(P_{i}-M_{i}\right)\quad\vspace{1ex} 
\end{cases}
\begin{pmatrix} i=lacI, tetR, cI \\ j=cI, lacI, tetR \end{pmatrix}
\label{eq:repressilator6}
\end{equation}
In Eqs.~\ref{eq:repressilator6}, $P_{i}$ denotes the repressor-protein concentrations, and $M_{i}$ represents the corresponding mRNA concentrations, where $i$ is $lacI$, $tetR$, or $cI$. If there are saturating amounts of repressor, the number of protein copies produced from a given promoter type is $\alpha_{0}$. Otherwise, this number would be $\alpha+\alpha_{0}$ per cell. $\beta$ represents the ratio of the protein decay rate to the mRNA decay rate, and $n$ is a Hill coefficient.

\section{Repressilator: Protein Only (Rep3)}
\label{appendix:rep3}
If we focus exclusively on the proteins, we can simplify the Repressilator model, which includes both mRNA and protein (Rep6), to a protein-only version (Rep3) as shown below:
\begin{equation}
% \begin{cases}
\dfrac{dP_{i}}{dt} = \dfrac{\beta}{1+P^{n}_{j}} - P_{i}\quad\vspace{1ex} 
% \end{cases}
\begin{pmatrix} i=lacI, tetR, cI \\ j=cI, lacI, tetR \end{pmatrix}
\label{eq:repressilator3}
\end{equation}
The variables utilized in Eqs.~\ref{eq:repressilator3} have the same connotations as those in Eqs.~\ref{eq:repressilator6}.


\section{SIR Model}
\label{appendix:sir}
The SIR \cite{anderson1991discussion} is a classic compartmental model in epidemiology to simulate the spread of infectious disease. The basic SIR model considers a closed population with three different labels susceptible (S), infectious (I), and recovered (R).
The evolution of the three interacting groups is predicted by the following equations:
\begin{equation}
\begin{cases}
\dfrac{dS}{dt} = -\dfrac{\beta IS}{N} \vspace{1ex} \\
\dfrac{dI}{dt} = \dfrac{\beta IS}{N} - \gamma I\vspace{1ex} \\
\dfrac{dR}{dt} = \gamma I \vspace{1ex}
\end{cases}
\label{eq:sir}
\end{equation}
where $S$, $I$, and $R$ represent the susceptible, infected, and remove (either by death or recovery) populations, respectively. $\beta$ refers to the contact rate between the susceptible and infected individuals, and $\gamma$ refers to the removal rate of the infected population. 


\section{Age-structured SIR Model (A-SIR)}
\label{appendix:asir}
Many infectious diseases, such as COVID-19, have dramatically different effects on individuals of different ages. 
According to this, an age-structured SIR model was proposed to take into consideration the age-group difference~\cite{ram2021modified}. 
In the age-structured SIR (ASIR) model (see below Eqs.~\ref{eq:sir-ages}), the functions $S_i(t)$, $I_i(t)$, and $R_i(t)$ represent the susceptible, infectious, and removed population at the $i$th age-bracket for $1\leq i \leq n$, where $n$ is the number of age-groups. $\mathcal{M}\in \mathbb{R}^{n\times n}$ is an age-contact matrix describing the rate of contact between each pair of age-groups, and the value used comes from~\cite{ram2021modified}.
\begin{equation}
\begin{cases}
\dfrac{dS_{i}}{dt} = - \dfrac{\beta S_{i}}{N}\cdot \sum ^{n}_{j=1}\mathcal{M}_{ij} I_{j}\vspace{1ex} \\
\dfrac{dI_{i}}{dt} =  \dfrac{\beta S_{i}}{N}\cdot \sum ^{n}_{j=1}\mathcal{M}_{ij} I_{j} - \gamma I_{i}\vspace{1ex} \\
\dfrac{dR_{i}}{dt} = \gamma I_{i} \vspace{1ex}
\end{cases}
\label{eq:sir-ages}
\end{equation}

% \begin{figure}[h]
% \centering
% \includegraphics[width=0.4\textwidth]{fig/SIR_Ages_matrix.png}
% \caption{An example for the age-contact matrix $\mathcal{M}$ in the age-structured SIR model.}
% \label{fig:sirages_matrix}
% \end{figure}




\section{Turing Model (1D and 2D)}
\label{appendix:turing}
Turing patterns, such as spots and stripes observed in nature, arise spontaneously and autonomously from initially uniform states and can be mathematically described by reaction-diffusion systems involving two interacting and diffusive substances~\cite{turing1990chemical}. These models can exhibit substantial complexity and are highly dynamic. In this study, we focus on a two-dimensional Turing model known as the Schnakenberg kinetics \cite{maini2012turing}:
% maini2012turing  Schnakenberg model https://core.ac.uk/download/pdf/8791221.pdf
\begin{equation}
\begin{cases}
\dfrac{\partial U}{\partial t}=c_{1}-c_{\text{-}1}U+c_{3}U^{2}V+d_{1}\nabla^{2} U \vspace{1ex}\\
\dfrac{\partial V}{\partial t}=c_{2}-c_{3}u^{2}v +d_{2}\nabla^{2} V \vspace{1ex}
\end{cases}
\end{equation}
where $U$ and $V$ are the concentrations of two diffusible substances, $c_{\text{-}1}, c_{1}, c_{3}$ represent the deterministic reaction rates, $d_{1}$, $d_{2}$ are diffusion rates, and the rest are reaction terms for the two substances.

The Turing model is commonly visualized in two dimensions (2D Turing), but it can be challenging to display 2D predictions for the entire time domain. To overcome this challenge, we limit the model to one dimension, so that at each time step within the time domain, a prediction can be made in one dimension. The predictions can then be arranged along the time domain to form a figure, which is named as the 1D Turing model.
