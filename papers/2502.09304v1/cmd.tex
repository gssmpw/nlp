\usepackage{xspace,balance,tabularx,multirow}
%\usepackage{soul} 
\usepackage{flushend}
\usepackage{tikz}
\usepackage{pgfplots}
\usetikzlibrary{patterns}
\pgfplotsset{compat=1.16}
\usepackage[ruled, vlined, linesnumbered]{algorithm2e}
\usepackage[noend]{algpseudocode}
\usepackage{xcolor}
\usepackage{float}
\usepackage{colortbl}
\usepackage{hyperref}
% \hypersetup{draft}
\usepackage{bbold}
\usepackage{textcomp}
\usepackage{pifont}
\SetKwComment{Comment}{$\triangleright$\ }{}
\usepackage{enumitem}
\usepackage{xcolor,framed}
\usepackage[aboveskip=-4pt]{subcaption}
\usepackage[normalem]{ulem}
\usepackage{tikz-3dplot}
\useunder{\uline}{\ul}{}

% comments
\newcommand{\yiqian}[1]{{\color{magenta}{#1}}}
\newcommand{\shiqi}[1]{{\color{blue}{[SQ: #1]}}}
\newcommand{\todo}[1]{{\color{blue}{[TODO: #1]}}}
\newcommand{\argmax}[1]{\underset{#1}{\operatorname{arg}\,\operatorname{max}}\;}
\newcommand{\argmin}[1]{\underset{#1}{\operatorname{arg}\,\operatorname{min}}\;}
\newcommand{\argmaxk}[1]{\underset{#1}{\operatorname{arg}\,\operatorname{max-k}}\;}
\newcommand{\argmink}[1]{\underset{#1}{\operatorname{arg}\,\operatorname{min-k}}\;}
\newcommand{\eat}[1]{} %comments out the arument. 

\makeatletter
\newcommand*\bigcdot{\mathpalette\bigcdot@{.5}}
\newcommand*\bigcdot@[2]{\mathbin{\vcenter{\hbox{\scalebox{#2}{$\m@th#1\bullet$}}}}}
\makeatother

\def\done{\hspace*{\fill} {$\square$}}
\def\header{\vspace{1mm} \noindent}

\SetKw{Break}{break}

% general
\newcommand{\ie}{{i.e.},\xspace}
\newcommand{\eg}{{e.g.},\xspace}
\newcommand{\etal}{{et al.},\xspace}
\newcommand{\stitle}[1]{\noindent{\bf #1.\/}}
\newcommand\lnorm[1]{\left\lVert#1\right\rVert}
\newtheorem{corollary}{Corollary}
\newtheorem{problem}{Problem}
\newtheorem{proposition}{proposition}
\newtheorem{observation}{Observation}
\newtheorem{property}{Property}

\newcommand{\D}{\mathcal{D}\xspace}
\newcommand{\G}{\mathcal{G}\xspace}
\newcommand{\W}{\mathcal{W}\xspace}
\newcommand{\C}{\mathcal{C}\xspace}
\newcommand{\N}{\mathcal{N}\xspace}
\newcommand{\Sset}{\mathcal{S}\xspace}

% \newcommand{\Lset}{\mathsf{L}\xspace}
% \newcommand{\A}{\mathcal{A}\xspace}
\newcommand{\Tset}{\mathcal{T}\xspace}
\newcommand{\temb}[1]{\boldsymbol{\upphi}(#1)}
\newcommand{\V}{\mathcal{V}\xspace}
\newcommand{\Vtext}{\mathcal{X}\xspace}
% \newcommand{\xemb}{\mathbf{x}}
\newcommand{\E}{\mathcal{E}\xspace}
\newcommand{\Etext}{\mathcal{R}\xspace}
% \newcommand{\remb}{\mathbf{r}}
% \newcommand{\I}{\mathcal{I}\xspace}
% \newcommand{\indicator}{\mathbb{I}}
% \newcommand{\C}{\mathcal{C}\xspace}
% \newcommand{\Sset}{\mathcal{S}\xspace}
% \newcommand{\Rset}{\mathcal{R}\xspace}

\newcommand{\M}{\mathsf{M}\xspace}
\newcommand{\pr}{\mathsf{Pr}\xspace}
\newcommand{\clen}{\ell\xspace}

\newcommand{\textgraph}{TAG}
\newcommand{\textrag}{\textsf{Text-RAG}\xspace}
\newcommand{\graphrag}{\textsf{Graph-RAG}\xspace}
\newcommand{\kgrag}{\textsf{KG-RAG}\xspace}
\newcommand{\hybridrag}{\textsf{Hybrid-RAG}\xspace}
\newcommand{\knnrag}{\textsf{KNNG-RAG}\xspace}
\newcommand{\keyrag}{\textsf{Keyword-RAG}\xspace}
\newcommand{\skeletonragu}{\textsf{Skeleton-RAG-U}\xspace}
\newcommand{\skeletonrag}{\textsf{Skeleton-RAG}\xspace}
\newcommand{\skeletonragp}{\textsf{Skeleton-RAG-P}\xspace}
\newcommand{\graphindex}{\textsf{KG-Index}\xspace}
\newcommand{\graphretrieval}{\textsf{KG-Retrieval}\xspace}
\newcommand{\sketrag}{\textsf{KET-RAG}\xspace}
\newcommand{\sketragu}{\textsf{KET-RAG-U}\xspace}
\newcommand{\sketragp}{\textsf{KET-RAG-P}\xspace}
\newcommand{\sketindex}{\texttt{KET-Index}\xspace}
\newcommand{\sketretrieval}{\texttt{KET-Retrieval}\xspace}

\newcommand*{\defeq}{\mathrel{\vcenter{\baselineskip0.5ex \lineskiplimit0pt
                     \hbox{\scriptsize.}\hbox{\scriptsize.}}}%
                     =}

\definecolor{Red}{HTML}{E81123}
\definecolor{Orange}{HTML}{FF8C00}
\definecolor{Green}{HTML}{009E49}
\definecolor{LightBlue}{HTML}{00BCF2}
\definecolor{DeepBlue}{HTML}{001BA3}
\definecolor{Pink}{HTML}{F2028F}

% argument #1: any options
\newenvironment{customlegend}[1][]{%
    \begingroup
    % inits/clears the lists (which might be populated from previous
    % axes):
    \csname pgfplots@init@cleared@structures\endcsname
    \pgfplotsset{#1}%
}{%
    % draws the legend:
    \csname pgfplots@createlegend\endcsname
    \endgroup
}%

% makes \addlegendimage available (typically only available within an
% axis environment):
\def\addlegendimage{\csname pgfplots@addlegendimage\endcsname}