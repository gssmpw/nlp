\section{Omitted proof in \cref{subsubsec: general cost manipulation}}\label{appendix:general costs}
% \paragraph{Manipulation}
\begin{proof}[Proof of \cref{prop:optimal simultaneous manipulation general cost}]
Consider any simultaneous mechanism that uses two tests $\tilde \classifier_A$ and $\tilde \classifier_B$.
Denote the intersecting point of the boundary lines of $\tilde \classifier_A$ and $\tilde \classifier_B$ by $\tilde O$.

Since program \ref{max qualified} requires feasible simultaneous mechanism not to accept unqualified agent, we must have the set of attributes that satisfy both $\tilde \classifier_A$ and $\tilde \classifier_B$ is contained in the set of attributes that satisfy both $ \classifier_A$ and $\classifier_B$, i.e., $\tilde \classifier_A \cap \tilde \classifier_B \subset \classifier_A\cap \classifier_B$.

Denote the angle between $ \tilde\classifier_A$ and $\classifier_B$ by $\theta( \tilde\classifier_A,\classifier_B)$.
Denote the angle between $ \classifier_A$ and $\tilde\classifier_B$ by $\theta( \classifier_A,\tilde\classifier_B)$.

\paragraph{Step 1:} Any feasible simultaneous mechanism $(\tilde \classifier_A,\tilde \classifier_B)$ must satisfy $\theta( \tilde\classifier_A,\classifier_B)\leq\theta$ and $\theta( \classifier_A,\tilde\classifier_B)\leq \theta$.

Suppose $\theta( \tilde\classifier_A,\classifier_B)>\theta$. 
This implies that the boundary lines of $\tilde \classifier_A$ and $ \classifier_A$ intersect at some point that is in $ \classifier_A\cap \classifier_B$.
This further implies that $\tilde \classifier_A\cap \tilde \classifier_B$ includes some attributes that are outside $ \classifier_A\cap \classifier_B$.
The simultaneous mechanism $(\tilde \classifier_A,\tilde \classifier_B)$ cannot be feasible. A contradiction.
Analogously, we can show that $\theta( \classifier_A,\tilde\classifier_B)>\theta$ is not possible.



\paragraph{Step 2:} Any simultaneous mechanism $(\tilde \classifier_A,\tilde \classifier_B)$ such that $\theta( \tilde\classifier_A,\classifier_B)<\theta$ or $\theta( \tilde\classifier_A,\classifier_B)<\theta$ cannot be optimal.

Suppose $\theta( \tilde\classifier_A,\classifier_B)<\theta$.
Let $\classifier_A^+$ be the half plane obtained by shifting $\classifier_A$ along $\weights_A$ such that $\tilde O$ falls on the boundary line of  $\classifier_A^+$.

\begin{claim}\label{claim 1}
    The simultaneous mechanism $(\classifier_A^+,\tilde \classifier_B)$ is feasible.
\end{claim}

%-----------------------------------
\begin{proof}[Proof of \cref{claim 1}]
    Suppose not. Then there exists some unqualified attributes $\features$ that are accepted by this simultaneous mechanism $(\classifier_A^+,\tilde \classifier_B)$.
    Then there must exist some attributes $\features'$ in the set of attributes that satisfy both tests, i.e., $\features'\in\classifier_A^+\cap\tilde \classifier_B$, such that $\onecost(\features,\features')\leq 1$.
    Moreover, such attributes $\features'$ are either on the boundary line of $\tilde\classifier_B$ or on the boundary line of $\classifier_A^+$.
    Suppose the attributes $\features'$ are not on the boundary line of  $\classifier_A^+$ ($\tilde\classifier_B$). 
    Then, the line segment between $\features$ and $\features'$ intersects the boundary line of $\classifier_A^+$ ($\tilde\classifier_B$) at some attributes $\hat \features$. 
    Since the cost function is absolute homogeneous, we have  
    $
        \onecost(\features,\features') 
        > \onecost(\features,\hat\features) 
    $.
    This implies that the agent will be better off choosing $\hat\features$, which gives a contradiction.
    
    Next, we analyze each case separately.

    \textbf{Case 1}:  $\features'$ are on the boundary line of $\tilde\classifier_B$. 
    This immediately implies that the unqualified attributes $\features$ are accepted in the simultaneous mechanism $(\tilde \classifier_A,\tilde \classifier_B)$.
    A contradiction.

\textbf{Case 2}:  $\features'$ are on the boundary line of $\classifier_A^+$.
Define the vector$ v$ as $v=\features'-\tilde O$.
Then $v$ is a vector parallel to the boundary line of $\classifier_A$ and $\classifier_A^+$ and it points towards $\tilde O$.
This implies that $\features+v$ are also unqualified.
However, since the cost is translation invariant, we can infer that $\onecost(\features+v,\tilde O)=\onecost(\features,\features')\leq 1$.
In other words, the unqualified attributes $\features+v$ are accepted in the simultaneous mechanism $(\tilde \classifier_A,\tilde \classifier_B)$.
This again contradicts that the simultaneous mechanism $(\tilde \classifier_A,\tilde \classifier_B)$ is feasible.
    
\end{proof}

%--------------------------------

By construction, $\tilde \classifier_A\cap\tilde \classifier_B\subset \classifier_A^+\cap\tilde \classifier_B$.
Hence the any attributes that can move to some attributes in $\tilde\classifier_A\cap\tilde \classifier_B$ with a cost no greater than one  can also move to some attributes in $\classifier_A^+\cap\tilde \classifier_B$ with a cost no greater than one.
Moreover, 
since the cost function is absolute homogeneous, the set of attributes that can move to $\classifier_A^+\cap\tilde \classifier_B$ with a cost no greater than one is strictly larger when $\tilde \classifier_A\cap\tilde \classifier_B$ is a strict subset of $\classifier_A^+\cap\tilde \classifier_B$.
This implies that the simultaneous mechanism $(\classifier_A^+,\tilde \classifier_B)$ is better than $(\tilde \classifier_A,\tilde \classifier_B)$.

Suppose $\theta( \classifier_A,\tilde\classifier_B)<\theta$.
Let $\classifier_B^+$ be the half plane obtained by shifting $\classifier_B$ along $\weights_B$ such that $\tilde O$ falls on the boundary line of  $\classifier_B^+$.
We can use analogous argument to show that the simultaneous mechanism $(\classifier_A^+, \classifier_B^+)$ is feasible and better than $( \classifier_A^+,\tilde \classifier_B)$.

\end{proof}


%-----------------------------------

\begin{proof}[Proof of \cref{lem:fix-simul}]
We consider the fixed order mechanism $(\classifier_A^{+},\classifier_B^{+},1)$ that first uses test $\classifier_A^{+}$ and then uses test $\classifier_B^{+}$. 
Another fixed order mechanism can be analyzed similarly.

\paragraph{Step 1:}
   We first show that these two fixed-order mechanisms are feasible for~\ref{max qualified}, which means they will not accept unqualified agents. 
    
    Suppose not. Then there exists an unqualified agent with attributes $\features$ that are accepted by this fixed-order mechanism. 
    Let $\firstfeatures, \secondfeatures$ be the attributes provided by this agent in the first test and the second test respectively.
    Then, the manipulation cost of this agent is $c(\features, \firstfeatures, \secondfeatures) \leq 1$.
    
    \textbf{Case 1:} Suppose the attributes $\features$ does not satisfy $\classifier_A$.  
    
    If the attributes $\firstfeatures$ also satisfy $ \classifier_B^+$, then this unqualified agent can be accepted by the simultaneous mechanism with tests $\classifier_A^{+},\classifier_B^{+}$. 
    A contradiction.
    
    If this attributes $\firstfeatures$ do not satisfy $\classifier_B^+$, then we can find a vector $\genericfeatures$ parallel to the boundary lines of $\classifier_A$ and $\classifier_A^+$ such that $\firstfeatures + \genericfeatures$ satisfy both $ \classifier_A^+$ and $\classifier_B^+$. 
    This implies that $\firstfeatures + \genericfeatures$ are accepted by the simultaneous mechanism $(\classifier_A^{+},\classifier_B^{+})$.
    Since the simultaneous mechanism $(\classifier_A^{+},\classifier_B^{+})$ is feasible, $\firstfeatures + \genericfeatures$ are qualified.
    Since $\features$ do not satisfy $\classifier_A$ and the vector $\genericfeatures$ is parallel to the boundary line of $\classifier_A$, we can infer that the shifted attributes $\features + \genericfeatures$ also do not satisfy $\classifier_A$, i.e.,  $\features + \genericfeatures$ are not qualified. 
    Since the cost function is translation invariant and is monotone, we have 
    $$
    \onecost(\features+\genericfeatures, \firstfeatures+\genericfeatures) = \onecost(\features, \firstfeatures) \leq c(\features, \firstfeatures, \secondfeatures) \leq 1.
    $$
    Thus, the unqualified agent with attributes $\features+\genericfeatures$ will be accepted by the simultaneous mechanism, which gives a contradiction. 

    \textbf{Case 2:} Suppose the attributes $\features$ does not satisfy $\classifier_B$. 

    If the attributes $\secondfeatures$ also satisfy $\classifier_A^+$, by triangle inequality, $\onecost(\features,\secondfeatures) \leq \cost(\features,\firstfeatures,\secondfeatures) \leq 1$.
    This implies that the attributes $\features$ can get accepted by the simultaneous mechanism with tests $\classifier_A^{+},\classifier_B^{+}$. 
    A contradiction.
    
    If this attributes $\secondfeatures$ do not satisfy $\classifier_A^+$, then we can find a vector $\genericfeatures$ parallel to the boundary lines of $\classifier_B$ and $\classifier_B^+$ such that $\secondfeatures + \genericfeatures$ satisfy both $\classifier_A^+$ and $ \classifier_B^+$. 
This implies that $\secondfeatures + \genericfeatures$ are accepted by the simultaneous mechanism $(\classifier_A^{+},\classifier_B^{+})$.
    Since the simultaneous mechanism $(\classifier_A^{+},\classifier_B^{+})$ is feasible, $\secondfeatures + \genericfeatures$ are qualified.
    Since $\features$ do not satisfy $\classifier_B$ and the vector $\genericfeatures$ is parallel to the boundary line of $\classifier_B$, we can infer that the shifted attributes $\features + \genericfeatures$ also do not satisfy $\classifier_B$, i.e.,  $\features + \genericfeatures$ are not qualified. 
    Since the cost function is translation invariant and satisfies triangle inequality, we have 
    $$
    \onecost(\features+\genericfeatures, \secondfeatures+\genericfeatures) = \onecost(\features, \secondfeatures) \leq c(\features, \firstfeatures, \secondfeatures) \leq 1.
    $$
    Thus, the unqualified agent with attributes $\features+\genericfeatures$ will be accepted by the simultaneous mechanism, which gives a contradiction. 

    


\paragraph{Step 2:}
    These two fixed-order mechanisms are not worse than the optimal simultaneous mechanism. 
    This is because for any agent accepted by the simultaneous mechanism, the agent can provide the attributes in $\classifier_A^{+}\cap\classifier_B^{+}$ to get accepted in the fixed-order mechanisms.
    That is, any qualified agent accepted by the simultaneous mechanism is also accepted by these two fixed-order mechanisms.
\end{proof}

%-----------------------------------

\begin{proof}[Proof of \cref{lem:stringent tests}]
We prove the contra-positive of the statement.
    Consider any mechanism that uses tests $\tilde\classifier_A$ and $\tilde\classifier_B$ such that $\tilde\classifier_A\cup\tilde\classifier_B\subsetneq \classifier_A\cup\classifier_B$ is not satisfied, which has three possibilities:
    \begin{enumerate}
        \item the set $\tilde \classifier_A\cap \tilde\classifier_B$ covers the qualified region $\classifier_A\cap\classifier_B$, i.e., $\classifier_A\cap\classifier_B \subset\tilde \classifier_A\cap \tilde\classifier_B  $ (\textbf{case 1}); 
        \item the set $\tilde \classifier_A\cap \tilde\classifier_B$ and the qualified region $\classifier_A\cap\classifier_B$ have no inclusion relation, i.e.,$\classifier_A\cap\classifier_B \not\subset\tilde \classifier_A\cap \tilde\classifier_B  $ and  $\tilde \classifier_A\cap \tilde\classifier_B  \not\subset \classifier_A\cap\classifier_B $(\textbf{case 2}).
        \item the set $\tilde \classifier_A\cap \tilde\classifier_B$ and the qualified region $\classifier_A\cap\classifier_B$ coincide, i.e., $\tilde \classifier_A\cap \tilde\classifier_B = \classifier_A\cap\classifier_B$  (\textbf{case 3});   
    \end{enumerate} 
      
    
    We want to show that in any of the three cases, some unqualified attributes are selected and hence none of these mechanisms is feasible. 
    

    
    Case 1 implies that the intersection of the set
    $\tilde \classifier_A\cap \tilde\classifier_B$ and the set $ \classifier_A^\compl $ is not empty.
     Notice that any attributes in $\tilde \classifier_A\cap \tilde\classifier_B$ are selected under any mechanism that uses these two tests.
    Hence there exists some unqualified attributes in the intersection, i.e.,$\features \in \tilde \classifier_A\cap \tilde\classifier_B  \cap\classifier_A^\compl$, such that $\features$ are selected automatically under the mechanism that uses tests $\tilde \classifier_A, \tilde\classifier_B$.
    This shows that any mechanism that uses tests in case 1 is not feasible.

    Case 2 implies that the intersection of the set
    $\tilde \classifier_A\cap \tilde\classifier_B$ and the set $ \classifier_i^\compl $ is not empty for at least one $i\in \{A,B\}$.
    Hence there exists some unqualified attributes in the intersection, i.e.,$\features \in \tilde \classifier_A\cap \tilde\classifier_B  \cap \classifier_i^\compl$, such that $\features$ are selected automatically under the mechanism that uses tests $\tilde \classifier_A, \tilde\classifier_B$.
    This shows that any mechanism that uses tests in case 1 is not feasible.


    % \textbf{Step 1: unqualified agent is selected if test $\classifier_i$ is used.}
    Lastly, we want to show case 3 is not feasible. 
    
    By \cref{def: existence of minimum cost}, 
    there exists some attributes that do not satisfy $\classifier_A$, i.e., $\firstfeatures\in \classifier_A^\compl$ such that the minimum cost from changing such attributes to another that passes $\classifier_A$ is less than one, i.e.,  $\min_{\genericfeatures\in \classifier_A}\cost(\firstfeatures,\genericfeatures)\leq 1$.
     This means that there exists some attributes $\genericfeatures^1$ in $\classifier_A$  such that the minimum cost from changing $\firstfeatures$ to  pass $\classifier_A$ is achieved, i.e., $\cost(\firstfeatures,\genericfeatures^1)\leq 1$. 
     
     Suppose $\genericfeatures^1$ is also in $\classifier_B$, then we have found unqualified attributes $\firstfeatures$ that could move to  $\tilde\classifier_A\cap \tilde\classifier_B$ with cost no greater than one.
     That is, such attributes have at least one profitable strategy to get selected.
     Such a mechanism is not feasible. 

Suppose $\genericfeatures^1$ is not in $\classifier_B$.
Then we can find some non-zero vector $\genericfeatures^\perp$ such that (1) $\genericfeatures^\perp\cdot \weights_A=0$, i.e., $\genericfeatures^\perp$ is parallel to line $\Line_A: \weights_A\cdot \features =0$, and (2) the shifted attributes $\genericfeatures^1+\genericfeatures^\perp$ also satisfy $\classifier_B$.
    Since the cost function is translation invariant, we must have $\cost(\firstfeatures+\genericfeatures^\perp,\genericfeatures^1+\genericfeatures^\perp)\leq 1$.
    Hence we have found unqualified attributes $\firstfeatures+\genericfeatures^\perp$ that could move to  $\tilde\classifier_A\cap \tilde\classifier_B$ with cost no greater than one.
     That is, such attributes have at least one profitable strategy to get selected.
     Such a mechanism is not feasible.
     
     We conclude that any mechanism that uses tests in case 3 is not feasible. 
    
\end{proof}



Now we prove the main theorem.
\begin{proof}[Proof of \cref{thm:opt_manipulation}]
   

    \cref{prop:optimal simultaneous manipulation general cost} and  \cref{lem:fix-simul} together imply that there exists a feasible fixed-order mechanism that is no worse than the optimal simultaneous mechanism. Hence the optimal mechanism must be a sequential one.
    \cref{lem:stringent tests} shows that the optimal mechanism must use stringent tests.
\end{proof}
%--------------------------------------

