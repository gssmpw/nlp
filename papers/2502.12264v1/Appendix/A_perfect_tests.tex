\section{Omitted proof in \cref{sec: perfect tests}}\label{appendix: perfect tests}


\begin{proof}[Proof of \cref{thm: sequential perfect tests}]
     Consider the following fixed-order sequential mechanism with two tests.
     For any message $m\in \Featurespace$ sent by the agent, let $\firstfeatures(m)$ be the  first test and $\secondfeatures(m)$ be the second test.
      
     If $m \in \classifier_A^+\cap \classifier_B^+$, then $\firstfeatures(m)=\secondfeatures(m)=m$.
     If $m\not\in \classifier_A \cap \classifier_B$, then $\firstfeatures(m)=\secondfeatures(m)=\features$ for some $\features \in \classifier_A^+\cap \classifier_B^+$.
     Call  $\classifier_A^+\cap \classifier_B^+$ the immediate accepted region.
    If $m\in (\classifier_A \cap \classifier_B) \setminus (\classifier_A^+\cap \classifier_B^+)$, then $\firstfeatures(m)=m$. 
    
    % \emph{First test.} 
    
    
    % Any agent with  $\firstfeatures \in \classifier_A^+\cap \classifier_B^+$ is accepted immediately.
    %  Any agent with $\firstfeatures \in (\classifier_A \cap \classifier_B) \setminus (\classifier_A^+\cap \classifier_B^+)$ will be accepted if they participate the second test and pass it. 
    %  Any agent with $\firstfeatures \notin \classifier_A \cap \classifier_B$ is rejected immediately.
    
    % \emph{Second test.} 
    Let  $\onecost(\firstfeatures, \classifier_A^+\cap \classifier_B^+) = \min_{\genericfeatures \in \classifier_A^+\cap \classifier_B^+} \eta \cdot \metric(\firstfeatures,\genericfeatures) $ denote the lowest cost from the first test $\firstfeatures$ to the immediate accepted region.
    Then, for any $m\in (\classifier_A \cap \classifier_B) \setminus (\classifier_A^+\cap \classifier_B^+)$, 
     (1) if $\onecost(m, \classifier_A^+\cap \classifier_B^+) \leq 1$, the second test
     $\secondfeatures(m)= \argmin_{\genericfeatures \in \classifier_A^+\cap \classifier_B^+} \eta \cdot \metric(\firstfeatures,\genericfeatures) $; (2) if $\onecost(m, \classifier_A^+\cap \classifier_B^+) > 1$,  the second test is some $\secondfeatures(m)\in \classifier_A \cap \classifier_B$ such that the  cost for the second test $\onecost(\firstfeatures(m),\secondfeatures(m))$ is one.  


    \textbf{Step 1: feasibility.}  Consider any agent with attribute $\features \not \in \classifier_A \cap \classifier_B$.
    Then the minimum cost for such $\features$ to any type in $\classifier_A^+\cap \classifier_B^+$ is at least $1$.
    This implies that it is not profitable for such $\features$ to take any first test in $\classifier_A^+\cap \classifier_B^+$.
    Suppose this agent reports some type $m \in (\classifier_A \cap \classifier_B) \setminus (\classifier_A^+\cap \classifier_B^+)$.  If $\onecost(m, \classifier_A^+\cap \classifier_B^+) > 1$, then this agent has to pay a cost one in the second test to get accepted.  Since the cost for the first test is non-negative, the total cost exceeds one.
    If $\onecost(m, \classifier_A^+\cap \classifier_B^+) \leq 1$, then to get accepted, this agent must produce $\secondfeatures \in \classifier_A^+\cap \classifier_B^+$. By the triangle inequality, the total cost for this agent to get accepted is at least $\onecost(\features,\firstfeatures) + \onecost(\firstfeatures,\secondfeatures) \geq \onecost(\features, \secondfeatures) > 1$. Therefore, any unqualified agent has no profitable strategy to get accepted. 

    \textbf{Step 2: first best.} For any qualified agent, to get selected, he  truthfully reports his type.
    Any agent with  $\firstfeatures \in \classifier_A^+\cap \classifier_B^+$ gets accepted with zero cost.
    We only need to show that for  any agent with attributes $\features \in (\classifier_A \cap \classifier_B) \setminus (\classifier_A^+\cap \classifier_B^+)$, the cost to get accepted is no greater than one.
    This is true by the construction of the mechanism.
\end{proof}

% \begin{proof}[Proof of \cref{thm: sequential perfect tests}]
%     We first introduce some useful notations. For two classifiers $\classifier_A, \classifier_B$, let $\tilde{h}_1$ and $\tilde{h}_2$ be the classifiers by shifting $\classifier_A$ and $\classifier_B$ towards their normal directions $w_1$ and $w_2$ with distance $1/\eta$ respectively. For all attributes that do not satisfy both $\tilde{h}_1,\tilde{h}_2$, we divide them into different sets according to their minimum distance to $\tilde\classifier_A \cap \tilde\classifier_B$. For any $r > 0$, we use $R(\tilde\classifier_A, \tilde\classifier_B, r) = \{\features : d(\features, \tilde\classifier_A \cap \tilde\classifier_B)) = r\}$ to denote all attributes with minimum distance $r$ to the region $\tilde\classifier_A \cap \tilde\classifier_B$.

%     \textbf{Constructing the first best sequential mechanism.} The proof is constructive. 
%     Let's consider the following sequential mechanism with two tests. 
%     Let $\firstfeatures$ be the attributes revealed from the first test. 
%     Let $\secondfeatures$ be the revealed attributes from the second test.
    
%     \emph{First test.} Any agent with $\firstfeatures \in \classifier_A \cap \classifier_B$ passes the first test and is allowed to participate the second test. 
%     In other words, the mechanism rejects those agent with $\firstfeatures \notin \classifier_A \cap \classifier_B$ immediately.
%     The mechanism accepts immediately any agent with revealed attributes $\firstfeatures \in \tilde\classifier_A \cap \tilde\classifier_B$ in the first test.
%      For those with revealed attributes $\firstfeatures \in (\classifier_A \cap \classifier_B) \setminus (\tilde\classifier_A \cap \tilde\classifier_B)$ in the first test, the mechanism would accept them only if they participate the second test and pass it. 
    
%     \emph{Second test.} 
%     Let  $r_1 = \inf_{\genericfeatures \in \tilde\classifier_A \cap \tilde\classifier_B} \metric(\firstfeatures,\genericfeatures)   $ be the distance between the first revealed attributes $\firstfeatures$ and the immediate accepted region $\tilde\classifier_A \cap \tilde\classifier_B$.
%     % $\in R(\tilde\classifier_A,\tilde\classifier_B, r_1)$ for some . 
%     % Suppose the first revealed attributes $\firstfeatures \in R(\tilde\classifier_A,\tilde\classifier_B, r_1)$ for some $r_1 > 0$. 
%     Then, any agent with revealed attributes $\firstfeatures \in (\classifier_A \cap \classifier_B) \setminus (\tilde\classifier_A \cap \tilde\classifier_B)$ in the first test is accepted if and only if:
%     either (1) the second revealed attributes $\secondfeatures \in R(\tilde\classifier_A, \tilde\classifier_B,r_1 - 1/\eta)$ for $r_1 > 1/\eta$, i.e., the second test exhausts the benefit of getting accepted when the agent's first revealed test is far away from the immediate accepted region $\tilde\classifier_A \cap \tilde\classifier_B$, or (2) the second revealed attributes
%      $\secondfeatures \in \tilde\classifier_A \cap \tilde\classifier_B$ for $r_1 \leq 1/\eta$, i.e., the second test requires the agent to move to the immediate accepted region $\tilde\classifier_A \cap \tilde\classifier_B$ if the total cost does not exceed the benefit of getting accepted.
    
    
%     % For any $x \in R(\tilde{h}_1,\tilde{h}_2)$, let the requirement $f(x) = x$ and the acceptance probability $p(x) = 1$. We now consider the region $R(\tilde{h}_1,\tilde{h}_2, r)$ for any $r \in (0,1/\eta]$. For any $x \in R(\tilde{h}_1,\tilde{h}_2, r)$ with $r \in (0,1/\eta]$, we set the requirement $f(x) = x'$ where $x'$ is the closest point to $x$ in $R(\tilde{h}_1,\tilde{h}_2)$ and $p(x) = 1$. Then, we consider any attribute $x \in R(\tilde{h}_1,\tilde{h}_2, r) \cap R(\classifier_A,\classifier_B)$ for $r >1/eta$. We set the requirement $f(x) = x'$  where $x'$ is the closest point to $x$ in $R(\tilde{h}_1,\tilde{h}_2, r')$ with $r' = r-1/\eta$ and the acceptance probability $p(x) = 1$. Note that the distance between $x$ and $x'$ is $d(x,x') = 1/\eta$. For any $x \not \in R(\classifier_A,\classifier_B)$, we set $f(x) = x$ and $p(x) = 0$. 


%     \textbf{Step 1: We first show that this mechanism does not accept any unqualified agent.}  Consider any agent with attribute $\features \not \in \classifier_A \cap \classifier_B$.
%     To pass the first test, this agent has to move to the attributes $\firstfeatures \in \classifier_A\cap \classifier_B$. 
%     Since $\features \not \in \classifier_A \cap \classifier_B$, the minimum distance from $\features$ to $\tilde\classifier_A \cap \tilde\classifier_B$ is at least $1/\eta$, i.e., $\inf_{\genericfeatures \in \tilde\classifier_A \cap \tilde\classifier_B} \onecost(\features,\genericfeatures)  > 1$. 
%     Thus, this agent has no incentive to produce $\firstfeatures \in \tilde\classifier_A \cap \tilde\classifier_B$, which means this agent can not get accepted through only the first test.
%     Suppose the agent moves to a attribute $\firstfeatures \in (\classifier_A \cap \classifier_B) \setminus (\tilde\classifier_A \cap \tilde\classifier_B)$. Then, we know the first observed attributes $\firstfeatures \in R(\tilde\classifier_A,\tilde\classifier_B, r_1)$ for $r_1= \inf_{\genericfeatures \in \tilde\classifier_A \cap \tilde\classifier_B} \metric(\firstfeatures,\genericfeatures)  > 0$. If $r_1 > 1/\eta$, then this agent has to move to $\secondfeatures \in R(\tilde\classifier_A,\tilde\classifier_B,r_1 - 1/\eta)$ to get accepted. Thus, the cost $\onecost(\firstfeatures,\secondfeatures) = \eta \cdot d(\firstfeatures,\secondfeatures) \geq 1$. This requires a total cost $\onecost(\features,\firstfeatures) + \onecost(\firstfeatures,\secondfeatures) > 1$, which is larger than the reward. If $r_1 \leq 1/\eta$, then to get accepted, this agent must produce $\secondfeatures \in \tilde\classifier_A \cap \tilde\classifier_B$. By the triangle inequality, the total cost for this agent to get accepted is at least $\onecost(\features,\firstfeatures) + \onecost(\firstfeatures,\secondfeatures) \geq \onecost(\features, \secondfeatures) > 1$. Therefore, this agent has no incentive to manipulate the attributes to get accepted. 

%     \textbf{Step 2: We now show that this mechanism accepts all qualified agents.} All agents with attributes in $\tilde\classifier_A \cap \tilde\classifier_B$ get accepted by providing their original attributes $\features$ in the first test with no cost. Consider any agent with attributes $\features \in (\classifier_A \cap \classifier_B) \setminus (\tilde \classifier_A \cap \tilde \classifier_B)$. Suppose $\features \in R(\tilde\classifier_A,\tilde\classifier_B, r_1)$. Let $\features'$ be the closest point in $\tilde \classifier_A \cap \tilde\classifier_B$ to this point $\features$. If $r_1 \leq 1/\eta$, then this agent can get accepted by providing $\firstfeatures = \features$ and $\secondfeatures = \features'$ in two tests sequentially. The total cost is $\onecost(\features, \features') \leq 1$. If $r_1 > 1/\eta$, then there exists a point $\features'' \in R(\tilde\classifier_A, \tilde\classifier_B, r_1 - 1/\eta)$ which lies on the line connecting $\features$ and $\features'$ due to the convexity of the region $\classifier_A \cap \classifier_B$. Thus, this agent will be accepted by producing $\firstfeatures = \features$ and $\secondfeatures = \features''$ with cost $1$. Therefore, under their best response, all qualified agents will be accepted. 
%     % we know that $x$ is in $R(\tilde{h}_1,\tilde{h}_2,r)$ for some $r > 1/\eta$. We consider the following three cases for the agent's report: (1) $m(x) \in R(\tilde{h}_1,\tilde{h}_2)$; (2) $m(x) \in R(\tilde{h}_1,\tilde{h}_2, r')$ for $r' \leq r$; (3) $m(x) \in R(\tilde{h}_1,\tilde{h}_2,r')$ for $r' > r$. In the first case, the cost for the agent is $c(x, f(m(x))) = \eta \cdot d(x, m(x)) > 1$ since $r > 1/\eta$. In the second case, we know that $f(m(x))$ is in either $R(\tilde{h}_1,\tilde{h}_2)$ or $R(\tilde{h}_1,\tilde{h}_2,r'')$ for $r'' = r' - 1/\eta$. If $r' < r$, we have the cost for the agent is greater than $1$. Note that for any point $x \in R(\tilde{h}_1,\tilde{h}_2,r')$, the closest point to $x$ in $R(\tilde{h}_1,\tilde{h}_2,r' - 1/\eta)$ is unique. If $r'' = r'$, we also have the cost $c(x,f(m(x))) > 1$. In the third case, 
% \end{proof}

\begin{proof}[Proof of \cref{thm: perfect tests arbitrary qualified region}]
     Let $\qualregion$ be the convex qualified region. We use $\partial \qualregion$ to denote the boundary of this region $\qualregion$. 
    % Let $\tilde \qualregion = \{\features \in \qualregion : \metric(\features, \partial \qualregion) \geq 1/\mc\}$ be the qualified attributes (points inside $\qualregion$) whose distance to the boundary $\partial \qualregion$ is at least $1/\mc$.
    Let $\qualified = \{\features \in \qualregion : \metric(\features, \partial \qualregion) \geq 1/\mc\}$ be the qualified attributes (points inside $\qualregion$) whose distance to the boundary $\partial \qualregion$ is at least $1/\mc$.
    For any $r > 0$, we use $\tilde \qualified(r) = \{\features: \metric(\features, \qualified) = r\}$ to be the set of all attributes with minimum distance $r$ to the region $\qualified$.

 We consider the following sequential mechanism with two tests. 
    
    \emph{First test.}
    Let $\firstfeatures$ be the attribute observed from the first test. 
    An agent with attribute $\firstfeatures$ passes the first test if and only if this attribute is contained in the qualified region $\firstfeatures 
    \in \qualregion$.
    The mechanism immediately rejects every agent with $\firstfeatures \not \in \qualregion$.
    Among the agents who passed the first test, the mechanism immediately accepts those agents with attribute $\firstfeatures \in \qualified$. Every agent with attribute $\firstfeatures \in \qualregion \setminus \qualified$ passes the first test and has to participate in the second test. 

    \emph{Second test.} 
    Let  $r_1 = \inf_{\tilde \qualregion} \metric(\firstfeatures,\genericfeatures)$ be the distance between the first revealed attributes $\firstfeatures$ and the immediately accepted region $\qualified$.
    % $\in R(\tilde\classifier_A,\tilde\classifier_B, r_1)$ for some . 
    % Suppose the first revealed attributes $\firstfeatures \in R(\tilde\classifier_A,\tilde\classifier_B, r_1)$ for some $r_1 > 0$. 
    Then, any agent with revealed attributes $\firstfeatures \in \qualregion \setminus \qualified$ in the first test is accepted if and only if:
    either (1) the second revealed attributes $\secondfeatures \in \tilde \qualified(r_1 - 1/\eta)$ for $r_1 > 1/\eta$, i.e., the second test exhausts the benefit of getting accepted when the agent's first revealed test is far away from the immediately accepted region $\qualified$, or (2) the second revealed attributes
    $\secondfeatures \in \qualified$ for $r_1 \leq 1/\eta$, i.e., the second test requires the agent to move to the immediately accepted region $\qualified$ when the agent's first revealed test is close to the immediately accepted region $\qualified$.

    \textbf{Step 1: We first show that this mechanism does not accept any unqualified agent.}  Consider any agent with attribute $\features \not \in \qualregion$.
    To pass the first test, this agent has to move to the attributes $\firstfeatures \in \qualregion$. 
    Since $\features \not \in \qualregion$, the minimum distance from $\features$ to the immediately qualified region $\qualified$ is at least $1/\eta$, i.e., $\inf_{\genericfeatures \in \qualified} \onecost(\features,\genericfeatures)  > 1$. 
    Thus, this agent has no incentive to produce $\firstfeatures \in \qualified$, which means this agent can not get accepted through only the first test.
    Suppose the agent moves to a attribute $\firstfeatures \in \qualregion \setminus \qualified$. Then, we know the first observed attributes $\firstfeatures \in \tilde \qualified(r_1)$ for $r_1= \inf_{\genericfeatures \in \qualified} \metric(\firstfeatures,\genericfeatures)  > 0$. If $r_1 > 1/\eta$, then this agent has to move to $\secondfeatures \in \tilde \qualified (r_1 - 1/\eta)$ to get accepted. Thus, the cost $\onecost(\firstfeatures,\secondfeatures) = \eta \cdot \metric(\firstfeatures,\secondfeatures) \geq 1$. This requires a total cost $\onecost(\features,\firstfeatures) + \onecost(\firstfeatures,\secondfeatures) > 1$, which is larger than the reward. If $r_1 \leq 1/\eta$, then to get accepted, this agent must produce $\secondfeatures \in \qualified$. By the triangle inequality, the total cost for this agent to get accepted is at least $\onecost(\features,\firstfeatures) + \onecost(\firstfeatures,\secondfeatures) \geq \onecost(\features, \secondfeatures) > 1$. Therefore, this agent has no incentive to manipulate the attributes to get accepted. 

    \textbf{Step 2: We now show that this mechanism accepts all qualified agents.} Every agent with attributes in $\qualified$ gets accepted by providing their original attributes $\features$ in the first test with no cost. Consider any agent with attributes $\features \in \qualregion \setminus \qualified$. Suppose $\features \in \tilde \qualified(r_1)$. Let $\features'$ be the closest point in $\qualified$ to this point $\features$. If $r_1 \leq 1/\eta$, then this agent can get accepted by providing $\firstfeatures = \features$ and $\secondfeatures = \features'$ in two tests sequentially. The total cost is $\onecost(\features, \features') \leq 1$. If $r_1 > 1/\eta$, then there exists a point $\features'' \in \tilde \qualregion(r_1 - 1/\eta)$ which lies on the line connecting $\features$ and $\features'$ due to the convexity of the region $\qualregion$. Thus, this agent will be accepted by producing $\firstfeatures = \features$ and $\secondfeatures = \features''$ with cost $1$. Therefore, under their best response, all qualified agents will be accepted. 

\end{proof}

\begin{proof}[Proof of \cref{thm: single-test mechanism perfect test}]
    
    % We now consider the special case where the qualified region $\qualregion$ is the convex hull of the union of balls with radius $1/\eta$. 
    
    \textbf{Part 1: sufficiency.}
    Suppose the qualified region $\qualregion$ is the convex hull of the union of balls with radius $1/\eta$. Consider a single-test mechanism that accepts an agent if and only if the first revealed attributes is in the immediately accepted region $\firstfeatures \in \qualified$.
    We show such a single-round test mechanism accepts all qualified agents and no unqualified agents.  
    
    \textbf{Step 1: this mechanism accepts every qualified agent.} 
    To do this, we only need to show that the distance between every attribute $\features$ in $\qualregion$ and the accepted region $\qualified$ is at most $1/\eta$, $\inf_{\genericfeatures \in \qualified} \metric(\features,\genericfeatures) \leq 1/\eta$. 
    Since every attribute in $\qualregion$ has a distance at most $1/\eta$ to $\qualified$, any qualified agent prefers to move to $\qualified$ and get accepted with cost at most $1$.  
    
    Let $\{\ballcenter_i\}$ be the centers of radius $1/\eta$ balls in the convex region $\qualregion$. 
    By assumption, we know that the qualified region $\qualregion = \conv(\bigcup_{i} B(\ballcenter_i, 1/\eta))$.
    We know that the distance between the center $\ballcenter_i$ and the boundary of $\qualregion$ is at least $d(\ballcenter_i,\partial \qualregion) \geq 1/\eta$. Thus, all ball centers $\ballcenter_i$ are contained in the accepted region $\qualified$. 
    
    Let $\conv(\{\ballcenter_i\})$ be the convex hull of these centers $\{\ballcenter_i\}$.

    We want to show that the convex hull $\conv(\{\ballcenter_i\})$ is contained in the accepted region $\qualified$.
    Consider any attribute $\features \in \conv(\{\ballcenter_i\})$.
    This attribute $\features$ can be written as a linear combination of two centers $\features = \lambda \ballcenter_i + (1-\lambda)\ballcenter_j$. Then, any attribute $\features' = \features + \boldsymbol{y}$ in the ball centered at $\features$ with radius $1/\eta$ can be written as the linear combination $\features' = \lambda(\ballcenter_i + \boldsymbol{y}) + (1-\lambda) (\ballcenter_j + \boldsymbol{y})$, where $\ballcenter_i + \boldsymbol{y}$ is in the ball with radius $1/\eta$ centered at $\ballcenter_i$. 
    We conclude that this attribute $\features'$ is contained in $\qualregion$, because the region $\qualregion$ is the convex hull of these balls centered at $\ballcenter_i$. Thus, the ball centered at $\features$ with radius $1/\eta$ is contained in the region $\qualregion$. 
    
    Then, this attribute $\features$ is also contained in the (immediately) accepted region $\qualified$. Hence, the convex hull $\conv(\{\ballcenter_i\})$ is contained in the accepted region $\qualified$. 
    
    Consider the union of balls centered at any attribute $\features \in \conv(\{\ballcenter_i\})$ with radius $1/\eta$, $\Sigma = \bigcup_{\features \in \conv(\{\ballcenter_i\})} B(\features, 1/\eta)$. This set $\Sigma$ is convex. The region $\qualregion$ is contained in the convex hull of this set $\Sigma$. Thus, we know that the region $\qualregion$ is contained in $\Sigma$, which means any attribute $\features \in \qualregion$ has a distance at most $1/\eta$ to the set $\conv(\{\ballcenter_i\})$. Since $\conv(\{\ballcenter_i\})$ is contained in $\qualified$, any attribute in $\qualregion$ has a distance at most $1/\eta$ to $\qualified$.
    
    \textbf{Step 2: we now show that this mechanism rejects every unqualified agent.} Note that $\qualified = \{\features \in \qualregion : \metric(\features, \partial \qualregion) \geq 1/\eta\}$. Every unqualified agent with attributes $\features \not \in \qualregion$ has a  minimum distance to $\qualified$ that is greater than $1/\eta$. This implies that the cost any unqualified agent to get accepted exceed the benefit, i.e., any unqualified agent has no incentive to move. 

    \paragraph{Part 2: necessity.} Suppose the qualified region $\qualregion$ is not the convex hull of a union of balls with radius $1/\eta$. We show that there does not exist a single-round test mechanism that accepts all qualified agents but no unqualified agents. 

    We prove this by contradiction. Suppose there is a single-round test mechanism that accepts all qualified agents but no unqualified agents. 
    First, this mechanism must make deterministic decisions based on observed attributes, i.e., it either accepts an agent with probability one or does not accept an agent with probability one. Suppose not, and this mechanism accepts an agent with a probability $p \in (0,1)$, then it will reject qualified agents or accept unqualified agents. 

    % Let $\calA$ be the set of observed attributes that will be accepted by this mechanism.\xqcomment{what is the purpose of defining this set $\calA$?}
    Since the qualified region $\qualregion$ is not the convex hull of the union of balls with radius $1/\eta$, we know there exist at least one attributes $\features \in \qualregion\setminus \qualified$ with distance to $\qualified$ greater than $1/\eta$.
    This is true because otherwise $\qualregion$ can be written as the union of balls with radius $1/\eta$ centered at attributes in $\qualified$, which is a contradiction. 
    
    Consider one such attribute $\features \in \qualregion \setminus \qualified$ with distance to $\qualified$ greater than $1/\eta$. 
    By assumption, the mechanism accepts qualified agent with this attribute $\features$.
    For this to happen, this mechanism must accept some attributes $\genericfeatures$ whose  distance to $\features$ is at most $1/\eta$, so that it is profitable for any agent with attribute $\features$ to move to $\genericfeatures$. 
    However, the distance from $x$ to $\qualified$ is greater than $1/\eta$ implies that $\genericfeatures$ is not in $\qualified$.
    This further implies that the ball with radius $1/\eta$ centered at $\genericfeatures$ also contains unqualified attributes. Otherwise, $\genericfeatures$ should be contained in $\qualified$.
    Thus, this mechanism will accept unqualified agents, which is a contradiction.   

    
    % We consider the following two cases. Suppose this accepted region $\calA \subseteq \tilde \qualregion$. Then, this mechanism can only accept agents with attributes in $\tilde \calA = \{\features : d(\features, \calA) \leq 1/\eta\}$.  
    % However, we know that there must exist attributes in $\qualregion \setminus \tilde \calA$, otherwise $\qualregion$ can be written as the convex hull of the union of balls with radius $1/\eta$ with ball centers in $\tilde \calA$. Suppose this accepted region $\calA \not\subseteq \qualregion$.
\end{proof}
