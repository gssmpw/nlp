\section{Omitted proof in \cref{sec: distance cost} (manipulation)}\label{appendix: distance cost}

\subsection{Preparation: characterization of Agent's Response}\label{sec: characterization BR}


In this subsection, we characterize the agent's best response in any random-order mechanism without disclosure.
% This characterization is mainly used to prove  \cref{lem:feasible-uninformed-rand-distance cost}.
% Readers who are not particularly interested in the proof of \cref{lem:feasible-uninformed-rand-distance cost} are safe to skip this section.
% , i.e.,  when the each of the two tests $\tilde \classifier_i$ used in the uninformed random order mechanism is parallel to the principal's true requirement $\classifier_i$ for $i\in \{A,B\}$.

Consider a random-order mechanism without disclosure $(\tilde \classifier_A, \tilde \classifier_B, q, \varnothing)$. 
We use $\manipulation_{\probprincipal}(\tilde\classifier_A,\tilde\classifier_B)$ to denote the set of attributes that the agent does not initially pass the sequence of tests but could profitably manipulate his attribute so as to pass the selection procedure. 
To prove the main theorem, we show the following coverage property of the manipulation sets for random-order mechanisms without disclosure with different probability $q$. 
\begin{proposition}\label{prop: coverage of Mq}
     For any fixed classifiers $\tilde\classifier_A$ and $\tilde\classifier_B$, the manipulation sets for random-order mechanisms without disclosure $(\tilde\classifier_A,\tilde\classifier_B,q,\nullset)$ satisfy:
    \begin{enumerate}
        \item $\manipulation_{\probprincipal}(\tilde \classifier_A, \tilde \classifier_B) \subset \manipulation_{0} (\tilde \classifier_A, \tilde \classifier_B)$ for any $0\leq \probprincipal \leq 1/2$;
        \item $\manipulation_{\probprincipal} (\tilde \classifier_A, \tilde \classifier_B) \subset \manipulation_{1} (\tilde \classifier_A, \tilde \classifier_B)$ for any $1/2 \leq \probprincipal \leq  1$;
    \end{enumerate}  
\end{proposition}



% \begin{lemma}[zig-zag strategy]\label{lem:zig-zag}
%     Suppose  $\probprincipal=1$. 
%     For any $\features\in R_4\cap \classifier_1^C$, let $\symmetric_{\classifier_1}\features$ be the point that is symmetric of $\features$ over $\classifier_1$.
%     Let $\Pi_{\classifier_2}(\symmetric_{\classifier_1}\features)$ be the projection of $\symmetric_{\classifier_1}\features$ on $\classifier_2$, i.e., the line connecting $\symmetric_{\classifier_1}\features$ and $\Pi_{\classifier_2}(\symmetric_{\classifier_1}\features)$ is perpendicular to  $\classifier_2$.
%     Let $\features'$ be the point that is on the boundary of $\classifier_1$ and intersects the line connecting $\symmetric_{\classifier_1}\features$ and $\Pi_{\classifier_2}(\symmetric_{\classifier_1}\features)$.
%     Then the best zig-zag strategy is to first move to $\features'$  and then to  $\Pi_{\classifier_2}(\symmetric_{\classifier_1}\features)$; moreover the cost of the best zig-zag strategy is $\onecost(\symmetric_{\classifier_1}\features,\Pi_{\classifier_2}(\symmetric_{\classifier_1}\features))$.
%     For any $\features\in R_4\cap \classifier_1$, let $\Pi_{\classifier_2}\features$ be the projection of $\features$ on $\classifier_2$, and let $\features'$ be the point that is on the boundary of $\classifier_1$ and intersects the line connecting $\features$ and $\Pi_{\classifier_2}\features$.
%     Then the best zig-zag strategy is to first move to $\features'$  and then to  $\Pi_{\classifier_2}\features$; moreover the cost of the best zig-zag strategy is $\onecost(\features,\Pi_{\classifier_2}\features)$.
% \end{lemma}
We first formally characterize the zig-zag strategy of agents in the sequential mechanism. For any attributes $\features$ and any classifier $\tilde \classifier_A$, we use $\symmetric_{\tilde\classifier_A} \features$ to denote its symmetric point with respect to the boundary of $\tilde\classifier_A$. We use $\Pi_{\tilde\classifier_A}\features$ to denote its projection onto $\tilde\classifier_A$, which is the closest point to $\features$ in $\tilde \classifier_A$. We call a two-step strategy $(\features,\features_1,\features_2)$ a zig-zag strategy if three attributes $\features$, $\features_1$, and $\features_2$ are not on a line. 

\begin{lemma}[zig-zag strategy]\label{lem:zig-zag}
    %Consider any sequential mechanism with two tests $\tilde \classifier_A$ and $\tilde \classifier_B$ such that the angle between two tests is in $(0,90^{\circ})$.
    Consider any attributes $\features \in \tilde \classifier_A^\compl \cup \tilde \classifier_B^\compl$.  
    If  $\Pi_{\classifier_B}(\symmetric_{\classifier_A}\features)$ does not satisfy $\tilde\classifier_A$, then the best strategy to first satisfy $\tilde\classifier_A$ is a zig-zag strategy $(\features,\features_1, \features_2)$, where $\features_2 = \Pi_{\classifier_B}(\symmetric_{\classifier_A}\features)$and $\features_1$ is the intersection point of $(\symmetric_{\classifier_A}\features)\features_2$ and the boundary of $\tilde\classifier_A$.
\end{lemma}

\begin{proof}[Proof of \cref{lem:zig-zag}]
    %Since the angle between two classifiers $\tilde\classifier_A$ and $\tilde\classifier_B$ is in $(0,90^{\circ})$, the line segment $(\symmetric_{\classifier_A}\features)\features_2$ intersects with the boundary of $\tilde\classifier_A$. 
    We first show that this zig-zag strategy is well-defined. Since $\features \in \tilde \classifier_A^C \cup \tilde \classifier_B^C$, we have $\symmetric_{\classifier_A}\features \in \tilde\classifier_A$. 
    If $\features_2 = \Pi_{\classifier_B}(\symmetric_{\classifier_A}\features)$ does not satisfy $\tilde\classifier_A$, then the line segment $(\symmetric_{\classifier_A}\features)\features_2$ intersects with $\tilde\classifier_A$. 
    Thus, this zig-zag strategy $(\features,\features_1,\features_2)$ is well-defined.
    
    For the zig-zag strategy $(\features,\features_1,\features_2)$, the cost of this strategy is $\cost(\features,\features_1,\features_2) = \onecost(\features,\features_1) + \onecost(\features_1,\features_2)$. Since $\features_1$ is on the boundary of $\tilde\classifier_A$, by symmetry, we have $\onecost(\symmetric_{\tilde\classifier_A}\features,\features_1) = \onecost(\features,\features_1)$. Since $\features_1$ is on the line $(\symmetric_{\tilde\classifier_A}\features)\features_2$, we have
    $$    \cost(\features,\features_1,\features_2) = \onecost(\symmetric_{\tilde\classifier_A}\features,\features_1) + \onecost(\features_1,\features_2) = \onecost(\symmetric_{\tilde\classifier_A}\features, \features_2).
    $$
    
    Consider any zig-zag strategy $(\features, \features',\features'')$ that first satisfies $\tilde\classifier_A$ and then $\tilde\classifier_B$.
    Since the attributes $\features$ do not satisfy $\tilde\classifier_1$, its symmetric point $\symmetric_{\tilde\classifier_A} \features$ satisfy $\tilde\classifier_A$.
    Since $\features'$ satisfies $\tilde \classifier_A$, by symmetry, we have $\onecost(\symmetric_{\tilde\classifier_A}\features, \features') \leq \onecost(\features, \features')$, which implies $\onecost(\symmetric_{\tilde\classifier_A}\features, \features') \leq \onecost(\features, \features')$. 
    By the triangle inequality of the Euclidean distance, the cost of this strategy is at least
    $$
    \cost(\features,\features',\features'') \geq \onecost(\symmetric_{\tilde\classifier_A}\features, \features') + \onecost(\features',\features'') \geq \onecost(\symmetric_{\tilde\classifier_A}\features, \features'').
    $$
    Since $\features_2$ is the projection of $\symmetric_{\tilde\classifier_A} \features$ onto the boundary of $\tilde\classifier_B$, we have $$\cost(\features,\features',\features'') \geq \onecost(\symmetric_{\tilde\classifier_A}\features, \features'') \geq \onecost(\symmetric_{\tilde\classifier_A}\features, \features_2) = \cost(\features,\features_1,\features_2),$$
    which completes the proof.
\end{proof}

\begin{remark}
    Consider any sequential mechanism with two tests $\tilde \classifier_A$ and $\tilde \classifier_B$ such that the angle between two tests is in $[90^{\circ}, 180^{\circ})$.
    For any attributes $\features \in \tilde \classifier_A^\compl \cup \tilde \classifier_B^\compl$, we have  $\Pi_{\classifier_B}(\symmetric_{\classifier_A}\features)$ satisfies $\tilde\classifier_A$. Thus, for such an agent, the best strategy to first satisfy $\tilde\classifier_A$ is not a zig-zag strategy. In this case, it is easy to show that the best strategy is a one-step strategy. An alternative algebraic proof is provided by Theorem 3.7 in \citet{zigzag}.
\end{remark}

In the following analysis, we only consider the sequential mechanism with two tests $\tilde \classifier_A$ and $\tilde \classifier_B$ such that the angle between two tests is in $(0,90^{\circ})$.
We now characterize the manipulation set $\manipulation_{\probprincipal}(\tilde\classifier_A,\tilde\classifier_B)$ for random-order mechanism without disclosure $(\tilde\classifier_A, \tilde\classifier_B,q,\varnothing)$.
Let $L_A$ and $L_B$ be the boundary lines of classifiers $\tilde \classifier_A$ and $\tilde \classifier_B$ respectively. 
Let $O$ be the intersection point of $L_A$ and $L_B$.
Let $\Line_A^+=\Line_A\cap \classifier_B$  and $\Line_B^+=\Line_B\cap \classifier_A$ be the part of $L_A$ and $L_B$ in the qualified region, respectively.
Let $\setperp_A(\tilde \classifier_A, \tilde \classifier_B)=\{\orifeatures\notin \tilde\classifier_A\cap \tilde\classifier_B: \min_{\genericfeatures\in\Line_A^+}\onecost(\orifeatures,\genericfeatures)\leq 1 \}$ be the set of candidates whose true attributes are not qualified but have cost less than one to adopt attributes on $\Line_A^+$.
Similarly, let $\setperp_B(\tilde \classifier_A, \tilde \classifier_B)=\{\orifeatures\notin \tilde \classifier_A\cap \tilde \classifier_B: \min_{\genericfeatures\in\Line_B^+}\onecost(\orifeatures,\genericfeatures)\leq 1 \}$ be the set of candidates whose true attributes are not qualified but have cost less than one to adopt attributes on $\Line_B^+$.

Let $\Omega(\tilde\classifier_A,\tilde\classifier_B) = \bbR^2 \setminus ((\tilde\classifier_A\cap\tilde\classifier_B)\cup \setperp_A(\tilde \classifier_A, \tilde \classifier_B)\cup \setperp_B(\tilde \classifier_A, \tilde \classifier_B))$.
Without loss of generality, we assume the unit normal vector of $\tilde\classifier_A$ is $\weights_A = (1,0)$.
Let $\util_{AB}$ be the agent's best utility among strategies that first pass only $\tilde \classifier_A$ but not $\tilde \classifier_B$ and then pass $\tilde \classifier_B$. 
We define the set $\setonetwo_{\probprincipal}(\tilde\classifier_A,\tilde\classifier_B)=\{\features\in\Omega(\tilde\classifier_A,\tilde\classifier_B): \util_{AB}\geq 0\}$.
Similarly, let $\util_{BA}$ be the agent's best utility among all strategies that first pass only $\tilde\classifier_B$ and then pass $\tilde\classifier_A$. 
Define the set $\settwoone_{\probprincipal}(\tilde\classifier_A,\tilde\classifier_B)=\{\features\in\Omega(\tilde\classifier_A,\tilde\classifier_B): \util_{BA}\geq 0\}$.
Let the agent's utility of directly moving to point $O$ be $\util_{0}$.
We define the set $\setO_{\probprincipal}(\tilde\classifier_A,\tilde\classifier_B)=\{\features\in\Omega(\tilde\classifier_A,\tilde\classifier_B): \util_{0}\geq 0 \}$.
% We use $\manipulation_{\probprincipal}(\tilde\classifier_A,\tilde\classifier_B)$ to denote the set of attributes in $\Omega(\tilde\classifier_A,\tilde\classifier_2)$ that the agent does not initially pass the sequence of tests but could profitably manipulate his attribute so as to pass the selection procedure.
When there is no ambiguity about the two classifiers $\tilde \classifier_A$ and $\tilde \classifier_B$ used in the mechanism, we use $\setO$, $\setonetwo$, and $\settwoone$ to denote these sets. 
Then, we have the manipulation set $\manipulation_{\probprincipal}(\tilde \classifier_A,\tilde\classifier_B)=\setperp_A \cup \setperp_B \cup \setonetwo_{\probprincipal}\cup \settwoone_{\probprincipal}\cup \setO_{\probprincipal}$.
% By \cref{prop: q=1}, we know that under fixed order and announcing the true selection criteria, $\manipulation_{1}= R_3\cup R_4$.
% Notice that the set of attributes where the agent does not initially pass the sequence of tests but could profitably manipulate his attribute so as to pass the selection procedure is $\allmanipulation_{\probprincipal}(\tilde\classifier_1,\tilde\classifier_2)=\manipulation_{\probprincipal}(\tilde\classifier_1,\tilde\classifier_2) \cup R_1(\tilde\classifier_1) \cup R_2(\tilde\classifier_2) $.

% Denote the set of attributes where the agent passes the sequence of tests without manipulating his attribute by $\qualified(\tilde\classifier_1,\tilde\classifier_2) = \tilde\classifier_1\cap\tilde\classifier_2$.
% Then the set of attributes that would eventually be selected is $\selected_{\probprincipal}(\tilde\classifier_1,\tilde\classifier_2)=\allmanipulation_{\probprincipal}(\tilde\classifier_1,\tilde\classifier_2)\cup \qualified(\tilde\classifier_1,\tilde\classifier_2)$.

First, it is easy to see that the set $\setO_{\probprincipal}$ is invariant with the probability $q$.

\begin{lemma}\label{lmm: Bq invariant with q}
    $ \setO_{\probprincipal} = \setO =\arc{OAB}$ for any $\probprincipal\in [0,1]$.
\end{lemma}

\begin{proof}
    The ball $B(O,1/\eta)$ contains all attributes $\features$ such that the cost $\onecost(\features,O) \leq 1$. We have $\setO_{\probprincipal} = B(O, 1/\eta) \cap \Omega(\tilde\classifier_A,\tilde\classifier_B) = \arc{OAB}$.
\end{proof}

Next, we characterize the sets $\setonetwo_{\probprincipal}$ and $\settwoone_{\probprincipal}$. Let the point $\pointonetwo_{\probprincipal}$ be the point whose distance to point $O$ is $\probprincipal/\eta$ and whose projection on $\tilde\classifier_B$ is point $O$. Then $\pointonetwo_{\probprincipal} = -\frac{\probprincipal}{\mc}\cdot \weights_B$, where $\weights_B$ is the unit normal vector of $\tilde\classifier_B$.
Note that for the agent with attributes $\pointonetwo_{\probprincipal}$, the cost for first passing $\tilde\classifier_A$ but not $\tilde \classifier_B$ and then passing $\tilde\classifier_B$ is $\probprincipal$.
Let its symmetric point with respect to the boundary of $\tilde\classifier_A$ be $\pointonetwo_{\probprincipal}' = \symmetric_{\tilde\classifier_A} \pointonetwo_{\probprincipal}$.
Let the point $\tilde \pointonetwo_{\probprincipal}  = (0,-\frac{\probprincipal}{\mc\sin{\theta}})$ be the point that falls on the boundary of $\tilde\classifier_A$ with distance $\probprincipal/\eta$ to $\tilde\classifier_B$.

\begin{lemma}\label{lem: Cq}
    $\setonetwo_{\probprincipal} = O \pointonetwo_{\probprincipal} \tilde \pointonetwo_{\probprincipal} \pointonetwo_{\probprincipal}'$.
\end{lemma}

\begin{proof}
    We first consider the agent with attributes $\features$ in the triangle region $O\pointonetwo_{\probprincipal} \tilde \pointonetwo_{\probprincipal}$. These attributes already satisfy the classifier $\tilde\classifier_A$. Since both $\pointonetwo_{\probprincipal}$ and $\tilde \pointonetwo_{\probprincipal}$ have distance $\probprincipal/\eta$ to $\tilde\classifier_B$, we have the line $\pointonetwo_{\probprincipal}\tilde \pointonetwo_{\probprincipal}$ is parallel to the boundary of $\tilde\classifier_B$. Thus, all attributes in $O\pointonetwo_{\probprincipal} \tilde \pointonetwo_{\probprincipal}$ have distance at most $\probprincipal/\eta$ to $\tilde\classifier_B$. For the agent with these attributes, the utility for providing $\features$ in the first test and then moving to pass $\tilde \classifier_B$ is non-negative. 

    Then, we consider the triangle region $O\pointonetwo_{\probprincipal}' \tilde \pointonetwo_{\probprincipal}$. Note that for each attributes $\features$ in $O\pointonetwo_{\probprincipal}' \tilde \pointonetwo_{\probprincipal}$, its symmetric point with respect to the boundary of $\tilde\classifier_A$ is in $O\pointonetwo_{\probprincipal} \tilde \pointonetwo_{\probprincipal}$. Thus, by Lemma~\ref{lem:zig-zag}, the agent with attributes in $O\pointonetwo_{\probprincipal}' \tilde \pointonetwo_{\probprincipal}$ also has the utility $u_{AB} \geq 0$.
\end{proof}

Similarly, let $\pointtwoone_{1-\probprincipal} = -\frac{1-\probprincipal}{\mc}\cdot \weights_A$ be the point whose distance to point $O$ is $1-\probprincipal$ and whose projection on $\tilde\classifier_A$ is point $O$. 
Let its symmetric point over the boundary of $\tilde\classifier_B$ be $\pointtwoone_{1-\probprincipal}' = \symmetric_{\tilde\classifier_B} \pointtwoone_{1-\probprincipal}$.
Let point $\tilde \pointtwoone_{1-\probprincipal}  = (-\frac{ 1-\probprincipal}{\mc},-\frac{1-\probprincipal}{\mc\tan{\theta}})$ be the point that falls on the boundary of $\tilde\classifier_B$ with distance $\frac{1-\probprincipal}{\eta}$ to $\tilde\classifier_A$. With a similar analysis as in Lemma~\ref{lem: Cq}, we characterize $\settwoone_{\probprincipal}$ as follows.

\begin{lemma}\label{lem: Dq}
    $\settwoone_{\probprincipal} = O \pointtwoone_{1-\probprincipal} \tilde \pointtwoone_{1-\probprincipal} \pointtwoone_{1-\probprincipal}'$.
\end{lemma}

% \begin{proof}
    
% \end{proof}

\begin{lemma}\label{lmm:Cq < Dq+Bq when q is small}
   $\setonetwo_{\probprincipal}\subset \settwoone_{0}\cup \setO$ if $\probprincipal\leq \frac{1}{2\cos{\theta}}$.
   %:=f(\theta)$.
   %This implies that $\manipulation_{\probprincipal}=\settwoone_{\probprincipal}\cup \setO$ if and only if $\probprincipal\leq f(\theta)$.
\end{lemma}

\begin{proof}%[Proof of \cref{lmm:Cq < Dq+Bq when q is small}]
    % Given $\probprincipal, \tilde\classifier_1,\tilde\classifier_2$, $\util_{12}(\orifeatures)\geq 0\Leftrightarrow \cost_{12}(\orifeatures)\leq \probprincipal$, where $\cost_{12}(\orifeatures)$ is the smallest cost of an agent with initial attributes $\orifeatures$ to first move to pass $\tilde\classifier_1$ and then move to pass $\tilde\classifier_2$.

Let $G = (0, -\frac{1}{\mc\sin{2\theta}})$ be the point where $\tilde \pointtwoone_{1} \pointtwoone_{1}'$ intersects with the boundary of $\tilde\classifier_A$.
% Let  $G' = (0, -\frac{1}{\mc})$ be the point where $\setO$ intersects with the boundary of $\tilde\classifier_1$.
When  $\probprincipal\leq \sin{\theta}$, we have $\frac{\probprincipal}{\mc\sin{\theta}}\leq \frac{1}{\mc\sin{2\theta}}$.
Since the point $\tilde \pointonetwo_{\probprincipal} = (0, - \frac{\probprincipal}{\mc\sin{\theta}})$, we have $O\tilde \pointonetwo_{\probprincipal}\subset OG$.

It is easy to see that $O\pointonetwo_{\probprincipal}\subset \setO$ for any $\probprincipal\leq 1$. Since $\pointonetwo_{\probprincipal}'$ is the symmetric point of $\pointonetwo_{\probprincipal}$ with respect to $\tilde\classifier_1$, we have $O\pointonetwo_{\probprincipal}'\subset \setO$ for any $\probprincipal\leq 1$. 

Note that $\settwoone_{0}\cup \setO$ is a convex set. Since four points $O$, $\pointonetwo_{\probprincipal}$, $\pointonetwo_{\probprincipal}'$, and $\tilde \pointonetwo_{\probprincipal}$ are contained in $\settwoone_{0}\cup \setO$, By Lemma~\ref{lem: Cq}, we have that $\setonetwo_{\probprincipal}\subset \settwoone_{\probprincipal}\cup \setO$.
\end{proof}

Similarly, we can show the following coverage property for $\settwoone_{\probprincipal}$.

\begin{lemma}\label{lmm:Dq < Cq+Bq when q is large}
   $\settwoone_{\probprincipal}\subset \setonetwo_{1}\cup \setO$ if and only if $\probprincipal\geq 1 - \frac{1}{2\cos{\theta}}$.
   %=1-h(\theta)$.
   %This implies that $\manipulation_{\probprincipal}=\setonetwo_{\probprincipal}\cup \setO$ if and only if $\probprincipal\geq 1-h(\theta)$.
\end{lemma}

\begin{proof}
    Let $E = (-\frac{1}{2\eta\cos\theta}, -\frac{1}{2\eta\sin{\theta}})$ be the point where $\tilde \pointonetwo_{1} \pointonetwo_{1}'$ intersects with the boundary of $\tilde\classifier_B$. Note that the point $\tilde \pointtwoone_{1-q} = (-\frac{1-q}{\eta},-\frac{1-q}{\eta\tan\theta}) = 2\cos \theta (1-q) E$. When $q \geq 1-\frac{1}{2\cos \theta}$, we have $2\cos \theta (1-q) \leq 1$ , which implies $O\tilde \pointtwoone_{1-q} \subset OE$.
    Since $\pointtwoone_{1-q}$ and $\pointtwoone_{1-q}'$ is in $\setO$ and $\setonetwo_{1}\cup \setO$ is convex, we have $\settwoone_{\probprincipal}\subset \setonetwo_{1}\cup \setO$.
\end{proof} 

% \paragraph{Set order under fixed $\tilde\classifier_1,\tilde\classifier_2$.} The following results concern the order of the manipulation set when the announced criteria are fixed.

Now, we prove the coverage proposition of the manipulation sets for sequential mechanisms.

\begin{proof}[Proof of \cref{prop: coverage of Mq}]
    We first consider the case where $q \leq 1/2$. 
    By \cref{lmm:Cq < Dq+Bq when q is small}, we know that $\manipulation_{\probprincipal}=\setperp_A \cup \setperp_B \cup \setonetwo_q \cup \settwoone_{\probprincipal}\cup \setO \subset \setperp_A \cup \setperp_B \cup\settwoone_0 \cup \setO \subset \manipulation_0$.
    % It suffices to show that $\settwoone_{\probprincipal} \subset \settwoone_{\probprincipal'}$ for any $0\leq \probprincipal'\leq \probprincipal\leq h(\theta)$.
    % This is true because for every edge of $\settwoone_{\probprincipal}$, for instance $O \pointtwoone_{1-\probprincipal}  \subset O \pointtwoone_{1-\probprincipal'} $ for any $0\leq \probprincipal'\leq \probprincipal\leq h(\theta)$.

    We then consider the case where $q \geq 1/2$. 
    By \cref{lmm:Dq < Cq+Bq when q is large}, we know that $\manipulation_{\probprincipal}=\setperp_A \cup \setperp_B \cup\setonetwo_{\probprincipal}\cup \settwoone_{\probprincipal} \cup \setO \subset \setperp_A \cup \setperp_B \cup\setonetwo_1 \cup \setO \subset \manipulation_1$.
    %for any $ \probprincipal\geq 1-h(\theta)$.
    % It suffices to show that $\setonetwo_{\probprincipal} \subset \setonetwo_{\probprincipal'}$ for any $ \probprincipal'\geq \probprincipal\geq 1- h(\theta)$.
    %  This is true because for every edge of $\setonetwo_{\probprincipal}$, for instance $O \pointonetwo_{\probprincipal}  \subset O \pointonetwo_{\probprincipal'} $ for any $ \probprincipal'\geq \probprincipal\geq 1- h(\theta)$.
\end{proof}



\begin{figure}[t]
\centering
% \begin{subfigure}[b]{0.3\linewidth}
% \begin{tikzpicture}[xscale=3.5,yscale=3.5,
%     pics/legend entry/.style={code={%   
%         \draw[pic actions] 
%         (-0.25,0.25) -- (0.25,0.25);}}]]
\begin{tikzpicture}[xscale=6,yscale=6,
    pics/legend entry/.style={code={%   
        \draw[pic actions] 
        (-0.25,0.25) -- (0.25,0.25);}}]]

\draw [domain=0.8:1.16, thick] plot (\x, {tan(deg(0.4*pi))*(\x-1)+1});%less than pi/9
\node [above] at (1.16, 1.5 ) {$\tilde\classifier_2$};
\draw [thick] (1,0.15) -- (1,1.5);
\node [above] at (1, 1.5 ) {$\tilde\classifier_1$};%: \feature_1 \geq 1


\draw [domain={1+0.25*cos(deg(0.1*pi))}:1.4, densely dotted] plot (\x, {tan(deg(0.4*pi))*(\x-1)+1-0.809}); % 1/c = 1/4
%\node [above] at (1.4, 1.5 ) {$\classifier_2^-$};
\draw [densely dotted] (1-0.25,1) -- (1-0.25,1.5);
%\node [above] at (1-0.25, 1.5) {$\classifier_1^-$};
\draw[red] (1-0.25,1) arc (180:240:0.25);
\draw[red] (1,1) -- ++(180:0.25);
\draw[red] (1,1) -- ++(240:0.25) ;

\draw[red] ({1+0.25*cos(deg(0.1*pi))},{1-0.25*sin(deg(0.1*pi))}) arc (360-18:360-78:0.25);
\draw[red] (1,1) -- ++(360-18:0.25);
\draw[red] (1,1) -- ++(360-78:0.25) ;

\draw [red] (1-0.125,{1-0.125*sqrt(3)}) -- (1-0.125,0.61625);
\draw [domain=0.875:0.957, red]  plot(\x,{tan(deg(0.3*pi))*(\x-0.875)+0.61625});

\draw [domain=0.956:1,red]  plot(\x,{-tan(deg(0.4*pi))*(\x-1)+1-0.125/sin(deg(0.1*pi))});
\draw [domain=1:1.053,red]  plot(\x,{tan(deg(0.4*pi))*(\x-1)+1-0.125/sin(deg(0.1*pi))});


\draw[blue] (1,1) -- ++(192:0.25) ;
\draw [domain=1:{1+0.25*cos(deg(0.1*pi))}, blue] plot (\x, {tan(deg(0.4*pi))*(\x-1)+1-0.809});
\draw [domain=1-0.246:1,blue]  plot(\x,{-tan(deg(0.4*pi))*(\x-1)+1-0.809});

\draw [domain=1:{1+0.25*0.8*cos(deg(0.1*pi))}, teal] plot (\x, {tan(deg(0.4*pi))*(\x-1)+1-0.8*0.25/sin(deg(0.1*pi))});
\draw [domain=1-0.196:1,teal]  plot(\x,{-tan(deg(0.4*pi))*(\x-1)+1-0.8*0.25/sin(deg(0.1*pi))});


\node [above] at (1.06, 1.01 ) {\footnotesize$O$};

\matrix [draw, above right] at (1.6,1) {
 \pic[blue]{legend entry}; &  \node[blue,font=\tiny] {$\probprincipal=1$}; \\
 \pic[red]{legend entry}; &  \node[red,font=\tiny] {$\probprincipal=\frac12$}; \\
 \pic[teal]{legend entry}; &  \node[teal,font=\tiny] {$\probprincipal=0.8$}; \\
};

\end{tikzpicture}
\caption{ random-order mechanisms without disclosure vs fixed-order mechanism}
% \rule{0in}{1.2em}$^\dag$\scriptsize 
% In this graph, $\theta=\frac{\pi}{10}$. Notice that $\manipulation_{\frac12}\subset M_1$ and $M_{0.8}\subset M_1$. However, we cannot rank the set $M_{0.8}$ and $\manipulation_{\frac12}$.\\
\end{figure}


% %--------------------------------------
\begin{figure}[t]
\centering
\begin{subfigure}[b]{0.4\linewidth}
\begin{tikzpicture}[xscale=3.5,yscale=3.5,
    pics/legend entry/.style={code={%   
        \draw[pic actions] 
        (-0.25,0.25) -- (0.25,0.25);}}]]
% \begin{tikzpicture}[xscale=6,yscale=6,
%     pics/legend entry/.style={code={%   
%         \draw[pic actions] 
%         (-0.25,0.25) -- (0.25,0.25);}}]]

% \draw [<->] (0,2.1) -- (0,0) -- (2.1,0);
% \node [right] at (2.15, 0 ) {$\feature_1$};
% \node [left] at (0,2.15) {$\feature_2$};


\draw [domain=0.8:1.16, thick] plot (\x, {tan(deg(0.4*pi))*(\x-1)+1});%less than pi/9
\node [above] at (1.16, 1.5 ) {$\classifier_2$};
\draw [thick] (1,0.15) -- (1,1.5);
\node [above] at (1, 1.5 ) {$\classifier_1$};%: \feature_1 \geq 1


\draw [domain={1+0.25*cos(deg(0.1*pi))}:1.4, densely dotted] plot (\x, {tan(deg(0.4*pi))*(\x-1)+1-0.809}); % 1/c = 1/4
\node [above] at (1.4, 1.5 ) {$\classifier_2^-$};
\draw [densely dotted] (1-0.25,1) -- (1-0.25,1.5);
\node [above] at (1-0.25, 1.5) {$\classifier_1^-$};
\draw[blue] (1-0.25,1) arc (180:198:0.25);
\draw[blue] (1,1) -- ++(180:0.25);

\draw[teal] ({1-0.25*cos(deg(0.1*pi))},{1-0.25*sin(deg(0.1*pi))}) arc (198:198+36:0.25);
\draw[teal] (1,1) -- ++(198+36:0.25) ;


\draw[teal] ({1+0.25*cos(deg(0.1*pi))},{1-0.25*sin(deg(0.1*pi))}) arc (360-18:360-54:0.25);
\draw[teal] (1,1) -- ++(360-18:0.25);
\draw[teal] (1,1) -- ++(360-54:0.25) ;




\draw[blue] (1,1) -- ++(192:0.25) ;
\draw [domain=1:{1+0.25*cos(deg(0.1*pi))}, blue] plot (\x, {tan(deg(0.4*pi))*(\x-1)+1-0.809});
\draw [domain=1-0.246:1,blue]  plot(\x,{-tan(deg(0.4*pi))*(\x-1)+1-0.809});

\draw [domain=1:{1+0.25*0.8*sin(47)}, teal] plot (\x, {tan(deg(0.4*pi))*(\x-1)+1-0.8*0.25/sin(deg(0.1*pi))});
\draw [domain={1-0.25*0.8*cos(43)}:1,teal]  plot(\x,{-tan(deg(0.4*pi))*(\x-1)+1-0.8*0.25/sin(deg(0.1*pi))});


\node [above,font=\tiny] at (1.06, 1.01 ) {$O$};

\node [right,font=\tiny] at ({1+0.25*cos(deg(0.1*pi))},{1-0.25*sin(deg(0.1*pi))} ) {$\pointonetwo_{\probprincipal=1}$};
\node [left,font=\tiny] at ({1-0.25*cos(deg(0.1*pi))},{1-0.25*sin(deg(0.1*pi))}) {$\pointonetwo_{\probprincipal=1}'$};
\node [below,font=\tiny] at (1, {1-0.25/sin(deg(0.1*pi))} ) {$\tilde \pointonetwo_{\probprincipal=1}$};


\matrix [draw, above right] at (1.5,0.2) {
 \pic[blue]{legend entry}; &  \node[blue,font=\tiny] {$\probprincipal=1$}; \\
 \pic[teal]{legend entry}; &  \node[teal,font=\tiny] {$\probprincipal=0.8$}; \\
};

\end{tikzpicture}
\caption{Random order $\probprincipal=0.8$} \label{fig:q>1-h(theta)}  
\end{subfigure}
\begin{subfigure}[b]{0.4\linewidth}
\begin{tikzpicture}[xscale=3.5,yscale=3.5,
    pics/legend entry/.style={code={%   
        \draw[pic actions] 
        (-0.25,0.25) -- (0.25,0.25);}}]]

\draw [domain=0.74:1.16, thick] plot (\x, {tan(deg(0.4*pi))*(\x-1)+1});%less than pi/9
\node [above] at (1.16, 1.5 ) {$\classifier_2$};
\draw [thick] (1,0.15) -- (1,1.5);
\node [above] at (1, 1.5 ) {$\classifier_1$};%: \feature_1 \geq 1


\draw [domain={1+0.25*cos(deg(0.1*pi))}:1.4, densely dotted] plot (\x, {tan(deg(0.4*pi))*(\x-1)+1-0.809}); % 1/c = 1/4
\node [above] at (1.4, 1.5 ) {$\classifier_2^-$};
\draw [densely dotted] (1-0.25,1) -- (1-0.25,1.5);
\node [above] at (1-0.25, 1.5) {$\classifier_1^-$};
\draw[teal] (1-0.25,1) arc (180:180+37:0.25);
\draw[teal] (1,1) -- ++(180:0.25);
\draw[teal] (1,1) -- ++(180+37:0.25) ;

\draw[red] ({1+0.25*cos(deg(0.1*pi))},{1-0.25*sin(deg(0.1*pi))}) arc (360-18:360-36:0.25);
\draw[red] (1,1) -- ++(360-18:0.25);
\draw[teal] ({1+0.25*cos(deg(0.2*pi))},{1-0.25*sin(deg(0.2*pi))}) arc (360-36:360-73:0.25);
\draw[teal] (1,1) -- ++(360-73:0.25);
% \draw[red] (1,1) -- ++(360-78:0.25) ;

\draw [teal] (1-0.25*0.8,{1-0.25*0.8*cos(44)}) -- (1-0.25*0.8,{1-0.25*0.8*tan(deg(0.4*pi))});
\draw [domain=1-0.25*0.8:{1+0.25*0.29}, teal]  plot(\x,{tan(deg(0.3*pi))*(\x-1+0.25*0.8)+1-0.25*0.8*tan(deg(0.4*pi))});


\draw[red] (1,1) -- ++(360-36:0.25) ;
\draw[red] (1-0.25,1) -- (1-0.25, {1- 0.25*tan(deg(0.4*pi))}) ;
\draw [domain=1-0.25:{1+0.25*cos(36)}, red] plot (\x, {tan(deg(0.3*pi))*(\x-1+0.25)+1-0.25*tan(deg(0.4*pi))});




\node [above,font=\tiny] at (1.06, 1.01 ) {$O$};

\node [left,font=\tiny] at ({1-0.25},1) {$\pointtwoone_{1-\probprincipal=1}$};
\node [right,font=\tiny] at ({1+0.25*cos(deg(0.2*pi))},{1-0.25*sin(deg(0.2*pi))}) {$\pointtwoone_{1-\probprincipal=1}'$};
\node [below,font=\tiny] at (1-0.25, {1-0.25*tan(deg(0.4*pi))} ) {$\tilde \pointtwoone_{1-\probprincipal=1}$};

\matrix [draw, above right] at (1.5,0) {
 \pic[red]{legend entry}; &  \node[red,font=\tiny] {$\probprincipal=0$}; \\
 \pic[teal]{legend entry}; &  \node[teal,font=\tiny] {$\probprincipal=0.2$}; \\
};

\end{tikzpicture}
\caption{Random order $\probprincipal=0.2$} \label{fig:q<h(theta)}  
\end{subfigure}

% \rule{0in}{1.2em}$^\dag$\scriptsize 
% In this graph, $\theta=\frac{\pi}{10}$. Notice that $\manipulation_{\frac12}\subset M_1$ and $M_{0.8}\subset M_1$. However, we cannot rank the set $M_{0.8}$ and $\manipulation_{\frac12}$.\\
\end{figure}

% %-----------------------------------------------
% \paragraph{Set order with varying $\tilde\classifier_1,\tilde\classifier_2$.}
% The following result shows that given a set of selected attributes from any mechanism with random order, there exists another mechanism with fixed order whose corresponding set of selected attributes is nested in the former set.

% \begin{proposition}[Nested structure of $\selected_{\probprincipal}$ with varying $\tilde\classifier_1,\tilde\classifier_2$]\label{prop: nested structure}
% Suppose $0< \theta<\pi$. Then for any $\probprincipal\in (0,1)$ and $\tilde\classifier_1,\tilde\classifier_2$, there exists $\hat{\probprincipal}\in \{0,1\}$ and $\hat\classifier_1,\hat\classifier_2$ such that
% \begin{equation}
%     \selected_{\hat{\probprincipal}}(\hat\classifier_1,\hat\classifier_2)\subset \selected_{\probprincipal}(\tilde\classifier_1,\tilde\classifier_2).
% \end{equation}
% Moreover, given $\classifier_i,i\in \{1,2\}$ whose boundary is parallel to that of $\tilde\classifier_i,i\in \{1,2\}$, suppose $\qualified(\classifier_1,\classifier_2)\subset \selected_{\probprincipal}(\tilde\classifier_1,\tilde\classifier_2)$ for some $\probprincipal\in (0,1)$, then there exists $\hat{\probprincipal}\in \{0,1\}$ and $\hat\classifier_1,\hat\classifier_2$ such that
% \begin{equation}
% \qualified(\classifier_1,\classifier_2)\subset\selected_{\hat{\probprincipal}}(\hat\classifier_1,\hat\classifier_2)\subset \selected_{\probprincipal}(\tilde\classifier_1,\tilde\classifier_2).
% \end{equation}
% \end{proposition}


% \begin{proof}[Proof of \cref{prop: nested structure}]
%     The proof is constructive. Suppose $\tilde\classifier_1: \tilde\weights_1^T\features\geq 0$ and $\tilde\classifier_2: \tilde\weights_2^T\features\geq 0$.
    
%     \textbf{Part 1.} Consider the following cases:
%     \begin{itemize}
%         \item \textbf{Case 1: $\probprincipal\leq \frac12$.} Let $\hat{\probprincipal}=0$, $\hat\classifier_1:\tilde\weights_1^T(\features+\frac{\probprincipal}{\mc}\cdot v_1)\geq 0$ and
%         $\hat\classifier_2:\tilde\classifier_2$.
%         \item \textbf{Case 2: $\probprincipal> \frac12$.} Let $\hat{\probprincipal}=1$, $\hat\classifier_2:\tilde\weights_2^T(\features+\frac{1-\probprincipal}{\mc}\cdot v_1)\geq 0$ and
%         $\hat\classifier_1=\tilde\classifier_1$.
%     \end{itemize}
%     Next we want to show that under each case, $\selected_{\hat{\probprincipal}}(\hat\classifier_1,\hat\classifier_2)\subset \selected_{\probprincipal}(\tilde\classifier_1,\tilde\classifier_2).$ To show this, we just need to show that every edge of $\selected_{\hat{\probprincipal}}(\hat\classifier_1,\hat\classifier_2)$ is contained in the set $\selected_{\probprincipal}(\tilde\classifier_1,\tilde\classifier_2)$.

%     \textbf{Case 1: $\probprincipal\leq \frac12$.}
%     ...

%      \textbf{Part 2.} Call the intersecting point of the boundaries of $\classifier_i,i\in \{1,2\}$ point $O$. 
%      \emph{First we consider the scenario where point $O$ is at the boundary of set $\selected_{\probprincipal}(\tilde\classifier_1,\tilde\classifier_2)$.}
%      Recall that $\selected_{\probprincipal}(\tilde\classifier_1,\tilde\classifier_2) = \qualified_{\probprincipal}(\tilde\classifier_1,\tilde\classifier_2)\cup R_1(\tilde\classifier_1) \cup R_2(\tilde\classifier_2) \cup \manipulation_{\probprincipal}(\tilde\classifier_1,\tilde\classifier_2)$, where $\manipulation_{\probprincipal}(\tilde\classifier_1,\tilde\classifier_2)=\setonetwo_{\probprincipal}(\tilde\classifier_1,\tilde\classifier_2)\cup\settwoone_{\probprincipal}(\tilde\classifier_1,\tilde\classifier_2)\cup\setO(\tilde\classifier_1,\tilde\classifier_2)$.
     
%      \textbf{Case 1: $\probprincipal\leq h(\theta)$.} From \cref{lmm:Cq < Dq+Bq when q is small}, we know that $\selected_{\probprincipal}(\tilde\classifier_1,\tilde\classifier_2) = \qualified_{\probprincipal}(\tilde\classifier_1,\tilde\classifier_2)\cup R_1(\tilde\classifier_1) \cup R_2(\tilde\classifier_2)\cup\settwoone_{\probprincipal}(\tilde\classifier_1,\tilde\classifier_2)\cup\setO(\tilde\classifier_1,\tilde\classifier_2)$.
%      Let $\hat{\probprincipal}=0$, $\hat\classifier_1:\weights_1^T(\features-\frac{1}{\mc}\cdot v_1)\geq 0$, i.e.,  $\classifier_1$ shifting inward by a distance of $\frac{1}{\mc}$, and
%         $\hat\classifier_2:\weights_2^T(\features-\frac{1}{\mc}\cdot v_2)\geq 0$, i.e., $\classifier_2$ shifting inward by a distance of $\frac{1}{\mc}$.
%      To argue that $\qualified(\classifier_1,\classifier_2)\subset\selected_{\hat{\probprincipal}}(\hat\classifier_1,\hat\classifier_2)\subset \selected_{\probprincipal}(\tilde\classifier_1,\tilde\classifier_2)$, we further distinguish the following cases:
   
%    \textbf{Case 1.a.:} point $O$ is at the boundary of $R_1(\tilde\classifier_1)$. 
%         This is only possible if and only if the boundary of $R_1(\tilde\classifier_1)$ and $\classifier_1$ coincide.

%         ...

%     \textbf{Case 1.b.:} point $O$ is at the boundary of $\settwoone_{\probprincipal}(\tilde\classifier_1,\tilde\classifier_2)$.
%     The construction and argument is the same as the above case.

%     ...



%     \textbf{Case 1.c.:} point $O$ is at the boundary of $\setO(\tilde\classifier_1,\tilde\classifier_2)$.
    

    
%     \textbf{Case 1.d.:} point $O$ is at the boundary of $R_2(\tilde\classifier_2)$.
        
%     ...

%     \textbf{Case 2: $\probprincipal> 1-h(\theta)$.} Let $\hat{\probprincipal}=1$, $\hat\classifier_1:\weights_1^T(\features-\frac{1}{\mc}\cdot v_1)\geq 0$, i.e.,  $\classifier_1$ shifting inward by a distance of $\frac{1}{\mc}$, and
%         $\hat\classifier_2:\weights_2^T(\features-\frac{1}{\mc}\cdot v_2)\geq 0$, i.e., $\classifier_2$ shifting inward by a distance of $\frac{1}{\mc}$. Then we can argue in a similar way...

%         \textbf{Case 3: $h(\theta) \leq \probprincipal\leq 1-h(\theta)$.} 
%         ...most complicated case...


%         \emph{Lastly when point $O$ is not at the boundary of set $\selected_{\probprincipal}(\tilde\classifier_1,\tilde\classifier_2)$,} the same construction we have discussed would go through. 
% \end{proof}

\subsection{Proof of main results in \cref{subsec:seq manipulation}}

%--------------------informed------------------

\begin{proof}[Proof of \cref{lem:gain non-parallel tests}]
   From \cref{sec: characterization BR} we know that the set of attributes selected is $\manipulation_1=\setperp_A \cup \setperp_B \cup\setonetwo_1 \cup \setO $.
   Apply \cref{lmm: Bq invariant with q} and \cref{lem: Cq}. 
\end{proof}

\begin{proof}[Proof of \cref{lem:loss non-parallel tests}]
    The proof uses similar argument as the previous one and hence is omitted.
\end{proof}

\begin{proof}[Proof of \cref{lem:feasible-informed-rand-distance cost}]
First, we  show that the fixed order mechanism $(\tilde\classifier_A,\tilde\classifier_B,1)$ is feasible.
That is, we want to show that for $\features$ that are accepted by $(\tilde\classifier_A,\tilde\classifier_B,1)$, they are also qualified.
We partition the set of attributes that are accepted by $(\tilde\classifier_A,\tilde\classifier_B,1)$ into two subsets $F_1$ and $F_2$. 
We show that for any attributes in the first subset $F_1$ are also accepted by the informed random order mechanism (Step 1).
Since the informed random order mechanism is feasible, we know that any attributes in the first subset $F_1$ are qualified.
As for the second subset $F_2$, we show that it is contained in a convex set, whose extreme points are in $F_1$ and hence are qualified(Step 2).
Since the qualified region is also convex, we can infer that the second subset $F_2$ is contained in the qualified region.
Hence any attributes in the second subset are also qualified.
The feasibility of $(\tilde\classifier_A,\tilde\classifier_B,0)$ can be shown analogously.
Lastly, we show that one of the two fixed order mechanisms is no worse than the informed random-order mechanism (Step 3).


\paragraph{Step 1} We first show that the set of attributes that (1) are accepted by informed random order mechanism  $(\tilde\classifier_A,\tilde\classifier_B,q,\test_1)$ and (2) satisfy either $\tilde\classifier_A$ or $\tilde\classifier_B$, contains the set of attributes that (1) are accepted by the fixed order mechanism  $(\tilde\classifier_A,\tilde\classifier_B,1)$ and (2) satisfy either $\tilde\classifier_A$ or $\tilde\classifier_B$.
We call the former set set $I$, the latter set set $F_1$.

Consider any $\features$ that are in set $F_1$, i.e.,  any $\features$ that (1) are accepted by the fixed order mechanism  $(\tilde\classifier_A,\tilde\classifier_B,1)$ and (2) satisfy either $\tilde\classifier_A$ or $\tilde\classifier_B$.
We distinguish three cases.

\textbf{Case 1:} Consider $\features$ that satisfy both $\tilde\classifier_A$ and $\tilde\classifier_B$.
Under the strategy $\strategies=(\features,\features)$, 
such attributes are accepted by both the informed random order mechanism and the fixed order mechanism.

\textbf{Case 2:} Consider $\features$ that satisfy $\tilde\classifier_A$ but not $\tilde\classifier_B$.
We can infer that either (1) the best response is a one-step strategy, i.e.,  $\firstfeatures=\secondfeatures$ that satisfy both $\tilde\classifier_A$ and $\tilde\classifier_B$, or (2) the best response  is a two-step strategy, i.e., $\firstfeatures$  satisfy $\tilde\classifier_A$ but not $\tilde\classifier_B$, and $\secondfeatures$ satisfy $\tilde\classifier_B$ but not $\tilde\classifier_A$.
If the former case is true, then such attributes are also accepted by the informed random order mechanism and hence are also in set $I$.
If the latter case is true, then we must have $\firstfeatures=\features$ because of triangle inequality.
Moreover, we can infer that such a strategy is profitable, i.e., $1-c(\features,\features,\secondfeatures)\geq 0$.
This implies that when such attributes use the same strategy in the informed random order mechanism, the expected utility is $q[1-c(\features,\features,\secondfeatures)]\geq 0$ and they get accepted by the informed random order mechanism with probability at least $q$.
Therefore, such attributes are also in set  $I$.

\textbf{Case 3:} Consider $\features$ that satisfy $\tilde\classifier_B$ but not $\tilde\classifier_A$.
Similarly, we can infer that either (1) it is a one-step strategy, i.e.,  $\firstfeatures=\secondfeatures$ that satisfy both $\tilde\classifier_A$ and $\tilde\classifier_B$, or (2) it is a two-step strategy, i.e., $\firstfeatures$  satisfy $\tilde\classifier_A$ but not $\tilde\classifier_B$, and $\secondfeatures$ satisfy $\tilde\classifier_B$ but not $\tilde\classifier_A$.
If the former case is true, then such attributes are also accepted by the informed random order mechanism and hence are also in set  $I$.
If the latter case is true, then we can infer that such a strategy is profitable, i.e., $1-c(\features,\firstfeatures,\secondfeatures)\geq 0$.
This implies that such attributes can use a strategy $(\features,\firstfeatures,\firstfeatures)$ in the informed random order mechanism, the expected utility is $(1-q)[1-\onecost(\features,\firstfeatures)]\geq 0$, because $\onecost(\features,\firstfeatures)\leq c(\features,\firstfeatures,\secondfeatures)$ by monotonicity.
Hence such attributes are  accepted by the informed random order mechanism with probability at least $1-q$.
Therefore, such attributes are also in set  $I$.



\paragraph{Step 2} By the first characterization, we know that $\setonetwo_1\cup \setO$ contains the set of attributes that (1) are accepted by the fixed order mechanism  $(\tilde\classifier_A,\tilde\classifier_B,1)$, and (2) satisfy neither $\tilde\classifier_A$ nor $\tilde\classifier_B$.
Call the latter set set $F_2$.

Notice that $\setonetwo_1\cup \setO$ is convex and the qualified region is convex.
To show that $F_2$ is contained in the qualified region, 
it suffices to show that the extreme points of $\setonetwo_1\cup \setO$ are qualified.
This is true because the extreme points of $\setonetwo_1\cup \setO$ are in set $F_1$.
To see this, denote the intersecting point of the boundary lines of $ \tilde\classifier_A$ and $\tilde\classifier_B$ by $\tilde O$.
The extreme points of $\setonetwo_1\cup \setO$ are $\tilde O$, $\tilde\pointonetwo_1$ and another point in $ \tilde\classifier_B$.

\paragraph{Step 3}
 we introduce a mixed mechanism that mixes two fixed-order mechanisms as follows:
     announce the fixed-order mechanism $(\tilde\classifier_A,\tilde\classifier_B,1)$ with probability $q$; and announce the other fixed-order mechanism $(\tilde\classifier_A,\tilde\classifier_B,0)$ with probability $1-q$. 
     Such a mixed mechanism only accepts qualified agent.
    
    We first show that this mixed fixed-order mechanism is better than the random-order mechanism.  
    More specifically, this mixed fixed-order mechanism accepts all attributes that are accepted by the random-order mechanism with weakly higher probability.
    Suppose the agent is accepted with probability one in the random-order mechanism. Then, this agent must use a one-step strategy and provide attributes that satisfy both $\tilde \classifier_A$ and $\tilde \classifier_B$ in the first test. In this case, the agent can always get accepted in both fixed-order mechanisms by adopting the same strategy. Thus, this agent is accepted with probability one in the mixed mechanism. 
    
    Suppose the agent gets accepted with probability $q$ in the random-order mechanism. This means the agent chooses a two-step strategy that first provides attributes $\firstfeatures$ that satisfy only test $\tilde \classifier_A$ and then provides $\secondfeatures$ that satisfy another test $\tilde \classifier_B$. 
    The expected utility of this agent is $q- d(\features,\firstfeatures)- q\times d(\firstfeatures,\secondfeatures) \geq 0$.  
    Since the cost is non-negative, we have $1- d(\features,\firstfeatures)-  d(\firstfeatures,\secondfeatures) \geq 0$.
    In the fixed-order mechanism $(\tilde\classifier_A,\tilde\classifier_B,1)$, this agent can adopt the same strategy to get utility $1- d(\features,\firstfeatures)-  d(\firstfeatures,\secondfeatures) \geq 0$.
    Thus, this agent gets accepted with probability at least $q$ in the mixed mechanism. 
    Similarly, we can show that for any agent who is accepted by the random-order mechanism with probability $1-q$ is also accepted with probability at least $1-q$ in the mixed mechanism.
    Therefore, the mixed fixed-order mechanism is no worse than the random-order mechanism. 
    
    Finally, it is easy to see that one (the better one) of the two fixed-order mechanisms is no worse than the mixed mechanism.
    

%     Denote the angle between $ \tilde\classifier_A$ and $\classifier_B$ by $\theta( \tilde\classifier_A,\classifier_B)$.
%     Denote the angle between $ \classifier_A$ and $\tilde\classifier_B$ by $\theta( \classifier_A,\tilde\classifier_B)$.

% \paragraph{Step 1:} Any feasible simultaneous mechanism $(\tilde \classifier_A,\tilde \classifier_B)$ must satisfy $\theta( \tilde\classifier_A,\classifier_B)\leq\theta$ and $\theta( \classifier_A,\tilde\classifier_B)\leq \theta$.

% Suppose $\theta( \tilde\classifier_A,\classifier_B)>\theta$. 
% This implies that the boundary lines of $\tilde \classifier_A$ and $ \classifier_A$ intersect at some point that is in $ \classifier_A\cap \classifier_B$.
% This further implies that $\tilde \classifier_A\cap \tilde \classifier_B$ includes some attributes that are outside $ \classifier_A\cap \classifier_B$.
% The informed random order mechanism  $(\tilde\classifier_A,\tilde\classifier_B,q,\test_1)$ cannot be feasible. A contradiction.
% Analogously, we can show that $\theta( \classifier_A,\tilde\classifier_B)>\theta$ is not possible.

% \paragraph{Step 2:}


\end{proof}

%---------------------------------------------
%------------proofs of lemmas for thm 1-------
\begin{proof}[Proof of \cref{lem:feasible-uninformed-rand-distance cost}]

We first show that two fixed-order mechanisms $(\tilde\classifier_A,\classifier_B',1)$ and $(\classifier_A',\tilde\classifier_B,0)$ are feasible.
Consider the fixed-order mechanism $(\tilde\classifier_A,\classifier_B',1)$.
By the characterization in Section~\ref{sec: characterization BR}, the manipulation set for this mechanism consists of sets $\setO$, $\setonetwo_1(\tilde\classifier_A,\classifier_B')$, $\setperp_A(\tilde\classifier_A,\classifier_B')$, and $\setperp_B(\tilde\classifier_A,\classifier_B')$.
Since the uninformed random-order mechanism $(\tilde\classifier_A,\tilde\classifier_B,q,\test_1)$ is feasible, we have the intersection point $O$ has a distance at least $1/\eta$ to the boundary of both classifiers $\tilde\classifier_A$ and $\tilde\classifier_B$, otherwise there exists unqualified agent can directly move to point $O$ with cost $1$.
Note that the classifier $\classifier_B'$ has the same normal vector $\weights_B$ as the classifier $\classifier_B$. 
Thus, any attributes in $\tilde \classifier_A \cap \classifier_B'$ have a distance at least $1/\eta$ to the boundary of $\classifier_A$ and $\classifier_B$.
Therefore, we have $\setperp_B(\tilde \classifier_A, \classifier_B', 1)$ and $\setO$ are contained in $\classifier_A \cap \classifier_B$.
By Lemma~\ref{lem: Cq}, we have $\setonetwo_1(\tilde \classifier_A, \classifier_B') = OE_1\tilde E_1 E_1'$.
Note that the boundary of $\classifier_B'$ is parallel to the boundary of $\classifier_B$.
Thus, we have attributes $E_1$ and $E_1'$ are
contained in $\classifier_A \cap \classifier_B$.
The attributes $\tilde E_1$ is the intersection of $\tilde\classifier_A$ and $\classifier_B$, which implies the attributes $\tilde E_1$ is also in $\classifier_A \cap \classifier_B$.
Since the qualified region is convex, we have the set $\setonetwo_1 \subset \classifier_A \cap \classifier_B$.
Thus, the fixed-order mechanism $(\tilde\classifier_A,\classifier_B',1)$ is feasible. 
With a similar analysis, we have $(\classifier_A', \tilde\classifier_B, 0)$ is also feasible. 

We next show that one of these two fixed-order mechanisms is no worse than the uninformed random-order mechanism. 
Suppose the boundaries of two tests $\tilde \classifier_A$ and $\tilde \classifier_B$ are parallel to the boundaries of $\classifier_A$ and $\classifier_B$ respectively. 
By Proposition~\ref{prop: coverage of Mq}, in this case, the manipulation set of the uninformed random-order mechanism $(\tilde\classifier_A,\tilde\classifier_B,q,\varnothing)$ is contained in the manipulation set of one fixed-order mechanism.

We now consider the case where the boundaries of $\tilde\classifier_A$ and $\tilde\classifier_B$ are not parallel to the boundaries of $\classifier_A$ and $\classifier_B$.
We show that a mixed mechanism that announces the fixed-order mechanism  $(\tilde\classifier_A,\classifier_B',1)$ with probability $q$ and the fixed-order mechanism $(\classifier_A', \tilde\classifier_B, 0)$ with probability $1-q$ is no worse than the uninformed random order mechanism $(\tilde\classifier_A,\tilde\classifier_B,q,\varnothing)$.
By Lemma~\ref{lem: Cq}, we have $\setonetwo_q(\tilde \classifier_A, \tilde \classifier_B) = OE_q \tilde E_q E_q'$. Since $\tilde E_q$ is on the boundary of $\tilde\classifier_A$, we must have $\tilde E_q$ is in $O \tilde E_1$, otherwise $\tilde E_q$ is not qualified. 
We also know that $E_q$ and $E_q'$ have a distance $q$ to the intersection $O$, which means $E_q$ and $E_q'$ are accepted by the fixed-order mechanism. 
Note that the set of attributes $\calA_1$ accepted by the fixed-order mechanism $(\tilde\classifier_A, \classifier_B', 1)$ is convex. 
Since attributes $O$, $E_q$, $\tilde E_q$, and $E_q'$ are contained in $\calA_1$, we have $\setonetwo_q(\tilde \classifier_A, \tilde \classifier_B) = OE_q\tilde E_qE_q'$ is contained in $\calA_1$.
Similarly, we have the set $\settwoone_q$ is contained in the set of attributes $\calA_2$ accepted by the fixed-order mechanism $(\classifier_A',\tilde \classifier_B,0)$.
If an agent is accepted by the random order mechanism $(\tilde\classifier_A, \tilde \classifier_B,q,\varnothing)$ with probability $1$, then the attributes of this agent must be in $\setO \cup (\tilde\classifier_A \cap \tilde \classifier_B) \cup \setperp_A(\tilde\classifier_A, \tilde \classifier_B) \cup \setperp_B(\tilde\classifier_A, \tilde \classifier_B)$. 
Note that $\setO$, $\tilde\classifier_A \cap \tilde \classifier_B$, $\setperp_A(\tilde\classifier_A, \tilde \classifier_B)$, and $\setperp_B(\tilde\classifier_A, \tilde \classifier_B)$ are all contained by both $\calA_1$ and $\calA_2$. 
Thus, these agents are also accepted with probability $1$ in the mixed fixed-order mechanism. 
If an agent is accepted by the random order mechanism $(\tilde\classifier_A, \tilde \classifier_B,q,t_1)$ with probability $q$, then the attributes of this agent must be in $\setonetwo_q(\tilde\classifier_A,\tilde\classifier_B)$, which is contained in $\calA_1$.
Thus, this agent is accepted with probability at least $q$ in the mixed mechanism.
Similarly, if the agent is accepted by the random order mechanism $(\tilde\classifier_A, \tilde \classifier_B,q,\varnothing)$ with probability $1-q$, then the attributes of this agent must be in $\settwoone_q(\tilde\classifier_A,\tilde\classifier_B)\subset \calA_2$.
Thus, this agent is accepted by the mixed mechanism with probability at least $1-q$.
Since one of the two fixed-order mechanisms is no worse than the mixed mechanism, we get the conclusion.
\end{proof}


We now prove our main theorem. 

\begin{proof}[Proof of Theorem~\ref{thm: optimal max qualified}]
    By \cref{lem:feasible-informed-rand-distance cost} and \cref{lem:feasible-uninformed-rand-distance cost},  we have the best sequential mechanism is a fixed order mechanism. Moreover, to avoid selecting any unqualified agent, we must have $\tilde \classifier_A \cap \tilde \classifier_B \subset \classifier_A \cap \classifier_B$.
\end{proof}

\subsection{Omitted proof in \cref{subsec: simultaneous manipulation}}
\begin{proof}[Proof of \cref{lem:fix-simultaneous-distance cost}]
      This result is implied by \cref{thm:opt_manipulation}.
\end{proof}

\subsection{Omitted proof in \cref{subsec:cheap talk manipulation}}\label{appendix:cheap talk manipulation}

First we state the following two lemmas (partially), which characterize the set of attributes that can get selected by these two fixed-order mechanisms.


\begin{lemma}\label{lem:gain non-parallel tests}
 Under the fixed-order procedure $(\classifier_A^+,\widehat\classifier_B,0)$, we have the following:
 \begin{itemize}
     \item Each type in the triangle $\Updelta\text{OAB}$ has a profitable strategy to get selected.
     \item No unqualified type has a strictly profitable strategy to get selected.
 \end{itemize}
\end{lemma}

\begin{lemma}\label{lem:loss non-parallel tests}
 Under the fixed-order procedure $(\classifier_A^+,\classifier_B^+,1)$, we have the following:
 \begin{itemize}
     \item Each type in $\classifier_A\cap\classifier_B\setminus \Updelta\text{OAB}$ has a profitable strategy to get selected.
     \item Each type in the triangle $\Updelta\text{OAB}$ does not have a profitable strategy to get selected.
     \item No unqualified type has a strictly profitable strategy to get selected.
 \end{itemize}
\end{lemma}

\begin{proof}[Proof of \cref{thm: optimal max qualified cheap talk}]
The fixed-order mechanism is described as the following: 
If the agent reports a type in the triangle $\Updelta \text{OAB}\subset \classifier_A\cap\classifier_B$, then the principal announces the fixed-order mechanism $(\classifier_A^+,\widehat\classifier_B,0)$.
 If the agent reports a type in $ \classifier_A\cap\classifier_B\setminus \Updelta \text{OAB}$, then the principal announces the fixed-order mechanism $(\classifier_A^+,\classifier_B^+,1)$. 
 If the agent reports a type that is not in $ \classifier_A\cap\classifier_B$, then the principal randomly announces either $(\classifier_A^+,\widehat\classifier_B,0)$ or $(\classifier_A^+,\classifier_B^+,1)$.

First, we show that this fixed-order mechanism is feasible. 
By \cref{lem:gain non-parallel tests} and \cref{lem:loss non-parallel tests}, no unqualified type has a profitable strategy to get selected under either $(\classifier_A^+,\widehat\classifier_B,0)$ or $(\classifier_A^+,\classifier_B^+,1)$.
Hence no unqualified type has a profitable strategy to get selected regardless of the reporting strategy.

Next we show that under this fixed-order mechanism, every qualified agent is selected.
By \cref{lem:gain non-parallel tests} and \cref{lem:loss non-parallel tests}, each type in the triangle $\Updelta \text{OAB}$ has a profitable strategy to get selected by the fixed-order mechanism $(\classifier_A^+,\widehat\classifier_B,0)$ but not in the fixed-order mechanism $(\classifier_A^+,\classifier_B^+,1)$.
Hence each type in the triangle $\Updelta \text{OAB}$ has incentive to truthfully report his type and gets selected. Again, by \cref{lem:loss non-parallel tests}, each type in $\classifier_A\cap\classifier_B\setminus \Updelta\text{OAB}$ has a profitable strategy to get selected by the fixed-order mechanism $(\classifier_A^+,\classifier_B^+,1)$ if he reports truthfully.\footnote{Here some types in $\classifier_A\cap\classifier_B\setminus \Updelta\text{OAB}$ might have incentive to misreport if they can get selected by the fixed-order mechanism $(\widehat\classifier_A,\classifier_B^+,1)$ with a lower cost. }
\end{proof}