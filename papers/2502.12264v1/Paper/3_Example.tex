\section{An example and preliminary analysis}\label{subsec:example}
In this section, we provide a stylized example to motivate our model. We discuss other applications in \cref{sec:applications}. Suppose a biotech company X is seeking candidates for a senior scientist
position. Senior scientists must balance technical expertise and domain
knowledge with managerial skills and leadership. Suppose X is looking for
candidates that have some management experience and especially strong
scientific expertise. Suppose, not unrealistically, that excessive
experience as a manager may be an indication of career goal as a manager
rather than a scientist, and even be a signal of declining technical
expertise.

We use \cref{fig: hiring} to illustrate X's requirements. The horizontal axis 
represents managerial experience and the vertical axis  represents
scientific experience. The blue vertical line represents the minimum
requirement on managerial experience, say managing one direct report. The
red diagonal line represents the joint requirement on both aspects. The purple dashed region represents the set of
attributes that X's considers as qualified for senior scientist position.


\begin{figure}[h] 
\centering 
\begin{tikzpicture}[xscale=7.8,yscale=7.8]
\draw [thick,->] (0.6,0.8) -- (1.5,0.8);
\draw [thick, ->] (0.9,0.66) -- (0.9,1.33);
\node [below] at (1.5,0.8) {$\feature_A=$managerial experience};
\node [above] at (0.9,1.33) {$\feature_B=$scientific experience};

\draw [domain=0.66:1.4, thick, red] plot (\x, {3/4*\x+1/4});
\draw [thick, blue] (1,0.66) -- (1,5.2/4);
% \node [left] at (1.3, 1.25 ) {$\classifier_B$};
% \node [right] at (1, 1.25) {$\classifier_A$};
\node [left] at (1.2,1.15) {$+$};
\node [right] at (1, 1.15) {$+$};
\node [left,font=\tiny] at (0.9, 0.78 ) {\footnotesize$O$};
% \fill [purple!60,nearly transparent] (1,5.2/4) -- (1,1)-- (1.4,5.2/4)-- cycle;
\path[pattern color=purple,pattern=north west lines] (1,5.2/4) -- (1,1)-- (1.4,5.2/4); 
% \node [blue] at (1, 1 ) {\textbullet};
% \node [right,font=\tiny] at (1, 1 ) {\footnotesize$(1,5)$};

\end{tikzpicture}
\caption{Hiring requirements for a senior scientist position}
\label{fig: hiring}
  \end{figure}
  
 Past experience in other biotech companies can reflect a candidate's
 suitability for the position. The resume provides some information, but does not fully reveal which of the two functions (manager or scientist) was the candidate's main duty. So, the company finds a verbal interview helpful to assessing the candidates' qualifications. 
Consider two types of hiring
procedures: one consists of resume screening, followed by an interview of
each candidate who was not rejected in reviewing resumes (sequential
procedure), and a joint procedure in which a committee reviews resumes and
interviews candidates at the same time (simultaneous procedure).

Candidates can obviously tailor their resumes and answers for common
interview questions to influence the interviewers' impression of their
profiles. So, we analyze this example within our manipulation setting.
Suppose that the cost of tailoring is $\onecost(\orifeatures,\features)$, where $%
\orifeatures=(x_{A}^{0},x_{B}^{0})$ are the true attributes and $\features=(\feature_{A},\feature_{B})$ are
the attributes tailored for the purpose of hiring, is equal to the Euclidean
distance between $\orifeatures$ and $\features$. Under the sequential procedure, if a
candidate chooses to tailor his resume, for example, to emphasize managerial
experience, then he is constrained during the interview. He cannot present
himself in a way that contradicts his resume. Hence, the total cost of
manipulating is $\onecost(\orifeatures,\firstfeatures)+\onecost(\firstfeatures,\secondfeatures)$, where $\firstfeatures$ stands for
attributes presented in the resume and $\secondfeatures$ stands the attributes presented
in the interview. The candidate's payoff is the probability of being hired
minus the cost of tailoring, be it one-step or two-step.

Suppose that as in our model, the company's preferences are lexicographic:
avoiding unqualified applicants is a primary objective, perhaps because
hiring an unqualified candidate is very costly. For the sake of execution,
the company must set some specific cutoffs in testing in practice. The
question the company faces is whether to use a simultaneous or a sequential
procedure (i.e., mechanism).\bigskip 

% \subsection{Simultaneous mechanisms}

% Suppose for a moment that the company decides to apply a simultaneous
% mechanism. It announces its criteria by posting that candidates who can
% demonstrate that their managerial experience and scientific expertise are
% determined to belong to $h_{A}\cap h_{B}$ will be hired. Recall that $%
% h_{A}\cap h_{B}$ is the region of attributes that the company finds qualified.

% This implies that any candidate with true attributes $\orifeatures\in h_{A}\cap h_{B}$
% will not manipulate and will be selected. In addition, candidates with true
% attributes $\orifeatures\notin h_{A}\cap h_{B}$ will be selected only if they pay a
% preparation cost to adopt attributes that are in the qualified region. Since
% the benefit of doing so (i.e., getting selected) is 1, only those with cost
% no greater than 1 will be willing to do so. The set of candidates who are
% willing to do such preparation is 
% \[
% \manipulation=\left\{ \orifeatures\notin h_{A}\cap h_{B}:\min_{\features\in h_{A}\cap
% h_{B}}\metric(\orifeatures,\features)\leq 1\right\} \text{,}
% \]%
% where $\metric(\cdot,\cdot)$ is the Euclidean distance.
% This set $\manipulation$ comprises all points whose distance to either the two edges of
% the qualified region $h_{A}\cap h_{B}$ is no greater than 1.
% % and is depicted in blue in Panel (a) of Figure 3.

% The simultaneous mechanism that passes candidates in the qualified region $%
% h_{A}\cap h_{B}$ selects some unqualified candidates, so does not satisfy
% the company's objective. In order not to select any unqualified candidate,
% the company must choose a more stringent test. For example, such a more
% stringent test would be $h_{A}^{+}\cap h_{B}^{+}$, where $h_{i}^{+}$ is
% obtained by a parallel shift of $h_{i}$ by a distance of 1. It turns out, and is fairly easy to verify that $h_{A}^{+}\cap
% h_{B}^{+}$ is the optimal simultaneous mechanism.\bigskip 

% \subsection{Fixed-order sequential mechanisms}\label{subsec:ex fixed order}

% Suppose now that the company decides to apply a sequential mechanism that
% screens managerial experience first by reviewing resumes, and next interview
% the candidates who pass the resume screening. In other words, the sequential
% procedure is a two-stage screening: (1) first select candidates whose
% attributes belong to $h_{A}$ (who pass test $h_{A}$), and (2) among those
% selected in (1), those whose attributes belong to $h_{B}$ will be eventually
% hired.

% As before, any candidate with true attributes $\orifeatures\in h_{A}\cap h_{B}$ will
% not manipulate and will be selected. However, the sequential mechanism with
% fixed order of tests has enriched the candidates' strategy space. Consider
% any candidate with unqualified true attributes $\orifeatures\notin h_{A}\cap h_{B}$.
% He has now two options of getting hired: (1) to adopt attributes that are in
% the qualified region, and (2) to adopt attributes $\firstfeatures\in h_{A}$ for the
% first test, and after passing the first test, to adopt yet other attributes $%
% \secondfeatures\in h_{B}$. To illustrate the two types of strategies, consider a
% candidate with attributes $\orifeatures$ in \cref{fig: zig zag}. The least costly one-step
% strategy that guarantees selection is to move to the  point $O$, at which
% the boundaries of $h_{A}$ and $h_{B}$ intersect. The utility of adopting
% this one-step strategy is $1-\onecost(\orifeatures,O)$.

% \begin{figure}[t]
% \centering
% \begin{tikzpicture}[xscale=10,yscale=10]

% \draw [domain=0.66:1.36, thick] plot (\x, {3/4*\x+1/4});
% \node [left] at (1.32, 1.25 ) {$\classifier_B$};%:\feature_2\geq \frac34 \feature_1 +\frac14
% \draw [thick] (1,0.66) -- (1,1.25);
% \node [right] at (1, 1.25 ) {$\classifier_A$};%: \feature_1 \geq 1$
% \node [left] at (1.27,1.2) {$+$};
% \node [right] at (1, 1.2) {$+$};

% \draw [domain=1+0.25*0.6:1.45, loosely dashed] plot (\x, {3/4*\x+1/4-5/16}); % 1/c = 1/4
% \draw [red,ultra thick] (1,1) -- (1+0.25*0.6,1-0.25*0.8);
% \node [right] at (1+0.25*0.6,1-0.25*0.8) {\footnotesize$B$};
% \node [left] at (1-0.25*0.6,1-0.25*0.8) {\footnotesize$B'$};

% \draw [loosely dashed] (1-0.25,1) -- (1-0.25,1.25);

% \draw [loosely dashed] (1-0.25,1) -- (1,1);
% \node [left] at (1-0.25, 1 ) {\footnotesize$A$};

% \draw[blue,ultra thick] (1-0.25,1) arc (180:233.1:0.25);
% \draw[blue,ultra thick] (1,1) -- ++(180:0.25);
% \draw[blue,ultra thick] (1,1) -- ++(233.1:0.25) ;

% \draw [red,ultra thick] (1,1) -- (1-0.25*0.6,1-0.25*0.8);
% \draw [ red,ultra thick] (1,11/16) -- (1-0.25*0.6,1-0.25*0.8);
% \node [left] at (0.98,11/16) {\footnotesize$C$};
% \draw [ red,ultra thick] (1,11/16) -- (1+0.25*0.6,1-0.25*0.8);

% \draw [dotted, ultra thick] (1-0.1,1-0.2) -- (1+0.1,1-0.2);
% \node [below] at (1-0.1, 1-0.2 ) {\footnotesize$\orifeatures$};
% \draw [dotted,ultra thick]  (1+0.1,1-0.2) -- (1+0.1-5.45/24*0.6,1-0.2+5.45/24*0.8);
% \node [below] at (1+0.1,1-0.2) {\footnotesize$\tilde\features$};
% \node [left] at  (1+0.1-5.5/24*0.6,1-0.2+5.5/24*0.8) {\footnotesize$\secondfeatures$};
% \draw [dotted,ultra thick] (1-0.1,1-0.2) -- (1,1-0.072);
% \node [right] at (1, 1-0.072 ) {\footnotesize$\firstfeatures$};

% \node [above] at (1.016, 1 ) {\footnotesize$O$};
% \end{tikzpicture}
% \caption{Two-step strategy: $\orifeatures\rightarrow \firstfeatures\rightarrow \secondfeatures$}
% \label{fig: zig zag}
% \end{figure}


% What is the least costly two-step strategy such that first attributes $%
% \firstfeatures\in h_{A}$ and next attributes $\secondfeatures\in h_{B}$ are attained? This question has been solved by \citet{zigzag}.  Here we provide a geometric intuition for our proof. Observe that
% the two chosen attributes $\firstfeatures$ and $\secondfeatures$ must belong to the boundary of $%
% h_{A}$ and $h_{B}$, respectively. Reflect point $\orifeatures$ over line $h_{A}$.
% This gives point $\widetilde{\features}$ in \cref{fig: zig zag}. Under the additive costs $%
% \cost(\orifeatures,\firstfeatures,\secondfeatures)=\onecost(\orifeatures,\firstfeatures)+\onecost(\firstfeatures,\secondfeatures)$, where $\onecost(\features^{k},x^{l})$
% is equal to the Euclidean distance between $\features^{k}$ and $\features^{l}$, 
% \[
% \cost(\orifeatures,\firstfeatures,\secondfeatures)=\cost(\widetilde{\features},\firstfeatures,\secondfeatures)\text{,}
% \]%
% and by the triangle inequality, $\cost(\widetilde{\features},\firstfeatures,\secondfeatures)$ is the
% shortest when $\widetilde{\features}$, $\firstfeatures$ and $\secondfeatures$ are co-linear.\footnote{This is also equivalent to the following shortest path problem in optics.
% Suppose there is a light emitting from point $\orifeatures$.
% The light has to first reach mirror $A$ (line $A$) before it reaches mirror $B$ (line $A$). We find the shortest path in a similar way.}

% In order not to select any unqualified agent, the company needs to choose
% more stringent tests instead of $h_{A}$ and $h_{B}$. It turns out that depending on the distribution, different stringent tests 
% could be optimal for
% sequential mechanisms. We postpone the discussion to  \cref{sec: distance cost}.

% \subsection{A comparison of simultaneous and fixed-order sequential
% mechanisms}

% But which mechanism: simultaneous $h_{A}^{+}\cap h_{B}^{+}$ or fixed-order sequential $%
% h_{A}^{+}$ and $h_{B}^{+}$ is preferred by the principal? This
% problem has been posed by \citet{zigzag}; however, the statement of their main result (Theorem 4.4) is inaccurate and there is an error in their proof, which we correct in \cref{thm:opt_manipulation} (\cref{sec: distance cost}). To prepare the readers for the main analysis, we provide a simple analysis here.
% Observe that under the fixed-order sequential procedure $h_{A}$ and $h_{B}$,
% there exists some types such that the least costly two-step strategy of the agent is always cheaper than the
% least costly one-step strategy (for example, any type in $OBCB'$ in \cref{fig: zig zag}). As we have already noticed, for these types, the cost of the
% former strategy is $\cost(\widetilde{\features},\firstfeatures,\secondfeatures)$ when $\widetilde{\features}$, $%
% \firstfeatures$ and $\secondfeatures$ are co-linear. The cost of the latter strategy is $%
% \onecost(\orifeatures,O)=\onecost(\widetilde{\features},O)$. And since points $\widetilde{\features}$, $\secondfeatures$
% and $O$ form a triangle, $\onecost(\widetilde{\features},O)>\cost(\widetilde{\features},\firstfeatures,\secondfeatures)$
% by triangle inequality, unless $\secondfeatures=O$ (see points $B$ and $B^{\prime }$
% in \cref{fig: zig zag}), in which case $\onecost(\widetilde{\features},O)=\cost(\widetilde{\features},\firstfeatures,\secondfeatures)$%
% .

% This analysis implies that by using the fixed-order sequential mechanism $%
% h_{A}^{+}$ and $h_{B}^{+}$, the company hires more candidates than
% by using the simultaneous mechanism $h_{A}^{+}\cap h_{B}^{+}$. This is
% because the fixed-order sequential mechanism enables candidates to use two-step
% strategies, which are cheaper (sometimes strictly) than the best one-step
% strategy. 
% It remains to check that all candidates hired by the fixed-order sequential mechanism are qualified under tests $%
% h_{A}^{+}$ and $h_{B}^{+}$.
% In this particular example, this is indeed the case.
% However, in general, for tests feasible for the simultaneous mechanism, they are not necessarily feasible for the fixed-order sequential mechanisms, i.e., the latter may select some unqualified candidates under the same tests.
% We postpone the more involved analysis to \cref{sec: distance cost}.
\paragraph{Preliminary analysis}
Next we provide a reflection argument, which is the key to characterize the set of attributes that are selected under any fixed-order mechanism. Suppose now the principal uses a sequential mechanism that (1) offers test $h_{A}$ at the first stage, and (2) among those
selected in (1), those whose attributes belong to $h_{B}$ will be eventually
selected.

Notice that any agent with true attributes $\orifeatures\in h_{A}\cap h_{B}$ will
not manipulate and will be selected.  Consider
any agent with unqualified true attributes $\orifeatures\notin h_{A}\cap h_{B}$.
He has now two options of getting selected: (1) to adopt attributes that are in
the qualified region, and (2) to adopt attributes $\firstfeatures\in h_{A}$ for the
first test, and after passing the first test, to adopt yet other attributes $%
\secondfeatures\in h_{B}$. To illustrate the two types of strategies, consider an agent with attributes $\orifeatures$ in \cref{fig: zig zag}. The least costly one-step
strategy that guarantees selection is to move to the  point $O$, at which
the boundaries of $h_{A}$ and $h_{B}$ intersect. The utility of adopting
this one-step strategy is $1-\onecost(\orifeatures,O)$.

\begin{figure}[t]
\centering
\begin{tikzpicture}[xscale=10,yscale=10]

\draw [domain=0.66:1.36, thick] plot (\x, {3/4*\x+1/4});
\node [left] at (1.32, 1.25 ) {$\classifier_B$};%:\feature_2\geq \frac34 \feature_1 +\frac14
\draw [thick] (1,0.66) -- (1,1.25);
\node [right] at (1, 1.25 ) {$\classifier_A$};%: \feature_1 \geq 1$
\node [left] at (1.27,1.2) {$+$};
\node [right] at (1, 1.2) {$+$};

\draw [domain=1+0.25*0.6:1.45, loosely dashed] plot (\x, {3/4*\x+1/4-5/16}); % 1/c = 1/4
\draw [red,ultra thick] (1,1) -- (1+0.25*0.6,1-0.25*0.8);
\node [right] at (1+0.25*0.6,1-0.25*0.8) {\footnotesize$B$};
\node [left] at (1-0.25*0.6,1-0.25*0.8) {\footnotesize$B'$};

\draw [loosely dashed] (1-0.25,1) -- (1-0.25,1.25);

\draw [loosely dashed] (1-0.25,1) -- (1,1);
\node [left] at (1-0.25, 1 ) {\footnotesize$A$};

\draw[blue,ultra thick] (1-0.25,1) arc (180:233.1:0.25);
\draw[blue,ultra thick] (1,1) -- ++(180:0.25);
\draw[blue,ultra thick] (1,1) -- ++(233.1:0.25) ;

\draw [red,ultra thick] (1,1) -- (1-0.25*0.6,1-0.25*0.8);
\draw [ red,ultra thick] (1,11/16) -- (1-0.25*0.6,1-0.25*0.8);
\node [left] at (0.98,11/16) {\footnotesize$C$};
\draw [ red,ultra thick] (1,11/16) -- (1+0.25*0.6,1-0.25*0.8);

\draw [dotted, ultra thick] (1-0.1,1-0.2) -- (1+0.1,1-0.2);
\node [below] at (1-0.1, 1-0.2 ) {\footnotesize$\orifeatures$};
\draw [dotted,ultra thick]  (1+0.1,1-0.2) -- (1+0.1-5.45/24*0.6,1-0.2+5.45/24*0.8);
\node [below] at (1+0.1,1-0.2) {\footnotesize$\tilde\features$};
\node [left] at  (1+0.1-5.5/24*0.6,1-0.2+5.5/24*0.8) {\footnotesize$\secondfeatures$};
\draw [dotted,ultra thick] (1-0.1,1-0.2) -- (1,1-0.072);
\node [right] at (1, 1-0.072 ) {\footnotesize$\firstfeatures$};

\node [above] at (1.016, 1 ) {\footnotesize$O$};
\end{tikzpicture}
\caption{Two-step strategy: $\orifeatures\rightarrow \firstfeatures\rightarrow \secondfeatures$}
\label{fig: zig zag}
\end{figure}


What is the least costly two-step strategy such that first attributes $%
\firstfeatures\in h_{A}$ and next attributes $\secondfeatures\in h_{B}$ are attained? This question has been solved by \citet{zigzag} under the same additive Euclidean cost function.  Here we provide a geometric intuition. Observe that
the two chosen attributes $\firstfeatures$ and $\secondfeatures$ must belong to the boundary of $%
h_{A}$ and $h_{B}$, respectively. Reflect point $\orifeatures$ over line $h_{A}$.
This gives point $\widetilde{\features}$ in \cref{fig: zig zag}. Under the additive costs $%
\cost(\orifeatures,\firstfeatures,\secondfeatures)=\onecost(\orifeatures,\firstfeatures)+\onecost(\firstfeatures,\secondfeatures)$, where $\onecost(\features^{k},x^{l})$
is equal to the Euclidean distance between $\features^{k}$ and $\features^{l}$, 
\[
\cost(\orifeatures,\firstfeatures,\secondfeatures)=\cost(\widetilde{\features},\firstfeatures,\secondfeatures)\text{,}
\]%
and by the triangle inequality, $\cost(\widetilde{\features},\firstfeatures,\secondfeatures)$ is the
shortest when $\widetilde{\features}$, $\firstfeatures$ and $\secondfeatures$ are co-linear.\footnote{This is also equivalent to the following shortest path problem in optics.
Suppose there is a light emitting from point $\orifeatures$.
The light has to first reach mirror $A$ (line $A$) before it reaches mirror $B$ (line $A$). We find the shortest path in a similar way.}

Observe that 
there exists some types such that the least costly two-step strategy of the agent is always cheaper than the
least costly one-step strategy (for example, any type in $OBCB'$ in \cref{fig: zig zag}). As we have already noticed, for these types, the cost of the
former strategy is $\cost(\widetilde{\features},\firstfeatures,\secondfeatures)$ when $\widetilde{\features}$, $%
\firstfeatures$ and $\secondfeatures$ are co-linear. The cost of the latter strategy is $%
\onecost(\orifeatures,O)=\onecost(\widetilde{\features},O)$. And since points $\widetilde{\features}$, $\secondfeatures$
and $O$ form a triangle, $\onecost(\widetilde{\features},O)>\cost(\widetilde{\features},\firstfeatures,\secondfeatures)$
by triangle inequality, unless $\secondfeatures=O$ (see points $B$ and $B^{\prime }$
in \cref{fig: zig zag}), in which case $\onecost(\widetilde{\features},O)=\cost(\widetilde{\features},\firstfeatures,\secondfeatures)$%
.

Despite its equivalence to the algebraic argument provided by \citet{zigzag}, we  highlight that this geometric intuition is very convenient and powerful when characterizing the agent's best response in any sequential mechanism, especially those with randomization.