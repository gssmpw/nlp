\section{Introduction}
Assessing job candidates through interviews, evaluating banks' financial health, judging the legitimacy of firms' activities, and classifying data with algorithms are different forms of testing.
Testing is an imperfect means to evaluate another party's information, especially when such information is unverifiable. 
Such situations arise when the interested party can fabricate their type at a cost \citep{perez2022test,perez2024score,li2023screening}.
This type of information is termed semi-hard, lying between hard information, which cannot be fabricated \citep{green1986partially}, and soft information, which can be fabricated at no cost.\footnote{This is usually the case in classic mechanism design settings and in cheap-talk communication settings where each agent can freely report their own type.}

Within the context of testing, many real-world scenarios involve multiple principals, each with different criteria and the power to veto the participant being tested.
Examples include a hiring committee with multiple members and multiple agencies collaborating on regulations. 
  Monetary transfer is typically unavailable. 
  More realistically, the principal(s) can use two tools that are more restricted: (1) tests, which establish the standards for what is deemed acceptable, and (2) a testing procedure, which outlines how tests are conducted, either simultaneously (all tests conducted together) or sequentially (tests conducted one after another).  For example, governments regulate firms to ensure they act in the best interests of both consumers and shareholders. This regulatory task can be carried out either by a single agency evaluating both criteria simultaneously or by two separate agencies, each focusing on one criterion and evaluating them sequentially.\footnote{For example, the Japan Financial Services Agency (FSA) works in coordination with the the Japan Fair Trade Commission (JFTC) to ensure that M\&A activities are compliant with both securities and antitrust laws.}\footnote{This also includes the special case of one principal with multiple criteria. For example, data classification with multiple classifiers and bank regulation, where the regulator uses leverage and liquidity ratios to assess banks' financial health. In Europe, banks report these ratios separately, whereas in the U.S., banks report them together.}
% Knowing the screening rules, participants can strategically improve their apparent type to pass the tests. 

Given this, we ask: what are the optimal tests and testing procedure to select a strategic participant meeting multiple criteria? Specifically, the choice of testing procedures can also be viewed as the arrangement of agencies responsible for administering the tests. While \citet{perez2022test} have studied the optimal test when both the screening criterion (or the participant's type) and the test are one-dimensional, an interesting trade-off concerning the choice of testing procedure arises when both the type and the tests are multi-dimensional.\footnote{The problem is trivial when only one side (either the type or the tests) is multi-dimensional. See \cref{sec:discussion}.}
 % Hence, the core of our question can be stated as: what is the optimal arrangement of agencies to conduct tests? 
 

While manipulation, either by costly lying or by costly fabricating the truth, is common in many of these scenarios,\footnote{Examples include job applicants tailoring their profiles to stand out in interviews, firms misrepresenting their impacts to bypass regulatory scrutiny, or banks involving in `window dressing' by temporarily adjusting their financial statements right before the reporting dates to appear safer than they actually are. See \citet{perez2022test,perez2024score, hardt2016strategic} for more examples.} it is also common for participants to invest in improving their actual types. For instance, job candidates seeking promotion acquire new skills, while firms and banks may restructure their business models to meet regulatory requirements.
Therefore, we compare the optimal policy under two different technologies: (1) the participant misrepresents themselves (\emph{manipulation}), or (2) the participant improves their actual type (\emph{investment}).\footnote{In \cref{sec:discussion}, we consider allowing the participant to choose between manipulation and investment. We argue that it does not add to the analysis. }
% We find that when the participant improves their apparent performance (\emph{manipulation}), 
% the optimal testing procedure is sequential  with fixed order of tests, and the optimal tests have higher standard than the true criteria (\emph{stringent tests}); 
% while when the participant improves their actual performance (\emph{investment}), the optimal testing procedure is simultaneous and the optimal tests coincide with the true criteria.

% Consider the following examples.

% \begin{example}[Firm regulations]
%      Governments often regulate businesses to ensure they act in the best interests of both consumers and shareholders. This regulatory task can be carried out either by a single agency evaluating both criteria simultaneously or by two separate agencies, each focusing on one of the criteria, evaluating them sequentially. 
% \end{example}

% \begin{example}[Banking regulations]
%     Central banks use leverage and liquidity ratios to regulate banks' financial health. In Europe, banks are required to report the two ratios separately.\footnote{Empirical evidence \citep{window_dressing,ECB_window} suggests that European banks take advantage of this fact to temporarily manipulate liquid assets, termed `window dressing'.} In contrast, US banks are required to report these ratios together.
% \end{example}

In practice, we observe that  the arrangement of agencies is different in different settings. We argue that the critical difference is whether participants manipulate or invest in their type. To show this, we build on a stylized principal-agent framework introduced by \citet{zigzag} and conduct a comprehensive mechanism design analysis. 
The principal (hereafter she) only wants to select an agent (he) whose true (two-dimensional) type meets her criteria.
The principal uses two tests to judge the agent's true type, together with a testing procedure, which could be simultaneous, sequential with a fixed order or sequential with a random order combined with a disclosure decision.\footnote{\citet{zigzag} did not study all possible sequential mechanisms. In contrast, we extend the analysis to the entire class of sequential mechanisms (\cref{sec: distance cost}).}
The principal commits to select the agent if and only if he passes both tests.
The agent seeks to be selected and could change his type at a type-dependent cost before each test either by manipulation or by investment. When the type is changed multiple times, the cost adds up.


Within this framework, we first look for the optimal sequential mechanism.
Consider the manipulation setting.
Intuitively, to exclude unqualified agent, the principal could use either tests stricter than her true criteria (\emph{stringent} tests) or a testing procedure that makes manipulation more costly (a \emph{stringent} procedure), given that every agent can pay a type-dependent cost to inflate his type.
For example, a procedure with fixed order of tests is less stringent than one that randomizes over the sequence of tests.\footnote{In the evidence literature \citep[e.g.,][]{glazer2004optimal, sher2014persuasion}, randomization takes a different form and usually refers to which attribute to check because the principal can check only one attribute.}
The reason is as follows.
In any sequential mechanism, the agent can choose a strategy to pass one test (potentially failing another) at a time. 
While the way the agent manipulates his type to pass one test is quite different from that to pass another test, such a strategy results in a lower chance of selection when  the order of tests is sometimes secretly reversed.
Hence, the agent may switch to passing both tests together under randomization, which is more costly but guarantees selection.
% than passing one test at a time .
Similarly, not disclosing the first (realized) test after it takes place makes a random-order procedure even more stringent because the agent has to accomplish the second test even after he fails the first one.
While \citet{perez2022test,zigzag} show that stringent tests can enhance outcomes by shutting down the channel of testing procedures, it is unclear how tests and a testing procedure interact and which combination is optimal.
% Specifically, both randomization over the sequence of tests and the disclosure decision can make the testing procedure more difficult.
 
We show that a fixed-order sequential mechanism with stringent tests is optimal in the manipulation setting (\cref{thm: optimal max qualified}).
% Specifically, we examine the benefits of randomizing over the sequence of tests and the disclosure decision. In any sequential mechanism, the agent can choose a strategy to pass one test (potentially failing another) at a time. 
% However, the way the agent falsify his type to pass one test is quite different from that to pass another test.
% Thus, if the order of tests is sometimes secretly reversed, the agent may hesitate to falsify his type due to the uncertainty. 
% Surprisingly, we show that randomization is not optimal in the manipulation setting.
% To select only qualified agent, the principal has to choose stringent tests in the optimal mechanism because every agent can pay a type-dependent cost to inflate his type \citep{perez2022test,zigzag}.
This is because any testing procedure has to be combined with stringent tests to exclude any unqualified agent.
However, stringent tests also exclude some qualified agents, which needs to be compensated by a less stringent procedure.
% i.e., one with fixed order of tests.
% Under the same stringent tests, a procedure with fixed order of tests
%  selects more potential qualified agents than one with random order of tests, because it provides them cheaper ways to manipulate their types to pass the stringent tests.
This is a simplified intuition because it separates the choice of tests and testing procedures.
In general, the set of tests available under random-order procedures is larger than that under fixed-order procedures.  
 In the formal analysis, we offer a new constructive argument to show that, given any feasible random-order mechanism with arbitrary tests, there exists a feasible fixed-order mechanism that dominates it.
 To show dominance, we introduce a feasible mechanism that clearly dominates the random-order mechanism using a set inclusion argument and is dominated by a fixed-order mechanism using a probability argument.
 The key challenge comes from constructing such a feasible mechanism, which will be discussed in details in \cref{sec: distance cost}.
  We further show that this result generalizes to  \emph{perfect tests} (tests that output the agent's attributes; \cref{thm: sequential perfect tests}), and that when cheap-talk communication is allowed upfront, a fixed-order sequential mechanism can achieve the first best (\cref{thm: optimal max qualified cheap talk}).


% but may achieve optimum in the investment setting.
% In addition, our results imply that disclosing the test to the agent is optimal in the manipulation setting but not optimal in the investment setting.

 
% To elaborate, we show that a fixed-order sequential mechanism with stringent tests is optimal in the manipulation setting (\cref{thm: optimal max qualified}).
%  To select only qualified agent, the principal has to choose stringent tests in the optimal mechanism because  every agent can pay a type-dependent cost to inflate his type \citep{perez2022test,zigzag}.
% However, these stringent tests exclude some qualified agents.
% % Above, we have reasoned that holding tests fixed, no randomization introduces more manipulation behavior by eliminating strategic uncertainty.


%  However, we show that this intuition is incomplete because the set of tests available to the principal is larger in random-order sequential mechanisms than in fixed-order sequential mechanisms.  
%  In the formal analysis, we offer a new constructive argument to show that, given any feasible random-order mechanism, there exists a feasible fixed-order mechanism that dominates it.
%  The key idea is to introduce a mixed mechanism that is a mixture over two feasible fixed-order mechanisms.
%  To show dominance, first, we use a straightforward set inclusion argument to show that the mixed mechanism dominates the random-order mechanism.
% Second, we use a probability argument to show that one of the fixed-order mechanisms dominates the mixed mechanism.
%  The remaining challenge comes from showing feasibility.
%  It requires a
%  careful construction of two feasible fixed-order mechanisms, which will be discussed in more details in \cref{sec: distance cost}.
 % We further show that when cheap-talk communication is allowed upfront, there exists a fixed-order sequential mechanism that achieves the first best (\cref{thm: optimal max qualified cheap talk}), and that this result generalizes to  \emph{perfect tests} (tests that output the agent's attributes; \cref{thm: sequential perfect tests}).

 In contrast,  we show that under mild conditions, a random-order mechanism using  tests that coincide with the true criteria (\emph{non-stringent} tests) can achieve the first best in the investment setting (\cref{thm: true effort sequential}).
As discussed earlier, the agent may switch to passing both tests together facing randomization.\footnote{Whether to disclose the first test to the agent also affects his preference over different strategies. Detailed discussions are in \cref{sec: distance cost}.}
When this happens and when \emph{non-stringent} tests are used, the agent is selected if and only if he invests to some type satisfying the true criteria.
% In contrast, a fixed-order mechanism allows unqualified agents to pass one test at a time, which distorts their incentive to invest.
% As a result, stringent tests are still needed in fixed-order mechanisms, losing some agents who would have invested.  
We show that this result generalizes to other cost functions (\cref{thm:optimal sequential metric investement}).


 
The second question we ask is whether the simultaneous procedure can achieve a better outcome than any sequential one.
Consider bank regulations where the central bank sets minimum requirements on liquidity and leverage ratios.
Suppose the liquidity ratio is tested before the leverage ratio.
A bank can temporarily acquire more liquid assets before the first test to increase the liquidity ratio and after passing the first test, it can get rid of extra liquid assets to boost its leverage ratio for the second test.
However, if the two tests are conducted simultaneously, such a manipulation strategy would not be available.
This suggests that a simultaneous procedure is more stringent than a sequential one.
\citet{zigzag} argue that in the manipulation setting, there is a fixed-order sequential mechanism better than any simultaneous mechanism. We give a more precise proof of this claim (\cref{lem:fix-simultaneous-distance cost}).
On the other hand, we prove that in the investment setting, a simultaneous procedure combined with non-stringent tests can achieve the first best (\cref{thm:optimal investment}).
Since the agent is forced to pass both tests together in any simultaneous mechanism, for the same reason discussed earlier, everyone selected becomes qualified under non-stringent tests.
We also show that these results generalize to general cost functions (\cref{thm:opt_manipulation}), \emph{perfect tests} and convex qualified region (\cref{thm: perfect tests arbitrary qualified region}), and an alternative design objective (\cref{prop:Pii manipulation} and \ref{prop:Pii investment}).
% to select every qualified agent and avoid unqualified agent as much as possible; 

To summarize, we study the optimal tests and testing procedure involving multiple criteria. 
We conduct a non-traditional mechanism design analysis in the context of testing. 
Unlike the classic framework, we study how the stringency of the tests interact with the testing procedure. 
Our theoretical contribution is a new constructive argument to tackle this problem.











