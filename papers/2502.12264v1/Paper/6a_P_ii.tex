\subsection{Robustness to objective}\label{sec:Pii}
In this subsection, we consider the following alternative objective.
% \footnote{When $\alpha=\infty$, \ref{obj: weighted average} puts infinity weight on Type II error and it becomes \ref{min unqualified}.}
% When regulating innovations like business models or technological inventions that require more leniency, the principal wants to minimize the likelihood of passing unqualified student conditional on passing the qualified student with probability one, i.e., 
\begin{equation}\tag{$\mathcal{P}_{II}$}\label{min unqualified}
    \begin{aligned}
        \min & \quad \pr[\text{selecting unqualified agent}] \\
        s.t. & \quad \text{ selecting all qualified agent}.\\
        % \Leftrightarrow \quad \min & \quad \text{Type I error} \\
        % s.t. & \quad \text{ zero Type II error}.
    \end{aligned}
\end{equation}

\paragraph{Manipulation}
Let $\density$ denote the density function of the distribution over initial type $\dist$.
Let $\nabla \density$ denote the gradient vector of $\density$.
% \begin{definition}[Strong regularity]
%     The cost function $c : \calS \to \bbR$ satisfies strong regularity if it is regular and satisfies the following two properties:
% \begin{itemize}
%     \item \textbf{Monotonicity (b).} $\onecost(\features,\genericfeatures)> \onecost(\features,\boldsymbol{y})$ for any $\lambda\in (0,1)$, for any $\features\neq\genericfeatures$, where $\boldsymbol{y}=\lambda \features +(1-\lambda)\genericfeatures$.
%     \item \textbf{Continuity.} Fix any attributes $\features$, $\onecost(\features,\genericfeatures)$ is continuous in $\genericfeatures$.
%     \item \textbf{Boundedness.}  Given any half plane $\classifier:\weights\cdot\features\geq 0$, there exists some $\features\in \classifier^\compl$ such that $\min_{\genericfeatures\in \classifier}\onecost(\features,\genericfeatures)\geq 1$.
% \end{itemize} 
% \end{definition}

\begin{proposition}\label{prop:Pii manipulation}
    Consider the manipulation setting and program \ref{min unqualified}.
    Suppose the cost function is induced by the Euclidean distance and is additive.
    Suppose $\nabla \density \cdot \weights_A \leq 0$ and $\nabla \density \cdot \weights_B \geq 0$.
    Then there exists a fixed-order sequential mechanism that outperforms the optimal simultaneous mechanism.
\end{proposition}


The alternative objective complicates the analysis under manipulation.
In general, which mechanism is optimal depends on the distribution.
Here we provide a relatively simple and easy-to-interpret condition $\nabla \density \cdot \weights_A \leq 0$ and $\nabla \density \cdot \weights_B \geq 0$, under which we can show a fixed-order mechanism dominates all simultaneous mechanisms. 
This condition quantifies the tradeoff between the two criteria.
It can be understood as the more qualified type according to one criterion (along 
the direction of $\weights_B$) has decreasing density while the more qualified type according to another criterion (along 
the direction of $\weights_A$) has increasing density. 
Under this condition, we can easily pin down the optimal tests under the optimal simultaneous mechanism.
Then, we can construct a fixed-order sequential mechanism that dominates it.
We defer the proof to \cref{appendix: alternative objective}. 



\paragraph{Investment} The same simultaneous mechanism continues to achieve the first best under the alternative objective in the investment setting. 
\begin{proposition}\label{prop:Pii investment}
    Consider  the investment setting and program \ref{min unqualified}.
    For any distribution $\dist$ and any cost function, the optimal  simultaneous mechanism is one that uses the true requirements and  achieves the first best.
\end{proposition}

The proof of \cref{prop:Pii investment} is very similar to the proof of \cref{thm:optimal investment}.
