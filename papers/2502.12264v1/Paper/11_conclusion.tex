\section{Conclusion}
We study the optimal testing procedures in two extreme environments: the agent improves either  apparent performance (manipulate) or actual performance (invest) but never both. We show that the optimal testing procedure is sequential with fixed order when the agent manipulates and it is simultaneous when the agent invests.
We apply our model to explain the different banking regulatory practices in Europe and in the US, the joint regulation of merger and acquisition across departments, and the collective decision making involving subjective opinions.
Our paper points out several potential paths of future research.
First, the comparison between the incentive schemes or institutional designs facing a manipulating agent and an investing agent is fruitful. 
Second, we consider a setting where a principal has multiple requirements which can potentially contradict each other, but it is also interesting to relax the commitment assumption to study conflicting principals.
Third, one can also consider an agent with richer technology space, i.e., an agent that can both manipulate and invest.