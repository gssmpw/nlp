\section{Relation to the literature}
% The comparison between sequential and simultaneous procedures involve a new trade off in our setting.
% In many economic problems, usually only one party (agent) has private information, while another party (principal) designs incentive schemes to elicit the private information.
% Examples include auction, search problems among others.
% In these settings, although a simultaneous procedure usually has cost (time) advantages, 
% a sequential procedure usually provides the principal informational advantages.
% For instance, when there is no search or waiting cost, a static auction can be implemented by asking each bidder's bid sequentially.
% In information acquisition problems, it is without loss to look at sequential mechanisms (\xqcomment{Ben-Porath et al. 2023}).
% In persuasion problems, a static protocol is a special case of a sequential protocol (\citet{sher2014persuasion}).
% However, in our current problem, 
% the testees of course have private information that is useful for the tester, but the tester could also reveal or hide information about the standards that could change the testees' behaviors.    
% In other words, the principal could also create private information in a sequential mechanism.
% It becomes unclear whether it is out of the interest of the principal to create such private information.
%
% This points to another  tension in designing a procedure with multiple tests: the relation between opacity of the procedure and the gaming incentives.  
% The design of testing procedure is usually associated with setting and publicizing standards.
% Knowing the procedures and standards, strategic agents might manipulate their attributes to game the rules.
% Such is a well-known phenomenon not only in designing test (\citet{perez2022test}), but also in designing algorithms (\citet{hardt2016strategic}), and in financial policy making which is called Goodhart's law.  
% The design of a testing procedure with multiple tests is subject to the same issue. 
% When offering multiple tests sequentially, revealing the order of the tests completely might introduce more manipulation behaviors from the testees.
% Intuitively, such manipulation behaviors could worsen principal's decision making since they result in worse information accuracy.
% One might be tempted to conclude that lack of opacity of the testing procedure is undesirable.
% However, we will show that this intuitive reasoning is incorrect.
\paragraph{Literature on test design facing a strategic agent}
Our paper is closely related to the literature on test design where the agent can take hidden action to affect the outcome of the test (\citet{perez2022test, deb2018optimal}).
\citet{perez2022test} study  a setting where the principal employs a test, which is an exogenous Blackwell experiment, to provide information for the true, one-dimensional state of the world, but the agent can costly falsify the inputs into the test.
The agent only cares about getting approved by the principal while the principal only wants to approve when the state of the world is high, analogous to our qualified region.
They show that the optimal test design involves `productive falsification', i.e., it sets the passing standard higher than the true qualified threshold so that even initially qualified agent has to exert positive effort to get approved.
We recover the similar finding in our manipulation setting, which is similar to their setting except that in our setting the agent's attributes are two-dimensional and the principal uses two tests to select the agent, while they study one-dimensional attribute and one test.
We also study a setting where the agent invests, which is not studied by them.

\citet{deb2018optimal} study a setting where the principal selects a sequence of tasks to learn about the agent's true one-dimensional type, while the agent can take hidden action to affect the outcome of the task, which is jointly determined by the agent's private type and hidden action.
The principal decides whether to pass the agent at the end based on the history of performance.
The principal cares about making the right decision, i.e., passing the good type and failing the bad type, while the agent only cares about passing the test.
They identify a sufficient condition under which the principal selects the most informative task. When the condition fails to hold, they show cases where the principal prefers using less informative tasks.
Although we both study sequential testing, our focus is very different from them. 
They study the optimal choice of the informativeness of the tasks, which they call adaptive testing, while we assume that the informativeness of the tests is exogenous and deterministic.

\citet{frankel2022improving,ballscoring} study the optimal design of linear scoring rules in a setting where the principal wants to infer the agent's fundamental type from a score that reflects both the agent's fundamental type and gaming type.
The fundamental type in their settings is similar to the initial attributes in our manipulation setting, but our model is very different from them.

\paragraph{Literature on persuasion with hard evidence}
\citet{glazer2004optimal,glazer2006study,sher2014persuasion}  study persuasion problems with hard evidence where the agent's private information also has two dimensions. 
 The agent (he) always prefers being accepted, whereas the principal (she) prefers accepting only when the state of the world satisfies some conditions.
 In their settings, the agent needs to provide hard evidence to persuade the principal.
 The principal chooses a persuasion rule to minimize the probability of an error.
 Their problems can be viewed as an extreme case of ours, where (1) the agent's cost of changing attributes is infinitely high, and (2) the principal can use \emph{perfect} tests, i.e., principal can observe agent's attributes in each round of communication, while we (1) allow the agent to change his attributes with some cost, and (2) consider a  more restrictive principal's objective due to tractability. 
\citet{glazer2004optimal} shows that it is beneficial for the receiver to randomize the requested evidence after allowing the sender to send a cheap talk message.
 \citet{glazer2006study} shows that neither the commitment to the decision nor randomization have value.
\citet{sher2014persuasion} shows that (1) commitment has no value, and (2) under some conditions, randomization has no value, implying that static persuasion rules coincide with dynamic persuasion rules.
In contrast, we show that (1) randomization has no value regardless of whether or not cheap talk message is allowed, and (2) the optimal simultaneous (or single-test) mechanism, which induce a static game, can be either better or worse than the optimal sequential mechanism, which induce a dynamic game, depending on  whether the agent manipulates or invests.
\citet{Carroll_Egorov} consider a richer class of payoff functions for the agent and they focus on the possibility of full learning by randomization.


\paragraph{Literature on contracting with gaming behaviors}
We are related to a large literature on how to design contracts to induce productive effort and deter gaming. 
\citet{holmstrom1987aggregation} is one seminal paper that studies the design of contract when agents can exert effort among multiple activities.
\citet{ederer2018gaming} study how randomization over compensation scheme could deter gaming, i.e., the agent diverts effort away from valuable but difficult to measure activities to less costly and easy to measure activities. 
\citet{li2021learning} study the optimal relational contract when the agent can privately learn which activity is more important and suggest that intermittent replacement of existing measures could deter such a learning-by-shirking effect.
We differ from these papers by studying how the principal can use test design, instead of using monetary payment to either provide incentives in the investment setting, or conduct screening in the manipulation setting.
Moreover, we also differ from these paper by considering a distinct type of gaming behaviors: the kind of gaming behaviors considered by the above mentioned papers are closer to the phenomenon of multi-tasking as initiated by \citet{holmstrom1991multitask}, while we consider the gaming behaviors that exploit the rules of classification to improve apparent performance as nicely summarized in \citet{ederer2018gaming}.
In contrast to \citet{ederer2018gaming}, randomization over the order of tests is never beneficial to deter gaming, or manipulation in our model.


\paragraph{Literature on multiple attribute search}
 We are related to the literature on multiple attribute search (\citet{olszewski2016search, sanjurjo2017search}) in the sense that the principal can be viewed as a searcher and the agent's attributes are analogous to the attributes.
 In this literature, the consensus is that when there is no cost for the searcher, sequential search is weakly better than simultaneous search.
One important difference is that the agent in our setting can take strategic actions, while the object is passive in the search literature.
Moreover, we find that even when there is no cost for the principal to offer test, simultaneous procedures could be better when the agent invests.



 
 \paragraph{Computer science literature on strategic classification in machine learning}
 We also contribute to a computer science literature that studies how to design algorithms facing strategic agents, stemming from \citet{hardt2016strategic}.
 % The key implication of our paper is that when using algorithms to classify high-dimensional data, if the agent can only manipulate, then it is optimal to use a sequential procedure to collect data and make decisions, while if the agent can only invest, then it is optimal to use a simultaneous procedure to collect data and make decisions.
 
Our work is most closely related to  \citet{zigzag}, who study fixed-order sequential mechanisms when the agent can manipulate his type according to an additive Euclidean cost function, and explore how the agent can exploit the sequential ordering of tests to achieve a favorable outcome at a limited cost.

Building on \citet{zigzag}, we conduct a comprehensive mechanism design analysis that encompasses random sequential mechanisms---which are central to two key results in our paper (\cref{thm: optimal max qualified} and \cref{thm: true effort sequential})---and complements \citet{zigzag}'s finding that a fixed-order sequential mechanism outperforms simultaneous mechanisms.\footnote{We also identify an error in Theorem 4.4 of \citet{zigzag} and offer a corrected version of their result in \cref{lem:fix-simul} and in \cref{thm:opt_manipulation} of our paper.}
 
 Other papers (\citet{ahmadi2022classificationstrategicagentsgame,miller2020strategicclassificationcausalmodeling,haghtalab2020maximizingwelfareincentiveawareevaluation,kleinberg2019classifiersinduceagentsinvest}) have also studied settings where the agent can both manipulate and invest at the same time. 
 The only exception is \citet{ahmadi2022classificationstrategicagentsgame}, who also compare the setting where the agent can manipulate to that where the agent can invest. 
 However, these papers have very different focus. They answer questions primarily related to computational hardness and algorithmic approximation, while we study the optimal mechanisms.

%  \paragraph{Literature on treatment allocation facing strategic agents}
% We are also related to an Econometric literature on regression estimation when agents strategically respond to the treatment allocation (\citet{munro2024treatment}). 
% The main difference is that we focus on a binary decision while this literature focuses on regression estimation. 

