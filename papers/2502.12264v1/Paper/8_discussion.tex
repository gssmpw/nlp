\section{Discussions}\label{sec:discussion}
% \paragraph{Strategy space: one-step strategies vs two-step strategies.}
% In a sequential mechanism, depending on whether the agent's choice of first and second attributes coincide, we can distinguish 
% \emph{one-step strategies} and \emph{two-step strategies}.
% Consider an arbitrary fixed order sequential mechanism $(\tilde\classifier_A,\tilde\classifier_B,1)$ where $\tilde\classifier_A$ is offered as the first test.
% Consider an agent with initial attributes $\orifeatures\notin \tilde\classifier_A\cap\tilde\classifier_B$, i.e., such an agent fails both tests initially.
% Knowing that $\tilde\classifier_A$ is offered as the first test, this agent can choose the first attributes $\firstfeatures\in \tilde \classifier_A$ in order to pass the first test.
% Before the second test $\tilde\classifier_B$ takes place, he can again choose some  $\secondfeatures\in \tilde \classifier_B$ to pass the second test. 
% We call such a strategy a \emph{two-step strategy} (See \cref{subfig:two step}).
% Alternatively, the agent can choose some attributes  $\tilde\features^1\in \tilde\classifier_A\cap\tilde\classifier_B$ before both tests.
% Since $\tilde\features^1$ is in the intersection of both tests, the agent can pass both tests and get selected.
% We call such a strategy a \emph{one-step strategy} (See  \cref{subfig:one step}).
% Both \emph{one-step strategies} and \emph{two-step strategies} are feasible in sequential mechanisms.
% However, in simultaneous mechanisms, only one-step strategies are feasible.
% Hence the major difference between simultaneous mechanisms and sequential mechanisms is the strategy space of the agent.



% \begin{figure}[h]  \label{fig: 1-step vs 2-step}
% 	\centering 
% 	\begin{subfigure}[b]{0.45\linewidth}
% 		\begin{tikzpicture}[xscale=7.8,yscale=7.8]
		
% 		\draw [domain=0.68:1.4, thick] plot (\x, {3/4*\x+1/4});
% 		\node [left] at (1.38, 1.3 ) {$\tilde \classifier_B$};%:\feature_2\geq \frac34 \feature_1 +\frac14
% 		\draw [thick] (1,0.68) -- (1,1.3);
% 		\node [right] at (1, 1.3) {$\tilde\classifier_A$};%: \feature_1 \geq 1$
		
% 		\node [left] at (1.33,1.25) {$+$};
% 		\node [right] at (1, 1.25) {$+$};
		
% 		\node[red] at (1-0.1, 1-0.2 ) {$\circ$};
% 		\node [ below] at (1-0.1, 1-0.2 ) {$\orifeatures$};
		
% 		\node [red] at  (1,1) {$\circ$};
% 		\node [right, red] at (1, 1 ) {$\tilde\features^1$};
		
% 		\draw [densely dashdotted, red] (1-0.1, 1-0.2 ) --(1,1);
% 		\end{tikzpicture}
% 		\caption{One-step strategy}  
% 		\label{subfig:one step}
% 	\end{subfigure}
% 	\begin{subfigure}[b]{0.45\linewidth}
% 		\begin{tikzpicture}[xscale=7.8,yscale=7.8]
		
% 		\draw [domain=0.68:1.4, thick] plot (\x, {3/4*\x+1/4});
% 		\node [left] at (1.38, 1.3 ) {$\tilde \classifier_B$};%:\feature_2\geq \frac34 \feature_1 +\frac14
% 		\draw [thick] (1,0.68) -- (1,1.3);
% 		\node [right] at (1, 1.3) {$\tilde\classifier_A$};%: \feature_1 \geq 1$
		
% 		\node [left] at (1.33,1.25) {$+$};
% 		\node [right] at (1, 1.25) {$+$};
		
% 		\node[blue] at (1-0.1, 1-0.2 ) {\textbullet};
% 		\node [ below] at (1-0.1, 1-0.2 ) {$\orifeatures$};
		
% 		\node [blue] at (1, 1-0.072 ) {\textbullet};
% 		\node [blue,right] at (1, 1-0.072 ) {$\firstfeatures$};
		
% 		\node [blue] at  (1+0.1-5.5/24*0.6,1-0.22+5.5/24*0.8) {\textbullet};
% 		\node [blue,above] at  (1+0.1-5.5/24*0.6,1-0.22+5.5/24*0.8) {$\secondfeatures$};
		
		
% 		\draw [densely dashdotted, blue] (1-0.1,1-0.2) -- (1,1-0.072);
% 		\draw [densely dashdotted, blue]  (1,1-0.072) -- (1+0.1-5.5/24*0.6,1-0.22+5.5/24*0.8);
		
% 		\end{tikzpicture}
% 		\caption{Two-step strategy}
% 		\label{subfig:two step}
% 	\end{subfigure}
% 	\caption{Example of two different types of strategies}
% 	\rule{0in}{1.2em}$^\dag$\scriptsize In principle, using a one-step strategy in a simultaneous mechanism could incur a different cost compared to using it in a sequential mechanism.
% 	This is because in the sequential mechanism, there could be a positive cost of maintaining the attributes for extra time.
% \end{figure}  


% Subtly, in a sequential mechanism, even when agent chooses the same attributes for both tests and he does not opt out after the first test, it could incur a positive cost of maintaining the attributes.
% However, this creates unnecessary complication and would not add any new insight to the analysis. 
% \cref{assump: cost function one step} 
% makes sure that once the agent chooses some attributes, there is no cost of maintaining it. 
% Therefore, we can safely write any \emph{one-step strategy} $\strategies=\tilde\features^1$  as 
% $\strategies=(\tilde\features^1,\tilde\features^1)$.

\paragraph{Dimensionality and degenerate sequential mechanisms.}
Our model cannot be reduced to a problem where agent has a one-dimension attribute and the principal offers two selection cutoffs.
Consider a setup where the agent has a one-dimensional attribute $\feature\in \bbR$.
The principal would like to select any agent with attribute $\feature\in [b_1,b_2]$, where $b_1<b_2$.
Suppose now the principal announces that any agent with attribute belonging to $[\tilde b_1,\tilde b_2]$ will be selected, with $\tilde b_1<\tilde b_2$. 
The corresponding tests would be $\tilde\classifier_A: \feature\geq \tilde b_1$ and $\tilde\classifier_B: \feature\leq \tilde b_2$.
Then no matter what the order of the tests is, the agent's best response is fixed: for any $\feature < \tilde b_1$, $\strategies(\feature)=(\tilde b_1,\tilde b_1)$; for any $\feature > \tilde b_2$, $\strategies(\feature)=(\tilde b_2,\tilde b_2)$; for any $\tilde b_1\leq \feature \leq \tilde b_2$, $\strategies(\feature)=(\feature,\feature)$.

Next we introduce the concept of degenerate testing mechanisms. 
\begin{definition}[degenerate mechanism]
	A sequential mechanism $(\tilde\classifier_A,\tilde\classifier_B,\probprincipal,\disclose)$ is said to be \emph{degenerate} if A's best response is the same regardless of the order of the tests.
\end{definition}

We summarize the above observation in the following proposition.
\begin{proposition}
	If the agent's attribute is one-dimension, then any testing mechanism $(\tilde\classifier_A,\tilde\classifier_B,\probprincipal,\disclose)$ is degenerate.
\end{proposition}

\citet{perez2022test} studies a test design problem where the agent's attribute is one dimensional.
The above observation provides a motivation to study the case where agent has multi dimensional attributes and how the order of the tests play a role in such settings. 
As a starting point, we study the simplest case where agent has two dimensional attributes and principal has two tests.




\paragraph{Modeling tests.}
We interpret a test as an evaluation of whether one (or some) aspect(s) of the agent meets the the principal's requirement.
We assume that the agent does not which aspect(s) he is being evaluated in each test.
We argue that this is a reasonable assumption.
For example, in a math test, the students taking the test do not know whether they are being evaluated on their mathematical thinking skill or their mastery of mathematical tools.
This slightly differs from the way we use the word test in English.
When we say there is a math test, the math test is a subject that includes at least the above mentioned two \emph{aspects of ability} being evaluated.
Hence if we interpret a test as an evaluation of aspect(s) of ability, it is unlikely that the students know how each aspect of ability is evaluated (or graded) unless they are told.
% Before we proceed, we point out one subtle but important distinction between the \emph{tests} in our model and the tests we use in day-to-day life.
% In day-to-day life, when we refer to a test or a school test, we usually refer to a specific subject or a topic, for example, math test, English test, etc..
% The testee usually knows which \emph{subject} they are being tested.
% However, in our model, a \emph{test} is an evaluation to one skill or multiple skills.
% Therefore, even if the testee knows which \emph{subject} they are being tested, they may still not know which \emph{aspect(s) of ability} they are being tested. 


\paragraph{Simultaneous mechanisms are not special case of sequential mechanisms.}
Although in simultaneous mechanisms the order of the tests does not matter, they cannot be viewed as special cases of sequential mechanisms, i.e.,  $\simultaneous\not\subset \sequential$.
First of all, in any game specified by a simultaneous mechanism, the agent only has one information node.
Therefore, a simultaneous mechanism cannot be achieved by any fixed order mechanism or random order mechanism with disclosure.
Second, even though in the game specified by a random order mechanism without disclosure, the agent also has one information node, his strategy space is much larger.
This is because in any simultaneous mechanism, the agent is essentially restricted to use \emph{one-step} strategies, while in any random order mechanism without disclosure, the agent can use either \emph{one-step} strategies or \emph{two-step} strategies. 
An alternative way to see this is that in any random order mechanism without disclosure, the agent is evaluated by a \emph{linear} test each time.
However, in any simultaneous mechanism, the agent is evaluated by the intersection of two \emph{linear} tests each time. 
More formally, let $\twolineartest=\{\tilde\classifier_1\cap\tilde\classifier_2:\tilde\classifier_i\in \lineartest, i=1,2\}$ be the space of tests that can be used in any simultaneous mechanism.
Then $\lineartest\subset \twolineartest$.
Therefore, by changing the timing to offer the tests, simultaneous mechanisms are essentially enlarging the space of tests that can be used.

\paragraph{Dynamic nature of sequential mechanisms.}
To further illustrate why the game induced by any sequential mechanism is not static, let's consider the cost function $\cost(\orifeatures,\firstfeatures,\secondfeatures)=\metric(\orifeatures,\firstfeatures)+\metric(\firstfeatures,\secondfeatures)$, where $\metric(\cdot,\cdot)$ is the Euclidean distance.
Consider two fixed order mechanisms that use the same two tests $\tilde\classifier_A$ and $\tilde\classifier_B$ and they only differ in terms of the order of the two tests.
In \cref{subfig:two step A-B}, the mechanism offers $\tilde\classifier_A$ as the first test, while in \cref{subfig:two step B-A}, the mechanism offers $\tilde\classifier_B$ as the first test.
Consider an agent with attributes $\orifeatures$ as in \cref{subfig:two step A-B}.
Notice that $\orifeatures$ pass test $\tilde\classifier_A$ but fail test $\tilde\classifier_B$.
Under the fixed order mechanism $(\tilde\classifier_A,\tilde\classifier_B,1)$, the optimal strategy of such attributes is to choose $\firstfeatures=\orifeatures$ and a different $\secondfeatures$ that are the projection of $\orifeatures$ on the boundary line of $\tilde\classifier_B$ (See \cref{subfig:two step A-B}).
% We call this a \emph{two-step} strategy.
Such a strategy is the least costly for the agent and guarantees that the agent is selected under this fixed order mechanism.
However, under another fixed order mechanism $(\tilde\classifier_B,\tilde\classifier_A,1)$, the optimal strategy of such attributes is to choose $\firstfeatures=\secondfeatures$ that are the intersection of  the boundary line of $\tilde\classifier_A$ and $\tilde\classifier_B$ (See \cref{subfig:two step A-B}).
This is because now $\tilde\classifier_B$ is the first test the agent needs to pass and $\orifeatures$  are so far away from  $\tilde\classifier_B$ that it is less costly to pass the two tests at the same time.
% We call this a \emph{one-step} strategy.
This example shows that a sequential mechanism is naturally dynamic and as a result, the agent's strategy is also dynamic. 



%---------------------------------------
\begin{figure}[h]  \label{fig: sequential is dynamic}
	\centering 
	\begin{subfigure}[b]{0.45\linewidth}
		\begin{tikzpicture}[xscale=7.8,yscale=7.8]
		
		\draw [domain=0.68:1.4, thick] plot (\x, {3/4*\x+1/4});
		\node [left] at (1.38, 1.3 ) {$\tilde \classifier_B$};%:\feature_2\geq \frac34 \feature_1 +\frac14
		\draw [thick, red] (1,0.68) -- (1,1.3);
		\node [right] at (1, 1.3) {$\tilde\classifier_A$};%: \feature_1 \geq 1$
		
		\node [left] at (1.33,1.25) {$+$};
		\node [right] at (1, 1.25) {$+$};
		
		\node[black] at (1, 1-0.2 ) {\textbullet};
		\node [ right] at (1, 1-0.2 ) {$\orifeatures\textcolor{blue}{=\firstfeatures}$};
		
		\node [blue] at  (1-12/125,1-9/125) {\textbullet};
		\node [left, blue] at (1-12/125,1-9/125 ) {$\secondfeatures$};
		
		\draw [densely dashdotted, blue] (1, 1-0.2 ) --(1-12/125,1-9/125);
		\end{tikzpicture}
		\caption{Two-step strategy: $\tilde\classifier_A\rightarrow\tilde\classifier_B$}  
		\label{subfig:two step A-B}
	\end{subfigure}
	\begin{subfigure}[b]{0.45\linewidth}
		\begin{tikzpicture}[xscale=7.8,yscale=7.8]
		
		\draw [domain=0.68:1.4, thick, red] plot (\x, {3/4*\x+1/4});
		\node [left] at (1.38, 1.3 ) {$\tilde \classifier_B$};%:\feature_2\geq \frac34 \feature_1 +\frac14
		\draw [thick] (1,0.68) -- (1,1.3);
		\node [right] at (1, 1.3) {$\tilde\classifier_A$};%: \feature_1 \geq 1$
		
		\node [left] at (1.33,1.25) {$+$};
		\node [right] at (1, 1.25) {$+$};
		
		\node[blue] at (1, 1-0.2 ) {\textbullet};
		\node [ right] at (1, 1-0.2 ) {$\orifeatures$};
		
		% \node [blue] at (1-12/125,1-9/125) {\textbullet};
		% \node [blue,left] at (0.8,1 ) {$\tilde\features$};
		
		\node [blue] at  (1,1) {\textbullet};
		\node [blue,right] at  (1,1) {$\firstfeatures=\secondfeatures$};
		
		
		\draw [densely dashdotted, blue] (1,1-0.2) -- (1,1);
		% \draw [densely dashdotted, blue]  (1,1) -- (0.8,1);
		
		\end{tikzpicture}
		\caption{Two-step strategy: $\tilde\classifier_B\rightarrow\tilde\classifier_A$}
		\label{subfig:two step B-A}
	\end{subfigure}
	\caption{Dynamic nature of sequential mechanisms}
	\rule{0in}{1.2em}$^\dag$\scriptsize Suppose $\cost(\orifeatures,\firstfeatures,\secondfeatures)=\metric(\orifeatures,\firstfeatures)+\metric(\firstfeatures,\secondfeatures)$, where $\metric(\cdot,\cdot)$ is the Euclidean distance. The left panel shows the optimal strategy under the fixed order mechanism that offers test $\tilde\classifier_A$ as the first test, while the right panel shows the optimal strategy under the fixed order mechanism that offers test $\tilde\classifier_B$ as the first test.
\end{figure}  

\paragraph{More tests do not help.}
In the manipulation setting, adding more tests only makes a procedure more stringent and it is counter-productive. 
Suppose the principal offers test $\classifier_1$ and $\classifier_2$ repeatedly. For simplicity, suppose the first test is $\classifier_1$, the second is $\classifier_2$ and the third is $\classifier_1$. For those types that find it profitable to pass both tests together under a fixed-order procedure with two tests, adding an extra test does not change their incentives.
For those types that find it profitable to pass one test at a time, adding a third test only make it more costly to pass one test at a time.
Hence, the benefit of the fixed-order procedure is diminished under more tests.

In the investment setting, under the optimal simultaneous mechanism, adding one more test cannot improve the outcome since using two tests already achieve the first best.


\paragraph{Endogenous agent's technology.}
We assume that the agent's technology, whether he manipulates or invests, is exogenous. 
This usually can be explained by the different monitoring strength of the economic environment.
In the banking application, US banks are tested daily and therefore there is no room for manipulation.
In contrast, European banks are tested monthly or quarterly.
The low monitoring intensity allows banks to temporarily change their balance sheet.

Suppose instead now when the agent changes his type from $\orifeatures$ to $\firstfeatures$, he can choose between  manipulation, which costs $0.2\cdot\onecost(\orifeatures,\firstfeatures)$ and investment, which costs $0.5\cdot\onecost(\orifeatures,\firstfeatures)$.
Then the only equilibrium is such that every type chooses manipulation and the principal chooses the optimal fixed-order sequential mechanism under manipulation, given that the principal's objective is \ref{max qualified}. 
This is because (1) whichever mechanism the principal chooses, the agent prefers manipulation over investment, and (2) no type can credibly commits to investment. 
As a result, although every type would have been better-off by choosing investment, it is not an equilibrium.



\paragraph{Informational robustness.}
When the agent manipulates , the optimal tests are chosen based on the information about the agent's cost.
Whenever there is a small amount of uncertainty in the agent's cost function,  the principal needs to choose even more stringent tests so as not to select any unqualified agent in the worst case where the agent's cost happens to be least costly one.
In contrast, when the agent invests, the optimal tests coincide with the qualified region. 
Hence the optimal mechanism in the investment setting is more robust to the agent's information.

% \paragraph{Stringency of testing procedures}
% Fixing the tests, we use how difficult it is for the agent to to manipulate to measure the stringency of a testing procedure.
% Roughly speaking, a simultaneous procedure is weakly more stringent than a random-order sequential procedure without disclosure, which is  more stringent than a random-order sequential procedure with disclosure,  which is  more stringent than a fixed-order sequential procedure.

% \paragraph{Noisy tests}
% A stochastic \emph{linear} test $\tilde\classifier$ is a half plane and a random variable $\epsilon$ such that the agent passes the test if and only if agent's attributes $\features +\epsilon \in \tilde\classifier$.
% The test result of an agent with attributes $\features$ under test $\tilde\classifier$ is $\test= \indicate{\features+\epsilon\in \tilde\classifier}$.
% Incorporating stochasticity into the test does not change our results.


% \paragraph{Randomizing over multiple tests}




% \paragraph{Connections to persuasion problems.}
% \citet{glazer2004optimal} and \citet{sher2014persuasion} study persuasion problems with evidence where the agent's private information also has two dimensions. 
% Instead of passing tests like in our setting, the agent needs to provide hard evidence to persuade the principal.
% The main difference of our paper compared to these two is that we adopt a slightly more general approach to model what information the agent can use to persuade the principal, which is similar to the modeling approached used in \citet{perez2022test}.
% Evidence usually contains the truth and therefore cannot be fabricated. 
% Or rather, we can view it as the cost of adopting different attributes is infinity.
% In this sense, it is more general to assume that agent can adopt different attributes to present to the principal with some cost. 
% Then persuasion with evidence can be viewed as (1) the agent's cost of changing attributes is infinitely high, and (2) the principal can use \emph{perfect} tests, i.e., principal can observe agent's attributes in each round of communication.
