\subsection{Investment}\label{sec:optimal investment}
In this subsection, we first show that the optimal mechanism among all simultaneous and sequential mechanisms is a simultaneous mechanism that uses two tests that coincide with the true requirements (\cref{thm:optimal investment}). 
The optimal simultaneous mechanism is very powerful in the investment setting because it accepts every qualified agent and every unqualified agent that can improve to some qualified attributes with a cost less than one.
Despite its power, we continue to study the optimal sequential mechanism under investment for two reasons.
First, practically speaking, it is sometimes hard to implement simultaneous mechanisms because of physical constraints.
Second, it helps us to understand better how the two classes of mechanisms work under investment.



\begin{proposition}\label{thm:optimal investment}
    Consider  the investment setting.  
    For any distribution $\dist$ and any cost function $\cost$, the optimal simultaneous  mechanism uses two tests that coincide with the true requirement: $\classifier_A$ and $\classifier_B$. Moreover, it achieves the first best (upper bound of the objective value).
    % Moreover, such a mechanism achieves the first best (upper bound of the objective value) under objective \ref{max qualified}. 
\end{proposition}

\begin{proof}[Proof of \cref{thm:optimal investment}]
    We show that the described simultaneous mechanism achieves the upper bound of the value of program \ref{max qualified}. 
    First notice that any qualified agent gets selected with zero cost under the described mechanism.
    Second, we show that the described mechanism accepts every unqualified agent who can improve to another qualified attributes with cost less than one.
    Consider an unqualified agent with attributes $\features\not\in\classifier_A \cap \classifier_B$. 
    Suppose there exists attributes $\features'$ such that (1)  $\features'$ satisfy both tests $\classifier_A$ and $\classifier_B$; and (2)  such an agent can move to $\features'$ with cost $c(\features, \features', \features') \leq 1$.
    Then this unqualified agent has an incentive  to improve his attributes to $\features'$ and get accepted.  
    However, if instead such attributes $\features'$ do not exist and for any attributes $\features''$ in $\classifier_A \cap \classifier_B$, the cost for this agent with attributes $\features$ to improve to $\features''$ is strictly larger than one, then this unqualified agent has no incentive to move to the qualified region $\classifier_A \cap \classifier_B$ in any mechanism.
    % While any unqualified agent with attributes in $\{\features : d(\features, \classifier_1\cap \classifier_2) > 1/\eta\}$ needs to pay a cost strictly greater than $1$ to reach qualified region $\classifier_1 \cap \classifier_2$.   
    Thus, the simultaneous mechanism that uses tests $\classifier_A$ and $\classifier_B$ accepts every agent who is either qualified or can improve their attributes to the qualified region with a cost less than one. 
    This is the upper bound of the value of program \ref{max qualified}.
    Hence the described simultaneous mechanism is optimal.
\end{proof}

The proof shows that the optimal simultaneous mechanism achieves the upper bound of the value under \ref{max qualified}, i.e., the first best scenario.
Intuitively, this is because by offering two tests simultaneously, a simultaneous mechanism essentially offers a stricter test that coincides with the qualified region. 
Not only it selects every qualified agent, but it also induces the initially unqualified agent to invest and become qualified.
Moreover, the strictness notion here is different from the notion of  stringent tests we used in the previous section.
Here, strictness refers to the definition of the test, while before, a stringent test refers to the standard of a linear test.
Setting higher standard for any linear test does not help in the investment setting, but using two linear tests simultaneously encourages most investment.

% \begin{proof}[Proof of \cref]
%     We show that the described simultaneous mechanism achieves the upper bound of the value of program \ref{max qualified}. 
%     First notice that any qualified agent gets selected with zero cost under the described mechanism.
%     Second, we show that the described mechanism accepts every qualified agent who can improve to another qualified attributes with cost less than one.
%     Consider an unqualified agent with attributes $\features\not\in\classifier_A \cap \classifier_B$. 
%     Suppose there exists attributes $\features'$ such that (1)  $\features'$ satisfy both tests $\classifier_A$ and $\classifier_B$; and (2)  such an agent can move to $\features'$ with cost $c(\features, \features', \features') \leq 1$.
%     Then this unqualified agent has an incentive  to improve his attributes to $\features'$ and get accepted.  
%     However, if instead such attributes $\features'$ do not exist and for any attributes $\features''$ in $\classifier_A \cap \classifier_B$, the cost for this agent with attributes $\features$ to improve to $\features''$ is strictly larger than one, then this unqualified agent has no incentive to move to the qualified region $\classifier_A \cap \classifier_B$ in any mechanism.
%     % While any unqualified agent with attributes in $\{\features : d(\features, \classifier_1\cap \classifier_2) > 1/\eta\}$ needs to pay a cost strictly greater than $1$ to reach qualified region $\classifier_1 \cap \classifier_2$.   
%     Thus, the simultaneous mechanism that uses tests $\classifier_A$ and $\classifier_B$ accepts every agent who is either qualified or can improve their attributes to the qualified region with a cost less than one. 
%     This is the upper bound of the value of program \ref{max qualified}.
%     Hence the described simultaneous mechanism is optimal.
% \end{proof}


% The proof shows that the optimal simultaneous mechanism achieves the upper bound of the value under \ref{max qualified}, i.e., the first best scenario.
% Intuitively, this is because by offering two tests simultaneously, a simultaneous mechanism essentially offers a stricter test that coincides with the qualified region. 
% Not only it selects every qualified agent, but it also induces the initially unqualified agent to invest and become qualified.
% Moreover, the strictness notion here is different from the notion of  stringent tests we used in the previous section.
% Here, strictness refers to the definition of the test, while before, a stringent test refers to the standard of a linear test.
% Setting higher standard for any linear test does not help in the investment setting, but using two linear tests simultaneously encourages most investment.

% Next, we study the optimal sequential mechanism under investment.

%--------------------------------------
% \subsection{Optimal Sequential Mechanism}
