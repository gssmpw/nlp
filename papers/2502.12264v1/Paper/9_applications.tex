\section{Other applications}\label{sec:applications}
\paragraph{Collective decision making with diverging opinions}

Consider a search committee deciding whether to hire a candidate.
To be more concrete, consider an academic department hiring a scientific researcher, where candidates usually need to present their research plan for the next five to ten years and supporting evidence that such a plan is feasible and promising, on top of past work to demonstrate competence. 
Each member in the committee cares both about the objective aspect of the candidate, i.e., competence, and the subjective aspect of the candidate, i.e., how risky or ambitious his research plan is.
Everyone likes to hire a competent candidate, but some members prefer more ambitious candidates while others prefer less ambitious candidates.
Consider an unanimous or equal veto decision rule. 
We can view the whole committee as the principal.
And the committee's qualified region is defined by the two members with the most extreme subjective preferences, say Mr. R who prefers ambitious candidates and Mr. L who prefers less ambitious candidates (See \cref{fig: search committee}).

\begin{figure}[h] 
\centering 
\begin{tikzpicture}[xscale=7.8,yscale=7.8]
\draw [thick,<->] (-0.1,0) -- (0.8,0);
\draw [thick, ->] (6/25,0) -- (6/25,0.8);
\node [below] at (0.8,0) {more ambitious};
\node [below] at (-0.2,0) {less ambitious};
\node [above] at (6/25,0.8) {competence};

\draw [domain=0:8/15, thick, red] plot (\x, {1.5*\x});
\draw [domain=0:0.6, thick, blue] plot (\x, {-\x +0.6});

% \node [left] at (1.2,1.15) {$+$};
% \node [right] at (1, 1.15) {$+$};
% \node [left,font=\tiny] at (0,-0.05) {\footnotesize$O$};
% \fill [purple!60,nearly transparent] (1,5.2/4) -- (1,1)-- (1.4,5.2/4)-- cycle;
\path[pattern color=yellow,pattern=north west lines] (0,0.8) -- (0,0.6) -- (6/25,6/25*1.5)-- (8/15,0.8); 
\node [thick,font=\tiny] at (0.1,0.65) {\footnotesize$\text{approval zone}$};

\fill [red!60,nearly transparent]  (8/15,0.8) -- (0,0) -- (8/15,0) -- cycle;
\node [thick,font=\tiny, red] at (0.53,0.3) {\footnotesize$\text{Mr. L's rejection zone}$};

\fill [blue!60,nearly transparent]  (0,0.6) -- (0,0) -- (0.6,0)-- cycle;
\node [thick,font=\tiny, blue] at (0,0.25) {\footnotesize$\text{Mr. R's rejection zone}$};



\end{tikzpicture}
\caption{A search committee with diverging preferences}
\label{fig: search committee}
  \end{figure}
  
A test is  a joint assessment on the candidate's multiple attributes by each member, which could be a conversation, a discussion, a speech, a presentation, or an interview.
Here only the two members with the most extreme preferences matter, i.e., Mr. L and Mr. R.
A candidate passes Mr. L's (or Mr. R's) test if he passes the overall assessment or judgement made by Mr. L (or Mr. R). 

 A simultaneous mechanism is an interview with both Mr. L and Mr. R in the same room.
 A sequential mechanism with fixed order consists of  two separate interviews Mr. L and Mr. R in a  fixed order.

The candidate could inflate or deflate his attitude towards taking risk when talking to different interviewers.
Hence this fits into the manipulation setting.
The cost of changing apparent performance is arguably path dependent for multiple reasons.
A candidate usually needs to provide resumes and personal statement that are assessed by all interviewers. This could constrain how far the candidate can misrepresent themselves.

\paragraph{Banking Regulation}
After the recent global financial crisis in 2008, one major responsibility of central bank is to stabilize the banking system. 
The central bank sets multiple regulatory objectives based on commercial banks' financial statements. 
However, some of these metrics are in conflict with each other.
Consider the leverage ratio and liquidity ratio.
The leverage ratio is the ratio of Tier 1 capital to total asset.
The liquidity ratio is the ratio of liquid asset to net cash outflows over a 30-day stress period.
These two ratios are interconnected through liquid asset.
There are two reasons for the interconnectedness of the two ratios.
First, the total asset is the sum of liquid asset and illiquid asset.
Second, liquid asset is the item that is easiest to change in a short period of time among the items appearing on the bank's financial statements.

The central bank publicly sets the minimum requirement on leverage ratio and liquidity ratio. To visualize the central banks' requirements, we project the requirements on the two ratios on the space of Tier 1 capital and liquid asset. 
Suppose the minimum requirement on leverage ratio is $3\%$. 
Then the minimum requirement on leverage ratio is equivalent to requiring $\text{Tier 1 capital }\geq 3\% \times \text{ liquid asset } + 3\% \times\text{ other asset}$, which is represented by the red line in \cref{fig: banking regulation}.
The liquidity ratio requirement equivalent to requiring $\text{ liquid asset }\geq \text{ net cash outflow }$, which is represented by the blue line in \cref{fig: banking regulation}.
Notice that both of the requirements are bank specific.
Hence \cref{fig: banking regulation} represents the central bank's regulatory objective to one bank.
In Europe, the Basel committee on Banking Supervision (BCBS) decides that commercial banks need to report their leverage ratio at the quarter end and report their liquidity ratio at the end of each month.
In contrast, in the US, big commercial banks are required to report both their leverage ratio and liquidity ratio daily.

We argue that 
the European regulatory framework is essentially a sequential one with raised standards,\footnote{Encountering European banks' `window dressing' behaviors, European central bank decided to raise regulatory requirements on selected banks. In particular, `for six banks, a P2R leverage ratio add-on was applied on top of the 3\% leverage ratio requirement.' ( \citet{ECBraiserequirement})} while the 
% US  banking regulation fits into the investment setting and the
US regulatory framework is essentially a simultaneous procedure. 
We further argue that both practices could be compatible to our model if the European central bank prioritizes reducing gaming behaviors from commercial banks and the US central bank prioritizes boosting investment from commercial banks.\footnote{An alternative possibility is that European central bank believes that  commercial banks are primarily  gaming, while the US central bank believes that  commercial banks are primarily investing.} 


\begin{figure}[h] 
\centering 
\begin{tikzpicture}[xscale=7.8,yscale=7.8]
\draw [thick,->] (-0.1,0) -- (0.8,0);
\draw [thick, ->] (0,-0.1) -- (0,0.8);
\node [below] at (0.86,0) {liquid asset};
\node [above] at (0,0.8) {Tier 1 capital};

\draw [domain=0:0.8, thick, red] plot (\x, {0.2*\x+0.3});
% \draw [domain=0.3:0.8, thick, blue] plot (\x, {0.4*\x -0.12});
\draw [ thick, blue] (0.5,0) --(0.5,0.8);

\node [below] at (0.46,0) {net cash outflow};
\node [right] at (0.5,0.8) {liquidity ratio};
\node [right] at (0.8, 0.46) {leverage ratio };
\node [left,font=\tiny] at (0,-0.05) {\footnotesize$O$};
% \path[pattern color=purple, pattern=north west lines] (0.8,0.46) -- (0.5.0.4) -- (0.5,0.8); 
\path[pattern color=purple,pattern=north west lines] (0.8,0.46)--(0.5,0.4)--(0.5,0.8); 

\end{tikzpicture}
\caption{Regulation on a commercial bank}
\label{fig: banking regulation}
  \end{figure}


In practice, more European banks are reported to manipulate their financial statements by temporarily changing their liquid asset holding in a short period of time. We consider such temporary change manipulation.
In contrast, US banks are not detected to be involved into this kind of short term manipulation behaviors. 
Instead, when US banks change their financial statements, they tend to maintain such a change for a long time. We consider such long term maintenance of change an investment.

Empirical evidence shows that under the current regulatory framework, European banks are involved into window dressing behaviors by contracting repurchase agreement transactions right before the quarter end, and expand repurchase agreement  transactions right after (\citet{window_dressing} and \citet{ECB_window}).
A repurchase agreement usually involves two parties: a borrower of liquidity and a lender.
From the perspective of a borrow, the agreement is referred to as a repo while from the perspective of a lender, the same agreement is called a reverse repo.
A repo transaction could impact the balance sheet of the borrower while a reverse repo transaction has no impact on the lender's balance sheet.
When a bank enters a repo transaction, they receive cash, which enters the total asset and hence decreases the leverage ratio.
Hence contracting repo transactions can  temporarily boost a bank's leverage ratio, while expanding repo transactions enables banks to borrow liquidity, which could enhance their liquidity ratio.


In the US, banks are required to report their daily and quarterly average leverage (liquidity) ratio.
This type of practice prevents banks from temporarily improving their leverage (liquidity) ratio.




\paragraph{Regulation on merger or acquisition approval}
For a merger or acquisition to happen, the proposed firm needs to first submit a public filing to the Securities and Exchange Commission (SEC) and disclose information to investors so that they can evaluate the profitability of the merger or acquisition transaction.
SEC's regulatory goal is to inform shareholders about the potential profitability consequence of the merger or acquisition.
Hence the proposed firm usually need to justify that the merger or acquisition will increase profit, either because it introduces production efficiency by reducing cost significantly or because it enables the firm to set higher prices through larger market power.\footnote{The SEC mandates that public companies disclose detailed financial information related to M\&A transactions to ensure transparency and protect investors. These disclosures often include financial metrics such as projections of future performance, required returns, and cost of capital \citep{SEC_Regulation_SK, SEC_Form_S4}. In M\&A transactions, financial decisions are often based on the comparison between the expected return and the cost of capital (the hurdle rate). This financial analysis determines whether the deal adds value to the acquiring company’s shareholders \citep{jensen1983market}.}
Notice that although increasing market power is a sever anti-competitiveness concern, it is usually not an issue with the SEC since the SEC only cares about the investors.
The regulation from the SEC is represented by the blue line in \cref{fig: FTC-SEC}.
Basically, SEC requires that a company needs to reduce costs and/or increase revenue in some way above a hurdle rate.

After the firm's proposal gets approved by the SEC, the firm needs to further justify to the Federal Trade Commission (FTC) that the merger or acquisition does not trigger antitrust intervention. 
The FTC's regulatory goal is in conflict with that of the SEC. 
FTC usually approves a merger or an acquisition if the increase in market power is offset by the cost reduction, or improved production efficiency, so that overall prices do not actually increase. 
The FTC's regulatory goal is represented by the red line in \cref{fig: FTC-SEC}.
Depending on how much costs are  passed through to the consumers, the slope of the red line could be steeper or flatter.
The region where a merger or an acquisition that gets approved (qualified region) is colored in yellow in \cref{fig: FTC-SEC}.

A test chosen by the SEC is usually a public filing where the firm needs to justify the benefits of the merger or acquisition to the shareholders,
while a test chosen by the FTC is usually a threshold on how much estimated increase in market power  compared to the estimated  cost reduction can be tolerated, for example, price increase due to increased power cannot exceed $5\%$ of cost reduction.
 
Usually, firms could inflate the increased benefit from the merger or acquisition to the SEC to pass the SEC's scrutiny.
In contrast, when presenting to the FTC, firms could 
could exaggerate the uncertainty on potential price increase.
If firms are sued by the FTC, they could also hire experts to argue that the market is larger and hence  to deflate the potential increase of price or the harm to the consumers, or they could hire experts to manipulate cost information to argue that cost reduction is much higher than estimated. \footnote{Sometimes, firms can also hide their data or throw shade on the legitimacy of their data to influence the FTC's estimation.}
However, when using such strategies, they cannot contradict to their public filing submitted to the SEC. So the cost function of the firms exhibit some path dependence. 


\begin{figure}[h] 
\centering 
\begin{tikzpicture}[xscale=7.8,yscale=7.8]
\draw [thick,->] (-0.1,0) -- (0.8,0);
\draw [thick, ->] (0,-0.1) -- (0,0.8);
\node [below] at (0.8,0) {market power};
\node [above] at (0,0.8) {cost reduction};

\draw [domain=0:8/15, thick, red] plot (\x, {1.5*\x});
\draw [domain=0:0.6, thick, blue] plot (\x, {-\x +0.6});

% \node [left] at (1.2,1.15) {$+$};
% \node [right] at (1, 1.15) {$+$};
\node [left,font=\tiny] at (0,-0.05) {\footnotesize$O$};
% \fill [purple!60,nearly transparent] (1,5.2/4) -- (1,1)-- (1.4,5.2/4)-- cycle;
\path[pattern color=yellow,pattern=north west lines] (0,0.8) -- (0,0.6) -- (6/25,6/25*1.5)-- (8/15,0.8); 
\node [thick,font=\tiny] at (0.25,0.6) {\footnotesize$\text{safe zone}$};

\fill [red!60,nearly transparent]  (8/15,0.8) -- (0,0) -- (8/15,0) -- cycle;
\node [thick,font=\tiny, red] at (0.5,0.3) {\footnotesize$\text{FTC danger zone}$};

\fill [blue!60,nearly transparent]  (0,0.6) -- (0,0) -- (0.6,0)-- cycle;
\node [thick,font=\tiny, blue] at (0.16,0.2) {\footnotesize$\text{SEC danger zone}$};



\end{tikzpicture}
\caption{Regulation on firms during merger and acquisition}
\label{fig: FTC-SEC}
  \end{figure}



% Another reason is that different members in the committee can potentially coordinate and exchange information, which also imposes disciplines on how much the candidate can contradict himself when talking to different interviewers.

  
%  \paragraph{Hiring vs promotion}
%  Suppose a biotech company X is seeking candidates for a senior scientist
% position. Senior scientists must balance technical expertise and domain
% knowledge with managerial skills and leadership. Suppose X is looking for
% candidates that have some management experience and especially strong
% scientific expertise. Suppose, not unrealistically, that excessive
% experience as a manager may be an indication of career goal as a manager
% rather than a scientist, and even be a signal of declining technical
% expertise.

% We use \cref{fig: hiring} to illustrate X's requirements. The horizontal axis 
% represents managerial experience and the vertical axis  represents
% scientific experience. The blue vertical line represents the minimum
% requirement on managerial experience, say managing one direct report. The
% red diagonal line represents the joint requirement on both aspects. The purple dashed region represents the set of
% attributes that X's considers as qualified for senior scientist position.


% \begin{figure}[h] 
% \centering 
% \begin{tikzpicture}[xscale=7.8,yscale=7.8]
% \draw [thick,->] (0.6,0.8) -- (1.5,0.8);
% \draw [thick, ->] (0.9,0.66) -- (0.9,1.33);
% \node [below] at (1.5,0.8) {$\feature_A=$managerial experience};
% \node [above] at (0.9,1.33) {$\feature_B=$scientific experience};

% \draw [domain=0.66:1.4, thick, red] plot (\x, {3/4*\x+1/4});
% \draw [thick, blue] (1,0.66) -- (1,5.2/4);
% % \node [left] at (1.3, 1.25 ) {$\classifier_B$};
% % \node [right] at (1, 1.25) {$\classifier_A$};
% \node [left] at (1.2,1.15) {$+$};
% \node [right] at (1, 1.15) {$+$};
% \node [left,font=\tiny] at (0.9, 0.78 ) {\footnotesize$O$};
% % \fill [purple!60,nearly transparent] (1,5.2/4) -- (1,1)-- (1.4,5.2/4)-- cycle;
% \path[pattern color=purple,pattern=north west lines] (1,5.2/4) -- (1,1)-- (1.4,5.2/4); 
% % \node [blue] at (1, 1 ) {\textbullet};
% % \node [right,font=\tiny] at (1, 1 ) {\footnotesize$(1,5)$};

% \end{tikzpicture}
% \caption{Hiring requirements for a senior scientist position}
% \label{fig: hiring}
%   \end{figure}
  
%  Past experience in other biotech companies can reflect a candidate's
%  suitability for the position. The resume provides some information, but does not fully reveal which of the two functions (manager or scientist) was the candidate's main duty. So, the company finds a verbal interview helpful to assessing the candidates' qualifications. 
% Consider two types of hiring
% procedures: one consists of resume screening, followed by an interview of
% each candidate who was not rejected in reviewing resumes (sequential
% procedure), and a joint procedure in which a committee reviews resumes and
% interviews candidates at the same time (simultaneous procedure).

% Candidates can obviously tailor their resumes and answers for common
% interview questions to influence the interviewers' impression of their
% profiles. So, we analyze this example within our manipulation setting.
% Suppose that the cost of tailoring is $\onecost(\orifeatures,\features)$, where $%
% \orifeatures=(x_{A}^{0},x_{B}^{0})$ are the true attributes and $\features=(\feature_{A},\feature_{B})$ are
% the attributes tailored for the purpose of hiring, is equal to the Euclidean
% distance between $\orifeatures$ and $\features$. Under the sequential procedure, if a
% candidate chooses to tailor his resume, for example, to emphasize managerial
% experience, then he is constrained during the interview. He cannot present
% himself in a way that contradicts his resume. Hence, the total cost of
% manipulating is $\onecost(\orifeatures,\firstfeatures)+\onecost(\firstfeatures,\secondfeatures)$, where $\firstfeatures$ stands for
% attributes presented in the resume and $\secondfeatures$ stands the attributes presented
% in the interview. The candidate's payoff is the probability of being hired
% minus the cost of tailoring, be it one-step or two-step.

% Suppose that as in our model, the company's preferences are lexicographic:
% avoiding unqualified applicants is a primary objective, perhaps because
% hiring an unqualified candidate is very costly. For the sake of execution,
% the company must set some specific cutoffs in testing in practice. The
% question the company faces is whether to use a simultaneous or a sequential
% procedure (i.e., mechanism).\bigskip 
% Suppose a biotech company X is seeking candidates for a senior scientist
% position. Senior scientists must balance technical expertise and domain
% knowledge with managerial skills and leadership. Suppose X is looking for
% candidates that have some management experience and especially strong
% scientific expertise. Suppose, not unrealistically, that excessive
% experience as a manager may be an indication of career goal as a manager
% rather than a scientist, and even be a signal of declining technical
% expertise.

% We use \cref{fig: hiring} to illustrate X's requirements. The horizontal axis 
% represents managerial experience and the vertical axis  represents
% scientific expertise. The blue vertical line represents the minimum
% requirement on managerial experience, say managing one direct report. The
% red diagonal line represents the joint requirement on scientific expertise
% and managerial experience. The purple dashed region represents the set of
% attributes that X's considers as qualified for senior scientist position.

% \begin{figure}[h] 
% \centering 
% \begin{tikzpicture}[xscale=7.8,yscale=7.8]
% \draw [thick,->] (0.6,0.8) -- (1.5,0.8);
% \draw [thick, ->] (0.9,0.66) -- (0.9,1.33);
% \node [below] at (1.5,0.8) {management experience};
% \node [above] at (0.9,1.33) {science experience};

% \draw [domain=0.66:1.4, thick, red] plot (\x, {3/4*\x+1/4});
% \draw [thick, blue] (1,0.66) -- (1,5.2/4);
% % \node [left] at (1.3, 1.25 ) {$\classifier_B$};
% % \node [right] at (1, 1.25) {$\classifier_A$};
% \node [left] at (1.2,1.15) {$+$};
% \node [right] at (1, 1.15) {$+$};
% \node [left,font=\tiny] at (0.9, 0.78 ) {\footnotesize$O$};
% % \fill [purple!60,nearly transparent] (1,5.2/4) -- (1,1)-- (1.4,5.2/4)-- cycle;
% \path[pattern color=purple,pattern=north west lines] (1,5.2/4) -- (1,1)-- (1.4,5.2/4); 
% % \node [blue] at (1, 1 ) {\textbullet};
% % \node [right,font=\tiny] at (1, 1 ) {\footnotesize$(1,5)$};

% \end{tikzpicture}
% \caption{Hiring requirements for a senior scientist position}
% \label{fig: hiring}
%   \end{figure}
  
% Past experience in other biotech companies can reflect a candidate's
% suitability for the position. The resume provides some information, but does
% not fully reveal which of the two functions (manager or scientist) was the
% candidate's main duty. So, the company finds a verbal interview helpful to
% assessing the candidates' qualifications. Consider two types of hiring
% procedures: one consists of resume screening, followed by an interview of
% each candidate who was not rejected in reviewing resumes (sequential
% procedure), and a joint procedure in which a committee reviews resumes and
% interviews candidates at the same time (simultaneous procedure).

% Candidates can obviously tailor their resumes and answers for common
% interview questions to influence the interviewers' impression of their
% profiles. So, we analyze this example within our manipulation setting.
% Suppose that the cost of tailoring is $\onecost(\orifeatures,x)$, where $%
% \orifeatures$ are the true attributes and $\features$ are
% the attributes tailored for the purpose of hiring, is equal to the Euclidean
% distance between $\orifeatures$ and $\features$. Under the sequential procedure, if a
% candidate chooses to tailor his resume, for example, to emphasize managerial
% experience, then he is constrained during the interview. He cannot present
% himself in a way that contradicts his resume. Hence, the total cost of
% manipulating is $\onecost(\orifeatures,\firstfeatures)+\onecost(\firstfeatures,\secondfeatures)$, where $\firstfeatures$ stands for
% attributes presented in the resume and $\secondfeatures$ stands the attributes presented
% in the interview. The candidate's payoff is the probability of being hired
% minus the cost of tailoring, be it one-step or two-step.

% Suppose that as in our model, the company's preferences are lexicographic:
% avoiding unqualified applicants is a primary objective, perhaps because
% hiring an unqualified candidate is very costly. For the sake of execution,
% the company must set some specific cutoffs in testing in practice. The
% question the company faces is whether to use a simultaneous or a sequential
% procedure (i.e., mechanism).
% In practice, we often see that the interview process is sequential. This is consistent with our prediction that sequential mechanism is better when candidates manipulate their resume.

% \textbf{Promotion.}
% Now suppose the biotech company X is considering promoting a senior scientist to the principal scientist position.
% One crucial difference between promoting and hiring is that $\firstfeatures$ and $\secondfeatures$ are now true skills while they were fake in the hiring example.         
% In this case, the company cares about the candidates' eventual skills instead of their initial skills.
% % We  use \cref{fig: hiring} to illustrate. 
%  Everything else is the same as in the hiring example except that $\firstfeatures$ and $\secondfeatures$ are  true skills while they were fake in the hiring example. 


% The company faces a similar problem of designing evaluation procedures.
% The company has several options as candidate evaluation procedures: (1) simultaneous mechanisms: let the leadership set one deadline to evaluate both dimensions; (2) sequential mechanisms: set a deadline to evaluate the candidates' minimum management experience, and candidates who pass the first stage evaluation can go to the second stage evaluation on science experience with a different deadline. 
% In practice we often see that promotion evaluation is simultaneous. This is consistent with our model prediction that simultaneous mechanism is optimal when candidates invest to improve their true type.

% \paragraph{Education.}
% Take math test as an example.
% Math ability consists of at least two aspects.
% The first aspect is familiarity of math tools, including understanding of existing techniques and problems.
% We refer to the first aspect as technical skill.
% The second aspect is the ability to conduct mathematical thinking, which includes abstract thinking and logical reasoning ability.
% We refer to the second aspect as analytical skill.
% In some situations, math tests are offered to select individuals with qualified math ability, in which case it is the individuals' initial attributes that matter.
% In others, for example, in the course-end evaluations, math tests may be offered to encourage learning over the course.
% In this case, it is the individuals' eventual attributes that matter.







