\subsection{Robustness to tests and qualified region}\label{sec: perfect tests}

In this subsection, we assume that the principal only uses perfect tests, i.e., for any agent with attributes $\features$, a perfect test outputs $\features$.
This is motivated by the literature on persuasion with evidence \citep{glazer2004optimal,sher2014persuasion}.
In the persuasion games with evidence, (1) the agent's cost of changing attributes is infinitely high, and (2) the principal can use \emph{perfect} tests, i.e., principal can observe agent's attributes in each round of communication.
The main difference is that in our manipulation setting, the agent can fabricate their type at a cost.

All results in this subsection is derived under the manipulation setting and program \ref{max qualified}.
All proof is omitted from this section and provided in \cref{appendix: perfect tests}.

First we show that when the principal can use perfect tests and an upfront cheap-talk communication is feasible, she can achieve the first best.

\begin{proposition}\label{thm: sequential perfect tests}
% Consider the manipulation setting.
Suppose $M=\Featurespace$.
     % When the principal can use perfect tests, 
     For any distribution $\dist$, there exists a feasible fixed-order sequential mechanism that achieves the first best.   
     Moreover, this mechanism is adaptive, i.e., the second test depends on the first test.
\end{proposition}


The next corollary points out that a perfect test can convey information as in a cheap-talk communication. 

\begin{corollary}\label{cor: sequential perfect tests}
% Consider the manipulation setting.
Suppose $M=\nullset$.
     % Suppose the principal can use perfect tests and 
     Suppose the principal lets the agent to choose the first test, based on which the principal chooses the second test. For any distribution $\dist$, there exists a feasible fixed-order sequential mechanism that achieves the first best.   
\end{corollary}

 \cref{cor: sequential perfect tests} is a direct implication of \cref{thm: sequential perfect tests}. Under perfect tests, when the principal allows the agent to choose the first test, every agent chooses the first test that coincides with his true type.
 Based on the first test, the principal 
 can implement the same mechanism constructed in \cref{thm: sequential perfect tests}.
 
Next, we show that \cref{thm: sequential perfect tests} is not confined to the specific shape of the qualified region. As a matter of fact, as long as the qualified region is convex, the principal achieve the first best by using a sequential mechanism that offers tests in an adaptive manner.

\begin{proposition}\label{thm: perfect tests arbitrary qualified region}
% Consider the manipulation setting.
% When principal can use perfect tests and 
When the (true) qualified region $\qualregion$ is convex,  a sequential mechanism that uses two tests  achieves the first best.
\end{proposition}


Finally, we show that under a condition on the shape of the qualified region, a single-test mechanism, an analogy to a simultaneous mechanism, is strictly dominated by the above described sequential mechanism.


\begin{proposition}\label{thm: single-test mechanism perfect test}
% Consider the manipulation setting.
     There exists a single-test mechanism that achieves the first best if and only if the qualified region $\qualregion$ is the convex hull of the union of balls with radius $1/\mc$.
\end{proposition}



\cref{thm: single-test mechanism perfect test} implies that whenever there is a sharp corner (for instance, $\qualregion=\classifier_A\cap \classifier_B$, where $\classifier_i, i\in \{A,B\}$ is a half plane) in the true qualified region, a single-test mechanism cannot accept every qualified agent without accepting any unqualified agent. 
In other words, the best sequential mechanism (with two tests) strictly out performs a single-test mechanism under objective \ref{max qualified} and when effort is manipulation.
Only in those cases where the true qualified region has only round corner(s),  a single-test mechanism is equally good as the best sequential mechanism (with two tests).
Such examples include circles, rectangles with round corners, etc..