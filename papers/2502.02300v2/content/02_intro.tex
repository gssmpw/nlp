
\section{Introduction}
\label{sec:introduction}

Estimating the ratio of two densities is a fundamental task in machine learning, with diverse applications \citep{Sugiyama2010}. 
For instance, by assuming one of the densities to be of a tractable density, often a standard Gaussian, we can construct an estimator for an unknown density we can only sample from by estimating their ratio~\citep{gutmann2012nce,gao2019noiseadaptivence,Rhodes2020,choi2022densityratio}. It is also possible to consider a scenario where both sides are not tractable. As noted by previous works \citep{choi2022densityratio}, density ratio estimation finds broad applications across machine learning, from mutual information estimation \citep{song2020vmie}, generative modelling \citep{goodfellow2020generative}, importance sampling \citep{sinha2020neuralbridge}, likelihood-free inference \citep{izbicki2014} to domain adaptation \citep{Wang2023}.

\begin{figure}[ht]
\vskip 0.2in
\begin{center}
\centerline{\includegraphics[width=0.6\columnwidth]{image/illustration_time_score.pdf
}}
\caption{
Densities are shown in blue.
\textit{Left}: A bi-modal probability path transitioning from a Gaussian distribution (\(t = 0\)) to a mixture of Diracs (\(t = 1\)). This path is estimated using ``time scores", which are not available in closed form; they are depicted by arrows, with  magnitudes ranging from low (gray) to high (red).  
\textit{Right}: A useful decomposition of the probability path and time scores is obtained by \textit{conditioning} on a final data point. The ensuing \textit{conditional} density is Gaussian, and thus, the ensuing \textit{conditional} time scores are analytically tractable. We propose to use this decomposition to estimate the ``time scores". 
}
\label{fig:illustration}
\end{center}
\vskip -0.2in
\end{figure}


The seminal work by~\citet{gutmann2012nce} proposed a learning objective for estimating the ratio of two densities, 
by identifying from which density a sample is drawn. This can be done by binary classification. However, their estimator has a high variance when the densities have little overlap, which makes it impractical for problems in high dimensions~\citep{lee2023ncevariance,Chehab2023provable}.

To address this issue, \citet{Rhodes2020} proposed connecting the two densities with a probability path and estimating density ratios between consecutive distributions. Since two consecutive distributions are ``close" to each other, the statistical efficiency may improve at the cost of increased computation, as there are multiple binary classification tasks to solve. \citet{choi2022densityratio} examined the limiting case where the intermediate distributions become infinitesimally close. In this limit, the density ratio converges to a quantity known as the time score, which is learnt by optimizing a Time Score Matching (TSM) objective. While this limiting case leads to empirical improvements, the TSM objective is computationally inefficient to optimize, and the resulting estimator may be inaccurate. Moreover, it is unclear what are the theoretical guarantees associated with the estimators.

In this work, we address these limitations. First, in Section~\ref{sec:estimating_time_score} we introduce a novel learning objective for the time score, which we call \textit{Conditional Time Score Matching (CTSM)}. It is based on recent advancements in generative modeling \citep{vincent2011denoisingscorematching,pooladian2023conditionalflowmatching,tong2024conditionalflowmatching}, which consider probability paths that are explicitly decomposed into mixtures of simpler paths, and where the time score is obtained in closed form. We demonstrate empirically that the CTSM objective significantly accelerates optimization in high-dimensional settings, and is several times faster 
compared to TSM. 

Second, in Section~\ref{sec:design_choices} we modify our CTSM objective with a number of techniques that are popular in generative modeling~\citep{song2021sde,choi2022densityratio,tong2024conditionalflowmatching} to ease the learning. In particular, we derive a closed-form weighting function for the objective, as well as a vectorized version of the objective which we call \textit{Vectorized Conditional Time Score Matching (CTSM-v)}. Together, these modifications substantially improve the estimation of the density-ratio in high dimensions, leading to stable estimators and significant speedups. 

Third, in Section~\ref{sec:theoretical_guarantees} we provide theoretical guarantees for density ratio estimation using probability paths, addressing a gap in prior works~\citep{Rhodes2020,choi2022densityratio}. 



