

\section{Discussion}

\paragraph{Other estimators of time score}
In this paper, we compare time score estimators based on different learning objectives. An alternative is to use a simple Monte Carlo estimator, replacing the expectation in~\eqref{eq:mixture_score} with finite samples. Similarly, Monte Carlo methods can estimate other quantities like the Stein score~\citet{Scarvelis2024}, though they are rarely used in practice. Recent works suggest that estimators obtained by minimizing a learning objective are preferable when the neural network architecture is well-suited to modeling the time score~\citep{kamb2024analytic} or the Stein score~\citep{Kadkhodaie2024}. A more careful exploration of these estimation methods is left for future work.

\paragraph{Connections with generative modeling literature}
The learning objectives in this paper rely on probability paths that can be explicitly decomposed into mixtures of simpler probability paths. We used such simpler paths to compute the time score in closed form. Related literature has used these simpler paths to compute other quantities in closed form, such as the Stein score $\partial_{\bx} \log p_t(\bx | \bz)$~\citep{song2021sde}, or the velocity~\citep{lipman2023conditionalflowmatching,liu2023,Albergo2023,pooladian2023conditionalflowmatching,tong2024conditionalflowmatching} which is a vector field that transports samples from $p_0$ to $p_1$. 

\paragraph{Connections with multi-class classification}
Recent works have proposed to perform density ratio estimation by learning a multi-class classifier between \textit{all} intermediate distributions, instead of multiple binary classifiers between \textit{consecutive} intermediate distributions~\citep{Srivastava2023,yair2023multiclass,yadin2024classification}. Multi-class classification seems to empirically improve the estimation of the density ratio, but compared with TSM, it has limitations in high dimensions~\citep{Srivastava2023}. The limiting case where the intermediate distributions are infinitesimally close is an interesting direction for future work.

\paragraph{Optimal design choices}
In this work, we introduce novel estimators of the time score that depend on many design choices. One of them is the choice of probability path. \citet{Xu2024} considered using the learned approximate optimal transport path, \citet{Wu2024} considered using the learned approximate probability path given by annealing and \citet{Kimura2024} considered an information geometry formulation. Finding optimal probability paths, in the sense that the final error is minimized, is an active area of research, for example applied to estimating normalizing constants~\citet{Chehab2023optimizing},or sampling from challenging distributions~\citep{guo2024provablebenefitannealedlangevin}. Another important design choice is the weighting function that has been empirically investigated in related literature~\citep{kingma2023diffusion,chen2023noisescheduling}. A rigorous study of which design choice influences the final performance is left for future work.

\section{Conclusion} We propose a new method for learning density ratios. We address a number of problems in previous work \citep{Rhodes2020,choi2022densityratio} that culminated in the TSM objective. First, TSM is computationally inefficient, second, the resulting estimator can be inaccurate, and third, the theoretical guarantees are not clear. Inspired by recent advances in diffusion models and flow matching, we propose the CTSM objective and directly address these three limitations. CTSM drastically reduces the running times while improving the estimation accuracy of the density ratio, especially in higher dimensions. Additionally, we develop techniques for increasing the numerical stability through, for example, novel weighting functions. Finally, we provide theoretical guarantees on the resulting estimators.







