%File: anonymous-submission-latex-2025.tex
\documentclass[letterpaper]{article} % DO NOT CHANGE THIS
\usepackage{aaai25}  % DO NOT CHANGE THIS
\usepackage{times}  % DO NOT CHANGE THIS
\usepackage{helvet}  % DO NOT CHANGE THIS
\usepackage{courier}  % DO NOT CHANGE THIS
\usepackage[hyphens]{url}  % DO NOT CHANGE THIS
\usepackage{graphicx} % DO NOT CHANGE THIS
\urlstyle{rm} % DO NOT CHANGE THIS
\def\UrlFont{\rm}  % DO NOT CHANGE THIS
\usepackage{natbib}  % DO NOT CHANGE THIS AND DO NOT ADD ANY OPTIONS TO IT
\usepackage{caption} % DO NOT CHANGE THIS AND DO NOT ADD ANY OPTIONS TO IT
\frenchspacing  % DO NOT CHANGE THIS
\setlength{\pdfpagewidth}{8.5in} % DO NOT CHANGE THIS
\setlength{\pdfpageheight}{11in} % DO NOT CHANGE THIS
%
% These are recommended to typeset algorithms but not required. See the subsubsection on algorithms. Remove them if you don't have algorithms in your paper.
\usepackage{algorithm}
\usepackage{algorithmic}

\usepackage{multirow}
\usepackage{colortbl}
\usepackage{xcolor}
\usepackage{amsmath}
\usepackage{amssymb}

% 
% These are are recommended to typeset listings but not required. See the subsubsection on listing. Remove this block if you don't have listings in your paper.
\usepackage{newfloat}
\usepackage{listings}
\DeclareCaptionStyle{ruled}{labelfont=normalfont,labelsep=colon,strut=off} % DO NOT CHANGE THIS
\lstset{%
	basicstyle={\footnotesize\ttfamily},% footnotesize acceptable for monospace
	numbers=left,numberstyle=\footnotesize,xleftmargin=2em,% show line numbers, remove this entire line if you don't want the numbers.
	aboveskip=0pt,belowskip=0pt,%
	showstringspaces=false,tabsize=2,breaklines=true}
\floatstyle{ruled}
\newfloat{listing}{tb}{lst}{}
\floatname{listing}{Listing}
%
% Keep the \pdfinfo as shown here. There's no need
% for you to add the /Title and /Author tags.
\pdfinfo{
/TemplateVersion (2025.1)
}

% DISALLOWED PACKAGES
% \usepackage{authblk} -- This package is specifically forbidden
% \usepackage{balance} -- This package is specifically forbidden
% \usepackage{color (if used in text)
% \usepackage{CJK} -- This package is specifically forbidden
% \usepackage{float} -- This package is specifically forbidden
% \usepackage{flushend} -- This package is specifically forbidden
% \usepackage{fontenc} -- This package is specifically forbidden
% \usepackage{fullpage} -- This package is specifically forbidden
% \usepackage{geometry} -- This package is specifically forbidden
% \usepackage{grffile} -- This package is specifically forbidden
% \usepackage{hyperref} -- This package is specifically forbidden
% \usepackage{navigator} -- This package is specifically forbidden
% (or any other package that embeds links such as navigator or hyperref)
% \indentfirst} -- This package is specifically forbidden
% \layout} -- This package is specifically forbidden
% \multicol} -- This package is specifically forbidden
% \nameref} -- This package is specifically forbidden
% \usepackage{savetrees} -- This package is specifically forbidden
% \usepackage{setspace} -- This package is specifically forbidden
% \usepackage{stfloats} -- This package is specifically forbidden
% \usepackage{tabu} -- This package is specifically forbidden
% \usepackage{titlesec} -- This package is specifically forbidden
% \usepackage{tocbibind} -- This package is specifically forbidden
% \usepackage{ulem} -- This package is specifically forbidden
% \usepackage{wrapfig} -- This package is specifically forbidden
% DISALLOWED COMMANDS
% \nocopyright -- Your paper will not be published if you use this command
% \addtolength -- This command may not be used
% \balance -- This command may not be used
% \baselinestretch -- Your paper will not be published if you use this command
% \clearpage -- No page breaks of any kind may be used for the final version of your paper
% \columnsep -- This command may not be used
% \newpage -- No page breaks of any kind may be used for the final version of your paper
% \pagebreak -- No page breaks of any kind may be used for the final version of your paperr
% \pagestyle -- This command may not be used
% \tiny -- This is not an acceptable font size.
% \vspace{- -- No negative value may be used in proximity of a caption, figure, table, section, subsection, subsubsection, or reference
% \vskip{- -- No negative value may be used to alter spacing above or below a caption, figure, table, section, subsection, subsubsection, or reference


\newif\ifarxiv
\arxivtrue

\ifarxiv
\usepackage{fancyhdr}
\fi

\setcounter{secnumdepth}{0} %May be changed to 1 or 2 if section numbers are desired.

% The file aaai25.sty is the style file for AAAI Press
% proceedings, working notes, and technical reports.
%

% Title

% Your title must be in mixed case, not sentence case.
% That means all verbs (including short verbs like be, is, using,and go),
% nouns, adverbs, adjectives should be capitalized, including both words in hyphenated terms, while
% articles, conjunctions, and prepositions are lower case unless they
% directly follow a colon or long dash
\title{Wavelet-Assisted Multi-Frequency Attention Network for Pansharpening}

\author{
    Jie Huang\textsuperscript{\rm 1}\thanks{Co-first authors contributed equally.},
    Rui Huang\textsuperscript{\rm 1}\footnotemark[1],
    Jinghao Xu\textsuperscript{\rm 1},
    Siran Peng\textsuperscript{\rm 2, \rm 3},
    Yule Duan\textsuperscript{\rm 1},
    Liangjian Deng\textsuperscript{\rm 1}\thanks{ Corresponding author.}
}


\affiliations{
    %Afiliations
    \textsuperscript{\rm 1}University of Electronic Science and Technology of China\\
    \textsuperscript{\rm 2}Institute of Automation, Chinese Academy of Sciences\\
    \textsuperscript{\rm 3}School of Artificial Intelligence, University of Chinese Academy of Sciences\\
    \{jayhuang,huang\_rui,2023040901014\}@std.uestc.edu.cn,pengsiran2023@ia.ac.cn, yule.duan@outlook.com,liangjian.deng@uestc.edu.cn
}




% \ifarxiv
% {
%   \chead{\footnotesize This is a pre-print of the original paper accepted at the AAAI Conference on Artificial Intelligence 2025.}
%   \lhead{}
%   \thispagestyle{fancy}
% }
% \fi
\usepackage{fancyhdr}
\thispagestyle{fancy} % 仅作用于当前页面
\fancyhead{} % 清空默认页眉
\fancyhead[C]{\footnotesize This is a pre-print of the original paper accepted at the AAAI Conference on Artificial Intelligence 2025.}


\begin{document}

\maketitle

\begin{abstract}
Pansharpening aims to combine a high-resolution panchromatic (PAN) image with a low-resolution multispectral (LRMS) image to produce a high-resolution multispectral (HRMS) image. Although pansharpening in the frequency domain offers clear advantages, most existing methods either continue to operate solely in the spatial domain or fail to fully exploit the benefits of the frequency domain. To address this issue, we innovatively propose Multi-Frequency Fusion Attention (MFFA), which leverages wavelet transforms to cleanly separate frequencies and enable lossless reconstruction across different frequency domains. Then, we generate Frequency-Query, Spatial-Key, and Fusion-Value based on the physical meanings represented by different features, which enables a more effective capture of specific information in the frequency domain. Additionally, we focus on the preservation of frequency features across different operations. On a broader level, our network employs a wavelet pyramid to progressively fuse information across multiple scales. Compared to previous frequency domain approaches, our network better prevents confusion and loss of different frequency features during the fusion process. Quantitative and qualitative experiments on multiple datasets demonstrate that our method outperforms existing approaches and shows significant generalization capabilities for real-world scenarios. % Our code is available at \url{https://github.com/Jie-1203/WFANet}.

\end{abstract}

% Uncomment the following to link to your code, datasets, an extended version or similar.

\begin{links}
    \link{Code}{https://github.com/Jie-1203/WFANet}
    % \link{Datasets}{https://aaai.org/example/datasets}
    % \link{Extended version}{https://aaai.org/example/extended-version}
\end{links}


\section{Introduction}
High-resolution multispectral (HRMS) images are vital for applications like environmental monitoring and urban planning. Due to hardware constraints, satellites typically capture low-resolution multispectral (LRMS) and high-resolution panchromatic (PAN) images. Pansharpening fuses these to produce HRMS, combining their strengths to enhance spatial and spectral resolution.

To obtain high-resolution multispectral (HRMS) images, pansharpening methods are categorized into traditional and deep learning-based approaches. Traditional methods are divided into three groups \cite{overall}: Component Substitution (CS) \cite{CS},
\begin{figure}[!h]
\centering
\includegraphics[width=\columnwidth]{1.pdf} 
\caption{The comparison covers four methods across two dimensions: (a) Convolutional network in the spatial domain,
(b) Convolutional network in different frequency domains,
(c) Attention mechanism in the spatial domain, 
and (d) Our proposed method which forms the primary motivation for this paper: 1) utilizing wavelet transforms to process in different frequency domains; 2) designing an attention method with clear physical significance to leverage the advantages of frequency domain processing.}
\label{your_label}
\end{figure}
 Multi-Resolution Analysis (MRA) \cite{MRA}, and Variational Optimization-based (VO) \cite{VO} techniques. In recent years, with the rapid development of deep learning, many deep learning methods \cite{SSconv, Triple} have been proposed for pansharpening using convolutional neural networks (CNN), as shown in Fig.~\ref{your_label} (a), such as PNN \cite{PNN}, DiCNN \cite{DiCNN}, and LAGConv \cite{LAGConv}. These methods underscore deep learning's potential to improve pansharpening accuracy and efficiency. However, most existing methods do not process images in different frequency domains but instead operate in the original single spatial domain, thereby limiting the potential for improving fusion quality.

\begin{figure}[t]
\centering
\includegraphics[width=0.99\columnwidth]{2.pdf} 
\caption{
(a) DWT decomposes the image into four different frequency components. IDWT is the lossless inverse process of DWT. Multiple applications of DWT produce a multi-scale wavelet pyramid. 
(b) Simplified illustration of MFFA. Fusion-Value, Spatial-Key, and Frequency-Query are derived from the information indicated by the arrows. These components are then processed through an attention mechanism, enabling the reconstruction of features across different frequencies that integrate both spectral and spatial information.}
\label{fig1}
\end{figure}

Direct fusion in the spatial domain methods can often result in detail loss or blurring due to the imprecise separation of frequency information. 
In contrast, frequency-based methods can separate different frequencies for targeted processing, which better preserves hard-to-capture high-frequency information while effectively preventing interference between different frequencies.
Consequently, processing in different frequency domains can be a more effective approach for achieving better fusion results compared to spatial domain fusion methods, and some works \cite{jin2022laplacian, GuidedNet} have already attempted this approach. 
For example, AFM-DIN \cite{detail-injection} introduces a high-frequency injection module to enhance LRMS features with PAN details, but it may lose low-frequency information. FAMENet \cite{MOE} uses an expert mixture model to fuse different frequencies, effectively balancing both high-frequency and low-frequency information.
However, these methods often struggle to achieve clean separation, leading to interference and information loss due to the neural network's tendency to slightly blend frequency components together \cite{shan2021decouple}. To address these issues, we adopt a method that cleanly separates frequencies and enables lossless reconstruction.
As shown in Fig.~\ref{fig1} (a), the Discrete Wavelet Transform (DWT) cleanly separates frequency components \cite{wave1,wave2}. The Inverse Discrete Wavelet Transform (IDWT) is lossless, preserving all information. Repeated DWT builds a wavelet pyramid \cite{wave3}, efficiently handling multi-scale features and enhancing detail detection, offering advantages over other methods that attempt to separate 
frequency components.

Using wavelet transforms to fuse information across different frequency domains is an innovative approach. Designing appropriate networks and modules to effectively extract and combine these features is therefore crucial for achieving optimal fusion results. Some past methods have employed wavelet transforms for pansharpening, as shown in Fig.~\ref{your_label} (b), 
such as FAFNet \cite{FAFNet}, which utilizes DWT layers to extract and manage frequency-domain features, followed by IDWT layers that reconstruct these features into the spatial domain, with the final fusion achieved through a convolutional block that produces the high-quality fused image.
However, convolutional neural networks inherently excel at capturing low-frequency information and are less effective in capturing high-frequency details \cite{xu2019,CVIP}. This limitation leads to decreased fusion quality. 
To address this issue, we consider leveraging attention mechanisms that can more flexibly capture specific features \cite{soydaner2022attention}.
Some previous methods \cite{hou2024linearly,bidirectional} have attempted to use attention mechanisms in the spatial domain, as shown in Fig.~\ref{your_label} (c).
For example, PanFormer \cite{panformer} employs a customized Transformer architecture, using panchromatic (PAN) and low-resolution multispectral (LRMS) features as queries and keys for joint feature learning, modeling long-range dependencies to produce high-quality pan-sharpened results. 
However, such a design does not explicitly capture and fuse information from different frequencies.
In other visual domains, combining wavelet transforms with attention mechanisms, such as Wave-ViT \cite{wave-vit}, integrates wavelet transforms with Transformers. By performing invertible down-sampling within Transformer blocks, this method improves image recognition and enhances visual representation accuracy.


\begin{figure*}[!h]
\centering
\includegraphics[width=0.99\textwidth]{3.pdf} 
\caption{The overall workflow of our WFANet. Our network processes the data using multiple scales (only two scales are illustrated here for simplicity). WFANet consists of two sub-modules: the Multi-Frequency Fusion Attention (MFFA) and the Spatial Detail Enhancement Module (SDEM). The illustration of frequency features is shown on both sides of the figure.} 
\label{fig3}
\end{figure*}


Inspired by the above discussion, we design an innovative Wavelet-Assisted Multi-Frequency Attention Network called WFANet, \textit{with the core component being the Multi-Frequency Fusion Attention (MFFA).}
In our proposed MFFA, \textit{we introduce the concept of a Frequency Attention Triplet, which consists of Frequency-Query, Spatial-Key, and Fusion-Value}. As shown in Fig.~\ref{fig1} (b), we apply DWT to the PAN image, achieving a clean separation of different frequency features, effectively avoiding interference between them. To process and fuse this separated frequency information, unlike conventional attention mechanisms \cite{vaswani2017attention,soydaner2022attention}, our Frequency Attention Triplet has distinct physical significance: Frequency-Query represents frequency features, Spatial-Key encodes spatial information, and Fusion-Value represents the preliminary fusion of spatial and spectral features. This design effectively guides the information fusion process. The attention mechanism then captures correlations to achieve the initial fusion of frequency features, and IDWT is used for lossless reconstruction. Through MFFA, we can more effectively fuse information across different frequency domains and prevent the loss of frequency.
Additionally, inspired by previous methods \cite{FUsionNet, hou2023bidomain, Variational}, which demonstrated that enhancing spatial details significantly improves restoration when a module is primarily focused on fusion, as realized in our MFFA, \textit{we design the Spatial Detail Enhancement Module (SDEM) to focus on the extraction and enhancement of spatial details}.
In designing SDEM, we compare different operations for their adaptability to the frequency domain, ensuring better preservation of spatial details.
Moreover, our overall framework is a multi-scale progressive reconstruction framework, fully utilizing the inherent multi-scale nature of the wavelet pyramid.

In summary, the contributions of this work are as follows:
\begin{itemize}
    \item  
We introduce the Multi-Frequency Fusion Attention (MFFA), utilizing wavelet transforms to cleanly separate and accurately reconstruct frequency components. This approach integrates the Frequency-Query, Spatial-Key, and Fusion-Value triplet to enhance feature fusion precision across different frequency domains, effectively reducing confusion and information loss.
    \item Additionally, we focus on how different operations preserve frequency features and utilize the wavelet pyramid for progressive, multi-scale fusion. The effectiveness of these strategies has been validated and demonstrated through extensive ablation experiments.
    \item Our method achieves state-of-the-art performance on three diverse pansharpening datasets, demonstrating high-quality fusion results supported by both quantitative and qualitative experimental evidence.
\end{itemize}


\begin{figure*}[!h]
\centering
\includegraphics[width=0.99\textwidth]{4.pdf} 
\caption{
The MFFA workflow involves two phases. First, in the FATG phase, the Frequency Attention Triplet with specific physical significance is generated. Then, in the ADFR phase, the obtained Frequency Attention Triplet is processed to reconstruct the features at different frequencies. \( Q \), \( S \), and \( I \) are shown as four colored blocks, representing features from different frequency domains after DWT. The data dimensions are exemplified using the largest scale.}
\label{fig4}
\end{figure*}

\section{Proposed Method}
This section introduces the proposed WFANet, detailing its two key components: Multi-Frequency Fusion Attention (MFFA) and the Spatial Detail Enhancement Module (SDEM), followed by the overall multi-scale framework. Fig.~\ref{fig3} illustrates the workflow of WFANet.


\subsection{Multi-Frequency Fusion Attention (MFFA)}
To fuse information across frequencies, we propose the MFFA, which is composed of two phases: Frequency Attention Triplet Generation (FATG) and Attention-Driven Frequency Reconstruction (ADFR). Details are shown in Fig.~\ref{fig4}.


\subsubsection{(I) Frequency Attention Triplet Generation}

As in the typical attention mechanism \cite{vaswani2017attention,soydaner2022attention}
, Query, Key, and Value are the three components of attention. Query represents the information we seek, Key is the index of this information, and Value is the specific content. To better adapt to different frequency domains, we design the Frequency Attention Triplet. Specifically, different frequency features are the information we seek, the overall spatial features are the indices for querying different frequency features, and the specific content is the fusion of spectral and spatial information. Therefore, we design Frequency-Query, Spatial-Key, and Fusion-Value with specific physical meanings. We first perform a DWT on \( P \), which is the feature of the panchromatic (PAN) image after convolution, as shown below:

\begin{equation}
P_{LL}, P_{LH}, P_{HL}, P_{HH} = \operatorname{DWT}(P)
\label{DWT}
\end{equation}
where \( P_{LL} \) represents the low-frequency details of \( P \), and \( P_{LH} \), \( P_{HL} \), and \( P_{HH} \) represent the high-frequency details of \( P \) in the horizontal, vertical, and diagonal directions, respectively.
\textit{For convenience, \( i = LL, LH, HL, HH \) corresponds to the four frequency features mentioned above, respectively.}
The DWT operation has already separated the frequency features adequately, 
thus we directly use \( P_i \), representing different frequency features, as \( Q_i \).
Low-frequency features represent the overall spatial appearance of the image, 
while high-frequency information represents edge details and fine textures \cite{citation1}.
Therefore, to use the overall spatial features as our key, 
we directly take the low-frequency features represented by \( P_{LL} \) as the Spatial-Key.
Then, we design an \( f_v \) to fuse spatial and spectral information as the Fusion-Value.
Next, the three components are normalized using LayerNorm and processed through an MLP to enhance their expressiveness.
The specific process can be described by the following equations:
\begin{equation}
\begin{aligned}
    &Q_i = \operatorname{MLP}(\operatorname{LN}(P_i)) \\
    &K = \operatorname{MLP}(\operatorname{LN}(P_{LL})) \\
    &V = \operatorname{MLP}(\operatorname{LN}(f_v(M, P_{LL})))
\end{aligned}
\end{equation}
where  \( M \) represents the feature of the low-resolution multispectral (LRMS) image after convolution and \( f_v \) represents the process of obtaining the Fusion-Value by combining \( M \) and \( P_{LL} \) through convolution.


\subsubsection{(II) Attention-Driven Frequency Reconstruction}
After obtaining the Frequency Attention Triplet, we use the attention mechanism to reconstruct the fused features at different frequencies.
First, calculate the frequency correlation \( R_i \) between the Frequency-Query and Spatial-Key, representing the correlation between different frequency features and the overall spatial features. Then, apply softmax to obtain the Frequency Attention Map \( S_i \), which highlights the importance of different frequency features relative to the overall spatial features. The process is as follows:
\begin{equation}
\begin{aligned}
R_i &= Q_i \otimes K \\
S_i &= \operatorname{softmax}(R_i)
\end{aligned}
\end{equation}
where  \(\otimes\) represents matrix multiplication and softmax is an operation that converts input values into a probability distribution.
Next, the Frequency Attention Map \( S_i \) of different frequencies is separately multiplied by \( V \), which is the fusion of spectral information and low-frequency spatial information. Then, the result is processed through MLPs and residual connection to obtain the reconstructed features of different frequencies containing spectral information \( I_i \). This process can be described by the following equation:

\begin{figure}[t]
\centering
\includegraphics[width=0.75\columnwidth]{5.pdf} 
\caption{Comparison of two network architectures for the SDEM:
(a) Frequency Adaptation Block (FAB), which is used in the SDEM.
(b) Convolution Block (CB).
} 
\label{comparison}
\end{figure}

\begin{equation}
I_i = f_I(S_i \otimes V)
\end{equation}
where \( f_I \) represents the process of applying MLPs and residual connection to the result of the multiplication.
Finally, the preliminary reconstructed image \( F_M \) is obtained by leveraging the lossless property of the IDWT, as shown below:

\begin{equation}
F_M = \operatorname{IDWT}(I_{LL}, I_{LH}, I_{HL}, I_{HH})
\end{equation}
where \( I_{LL} \), \( I_{LH} \), \( I_{HL} \), and \( I_{HH} \) represent the features reconstructed at different frequencies. 


\subsection{Spatial Detail Enhancement Module (SDEM)}
% 13

The core component, MFFA, achieves the fusion of information across different frequency domains. 
In contrast, SDEM focuses on extracting and enhancing spatial detail information within these frequency domains.
First, we decompose \( P \) according to Eq.\ref{DWT} to obtain \( P_i \), features containing information from different frequencies, helping to prevent interference between them. Next, we extract \( f_i \), representing spatial information for different frequency features, separately using several Frequency Adaptation Blocks (FABs). An FAB is a block capable of adapting to different frequencies and is composed of a linear layer and a sigmoid activation function. This process is illustrated below:


\begin{equation}
\begin{aligned}
    f_i &= \operatorname{FABs}(P_i)
\end{aligned}
\end{equation}
where \( i = LL, LH, HL, HH \) correspond to the four different frequency features, respectively. As illustrated in Fig.~\ref{comparison}, we do not choose the Convolution Block. Given that convolutional networks struggle with extracting high-frequency information and perform poorly when processing different frequency domains \cite{xu2019,CVIP}, we opt for linear layers, which, as demonstrated by our ablation experiments, better adapt to different frequency domains. After extracting features in different frequency domains, we use the lossless IDWT to recover the complete spatial details \( F_S \), as illustrated below:

\begin{equation}
F_{S} = \operatorname{IDWT}(f_{LL}, f_{LH}, f_{HL}, f_{HH})
\end{equation}


\subsection{Network Framework and Loss}

This section describes how to utilize the wavelet pyramid to 
construct the multi-scale network architecture of WFANet with MFFA and SDEM.
Our network employs a multi-scale structure with \( N \) layers (limited by the dataset, we use two scales in this paper).
First, we construct a wavelet pyramid by repeatedly applying DWT, as follows:
\begin{equation}
P_{k} = \operatorname{DWT}(P_{k+1})
\label{down}
\end{equation}
where \( P_{k+1} \) represents the four different frequency features of the previous larger scale, and \( P_k \) represents the frequency features of the next smaller scale.
Then, we progressively fuse from the smallest scale.
\( P_k \) and \( M_k \) are the inputs at the \( k \)-th scale.
The output of each layer is obtained by adding the output \( F_M \) from the MFFA and the output \( F_S \) from the SDEM.
The output of the \( k \)-th layer serves as the input \( M_{k+1} \) for the next layer, which in turn enables the process of progressive reconstruction, expressed as follows:


\begin{equation}
\left\{
\begin{aligned}
    &M_1 = f(M_0, P_0) \\
    &M_2 = f(M_1, P_1) \\
    &\phantom{M_1, P_1} \vdots \\
    &M_n = f(M_{n-1}, P_{n-1}) 
\end{aligned}
\right.
\end{equation}
where \(n\) represents the number of scales. \(M_n\) is then convolved to get the final high-resolution multispectral (HRMS) image \(\hat{M}\). 
We choose the simple $\ell_{1}$ loss function since it is sufficient to yield consistently good outcomes: 
\begin{equation}
\label{loss}
\mathcal{L} = \frac{1}{K}\sum_{i=1}^{K} \| \hat{M}^{\{i\}} - I^{\{i\}} \|_1 
\end{equation}
where \(K\) is the number of training data, \( I^{\{i\}} \) denotes the $i$-th ground truth image, and \( \|\cdot\|_1 \) represents the $\ell_{1}$ norm.


\section{Experiments}


\begin{figure*}[!h]
\centering
\includegraphics[width=0.99\textwidth]{exp1.png} 
\caption{ The visual results (Top) and residuals (Bottom) of all compared approaches on the WV3 reduced-resolution dataset.} 
\label{qualitative-wv3}
\end{figure*}


\begin{table*}[h]
\centering
\begin{tabular}{c@{\hskip 0.02in}c@{\hskip 0.02in}c@{\hskip 0.02in}c@{\hskip 0.02in}c|c@{\hskip 0.02in}c@{\hskip 0.02in}c@{\hskip 0.02in}c|c@{\hskip 0.02in}c@{\hskip 0.02in}c@{\hskip 0.02in}c}
\hline
\multirow{2}{*}{\textbf{Methods}} & \multicolumn{4}{c|}{\textbf{WV3 }} & \multicolumn{4}{c|}{\textbf{QB }} & \multicolumn{4}{c}{\textbf{GF2 }} \\  
 & \textbf{PSNR$\uparrow$} & \textbf{SAM$\downarrow$} & \textbf{ERGAS$\downarrow$} & \textbf{Q8$\uparrow$} & \textbf{PSNR$\uparrow$} & \textbf{SAM$\downarrow$} & \textbf{ERGAS$\downarrow$} & \textbf{Q4$\uparrow$} & \textbf{PSNR$\uparrow$} & \textbf{SAM$\downarrow$} & \textbf{ERGAS$\downarrow$} & \textbf{Q4$\uparrow$} \\ \hline
MTF-GLP-FS & 32.963 & 5.316 & 4.700 & 0.833 & 32.709 & 7.792 & 7.373 & 0.835 & 35.540 & 1.655 & 1.589 & 0.897 \\  
BDSD-PC & 32.970 & 5.428 & 4.697 & 0.829 & 32.550 & 8.085 & 7.513 & 0.831 & 35.180 & 1.681 & 1.667 & 0.892 \\ 
TV & 32.381 & 5.692 & 4.855 & 0.795 & 32.136 & 7.510 & 7.690 & 0.821 & 35.237 & 1.911 & 1.737 & 0.907 \\ \hline

PNN & 37.313 & 3.677 & 2.681 & 0.893 & 36.942 & 5.181 & 4.468 & 0.918 & 39.071 & 1.048 & 1.057 & 0.960 \\ 
PanNet & 37.346 & 3.613 & 2.664 & 0.891 & 34.678 & 5.767 & 5.859 & 0.885 & 40.243 & 0.997 & 0.919 & 0.967 \\ 
DiCNN & 37.390 & 3.592 & 2.672 & 0.900 & 35.781 & 5.367 & 5.133 & 0.904 & 38.906 & 1.053 & 1.081 & 0.959 \\ 
FusionNet & 38.047 & 3.324 & 2.465 & 0.904 & 37.540 & 4.904 & 4.156 & 0.925 & 39.639 & 0.974 & 0.988 & 0.964 \\ 


U2Net & \underline{39.117} & \underline{2.888} & \underline{2.149} & \underline{0.920} & 38.065 & 4.642 & 3.987 & 0.931 & 43.379 & 0.714 & 0.632 & 0.981 \\ 
PanMamba & 39.012 & 2.913 & 2.184 & 0.920 & 37.356 & 4.625 & 4.277 & 0.929 & 42.907 & 0.743 & 0.684 & 0.982 \\ 
CANNet & 39.003 & 2.941 & 2.174 & 0.920 & \underline{38.488} & \underline{4.496} & \underline{3.698} & \underline{0.937} & \underline{43.496} & \underline{0.707} & \underline{0.630} & \underline{0.983} \\  
 \textbf{Proposed} & \textbf{39.345} & \textbf{2.849} & \textbf{2.093} & \textbf{0.922} & \textbf{38.822} & \textbf{4.392} & \textbf{3.556} & \textbf{0.940} & \textbf{43.913} & \textbf{0.685} & \textbf{0.597} & \textbf{0.985} \\ \hline
\end{tabular}
\caption{Comparisons on WV3, QB, and GF2 datasets with 20 reduced-resolution samples, respectively. Best: \textbf{bold}, and second-best: \underline{underline}.}
\label{reduced}
\end{table*}


\subsection{Datasets and Benchmark}

To benchmark the effectiveness of our network for pansharpening, 
we adopt various datasets, 
including datasets captured by the WorldView-3 (WV3), GaoFen-2 (GF2), and QuickBird (QB) sensors.
Since ground truth (GT) images are not available, Wald's protocol \cite{wald1997fusion} is applied. Each training dataset consists of PAN, LRMS, and GT image pairs with sizes of 64 × 64, 16 × 16 × 8, and 64 × 64 × 8, respectively. 
We obtain our datasets and data processing methods from the PanCollection repository\footnote{\url{https://github.com/liangjiandeng/PanCollection}} 
\cite{deng2022machine}.
To evaluate the proposed method, several state-of-the-art pansharpening methods are selected, including three traditional methods, MTF-GLP-FS \cite{MRA}, BDSD-PC \cite{CS}, and TV \cite{TV}, and seven deep-learning methods, including PNN, PanNet \cite{PanNet}, DiCNN, FusionNet \cite{FUsionNet}, U2Net \cite{U2Net}, PanMamba \cite{panMamba}, and CANNet \cite{CANNet}.


\subsection{Implementation Details}
We implement our network using the PyTorch framework on an RTX 4090D GPU. The learning rate is set to \( 9 \times 10^{-4} \) and is halved every 90 epochs. The model is trained for 360 epochs with a batch size of 32. The Adam optimizer is employed. Our method's performance is assessed using standard pansharpening metrics including SAM \cite{SAM}, ERGAS \cite{ERGAS}, and Q4/Q8 \cite{Q4} for reduced-resolution datasets, and HQNR \cite{HQNR}, D$_s$, and D$_\lambda$ for full-resolution datasets.


\subsection{Comparison with State-of-the-Art Methods}

\begin{figure*}[!h]
\centering
\includegraphics[width=0.99\textwidth]{exp2.png} 
\caption{The visual results (Top) and residuals (Bottom) of all compared approaches on the GF2 reduced-resolution dataset.} 
\label{qualitative-gf2}
\end{figure*}


\begin{table*}[h]
\centering
\begin{tabular}{ c| c@{\hskip 0.05in}c@{\hskip 0.05in}c@{\hskip 0.05in}c@{\hskip 0.05in}c@{\hskip 0.05in}c@{\hskip 0.05in}c@{\hskip 0.05in}c@{\hskip 0.05in}c@{\hskip 0.05in}c@{\hskip 0.05in}c }
\hline
\textbf{Metric} & \textbf{MTF-GLP-FS} & \textbf{BDSD-PC} & \textbf{TV} & \textbf{PNN} & \textbf{PanNet} & \textbf{DiCNN} & \textbf{FusionNet} & \textbf{U2Net} & \textbf{PanMamba} & \textbf{CANNet} & \textbf{Proposed} \\ \hline
\textbf{D$_\lambda \downarrow$} & 0.020 & 0.063 & 0.023 & 0.021 & \textbf{0.017} & 0.036 & 0.024 & 0.020 & 0.018 & 0.020 & \textbf{0.017} \\ 
\textbf{D$_s \downarrow$} & 0.063 & 0.073 & 0.039 & 0.043 & 0.047 & 0.046 & 0.036 & \underline{0.028} & 0.053 & 0.030 & \textbf{0.027} \\ 
\textbf{HQNR$\uparrow$} & 0.919 & 0.870 & 0.938 & 0.937 & 0.937 & 0.920 & 0.941 & \underline{0.952} & 0.930 & 0.951  &  \textbf{0.957} \\ \hline

\end{tabular}
\caption{Quantitative comparisons on the WV3 full-resolution dataset.}
\label{WV3_full}
\end{table*}


\subsubsection{Reduced-Resolution Assessment}
Table \ref{reduced} clearly shows a comparison of our proposed method with the current best methods across three datasets. Our method
consistently achieves the best results across all metrics. Specifically, our method achieves a PSNR improvement of 0.228dB, 0.334dB, and 0.417dB on the WV3, QB, and GF2 datasets, respectively, compared to the second-best results. These improvements highlight the clear advantages of our method, confirming its competitiveness in the field. Fig.~\ref{qualitative-wv3} and Fig.~\ref{qualitative-gf2} show the qualitative assessment results for two datasets and their corresponding ground truth (GT). By comparing the MSE residuals between the pan-sharpened results and the ground truth, it is evident that our residual maps are the darkest, indicating that our method outperforms others. The experimental results above demonstrate that our method is superior to the latest state-of-the-art pansharpening methods.

\begin{table}[H]
\centering
\setlength{\tabcolsep}{4pt}
\begin{tabular}{c c c c c}
\hline
\textbf{Ablation} & \textbf{PSNR$\uparrow$} & \textbf{SAM$\downarrow$} & \textbf{ERGAS$\downarrow$} & \textbf{Q8$\uparrow$} \\ \hline
Frequency-Query & 38.884 & 2.968 & 2.206 & 0.918 \\ 
Spatial-Key & 39.156 & 2.901 & 2.138 & 0.921 \\ 
Fusion-Value & 38.921 & 2.971 & 2.203 & 0.918 \\ 
\textbf{Ours} & \textbf{39.345} & \textbf{2.849} & \textbf{2.093} & \textbf{0.922} \\ 
\hline
\end{tabular}
\caption{Ablation experiment about Attention Triplet on WV3 reduced-resolution dataset.}
\label{sao}
\end{table}

\subsubsection{Full-Resolution Assessment}
To demonstrate the generalization ability of our method, we conduct experiments on full-resolution samples of WV3. The quantitative evaluation results are shown in Table \ref{WV3_full}. Our method achieves the best HQNR results, which reflects its ability to balance spectral and spatial distortions, demonstrating its high application value.


\begin{table}[h]
\centering
\begin{tabular}{ l c c c c}
\hline
\textbf{Ablation} & \textbf{PSNR$\uparrow$} & \textbf{SAM$\downarrow$} & \textbf{ERGAS$\downarrow$} & \textbf{Q8$\uparrow$} \\ \hline
MFFA & 38.486 & 3.148 & 2.349 & 0.915 \\ 
SDEM & 38.993 & 2.937 & 2.178 & 0.918 \\ 
Multi-Scale & 38.851 & 2.995 & 2.214 & 0.916 \\ 
FAB & 39.074 & 2.919 & 2.155 & 0.919 \\ 
 \textbf{Ours} & \textbf{39.345} & \textbf{2.849} & \textbf{2.093} & \textbf{0.922} \\ \hline
\end{tabular}
\caption{Ablation experiment about key components and strategy on WV3 reduced-resolution dataset.}
\label{ablation}
\end{table}


\subsection{Ablation Study}
This section explores the rationale behind the design of the Frequency Attention Triplet and the roles of key components and strategies in WFANet. We conduct a series of ablation experiments on the WV3 dataset to demonstrate their effectiveness and validity. First, Table \ref{sao} presents three sets of ablation experiments for Frequency Attention Triplet, followed by Table \ref{ablation}, which shows four sets of ablation experiments for WFANet. \textit{More experiments and detailed ablation settings can be found in the supplementary materials.}

\subsubsection{Frequency Attention Triplet}
To demonstrate the effectiveness of Frequency-Query, we remove the DWT operation and directly use a spatial domain \(Q\) instead of using \(Q_i\) in the different frequency domains. For Spatial-Key, we no longer use \(P_{LL}\) obtained by DWT, which represents the overall spatial features, but instead use features from \(P\) after convolution, introducing interference from high-frequency information. Regarding Fusion-Value, we no longer use the fusion of LRMS and \(P_{LL}\), but instead include only the information from LRMS, thereby lacking the low-frequency spatial information represented by \(P_{LL}\). The results in Table \ref{sao} demonstrate the effectiveness of each component in the Frequency Attention Triplet.

\subsubsection{Multi-Frequency Fusion Attention}
To demonstrate the effectiveness of the attention mechanism in MFFA, we replace it with the convolutional network, where HRMS and different frequency features are concatenated and then processed through convolution. The results in Table \ref{ablation} prove it.

\subsubsection{Spatial Detail Enhancement Module}
The SDEM enhances reconstructed images by injecting spatial details from frequency domains. We remove the SDEM while retaining the MFFA, and the lack of spatial detail is noticeable. The results in Table \ref{ablation} confirm the effectiveness of the SDEM.

\subsubsection{The Multi-Scale Training Strategy}
To demonstrate the effectiveness of the multi-scale strategy, we replace the multi-scale network in this paper with a single-scale network, aligning the sizes using convolution operations. The results in Table \ref{ablation} confirm the significance of this strategy.

\subsubsection{Frequency Adaptation Block}
We explore how operations adapt to frequency domains. Fig.~\ref{comparison} shows two network architectures for the SDEM. We replace the FABs with Convolution Blocks, and Table \ref{ablation} shows the superior feature extraction across frequencies provided by the FABs.


\section{Conclusion}
In this paper, we propose a novel approach, Multi-Frequency Fusion Attention (MFFA), which leverages an effective method for frequency decomposition and reconstruction. By designing Frequency-Query, Spatial-Key, and Fusion-Value with clear physical significance, MFFA achieves more effective and precise fusion in the frequency domain. We also emphasize the adaptation of different operations to the frequency domain and have designed a comprehensive multi-scale fusion strategy. Ablation experiments further confirm the effectiveness of our approach. Extensive experiments on three different satellite datasets demonstrate that our model outperforms state-of-the-art methods.


\section{Acknowledgments}
This research is supported by the National Natural Science Foundation of China (12271083), and the Natural Science Foundation of Sichuan Province (2024NSFSC0038).

\bibliography{aaai25}

% \bigskip
% \noindent 

% \bibliography{aaai25}

% % ICCV 2025 Paper Template; see https://github.com/cvpr-org/author-kit

\documentclass[10pt,twocolumn,letterpaper]{article}

%%%%%%%%% PAPER TYPE  - PLEASE UPDATE FOR FINAL VERSION
% \usepackage{iccv}              % To produce the CAMERA-READY version
\usepackage[review]{iccv}      % To produce the REVIEW version
% \usepackage[pagenumbers]{iccv} % To force page numbers, e.g. for an arXiv version

% Import additional packages in the preamble file, before hyperref
%
% --- inline annotations
%
\newcommand{\red}[1]{{\color{red}#1}}
\newcommand{\todo}[1]{{\color{red}#1}}
\newcommand{\TODO}[1]{\textbf{\color{red}[TODO: #1]}}
% --- disable by uncommenting  
% \renewcommand{\TODO}[1]{}
% \renewcommand{\todo}[1]{#1}



\newcommand{\VLM}{LVLM\xspace} 
\newcommand{\ours}{PeKit\xspace}
\newcommand{\yollava}{Yo’LLaVA\xspace}

\newcommand{\thisismy}{This-Is-My-Img\xspace}
\newcommand{\myparagraph}[1]{\noindent\textbf{#1}}
\newcommand{\vdoro}[1]{{\color[rgb]{0.4, 0.18, 0.78} {[V] #1}}}
% --- disable by uncommenting  
% \renewcommand{\TODO}[1]{}
% \renewcommand{\todo}[1]{#1}
\usepackage{slashbox}
% Vectors
\newcommand{\bB}{\mathcal{B}}
\newcommand{\bw}{\mathbf{w}}
\newcommand{\bs}{\mathbf{s}}
\newcommand{\bo}{\mathbf{o}}
\newcommand{\bn}{\mathbf{n}}
\newcommand{\bc}{\mathbf{c}}
\newcommand{\bp}{\mathbf{p}}
\newcommand{\bS}{\mathbf{S}}
\newcommand{\bk}{\mathbf{k}}
\newcommand{\bmu}{\boldsymbol{\mu}}
\newcommand{\bx}{\mathbf{x}}
\newcommand{\bg}{\mathbf{g}}
\newcommand{\be}{\mathbf{e}}
\newcommand{\bX}{\mathbf{X}}
\newcommand{\by}{\mathbf{y}}
\newcommand{\bv}{\mathbf{v}}
\newcommand{\bz}{\mathbf{z}}
\newcommand{\bq}{\mathbf{q}}
\newcommand{\bff}{\mathbf{f}}
\newcommand{\bu}{\mathbf{u}}
\newcommand{\bh}{\mathbf{h}}
\newcommand{\bb}{\mathbf{b}}

\newcommand{\rone}{\textcolor{green}{R1}}
\newcommand{\rtwo}{\textcolor{orange}{R2}}
\newcommand{\rthree}{\textcolor{red}{R3}}
\usepackage{amsmath}
%\usepackage{arydshln}
\DeclareMathOperator{\similarity}{sim}
\DeclareMathOperator{\AvgPool}{AvgPool}

\newcommand{\argmax}{\mathop{\mathrm{argmax}}}     



% It is strongly recommended to use hyperref, especially for the review version.
% hyperref with option pagebackref eases the reviewers' job.
% Please disable hyperref *only* if you encounter grave issues, 
% e.g. with the file validation for the camera-ready version.
%
% If you comment hyperref and then uncomment it, you should delete *.aux before re-running LaTeX.
% (Or just hit 'q' on the first LaTeX run, let it finish, and you should be clear).
\definecolor{iccvblue}{rgb}{0.21,0.49,0.74}
\usepackage[pagebackref,breaklinks,colorlinks,allcolors=iccvblue]{hyperref}
\usepackage{overpic}
\usepackage{makecell}
\usepackage{adjustbox} 
%\usepackage{float}
%\usepackage{arydshln}
\usepackage{multirow}
\usepackage{float}
\usepackage{bbding}
\usepackage{times}
\usepackage{epsfig}
\usepackage{graphicx}
\usepackage{amsmath}
\usepackage{amsthm}
\usepackage{amssymb}
\usepackage{booktabs}
\usepackage{xcolor}
\usepackage{enumitem}
\usepackage{lipsum}
\usepackage{diagbox}

\usepackage{algorithmic} % 引入 algorithmic 包
\usepackage{algpseudocode} % 引入 algpseudocode 包(如果需要更现代的控制结构)

\newcommand\blfootnote[1]{%
  \begingroup
  \renewcommand\thefootnote{}\footnote{#1}%
  \addtocounter{footnote}{-1}%
  \endgroup
}

\usepackage[ruled,vlined,linesnumbered]{algorithm2e}
\makeatletter
\newcommand{\algorithmfootnote}[2][\footnotesize]{%
  \let\old@algocf@finish\@algocf@finish% Store algorithm finish macro
  \def\@algocf@finish{\old@algocf@finish% Update finish macro to insert "footnote"
    \leavevmode\rlap{\begin{minipage}{\linewidth}
    #1#2
    \end{minipage}}%
  }%
}
\makeatother
%%%%%%%%% PAPER ID  - PLEASE UPDATE
\def\paperID{5291} % *** Enter the Paper ID here
\def\confName{ICCV}
\def\confYear{2025}
\newtheorem{theorem}{Theorem}[section]
\newtheorem{lemma}[theorem]{Lemma}
%%%%%%%%% TITLE - PLEASE UPDATE
\title{Q-PETR: Quant-aware Position Embedding Transformation for Multi-View 3D Object Detection\\------------ Supplementary Material ------------}

%%%%%%%%% AUTHORS - PLEASE UPDATE
\author{First Author\\
Institution1\\
Institution1 address\\
{\tt\small firstauthor@i1.org}
% For a paper whose authors are all at the same institution,
% omit the following lines up until the closing ``}''.
% Additional authors and addresses can be added with ``\and'',
% just like the second author.
% To save space, use either the email address or home page, not both
\and
Second Author\\
Institution2\\
First line of institution2 address\\
{\tt\small secondauthor@i2.org}
}

\begin{document}
\maketitle

\section{Preliminaries}
\label{sec:petr_preliminaries}

\paragraph{PETR} enhances 2D image features with 3D position-aware properties using camera-ray positional encoding (PE), enabling refined query updates for 3D bounding box prediction. Specifically, surround-view images $\mathbf{I}$ pass through a backbone to generate 2D features $\mathbf{f}_{2D}$, while camera-ray PE $\mathbf{p}_c$ is computed using camera intrinsics and extrinsics. The learnable query embeddings $q$ serve as the initial queries $\mathbf{Q}$ for the decoder. Here, $\mathbf{f}_{2D}$ serves as the values $\mathbf{V}$, and adding $\mathbf{p}_c$ to $\mathbf{f}_{2D}$ element-wise forms the 3D position-aware keys $\mathbf{K}$.

The decoder updates the queries using these key-value pairs through self-attention, cross-attention, and feed-forward network (FFN) modules. The updated query vectors are passed through an MLP to predict 3D bounding box categories and attributes, repeating for $L$ cycles. The entire PETR process is summarized in Algorithm~\ref{algo:algorithm_petr}.
\begin{algorithm}[htb]
\SetAlgoLined
\DontPrintSemicolon
\SetNoFillComment
\SetInd{1em}{1em} % <-- 关键:取消缩进
\footnotesize

\KwData{Surround-view images $\mathbf{I}$, camera intrinsics and extrinsics}
\KwResult{3D bounding boxes $\mathbf{b}^l$, categories $\mathbf{c}^l$ for $l = 1$ to $L$}

Compute image features: $\mathbf{f}_{2D} = \text{Backbone}(\mathbf{I})$\\
Compute camera-ray PE $\mathbf{p}_c$ using camera intrinsics and extrinsics\\
Form 3D position-aware keys: $\mathbf{K} = \mathbf{f}_{2D} + \mathbf{p}_c$ \tcp{Element-wise addition}
Set values: $\mathbf{V} = \mathbf{f}_{2D}$ \\
Initialize queries: $\mathbf{Q} = q$ (For simplicity, omit \(\mathbf{Q}\)'s encoding.)\\
\For{$l = 1$ to $L$}{
  $\mathbf{Q} \gets \texttt{QProj}(\mathbf{Q})$; 
  $\mathbf{K} \gets \texttt{KProj}(\mathbf{K})$; 
  $\mathbf{V} \gets \texttt{VProj}(\mathbf{V})$ \\ 
  $\mathbf{A}_s = \texttt{MultiHeadAtt}(\mathbf{Q}, \mathbf{Q}, \mathbf{Q})$ \tcp{Self-Attn}
  $\mathbf{A}_c = \texttt{MultiHeadAtt}(\mathbf{A}_s, \mathbf{K}, \mathbf{V})$ \tcp{Cross-Attn}
  $\mathbf{Q} \gets \texttt{FFN}(\mathbf{Q} + \mathbf{A}_c)$ \\
  $\mathbf{b}^{l} \gets \texttt{MLP}(\mathbf{Q})$; 
  $\mathbf{c}^{l} \gets \texttt{MLP}(\mathbf{Q})$ \\
}
\Return{$(\mathbf{b}^l, \mathbf{c}^l)$ for $l = 1$ to $L$}

\caption{Pseudo-code of PETR.}
\label{algo:algorithm_petr}
\end{algorithm}



\iffalse
\section{Quantization Failure of PETR}
\label{sec:quantization_failure_of_petr}

We evaluate the performance of several PETR configurations~\cite{liu2022petr} using the official code. Under standard 8-bit symmetric per-tensor post-training quantization (PTQ), PETR suffers significant performance degradation, with an average drop of 58.2\% in mAP and 36.9\% in NDS on the nuScenes validation dataset (see Table~\ref{tab:performance_drop_for_ptq_on_raw_petr}). 

\begin{table}[htb] %{0.45\linewidth}
    %\tiny
    %\scriptsize
    \footnotesize
    \setlength{\tabcolsep}{1.4mm}
    %\small
    %\setlength{\tabcolsep}{2.5mm}
    %\normalsize
    %\large
    \centering
    \begin{tabular}{l|c|c|c|c|c|c}
    \toprule[1.5pt]
    \multirow{2}{*}{Bac} & \multirow{2}{*}{Size} & \multirow{2}{*}{Feat} & \multicolumn{2}{c|}{FP32 Acc}                               & \multicolumn{2}{c}{INT8 Acc}   \\ \cline{4-7}
     & & & \multicolumn{1}{c|}{mAP} & \multicolumn{1}{c|}{NDS} & \multicolumn{1}{c|}{mAP} & \multicolumn{1}{c}{NDS} \\ 
    \midrule
    \textcolor{white}{0}R50\textcolor{white}{0} & 1408$\times$512 & c5 & 30.5 & 35.0 & 18.4(12.1$\downarrow$) & 27.3(\textcolor{white}{0}7.7$\downarrow$) \\
    \textcolor{white}{0}R50\textcolor{white}{0} & 1408$\times$512 & p4 & 31.7 & 36.7 & 15.7(16.0$\downarrow$) & 26.1(10.6$\downarrow$) \\
    V2-99 & \textcolor{white}{0}800$\times$320 & p4 & 37.8 & 42.6 & 10.9(26.9$\downarrow$) & 23.6(19.0$\downarrow$) \\
    V2-99 & 1600$\times$640 & p4 & 40.4 & 45.5 & 11.3(29.1$\downarrow$) & 23.9(21.6$\downarrow$) \\
    \bottomrule[1.5pt]
    \end{tabular}
    \vspace{-0.3cm}
    \caption{
    PETR's performance of 3D object detection on nuSences val set, directly utilizing the pre-trained parameters from the official repository.
    }
    %\vspace{-0.5cm}
    \label{tab:performance_drop_for_ptq_on_raw_petr}
\end{table}

\iffalse
Enlightened by the existing state-of-the-art post-training quantization methods~\cite{dong2019hawq1,dong2020hawq2,hubara2021adaq,li2021brecq,liu2021ptqvit,nagel2020up,yao2021hawq3}, the adaptive rounding proxy objective is introduced to measure the quantization performance degradation:
\begin{equation}
\begin{aligned}
    \label{eqnAdaptiveEll}
    \textstyle
    \ell(\widetilde W)
    &=\mathbf{E}_x\left[ \Vert{ (W- \widetilde W) x \Vert}^2 \right] \\
    &=tr\left( (W - \widetilde W) H (W - \widetilde W)^T \right)
\end{aligned}
\end{equation}
Where $W$ is the float weight tensor of a learnable operator, $\widetilde W$ are the quantized weight, 
$x$ is an input tensor sampled randomly from a calibration set, 
and $H$ is the second moment matrix of these vectors, interpreted as a proxy Hessian.
\fi

\paragraph{Layer-wise Quantization Error Analysis.} Quantizing a pre-trained network introduces output noise, degrading performance. To identify the root causes of quantization failure, we employ the signal-to-quantization-noise ratio (SQNR), inspired by recent PTQ advancements~\cite{pandey2023practical, yang2023efficient, pagliari2023plinio}:

\begin{equation}\label{eq:sqnr}
SQNR_{q,b} = 10\log_{10} \left( \frac{ \sum_{i=1}^N \mathbb{E}[{\mathcal{F}}{\theta}(x_{i})^{2} ] }{ \sum_{i=1}^N \mathbb{E}[ e(x_{i})^{2} ] } \right)
\end{equation}

Here, $N$ is the number of calibration data points; $\mathcal{F}{\theta}$ denotes the full-precision network; the quantization error is $e(x_i) = \mathcal{F}{\theta}(x_i) - \mathcal{Q}{q,b}(\mathcal{F}{\theta}(x_i))$; and $\mathcal{Q}_{q,b}(\mathcal{F}_\theta)$ denotes the network output when only the target layer is quantized to $b$ bits, with all other layers kept at full precision.

Since 8-bit weight quantization results in only a minor loss of precision, we focus on quantization errors arising from operator inputs. Using the PETR configuration from the first row of Table~\ref{tab:performance_drop_for_ptq_on_raw_petr}, we obtain layer-wise SQNRs, depicted in Fig.~\ref{fig:ASQNR}. From these results, we identify three main factors contributing to quantization errors:

\paragraph{Observation 1: Position Encoding Design Flaws Lead to Quantization Difficulties.}
Our in-depth analysis reveals that the quantization issues in PETR fundamentally stem from a design flaw in the positional encoding module, which manifests in two interrelated aspects. \textbf{(a) The inverse-sigmoid operator disrupts feature distribution balance.} As indicated by the red arrow in Fig.\ref{fig:ASQNR}, quantization difficulties stem from PETR's positional encoding module. Analyzing its construction (Fig.\ref{fig:pe_compare}(a)), we find that the inverse-sigmoid operation induces an imbalanced feature distribution. Specifically, Fig.~\ref{fig:distribution_before_and_after_insigmoid} shows that before applying inverse-sigmoid, the feature distribution is balanced and quantization-friendly, whereas afterward, it exhibits significant outliers. \textbf{(b) Magnitude Disparity between Camera-ray PE and Image Features.} As highlighted by the purple arrow in Fig.~\ref{fig:ASQNR}, applying 8-bit symmetric linear quantization to the 3D position-aware key $\mathbf{K}$ leads to significant performance degradation. To investigate this phenomenon, we conduct a statistical analysis of the magnitude distributions between image features and camera-ray positional encodings (PE). As illustrated in Fig.~\ref{fig:pe_img_compare}, both token-wise and channel-wise comparisons reveal that camera-ray PE exhibits an order-of-magnitude larger dynamic range (typically within $\pm$120) compared to image features (confined to $\pm$3). This severe imbalance creates a critical issue during quantization. When applying symmetric linear quantization with an 8-bit integer range (-128 to 127), the scaling factor \( s \) in Eq~\ref{e:eq2} becomes dominated by the extreme values of PE. Consequently, image features---occupying only ~2.5\% of the total dynamic range—are compressed into merely 7 discrete quantization bins (-3 to +3). As visualized in Fig.~\ref{fig:magnitude_distributions_of_image_feature_and_camera_ray_PE}, over 95\% of the original image feature variations collapse into the zero-centered bins, resulting in catastrophic information loss. This quantization artifact directly explains the observed performance drop in Table~\ref{tab:performance_drop_for_ptq_on_raw_petr}. To mitigate this issue, we propose two essential modifications: 1) eliminating the inverse-sigmoid operation that exacerbates outlier magnitudes, and 2) redesigning the positional encoding architecture to align its magnitude distribution with that of image features. These adaptations ensure balanced quantization resolution allocation, preserving critical information in both PE and image features. To mitigate this issue, we propose two essential modifications: 1) eliminating the inverse-sigmoid operation that exacerbates outlier magnitudes, and 2) redesigning the positional encoding architecture to align its magnitude distribution with that of image features. These adaptations ensure balanced distribution for quantization, preserving critical information in both PE and image features.
% To address these challenges, it is imperative to avoid the inverse-sigmoid operation and to redesign the positional encoding so that its magnitude distribution aligns better with that of the image features, thereby enhancing overall quantization-friendliness.


\begin{figure}[htb]
\centering
	\includegraphics[width=0.8\linewidth]{./figs/coor3d_before_and_after_inversesigmoid.pdf}
    %\vspace{-0.5cm}
	\caption{Feature distribution before and after the inverse-sigmoid operator. Red arrows highlight outliers.}
	\label{fig:distribution_before_and_after_insigmoid}
\end{figure}
\vspace{-0.5cm}
\begin{figure}[htb]
\centering
	\includegraphics[width=0.75\linewidth]{./figs/pe_img_compare.pdf}
    %\vspace{-0.3cm}
	\caption{Magnitude Distribution of Image Features and Positional Encodings: A Token-wise and Channel-wise Comparison}
	\label{fig:pe_img_compare}
\end{figure}


\begin{figure}[htb]
\centering
	\includegraphics[width=0.8\linewidth]{./figs/img_feat_and_pe_distribution.pdf}
    \vspace{-0.3cm}
	\caption{The distributions of image features and camera-ray position encodings after symmetric quantization using the quantization parameters derived from the 3D position-aware $\mathbf{K}$.}
	\label{fig:magnitude_distributions_of_image_feature_and_camera_ray_PE}
\end{figure}
\vspace{-0.5cm}


\paragraph{Observation 2: Dual-Dimensional Heterogeneity in Cross-Attention Leads to Quantization Bottlenecks.}

As evidenced by the green arrow in Fig.~\ref{fig:ASQNR} and further clarified in Fig.~\ref{fig:scaled_dot_product}, the scaled dot-product in cross-attention exhibits pronounced heterogeneity on two levels. First, the inter-head variance spans 2–3 orders of magnitude, while within each head, the value distribution is extremely broad (e.g., ranging beyond [$-10^3$, $10^3$]). We merge the head and query dimensions to directly reveal the row-wise feature distribution. The results show that regardless of whether quantization is performed per head, per token, or on the entire tensor, the excessively large softmax inputs result in significant quantization errors. This confirms that existing quantization paradigms are fundamentally inadequate for handling the severe amplitude disparities in the cross-attention mechanism.



\begin{figure}[htb]
\centering
	\includegraphics[width=0.8\linewidth]{./figs/softmax_input.pdf}
    %\vspace{-0.3cm}
	\caption{The distributions of scaled dot-product in cross-attention. There are significant amplitude fluctuations along the head dimension.}
	\label{fig:scaled_dot_product}
\end{figure}


\fi

\section{Experimental Setup}\label{sec:exp_setup}
\textbf{Benchmark.}
We use the nuScenes dataset, a comprehensive autonomous driving dataset covering object detection, tracking, and LiDAR segmentation. The vehicle is equipped with one LiDAR, five radars, and six cameras providing a 360-degree view. The dataset comprises 1,000 driving scenes split into training (700 scenes), validation (150 scenes), and testing (150 scenes) subsets. Each scene lasts 20 seconds, annotated at 2 Hz.

\textbf{Metrics.}
Following the official evaluation protocol, we report the nuScenes Score (NDS), mean Average Precision (mAP), and five true positive metrics: mean Average Translation Error (mATE), Scale Error (mASE), Orientation Error (mAOE), Velocity Error (mAVE), and Attribute Error (mAAE).


\textbf{Experimental Details.}
Our experiments encompass both floating-point training and quantization configurations. For floating-point training, we follow PETR series settings, using PETR with an R50dcn backbone unless specified, and utilize the C5 feature (1/32 resolution output) as the 2D feature. Input images are at $1408 \times 512$ resolution. Both the lidar-ray PE and QD-aware lidar-ray PE use a pixel-wise depth of 30m with three anchor embeddings per axis. The 3D perception space is defined as $[-61.2, 61.2]$m along the X and Y axes, and $[-10, 10]$m along the Z axis. We also compare these positional encodings on StreamPETR, using a V2-99 backbone and input images of $800 \times 320$ resolution.

Training uses the AdamW optimizer (weight decay 0.01) with an initial learning rate of $2.0 \times 10^{-4}$, decayed via a cosine annealing schedule. We train for 24 epochs with a batch size of 8 on four NVIDIA RTX 4090 GPUs. No test-time augmentation is applied.

For quantization, we adopt 8-bit symmetric per-tensor post-training quantization, using 32 randomly selected training images for calibration. When quantizing the scaled dot-product in cross-attention, we define a candidate set of 20 scaling factors.


\section{Theoretical Analysis of Magnitude Bounds in Position Encodings}
\label{sec:mag_analysis}

\subsection{Normalization Framework and Input Conditioning}
\label{subsec:normalization}
To establish a unified analytical framework, we first formalize the spatial normalization process for various ray-based position encodings. Let $\mathbf{p} = (x, y, z)$ denote the 3D coordinates within the perception range $x, y \in [-51.2, 51.2]$ meters and $z \in [-5, 3]$ meters. The normalized coordinates $\mathbf{v} \in [0, 1]^3$ are computed as:

\begin{equation}
    \mathbf{v} = \left( \frac{x + 51.2}{102.4}, \frac{y + 51.2}{102.4}, \frac{z + 5.0}{8.0} \right)
    \label{eq:normalization}
\end{equation}

Noting that $\mathbf{v}$ is clamped to $\mathbf{v}_c$ within the range $[0, 1]$, the distribution ranges of the normalized sampled points in positional encodings are characterized as follows:
\begin{itemize}
    \item For the sampled point of Camera-Ray PE, denoted as $\mathbf{v}_c^{CR}$, the distribution spans the unit cube, i.e., $[0, 1] \times [0, 1] \times [0, 1]$.
    \item For the sampled points of LiDAR-Ray PE and QDPE, denoted as $\mathbf{v}_c^{LR}$ and $\mathbf{v}_c^{QD}$ respectively, the distributions are constrained to $[0, 0.79] \times [0, 0.79] \times [0, 1]$.
\end{itemize}
Here, the value $0.79$ is derived from the ratio $30/51.2$, where $30$ corresponds to the fixed depth setting in the encoding process. This distinction highlights the inherent differences in spatial coverage and normalization strategies employed by these positional encodings.

\subsection{Magnitude Propagation Analysis}
\label{subsec:magnitude_propagation}

\subsubsection{Camera-Ray Position Encoding}
\label{subsubsec:camera_pe}
As illustrated in Fig.~\ref{fig:pe_compare} (a), the encoding pipeline consists of two critical stages:

\textbf{Stage 1: Inverse Sigmoid Transformation}
\begin{equation}
    \hat{\mathbf{v}}^{CR} = \ln\left(\frac{\mathbf{v}_c^{CR} + \epsilon}{1 - (\mathbf{v}_c^{CR} + \epsilon)}\right), \quad \epsilon = 10^{-5}
    \label{eq:logit_transform_cr}
\end{equation}
Empirical analysis reveals a maximum magnitude $\eta_{\text{max}} = \max(\|\hat{\mathbf{v}}^{CR}\|_\infty) \approx 11.5$.

\textbf{Stage 2: MLP Projection} (Through Two Fully-Connected Layers)
\begin{equation}
    \text{PE}_{\text{CR}} = \mathbf{W}_2 \sigma(\mathbf{W}_1 \hat{\mathbf{v}}^{CR} + \mathbf{b}_1) + \mathbf{b}_2
    \label{eq:mlp_transform_cr}
\end{equation}
where $\sigma$ denotes the ReLU activation function. Let $\Gamma = \max(\|\mathbf{W}_1\|_{\max}, \|\mathbf{W}_2\|_{\max})$ be the maximum weight magnitude. We derive the upper bound:
\begin{equation}
    \| \text{PE}_{\text{CR}} \|_\infty \leq 256 \cdot 192 \cdot \Gamma^2 \cdot 11.5
    \label{eq:cr_bound}
\end{equation}
where $192$ and $256$ denote the input tensor channels for $\mathbf{W}_1$ and $\mathbf{W}_2$, respectively.

\subsubsection{LiDAR-Ray Position Encoding}
\label{subsubsec:lidar_pe}
Unlike Camera-Ray PE, the encoding process of LiDAR-Ray PE introduces sinusoidal modulation between the inverse sigmoid transformation and MLP projection, as shown in Fig.~\ref{fig:pe_compare} (b). The magnitude propagation for LiDAR-Ray PE is as follows:

\textbf{Stage 1: Inverse Sigmoid Transformation}
\begin{equation}
    \hat{\mathbf{v}}^{LR} = \ln\left(\frac{\mathbf{v}_c^{LR} + \epsilon}{1 - (\mathbf{v}_c^{LR} + \epsilon)}\right), \quad \epsilon = 10^{-5}
    \label{eq:logit_transform_lr}
\end{equation}
Empirical analysis reveals a maximum magnitude $\eta_{\text{max}} = \max(\|\hat{\mathbf{v}}^{LR}\|_\infty) \approx 1.8$.

\textbf{Stage 2: Spectral Embedding}
\begin{equation}
    \phi(\hat{\mathbf{v}}^{LR}) = \bigoplus_{k=1}^{32} \left[\sin(\omega_k \hat{\mathbf{v}}^{LR}), \cos(\omega_k \hat{\mathbf{v}}^{LR})\right]
    \label{eq:sinusoidal}
\end{equation}
where $\bigoplus$ denotes concatenation. This ensures:
\begin{equation}
    \| \phi(\hat{\mathbf{v}}^{LR}) \|_\infty \leq 1.0
    \label{eq:sin_bound}
\end{equation}

\textbf{Stage 3: MLP Projection} (Following setting in Camera-Ray PE)
\begin{equation}
    \| \text{PE}_{\text{LR}} \|_\infty \leq 256 \cdot 192 \cdot \Gamma^2 \cdot 1.0
    \label{eq:lr_bound}
\end{equation}

\subsubsection{Ours QD-PE}
\label{subsubsec:qd_pe}
The proposed encoding introduces anchor-based constraints, as depicted in Fig.~\ref{fig:pe_compare} (c):

\textbf{Stage 1: Anchor Interpolation} (For Each Axis $\alpha \in \{x, y, z\}$)
\begin{equation}
    \mathbf{e}_\alpha = \frac{p_\alpha - L_\alpha^i}{\Delta L_\alpha} \mathbf{E}_\alpha^{i+1} + \frac{L_\alpha^{i+1} - p_\alpha}{\Delta L_\alpha} \mathbf{E}_\alpha^i
    \label{eq:anchor_interp}
\end{equation}
where $\mathbf{E}_\alpha^i$ denotes learnable anchor embeddings.
Via Theorem~\ref{thm:anchor}, the magnitude is constrain to:
\begin{equation}
    \| \mathbf{e}_\alpha \|_\infty \leq \gamma \quad 
    \label{eq:anchor_bound}
\end{equation}

\textbf{Stage 2: MLP Projection}
\begin{equation}
    \| \text{PE}_{\text{QD}} \|_\infty \leq 256 \cdot 192 \cdot \Gamma^2 \cdot 0.8
    \label{eq:qd_bound}
\end{equation}

\subsection{Comparative Magnitude Analysis}
\label{subsec:comparative}
The derived bounds reveal fundamental differences in magnitude scaling:
\begin{align}
    \frac{\| \text{PE}_{\text{CR}} \|}{\| \text{PE}_{\text{LR}} \|} &\approx \frac{11.5}{1.0} = 11.5 
    \label{eq:ratio_lidar} \\
    \frac{\| \text{PE}_{\text{CR}} \|}{\| \text{PE}_{\text{QD}} \|} &\approx \frac{11.5}{0.8} = 14.3
    \label{eq:ratio_qd}
\end{align}
This analysis demonstrates that QD-PE requires $14\times$ less quantization range than Camera-Ray PE.

\subsection{Theoretical Guarantee of Magnitude Constraints}
\label{subsec:theorem}
\begin{theorem}[Anchor Embedding Magnitude Bound]
\label{thm:anchor}
Let $\mathbf{E}_\alpha^i, \mathbf{E}_\alpha^{i+1}$ be adjacent anchor embeddings with $\|\mathbf{E}_\alpha^i\|_\infty \leq \gamma$. For any point $p_\alpha \in [L_\alpha^i, L_\alpha^{i+1}]$, its interpolated embedding satisfies:
\begin{equation}
    \| \mathbf{e}_\alpha \|_\infty \leq \gamma
\end{equation}
\end{theorem}

\begin{proof}
Let $\lambda = \frac{p_\alpha - L_\alpha^i}{\Delta L_\alpha} \in [0, 1]$. The interpolated embedding becomes:
\begin{equation}
    \mathbf{e}_\alpha = \lambda \mathbf{E}_\alpha^{i+1} + (1 - \lambda) \mathbf{E}_\alpha^i
\end{equation}
For any component $k$:
\begin{equation}
    |e_{\alpha,k}| \leq \lambda |E_{\alpha,k}^{i+1}| + (1 - \lambda) |E_{\alpha,k}^i| \leq \lambda \gamma + (1 - \lambda) \gamma = \gamma
\end{equation}
Thus, $\|\mathbf{e}_\alpha\|_\infty \leq \gamma$ holds for all dimensions.
\end{proof}
Through the application of regularization (e.g., L2 constraint) on the anchor embeddings $\mathbf{E}_\alpha^i$
during training, the magnitude of $\gamma$ can be explicitly controlled.
Empirically, we find that this value converges to approximately 0.8 in our experiments.


\section{More Ablation Study}\label{sec:more_ablation}

\begin{figure*}[htb]
\centering
	\includegraphics[width=0.8\linewidth]{./figs/pe visual.pdf}
    %\vspace{-0.8cm}
	\caption{Qualitative comparison of the local similarity.}
	\label{fig:pe_visual_compare}
\end{figure*}

\subsection{Local Similarity of Position Encoding Features}
%\textbf{Local Similarity of Position Encoding Features.} 
Fig.~\ref{fig:pe_visual_compare} shows that QD-PE significantly outperforms 3D point PE and cameraray PE in local similarity of position encoding. Its similarity distribution appears more compact and concentrated, validating the method's superiority in local spatial information modeling and its capability to precisely capture neighborhood spatial relationships around target pixels.
% is a qualitative comparison of QD-PE, 3D point PE, and cameraray PE from the back perspective of a 3D surround-view system, focusing on the local similarity of position encoding features. The red box in the top row marks a selected pixel, and the similarity maps illustrate how each method encodes spatial relationships around that pixel. A more concentrated similarity distribution indicates a stronger position encoding. As shown, our method achieves a more compact and focused similarity, highlighting its effectiveness in capturing local spatial information compared to other approaches.

\section{Limitations}
Although our method incurs almost no quantization accuracy loss, 
users need to replace the camera-ray in the original PETR series with our proposed QDPE. 
The only drawback is that this requires retraining. 
However, from the perspective of quantization deployment,
this retraining is beneficial, 
and the floating-point precision can even be improved.


\end{document}


% \end{document}

\clearpage
\twocolumn[\begin{minipage}{.99\textwidth}\centering%
    \LARGE\bf Wavelet-Assisted Multi-Frequency Attention Network for Pansharpening     Supplemental Material\vspace*{2ex}

    % \Large Ryan Rabinowitz, Steve Cruz, Manuel G\"unther, and Terrance E. Boult\vspace*{3ex}
\end{minipage}]
% % ICCV 2025 Paper Template; see https://github.com/cvpr-org/author-kit

\documentclass[10pt,twocolumn,letterpaper]{article}

%%%%%%%%% PAPER TYPE  - PLEASE UPDATE FOR FINAL VERSION
% \usepackage{iccv}              % To produce the CAMERA-READY version
\usepackage[review]{iccv}      % To produce the REVIEW version
% \usepackage[pagenumbers]{iccv} % To force page numbers, e.g. for an arXiv version

% Import additional packages in the preamble file, before hyperref
%
% --- inline annotations
%
\newcommand{\red}[1]{{\color{red}#1}}
\newcommand{\todo}[1]{{\color{red}#1}}
\newcommand{\TODO}[1]{\textbf{\color{red}[TODO: #1]}}
% --- disable by uncommenting  
% \renewcommand{\TODO}[1]{}
% \renewcommand{\todo}[1]{#1}



\newcommand{\VLM}{LVLM\xspace} 
\newcommand{\ours}{PeKit\xspace}
\newcommand{\yollava}{Yo’LLaVA\xspace}

\newcommand{\thisismy}{This-Is-My-Img\xspace}
\newcommand{\myparagraph}[1]{\noindent\textbf{#1}}
\newcommand{\vdoro}[1]{{\color[rgb]{0.4, 0.18, 0.78} {[V] #1}}}
% --- disable by uncommenting  
% \renewcommand{\TODO}[1]{}
% \renewcommand{\todo}[1]{#1}
\usepackage{slashbox}
% Vectors
\newcommand{\bB}{\mathcal{B}}
\newcommand{\bw}{\mathbf{w}}
\newcommand{\bs}{\mathbf{s}}
\newcommand{\bo}{\mathbf{o}}
\newcommand{\bn}{\mathbf{n}}
\newcommand{\bc}{\mathbf{c}}
\newcommand{\bp}{\mathbf{p}}
\newcommand{\bS}{\mathbf{S}}
\newcommand{\bk}{\mathbf{k}}
\newcommand{\bmu}{\boldsymbol{\mu}}
\newcommand{\bx}{\mathbf{x}}
\newcommand{\bg}{\mathbf{g}}
\newcommand{\be}{\mathbf{e}}
\newcommand{\bX}{\mathbf{X}}
\newcommand{\by}{\mathbf{y}}
\newcommand{\bv}{\mathbf{v}}
\newcommand{\bz}{\mathbf{z}}
\newcommand{\bq}{\mathbf{q}}
\newcommand{\bff}{\mathbf{f}}
\newcommand{\bu}{\mathbf{u}}
\newcommand{\bh}{\mathbf{h}}
\newcommand{\bb}{\mathbf{b}}

\newcommand{\rone}{\textcolor{green}{R1}}
\newcommand{\rtwo}{\textcolor{orange}{R2}}
\newcommand{\rthree}{\textcolor{red}{R3}}
\usepackage{amsmath}
%\usepackage{arydshln}
\DeclareMathOperator{\similarity}{sim}
\DeclareMathOperator{\AvgPool}{AvgPool}

\newcommand{\argmax}{\mathop{\mathrm{argmax}}}     



% It is strongly recommended to use hyperref, especially for the review version.
% hyperref with option pagebackref eases the reviewers' job.
% Please disable hyperref *only* if you encounter grave issues, 
% e.g. with the file validation for the camera-ready version.
%
% If you comment hyperref and then uncomment it, you should delete *.aux before re-running LaTeX.
% (Or just hit 'q' on the first LaTeX run, let it finish, and you should be clear).
\definecolor{iccvblue}{rgb}{0.21,0.49,0.74}
\usepackage[pagebackref,breaklinks,colorlinks,allcolors=iccvblue]{hyperref}
\usepackage{overpic}
\usepackage{makecell}
\usepackage{adjustbox} 
%\usepackage{float}
%\usepackage{arydshln}
\usepackage{multirow}
\usepackage{float}
\usepackage{bbding}
\usepackage{times}
\usepackage{epsfig}
\usepackage{graphicx}
\usepackage{amsmath}
\usepackage{amsthm}
\usepackage{amssymb}
\usepackage{booktabs}
\usepackage{xcolor}
\usepackage{enumitem}
\usepackage{lipsum}
\usepackage{diagbox}

\usepackage{algorithmic} % 引入 algorithmic 包
\usepackage{algpseudocode} % 引入 algpseudocode 包(如果需要更现代的控制结构)

\newcommand\blfootnote[1]{%
  \begingroup
  \renewcommand\thefootnote{}\footnote{#1}%
  \addtocounter{footnote}{-1}%
  \endgroup
}

\usepackage[ruled,vlined,linesnumbered]{algorithm2e}
\makeatletter
\newcommand{\algorithmfootnote}[2][\footnotesize]{%
  \let\old@algocf@finish\@algocf@finish% Store algorithm finish macro
  \def\@algocf@finish{\old@algocf@finish% Update finish macro to insert "footnote"
    \leavevmode\rlap{\begin{minipage}{\linewidth}
    #1#2
    \end{minipage}}%
  }%
}
\makeatother
%%%%%%%%% PAPER ID  - PLEASE UPDATE
\def\paperID{5291} % *** Enter the Paper ID here
\def\confName{ICCV}
\def\confYear{2025}
\newtheorem{theorem}{Theorem}[section]
\newtheorem{lemma}[theorem]{Lemma}
%%%%%%%%% TITLE - PLEASE UPDATE
\title{Q-PETR: Quant-aware Position Embedding Transformation for Multi-View 3D Object Detection\\------------ Supplementary Material ------------}

%%%%%%%%% AUTHORS - PLEASE UPDATE
\author{First Author\\
Institution1\\
Institution1 address\\
{\tt\small firstauthor@i1.org}
% For a paper whose authors are all at the same institution,
% omit the following lines up until the closing ``}''.
% Additional authors and addresses can be added with ``\and'',
% just like the second author.
% To save space, use either the email address or home page, not both
\and
Second Author\\
Institution2\\
First line of institution2 address\\
{\tt\small secondauthor@i2.org}
}

\begin{document}
\maketitle

\section{Preliminaries}
\label{sec:petr_preliminaries}

\paragraph{PETR} enhances 2D image features with 3D position-aware properties using camera-ray positional encoding (PE), enabling refined query updates for 3D bounding box prediction. Specifically, surround-view images $\mathbf{I}$ pass through a backbone to generate 2D features $\mathbf{f}_{2D}$, while camera-ray PE $\mathbf{p}_c$ is computed using camera intrinsics and extrinsics. The learnable query embeddings $q$ serve as the initial queries $\mathbf{Q}$ for the decoder. Here, $\mathbf{f}_{2D}$ serves as the values $\mathbf{V}$, and adding $\mathbf{p}_c$ to $\mathbf{f}_{2D}$ element-wise forms the 3D position-aware keys $\mathbf{K}$.

The decoder updates the queries using these key-value pairs through self-attention, cross-attention, and feed-forward network (FFN) modules. The updated query vectors are passed through an MLP to predict 3D bounding box categories and attributes, repeating for $L$ cycles. The entire PETR process is summarized in Algorithm~\ref{algo:algorithm_petr}.
\begin{algorithm}[htb]
\SetAlgoLined
\DontPrintSemicolon
\SetNoFillComment
\SetInd{1em}{1em} % <-- 关键:取消缩进
\footnotesize

\KwData{Surround-view images $\mathbf{I}$, camera intrinsics and extrinsics}
\KwResult{3D bounding boxes $\mathbf{b}^l$, categories $\mathbf{c}^l$ for $l = 1$ to $L$}

Compute image features: $\mathbf{f}_{2D} = \text{Backbone}(\mathbf{I})$\\
Compute camera-ray PE $\mathbf{p}_c$ using camera intrinsics and extrinsics\\
Form 3D position-aware keys: $\mathbf{K} = \mathbf{f}_{2D} + \mathbf{p}_c$ \tcp{Element-wise addition}
Set values: $\mathbf{V} = \mathbf{f}_{2D}$ \\
Initialize queries: $\mathbf{Q} = q$ (For simplicity, omit \(\mathbf{Q}\)'s encoding.)\\
\For{$l = 1$ to $L$}{
  $\mathbf{Q} \gets \texttt{QProj}(\mathbf{Q})$; 
  $\mathbf{K} \gets \texttt{KProj}(\mathbf{K})$; 
  $\mathbf{V} \gets \texttt{VProj}(\mathbf{V})$ \\ 
  $\mathbf{A}_s = \texttt{MultiHeadAtt}(\mathbf{Q}, \mathbf{Q}, \mathbf{Q})$ \tcp{Self-Attn}
  $\mathbf{A}_c = \texttt{MultiHeadAtt}(\mathbf{A}_s, \mathbf{K}, \mathbf{V})$ \tcp{Cross-Attn}
  $\mathbf{Q} \gets \texttt{FFN}(\mathbf{Q} + \mathbf{A}_c)$ \\
  $\mathbf{b}^{l} \gets \texttt{MLP}(\mathbf{Q})$; 
  $\mathbf{c}^{l} \gets \texttt{MLP}(\mathbf{Q})$ \\
}
\Return{$(\mathbf{b}^l, \mathbf{c}^l)$ for $l = 1$ to $L$}

\caption{Pseudo-code of PETR.}
\label{algo:algorithm_petr}
\end{algorithm}



\iffalse
\section{Quantization Failure of PETR}
\label{sec:quantization_failure_of_petr}

We evaluate the performance of several PETR configurations~\cite{liu2022petr} using the official code. Under standard 8-bit symmetric per-tensor post-training quantization (PTQ), PETR suffers significant performance degradation, with an average drop of 58.2\% in mAP and 36.9\% in NDS on the nuScenes validation dataset (see Table~\ref{tab:performance_drop_for_ptq_on_raw_petr}). 

\begin{table}[htb] %{0.45\linewidth}
    %\tiny
    %\scriptsize
    \footnotesize
    \setlength{\tabcolsep}{1.4mm}
    %\small
    %\setlength{\tabcolsep}{2.5mm}
    %\normalsize
    %\large
    \centering
    \begin{tabular}{l|c|c|c|c|c|c}
    \toprule[1.5pt]
    \multirow{2}{*}{Bac} & \multirow{2}{*}{Size} & \multirow{2}{*}{Feat} & \multicolumn{2}{c|}{FP32 Acc}                               & \multicolumn{2}{c}{INT8 Acc}   \\ \cline{4-7}
     & & & \multicolumn{1}{c|}{mAP} & \multicolumn{1}{c|}{NDS} & \multicolumn{1}{c|}{mAP} & \multicolumn{1}{c}{NDS} \\ 
    \midrule
    \textcolor{white}{0}R50\textcolor{white}{0} & 1408$\times$512 & c5 & 30.5 & 35.0 & 18.4(12.1$\downarrow$) & 27.3(\textcolor{white}{0}7.7$\downarrow$) \\
    \textcolor{white}{0}R50\textcolor{white}{0} & 1408$\times$512 & p4 & 31.7 & 36.7 & 15.7(16.0$\downarrow$) & 26.1(10.6$\downarrow$) \\
    V2-99 & \textcolor{white}{0}800$\times$320 & p4 & 37.8 & 42.6 & 10.9(26.9$\downarrow$) & 23.6(19.0$\downarrow$) \\
    V2-99 & 1600$\times$640 & p4 & 40.4 & 45.5 & 11.3(29.1$\downarrow$) & 23.9(21.6$\downarrow$) \\
    \bottomrule[1.5pt]
    \end{tabular}
    \vspace{-0.3cm}
    \caption{
    PETR's performance of 3D object detection on nuSences val set, directly utilizing the pre-trained parameters from the official repository.
    }
    %\vspace{-0.5cm}
    \label{tab:performance_drop_for_ptq_on_raw_petr}
\end{table}

\iffalse
Enlightened by the existing state-of-the-art post-training quantization methods~\cite{dong2019hawq1,dong2020hawq2,hubara2021adaq,li2021brecq,liu2021ptqvit,nagel2020up,yao2021hawq3}, the adaptive rounding proxy objective is introduced to measure the quantization performance degradation:
\begin{equation}
\begin{aligned}
    \label{eqnAdaptiveEll}
    \textstyle
    \ell(\widetilde W)
    &=\mathbf{E}_x\left[ \Vert{ (W- \widetilde W) x \Vert}^2 \right] \\
    &=tr\left( (W - \widetilde W) H (W - \widetilde W)^T \right)
\end{aligned}
\end{equation}
Where $W$ is the float weight tensor of a learnable operator, $\widetilde W$ are the quantized weight, 
$x$ is an input tensor sampled randomly from a calibration set, 
and $H$ is the second moment matrix of these vectors, interpreted as a proxy Hessian.
\fi

\paragraph{Layer-wise Quantization Error Analysis.} Quantizing a pre-trained network introduces output noise, degrading performance. To identify the root causes of quantization failure, we employ the signal-to-quantization-noise ratio (SQNR), inspired by recent PTQ advancements~\cite{pandey2023practical, yang2023efficient, pagliari2023plinio}:

\begin{equation}\label{eq:sqnr}
SQNR_{q,b} = 10\log_{10} \left( \frac{ \sum_{i=1}^N \mathbb{E}[{\mathcal{F}}{\theta}(x_{i})^{2} ] }{ \sum_{i=1}^N \mathbb{E}[ e(x_{i})^{2} ] } \right)
\end{equation}

Here, $N$ is the number of calibration data points; $\mathcal{F}{\theta}$ denotes the full-precision network; the quantization error is $e(x_i) = \mathcal{F}{\theta}(x_i) - \mathcal{Q}{q,b}(\mathcal{F}{\theta}(x_i))$; and $\mathcal{Q}_{q,b}(\mathcal{F}_\theta)$ denotes the network output when only the target layer is quantized to $b$ bits, with all other layers kept at full precision.

Since 8-bit weight quantization results in only a minor loss of precision, we focus on quantization errors arising from operator inputs. Using the PETR configuration from the first row of Table~\ref{tab:performance_drop_for_ptq_on_raw_petr}, we obtain layer-wise SQNRs, depicted in Fig.~\ref{fig:ASQNR}. From these results, we identify three main factors contributing to quantization errors:

\paragraph{Observation 1: Position Encoding Design Flaws Lead to Quantization Difficulties.}
Our in-depth analysis reveals that the quantization issues in PETR fundamentally stem from a design flaw in the positional encoding module, which manifests in two interrelated aspects. \textbf{(a) The inverse-sigmoid operator disrupts feature distribution balance.} As indicated by the red arrow in Fig.\ref{fig:ASQNR}, quantization difficulties stem from PETR's positional encoding module. Analyzing its construction (Fig.\ref{fig:pe_compare}(a)), we find that the inverse-sigmoid operation induces an imbalanced feature distribution. Specifically, Fig.~\ref{fig:distribution_before_and_after_insigmoid} shows that before applying inverse-sigmoid, the feature distribution is balanced and quantization-friendly, whereas afterward, it exhibits significant outliers. \textbf{(b) Magnitude Disparity between Camera-ray PE and Image Features.} As highlighted by the purple arrow in Fig.~\ref{fig:ASQNR}, applying 8-bit symmetric linear quantization to the 3D position-aware key $\mathbf{K}$ leads to significant performance degradation. To investigate this phenomenon, we conduct a statistical analysis of the magnitude distributions between image features and camera-ray positional encodings (PE). As illustrated in Fig.~\ref{fig:pe_img_compare}, both token-wise and channel-wise comparisons reveal that camera-ray PE exhibits an order-of-magnitude larger dynamic range (typically within $\pm$120) compared to image features (confined to $\pm$3). This severe imbalance creates a critical issue during quantization. When applying symmetric linear quantization with an 8-bit integer range (-128 to 127), the scaling factor \( s \) in Eq~\ref{e:eq2} becomes dominated by the extreme values of PE. Consequently, image features---occupying only ~2.5\% of the total dynamic range—are compressed into merely 7 discrete quantization bins (-3 to +3). As visualized in Fig.~\ref{fig:magnitude_distributions_of_image_feature_and_camera_ray_PE}, over 95\% of the original image feature variations collapse into the zero-centered bins, resulting in catastrophic information loss. This quantization artifact directly explains the observed performance drop in Table~\ref{tab:performance_drop_for_ptq_on_raw_petr}. To mitigate this issue, we propose two essential modifications: 1) eliminating the inverse-sigmoid operation that exacerbates outlier magnitudes, and 2) redesigning the positional encoding architecture to align its magnitude distribution with that of image features. These adaptations ensure balanced quantization resolution allocation, preserving critical information in both PE and image features. To mitigate this issue, we propose two essential modifications: 1) eliminating the inverse-sigmoid operation that exacerbates outlier magnitudes, and 2) redesigning the positional encoding architecture to align its magnitude distribution with that of image features. These adaptations ensure balanced distribution for quantization, preserving critical information in both PE and image features.
% To address these challenges, it is imperative to avoid the inverse-sigmoid operation and to redesign the positional encoding so that its magnitude distribution aligns better with that of the image features, thereby enhancing overall quantization-friendliness.


\begin{figure}[htb]
\centering
	\includegraphics[width=0.8\linewidth]{./figs/coor3d_before_and_after_inversesigmoid.pdf}
    %\vspace{-0.5cm}
	\caption{Feature distribution before and after the inverse-sigmoid operator. Red arrows highlight outliers.}
	\label{fig:distribution_before_and_after_insigmoid}
\end{figure}
\vspace{-0.5cm}
\begin{figure}[htb]
\centering
	\includegraphics[width=0.75\linewidth]{./figs/pe_img_compare.pdf}
    %\vspace{-0.3cm}
	\caption{Magnitude Distribution of Image Features and Positional Encodings: A Token-wise and Channel-wise Comparison}
	\label{fig:pe_img_compare}
\end{figure}


\begin{figure}[htb]
\centering
	\includegraphics[width=0.8\linewidth]{./figs/img_feat_and_pe_distribution.pdf}
    \vspace{-0.3cm}
	\caption{The distributions of image features and camera-ray position encodings after symmetric quantization using the quantization parameters derived from the 3D position-aware $\mathbf{K}$.}
	\label{fig:magnitude_distributions_of_image_feature_and_camera_ray_PE}
\end{figure}
\vspace{-0.5cm}


\paragraph{Observation 2: Dual-Dimensional Heterogeneity in Cross-Attention Leads to Quantization Bottlenecks.}

As evidenced by the green arrow in Fig.~\ref{fig:ASQNR} and further clarified in Fig.~\ref{fig:scaled_dot_product}, the scaled dot-product in cross-attention exhibits pronounced heterogeneity on two levels. First, the inter-head variance spans 2–3 orders of magnitude, while within each head, the value distribution is extremely broad (e.g., ranging beyond [$-10^3$, $10^3$]). We merge the head and query dimensions to directly reveal the row-wise feature distribution. The results show that regardless of whether quantization is performed per head, per token, or on the entire tensor, the excessively large softmax inputs result in significant quantization errors. This confirms that existing quantization paradigms are fundamentally inadequate for handling the severe amplitude disparities in the cross-attention mechanism.



\begin{figure}[htb]
\centering
	\includegraphics[width=0.8\linewidth]{./figs/softmax_input.pdf}
    %\vspace{-0.3cm}
	\caption{The distributions of scaled dot-product in cross-attention. There are significant amplitude fluctuations along the head dimension.}
	\label{fig:scaled_dot_product}
\end{figure}


\fi

\section{Experimental Setup}\label{sec:exp_setup}
\textbf{Benchmark.}
We use the nuScenes dataset, a comprehensive autonomous driving dataset covering object detection, tracking, and LiDAR segmentation. The vehicle is equipped with one LiDAR, five radars, and six cameras providing a 360-degree view. The dataset comprises 1,000 driving scenes split into training (700 scenes), validation (150 scenes), and testing (150 scenes) subsets. Each scene lasts 20 seconds, annotated at 2 Hz.

\textbf{Metrics.}
Following the official evaluation protocol, we report the nuScenes Score (NDS), mean Average Precision (mAP), and five true positive metrics: mean Average Translation Error (mATE), Scale Error (mASE), Orientation Error (mAOE), Velocity Error (mAVE), and Attribute Error (mAAE).


\textbf{Experimental Details.}
Our experiments encompass both floating-point training and quantization configurations. For floating-point training, we follow PETR series settings, using PETR with an R50dcn backbone unless specified, and utilize the C5 feature (1/32 resolution output) as the 2D feature. Input images are at $1408 \times 512$ resolution. Both the lidar-ray PE and QD-aware lidar-ray PE use a pixel-wise depth of 30m with three anchor embeddings per axis. The 3D perception space is defined as $[-61.2, 61.2]$m along the X and Y axes, and $[-10, 10]$m along the Z axis. We also compare these positional encodings on StreamPETR, using a V2-99 backbone and input images of $800 \times 320$ resolution.

Training uses the AdamW optimizer (weight decay 0.01) with an initial learning rate of $2.0 \times 10^{-4}$, decayed via a cosine annealing schedule. We train for 24 epochs with a batch size of 8 on four NVIDIA RTX 4090 GPUs. No test-time augmentation is applied.

For quantization, we adopt 8-bit symmetric per-tensor post-training quantization, using 32 randomly selected training images for calibration. When quantizing the scaled dot-product in cross-attention, we define a candidate set of 20 scaling factors.


\section{Theoretical Analysis of Magnitude Bounds in Position Encodings}
\label{sec:mag_analysis}

\subsection{Normalization Framework and Input Conditioning}
\label{subsec:normalization}
To establish a unified analytical framework, we first formalize the spatial normalization process for various ray-based position encodings. Let $\mathbf{p} = (x, y, z)$ denote the 3D coordinates within the perception range $x, y \in [-51.2, 51.2]$ meters and $z \in [-5, 3]$ meters. The normalized coordinates $\mathbf{v} \in [0, 1]^3$ are computed as:

\begin{equation}
    \mathbf{v} = \left( \frac{x + 51.2}{102.4}, \frac{y + 51.2}{102.4}, \frac{z + 5.0}{8.0} \right)
    \label{eq:normalization}
\end{equation}

Noting that $\mathbf{v}$ is clamped to $\mathbf{v}_c$ within the range $[0, 1]$, the distribution ranges of the normalized sampled points in positional encodings are characterized as follows:
\begin{itemize}
    \item For the sampled point of Camera-Ray PE, denoted as $\mathbf{v}_c^{CR}$, the distribution spans the unit cube, i.e., $[0, 1] \times [0, 1] \times [0, 1]$.
    \item For the sampled points of LiDAR-Ray PE and QDPE, denoted as $\mathbf{v}_c^{LR}$ and $\mathbf{v}_c^{QD}$ respectively, the distributions are constrained to $[0, 0.79] \times [0, 0.79] \times [0, 1]$.
\end{itemize}
Here, the value $0.79$ is derived from the ratio $30/51.2$, where $30$ corresponds to the fixed depth setting in the encoding process. This distinction highlights the inherent differences in spatial coverage and normalization strategies employed by these positional encodings.

\subsection{Magnitude Propagation Analysis}
\label{subsec:magnitude_propagation}

\subsubsection{Camera-Ray Position Encoding}
\label{subsubsec:camera_pe}
As illustrated in Fig.~\ref{fig:pe_compare} (a), the encoding pipeline consists of two critical stages:

\textbf{Stage 1: Inverse Sigmoid Transformation}
\begin{equation}
    \hat{\mathbf{v}}^{CR} = \ln\left(\frac{\mathbf{v}_c^{CR} + \epsilon}{1 - (\mathbf{v}_c^{CR} + \epsilon)}\right), \quad \epsilon = 10^{-5}
    \label{eq:logit_transform_cr}
\end{equation}
Empirical analysis reveals a maximum magnitude $\eta_{\text{max}} = \max(\|\hat{\mathbf{v}}^{CR}\|_\infty) \approx 11.5$.

\textbf{Stage 2: MLP Projection} (Through Two Fully-Connected Layers)
\begin{equation}
    \text{PE}_{\text{CR}} = \mathbf{W}_2 \sigma(\mathbf{W}_1 \hat{\mathbf{v}}^{CR} + \mathbf{b}_1) + \mathbf{b}_2
    \label{eq:mlp_transform_cr}
\end{equation}
where $\sigma$ denotes the ReLU activation function. Let $\Gamma = \max(\|\mathbf{W}_1\|_{\max}, \|\mathbf{W}_2\|_{\max})$ be the maximum weight magnitude. We derive the upper bound:
\begin{equation}
    \| \text{PE}_{\text{CR}} \|_\infty \leq 256 \cdot 192 \cdot \Gamma^2 \cdot 11.5
    \label{eq:cr_bound}
\end{equation}
where $192$ and $256$ denote the input tensor channels for $\mathbf{W}_1$ and $\mathbf{W}_2$, respectively.

\subsubsection{LiDAR-Ray Position Encoding}
\label{subsubsec:lidar_pe}
Unlike Camera-Ray PE, the encoding process of LiDAR-Ray PE introduces sinusoidal modulation between the inverse sigmoid transformation and MLP projection, as shown in Fig.~\ref{fig:pe_compare} (b). The magnitude propagation for LiDAR-Ray PE is as follows:

\textbf{Stage 1: Inverse Sigmoid Transformation}
\begin{equation}
    \hat{\mathbf{v}}^{LR} = \ln\left(\frac{\mathbf{v}_c^{LR} + \epsilon}{1 - (\mathbf{v}_c^{LR} + \epsilon)}\right), \quad \epsilon = 10^{-5}
    \label{eq:logit_transform_lr}
\end{equation}
Empirical analysis reveals a maximum magnitude $\eta_{\text{max}} = \max(\|\hat{\mathbf{v}}^{LR}\|_\infty) \approx 1.8$.

\textbf{Stage 2: Spectral Embedding}
\begin{equation}
    \phi(\hat{\mathbf{v}}^{LR}) = \bigoplus_{k=1}^{32} \left[\sin(\omega_k \hat{\mathbf{v}}^{LR}), \cos(\omega_k \hat{\mathbf{v}}^{LR})\right]
    \label{eq:sinusoidal}
\end{equation}
where $\bigoplus$ denotes concatenation. This ensures:
\begin{equation}
    \| \phi(\hat{\mathbf{v}}^{LR}) \|_\infty \leq 1.0
    \label{eq:sin_bound}
\end{equation}

\textbf{Stage 3: MLP Projection} (Following setting in Camera-Ray PE)
\begin{equation}
    \| \text{PE}_{\text{LR}} \|_\infty \leq 256 \cdot 192 \cdot \Gamma^2 \cdot 1.0
    \label{eq:lr_bound}
\end{equation}

\subsubsection{Ours QD-PE}
\label{subsubsec:qd_pe}
The proposed encoding introduces anchor-based constraints, as depicted in Fig.~\ref{fig:pe_compare} (c):

\textbf{Stage 1: Anchor Interpolation} (For Each Axis $\alpha \in \{x, y, z\}$)
\begin{equation}
    \mathbf{e}_\alpha = \frac{p_\alpha - L_\alpha^i}{\Delta L_\alpha} \mathbf{E}_\alpha^{i+1} + \frac{L_\alpha^{i+1} - p_\alpha}{\Delta L_\alpha} \mathbf{E}_\alpha^i
    \label{eq:anchor_interp}
\end{equation}
where $\mathbf{E}_\alpha^i$ denotes learnable anchor embeddings.
Via Theorem~\ref{thm:anchor}, the magnitude is constrain to:
\begin{equation}
    \| \mathbf{e}_\alpha \|_\infty \leq \gamma \quad 
    \label{eq:anchor_bound}
\end{equation}

\textbf{Stage 2: MLP Projection}
\begin{equation}
    \| \text{PE}_{\text{QD}} \|_\infty \leq 256 \cdot 192 \cdot \Gamma^2 \cdot 0.8
    \label{eq:qd_bound}
\end{equation}

\subsection{Comparative Magnitude Analysis}
\label{subsec:comparative}
The derived bounds reveal fundamental differences in magnitude scaling:
\begin{align}
    \frac{\| \text{PE}_{\text{CR}} \|}{\| \text{PE}_{\text{LR}} \|} &\approx \frac{11.5}{1.0} = 11.5 
    \label{eq:ratio_lidar} \\
    \frac{\| \text{PE}_{\text{CR}} \|}{\| \text{PE}_{\text{QD}} \|} &\approx \frac{11.5}{0.8} = 14.3
    \label{eq:ratio_qd}
\end{align}
This analysis demonstrates that QD-PE requires $14\times$ less quantization range than Camera-Ray PE.

\subsection{Theoretical Guarantee of Magnitude Constraints}
\label{subsec:theorem}
\begin{theorem}[Anchor Embedding Magnitude Bound]
\label{thm:anchor}
Let $\mathbf{E}_\alpha^i, \mathbf{E}_\alpha^{i+1}$ be adjacent anchor embeddings with $\|\mathbf{E}_\alpha^i\|_\infty \leq \gamma$. For any point $p_\alpha \in [L_\alpha^i, L_\alpha^{i+1}]$, its interpolated embedding satisfies:
\begin{equation}
    \| \mathbf{e}_\alpha \|_\infty \leq \gamma
\end{equation}
\end{theorem}

\begin{proof}
Let $\lambda = \frac{p_\alpha - L_\alpha^i}{\Delta L_\alpha} \in [0, 1]$. The interpolated embedding becomes:
\begin{equation}
    \mathbf{e}_\alpha = \lambda \mathbf{E}_\alpha^{i+1} + (1 - \lambda) \mathbf{E}_\alpha^i
\end{equation}
For any component $k$:
\begin{equation}
    |e_{\alpha,k}| \leq \lambda |E_{\alpha,k}^{i+1}| + (1 - \lambda) |E_{\alpha,k}^i| \leq \lambda \gamma + (1 - \lambda) \gamma = \gamma
\end{equation}
Thus, $\|\mathbf{e}_\alpha\|_\infty \leq \gamma$ holds for all dimensions.
\end{proof}
Through the application of regularization (e.g., L2 constraint) on the anchor embeddings $\mathbf{E}_\alpha^i$
during training, the magnitude of $\gamma$ can be explicitly controlled.
Empirically, we find that this value converges to approximately 0.8 in our experiments.


\section{More Ablation Study}\label{sec:more_ablation}

\begin{figure*}[htb]
\centering
	\includegraphics[width=0.8\linewidth]{./figs/pe visual.pdf}
    %\vspace{-0.8cm}
	\caption{Qualitative comparison of the local similarity.}
	\label{fig:pe_visual_compare}
\end{figure*}

\subsection{Local Similarity of Position Encoding Features}
%\textbf{Local Similarity of Position Encoding Features.} 
Fig.~\ref{fig:pe_visual_compare} shows that QD-PE significantly outperforms 3D point PE and cameraray PE in local similarity of position encoding. Its similarity distribution appears more compact and concentrated, validating the method's superiority in local spatial information modeling and its capability to precisely capture neighborhood spatial relationships around target pixels.
% is a qualitative comparison of QD-PE, 3D point PE, and cameraray PE from the back perspective of a 3D surround-view system, focusing on the local similarity of position encoding features. The red box in the top row marks a selected pixel, and the similarity maps illustrate how each method encodes spatial relationships around that pixel. A more concentrated similarity distribution indicates a stronger position encoding. As shown, our method achieves a more compact and focused similarity, highlighting its effectiveness in capturing local spatial information compared to other approaches.

\section{Limitations}
Although our method incurs almost no quantization accuracy loss, 
users need to replace the camera-ray in the original PETR series with our proposed QDPE. 
The only drawback is that this requires retraining. 
However, from the perspective of quantization deployment,
this retraining is beneficial, 
and the floating-point precision can even be improved.


\end{document}


% \clearpage
% \setcounter{page}{1}
% \maketitlesupplementary
\maketitle
\begin{abstract}
 
The supplementary materials provide additional insights into the method proposed in our paper.
First, we offer a more detailed explanation of the discrete wavelet transform (DWT) and multi-scale strategy employed in this work.
Furthermore, we present additional experiments and discussions, including a comparison of parameter numbers and further validation of the Frequency Attention Triplet.
Lastly, we provide a more comprehensive overview of the ablation experiment settings and offer additional quantitative and qualitative comparisons of different methods.

\end{abstract}


\section{Method Supplementary}

\subsection{Detailed Explanation of Wavelet Transform}
There are various forms of wavelet transform\cite{wave1,wave2,wave3}, and in our method, we utilize the discrete wavelet transform (DWT). This process can be illustrated with a simple example.
Consider an image, \( I \), represented by the following matrix:
\begin{equation}
I =
\begin{bmatrix}
a_{11} & a_{12} \\
a_{21} & a_{22} 
\end{bmatrix}
\end{equation}
Upon applying the DWT, the resulting low-frequency component can be expressed as:
\begin{equation}
LL = \frac{a_{11} + a_{12} + a_{21} + a_{22}}{4}
\end{equation}
The high-frequency components in the horizontal, vertical, and diagonal directions are represented by the LH, HL, and HH components, respectively, and are computed as follows:
\begin{equation}
LH = \frac{a_{11} + a_{12} - a_{21} - a_{22}}{4}
\end{equation}

\begin{equation}
HL = \frac{a_{11} - a_{12} + a_{21} - a_{22}}{4}
\end{equation}

\begin{equation}
HH = \frac{a_{21} + a_{22} - a_{11} - a_{12}}{4}
\end{equation}
\begin{figure}[!h]
\centering
\includegraphics[width=\columnwidth]{scale.png} 
\caption{The explanation of the multi-scale processing adopted by our method. The dimensions above the arrows represent the data dimensions entering the module to which the arrow points. Each scale differs by a factor of 2, so \(r\) is a multiple of 2. In this paper, \(C = 32\).}
\label{scale}
\end{figure}
For larger matrices, the image is divided into multiple 2x2 regions, and each region is processed individually using the above method. 
The inverse discrete wavelet transform (IDWT) reconstructs the original image by combining the LL, LH, HL, and HH components through a system of equations, ensuring the accurate and lossless recovery of the image.
To create a wavelet pyramid, the DWT operation is applied recursively to the LL component from the previous scale, constructing the pyramid through multiple iterations.
In practice, these operations can be efficiently implemented using fixed-value convolution and deconvolution kernels.


\subsection{Multi-Scale Strategy Details}
Previous works \cite{8281501,jin2022laplacian} have explored the advantages of leveraging this multi-scale property.
The multi-scale characteristic of the wavelet pyramid allows for more effective utilization of features across different scales. The multi-scale fusion process we adopt is illustrated in Fig.~\ref{scale}. Using multiple DWT operations, we obtain PAN image features with gradually decreasing scales until they match the scale of the LRMS. Fusion begins at the smallest scale and progressively transitions to the largest scale. At each scale, the input features have the same dimensions, and the output dimensions are twice those of the input, thereby achieving a progressive fusion process.


\begin{table*}[h]
\centering
\begin{tabular}{ c|c c c c c c c c c }
\hline
\textbf{Methods} & \textbf{PNN} & \textbf{PanNet} & \textbf{DiCNN} & \textbf{FusionNet} & \textbf{U2Net} & \textbf{PanMamba} & \textbf{CANNet} & \textbf{WFANet} & \textbf{WFANet-L}  \\ \hline
\textbf{Params(M)} & 0.10 & 0.08 & 0.04 & 0.08 & 0.66 & 0.48 & 0.78 & 0.52 & 0.07 \\ \hline
\end{tabular}
\caption{Comparison of parameters for different methods.}
\label{params}
\end{table*}



\section{Experimental Supplementary}

\subsection{ Comparison of Parameter Numbers
}
In this section, we compare the parameter numbers of various DL-based pansharpening methods, as illustrated in Table \ref{params}.
We divide DL-based pansharpening methods into two categories based on their number of parameters. Models with no more than 0.10M parameters are designated as lightweight networks,
 whereas those exceeding 0.10M parameters are classified as heavyweight networks. WFANet belongs to the heavyweight category, and we also designed a lightweight version, WFANet-L.
To reduce network parameters while maintaining performance, we decreased the common channel size from 32 to 24 and simplified several MLP layers. 
To ensure a fair comparison, Fig.~\ref{comparison} shows the results of the lightweight networks in the left half and the heavyweight networks in the right half, with PSNR representing the model performance\cite{8281501}.
 Both WFANet-L and WFANet achieve strong performance while maintaining a relatively low number of parameters. These results demonstrates that our method effectively balances model performance with manageable complexity.


\subsection{Further Validation of the Frequency Attention Triplet}
To further validate the effectiveness of the Frequency Attention Triplet design, we conducted experiments where we systematically swapped the roles of Frequency-Query, Spatial-Key, and Fusion-Value as the Query, Key, and Value in the attention mechanism. This resulted in six different configurations. The original configuration is labeled as Ours, while the alternative configurations, named V1 through V5, \textit{each represents a specific permutation of Frequency-Query, Spatial-Key, and Fusion-Value serving as Query, Key, and Value, respectively}. As shown in Table \ref{qkv}, the experimental results clearly demonstrate that our method achieves the best performance across all metrics, which can be attributed to the thoughtful design based on their physical significance.

\begin{figure}[t]
\centering
\includegraphics[width=0.99\columnwidth]{1.png} 
\caption{Trade-off between parameter numbers and PSNR on the WV3 reduced-resolution dataset. The left side shows lightweight networks, while the right side shows heavyweight networks.
} 
\label{comparison}
\end{figure}


\begin{table}[t]
\centering
\begin{tabular}{ c c c c c }
\hline
\textbf{Method} & \textbf{PSNR$\uparrow$} & \textbf{SAM$\downarrow$} & \textbf{ERGAS$\downarrow$} & \textbf{Q8$\uparrow$} \\ \hline
V1 & 39.092 & 2.925 & 2.161 & 0.920 \\ 
V2 & 38.954 & 2.946 & 2.190 & 0.917 \\ 
V3 & 38.916 & 2.954 & 2.198 & 0.918 \\ 
V4 & 39.056 & 2.914 & 2.163 & 0.920 \\ 
V5 & 38.573 & 3.104 & 2.316 & 0.915 \\ 
\textbf{Ours} & \textbf{39.345} & \textbf{2.849} & \textbf{2.093} & \textbf{0.922} \\ \hline
\end{tabular}
\caption{Comparison of different methods for further validation of the Frequency Attention Triplet.}
\label{qkv}
\end{table}

\begin{figure}[t]
\centering
\includegraphics[width=0.99\columnwidth]{ablation.png} 
\caption{A detailed explanation of the ablation experiments: (a) Frequency-Query ablation, (b) Spatial-Key ablation, (c) Fusion-Value ablation, and (d) MFFA ablation.} 
\label{ablation}
\end{figure}

\begin{table*}[t]
\centering
\begin{tabular}{ c| c@{\hskip 0.05in}c@{\hskip 0.05in}c@{\hskip 0.05in}c@{\hskip 0.05in}c@{\hskip 0.05in}c@{\hskip 0.05in}c@{\hskip 0.05in}c@{\hskip 0.05in}c@{\hskip 0.05in}c@{\hskip 0.05in}c }
\hline
\textbf{Metric} & \textbf{MTF-GLP-FS} & \textbf{BDSD-PC} & \textbf{TV} & \textbf{PNN} & \textbf{PanNet} & \textbf{DiCNN} & \textbf{FusionNet} & \textbf{U2Net} & \textbf{PanMamba} & \textbf{CANNet} & \textbf{Proposed} \\ \hline
\textbf{D$_\lambda \downarrow$} & 0.035 & 0.076 & 0.055 & 0.032 & \underline{0.018} & 0.037 & 0.035 & 0.024 & 0.023 & 0.019 & \textbf{0.017} \\ 
\textbf{D$_s \downarrow$} & 0.143 & 0.155 & 0.112 & 0.094 & 0.080 & 0.099 & 0.101 & \textbf{0.051} & 0.057 & 0.063 & \underline{0.052} \\ 
\textbf{HQNR$\uparrow$} & 0.828 & 0.781 & 0.839 & 0.877 & 0.904 & 0.868 & 0.867 & \underline{0.927} & 0.921 & 0.919 & \textbf{0.932} \\ \hline

\end{tabular}
\caption{Quantitative comparisons on the GF2 full-resolution dataset.}
\label{WV3_full}
\end{table*}

\begin{figure*}[!h]
\centering
\includegraphics[width=0.99\textwidth]{qb.png} 
\caption{The visual results (Top) and residuals (Bottom) of all compared approaches on the QB reduced-resolution dataset.} 
\label{reduce-wv3}
\end{figure*}

\begin{figure*}[!h]
\centering
\includegraphics[width=0.99\textwidth]{f1.png} 
\caption{The visual results (Top) and  HQNR maps (Bottom) of all compared approaches on the WV3 full-resolution dataset. } 
\label{full-wv3}
\end{figure*}


\begin{figure*}[!h]
\centering
\includegraphics[width=0.99\textwidth]{f2.png} 
\caption{The visual results (Top) and  HQNR maps (Bottom) of all compared approaches on the GF2 full-resolution dataset. } 
\label{full-GF2}
\end{figure*}

\subsection{Ablation Settings Details}
This section provides a more detailed description of some ablation experiments.
\subsubsection{Frequency Attention Triplet}
First, without any ablation, the generation process of our Frequency Attention Triplet is as follows:
\begin{equation}
\begin{aligned}
    &Q_i = \operatorname{MLP}(\operatorname{LN}(P_i)) \\
    &K = \operatorname{MLP}(\operatorname{LN}(P_{LL})) \\
    &V = \operatorname{MLP}(\operatorname{LN}(f_v(M, P_{LL})))
\end{aligned}
\end{equation}
\textit{We separately altered the generation method of one component within the Frequency Attention Triplet while keeping the others unchanged}. Fig.~\ref{comparison} (a)-(c) correspond to the three different ablation settings, where each substitutes \( \overline{Q} \), \( \overline{K} \), or \( \overline{V} \) for the original component while keeping the other two components unchanged.
The process can be obtained using the following equations:
\begin{equation}
\begin{aligned}
    &\overline{Q} = \operatorname{MLP}(\operatorname{LN}(\operatorname{Conv}(P))) \\
    &\overline{K} = \operatorname{MLP}(\operatorname{LN}(\operatorname{Conv}(P))) \\
    &\overline{V} = \operatorname{MLP}(\operatorname{LN}(\operatorname{Conv}(M)))
\end{aligned}
\end{equation}

\subsubsection{Multi-Frequency Fusion Attention}
As illustrated in Fig.~\ref{comparison} (d), we respectively concatenate the convolved \(M\) with the features in different frequency domains and then extract features through a convolutional network. The design of this convolutional network follows the classical PNN approach \cite{PNN}. 


\subsection{Additional Results}
In this section, we present additional qualitative and quantitative results. 
Table \ref{WV3_full} presents the results on the GF2 full-resolution dataset. Fig.~\ref{reduce-wv3} illustrates the visualization results on the QB reduced dataset. Fig.~\ref{full-wv3} and Fig.~\ref{full-GF2} present the visualization results of the WV3 and GF2 full-resolution datasets. 
As depicted in the second row, the redder areas indicate better performance, while the bluer areas indicate poorer performance. Among the methods compared, ours shows the largest and deepest red area, indicating the best performance.




\clearpage
% \bibliography{suppl}
% %\flush
% \clearpage

\end{document}