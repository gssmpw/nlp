\documentclass{article} % For LaTeX2e
\usepackage[dvipsnames]{xcolor} % Load xcolor first with the desired options

% Define custom colors
\definecolor{mydarkred}{rgb}{0.6,0,0}
\definecolor{myblue}{HTML}{268BD2}
\definecolor{mygreen}{HTML}{658354}
\definecolor{orangeinplot}{HTML}{e29c7a}
\definecolor{purpleinplot}{HTML}{7676a4}
\definecolor{greeninplot}{HTML}{288308}

% Load other packages
\usepackage{iclr2025_conference,times}
\usepackage{amsfonts}
\usepackage{amsmath}
\usepackage{bm}
\usepackage{natbib}
\usepackage{algorithm}
\usepackage{algpseudocode}
\usepackage{verbatim}
\usepackage{float}
\usepackage{multirow}
\usepackage{xspace}
\usepackage{pifont}
\usepackage{wrapfig}
\usepackage{tikz}
\usepackage{ctable}
\usepackage{graphicx}
\usepackage{caption}
\usepackage{subcaption}
\usepackage{amsthm}
\usepackage[most]{tcolorbox}
\usepackage{url}

% Configure hyperref
\usepackage[colorlinks,
linkcolor=mydarkred,
urlcolor=RoyalBlue,
citecolor=myblue]{hyperref}
\usepackage{cleveref}

% Configure algorithmic
\renewcommand{\algorithmicrequire}{\textbf{Input:}}
\renewcommand{\algorithmicensure}{\textbf{Output:}}

% Define custom commands
\newcommand{\xmark}{\ding{55}}
\newcommand{\model}{\textsc{GNNSafe}\xspace}
\newcommand{\modelplus}{\textsc{GNNSafe++}\xspace}

% Define tcolorbox for propositions
\definecolor{propbg}{HTML}{F2F2F9}
\definecolor{propfr}{HTML}{00007B}
\newtcolorbox{proposition}{
  enhanced,
  boxrule=0pt,frame hidden,
  borderline west = {2pt}{0pt}{propfr},
  colback=propbg,
  sharp corners
}

% Define custom command for red circles
\definecolor{darkred}{rgb}{0.9, 0, 0}
\newcommand{\redcircle}[1]{%
  \tikz[baseline=(char.base)]\node[shape=circle,fill=darkred,text=white,inner sep=0.05pt] (char) {\small #1};%
}

%\documentclass{article} % For LaTeX2e
% \usepackage{iclr2025_conference,times}
% % Optional math commands from https://github.com/goodfeli/dlbook_notation.
% % %%%%% NEW MATH DEFINITIONS %%%%%

% \usepackage{amsmath,amsfonts,bm}
\usepackage{amsmath,amsfonts}

\usepackage{pifont}


\newcommand{\R}{\mathbb{R}}


\def\va{{\mathbf{a}}}
\def\vg{{\mathbf{g}}}

% Sets
\def\sR{\mathbb{R}}
\def\sC{\mathbb{C}}
\def\sZ{\mathbb{Z}}
\def\sN{\mathbb{N}}
\def\sQ{\mathbb{Q}}

\def\sS{\mathcal{S}}



% Vectors
\def\vzero{{\mathbf{0}}}
\def\vone{{\mathbf{1}}}
\def\vmu{{\mathbf{\mu}}}
\def\vtheta{{\mathbf{\theta}}}
\def\va{{\mathbf{a}}}
\def\vb{{\mathbf{b}}}
\def\vc{{\mathbf{c}}}
\def\vd{{\mathbf{d}}}
\def\ve{{\mathbf{e}}}
\def\vf{{\mathbf{f}}}
\def\vg{{\mathbf{g}}}
\def\vh{{\mathbf{h}}}
\def\vi{{\mathbf{i}}}
\def\vj{{\mathbf{j}}}
\def\vk{{\mathbf{k}}}
\def\vl{{\mathbf{l}}}
\def\vm{{\mathbf{m}}}
\def\vn{{\mathbf{n}}}
\def\vo{{\mathbf{o}}}
\def\vp{{\mathbf{p}}}
\def\vq{{\mathbf{q}}}
\def\vr{{\mathbf{r}}}
\def\vs{{\mathbf{s}}}
\def\vt{{\mathbf{t}}}
\def\vu{{\mathbf{u}}}
\def\vv{{\mathbf{v}}}
\def\vw{{\mathbf{w}}}
\def\vx{{\mathbf{x}}}
\def\vy{{\mathbf{y}}}
\def\vz{{\mathbf{z}}}
\def\vzeta{{\mathbf{\zeta}}}

% Matrix
\def\mA{{\mathbf{A}}}
\def\mB{{\mathbf{B}}}
\def\mC{{\mathbf{C}}}
\def\mD{{\mathbf{D}}}
\def\mE{{\mathbf{E}}}
\def\mF{{\mathbf{F}}}
\def\mG{{\mathbf{G}}}
\def\mH{{\mathbf{H}}}
\def\mI{{\mathbf{I}}}
\def\mJ{{\mathbf{J}}}
\def\mK{{\mathbf{K}}}
\def\mL{{\mathbf{L}}}
\def\mM{{\mathbf{M}}}
\def\mN{{\mathbf{N}}}
\def\mO{{\mathbf{O}}}
\def\mP{{\mathbf{P}}}
\def\mQ{{\mathbf{Q}}}
\def\mR{{\mathbf{R}}}
\def\mS{{\mathbf{S}}}
\def\mT{{\mathbf{T}}}
\def\mU{{\mathbf{U}}}
\def\mV{{\mathbf{V}}}
\def\mW{{\mathbf{W}}}
\def\mX{{\mathbf{X}}}
\def\mY{{\mathbf{Y}}}
\def\mZ{{\mathbf{Z}}}
\def\mBeta{{\mathbf{\beta}}}
\def\mPhi{{\mathbf{\Phi}}}
\def\mLambda{{\mathbf{\Lambda}}}
\def\mSigma{{\mathbf{\Sigma}}}


% Expectation
% \def\eE{\mathop{\mathbb{E}}\limits}
\def\eE{\mathbb{E}}

% Probability
\def\pP{\mathbb{P}}

% Tilde
\def\tf{\tilde{f}}
\def\tS{\tilde{S}}
\def\wtF{\widetilde{\mathcal{F}}}
\def\whR{\widehat{R}}
\def\tvx{\tilde{\mathbf{x}}}
\def\ty{\tilde{y}}


\def\defeq{\overset{\textup{def}}{=}}
% \def\defeq{\overset{.}{=}}
\def\defone{\overset{\text{\ding{172}}}{=}}
\def\deftwo{\overset{\text{\ding{173}}}{=}}
\def\leqone{\overset{\text{\ding{172}}}{\leq}}
\def\leqtwo{\overset{\text{\ding{173}}}{\leq}}
\def\leqthree{\overset{\text{\ding{174}}}{\leq}}
\def\leqfour{\overset{\text{\ding{175}}}{\leq}}
\def\eqone{\overset{\text{\ding{172}}}{=}}
\def\eqtwo{\overset{\text{\ding{173}}}{=}}
\def\eqthree{\overset{\text{\ding{174}}}{=}}
\def\eqfour{\overset{\text{\ding{175}}}{=}}
\def\geqfive{\overset{\text{\ding{176}}}{\geq}}
% \usepackage{amsfonts}
% \usepackage[dvipsnames]{xcolor}

% \definecolor{mydarkred}{rgb}{0.6,0,0}
% \definecolor{myblue}{HTML}{268BD2}
% \definecolor{mygreen}{HTML}{658354}
% \definecolor{orangeinplot}{HTML}{e29c7a}
% \definecolor{purpleinplot}{HTML}{7676a4}
% \definecolor{greeninplot}{HTML}{288308}
% \usepackage[colorlinks,
% linkcolor=mydarkred,
% urlcolor = RoyalBlue,
% citecolor=myblue]{hyperref}

% % \usepackage[hidelinks]{hyperref}
% \usepackage{url}

% % Added packages
% \usepackage{amsmath}
% \usepackage{bm}
% \usepackage{natbib}

% \usepackage{algorithm}
% % \usepackage{algorithmic}
% \usepackage{algpseudocode}
% \renewcommand{\algorithmicrequire}{\textbf{Input:}}
% \renewcommand{\algorithmicensure}{\textbf{Output:}}
% \usepackage{cleveref}
% \usepackage{verbatim}
% % \usepackage{graphicx}
% % \graphicspath{ {./} }
% \usepackage{tgcursor}
% % \usepackage[dvipsnames]
% % \usepackage{color,soul}

% % Packages for table
% \usepackage{float}
% \usepackage{multirow}
% \usepackage{xspace}
% \usepackage{pifont}
% \usepackage{wrapfig}
% \usepackage{tikz}
% \usepackage{ctable}
% \newcommand{\xmark}{\ding{55}}
% \newcommand{\model}{\textsc{GNNSafe}\xspace}
% \newcommand{\modelplus}{\textsc{GNNSafe++}\xspace}

% % Packages for graph
% \usepackage{graphicx}
% \usepackage{caption}
% \usepackage{subcaption}

% \usepackage{amsthm}

% % \newtheorem{proposition}{Proposition}
% \usepackage[most]{tcolorbox}
% \definecolor{propbg}{HTML}{F2F2F9}
% \definecolor{propfr}{HTML}{00007B}

% \newtcolorbox{proposition}{
% enhanced,
% boxrule=0pt,frame hidden,
% borderline west = {2pt}{0pt}{propfr},
% colback=propbg,
% sharp corners
% }

% \definecolor{darkred}{rgb}{0.9, 0, 0}
% \newcommand{\redcircle}[1]{%
%   \tikz[baseline=(char.base)]\node[shape=circle,fill=darkred,text=white,inner sep=0.05pt] (char) {\small #1};%
% }

\title{GOLD: Graph Out-of-Distribution Detection via Implicit Adversarial Latent Generation}

% Authors must not appear in the submitted version. They should be hidden
% as long as the \iclrfinalcopy macro remains commented out below.
% Non-anonymous submissions will be rejected without review.

\author{Danny Wang, Ruihong Qiu, Guangdong Bai, Zi Huang\\
% \thanks{ Use footnote for providing further information
% about author (webpage, alternative address)---\emph{not} for acknowledging
% funding agencies.  Funding acknowledgements go at the end of the paper.} \\
The University of Queensland\\
\texttt{\{danny.wang,r.qiu,g.bai,helen.huang\}@uq.edu.au}
}

% The \author macro works with any number of authors. There are two commands
% used to separate the names and addresses of multiple authors: \And and \AND.
%
% Using \And between authors leaves it to \LaTeX{} to determine where to break
% the lines. Using \AND forces a linebreak at that point. So, if \LaTeX{}
% puts 3 of 4 authors names on the first line, and the last on the second
% line, try using \AND instead of \And before the third author name.

\newcommand{\fix}{\marginpar{FIX}}
\newcommand{\new}{\marginpar{NEW}}

\iclrfinalcopy % Uncomment for camera-ready version, but NOT for submission.
\begin{document}


\maketitle

\begin{abstract}
% The abstract paragraph should be indented 1/2~inch (3~picas) on both left and
% right-hand margins. Use 10~point type, with a vertical spacing of 11~points.
% The word \textsc{Abstract} must be centered, in small caps, and in point size 12. Two
% line spaces precede the abstract. The abstract must be limited to one
% paragraph.
Despite graph neural networks' (GNNs) great success in modelling graph-structured data, out-of-distribution (OOD) test instances still pose a great challenge for current GNNs. One of the most effective techniques to detect OOD nodes is to expose the detector model with an additional OOD node-set, yet the extra OOD instances are often difficult to obtain in practice. Recent methods for image data address this problem using OOD data synthesis, typically relying on pre-trained generative models like Stable Diffusion. However, these approaches require vast amounts of additional data, as well as one-for-all pre-trained generative models, which are not available for graph data.
% While recent methods for image data attempt to resolve this problem with data synthesis approaches, this OOD synthesis generally relies on pre-trained generative models, e.g., Stable Diffusion, that inevitably necessitate huge amounts of additional data. Further, such synthesis methods are not easy to extend to graph data given that there is no one-for-all pre-trained generative model in graph.
% \textcolor{red}{These challenges highlight the need for a graph OOD detector with effective OOD data synthesis.}
Therefore, we propose the GOLD framework for graph OOD detection, an implicit adversarial learning pipeline with synthetic OOD exposure without pre-trained models.
% method for OOD detection with OOD data synthesis under an adversarial learning pipeline. %an adversarial data synthesis-based framework for OOD detection, guided by an energy scoring function derived from a well-trained GNN. 
The implicit adversarial training process employs a novel alternating optimisation framework by training: (1) a latent generative model to regularly imitate the in-distribution (ID) embeddings from an evolving GNN, and (2) a GNN encoder and an OOD detector to accurately classify ID data while increasing the energy divergence between the ID embeddings and the generative model's synthetic embeddings. This novel approach implicitly transforms the synthetic embeddings into pseudo-OOD instances relative to the ID data, effectively simulating exposure to OOD scenarios without auxiliary data.
% Specifically, the adversarial pipeline is proposed to train (1) a standard node-level classifier to classify the data from in-distribution; (2) a latent \textcolor{red}{generative} model to effectively synthesise \textcolor{red}{pseudo-OOD} data; and (3) a novel energy-based detector\textcolor{red}{, under an alternating optimisation framework.}
% Specifically, the latent diffusion module is used to efficiently and effectively mimic training data. These synthetic data will be optimised to possess OOD characteristics. % to deviate from ID data. 
Extensive OOD detection experiments are conducted on five benchmark graph datasets, verifying the superior performance of GOLD without using real OOD data compared with the state-of-the-art OOD exposure and non-exposure baselines.
% which encompass both synthetic and real-world data distribution shifts. The results demonstrate the effectiveness of GOLD over the state-of-the-art.
\footnote{Code is available at \url{https://github.com/DannyW618/GOLD}.}
\end{abstract}

\section{Introduction}
The proliferation of Graph Neural Networks (GNNs) across diverse domains and real-world applications has underscored the importance of robust and reliable predictive systems~\citep {GCN,GraphSage}. %Central to the effectiveness of these models is their ability to accurately generalise and identify instances lying within and outside of the training distribution. 
%Although GNNs, by their very nature, excel in capturing intricate relationships and patterns within graph-structured data, 
Their performance relies crucially on the assumption that the testing data follows the same distribution as the training data~\citep{OOD-GNN, GCN, LiSA, GraphSage}. This assumption is frequently violated in practice, as real-world graph data is generally filled with out-of-distribution (OOD) instances~\citep{GOOD-cert, CIGA, GDSBenchmark, OODLink, GOODSurvey, MoleOOD}. Consequently, inaccurate predictions will inevitably be made by the deployed models, which can be detrimental in critical areas like medical diagnosis and drug discovery~\citep{medicalDiagnoisis,GraphMedicalDiagnosis, HealthOOD, HealthForecast, DrugOOD}. Thus, it is necessary to develop OOD detection methods to identify out-of-distribution instances that deviate from the training distribution~\citep{OODD, LRatio_OODD, NLPOODD, GOODD-uncertainty}.

Recent work has made significant strides in developing OOD detection techniques tailored for graph-structured data, primarily in three categories~\citep{OODGAT, LMN, GNNSafe, GPN}. (1) General OOD detection methods train the detector only with in-distribution (ID) data from the training set~\citep{SGOOD, grasp, GOOD-D, GOODAT}. This process involves fine-tuning a classifier and learning graph representations to improve the model's OOD detection performance using various scoring metrics. (2) A more effective method for OOD detection is OOD exposure, which takes advantage of exposing the detector with additional OOD samples during training~\citep{SLW, GNNSafe, OE, GenOE}. These methods generally require an extra dataset containing OOD samples and the detector is trained to discriminate the ID training data with these OOD data. (3) More recently, OOD synthesis methods have been proposed for image data, mainly leveraging pre-trained generative models, e.g., Stable Diffusion~\citep{sd}, to create OOD samples that lie on the boundary of ID data~\citep{NPOS, Dream-OOD, DFDD, ATOL}.

% \textcolor{red}{1. OOD exposure uses additional real data: (1) inaccessible; (2) inaccurately reflect true OOD. 2. OOD synthesis: (2) pre-trained inaccessible and no graph.}

Despite the effectiveness of OOD exposure-based methods over general OOD detection methods, two challenges remain: (1) For the OOD exposure approaches using a real and additional OOD dataset, acquiring these extra OOD samples is often infeasible during model training in the real world. Furthermore, relying on the additional OOD dataset to guide the detector in distinguishing the ID and OOD data could lead to an inaccurate decision boundary. This is because the training logic assumes that the exposed OOD data can represent the distribution of OOD data from test scenarios, which has no guarantee in real-world~\citep{VOS, manifold}. (2) Although OOD synthesis-based approaches have been proposed to resolve the lack of unknown data, these methods typically rely on pre-trained models built upon substantial amounts of auxiliary data~\citep{regOOD, NPOS, Dream-OOD, MOL_DIF}. %Generally, the synthesis of OOD data is based on the boundary instances from the ID data, which could misrepresent the semantics of real OOD data. 
Moreover, the lack of a one-for-all pre-trained generative model for graph data hinders the synthesis of OOD data using simple plug-and-play models~\citep{GFM}. Thus, this presents the key motivation:
% Despite the effectiveness of OOD exposure-based methods over general OOD detection methods, there are still two challenges remained. the implementation requires access to real OOD data during training.
% Such data is typically sourced by partitioning and modifying the ID dataset based on specific attributes or by collecting it from distinct datasets~\citep{GNNSafe, OODGAT, OE}.
% Nonetheless, these extra OOD samples are often infeasible to collect during model training in real-world. The inclusion of OOD samples in training enables the model to distinguish between ID and OOD data, improving its OOD detection capabilities. However, using OOD datasets can introduce bias, potentially causing the model to perform better on similarly distributed OOD test sets but worse on dissimilar sets~\citep{VOS, manifold}. Although data synthesis-based approaches have been proposed to resolve the lack of unknown data, these methods typically rely on pre-trained models, which require substantial amounts of auxiliary data~\citep{regOOD, NPOS, Dream-OOD, MOL_DIF}. Similar to using real OOD instances, leveraging pre-trained generative models built upon an additional huge amount of data may introduce bias into the generation process. Specifically, the generated samples can be influenced by the distribution of the auxiliary data, which may not accurately reflect the characteristics of true OOD samples relative to ID data. Thus, this presents the key challenge:
% \vspace{-0.3cm}
\begin{center}
    \textbf{\textit{How to enhance graph OOD detection by exposing to OOD scenarios without auxiliary data?}}
\end{center}
% \vspace{-0.3cm}
%This presents challenges in terms of both acquiring OOD data (and assessing its nature. In real-world deployment, it is often difficult to obtain OOD data  How can we discern which OOD data might aid or hinder OOD detection?)

\begin{wrapfigure}{R}{0.5\textwidth}
% \begin{figure}[h!]
    \vspace{-0.5cm}
    \centering
    \begin{subfigure}{0.24\textwidth}
        \centering
        \includegraphics[width=1\textwidth]{fig/Twitch-initial-1_edited.png}
        \vspace{-0.6cm}
        \caption*{(a) Initial energy.}
    \end{subfigure}
    \begin{subfigure}{0.24\textwidth}
        \centering
        \includegraphics[width=\linewidth]{fig/Twitch-during-training6-final-1.png}
         \vspace{-0.6cm}
        \caption*{(b) Post-training energy.}
    \end{subfigure}
    
    \begin{subfigure}{0.25\textwidth}
        \centering
        \includegraphics[width=\linewidth]{fig/umap_twitch_initial_v3_reduced_resolution.png}
         \vspace{-0.6cm}
        \caption*{(c) Initial embeds.}
    \end{subfigure}%
    \begin{subfigure}{0.26\textwidth}
        \centering
        \includegraphics[width=\linewidth]{fig/umap_twitch_post_v3_reduced_resolution.png}
         \vspace{-0.6cm}
        \caption*{(d) Post-training embeds.}
    \end{subfigure}%
\caption{Motivation of GOLD: The initially close energy distributions (a) after training the latent generative model, become separated  after training GOLD (b), where the initial pseudo-OOD (p-OOD) embeddings (embeds.,) (c) implicitly diverges from the ID data and resembles real OOD instances (d).}
% \end{figure}
\vspace{-0.5cm}
\label{fig:GOLD Motivation}
\end{wrapfigure}

In light of the above challenges, the intuition of this work is to generate and expose pseudo-OOD samples solely based on the ID training data to ensure effective OOD detection. To achieve this, we propose an implicit adversarial training framework with a novel alternating optimisation schema by training: (1) a latent generative model (LGM) to \textbf{regularly generate embeddings similar to the in-distribution (ID) embeddings from an evolving GNN}, and (2) a GNN encoder and an OOD detector to accurately classify ID data while \textbf{increasing the energy divergence between these generated embeddings and the ID embeddings}. This novel approach implicitly transforms synthetic embeddings into pseudo-OOD instances relative to the ID data, effectively simulating OOD exposure without auxiliary data. Evident in Figure \ref{fig:GOLD Motivation}, the initially similar energy distributions after LGM training diverge post-training, which implicitly separates the embedding distributions, ensuring the pseudo-OOD data resemble close to the real OOD instances. The main contributions of this paper are summarised as follows:
% In light of the above challenges, the intuition of this work is to generate and expose pseudo-OOD samples that are solely based on the ID training data to ensure effective OOD detection. In achieving this, we initiate the process by generating samples akin to the ID data via a latent generative model. Subsequently, we incrementally diverge the synthetic data from the training data toimplicitly transform them into pseudo-OOD samples. \textcolor{red}{Thus, the transformed pseudo-OOD data would facilitate as effective} exposure samples to guide the classifier in OOD detection. To this end, we propose the GOLD framework to execute the pseudo-OOD exposure method through \textcolor{red}{a novel implicit adversarial training framework}. % efficiently. Firstly, a GNN classifier undergoes training with standard classification loss, capturing the intricate relationships and patterns inherent in graph-structured data. Moreover, 
% Specifically, a latent \textcolor{red}{generative} module is integrated to generate imitations of training data, which will subsequently be transformed into pseudo-OOD samples. %To improve efficiency, we implement this diffusion process within the latent space to mimic the learnt graph representations of the GNN. 
% To enable OOD detection, we propose a novel energy objective derived from the prediction logits of a GNN classifier to achieve a high distinguishability of data distribution while maintaining a comparable ID performance.
% % The energy score will serve as a non-probabilistic score, effectively signalling whether each node instance is OOD or ID. Additionally, a GNN and detector module is designed for effective ID classification and OOD detection.
% Uniting all components, an adversarial training paradigm is formulated to optimise the various models, generating a more discernible energy score margin between the ID and synthesised OOD data. 
% %Particularly, an energy-guided objective function consisting of the standard classification losses and energy regularisation losses is used to train the GNN classifier and the MLP module. An additional diffusion loss is also utilised to train the diffusion model. 
% Five benchmark datasets were evaluated, demonstrating the state-of-the-art performance of GOLD on the task of OOD detection. The main contributions of this paper are summarised as follows:
\begin{itemize}
  % \item We investigate the prevailing problem of OOD detection for graph-structured data, exploring the challenge of synthesising pseudo-OOD data in the absence of additional data access.
  \item We propose GOLD, a novel non-OOD exposed synthesis-based framework for graph OOD detection. GOLD includes a unique implicit adversarial training paradigm for effective pseudo-OOD synthesis, which is achieved by a latent generative model and a novel detector.
  % , with the best improvement of the false positive rate (FPR) at 95\% reduced from 33.57\% to 1.78\%.
  %which eliminates the requirement for auxiliary data in synthesising or employing real OOD data to assist training.
  \item We conducted extensive experiments on five benchmark datasets. Without auxiliary OOD data, GOLD achieves state-of-the-art performance compared with non-OOD and OOD exposure methods, with the best improvement of FPR95 reduced from 33.57\% to 1.78\%.
  % \item To our knowledge, this work is among the first to introduce a synthetic-based OOD detection method for graph-structured data that operates without any auxiliary data. Our approach uniquely benefits from OOD exposure to enhance its effectiveness.
\end{itemize}

% \begin{wrapfigure}{R}{0.6\textwidth}
%     \centering
%     \vspace{-1em} 
%     \begin{subfigure}{0.55\textwidth}
%     \includegraphics[width=1\linewidth]{fig/GOLD_motivation_pre.pdf} 
%     \caption{Embeds. \& energy distribution of initial generator training.}
%     \end{subfigure}

    
%     \vspace{1em}
%     \begin{subfigure}{0.55\textwidth}
%     \includegraphics[width=1\linewidth]{fig/GOLD_Motivation_post.pdf} 
%     \caption{Embeds. \& energy distribution post training.}
%      \end{subfigure}

%     \caption{GOLD Motivation: The initially close energy distributions after generator training have been separated post GOLD training, where the pseudo-OOD embeddings (embeds.) are implicitly diverged from ID data and become closer to real OOD instances.}
%     \label{fig:GOLD Motivation}
% \end{wrapfigure}


\section{Preliminary}
Generally, for a node classification problem, a graph is denoted as $ \mathcal{G} = (\mathbf{X},\mathbf{A})$, where $\mathbf{X}\in\mathbb{R}^{n\times d}$ is the node feature matrix with $n$ nodes and feature dimension $d$, and $\mathbf{A}\in\mathbb{R}^{n\times n}$ is an adjacency matrix indicating the connection among nodes. Each node is associated with a label $y \in \{1, 2, ..., C\}$ indicating a total of $C$ classes. For out-of-distribution detection, there are generally two main tasks:

\paragraph{Task I: In-distribution classification.} To formulate the node classification problem for in-distribution data, given test nodes from the same distribution as training nodes, $ P_{train}(\mathbf{X, A}) = P_{test}(\mathbf{X, A})$ and the conditional distribution $P_{train}(\mathbf{y}|\mathbf{X, A}) = P_{test}(\mathbf{y}|\mathbf{X, A})$, the task is to develop an $L$-layer GNN classifier to predict the label $\mathbf{y}\in\mathbb{R}^n$ for the testing nodes with trainable parameters in the GNN classifier (see Appendix~\ref{Appendix:GNN} for details of GNN):
\begin{equation}
    \mathbf{y}=\text{Softmax}(\text{GNN}(\mathbf{X},\mathbf{A})).
\end{equation}
\paragraph{Task II: Out-of-distribution detection.} To detect the testing nodes coming from a different distribution from the training data, where $ P_{train}(\mathbf{X, A}) \neq P_{test}(\mathbf{X, A})$ and the conditional distribution $P_{train}(\mathbf{y}|\mathbf{X, A}) \neq P_{test}(\mathbf{y}|\mathbf{X, A})$, the task is to require an OOD detector $F$ to output a binary prediction for the testing nodes. $F$ is usually built upon the output from the classifier GNN with $F(\mathbf{x},\mathbf{A};\text{GNN})=1$ for data from in-distribution and $F(\mathbf{x},\mathbf{A};\text{GNN})=0$ for data from out-of-distribution~\citep{GNNSafe, energy, FS-OOD}.

\paragraph{Energy Score-based Detector.} Recent work indicated that using the energy score from logits in the classifier can benefit OOD detection~\citep{energy, GNNSafe, EBM-NN}. The energy score $e$ for a node $i$ is defined as:
\begin{equation}
    e_i = - \log \sum\nolimits_{c=0}^{C-1}{\exp(\mathbf{z}_{i,c})},
    \label{eq: energy_gcn}
\end{equation}
where $e_i\in\mathbb{R}$ is the energy score of node $i$, $\mathbf{z}_i\in\mathbb{R}^C$ is the logits for node $i$ output from the classifier $\mathbf{Z}=\text{GNN}(\mathbf{X},\mathbf{A})\in\mathbb{R}^{n\times C}$, and $c$ is to select the logit of the $c$-th element of $\mathbf{z}_i$. Therefore, the energy score-based OOD detector for a node $i$ is instantiated with a threshold $\tau$ as:
\begin{equation}
    F\left(\mathbf{x}_i, \mathbf{A}; \text{GNN}\right)= \begin{cases}0, & \text { if } \quad e_i \ge \tau, \\ 1, & \text { if } \quad e_i < \tau.\end{cases}
    \label{eq:ood_criteria}
\end{equation}
The training of this energy score-based OOD detector is generally based on an energy regulariser~\citep{energy}, which maximises the difference between the energy scores from in-distribution data ($P_\text{ID}$) and out-of-distribution data ($P_\text{OOD}$) with two scalar thresholds, $t_\text{ID}$ and $t_\text{OOD}$:
\begin{equation}
    \begin{split}
    \max_\text{GNN}\mathcal{L}_\text{EReg},    \text{ where }\mathcal{L}_\text{EReg} = 
    \mathbb{E}_{i\sim P_\text{ID}}\left[\operatorname{max}\left(0, t_{\text {ID}}-e_i\right)\right]^2
    + \mathbb{E}_{j\sim P_\text{OOD}}\left[\operatorname{max}\left(0, e_j-t_{\text{OOD}}\right)\right]^2.
    \end{split} 
    \label{eq: ereg}
\end{equation}
% \paragraph{\textcolor{red}{Energy Uncertainty Regularisation.}} \textcolor{red}{To further enhance the effectiveness of energy-based OOD detector, an uncertainty based regulariser is developed in VOS~\citep{VOS} with an extra MLP function mapping the scalar energy to a two dimensional vector for further divergence, $\text{MLP}:\mathbb{R}\mapsto\mathbb{R}^2$:
% \begin{equation}
% \begin{split}
%     \max_\text{MLP}\quad&\mathcal{L}_\text{Uncertainty}\\
%     \text{where}\quad&\mathcal{L}_\text{Uncertainty}=\text{Sigmoid}(\text{MLP}(e))
% \end{split}
% \end{equation}
% }
\paragraph{Energy Propagation for OOD Detector.} To facilitate the energy score for graph data, \textsc{GNNSafe}~\citep{GNNSafe} proposes an energy propagation schema that emulates label propagation for effective OOD detection. This propagated energy is then fed into the objective in Eq.~\ref{eq: ereg}:
\begin{equation}
\mathbf{e}^{(k)}=\alpha \mathbf{e}^{(k-1)}+(1-\alpha) \mathbf{D}^{-1} \mathbf{A} \mathbf{e}^{(k-1)},
\label{eq:eprop}
\end{equation}

where $\mathbf{e}^{(k)}\in\mathbb{R}^{n\times 1}$ is the energy scores for $n$ nodes after $k$-th energy propagation with $\alpha \in [0,1]$ controlling the concentration of energy. $\mathbf{D}$ is the degree matrix of graph $\mathcal{G}$. In the following, the energy scores in our framework will be the propagated energy scores and will be used interchangeably.


\section{GOLD} \label{sec:GOLD_Method}
% \textcolor{red}{Changing from here and leave Section 4 untouched.}
In this section, the GOLD framework for graph OOD detection is described with illustration in Figure~\ref{fig:pipline}. 
In summary, GOLD is trained with a novel implicit adversarial objective that optimises a latent generative model (LGM), a GNN classifier, and an OOD detector. The LGM aims to generate embeddings akin to ID data, while the implicit adversarial objective encourages divergence between the ID and OOD energy scores derived from GNN and detector. This process implicitly transforms the synthetic embeddings into pseudo-OOD, effectively facilitating synthetic OOD exposure. %\textcolor{red}{!!!XXX IF I WERE YOU, I WOULD TALK ABOUT THE IMPLICIT GENERATION LOGIT to replace the following ``three component'' sentence.} The proposed GOLD consists of three components: a GNN classifier for in-distribution data classification, an OOD detector for OOD data detection, and a latent generative model to synthesise pseudo-OOD data for detector training.

The GNN classifier is trained to maximise the log probability of the ground truth classification label:
\begin{equation}\max_\text{GNN}\mathcal{L}_\text{CLS},    \text{ where }\mathcal{L}_\text{CLS} = \log p(y| \mathbf{x}, \mathcal{G}_{\mathbf{x}}).
\label{eq:cls}
\end{equation}
% can be represented as:
% \begin{equation}
%     y=\text{Softmax}(\text{GNN}(\mathbf{x},G_x)).
% \end{equation}
In the following, the detector and the latent generator are both built upon this GNN in GOLD.

% In this section, the GOLD framework for graph OOD detection is described with illustration in Figure~\ref{fig:pipline}. %(We focus on exploring the effectiveness of simulating pseudo-OOD data to improve the discrimination of ID and OOD data.) 
% We first present the energy-based OOD detector in Section \ref{sec:ood detector}, followed by a latent diffusion model for generating the pseudo-OOD instances in Section \ref{sec:latent diffusion model}. Lastly, an adversarial optimisation strategy will be discussed in Section \ref{sec:adversarial paradigm}. 
% The detection model in GOLD consists of two components: a GNN model to effectively capture the intricate relationships and patterns within graph-structured data, and a detector module to improve OOD detection. Using prior energy-based classifier~\citep{GNNSafe}, the encoded latent features $\mathbf{h}_{gnn}$ can be obtained from the $(L-1)^{th}$ layer of a $L$-layer GNN model $f_\theta$, while the logits $\mathbf{z_{gnn}}$ can be derived from the last layer $L$. 

% \begin{equation}
%     \mathbf{h_{gnn}}=f_{{\theta}_{L-1}}(\mathbf{x}, G_{\mathbf{x}}),\quad
%     \mathbf{z_{gnn}}=f_{{\theta}}(\mathbf{x}, G_{\mathbf{x}})
% \end{equation}

% Energy-based training strategies presented in \cref{eq: energy_gcn,eq:eprop,eq: greg} are utilised to enable OOD detection.

\begin{figure}[!t]
\includegraphics[width=0.8\linewidth]{fig/GOLD_ICLR_V2.pdf}
\centering
\caption{Overview of GOLD. Given an input graph, GOLD consists of two components: \textbf{Step} \protect\redcircle{1} trains a latent generative model using hidden representation $\mathbf{H}$ from a frozen GNN.
% , aiming to learn the data distribution by minimising the diffusion loss ($\mathcal{L}_{dif}$).
\textbf{Step} \protect\redcircle{2} trains a GNN classifier and an OOD detector based on the ID data $\mathbf{H}$ and the latent generator generated pseudo data $\mathbf{H}_\text{p-OOD}$. The overall training is in an adversarial manner.
% which includes a trainable GNN, a diffusion model that generates synthetic latent representations ($\mathbf{h_{syn}}$) from random inputs, and a detector model to effectively distinguish ID and synthetic (pseudo-OOD) energy scores, thereby improving OOD detection. The joint training objective combines diffusion loss, classification loss, and a divergence energy term, with gradient updates indicated by dashed lines.
}%, to effectively detect OOD samples using both ID and pseudo-OOD data.}
\vspace{-0.3cm}
\label{fig:pipline}
\end{figure}

\subsection{Latent Generative Model as Pseudo-OOD Generator} \label{sec:latent diffusion model}
% Discuss latent diffusion model (DDPM) 
% So far, we have presented the classification model and the application of OE for OOD detection. In light of our key motivation of exposing the model to OOD scenarios without auxiliary data, we adopt the use of a latent diffusion model for pseudo-OOD synthesis. The latent diffusion model would take input from the encoded node representations of the GNN module, which captures both the global and local information.

In light of the key motivation of exposing the model to OOD scenarios with generated data, an LGM is employed for pseudo-OOD synthesis. The model would take input from the encoded node representations $\mathbf{H} \in \mathbb{R}^{n\times d'}$, where $d'$ is the hidden dimension, of the GNN module after the $(L-1)$-th layer, which captures both the global and local information~\citep{GCN}:
\begin{equation}
    \mathbf{H}=\text{GNN}_{1:L-1}(\mathbf{X},\mathbf{A}),\quad\mathbf{Z}=\text{GNN}_{L}(\mathbf{H},\mathbf{A})=\text{GNN}(\mathbf{X},\mathbf{A}). \label{eq:GCN_emb_logits}
\end{equation}
The LGM aims to mimic and generate latent embeddings close to the ID representations. This is typically achieved by minimising a reconstruction loss or distance between a target and predicted embedding, i.e., if a latent diffusion model (LDM)~\citep{DDPM} or a variational autoencoder (VAE)~\citep{VAE} is used as the pseudo-OOD generator, the objective is given by:

\vspace{-0.5cm}
\begin{equation}
\min_D\mathcal{L}_{\mathrm{Gen}},
\label{eq:latent}
\end{equation}
\begin{equation}
\notag
\text{ where }\mathcal{L}_{\mathrm{Gen}} = 
\begin{cases}
\mathbb{E}_{\mathbf{h}_0, \bm{\epsilon}, t}\left[\left\|\bm{\epsilon} -D\left(\sqrt{\bar{a}_t} \mathbf{h}_0 +\sqrt{\left(1-\bar{a}_t\right)} \bm{\epsilon}, t\right)\right\|_2^2\right],& \text{if LDM}\\
 \mathbb{E}_{q_{D}(\mathbf{h}_\text{p-OOD}|\mathbf{h})} \left[ \log p(\mathbf{h} |\mathbf{h}_\text{p-OOD}) \right] - \text{KL} \left[ {q_{D}}(\mathbf{\mathbf{h}_\text{p-OOD}} | \mathbf{h}) \| p(\mathbf{h}_\text{p-OOD}) \right] ,& \text{if VAE}
\end{cases}
\end{equation}
% where $\mathbf{h}$ denotes the latent vectors, and $D$ is the decoder. 
with latent vectors $\mathbf{h}$ and decoder $D$. For LDM, $D$ is a denoising network that predicts and progressively removes noise during the backward denoising step. Contrary, VAE minimises a reconstruction and embedding distance loss, based on the input latent embeddings and the embeddings generated by the decoder $D$. The pseudo-OOD latent embeddings $\mathbf{h}_\text{p-OOD}$ can thus be generated using the decoder network $D$ with noise vector sampled from a normal distribution $\mathbf{h}_N\sim\mathcal{N}(0,\mathbf{I})$ via:
\begin{equation}
\label{eq:p-ood-dist}
    \mathbf{h}_\text{p-OOD}\sim P_\text{p-OOD},\text{ and }P_\text{p-OOD}=P_D(\mathbf{h}_{\text{p-OOD}}|\mathbf{h}_N)
\end{equation}

% \textcolor{blue}{Till now, the pseudo-OOD data drawn from the distribution of the latent generative model still mimics the in-distribution data. In the next subsection, the process of separating the generated pseudo-OOD data and the ID data will be illustrated.}

For comparison, VAE presents competitive performance and faster training time due to model design, while LDM remains efficient and performs better among (non-) OOD exposure methods when used in GOLD. Moreover, GOLD achieves the same inference time as SOTA baselines with any LGMs. This is because the generative model is not involved during inference. A detailed description of the two generative methods and their corresponding objective is provided in Appendix \ref{Appendix:latent_generative_model}, and further results about effectiveness and efficiency will be discussed in Section \ref{sec:overall_performance} and \ref{sec:computational_cost}. 
%\textcolor{red}{!!!XXXSHOULD BE MORE SPECIFIC SECTION}

Note that at this stage, the synthetic embeddings generated by the LGM still imitate the ID data. In the following subsections, with a novel detector, an implicit adversarial training process will be introduced, which separates the synthetic embeddings from the ID representations, transforming it into pseudo-OOD instances.
%Without loss of generality, a latent diffusion model will be used as the latent generative model in the following.}

\subsection{OOD Detector for ID and Pseuro-OOD Separation}\label{sec:ood detector}
Given that a trained latent generator can synthesise latent representations akin to the ID embeddings, the OOD detector is designed to pull apart the energy scores of ID instances from those of the generated data. This ensures a clear separation between the distributions, and through gradient flow to the trainable GNN encoder, it implicitly separates the synthetic embeddings from the ID embeddings. Hence, the synthetic data is effectively transformed into pseudo-OOD instances relative to ID data, as shown in Figure~\ref{fig:GOLD Motivation}, allowing the model to be exposed to OOD scenarios without the need for real OOD data.
% (the OOD detector is designed to implicitly pull away the ID embeddings from the generated pseudo-OOD embeddings from the perspective of OOD detection and energy scores \textcolor{red}{///the OOD detector is designed to pull away the energy score, as well as the embeddings of ID data from the ones of the generated pseudo-OOD data. (the pull away is explicit, it's not implicit???)}). 
% This ensures clear separation in the distributions, allowing the generated embeddings being implicitly transformed into pseudo-OOD embeddings relative to the ID data, thereby facilitating exposure to OOD scenarios without having real OOD data. 
For clarity, in the following, the OOD exposure in previous methods will be replaced with the pseudo-OOD data generated by the LGM from Eq.~\ref{eq:p-ood-dist}.

In the general design of an energy-based OOD detector as in Eq.~\ref{eq: energy_gcn}, the energy score is a combination of the prediction logits. To overcome the potential difficulties when the number of classes increases or when certain classes are unable to be accurately distinguished by the model, an MLP is applied to the energy and trained with an uncertainty loss as in~\cite{VOS} with $\phi$ as the softmax function:
\begin{equation}
\label{eq:uncertainty}
    \max_\text{GNN, MLP}\mathcal{L}_\text{Unc},\text{ where }\mathcal{L}_\text{Unc}=\mathbb{E}_{i\sim P_\text{ID}} \log[\phi(\text{MLP}(e_i))_{{[0]}}] + \mathbb{E}_{j\sim P_\text{p-OOD}}\log[\phi(\text{MLP}(e_j))_{{[1]}}].
\end{equation}
The subscripts indicate the label of the corresponding logit value from the MLP model after applying $\phi$ (i.e., [0] represents the ID Class 0 and [1] represents the OOD Class 1).
In addition to using this uncertainty objective, we aim to further transform the energy with the classifier output to enhance the separability of the energy:
\begin{equation}
    e'_i = - \log [ e^{\text{MLP}(e_i)_{{[0]}}} + e^{\text{MLP}(e_i)_{{[1]}}}]. \label{eq:denergy}
\end{equation}
With the transformed energy $\mathbf{e}'$, we propose a new divergence regularisation to obtain a more diverged energy score distribution than the pre-transformed energy $e$ from the GNN classifier:
\begin{equation}
    \max_\text{GNN, MLP}\mathcal{L}_\text{DReg},\text{ where }\mathcal{L}_\text{DReg} =  \mathbb{E}_{i\sim P_\text{ID}}\operatorname{max}\left( 0, e_i - e'_i\right)^2+\mathbb{E}_{j\sim P_\text{p-OOD}}\operatorname{max}\left(0, e'_j -e_j\right)^2. \label{eq:dreg}
\end{equation}
We next show that the combination of the two proposed losses with pseudo-OOD data could enable the detector to produce more distinctive energy scores between distributions to assist OOD detection.
\vspace{-0.4cm}
\begin{proposition}
    \vspace{-0.2cm}
    \textbf{Proposition 1.} \textit{The gradient descent on $\mathcal{L}_\text{Unc}$ and $\mathcal{L}_\text{DReg}$ will overall decrease (increase) the transformed energy $\mathbf{e}'$ for in-distribution (pseudo-out-of-distribution) instance, bounded by the given initial energy $\mathbf{e}\sim P_\text{ID}$ ($P_\text{p-OOD}$), respectively, for the detector MLP model.}
    \vspace{-0.2cm}
\end{proposition}
% \vspace{-0.3cm}

% \paragraph{Proposition 1.} The gradient descent on $\mathcal{L}_\text{Unc}$ and $\mathcal{L}_\text{DReg}$ will overall decrease (increase) the transformed energy $\mathbf{e}'$ for any in-distribution ( pseudo-out-of-distribution) instance, bounded by the given energy $\mathbf{e}\sim P_\text{ID}$ ($P_\text{p-OOD}$), for the detector MLP model, respectively.

The proof is provided in Appendix~\ref{Appendix:Proof}. Intuitively, the $\mathcal{L}_\text{Unc}$ aims to train the detector to classify the ID and OOD data with high probability under binary classification, which ensures the separability of embeddings. While the $\mathcal{L}_\text{DReg}$ aims to diverge the energy of ID and OOD based on the logits from this same detector. Therefore, the logits of ID data are expected to have a larger scale than the logits of OOD data, which leads to the energy score based on the logits for this binary classification providing a greater discrepancy between ID and OOD data. The empirical visualisation is shown in Figure~\ref{fig:Logits_smf} of Appendix~\ref{Appendix:Logits vs. SFM}.
% Intuitively, an optimised detector trained using $\mathcal{L}_\text{Unc}$ and $\mathcal{L}_\text{DReg}$ tends to assign larger logits to the ID class and smaller logits to the p-OOD class for ID data while yielding relatively smaller logits in both classes for p-OOD instances, under the binary classification. This leads to more distinguishable energy scores suitable for OOD detection, as opposed to the consistently high softmax scores illustrated empirically in Appendix~\ref{Appendix:Logits vs. SFM}. Hence, this proposition expects greater discrepancy between the transformed energy scores of the ID and p-OOD instances derived from the detector logits, compared to the initial energy scores derived from GNN logits.

\textbf{\textit{Energy Divergence Objective}}: Replacing $P_\text{OOD}$ with $P_\text{p-OOD}$, additionally with the $\mathcal{L}_\text{EReg}$ from Eq.~\ref{eq: ereg}, the objective to diverge the energy for the OOD detector is a combination with weight $\mu, \lambda, \gamma \in \mathbb{R}$:
\begin{equation}
\label{eq:div}
    \max_\text{GNN, MLP}\mathcal{L}_\text{Div},\text{ where }\mathcal{L}_\text{Div}=\mu\mathcal{L}_\text{EReg}+\lambda\mathcal{L}_\text{Unc}+\gamma\mathcal{L}_\text{DReg}.
\end{equation}
After optimising the MLP detector and GNN classifier with the final energy divergence objective $\mathcal{L}_\text{Div}$, the embeddings generated by the fixed LGM will implicitly diverge from the ID embeddings produced by the updated classifier. This divergence occurs because the energy scores generated by the detector are separated by the optimised objective, which will further train the GNN classifier via gradient flow. As a result, the LGM would effectively function as a pseudo-OOD generator.


\subsection{Implicit Adversarial Objective}\label{sec:adversarial paradigm}
To accomplish the ID classification and the OOD detection, the overall objective of the pseudo-OOD synthesis and OOD detector can be formulated in an adversarial style, by combining Eq.~\ref{eq:latent},~\ref{eq:cls} and~\ref{eq:div}:
\begin{equation}
\min_{D}\max_{\text{GNN}, \text{MLP}}\quad \mathcal{L}_\text{Gen}+\mathcal{L}_\text{CLS}+\mathcal{L}_\text{Div}\label{eq:jointObj}
\end{equation}
The intuition of this adversarial objective stems from the contradictory optimisation purpose from the individual objectives. When fixing the GNN encoder, $\mathcal{L}_\text{Gen}$ aims to optimise the LGM to minimise the gap between the generated pseudo-OOD embeddings and ID embeddings. This ensures the LGM can generate meaningful representations that are initially close to ID data, instead of generating meaningless and far away pseudo-OOD data. When fixing the generator, $\mathcal{L}_\text{CLS}$ and $\mathcal{L}_\text{Div}$ aims to optimise the GNN encoder and the MLP detector to maximise the discrepancy in the energy score between the pseudo-OOD data and the ID data, while keeping the GNN encoding for ID data meaningful for classification. Notably, GOLD does not directly generate pseudo-embeddings from the LGM against the ID embeddings but instead, the encoder and the detector implicitly pull the ID embeddings away from the generated pseudo-OOD embeddings via the energy score divergence.

% The intuition of this adversarial objective stems from the tendency of LGM to produce samples that closely resemble ID embeddings, resulting in similarly distributed energy scores. This ensures we can obtain meaningful representations that are initially close to ID data. Conversely, the OOD exposed energy-regularisation constraints proposed in Eq.~\ref{eq: ereg} and~\ref{eq:dreg} aim to differentiate between ID and OOD energy scores, implicitly transforming the ID-like synthetic embeddings to become pseudo-OOD for OOD exposure. Thus, this naturally leads to an adversarial training paradigm. This paradigm seeks to pull the generated embeddings and ID representations together while maximising the energy score discrepancy between ID and these generated pseudo-OOD data, allowing the model to implicitly expose to OOD scenarios. Notably, GOLD does not directly generate pseudo-embeddings from the LGM against the ID embeddings. Instead, it implicitly achieves this with two steps: \textbf{(1)} optimising the LGM to produce embeddings closer to the ID data by $\mathcal{L}_\text{Gen}$, and \textbf{(2)} enlarging the energy score gap between the distributions, which implicitly increases the distance in the embedding space by updating the GNN via gradient flow from $\mathcal{L}_\text{Div}$.

We note that adversarial training schemas were also previously developed by \cite{ConfOOD} and \cite{BadGAN} to generate OOD data or outliers without pre-training. These methods present a GAN-based model with generator and discriminator components, alongside additional density estimation methods, such as a confidence classifier and pre-trained models for OOD prediction. These methods are shown to be not comparable to the OOD exposure-based detector~\citep{Dream-OOD}. In contrast, the key adversarial aspect of our framework is that the LGM generates embeddings resembling ID data, while the detector and classifier amplify the energy score gap between these generated pseudo-OOD embeddings and ID samples, implicitly affecting the embeddings' distribution.

\subsection{Alternating Optimisation}
\begin{wrapfigure}{R}{0.55\textwidth}
\begin{minipage}{0.55\textwidth}
\vspace{-1.7cm}
\begin{algorithm}[H] % !h doesn't work?
\caption{Adversarial Optimisation of GOLD}\label{alg:GOLD}
\begin{algorithmic}[1]
\Require {ID graph $\mathcal{G} = (\mathbf{A}, \mathbf{X})$, randomly initialised GNN, MLP detector, and latent generator $D$, epoch numbers $M_1$ and $M_2$, loss coefficients $\lambda, \mu, \gamma$.}
\Ensure Optimised GNN, MLP detector, and generative model $D$
\While{$train$}
\State Obtain $\mathbf{H}$ for ID data with GNN from Eq.~\ref{eq:GCN_emb_logits}
\For{$epoch = 1, \dots, M_1$} \hfill{\textcolor{blue}{// \textbf{Step \protect\redcircle{1}}}}
    \State Train $D$ with $\mathcal{L}_\text{Gen}$ and $\mathbf{H}$ from Eq.~\ref{eq:latent}
\EndFor

\State Sample noise $\mathbf{H}_N$ from Normal distribution
\State Generate pseudo-OOD $\mathbf{H}_\text{p-OOD}$ with $D$ and $\mathbf{H}_T$ from Eq.~\ref{eq:p-ood-dist}
\For{$epoch = 1, \dots, M_2$} \hfill{\textcolor{blue}{// \textbf{Step \protect\redcircle{2}}}}
    \State Train GNN with $\mathcal{L}_\text{CLS}$ and $\mathbf{H}$ Eq.~\ref{eq:cls}
    \State Train GNN and MLP with $\mathcal{L}_\text{Div}$, $\mathbf{H}$ and $\mathbf{H}_\text{p-OOD}$ from Eq.~\ref{eq:div}
\EndFor
\EndWhile
\end{algorithmic}
\end{algorithm}
\vspace{-1cm}
\end{minipage}
\end{wrapfigure}

To facilitate effective optimisation of the implicit adversarial objective in Eq.~\ref{eq:jointObj}, we propose an alternating training schema for GOLD, shown in Algorithm \ref{alg:GOLD}. The intuition is to repetitively generate samples close to the ID embeddings of GNN, and then diverge the energy score distribution of the ID and the synthesised pseudo-OOD data. This process ensures the pseudo-OOD data do not diverge too far from ID, enabling effective pseudo-OOD exposure.

Thus, GOLD consists of two alternating optimisation steps: \textbf{Step \protect\redcircle{1}}: \textbf{LGM mimicking embeddings of evolving GNN}: A latent generative model is trained to mimic the ID embeddings $\mathbf{H}$, extracted from the $(L-1)$-th layer of an evolving GNN encoder trained in Step 2, to generate pseudo-OOD embeddings. This ensures the pseudo-OOD, $\mathbf{H}_\text{p-OOD}$, is sufficiently close to the ID embeddings and avoids far away and meaningless generation before subsequent divergence (Line 2-7 in Algorithm \ref{alg:GOLD}); \textbf{Step \protect\redcircle{2}}: \textbf{Detector and GNN diverging energy of evolving generator}: The GNN encoder and OOD detector are trained to diverge the energy between the ID embeddings and the pseudo-OOD embeddings generated by the evolving LGM trained in Step 1 (Line 8-11 in Algorithm \ref{alg:GOLD}). Figure \ref{fig:adv_energy_vis} illustrates the adversarial essence of this training paradigm.

% Ultimately, this implicitly transforms the synthetic data to inherit OOD characteristics w.r.t.\ the training ID data, thus becoming effective pseudo-OOD data for OOD exposure.
% Specifically, in the first main loop (Line 3-5), the latent representations $\mathbf{H}$ extracted from the $L-1$-th layer of the GNN model is used to train the latent diffusion model according to Eq.~\ref{eq:latent}. This ensures we can generate pseudo-OOD representations, $\mathbf{H}_\text{p-OOD}$, sufficiently close to the ID data representation to increase the difficulty for the OOD detector.
% assuring that the energy scores are not diverged too far from ID data after applying the following adversarial process.
% Moreover, we perform in-distribution classification and the energy divergence procedures in the second main loop (Line 8-11).


\begin{figure}[!h]
    \centering
    % \vspace{-0.2cm}
    \begin{subfigure}[b]{0.24\textwidth}
        \centering
        \includegraphics[width=\textwidth]{fig/Twitch-initial.pdf}
        \caption{Ep 1, Tr. LGM.}
        % \label{fig:twitch_score_gap}
    \end{subfigure}
    \hfill
    \begin{subfigure}[b]{0.24\textwidth}
        \centering
        \includegraphics[width=\textwidth]{fig/Twitch-during-training3.pdf}
        \caption{Ep 14, Tr. GNN \& Det.}
        % \label{fig:twitch_id_ood}
    \end{subfigure}
    \hfill
    \begin{subfigure}[b]{0.24\textwidth}
        \centering
        \includegraphics[width=\textwidth]{fig/Twitch-during-training5.pdf}
        \caption{Ep 15, Tr. LGM.}
        % \label{fig:cora_id_ood}
    \end{subfigure}
    \hfill
    \begin{subfigure}[b]{0.24\textwidth}
        \centering
        \includegraphics[width=\textwidth]{fig/Twitch-during-training6-final.pdf}
        \caption{Ep 22, Tr. GNN \& Det.}
        % \label{fig:cora_score_gap}
    \end{subfigure}
    \caption{Transformed energy $\mathbf{e}'$ distribution during adversarial training (Tr.) on the {\fontfamily{qcr}\selectfont Twitch} dataset for \textcolor{LimeGreen}{in-distribution (ID)}, \textcolor{red}{pseudo (p-)OOD}, and \textcolor{violet}{real OOD} across epochs. \textbf{(a)} shows that after LGM trains to mimic ID data, energy scores are overlapped for ID, p-OOD, and OOD in the initial stages. \textbf{(b)} indicates that after training GNN and the detector to separate the energy of ID and the p-OOD, the real OOD energy cannot be effectively separated from ID. This is a similar situation to the OOD exposure for GNNSafe as in Figure~\ref{fig:twitch_gnnsafe_id_ood}. \textbf{(c)} shows that under adversarial learning, the LGM will be updated to generate p-OOD closer to the updated ID data, preventing it from being too far away from ID data with ineffective OOD learning. \textbf{(d)} displays the final energy distribution after convergence, with real OOD and ID being well separated, while p-OOD and OOD being well aligned.}
    \vspace{-0.5cm}
    \label{fig:adv_energy_vis}
\end{figure}

\section{Experiments} \label{sec:Experiments}
% In this section, we present the experiments to evaluate the performance of GOLD under two tasks: OOD detection and 2) ID classification.
% We follow the evaluation protocol proposed in~\citep{GNNSafe}.
% Further experiments including hyper-parameter sensitivity analysis are provided in Appendix \ref{Appendix:Additional_exp}.
% \subsection{Setup} \label{sec:setup}

\paragraph{Datasets.} \label{sec:datasets}
Following \cite{GNNSafe}, five benchmark datasets are used for OOD detection evaluation, including four single-graph datasets: (1) {\fontfamily{qcr}\selectfont Cora}, (2) {\fontfamily{qcr}\selectfont Amazon-Photo}, (3) {\fontfamily{qcr}\selectfont Coauthor-CS}, with synthetic OOD data created via: structure manipulation, feature interpolation, and label leave-out; and (4) {\fontfamily{qcr}\selectfont ogbn-Arxiv}, OOD by year, and (5) one multi-graph scenario: {\fontfamily{qcr}\selectfont TwitchGamers-Explicit}, OOD by different graphs. Detailed splits are provided in Appendix~\ref{Appendix:dataset}.

\paragraph{Baselines.}
We compared GOLD with 12 baseline models, classed into three categories. (1) General non-OOD exposed methods: MSP~\citep{MSP}, ODIN~\citep{ODIN}, Mahalanobis (short for Maha)~\citep{Mahalanobis}, and Energy~\citep{energy}, with GNN used as backbone. 
%These methods are modified to adopt a GNN as the backbone, as detailed in \cite{GNNSafe}. 
(2) Graph-specific non-OOD exposed detection methods: GKDE~\citep{GKDE}, GPN~\citep{GPN}, \textsc{GNNSafe}~\citep{GNNSafe}, and \textsc{NODESafe}~\citep{NODESAFE}. (3) Real OOD exposed methods: adopts techniques from computer vision, such as OE~\citep{OE} and Energy FT~\citep{energy}, along with the state-of-the-arts {GNNSafe++}~\citep{GNNSafe} and \textsc{NODESafe++}~\citep{NODESAFE} for graph data. Note that OOD synthesis methods from computer vision~\citep{VOS,Dream-OOD,ConfOOD,NPOS} are not compared due to the non-trivial application from image to graph.
% \textcolor{red}{WRITE FROM HERE!!!!!!XXXXXX}. In addition, we also compared against three graph-specific SOTA OOD detection methods: GKDE~\citep{GKDE}, GPN~\citep{GPN}, and two variants of energy-based detection model \textsc{GNNSafe} and \textsc{GNNSafe++}~\citep{GNNSafe}.
\vspace{-0.1cm}
\paragraph{Metrics.}
The following common practice metrics are used for evaluation: AUROC, AUPR, and FPR95 for OOD detection and Accuracy for ID classification. Metric details are in Appendix \ref{Appendix:Metrics}.
\paragraph{Implementations.}
For a fair comparison, GCN is used as the backbone across all methods, with a layer depth of $2$ and a hidden size of $64$. The propagation iteration $k$ in Eq.~\ref{eq:eprop} is set to $2$, and the controlling parameter $\alpha$ of $0.5$ is used. For LDM, the timestep $T$ is configured within \{$600$, $800$, $1000$\}, $\beta_1 = 10^{-4}$, and $\beta_T = 0.02$. The denoising network $D$ and the MLP detector model are implemented with varying layer and hidden dimension sizes within \{$2$, $3$\} and \{$128$, $256$, $512$\} respectively, subject to the dataset. Additional hyperparameter analysis and parameter details are provided in Appendix \ref{Appendix:latent_generative_model}. We use the Adam optimizer for optimisation~\citep{Adam}.

\begin{table}[t!]
    \centering
    \caption{Model performance comparison: out-of-distribution detection results are measured by \textbf{AUROC} ($\uparrow$) $/$ \textbf{AUPR} ($\uparrow$) $/$ \textbf{FPR95} ($\downarrow$) ($\%$) and in-distribution classification results are measured by accuracy \textbf{(ID ACC)} ($\uparrow$). The average performance of the OOD test sets is reported, with variance reflecting performance differences across distinct test sets. Detailed results for individual subsets are reported in Appendix \ref{Appendix:Additional_exp}. OOD detection performance was prioritised, with the detection results of our Non-OOD exposed GOLD against Non- (Real-) OOD Exposure methods highlighted by {\color{teal}{best}} and {\color{purple}{runner-up}} (\textbf{best} and {\underline{runner-up}}), respectively. Dashed line indicates unavailability.}
    % \vspace{-0.3cm}
% \label{tab:ood-result1}
    \resizebox{1\linewidth}{!}{
    \begin{tabular}{c|c|cccccccc|cccc|cc}
    \specialrule{.1em}{.05em}{.05em} 
    \multirow{2}{*}{} & \multirow{2}{*}{\textbf{Metrics}} & \multicolumn{8}{c|}{\textbf{Non-OOD Exposure}} & \multicolumn{4}{c|}{\textbf{Real OOD Exposure}} & \multicolumn{2}{c}{\textbf{GOLD (Non-OOD)}}\\
    % \cmidrule(lr){3-9} \cmidrule(lr){10-12} \cmidrule(lr){13-13}
    % & & MSP & ODIN & Mahalanobis & Energy & GKDE & GPN & \textsc{GNNSafe} & OE & Energy FT & \textsc{GNNSafe++} & GOLD\\
    & & \textbf{MSP} & \textbf{ODIN} & \textbf{Maha} & \textbf{Energy} & \textbf{GKDE} & \textbf{GPN} & \textbf{\textsc{GNNSafe}} & \textbf{\textsc{NODESafe}} & \textbf{OE} & \textbf{Energy FT} & \textbf{\textsc{GNNSafe++}} & \textbf{\textsc{NODESafe++}} & \textbf{w/ VAE} & \textbf{w/ LDM}\\
    \midrule
    \multirow{4}{*}{\rotatebox{90}{{\fontfamily{qcr}\selectfont Twitch}}} 
        & AUROC & 33.59 & 58.16 & 55.68 & 51.24 & 46.48 & 51.73 & 66.82 & \textcolor{purple}{89.99} & 55.72 & 84.50 & 95.36 & \underline{98.50} & 99.26 & \textcolor{teal}{\textbf{99.46 $\pm$ 0.09}} \\
        & AUPR  & 49.14 & 72.12 & 66.42 & 60.81 & 62.11 & 66.36 & 70.97 & \textcolor{purple}{93.33} & 70.18 & 88.04 & 97.12 & \underline{99.18} & 98.54 & \textcolor{teal}{\textbf{99.62 $\pm$ 0.06}} \\
        & FPR95   & 97.45 & 93.96 & 90.13 & 91.61 & 95.62 & 95.51 & 76.24 & \textcolor{purple}{47.00} & 95.07 & 61.29 & 33.57 & \underline{3.43} & 3.03 & \textcolor{teal}{\textbf{1.78 $\pm$ 0.43}} \\
        & ID ACC & 68.72 & 70.79 & 70.51 & 70.40 & 67.44 & 68.09 & 70.40 & 71.79 & 70.73 & 70.52 & 70.18 & 71.85 & 68.50 & 68.49 $\pm$ 0.13 \\
    \midrule
    \multirow{4}{*}{\rotatebox{90}{{\fontfamily{qcr}\selectfont Cora}}} 
        & AUROC & 82.55 & 49.87 & 54.74 & 83.09 & 69.54 & 84.56 & 91.25 & \textcolor{purple}{94.39} & 79.76 & 85.13 & 92.98 & \underline{95.36} & 89.96 & \textcolor{teal}{\textbf{95.84 $\pm$ 0.69}} \\
        & AUPR  & 65.82 & 26.08 & 34.43 & 66.21 & 46.09 & 68.02 & 82.62 & \textcolor{purple}{86.01} & 64.93 & 67.89 & 84.93 & \underline{88.08} & 93.19 & \textcolor{teal}{\textbf{91.17 $\pm$ 2.59}}\\
        & FPR95   & 62.39 & 100.00 & 96.30 & 65.21 & 80.51 & 58.30 & 47.38 & \textcolor{purple}{26.04} & 75.22 & 51.03 & 38.44 & \underline{20.20} & 28.66 & \textcolor{teal}{\textbf{17.83 $\pm$ 3.78}} \\
        & ID ACC & 79.91 & 79.61 & 79.57 & 80.34 & 79.86 & 81.65 & 80.37 & 81.92 & 77.69 & 80.44 & 81.45 & 81.65 & 76.79 & 81.66 $\pm$ 7.94 \\
    \midrule
    \multirow{4}{*}{\rotatebox{90}{{\fontfamily{qcr}\selectfont Amazon}}}    
        & AUROC & 96.52 & 80.12 & 73.81 & 96.73 & 66.98 & 92.60 & \textcolor{purple}{98.49} & - & 97.79 & 98.04 & \textbf{98.99} & - & 98.68 & \textcolor{teal}{\underline{98.81 $\pm$ 1.40}} \\
        & AUPR  & 95.01 & 77.18 & 72.35 & 95.16 & 71.18 & 90.50 & \textcolor{purple}{98.62} & - & 97.26 & 96.96 & \underline{98.88} & - & 98.89 & \textcolor{teal}{\textbf{98.92 $\pm$ 1.31}} \\
        & FPR95   & 13.83 & 85.22 & 83.44 & 13.15 & 98.47 & 32.64 & \textcolor{purple}{2.30} & - &  7.52 &  5.98 &  \underline{2.10} & - & 5.11 & \textcolor{teal}{\textbf{2.07 $\pm$ 3.46}} \\
        & ID ACC & 93.83 & 93.88 & 93.80 & 93.85 & 87.71 & 89.54 & 93.70 & - & 93.54 & 93.38 & 93.48 & - & 89.91 & 92.99 $\pm$ 1.90 \\
    \midrule
    \multirow{4}{*}[0.22em]{\rotatebox{90}{{\fontfamily{qcr}\selectfont Coauthor}}} 
        & AUROC & 95.74 & 51.71 & 82.02 & 96.64 & 69.24 & 69.89 & \textcolor{purple}{98.82} & - & 97.65 & 98.17 & \textbf{99.28} & - & 98.78 & \textcolor{teal}{\underline{99.00 $\pm$ 1.19}} \\
        & AUPR  & 96.43 & 56.37 & 87.05 & 97.09 & 80.17 & 72.77 & \textcolor{purple}{99.44} & - & 98.04 & 98.51 & \textbf{99.73} & - & 96.40 & \textcolor{teal}{\underline{99.56 $\pm$ 0.43}} \\
        & FPR95   & 21.37 & 99.97 & 48.09 & 15.49 & 97.04 & 69.60 &  \textcolor{purple}{4.28} & - & 10.61 &  7.76 &  \underline{3.18} & - & 4.66 & \textcolor{teal}{\textbf{3.16 $\pm$ 5.46}} \\
        & ID ACC & 93.37 & 93.29 & 93.29 & 93.57 & 87.74 & 89.39 & 93.56 & - & 93.41 & 93.44 & 93.68 & - & 92.22 & 92.69 $\pm$ 1.87 \\
    \midrule
    \multirow{4}{*}{\rotatebox{90}{{\fontfamily{qcr}\selectfont Arxiv}}}    
        & AUROC & 63.91 & 55.07 & 56.92 & 64.20 & 58.32 & OOM & 71.06 & \textcolor{purple}{72.44} & 69.80 & 71.56 & 74.77 & \textbf{75.49} & 71.52 & \textcolor{teal}{\underline{73.90 $\pm$ 0.11}} \\
        & AUPR  & 75.85 & 68.85 & 69.63 & 75.78 & 72.62 & OOM & 80.44 & \textcolor{purple}{81.51} & 80.15 & 80.47 & 83.21 & \textbf{83.71} & 80.25 & \textcolor{teal}{\underline{82.52 $\pm$ 0.12}} \\
        & FPR95   & 90.59 & 100.0 & 94.24 & 90.80 & 93.84 & OOM & 87.01 & \textcolor{purple}{84.27}& 85.16 & 80.59 & 77.43 & \textbf{75.24} & 81.95 & \textcolor{teal}{\underline{80.57 $\pm$ 0.32}} \\
        & ID ACC & 53.78 & 51.39 & 51.59 & 53.36 & 50.76 & OOM & 53.39 & 51.20 & 52.39 & 53.26 & 53.50 & 52.93 & 49.70 & 50.59 $\pm$ 0.53 \\
    \specialrule{.1em}{.05em}{.05em} 
     \end{tabular}}
    \vspace{-0.3cm}
    \label{Table:overall_performance}
\end{table}

\subsection{Overall Performance} \label{sec:overall_performance}
% \textcolor{red}{Overall comparisons with all baselines. One wide table.}

\paragraph{Our Non-OOD exposed GOLD can outperform Non-OOD exposure methods and is competitive with Real OOD exposed methods.} As shown in Table~\ref{Table:overall_performance}, GOLD with LDM consistently surpasses the state-of-the-art non-OOD exposure methods \textsc{NODESafe} and \textsc{GNNSafe} by a large margin across all datasets, as indicated by the teal colouring. When using VAE as LGM, the OOD detection performance is very close while being more lightweight due to the model design. GOLD with VAE can achieve state-of-the-art performance especially when the datasets are challenging for general methods, like {\fontfamily{qcr}\selectfont Twitch} and {\fontfamily{qcr}\selectfont Arxiv}. Additionally, considering LDM as the generative model, GOLD can largely outperform \textsc{GNNSafe++} and achieves better performance than the SOTA \textsc{NODESafe++} in OOD detection for {\fontfamily{qcr}\selectfont Twitch} and {\fontfamily{qcr}\selectfont CORA}, as highlighted in bold. While for {\fontfamily{qcr}\selectfont Amazon}, {\fontfamily{qcr}\selectfont Coauthor}, and {\fontfamily{qcr}\selectfont Arxiv} dataset, GOLD can achieve a comparable performance with \textsc{GNNSafe++} while not significantly surpassing them. The reason can be two-fold. For {\fontfamily{qcr}\selectfont Amazon} and {\fontfamily{qcr}\selectfont Coauthor}, the classifier and the OOD detector are already in high performance, which leads to the fact that the energy from the classifier and the information given by the real OOD data have already been well utilised. The pseudo-OOD generation in GOLD cannot provide much more useful supervision signals for the detector. Nonetheless, GOLD still largely outperforms the non-OOD exposure. While for the {\fontfamily{qcr}\selectfont Arxiv} dataset, the OOD situation is defined by time, which leads to a huge boost of OOD information when exposing a real OOD dataset. In contrast, for GOLD, the pseudo-OOD generation is largely limited by the ID accuracy of the classifier at 50\%. A more detailed table with individual OOD test set performance and variance can be found in Appendix \ref{Appendix:Additional_exp}.

% \textcolor{red}{To demonstrate the effectiveness of GOLD, additional experiments were conducted by substituting the latent diffusion with a much more lightweight latent generator, Variational Auto-Encoder, in Table \ref{Table:overall_performance}%\ref{Appendix:latent_generative_model}
% ~\citep{VAE}. The OOD detection performance is very close while being more lightweight due to the model design. This demonstrates that our adversarial training pipeline can use different latent generators to achieve the goal of pseudo-OOD generation for OOD detection.}

Since GOLD uses \textsc{GNNSafe} as the backbone, the following detailed experiments are mainly conducted based on the comparison with the base \textsc{GNNSafe/++} approach. GOLD with LDM is used as the default model without specific notation.

\subsection{Visualisation of Energy Score Gap} \label{sec:vis}
This experiment presents the energy distribution of GOLD and \textsc{GNNSafe}. Figures \ref{fig:twitch_score_gap} and \ref{fig:cora_score_gap} display \textbf{a distinct separation in the energy scores of ID and p-OOD, as well as ID and OOD, produced by the detector}, exemplifying the effectiveness of GOLD in distinguishing and amplifying the energy margin between ID and (p-)OOD data. Furthermore, Figures \ref{fig:twitch_id_ood} and \ref{fig:cora_id_ood} illustrate the energy score distributions of the test ID data, synthetic OOD data, and the test OOD data. These figures reveal an optimal and almost disjoint between ID and OOD data, where the thresholds $t_\text{ID}$ and $t_\text{OOD}$ indicate a clear energy boundary, thereby indicating the efficacy of GOLD in simulating pseudo-OOD data to facilitate effective OOD detection. Compared with the energy distribution of test ID data, test OOD data and exposed OOD data from \textsc{GNNSafe} in Figures~\ref{fig:twitch_gnnsafe_id_ood} and~\ref{fig:coral_gnnsafe_id_ood}, GOLD can further separate the energy scores between the test ID and OOD data with the pseudo-OOD data.

\begin{figure}[!h]
    \centering
    \begin{subfigure}[b]{0.3\textwidth}
        \centering
        \includegraphics[width=\textwidth]{fig/Twitch-EnergySeparation_Enlarged_v2.pdf}
        \caption{{\fontfamily{qcr}\selectfont Twitch} training (GOLD).}
        \label{fig:twitch_score_gap}
    \end{subfigure}
    \hfill
    \begin{subfigure}[b]{0.3\textwidth}
        \centering
        \includegraphics[width=\textwidth]{fig/Twitch-IDvsOOD_Enlarged_v2.pdf}
        \caption{{\fontfamily{qcr}\selectfont Twitch} test (GOLD).}
        \label{fig:twitch_id_ood}
    \end{subfigure}
    \hfill
    \begin{subfigure}[b]{0.3\textwidth}
        \centering
        \includegraphics[width=\textwidth]{fig/Twitch-GNNSAFE.pdf}
        \caption{{\fontfamily{qcr}\selectfont Twitch} test (GNNSafe++).}
        \label{fig:twitch_gnnsafe_id_ood}
    \end{subfigure}
    \hfill
    \begin{subfigure}[b]{0.3\textwidth}
        \centering
        \includegraphics[width=\textwidth]{fig/CoraL-EnergySeparation_Enlarged_v2.pdf}
        \caption{{\fontfamily{qcr}\selectfont Cora-L} training (GOLD).}
        \label{fig:cora_score_gap}
    \end{subfigure}
    \hfill
    \begin{subfigure}[b]{0.3\textwidth}
        \centering
        \includegraphics[width=\textwidth]{fig/CoraL-IDvsOOD_Enlarged_v2.pdf}
        \caption{{\fontfamily{qcr}\selectfont Cora-L} test (GOLD).}
        \label{fig:cora_id_ood}
    \end{subfigure}
    \hfill
    \begin{subfigure}[b]{0.3\textwidth}
        \centering
        \includegraphics[width=\textwidth]{fig/CORAL-GNNSAFE.pdf}
        \caption{{\fontfamily{qcr}\selectfont Cora-L} test (GNNSafe++).}
        \label{fig:coral_gnnsafe_id_ood}
    \end{subfigure}
    \caption{Energy score distributions for {\fontfamily{qcr}\selectfont Twitch} and {\fontfamily{qcr}\selectfont Cora-L} with GOLD and \textsc{GNNSafe++}. The vertical green (red) dashed lines represent the thresholds $t_\text{ID}$ ($t_\text{OOD}$) from Eq.~\ref{eq: ereg}. $\mathbf{e}$ denotes original energy scores from the GNN, while $\mathbf{e}'$ are transformed scores from the detector, with subscripts for ID, OOD, p-OOD (pseudo), and e-OOD (exposed) data. \textbf{(a) \& (d)} show that transformed energy $\mathbf{e}'$ (\textcolor{LimeGreen}{green} and \textcolor{red}{red}) can be further diverged from the original energy $\mathbf{e}$ (\textcolor{Cyan}{blue} and \textcolor{orange}{orange}). \textbf{(b) \& (e)} indicate that GOLD can align the transformed energy $\mathbf{e}'$ for pseudo OOD (\textcolor{red}{red}) and real OOD (\textcolor{violet}{purple}) in testing. At the same time, the transformed energy $\mathbf{e}'$ of ID (\textcolor{LimeGreen}{green}) can be separated. \textbf{(c) \& (f)} demonstrate that energy separation of test ID (\textcolor{Cyan}{blue}) and OOD (\textcolor{Thistle}{pink}) in \textsc{GNNSafe++} is not effective, such that although the exposed OOD (\textcolor{orange}{orange}) can diverge far away from the ID (\textcolor{Cyan}{blue}), the real OOD (\textcolor{Thistle}{pink}) is still closer to the ID (\textcolor{Cyan}{blue}).}
    \label{fig:Energy_Score_Distribution}
\end{figure}

\begin{wraptable}{r}{0.5\linewidth}
    \centering
    \vspace{-1.3cm}
    \caption{Ablation study.}
    \vspace{-0.3cm}
    \resizebox{1\linewidth}{!}{
    \begin{tabular}{c|c|cc|cc|c}
    % \specialrule{.1em}{.05em}{.05em} 
    \toprule
    & \textbf{Metrics} & \textbf{\textsc{GNNSafe}} & \textbf{\textsc{GNNSafe++}} & \textbf{w/o Adv.} & \textbf{w/o Det.} & \textbf{GOLD}\\
    % \cmidrule(lr){3-4} \cmidrule(lr){5-6} \cmidrule(lr){7-7}
    % & & \textbf{\textsc{GNNSafe}} & \textbf{\textsc{GNNSafe++}} & \textbf{w/o Adv.} & \textbf{w/o Det.} & \textbf{GOLD}\\
    \midrule
    \multirow{4}{*}{\rotatebox{90}{\fontfamily{qcr}\selectfont Twitch}} 
        & AUROC & 66.82 & \underline{95.36} & 84.59 & \textcolor{purple}{77.70} & \textcolor{teal}{\textbf{99.46}} \\
        & AUPR  & 70.97 & \underline{97.12} & 88.69 & \textcolor{purple}{83.91} & \textcolor{teal}{\textbf{99.62}}\\
        & FPR95 & \textcolor{purple}{76.24} & \underline{33.57} & 59.71  & 79.84 & \textcolor{teal}{\textbf{1.78}}\\
        & ID ACC & 70.40 & 70.18 & 70.97 & 70.97  & 68.49\\
    \midrule
    \multirow{4}{*}{\rotatebox{90}{\fontfamily{qcr}\selectfont Cora}} 
        & AUROC & 91.25 & 92.98 & 89.64 & \textcolor{purple}{\underline{93.43}} & \textcolor{teal}{\textbf{95.84}} \\
        & AUPR  & 82.62 & 84.93 & 80.22 & \textcolor{purple}{\underline{86.78}} & \textcolor{teal}{\textbf{91.17}} \\
        & FPR95 & 47.38 & 38.44 & 46.33 & \textcolor{purple}{\underline{34.01}} & \textcolor{teal}{\textbf{17.83}} \\
        & ID ACC & 80.37 & 81.45 & 77.60 & 80.70 & 81.66\\
    \midrule                        
    \multirow{4}{*}{\rotatebox{90}{\fontfamily{qcr}\selectfont Arxiv}}
        & AUROC & \textcolor{purple}{71.06} & \textbf{74.77} & 69.76 & 69.91 & \textcolor{teal}{\underline{73.90}}\\
        & AUPR  & \textcolor{purple}{80.44} & \textbf{83.21} & 78.93 & 79.05 & \textcolor{teal}{\underline{82.52}}\\
        & FPR95 & \textcolor{purple}{87.01} & \textbf{77.43} & 88.16 & 89.67 & \textcolor{teal}{\underline{80.57}}\\
        & ID ACC & 53.39 & 53.50 & 49.89 & 49.66 & 50.59\\
    \specialrule{.1em}{.05em}{.05em} 
    \end{tabular}}
    \label{Table:Ablation_general}
    \vspace{-0.5cm}
\end{wraptable}
\subsection{Ablation Study}
\label{sec:ablation}
In the ablation study, two variants are studied: (1) pre-training the LDM without the adversarial pipeline (w/o Adv.), and (2) removing the MLP detector, using GNN energy scores instead (w/o Det.). In Table~\ref{Table:Ablation_general}, \textbf{both the adversarial training paradigm and the new detector significantly contribute to the GOLD}. The results reveal that without adversarial learning, the OOD detection performance has a significant drop for all situations. This underscores the efficacy of the adversarial framework in pseudo-OOD exposure. Moreover, we observe a more unstable performance when removing the detector. In this scenario, the model can still surpass other baselines on datasets like {\fontfamily{qcr}\selectfont Cora}, but it shows a significant drop on others. This juxtaposition exemplifies the necessity of the detector model in GOLD. Nonetheless, the results illustrate the importance of integrating all components to enhance the model's OOD detection capabilities. We provide additional explanation and visualisation of the ablation study in Appendix \ref{Appendix:ablation_vis}.

\begin{wraptable}{r}{0.5\linewidth}
    \vspace{-1.3cm}
    \centering
    \caption{Adversarial training analysis.}
    \vspace{-0.3cm}
    \resizebox{1\linewidth}{!}{
    \begin{tabular}{c|c|c|ccc|c}
    % \specialrule{.1em}{.05em}{.05em} 
    \toprule
    & \textbf{Metrics} & \textbf{\textsc{GNNSafe++}}& \textbf{Gen.\ Once}  & \textbf{Gen.\ Multi} & \textbf{Real OOD} & \textbf{GOLD}\\
    % \multirow{2}{*}{\textbf{Dataset}} & \multirow{2}{*}{\textbf{Metrics}} & \multicolumn{2}{c|}{\textbf{Baselines}} & \multicolumn{2}{c|}{\textbf{Modified Architecture}} & \multicolumn{1}{c}{\textbf{Default}}\\
    % \cmidrule(lr){3-4} \cmidrule(lr){5-6} \cmidrule(lr){7-7}
    % & & \textbf{w/o Adv.} & \textbf{\textsc{GNNSafe++}} & \textbf{Pre-trained Dif.} & \textbf{Real OOD} & \textbf{GOLD}\\
    \midrule
    \multirow{4}{*}{\rotatebox{90}{\fontfamily{qcr}\selectfont Twitch}} 
        & AUROC & 95.36& 84.59  & 84.33 & \underline{97.58} & \textbf{99.46}\\
        & AUPR  & 97.12& 88.69  & 88.38 & \underline{98.50} & \textbf{99.62}\\
        & FPR95  & 33.57& 59.71  & 57.00 & \underline{14.39} & \textbf{1.78}\\
        & ID ACC & 70.18& 70.97  & 71.12 & 70.45 & 68.49\\
    \midrule
    \multirow{4}{*}{\rotatebox{90}{\fontfamily{qcr}\selectfont Cora}} 
        & AUROC & 92.98& 89.64  & 92.83 & \underline{95.59}   & \textbf{95.84} \\
        & AUPR  & 84.93 & 80.22 & 85.09 & \underline{90.05} & \textbf{91.17} \\
        & FPR95 & 38.44& 46.33  & 30.11 & \underline{21.54}  & \textbf{17.83} \\
        & ID ACC & 81.45& 77.60  & 80.56 & 78.42  & 81.66\\
    \midrule    
    \multirow{4}{*}{\rotatebox{90}{\fontfamily{qcr}\selectfont Arxiv}}   
        & AUROC & \underline{74.77} & 69.76   & 72.15  & \textbf{78.90} & 73.90\\
        & AUPR  & \underline{83.21}& 78.93 & 80.57  & \textbf{85.46} &82.52\\
        & FPR95  & \underline{77.43}& 88.16  & 82.02  & \textbf{68.94} & 80.57\\
        & ID ACC & 53.50& 49.89 & 50.77 & 49.99 & 50.59 \\
    \specialrule{.1em}{.05em}{.05em} 
    \end{tabular}} \label{Table:Ablation_adversarial}
    \vspace{-0.3cm}
% \end{table}
\end{wraptable}
\subsection{Adversarial Training Analysis}
To further assess the proposed adversarial training framework, three variants of (pseudo-) OOD exposure are studied: (1) using an ID-pretrained LDM to generate once to train GOLD (Gen.\ Once), which is the same as w/o Adv. in Section~\ref{sec:ablation}; (2) using an ID-pretrained LDM to generate multiple rounds of pseudo-OOD along the GOLD training loops (Gen.\ Multi); (3) using real OOD data instead of pseudo-OOD to train GOLD (Real OOD).
% To facilitate this, we pre-trained the diffusion model on latent features derived from a well-trained GNN based on ID data. This diffusion model will remain frozen during the adversarial training phase. 
The results are detailed in Table \ref{Table:Ablation_adversarial}, which highlights that \textbf{using an ID-pre-trained generative model would not improve OOD detection performance}. This is because without the adversarial training, the detector will be biased by a set of inaccurate and close-to-ID pseudo-OOD data generated by the pre-trained diffusion model. When incorporating real OOD data to substitute the pseudo-OOD in our framework, the Real OOD variant can achieve consistently better performance than Gen.\ Once and Gen.\ Multi. For {\fontfamily{qcr}\selectfont Arxiv}, Real OOD can surpass our default GOLD model with the advantage of OOD exposure in this dataset. Furthermore, a comparison of the results after removing the adversarial process highlights the superiority of the adversarial framework, as all adversarial-based methods outperform their non-adversarial baselines. This robust set of results validates the efficacy of our adversarial training paradigm in enhancing model performance for OOD detection. Despite these modifications, our synthetic-based OOD detection continues to maintain strong performance.

\subsection{Effectiveness of Energy Regulariser}
Extending beyond the previous analysis, we observed that the energy regularisers in GOLD are important factors for OOD detection, especially for the divergence regularisation. We provide a comprehensive assessment of the energy regularisers, $\mathcal{L}_\text{Unc}$ from Eq.~\ref{eq:uncertainty}, $\mathcal{L}_\text{EReg}$ from Eq.~\ref{eq: ereg}, and $\mathcal{L}_\text{DReg}$ from Eq.~\ref{eq:dreg}, across three datasets: {\fontfamily{qcr}\selectfont Twitch}, {\fontfamily{qcr}\selectfont Cora}, and {\fontfamily{qcr}\selectfont Amazon}, reporting the average performance across subsets in Table~\ref{Table:Ablation_regulariser}. \textbf{The default GOLD that incorporates all regularisers, consistently shows superior performance across all datasets, effectively indicating the contribution of the energy regularisers in OOD detection}.  Notably, each dataset exhibits different sensitivities to the absence or presence of specific regularisers. For instance, all datasets are significantly affected by the removal of $\mathcal{L}_\text{DReg}$, highlighting its critical role. There is a substantial performance drop for {\fontfamily{qcr}\selectfont Cora} without $\mathcal{L}_\text{EReg}$. Additionally, individual regulariser performance is context-dependent, with $\mathcal{L}_\text{DReg}$ emerging as particularly impactful, often driving better outcomes when combined with either of the other two regularisers. This is reflected in the best runner-up results, where $\mathcal{L}_\text{DReg}$ is combined with another regulariser, underscoring its influence as the most impactful of the three. Nonetheless, this analysis demonstrates the effectiveness of a holistic approach of combining all proposed regularisers, as shown by GOLD's consistently high performance across all metrics and datasets. The extended performance of each subset is provided in Appendix \ref{Appendix:Additional_exp}.

\begin{table}[!h]
    \centering
    \vspace{-0.1cm}
    \caption{Energy regulariser analysis.}
    \vspace{-0.3cm}
    \resizebox{1\linewidth}{!}{
    \begin{tabular}{ccc|cccc|cccc|cccc}
    % \specialrule{.1em}{.05em}{.05em} 
    \toprule
    \multirow{2}{*}{$\mathcal{L}_\text{Unc}$} & \multirow{2}{*}{$\mathcal{L}_\text{EReg}$} & \multirow{2}{*}{$\mathcal{L}_\text{DReg}$} & \multicolumn{4}{c|}{{\fontfamily{qcr}\selectfont Twitch}} & \multicolumn{4}{c|}{{\fontfamily{qcr}\selectfont Cora}}& \multicolumn{4}{c}{{\fontfamily{qcr}\selectfont Amazon}}\\
    & & & \textbf{AUROC} & \textbf{AUPR} & \textbf{FPR} & \textbf{ID Acc} & \textbf{AUROC} & \textbf{AUPR} & \textbf{FPR} & \textbf{ID Acc} & \textbf{AUROC} & \textbf{AUPR} & \textbf{FPR} & \textbf{ID Acc}\\
    % \hline
    \midrule
     &  &  & 
    86.44 & 80.64 & 79.84 & 68.97 &
    61.14 & 57.82 & 89.70 & 76.23 & 
    64.17 & 72.67 & 46.72 & 92.07 \\
    % \hline
    % \cline{1-3}
    \midrule
    \checkmark & & & 
    10.18 & 40.62 & 97.84 & 70.15 &
    70.76 & 65.12 & 94.37 & 81.05& 
    67.20 & 71.73 & 68.13 & 93.70 \\
   % \hline
    % \cline{1-3}
     & \checkmark& & 
    78.02& 83.37& 78.90& 70.98&
    66.00 & 63.32 & 54.06 & 80.44 & 
    48.65 & 61.93 & 77.91 & 93.45 \\
    % \hline
    % \cline{1-3}
     &  & \checkmark& 
    69.04 & 76.88& 44.54& 70.79&
    84.39 & 74.57 & 68.49 & 76.08 & 
    97.72 & 96.83 & 8.28 & 92.79\\
   % \hline
    % \cline{1-3}
    \midrule
    \checkmark & \checkmark&  &
    76.88 & 81.49 & 76.14 & 70.99 &
    34.63 & 39.27 & 96.84 &81.25 & 
    71.15  & 63.84 & 74.85 & 93.30\\
    % \hline
    % \cline{1-3}
    \checkmark &  & \checkmark& 
    64.43 & 75.46 & 45.95 & 70.90&
    \textcolor{purple}{94.03} & 73.89 & 73.54 & 74.53 & 
    97.91 & 97.03 & 4.54 & 93.20 \\
    % \hline
    % \cline{1-3}
    & \checkmark& \checkmark& 
    \textcolor{purple}{89.58} & \textcolor{purple}{93.12} & \textcolor{purple}{43.78} & 69.64 &
    93.28 & \textcolor{purple}{87.50} & \textcolor{purple}{31.04} & 79.88& 
    \textcolor{purple}{98.02} & \textcolor{purple}{98.42} & \textcolor{purple}{3.40} & 92.81 \\
    \midrule
    % \midrule
    & GOLD & & 
    \textcolor{teal}{99.46} & \textcolor{teal}{99.62} & \textcolor{teal}{1.78} & 68.49 & 
    \textcolor{teal}{95.84} & \textcolor{teal}{91.17} & \textcolor{teal}{17.83} & 81.66& 
    \textcolor{teal}{98.81} & \textcolor{teal}{98.92} & \textcolor{teal}{2.07} & 92.99\\
    \specialrule{.1em}{.05em}{.05em} 
    \end{tabular}}
    \label{Table:Ablation_regulariser}
    \vspace{-0.3cm}
\end{table}

% \begin{wraptable}{r}{0.5\linewidth}
%     \vspace{-1.6cm}
%     \centering
%     \caption{Inference speed against SOTA Baselines.}
%     \vspace{-0.2cm}
%     \resizebox{\linewidth}{!}{
%     \begin{tabular}{lccc}
%     \hline
%     Inference(s)      & \textsc{GNNSafe} & \textsc{GNNSafe++} & GOLD \\
%     \hline
%     {{\fontfamily{qcr}\selectfont Twitch}}       & 0.08    & 0.09      & 0.10 \\
%     {{\fontfamily{qcr}\selectfont Cora-F}}     & 0.03    & 0.03      & 0.04 \\
%     {{\fontfamily{qcr}\selectfont Amazon-F }}    & 0.04    & 0.05      & 0.07 \\
%     {{\fontfamily{qcr}\selectfont Coauthor-F}}   & 0.35    & 0.36      & 0.37 \\
%     {{\fontfamily{qcr}\selectfont Arxiv}}        & 0.40    & 0.40      & 0.47 \\
%     \hline
%     \label{Table:Inference_speed}
%     \end{tabular}
%     } 
%     \vspace{-0.8cm}
% \end{wraptable}

\begin{wraptable}{r}{0.6\linewidth}
    \vspace{-1.2cm}
    \centering
    \caption{Inference and training time (s) of GOLD.}
    \vspace{-0.2cm}
    \resizebox{\linewidth}{!}{
    \begin{tabular}{l|cc|cc|cc|cc}
    \toprule
       & \multicolumn{2}{c|}{\textsc{GNNSafe}} & \multicolumn{2}{c|}{\textsc{GNNSafe++}} & \multicolumn{2}{c|}{GOLD w/ VAE} & \multicolumn{2}{c}{GOLD w/ LDM}\\
        & Inf. & Train.  & Inf. & Train.  & Inf. & Train. & Inf. & Train.\\
    \midrule
    \texttt{Twitch}  & \textbf{\textcolor{ForestGreen}{0.08}}  & 2.41  & \textbf{\textcolor{ForestGreen}{0.09}}  & 4.74    & \textbf{\textcolor{ForestGreen}{0.09}}  & 2.78  & \textbf{\textcolor{ForestGreen}{0.10}}  & 8.96   \\
    \texttt{Cora-F}  & \textbf{\textcolor{ForestGreen}{0.03}}  & 4.40  & \textbf{\textcolor{ForestGreen}{0.03}}  & 5.32  & \textbf{\textcolor{ForestGreen}{0.04}}  & 3.91  & \textbf{\textcolor{ForestGreen}{0.04}}  & 5.93   \\
    \texttt{Amazon-F}     & \textbf{\textcolor{ForestGreen}{0.04}}  & 13.51     & \textbf{\textcolor{ForestGreen}{0.05}}  & 18.40   & \textbf{\textcolor{ForestGreen}{0.05}}  & 12.52 & \textbf{\textcolor{ForestGreen}{0.07}}  & 39.04    \\
    \texttt{Coauthor-F}   & \textbf{\textcolor{ForestGreen}{0.35}}  & 57.80    & \textbf{\textcolor{ForestGreen}{0.36}}  & 67.83   & \textbf{\textcolor{ForestGreen}{0.35}}  & 55.65 & \textbf{\textcolor{ForestGreen}{0.37}}  & 89.74 \\
    \texttt{Arxiv}        & \textbf{\textcolor{ForestGreen}{0.40}}  & 85.23  & \textbf{\textcolor{ForestGreen}{0.40 }} & 132.36  & \textbf{\textcolor{ForestGreen}{0.45}}  & 80.77 & \textbf{\textcolor{ForestGreen}{0.47}}  & 244.95  \\
    \bottomrule
    \end{tabular}
    }
    \label{Table:Inference_speed}
    \vspace{-0.5cm}
\end{wraptable}


\subsection{Computational Cost} \label{sec:computational_cost}
% \textcolor{red}{May need to rewrite this .}
Table~\ref{Table:Inference_speed} shows that \textbf{GOLD generally achieves a very close inference time, and a faster (w/ VAE) or comparable (w/LDM) training time relative to the GNNSafe(++) baseline}. This is under the situation that the non-OOD exposed GOLD outperforms the existing non-OOD exposure methods, while matching or surpassing the real-OOD exposed SOTA baselines, all under the same backbone. In addition to the high-performing LDM variant, a lightweight VAE is also experimented, providing an efficient alternative with comparable performance. Thus, we consider this training cost as an acceptable trade-off for improved OOD detection performance, and is discussed in Appendix~\ref{Appendix:Limitation}. However, we highlight that GOLD can achieve a similar inference time as the baselines, regardless of the LGM, as shown in Table~\ref{Table:Inference_speed}. This reveals a competitive application of GOLD while having a strong performance. We provide detailed results in Appendix~\ref{Appendix:computational_cost}.

% \subsection{Effectiveness of Latent Generative model}
% \textcolor{red}{To demonstrate the effectiveness of GOLD, additional experiments were conducted by substituting the latent diffusion with a much more lightweight latent generator, Variational Auto-Encoder, in Table \ref{Table:overall_performance}%\ref{Appendix:latent_generative_model}
% ~\citep{VAE}. The OOD detection performance is very close while being more lightweight due to the model design. This demonstrates that our adversarial training pipeline can use different latent generators to achieve the goal of pseudo-OOD generation for OOD detection.}

\section{Related Work}
Our work intersects with three major research areas: \textbf{1) Non-OOD-Exposure OOD Detection} that purely relies on ID data for detecting OOD instances, this involves score-based methods, feature learning, and techniques specific for graph-structured data~\citep{ConfOOD, Hendrycks17softmax, MSP, GenOE, energy, SGOOD, grasp, GKDE, GOOD-D, GraphDE, GNNSafe, NODESAFE}; \textbf{2) OOD Exposure-Based OOD Detection}, a prominent line of work that adopts auxiliary OOD data to assist training, often achieving higher performance than non-OOD-exposure based methods~\citep{OE, energy, Textual-OODExposure, DivOE, ATOL, SAL,GNNSafe, GDE_OOD}; and \textbf{3) OOD Generation,} a more recent field that aims to synthesise OOD-like data to assist OOD detection~\citep{manifold, Likelihood, LRegret, GenAnalysis, GenUnknown, Hierarc,ConfOOD,VOS,NPOS}. Notably for graph data, \textsc{GNNSafe} considers the inter-dependence nature of node instances and proposes an energy propagation schema, and explores an OOD-exposed variant \textsc{GNNSafe++}~\citep{GNNSafe}. \textsc{NODESafe/++} builds upon \textsc{GNNSafe/++} and proposes additional regularisation terms to reduce and bound the generation of extreme energy scores~\citep{NODESAFE}. \cite{GDE_OOD} proposes a generalised Dirichlet energy score for graph OOD detection. A detailed review of related work is provided in Appendix~\ref{Appendix:related_work}.

% \paragraph{Non-OOD-Exposure OOD Detection.} OOD detection is a fundamental task extensively studied in diverse machine learning domains~\citep{ConfOOD, EBFeat, RPRW, regOOD, MSP, Mahalanobis, GenOE, OECC, FocalOE}. A representative line of work that relies on purely ID data is based on the model's output including using softmax score~\citep{Hendrycks17softmax, ODIN}, using energy score~\citep{energy, Wang21energy, NODESAFE}, and activation pruning-based methods~\citep{ASH,DICE, React}. Other approaches involve confidence enhancement~\citep{Gen-ODIN, WhyReLU, Ensemble}, feature learning~\citep{MOOD, NMD}, and adversarial strategies~\citep{GOOD-cert, ATOM, RND}. More recent studies have applied OOD detection to graph-structured data~\citep{SGOOD, grasp, GOODAT, LMN, UGTs, G_UQ, EMP, advOOD, SLW,GOOD,GOODD-uncertainty}. For node-level detection, \textsc{GNNSafe} considers the inter-dependence nature of node instances and proposes an energy propagation schema~\citep{GNNSafe}. \textsc{NODESafe} builds upon \textsc{GNNSafe} and proposes additional regularisation terms to reduce and bound the generation of extreme energy scores~\citep{NODESAFE}. GKDE proposes a multi-source uncertainty framework to estimate the node-level Dirichlet distributions to assist OOD detection~\citep{GKDE}. GPN applies Bayesian posterior and density estimation to estimate the uncertainty for each node~\citep{GPN}. For graph-level detection, recent methods includes modelling distribution shifts through a graph generative process, overseeing from a data-centric perspective, and unsupervised methods~\citep{GraphDE, AAGOD, GOOD-D}. %GraphDE proposes to model the distribution shifts through a graph generative process and derives a posterior distribution for graph-level OOD detection~\citep{GraphDE}. Additionally, AAGOD presents a data-centric method for detecting graph-level OOD~\citep{AAGOD}.
% % which aims to learn structural patterns in the graph data through a learnable amplifier matrix
% %GOOD-D is an unsupervised method that considers purely the performance of graph-level OOD detection by applying contrastive learning~\citep{GOOD-D}. Other uncertainty estimation-based approaches have also been proposed for OOD detection~\citep{GOODD-uncertainty, GPN, GKDE}.

% \paragraph{OOD Exposure-Based OOD Detection.}
% OOD exposure is another prominent line of work that adopts auxiliary OOD data to assist training~\citep{OE, energy, Textual-OODExposure, DivOE, ATOL, SAL,GNNSafe}. The aforementioned \textsc{GNNSafe} model also considers an additional version \textsc{GNNSafe++} to adopt OOD exposure and has shown greater performance than standard model~\citep{GNNSafe}. \cite{NODESAFE} also presents \textsc{NODESafe++} as an extended OOD exposed version. \cite{GDE_OOD} proposes a generalised Dirichlet energy score for OOD detection. Our proposed GOLD method attempts to take advantage of the effectiveness of OOD exposure through synthesising samples that exhibit OOD characteristics. Thus, avoiding the necessity of real OOD data during training.

% \paragraph{OOD Generation.}
% Recent studies begin to work on synthesising OOD data~\citep{manifold, Likelihood, LRegret, GenAnalysis, GenUnknown, Hierarc,ConfOOD,VOS,NPOS}. A GAN-based approach is proposes to generate OOD data by jointly training a confidence classifier~\citep{ConfOOD}. VOS generates synthetic outliers from low-probability regions of multivariate Gaussian distributions~\citep{VOS}. Recently,pre-trained  diffusion models have been widely employed for OOD generation including DFDD~\citep{DFDD}, Dream-OOD~\citep{Dream-OOD}. Several initial graph-level OOD studies have been initiated, predominantly for molecule~\citep{MOOD-Molecule,MOL_DIF}. A score-based OOD molecule generation model is proposed by MOOD~\citep{MOOD-Molecule}, which employs an OOD-controlled reverse-time diffusion. A recent work PGR-MOOD~\citep{MOL_DIF} proposes to rely on a pre-trained molecule diffusion for generation. These methods typically rely on pre-trained models that are trained with additional data. In contrast, GOLD does not rely on pre-trained generative models to synthesise pseudo-OOD data.

\section{Conclusion} \label{sec:conclusion}
% \textcolor{red}{MAY NEED REWRITE with new logic.}
In this paper, we propose GOLD, a novel graph OOD detection framework with a latent generative model trained in a novel implicit adversarial paradigm. Unlike methods that rely on pre-trained generative models or real OOD data requiring auxiliary data inputs, GOLD synthesises pseudo-OOD data to inherit OOD characteristics through the implicit adversarial framework, solely based on ID data. An effective OOD detector head is further designed to address the difficulties with multiple classes in the logit space, optimising the energy score for improved detection. 
Extensive experiments show the efficacy of GOLD, outperforming SOTA non- and OOD-exposed methods.
% While it exhibits slightly higher computational costs and less improvement with weaker base classifiers, GOLD demonstrates significant success and potential in graph OOD detection across various scenarios.
% Our proposed method demonstrates significant success but also has limitations, including the computational cost of training the generative model and a less significant model performance under a less competitive base classifier. However, the great potential of the proposed GOLD framework in graph OOD detection is still evidenced by the improvement over different scenarios. 
We hope this work inspires future synthetic-based graph OOD detection research for real-world applications.

\newpage
\section{Acknowledgement}
This research has been partially supported by Australian Research Council Discovery Projects (DP230101196, DP24010306, DE250100919 and CE200100025).

\section{Reproducibility Statement}
To support reproducible research, we summarise our efforts as below:
\begin{enumerate}
    \item \textbf{Baselines \& Datasets.} We follow the baseline from~\citep{GNNSafe} and utilise publicly available datasets. The details are described in Section \ref{sec:Experiments} and Appendix \ref{Appendix:dataset}.
    \item \textbf{Model training.} Our implementation of the energy-based OOD detector builds upon the open-sourced work \textsc{GNNSafe} by \cite{GNNSafe}, \url{https://github.com/qitianwu/GraphOOD-GNNSafe}. Detailed implementation setting is provided in Section \ref{sec:Experiments} and Appendix \ref{Appendix:implementation_details}.
    \item \textbf{Methodology.} Our GOLD framework is fully documented in Section \ref{sec:GOLD_Method}. In addition, we provide a detailed pseudo code in Algorithm \ref{alg:GOLD}.
    \item \textbf{Evaluation Metrics.} We discuss the evaluation metrics used in Section \ref{sec:Experiments} and Appendix \ref{Appendix:Metrics}.
    % \item \textbf{Open Source.} The code and dataset will be released upon acceptance.
\end{enumerate}

\newpage
% \bibliographystyle{abbrv}
\bibliography{iclr2025_conference}
\bibliographystyle{iclr2025_conference}


\newpage
\appendix
\section{Appendix}

In the Appendix, we provide additional supplementary material to the main paper. The structure is as follows:
\begin{itemize}
    \item We provide the proof for Proposition 1. in \ref{Appendix:Proof}.
    \item An extended related work is detailed in \ref{Appendix:related_work}.
    \item Potential Limitations is discussed in \ref{Appendix:Limitation}.
    \item Preliminary GNN description is described in \ref{Appendix:GNN}.
    \item We provide the description of datasets in \ref{Appendix:dataset}.
    \item The evaluation metrics and implementation details are provided in \ref{Appendix:Metrics} and \ref{Appendix:implementation_details}.
    \item Additional experiment results, including extended subset performance, ablation study visualisations, empirical evaluations of logits vs. softmax scores, and computational cost are detailed in \ref{Appendix:Additional_exp}, 
    \ref{Appendix:ablation_vis}
    \ref{Appendix:Logits vs. SFM}, \ref{Appendix:computational_cost}.
    \item Descriptions of the latent generative models: 1) Latent diffusion model, and 2) Variational autoencoder are provided in \ref{Appendix:latent_generative_model}.
    
\end{itemize}

\subsection{Proof for Proposition 1.} \label{Appendix:Proof}
\begin{proof}
Let $l_{\theta_{[y]}}$ denote the logits of the MLP detector with parameter $\theta$ for class y, $\phi$ denote the softmax function. Assume the hyper-parameters $\lambda = \gamma = 1$.

Note that:
\begin{equation}
\begin{split}
\frac{\partial \log (e^a_\theta + e^b_\theta)}{\partial \theta} = \frac{e^a_\theta \frac{\partial a_\theta}{\partial \theta} + e^b_\theta \frac{\partial b_\theta}{\partial \theta}}{e^a_\theta + e^b_\theta}\\
\end{split}
\end{equation}

The gradient of $\min_{\text{MLP}} -(\mathcal{L}_{\text{Unc}} + \mathcal{L}_{\text{DReg}})$ w.r.t $\theta$ is given by:  
\begin{equation}
\begin{split}
- \frac{\partial \mathcal{L}_{\text{Unc}}}{\partial \theta} & = - \mathbb{E}_{i\sim P_\text{ID}} \frac{\partial \log[\phi(l_\theta(e_i))_{[0]}]}{\partial \theta} - \mathbb{E}_{j\sim P_\text{p-OOD}} \frac{\partial \log[\phi(l_\theta(e_j))_{[1]}]}{\partial \theta} \\
& = - \mathbb{E}_{i\sim P_\text{ID}} \frac{\partial \log[\frac{e^{l_\theta(e_i)_{[0]}}}{e^{l_\theta(e_i)_{[0]}} + e^{l_\theta(e_i)_{[1]}}}]}{\partial \theta} - \mathbb{E}_{j\sim P_\text{p-OOD}} \frac{\partial \log[\frac{e^{l_\theta(e_j)_{[1]}}}{e^{l_\theta(e_j)_{[0]}} + e^{l_\theta(e_j)_{[1]}}}]}{\partial \theta} \\
& = - \mathbb{E}_{i\sim P_\text{ID}} \frac{\partial [l_\theta(e_i)_{[0]} -\log (e^{l_\theta(e_i)_{[0]}} + e^{l_\theta(e_i)_{[1]}})]}{\partial \theta} \\
& - \mathbb{E}_{j\sim P_\text{p-OOD}} \frac{[\partial l_\theta(e_j)_{[1]} - \log (e^{l_\theta(e_j)_{[0]}} + e^{l_\theta(e_j)_{[1]}})]}{\partial \theta} \\
& = \mathbb{E}_{i\sim P_\text{ID}} \left[ -\frac{\partial l_\theta(e_i)_{[0]}}{\partial \theta} + \frac{e^{\l_\theta(e_i)_{[0]}} \frac{\partial \l_\theta(e_i)_{[0]}}{\partial \theta} + e^{\l_\theta(e_i)_{[1]}} \frac{\partial \l_\theta(e_i)_{[1]}}{\partial \theta}}{e^{\l_\theta(e_i)_{[0]}} + e^{\l_\theta(e_i)_{[1]}}} \right]\\
& +\mathbb{E}_{j\sim P_\text{p-OOD}} \left[- \frac{\partial l_\theta(e_j)_{[1]}}{\partial \theta} +\frac{e^{\l_\theta(e_j)_{[0]}} \frac{\partial \l_\theta(e_j)_{[0]}}{\partial \theta} + e^{\l_\theta(e_j)_{[1]}} \frac{\partial \l_\theta(e_j)_{[1]}}{\partial \theta}}{e^{\l_\theta(e_j)_{[0]}} + e^{\l_\theta(e_j)_{[1]}}} \right]
\end{split}
\end{equation}

Notice that $\max(0, e_i - e'_i)$ and $\max(0, e'_j - e_j)$ are positive and monotonic, the optimised $\theta$ that minimises the functions (arg min) would also minimise $\max(0, e_i - e'_i)^2$ and $\max(0, e'_j - e_j)^2$, thus, we consider the gradient of a surrogate function of $\mathcal{L}_{\text{DReg}}$ as $\mathcal{L}_{\text{DReg}_S}$:

\begin{equation}
\begin{split}
- \frac{\partial \mathcal{L}_{\text{DReg}_S}}{\partial \theta} & = -\frac{\partial}{\partial \theta} \mathbb{E}_{i\sim P_\text{ID}}\operatorname{max}\left( 0, e_i - e'_i\right) - \frac{\partial}{\partial \theta} \mathbb{E}_{j\sim P_\text{p-OOD}}\operatorname{max}\left(0, e'_j -e_j\right) \\
& \text{If $e_i - e'_i \le 0$ or $e'_j - e_j \le 0$ the gradient is 0, else:}\\
& = -\frac{\partial}{\partial \theta} \mathbb{E}_{i\sim P_\text{ID}} \left[e_i + \log(e^{\l_\theta(e_i)_{[0]}} + e^{\l_\theta(e_i)_{[1]}}) \right] \\
& - \frac{\partial}{\partial \theta} \mathbb{E}_{j\sim P_\text{p-OOD}} \left[- \log(e^{\l_\theta(e_j)_{[0]}}  + e^{\l_\theta(e_j)_{[1]}}) -e_j \right]\\
& = -\mathbb{E}_{i\sim P_\text{ID}} \frac{\partial \log(e^{\l_\theta(e_i)_{[0]}} + e^{\l_\theta(e_i)_{[1]}})}{\partial \theta} \\
& - \mathbb{E}_{j\sim P_\text{p-OOD}}  \frac{\partial - \log(e^{\l_\theta(e_j)_{[0]}}  + e^{\l_\theta(e_j)_{[1]}})}{\partial \theta} \\
& = -\mathbb{E}_{i\sim P_\text{ID}} \frac{e^{\l_\theta(e_i)_{[0]}} \frac{\partial \l_\theta(e_i)_{[0]}}{\partial \theta} + e^{\l_\theta(e_i)_{[1]}} \frac{\partial \l_\theta(e_i)_{[1]}}{\partial \theta}}{e^{\l_\theta(e_i)_{[0]}} + e^{\l_\theta(e_i)_{[1]}}}\\
& + \mathbb{E}_{j\sim P_\text{p-OOD}}  \frac{e^{\l_\theta(e_j)_{[0]}} \frac{\partial \l_\theta(e_j)_{[0]}}{\partial \theta} + e^{\l_\theta(e_j)_{[1]}} \frac{\partial \l_\theta(e_j)_{[1]}}{\partial \theta}}{e^{\l_\theta(e_j)_{[0]}} + e^{\l_\theta(e_j)_{[1]}}}
\end{split}
\end{equation}

\begin{equation}
\begin{split}
- (\frac{\partial \mathcal{L}_{\text{Unc}}}{\partial \theta} + \frac{\partial \mathcal{L}_{\text{DReg}_S}}{\partial \theta}) & = \mathbb{E}_{i\sim P_\text{ID}} \left[-\frac{\partial l_\theta(e_i)_{[0]}}{\partial \theta} + \frac{e^{\l_\theta(e_i)_{[0]}} \frac{\partial \l_\theta(e_i)_{[0]}}{\partial \theta} + e^{\l_\theta(e_i)_{[1]}} \frac{\partial \l_\theta(e_i)_{[1]}}{\partial \theta}}{e^{\l_\theta(e_i)_{[0]}} + e^{\l_\theta(e_i)_{[1]}}} \right]\\
& +\mathbb{E}_{j\sim P_\text{p-OOD}} \left[ - \frac{\partial l_\theta(e_j)_{[1]}}{\partial \theta} +\frac{e^{\l_\theta(e_j)_{[0]}} \frac{\partial \l_\theta(e_j)_{[0]}}{\partial \theta} + e^{\l_\theta(e_j)_{[1]}} \frac{\partial \l_\theta(e_j)_{[1]}}{\partial \theta}}{e^{\l_\theta(e_j)_{[0]}} + e^{\l_\theta(e_j)_{[1]}}} \right] \\
&-\mathbb{E}_{i\sim P_\text{ID}} \frac{e^{\l_\theta(e_i)_{[0]}} \frac{\partial \l_\theta(e_i)_{[0]}}{\partial \theta} + e^{\l_\theta(e_i)_{[1]}} \frac{\partial \l_\theta(e_i)_{[1]}}{\partial \theta}}{e^{\l_\theta(e_i)_{[0]}} + e^{\l_\theta(e_i)_{[1]}}}\\
& + \mathbb{E}_{j\sim P_\text{p-OOD}}  \frac{e^{\l_\theta(e_j)_{[0]}} \frac{\partial \l_\theta(e_j)_{[0]}}{\partial \theta} + e^{\l_\theta(e_j)_{[1]}} \frac{\partial \l_\theta(e_j)_{[1]}}{\partial \theta}}{e^{\l_\theta(e_j)_{[0]}} + e^{\l_\theta(e_j)_{[1]}}}\\
& \text{Define the energy w.r.t. label y as $E(e, y) = - l_\theta(e)_{[y]}$}\\
& = \mathbb{E}_{i\sim P_\text{ID}} \frac{\partial E(e_i, 0)}{\partial \theta} +\mathbb{E}_{j\sim P_\text{p-OOD}} \frac{\partial E(e_j, 1)}{\partial \theta} \\
& - 2 \left(\phi(\l_\theta(e_j))_{[0]} \frac{\partial E(e_j, 0)}{\partial \theta} + \phi(\l_\theta(e_j))_{[1]} \frac{\partial E(e_j, 1)}{\partial \theta}\right)\\
& = \mathbb{E}_{i\sim P_\text{ID}} \frac{\partial E(e_i, 0)}{\partial \theta} +\mathbb{E}_{j\sim P_\text{p-OOD}} (1-2\phi(\l_\theta(e_j))_{[1]}) \frac{\partial E(e_j, 1)}{\partial \theta} \\
& - 2 \phi(\l_\theta(e_j))_{[0]} \frac{\partial E(e_j, 0)}{\partial \theta}
\end{split}
\end{equation}

From the above equation, the training procedure that overall minimises the first order gradient of the negative sum of $\mathcal{L}_{\text{Unc}}$ and the surrogate function of $\mathcal{L}_{\text{DReg}}$ will decrease the energy score $E(e_i, 0; l_\theta)$ for in-distribution data, and increase the energy score $E(e_j, 1; l_\theta)$ and $E(e_j, 0; l_\theta)$ for pseudo-OOD data, given $\phi(\l_\theta(e_j))_{[1]} > 0.5$ as the detector continues to improve detection performance.
\end{proof}

\subsection{Logits vs. Softmax discrepancy} \label{Appendix:Logits vs. SFM}
In this section, we present an empirical evaluation of the Logits vs. Softmax discrepancy between ID and OOD data from the detector. It is evident from Figure~\ref{fig:Logits_smf}, while the softmax confidence scores present high confidence for ID and OOD instances, where the majority of the scores corresponding to the respective class were close to 1 (based on the marginal distribution), the energy scores provide more meaningful information for distinguishing between them. Notably, ID data typically possess higher and positive ID Logits and lower OOD logits than OOD data. Thereby, leading to more distinguishable energy scores than softmax for OOD detection. 

\begin{figure}[h!]
    \centering
    \begin{subfigure}{0.45\linewidth}
        \includegraphics[width=\linewidth]{fig/Twtich_JointPlot_Logits_v3.png}
        \caption{Logits Joint distribution}
        \label{fig:figure1}
    \end{subfigure}
    \hspace{0.05\linewidth} % Adjust the space between the figures
    \begin{subfigure}{0.45\linewidth}
        \includegraphics[width=\linewidth]{fig/Twtich_JointPlot_softmax_v4.png} % Replace with your second figure
        \caption{Softmax Joint distribution}
        \label{fig:figure2}
    \end{subfigure}
    \caption{Logits vs. Softmax joint distribution plot for Twitch dataset.}
    \label{fig:Logits_smf}
\end{figure}

\subsection{Extended Related Work} \label{Appendix:related_work}
\paragraph{Non-OOD-Exposure OOD Detection.} OOD detection is a fundamental task extensively studied in diverse machine learning domains~\citep{ConfOOD, EBFeat, RPRW, regOOD, MSP, Mahalanobis, GenOE, OECC, FocalOE}. A representative line of work that relies on purely ID data is based on the model's output including using softmax score~\citep{Hendrycks17softmax, ODIN}, using energy score~\citep{energy, Wang21energy, NODESAFE}, and activation pruning-based methods~\citep{ASH,DICE, React}. Other approaches involve confidence enhancement~\citep{Gen-ODIN, WhyReLU, Ensemble}, feature learning~\citep{MOOD, NMD}, and adversarial strategies~\citep{GOOD-cert, ATOM, RND}. More recent studies have applied OOD detection to graph-structured data~\citep{SGOOD, grasp, GOODAT, LMN, UGTs, G_UQ, EMP, advOOD, SLW,GOOD,GOODD-uncertainty}. For node-level detection, \textsc{GNNSafe} considers the inter-dependence nature of node instances and proposes an energy propagation schema~\citep{GNNSafe}. \textsc{NODESafe} builds upon \textsc{GNNSafe} and proposes additional regularisation terms to reduce and bound the generation of extreme energy scores~\citep{NODESAFE}. GKDE proposes a multi-source uncertainty framework to estimate the node-level Dirichlet distributions to assist OOD detection~\citep{GKDE}. GPN applies Bayesian posterior and density estimation to estimate the uncertainty for each node~\citep{GPN}. For graph-level detection, recent methods include modelling distribution shifts through a graph generative process, overseeing from a data-centric perspective, and unsupervised methods~\citep{GraphDE, AAGOD, GOOD-D}. %GraphDE proposes to model the distribution shifts through a graph generative process and derives a posterior distribution for graph-level OOD detection~\citep{GraphDE}. Additionally, AAGOD presents a data-centric method for detecting graph-level OOD~\citep{AAGOD}.
% which aims to learn structural patterns in the graph data through a learnable amplifier matrix
%GOOD-D is an unsupervised method that considers purely the performance of graph-level OOD detection by applying contrastive learning~\citep{GOOD-D}. Other uncertainty estimation-based approaches have also been proposed for OOD detection~\citep{GOODD-uncertainty, GPN, GKDE}.

\paragraph{OOD Exposure-Based OOD Detection.}
OOD exposure is another prominent line of work that adopts auxiliary OOD data to assist training~\citep{OE, energy, Textual-OODExposure, DivOE, ATOL, SAL,GNNSafe}. The aforementioned \textsc{GNNSafe} model also considers an additional version \textsc{GNNSafe++} to adopt OOD exposure and has shown greater performance than the standard model~\citep{GNNSafe}. \cite{NODESAFE} also presents \textsc{NODESafe++} as an extended OOD exposed version. \cite{GDE_OOD} proposes a generalised Dirichlet energy score for OOD detection. Our proposed GOLD method attempts to take advantage of the effectiveness of OOD exposure by synthesising samples that exhibit OOD characteristics. Thus, avoiding the necessity of real OOD data during training.

\paragraph{OOD Generation.}
Recent studies begin to work on synthesising OOD data~\citep{manifold, Likelihood, LRegret, GenAnalysis, GenUnknown, Hierarc,ConfOOD,VOS,NPOS}. A GAN-based approach is proposed to generate OOD data by jointly training a confidence classifier~\citep{ConfOOD}. VOS generates synthetic outliers from low-probability regions of multivariate Gaussian distributions~\citep{VOS}. Recently,pre-trained diffusion models have been widely employed for OOD generation including DFDD~\citep{DFDD}, Dream-OOD~\citep{Dream-OOD}. Several initial graph-level OOD studies have been initiated, predominantly for molecule~\citep{MOOD-Molecule,MOL_DIF}. A score-based OOD molecule generation model is proposed by MOOD~\citep{MOOD-Molecule}, which employs an OOD-controlled reverse-time diffusion. A recent work PGR-MOOD~\citep{MOL_DIF} proposes to rely on a pre-trained molecule diffusion for generation. These methods typically rely on pre-trained models that are trained with additional data. In contrast, GOLD does not rely on pre-trained generative models to synthesise pseudo-OOD data.

\subsection{Potential Limitations} \label{Appendix:Limitation}
In our concluding remarks, we highlight that our methodology leverages a generative model (specifically, diffusion model) to generate effective pseudo-OOD instances for OOD detection. To curb computational expenses, we employ a latent diffusion model, which reduces the computational demands of direct input space manipulation. Despite this, training-time efficiency may still be impacted. Nonetheless, during the inference phase, our model does not necessitate the generation of extra data, thus mitigating the impact of high latency. Moreover, we have experimented with a lightweight VAE as the latent generative model, which can achieve a competitive computational time as the standard SOTA baselines. Additionally, our approach currently targets node-level prediction tasks; however, we envisage its applicability to graph-level OOD detection, which we leave for future research. Following the propositions of \cite{energy} and \cite{GNNSafe}, our framework incorporates an energy-bounded regulariser that ideally ensures ID scores are lower than those of OOD samples, as illustrated in our visualisations in Section \ref{sec:vis}. Extended experiments detailed in Table~\ref{Appendix:Cora_reg_effectiveness} reveal that using only the energy regulariser results in AUROC scores near the single-digit range. This outcome highlights the regulariser's limitations and challenges the assumption that OOD energy scores consistently exceed ID scores, thereby undermining the effectiveness of the OOD metric in true detection performance. Nevertheless, our framework introduces an additional regulariser, which effectively addresses these discrepancies, as showcased by our consistently positive results.

\subsection{GNN} \label{Appendix:GNN}
GNNs, by their very nature, excel in modelling the complex relationship of node-dependence in graphs. Central to their success is the message-passing mechanism, which iteratively aggregates neighbouring information towards the centre node to capture both local and global knowledge. Denote the learnt representation of node $i$ at the $l$-th layer as $\mathbf{h}_i^{(l)}$, a typical Graph Convolutional Network (GCN) executes recursive layer propagation via:
\begin{equation}
\mathbf{H}^{(l)} = \sigma ( \mathbf{D}^{-1/2} \mathbf{\Tilde{A}}\mathbf{D}^{-1/2} \mathbf{H}^{(l - 1)} \mathbf{W}^{(l)}), \mathbf{H}^{l-1} = [{\mathbf{h}}_i^{l-1}], \mathbf{H}^{(0)} = {\mathbf{X}} \label{eq:GCN} 
\end{equation}
with $\mathbf{\Tilde{A}} = \mathbf{A} + \mathbf{I}$, where $\mathbf{I}$ is the identity matrix, $\mathbf{D}$ is the diagonal degree matrix of $\mathbf{\Tilde{A}} $, $\sigma$ is a non-linear activation function (i.e., ReLU), and $\mathbf{W}^{(l)}$ is the corresponding weight matrix at layer $l$~\citep{GCN}.

% \section{Baselines}

\subsection{Description of Datasets} \label{Appendix:dataset}
The datasets utilised in this study are publicly available benchmark datasets for graph learning. We follow the same data collection and processing protocol in \cite{GNNSafe} and utilised the data loader for the {\fontfamily{qcr}\selectfont ogbn-Arxiv} dataset provided by the OGB package\footnote{https://github.com/snap-stanford/ogb?tab=readme-ov-file}, and others from the Pytorch Geometric Package\footnote{https://pytorch-geometric.readthedocs.io/en/latest/modules/datasets.html}. For all datasets, we follow the provided splits and generation process in \cite{GNNSafe}. We provide a brief description of the datasets below:

The {\fontfamily{qcr}\selectfont TwitchGamers - Explicit} dataset consists of multiple subgraphs, each representing a social network from a different region~\citep{Twitch_Gamers}. The nodes within these subgraphs indicate Twitch gamers, while the edges depict the follower relationships between two users. Node features include embeddings based on the games played by Twitch users, and for this study, we focus on the label that indicates whether a user broadcasts mature content (i.e., Explicit). We utilise subgraph DE as ID data, and subgraphs ES, FR, RU as testing data. Dataset details are provided in Table \ref{appendix:twitch_table}.

\begin{table} [H]
    \centering
    \resizebox{0.8\linewidth}{!}{\begin{tabular}{c|c|c|c|c|c}
        \toprule
        {\fontfamily{qcr}\selectfont Twitch} & Splits & $\#$ Nodes & $\#$ Edges & Feature Dimension & $\#$ Classes\\
        \midrule
        {\fontfamily{qcr}\selectfont Twitch-DE} & ID& 9498 & 315774 & 128 & 2\\
        {\fontfamily{qcr}\selectfont Twitch-ES} & OOD&4648 & 123412 & 128 & 2\\
       {\fontfamily{qcr}\selectfont Twitch-FR} & OOD&6551 & 231883 & 128 & 2 \\
        {\fontfamily{qcr}\selectfont Twitch-RU}& OOD&4385 & 78993 & 128 & 2 \\
        \bottomrule
    \end{tabular}}
    \caption{{\fontfamily{qcr}\selectfont Twitch} dataset overview}
    \label{appendix:twitch_table}
\end{table}

The {\fontfamily{qcr}\selectfont Cora} dataset is a citation network where each node represents a published paper, and each edge reflects a citation relationship between papers~\citep{Cora}. The dataset consists of seven labels. Since {\fontfamily{qcr}\selectfont Cora} does not contain an explicit domain attribute to partition into OOD subgraphs, we follow the provided protocol in \cite{GNNSafe}, and synthetically create the OOD data as mentioned in Section \ref{sec:datasets}. Dataset details are provided in Table \ref{appendix:cora_table}.

\begin{table} [H]
    \centering
    \resizebox{0.8\linewidth}{!}{\begin{tabular}{c|c|c|c|c|c}
        \toprule
        {\fontfamily{qcr}\selectfont Cora} & Splits&$\#$ Nodes & $\#$ Edges & Feature Dimension & $\#$ Classes\\
        \midrule
        {\fontfamily{qcr}\selectfont Cora-S} &ID & 2708  & 10556  & 1433 & 7\\
        {\fontfamily{qcr}\selectfont Cora-S} &OOD & 2708  & 6696 & 1433 & 7\\
        \midrule
        {\fontfamily{qcr}\selectfont Cora-F} &ID & 2708  & 10556  & 1433 & 7\\
        {\fontfamily{qcr}\selectfont Cora-F} &OOD & 2708  & 10556 & 1433 & 7\\
        \midrule
        {\fontfamily{qcr}\selectfont Cora-L} &ID & 904   & 10556  & 1433 & 3\\
        {\fontfamily{qcr}\selectfont Cora-L} &OOD & 986   & 10556 & 1433 & 3\\
        
        \bottomrule
    \end{tabular}}
    \caption{{\fontfamily{qcr}\selectfont Cora} dataset overview}
    \label{appendix:cora_table}
\end{table}

The {\fontfamily{qcr}\selectfont Amazon-Photo} dataset forms an item co-purchasing network on Amazon, where each node represents a product and each edge signifies that the linked products are frequently bought together~\citep{AmazonPhoto}. Node labels categorise the products. Similar to the {\fontfamily{qcr}\selectfont Cora} dataset, we employ three synthetic methods to create the OOD data due to the lack of a clear domain for partition. Dataset details are provided in Table \ref{appendix:amazon_table}.

\begin{table} [H]
    \centering
    \resizebox{0.8\linewidth}{!}{\begin{tabular}{c|c|c|c|c|c}
        \toprule
        {\fontfamily{qcr}\selectfont Amazon-Photo} & Splits&$\#$ Nodes & $\#$ Edges & Feature Dimension & $\#$ Classes\\
        \midrule
        {\fontfamily{qcr}\selectfont Amazon-S} &ID & 7650    & 238162  & 745 & 8\\
        {\fontfamily{qcr}\selectfont Amazon-S} &OOD & 7650   & 149168 & 745 & 8\\
        \midrule
        {\fontfamily{qcr}\selectfont Amazon-F} &ID & 7650   & 238162  & 745 & 8\\
        {\fontfamily{qcr}\selectfont Amazon-F} &OOD & 7650   & 238162 & 745 & 8\\
        \midrule
        {\fontfamily{qcr}\selectfont Amazon-L} &ID & 3095    & 238162  & 745 & 3\\
        {\fontfamily{qcr}\selectfont Amazon-L} &OOD & 3673     & 238162 & 745 & 4\\
        
        \bottomrule
    \end{tabular}}
    \caption{{\fontfamily{qcr}\selectfont Amazon-Photo} dataset overview}
    \label{appendix:amazon_table}
\end{table}

The {\fontfamily{qcr}\selectfont Coauthor-CS} dataset describes a network of computer science coauthors. Such that each node represents an author, and edges connect any two authors who have collaborated on a paper. The dataset aims to classify authors into their respective fields of study based on the keywords from their publications, which are also used as node features. Due to the lack of a clear domain to split the data, OOD graphs were constructed following the same protocol aforementioned. Dataset details are provided in Table \ref{appendix:coauthor_table}.

\begin{table} [H]
    \centering
    \resizebox{0.8\linewidth}{!}{\begin{tabular}{c|c|c|c|c|c}
        \toprule
        {\fontfamily{qcr}\selectfont Coauthor-CS} & Splits&$\#$ Nodes & $\#$ Edges & Feature Dimension & $\#$ Classes\\
        \midrule
        {\fontfamily{qcr}\selectfont Coauthor-S} &ID & 18333 & 163788 & 6805 & 15\\
        {\fontfamily{qcr}\selectfont Coauthor-S} &OOD & 18333 & 92802 & 6805 & 15\\
        \midrule
        {\fontfamily{qcr}\selectfont Coauthor-F} &ID & 18333  & 163788  & 6805 & 15\\
        {\fontfamily{qcr}\selectfont Coauthor-F} &OOD & 18333 & 163788  & 6805 & 15 \\
        \midrule
        {\fontfamily{qcr}\selectfont Coauthor-L} &ID & 13290 & 163788  & 6805 & 10\\
        {\fontfamily{qcr}\selectfont Coauthor-L} &OOD & 3649  & 163788 & 6805 & 4\\
        
        \bottomrule
    \end{tabular}}
    \caption{{\fontfamily{qcr}\selectfont Coauthor-CS} dataset overview}
    \label{appendix:coauthor_table}
\end{table}

The {\fontfamily{qcr}\selectfont ogbn-Arxiv} dataset curated an extensive dataset from 1960 to 2020, where each node represents a paper, labelled by its subject area for classification~\citep{Arxiv}. Edges reflect the citation relationships among papers, and each node is associated with a 128-dimensional vector derived from word embeddings of its title and abstract. Following \cite{GNNSafe}, we utilise time information to partition the graph, where data before 2015 are used as ID data and papers published after 2017 are used as OOD data. Dataset details are provided in Table \ref{appendix:arxiv_table}.

\begin{table} [H]
    \centering
    \resizebox{0.8\linewidth}{!}{\begin{tabular}{c|c|c|c|c|c}
        \toprule
        {\fontfamily{qcr}\selectfont ogbn-Arxiv} & Splits&$\#$ Nodes & $\#$ Edges & Feature Dimension & $\#$ Classes\\
        \midrule
        {\fontfamily{qcr}\selectfont Arxiv-2015} &ID & 53160   & 152226  & 128 & 40 \\
        {\fontfamily{qcr}\selectfont Arxiv-2018} &OOD & 29799  & 622466 & 128 & 40\\
        {\fontfamily{qcr}\selectfont Arxiv-2019} & OOD & 39711   & 1061197  & 128 & 40\\
        {\fontfamily{qcr}\selectfont Arxiv-2020} &OOD & 8892   & 1166243 & 128 & 40\\
        
        \bottomrule
    \end{tabular}}
    \caption{{\fontfamily{qcr}\selectfont ogbn-Arxiv} dataset overview}
    \label{appendix:arxiv_table}
\end{table}

\subsection{Evaluation Metrics} \label{Appendix:Metrics}
In this section, we provide a detailed description of the metrics used for evaluation. Following common practice in OOD detection~\citep{GNNSafe, energy, GOOD-D}, we employed three key metrics to measure the performance of detecting OOD instances: (1) the Area Under the Receiver Operating Characteristic curve (AUROC); (2) the Area Under the Precision-Recall curve (AUPR); and (3) the false positive rate (FPR95) of OOD examples when the true positive rate of ID examples is 95\%. AUROC measures the trade-off between the true positive rate (TPR) and the false positive rate (FPR) at different threshold levels, providing insights into the model's ability to accurately distinguish between ID and OOD instances. However, in highly imbalanced datasets with only a few OOD instances, AUROC might be overly optimistic. AUPR, on the other hand, offers a more realistic performance measure by accounting for both precision and recall. FPR95 provides further insight into the model's performance under high-sensitivity conditions, indicating the probability of misclassifying in-distribution samples as OOD when the TPR is 95\%.

\subsection{Implementation Details} \label{Appendix:implementation_details}
We utilised the publicly available benchmark (i.e., datasets and baselines) provided by \cite{GNNSafe}, and fully respect their CC-BY 4.0 license. The experiments were conducted using Python 3.8.0 and PyTorch 2.2.2 with Cuda 12.1, using Tesla V100 GPUs with 32GB memory for experiments. The datasets were obtained from Pytorch Geometric 2.0.3 and OGB 1.3.3 under the MIT license. Extending beyond the thresholds provided in \cite{GNNSafe}, we tuned the margins $t_\text{ID}$ and $t_\text{OOD}$ with various ranges for different dataset (i.e., for {\fontfamily{qcr}\selectfont Twitch} $t_\text{ID}\in\{-5, -4, -3\}, t_\text{OOD} \in \{1, 2, 3\} $). The detector loss weights $\lambda,\mu,\gamma$ are tuned in the range of $\{0, 0.3, 0.5, 0.7, 1, 1.5\}$, depending on the dataset. Hyperparameter sensitivity analysis for the detector and classifier loss objective can be found in Figure \ref{Appendix:Hyperparam}. The LGM training step $M_1$ is configured in the range of $\{100, 200, 600, 800\}$, and the classifier and detector update $M_2$ is tuned from $\{5-20\}$ subject to the dataset, with early stopping applied to ensure the ID accuracy does not reduce significantly. Regarding baseline models, we utilised the provided benchmark in \cite{GNNSafe}, which includes modified versions of the different baseline models.
This involves adapting to the same encoder GCN backbone (i.e., a hidden size of 64 and layer number of 2) for MSP, ODIN, Mahalanobis, Energy and Energy FT. We also considered latest SOTA OOD Detection method by \cite{NODESAFE} using their reported results.

\subsection{Additional Experiment Results} \label{Appendix:Additional_exp}
In this section, we provide additional experimental results to supplement the results provided in the maintext. Specifically, we present detailed OOD detection performance of the subsets for each OOD dataset (i.e., subgraphs for {\fontfamily{qcr}\selectfont Twitch}, three types of OOD data for {\fontfamily{qcr}\selectfont Cora}, {\fontfamily{qcr}\selectfont Amazon}, and {\fontfamily{qcr}\selectfont Coauthor}, and different years for {\fontfamily{qcr}\selectfont Arxiv}) in Tables \ref{Appendix:twitch_full} to \ref{Appendix:arxiv_full}, complementing Table \ref{Table:overall_performance} in the main text. Furthermore, in Tables \ref{Appendix:Full_ablation_general} and \ref{Appendix:Full_ablation_adversarial}, we report an extended version of the ablation study and adversarial training effectiveness, covering the subsets of {\fontfamily{qcr}\selectfont Twitch}, {\fontfamily{qcr}\selectfont Cora}, and {\fontfamily{qcr}\selectfont Arxiv}, supplementing Table \ref{Table:Ablation_general} and \ref{Table:Ablation_adversarial} in the maintext. Lastly, we provide the full tables for the energy regulariser analysis in Table \ref{Table:Ablation_regulariser} for {\fontfamily{qcr}\selectfont Twitch}, {\fontfamily{qcr}\selectfont Cora}, and {\fontfamily{qcr}\selectfont Amazon} in Tables \ref{Appendix:Twtich_reg_effectiveness}, \ref{Appendix:Cora_reg_effectiveness}, and \ref{Appendix:Amazon_reg_effectiveness}, respectively. 


\begin{table}[H]
    \centering
    \caption{Model performance on OOD sub-graphs ES, FR and RU of {\fontfamily{qcr}\selectfont Twitch} dataset.}
    \resizebox{1\linewidth}{!}{
    \begin{tabular}{c|c|ccccccc|ccc|c}
    \specialrule{.1em}{.05em}{.05em} 
    \multirow{2}{*}{\textbf{Dataset}} & \multirow{2}{*}{\textbf{Metrics}} & \multicolumn{7}{c}{\textbf{Non-OOD Exposure}} & \multicolumn{3}{c}{\textbf{Real OOD Exposure}} & \multicolumn{1}{c}{\textbf{Ours}}\\
    % \cmidrule(lr){3-9} \cmidrule(lr){10-12} \cmidrule(lr){13-13}
    & & \textbf{MSP} & \textbf{ODIN} & \textbf{Mahalanobis} & \textbf{Energy} & \textbf{GKDE} & \textbf{GPN} & \textbf{\textsc{GNNSafe}} & \textbf{OE} & \textbf{Energy FT} & \textbf{\textsc{GNNSafe++}} & \textbf{GOLD}\\
    \midrule
    \multirow{4}{*}{{\fontfamily{qcr}\selectfont Twitch-ES}} & AUROC & 37.72 & 83.83 & 45.66 & 38.80 & 48.70 & 53.00 & 49.07 & 55.97 & 80.73 & 94.54 & 99.72 $\pm$ 0.03 \\
                            & AUPR  & 53.08 & 80.43 & 58.82 & 54.26 & 61.05 & 64.24 & 57.62 & 69.49 & 87.56 & 97.17 & 99.82 $\pm$ 0.02 \\
                            & FPR95   & 98.09 & 33.28 & 95.48 & 95.70 & 95.37 & 95.05 & 93.98 & 94.94 & 76.76 & 44.06 & 0.44 $\pm$0.13 \\
                            & ID ACC & 68.72 & 70.79 & 70.51 & 70.40 & 67.44 & 68.09 & 70.40 & 70.73 & 70.52 & 70.18 & 68.49 $\pm$ 0.13 \\
    \midrule
    \multirow{4}{*}{{\fontfamily{qcr}\selectfont Twitch-FR}} & AUROC & 21.82 & 59.82 & 40.40 & 57.21 & 49.19 & 51.25 & 63.49 & 45.66 & 79.66 & 93.45 & 99.08 $\pm$ 0.19 \\
                            & AUPR  & 38.27 & 64.63 & 46.69 & 61.48 & 52.94 & 55.37 & 66.25 & 54.03 & 81.20 & 95.44 & 99.25 $\pm$ 0.15 \\
                            & FPR95   & 99.25 & 92.57 & 95.54 & 91.57 & 95.04 & 93.92 & 90.80 & 95.48 & 76.39 & 51.06 & 3.77$\pm$ 0.92 \\
                            & ID ACC & 68.72 & 70.79 & 70.51 & 70.40 & 67.44 & 68.09 & 70.40 & 70.73 & 70.52 & 70.18 & 68.49 $\pm$ 0.13 \\
    \midrule
    \multirow{4}{*}{{\fontfamily{qcr}\selectfont Twitch-RU}} & AUROC & 41.23 & 58.67 & 55.68 & 57.72 & 46.48 & 50.89 & 87.90 & 55.72 & 93.12 & 98.10 & 99.58 $\pm$ 0.06 \\
                            & AUPR  & 56.06 & 72.58 & 66.42 & 66.68 & 62.11 & 65.14 & 89.05 & 70.18 & 95.36 & 98.74 & 99.78$\pm$ 0.04 \\
                            & FPR95   & 95.01 & 93.98 & 90.13 & 87.57 & 95.62 & 99.93 & 43.95 & 95.07 & 30.72 & 5.59 & 
                            1.14 $\pm$ 0.35 \\
                            & ID ACC & 68.72 & 70.79 & 70.51 & 70.40 & 67.44 & 68.09 & 70.40 & 70.73 & 70.52 & 70.18 & 68.49 $\pm$ 0.13 \\
    \specialrule{.1em}{.05em}{.05em} 
\end{tabular}} \label{Appendix:twitch_full}
\end{table}

\begin{table}[H]
    \centering
    \caption{Model performance on {\fontfamily{qcr}\selectfont Cora} with three types of OOD (\textbf{S}tructure manipulation, \textbf{F}eature interpolation, and \textbf{L}abel leave-out).}
    \resizebox{1\linewidth}{!}{
    \begin{tabular}{c|c|ccccccc|ccc|c}
    \specialrule{.1em}{.05em}{.05em} 
    \multirow{2}{*}{\textbf{Dataset}} & \multirow{2}{*}{\textbf{Metrics}} & \multicolumn{7}{c}{\textbf{Non-OOD Exposure}} & \multicolumn{3}{c}{\textbf{Real OOD Exposure}} & \multicolumn{1}{c}{\textbf{Ours}}\\
    % \cmidrule(lr){3-9} \cmidrule(lr){10-12} \cmidrule(lr){13-13}
    & & \textbf{MSP} & \textbf{ODIN} & \textbf{Mahalanobis} & \textbf{Energy} & \textbf{GKDE} & \textbf{GPN} & \textbf{\textsc{GNNSafe}} & \textbf{OE} & \textbf{Energy FT} & \textbf{\textsc{GNNSafe++}} & \textbf{GOLD}\\
    \midrule
\multirow{4}{*}{{\fontfamily{qcr}\selectfont Cora-S}} & AUROC & 70.90 & 49.92 & 46.68 & 71.73 & 68.61 & 77.47 & 87.52 & 67.98 & 75.88 & 90.62 & 95.48 $\pm$ 0.28 \\
                            & AUPR  & 45.73 & 27.01 & 29.03 & 46.08 & 44.26 & 53.26 & 77.46 & 46.93 & 49.18 & 81.88 & 91.06 $\pm$ 0.32 \\
                            & FPR95   & 87.30 & 100.00 & 98.19 & 88.74 & 84.34 & 76.22 & 73.15 & 95.31 & 67.73 & 53.51 & 21.86 $\pm$ 0.97 \\
                            & ID ACC & 75.50 & 74.90 & 74.90 & 76.00 & 73.70 & 76.50 & 75.80 & 71.80 & 75.50 & 76.10 & 77.4 $\pm$ 0.56 \\
\midrule
\multirow{4}{*}{{\fontfamily{qcr}\selectfont Cora-F}} & AUROC & 85.39 & 49.88 & 49.93 & 86.15 & 82.79 & 85.88 & 93.44 & 81.83 & 88.15 & 95.56 & 96.64 $\pm$ 0.15 \\
                            & AUPR  & 73.70 & 26.96 & 31.95 & 74.42 & 66.52 & 73.79 & 88.19 & 70.84 & 75.99 & 90.27 & 93.82 $\pm$ 0.24 \\
                            & FPR95   & 64.88 & 100.00 & 99.93 & 65.81 & 68.24 & 56.17 & 38.92 & 83.79 & 47.53 & 27.73 & 14.35 $\pm$ 2.05 \\
                            & ID ACC & 75.30 & 75.00 & 74.90 & 76.10 & 74.80 & 77.00 & 76.40 & 73.30 & 75.30 & 76.80 & 76.77 $\pm$ 0.21\\
\midrule
\multirow{4}{*}{{\fontfamily{qcr}\selectfont Cora-L}} & AUROC & 91.36 & 49.80 & 67.62 & 91.40 & 57.23 & 90.34 & 92.80 & 89.47 & 91.36 & 92.75 & 95.40 $\pm$ 0.17 \\
                            & AUPR  & 78.03 & 24.27 & 42.31 & 78.14 & 27.50 & 77.40 & 82.21 & 77.01 & 78.49 & 82.64 & 88.65 $\pm$ 0.25 \\
                            & FPR95   & 34.99 & 100.00 & 90.77 & 41.08 & 88.95 & 37.42 & 30.83 & 46.55 & 37.83 & 34.08 & 17.28 $\pm$ 0.50 \\
                            & ID ACC & 88.92 & 88.92 & 88.92 & 88.92 & 89.87 & 91.46 & 88.92 & 87.97 & 90.51 & 91.46 & 90.82 $\pm$ 0.55 \\
\specialrule{.1em}{.05em}{.05em} 
    \end{tabular}} \label{Appendix:cora_full}
\end{table}

\begin{table}[H]
    \centering
    \caption{Model performance on {\fontfamily{qcr}\selectfont Amazon} with three types of OOD  (\textbf{S}tructure manipulation, \textbf{F}eature interpolation, and \textbf{L}abel leave-out).}
    \resizebox{1\linewidth}{!}{
    \begin{tabular}{c|c|ccccccc|ccc|c}
    \specialrule{.1em}{.05em}{.05em} 
    \multirow{2}{*}{\textbf{Dataset}} & \multirow{2}{*}{\textbf{Metrics}} & \multicolumn{7}{c}{\textbf{Non-OOD Exposure}} & \multicolumn{3}{c}{\textbf{Real OOD Exposure}} & \multicolumn{1}{c}{\textbf{Ours}}\\
    & & \textbf{MSP} & \textbf{ODIN} & \textbf{Mahalanobis} & \textbf{Energy} & \textbf{GKDE} & \textbf{GPN} & \textbf{\textsc{GNNSafe}} & \textbf{OE} & \textbf{Energy FT} & \textbf{\textsc{GNNSafe++}} & \textbf{GOLD}\\
    \midrule
\multirow{4}{*}{{\fontfamily{qcr}\selectfont Amazon-S}} & AUROC & 98.27 & 93.24 & 71.69 & 98.51 & 76.39 & 97.17 & 99.58 & 99.60 & 98.83 & 99.82 & 99.99 $\pm$ 0.03 \\
 & AUPR & 98.54 & 95.26 & 79.01 & 98.72 & 81.58 & 96.39 & 99.76 & 99.61 & 99.14 & 99.89 & 99.99 $\pm$ 0.02 \\
 & FPR95 & 6.13 & 65.44 & 99.91 & 4.97 & 99.25 & 11.65 & 0.00 & 0.51 & 1.31 & 0.00 & 0 $\pm$ 0 \\
 & ID ACC & 92.84 & 92.84 & 92.79 & 92.86 & 87.57 & 88.51 & 92.53 & 92.61 & 92.79 & 92.22 & 92.03 $\pm$ 0.24 \\

\midrule
\multirow{4}{*}{{\fontfamily{qcr}\selectfont Amazon-F}} & AUROC & 97.31 & 81.15 & 76.50 & 97.87 & 58.96 & 87.91 & 98.55 & 98.39 & 98.68 & 99.64 & 99.17 $\pm$ 0.02 \\
 & AUPR & 95.16 & 78.47 & 71.14 & 95.64 & 66.76 & 84.77 & 98.99 & 96.24 & 96.82 & 99.68 & 99.31 $\pm$ 0.06 \\
 & FPR95 & 8.72 & 100.0 & 76.12 & 6.00 & 99.28 & 49.11 & 0.31 & 4.34 & 2.84 & 0.13 & 0.14 $\pm$ 0.03 \\
 & ID ACC & 92.89 & 92.71 & 92.86 & 92.96 & 86.18 & 90.05 & 92.81 & 92.30 & 92.52 & 92.39 & 91.76 $\pm$ 0.57 \\

\midrule
\multirow{4}{*}{{\fontfamily{qcr}\selectfont Amazon-L}} & AUROC & 93.97 & 65.97 & 73.25 & 93.81 & 65.58 & 92.72 & 97.35 & 95.39 & 96.61 & 97.51 & 97.26 $\pm$ 0.27  \\
 & AUPR & 91.32 & 57.80 & 66.89 & 91.13 & 65.20 & 90.34 & 97.12 & 92.53 & 94.92 & 97.07 & 97.46 $\pm$ 0.29 \\
 & FPR95 & 26.65 & 90.23 & 74.30 & 28.48 & 96.87 & 37.16 & 6.59 & 17.72 & 13.78 & 6.18 & 6.06 $\pm$ 1.81 \\
 & ID ACC & 95.76 & 96.08 & 95.76 & 95.72 & 89.37 & 90.07 & 95.76 & 95.72 & 94.83 & 95.84 & 95.18 $\pm$ 0.81\\
\specialrule{.1em}{.05em}{.05em} 
    \end{tabular}} \label{Appendix:amazon_full}
\end{table}

\begin{table}[H]
    \centering
    \caption{Model performance on {\fontfamily{qcr}\selectfont Coauthor} with three types of OOD  (\textbf{S}tructure manipulation, \textbf{F}eature interpolation, and \textbf{L}abel leave-out).}
    \resizebox{1\linewidth}{!}{
    \begin{tabular}{c|c|ccccccc|ccc|c}
    \specialrule{.1em}{.05em}{.05em} 
    \multirow{2}{*}{\textbf{Dataset}} & \multirow{2}{*}{\textbf{Metrics}} & \multicolumn{7}{c}{\textbf{Non-OOD Exposure}} & \multicolumn{3}{c}{\textbf{Real OOD Exposure}} & \multicolumn{1}{c}{\textbf{Ours}}\\
    & & \textbf{MSP} & \textbf{ODIN} & \textbf{Mahalanobis} & \textbf{Energy} & \textbf{GKDE} & \textbf{GPN} & \textbf{\textsc{GNNSafe}} & \textbf{OE} & \textbf{Energy FT} & \textbf{\textsc{GNNSafe++}} & \textbf{GOLD}\\
    \midrule
    \multirow{4}{*}{{\fontfamily{qcr}\selectfont Coauthor-S}} & AUROC & 95.30 & 52.14 & 80.46 & 96.18 & 65.87 & 34.67 & 99.60 & 97.86 & 98.84 & 99.99 & 99.62 $\pm$ 0.02 \\
 & AUPR & 94.37 & 48.83 & 76.65 & 95.25 & 72.65 & 40.21 & 99.69 & 96.81 & 97.78 & 99.99 & 99.78 $\pm$ 0.01 \\
 & FPR95 & 24.75 & 99.92 & 70.75 & 18.02 & 99.48 & 99.57 & 0.26 & 9.23 & 3.97 & 0.02 & 0.01 $\pm$ 0.01 \\
 & ID ACC & 92.47 & 92.34 & 92.33 & 92.75 & 88.62 & 89.45 & 92.73 & 92.60 & 92.61 & 92.92 & 91.41 $\pm$ 0.16 \\
\midrule
\multirow{4}{*}{{\fontfamily{qcr}\selectfont Coauthor-F}} & AUROC & 97.05 & 51.54 & 93.23 & 97.88 & 80.69 & 81.77 & 99.64 & 99.04 & 99.43 & 99.97 & 99.78 $\pm$ 0.15 \\
 & AUPR & 96.93 & 45.50 & 90.88 & 97.69 & 86.47 & 80.56 & 99.66 & 98.80 & 99.25 & 99.95 & 99.86 $\pm$ 0.09 \\
 & FPR95 & 15.55 & 100.0 & 28.10 & 9.75 & 96.57 & 74.46 & 0.51 & 4.44 & 2.25 & 0.09 & 0.03 $\pm$ 0.01 \\
 & ID ACC & 92.45 & 92.39 & 92.34 & 92.75 & 84.72 & 87.05 & 92.73 & 92.64 & 92.50 & 92.87 & 91.81 $\pm$ 0.26 \\
\midrule
\multirow{4}{*}{{\fontfamily{qcr}\selectfont Coauthor-L}} & AUROC & 94.88 & 51.44 & 85.36 & 95.87 & 61.15 & 93.24 & 97.23 & 96.04 & 96.23 & 97.89 & 97.63 $\pm$ 0.16 \\
 & AUPR & 97.99 & 74.79 & 93.61 & 98.34 & 81.39 & 97.55 & 98.98 & 98.50 & 98.51 & 99.24 & 99.06 $\pm$ 0.07 \\
 & FPR95 & 23.81 & 100.0 & 45.41 & 18.69 & 94.60 & 34.78 & 12.06 & 18.17 & 17.07 & 9.43 & 9.46 $\pm$ 0.3 \\
 & ID ACC & 95.18 & 95.15 & 95.19 & 95.20 & 89.05 & 91.68 & 95.21 & 95.10 & 95.20 & 95.24 & 94.84 $\pm$ 0.03 \\
\specialrule{.1em}{.05em}{.05em} 
    \end{tabular}} \label{Appendix:coauthor_full}
\end{table}

\begin{table}[H]
    \centering
    \caption{Model performance on OOD datasets of paper published in 2018, 2019, and 2020 on {\fontfamily{qcr}\selectfont Arxiv}.}
    \resizebox{1\linewidth}{!}{
    \begin{tabular}{c|c|ccccccc|ccc|c}
    \specialrule{.1em}{.05em}{.05em} 
    \multirow{2}{*}{\textbf{Dataset}} & \multirow{2}{*}{\textbf{Metrics}} & \multicolumn{7}{c}{\textbf{Non-OOD Exposure}} & \multicolumn{3}{c}{\textbf{Real OOD Exposure}} & \multicolumn{1}{c}{\textbf{Ours}}\\
    & & \textbf{MSP} & \textbf{ODIN} & \textbf{Mahalanobis} & \textbf{Energy} & \textbf{GKDE} & \textbf{GPN} & \textbf{\textsc{GNNSafe}} & \textbf{OE} & \textbf{Energy FT} & \textbf{\textsc{GNNSafe++}} & \textbf{GOLD}\\
    \midrule
        \multirow{4}{*}{{\fontfamily{qcr}\selectfont Arxiv-2018}} & AUROC & 61.66 & 53.49 & 57.08 & 61.75 & 56.29 & OOM & 66.47 & 67.72 & 69.58 & 70.40 & 69.74 $\pm$ 0.28 \\
        & AUPR & 70.63 & 63.06 & 65.09 & 70.41 & 66.78 & OOM & 74.99 & 75.74 & 76.31 & 78.62 & 77.12 $\pm$ 0.23 \\
        & FPR95 & 91.67 & 100.0 & 93.69 & 91.74 & 94.31 & OOM & 89.44 & 86.67 & 82.10 & 81.47 & 83.20 $\pm$ 0.57 \\
        & ID ACC & 53.78 & 51.39 & 51.59 & 53.36 & 50.76 & OOM & 53.39 & 52.39 & 53.26 & 53.50 & 50.59 $\pm$ 0.53 \\
        \midrule
        \multirow{4}{*}{{\fontfamily{qcr}\selectfont Arxiv-2019}} & AUROC & 63.07 & 53.95 & 56.76 & 63.16 & 57.87 & OOM & 68.36 & 69.33 & 70.58 & 72.16 & 72.46 $\pm$ 0.35 \\
        & AUPR & 66.00 & 56.07 & 57.85 & 65.78 & 62.34 & OOM & 71.57 & 72.15 & 72.03 & 75.43 & 75.41 $\pm$ 0.38 \\
        & FPR95 & 90.82 & 100.0 & 94.01 & 90.96 & 93.97 & OOM & 88.02 & 85.52 & 81.30 & 79.33 & 81.16 $\pm$ 0.58 \\
        & ID ACC & 53.78 & 51.39 & 51.59 & 53.36 & 50.76 & OOM & 53.39 & 52.39 & 53.26 & 53.50 & 50.59 $\pm$ 0.53 \\
        \midrule
        \multirow{4}{*}{{\fontfamily{qcr}\selectfont Arxiv-2020}} & AUROC & 67.00 & 55.78 & 56.92 & 67.70 & 60.79 & OOM & 78.35 & 72.35 & 74.53 & 81.75 & 79.50 $\pm$ 0.11 \\
        & AUPR & 90.92 & 87.41 & 85.95 & 91.15 & 88.74 & OOM & 94.76 & 92.57 & 93.08 & 95.57 & 95.02 $\pm$ 0.04 \\
        & FPR95 & 89.28 & 100.0 & 95.01 & 89.69 & 93.31 & OOM & 83.57 & 83.28 & 78.36 & 71.50 & 77.36 $\pm$ 0.75 \\
        & ID ACC & 53.78 & 51.39 & 51.59 & 53.36 & 50.76 & OOM & 53.39 & 52.39 & 53.26 & 53.50 & 50.59 $\pm$ 0.53 \\
    \specialrule{.1em}{.05em}{.05em} 
    \end{tabular}}\label{Appendix:arxiv_full}
\end{table}


\begin{table}[H]
    \centering
    \caption{Extended ablation performance of individual subsets.}
    \resizebox{0.8\linewidth}{!}{
    \begin{tabular}{c|c|cc|cc|c}
    % \specialrule{.1em}{.05em}{.05em} 
\toprule
    \textbf{Dataset} & \textbf{Metrics} & \textbf{\textsc{GNNSafe}} & \textbf{\textsc{GNNSafe++}} & \textbf{w/o Adv.} & \textbf{w/o Det.} & \textbf{GOLD}\\
    \midrule
    \multirow{4}{*}{{\fontfamily{qcr}\selectfont Twitch-ES}} 
        & AUROC & 49.07 & 94.54 & 69.10 & 57.65 & 99.72 \\
        & AUPR  & 57.62 & 97.17 & 75.86 & 65.82 & 99.82\\
        & FPR95 & 93.98 & 44.06 & 85.82 & 91.65 & 0.44 \\
        & ID ACC & 70.40 & 70.18 & 70.97 & 70.97  & 68.49\\
    \midrule
    \multirow{4}{*}{{\fontfamily{qcr}\selectfont Twitch-FR}} 
        & AUROC & 63.49 & 93.45 & 93.86 & 88.98 & 99.08 \\
        & AUPR  & 66.25 & 95.44 & 95.45 & 92.61 & 99.25\\
        & FPR95 & 90.80 & 51.06 & 39.44  & 70.84 & 3.77\\
        & ID ACC & 70.40 & 70.18 & 70.97 & 70.97  & 68.49\\
    \midrule
    \multirow{4}{*}{{\fontfamily{qcr}\selectfont Twitch-RU}} 
        & AUROC & 87.90 & 98.10 & 90.81 & 86.48 & 99.58 \\
        & AUPR  & 89.05 & 98.74 & 94.75 & 93.30 & 99.78\\
        & FPR95 & 43.95 & 5.59 & 53.87 & 77.01 & 1.14\\
        & ID ACC & 70.40 & 70.18 & 70.97 & 70.97  & 68.49\\
    \midrule    
    \multirow{4}{*}{{\fontfamily{qcr}\selectfont Cora-S}} & AUROC & 87.52 & 90.62 & 90.05  & 93.33   & 95.48 \\
                            & AUPR   & 77.46 & 81.88 & 83.04 & 87.13 & 91.06 \\
                            & FPR95   & 73.15 & 53.51 & 59.45 & 31.98  & 21.86 \\
                            & ID ACC & 75.80 & 76.10 & 67.70  & 75.60  & 77.40\\
\midrule
    \multirow{4}{*}{{\fontfamily{qcr}\selectfont Cora-F}} & AUROC & 93.44 & 95.56 & 94.43 & {95.24}  & 96.64\\
                            & AUPR  & 88.19 & 90.27 & 91.74& 91.23 & 93.82\\
                            & FPR95   & 38.92 & 27.73 & 29.54  & 26.74 & 14.35\\
                            & ID ACC & 76.40 & 76.80 & 76.50 & 75.70  & 76.77 \\
    \midrule
    \multirow{4}{*}{{\fontfamily{qcr}\selectfont Cora-L}} & AUROC & 89.47 & 92.75 & 84.45  & 91.71& 95.40\\
                            & AUPR  & 82.21 & 82.64 & 65.90  & 81.98& 88.65 \\
                            & FPR95  & 30.83 & 34.08 &50.00 & 43.31& 17.28 \\
                            & ID ACC & 88.92 & 91.46 & 88.60  &90.80  & 90.82\\
\midrule
\multirow{4}{*}{{\fontfamily{qcr}\selectfont Arxiv-2018}}
        & AUROC & 66.47 & 70.40 & 64.97  & 65.55  & 69.74\\
        & AUPR  & 74.99 & 78.62 & 72.71 & 73.63  & 77.12\\
        & FPR95  & 89.44 & 81.47 & 90.12 & 91.19   & 83.20 \\
        & ID ACC & 53.39 & 53.50 & 49.89 & 49.66  & 50.59 \\
    \midrule
\multirow{4}{*}{{\fontfamily{qcr}\selectfont Arxiv-2019}}    
        & AUROC & 68.36 & 72.16 & 67.21  & 67.13  & 72.46\\
        & AUPR  & 71.57 & 75.43 & 69.70 & 69.06  & 75.41\\
        & FPR95  & 88.02 & 79.33 & 88.97 & 90.27  & 81.16\\
        & ID ACC & 53.39 & 53.50 & 49.89 & 49.66  & 50.59 \\
        \midrule
\multirow{4}{*}{{\fontfamily{qcr}\selectfont Arxiv-2020}}    
        & AUROC & 78.35 & 81.75 & 77.11  & 77.04  & 79.50\\
        & AUPR  & 94.76 & 95.57 & 94.39 & 94.45  & 95.02\\
        & FPR95  & 83.57 & 71.50 & 85.40 & 87.54  & 77.36\\
        & ID ACC & 53.39 & 53.50 & 49.89 & 49.66  & 50.59 \\
    \specialrule{.1em}{.05em}{.05em} 
    \end{tabular}
    } 
    \label{Appendix:Full_ablation_general}
\end{table}


\begin{table}[H]
    \centering
    \caption{Extended adversarial training effectiveness analysis of individual subsets.}
    \resizebox{0.8\linewidth}{!}{
    \begin{tabular}{c|c|c|ccc|c}
    \toprule
    \textbf{Dataset} & \textbf{Metrics} & \textbf{\textsc{GNNSafe++}}& \textbf{Dif.\ Once}  & \textbf{Dif.\ Multi} & \textbf{Real OOD} & \textbf{GOLD}\\
    \midrule
    \multirow{4}{*}{{\fontfamily{qcr}\selectfont Twitch-ES}} 
        & AUROC & 94.54& 69.10  & 66.52 & 98.99 & 99.72 \\
        & AUPR   & 97.17& 75.86 & 73.49 & 99.52 & 99.82\\
        & FPR95 & 44.06 & 85.82 & 87.07 & 1.38 & 0.44 \\
        & ID ACC & 70.18 & 70.40  & 71.12 & 70.45  & 68.49\\
    \midrule
    \multirow{4}{*}{{\fontfamily{qcr}\selectfont Twitch-FR}} 
        & AUROC & 93.45& 93.86  & 95.20 & 94.51 & 99.08 \\
        & AUPR   & 95.44& 95.45 & 96.54 & 96.33 & 99.25\\
        & FPR95 & 51.06 & 39.44 & 31.08  & 40.62 & 3.77\\
        & ID ACC & 70.18 & 70.40  & 71.12 & 70.45  & 68.49\\
    \midrule
    \multirow{4}{*}{{\fontfamily{qcr}\selectfont Twitch-RU}} 
        & AUROC & 98.10& 90.81  & 91.28 & 99.24 & 99.58 \\
        & AUPR  & 98.74& 94.75  & 95.10 & 99.65 & 99.78\\
        & FPR95 & 5.59 & 53.87 & 82.84 & 1.16 & 1.14\\
        & ID ACC & 70.18 & 70.40  & 71.12 & 70.45  & 68.49\\
    \midrule    
    \multirow{4}{*}{{\fontfamily{qcr}\selectfont Cora-S}} 
        & AUROC & 90.62 & 90.05  & 95.69  & 94.12 & 95.48 \\
        & AUPR  & 81.88 & 83.04  & 91.59 & 89.91 & 91.06 \\
        & FPR95   & 53.51 & 59.45 & 22.30 &  37.11 & 21.86 \\
        & ID ACC & 76.10& 67.70  & 76.10 & 75.90 & 77.40\\
    \midrule
    \multirow{4}{*}{{\fontfamily{qcr}\selectfont Cora-F}} 
        & AUROC & 95.56& 94.43  & 96.02 & 97.60 & 96.64\\
        & AUPR  & 90.27& 91.74  & 92.99  & 94.23 & 93.82\\
        & FPR95  & 27.73 & 29.54 & 18.94 &10.27 & 14.35\\
        & ID ACC & 76.80 & 76.50 & 77.30  & 71.70 & 76.77 \\
    \midrule
    \multirow{4}{*}{{\fontfamily{qcr}\selectfont Cora-L}} 
        & AUROC & 92.75& 84.45  & 86.76 & 95.04 & 95.40 \\
        & AUPR   & 82.64 & 65.90 & 70.70 & 86.01& 88.65 \\
        & FPR95  & 34.08 & 50.00 & 49.09 & 17.24 & 17.28 \\
        & ID ACC & 91.46  & 88.60 & 88.29 & 87.65 & 90.82\\
    \midrule
\multirow{4}{*}{{\fontfamily{qcr}\selectfont Arxiv-2018}}
        & AUROC & 70.40& 64.97  & 67.26  & 75.32  & 69.74\\
        & AUPR  & 78.62& 72.71  & 74.72 & 80.89  & 77.12\\
        & FPR95  & 81.47 & 90.12 & 85.23 & 72.40   & 83.20 \\
        & ID ACC & 53.50 & 53.39  & 50.77 & 49.99  & 50.59 \\
    \midrule
\multirow{4}{*}{{\fontfamily{qcr}\selectfont Arxiv-2019}}    
        & AUROC & 72.16 & 67.21 & 69.66  & 77.98 & 72.46\\
        & AUPR  & 75.43 & 69.70 & 72.05 & 79.56  & 75.41\\
        & FPR95  & 79.33& 88.97  & 83.33 & 95.92  & 81.16\\
        & ID ACC & 53.50& 53.39  & 50.77 & 49.99   & 50.59 \\
        \midrule
\multirow{4}{*}{{\fontfamily{qcr}\selectfont Arxiv-2020}}    
        & AUROC & 81.75& 77.11   & 79.52  &83.41 & 79.50\\
        & AUPR  & 95.57& 94.39  & 94.95 & 95.92  & 95.02\\
        & FPR95  & 71.50& 85.40  & 77.52 & 64.93  & 77.36\\
        & ID ACC & 53.50& 53.39  & 50.77 &49.99   & 50.59 \\
    \specialrule{.1em}{.05em}{.05em} 
    \end{tabular}} \label{Appendix:Full_ablation_adversarial}
\end{table}

\begin{table}[H]
    \centering
    \caption{Extended energy regulariser effectiveness analysis on {\fontfamily{qcr}\selectfont Twitch}.}
    \resizebox{1\linewidth}{!}{
    \begin{tabular}{ccc|cccc|cccc|cccc}
    \toprule
    \multirow{2}{*}{$\mathcal{L}_\text{Unc}$} & \multirow{2}{*}{$\mathcal{L}_\text{EReg}$} & \multirow{2}{*}{$\mathcal{L}_\text{DReg}$} & \multicolumn{4}{c|}{{\fontfamily{qcr}\selectfont Twitch-ES}}& \multicolumn{4}{c|}{{\fontfamily{qcr}\selectfont Twitch-FR}}& \multicolumn{4}{c}{{\fontfamily{qcr}\selectfont Twitch-RU}}\\
    & & & AUROC & AUPR & FPR & ID Acc & AUROC & AUPR & FPR & ID Acc& AUROC & AUPR & FPR & ID Acc\\
    \midrule
     & & &
    74.33 & 76.23 & 52.97& 68.97& 
   98.34 & 98.72 & 2.58& 68.97 &
   26.92& 48.67& 91.70 & 68.97\\
    \midrule
     \checkmark& & & 
    17.39 & 44.31 & 98.04 & 70.15& 
   5.69 & 34.24& 99.07& 70.15 &
   7.48 & 43.30 & 96.42 &70.15 \\
    & \checkmark& & 
    60.03 & 68.63& 92.36& 70.98 & 
    90.21& 92.64 & 66.57 & 70.98 &
    83.81 & 88.85 & 77.77& 70.98 \\
    & & \checkmark& 
    18.69 & 44.81 & 97.89 & 70.79& 
    96.40 & 94.65 & 8.85 & 70.79 &
    92.04 & 91.19 & 26.89 & 70.79\\
    \midrule
     \checkmark& \checkmark& &
    59.40  & 68.20 & 92.77 & 70.99 & 
    91.91& 93.23 & 56.07 & 70.99 0&
   79.32 & 83.03 & 79.59 & 70.99 \\
     \checkmark& & \checkmark& 
    7.50 & 42.10 & 99.07 & 70.90 & 
    99.04 & 98.15 & 1.30 & 70.90 &
    86.76 & 86.13 & 37.49 & 70.90 \\
    & \checkmark& \checkmark& 
    90.55 & 94.20 & 41.98 & 69.64& 
    88.95 & 91.30 & 45.86 & 69.64 &
    90.35 & 80.94 & 42.09 & 69.64 \\
    \midrule
    & GOLD & & 
    99.72 &99.82 & 0.44& 68.49& 
    99.08 & 99.25 & 3.77 & 68.49& 
    99.58 & 99.78 & 1.14& 68.49\\
    \specialrule{.1em}{.05em}{.05em} 
    \end{tabular}} \label{Appendix:Twtich_reg_effectiveness}
\end{table}

\begin{table}[H]
    \centering
    \caption{Extended energy regulariser effectiveness analysis on {\fontfamily{qcr}\selectfont Cora}.}
    \resizebox{1\linewidth}{!}{
    \begin{tabular}{ccc|cccc|cccc|cccc}
    \toprule
    \multirow{2}{*}{$\mathcal{L}_\text{Unc}$} & \multirow{2}{*}{$\mathcal{L}_\text{EReg}$} & \multirow{2}{*}{$\mathcal{L}_\text{DReg}$} & \multicolumn{4}{c|}{{\fontfamily{qcr}\selectfont Cora-S}}& \multicolumn{4}{c|}{{\fontfamily{qcr}\selectfont Cora-F}}& \multicolumn{4}{c}{{\fontfamily{qcr}\selectfont Cora-L}}\\
    & & & AUROC & AUPR & FPR & ID Acc & AUROC & AUPR & FPR & ID Acc& AUROC & AUPR & FPR & ID Acc\\
    \midrule
     & & &
    86.44 & 91.26 & 35.20 & 67.70 & 
    13.00 & 16.79 & 99.19 & 70.80 &
    83.98 & 76.01 & 90.06 & 90.19\\
    \midrule
     \checkmark& & & 
    68.06 & 67.14 & 89.18 & 80.00& 
    68.01 & 77.40 & 45.70 & 75.80 &
    76.20 & 60.36 & 96.65 & 87.34 \\
    & \checkmark& & 
    94.70 & 90.09 & 30.54 & 74.30 & 
    10.41 & 15.72 & 100 & 76.20 &
    92.90 & 84.14 & 31.64 & 90.82 \\
    & & \checkmark& 
    89.18 & 80.96 & 65.18 & 72.80& 
    79.83 & 74.36 & 90.69 & 70.00&
    84.16 & 68.39 & 49.59 & 85.44\\
    \midrule
     \checkmark& \checkmark& &
    47.95  & 49.68 & 91.32 & 77.60 & 
    47.63 & 54.31 & 99.41 & 76.60&
    8.30 & 13.82 & 99.80 & 89.55\\
     \checkmark& & \checkmark& 
    86.26 & 76.48 & 64.59 & 69.70 & 
    79.40 & 73.66 & 93.65 & 67.50 &
    86.43 & 71.52 & 62.37 & 86.39 \\
    & \checkmark& \checkmark& 
    93.94 & 88.86 & 28.99 & 75.60 & 
    95.54 & 92.69 & 22.05 & 76.70 &
    90.35 & 80.94 & 42.09 & 87.34 \\
    \midrule
    & GOLD & & 
    95.48 & 91.06 & 21.86 & 77.40& 
    96.64 & 93.82 & 14.35 & 76.77& 
    95.40 & 88.65 & 17.28 & 90.82\\
    \specialrule{.1em}{.05em}{.05em} 
    \end{tabular}} \label{Appendix:Cora_reg_effectiveness}
\end{table}


\begin{table}[H]
    \centering
    \caption{Extended energy regulariser effectiveness analysis on {\fontfamily{qcr}\selectfont Amazon}.}
    \resizebox{1\linewidth}{!}{
    \begin{tabular}{ccc|cccc|cccc|cccc}
    \toprule
    \multirow{2}{*}{$\mathcal{L}_\text{Unc}$} & \multirow{2}{*}{$\mathcal{L}_\text{EReg}$} & \multirow{2}{*}{$\mathcal{L}_\text{DReg}$} & \multicolumn{4}{c|}{{\fontfamily{qcr}\selectfont Amazon-S}}& \multicolumn{4}{c|}{{\fontfamily{qcr}\selectfont Amazon-F}}& \multicolumn{4}{c}{{\fontfamily{qcr}\selectfont Amazon-L}}\\
    & & & AUROC & AUPR & FPR & ID Acc & AUROC & AUPR & FPR & ID Acc& AUROC & AUPR & FPR & ID Acc\\
    \midrule
     & & &
    96.67 & 97.48 & 20.16 & 88.35 & 
    1.21 & 26.70 & 99.35 & 92.11 &
    94.64 & 93.84 & 20.66 & 95.76 \\
    \midrule
     \checkmark& & & 
    84.90 & 90.80 & 100.00 & 92.17 & 
    21.53 & 30.92 & 86.18 & 92.96 &
    95.17 & 93.45 & 18.21 & 95.96 \\
    & \checkmark& & 
    52.82 & 69.21 & 100.00 & 92.74 & 
    1.58 & 26.81 & 98.47 & 91.89 &
    91.53 & 89.77 & 35.26 & 95.72 \\
    & & \checkmark& 
    100.00 & 100.00 & 0.00 & 91.35 & 
    99.76 & 99.55 & 0.63 & 91.62 &
    93.39 & 90.94 & 24.20 & 95.39 \\
    \midrule
     \checkmark& \checkmark& &
    85.72 & 91.06 & 99.97 & 92.48 & 
    57.99 & 44.03 & 55.73 & 92.63 &
    69.74 & 56.45 & 68.85 & 94.79 \\
     \checkmark& & \checkmark& 
    100.00 & 100.00 & 0.00 & 92.33 & 
    98.61 & 99.09 & 0.29 & 91.76 &
    95.11 & 91.99 & 13.34 & 95.52 \\
    & \checkmark& \checkmark& 
    98.58 & 99.21 & 0.00 & 92.25 & 
    98.36 & 98.75 & 0.60 & 92.04 &
    97.13 & 97.29 & 9.61 & 94.14 \\
    \midrule
    & GOLD & & 
    99.98 & 99.99 & 0.00 & 92.03& 
    99.17 & 99.31 & 0.14 & 91.76& 
    97.26 & 97.46 & 6.06 & 95.18\\
    \specialrule{.1em}{.05em}{.05em} 
    \end{tabular}} \label{Appendix:Amazon_reg_effectiveness}
\end{table}

\begin{figure}[!h]
    \centering
    % \begin{subfigure}[b]{0.32\textwidth}
    %     \centering
    %     \includegraphics[width=\textwidth]{fig/Twitch-Hyperparam_freq.pdf}
    %     \caption{}
    %     \label{fig:FREQ_HYPER}
    % \end{subfigure}
    % \hfill
    \begin{subfigure}[b]{0.35\textwidth}
        \centering
        \includegraphics[width=\textwidth]{fig/Twitch-Hyperparam_AUROC.pdf}
        \caption{}
        \label{fig:AUROC_HYPER}
    \end{subfigure}
    \begin{subfigure}[b]{0.35\textwidth}
        \centering
        \includegraphics[width=\textwidth]{fig/Twitch-Hyperparam_FPR.pdf}
        \caption{}
        \label{fig:FPR_HYPER}
    \end{subfigure}
    \caption{The {\fontfamily{qcr}\selectfont Twitch} dataset was utilised for conducting hyper-parameter sensitivity analysis. (\ref{fig:AUROC_HYPER}) and (\ref{fig:FPR_HYPER}) are Hyper-parameter sensitivity of different weights in Eq.~\ref{eq:div} for Detector measured by AUROC and FPR95.} 
    \label{Appendix:Hyperparam}
\end{figure}

% \begin{figure}[!h]
%     \centering
%     \begin{subfigure}[b]{0.21\textwidth}
%         \centering
%         \includegraphics[width=\textwidth]{fig/Twitch-initial.pdf}
%         \caption{Ep 1, train diffusion.}
%         % \label{fig:twitch_score_gap}
%     \end{subfigure}
%     \hfill
%     \begin{subfigure}[b]{0.22\textwidth}
%         \centering
%         \includegraphics[width=\textwidth]{fig/Twitch-during-training3.pdf}
%         \caption{Ep 14, train GNN and Det.}
%         % \label{fig:twitch_id_ood}
%     \end{subfigure}
%     % \hfill
%     % \begin{subfigure}[b]{0.3\textwidth}
%     %     \centering
%     %     \includegraphics[width=\textwidth]{fig/Twitch-during-training3.pdf}
%     %     % \caption{{\fontfamily{qcr}\selectfont Twitch} test (GNNSafe++).}
%     %     % \label{fig:twitch_gnnsafe_id_ood}
%     % \end{subfigure}
%     \hfill
%     \begin{subfigure}[b]{0.22\textwidth}
%         \centering
%         \includegraphics[width=\textwidth]{fig/Twitch-during-training5.pdf}
%         \caption{Ep 15, train diffusion.}
%         % \label{fig:cora_id_ood}
%     \end{subfigure}
%     \hfill
%     \begin{subfigure}[b]{0.22\textwidth}
%         \centering
%         \includegraphics[width=\textwidth]{fig/Twitch-during-training6-final.pdf}
%         \caption{Ep 22, train GNN and Det.}
%         % \label{fig:cora_score_gap}
%     \end{subfigure}
%     \caption{
%     % \textcolor{red}{May need to show the specific number of epoch of each fig, instead of ``initial'', ``during'' and ``final''.} \textcolor{blue}{I have thought of this before, but i think it might be less intuitive since the diffusion model is trained separately with the others, so the first figure is the diffusion model being trained first to mimic gnn, then the following is training gnn etc, then training diffusion to become closer then gnn will again learn to distinguish better. }
%     \textbf{Energy Score distribution throughout the adversarial training for the {\fontfamily{qcr}\selectfont Twitch} dataset, for the in-distribution data (\textcolor{green}{ID}), the synthetic pseudo OOD (\textcolor{red}{p-OOD}), and the real \textcolor{violet}{OOD} in different epochs (Ep).} (a) shows that the initial energy scores of ID, p-OOD and OOD cannot be separated when the diffusion model is firstly trained to mimic the latent distribution of ID data. (b) represents the adversarial energy divergence between ID and p-OOD data during OOD detection training for the backbone GNN and the detector (Det). (c) illustrates the alternating optimisation of the diffusion model to imitate the updated latent distribution of ID data, preventing the p-OOD from diverging too far as shown in (b). (d) is the final converged model's energy distribution.} 
%     \vspace{-0.5cm}
%     % The green dashed line is the threshold $t_\text{ID}$ and the red dashed line is the threshold $t_\text{OOD}$ in Eq.~\eqref{eq: ereg}. (a) and (d) illustrate the energy score gaps between the ID and pseudo-OOD data obtained from the GNN and MLP models for training set. (b) and (e) display the energy score distributions comparing ID, pseudo-OOD, and test OOD data for test set. (c) and (f) illustrate that the energy of test ID data, test OOD data, and the additionally exposed real OOD as in \textsc{GNNSafe++}. GOLD is more effective in separating the energy of test OOD data from test ID data.}
%     \label{Appendix:Energy_Score_Distribution}
% \end{figure}

\subsection{Ablation study visualisation} \label{Appendix:ablation_vis}
To explore how the different modules contribute to OOD detection, we present further energy distribution visualisations in Figure \ref{Appendix:Additional_Ablation_fig}.
The adversarial training can help to maintain the closeness between synthetic and ID data, preventing the synthetic samples from diverging too far from real ID/OOD data to bias the detector. Our method alternates between two tasks: (1) the latent diffusion model pulls latent embeddings of ID data and generated pseudo-OOD embeddings closer, while (2) the GNN \& detector push their energies apart. Without this component, the pseudo-OOD distribution diverges significantly compared to GOLD in Figure \ref{Appendix:Additional_Ablation_fig} c/f, where synthetic embeddings are regularly updated. This divergence makes OOD data indistinct from ID data, leading to poor performance in the ablation study.
The detector can decrease the overlap of energy distribution between the ID and OOD samples, leading to better energy-based OOD detection. Figure \ref{Appendix:Additional_Ablation_fig} b/e shows that w/o detector will lead to a large overlap of energy distribution between ID and OOD samples. This overlap occurs because the energy scores, derived from prediction logits of the GNN classifier, become indistinct as the number of predicted classes increases and when the classifier struggles to distinguish certain classes. Thus, introducing a dedicated detector to further discern energy scores enhances detection by reducing the number of output classes.

% Energy visualisation for training w/o adv & w/o det
\begin{figure}[h]
    \centering
    \begin{subfigure}[b]{0.25\textwidth}
        \centering
        \includegraphics[width=\textwidth]{fig/Twitch-wo-adv.pdf}
        \caption{Twitch w/o Adv.}
        % \label{fig:twitch_score_gap}
    \end{subfigure}
    \hfill
    \begin{subfigure}[b]{0.25\textwidth}
        \centering
        \includegraphics[width=\textwidth]{fig/Twitch-wo-detector.pdf}
        \caption{Twitch w/o Det.}
        % \label{fig:twitch_id_ood}
    \end{subfigure}
    \hfill
    \begin{subfigure}[b]{0.25\textwidth}
        \centering
        \includegraphics[width=\textwidth]{fig/Twitch-IDvsOOD_Enlarged_v2.pdf}
        \caption{Twtich GOLD}
        % \label{fig:cora_id_ood}
    \end{subfigure}
    \hfill
    \begin{subfigure}[b]{0.25\textwidth}
        \centering
        \includegraphics[width=\textwidth]{fig/CoraL-wo-adv.pdf}
        \caption{Cora-L w/o Adv.}
        % \label{fig:cora_score_gap}
    \end{subfigure}
    \hfill
    \begin{subfigure}[b]{0.25\textwidth}
        \centering
        \includegraphics[width=\textwidth]{fig/CoraL-wo-detector.pdf}
        \caption{Cora-L w/o Det.}
        % \label{fig:cora_score_gap}
    \end{subfigure}
    \hfill
    \begin{subfigure}[b]{0.25\textwidth}
        \centering
        \includegraphics[width=\textwidth]{fig/CoraL-IDvsOOD_Enlarged_v2.pdf}
        \caption{Cora-L GOLD}
        % \label{fig:cora_score_gap}
    \end{subfigure}
    \caption{
    %\textcolor{red}{I think they may not be happy to hear the word ``ablation'', because they may think ablation is just to remove something and give them a result. I prefer just to say visualisation personally.} 
    \textbf{Visualisation of the energy score distributions of GOLD without adversarial training or the use of detector for {\fontfamily{qcr}\selectfont Twitch} and {\fontfamily{qcr}\selectfont Cora-L} datasets.} (a) and (d) illustrate the energy score gaps w/o adversarial training, where the energy of p-OOD data will be diverged too far and fail to diverge the energy of real OOD. (b) and (e) shows the energy scores derived from the GNN classifier without our proposed detector, where the energy scores cannot be effectively separated. (c) and (f) demonstrates the ability of GOLD to effectively distinguish the ID and OOD energy distributions, illustrating the effectiveness of the adversarial and detector components.}
    \vspace{-0.5cm}
    \label{Appendix:Additional_Ablation_fig}
\end{figure}

\subsection{Latent Generative Model} \label{Appendix:latent_generative_model}
In this section, we provide description of the two latent generative models utilised: The variational autoencoder and the latent diffusion model.

\subsubsection{Variational Autoencoder}
The variational autoencoder (VAE) is a generative model consisting of an encoder that learns latent variables from training data and a decoder that then uses those latent variables to reconstruct the input data. VAEs are trained to optimise a lower bound on the marginal log-likelihood $\log p_\theta(x)$ over the data by using a learned approximate posterior $q_\phi(h|x)$, as follows:

$$\mathcal{L}(\theta, \phi; x) = \mathbb{E}_{q_\phi(h|x)}[\log p_\theta(x|h)] - D_{KL}(q_\phi(h|x) || p(h))$$
s.t the first term is the reconstruction loss, and the second term is the KL divergence of the approximate from the true posterior. The trained approximate posterior $q_\phi(h|x)$ would thus act as an encoder that maps the data $x$ to a lower dimensional latent representation, and latent samples $h$ can be drawn via the reparametrisation trick: 
$$ h = \mu_\phi(x) + \sigma_\phi(x) \odot \epsilon, \text{ where } \epsilon \sim \mathcal{N}(0, I) \text{ if the models are Gaussian}. $$

We set the encoder hidden dimension size to be 512, the decoder dimension to be 256, and layer sizes to be 2.

\subsubsection{Latent Diffusion Model}
The latent diffusion model consists of a forward diffusion and a backward denoising process on a set of latent representations~\citep{DDPM, LatentDIF, NGG}. In our GOLD, a latent node representation $\mathbf{h}_0\in\mathbb{R}^{d'}$ is initialised at timestep 0 from the GNN encodings $\mathbf{H}$ in Eq.~\ref{eq:GCN_emb_logits}. At the forward process, the model progressively adds Gaussian noise to the latent node representation  $\mathbf{h}_0$, according to a known variance schedule $\beta_1, \cdots, \beta_T$, for $0 < \beta_1 < . . . \beta_T < 1$. This process will produce a sequence of increasingly noisy vectors $(\mathbf{h}_1, \cdots \mathbf{h}_T)$ with timestep $t = \{1,2,3, \dots, T\}$. Denoting $a_t=1-\beta_t$ and $\bar{a}_t=\prod_{i=1}^t a_i$, we can derive a closed form for obtaining the representation at any timestep $t$ given the initial representation $\mathbf{h}_0$:
\begin{equation}
\mathbf{h}_t \sim \mathcal{N}\left(\sqrt{\bar{a}_t} \mathbf{h_0},\left(1-\bar{a}_t\right) \mathbf{I}\right).
\end{equation}
The backward denoising process involves predicting the noise added to the representation at a given timestep via a denoising model $D$ (e.g., MLP). To train the latent diffusion model, we minimise the mean squared error loss between the added noise $\bm{\epsilon} ~ \sim \mathcal{N}(\mathbf{0},\mathbf{I})$ and the predicted noise from the noisy representation $\mathbf{h}_t$ at a given timestep $t$ with the reparameterisation trick: 
\begin{equation}
\min_D\mathcal{L}_{\mathrm{{Gen}}},\text{ where }\mathcal{L}_{\mathrm{{Gen}}} =\mathbb{E}_{\mathbf{h}_0, \bm{\epsilon}, t}\left[\left\|\bm{\epsilon} -D\left(\sqrt{\bar{a}_t} \mathbf{h}_0 +\sqrt{\left(1-\bar{a}_t\right)} \bm{\epsilon}, t\right)\right\|_2^2\right]. \label{eq:latent_appendix}
\end{equation}

\begin{table}[h!]
\centering
\resizebox{0.6\linewidth}{!}{\begin{tabular}{c|c|c|c|c|c}
\toprule
\textbf{Hyperparams} & \textbf{} & \textbf{AUROC} & \textbf{AUPR} & \textbf{FPR} & \textbf{ID ACC} \\
\midrule
\multirow{4}{*}{$\beta_1$} & 0.00001 & 99.43 & 99.59 & 1.77 & 68.48 \\
          & \textbf{0.0001}  & 99.46 & 99.62 & 1.78 & 68.49 \\
          & 0.001   & 99.52 & 99.66 & 1.44 & 68.11 \\
          & 0.01    & 99.46 & 99.60 & 1.99 & 67.71 \\
\midrule
\multirow{3}{*}{$\beta_T$} & 0.005   & 85.91 & 88.79 & 44.77 & 71.04 \\
          & \textbf{0.02}    & 99.46 & 99.62 & 1.78 & 68.49 \\
          & 0.1     & 82.02 & 86.79 & 55.27 & 68.40 \\
\midrule
\multirow{6}{*}{$T$}       & 400     & 96.59 & 97.84 & 18.23 & 69.43 \\
          & 500     & 95.15 & 96.38 & 21.32 & 68.99 \\
          & \textbf{600}     & 99.46 & 99.62 & 1.78 & 68.49 \\
          & 700     & 98.80 & 99.27 & 4.07 & 68.09 \\
          & 800     & 92.02 & 95.68 & 66.94 & 68.85 \\
          & 1000    & 17.35 & 45.22 & 99.57 & 71.20 \\
\bottomrule
\end{tabular}}
\caption{Performance comparison for different hyperparameters for Diffusion model on Twitch dataset. Default values are highlighted in \textbf{Bold}.}
\label{Appendix:hyperparam_dif}
\end{table}
We set the diffusion model parameter $\beta$ to be a sequence of linearly increasing constants from $\beta_1=10^{-4}$ to $\beta_T=0.02$ as presented in \citep{DDPM,sd}.
A hyperparameter sensitivity experiment on the Twitch dataset is provided in Table \ref{Appendix:hyperparam_dif}. Generally, a larger (smaller) $\beta$ adds or removes more (less) noise at each step. A larger $T$ increases noise corruption, making recovery harder but with more output variation, while a smaller $T$ reduces noise corruption, making recovery easier but limiting variation. $\beta_1$ typically does not affect performance, while $\beta_T$ is more sensitive, reflecting higher/lower corruption at the end of the timestep. The value of $T$ also impacts performance, with non-default values either limiting or excessively diversifying the synthetic samples.

% We provide experiments by substituting the latent diffusion with a much more lightweight latent generator, Variational Auto-Encoder (VAE). The OOD detection performance is very close while being more lightweight due to the model design. This demonstrates that our adversarial training pipeline can use different latent generators to achieve the goal of pseudo-OOD generation for OOD detection.

% % VAE results
% \begin{table}[!h]
%     \centering
%     \caption{\textbf{Performance of GOLD with different latent generative models.} Latent diffusion model (LDM) is the default for GOLD, which can be replaced by Variational Graph Auto-Encoder (VAE). The \textbf{Non-OOD exposed} GOLD framework outperforms both \textbf{Non- and Real-OOD exposed} SOTA baselines with both generators.}
%     \resizebox{0.8\linewidth}{!}{
%     \begin{tabular}{c|cccc|cccc}
%         \toprule
%          \multirow{2}{*}{Model Performance} & \multicolumn{4}{c|}{{\fontfamily{qcr}\selectfont Twitch}} & \multicolumn{4}{c}{{\fontfamily{qcr}\selectfont Cora}}\\
%          & AUROC($\uparrow$) & AUPR($\uparrow$) & FPR95($\downarrow$) & ID ACC($\uparrow$) & AUROC($\uparrow$) & AUPR($\uparrow$) & FPR95($\downarrow$) & ID ACC($\uparrow$) \\
%          \midrule
%          \textsc{GNNSafe} (Non) & 66.82 & 70.97 & 76.24 & 70.40  & 91.25 & 82.62 & 47.38 & 80.37 \\
%          \midrule
%          \textsc{GNNSafe++} (Real) & 95.36 & 97.12 & 33.57 & 70.18 & 92.98 & 84.93 & 38.44 & 81.45 \\
%          \midrule
%          \midrule
%          GOLD-VAE (Non) & 99.26 & 98.54 & 3.03 & 68.50 & 89.96 & 93.19 & 28.66 & 76.79 \\
%          \midrule
%          % \textcolor{red}{\textbf{GOLD-LDM}} & \textcolor{red}{\textbf{99.46}} & \textcolor{red}{\textbf{99.62}} & \textcolor{red}{\textbf{1.78}} & 68.49 & \textcolor{red}{\textbf{95.84}} & \textcolor{red}{\textbf{91.17}} & \textcolor{red}{\textbf{17.83}} & 81.66 \\
%          GOLD-LDM (Non) & 99.46 & 99.62 & 1.78 & 68.49 & 95.84 & 91.17 & 17.83 & 81.66 \\
%          \bottomrule
%     \end{tabular}
%     }
%     \vspace{-0.5cm}
%     \label{Appendix:VAE}
% \end{table}

\subsection{Computational Cost} \label{Appendix:computational_cost}
In this section, we provide the computational cost of GOLD against SOTA baselines. GOLD outperforms the baselines with a rough trade-off of 2x training time and memory usage.
\begin{table}[!h]
    \centering
    \caption{\textbf{Computation cost (one 32GB (32768MiB) NVIDIA V100 GPU) and OOD detection performance of GOLD (Non-OOD Exposed) against both \textbf{Non- and Real-OOD exposed} SOTA baselines.} The `Train' column is the training convergence time in seconds. The `Test' column is the inference time in seconds. The `Mem.' column is the maximum memory usage in Mebibytes (MiB). The `FPR95' column is the OOD detection performance in \%, the lower the better. \textbf{The inference time of these methods is the same with the same backbone GNN.}}
    \resizebox{1\linewidth}{!}{
    \begin{tabular}{c|cccc|cccc|cccc|cccc|cccc}
        \toprule
        &\multicolumn{4}{c|}{\fontfamily{qcr}\selectfont Twitch}&\multicolumn{4}{c|}{\fontfamily{qcr}\selectfont Cora-F}&\multicolumn{4}{c|}{\fontfamily{qcr}\selectfont Amazon-F}&\multicolumn{4}{c|}{\fontfamily{qcr}\selectfont Coauthor-F}&\multicolumn{4}{c}{\fontfamily{qcr}\selectfont Arxiv}\\
        &Train&Test&Mem.&FPR95$(\downarrow)$&Train&Test&Mem.&FPR95$(\downarrow)$&Train&Test&Mem.&FPR95$(\downarrow)$&Train&Test&Mem.&FPR95$(\downarrow)$&Train&Test&Mem.&FPR95$(\downarrow)$\\
        \midrule
        \textsc{GNNSafe} (Non) 
        & 2.41 & 0.08 & 667 & 76.24 
        & 4.40 & 0.03 & 465 & 38.92 
        & 13.51 & 0.04 & 665 & 0.31
        & 57.80 & 0.35 &  1523 & 0.51
        & 85.23 & 0.40 &  3370 & 87.01\\     
        % \textsc{GNNSafe}&\multicolumn{3}{c|}{ }&\multicolumn{3}{c|}{ }&\multicolumn{3}{c|}{ }&\multicolumn{3}{c|}{ }\\ 
        
        \midrule
        \textsc{GNNSafe++} (Real)
        & 4.74 & 0.09 & 667 &33.57
        & 5.32 & 0.03 & 465 & 27.73
        & 18.40 & 0.05 & 665 & 0.13
        & 67.83 & 0.36 & 1523 & 0.09
        & 132.36 & 0.40 & 3370 & 77.43\\
        % \textsc{GNNSafe++}&\multicolumn{3}{c|}{ }&\multicolumn{3}{c|}{ }&\multicolumn{3}{c|}{ }&\multicolumn{3}{c|}{ }\\
        \midrule
        GOLD w/ VAE (Non)
        & 2.78 & 0.09 & 1427 & 3.03
        & 3.91 & 0.04 & 1081 & 23.60
        & 12.52 & 0.05 & 1319 & 0.15
        & 55.65 & 0.35 & 2439 & 0.23
        & 80.77 & 0.45 & 9039 & 81.95\\
        \midrule
        GOLD w/ LDM (Non)
        & 8.96 & 0.10 & 1452 & 1.71
        & 5.93 & 0.04 & 1083 & 14.51
        & 39.04 & 0.07 & 1347 & 0.11
        & 89.74 & 0.37 & 2515 & 0.01
        & 244.95 & 0.47 & 10579 & 80.35\\
        % \textbf{GOLD}&\multicolumn{3}{c|}{ }&\multicolumn{3}{c|}{ }&\multicolumn{3}{c|}{ }&\multicolumn{3}{c|}{ }\\
        \bottomrule
        \end{tabular}
            }
\end{table}

\subsection{Ablation with Additional Backbone}\label{Appendix: Backbone}
We provide the following experiments with two additional backbones: GAT~\citep{GAT} and MixHop~\citep{MixHop}. We compare these architectures against GNNSafe and NodeSafe and their OOD-exposed variants. To ensure a fair comparison, we maintain the same configuration as the original GCN implementation, with a hidden dimension of 64, two layers, 8 attention heads for GAT, and two hops for MixHop. The results shown in Table~\ref{tab:backbone}, demonstrate that GOLD outperforms other methods across the evaluated backbones.

\begin{table}[h!]
\centering
\caption{Ablation of different backbones}

\resizebox{1\linewidth}{!}{
\begin{tabular}{c|c|c|c|c|c|c|c}
\toprule
\textbf{Dataset} & \textbf{Backbone} & \textbf{Metrics} & \textbf{\textsc{GNNSafe}} & \textbf{\textsc{GNNSafe++}} & \textbf{\textsc{NodeSafe}} & \textbf{\textsc{NodeSafe++}} & \textbf{GOLD} \\ \midrule
\multirow{6}{*}{{\fontfamily{qcr}\selectfont Twitch}}           
    & \multirow{3}{*}{MixHop} & AUROC & \textcolor{purple}{72.08} & 95.07 & 57.91 & \underline{95.08}  & \textcolor{teal}{\textbf{96.94}}  \\ \cline{3-8} 
        &   & FPR95 & \textcolor{purple}{73.70} & 33.46 & 93.76 & \underline{30.71}  & \textcolor{teal}{\textbf{17.98}}  \\ \cline{3-8} 
        &   & ID Acc & 69.66 & 66.04 & 70.09 & 70.56 & 67.58  \\ \cline{2-8}
    & \multirow{3}{*}{GAT}  & AUROC  & \textcolor{purple}{83.08} & \underline{97.51}  & 54.78 & 95.07                & \textcolor{teal}{\textbf{98.64}}  \\ \cline{3-8} 
    &   & FPR95 & \textcolor{purple}{50.46}  & \underline{20.43}  & 93.24 & 30.71                & \textcolor{teal}{\textbf{1.42}}   \\ \cline{3-8} 
    &  & ID Acc & 68.21 & 68.54 & 68.40 & 70.56 & 67.32 \\ \midrule
\multirow{6}{*}{{\fontfamily{qcr}\selectfont Cora}} 
    & \multirow{3}{*}{MixHop}  & AUROC  & \textcolor{purple}{88.65}  & 91.33 & 82.60 & \textbf{92.79}  & \textcolor{teal}{\underline{91.42}}  \\ \cline{3-8} 
    &         & FPR95  & \textcolor{purple}{59.08}  & 44.59 & 60.22 & \underline{38.63}  & \textcolor{teal}{\textbf{25.09}}  \\ \cline{3-8} 
    &        & ID Acc  & 79.52  & 80.66 & 82.16 & 81.45 & 80.67  \\ \cline{2-8}
    & \multirow{3}{*}{GAT} & AUROC & \textcolor{purple}{91.62}  & \underline{92.50} & 85.55 & 92.32                & \textcolor{teal}{\textbf{94.66}}  \\ \cline{3-8} 
    &                      & FPR95 & \textcolor{purple}{33.81}  & \underline{33.44} & 55.20 & 34.93                & \textcolor{teal}{\textbf{19.63}}  \\ \cline{3-8} 
    &                      & ID Acc & 79.44 & 79.52 & 81.06 & 80.23 & 78.40 \\ 
    \bottomrule
\end{tabular}
}
\label{tab:backbone}
\end{table}





















\begin{comment}
\section{Submission of conference papers to ICLR 2025}

ICLR requires electronic submissions, processed by
\url{https://openreview.net/}. See ICLR's website for more instructions.

If your paper is ultimately accepted, the statement {\tt
  {\textbackslash}iclrfinalcopy} should be inserted to adjust the
format to the camera ready requirements.

The format for the submissions is a variant of the NeurIPS format.
Please read carefully the instructions below, and follow them
faithfully.

\subsection{Style}

Papers to be submitted to ICLR 2025 must be prepared according to the
instructions presented here.

%% Please note that we have introduced automatic line number generation
%% into the style file for \LaTeXe. This is to help reviewers
%% refer to specific lines of the paper when they make their comments. Please do
%% NOT refer to these line numbers in your paper as they will be removed from the
%% style file for the final version of accepted papers.

Authors are required to use the ICLR \LaTeX{} style files obtainable at the
ICLR website. Please make sure you use the current files and
not previous versions. Tweaking the style files may be grounds for rejection.

\subsection{Retrieval of style files}

The style files for ICLR and other conference information are available online at:
\begin{center}
   \url{http://www.iclr.cc/}
\end{center}
The file \verb+iclr2025_conference.pdf+ contains these
instructions and illustrates the
various formatting requirements your ICLR paper must satisfy.
Submissions must be made using \LaTeX{} and the style files
\verb+iclr2025_conference.sty+ and \verb+iclr2025_conference.bst+ (to be used with \LaTeX{}2e). The file
\verb+iclr2025_conference.tex+ may be used as a ``shell'' for writing your paper. All you
have to do is replace the author, title, abstract, and text of the paper with
your own.

The formatting instructions contained in these style files are summarized in
sections \ref{gen_inst}, \ref{headings}, and \ref{others} below.

\section{General formatting instructions}
\label{gen_inst}

The text must be confined within a rectangle 5.5~inches (33~picas) wide and
9~inches (54~picas) long. The left margin is 1.5~inch (9~picas).
Use 10~point type with a vertical spacing of 11~points. Times New Roman is the
preferred typeface throughout. Paragraphs are separated by 1/2~line space,
with no indentation.

Paper title is 17~point, in small caps and left-aligned.
All pages should start at 1~inch (6~picas) from the top of the page.

Authors' names are
set in boldface, and each name is placed above its corresponding
address. The lead author's name is to be listed first, and
the co-authors' names are set to follow. Authors sharing the
same address can be on the same line.

Please pay special attention to the instructions in section \ref{others}
regarding figures, tables, acknowledgments, and references.


There will be a strict upper limit of 10 pages for the main text of the initial submission, with unlimited additional pages for citations. 

\section{Headings: first level}
\label{headings}

First level headings are in small caps,
flush left and in point size 12. One line space before the first level
heading and 1/2~line space after the first level heading.

\subsection{Headings: second level}

Second level headings are in small caps,
flush left and in point size 10. One line space before the second level
heading and 1/2~line space after the second level heading.

\subsubsection{Headings: third level}

Third level headings are in small caps,
flush left and in point size 10. One line space before the third level
heading and 1/2~line space after the third level heading.

\section{Citations, figures, tables, references}
\label{others}

These instructions apply to everyone, regardless of the formatter being used.

\subsection{Citations within the text}

Citations within the text should be based on the \texttt{natbib} package
and include the authors' last names and year (with the ``et~al.'' construct
for more than two authors). When the authors or the publication are
included in the sentence, the citation should not be in parenthesis using \verb|\citet{}| (as
in ``See \citet{Hinton06} for more information.''). Otherwise, the citation
should be in parenthesis using \verb|\citep{}| (as in ``Deep learning shows promise to make progress
towards AI~\citepp{Bengio+chapter2007}.'').

The corresponding references are to be listed in alphabetical order of
authors, in the \textsc{References} section. As to the format of the
references themselves, any style is acceptable as long as it is used
consistently.

\subsection{Footnotes}

Indicate footnotes with a number\footnote{Sample of the first footnote} in the
text. Place the footnotes at the bottom of the page on which they appear.
Precede the footnote with a horizontal rule of 2~inches
(12~picas).\footnote{Sample of the second footnote}

\subsection{Figures}

All artwork must be neat, clean, and legible. Lines should be dark
enough for purposes of reproduction; art work should not be
hand-drawn. The figure number and caption always appear after the
figure. Place one line space before the figure caption, and one line
space after the figure. The figure caption is lower case (except for
first word and proper nouns); figures are numbered consecutively.

Make sure the figure caption does not get separated from the figure.
Leave sufficient space to avoid splitting the figure and figure caption.

You may use color figures.
However, it is best for the
figure captions and the paper body to make sense if the paper is printed
either in black/white or in color.
\begin{figure}[h]
\begin{center}
%\framebox[4.0in]{$\;$}
\fbox{\rule[-.5cm]{0cm}{4cm} \rule[-.5cm]{4cm}{0cm}}
\end{center}
\caption{Sample figure caption.}
\end{figure}

\subsection{Tables}

All tables must be centered, neat, clean and legible. Do not use hand-drawn
tables. The table number and title always appear before the table. See
Table~\ref{sample-table}.

Place one line space before the table title, one line space after the table
title, and one line space after the table. The table title must be lower case
(except for first word and proper nouns); tables are numbered consecutively.

\begin{table}[t]
\caption{Sample table title}
\label{sample-table}
\begin{center}
\begin{tabular}{ll}
\multicolumn{1}{c}{\bf PART}  &\multicolumn{1}{c}{\bf DESCRIPTION}
\\ \hline \\
Dendrite         &Input terminal \\
Axon             &Output terminal \\
Soma             &Cell body (contains cell nucleus) \\
\end{tabular}
\end{center}
\end{table}

\section{Default Notation}

In an attempt to encourage standardized notation, we have included the
notation file from the textbook, \textit{Deep Learning}
\cite{goodfellow2016deep} available at
\url{https://github.com/goodfeli/dlbook_notation/}.  Use of this style
is not required and can be disabled by commenting out
\texttt{math\_commands.tex}.


\centerline{\bf Numbers and Arrays}
\bgroup
\def\arraystretch{1.5}
\begin{tabular}{p{1in}p{3.25in}}
$\displaystyle a$ & A scalar (integer or real)\\
$\displaystyle \va$ & A vector\\
$\displaystyle \mA$ & A matrix\\
$\displaystyle \tA$ & A tensor\\
$\displaystyle \mI_n$ & Identity matrix with $n$ rows and $n$ columns\\
$\displaystyle \mI$ & Identity matrix with dimensionality implied by context\\
$\displaystyle \ve^{(i)}$ & Standard basis vector $[0,\dots,0,1,0,\dots,0]$ with a 1 at position $i$\\
$\displaystyle \text{diag}(\va)$ & A square, diagonal matrix with diagonal entries given by $\va$\\
$\displaystyle \ra$ & A scalar random variable\\
$\displaystyle \rva$ & A vector-valued random variable\\
$\displaystyle \rmA$ & A matrix-valued random variable\\
\end{tabular}
\egroup
\vspace{0.25cm}

\centerline{\bf Sets and Graphs}
\bgroup
\def\arraystretch{1.5}

\begin{tabular}{p{1.25in}p{3.25in}}
$\displaystyle \sA$ & A set\\
$\displaystyle \R$ & The set of real numbers \\
$\displaystyle \{0, 1\}$ & The set containing 0 and 1 \\
$\displaystyle \{0, 1, \dots, n \}$ & The set of all integers between $0$ and $n$\\
$\displaystyle [a, b]$ & The real interval including $a$ and $b$\\
$\displaystyle (a, b]$ & The real interval excluding $a$ but including $b$\\
$\displaystyle \sA \backslash \sB$ & Set subtraction, i.e., the set containing the elements of $\sA$ that are not in $\sB$\\
$\displaystyle \gG$ & A graph\\
$\displaystyle \parents_\gG(\ervx_i)$ & The parents of $\ervx_i$ in $\gG$
\end{tabular}
\vspace{0.25cm}


\centerline{\bf Indexing}
\bgroup
\def\arraystretch{1.5}

\begin{tabular}{p{1.25in}p{3.25in}}
$\displaystyle \eva_i$ & Element $i$ of vector $\va$, with indexing starting at 1 \\
$\displaystyle \eva_{-i}$ & All elements of vector $\va$ except for element $i$ \\
$\displaystyle \emA_{i,j}$ & Element $i, j$ of matrix $\mA$ \\
$\displaystyle \mA_{i, :}$ & Row $i$ of matrix $\mA$ \\
$\displaystyle \mA_{:, i}$ & Column $i$ of matrix $\mA$ \\
$\displaystyle \etA_{i, j, k}$ & Element $(i, j, k)$ of a 3-D tensor $\tA$\\
$\displaystyle \tA_{:, :, i}$ & 2-D slice of a 3-D tensor\\
$\displaystyle \erva_i$ & Element $i$ of the random vector $\rva$ \\
\end{tabular}
\egroup
\vspace{0.25cm}


\centerline{\bf Calculus}
\bgroup
\def\arraystretch{1.5}
\begin{tabular}{p{1.25in}p{3.25in}}
% NOTE: the [2ex] on the next line adds extra height to that row of the table.
% Without that command, the fraction on the first line is too tall and collides
% with the fraction on the second line.
$\displaystyle\frac{d y} {d x}$ & Derivative of $y$ with respect to $x$\\ [2ex]
$\displaystyle \frac{\partial y} {\partial x} $ & Partial derivative of $y$ with respect to $x$ \\
$\displaystyle \nabla_\vx y $ & Gradient of $y$ with respect to $\vx$ \\
$\displaystyle \nabla_\mX y $ & Matrix derivatives of $y$ with respect to $\mX$ \\
$\displaystyle \nabla_\tX y $ & Tensor containing derivatives of $y$ with respect to $\tX$ \\
$\displaystyle \frac{\partial f}{\partial \vx} $ & Jacobian matrix $\mJ \in \R^{m\times n}$ of $f: \R^n \rightarrow \R^m$\\
$\displaystyle \nabla_\vx^2 f(\vx)\text{ or }\mH( f)(\vx)$ & The Hessian matrix of $f$ at input point $\vx$\\
$\displaystyle \int f(\vx) d\vx $ & Definite integral over the entire domain of $\vx$ \\
$\displaystyle \int_\sS f(\vx) d\vx$ & Definite integral with respect to $\vx$ over the set $\sS$ \\
\end{tabular}
\egroup
\vspace{0.25cm}

\centerline{\bf Probability and Information Theory}
\bgroup
\def\arraystretch{1.5}
\begin{tabular}{p{1.25in}p{3.25in}}
$\displaystyle P(\ra)$ & A probability distribution over a discrete variable\\
$\displaystyle p(\ra)$ & A probability distribution over a continuous variable, or over
a variable whose type has not been specified\\
$\displaystyle \ra \sim P$ & Random variable $\ra$ has distribution $P$\\% so thing on left of \sim should always be a random variable, with name beginning with \r
$\displaystyle  \E_{\rx\sim P} [ f(x) ]\text{ or } \E f(x)$ & Expectation of $f(x)$ with respect to $P(\rx)$ \\
$\displaystyle \Var(f(x)) $ &  Variance of $f(x)$ under $P(\rx)$ \\
$\displaystyle \Cov(f(x),g(x)) $ & Covariance of $f(x)$ and $g(x)$ under $P(\rx)$\\
$\displaystyle H(\rx) $ & Shannon entropy of the random variable $\rx$\\
$\displaystyle \KL ( P \Vert Q ) $ & Kullback-Leibler divergence of P and Q \\
$\displaystyle \mathcal{N} ( \vx ; \vmu , \mSigma)$ & Gaussian distribution %
over $\vx$ with mean $\vmu$ and covariance $\mSigma$ \\
\end{tabular}
\egroup
\vspace{0.25cm}

\centerline{\bf Functions}
\bgroup
\def\arraystretch{1.5}
\begin{tabular}{p{1.25in}p{3.25in}}
$\displaystyle f: \sA \rightarrow \sB$ & The function $f$ with domain $\sA$ and range $\sB$\\
$\displaystyle f \circ g $ & Composition of the functions $f$ and $g$ \\
  $\displaystyle f(\vx ; \vtheta) $ & A function of $\vx$ parametrized by $\vtheta$.
  (Sometimes we write $f(\vx)$ and omit the argument $\vtheta$ to lighten notation) \\
$\displaystyle \log x$ & Natural logarithm of $x$ \\
$\displaystyle \sigma(x)$ & Logistic sigmoid, $\displaystyle \frac{1} {1 + \exp(-x)}$ \\
$\displaystyle \zeta(x)$ & Softplus, $\log(1 + \exp(x))$ \\
$\displaystyle || \vx ||_p $ & $\normlp$ norm of $\vx$ \\
$\displaystyle || \vx || $ & $\normltwo$ norm of $\vx$ \\
$\displaystyle x^+$ & Positive part of $x$, i.e., $\max(0,x)$\\
$\displaystyle \1_\mathrm{condition}$ & is 1 if the condition is true, 0 otherwise\\
\end{tabular}
\egroup
\vspace{0.25cm}



\section{Final instructions}
Do not change any aspects of the formatting parameters in the style files.
In particular, do not modify the width or length of the rectangle the text
should fit into, and do not change font sizes (except perhaps in the
\textsc{References} section; see below). Please note that pages should be
numbered.

\section{Preparing PostScript or PDF files}

Please prepare PostScript or PDF files with paper size ``US Letter'', and
not, for example, ``A4''. The -t
letter option on dvips will produce US Letter files.

Consider directly generating PDF files using \verb+pdflatex+
(especially if you are a MiKTeX user).
PDF figures must be substituted for EPS figures, however.

Otherwise, please generate your PostScript and PDF files with the following commands:
\begin{verbatim}
dvips mypaper.dvi -t letter -Ppdf -G0 -o mypaper.ps
ps2pdf mypaper.ps mypaper.pdf
\end{verbatim}

\subsection{Margins in LaTeX}

Most of the margin problems come from figures positioned by hand using
\verb+\special+ or other commands. We suggest using the command
\verb+\includegraphics+
from the graphicx package. Always specify the figure width as a multiple of
the line width as in the example below using .eps graphics
\begin{verbatim}
   \usepackage[dvips]{graphicx} ...
   \includegraphics[width=0.8\linewidth]{myfile.eps}
\end{verbatim}
or % Apr 2009 addition
\begin{verbatim}
   \usepackage[pdftex]{graphicx} ...
   \includegraphics[width=0.8\linewidth]{myfile.pdf}
\end{verbatim}
for .pdf graphics.
See section~4.4 in the graphics bundle documentation (\url{http://www.ctan.org/tex-archive/macros/latex/required/graphics/grfguide.ps})

A number of width problems arise when LaTeX cannot properly hyphenate a
line. Please give LaTeX hyphenation hints using the \verb+\-+ command.

\subsubsection*{Author Contributions}
If you'd like to, you may include  a section for author contributions as is done
in many journals. This is optional and at the discretion of the authors.

\subsubsection*{Acknowledgments}
Use unnumbered third level headings for the acknowledgments. All
acknowledgments, including those to funding agencies, go at the end of the paper.


% \bibliography{iclr2025_conference}
% \bibliographystyle{iclr2025_conference}

\appendix
\section{Appendix}
You may include other additional sections here.
\end{comment}

\end{document}
