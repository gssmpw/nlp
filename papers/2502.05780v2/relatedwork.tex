\section{Related Work}
Our work intersects with three major research areas: \textbf{1) Non-OOD-Exposure OOD Detection} that purely relies on ID data for detecting OOD instances, this involves score-based methods, feature learning, and techniques specific for graph-structured data~\citep{ConfOOD, Hendrycks17softmax, MSP, GenOE, energy, SGOOD, grasp, GKDE, GOOD-D, GraphDE, GNNSafe, NODESAFE}; \textbf{2) OOD Exposure-Based OOD Detection}, a prominent line of work that adopts auxiliary OOD data to assist training, often achieving higher performance than non-OOD-exposure based methods~\citep{OE, energy, Textual-OODExposure, DivOE, ATOL, SAL,GNNSafe, GDE_OOD}; and \textbf{3) OOD Generation,} a more recent field that aims to synthesise OOD-like data to assist OOD detection~\citep{manifold, Likelihood, LRegret, GenAnalysis, GenUnknown, Hierarc,ConfOOD,VOS,NPOS}. Notably for graph data, \textsc{GNNSafe} considers the inter-dependence nature of node instances and proposes an energy propagation schema, and explores an OOD-exposed variant \textsc{GNNSafe++}~\citep{GNNSafe}. \textsc{NODESafe/++} builds upon \textsc{GNNSafe/++} and proposes additional regularisation terms to reduce and bound the generation of extreme energy scores~\citep{NODESAFE}. \cite{GDE_OOD} proposes a generalised Dirichlet energy score for graph OOD detection. A detailed review of related work is provided in Appendix~\ref{Appendix:related_work}.

% \paragraph{Non-OOD-Exposure OOD Detection.} OOD detection is a fundamental task extensively studied in diverse machine learning domains~\citep{ConfOOD, EBFeat, RPRW, regOOD, MSP, Mahalanobis, GenOE, OECC, FocalOE}. A representative line of work that relies on purely ID data is based on the model's output including using softmax score~\citep{Hendrycks17softmax, ODIN}, using energy score~\citep{energy, Wang21energy, NODESAFE}, and activation pruning-based methods~\citep{ASH,DICE, React}. Other approaches involve confidence enhancement~\citep{Gen-ODIN, WhyReLU, Ensemble}, feature learning~\citep{MOOD, NMD}, and adversarial strategies~\citep{GOOD-cert, ATOM, RND}. More recent studies have applied OOD detection to graph-structured data~\citep{SGOOD, grasp, GOODAT, LMN, UGTs, G_UQ, EMP, advOOD, SLW,GOOD,GOODD-uncertainty}. For node-level detection, \textsc{GNNSafe} considers the inter-dependence nature of node instances and proposes an energy propagation schema~\citep{GNNSafe}. \textsc{NODESafe} builds upon \textsc{GNNSafe} and proposes additional regularisation terms to reduce and bound the generation of extreme energy scores~\citep{NODESAFE}. GKDE proposes a multi-source uncertainty framework to estimate the node-level Dirichlet distributions to assist OOD detection~\citep{GKDE}. GPN applies Bayesian posterior and density estimation to estimate the uncertainty for each node~\citep{GPN}. For graph-level detection, recent methods includes modelling distribution shifts through a graph generative process, overseeing from a data-centric perspective, and unsupervised methods~\citep{GraphDE, AAGOD, GOOD-D}. %GraphDE proposes to model the distribution shifts through a graph generative process and derives a posterior distribution for graph-level OOD detection~\citep{GraphDE}. Additionally, AAGOD presents a data-centric method for detecting graph-level OOD~\citep{AAGOD}.
% % which aims to learn structural patterns in the graph data through a learnable amplifier matrix
% %GOOD-D is an unsupervised method that considers purely the performance of graph-level OOD detection by applying contrastive learning~\citep{GOOD-D}. Other uncertainty estimation-based approaches have also been proposed for OOD detection~\citep{GOODD-uncertainty, GPN, GKDE}.

% \paragraph{OOD Exposure-Based OOD Detection.}
% OOD exposure is another prominent line of work that adopts auxiliary OOD data to assist training~\citep{OE, energy, Textual-OODExposure, DivOE, ATOL, SAL,GNNSafe}. The aforementioned \textsc{GNNSafe} model also considers an additional version \textsc{GNNSafe++} to adopt OOD exposure and has shown greater performance than standard model~\citep{GNNSafe}. \cite{NODESAFE} also presents \textsc{NODESafe++} as an extended OOD exposed version. \cite{GDE_OOD} proposes a generalised Dirichlet energy score for OOD detection. Our proposed GOLD method attempts to take advantage of the effectiveness of OOD exposure through synthesising samples that exhibit OOD characteristics. Thus, avoiding the necessity of real OOD data during training.

% \paragraph{OOD Generation.}
% Recent studies begin to work on synthesising OOD data~\citep{manifold, Likelihood, LRegret, GenAnalysis, GenUnknown, Hierarc,ConfOOD,VOS,NPOS}. A GAN-based approach is proposes to generate OOD data by jointly training a confidence classifier~\citep{ConfOOD}. VOS generates synthetic outliers from low-probability regions of multivariate Gaussian distributions~\citep{VOS}. Recently,pre-trained  diffusion models have been widely employed for OOD generation including DFDD~\citep{DFDD}, Dream-OOD~\citep{Dream-OOD}. Several initial graph-level OOD studies have been initiated, predominantly for molecule~\citep{MOOD-Molecule,MOL_DIF}. A score-based OOD molecule generation model is proposed by MOOD~\citep{MOOD-Molecule}, which employs an OOD-controlled reverse-time diffusion. A recent work PGR-MOOD~\citep{MOL_DIF} proposes to rely on a pre-trained molecule diffusion for generation. These methods typically rely on pre-trained models that are trained with additional data. In contrast, GOLD does not rely on pre-trained generative models to synthesise pseudo-OOD data.