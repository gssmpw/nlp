% This must be in the first 5 lines to tell arXiv to use pdfLaTeX, which is strongly recommended.
\pdfoutput=1
% In particular, the hyperref package requires pdfLaTeX in order to break URLs across lines.

\documentclass[11pt]{article}

% Change "review" to "final" to generate the final (sometimes called camera-ready) version.
% Change to "preprint" to generate a non-anonymous version with page numbers.
\usepackage{acl}

% Standard package includes
\usepackage{times}
\usepackage{latexsym}

% Draw tables
\usepackage{booktabs}
\usepackage{multirow}
\usepackage{xcolor}
\usepackage{colortbl}
\usepackage{array} 
\usepackage{amsmath}

\newcolumntype{C}{>{\centering\arraybackslash}p{0.07\textwidth}}
% For proper rendering and hyphenation of words containing Latin characters (including in bib files)
\usepackage[T1]{fontenc}
% For Vietnamese characters
% \usepackage[T5]{fontenc}
% See https://www.latex-project.org/help/documentation/encguide.pdf for other character sets
% This assumes your files are encoded as UTF8
\usepackage[utf8]{inputenc}

% This is not strictly necessary, and may be commented out,
% but it will improve the layout of the manuscript,
% and will typically save some space.
\usepackage{microtype}
\DeclareMathOperator*{\argmax}{arg\,max}
% This is also not strictly necessary, and may be commented out.
% However, it will improve the aesthetics of text in
% the typewriter font.
\usepackage{inconsolata}

%Including images in your LaTeX document requires adding
%additional package(s)
\usepackage{graphicx}
% If the title and author information does not fit in the area allocated, uncomment the following
%
%\setlength\titlebox{<dim>}
%
% and set <dim> to something 5cm or larger.

\title{Wi-Chat: Large Language Model Powered Wi-Fi Sensing}

% Author information can be set in various styles:
% For several authors from the same institution:
% \author{Author 1 \and ... \and Author n \\
%         Address line \\ ... \\ Address line}
% if the names do not fit well on one line use
%         Author 1 \\ {\bf Author 2} \\ ... \\ {\bf Author n} \\
% For authors from different institutions:
% \author{Author 1 \\ Address line \\  ... \\ Address line
%         \And  ... \And
%         Author n \\ Address line \\ ... \\ Address line}
% To start a separate ``row'' of authors use \AND, as in
% \author{Author 1 \\ Address line \\  ... \\ Address line
%         \AND
%         Author 2 \\ Address line \\ ... \\ Address line \And
%         Author 3 \\ Address line \\ ... \\ Address line}

% \author{First Author \\
%   Affiliation / Address line 1 \\
%   Affiliation / Address line 2 \\
%   Affiliation / Address line 3 \\
%   \texttt{email@domain} \\\And
%   Second Author \\
%   Affiliation / Address line 1 \\
%   Affiliation / Address line 2 \\
%   Affiliation / Address line 3 \\
%   \texttt{email@domain} \\}
% \author{Haohan Yuan \qquad Haopeng Zhang\thanks{corresponding author} \\ 
%   ALOHA Lab, University of Hawaii at Manoa \\
%   % Affiliation / Address line 2 \\
%   % Affiliation / Address line 3 \\
%   \texttt{\{haohany,haopengz\}@hawaii.edu}}
  
\author{
{Haopeng Zhang$\dag$\thanks{These authors contributed equally to this work.}, Yili Ren$\ddagger$\footnotemark[1], Haohan Yuan$\dag$, Jingzhe Zhang$\ddagger$, Yitong Shen$\ddagger$} \\
ALOHA Lab, University of Hawaii at Manoa$\dag$, University of South Florida$\ddagger$ \\
\{haopengz, haohany\}@hawaii.edu\\
\{yiliren, jingzhe, shen202\}@usf.edu\\}



  
%\author{
%  \textbf{First Author\textsuperscript{1}},
%  \textbf{Second Author\textsuperscript{1,2}},
%  \textbf{Third T. Author\textsuperscript{1}},
%  \textbf{Fourth Author\textsuperscript{1}},
%\\
%  \textbf{Fifth Author\textsuperscript{1,2}},
%  \textbf{Sixth Author\textsuperscript{1}},
%  \textbf{Seventh Author\textsuperscript{1}},
%  \textbf{Eighth Author \textsuperscript{1,2,3,4}},
%\\
%  \textbf{Ninth Author\textsuperscript{1}},
%  \textbf{Tenth Author\textsuperscript{1}},
%  \textbf{Eleventh E. Author\textsuperscript{1,2,3,4,5}},
%  \textbf{Twelfth Author\textsuperscript{1}},
%\\
%  \textbf{Thirteenth Author\textsuperscript{3}},
%  \textbf{Fourteenth F. Author\textsuperscript{2,4}},
%  \textbf{Fifteenth Author\textsuperscript{1}},
%  \textbf{Sixteenth Author\textsuperscript{1}},
%\\
%  \textbf{Seventeenth S. Author\textsuperscript{4,5}},
%  \textbf{Eighteenth Author\textsuperscript{3,4}},
%  \textbf{Nineteenth N. Author\textsuperscript{2,5}},
%  \textbf{Twentieth Author\textsuperscript{1}}
%\\
%\\
%  \textsuperscript{1}Affiliation 1,
%  \textsuperscript{2}Affiliation 2,
%  \textsuperscript{3}Affiliation 3,
%  \textsuperscript{4}Affiliation 4,
%  \textsuperscript{5}Affiliation 5
%\\
%  \small{
%    \textbf{Correspondence:} \href{mailto:email@domain}{email@domain}
%  }
%}

\begin{document}
\maketitle
\begin{abstract}
Recent advancements in Large Language Models (LLMs) have demonstrated remarkable capabilities across diverse tasks. However, their potential to integrate physical model knowledge for real-world signal interpretation remains largely unexplored. In this work, we introduce Wi-Chat, the first LLM-powered Wi-Fi-based human activity recognition system. We demonstrate that LLMs can process raw Wi-Fi signals and infer human activities by incorporating Wi-Fi sensing principles into prompts. Our approach leverages physical model insights to guide LLMs in interpreting Channel State Information (CSI) data without traditional signal processing techniques. Through experiments on real-world Wi-Fi datasets, we show that LLMs exhibit strong reasoning capabilities, achieving zero-shot activity recognition. These findings highlight a new paradigm for Wi-Fi sensing, expanding LLM applications beyond conventional language tasks and enhancing the accessibility of wireless sensing for real-world deployments.
\end{abstract}

\section{Introduction}\label{sec:introduction}
% -- Outline
% ---- LLMs are popular
% ---- There're many stakeholders in the training and inference loop
% ---- Adversaries in the training loop are a problem -- malpractice, poisoning
% ---- Also, showing compliance
% ---- Need a framework to prove the integrity of the pipeline
% ---- Enter Atlas

% ---- LLMs are popular
In recent years, machine learning (ML) models, have become increasingly popular.
The pervasive use of large language models (LLMs), in particular, and multi-stakeholder
involvement in model creation and deployment exacerbate security and privacy risks.
These considerations are emphasized by the global nature and the complexity of
large-scale ML deployments with different lifecycle stages:
%gathering and sanitizing the data from different sources,
%training and inferencing across many data centers,
%compliance with local laws or corporate policies.

% ---- There're many stakeholders in the training and inference loop
%Additionally, different stages of the ML development pipeline come with their own stakeholders:
\begin{enumerate}[label=\arabic*)]
    \item Collection and sanitation of a \emph{training} dataset from several public and proprietary sources.
    %\item Solicitation and facilitation of training.
    \item Provisioning of the training environment (hardware and software).
    \item Execution of training across many data centers.
    \item Construction of a \emph{testing} dataset from several sources, and the evaluation.
    \item Deployment and use of the model for inference that is compliant with local laws or corporate policies.
    %\item Use of the model in compliance with local laws or corporate policies.
\end{enumerate}

% ---- Adversaries in the training loop are a problem -- malpractice, poisoning
Each of these stages is vulnerable to malicious or dishonest parties.
For example, data can be poisoned~\cite{biggio2012poisoning,carlini2024poisoning} during collection or training.
Service providers executing outsourced training can shorten or omit critical steps to reduce their cost.
Model providers can serve smaller models in SaaS, or even distribute malicious ones.

% ---- Also, showing compliance
On the other hand, responsible model builders and other stakeholders may be incentivised or required to provide security and trust guarantees.
They may want to prove low bias in their training data, offer easily verifiable performance claims, or guarantee end-to-end integrity of the model creation in high risk domains.

% ---- Need a framework to prove the integrity of the pipeline
To address these challenges, it is necessary to guarantee the integrity of the entire ML lifecycle --
beginning with the data, through the training, and finally, the evaluation and deployment.
Was the data modified?
Did the hardware and software environment adhere to the specification?
Did the contractor follow the specified training procedure?
Can I trust the evaluation?
How can I guarantee that I am interacting with the intended model?
These are example questions that showcase the breadth of the involved challenges that must be tackled to provide end-to-end security.

% --- Enter Atlas
In this work, we introduce \atlas, a framework for enhancing the security and transparency of the lifecycle of ML models.
\atlas establishes the baseline of fundamental components and capabilities needed for comprehensive provenance tracking
at each stage of the ML lifecycle.
Subsequently, \atlas defines the core integrity requirements for verifiable ML lifecycle transparency.
We provide a reference implementation that instantiates \atlas using hardware-based security mechanisms -- with trusted execution environment (TEE),
including attestations.% , and comprehensive metadata-based provenance tracking.
%Our implementation satisfies all \atlas requirements.

We claim the following contributions:
\begin{enumerate}[label=\arabic*.]\label{sec:introduction:contributions}
    \item We introduce \atlas, a framework designed for end-to-end ML lifecycle transparency.
    \item We instantiate \atlas using TEEs and metadata-based provenance tracking.
    \item We evaluate our \atlas prototype through two case studies:
        \begin{enumerate*}[label=\arabic*)]
            \item fine-tuning of a BERT model~\cite{lin2023metabert, lin2023metabertimpl};
            \item fine-tuning of a bge-reranker model~\cite{chen2023bge}
        \end{enumerate*}.
\end{enumerate}

%\msm{revise: Integrate this motivation into intro}
%Organizations frequently leverage pre-trained models, outsource training processes, and integrate components from multiple sources,
%making it difficult to verify the authenticity and trustworthiness of their ML systems. This complexity is further compounded
%by the potential for malicious modifications at various stages of the model lifecycle, from data preparation through deployment.
%The involvement of various third parties in ML model development and deployment
%creates critical challenges in ensuring supply chain integrity.
%
%While Software Bills of Materials (SBOMs) and AI Bills of Materials (AI BOMs) provide basic inventory tracking for model components,
%they fall short in addressing the dynamic nature of ML pipelines. These approaches typically offer point-in-time snapshots but
%fail to capture the complex transformations, fine-tuning operations, and runtime modifications that characterize modern ML workflows.
%Additionally, they lack cryptographic guarantees about the integrity of recorded information and cannot effectively track the provenance
% of model weights and training data.
%
% These approaches demonstrate the growing importance of ML supply chain security.
% However, they are typically applied in an ad-hoc fashion, highlighting the need
% for a more integrated approach that combines comprehensive lineage tracking,
% strong cryptographic properties, and practical integration capabilities with existing ML development and deployment pipelines.
%
%A comprehensive solution requires not just documentation of components, but verifiable evidence of their origins,
%transformations, and integrity throughout the entire model lifecycle. This need has driven interest in more robust
%provenance tracking mechanisms that can:
%
%\begin{itemize}
%\item Provide cryptographic proof of model lineage
%\item Track and verify all pipeline transformations
%\item Maintain tamper-evident records of training processes
%\item Ensure integrity of model artifacts across organizational boundaries
%\end{itemize}
%
%Several existing tools and frameworks
%commonly focusing on different components of the model lifecycle and provenance tracking.
%While these solutions offer valuable capabilities, they often address only specific parts of the end-to-end ML
%supply chain rather than providing comprehensive coverage.
%\msm{end-revise}
%
%\todo{add discussion of EU-CRA AI Act requirements for model documentation and FDA guidelines for AI/ML in healthcare}

%The remainder of this paper is organized as follows:
%in Section~\ref{sec:background-related} we provide an overview of the necessary background, and the related work;
%Section~\ref{sec:problem} presents the challenge of providing integrity in the ML pipeline, the threat model, and the system assumptions;
%in Section~\ref{sec:framework} we present \atlas -- our framework for providing ML integrity;
%Section~\ref{sec:implementation} covers implementation details;
%in Section~\ref{sec:eval}, we show that \atlas is effective across three dimensions: training overhead $<8\%$, the verification time increases linearly with the size of the model, and it is compatible with PyTorch and Tensorflow;
%in Section~\ref{sec:casestudies} we present the case studies;
%in Section~\ref{sec:discussion} we discuss additional considerations for \atlas,
%and Section~\ref{sec:conclusion} concludes the paper and provides directions for future work.

\section{Related Work}
\label{sec:related}

Recent advances~\cite{lecun2015deep, zaidi2022survey} in deep learning have vastly improved object detection and instance segmentation results in the terrestrial domain. 
Such progress has been achieved by developing effective designs of models and training them with large datasets~\cite{lin2014microsoft, russakovsky2015imagenet} containing millions of images and corresponding labels. 
Even with such advances, detecting underwater debris still remains challenging. 
While~\cite{fulton2019robotic} presents the first deep learning based approach to detect underwater debris and outperforms previous non deep learning approaches, the accuracy is worse than general object detection tasks due to a small training dataset. 
To increase the debris detection accuracy,~\cite{hong2020trashcan} proposes a larger dataset, TrashCan, which has both bounding box and pixel-level annotations for object detection and instance segmentation along with baseline results using Mask R-CNN~\cite{he2017mask} and Faster R-CNN~\cite{ren2015faster}. 
However, increasing the dataset size to improve debris detection accuracy further is not scalable due to debris data scarcity and labeling costs. 
To overcome the data scarcity issue,~\cite{hong_generative_2020} proposes a generative method, augmenting the existing dataset with synthetic underwater debris images. 
While the method can create realistic synthetic images, it still requires additional labeling efforts to be used for training detectors. 

Style transfer~\cite{singh_neural_2021,jing_neural_2020} is an approach for changing the appearance of one image based on the visual style of another. 
\cite{rodriguez_domain_2019, yu_sc-uda_2022} use this to improve detection in images taken from various domains (\eg different light conditions and image clarity). 
They aim to account for low-level texture changes in images by updating them to have the same style throughout the data. 
\cite{kadish_improving_2021} also attempts to improve detection using style transfer, by having the detector learn high-level features (\eg object shape) instead of low-level features (\eg the texture of paintings). 
\cite{amirkhani_enhancing_2021} uses style transfer to simulate various types of noise that may be present in real-world data. 
\cite{lin_gan-based_2021,liu_lane_2020} use style transfer to imitate varying light conditions. 
Style transfer has been applied beyond RGB images; \eg\cite{cygert_style_2019} converts RGB images from COCO dataset~\cite{lin2014microsoft} to thermal images and uses them to train a thermal image detector. 
While style transfer works well in augmenting the appearance of an image, it does not add new objects to our data.


Unlike style transfer, image blending based methods allow placing new objects anywhere on target background images. 
\cite{perez_poisson_2003} introduces Poisson editing using Laplacian information to smooth the boundary between the image patches and target images. 
\cite{wu_gp-gan_2019} uses a GAN-based approach for image blending, producing realistic images; however, it requires image pairs of empty backgrounds and objects placed in the backgrounds to train, limiting its use when the source data is limited. 
\cite{georgakis_synthesizing_2017} modifies~\cite{perez_poisson_2003} to find spaces within a given image plane to blend an object. 
However, detectors trained with their synthetic data show degraded performance on real data due to the style discrepancy between the blended objects and backgrounds in the dataset.
\cite{zhang_training_2022} uses a harmonization blending approach to create new data for aerial search and rescue, but it does not blend the boundary of target objects. 

\cite{zhang_deep_2020} presents a two-stage deep network-based approach to blend an image patch onto a background. Unlike~\cite{wu_gp-gan_2019} their approach does not need additional training data to generate blended images.
\begin{figure}  
    \centering
    \scalebox{0.75}{\tikzset{every picture/.style={line width=0.75pt}} 

\begin{tikzpicture}[x=0.75pt,y=0.75pt,yscale=-1,xscale=1]
\draw (-490,101) node  {\includegraphics[width=0.25\textwidth]{imgs/IBURD_firstpass.png}};
\draw (-310,101) node  {\includegraphics[width=0.25\textwidth]{imgs/DIB_secondpass.png}};
\draw (-130,101) node  {\includegraphics[width=0.25\textwidth]{imgs/IBURD_secondpass.png}};

\draw (-560,192) node [anchor=north west][inner sep=0.75pt]   [align=left] {{\fontfamily{helvet}\selectfont Poisson Image Editing}};
\draw (-380,192) node [anchor=north west][inner sep=0.75pt]   [align=left] {{\fontfamily{helvet}\selectfont Deep Image Blending}};
\draw (-180,192) node [anchor=north west][inner sep=0.75pt]   [align=left] {{\fontfamily{helvet}\selectfont IBURD (Ours)}};


\end{tikzpicture}}
    \caption{Comparison of generated images using three approaches:  Poisson image editing~\cite{perez_poisson_2003}, Deep image blending~\cite{zhang_deep_2020} and our method, IBURD. In our approach, we can successfully prevent over-stylization of the blended objects.}  
    \label{fig:compare}
    \vspace{-4mm}
\end{figure}

They use the proposed method mainly for artistic purposes and it struggles with blending transparent source images onto background images, as seen in Fig.~\ref{fig:compare}. 
The method is only tested with $20$ images and takes approximately $4$ minutes to blend one object in an image of size $512\times512$ pixels.

Our proposed approach, IBURD, allows us to place source images at various locations and scales in target background images with relevant bounding box and pixel-level annotations within $50$ seconds, which is $5$ times faster than~\cite{zhang_deep_2020}. 
Our method addresses blending transparent objects using Poisson editing, a situation that previous methods fail to cover.
Additionally, IBURD deals with object distortion due to excessive style transfer using Fast Fourier Transform (FFT)~\cite{liu_image_2008} based weight adjustment for loss.

\section{Method}
\label{sec:method}

\subsection{Weaknesses of Previous Conditioning Methods}

The most popular form of latent image conditioning typically converts conditioning signals to images, before processing them with typical image processing models. While this approach is powerful, it exhibits limitations in handling complex image synthesis tasks, particularly when incorporating heterogeneous or sparse input conditions. Some approaches, such as \textit{LayoutDiffusion} \cite{zheng_layoutdiffusion_2024}, tackle this with custom attention modules that attend to bounding boxes with learned positional embeddings. However, these approaches neglect to include multiple modalities and the relationships between them, which overlooks nuanced interactions between conditioning signals i.e. disambiguating spatial ordering between overlapping boxes. 

% For example, interactions between conditions which may not explicitly exist in the discrete spatial image domain.

% These approaches force diverse modalities, like mixed spatial and categorical information directly into a unified image space, which overlooks nuanced interactions between conditioning signals. For example, interactions between conditions which may not explicitly exist in the discrete spatial image domain.

Previous conditional diffusion research that utilise graph data opt for complex multi-stage training procedures such as masked contrastive pre-training using graph triplets \cite{yang_diffusion-based_2022}. This is not only time-consuming, but also fails to exploit potential benefits of training an end-to-end system that integrates graph data directly into image processing. 
% Furthermore, other work has shown that the repeated conditioning diffusion models (i.e. time or text conditioning) is superior to simply providing   

We tackle these problems by representing images and their conditioning signals as a single graph, which is processed by a bespoke GNN architecture. This allows repeated interactions between conditioning signals and the image throughout the synthesis process, enabling more flexible and dynamic representations that account for both the current image features and interactions between conditioning signals. By maintaining separate pathways for distinct input types, our approach supports heterogeneous and sparse conditioning, leading to better generalisation, finer control, and more precise manipulation of generated images. This simple yet powerful method can be easily integrated into a wide range of existing vision models.

\begin{figure}
    \centering    \includegraphics[width=1\linewidth]{icml2023/hig_fig2.pdf}
\vspace{-20pt}
    \caption{(\textbf{a}) Overview of the proposed architecture. The HIG is encoded into a latent representation through a MP-GNN which is then used as a condition $c_f$ in a ControlNet. (\textbf{b}) Details of the MP-GNN module. Note: HMP is shorthand for heterogenous magnitude preserving operations applied across all nodes.}
    \label{fig:architecture}
\end{figure}

\subsection{Heterogeneous Image Graphs}

To improve on previous approaches we develop a new approach to condition images via the HIG representation. In this manner, we fully exploit variable-length and heterogeneous conditions to aid in image synthesis.

\textbf{Image Graphs.} When faced with the challenge of conditioning images with graphs we first convert images into representations amenable for graph processing. We reshape image features into image nodes pixel-wise in line with other works \cite{liu_cnn-enhanced_2021, han_vision_2022}. In practice, these nodes represent more than a single pixel, for example a latent image patch. This can be due to performing latent image diffusion \cite{rombach_high-resolution_2022, podell_sdxl_2023} where images are first pre-compressed to latent images, or due to prior processing by the image processing model. In contrast to other works \cite{tian_image_nodate, han_vision_2022, tarasiewicz_graph_2021}, we decide to leave image nodes unconnected; this loosely decouples image conditioning from processing. Image nodes are conditioned and later converted back into an image representation, allowing existing architectures to handle processing. Connecting image nodes in a locally dense fashion gains little benefit over highly optimised $3 \times 3$ convolutional operations. Formally, image nodes exist in a discrete space \( f : \mathbb{Z}^2 \to \mathbb{R}^C \). For an image of size \(M \times N\), we define \( f(i, j) \) where \( i, j \in \mathbb{Z} \) and \( 0 \leq i < M \), \( 0 \leq j < N \).

\textbf{Conditioning Graphs}. Conditioning graphs consist of nodes and edges, where each node has features defined as $ g : \mathcal{V} \to \mathbb{R}^F$, where $\mathcal{V}$ represents the set of nodes and $\mathbb{R}^F$ the feature space. Nodes may have spatial ties to the image domain, which we materialise via edges linking image and conditioning nodes. We use conditioning nodes to indicate semantics within the scene, for instance, a node may represent an object (e.g., a \textit{person}). Whereas we utilise different edge types to represent both spatial, abstract relationships and additional semantics. For instance, an edge between two object nodes may encode interactions or attributes (e.g., a person \textit{wearing} a {\textit{yellow}} hat). The graph structure reflects real-world data: often sparse and heterogeneous. We therefore construct graphs on a per task-basis to best leverage the available data and its dependencies.
Formally, each edge \( e \in \mathcal{E} \) connects two nodes \( (v_i, v_j) \in \mathcal{V} \times \mathcal{V} \) and represents a relationship between them. Edges represent any dependency, allowing for abstract relationships to be included.

% To continue the example, if spatial information for both the \textit{person} and the \textit{hat} is available, the graph would contain a node for each object and an edge connecting them, with the edge encoding the relationship \textit{wearing}. 


% \textbf{Conditioning Graphs.} In contrast, conditioning graphs are represented by sets of nodes and edges, with each node having associated features defined by a function $( g : \mathcal{V} \to \mathbb{R}^F$, where $\mathcal{V}$ represents the set of nodes and $\mathbb{R}^F$ the feature space. Although nodes \textit{may} have explicit spatial ties to the discrete image domain, we materialise these through edges between image and conditioning nodes. However, these relationships may be the product of spatial properties of conditioning nodes. As such, subsets of $\mathbb{R^F}$ may represent spatial coordinates \( (x, y) \in \mathbb{R}^2 \) that satisfy \( 0 \leq x < M \) and \( 0 \leq y < N \). Conditioning nodes are not restricted to pixel grid positions, nor the number of spatial dimensions e.g. nodes may represent 3D properties of the real world. Nodes and edges may represent properties independent of spatial dimensions. For example, nodes in the graph can represent concrete objects in the image (e.g., a \textit{person}), while edges between them may represent abstract interactions or attributes (e.g., a person \textit{wearing} a {\textit{yellow}} hat). The graph structure may be sparse, and heterogeneous (multiple types of nodes and edges). Conditioning graphs are constructed on a per-task basis to optimally leverage available data and its dependencies. Formally, each edge \( e \in \mathcal{E} \) connects two nodes \( (v_i, v_j) \in \mathcal{V} \times \mathcal{V} \) and represents a relationship between them. To continue the example, if spatial information for both the \textit{person} and the \textit{hat} is available, the graph would contain a node for each object and an edge connecting them, with the edge encoding the relationship \textit{wearing}. Edges can represent any dependency, allowing for abstract relationships to be included in the graph.

\textbf{Connecting Image and Conditioning Nodes.} With image and conditioning nodes defined, we are close to the complete HIG representation. To enable conditioning between the image and conditioning graphs, we must construct edges between the two. These connections are determined on a per-task basis, depending on the available data, with explicit choices described in Section 4. However, when spatial information is available i.e. segmentation masks or bounding boxes, it enables direct connections between the image graph and the conditioning graph. Specifically, edges are created between image nodes relevant to spatial conditionings (i.e. pixels within the bounding box) and conditioning nodes representing the corresponding semantic class (i.e. class label). This linkage facilitates information flow across the graphs, integrating pixel-level details with higher-level semantic representations. 

% Additionally, the flexibility of heterogeneous GNNs allows for connections from the image back to the graph with different sets of learned weights. This approach enables the image to influence the graph structure while leveraging the rich semantic details present in the image—such as color or object sub-class—throughout much of the diffusion training scheme, while still respecting the different types of information carried by the node types.

\subsection{Model Architecture}

To be compatible with the EDM2 U-Net architecture \footnote{\href{https://github.com/NVlabs/edm2}{https://github.com/NVlabs/edm2}}, we propose the addition of a magnitude-preserving \textit{Heterogenous Image Graph Neural Network} (HIGnn) as the conditioning network to be used in a ControlNet strategy.

\textbf{HIGnn.} The general architecture of the HIG conditioning block requires two primary capabilities: representation switching and HIG processing. To handle switching between image features and image nodes on the HIG we consider the update function $\mathcal{U}_{\text{i}\rightarrow\text{g}}$. This update functions reshapes image features $\mathbf{x_i} \in \mathbb{R}^{N \times C \times H \times W}$ into image nodes pixel wise $\mathbf{x_g} \in \mathbb{R}^{N\cdot H \cdot W \times C}$ and applies an optional projection to ensure correct dimensionality. For the current set of image pixels $\mathbf{x_i}$, we retrieve HIG image nodes $\mathbf{x_g}$ by
\begin{equation}
\mathbf{x_g} = \mathcal{U}_{\text{i}\rightarrow\text{g}}(\mathbf{x_i}) = \hat{W}R(\mathbf{x_i}),  
 \label{eq:HIG_update}
\end{equation}
where $R$ reshapes the image, and $\hat{W}$ is a learned projection with forced magnitude preservation from \cite{karras_analyzing_2024}. Refer to Appendix \ref{appendix:edm2_preliminaries} for greater detail into the mathematical preliminaries of \cite{karras_analyzing_2024}. We consider the reverse operation of converting from graph nodes to an image $\mathcal{U}_{\text{g}\rightarrow\text{i}}$ in a similiar fashion. 

Once we have the HIG updated with current image nodes we can process it with a GNN. We identify several areas where magnitudes can grow and address them each in turn. In practice many varieties of heterogenous message passing GNN could be used, we create our own magnitude preserving graph convolutional operator similiar to Hamilton et al. \cite{hamilton_inductive_2018} for its simplicity and stability. The basic approach propagates information through two branches, a pseudo `skip-connection' applied to the current node, and a learned pooling operation of the local neighbourhood, and we add the ability to include edge information in the neighbourhood pooling. If edge attributes $\mathbf{a}_i$ are present we integrate them via magnitude preserving concatenation to the pooling branch. Formally, the HIGConv operator applied per meta-path to get updated node embeddings $\mathbf{x}_i'$ is defined as:
% \begin{equation}
%     \mathbf{x}_i' = \psi\left(\hat{W}^{\Phi}_1 \mathbf{x}_i +^\text{mp} \hat{W}^{\Phi}_2 \cdot \frac{1}{\sqrt{|\mathcal{N}^{\Phi}|}} \sum_{j \in \mathcal{N}^{\Phi}(i)} [\mathbf{x}_j \|^\text{mp} \mathbf{a}_j] \right),
%     \label{eq:hignn_operator}
% \end{equation}
\begin{equation}
    \mathbf{x}_g' = \psi\left(\hat{W}^{\Phi}_1 \mathbf{x}_g 
    \underset{0 \text{ if } |\mathcal{N}^{\Phi}(i)| = 0}{\underbrace{+^\text{mp} \hat{W}^{\Phi}_2 \cdot \frac{1}{\sqrt{|\mathcal{N}^{\Phi}(i)|}} \sum_{j \in \mathcal{N}^{\Phi}(i)} [\mathbf{x}_j \|^\text{mp} \mathbf{a}_j]}}\right)    \label{eq:hignn_operator}
\end{equation}

% \[
%     \mathbf{x}_i' = \psi\left(\hat{W}^{\Phi}_1 \mathbf{x}_i +^\text{mp} 
%     \underset{+ 0 \text{ if } |\mathcal{N}^{\Phi}(i)| = 0}{\underbrace{\hat{W}^{\Phi}_2 \cdot \frac{1}{\sqrt{|\mathcal{N}^{\Phi}(i)|}} \sum_{j \in \mathcal{N}^{\Phi}(i)} [\mathbf{x}_j \|^\text{mp} \mathbf{a}_j]}}\right).
% \]

where we choose $\psi$ to be magnitude preserving SiLU operator, and $+^\text{mp}$ the magnitude preserving sum (See Appendix \ref{appendix:edm2_preliminaries}), and both meta-path weights $\hat{W}^{\Phi}_1$ and $\hat{W}^{\Phi}_2$ have forced magnitude. $\mathcal{N}$ indicates the local node neighbourhood and is defined by the connectivity of graph. In order to achieve magnitude preservation we first assume all neighbourhood features to be of unit length, we then summate them scale them by the square root of the neighbourhood size ($\sqrt{|\mathcal{N}^{\Phi}|}$), see Appendix \ref{appendix:sum_random} for details. It is important to address unconnected or `zero-degree' nodes, in this case we ignore the right hand side of the equation, and only take the residual path. Note that simply setting the  neighbourhood to zero unintentionally changes the feature magnitudes when mp-sum is applied, since it assumes both vectors to be of unit length. Finally to combine information across meta-paths, we use the same method and sum across paths before normalising by the inverse square root of the number of incoming meta-paths ($|\Phi_i| = |\{\Phi_k \mid x_i \in \Phi_k\}|$)

% To formulate a heterogeneous GNN with learned projections per meta-path ($\mathbf{\Phi} = \{\Phi_1 ... \Phi_n\}$), we must preserve magnitudes when combining meta-paths.

\begin{equation}
\Tilde{\mathbf{x}}_g = \frac{1}{\sqrt{|\Phi_g|}} \sum_{\Phi \in \Phi_g} \mathbf{x}'_g,
\label{eq:meta_path}
\end{equation}

We verify that this approach is guaranteed to maintain magnitudes under certain conditions of the underlying graph data. In particular, for graph-data of sufficient size this approach holds for graphs which do not have identical features attached to the same node since this breaks the independence assumption. 

% An interesting interpretation of this formulation with respect to image synthesis is to observe how different receptive fields change. The typical convolutional operator used in U-Net models define a local image receptive field $\mathcal{R}$, self-attention  defines a global image receptive field $\mathcal{A}$, and the HIGnn defines receptive fields over meta-path relationships $\mathcal{N}^{\Phi}$ for both the image and conditioning variables. We postulate this to an advantage over other conditioning methods as it allows instant communication between different conditioning signals and parts of the image whilst remaining computationally tractable.

\textbf{EDM2 ControlNet Integration.} To integrate conditioning into a generative model, we adopt a strategy similar to ControlNet \cite{zhang_adding_2023}, i.e. a frozen EDM2 pre-trained model, with a trainable copy the encoder integrated with the conditioning HIGnn. Refer to Figure \ref{fig:architecture} for an overview of our proposed architecture, we employ 4 HIG blocks for our base model. The EDM2 checkpoints are only available for class-conditional generation of the 1000 ImageNet classes, yet we find them easy to adapt to our natural image datasets.  To facilitate this we unfreeze the embedding network. To integrate features we adopt $1\times1$ convolutions with a learnable zero-gain in a similar fashion to the original ControlNet, but we note that traditional summation may damage feature magnitudes. We find that naively integrating is harmful to training. Instead, we apply magnitude preserving summation, which, in contrast to the original ControlNet paper, directly alters the primary network features. This yields poor generative quality at step 0, but proves to be quick to train and to be best in practice.

In the trainable encoder we integrate our proposed HIGnn after the initial convolution block. We opt to keep the dimension of the GNN matched to that of the generative model. Finally, to generate samples we opt for the non-stochastic EDM2 sampler, and use the recent advancements in auto-guidance \cite{karras_guiding_2024}, we use our control model as the primary network, and use the unconditional XS ImageNet checkpoint released with EDM2 as the guidance network \cite{karras_analyzing_2024, karras_guiding_2024}. 

% We do not use EMA
\section{Dataset Generation}
\label{sec:dataset}
\revise{
To train the proposed GNN, we constructed a dataset of building structures and a subset of these structures were subjected to fire simulations using FEA. The dataset generation process is illustrated in \figref{fig:dataset_generation_procedure}. Initially, a total of 33,000 building structures with geometrical details, material properties, and gravity loads were created. Due to randomness in generating these structures, a filter is applied to remove unreasonable data after gravity load simulation, which included 15,377 structures. A trade-off between computational feasibility and model performance is made among the remaining 17,623 structures. As further labeling structures with MIDR requires resource-intensive fire simulations via OpenSeesRT, a large proportion of 16,050 structures is selected as unlabeled dataset. On the other hand, each of the other 1,573 structures was further subjected to 30 different fire simulations, forming the labeled dataset containing $1,573\times 30 = 47,190$ fire cases.} This section details the step-by-step process for generating the dataset, including geometry creation, material property assignment, and simulations due to gravity loads and fire scenarios. 
% To train the proposed neural network, we constructed a dataset comprising building structure data and a subset of fire scenario data. The dataset generation process is illustrated in \figref{fig:dataset_generation_procedure}. 
% A total of 33,000 building structures with geometric details, material properties, and gravity loads were initially created. Out of these, 3,000 structures were selected as labeled data, and the remaining 30,000 were designated as unlabeled data. Further, about half of them filtered out due to instability under gravity loads only. 
\begin{figure*}[h!]
    \centering
    \includegraphics[width=0.8\linewidth]{figures/dataset_filter_procedure.pdf}
    \caption{Workflow for dataset generation (geometry, material property, gravity loads, and fire scenarios).}
    \label{fig:dataset_generation_procedure}
\end{figure*}

\subsection{Geometry Generation}
\label{subsec:geometry_generation}
The geometry of the building structures forms the foundation of the dataset. Regular 
\revise{3D structures} resembling multi-story parking structures or shopping malls were generated, with parameters such as building floor dimensions and story heights selected randomly. Each building structure is composed of multiple rooms, which serve as the basic unit in this study. A room herein is a cuboid space defined by specific length, width, and height. Within a structure, rooms of the same dimensions are uniformly arranged along the length, width, and height, corresponding to the $x$-, $y$-, and $z$-axes, respectively. Structures vary in room size and number of rooms along each axis. Specifically, the room length, width, and height are independently sampled from a uniform distribution within the interval $[2, 5]$ meters along the three directions of the structure. Similarly, the room number along each axis is uniformly sampled independently as an integer within the interval $[2, 7]$, i.e., the maximum number of stories of the buildings simulated in this study is 7.

To introduce variability and simulate real-world scenarios, approximately $8\%$ of structural elements (beams or columns) are randomly removed after initial geometry creation. 
\revise{Such removal is not fire-induced damage, but reflects functional diversity often observed in real buildings, such as open spaces designed for activities in shopping malls, e.g., ice skating rinks. Examples of the generated geometries are illustrated in \figref{fig:example_generated_geometry}, showcasing the diversity and realism of the dataset. This element removal does not affect the definition of room's geometry in the structure and nor does it affect the number of considered fire scenarios.} 

\revise{A range of coefficient of variation values ($3.3\%$ to $17.5\%$) was derived from prior studies that investigated the statistics of geometrical and material properties of structural components of buildings (e.g., \cite{mirza1979variations, lee2004probabilistic}). These studies provide empirical data on the natural variability in parameters such as Young's modulus, yield strength, and dimensions of structural elements due to manufacturing tolerances and material inconsistencies. By selecting $8\%$ for the removal of structural elements in our database, we aimed to maintain a level of variability that is representative of real-world uncertainties while ensuring computational feasibility. This choice ensures that the database captures realistic deviations without introducing extreme cases that may not be commonly encountered in practice.}

\begin{figure*}[h!]
    \centering
    \includegraphics[width=\linewidth]{figures/example_generated_geometry.pdf}
    \caption{Examples of generated structural geometry of different sizes (all dimensions in meters).}
    \label{fig:example_generated_geometry} 
\end{figure*}

{\blockRevise

In this study, we opted for a deterministic square, dimension of $0.1$ m, solid cross-sectional steel elements due to their simplicity in modeling and analysis. Square sections exhibit uniform geometrical properties in all directions, simplifying the computation of structural responses and avoiding complications associated with more complex shapes, such as wide-flange sections, facilitating the computational efficiency and scalability to generate a large dataset. This choice also helps to mitigate issues related to stress concentrations and facilitates a more straightforward representation of structural behavior under thermal loads. 

\textit{Remark:} The selected cross-section provides a comparable flexural rigidity to a $W 130 \times 130 \times 28.1$ wide-flange section (metric units), albeit with significantly higher axial rigidity. This cross-section is acceptable for gravity-load-designed frames under service loading conditions where the models assume fully rigid, moment-resisting beam-column connections for the evaluation of the IDR under thermal loading. This assumption is reasonable in this computational study where the primary interest is to understand the global deformation response of frames under fire conditions. The selection of uniform square cross-sections for both beams and columns, rather than adherence to standard capacity design principles, was made here primarily for computational efficiency and to reduce design parameters in the database generation process. This choice allows for simplified and scalable approach to analyze the fire-induced response of generic steel frames without the need for large section variations, where this study mainly focuses on the fire vulnerability assessment using ML-based predictions. However, if additional loading conditions, e.g., seismic or wind loads, were to be considered, larger sections, strong-column/weak-beam principle, and ductile detailing would be required in the generated buildings for realistic structural behavior under combined loading conditions. Future studies may also consider investigating the influence of variable cross-sectional dimensions and semi-rigid connections on the structural performance under fire conditions. 
} % blockRevise

\subsection{Material Properties}
Steel is chosen as the material for the structures. To reflect real-world variations, we randomly assign one of five slightly different steel material types to each structural element. \revise{
The ranges of material properties are provided in \tabref{tab:material_property_ranges} and the properties are sampled from uniform distributions of the corresponding ranges. These variations simulate differences arising from manufacturing batches or regional material properties. That these properties are at ambient temperature and change when the temperature rises due to a fire. The selection of materials with varying properties is aimed at increasing the diversity of the data. Our goal is to represent as wide a range of data as possible with a limited amount of building structure data, thereby enhancing the generalization ability of the GNN. Our assumed material property ranges are expected to be wider than the real-world conditions based on findings in \cite{mirza1979variations, lee2004probabilistic}. Therefore, we are essentially tackling a more challenging and general task. If we can solve this problem, we are confident that our method will perform equally well or even better in real-world scenarios.
}
\begin{table}[h!]
    \centering
    \caption{Material properties ranges for considered steel structures.}
    \begin{tabular}{lc}
        \toprule
        Property & Range \\
        \midrule
        Young's modulus & [168, 252] GPa \\
        Yield strength & [220, 330] MPa \\
        Strain-hardening ratio & [0.8, 1.2] \% \\
        \bottomrule
    \end{tabular}
    \label{tab:material_property_ranges}
\end{table}

\subsection{Gravity Loads}
Gravity loads are applied to columns and beams based on their \revise{influence (tributary) areas as typically conducted in structural analysis. The considered ``service'' load conditions include the column self-weight and the additional loads directly supported on the beams from their self-weight and weights of the reinforced concrete slabs, people as live load, and building content. An edge beam typically carries approximately half the gravity load supported by a parallel interior beam}. The ranges of gravity loads are listed in \tabref{tab:gravity_load_ranges}. \revise{The loads are sampled from uniform distributions of the corresponding ranges.} Structures that failed to meet an MIDR threshold of $1\%$ under gravity loads were deemed unacceptable designs and filtered out, as such configurations of randomly chosen geometry, material, and gravity load combinations were considered unrealistic from a regulatory and practicality points of view.
\begin{table}[h!]
    \centering
    \caption{Gravity load ranges for considered beams and columns.}
    \begin{tabular}{lc}
        \toprule
        Element & Range (kN/m)  \\
        \midrule
        Column & [0.5, 1.0]  \\
        Edge beam & [1.5, 4.5]  \\
        Interior beam & [3.0, 7.5]  \\
        \bottomrule
    \end{tabular}
    \label{tab:gravity_load_ranges}
\end{table} 

\subsection{Rule-based Thermal Load Generation}
\label{subsec:thermal_load_generation}
To evaluate a building's structural response during a fire event, we employed a simplified rule-based approach for thermal load generation. 
% Previous studies \cite{nan_structuralfire_2023} have demonstrated that steel structures rapidly equilibrate with surrounding gases temperatures due to efficient heat exchange. Consequently, gas temperatures can be directly used as inputs for FEA tools, e.g., OpenSees, simplifying the process of modeling thermal loads. 
% Accurately simulating temperature fields in fire scenarios poses significant challenges. Advanced thermodynamic simulations, such as those performed using Fire Dynamics Simulator (FDS) \cite{mcgrattan_fire_2000}, provide precise temperature predictions. However, these methods are hindered by high computational costs, prolonging execution times, and limited scalability, making them impractical for generating large datasets. Additionally, real-world fire loads often display substantial spatial variability across different rooms \cite{dundar_fire_2023}, resulting in scenario-specific temperature fields with limited generalizability. For example, studies on bridge fires \cite{he_study_2024} have demonstrated that environmental factors, such as wind speeds, can significantly influence temperature distributions. Furthermore, even within identical scenarios, variations in fire modeling methodologies can produce distinctly different temperature fields \cite{zhang_temperature_2020, du_new_2012}. These challenges emphasize the need for efficient and adaptable methods to generate fire temperature data.
% To address these issues, we adopted a rule-based approach to model temperature variations. 
According to \cite{spearpoint_fire_2008}, a typical fire development follows a predictable pattern. During the {\em{growth stage}}, the temperature rises slowly and approximately linearly after ignition. This is followed by the {\em{flashover stage}}, where temperatures increase rapidly to peak values. After reaching the peak, the temperature either stabilizes or continues to rise slowly until the {\em{decay stage}} begins. Inspired by this fire development pattern, we describe the temperature evolution in time, $t$, prior to the decay stage in two distinct stages:
\begin{enumerate}
    \item {\bf{Initial linear increase stage}}: For $t \in [0, t_1)$, temperature increases gradually and linearly as the fire spreads through the building. This stage represents the time before the fire directly affects a structural element.  
    \item {\bf{ISO 834 fire curve stage}}: For $t \in [t_1, t_{\thre}]$, temperature rises rapidly following the ISO 834 curve \cite{ISO834}, modeling the direct impact of the fire on the structural element. 
\end{enumerate}
The slope of the linear temperature increase, $c$, and the transition time, $t_1$, are influenced by the spatial relationship between the fire source and the structural element. For the second stage of temperature evolution, we utilize the ISO 834 curve, a widely accepted standard for fire resistance testing. This standardized fire curve describes the temperature rise over time, enabling rapid and consistent thermal fields across various scenarios. The duration of fire simulation in this study is set to $t_{\thre}=60$ minutes. This value represents the upper limit for the temperature evolution of each structural element, providing a consistent basis for analyzing the structural response to fire.

Let $(x, y, z)$ represents the midpoint of a structural element and $(x_{\subfire}, y_{\subfire}, z_{\subfire})$ the fire source point. \revise{Integer parameters $h$ and $h_{\subfire}$ correspond to the respective floor levels of the element and the fire source}. The temperature evolution for each element is expressed as follows:
\begin{enumerate}
    \item Linear increase stage ($0 < t < t_1$):
    \begin{equation}
    T(t) = c \cdot t,
    \end{equation}
    where $c$, the rate of temperature increase ($^\circ\mathrm{C}/\mathrm{min}$), depends on the height difference between the element, $h$, and the fire source, $h_{\subfire}$:
    \begin{equation}
        c = 
        \begin{cases} 
        5\left/\left(h - h_{\subfire} + 1\right)\right., & h \geq h_{\subfire}, \\
        2\left/\left(h_{\subfire} - h\right)\right., & h < h_{\subfire}.
        \end{cases}
    \end{equation}
     \item ISO 834 stage ($t \geq t_1$):
\begin{equation}
    T(t) = c \cdot t_1 + 345 \log_{10} \left(8 \left(t - t_1\right) + 1\right).
\end{equation}
\end{enumerate}

The transition (arrival) time $t_1$, marking the end of the linear stage, depends on the spatial distance between the fire source and the element. We define the following two Euclidean distances $L_p$ in the $xy$ plane and $L_s$ in the $xyz$ space:
\begin{eqnarray}
L_p & \triangleq & \sqrt{(x - x_{\subfire})^2 + (y - y_{\subfire})^2}, \\
\label{eq:Lp}
L_s & \triangleq & \sqrt{(x - x_{\subfire})^2 + (y - y_{\subfire})^2 + (z - z_{\subfire})^2}.
\label{eq:Ls}
\end{eqnarray}
Accordingly, the transition time, $t_1$, is expressed as follows:
\begin{equation}
    t_1 = 
    \begin{cases}
    \beta_{1} \cdot \left(1 - \exp\left\{- L_s\left/\alpha_{1}\right.\right\}\right), & h > h_{\subfire}, \\
    \beta_{2} \cdot \left(1 - \exp\left\{- L_p\left/\alpha_{2}\right.\right\}\right), & h = h_{\subfire}, \\
    \beta_{3} \cdot \left(1 - \exp\left\{- L_s\left/\alpha_{3}\right.\right\}\right), & h < h_{\subfire} .
    \end{cases}
    \label{eq:t1}
\end{equation}
The parameters $\beta_i$ and $\alpha_i$ for determining $t_1$ are summarized in Table~\ref{tab:fire_spread_parameters}. In this study, we take $r_{\mathrm{up}}=0.95$ and $r_{\mathrm{down}}=0.97$.
\begin{table}[ht]
    \centering
    \caption{Fire spread parameters for $t_1$ calculations.}
    \begin{tabular}{lcc}
        \toprule
        Case  & $\beta_i$ & $\alpha_i$  \\
        \midrule
        $i=1$, Upward spread & $16 \left.\left(1-r_{\mathrm{up}}^{\left|h-h_{\subfire}\right|}\right)\right/\left(1-r_{\mathrm{up}}\right)$ & $10$  \\
        $i=2$, Horizontal spread & $18$ & $18$  \\
        $i=3$, Downward spread & $30 \left.\left(1-r_{\mathrm{down}}^{\left|h-h_{\subfire}\right|}\right)\right/\left(1-r_{\mathrm{down}}\right)$ & $5$  \\
        \bottomrule
    \end{tabular}
    \label{tab:fire_spread_parameters}
\end{table}

\figref{fig:t1_curve} illustrates the $t_1$ curves for various fire scenarios: (1) fire originating on the lower floor, $h-h_{\subfire}=1$ with rapid upward spread, (2) fire on the same floor, $h=h_{\subfire}$ with the fastest spread, and (3) fire on the upper floor, $h_{\subfire}-h=1$ with slow downward spread. The exponential decay in $t_1$ reflects the accelerating fire propagation speed as the distance increases. \figref{fig:t1_curve} also indicates that the employed simplified model is consistent with the Markov chain-based dynamic model given by \cite{cheng_dynamic_2011}, where the rooms at the same floor of the fire point start flashover slightly before the corresponding upper floors. Additionally, $\beta_{1}$ and $\beta_{3}$ are the summation of a geometric sequence, where story level $h$ is the index. The common ratios $r_{\mathrm{up}}<1$ in $\beta_{1}$ and $r_{\mathrm{down}}<1$ in $\beta_{3}$ indicate that the fire speeds up to spread through the next story, which is consistent with the real-world fire spread mechanism given in \cite{hokugo_mechanism_2000}. The temperature profile within the range $t \in [0, t_{\thre}]$ is subsequently used as the thermal load in OpenSeesRT simulations to compute displacements at each structural node at time $t_{\thre}$.
\begin{figure}[h!]
    \centering
    \includegraphics[width=0.8\linewidth]{figures/m204_t1_curve.pdf}
    \caption{Three examples for the $t_1$ curve.}
    \label{fig:t1_curve}
\end{figure}

\revise{
\textit{Remark:} The effects of structural elements, such as concrete floor slabs and partitions, are not explicitly modeled in our approach. Instead, their influence is implicitly captured through the careful selection of the parameters $ \alpha, \beta, r_\mathrm{up} $, and $ r_\mathrm{down} $. This parameterization provides a unified framework for generating temperature fields. Indeed, fire propagation is governed by a multitude of factors and remains an open research question. For instance, if the fire resistance of a floor slab is enhanced by fire protective coating, the corresponding model can account for this by decreasing $\alpha_1$ \& $\alpha_3$, increasing $\beta_1$ \& $\beta_3$, and adopting larger values for $r_\mathrm{up}$ \& $r_\mathrm{down}$, which collectively slow down the vertical spread of fire. Conversely, scenarios involving higher amounts of combustible materials would warrant the opposite adjustments. This flexible and integrated approach avoids the need to design separate models for different fire propagation scenarios while still capturing the essential effects.
}

\revise{
In conclusion, our rule-based approach is a computationally efficient method for approximating fire temperature fields, enabling large-scale dataset generation to train predictive models. By combining ISO 834 fire curves with spatial considerations and embedding structural effects through parameter calibration, the method achieves a balanced trade-off between accuracy and scalability, making it a practical solution for thermal load modeling in fire scenarios. After generating the temperature of each beam or column according to the middle point, the temperature is applied as uniform thermal load to the elements of the structure in question using OpenSeesRT. 
}

% In conclusion, this rule-based approach is a computationally efficient method to approximate fire temperature fields, enabling large-scale dataset generation to train predictive models. By combining ISO 834 fire curves with spatial considerations, the method balances accuracy and scalability, making it a practical solution for thermal load modeling in fire scenarios.

% \subsection{Interstory Drift Ratio}
\subsection{OpenSeesRT Simulation}
\label{subsec:opensees_simulation}

The thermal and mechanical responses of 3D frame structures under combined fire and gravity loads are simulated using OpenSeesRT \cite{perez2024openseesrt}. \revise{In the simulation, the IDR of each node at $t_{\thre}$ is computed using the computed nodal displacements. Each structural model features six degrees of freedom per node (3 translational  and 3 rotational), with linear geometrical transformations (\texttt{geomTransf: Linear}) defining how the element local coordinate systems are mapped to the global coordinate system and assuming small displacements and rotations. Although OpenSeesRT allows a variety of options for modeling finite deformations, in the present simulations and mainly for simplicity, we did not consider large deformations. All bottom nodes (nodes on the ground) are fully constrained in all six degrees of freedom, while degrees of freedom os all other nodes are free.} Material behavior is temperature-dependent and modeled with \texttt{Steel01Thermal}, while fiber-based sections (\texttt{FiberThermal}) capture nonlinear interactions between thermal and mechanical responses at the cross-section level. \revise{Structural elements are represented as displacement-based Euler-Bernoulli beam-columns (\texttt{dispBeamColumnThermal}). This element  formulation accounts for thermal strains (temperature gradients) in the section, which is discretized into fibers. Numerical integration is used along the length of each element using three integration (Gauss) points, one at each end and the third in the middle of the element.}

{\revise{Thermal expansion of steel members plays a crucial role in IDR development. In reality, reinforced concrete floor slabs heat at a different rate than steel members due to their higher thermal mass and lower thermal conductivity. This differential heating can lead to restrained thermal expansion, introducing axial compression in beams and affecting the overall structural response. In this study, explicit {\em{composite action}} between steel members and concrete slabs is not modeled. Instead, our approach focuses on isolating the response of the steel structural frame, which is often the critical load-bearing component in fire scenarios. This assumption aligns with prior studies \cite{Possidente_2024} demonstrating that steel structures reach thermal equilibrium with surrounding gases quickly, allowing the use of uniform thermal loading in fire analysis. Future work could enhance this framework by incorporating slab-beam interaction effects, through a refined FEA for an extended dataset where constraints imposed by floor slabs are explicitly considered.}

The analysis begins with the application of gravity loads, followed by incremental thermal loads simulating the fire exposure. A static nonlinear solver using  \texttt{ExpressNewton} algorithm ensures convergence, while the \texttt{NormDispIncr} test maintains accuracy. An incremental \texttt{LoadControl} scheme with small step sizes is employed to guarantee numerical stability, using 10\% for gravity loads and 1\% for thermal loads. 

\revise{
In the thermal load analysis, uniform thermal load is applied to each beam or column, i.e., the temperature of each element is set to be that at the middle point, according to \secref{subsec:thermal_load_generation}. The \texttt{Steel01Thermal} material allows the properties (e.g., Young's modulus and yield strength) to be adjusted at increasing temperatures according to \cite{EN1993} using its Table 3.1: Reduction factors for the stress-strain relationship of carbon steel at elevated temperatures. For example, if the Young’s modulus at ambient temperature is $E_0$, then as the temperature ($T$) increases, the modulus changes as $E(T) = \eta (T) \times E_0$. \cite{EN1993} directly provides the values of $\eta(T) \in \left[0,1\right] $ at every $100 ^\circ\mathrm{C}$ interval and recommends using linear interpolation to obtain $\eta(T)$ for intermediate values of $T$.
} OpenSeesRT documentation \cite{OpenSeesThermalExamples} provides several examples of thermal analyses.

This modeling framework accommodates variations in material properties, cross-sectional geometries, and temperature profiles, providing robust simulations of structural behavior under fire conditions. The primary settings and configurations for the OpenSeesRT simulations are summarized in \tabref{tab:ops_detail}.
\begin{table}[h!]
    \centering
        \caption{Key settings of OpenSeesRT simulations.}
    \begin{tabular}{l|>{\raggedright\arraybackslash}p{0.6\linewidth}} %
    \toprule
    Modeling Aspect     & Details \\
    \midrule
    Geometry            & 3D models; 6 degrees of freedom per node \\
    Transformation      & geomTransf: Linear \\ 
    Material            & Steel01Thermal \\
    Section             & FiberThermal; Cross-section: $0.1$ m $\times$ $0.1$ m \\ 
    Element type        & {dispBeamColumnThermal} \\ 
    Loading             & Gravity loads: {beamUniform}; Thermal loads: {beamThermal} \\
    Integration scheme  & Incremental {LoadControl}; Step size: $10\%$ (gravity analysis), $1\%$ (thermal analysis) \\
    Nonlinear solver    & {ExpressNewton} algorithm; {UmfPack} solver; Convergence test: {NormDispIncr} tolerance: $10^{-8}$; Maximum \# iterations per step: $1000$. \\ 
    \bottomrule
    \end{tabular}
    \label{tab:ops_detail}
\end{table}

For each structure in the labeled dataset, 30 fire points are selected using a dual-granularity approach, \revise{i.e., two-stage sampling strategy,} to ensure they are well-distributed. Specifically, rooms are sequentially selected, with one fire point randomly chosen within each selected room. If a building is large and contains more than 30 rooms, we randomly select 30 rooms without replacement, i.e., ensuring that no more than one fire point is located in the same room. Conversely, if the building is small and has fewer than 30 rooms, all rooms are initially selected, with one fire point randomly assigned to each room. Additionally, rooms are then selected with replacement until a total of 30 fire points are assigned. \revise{The room-level sampling prioritizes selecting distinct rooms to avoid spatial clustering of fire points, while the point-level sampling ensures intra-room variability. This approach aligns with stratified sampling principles commonly used for efficient spatial representation, where multi-stage sampling strategies optimize coverage and variability, e.g., \cite{arunachalam_generalized_2023}, and enables a more comprehensive characterizing of how the structures respond under fire conditions.}
% This selection method prevents fire points from clustering too closely while maintaining an element of randomness. By distributing fire points in this manner, the 30 fire scenarios are effectively utilized, enabling a more comprehensive characterizing of how the structures respond under fire conditions.

\subsection{Summary of the Dataset Generation}
As discussed in this section and related to  \figref{fig:dataset_generation_procedure}, three key steps were considered in the development of the dataset: 
\begin{enumerate}
    \item {\bf{Filtering process}}: Structures with MIDR exceeding $1\%$ under gravity loads were excluded,  resulting in $1,573$ labeled structures retained for fire simulation and $16,050$ unlabeled structures for training the MFSP predictor.
    \item {\bf{Fire simulations}}: For each retained labeled structure, 30 fire scenarios were simulated using OpenSeesRT, yielding $47,190$ fire cases.
    \item {\bf{Data distribution check}}: MIDR distributions for labeled and unlabeled data under gravity loads were highly similar, because both datasets were generated using the same method. Under fire conditions, the MIDR distribution shifted, reflecting significant structural deformation with values reaching a maximum of about 6\%, an average of 1.70\%, and a standard deviation of 1.12\%. This step ensured a diverse and comprehensive dataset for the proposed predictive framework.
\end{enumerate}
The statistical distribution histograms for MIDR (after applying the $1\%$ filtering threshold \revise{for gravity load responses}) under different loading conditions are plotted in \figref{fig:histogram_mdr}. Figures \ref{fig:histogram_mdr}(a) and \ref{fig:histogram_mdr}(b) show the MIDR distributions of the labeled and unlabeled data, respectively, under gravity loads only. \figref{fig:histogram_mdr}(c) shows the MIDR distribution of the labeled data under the combined effects of gravity and fire loads. Fire load causes the structures to significantly deform, leading to a noticeably \revise{right-skewed} MIDR distribution.

\begin{figure*}[h!]
    \centering
    \includegraphics[width=\linewidth]{figures/histogram_mdr.pdf}
    \caption{Histograms of MIDR for labeled and unlabeled structures with gravity loads and fire cases.}
    \label{fig:histogram_mdr}
\end{figure*}

\revise{
This dataset provides the basis for training and testing the performance of the GNN-based framework. Although we employed a simplified rule-based thermal load generation method compared with conventional CFD-based simulations, the temperature field, the changes of the material properties, and the response of the structures, are all still highly nonlinear and complex. Therefore, it is still a challenging task for the NN to predict the MIDRs based on this dataset.
}
\section{Experiments}
\label{sec: exp}

In this section, we conduct experiments to answer the following research questions:
\begin{itemize}
\item \textbf{RQ1}: Can the proposed model effectively improve the performance of the original CDMs?  
\item \textbf{RQ2}: What is the impact of each component within the proposed method? 
\item \textbf{RQ3}: How does the proposed model perform on cold-start scenarios? 
% \item \textbf{RQ4}: What are the differences in diagnostic effectiveness when using different LLMs?
\item \textbf{RQ4}: How effective is the alignment of semantic and behavioral space embeddings during the cognitive level alignment process?
\end{itemize}

\subsection{Experimental Settings}

\subsubsection{Datasets}

\section{Baseline} \label{sec:splitgraph}

The baseline method for batch-$k$DP solves each query using flow-augmenting path-based methods, which rely on the concept of \textit{split-graphs}~\cite{baseline_moreverbose, baseline1step2, baselineOnlySplitP1}. 
% For each query, paths are iteratively found in a split-graph, which is updated after each iteration.
% A split-graph is constructed by two transformations of the original graph:
% (1) reversing result-set paths, simulating flow-augmentation, and 
% (2) splitting vertices within these paths, giving rise to the name ``split-graph."

\textbf{Definition: Split-Graph~\cite{baselineOnlySplitP1}} 
Given a graph \( G = (V, E) \) and a set \( P \) of disjoint paths from \( s \) to \( t \), the split-graph \( \iG_{G,P} = (\iV_{G,P}, \iE_{G,P}) \) is constructed as follows:
(1) Initializing \( \iV_{G,P} = V \) and \( \iE_{G,P} = E \).
(2) For each edge in \( E(P) \), reversing the corresponding edge in \( \iE_{G,P} \).
(3) Splitting vertices \(v \in V(P) \setminus \{s, t\}\) into \(v^{in}\) and \(v^{out}\), and connecting them accordingly.
(4) Replacing edges in \(\iE_{G,P}\) with updated vertex connections, preserving incoming and outgoing edges.

% \textbf{Example}: 
% Fig.~\ref{fig:eg_split} shows the split-graph construction for the graph \( G \) in Fig.~\ref{fig:g} with $P= \{p_1=\{a, e, d, h\}\}$. Changes are shown in red.


% \vspace{-10pt}
\begin{figure}[h!]
\newcommand{\mylinewidth}{\linewidth}
\centering
    \begin{subfigure}[t]{0.35\mylinewidth}
        \centering
        % \resizebox{\mylinewidth}{!}
        {\includegraphics[width=\linewidth]{pic/eg/g}}
        \caption{Disjoint paths for $(a, h)$.}
        \label{fig:g}
    \end{subfigure}
    \begin{subfigure}[t]{0.6\mylinewidth}
        \centering
        % \resizebox{\mylinewidth}{!}
        {\includegraphics[width=\linewidth]{pic/eg/steps_red_new.pdf}}
        \caption{Split-graph with $P= \{p_1=\{$a$, $e$, $d$, $h$\}\}$.}
        \label{fig:eg_split}
    \end{subfigure}
    \caption{Examples of disjoint paths and split-graph.}
    % \label{fig:fg_share_intuition}
\end{figure} 
% \vspace{-5pt}

% 删除 begin
Given a graph \( G \) and vertices \( s \) and \( t \), the algorithm proceeds as follows:
% (1) Initialize \( P = \emptyset \) and \( \iG_{G,P} = G \).
% (2) Find the first path \( p_1 \) using a path-finding algorithm (e.g., BFS) in \( \iG_{G,P} \) and update \( \iG_{G,P} \).
% (3) Find the second path \( p_2 \), update found paths following an approach similar to augmenting paths in the maximum flow problem~\cite{baseline_moreverbose}, then update \( \iG_{G,P} \). More paths are found in a similar manner.
(1) Initialize $P = \emptyset$ and $\iG_{G, P} = G$.
(2) Find the first path $p_1$ in $\iG_{G, P}$ using any path-finding algorithm (e.g., BFS), forming $P_1 = \{p_1\}$, and update $\iG_{G, P}$ to $\iG_{G, P_1}$.
(3) Search for $p_2$ in $\iG_{G, P_1}$, yielding $P_2 = \{p_1, p_2\}$, and adjust $P_2$ following an approach similar to augmenting flows~\cite{baseline_moreverbose}.
Then update $\iG_{G, P_1}$ to $\iG_{G, P_2}$.
(4) Search for $p_3$ in $\iG_{G, P_2}$. More paths are found in a similar manner.
% 删除 end
In our experiments, we utilize four courses, Python Programming (Python), Linux System (Linux), Database Technology and Application (Database), and Literature and History (Literature), from a publicly available dataset PTADisc~\cite{hu2023ptadisc}, which comes from real-world students' responses in the educational website PTA\footnote{\url{https://pintia.cn/}} and contains textual information of exercises and knowledge concepts. 
%Each response log in the dataset contains a student ID, an exercise ID, whether the student correctly answers the question, the content of the exercise, and the knowledge concepts related to the exercise.
The statistics of the datasets are presented in Table~\ref{tab: dataset}.
The datasets are divided into training, validation, and testing sets, with a ratio of 8:1:1.

\subsubsection{Evaluation Metrics}

Following previous works, we evaluate the students' cognitive status by predicting the performance of students on the testing set, as the cognitive status can not be directly observed. We adopt commonly used metrics, namely the Area Under a ROC Curve (AUC), the Prediction Accuracy (ACC), and the Root Mean Square Error (RMSE), to validate the effectiveness of the CDMs.
%In the subsequent tables, \textbf{bold} numbers represent the best performance, while \underline{underlined} numbers represent the second-best performance. 
For all the metrics, $\uparrow$ represents that a greater value is better, while $\downarrow$ represents the opposite.

\subsubsection{Baseline Methods}

To validate the effectiveness of the proposed method, we conduct experiments on several representative CDMs, including IRT~\cite{lord1952theory}, MIRT~\cite{reckase200618}, DINA~\cite{de2009dina}, NCD~\cite{wang2020neural}, RCD~\cite{gao2021rcd}, SCD~\cite{wang2023self} and ACD~\cite{wang2024boosting}.
 

\subsubsection{Implementation Details}

We utilize PyTorch to implement both the baseline methods and our proposed KCD framework. 
For the baseline models, We use the default hyper-parameters as stated in their papers and for KCD, we use the same hyper-parameter settings, such as training epoch, learning rate, and batch size.
We employ ChatGPT to represent LLMs (specifically, gpt-3.5-turbo-16k) and text-embedding-ada002 as the text embedding model. All the experiments are conducted on a GeForce RTX 3090 GPU.
We train the model on train set and at the end of each epoch, we evaluate the model on the validation set.
The hyper-parameter $\alpha$, $\beta$, and $\lambda$ was set to $0.04$, $0.015$, and $0.2$.
Since our dataset does not include affect labels, we utilize the unsupervised contrastive ACD model and employ NCD as the basic cognitive diagnosis module.
The behavioral space alignment approach is denoted as `-Beh' and the semantic space alignment approach is denoted as `-Sem'.
% We investigated the impact of the hyper-parameter $\lambda$, within the range $[0,0.2,\cdots,1]$ with a step size of $0.2$. Our analysis revealed that setting $\lambda$ to $0.1$ resulted in the best performance across all three datasets.

\begin{figure}[t]
  \centering
  
  \includegraphics[width=1.02\linewidth]{figs/experimentx.png}
  \caption{Performance comparison in cold (blue) and warm (red) scenarios on Python dataset.}
  \vspace{-2em}
\label{fig: experiment1}
\end{figure}

\subsection{Performance Comparison (RQ1)}
To demonstrate the effectiveness of our proposed method in improving cognitive diagnosis, we implement the framework on seven cognitive diagnosis models, and the results are shown in Table~\ref{tab:performance}. 
Additionally, we compared the performance of NCD in warm and cold scenarios, with the results illustrated in Figure~\ref{fig: experiment1}. Here we define the cold scenario as less than $3$ interactions in the training set for exercises and define the warm scenario as more than $10$ interactions in the training set for exercises. Following this definition, we divide the testing set into cold and warm subsets.
We have the following observations from the results: 

\begin{itemize}[leftmargin=*]
    \item[1)]  
    Both KCD-Beh and KCD-Sem achieve significant improvements compared to the basic CDMs.
    This indicates that our proposed framework is widely applicable to various CDMs, and both alignment methods can effectively align the behavioral space of CDMs and the semantic space of LLMs.
    In most models, the behavioral space alignment approach performs better, indicating that aligning in the behavioral space of CDMs can better align information from the semantic space of LLMs.
    \item[2)] Compared to basic CDMs, our proposed methods demonstrate improvements in both cold and warm scenarios, especially in cold scenarios. This indicates that our approach of introducing LLMs as knowledge enhancement effectively alleviates the cold-start issue.
\end{itemize}




\begin{table*}
  [t]
  \centering
  \resizebox{\textwidth}{!}{%
  \begin{tabular}{cccccccccccc}
    \toprule \multicolumn{2}{c}{Components}                                                             & \multicolumn{5}{c}{Re-executability Rate (\%)} & \multicolumn{5}{c}{Readability (\#)} \\
    \cmidrule(lr){1-2} \cmidrule(lr){3-7} \cmidrule(lr){8-12}        \hspace{8pt}\labelemoji\hspace{8pt}                                                                & \hspace{8pt}\toolemoji\hspace{8pt}                                      & O0                                 & O1             & O2             & O3             & AVG            & O0             & O1             & O2             & O3             & AVG            \\
    \hline
    \rowcolor[rgb]{0.93,0.93,0.93}\multicolumn{12}{c}{\textbf{Initialize with LLM4Decompile-End-6.7B~\citep{llm4decompile}}}   \\
    \xmark                                                                                              & \xmark                                    & 69.51                              & 46.95          & 50.61          & 46.34          & 53.35          & 3.98 & 3.41 & 3.44 & 3.38 & 3.55 \\
    \cmark                                                                                              & \xmark                                    & 75.61                              & 50.61          & 50.00          & 50.00          & 56.55          & 4.01 & 3.44 & 3.39 & \textbf{3.49} & 3.58 \\
    \xmark                                                                                              & \cmark                                    & 83.54                     & \textbf{56.10}          & 51.22          & 50.61 & 60.37 & 4.05 & 3.51 & 3.51 & 3.42 & 3.62 \\
    \cmark                                                                                              & \cmark                                    & \textbf{85.37}                            & \textbf{56.10}                     & \textbf{51.83} & \textbf{52.43}          & \textbf{61.43} & \textbf{4.13} & \textbf{3.60} & \textbf{3.54} & \textbf{3.49} & \textbf{3.69} \\

    \rowcolor[rgb]{0.93,0.93,0.93}\multicolumn{12}{c}{\textbf{Initialize with Deepseek-Coder-6.7B-base~\citep{deepseekcoder}}} \\
    \xmark                                                                                              & \xmark                                    & 59.15                              & 35.98          & 39.02          & 37.80          & 42.99          & 3.71 & 3.05 & 3.16 & 3.05 & 3.24 \\
    \cmark                                                                                              & \xmark                                    & 66.46                              & 41.46          & 38.41          & 36.59          & 45.73          & 3.76 & 3.17 & \textbf{3.21} & 3.08 & 3.31 \\
    \xmark                                                                                              & \cmark                                    & 70.73                              & 39.63          & 39.02          & 40.24          & 47.41          & 3.90 & 3.17 & 3.08 & 3.11 & 3.31 \\
    \cmark                                                                                              & \cmark                                    & \textbf{79.88}                     & \textbf{45.73} & \textbf{43.90} & \textbf{42.68} & \textbf{53.05} & \textbf{3.96} & \textbf{3.21} & 3.18 & \textbf{3.19} & \textbf{3.38} \\
    \bottomrule
  \end{tabular}%
  }
  \caption{The ablation study of different methods across four optimization levels
  (O0, O1, O2, O3), as well as their average scores (AVG). The results in bold represent the optimal performance. The ~\labelemoji~ and ~\toolemoji~ means Relabedling and Function Call. \textbf{Bold} denotes the best performance.}
  \label{tab:ablation}
\end{table*}
\subsection{Ablation Study (RQ2)}


To validate the effectiveness of different components of our proposed method, we conduct ablation experiments to verify several components utilized in LLM Diagnosis and Cognitive Level alignment, including the usage of collaborative information (denoted as `Coll. Info'), the local contrast and global contrast (denoted as `Local Con.' and `Global Con.'), and the dynamic masking strategy (denoted as `Dym. Mask').

Table~\ref{tab:ablation} demonstrates the results of the ablation study on Python dataset, comparing the model performance after removing specific components (denoted as `w/o'). `w/o Coll. Info' represents replacing collaborative information in the process of diagnosis generation and `w/o Dym. Mask' represents replacing dynamic masking strategy with a constant mask ratio.
Experimental results show that removing these components individually leads to a decline in the model's performance. This indicates that these components are crucial for the model's performance.


\begin{figure}[t]
  \centering
  
  \includegraphics[width=1\linewidth]{figs/drop.png}
  \caption{Performance on different dropout ratios.}
  
\label{fig: drop}
\end{figure}
\subsection{Performance on Cold-Start Scenarios (RQ3)}

we conduct additional experiments on sub-datasets with varying degrees of sparsity. Specifically, we apply random dropout to the training sets of the Python and Linux datasets at ratios of $10\%$, $20\%$, $30\%$, $40\%$, and $50\%$.

Figure~\ref{fig: drop} shows the results of the experiments on different dropout ratios. It is obvious that as the dropout ratio increases, both AUC and ACC decrease. This is because the training set becomes more sparse, approaching a cold-start scenario. 
Additionally, compared to ACC, AUC experiences a greater decline, which might be due to the different calculation methods of the two metrics. 
% For more sparse datasets, Python, AUC experience a more significant decrease compared to the Linux dataset. From the experimental results, it can be seen that our proposed method is effective across different dropout ratios, leading to significant improvements for CDMs. More specifically, from the different performances of NCD-Beh and NCD-Sem in the Linux and Python datasets, it can be seen that we can choose different alignment methods based on the dataset to achieve better diagnostic results.


\begin{figure}[t]
  \centering
  \vspace{-1em}
  \includegraphics[width=1\linewidth]{figs/experiment2.png}
  \caption{The t-SNE visualization of student embeddings on Literature dataset.}
  \vspace{-2em}
\label{fig: experiment2}
\end{figure}
\subsection{Visualization of Semantic and Behavioral Embeddings (RQ4)}


To validate the effectiveness of the two alignment processes, we utilize t-SNE~\cite{van2008visualizing} to visualize the distribution of features in LLMs semantic space and CDMs behavioral space. We randomly select 200 example students and map their behavioral embeddings and semantic embeddings to 2-dimensional space. NCD (w/o Alignment) represents the original CDMs without alignment.

Figure~\ref{fig: experiment2} demonstrates the integration of semantic and behavioral embeddings of NCD-Beh and NCD-Sem, with their distributions closely merged compared to original CDMs. This proves the effectiveness of the two alignment methods we proposed.

\begin{figure}[t]
  \centering
  
  \includegraphics[width=1\linewidth]{figs/case.png}
  \caption{The case study of a student on multiple knowledge concepts on Linux dataset.}
  \vspace{-2em}
\label{fig: case}
\end{figure}

\subsection{Case Study}


To more intuitively demonstrate the improvements our proposed methods bring to CDMs, we selected a diagnosis for a specific student in the Linux dataset and compared the prediction results of NCD with the diagnosis results of NCD-Beh.
As illustrated in Figure~\ref{fig: case}, we randomly choose a student, and list his mastery of some knowledge concepts predicted by NCD and our proposed NCD-Beh.
This student correctly answered the exercises related to `numerical encoding' and `process communication', showing mastery of these concepts. He answered other exercises incorrectly, indicating a lack of familiarity with the remaining knowledge concepts.
From the LLM's diagnostic results, it can be observed that the LLM captured similar question-answer information from the training set and made corresponding inferences. This played an important role in NCD-Beh's more accurate prediction of the student's mastery level.
%%%%%%%%%%%%
%  E2E     %
%%%%%%%%%%%%


\section{Conclusion}
In this paper, we introduced Wi-Chat, the first LLM-powered Wi-Fi-based human activity recognition system that integrates the reasoning capabilities of large language models with the sensing potential of wireless signals. Our experimental results on a self-collected Wi-Fi CSI dataset demonstrate the promising potential of LLMs in enabling zero-shot Wi-Fi sensing. These findings suggest a new paradigm for human activity recognition that does not rely on extensive labeled data. We hope future research will build upon this direction, further exploring the applications of LLMs in signal processing domains such as IoT, mobile sensing, and radar-based systems.

\section*{Limitations}
While our work represents the first attempt to leverage LLMs for processing Wi-Fi signals, it is a preliminary study focused on a relatively simple task: Wi-Fi-based human activity recognition. This choice allows us to explore the feasibility of LLMs in wireless sensing but also comes with certain limitations.

Our approach primarily evaluates zero-shot performance, which, while promising, may still lag behind traditional supervised learning methods in highly complex or fine-grained recognition tasks. Besides, our study is limited to a controlled environment with a self-collected dataset, and the generalizability of LLMs to diverse real-world scenarios with varying Wi-Fi conditions, environmental interference, and device heterogeneity remains an open question.

Additionally, we have yet to explore the full potential of LLMs in more advanced Wi-Fi sensing applications, such as fine-grained gesture recognition, occupancy detection, and passive health monitoring. Future work should investigate the scalability of LLM-based approaches, their robustness to domain shifts, and their integration with multimodal sensing techniques in broader IoT applications.


% Bibliography entries for the entire Anthology, followed by custom entries
%\bibliography{anthology,custom}
% Custom bibliography entries only
\bibliography{main}
\newpage
\appendix

\section{Experiment prompts}
\label{sec:prompt}
The prompts used in the LLM experiments are shown in the following Table~\ref{tab:prompts}.
\begin{tcolorbox}[title={The Prompt used for Translation}]
You are a highly skilled translator tasked with translating various types of content from English into \{\{ language \}\}. Follow these instructions carefully to complete the translation task.

You will receive a user-bot conversation in XML format. Please follow a three-step translation process:

\begin{enumerate}
  \item \textbf{Initial Translation:} Translate the input content into \{\{ language \}\}, preserving the original intent and keeping the original paragraph and text format unchanged. Do not delete or omit any content, and ensure that all original Markdown elements (e.g., images, code blocks) are preserved.
  \item \textbf{Reflection and Feedback:} Carefully review both the source text and your translation. Provide constructive criticism and specific suggestions to improve the translation in terms of:
    \begin{enumerate}[label=(\roman*)]
      \item \textbf{Accuracy:} Correct errors of addition, mistranslation, omission, or untranslated text.
      \item \textbf{Fluency:} Apply \{\{ language \}\} grammar, spelling, and punctuation rules while avoiding unnecessary repetitions.
      \item \textbf{Style:} Ensure that the translation reflects the style of the source text and considers any relevant cultural context.
    \end{enumerate}
  \item \textbf{Refinement:} Based on your reflections, refine and polish your translation.
  \item \textbf{Fallback:} If you are not confident in translating the conversation, please return ``\texttt{<stop></stop>}''.
\end{enumerate}

\bigskip
\textbf{Output:}

For each step of the translation process, output your results within the appropriate XML tags as follows:
\begin{verbatim}
<step1_initial_translation>
[Insert your initial translation here]
</step1_initial_translation>

<step2_reflection>
[Insert your reflection on the translation, including a list 
of specific, helpful, and constructive suggestions for 
improvement. Each suggestion should address a specific 
part of the translation.]
</step2_reflection>

<step3_refined_translation>
[Insert your refined and polished translation here]
</step3_refined_translation>
\end{verbatim}

Ensure that your final translation in step 3 accurately reflects the original meaning while sounding natural in \{\{ language \}\}.

Here is the original conversation:
\label{box:trans_prompt}
\end{tcolorbox}

% \section{Full Experiment Results}
% \begin{table*}[th]
    \centering
    \small
    \caption{Classification Results}
    \begin{tabular}{lcccc}
        \toprule
        \textbf{Method} & \textbf{Accuracy} & \textbf{Precision} & \textbf{Recall} & \textbf{F1-score} \\
        \midrule
        \multicolumn{5}{c}{\textbf{Zero Shot}} \\
                Zero-shot E-eyes & 0.26 & 0.26 & 0.27 & 0.26 \\
        Zero-shot CARM & 0.24 & 0.24 & 0.24 & 0.24 \\
                Zero-shot SVM & 0.27 & 0.28 & 0.28 & 0.27 \\
        Zero-shot CNN & 0.23 & 0.24 & 0.23 & 0.23 \\
        Zero-shot RNN & 0.26 & 0.26 & 0.26 & 0.26 \\
DeepSeek-0shot & 0.54 & 0.61 & 0.54 & 0.52 \\
DeepSeek-0shot-COT & 0.33 & 0.24 & 0.33 & 0.23 \\
DeepSeek-0shot-Knowledge & 0.45 & 0.46 & 0.45 & 0.44 \\
Gemma2-0shot & 0.35 & 0.22 & 0.38 & 0.27 \\
Gemma2-0shot-COT & 0.36 & 0.22 & 0.36 & 0.27 \\
Gemma2-0shot-Knowledge & 0.32 & 0.18 & 0.34 & 0.20 \\
GPT-4o-mini-0shot & 0.48 & 0.53 & 0.48 & 0.41 \\
GPT-4o-mini-0shot-COT & 0.33 & 0.50 & 0.33 & 0.38 \\
GPT-4o-mini-0shot-Knowledge & 0.49 & 0.31 & 0.49 & 0.36 \\
GPT-4o-0shot & 0.62 & 0.62 & 0.47 & 0.42 \\
GPT-4o-0shot-COT & 0.29 & 0.45 & 0.29 & 0.21 \\
GPT-4o-0shot-Knowledge & 0.44 & 0.52 & 0.44 & 0.39 \\
LLaMA-0shot & 0.32 & 0.25 & 0.32 & 0.24 \\
LLaMA-0shot-COT & 0.12 & 0.25 & 0.12 & 0.09 \\
LLaMA-0shot-Knowledge & 0.32 & 0.25 & 0.32 & 0.28 \\
Mistral-0shot & 0.19 & 0.23 & 0.19 & 0.10 \\
Mistral-0shot-Knowledge & 0.21 & 0.40 & 0.21 & 0.11 \\
        \midrule
        \multicolumn{5}{c}{\textbf{4 Shot}} \\
GPT-4o-mini-4shot & 0.58 & 0.59 & 0.58 & 0.53 \\
GPT-4o-mini-4shot-COT & 0.57 & 0.53 & 0.57 & 0.50 \\
GPT-4o-mini-4shot-Knowledge & 0.56 & 0.51 & 0.56 & 0.47 \\
GPT-4o-4shot & 0.77 & 0.84 & 0.77 & 0.73 \\
GPT-4o-4shot-COT & 0.63 & 0.76 & 0.63 & 0.53 \\
GPT-4o-4shot-Knowledge & 0.72 & 0.82 & 0.71 & 0.66 \\
LLaMA-4shot & 0.29 & 0.24 & 0.29 & 0.21 \\
LLaMA-4shot-COT & 0.20 & 0.30 & 0.20 & 0.13 \\
LLaMA-4shot-Knowledge & 0.15 & 0.23 & 0.13 & 0.13 \\
Mistral-4shot & 0.02 & 0.02 & 0.02 & 0.02 \\
Mistral-4shot-Knowledge & 0.21 & 0.27 & 0.21 & 0.20 \\
        \midrule
        
        \multicolumn{5}{c}{\textbf{Suprevised}} \\
        SVM & 0.94 & 0.92 & 0.91 & 0.91 \\
        CNN & 0.98 & 0.98 & 0.97 & 0.97 \\
        RNN & 0.99 & 0.99 & 0.99 & 0.99 \\
        % \midrule
        % \multicolumn{5}{c}{\textbf{Conventional Wi-Fi-based Human Activity Recognition Systems}} \\
        E-eyes & 1.00 & 1.00 & 1.00 & 1.00 \\
        CARM & 0.98 & 0.98 & 0.98 & 0.98 \\
\midrule
 \multicolumn{5}{c}{\textbf{Vision Models}} \\
           Zero-shot SVM & 0.26 & 0.25 & 0.25 & 0.25 \\
        Zero-shot CNN & 0.26 & 0.25 & 0.26 & 0.26 \\
        Zero-shot RNN & 0.28 & 0.28 & 0.29 & 0.28 \\
        SVM & 0.99 & 0.99 & 0.99 & 0.99 \\
        CNN & 0.98 & 0.99 & 0.98 & 0.98 \\
        RNN & 0.98 & 0.99 & 0.98 & 0.98 \\
GPT-4o-mini-Vision & 0.84 & 0.85 & 0.84 & 0.84 \\
GPT-4o-mini-Vision-COT & 0.90 & 0.91 & 0.90 & 0.90 \\
GPT-4o-Vision & 0.74 & 0.82 & 0.74 & 0.73 \\
GPT-4o-Vision-COT & 0.70 & 0.83 & 0.70 & 0.68 \\
LLaMA-Vision & 0.20 & 0.23 & 0.20 & 0.09 \\
LLaMA-Vision-Knowledge & 0.22 & 0.05 & 0.22 & 0.08 \\

        \bottomrule
    \end{tabular}
    \label{full}
\end{table*}




\end{document}
