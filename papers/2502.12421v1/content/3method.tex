
\section{Preliminary}

\subsection{Wi-Fi Sensing} \label{Preliminary}

In recent years, Wi-Fi sensing has garnered significant attention due to the widespread use of Wi-Fi devices and their ability to sense people and surrounding environments. In an indoor setting, Wi-Fi signals propagate through both direct (i.e., line-of-sight (LoS)) and reflected paths, bouncing off objects and humans, before reaching the receiver. When the sensing target is a person, changes in the signal can be leveraged to infer locations, activities, and even vital signs of the person, enabling contactless human sensing. Moreover, existing Wi-Fi infrastructure can be reused for sensing, allowing for seamless integration into an LLM-powered smart environment.

To capture changes in Wi-Fi signals caused by a target (e.g., a person), we utilize Channel State Information (CSI), which characterizes how signals are altered as they propagate through physical space. Specifically, Wi-Fi signals travel from a transmitter to a receiver through multiple paths, including a direct path (i.e., LoS propagation) and numerous reflected paths from objects such as walls, furniture, and the person. CSI represents the superposition of signals from all these paths and describes this multipath propagation, which can be expressed as: $H(f, t) = \sum^{N}_{i=1}a_{i}e^{-j2\pi \frac{d_{i}(t)}{\lambda}}$,
%\vspace{-5mm}
%\begin{equation}
%H(f, t) = \sum^{N}_{i=1}a_{i}e^{-j2\pi \frac{d_{i}(t)}{\lambda}},
%\end{equation}
where $a_{i}$ is the complex attenuation and $d_{i}(t)$ is the length of the $i^{th}$ path, $N$ is the number of paths, $\lambda$ is the wavelength, and $f$ is the signal frequency.



The CSI can be further decomposed into static and dynamic components. The static component consists of the LoS signals and reflections from stationary objects in the environment. In contrast, the dynamic component arises from reflections caused by the moving target. For simplicity, we assume that there is only a single signal reflection from the target. Thus, the CSI can be denoted as: 
\vspace{-2mm}
\begin{equation} \label{eq2}
\begin{split}
H(f, t) &= H_{s}(f, t) + H_{d}(f, t) \\
&= H_{s}(f, t) + a(f,t)e^{-j2\pi \frac{d(t)}{\lambda}},
\end{split}
\end{equation}
where $H_{s}(f, t)$ is the static component, $a(f,t)$, $e^{-j2\pi \frac{d(t)}{\lambda}}$, and $d(t)$ are the complex attenuation, phase shift and path length of dynamic component $H_{d}(f, t)$, respectively.


\subsection{Wi-Fi-based Human Activity Recognition} \label{systems}
% \vspace{-2mm}
The overall system flow for Wi-Fi-based human activity recognition is shown in Figure~\ref{systemflow}. In all systems, a Wi-Fi transmitter emits signals that are received by Wi-Fi receivers to probe human activities. The system processes Wi-Fi CSI measurements, which can be extracted from the network interface controllers of Wi-Fi devices.
In this work, we examine three distinct types of paradigms:

\textbf{Paradigm 1: Conventional Wi-Fi-based Systems.} Following Wi-Fi probing, raw Wi-Fi signals undergo processing in a signal-denoising module to reduce noise. This process may involve phase offset removal techniques to calibrate the signal phase~\cite{guo2017wifi, kotaru2015spotfi} and various filtering methods to eliminate outliers~\cite{ali2017recognizing}.
Next, signal transformation methods are applied for time-frequency analysis of the denoised Wi-Fi signals. Common techniques include Fast Fourier Transform (FFT), Short-Time Fourier Transform (STFT), and Discrete Wavelet Transform (DWT).
Next, a feature extraction step extracts relevant features from the preprocessed Wi-Fi signals. For instance, propagation distance features can be derived using the Power Delay Profile (PDP)~\cite{xie2015precise}. Additionally, Principal Component Analysis (PCA), Independent Component Analysis (ICA), and Singular Value Decomposition (SVD) are widely employed for feature extraction, signal separation, and dimensionality reduction.
Once features are extracted, deep learning or machine learning models are trained to learn a mapping function between Wi-Fi signals and corresponding ground truth activity labels. After training, the model can perform human activity recognition on newly captured Wi-Fi signals.

\textbf{Paradigm 2: Machine Learning Models with Raw Signals.}
An alternative approach is directly inputting raw Wi-Fi signals into machine learning models, including Convolutional Neural Networks (CNNs), Recurrent Neural Networks (RNNs), and Support Vector Machines (SVMs).
In this approach, we first apply a simple signal smoothing. The signals are then converted into textual or visual representations before being fed into the models. Similar to conventional systems, these models still require extensive manual labeling and training.

\textbf{Paradigm 3: LLM-based Zero-shot Inference System.} In Wi-Chat, the input data consists of textual or visual representations of raw Wi-Fi signals, processed with simple signal smoothing. The system aims to recognize human activities based on these signals.
To achieve this, we explicitly instruct LLMs to determine a person's activity by analyzing the Wi-Fi signals. The underlying idea is that different human activities induce distinct patterns in Wi-Fi signals. By integrating physical models of Wi-Fi sensing into prompts, we provide physical model guidance to LLMs to interpret the signals.
Unlike conventional Wi-Fi-based systems or machine learning models, Wi-Chat eliminates the need for complex signal processing and labor-intensive model training, offering a more efficient and scalable approach to activity recognition.

\begin{figure}[t]
    \centering
    \includegraphics[width=0.42\textwidth]{figures/csiwalk.pdf}
    \vspace{-3mm}
    \caption{Modeling the human walking scenario.}
    \vspace{-4mm}
    \label{csiwalk}
\end{figure}

\begin{figure}[t]
    \centering
    \includegraphics[width=0.42\textwidth]{figures/csifall.pdf}
    \vspace{-3mm}
    \caption{Modeling the human falling scenario.}
    \vspace{-7mm}
    \label{csifall}
\end{figure}

\begin{figure}[t]
    \centering
    \includegraphics[width=0.42\textwidth]{figures/csibreath.pdf}
    \vspace{-3mm}
    \caption{Modeling the human breathing scenario.}
    \vspace{-4mm}
    \label{csibreath}
\end{figure}

\begin{figure}[t]
    \centering
    \includegraphics[width=0.42\textwidth]{figures/csino.pdf}
    \vspace{-3mm}
    \caption{Modeling the no-event scenario.}
    \vspace{-7mm}
    \label{csino}
\end{figure}



\begin{figure*}[h]
    \centering
    \includegraphics[width=0.9\textwidth]{figures/signals.pdf}
    \caption{Real signals of different human activities.}
    %\vspace{-6mm}
    \label{signals}
\end{figure*}

\section{Method}
\subsection{Wi-Fi Physical Model Knowledge}
\label{phisical}
In this subsection, we construct the physical models of Wi-Fi sensing in terms of human walking, falling, breathing, and no-event scenarios. Then we derive the LLM prompting guidance accordingly.


%Then we can plot the signals according to the in-phase (I) and quadrature (Q) components~\cite{wang2015understanding} in the IQ plane as shown in Figure~\ref{csiwalk}(b). In theory, the static component vector ($H_s$) is fixed and the dynamic component vector ($H_d$) can change and rotate. The overall CSI $H$ is the addition of vectors $H_s$ and $H_d$. We can observe that when dynamic component and static component vectors on the IQ plane are in the same direction (e.g., $P_{0,2,4\dots}$), they add constructively and the CSI amplitude is maximum, while they add destructively and amplitude is minimum in the reverse direction (e.g., $P_{1,3,5\dots}$). In particular, when the path length of the dynamic component changes by one wavelength ($\lambda = 6cm$ for 5 GHz Wi-Fi), its phase rotates by $2\pi$~\cite{ren2021winect} as indicated in Equation~\ref{eq2}. The human walking activity is a large-scale activity and each step can result in changes of multiple wavelengths in the propagation path. Therefore, the phase of the dynamic component can rotate multiple times and this causes multiple peaks and troughs between $|H_s|+|H_d|$ and $|H_s|-|H_d|$ for CSI amplitude as shown in Figure~\ref{csiwalk}(c). Note that we only consider CSI amplitude in this work. The duration of walking $\Delta T$ could be a couple of seconds or minutes.

\textbf{Modeling the Human Walking Scenario.} As shown in Figure~\ref{csiwalk}(a), a person is walking and the Wi-Fi transmitter and receiver are placed at fixed locations. The Wi-Fi transmitter emits Wi-Fi signals which propagate through the multipath environments. As described in Section~\ref{Preliminary}, the static components (i.e., solid red lines) include the LoS signal and the signals reflected off the wall. The dynamic components (i.e., dotted blue lines) consist of the signals reflected off the human body. Assuming the person moves from location $P_A$ to location $P_N$, this movement results in a change in the path length of the dynamic component. The duration of this movement is denoted as $\Delta T$. 

%from $d_0$ to $d_n$, where $d_n = d_0 + n \frac{\lambda}{2}$, $n=1, 2, 3, \dots$, and $\lambda$ is the wavelength. The duration of this movement is denoted as $\Delta T$. 

%{$\vec a$}


We can plot the signals based on the in-phase (I) and quadrature (Q) components~\cite{wang2015understanding} in the IQ plane, as illustrated in Figure~\ref{csiwalk}(b). Theoretically, the static component vector ($\vec{H_s}$) remains fixed, while the dynamic component vector ($\vec{H_d}$) can change and rotate. The overall CSI $\vec{H}$ is the sum of vectors $\vec{H_s}$ and $\vec{H_d}$. When the dynamic and static component vectors on the IQ plane are aligned in the same direction (e.g., at $P_{N}$), they add constructively, resulting in the maximum CSI amplitude (i.e., $|\vec{H}|=|\vec{H_s}|+|\vec{H_d}|$). Conversely, when they are in opposite directions (e.g., at $P_{B}$), they add destructively, minimizing the CSI amplitude (i.e., $|\vec{H}|=|\vec{H_s}|-|\vec{H_d}|$). Note that when the path length of the dynamic component changes by one wavelength (e.g., one wavelength is about 6 $cm$ for 5 GHz Wi-Fi), its phase rotates by $2\pi$~\cite{ren2021winect, ren2022gopose} as indicated in Equation~\ref{eq2}. Since human walking is a large-scale activity, each step can cause changes of many wavelengths in the propagation path, resulting in multiple phase rotations in the dynamic components. This leads to multiple peaks ($|H_s|+|H_d|$) and troughs ($|H_s|-|H_d|$) for CSI amplitude ($|\vec{H}|$) as shown in Figure~\ref{csiwalk}(c). In this work, we focus exclusively on the CSI amplitude. Additionally, we note that human walking is a continuous activity with a duration $\Delta T$ that can range from a few seconds to several minutes.

Thus, we formulate the LLM prompting guidance for human walking as follows: 
\emph{``Walking is a large-scale activity that induces significant changes in the Wi-Fi CSI amplitude over time, characterized by the presence of numerous peaks and troughs.''}

%Thus, we derive the LLM prompting guidance for human walking that \emph{walking is a large-scale activity that induces significant changes in the Wi-Fi CSI amplitude over time, resulting in numerous peaks and troughs}. 

%Thus, we formulate the LLM prompting guidance for human walking as follows: "Walking is a large-scale activity that induces significant variations in the Wi-Fi CSI amplitude over time, characterized by the presence of multiple peaks and troughs."

%This will be used for physical guidance for LLM prompting.

%section 4.2 physical-aware prompting




\textbf{Modeling the Human Falling Scenario.} Similar to walking, the falling scenario is also a large-scale activity, where the person moves from $P_A$ to $P_N$, as depicted in Figure~\ref{csifall}(a). This movement can cause the overall CSI amplitude ($|\vec{H}|$) to reach both maximum and minimum values, corresponding to $|\vec{H_s}| + |\vec{H_d}|$ and $|\vec{H_s}| - |\vec{H_d}|$, respectively, as shown in Figures~\ref{csifall}(b) and (c). However, the duration $\Delta T$ of a fall can be very short (e.g., about 0.5 seconds)~\cite{choi2015kinematic}. As a result, the peaks and troughs caused by the fall are concentrated within a brief period. After signal smoothing, these rapid fluctuations can be considered as one significant peak/trough. Following the fall, the person may lose consciousness or remain motionless, leading to a static period after the fall.

Therefore, we can formulate the LLM prompting guidance for human falling as follows:
\emph{``Falling is a large-scale and sudden activity that induces a single significant peak/trough in the Wi-Fi CSI amplitude, followed by a relatively stable period.''}



\textbf{Modeling the Human Breathing Scenario.} Human breathing is a small-scale activity, as the typical range of chest expansion and contraction during a breath is only a few centimeters (from $P_A$ to $P_B$, as shown in Figure~\ref{csibreath}(a)). This movement leads to dynamic path length changes that are typically very small. Thus, the overall CSI amplitude may not reach both its maximum and minimum values, meaning that $(|\vec{H_s}| - |\vec{H_d}|)<|\vec{H}|<(|\vec{H_s}| + |\vec{H_d}|)$, as illustrated in Figures~\ref{csibreath}(b) and (c). Furthermore, breathing is a continuous and smooth activity, meaning its duration ($\Delta T$) is long. 

%Consequently, it causes gradual and smooth changes in the Wi-Fi CSI amplitude over time.

We formulate the LLM prompting guidance for human breathing as follows:
\emph{``Breathing is a small-scale and smooth activity that causes slow and gradual changes in Wi-Fi CSI amplitude over time, with a moderate variation range.''}


\textbf{Modeling the No-event Scenario.} In this scenario, only static signal components exist, such as the LoS signals and the signals reflected by stationary objects, as depicted in Figures~\ref{csino}(a) and (b). Since no movement occurs, no dynamic component is introduced into the Wi-Fi signal propagation. As a result, the overall CSI amplitude is determined solely by the amplitude of these static signal components and remains nearly constant over time (i.e., $(|\vec{H}|=|\vec{H_s}|$), as illustrated in Figure~\ref{csino}(c).

We formulate the LLM prompting guidance for the absence of motion as follows: 
\emph{``For the no-event scenario, the time-series CSI amplitude remains stable, meaning the variation range is very small.''}



We further illustrate examples of real Wi-Fi signals corresponding to different human activities in Figure~\ref{signals}. These signal patterns align well with our physical model for Wi-Fi sensing, validating its effectiveness in characterizing human activities through Wi-Fi CSI.
By leveraging these physical insights, our derived prompts offer explicit guidance to LLMs, enabling them to interpret Wi-Fi signals based on Wi-Fi sensing principles. This allows for accurate recognition of human activities in a zero-shot manner.











%Only one significant peak or trough in the time series CSI amplitude. And there is a relatively stable period at the end of the data with minor variations. It is because falling is a large-scale and rapid activity and a person will likely stay static after falling.

\vspace{-2mm}
\subsection{Wi-Chat: LLMs for Wi-Fi-Based Activity Recognition}

This section outlines different prompting strategies for leveraging LLMs in Wi-Fi-based human activity recognition. We aim to explore how LLMs can interpret Wi-Fi signals and improve activity classification without extensive model training or complex signal processing.

\textbf{Base.} 
For the base setting, we provide the LLM with raw CSI amplitude data, represented as a time series, and prompt it to recognize human activity labels, represented as $\hat{a}=\argmax_a p_{M}(a\mid s)$. Additionally, incorporating the physical model of Wi-Fi sensing, as described in Section~\ref{phisical}, as \textbf{domain knowledge} may enhance the LLM's interpretability and performance.

\textbf{In-context Learning.}
Recent studies have demonstrated that LLMs exhibit strong few-shot learning capabilities across various tasks, a phenomenon known as in-context learning (ICL)~\cite{brown2020language}. By learning from these exemplars within a single inference session, the model can recognize patterns in the signals and improve its classification accuracy without additional fine-tuning. The standard ICL prompts a large language model, $M$, with a set of $k$ exemplar and predicts a activity $\hat{a}$ for the Wi-Fi signal by:
\vspace{-3mm}
\begin{equation}
    \hat{a}=\argmax_a p_{M}(a\mid s,\{(s^1,a^1)...(s^k,a^k)\}).
\end{equation}

\textbf{CoT.} Beyond simple input-output mappings, incorporating chain-of-thought (CoT) reasoning into prompts can further enhance the model's interpretability~\cite{nye2021show, wei2022chain}. By including explicit intermediate steps, CoT prompting helps the model better capture the relationships between signal variations and human activities. It could be represented as:
\vspace{-3mm}
\begin{equation}
   \hat{a}=\argmax_a p_{M}(a\mid s,C),
\end{equation}
where $C=\{(s^1,e^1,a^1)...(s^k,e^k,a^k)\}$ is the set of input-explanation-output triplets in prompts. 

\textbf{Multi-modal.}
Since raw Wi-Fi signals in numerical form can be difficult for LLMs to interpret, we extend our approach by incorporating visual representations. Specifically, we generate signal plots and present them as additional inputs, allowing the model to process both textual and visual information. This multi-modal strategy leverages the LLM's capability to analyze images, potentially improving activity recognition by making signal variations more interpretable.
\vspace{-3mm}
\begin{equation}
   \hat{a}=\argmax_a p_{M}(a\mid s,v),
\end{equation}
where $v$ represents the visual representation of the signal. By incorporating these plots, we aim to improve the interpretability of the Wi-Fi signals for LLMs, enabling more accurate activity recognition.

By exploring these prompting strategies, we aim to assess the feasibility of LLMs for Wi-Fi-based activity recognition and understand how different types of input representations influence their performance. 