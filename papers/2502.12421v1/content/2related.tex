
\section{Related Work}


\textbf{Wi-Fi Sensing.} Wi-Fi sensing has been extensively explored for various applications~\cite{qian2017inferring, xie2019md, ren2020liquid, jiang2020towards, ren2023person, li2024spacebeat, wang2024multi}, particularly in human activity recognition~\cite{wang2015understanding, zhang2017toward, wang2014eyes}, due to its non-contact nature and low cost.
For example, E-eyes~\cite{wang2014eyes} was the first work to leverage Wi-Fi signals for recognizing daily human activities. \citet{zhang2017toward} theoretically analyzed the perception capabilities of Wi-Fi signals and introduced a Fresnel zone model for human activity sensing. WiG~\cite{he2015wig} utilizes support vector machines to extract activity-related features from Wi-Fi. CARM~\cite{wang2015understanding} employs a hidden Markov model to extract temporal features from Wi-Fi for activity recognition. Additionally, \citet{yang2019learning} applied convolutional and recurrent neural networks to extract distinguishing features from Wi-Fi signals for human activity recognition.
While these conventional Wi-Fi sensing systems achieve strong performance, they typically rely on multi-stage signal processing techniques and require large amounts of data to train deep learning or machine learning models.

\noindent
\textbf{Large Language Model Applications.} Recently, the emergence of LLMs has revolutionized both academic NLP research and industrial
products due to their remarkable ability to understand, analyze, and generate texts with vast
amounts of pre-trained knowledge~\cite{zhao2023survey,zhang2023summit,liu2023pre,zhang2024systematic}. By leveraging extensive corpora of text data, LLMs can capture
complex linguistic patterns, semantic relationships, and contextual cues, enabling them to produce
high-quality responses. LLMs have also been applied beyond the scope of NLP, as a powerful tool that has
propelled the field of healthcare, legal and finance~\cite{chen2024survey,yuan2024structure,he2025survey}. 
More recently, researchers have explored the application of LLMs in sensing-related domains. The concept of Penetrative AI~\cite{xu2024penetrative} has been introduced to integrate LLMs with the physical world, enabling the analysis of sensor data through LLMs. Additionally, HARGPT~\cite{ji2024hargpt} and LLMTrack~\cite{yang2024you} have demonstrated the potential of LLMs in recognizing human activities and tracking robotic movements by analyzing Inertial Measurement Unit (IMU) data. These advancements highlight the growing potential of bridging LLMs with physical world sensing applications.

%... \textcolor{red}{add papers}