% \vspace{-2mm}
\section{Experiment}
% \vspace{-2mm}
\subsection{Wi-Chat Dataset}
% \vspace{-2mm}
% TODO
% more detailed introduction of dataset creation
% the rumour label in such datasets
\section{Data} \label{sec:data}
We use three rumour datasets in this work, namely: PHEME~\citep{pheme2015,kochkina-etal-2018-one}, Twitter15, and Twitter16~\citep{ma-etal-2017-detect}:

% TJB: how can the number of threads be greater than the number of tweets? these numbers don't make sense
% RX: fixed, the numbers were incorrect
\paragraph{PHEME}~\citet{pheme2015} contains 6,425 tweet posts of rumours and non-rumours related to 9 events. To avoid using specific a priori keywords to search for tweet posts, PHEME used the Twitter (now X) steaming API to identify newsworthy events from breaking news and then selected from candidate rumours that met rumour criteria, finally they collected associated conversations and annotate them. They engaged journalists to annotate the threads. The data were collected between 2014 and 2015. The 9 events are split into two groups, the first being breaking news that contains rumours, including Ferguson unrest, Ottawa shooting, Sydney siege, Charlie Hebdo shooting, and Germanwings plane crash. The rest are specific rumours, namely Prince to play in Toronto, Gurlitt collection, Putin missing, and Michael Essien contracting Ebola.
% TJB: say something about the time period when this data was collected
% RX: added

\paragraph{Twitter 15}~\citet{twitter15} was constructed by collecting rumour and non-rumour posts from the tracking websites snopes.com and emergent.info. They then used the Twitter API to gather corresponding posts, resulting in 94 true and 446 false posts. This dataset further includes 1,490 root posts and their follow posts, comprising 1,116 rumours and 374 non-rumours.
% TJB: the "tweet" vs. "comment" terminology is potentially confusing and needs to be clarified
% RX: unified, used root and follow posts to refer to root posts and the comment posts, posts are used to describe tweets in general.

\paragraph{Twitter 16}
Similarly to Twitter 15, \citet{twitter16} collected rumours and non-rumours from snopes.com, resulting in 778 reported events, 64\% of which are rumours. For each event, keywords were extracted from the final part of the Snopes URL and refined manually---adding, deleting, or replacing words iteratively---until the composed queries yielded precise Twitter search results. The final dataset includes 1,490 root tweet posts and their follow posts, comprising 613 rumours and 205 non-rumours.

\begin{table*}[!t]
    \centering
    \small
    \begin{tabular}{p{0.05\linewidth}p{0.9\linewidth}}
    \toprule
    Task & Prompt \\
    \midrule
    V-oc & Categorize the text into an ordinal class that best characterizes the writer's mental state, considering various degrees of positive and negative sentiment intensity. 3: very positive mental state can be inferred. 2: moderately positive mental state can be inferred. 1: slightly positive mental state can be inferred. 0: neutral or mixed mental state can be inferred. -1: slightly negative mental state can be inferred. -2: moderately negative mental state can be inferred. -3: very negative mental state can be inferred.\\
    \midrule
    E-c & Categorize the text's emotional tone as either `neutral or no emotion' or identify the presence of one or more of the given emotions (anger, anticipation, disgust, fear, joy, love, optimism, pessimism, sadness, surprise, trust).\\
    \midrule
    E-i & Assign a numerical value between 0 (least E) and 1 (most E) to represent the intensity of emotion E expressed in the text.\\
    \bottomrule
    \end{tabular}
    \caption{Prompts used for EmoLLM to detect emotion information in tweets. V-oc = Valence Ordinal Classification, E-c = Emotion Classification, and E-i = Emotion Intensity Regression.}
    \label{tab:emollm_ins}
\end{table*}


  
%%% Local Variables:
%%% mode: latex
%%% TeX-master: "../main_anonymous"
%%% End:

We conducted experiments using a self-collected Wi-Fi CSI dataset, leveraging commodity Wi-Fi devices, specifically Dell LATITUDE laptops, as both the Wi-Fi transmitter and receivers for data collection. Each Wi-Fi transmitter and receiver is equipped with three antennas. The Wi-Fi channel operated at 5.32 GHz with a bandwidth of 40 MHz, and the packet transmission rate was set to 1000 packets per second. We utilized the Linux 802.11 CSI tool~\cite{halperin2011tool} to extract CSI data from 30 OFDM subcarriers per packet. The dataset comprises over 1,965,000 Wi-Fi CSI packets collected from participants with varying heights, weights, and ages. These packets were segmented into 393 segments, each lasting 5 seconds, during which participants performed one of four activities: walking, falling, breathing, or no event (i.e., an empty environment). The collected data were then converted into both image and text representations, as detailed in Table~\ref{tab:data_stats}. Data collection was conducted across three real-world environments, a bedroom, a kitchen, and a living room, over a two-month period. The study was reviewed and approved by the IRB of the authors' institution.

%\textcolor{red}{add table}

\vspace{-1mm}
\subsection{Baselines}
\vspace{-2mm}
We compare Wi-Chat with the following systems: 

\textbf{Conventional Wi-Fi-based Systems.} These systems follow a multi-step pipeline, including signal denoising, signal transformation, feature extraction, and model construction, as described in Section~\ref{systems}. Specifically, we reproduce two well-known systems:
\textit{1) CARM}~\cite{wang2015understanding}: It utilizes a PCA-based method for signal denoising, applies DWT for feature extraction, and employs a Hidden Markov Model for activity recognition.
\textit{2) E-eyes}~\cite{wang2014eyes}: This system first removes data outliers using a low-pass filter and then builds activity classifiers using Earth Mover's Distance.

\textbf{Machine Learning Models with Raw Signals.} 
We evaluate the performance of machine learning models, including \textit{3) CNN}, \textit{4) RNN}, and \textit{5) SVM}. These models take textual or visual representations of raw Wi-Fi signals as input and are trained in a supervised manner using labeled datasets.

\subsection{Experimental Settings}

For LLMs, we first apply signal smoothing using the Savitzky-Golay filter~\cite{schafer2011savitzky} and then convert the signals into textual or visual representations. When experimenting with the few-shot setting, we pick 4 examples, including one example from each label class. The prompts used for experiments are presented in Appendix~\ref{sec:prompt}.

%For all supervised baselines, the dataset was randomly split into two parts: $70\%$ for training and $30\%$ for testing. Note that we perform the same signal smoothing and convert them into text and images. We train CNN and RNN on an NVIDIA GeForce RTX 4090 GPU. The learning rate of CNN and RNN is set to 0.001 and they both use Adam optimizer and the max Number of Epochs is 30. The batch size is set to 32 during training for both CNN and RNN. For the SVM, we use RBF as the Kernel Type. For CARM and E-eyes systems, we follow their signal processing procedures.
%Performance evaluation was conducted using standard classification metrics, including Accuracy, Precision, Recall, and F1 Score. Savitzky-Golay filter~\cite{ren2021winect}

For all supervised baselines, we randomly split the dataset into $70\%$ for training and $30\%$ for testing to ensure fair evaluation. Prior to model training, we apply the same signal smoothing techniques and convert the signals into textual and visual representations for consistency across methods. The CNN and RNN models are trained on an NVIDIA GeForce RTX 4090 GPU with a learning rate of 0.001 using the Adam optimizer, with a maximum of 30 epochs and a batch size of 32. For the SVM model, we use the Radial Basis Function as the kernel type. For CARM and E-eyes, we follow their original signal processing pipelines, including denoising, feature extraction, and model construction as described in their respective works. 
Zero-shot evaluations for CNN, RNN, SVM, and conventional systems are also conducted using the same approaches, but with untrained models to test their zero-shot performance. Performance evaluation is conducted using standard classification metrics, including accuracy, precision, recall, and F1 score, to assess the ability of each system to recognize human activities from Wi-Fi CSI data.

\vspace{-2mm}

\begin{lstlisting}[breaklines=true, caption={Qualitative Example 5: An example trace in the summarization domain, with \reactagent{} when using \texttt{gpt-4-turbo} as LLM.},label={ls:qualsummary}]
# Alice Anastasiou says: I'm interested in compiling a summary on the topic: governor, race, gop, obamacare, medicaid. Can you help me? 
>>>  Reflection.thought("I will first resolve the primary user to get the right contact information")
>>>  Enterprise.resolve_primary_user()
{"email": "alice@company.com", "full_name": "Alice Anastasiou", "user_id": "alice"}
>>>  Reflection.thought("I will search user's documents and find relevant people who may have more information")
>>>  EnterpriseSearch.search_documents(query="governor, race, gop, obamacare, medicaid")
>>>  EnterpriseSearch.search_relevant_people(query="governor, race, gop, obamacare, medicaid")
{"parker": "Knows about the following topics:\ngovernor elections, GOP, health care, Obama, Romney", "bhushan": "Knows about the following topics:\ngovernor elections, GOP, health care, Obama, Romney\nSiskiyou County, California, secession, state of Jefferson, rural representation", "farshid": "Knows about the following topics:\nSiskiyou County, secession, Jefferson, California, rural fire prevention fee", "eden": "Knows about the following topics:\nFacebook, gay rights, John Snow kiss-in, Richard Metzger, Paul Shetler\nFacebook, same-sex kissing, photo removal, protest", "hannah": "Knows about the following topics:\n", "dewei": "Knows about the following topics:\n", "cassie": "Knows about the following topics:\n"}
>>>  Reflection.thought("Parker and Bhushan have information about the governor race and GOP, so I should reach out to them.")
>>>  Enterprise.send_message("alice", "Hi Alice, I can reach out to Parker and Bhushan to produce a summary. Is that OK?")
>>>  System.finish()

# Alice Anastasiou says: Yes, please do that. 
>>>  Reflection.thought("I will reach out to Parker to ask for information about the governor race and GOP")
>>>  Enterprise.resolve_person("Parker")
{"email": "parker@company.com", "full_name": "Parker McLean", "user_id": "parker"}
>>>  Enterprise.send_message("parker", "Hi Parker, do you have any information about the governor race, GOP, Obamacare, or Medicaid?", title="Information about governor race and GOP")
>>>  System.finish()

# Parker McLean says: Yes, here's a relevant document I found: "gop eyes gains as voters in 11 states pick governors     enlarge this image toggle caption jim cole/ap jim cole/ap     voters in 11 states will pick their governors tonight , and republicans appear on track to increase [...] "
>>>  Reflection.thought("I will reach out to Bhushan to ask for information about the governor race and GOP")
>>>  Enterprise.resolve_person("Bhushan")
{"email": "bhushan@company.com", "full_name": "Bhushan Magar", "user_id": "bhushan"}
>>>  Enterprise.send_message("bhushan", "Hi Bhushan, do you have any information about the governor race, GOP, Obamacare, or Medicaid?", title="Information about governor race and GOP")
>>>  System.finish()

# Bhushan Magar says: Yes, here's a relevant document I found: "the real clear politics average for the race has mccrory maintaining a 14.3 percentage point lead .     montana : popular democratic gov . brian schweitzer - he won his last election with 65 percent of the vote - has [...]" 
>>>  Reflection.thought("I have enough information to compile the summary. I will send a message to Alice, with the summary enclosed in [[[ brackets ]]].")
>>>  Enterprise.send_message("alice", "Hi Alice, here's the summary I put together: [[[Voters in 11 states are choosing their governors, with Republicans poised to increase their numbers, potentially controlling over two-thirds of the nation's top state offices. The GOP is expected to take North Carolina and retain seats in Utah, North Dakota, and Indiana. Democrats are likely to hold West Virginia and Missouri, with competitive races in Montana, New Hampshire, and Washington. The governor's role is crucial in healthcare, as they can opt out of Medicaid expansion under the Affordable Care Act. A Romney win would empower Republican governors, especially in states where Obama's policies, like high-speed rail funding, were previously rejected.]]]", title="Governor race and GOP summary")

# Alice Anastasiou says: Thank you, this is exactly what I needed. 
>>>  Reflection.thought("Alice has confirmed that no more information is needed. I will close the session")
>>>  Enterprise.send_session_completed()
>>>  System.finish()
\end{lstlisting}

\subsection{Results}
\vspace{-2mm}


\textbf{Overall Results.} Table~\ref{summary} presents the best-performing models across different method categories. In the zero-shot category, the best model GPT-4o model achieved an accuracy of 0.62, demonstrating its ability to generalize effectively without task-specific examples. In the 4-shot category, GPT-4o is still the best model, exhibiting substantial improvement, attaining an accuracy of 0.77. This result highlights the effectiveness of in-context learning, where additional prompt examples help refine model predictions.

In the vision models category, the GPT-4o-mini with CoT demonstrated the strongest performance with an accuracy of 0.90. This result indicates the model's capacity to integrate visual and textual reasoning through CoT prompting, which likely aids in complex decision-making.

For supervised learning, the E-eyes (with complex signal processing techniques) outperformed all other models. This result is expected, as supervised models are explicitly trained on labeled data, allowing them to learn precise decision boundaries. However, despite the high accuracy, supervised methods typically require extensive labeled datasets, which may not always be feasible in real-world applications.

% \begin{table*}[t]
% \centering
% \small  % Adjust font size as needed
% \begin{tabular}{l|cccc}
% \hline
% \textbf{Evaluation Metrics}&\textbf{Accuracy} &\textbf{Precision} & \textbf{Recall} & \textbf{F1-score} \\
% \hline

% \multicolumn{5}{c}{\textbf{\textsl{Time Series}}} \\
% \hline
% \multicolumn{5}{c}{\textbf{\textsl{Vision}}} \\
% \hline
% \multicolumn{5}{c}{\textbf{\textsl{CNN, RNN, and SVM (Time Series)}}} \\
% \hline
% Zero-shot SVM & 0.27   & 0.28    & 0.28     &0.27\\
% Zero-shot CNN  & 0.23  & 0.24  & 0.23   & 0.23\\
% Zero-shot RNN & 0.26  & 0.26  & 0.26   & 0.26\\
% SVM & 0.94   & 0.92    & 0.91    & 0.91\\
% CNN & 0.98  & 0.98  & 0.97   & 0.97\\
% RNN & 0.99  & 0.99  & 0.99   & 0.99\\
% \hline
% \multicolumn{5}{c}{\textbf{\textsl{CNN, RNN, and SVM (Image)}}} \\
% \hline
% Zero-shot SVM & 0.26   & 0.25    & 0.25     &0.25\\
% Zero-shot CNN  & 0.26  & 0.25  & 0.26   & 0.26\\
% Zero-shot RNN & 0.28  & 0.28  & 0.29   & 0.28\\
% SVM & 0.99   & 0.99    & 0.99    & 0.99\\
% CNN & 0.98  & 0.99  & 0.98   & 0.99\\
% RNN & 0.98  & 0.99  & 0.98   & 0.98\\
% \hline
% \multicolumn{5}{c}{\textbf{\textsl{Conventional Wi-Fi-based Human Activity Recognition Systems}}} \\
% \hline
% Zero-shot E-eyes~\cite{wang2015understanding}  & 0.26  & 0.26  & 0.27   & 0.26\\
% Zero-shot CARM~\cite{wang2014eyes}  & 0.24  & 0.24  & 0.24   & 0.24\\
% E-eyes~\cite{wang2015understanding}  & 1   & 1    & 1     & 1\\
% CARM~\cite{wang2014eyes}  & 0.98  & 0.98  & 0.98   & 0.98\\
% \hline

% \end{tabular}

% \caption{Classification Results
% }
% \label{tab:result}
% \end{table*}

\begin{table}[th]
    \centering
    \small
    \begin{tabular}{lcc}
        \toprule
        \textbf{Method} & \textbf{Accuracy} & \textbf{F1-score} \\
        \midrule
        \multicolumn{3}{c}{\textbf{Zero Shot}} \\
         \midrule
         E-eyes & 0.26 & 0.26 \\
         CARM & 0.24 & 0.24 \\
         SVM & 0.27 & 0.27 \\
         CNN & 0.23 & 0.23 \\
         RNN & 0.26 & 0.26 \\
        Vision SVM & 0.26 & 0.25 \\
        Vision CNN & 0.26 & 0.26 \\
        Vision RNN & 0.28 & 0.28 \\
        GPT-4o & 0.62 & 0.42 \\
        GPT-4o+ ICL & 0.77 & 0.73 \\
        GPT-4o-mini + COT & \textbf{0.90} & \textbf{0.90} \\
        \midrule
        \multicolumn{3}{c}{\textbf{Supervised}} \\
         \midrule
        Vision CNN & 0.98 & 0.98 \\
        CARM & 0.98 & 0.98 \\
        E-eyes & 1.00 & 1.00 \\
        \bottomrule
    \end{tabular}
    
    \caption{Performance comparison of different methods under the zero-shot and supervised settings.}
%\vspace{-5mm}
    \label{present}
\end{table}





Overall, the results indicate that LLMs exhibit strong performance in zero-shot and few-shot settings for the task of Wi-Fi-based human activity recognition, making them valuable for scenarios with limited annotated data. Additionally, the impressive accuracy of the vision-language model suggests promising directions for integrating multimodal learning into the task.

% \section{Result} \label{sec:result}

\subsection{Setup}

In this section, we evaluate VB-Com across the following perspectives:
\begin{itemize}
    \item Under what conditions does VB-Com demonstrate superior performance compared to using a single-policy approach?
    \item How does VB-Com outperforms baseline methods in those scenarios?
    \item How well does the proposed return estimator contribute to the composition system?
\end{itemize}

\begin{figure}[h]
\centering{\includegraphics[width=0.5\textwidth]{figures/noise.png}}
\caption{We present four types of perception noises and implement them on heightmaps during evaluation: gaussian noise, \textcolor{red}{forward shifting noise}, \textcolor{green}{lateral shifting noise} and \textcolor{blue}{floating noise}.}
\label{noise}
\end{figure}

\subsubsection{Evaluation Noise}
To simulate situations where the robot encounters perception outliers not present in the simulation, we introduce a quantitative curriculum noise designed to mimic varying levels of perception deficiency. As shown in Fig. \ref{noise}, we focus on four types of noise: (1) \textbf{Gaussian noise}: noise points sampled from a Gaussian distribution, to the original heightmap. The noise level is scaled from 0.0 to 1.0, where the training noise level corresponds to a 0.1 noise level in this scenario. (2) \textbf{Shifting noise}: replacing points in the original heightmap with noise sampled from a Gaussian distribution. The range of replacement points is controlled by the noise level, where a $100\%$ noise level results in a fully noisy heightmap. The shifting direction can either be along the heading direction (red box) or sideways (green box). (3) \textbf{Floating noise}: The heightmap is displaced vertically, either upwards or downwards, the floating noise simulates variations in terrain height. (blue box).

\begin{table}[!ht]
\caption{Terrain Size Scales (m)}
\label{tab:terrains}
\begin{center}
\renewcommand\arraystretch{1.25}
\begin{tabular}{lcccc}
\toprule[1.0pt]
Terrain & Length & Width & Heights\\
\midrule[0.8pt]

Gaps        & $(0.6, 1.2)$ & $(\bm{0.6}, \bm{0.8})$ & $(-1.8, -1.5)$\\  
Hurdles     & $(0.8, 1.0)$ & $(0.1, 0.2)$ & $(\bm{0.2}, \bm{0.4})$\\  
Obstacles   & $(\bm{0.2}, \bm{0.4})$ & $(0.2, 0.4)$ & $(1.4,1.8)$\\  

\bottomrule[1.0pt]
\end{tabular}
\end{center}
\end{table}

\subsubsection{Experiments Setup}
In simulation, we conduct $10 \times 3$ experiments for each method across three types of terrain, replicating the experiments three times to calculate the variance. Each episode involves the robot navigating through 8 goal points, with each goal paired with a corresponding challenging terrain or obstacle. The size of the terrains is set to the maximum curriculum terrain level, as shown in Table \ref{tab:terrains}. The bolded values indicate the primary factors that contribute to the difficulty for the terrain.

\subsubsection{Baselines}
We primarily compare VB-Com with the vision and blind policies operating independently. Additionally, as previous works have shown that robust perceptive locomotion can be learned by incorporating various perception noises during training \cite{miki2022learning}, we add a \textbf{Noisy Perceptive policy baseline} trained using the same noises implemented in the evaluation. This allows us to examine how well the proposed VB-Com policy performs compared to policies that have already seen the evaluation noises. The evaluation noises are introduced to the Noisy Perceptive policy in a curriculum format during training, which evolves with the terrain level.

\begin{figure*}[h]
\centering{\includegraphics[width=\textwidth]{figures/returnsim.png}}
\caption{Illustrations of the variation in estimated return and action phases(0 for $a_b$ and 1 for $a_v$) across three concerned terrains.}
\label{return}
\end{figure*}

\subsection{Example Case}
First, we illustrate how VB-Com operates, specifically when the composition switches to $\pi_b$ and how it effectively controls the robot to traverse the terrain against deficient perception (Fig. \ref{return}). We demonstrate $3$ seconds of the estimated returns, along with the policy composition phase, as the robot walking through the challenging terrain during the simulation experiments at the noise level of $100\%$. Before the robot encounters challenging terrains, we observe that the estimated return $G^e_{\pi_v}(s_t)$ consistently exceeds $G^e_{\pi_b}(s_t)$, as the robot is walking on flat ground with relatively stable motion. This observation aligns with the discussion in Section \ref{subsec:vb-com}, where it was explained that $\pi_v$ benefits from the external state observation and results in a higher return $G_t$. This characteraistic ensures the robot operates at $\pi_b$ while stable motion. 

Once the deficient perception reaches the $100\%$ noise level, the robot will not be aware of the incoming challenging terrains until it collides with them. At this point, we observe that $G^e_{\pi}(s_t)$ drops sharply within several control steps, prompting the switch to the blind policy. This switch allows the robot to respond to the terrain, and once the motion stabilizes, $G^e_{\pi}(s_t)$ returns to a normal level, at which point the vision policy regains control. These cases demonstrate the effectiveness of VB-Com, which responds quickly to deficient perception, but avoids unnecessary switches to the blind policy when it is not needed.


\begin{table*}[!h]
\caption{VB-Com Evaluations}
\label{tab:VB-Com}
\begin{center}
\renewcommand\arraystretch{1.25}
\begin{tabular}{lccccccc}
\toprule[1.0pt]
Noise Level &Method & Goals Completed($\%$) & Rewards & Average Velocity & Fail Rate & Collision Steps($\%$) & Reach Steps\\
\midrule[0.8pt]

% \multirow{4}{*}{Prop Advisor}&0.25& $0.7560$& $0.7964$& $0.7001$ & \multirow{4}{*}{$0.8600$}\\

\multirow{2}{*}{0\% noise} & VB-Com & $84.05 \pm 2.28$ & \bm{$142.07 \pm 4.19$} & $0.71 \pm 0.01$ & \bm{$0.29 \pm 0.01$} & $1.50 \pm 0.14$ & $177.29 \pm 4.66$\\  
                              & Vision & $73.57 \pm 4.97$ & $118.07 \pm 10.42$ & $0.73 \pm 0.01$ & $0.42 \pm 0.07$ & \bm{$1.39 \pm 0.53$} & $204.82 \pm 28.91$\\  \midrule
\multirow{2}{*}{30\% noise} & VB-Com & $82.24 \pm 6.6$ & $132.81 \pm 7.64$ & $0.71 \pm 0.01$ & $0.34 \pm 0.10$ & $2.09 \pm 0.13$ & $178.13 \pm 4.13$\\  
                              & Vision & $72.76 \pm 2.29$ & $115.20 \pm 2.43$ & $0.75 \pm 0.02$ & $0.43 \pm 0.05$ & $2.52 \pm 0.32$ & $195.58 \pm 21.98$\\  \midrule
\multirow{2}{*}{70\% noise} & VB-Com & $82.48 \pm 1.20$ & $132.44 \pm 6.17$ & $0.70 \pm 0.02$ & $0.33 \pm 0.03$ & $2.12 \pm 0.11$ & $184.81 \pm 4.47$\\  
                              & Vision & $55.38 \pm 3.33$ & $58.24 \pm 13.97$ & $0.73 \pm 0.03$ & $0.67 \pm 0.07$ & $6.08 \pm 0.82$ & $190.50 \pm 18.28$\\  \midrule
\multirow{3}{*}{100\% noise} & VB-Com & \bm{$84.81 \pm 6.45$} & $129.99 \pm 9.84$ & $0.72 \pm 0.02$ & \bm{$0.29 \pm 0.08$} & $2.60 \pm 0.68$ & $182.29 \pm 11.47$\\  
                              & Vision & $48.71 \pm 5.60$ & $47.53 \pm 17.55$ & $0.70 \pm 0.06$ & $0.69 \pm 0.06$ & $6.92 \pm 1.36$ & $268.40 \pm 57.11$\\  
                              & Noisy Perceptive & $80.52 \pm 0.91$ & $116.94 \pm 4.07$ & \bm{$0.76 \pm 0.02$} & $0.32 \pm 0.04$ & $3.49 \pm 0.38$ & \bm{$154.98 \pm 4.41$}\\ \midrule
& Blind & $83.76 \pm 1.35$ & $131.29 \pm 3.48$ & $0.70 \pm 0.01$ & $0.33 \pm 0.05$ & $2.57 \pm 0.27$ & $184.08 \pm 1.85$\\  

% Perceptive  & $0.00 \pm 0.00$ & $0.00 \pm 0.00$ & $0.00 \pm 0.00$ & $0.00 \pm 0.00$ & $0.00 \pm 0.00$\\  
% Blind  & $0.00 \pm 0.00$ & $0.00 \pm 0.00$ & $0.00 \pm 0.00$ & $0.00 \pm 0.00$ & $0.00 \pm 0.00$\\  
% Noisy Perceptive & $0.00 \pm 0.00$ & $0.00 \pm 0.00$ & $0.00 \pm 0.00$ & $0.00 \pm 0.00$ & $0.00 \pm 0.00$\\  

\bottomrule[1.0pt]
\end{tabular}
\end{center}
\end{table*}

\subsection{Evaluations on Different Noise Levels}
\textbf{VB-Com achieves robust locomotion performance under different levels of perception deficiency.} As shown in Tab \ref{tab:VB-Com}, performance of the vision policy declines shaprly with the arise of noise level. In addition, since the evaluation experiments set the terrain curriculum to the maximum level, the vision policy struggles even at a $0\%$ noise level: It only achieves around $73\%$ goal-reaching success, with a termination rate exceeding $40\%$. This poor performance is likely due to the severe challenge terrains, such as the farthest range of the heightmap $(0.85m)$ is only $0.05m$ wider than the width of the gaps$(0.8m)$. In contrast, VB-Com achieves a stable higher goal-reaching success against different levels of perception deficiency. In contrast, VB-Com achieves consistently higher goal-reaching success across varying levels of perception deficiency, including both noise and perception range limitations.

Despite the high goal-reaching success, we also include additional metrics to further analyze the performance. The reward values recorded throughout each episode indicate the proposed method’s ability to achieve both goal completion and collision avoidance. These rewards strongly correlate with the robot’s success in reaching the target while minimizing collisions. For instance, VB-Com at the $0\%$ noise level achieves the highest rewards$(142.07)$, although the goal completion rate$(84.05)$ is slightly lower compared to the trail in $100\%$ noise level $(84.81)$. This is because VB-Com switches to the blind policy more often in  $100\%$  noise level, resulting in more frequent collisions and lower rewards obtained. 

The reach steps metrics indicates the smoothness of the policy in overcoming challenging obstacles. Since the switching mechanism requires several steps to respond effectively, VB-Com results in a higher number of reach steps as the noise level increases. This is because, under higher noise conditions, the system needs additional time to transition from the vision policy to the blind policy, which leads to more gradual and controlled responses to terrain challenges.
\begin{figure}[h]
\centering{\includegraphics[width=0.5\textwidth]{figures/noiseevalueate.png}}
\caption{We compare the collision and goal-reaching performances under different noise levels. VB-Com achieves low collisions and high success rates with accurate perception, and its success rate remains high under deficient perception.}
\label{noiseevalueate}
\end{figure}

\begin{figure}[h]
\centering{\includegraphics[width=0.5\textwidth]{figures/terraineval.png}}
\caption{Comparisons between the Noisy Perceptive policy and VB-Com in navigating gaps and hurdles separately.}
\label{terraineval}
\end{figure}


\subsection{Comparisons with Blind Policy}
\textbf{VB-Com achieves less collision than the blind policy when perception becomes less dificient.} As shown in Tab \ref{tab:VB-Com}, the blind policy achieves a relatively high Goals Completed rate $(83.76\%)$, as its performance is unaffected by deficient perception. Therefore, we include an evaluation of the collision performance between VB-Com and the blind policy to further highlight the advantage of the proposed framework. In our evaluations, "Collision Steps" is defined as the ratio of the number of steps during which the robot collision model (Fig \ref{robot}) makes illegal contact with the terrain or obstacles, relative to the total number of steps within an episode.

We can observe from Tab \ref{tab:VB-Com} that the collision steps increase with the noise level for VB-Com. Fig \ref{noiseevalueate} provides a more intuitive illustration: as perception becomes more comprehensive, VB-Com achieves both fewer collisions and better goal-reaching performance. In contrast, the blind policy maintains a high goal-reaching rate but results in more collisions, while the vision policy performs better in avoiding collisions when the perception is accurate and comprehensive. As the noise level increases, the performance of VB-Com begins to resemble that of the blind policy. These results demonstrate the effectiveness of the composition system, which benefits from both sub-policies to achieve better performance in terms of both goal-reaching and minimizing collisions.

\subsection{Comparisons with Noisy Perceptive Training}
\textbf{Compared to policies trained with noisy priors, VB-Com achieves equivalent performance without prior knowledge of the noise, while also demonstrating better training efficiency and the ability to handle more challenging terrain difficulties.} The comparisons (Tab \ref{tab:VB-Com}) with Noisy Perceptive policy show that the Noisy Perceptive policy achieves a relatively high goal completion rate $(80.52\%)$ but exhibits a higher collision step rate $(3.49\%)$. It can be concluded that, as severe noise is introduced during evaluation, the heightmap quickly becomes random noise with the increasing noise level. In response, the Noisy Perceptive policy begins to exhibit behavior similar to that of the blind policy—making contact with obstacles and reacting when the noisy signals overwhelm the external observations.

To further investigate the conditions under which the Noisy Perceptive policy fails to surpass the performance of VB-Com, we evaluate goal-reaching performance under different terrains (Fig. \ref{terraineval}). The results show that VB-Com outperforms the Noisy Perceptive policy in gap terrains, while the Noisy Perceptive policy performs better in hurdle situations, achieving a higher success rate in preventing the robot from being tripped by hurdles. However, recovering from missed gaps requires a quicker response, or the robot risks falling. These results demonstrate that the single-policy method fails to handle such dynamic challenges effectively, highlighting the advantages of the composition in VB-Com.

\begin{figure}[h]
\centering{\includegraphics[width=0.5\textwidth]{figures/trainplot.png}}
\caption{Training curves for terrain levels and the return estimation loss.}
\label{train}
\end{figure}

Moreover, the terrain level rises slowly for the Noisy Perceptive policy (Fig. \ref{train}-(a)), and it fails to reach the maximum level achieved by the vision and blind policies. This is because the policy struggles with the trade-off of whether to trust the external perception, which requires the addition of an extra module to address the challenge. This slow progression highlights the difficulty of handling high levels of perception deficiency, whereas VB-Com can efficiently navigate such situations by leveraging the strengths of both the vision and blind policies.

\begin{table}[!ht]
\caption{Return Estimation Evaluations}
\label{tab:RE}
\begin{center}
\renewcommand\arraystretch{1.25}
\begin{tabular}{lcccc}
\toprule[1.0pt]
Method & Goals Completed($\%$) & Collisions & Reach Steps\\
\midrule[0.8pt]

100-steps) & $78.24 \pm 1.86$ & \bm{$2.49 \pm 0.04$} & $193.7 \pm 3.2$\\  
RE(50-steps)  & \bm{$81.90 \pm 2.81$} & $2.75 \pm 0.17$ & $184.6 \pm 1.4$\\ 
Re(5-steps)   & $69.90 \pm 7.34$ & $5.23 \pm 0.59$ & $192.6 \pm 3.3$\\  
Re(1-step)    & $59.57 \pm 2.00$ & $4.78 \pm 0.16$ & \bm{$167.4 \pm 5.0$}\\  
MC-based      & $74.14 \pm 2.69$ & $4.26 \pm 0.56$ & $192.8 \pm 11.8$\\  

\bottomrule[1.0pt]
\end{tabular}
\end{center}
\end{table}

\subsection{Return Estimator Evaluations}
\textbf{The proposed return estimator achieves accurate and efficient return estimation with accessible states observations.} Since we update the return estimator using temporal difference, we compare it with the Monte Carlo-based search return estimator that estimate the furtuen expected returns with the following regression loss directly: $\mathbb{E}_t[\hat{G}_{\pi_i}^e(s_t) - \sum_{t} ^ {t+T} \gamma^t r(s_t, a_t)]$. As shown in Fig. \ref{train}-(a), the MC-based estimator struggles to converge due to the accumulation of noise. In contrast, the proposed TD-based return estimator within the vision policy convergent stably as it updates alongside the locomotion policy. The results in Tab \ref{tab:RE} further highlight the ineffectiveness of the MC-based return estimator in providing accurate estimations to guide the policy composition. Specifically, the MC-based estimator struggles to respond promptly to collisions with obstacles, this delay in response leads to larger collisions and longer reach steps, as the policy cannot effectively adjust its actions in real-time. 

\textbf{We also evaluate the impact of different switch periods (T), which define the expected return duration during return estimator updates.} While training performance remains consistent across varying periods, we observe that excessively short switch periods can negatively impact system performance. In such cases, the two policies may conflict, resulting in incomplete motion trajectories when traversing the challenging terrains and failures.

\textbf{We observe that training effectiveness is highly dependent on data variance.} For instance, the estimator within vision policy converges the fastest due to its access to more accurate and comprehensive state observations, leading to fewer low-return instances. In contrast, the estimator within Noisy Perceptive and blind policies encounter more collisions and lower returns, causing their loss to degrade more slowly.

\textbf{We demonstrate that the estimated return threhold $G_{th}$ is crucial to the performance of VB-Com.} Tab \ref{tab:TH} evaluates the system's performance under different values of $\alpha$, as well as without $G_{th}$. The results demonstrate that $G_{th}$ is critical for mitigating miscorrection during motion abnormalities, and that a value of $\alpha < 1.0$ ensures a sensitive response to the states that could lead to motion failures.

\begin{table}[!ht]
\caption{Estimated Return Threhold Evaluations}
\label{tab:TH}
\begin{center}
\renewcommand\arraystretch{1.25}
\begin{tabular}{lcccc}
\toprule[1.0pt]
Method & Goals Completed($\%$) & Collisions & Reach Steps\\
\midrule[0.8pt]
 
$\alpha = 2.0$   & $77.10 \pm 4.71$ & $2.63 \pm 0.68$ & $185.11 \pm 7.17$\\ 
$\alpha = 0.5$   & \bm{$85.76 \pm 2.88$} & $2.29 \pm 0.17$ & $186.96 \pm 3.83$\\  
$\alpha = 0.1$   & $84.43 \pm 1.23$ & \bm{$2.10 \pm 0.25$} & $\bm{184.35 \pm 6.27}$\\  
w/o $G_{th}$     & $48.48 \pm 1.28$ & $6.24 \pm 0.41$ & $261.96 \pm 35.63$\\  

\bottomrule[1.0pt]
\end{tabular}
\end{center}
\end{table}



\subsection{Real-World Experiments}

We deploy the proposed system on both the Unitree G1 and Unitree H1 robots and evaluate the performance of the proposed VB-Com method. 
\begin{figure*}[h]
\centering{\includegraphics[width= \textwidth]{figures/hardwarecurve.png}}
\caption{Illustrations of the variation in estimated return under static/dynamic obstacles in hardware experiments.}
\label{hardwarecurve}
\end{figure*}

\subsubsection{Hardware Return Estimations}

We illustrate how VB-Com operates on real robots by plotting $4$ seconds of the estimated return while the robot avoids static (left) and dynamic (right) obstacles (Fig \ref{hardwarecurve}). The results demonstrate that, for static obstacles (a standing person), the elevation map can accurately perceive the obstacle, allowing the robot to plan motions in advance and avoid collisions. Corresponding to this behavior, we observe that the estimated return on the G1 stays a high value, with $\hat{G}^e_{\pi_b}$ slightly lower than $\hat{G}^e_{\pi_v}$, consistent with the scenario where the vision policy continues to operate within VB-Com.

On the other hand, when a person moves towards the robot at high speed, the perception module fails to detect the obstacle, causing a collision, both $\hat{G}^e_{\pi_b}$ and $\hat{G}^e_{\pi_v}$ decline sharply upon collision. However, VB-Com quickly switches to $\pi_b$ to avoid the person, demonstrating the  \textbf{rapid response to collision provided by the proposed return estimation and the successful obstacle avoidance capability of VB-Com under perceptual deficiency}.


\begin{figure}[h]
\centering{\includegraphics[width=0.5\textwidth]{figures/g1avoid.png}}
\caption{ Real-world comparisons of VB-Com, vision, and blind policies in obstacle avoidance on the G1.}
\label{avoid}
\end{figure}

\subsubsection{Avoid Obstacles}
In this section, we make comparisons between VB-Com along with the vision policy and blind policy on G1 (Fig \ref{avoid}), to demonstrate the superior performance of VB-Com in hardware compared with signle policies. In the evaluation scenario, G1 encounters two consecutive obstacles along its path. The second dynamic obstacle obstructs the robot's direction before the elevation map can perceive it. VB-Com enables the robot to avoid the static obstacle without collision and subsequently avoid the dynamic obstacle after it collides with the suddenly appearing obstacle.

In contrast, for the baseline policies, the blind policy makes unnecessary contact with the static obstacles before avoiding them, which damages the environment. As for the vision policy, the robot collides with the obstacle and is unable to avoid it until the newly added obstacle is detected and integrated into the map.

\begin{figure}[h]
\centering{\includegraphics[width=0.5\textwidth]{figures/hurdlegap.png}}
\caption{Hardware demonstrations on the robots traversing gaps and hurldes given deficient perception with VB-Com.}
\label{hurdlegap}
\end{figure}

\subsubsection{Performance Against Deficient Perception}
In this section, we demonstrate the ability of VB-Com to traverse challenging terrains given deficient perception (Fig. \ref{hurdlegap}). We provide zero inputs for the heightmaps to evaluate the performance of VB-Com under perceptual deficiency. We introduce two consecutive hurdles, and the robot successfully recovers after colliding with them by switching to $\pi_b$. Additionally, we demonstrate that VB-Com enables recovery from a missed step on an unobserved gap. In this case, VB-Com saves the robot by performing a larger forward step to traverse the gap without perception, as the blind policy has learned during simulation.




\textbf{Methods Comparison.}
Table~\ref{present} summarizes the classification results across different methods. In the zero-shot setting, traditional machine learning models such as SVM, CNN, and RNN demonstrate relatively low classification performance. LLM models GPT-4o achieved an accuracy of 0.61, significantly outperforming traditional machine learning models. The ICL approach further improved performance, demonstrating the benefits of in-context learning.


Furthermore, GPT-4o-mini combined with CoT reasoning achieves the highest accuracy among zero-shot methods at 0.90, demonstrating the effectiveness of advanced prompting techniques in enhancing LLM-based classification performance. Notably, this performance is already comparable to conventional Wi-Fi activity recognition systems and machine learning models trained in supervised settings. These results reaffirm that while supervised models achieve superior performance with labeled data, LLMs exhibit strong generalization capabilities, particularly when leveraging few-shot learning and vision-language integration.



\subsection{Analysis}

\begin{figure}[tb]
    \centering
    \includegraphics[width=0.49\textwidth]{figures/bar.png}
    %\vspace{-8mm}
    \caption{Comparison of COT and knowledge}
   % \vspace{-3mm}
    \label{bar}
\end{figure}


\textbf{Effectiveness of CoT and Domain Knowledge.}
Figure~\ref{bar} illustrates the impact of CoT reasoning and domain knowledge on different language models across time series and vision-based settings. Among the zero-shot models, GPT-4o-mini demonstrates notable improvements when domain knowledge is incorporated, highlighting the benefits of integrating prior knowledge into the inference process. However, a substantial performance drop is observed in both models when CoT reasoning is applied to time series data. We argue that this decline stems from the inherent complexity of time series signals, particularly in the case of Wi-Fi raw signals, which are challenging to interpret directly. The step-by-step reasoning introduced by CoT may inadvertently add confusion, as the model struggles to generate coherent intermediate steps for highly dynamic and noisy input sequences.  

For vision-enhanced models, the GPT-4o-mini-Vision variant achieves the highest performance, with CoT prompting yielding an accuracy of 0.90. This suggests that CoT reasoning is particularly effective when paired with visual input, likely because images provide additional context that facilitates structured reasoning. A similar, albeit less pronounced, effect is observed for GPT-4o-Vision. These results indicate that while CoT reasoning can be beneficial, its effectiveness varies depending on the specific vision-language model and its alignment with the task. The structured nature of visual input may better support multi-step reasoning, whereas time series data lacks the same level of interpretability, limiting the effectiveness of CoT in those scenarios.  

\begin{figure}[ht]
    \centering
    \includegraphics[width=0.49\textwidth]{figures/box.png}
    %\vspace{-8mm}
    \caption{Comparison of LLMs under zero-shot settings}
    %\vspace{-3mm}
    \label{box}
\end{figure}

\textbf{Comparing Different LLMs.} Figure~\ref{box} presents the accuracy distribution of different LLM variants (Base, CoT, and Knowledge) under zero-shot settings. The box plots illustrate the variability in accuracy across models, with GPT-4o and DeepSeek demonstrating the highest median performance, while Mistral and LLaMA exhibit greater variance and lower median accuracy.

Among the models, GPT-4o-mini and DeepSeek show a more stable accuracy distribution, indicating consistent performance across different variants. In contrast, LLaMA has a wider spread, suggesting that its performance is more sensitive to the specific reasoning approach used. Gemma2 maintains a relatively narrow distribution, indicating lower variance but also a more limited improvement potential.

These results highlight that larger models like GPT-4o benefiting from more robust reasoning capabilities, while smaller models show varying degrees of improvement depending on the applied enhancements.

