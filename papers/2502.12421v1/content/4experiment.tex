% \vspace{-2mm}
\section{Experiment}
% \vspace{-2mm}
\subsection{Wi-Chat Dataset}
% \vspace{-2mm}
\begin{figure*}
    \centering
    \includegraphics[width=1\linewidth]{bar2.pdf}
    \caption{(a) shows the bar chart of the raw data, (b) presents the results of applying Moving Average Smoothing to reduce anomalies in prediction percentages, and (c) highlights the reduction of visual clutter and emphasizes sequential behavior patterns after merging behaviors of the same category.}
    \label{fig:bar}
    \Description{(a) shows the bar chart of the raw data, (b) presents the results of applying Moving Average Smoothing to reduce anomalies in prediction percentages, and (c) highlights the reduction of visual clutter and emphasizes sequential behavior patterns after merging behaviors of the same category.}
\end{figure*}

\section{Data Collection and Processing}
\label{sec:data}
\RR{In this section, we provided an overview of the data collection context and introduced the collaborative programming performance framework along with its metric quantification methods.}

\subsection{Data Collection}
We collaborated with Professor E1, an expert in programming education, and teaching assistants (TA1 and TA2), experienced in Python, to collect data from E1's Spring 2023 Python course with 66 non-computer science freshmen in 22 groups. Using non-intrusive methods, we recorded group discussions, screen activities (without audio), and code submissions. Session lengths ranged from 10 to 60 minutes based on question completion. 
Due to data quality issues, we selected data from 19 groups (57 students) for analysis.


\subsection{Data Preprocessing}
In collaborative programming analysis, students' spoken content was key to understanding discussion and evaluating collaboration. We used the Faster-Whisper model~\cite{fasterwhisper} for speech recognition and the Pyannote-audio model~\cite{pyannoteaudio} for speaker diarization. 
For groups lacking clear problem-solving strategies, we used Tesseract OCR~\cite{tesseract} to analyze screen recordings and extract key frames through screenshots.

\subsection{Scope of Collaborative Programming Performance Framework}
Evaluating student and group performance in collaborative programming required considering multiple dimensions~\cite{hawlitschek2023empirical}.  
Building on literature and expert input (E1), we proposed the following comprehensive analytical framework to assess performance. 



\subsubsection{Student Performance Assessment}
\label{shema}
Previous research demonstrated that students' skills, backgrounds, and personalities in the classroom vary significantly, affecting their engagement and learning outcomes~\cite{wu2019analysing}. 
Therefore, we focus on each student's \textit{background} (prior academic performance and major), \textit{role transitions}, \textit{behavioral engagement}, and \textit{cognitive engagement}.






\textbf{Problem-solving Categorization:}
Based on previous frameworks~\cite{wu2019analysing}, team theory~\cite{zhao2023analysing}, and collaborative coding processes~\cite{sun2021three}, we developed a coding scheme (Fig.~\ref{fig:scheme}) to capture group problem-solving in collaborative programming. 
The scheme used four color-coded categories to represent discussion types. 
The first three categories followed a hierarchical structure, indicating discussion depth, while the green category focuses on situation awareness and specific behaviors.

Building on the scheme, we used tailored prompts with the ChatGPT-4o model~\cite{gpt4o} to classify behavioral patterns in transcribed dialogue \RR{(More details are in appendix B)}. 
\RR{The model provided a prediction percentage of uncertainty for each classification, improving result interpretability. }
To minimize anomalies, we applied a ``moving window'' technique with Moving Average Smoothing~\cite{chang2022muse}, stabilizing prediction percentages (Fig.\ref{fig:bar}-b). To reduce visual clutter in long time-series data, we aggregated consecutive instances of the same category, averaging prediction percentages (Fig.\ref{fig:bar}-c). These results were displayed in the timeline panel's progress bar, enabling detailed analysis by zooming into specific behavior categories in Sec.~\ref{barchart}. 




\textbf{Roles Extraction:}
We analyzed each speaker's dynamic roles (Driver, Navigator, and Monitor) during programming~\cite{lewis2011pair}. Using ChatGPT-4o and prompts based on the Thought Chain Model~\cite{wei2022chain}, we guided the model through step-by-step reasoning to generate role classifications. Prompts were iterated for clarity, and the model's responses were structured hierarchically and returned in JSON format. Each query was repeated ten times, with the majority result adopted for classification.

\RR{\textbf{Behavioral Engagement:} reflected the level of effort and participation students invested in learning~\cite{fredricks2022measurement}. 
In our study, we focused on the duration and frequency of student speech.} 
We extracted conversation data, excluding irrelevant chat, and divided each conversation into two parts: the first half and the full conversation. We then measured speaking duration, frequency, and degree centrality using co-occurrence networks~\cite{ng1999toward}. For each question, we created and normalized two networks, followed by Non-negative Matrix Factorization (NMF)~\cite{lee2000algorithms} to identify key behavioral patterns for dynamic group comparison.


\RR{\textbf{Cognitive Engagement:} referred to the cognitive investment students made in their learning. We highlighted the role changes and behavior frequencies of students during the collaborative process. }
To capture dynamic changes in student cognitive engagement, we split the dialogue for each question into two segments: the first half and the full dialogue. We extracted the frequency of each speaker's 14 behavioral categories and their roles at each timestamp. After normalizing these features for consistency, we applied NMF to reduce dimensionality and assess each speaker's cognitive engagement.

\begin{figure*}
  \includegraphics[width=\textwidth]{CPVis.pdf}
  \caption{\RR{A screenshot of Group 10 view.} \textit{CPVis} applies multimodal learning analysis to provide instructors with evidence for evaluating group and student performance. It consists of three views:
Filter View (A) Provides an overview and allows group selection. The selected groups appear in the lasso selection area (A2), and the similarity panel (A3) displays the most similar and different groups based on the search (A1a).
Content View (B) Displays group performance, with the B1 panel showing completed codes, the B3a panel illustrating the behavior sequence, and the B3b panel showing student engagement over time.
Detail View (C) Presents the group's collaborative programming video (C1) and raw conversation data (C2).}
  \Description{A screenshot of Group 10 view. \textit{CPVis} applies multimodal learning analysis to provide instructors with evidence for evaluating group and student performance. It consists of three views:
Filter View (A) Provides an overview and allows group selection. The selected groups appear in the lasso selection area (A2), and the similarity panel (A3) displays the most similar and different groups based on the search (A1a).
Content View (B) Displays group performance, with the B1 panel showing completed codes, the B3a panel illustrating the behavior sequence, and the B3b panel showing student engagement over time.
Detail View (C) Presents the group's collaborative programming video (C1) and raw conversation data (C2).}
  \label{fig:teaser}
  \end{figure*}

\subsubsection{Group Performance Assessment}
We evaluated group performance based on three dimensions: code quality, collaborative problem-solving, and teacher scaffolding. 
Through in-depth discussions with domain experts, we assessed how each dimension was valued and measured in the context of our study.




\label{code}
\textbf{Code quality}, reflecting students' mastery of course concepts, was a key metric for evaluating group performance. To assess student submissions, we used ChatGPT-4o~\cite{gpt4o} to evaluate dimensions such as problem-solving, code integrity, accuracy, and algorithmic innovation, scoring each on a 1–5 scale. After refining evaluation prompts, we ran the assessment ten times per submission, averaging the results to ensure consistency and reliability.





\textbf{Collaborative Problem-Solving (CPS):} 
Earlier studies categorized CPS into team effectiveness and task effectiveness~\cite{rosen2020towards}. Team effectiveness was measured by student engagement, while task effectiveness was assessed through code quality. %Our analysis captured problem-solving behaviors by frequency and sequence.
To evaluate CPS, we examined task effectiveness, represented by the average question score (\(\bar{s}\)), and team effectiveness, assessed through the standard deviation of engagement (\(\sigma_e\)) and the average engagement score (\(\bar{e}\)) as shown in Equation \ref{eq:1}. We then used the coefficient of variation (\(CV_e\)) \RR{to account for both engagement variability and engagement}. Finally, the overall collaboration quality was calculated using Equation \ref{eq:2}, combining question performance and engagement balance. 
\begin{equation}
\sigma_e = \sqrt{\frac{1}{n} \sum_{i=1}^{n} (e_i - \bar{e})^2}, \quad CV_e = \frac{\sigma_e}{\bar{e}}
\label{eq:1}
\end{equation}

\begin{equation}
Quality = \bar{s} \cdot (1 - CV_e)
\label{eq:2}
\end{equation}
As shown in Table \ref{table:comparison}, Group 19, despite achieving a respectable average score, exhibited imbalanced engagement, leading to a lower collaboration quality score. In contrast, Group 20 demonstrated more balanced and higher engagement, resulting in a better overall collaboration quality.
\begin{table}[htbp]
\centering
\begin{tabular}{cccccc}
\toprule
\textbf{Group} & \(\bar{s}\) & \textbf{Engagement Levels} & \(\sigma_e\) & \(\text{CV}_e\) & \textbf{CQ} \\
\midrule
Group 19 & \(4.11\) & (10.515, 9.725, 4.575) & \(2.80\) & \(0.24\) & \(2.80\) \\
Group 20 & \(4.14\) & (10.06, 9.32, 8.62) & \(0.73\) & \(0.08\) & \(3.88\) \\
\bottomrule
\end{tabular}
\caption{Comparison of Group 19 and Group 20 on Collaboration Quality (CQ).}
\label{table:comparison}
\end{table}

\textbf{Teacher Scaffolding,} categorized into cognitive (low, medium, high-control) and metacognitive forms~\cite{ouyang2022applying}, reflected the level of support provided to a group and its impact on programming performance. We evaluated four scaffolding dimensions, leveraging GPT-4o for annotation. By using targeted prompts and examples, we improved classification accuracy, while teacher scaffolding was categorized according to the type of support based on a semantic analysis of interactions.



We conducted experiments using a self-collected Wi-Fi CSI dataset, leveraging commodity Wi-Fi devices, specifically Dell LATITUDE laptops, as both the Wi-Fi transmitter and receivers for data collection. Each Wi-Fi transmitter and receiver is equipped with three antennas. The Wi-Fi channel operated at 5.32 GHz with a bandwidth of 40 MHz, and the packet transmission rate was set to 1000 packets per second. We utilized the Linux 802.11 CSI tool~\cite{halperin2011tool} to extract CSI data from 30 OFDM subcarriers per packet. The dataset comprises over 1,965,000 Wi-Fi CSI packets collected from participants with varying heights, weights, and ages. These packets were segmented into 393 segments, each lasting 5 seconds, during which participants performed one of four activities: walking, falling, breathing, or no event (i.e., an empty environment). The collected data were then converted into both image and text representations, as detailed in Table~\ref{tab:data_stats}. Data collection was conducted across three real-world environments, a bedroom, a kitchen, and a living room, over a two-month period. The study was reviewed and approved by the IRB of the authors' institution.

%\textcolor{red}{add table}

\vspace{-1mm}
\subsection{Baselines}
\vspace{-2mm}
We compare Wi-Chat with the following systems: 

\textbf{Conventional Wi-Fi-based Systems.} These systems follow a multi-step pipeline, including signal denoising, signal transformation, feature extraction, and model construction, as described in Section~\ref{systems}. Specifically, we reproduce two well-known systems:
\textit{1) CARM}~\cite{wang2015understanding}: It utilizes a PCA-based method for signal denoising, applies DWT for feature extraction, and employs a Hidden Markov Model for activity recognition.
\textit{2) E-eyes}~\cite{wang2014eyes}: This system first removes data outliers using a low-pass filter and then builds activity classifiers using Earth Mover's Distance.

\textbf{Machine Learning Models with Raw Signals.} 
We evaluate the performance of machine learning models, including \textit{3) CNN}, \textit{4) RNN}, and \textit{5) SVM}. These models take textual or visual representations of raw Wi-Fi signals as input and are trained in a supervised manner using labeled datasets.

\subsection{Experimental Settings}

For LLMs, we first apply signal smoothing using the Savitzky-Golay filter~\cite{schafer2011savitzky} and then convert the signals into textual or visual representations. When experimenting with the few-shot setting, we pick 4 examples, including one example from each label class. The prompts used for experiments are presented in Appendix~\ref{sec:prompt}.

%For all supervised baselines, the dataset was randomly split into two parts: $70\%$ for training and $30\%$ for testing. Note that we perform the same signal smoothing and convert them into text and images. We train CNN and RNN on an NVIDIA GeForce RTX 4090 GPU. The learning rate of CNN and RNN is set to 0.001 and they both use Adam optimizer and the max Number of Epochs is 30. The batch size is set to 32 during training for both CNN and RNN. For the SVM, we use RBF as the Kernel Type. For CARM and E-eyes systems, we follow their signal processing procedures.
%Performance evaluation was conducted using standard classification metrics, including Accuracy, Precision, Recall, and F1 Score. Savitzky-Golay filter~\cite{ren2021winect}

For all supervised baselines, we randomly split the dataset into $70\%$ for training and $30\%$ for testing to ensure fair evaluation. Prior to model training, we apply the same signal smoothing techniques and convert the signals into textual and visual representations for consistency across methods. The CNN and RNN models are trained on an NVIDIA GeForce RTX 4090 GPU with a learning rate of 0.001 using the Adam optimizer, with a maximum of 30 epochs and a batch size of 32. For the SVM model, we use the Radial Basis Function as the kernel type. For CARM and E-eyes, we follow their original signal processing pipelines, including denoising, feature extraction, and model construction as described in their respective works. 
Zero-shot evaluations for CNN, RNN, SVM, and conventional systems are also conducted using the same approaches, but with untrained models to test their zero-shot performance. Performance evaluation is conducted using standard classification metrics, including accuracy, precision, recall, and F1 score, to assess the ability of each system to recognize human activities from Wi-Fi CSI data.

\vspace{-2mm}
We have introduced iVISPAR, a novel interactive multi-modal benchmark designed to evaluate the spatial reasoning capabilities in 3D vision of VLMs acting as agents. The benchmark, centered on the Sliding Geom Puzzle, evaluates VLMs' abilities in logical planning, spatial awareness, and multi-step problem-solving, aiming to reflect real-world spatial reasoning. Our evaluation tested a suite of state-of-the-art open-source and closed-source VLMs on a dataset of board configurations, scaled across two levels of complexity. We compared them to baselines for human capabilities, optimal and random agents, providing insight into their performance under varying conditions.

\subsection{Results}
\vspace{-2mm}


\textbf{Overall Results.} Table~\ref{summary} presents the best-performing models across different method categories. In the zero-shot category, the best model GPT-4o model achieved an accuracy of 0.62, demonstrating its ability to generalize effectively without task-specific examples. In the 4-shot category, GPT-4o is still the best model, exhibiting substantial improvement, attaining an accuracy of 0.77. This result highlights the effectiveness of in-context learning, where additional prompt examples help refine model predictions.

In the vision models category, the GPT-4o-mini with CoT demonstrated the strongest performance with an accuracy of 0.90. This result indicates the model's capacity to integrate visual and textual reasoning through CoT prompting, which likely aids in complex decision-making.

For supervised learning, the E-eyes (with complex signal processing techniques) outperformed all other models. This result is expected, as supervised models are explicitly trained on labeled data, allowing them to learn precise decision boundaries. However, despite the high accuracy, supervised methods typically require extensive labeled datasets, which may not always be feasible in real-world applications.

% \begin{table*}[t]
% \centering
% \small  % Adjust font size as needed
% \begin{tabular}{l|cccc}
% \hline
% \textbf{Evaluation Metrics}&\textbf{Accuracy} &\textbf{Precision} & \textbf{Recall} & \textbf{F1-score} \\
% \hline

% \multicolumn{5}{c}{\textbf{\textsl{Time Series}}} \\
% \hline
% \multicolumn{5}{c}{\textbf{\textsl{Vision}}} \\
% \hline
% \multicolumn{5}{c}{\textbf{\textsl{CNN, RNN, and SVM (Time Series)}}} \\
% \hline
% Zero-shot SVM & 0.27   & 0.28    & 0.28     &0.27\\
% Zero-shot CNN  & 0.23  & 0.24  & 0.23   & 0.23\\
% Zero-shot RNN & 0.26  & 0.26  & 0.26   & 0.26\\
% SVM & 0.94   & 0.92    & 0.91    & 0.91\\
% CNN & 0.98  & 0.98  & 0.97   & 0.97\\
% RNN & 0.99  & 0.99  & 0.99   & 0.99\\
% \hline
% \multicolumn{5}{c}{\textbf{\textsl{CNN, RNN, and SVM (Image)}}} \\
% \hline
% Zero-shot SVM & 0.26   & 0.25    & 0.25     &0.25\\
% Zero-shot CNN  & 0.26  & 0.25  & 0.26   & 0.26\\
% Zero-shot RNN & 0.28  & 0.28  & 0.29   & 0.28\\
% SVM & 0.99   & 0.99    & 0.99    & 0.99\\
% CNN & 0.98  & 0.99  & 0.98   & 0.99\\
% RNN & 0.98  & 0.99  & 0.98   & 0.98\\
% \hline
% \multicolumn{5}{c}{\textbf{\textsl{Conventional Wi-Fi-based Human Activity Recognition Systems}}} \\
% \hline
% Zero-shot E-eyes~\cite{wang2015understanding}  & 0.26  & 0.26  & 0.27   & 0.26\\
% Zero-shot CARM~\cite{wang2014eyes}  & 0.24  & 0.24  & 0.24   & 0.24\\
% E-eyes~\cite{wang2015understanding}  & 1   & 1    & 1     & 1\\
% CARM~\cite{wang2014eyes}  & 0.98  & 0.98  & 0.98   & 0.98\\
% \hline

% \end{tabular}

% \caption{Classification Results
% }
% \label{tab:result}
% \end{table*}

\begin{table}[th]
    \centering
    \small
    \begin{tabular}{lcc}
        \toprule
        \textbf{Method} & \textbf{Accuracy} & \textbf{F1-score} \\
        \midrule
        \multicolumn{3}{c}{\textbf{Zero Shot}} \\
         \midrule
         E-eyes & 0.26 & 0.26 \\
         CARM & 0.24 & 0.24 \\
         SVM & 0.27 & 0.27 \\
         CNN & 0.23 & 0.23 \\
         RNN & 0.26 & 0.26 \\
        Vision SVM & 0.26 & 0.25 \\
        Vision CNN & 0.26 & 0.26 \\
        Vision RNN & 0.28 & 0.28 \\
        GPT-4o & 0.62 & 0.42 \\
        GPT-4o+ ICL & 0.77 & 0.73 \\
        GPT-4o-mini + COT & \textbf{0.90} & \textbf{0.90} \\
        \midrule
        \multicolumn{3}{c}{\textbf{Supervised}} \\
         \midrule
        Vision CNN & 0.98 & 0.98 \\
        CARM & 0.98 & 0.98 \\
        E-eyes & 1.00 & 1.00 \\
        \bottomrule
    \end{tabular}
    
    \caption{Performance comparison of different methods under the zero-shot and supervised settings.}
%\vspace{-5mm}
    \label{present}
\end{table}





Overall, the results indicate that LLMs exhibit strong performance in zero-shot and few-shot settings for the task of Wi-Fi-based human activity recognition, making them valuable for scenarios with limited annotated data. Additionally, the impressive accuracy of the vision-language model suggests promising directions for integrating multimodal learning into the task.

% \begin{figure}[b]
\vspace{-0.5cm}
    \centering
    \includegraphics[width=\linewidth]{figures/re.pdf}
    \vspace{-0.5cm}
    \caption{Visualization of different domain adaptation methods performance of two specific target tasks:
    WMH segmentation on MRI images and liver segmentation on CT images, both using MambaUNet. The pixels highlighted in red represent \textit{incorrect predictions}.
    }
    \label{result}
    \vspace{-0.5cm}
\end{figure}

\textbf{Methods Comparison.}
Table~\ref{present} summarizes the classification results across different methods. In the zero-shot setting, traditional machine learning models such as SVM, CNN, and RNN demonstrate relatively low classification performance. LLM models GPT-4o achieved an accuracy of 0.61, significantly outperforming traditional machine learning models. The ICL approach further improved performance, demonstrating the benefits of in-context learning.


Furthermore, GPT-4o-mini combined with CoT reasoning achieves the highest accuracy among zero-shot methods at 0.90, demonstrating the effectiveness of advanced prompting techniques in enhancing LLM-based classification performance. Notably, this performance is already comparable to conventional Wi-Fi activity recognition systems and machine learning models trained in supervised settings. These results reaffirm that while supervised models achieve superior performance with labeled data, LLMs exhibit strong generalization capabilities, particularly when leveraging few-shot learning and vision-language integration.



\subsection{Analysis}

\begin{figure}[tb]
    \centering
    \includegraphics[width=0.49\textwidth]{figures/bar.png}
    %\vspace{-8mm}
    \caption{Comparison of COT and knowledge}
   % \vspace{-3mm}
    \label{bar}
\end{figure}


\textbf{Effectiveness of CoT and Domain Knowledge.}
Figure~\ref{bar} illustrates the impact of CoT reasoning and domain knowledge on different language models across time series and vision-based settings. Among the zero-shot models, GPT-4o-mini demonstrates notable improvements when domain knowledge is incorporated, highlighting the benefits of integrating prior knowledge into the inference process. However, a substantial performance drop is observed in both models when CoT reasoning is applied to time series data. We argue that this decline stems from the inherent complexity of time series signals, particularly in the case of Wi-Fi raw signals, which are challenging to interpret directly. The step-by-step reasoning introduced by CoT may inadvertently add confusion, as the model struggles to generate coherent intermediate steps for highly dynamic and noisy input sequences.  

For vision-enhanced models, the GPT-4o-mini-Vision variant achieves the highest performance, with CoT prompting yielding an accuracy of 0.90. This suggests that CoT reasoning is particularly effective when paired with visual input, likely because images provide additional context that facilitates structured reasoning. A similar, albeit less pronounced, effect is observed for GPT-4o-Vision. These results indicate that while CoT reasoning can be beneficial, its effectiveness varies depending on the specific vision-language model and its alignment with the task. The structured nature of visual input may better support multi-step reasoning, whereas time series data lacks the same level of interpretability, limiting the effectiveness of CoT in those scenarios.  

\begin{figure}[ht]
    \centering
    \includegraphics[width=0.49\textwidth]{figures/box.png}
    %\vspace{-8mm}
    \caption{Comparison of LLMs under zero-shot settings}
    %\vspace{-3mm}
    \label{box}
\end{figure}

\textbf{Comparing Different LLMs.} Figure~\ref{box} presents the accuracy distribution of different LLM variants (Base, CoT, and Knowledge) under zero-shot settings. The box plots illustrate the variability in accuracy across models, with GPT-4o and DeepSeek demonstrating the highest median performance, while Mistral and LLaMA exhibit greater variance and lower median accuracy.

Among the models, GPT-4o-mini and DeepSeek show a more stable accuracy distribution, indicating consistent performance across different variants. In contrast, LLaMA has a wider spread, suggesting that its performance is more sensitive to the specific reasoning approach used. Gemma2 maintains a relatively narrow distribution, indicating lower variance but also a more limited improvement potential.

These results highlight that larger models like GPT-4o benefiting from more robust reasoning capabilities, while smaller models show varying degrees of improvement depending on the applied enhancements.

