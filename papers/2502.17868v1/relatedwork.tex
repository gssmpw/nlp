\section{Related Work}
\subsection{Swarm User Interfaces}

Distributed and closely cooperating groups are commonly referred to as swarms, which can accomplish tasks beyond the capabilities of individual members through collaboration.
Swarm robotics is an approach inspired by the collective behavior of animal groups, aiming to design robust, flexible, and scalable group behaviors by coordinating a large number of robots using simple rules and local interactions~\cite{Brambilla2013}.
The number of swarm robots varies depending on the research context; for instance, \textit{Kilobot}~\cite{Kilobot2014} presents a low-cost, easy-to-assemble swarm robot that can be used to test swarm algorithms with thousands of robots.
However, the maximum speed of \textit{Kilobot} is only 1 $cm/s$, which is relatively slow and unsuitable for real-time interaction with users.
The size of swarm robots ranges from very small, approximately 4 $mm$ in size~\cite{Wu2022}, to those similar in size to cleaning robots~\cite{RoomShift2020}.

In the field of Human-Computer Interaction, research involving swarm robots has been ongoing since the late 2010s.
Swarm UIs are a relatively new category of interfaces in which multiple self-propelled robots respond to user input and environmental changes~\cite{Zooids2016,Suzuki2022,HERMITS2020,UbiSwarm2017,ShapeBots2019,disappearables2022,HapticBots2021}.
One advantage of Swarm UIs is their composition of numerous interchangeable elements, enabling flexible adjustments to the overall size and shape of the interface.
For example, \textit{Zooids}~\cite{Zooids2016} can dynamically change the overall shape of the swarm by moving collectively, providing information through the spatial arrangement of robots.
Interactions in which individual robots control the behavior of the entire group are also possible.

Due to these characteristics, Swarm UIs are being explored for various applications, such as actuated tangible user interfaces~\cite{HERMITS2020,disappearables2022}, providing haptic feedback~\cite{SwarmHaptics2019,HapticBots2021}, creating physical displays~\cite{Alonso-Mora2011,Alonso-Mora2012,Alonso-Mora2015}, embodying the human body parts~\cite{SwarmBody}, manipulating objects on tabletop surfaces~\cite{Push-That-There2024}, and assisting digital fabrication machines~\cite{FabRobotics2024}.

Many Swarm UIs are designed primarily for tabletop use~\cite{Reactile2018,ShapeBots2019,Hiraki2016,disappearables2022,HERMITS2020}, as tabletops are considered suitable locations for human interaction.
For example, \textit{HERMITS}~\cite{HERMITS2020} expands the capabilities of individual robots by connecting detachable external parts, called mechanical shells, to modified small robots, thereby demonstrating the application of swarm robots through cooperative work.

Building on previous work in Swarm UIs, \textit{corobos} aims to extend the operational range of these interfaces by introducing a mechanism that facilitates transitions between tabletop and wall surfaces.


\subsection{Robots Adhering to Surfaces}
Swarm UIs primarily utilize two-wheeled robots on tabletops, but there are also robots with alternative locomotion capabilities suitable for different environments.

For instance, \textit{Griddrones}~\cite{Griddrones2018} are Swarm UIs that operate in the air.
These cube-shaped drones can create 3D shapes as voxels, with their size limited only by the available indoor space.
However, drones consume significant power for flight, making it challenging for them to remain airborne for extended periods.

Robots that adhere to walls or ceilings have been extensively investigated in the field of robotics~\cite{Inoue2006,Ahmed2022,Panich2010,Tripillar2011}.
Methods for adhering to walls and ceilings can be broadly classified into four categories: vacuum suction, adhesive materials, mechanical mechanisms, and magnetic adhesion.

Vacuum suction~\cite{Panich2010,suctioncup2010,EpidermalRobots} or air thrust~\cite{Vertigo2015} generates an attractive force through air pressure.
While this method can adhere to various flat wall materials, it has disadvantages, such as noise and increased device size, making it unsuitable for indoor swarm robots.

The adhesive material approach~\cite{Waalbot2007,Kim2008,Yan2021,HawkesGeckoHuman2015} uses dry or wet sticky substances to adhere to walls.
Although generally quieter than vacuum suction, its adhesion force is weaker because it depends on the surface friction of the wall material.

Mechanical mechanisms~\cite{Inoue2006,CLASH2011} allow robots to adhere by inserting claw-shaped parts into the unevenness of the wall surface, offering strong adhesion force and high reliability of movement.
However, this approach often results in larger, multi-legged robots due to its complexity.

Magnetic adhesion~\cite{iRobot2023,UbiSwarm2017,Ahmed2022,Takada2017,Matsumura2019,AeroRigUI_CHI2023,ThrowIO_CHI2023,Tripillar2011,R-Track2021,ThreadingSpace2024} is a commonly used method in various applications due to its silent operation and reliability on ferromagnetic surfaces.
For industrial purposes, inspection robots for steel structures (such as bridges) use magnetic adhesion to adhere to various surfaces~\cite{Ahmed2022}.
In educational settings, the iRobot Root\texttrademark~\cite{iRobot2023} adheres to whiteboards using magnets on its underside, allowing it to perform tasks like drawing and erasing lines with an attached pen.

In examples like \textit{Rovables}~\cite{Rovables2016} and \textit{Calico}~\cite{Calico}, robots can move on clothing.
These robots climb vertically without modifying the clothing by clamping their wheels onto the fabric using magnets or specially designed rails.
% Potential applications include measuring human movement, modular displays, haptic feedback, and interactive clothing.

Among Swarm UIs that move on walls, \textit{UbiSwarm}~\cite{UbiSwarm2017} is a notable example.
In \textit{UbiSwarm}, magnets attached to the robot’s bottom surface allow them to adhere to ferromagnetic wall surfaces.

\textit{AeroRigUI}~\cite{AeroRigUI_CHI2023}, \textit{ThrowIO}~\cite{ThrowIO_CHI2023}, and Threading Space~\cite{ThreadingSpace2024} also use magnetic adhesion to attach objects and robots to ceilings, enabling 3D mid-air interactions, throwing and catching interactions, or altering spatial perception with physical threads.

Although installing strong ferromagnetic materials like whiteboards (steel plates) on the wall surface is necessary, magnetic adhesion allows for smaller size, passive (non-powered) adhesion, noise-free operation, and stronger adhesion force than adhesive materials.
Consequently, for small-sized mobile robots in Swarm UIs, magnetic adhesion is a suitable choice, and thus, we have adopted this method for \textit{corobos}.


\subsection{Surface-Transitioning Robots}
To transition between perpendicular surfaces, such as a tabletop and a wall, a robot must rotate 90 degrees while adhering to the surface, necessitating a unique mechanism.
For example, Ahmed et al. developed a robot~\cite{Ahmed2022} for inspecting structures made of ferromagnetic materials.
This robot is equipped with joints that connect leg-like parts with magnets, enabling it to adhere to and transition on structures with noncontinuous surfaces.
Another approach involves using magnetic wheels~\cite{Matsumura2019,Tripillar2011}, which allow a robot to adhere to a structure’s surface and move freely in any direction by rotating the wheels.
However, these methods result in complex, high-performance robots, making them unsuitable for Swarm UIs.

Conversely, \textit{FreeBOT}~\cite{Liang2020} consists of multiple spherical robots that adhere to each other using magnets, allowing their overall shape to change dynamically.
\textit{FreeBOT} can also adhere to and transition on ferromagnetic walls while rotating on its own and can even ride on top of another robot.
However, this transition method is not ideal for interactions involving humans or objects, such as pushing or pulling in one direction, and accurately estimating the position is challenging due to the robots’ form factor.

In \textit{UbiSwarm}~\cite{UbiSwarm2017}, researchers proposed the concept of Ubiquitous Robotic Interfaces (URIs) to make robots ubiquitous and capable of interacting with humans and their environment.
Their vision involves groups of robots seamlessly transitioning from walls to tabletops and between different rooms at a speed comparable to the human eye’s refresh rate, providing virtually unlimited interaction space.
Although they mentioned a method involving a magnetic slope between walls and tabletops as a candidate for transitioning, it has not yet been realized in the paper.

\textit{R-Track}~\cite{R-Track2021} features modular robots equipped with magnetic tracks that can transition from wall to wall by connecting three robots in sequence.
These robots are capable of navigating both internal and external angles between two perpendicular surfaces.
However, their design is focused on inspection tasks, with a size of about 150 mm, making them less suitable for Swarm UI applications, which typically require smaller, more compact robots.
Additionally, \textit{R-Track} incorporates dedicated mechanisms like external connectors to facilitate wall-to-wall transitions, which, while effective, are unnecessary for simpler table-to-wall transitions in Swarm UI scenarios.
The simpler structure of \textit{corobos} provides a more practical approach for these applications by minimizing complexity and focusing on specific user interaction needs.

In \textit{(Dis)Appearables}~\cite{disappearables2022}, a small two-wheeled robot can transition between horizontal surfaces of different heights by using a ferromagnetic material ramp installed between the two surfaces.
While this method can be applied to \textit{corobos} system, it restricts transition opportunities to locations with ramps, and the ramp’s continuous presence on the desktop may interfere with human activities.

In our approach, we aim to develop a transition method that requires minimal modification to both the robots and the desktop environment.