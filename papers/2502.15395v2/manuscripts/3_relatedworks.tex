
%% Works Related to LLM assisted Decision Making 
\subsection{The Evolving Role of LLMs in Human Decision Making}
The use of LLMs in human decision-making has become increasingly prominent in recent years, with use cases ranging from everyday tasks to professional environments \cite{chkirbene2024applications}. Through their capacity to generate coherent text and context-relevant suggestions, LLMs play multiple roles in these decision-making processes, acting as both information providers and triggers for human self-reflection \cite{kim2022bridging, kmmer2024effects}. Despite the significant presence of LLMs and AI in various domains, there remains a notable gap in research that specifically addresses their effect on daily decision-making.

%% Works Related to LLM Overreliance (??) 
As LLM usage continues to grow, the corresponding reliance on them has also seen a sharp increase \cite{he2025conversational}. Existing work explores factors that shape human reliance on AI and LLMs \cite{eigner2024determinants}, along with the broader implications of this reliance \cite{kmmer2024effects}. However, current studies evaluate the outcomes of this dependence primarily through task performance and decision accuracy \cite{kim2025fostering}.
Questions persist about the types of decisions humans delegate to LLMs, as well as how these tools might augment or even replace users' own metacognitive processes. Moreover, the social aspects of LLM-assisted decision-making, including their effect on interpersonal dynamics in cases of levels of reliance, remain unexplored.

%% Tasks with LLMs (not just decision making - writing, coding...) 
%% What specific tasks they are analyzing, and how they are measuring the reliance or they cooperation with LLMs? 

%% Works related to Decision Making (not just LLMs) -> Theories on how different decision making tasks are divided (ex CCT)
\subsection{Human Decision Making: Key Factors and Frameworks}
Decision-making is a complex cognitive process that can be influenced by a range of internal (e.g., motivation, expertise, emotional state) and external (e.g., social context, time constraints) factors. Prior research in cognitive science and behavioral economics has highlighted multiple strategies people employ when making decisions \cite{dhami2012cct, offredy2008cct}, as well as frameworks for understanding these strategies. For instance, dual-process theories distinguish between intuitive, heuristic-based reasoning (System 1) and more analytical, deliberative reasoning (System 2) \cite{djulbegovic2012dualmedical, cardoso2020TheAB}. 

Social and environmental elements also play crucial roles. From social influence theories to collaborative cognition frameworks, studies show how peers, experts, and technology can shape individual judgment \cite{gracia2003groupdeliberation, ma2024deliberation}. These theories serve as a valuable lens through which we can interpret human-LLM interactions: if individuals increasingly rely on LLM outputs, it is vital to understand how this reliance interplays with their cognitive processes, and what conditions reinforce or mitigate it.

In this context, existing decision-making theories provide a baseline for evaluating LLM-assisted decisions. By integrating these theoretical perspectives with empirical observations of human-LLM interaction, we can more comprehensively assess not only what decisions are delegated to LLMs, but also why and how such delegation happens.
