\subsection{Data Gathering}
We identified heavy LLM users through a screening survey followed by semi-structured interviews. Heavy users were defined as individuals who regularly employed LLMs beyond work tasks, incorporating them into everyday decision-making processes and delegating various aspects of their decision-making to these systems.

\subsubsection{Participant Recruitment}
To identify heavy LLM users who regularly use these systems for everyday decisions, we sought participants who delegated decision-making tasks across both System 1 and System 2 thinking, as well as the spectrum between these two systems. To ensure comprehensive coverage of decision-making types, we based our screening survey on Hammond's Cognitive Continuum Theory (CCT) \cite{hammond2000human}. CCT classifies decision-making into four modes based on the balance between System 1 and System 2 thinking: Pure Intuition, Aided Intuition, Quasi-Rational Intuition, and Quasi-Rational Analysis. We developed 5-6 representative tasks for each mode. Examples include "deciding whether to speak up in class or meetings" (Pure Intuition), "choosing a recipe to cook" (Aided Intuition), "addressing interpersonal conflicts" (Quasi-Rational Intuition), "selecting elective courses" (Quasi-Rational Intuition), and "determining project scope" (Quasi-Rational Analysis). Survey participants indicated whether they had used LLMs for each task with three response options: "Yes, I have done this task with LLMs," "No, but I have made decisions like this before," or "Not Applicable to me." The complete task set for each mode is provided in Appendix~\ref{appendix:survey-tasks}.

\begin{table*}[htbp]
\begin{tabular}{llllcl}
\toprule
\textbf{Number} & \textbf{Age} & \textbf{Gender} & \textbf{Vocation} & \textbf{Tasks Done with LLMs} & \textbf{Usage} \\
\midrule
P1  & 36 & Male & Teacher (10 years experience) & 14 & 3-7 times per week \\
P2  & 25 & Female & 2nd Year Undergraduate (Environmental Science) & 17 & 3-7 times per week \\
P3  & 39 & Male & PhD Student / Part-time Lecturer & 18 & Daily \\
P4  & 20 & Female & 2nd Year Undergraduate (Economics) & 14 & Daily \\
P5  & 24 & Female & Job Seeker (Recent Graduate) & 13 & Daily \\
P6  & 26 & Male & 2nd Year PhD Student (Mechanical Engineering) & 18 & Daily \\
P7  & 20 & Male & 2nd Year Undergraduate (Computer Science) & 14 & Daily \\
\midrule
\textbf{N=7} & \textbf{27.1} & \textbf{M/F: 4/3} & & \textbf{15.4} & \textbf{Daily/Weekly: 5/2} \\
\bottomrule
\end{tabular}
\caption{Participant Profiles and LLM Usage}
\label{tab:participant-data}
\end{table*}

\subsubsection{Participants}
From 78 survey responses, we identified heavy users using three criteria. Participants needed to indicate LLM use in at least 13 of the 22 tasks, show balanced LLM use across different decision-making modes, and demonstrate LLM use for both work-related and personal tasks (e.g., recipe selection or outfit choices). Seven participants met these criteria (Table~\ref{tab:participant-data}).

\subsection{Semi-Structured Interview and Analysis}
We conducted an hour-long semi-structured interviews with each participant. Sessions began with an overview of the study's goals and an acknowledgment that there were no predetermined correct answers. The interview protocol addressed three main themes: participants' reasons for using LLMs in specific scenarios from the survey, their general motivation and usage patterns for LLMs, and their perceived changes in decision-making processes since adopting LLMs. Participants received KRW 15,000 (approximately 11 USD) as compensation. For the analysis, two researchers individually coded the interview transcripts using open coding. Then, a team of three researchers discussed the outcomes to resolve discrepancies and generate, review, and iterate on themes.
