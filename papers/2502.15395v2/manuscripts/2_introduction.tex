The ubiquitous question "How did I do this before ChatGPT?" has become a cultural touch point, highlighting how Large Language Models (LLMs) have gradually permeated people's everyday lives. While initially introduced as general-purpose chatbots, LLMs have been adopted in unexpectedly diverse ways \cite{chkirbene2024applications}. These systems now play multiple roles in decision-making processes and tasks, ranging from information providers to triggers for human self-reflection \cite{kim2022bridging, kmmer2024effects}. This widespread integration has raised the question about how users develop dependencies on and relationships with these AI systems \cite{he2025conversational}.

Previous Human-Computer Interaction (HCI) research has extensively examined domain-specific LLM applications \cite{jin2024teach, liu2024selenite}. These studies have yielded insights into specialized use cases and led to targeted interaction design improvements. However, the broader impact of LLMs on everyday decision-making and tasks remains under-explored. As users increasingly integrate these tools into their daily routines, understanding the tangible impacts of habitual use becomes crucial \cite{kmmer2024effects}. 

Recent studies have attempted to measure the impact of LLM use through quantitative metrics such as task performance and decision accuracy \cite{kim2025fostering}. However, these measurement-based approaches cannot fully capture how people delegate everyday decisions to LLMs or the resulting meta-cognitive effects. Furthermore, everyday decisions encompass a broad spectrum of choices, from routine task management to social interaction planning \cite{eigner2024determinants, dhami2012cct}, making them challenging to examine through purely quantitative and task-specific approaches.

To address this gap, we conducted a qualitative study examining heavy LLM users who regularly rely on these systems for everyday decisions and tasks. Through interviews and analysis, we explored how these users integrate LLMs into their decision-making processes, what types of decisions they choose to delegate, and how this delegation affects their cognitive patterns and decision-making confidence. Our study addressed three research questions: 

\begin{itemize}
\item RQ1: How do heavy LLM users integrate LLMs into their everyday decision-making process?
\item RQ2: What underlying needs do heavy LLM users seek to fulfill through LLM assistance?
\item RQ3: How do heavy LLM users conceptualize and evaluate their relationship with LLMs?
\end{itemize}

Through these research questions, we examine three key aspects of heavy LLM use. RQ1 explores emergent use cases and notable patterns in how users incorporate LLMs into their decision-making processes. RQ2 investigates the fundamental motivations and needs that drive sustained LLM usage. RQ3 examines how users develop their mental models of LLMs and reflect on their extensive interaction with these systems.