%%
%% This is file `sample-sigconf.tex',
%% generated with the docstrip utility.
%%
%% The original source files were:
%%
%% samples.dtx  (with options: `all,proceedings,bibtex,sigconf')
%% 
%% IMPORTANT NOTICE:
%% 
%% For the copyright see the source file.
%% 
%% Any modified versions of this file must be renamed
%% with new filenames distinct from sample-sigconf.tex.
%% 
%% For distribution of the original source see the terms
%% for copying and modification in the file samples.dtx.
%% 
%% This generated file may be distributed as long as the
%% original source files, as listed above, are part of the
%% same distribution. (The sources need not necessarily be
%% in the same archive or directory.)
%%
%%
%% Commands for TeXCount
%TC:macro \cite [option:text,text]
%TC:macro \citep [option:text,text]
%TC:macro \citet [option:text,text]
%TC:envir table 0 1
%TC:envir table* 0 1
%TC:envir tabular [ignore] word
%TC:envir displaymath 0 word
%TC:envir math 0 word
%TC:envir comment 0 0
%%
%% The first command in your LaTeX source must be the \documentclass
%% command.
%%
%% For submission and review of your manuscript please change the
%% command to \documentclass[manuscript, screen, review]{acmart}.
%%
%% When submitting camera ready or to TAPS, please change the command
%% to \documentclass[sigconf]{acmart} or whichever template is required
%% for your publication.
%%
%%
\documentclass[sigconf]{acmart}
\usepackage{titlesec}
%%
%% \BibTeX command to typeset BibTeX logo in the docs
\AtBeginDocument{%
  \providecommand\BibTeX{{%
    Bib\TeX}}}

%% Rights management information.  This information is sent to you
%% when you complete the rights form.  These commands have SAMPLE
%% values in them; it is your responsibility as an author to replace
%% the commands and values with those provided to you when you
%% complete the rights form.
% \setcopyright{acmlicensed}
% \copyrightyear{2025}
% \acmYear{2025}
% \acmDOI{XXXXXXX.XXXXXXX}
\setcopyright{none}
%% These commands are for a PROCEEDINGS abstract or paper.

% \acmConference[CHI'25 Workshop: Tools for Thought]{2025}{Yokohama, Japan}
\acmConference[CHI'25 Workshop on Tools for Thought]{Tools for Thought: Research and Design for Understanding, Protecting, and Augmenting Human Cognition with Generative AI on CHI 2025 Workshop}{April 26,2025}{Yokohama, JAPAN}
  
\settopmatter{printacmref=false}
\renewcommand\footnotetextcopyrightpermission[1]{}



%%
%%  Uncomment \acmBooktitle if the title of the proceedings is different
%%  from ``Proceedings of ...''!
%%
%%\acmBooktitle{Woodstock '18: ACM Symposium on Neural Gaze Detection,
%%  June 03--05, 2018, Woodstock, NY}
\acmISBN{978-1-4503-XXXX-X/2018/06}


%%
%% Submission ID.
%% Use this when submitting an article to a sponsored event. You'll
%% receive a unique submission ID from the organizers
%% of the event, and this ID should be used as the parameter to this command.
%%\acmSubmissionID{123-A56-BU3}

%%
%% For managing citations, it is recommended to use bibliography
%% files in BibTeX format.
%%
%% You can then either use BibTeX with the ACM-Reference-Format style,
%% or BibLaTeX with the acmnumeric or acmauthoryear sytles, that include
%% support for advanced citation of software artefact from the
%% biblatex-software package, also separately available on CTAN.
%%
%% Look at the sample-*-biblatex.tex files for templates showcasing
%% the biblatex styles.
%%

%%
%% The majority of ACM publications use numbered citations and
%% references.  The command \citestyle{authoryear} switches to the
%% "author year" style.
%%
%% If you are preparing content for an event
%% sponsored by ACM SIGGRAPH, you must use the "author year" style of
%% citations and references.
%% Uncommenting
%% the next command will enable that style.
%%\citestyle{acmauthoryear}

\usepackage{booktabs}
\usepackage{multirow}
% \usepackage[table]{xcolor}
\usepackage{array}
\usepackage{geometry}

%%
%% end of the preamble, start of the body of the document source.
\begin{document}

%%
%% The "title" command has an optional parameter,
%% allowing the author to define a "short title" to be used in page headers.
\title[Beyond Tools]{Beyond Tools: Understanding How Heavy Users Integrate LLMs into Everyday Tasks and Decision-Making}

%%
%% The "author" command and its associated commands are used to define
%% the authors and their affiliations.
%% Of note is the shared affiliation of the first two authors, and the
%% "authornote" and "authornotemark" commands
%% used to denote shared contribution to the research.
\author{Eunhye Kim}
\email{gracekim027@snu.ac.kr}
\affiliation{%
  \institution{Seoul National University}
  \city{Seoul}
  \country{Republic of Korea}
}

\author{Kiroong Choe}
\email{krchoe@hcil.snu.ac.kr}
\affiliation{%
  \institution{Seoul National University}
  \city{Seoul}
  \country{Republic of Korea}
}

\author{Minju Yoo}
\email{minjuu613@ewhain.net}
\affiliation{%
  \institution{Ewha Womans University}
  \city{Seoul}
  \country{Republic of Korea}
}

\author{Sadat Shams Chowdhury}
\email{sadatshams@kaist.ac.kr}
\affiliation{%
  \institution{School of Computing, KAIST}
  \city{Daejeon}
  \country{Republic of Korea}
}

\author{Jinwook Seo}
\email{jseo@hcil.snu.ac.kr}
\affiliation{%
  \institution{Seoul National University}
  \city{Seoul}
  \country{Republic of Korea}
}

%%
%% By default, the full list of authors will be used in the page
%% headers. Often, this list is too long, and will overlap
%% other information printed in the page headers. This command allows
%% the author to define a more concise list
%% of authors' names for this purpose.
\renewcommand{\shortauthors}{Kim et al.}

%%
%% The abstract is a short summary of the work to be presented in the
%% article.
\begin{abstract}
  \begin{abstract}

We introduce \ours, a novel framework for scene-level appearance transfer from a single style image to a real-world scene represented by multiple views. The method combines explicit semantic correspondences with multi-view consistency to achieve precise and coherent stylization.
Unlike conventional stylization methods that apply a reference style globally, \ours uses open-vocabulary segmentation to establish dense, instance-level correspondences between the style and real-world images. This ensures that each object is stylized with semantically matched textures.
\ours first transfers the style to a single view using a training-free semantic-attention mechanism in a diffusion model.
It then lifts the stylization to additional views via a learned warp-and-refine network guided by monocular depth and pixel-wise correspondences.
Experiments show that \ours consistently outperforms prior methods in structure preservation, perceptual style similarity, and multi-view coherence.
User studies further validate its ability to produce photo-realistic, semantically faithful results.
Our code, pretrained models, and dataset will be publicly released, to support new applications in interior design, virtual staging, and 3D-consistent stylization.

\end{abstract}

\end{abstract}

%%
%% The code below is generated by the tool at http://dl.acm.org/ccs.cfm.
%% Please copy and paste the code instead of the example below.
%%
\begin{CCSXML}
<ccs2012>
   <concept>
       <concept_id>10003120.10003121.10011748</concept_id>
       <concept_desc>Human-centered computing~Empirical studies in HCI</concept_desc>
       <concept_significance>500</concept_significance>
       </concept>
 </ccs2012>
\end{CCSXML}

\ccsdesc[500]{Human-centered computing~Empirical studies in HCI}

%%
%% Keywords. The author(s) should pick words that accurately describe
%% the work being presented. Separate the keywords with commas.
\keywords{Decision-Making, AI Delegation, Qualitative Study}
%% A "teaser" image appears between the author and affiliation
%% information and the body of the document, and typically spans the
%% page.
% \received{20 February 2007}
% \received[revised]{12 March 2009}
% \received[accepted]{5 June 2009}

%%
%% This command processes the author and affiliation and title
%% information and builds the first part of the formatted document.
\maketitle
\section{Introduction}

\begin{figure*}[t!]
    \centering
    \includegraphics[width=0.7\textwidth]{./Comparison.pdf}
    \caption{Comparison between conventional wireless system (left) and PASS (right).}
    \label{comparison}
    \vspace{-0.5cm}
\end{figure*} 

\section{Introduction} \label{sec:intro}

\IEEEPARstart{S}INCE Marconi demonstrated the feasibility of wireless communication in the late 19th century, the technology has undergone significant evolution and remarkable transformations. To address the unpredictable and dynamic nature of wireless channels, numerous advancements have been made in the air interface design, channel coding, source compression, and communication protocols for improving data rates and enhancing reliability. Among these advancements, multiple-input multiple-output (MIMO) has been one of the most important evolutionary techniques for wireless communication over the past few decades. By exploiting antenna arrays, MIMO brings about multiple benefits, such as enhanced signal strength through beamforming, mitigation of multi-path fading, and efficient spatial-domain multiplexing of users~\cite{bjornson2023twenty}. Since the advent of the third generation (3G) system, MIMO has been a fundamental component of wireless communication standards. However, during that era, the size of antenna arrays in MIMO systems was generally limited. The breakthrough came when Marzetta demonstrated the significant benefits of deploying an infinite number of antennas in 2010~\cite{marzetta2010noncooperative}, revealing the potential of MIMO to enhance communication performance while reducing system complexity. This revelation paved the way for the concept of massive MIMO, i.e., employing large-scale antenna arrays at base stations. Over time, massive MIMO has evolved into a key research focus and has become a reality with the deployment of 5G networks. 


However, massive MIMO has faced numerous challenges, as it is expected to transition from “Massive” in 5G (typically with 32-64 antennas) to “Gigantic” in 6G~\cite{Xtext, bjornson2024enabling}, where the number of antennas is expected to scale to hundreds or even thousands. One of the key obstacles is the complexity and cost of implementing massive MIMO since each antenna typically needs to be fed by a dedicated radio-frequency (RF) chain. Exploiting low-resolution analog-to-digital converters in RF chains or hybrid analog-digital antenna arrays with a limited number of RF chains were common methods to address this challenge, especially in the millimeter-wave band~\cite{heath2016overview}. More recently, advancements in metamaterials have paved the way for new antenna technologies, exemplified by waveguide-fed metasurface antennas~\cite{smith2017analysis, shlezinger2021dynamic, di2024reconfigurable}, which facilitate the ultra-dense deployment of antenna elements at a significantly lower cost and making massive MIMO implementation more feasible.

Flexible-antenna technique is a new evolution of MIMO. Unlike massive MIMO focusing on enlarging the wireless channel dimension, the flexible-antenna technique focuses on enabling the reconfiguration of the wireless channel. One of the most well-known approaches in this domain is the reconfigurable intelligent surface (RIS) technique~\cite{huang2019reconfigurable, wu2019intelligent, mu2021simultaneously}. By deploying RIS between transceivers, the wireless channel can be intelligently reconfigured by adjusting the phase shifts of the signals reflected/refracted by the RIS. More recently, fluid antennas~\cite{new2024tutorial} and movable antennas~\cite{zhu2023movable} have emerged as promising flexible-antenna technologies. The fundamental concept behind these approaches is to implement antenna arrays where individual antenna elements can dynamically adjust their positions within a spatial region, thus creating favorable channel conditions to enhance communication performance. 

Nevertheless, as shown on the left of Fig. \ref{comparison}, both massive MIMO and flexible-antenna techniques have limited capability in fundamentally addressing free-space pathloss and line-of-sight (LoS) blockage, two major causes of signal attenuation in wireless communications. While massive MIMO can achieve high beamforming gains to strengthen signals, it cannot combat LoS blockage and to effectively mitigate free-space pathloss, particularly for cell-edge users. RISs have been considered as a promising solution to overcome LoS blockage by creating virtual LoS paths. However, the double fading effect caused by signal reflection results in much higher pathloss compared to a direct LoS channel~\cite{ozdogan2019intelligent}. Additionally, fluid and movable antennas are typically capable of adjusting their positions only within a few wavelengths, making them more effective for mitigating small-scale fading rather than addressing large-scale pathloss. It is worthy to point out that all the aforementioned MIMO systems are lack of antenna array reconfigurability, i.e., once an antenna array is built, adding or removing antennas is no longer possible.

Pinching-Antenna SyStem (PASS) is a revolutionary technique for addressing the challenges of free-space pathloss and LoS blockage encountered by conventional multi-antenna technologies. This technique was originally proposed and prototyped by NTT DOCOMO in 2022~\cite{suzuki2022pinching}. As illustrated on the right of Fig. \ref{comparison}, PASS employs a dielectric waveguide as its primary transmission medium, which is known for its exceptionally low propagation loss (e.g., 0.01 dB/m \cite{pozar2021microwave}). By pinching a small separated dielectric element, referred to as a \emph{pinching antenna}, onto the waveguide, the system enables signal emission from the waveguide into the pinching antenna, which then radiates the signal into free space. Building on this principle, waveguides can be pre-deployed to extend service coverage, allowing pinching antennas to be placed at positions close to users. This strategic placement transforms the wireless system into a \emph{near-wired} system and hence establishes strong LoS links with users, effectively minimizing free-space path loss and mitigating blockage issues. Additionally, unlike existing MIMO systems, PASS allows both the number and positions of pinching antennas to be easily adjusted by simply pinching them to or releasing them from the waveguide~\cite{suzuki2022pinching}. This feature provides a low-cost and scalable approach to implementing MIMO while also facilitating the so-called \emph{pinching beamforming}, which enhances communication performance by dynamically optimizing antenna positions \cite{liu2025pinching}.

Given the successful prototyping of PASS by NTT DOCOMO, theoretical research on this topic has been steadily growing, though it remains in its early stages. The first theoretical study on PASS for the communication system design was presented in \cite{ding2024flexible}, where the authors provided a comprehensive analysis and developed low-complexity pinching beamforming designs for fundamental single-user and two-user scenarios. The array gain achieved by multiple pinching antennas on a waveguide was analyzed in \cite{ouyang2025array}, unveiling the optimal number of antennas and their spacing for maximizing the beamforming gain. 
% The authors of \cite{tegos2024minimum} studied an uplink PASS system and proposed an iterative antenna position optimization algorithm to maximize the sum rate under perfect phase alignment conditions. In \cite{wang2024antenna}, the authors investigated a downlink PASS system and introduced a matching theory-based optimization method for activating pinching antennas at preconfigured discrete positions. Their findings also highlighted the advantages of using non-orthogonal multiple access (NOMA) in PASS. Expanding on this,
The authors of \cite{bereyhi2025downlink} explored a downlink PASS architecture utilizing multiple waveguides, each equipped with a single pinching antenna, and proposed a greedy approach for jointly optimizing the transmit and pinching beamforming. Meanwhile, \cite{guo2025deep} examined a more generalized scenario, where multiple pinching antennas were deployed on each waveguide, and introduced a graph neural network (GNN)-based deep learning method to address the corresponding joint beamforming optimization problem.

Although PASS has attracted growing attention, several key challenges remain unsolved. On the one hand, the physics modeling of PASS is still underdeveloped, which is crucial for establishing an accurate signal model. In existing studies \cite{ding2024flexible, ouyang2025array, bereyhi2025downlink, guo2025deep}, it is commonly assumed that all signal power within the waveguide is fully radiated into free space and that each pinching antenna on a waveguide emits identical radiation power—an assumption analogous to conventional MIMO systems. However, pinching antennas operate fundamentally differently from traditional electronic antennas, and such assumptions may lack a solid physical foundation and fail to accurately reflect real-world behaviors. On the other hand, most existing works design PASS under simplified assumptions \cite{ding2024flexible, ouyang2025array, bereyhi2025downlink}, such as a single user, a single waveguide, a single pinching antenna per waveguide, or perfectly aligned signal phases. Although the GNN-based deep learning model proposed in \cite{guo2025deep} is capable of handling more complex scenarios with arbitrary numbers of users, waveguides, and pinching antennas, it suffers from a key limitation: the model parameters need to be retrained once the system configuration changes, limiting its generalization ability. Motivated by these challenges, this paper aims to develop a fundamental physics-based signal model for PASS and explore joint beamforming designs for more general scenarios. The key contributions of this work are summarized as follows:
\begin{itemize}
    \item We propose a physics-based hardware model for PASS, in which a pinching antenna is modeled as an open-ended directional waveguide coupler to facilitate the adjustment of radiation characteristics and simplify signal modeling. Based on this model, we characterize the relationship between the electromagnetic (EM) fields within the waveguide and those radiated by the pinching antennas using coupled-mode theory.
    \item We derive a novel signal model for PASS based on the proposed physics framework, revealing the inherent coupling effect between the radiation power of pinching antennas deployed on the same waveguide. Leveraging this coupling relationship, we introduce two simplified power models and their respective implementation methods: equal power and proportional power models.
    \item We formulate a joint transmit and pinching beamforming optimization problem to minimize the transmit power in a general PASS system with arbitrary numbers of users, waveguides, and pinching antennas, considering both continuous and discrete activation of pinching antennas. To solve this highly nonconvex, coupled, and multimodal optimization problem, we propose two algorithms: the penalty-based alternating optimization algorithm and the zero-forcing (ZF)-based low-complexity algorithm.
    \item We provide comprehensive numerical results to validate the advantages of PASS and the effectiveness of the proposed algorithm. The results demonstrate that 1) the ZF-based algorithm delivers performance comparable to the penalty-based algorithm but has a low complexity, 2) PASS significantly reduces transmit power, achieving a reduction of over 95\% compared to conventional and massive MIMO, 3) a dense set of available antenna positions is required for discrete activation to achieve similar performance to continuous activation, and 4) the proportional power model exhibits performance comparable to the equal power model.
\end{itemize}

The rest of this paper is structured as follows. Section \ref{sec:model} introduces the proposed physics-based hardware model and signal model for PASS. Section \ref{sec:beamforing} presents the general system model for downlink PASS and introduces a penalty-based alternating optimization method and a ZF-based algorithm for solving the joint beamforming optimization problem. Numerical evaluations and performance comparisons under various system configurations are presented in Section \ref{sec:results}. Finally, Section \ref{sec:conclusion} summarizes the findings and concludes the paper.


\emph{Notations:} Scalars are denoted using regular typeface, vectors and matrices are represented in boldface, and Euclidean subspaces are indicated with calligraphic letters. The set of complex and real numbers are denoted by $\mathbb{C}$ and $\mathbb{R}$, respectively. The inverse, conjugate, transpose, conjugate transpose, and trace operators are denoted by $(\cdot)^{-1}$, $(\cdot)^*$, $(\cdot)^T$, $(\cdot)^H$, and $\mathrm{tr}(\cdot)$, respectively. The absolute value, Euclidean norm, Frobenius norm, and maximum norm are denoted by $|\cdot|$, $\|\cdot\|$, $\|\cdot\|_F$, and $\|\cdot\|_\infty$ respectively. The real part of a complex number of demoted by $\Re \{\cdot\}$. The entry in the $n$-th row and $m$-th column of a matrix $\mathbf{X}$ is denoted by $[\mathbf{X}]_{n,m}$. An identity matrix of dimension $N \times N$ is denoted by $\mathbf{I}_N$. The big-O notation is given by $O(\cdot)$. A diagonal matrix with diagonal entries $x_1,\dots,x_N$ is denoted as $\mathrm{diag}(x_1,\dots,x_N)$.    




% \begin{figure*}[t!]
% \centering
% \begin{subfigure}[t]{0.48\textwidth}
%     \centering
%     \includegraphics[height=0.5\textwidth]{./Comparison_conventional.pdf}
% \end{subfigure}
% \hspace{-1.5cm}
% \begin{subfigure}[t]{0.48\textwidth}
%     \centering
%     \includegraphics[height=0.5\textwidth]{./Comparison_PASSpdf.pdf}
% \end{subfigure}
% \caption{Comparison between conventional wireless system (left) and PASS (right).}
% \end{figure*} 


\section{Methods}
\section{Methods}
\label{sec:methods}

We conducted interviews with mental health clinicians to explore how they would design health information technologies (HITs) that support value-based mental healthcare.
Methodologically, we were inspired by work in speculative design to imagine futures where VBC is mandated, and then brainstorm with participants how HITs could support VBC outcomes data storage, collection, and use \cite{hockenhull_speculative_2021, wong_speculative_2018}. 
In this section, we detail the study procedures, including participant recruitment (Section \ref{sec:methods:participants}), background information (Section \ref{sec:methods:participants-backgrounds}), how data was collected and analyzed (Section \ref{sec:methods:data}), and our positionality (Section \ref{sec:methods:positionality}). 
All study procedures were approved by the coauthors' institutional review board (IRB). 

\subsection{Participant Recruitment}
\label{sec:methods:participants}
We enrolled as participants mental health clinicians, specifically practicing psychiatrists, clinical psychologists, licensed clinical social workers (LCSWs), and licensed mental health counselors (LMHCs).
We intentionally recruited providers from these different clinical orientations to gather different perspectives on designing HITs \cite{mental_health_america_types_2024}. 
Participants were recruited via a combination of convenience, purposive, and snowball sampling \cite{etikan_comparison_2015, goodman_snowball_1961}.
Specifically, a recruitment email and flier were sent to staff working at academic medical centers across the United States. 
\rev{Recruitment emails were often forwarded to providers who worked in smaller, private practices or community health settings, to help us gain perspectives from mental health clinicians working in diverse settings, treating different types of patients.} 
Within the qualitative tradition \cite{braun_one_2021}, our goal for this work was not to gather perspectives representative of mental health clinicians as a whole, but instead to deep dive with our participants into the complexities of designing HITs that support VBC.

\subsection{Participants' Backgrounds}
\label{sec:methods:participants-backgrounds}

\rev{Table \ref{tab:participants} summarizes background information for the 30 mental health clinicians who participated in the study.
This background information was collected during an intake survey, which was administered after participants provided informed consent for our study.
Apart from data collected within this intake survey, we often asked participants during our study interviews to provide background information regarding their current payment arrangements.
Most of our participants took traditional, fee-for-service payments (public and private), or asked their private practice patients to pay for care out-of-pocket.
A few participants (eg, SW28) worked in health systems transitioning to value-based payments.
Many participants were unfamiliar with VBC.
}

\begin{table*}[t]
\begin{tabular}{ll}
\toprule
Number of participants & 30 mental health clinicians \\ 
\midrule
Clinical training           & 13 Clinical Psychology \rev{(CP)} \\
                            & 6 Psychiatry \rev{(PS)} \\    
                            & 8 Clinical Social Work \rev{(SW)} \\
                            & 2 Mental Health Counseling \rev{(MC)} \\
                            & 1 Family and Marriage Therapist \rev{(FT)} \\
\midrule
Practice setting            & 16 Academic Medical Center \\
                            & 14 Private Practice \\
                            & 5 Community Mental Health Center \\
                            & 2 Employee Assistance Program \\
\midrule
Geographic location (in the USA)    & 26 Northeast \\
                                    & 2 Southeast \\
                                    & 2 West Coast \\
\bottomrule
\end{tabular}
\caption{Background information of the study participants. Participants could list multiple practice settings.
\rev{Clinical training abbreviations (eg, ``CP'') are used within Section \ref{sec:findings}.}
}
% \Description{A table summarizing the backgrounds of the 30 participants we interviewed in this study. The table describes the clinical training of participants, the practice setting, and geographic location (in the United States) of each participant.}
\label{tab:participants}
\vspace{-5pt}
\end{table*}

\subsection{Data Collection and Analysis}
\label{sec:methods:data}

All participants were asked to provide informed consent after being provided complete information about the study procedures.
Interviews were held via Zoom over two 1-hour sessions attended by the first three authors, and participants were reimbursed \$30 per hour for their time.
The first session was a semi-structured interview where we asked clinicians about their current care practices, specifically how they used data -- defined broadly, collected with or without technology -- in care.
We specifically asked participants about their perspectives on \textit{measurement-based care} (MBC), the practice of collecting and using data in care that would power HITs supporting VBC \cite{kilbourne_measuring_2018}.
We then asked participants further questions about how they used this data to measure care outcomes, how technology was involved in this process, and whether providers were accountable to achieve certain care outcomes.
Interview questions were broad to allow for on-the-spot adaptation and probing \cite{barriball_collecting_1994}.

In the second session, participants completed two design prompts.
These prompts were motivated by work in speculative design \cite{hockenhull_speculative_2021, wong_influence_2008}, to imagine futures where MBC and VBC were mandated and to understand how clinicians would collect and report outcomes data as a part of these programs.
The first prompt asked participating clinicians to imagine a world where they were mandated to use outcomes data as a part of care, and to brainstorm what data they would prioritize.
The second prompt was motivated by the five-star quality rating system used by the United States Center for Medicare \& Medicaid services (CMS) \cite{center_for_medicare__medicaid_services_five-star_2022}.
Participating clinicians were asked to imagine that as a part of VBC, CMS wanted to design ``mental health quality star ratings'' to measure patient outcomes and care quality across clinics and health systems.
Participants were asked to brainstorm what data should be included in this new star rating program.
After responding to each prompt, we discussed with participants the data they included in their responses, and asked probing questions to further understand how HITs could support data storage, collection, and use.
Full interview guides can be found in Appendix \ref{appendix:guide}.

Interviews were recorded with participants' permission, transcribed by a professional service, and de-identified.
Transcripts were analyzed using a reflexive thematic analysis approach adopted from \cite{braun_using_2006}.
This approach combined both inductive and deductive elements.
Codes and themes arose from the data, but were guided by our research interests and the literature \cite{braun_one_2021}, specifically the stages of preparation, collection, and action from Li et al. \cite{li_stage-based_2010}.
The first author qualitatively coded all transcripts.
Codes were iteratively refined, resulting in a final codebook, and all transcripts were recoded using the final codebook.
Themes were developed from the codes by the first author, with support from the second and third authors who also participated in the interviews and validated that the themes represented participants' views.
The codebook used to generate each theme can be found in Appendix \ref{appendix:codebook}.

\subsection{Positionality}
\label{sec:methods:positionality}

The first, second, and third authors are graduate students in computer and information science. 
These authors recruited participants, collected, and analyzed all of the data. 
One author is a clinical researcher and practicing mental health clinician who worked with the first author on the study protocols, and did not participate in the study. 
Another author is a health policy researcher, who is an expert on both digital mental health and value-based care.
The final author is a researcher in computing and information science. 
All authors were based in the United States, and thus our findings and perspectives are greatly informed, and potentially limited by, our knowledge of the United States healthcare system.
\section{Results}
% \section{Simulation Evaluation \& Results}\label{sec:results}

\subsection{Baseline Planners}

To evaluate the performance of \PlannerName, we compare it against several baseline methods. In the following section, we describe these baselines, their implementation details, and their respective advantages and limitations, particularly in the context of information gathering in large, high-dimensional search spaces. The simulation framework and vehicle parameters remain consistent across all planners, and each method is allowed to replan during testing.

\subsubsection{Monte-Carlo Tree Search}

Monte Carlo Tree Search (MCTS) can be a powerful technique for finding feasible and optimal paths in complex environments. It is a heuristic search algorithm that builds a search tree incrementally through repeated simulations. At each iteration, it selects a node to explore based on a selection policy (often the Upper Confidence Bound or UCB1 algorithm), expands the tree by adding possible actions from that node, runs a simulation from the newly added node, and updates the statistics of nodes along the path traversed during the simulation. 

The UCB1 (Upper Confidence Bound) algorithm is a technique commonly used in the context of multi-armed bandit problems and Monte Carlo Tree Search (MCTS) for balancing exploration and exploitation. It helps in selecting actions or nodes that are likely to yield high rewards while also exploring less-frequented options to gather more information about their potential rewards. 

We formulate our UCB score in the following manner, \\
\begin{equation*}
    UCB_\text{node} = \frac{I(X_{\text{node}})}{\alpha} + C \times \sqrt{\frac{\ln(N_\text{tree})}{N_\text{node}}}
\end{equation*}
%  $
% UCB_\text{node} = \frac{\overline{X_\text{node}}}{\alpha} + C \times \sqrt{\frac{\ln(N_\text{tree})}{N_\text{node}}}
% $ \\
Here $I(X_{\text{node}})$ denotes the estimated information gain from the node, $\alpha$ denotes the normalization factor which is given by $\frac{B}{v_\text{desired}}$, $B$ being the maximum planning budget and $v_\text{desired}$ being the desired speed of our UAV. $C$ denotes the exploration weight, and $N_\text{tree}$ denotes the number of visits to the tree root node while $N_\text{node}$ denotes the number of times the present node has been visited.

After selecting a candidate node, if it has been visited before, it is expanded by applying motion primitives to generate child nodes, growing the tree. Unvisited nodes skip this step. Following expansion, either the unvisited candidate node or one of its children is selected for the simulation phase, where the future values of nodes along the path are estimated to update the total potential information gain. This informs the selection policy in subsequent iterations. Once planning time is exhausted, the path with the highest information gain is returned.

% with authors goes here
\begin{figure}[t]
\centering
\includegraphics[trim={.7cm 0cm .5cm 1.4cm},clip,width=\columnwidth]{figs/5_/Results1v3.pdf}
\caption{The Monte Carlo simulation results for the planners. The plots show the average percent reduction in entropy over the course of the simulations, and the shading shows the 95\% confidence intervals. IA-TIGRIS outperforms all of the baselines.}
\label{fig:mc_results}
\end{figure}

While MCTS is probabilistically guaranteed to converge to the optimal path \cite{mcts_ref_1}, it is constrained to actions within a predefined set of motion primitives. Its reliance on random sampling to estimate the future value of nodes can result in poor approximations, particularly in environments with sparse, localized pockets of high information gain. This limitation is especially pronounced in large search areas or scenarios with large budgets constraints, where estimating future node values becomes increasingly expensive. As a result, in such scenarios, MCTS is often implemented with a finite planning horizon, which can restrict its ability to account for long-term consequences or dependencies in the environment.

% This property of MCTS, which causes unguided exploration of the environment, leads to increased convergence times on the optimal path, as a result of a lot of budget being spent in exploring information sparse areas of the map. 
% Also, the computation time of MCTS increases exponentially with the depth of the search tree. The time complexity of MCTS is given by $\mathcal{O}(\frac{T}{t_\text{iter}} \cdot |A|^d)$. Here, $T$ is the total planning time and $t_\text{iter}$ is the time taken per iteration of the planning loop. $|A|$ is the number of actions and $d$ represents the average depth of the search tree. 

% The above limitations are not inconsequential in the context of performing informative path planning in large high-dimensional search spaces. We compare MCTS with \PlannerName, in \ref{}, and empirically demonstrate its drawbacks and how \PlannerName, is able to outperform MCTS in the context of the mission parameters we examine in this work.  

\subsubsection{Greedy}

For the greedy planner, we iterated through each cell within the search bounds and calculated the reward for a given cell $i$ as $g_i = R(X_i) / d_i$ where $R(X_i)$ is given through \eqref{equ:reward} and $d_i$ represents the Euclidean distance between the current position the robot at the current time $t$ and the closest viewpoint to the cell. To compute this viewpoint, the yaw between the current pose of the robot and the intersected cell is first calculated. Using the robot's sensor configuration and this yaw, $x$ and $y$ coordinates are calculated that view the cell at the desired flight altitude. With this formulation, the planner prioritizes regions with a high ratio of entropy to distance. This can lead to locally optimal choices that contradict with paths that lead to higher information gain over the entire trajectory. 

% without authors goes here
% \begin{figure}[t]
% \centering
% \includegraphics[trim={.7cm 0cm .5cm 1.4cm},clip,width=\columnwidth]{figs/5_/Results1v3.pdf}
% \caption{The Monte Carlo simulation results for the planners. The plots show the average percent reduction in entropy over the course of the simulations, and the shading shows the 95\% confidence intervals. IA-TIGRIS outperforms all of the baselines.}
% \label{fig:mc_results}
% \end{figure}


\begin{figure*}[t]
    \centering
    \begin{subfigure}[b]{0.99\textwidth}
        \centering
        \includegraphics[trim={0cm 0.3cm 0cm 0cm},clip,width=\textwidth]{figs/5_/Fig2v1_target.png}
        % \caption{Slice by targets}
        % \vspace{.1cm}
    \end{subfigure}
    
    \begin{subfigure}[b]{0.99\textwidth}
        \centering
        \includegraphics[trim={0cm 0cm 0cm 0cm},clip,width=\textwidth]{figs/5_/Fig2v1_sigma.png}
        % \caption{Slice by sigma }
    \end{subfigure}
    \caption{A comparison of the methods based on the number of sampled prior clusters and the standard deviation of sampled prior clusters. IA-TIGRIS is most effective compared to the baselines when there is high variation in the search space. As the search space prior information becomes more evenly spread out, the performance gap between the methods tends to decrease.}
    \label{fig:targets_sigmas}
\end{figure*}

\subsubsection{Random}

The random planner operates by iteratively sampling points within the defined search bounds and calculating the minimum-cost path to observe each sampled point. This process is repeated until the available budget is fully expended. The random planner does not utilize any prior information about the environment or target distribution. Additionally, it does not optimize the sequence of actions, instead treating each sampled point independently without considering the global structure of the search problem. This simplicity allows the random planner to highlight the performance benefits of more sophisticated methods by providing a lower-bound comparison for evaluation.

\subsubsection{Coverage}

The coverage planner generates a plan that systematically covers the entire search space using a straightforward lawn-mower pattern. The spacing between each pass is set to match the width of the projected observation footprint at 20\% from the bottom, ensuring that no grid cells are missed. This spacing also maintains a distance that enables high-quality sensor measurements. However, due to the size of the search spaces considered, the coverage planner spends significant time surveying empty regions. This approach results in inefficient use of the budget, as it prioritizes full coverage with safe sensor overlap, even in areas with little or no valuable information. While simple and robust, this method highlights the tradeoff between exhaustive coverage and efficient, targeted exploration.

% \subsubsection{Branch and Bound}
% The branch and bound baseline is based on motion primitive planning. In each future step the drone has a set of motion primitives with future states and each of these future states also has a set of motion primitives. In this way, a tree can be built with multiple path candidates. The path candidate with the highest information gain will be selected and form the output. 

% By adding branch and bound, there will be an estimation of a node's upper bound information reward, using the node's current information reward, updated information map and the remaining budget. If this upper bound is already lower than the information reward of any other node in the tree, the corresponding node will be closed and not expanded in the future to accelerate the expansion of the tree. 



\subsection{Tests and Analysis}
% To evaluate the efficacy of IA-TIGRIS compared to the baseline methods, we conduct Monte Carlo testing as well as analyze how the prior and budget affect the performance of each method. In all of these test cases, there are no time-based or priority rewards and have horizon lengths set to the full budget. All tests were performed using an Intel Xeon CPU E5-2620 v4 @ 2.10GHz.
To evaluate the efficacy of IA-TIGRIS against baseline methods, we perform Monte Carlo testing and analyze the impact of the prior and budget on the performance of each method. In all test cases, rewards are calculated using \eqref{equ:reward}, and horizon lengths are set to match the full budget. The tests are conducted on an Intel Xeon CPU E5-2620 v4 @ 2.10GHz, ensuring consistent computational conditions across all evaluations.

% Random sample across which parameters.

% Quantitative ideas. Look into number and std of prior (metric for this? std of grid cell values, mediuan, mean,). 
% Uniform prior? 
% Split distinct regions, not smooth. 
% Compare to coverage and amount of time to reach specific amount. 
% Compare with different budgets. 
% Repeatability test. 
% Graph size vs time. 
% Look at coverage with different altitudes or widths. Something that shows long horizon vs not nature of things?
% Shape of search space?
% Time/budget to get x\% of all info gain. Have to do moving horizon. 
% Targets detected? 

% Key thought for results where I show time, our optimization does not optimize for time, only final value. Key thing to show across the different budgets. 

% \BM{Qualitative. Nayana idea of plot with example sampled case. Should add one here.} 



\subsubsection{Monte Carlo Testing}
Our simulated testing environment is a $5000\times5000$ m square with Gaussian-distributed prior information randomly placed throughout the search space. The number of prior clusters was sampled uniformly between $[4,20]$, with standard deviations between $[60,450]$, and maximum value between $[0.05,0.5]$. 

The results of $100$ Monte Carlo tests are shown in Fig.~\ref{fig:mc_results}. IA-TIGRIS clearly outperforms the other methods, achieving nearly a $40\%$ greater reduction in entropy than the next best method. Early in the simulation, the greedy method initially gains information more quickly, as expected, but this does not translate to better long-term performance. Since our method optimizes for total information gain, it generates paths that maximize information collection over the entire budget. MCTS performed slightly worse than the greedy approach.

The random paths slightly outperformed the coverage paths. This is likely because the lawnmower strategy requires sufficient overlap between passes to avoid missing areas, and its long straight paths often lead to redundant observations due to the UAV’s forward-facing camera. Changing the heading of the UAV is beneficial to viewing more of the search space, which may explain why random paths performed better.

We also conducted Monte Carlo tests where either the number of prior clusters or their standard deviation was held constant to analyze how variations in the information map affect planner performance. The results, shown in Fig.~\ref{fig:targets_sigmas}, include two cases: the upper figure fixes the number of priors, while the lower figure fixes their standard deviation. All other agent and simulation parameters remained unchanged.


% The first thing to note from these results is that for all tests the proportional performance gap between IA-TIGRIS and the baselines increases as the number and standard deviation of the Gaussian priors decreases. As the search space becomes more uniformly filled with entropy in the information map, the need for longer-horizon planning decreases and other simple or random approaches can perform satisfactorily given the testing budget. As the information becomes more sparsely distribution in the space, such as when the information is contained in separated pockets of areas, there is a greater need to plan longer-horizon paths that reason about the given budget.
% \BM{Could have figures here or refer to others}

Across these tests, the performance gap between IA-TIGRIS and the baselines widens as the number and standard deviation of the Gaussian priors decrease. When entropy is more uniformly distributed across the search space, simpler methods perform reasonably well within the given budget. However, when information is concentrated in sparse, distinct regions, longer-horizon planning becomes essential. In such cases, IA-TIGRIS demonstrates a significant advantage by effectively reasoning about the budget and prioritizing high-value regions.

% Show plot of first plans expected info gain versus planning time. (plans not executed)


\subsubsection{Budget Analysis}
To evaluate the impact of budget constraints on performance, we conducted additional tests beyond our initial Monte Carlo experiments, evaluating budgets of $5000$ m, $10000$ m, $30000$ m, and $60000$ m. Table~\ref{tab:budgets} summarizes the average entropy reduction across these budgets.

\definecolor{tabfirst}{rgb}{1, 0.7, 0.7} % red
\definecolor{tabsecond}{rgb}{1, 0.85, 0.7} % orange
\definecolor{tabthird}{rgb}{1, 1, 0.7} % yellow
\begin{table}[t]
    \centering
    \resizebox{\linewidth}{!}{
    \begin{tabular}{l|ccccc}
    & $5000$ m & 10000 m  & 15000 m& 30000 m& 60000 m\\ \hline

    % \hline
    IA-TIGRIS  &  \cellcolor{tabfirst}$9.41\pm1.0$ &  \cellcolor{tabfirst}$18.28\pm1.8$ & \cellcolor{tabfirst}$25.36\pm2.3$ & \cellcolor{tabfirst}$41.08\pm2.9$ & \cellcolor{tabfirst}$58.85\pm2.9$ \\
    Greedy  &  \cellcolor{tabsecond}$6.99\pm0.8$ &  \cellcolor{tabsecond}$13.10\pm1.5$ & \cellcolor{tabsecond}$17.97\pm2.0$ & \cellcolor{tabthird}$30.00\pm2.3$ & \cellcolor{tabsecond}$49.38\pm3.5$ \\
    MCTS  &  \cellcolor{tabthird}$6.06\pm0.7$ &  \cellcolor{tabthird}$11.80\pm1.1$ & \cellcolor{tabthird}$17.11\pm1.4$ & \cellcolor{tabsecond}$30.21\pm2.2$ & \cellcolor{tabthird}$48.68\pm2.7$ \\
    Random  &  $2.19\pm0.3$ & $4.29\pm0.7$ & $6.61\pm0.6$ & $17.50\pm1.2$ & $22.47\pm1.4$ \\
    Coverage  &  $1.58\pm0.3$ &  $2.82\pm0.4$ & $4.09\pm0.7$ & $12.04\pm1.9$ & $16.77\pm2.4$ \\

    \end{tabular}
    }
    \caption{Monte Carlo testing results given different budgets. The values are the average percent reduction in entropy and the 95\% confidence bounds. \mbox{IA-TIGRIS} had the best performance for all budgets.}
    \label{tab:budgets}
\end{table}
%$\uparrow$ 

IA-TIGRIS consistently achieved the highest entropy reduction across all budget constraints, with a statistically significant margin over alternative methods. Greedy generally ranked second but was slightly outperformed by MCTS at the $30000$ m budget level. Greedy and MCTS exhibited comparable performance throughout the tests, with their results closely tracking each other. Consistent with our previous findings, Random and Coverage methods yielded the lowest results.


Among the tested methods, only IA-TIGRIS and MCTS explicitly incorporate budget constraints into their planning algorithms. Notably, at lower budgets ($5000$ m and $10000$ m), these methods achieved higher entropy reduction compared to the equivalent time steps ($200$ s and $400$ s) in the $15000$ m budget scenario shown in Fig.~\ref{fig:mc_results}. This improved performance stems from IA-TIGRIS's optimization of total path reward under budget constraints, contrasting with the myopic next-best-action approach of the greedy method. The remaining methods---Greedy, Random, and Coverage---maintain consistent behavior regardless of budget constraints, as their planning strategies do not account for resource limitations.


The performance gap between IA-TIGRIS and the next-best method varied with budget size, showing margins of $34.6\%$, $39.5\%$, $41.1\%$, $36.0\%$, and $19.2\%$ in ascending budget order. This gap widened through the first three budget levels as problem complexity increased, before declining significantly at higher budgets. This performance pattern suggests that implementing a planning horizon could enhance efficiency by limiting tree search depth, enabling the planner to prioritize path quality optimization over exhaustive space exploration.


% percent improved from next best
% 34.6, 39.5, 41.1, 36.0, 19.2
% reasons, too long horizon is a larger search space, so less quality paths closer. Or larger horizon, more packing in


% with authors goes here
\begin{figure}[t] 
    \centering
    \renewcommand\arraystretch{0} % Adjust the height between rows here
    \setlength{\tabcolsep}{1pt} % Adjust the column separation here
    \begin{tabular}{c}
        \begin{tikzpicture}
            \node[anchor=south west, inner sep=0] (image) at (0,0) {
                \includegraphics[width=0.9\linewidth]{figs/5_/google_earth_prior.png}
            };
            \begin{scope}[x={(image.south east)},y={(image.north west)}]
                % \fill[OrangeRed] (0.02, 0.03) circle (2pt); 
                % \fill[OrangeRed] (0.51, 0.04) circle (2pt); 
                % \fill[OrangeRed] (0.61, 0.04) arc (0:90:2pt); 
                \fill[Orange, opacity=0.8] (0.74, 0.45) circle (3pt); % Adjust 
                \fill[Orange, opacity=0.8] (0.27, 0.42) circle (3pt); % Adjust 
                \fill[Orange, opacity=0.8] (0.39, 0.63) circle (3pt); % Adjust 
            \end{scope}
        \end{tikzpicture} \\
        % \includegraphics[width=0.9\linewidth]{figs/5_/google_earth_prior.png} \\
        \\
        \includegraphics[width=0.9\linewidth]{figs/5_/google_earth_path.png} 
    \end{tabular}
    \caption{Google Earth screenshots illustrating the mission planning process and execution. Top: Areas of high entropy targeted for search are highlighted in red, representing regions with a binary occupied/unoccupied probability of 0.2. Three points of particular interest, each assigned a 0.5 probability, are marked in orange. Bottom: The executed drone flight path (yellow) shows the optimized path for maximum information gain across the search space.} 
    \label{fig:google_earth}
\end{figure}
\begin{figure}[t]
\centering
% https://docs.google.com/presentation/d/1RjI-QqHpBRLHN60UAxzmQYs4EaWaVCOoSBkEkA39kk0/edit?usp=sharing
\includegraphics[width=\columnwidth]{figs/5_/m600_labeled.jpg}
\caption{Hexarotor system (DJI M600 Pro) with onboard compute and camera. Left image shows drone on the ground, right image shows drone in flight.}
\label{fig:m600}
\end{figure}


\section{Field Deployments}\label{sec:field}


\subsection{Hexarotor Deployment}
The first field experiment that we present uses a hexarotor drone to cover an urban area shown in Fig.~\ref{fig:fig1}.
We designed this field experiment to simulate classifying where cars are within a search area.  
Hence, we set the plan request to focus on parking lots at the field test site (Fig.~\ref{fig:google_earth}, top), with the addition of three chosen grid cells within the parking lots being marked as having a higher uncertainty. The plan request boundaries and priors were created with GPS coordinates in Google Earth, exported as kml files, and then converted into our plan request message format. 

The following sections details the hardware, autonomy, and experimental results for our hexarotor deployments.

% without the authors goes here
% \begin{figure}[t] 
%     \centering
%     \renewcommand\arraystretch{0} % Adjust the height between rows here
%     \setlength{\tabcolsep}{1pt} % Adjust the column separation here
%     \begin{tabular}{c}
%         \begin{tikzpicture}
%             \node[anchor=south west, inner sep=0] (image) at (0,0) {
%                 \includegraphics[width=0.9\linewidth]{figs/5_/google_earth_prior.png}
%             };
%             \begin{scope}[x={(image.south east)},y={(image.north west)}]
%                 % \fill[OrangeRed] (0.02, 0.03) circle (2pt); 
%                 % \fill[OrangeRed] (0.51, 0.04) circle (2pt); 
%                 % \fill[OrangeRed] (0.61, 0.04) arc (0:90:2pt); 
%                 \fill[Orange, opacity=0.8] (0.74, 0.45) circle (3pt); % Adjust 
%                 \fill[Orange, opacity=0.8] (0.27, 0.42) circle (3pt); % Adjust 
%                 \fill[Orange, opacity=0.8] (0.39, 0.63) circle (3pt); % Adjust 
%             \end{scope}
%         \end{tikzpicture} \\
%         % \includegraphics[width=0.9\linewidth]{figs/5_/google_earth_prior.png} \\
%         \\
%         \includegraphics[width=0.9\linewidth]{figs/5_/google_earth_path.png} 
%     \end{tabular}
%     \caption{Google Earth screenshots illustrating the mission planning process and execution. Top: Areas of high entropy targeted for search are highlighted in red, representing regions with a binary occupied/unoccupied probability of 0.2. Three points of particular interest, each assigned a 0.5 probability, are marked in orange. Bottom: The executed drone flight path (yellow) shows the optimized path for maximum information gain across the search space.} 
%     \label{fig:google_earth}
% \end{figure}
% \begin{figure}[t]
% \centering
% % https://docs.google.com/presentation/d/1RjI-QqHpBRLHN60UAxzmQYs4EaWaVCOoSBkEkA39kk0/edit?usp=sharing
% \includegraphics[width=\columnwidth]{figs/5_/m600_labeled.jpg}
% \caption{Hexarotor system (DJI M600 Pro) with onboard compute and camera. Left image shows drone on the ground, right image shows drone in flight.}
% \label{fig:m600}
% \end{figure}

\subsubsection{Hardware System}
The hardware consists of the DJI M600 Pro, shown in Fig.~\ref{fig:m600}, along with the physical sensing and onboard computer payload. The DJI M600 Pro contains a flight controller that handles pose estimation and position-based control. The DJI M600 Pro’s flight controller also handles teleloperation if human intervention is necessary. Beneath the drone's base, we mount a custom hardware payload.
That payload consists of an onboard computer, a Jetson Xavier, to run the autonomy software shown in Fig.~\ref{fig:functional_diagram}.
The payload also contains a downward-facing a camera for sensing the environment. The camera is a Seek S304SP thermal camera.
The camera intrinsics are used to calculate the frustum's intersection with the search map's cells in IA-TIGRIS.

% without authors goes here
\begin{figure}[t]
\centering
% https://lucid.app/lucidchart/f750ddb4-2809-4773-8361-d5fbb1ba49eb/edit?viewport_loc=-257%2C-116%2C2219%2C1140%2C0_0&invitationId=inv_56e8a3a9-e8cf-4cad-a280-48bd967ff651
\includegraphics[trim={0cm 0cm 0cm 0cm},clip,width=\columnwidth]{figs/5_/functional_diagram.jpeg}
\caption{Functional diagram of the DJI M600 Pro autonomy software.}
\label{fig:functional_diagram}
\end{figure}
\begin{figure}[b]
    \centering
    \begin{subfigure}[b]{0.48\columnwidth}
        \centering
        \includegraphics[width=1.0\linewidth]{figs/5_/field_test_altitude_over_time.png}
        \caption{}
        \label{fig:m600_altitude_over_time}
    \end{subfigure}
    \begin{subfigure}[b]{0.48\columnwidth}
        \centering
        \includegraphics[width=1.0\linewidth]{figs/5_/field_test_entropy_over_time.png}
        \caption{}
        \label{fig:m600_entropy_over_time}
    \end{subfigure}
    \caption{The results for our hexarotor field deployment. (a) Plot of flown altitude over time, showing large variation throughout the experiment. (b) Reduction in entropy percentage over time of field experiment.}
\end{figure}

\subsubsection{Autonomy System}
Fig.~\ref{fig:functional_diagram} illustrates the functional system diagram for the real world field test on the DJI M600. The user specifies the initial plan request prior to takeoff. The TIGRIS planner makes an initial plan on that plan request and sends a global path to the waypoint manager. The waypoint manager tracks the current waypoint within the plan and sends the next waypoint to the DJI software development kit, which then sends actuation commands to the motors. The position of the drone is used to calculate the distance from the drone to the ground and sends that distance parameter to the sensor model. The sensor model's true positive and false positive rate is used to calculate the per-cell entropy updates in the search map manager. The search map manager publishes the current information map, and the replanning node sends an updated plan request to the IA-TIGRIS planner every ten seconds.

The drone started at an altitude of $50$ m above the origin of the reference frame. The informed sampler in IA-TIGRIS was set to add states at altitudes of either $30$ m or $60$ m, creating a trade-off between observation area and detector accuracy. The budget was $2000$ m, the planning horizon was $600$ m, and the planning time was $10$ seconds. 

% % without authors goes here
% \begin{figure}[t]
% \centering
% % https://lucid.app/lucidchart/f750ddb4-2809-4773-8361-d5fbb1ba49eb/edit?viewport_loc=-257%2C-116%2C2219%2C1140%2C0_0&invitationId=inv_56e8a3a9-e8cf-4cad-a280-48bd967ff651
% \includegraphics[trim={0cm 0cm 0cm 0cm},clip,width=\columnwidth]{figs/5_/functional_diagram.jpeg}
% \caption{Functional diagram of the DJI M600 Pro autonomy software.}
% \label{fig:functional_diagram}
% \end{figure}
% \begin{figure}[b]
%     \centering
%     \begin{subfigure}[b]{0.48\columnwidth}
%         \centering
%         \includegraphics[width=1.0\linewidth]{figs/5_/field_test_altitude_over_time.png}
%         \caption{}
%         \label{fig:m600_altitude_over_time}
%     \end{subfigure}
%     \begin{subfigure}[b]{0.48\columnwidth}
%         \centering
%         \includegraphics[width=1.0\linewidth]{figs/5_/field_test_entropy_over_time.png}
%         \caption{}
%         \label{fig:m600_entropy_over_time}
%     \end{subfigure}
%     \caption{The results for our hexarotor field deployment. (a) Plot of flown altitude over time, showing large variation throughout the experiment. (b) Reduction in entropy percentage over time of field experiment.}
% \end{figure}

\subsubsection{Experimental Results}


The bottom image of Fig.~\ref{fig:google_earth} shows the path selected by IA-TIGRIS in the search area. The figure highlights how the planner dynamically adjusts altitudes over time to balance coverage and sensing resolution, maximizing information gain. Higher altitudes allow for broader area coverage, while lower altitudes provide more detailed observations where needed. Additionally, the planner prioritizes revisiting the three regions of higher uncertainty, recognizing the need for repeated observations reduce entropy. This adaptive strategy ensures that uncertain areas receive sufficient attention to improve the belief map. As a result, the entropy of the information map decreases to near zero by the end of the mission, as shown in Fig.~\ref{fig:m600_entropy_over_time}, indicating that the planner has effectively gathered the necessary information. This behavior demonstrates the planner’s ability to optimize sensing actions, balancing altitude selection, revisit frequency, and exploration to maximize mission success.

\begin{figure}[t]
\centering
% \includegraphics[width=2.5in]{fig1}
\includegraphics[trim={4cm 4cm 0cm 4cm},clip,width=\columnwidth]{figs/5_/TL1.jpg}
\caption{Fixed-wing platform used for autonomous flights with an onboard camera pitched at 10 degrees\cite{alarewebsite}}
\label{fig:tl1}
\end{figure}






\subsection{Fixed-wing Deployments}

Our proposed approach was extensively tested on the fixed-wing AlareTech TL-1 UAV, shown in Fig.~\ref{fig:tl1}. The UAV is equipped with an onboard camera pitched at 10 degrees, which introduces a more challenging planning problem due to the non-holonomic motion model and the camera's field of view. Over more than 20 flight hours and 100 flights running IA-TIGRIS, we validated our approach with the objective to search for objects of interest in a large search space across a variety of test scenarios, including different terrain types, varying environmental conditions, and diverse target distributions. An example mission from these tests is shown in Fig.~\ref{fig:fwd}. In this scenario, the planner was given the search bounds and a designated high-priority region. The resulting flight path prioritized revisiting the high-priority area twice, optimizing sensor use and ensuring maximum information gain. This strategy led to the successful detection of the object of interest, with its estimated position marked by the red dot in the figure. 

The map on the upper right in Fig.~\ref{fig:fwd} shows the information map after plan execution was complete. Due to the UAV's limited budget, the upper right and lower left corners of the map are not searched by the agent. The budget is instead utilized to search over the area of higher priority two times. Compared to the paths in Fig.~\ref{fig:google_earth}, we observe that the paths for the fixed wing are smoother and have a larger turning radius, demonstrating how IA-TIGRIS respects the motion constraints of the vehicle. We can also see the effect of wind on the path execution, where the flown path shown in green deviates from the planned path shown in yellow. This illustrates the importance of online planning in the cases where this deviation is large or would accumulate over the course of a longer mission and cause the expected observed area to be much different than actual observed area. 

\begin{figure}[t]
\centering
% \includegraphics[width=2.5in]{fig1}
% [trim={left bottom right top},clip]
\includegraphics[trim={3.0cm, 1.0cm, 3.0cm, 1.0cm},clip,width=\columnwidth]{figs/5_/ONRFig_v3.pdf}
\caption{An example path generated for the fixed-wing platform conducting a large-area search for an object of interest. The larger black rectangle denotes the search bounds, while the smaller black rectangle highlights a region of higher uncertainty. The red dot marks the estimated position of the detected object based on image detections. The upper-right map displays the information state after planning is complete, while the middle plot shows the percent change in entropy over mission time. The flown path illustrates a balance between allocating resources to the high-priority region and exploring other areas within the search space.}
\label{fig:fwd}
\end{figure}

% Also tested extensively on the AlareTech TL-1 (citation?) tube launched UAV seen in Fig.~\ref{fig:tl1}.

% Talk about amount of flights, hours. Platform. Compute. Show visualization fo example flight. Talk about objects of interest in a broad sense (no mention of water/ocean/land for targets). Follow similar figure format as previous section. Main thing we want to highlight is the differences introduced in plans by having a fixed-wing platform compared to a drone. Include image of Alare TL-1 somewhere.

% One big figure showing all the info we want to convey. 

% \BM{Pitch 10 degrees, onboard computer type, etc}


% \subsection{VTOL?}
% what would it bring?


\section{Discussion}
\section{Discussion}
\label{sec:discussion}

% \TODO{Bryan}

Our multimodal data augmentation method is a plug-and-play method that can be applied to any future VLM. Also the T2I generation can be replaced by any future T2I model, thus the effectiveness of our method automatically improves along with the SOTA T2I model, making it future-proof.



Our main method, \textbf{Co}ntrastive Visual \textbf{D}ata \textbf{A}ugmentation (\textbf{CoDA}), is simple and easy to apply to LMMs in a variety of scenarios. Several components in the pipeline utilize existing off-the-shelf model components that can be easily swapped out for superior versions of similar models as research in their respective field progresses. Therefore, we expect the efficiency and effectiveness of \textbf{CoDA} to dramatically scale along with the advancement of relevant models. 



\bibliographystyle{ACM-Reference-Format}
\bibliography{references}

\appendix

\section{Metric}
\label{sec:metric}

\textbf{Mean Squared Error (MSE)} Mean Squared Error (MSE) is a common statistical metric used to assess the difference between predicted and actual values. The formula is:
\begin{equation}
    MSE = \frac{1}{n} \sum_{i=1}^{n} (y_i - \hat{y}_i)^2
\end{equation}
where $ n $ is the number of samples, $ y_i $ is the actual value, and $ \hat{y}_i $ is the predicted value.

\textbf{Relative L2 Error} Relative L2 error measures the relative difference between predicted and actual values, commonly used in time series prediction. The formula is:
\begin{equation}
    \text{Relative L2 Error} = \frac{\| Y_{\text{pred}} - Y_{\text{true}} \|_2}{\| Y_{\text{true}} \|_2}
\end{equation}
where $ Y_{\text{pred}} $ is the predicted value and $ Y_{\text{true}} $ is the actual value.

\textbf{Structural Similarity Index Measure (SSIM)} The Structural Similarity Index (SSIM) measures the similarity between two images in terms of luminance, contrast, and structure. The formula is:
\begin{equation}
    SSIM(x, y) = \frac{(2\mu_x \mu_y + C_1)(2\sigma_{xy} + C_2)}{(\mu_x^2 + \mu_y^2 + C_1)(\sigma_x^2 + \sigma_y^2 + C_2)}
\end{equation}
where $ \mu_x $ and $ \mu_y $ are the mean values, $ \sigma_x $ and $ \sigma_y $ are the standard deviations, $ \sigma_{xy} $ is the covariance.

\section{Related Work}
\subsection{Deep Learning based Weather Forecasting}
\textbf{Global Weather Forecasting.} Global weather forecasting has seen significant progress with deep learning models. FourCastNet, based on Fourier neural operators, provides global forecasts comparable to traditional numerical methods like IFS, but at much higher speeds~\cite{pathak2022fourcastnet}. Pangu, utilizing the Swin Transformer, exceeds NWP methods, incorporating earth-specific location embeddings for better performance~\cite{bi2023accurate}. The Spherical Fourier Neural Operator (SFNO) extends Fourier methods using spherical harmonics, offering more stable long-term predictions~\cite{bonev2023spherical}. FuXi focuses on long-term forecasting, achieving a 15-day forecasts comparable to ECMWF~\cite{chen2023fuxi}. GraphCast leverages message-passing networks to improve efficiency and forecasting accuracy~\cite{lam2023learning}, and GenCast builds on this to enhance ensemble forecasting~\cite{price2023gencast}. Further, diffusion models like those in~\cite{li2024generative} generate probabilistic ensembles by sampling, while NeuralGCM~\cite{kochkov2024neural} focuses on atmospheric circulation with a dynamic core, offering climate simulation capabilities but at higher training and inference costs. 

\textbf{Regional Weather Forecasting.} The goal of regional weather forecasting is to enhance local prediction accuracy with high-resolution models. CorrDiff~\cite{mardani2023generative} combines U-Net and diffusion models to improve local forecasts. MetaWeather~\cite{kim2024metaweather} adapts global forecasts to regional contexts using meta-learning. GNNs are also widely applied in regional forecasting, with Graphcast~\cite{lam2023learning} enhancing accuracy by modeling complex spatial dependencies. MetNet-3~\cite{espeholt2022deep} offers high-accuracy forecasts for weather variables, such as precipitation, temperature, and wind speed, at 2-minute intervals and 1–4 km resolution, outperforming traditional models like HRRR. NowcastNet~\cite{zhang2023skilful} and DGMR~\cite{ravuri2021skilful} excel in short-term extreme precipitation forecasts using deep generative models and radar data. In spatiotemporal prediction, NMO~\cite{wu2024neural} models the evolution of physical dynamics, providing new insights for local weather forecasting. Similarly, SimVP~\cite{gao2022simvp} and PastNet~\cite{wu2024pastnet} achieve good results in forecasting local precipitation evolution using spatiotemporal convolution methods.
    
% Despite these advances, none of these methods effectively address the challenge of balancing global and regional high-resolution forecasts or capturing the fine-grained, dynamic interactions important for extreme event prediction.
    
\subsection{Numerical analysis methods}
Multigrid methods~\cite{mccormick1987multigrid,wesseling1995introduction,hackbusch2013multi,bramble2019multigrid,hiptmair1998multigrid,brandt1983multigrid,borzi2009multigrid} and nested grid strategies~\cite{miyakoda1977one,zhang2012nested,sullivan1996grid} are widely used to solve PDEs and handle multi-scale problems~\cite{debreu2008two,xue2000advanced}. Multigrid methods use grids of different resolutions to transfer information and accelerate iterations. They efficiently solve large-scale problems and improve computational accuracy. By eliminating low-frequency errors on coarse grids and high-frequency errors on fine grids, multigrid methods effectively handle error convergence at different scales~\cite{he2019mgnet,he2023mgno,shao2022fast}. Nested grid strategies embed higher-resolution fine grids into regions of interest based on a global coarse grid to capture local complex physical phenomena in detail. In weather forecasting, this method provides large-scale background fields on a global scale while refining the grid for target regions to accurately simulate the evolution of local weather systems and the occurrence of extreme events~\cite{bacon2000dynamically}. 

% Our proposed neural nested grid method helps address challenges like boundary information loss in regional forecasting and multi-scale feature capture.

\section{Additional Results}
%
We present more additional results in Figure \ref{fig_0.25-day}, \ref{fig_0.5-day}, \ref{fig_1.0-day} \ref{fig_1.5-day}, \ref{fig_2.0-day}, \ref{fig_2.5-day}, \ref{fig_3.0-day}, \ref{fig_3.5-day}, \ref{fig_4.0-day}, \ref{fig_4.5-day}, \ref{fig_5.0-day}, \ref{fig_5.5-day}, \ref{fig_6.0-day}, \ref{fig_6.5-day}, \ref{fig_7.0-day}, \ref{fig_7.5-day},
\ref{fig_8.0-day}, \ref{fig_8.5-day}, \ref{fig_9.0-day}, \ref{fig_9.5-day},
\ref{fig_10.0-day}, including 18 variables that are importmant to weather forecasting, each with results ranging from 6 hours to 10 days. These additional results further demonstrate the effectiveness of OneForecast. Same as the Figure \ref{fig:visual_results}
, the initial conditions is 00:00 UTC, 1 January 2020.


\begin{figure*}[h]
\centering
\includegraphics[width=1\linewidth]{figures/fig_0.25-day.jpg}
\vspace{-20pt}
\caption{6-hour forecast results of different models.}
\label{fig_0.25-day}
\end{figure*}

\begin{figure*}[h]
\centering
\includegraphics[width=1\linewidth]{figures/fig_0.5-day.jpg}
\vspace{-20pt}
\caption{0.5-day forecast results of different models.}
\label{fig_0.5-day}
\end{figure*}

\begin{figure*}[h]
\centering
\includegraphics[width=1\linewidth]{figures/fig_1.0-day.jpg}
\vspace{-20pt}
\caption{1-day forecast results of different models.}
\label{fig_1.0-day}
\end{figure*}

\begin{figure*}[h]
\centering
\includegraphics[width=1\linewidth]{figures/fig_1.5-day.jpg}
\vspace{-20pt}
\caption{1.5-day forecast results of different models.}
\label{fig_1.5-day}
\end{figure*}

\begin{figure*}[h]
\centering
\includegraphics[width=1\linewidth]{figures/fig_2.0-day.jpg}
\vspace{-20pt}
\caption{2-day forecast results of different models.}
\label{fig_2.0-day}
\end{figure*}


\begin{figure*}[h]
\centering
\includegraphics[width=1\linewidth]{figures/fig_2.5-day.jpg}
\vspace{-20pt}
\caption{2.5-day forecast results of different models.}
\label{fig_2.5-day}
\end{figure*}

\begin{figure*}[h]
\centering
\includegraphics[width=1\linewidth]{figures/fig_3.0-day.jpg}
\vspace{-20pt}
\caption{3-day forecast results of different models.}
\label{fig_3.0-day}
\end{figure*}

\begin{figure*}[h]
\centering
\includegraphics[width=1\linewidth]{figures/fig_3.5-day.jpg}
\vspace{-20pt}
\caption{3.5-day forecast results of different models.}
\label{fig_3.5-day}
\end{figure*}

\begin{figure*}[h]
\centering
\includegraphics[width=1\linewidth]{figures/fig_4.0-day.jpg}
\vspace{-20pt}
\caption{4-day forecast results of different models.}
\label{fig_4.0-day}
\end{figure*}

\begin{figure*}[h]
\centering
\includegraphics[width=1\linewidth]{figures/fig_4.5-day.jpg}
\vspace{-20pt}
\caption{4.5-day forecast results of different models.}
\label{fig_4.5-day}
\end{figure*}


\begin{figure*}[h]
\centering
\includegraphics[width=1\linewidth]{figures/fig_5.0-day.jpg}
\vspace{-20pt}
\caption{5.0-day forecast results of different models.}
\label{fig_5.0-day}
\end{figure*}

\begin{figure*}[h]
\centering
\includegraphics[width=1\linewidth]{figures/fig_5.5-day.jpg}
\vspace{-20pt}
\caption{5.5-day forecast results of different models.}
\label{fig_5.5-day}
\end{figure*}

\begin{figure*}[h]
\centering
\includegraphics[width=1\linewidth]{figures/fig_6.0-day.jpg}
\vspace{-20pt}
\caption{6.0-day forecast results of different models.}
\label{fig_6.0-day}
\end{figure*}

\begin{figure*}[h]
\centering
\includegraphics[width=1\linewidth]{figures/fig_6.5-day.jpg}
\vspace{-20pt}
\caption{6.5-day forecast results of different models.}
\label{fig_6.5-day}
\end{figure*}

\begin{figure*}[h]
\centering
\includegraphics[width=1\linewidth]{figures/fig_7.0-day.jpg}
\vspace{-20pt}
\caption{7.0-day forecast results of different models.}
\label{fig_7.0-day}
\end{figure*}

\begin{figure*}[h]
\centering
\includegraphics[width=1\linewidth]{figures/fig_7.5-day.jpg}
\vspace{-20pt}
\caption{7.5-day forecast results of different models.}
\label{fig_7.5-day}
\end{figure*}

\begin{figure*}[h]
\centering
\includegraphics[width=1\linewidth]{figures/fig_8.0-day.jpg}
\vspace{-20pt}
\caption{8.0-day forecast results of different models.}
\label{fig_8.0-day}
\end{figure*}

\begin{figure*}[h]
\centering
\includegraphics[width=1\linewidth]{figures/fig_8.5-day.jpg}
\vspace{-20pt}
\caption{8.5-day forecast results of different models.}
\label{fig_8.5-day}
\end{figure*}

\begin{figure*}[h]
\centering
\includegraphics[width=1\linewidth]{figures/fig_9.0-day.jpg}
\vspace{-20pt}
\caption{9.0-day forecast results of different models.}
\label{fig_9.0-day}
\end{figure*}

\begin{figure*}[h]
\centering
\includegraphics[width=1\linewidth]{figures/fig_9.5-day.jpg}
\vspace{-20pt}
\caption{9.5-day forecast results of different models.}
\label{fig_9.5-day}
\end{figure*}

\begin{figure*}[h]
\centering
\includegraphics[width=1\linewidth]{figures/fig_10.0-day.jpg}
\vspace{-20pt}
\caption{10.0-day forecast results of different models.}
\label{fig_10.0-day}
\end{figure*}


\section{Detailed Mathematical Proof}
\label{sec:proof}
\textbf{Proof of Theorem 1}

Now we have N augmented data and we need to select the best from them. We consider both the quality and the diversity of these data and get the sampling strategy from an optimization problem.

We model the sampling strategy as a multinomial distribution supported on all the augmented data $S = \{\mathbf{X}_j\}_{j=1}^N$, which means that the sampling strategy $\pi=(\pi_1,...,\pi_N)^\top$ is the corresponding probabilities of selecting $\mathbf{X}_1,...,\mathbf{X}_N$, then we can model the expectation of the similarity as:
$$\begin{aligned}
 & \mathbb{E}_{Y_x,Y_{x^{\prime}}\in\mathcal{C}}\{g(x,x^{\prime})\mid S\} \\
 & =\quad\int g(\mathbf{x},\mathbf{x}^{\prime})\boldsymbol{\pi}(\mathbf{x})\mathrm{Pr}_{S}(Y_{x}\in\mathcal{C}\mid\boldsymbol{x}=\mathbf{x})\boldsymbol{\pi}(\mathbf{x}^{\prime})\mathrm{Pr}_{S}(Y_{x}\in\mathcal{C}\mid\boldsymbol{x}=\mathbf{x}^{\prime})d\mathbf{x}d\mathbf{x}^{\prime} \\
 & =\quad\sum_{i,j=1}^Ng(\mathbf{X}_i,\mathbf{X}_j)\pi_i\pi_j\mathrm{Pr}_{S}(Y_x\in\mathcal{C}\mid\boldsymbol{x}=\mathbf{X}_i)\mathrm{Pr}_{S}(Y_x\in\mathcal{C}\mid\boldsymbol{x}=\mathbf{X}_j),
\end{aligned}$$
where the set $\mathcal{C}$ denotes the criterion of selection we are using, the function $g$ can be chosen as any similarity metric function and $x$ means a random variable.

The core to solving the above optimization problem is to use predictive inference to approximate the conditional probability of $\{Y_x\in\mathcal{C}\}$ given $x = \mathbf{X}$
Let $\mu ( \mathbf{x} ) : = \mathbb{E} ( Y\mid \mathbf{X} = \mathbf{x} )$ be the oracle associated with $( \mathbf{X} , Y) .$ Denote $\theta_j=\mathbb{I}\{Y_j\in\mathcal{C}\}$. As the augmented data
$\mathbf{X}_1,...,\mathbf{X}_N$ are independently identically distributed, $\theta_1,...,\theta_N$ can be regarded as independent Bernoulli($q)$ variables with $q=\Pr(Y_j\in\mathcal{C}).$ The probability distribution of the predicted result $W_j$ for $j=1,...,N$ is
$$\Pr(W_j\mid\theta_j)=(1-\theta_j)f_0+\theta_jf_1,\quad$$
where $f_0$ and $f_1$ are the conditional distributions of $W_j$ on $Y_j \in \mathcal{C}$ or not.

Denote $T(w) = \frac{(1-q)f_0(W_j)}{f(W_j)}$, we can rewrite the expectation of the similarity as
$$\mathbb{E}_{Y_x,Y_{x^{\prime}}\in\mathcal{C}}\{g(x,x^{\prime})|S\}=\sum_{i,j=1}^Ng(\mathbf{X}_i,\mathbf{X}_j)\pi_i\pi_j(1-T_i)(1-T_j)=\boldsymbol{\pi}^\top A_\mathbb{T}\boldsymbol{\pi},$$

Next, we use the expectation to control the quality of the data.
$$\mathbb{E}\{\mathbb{I}(Y_x\not\in\mathcal{C})\mid S\}=\sum_{i=1}^N\Pr(Y_i\not\in\mathcal{C}\mid\mathbf{X}_i)\pi_i=\sum_{i=1}^N\pi_iT_i\leq\alpha,$$

In all, the optimization problem can be modeled as 
\begin{align}
    & \arg\min_{\boldsymbol{\pi}}\quad h(\boldsymbol{\pi},\mathbb{T}):=\boldsymbol{\pi}^\top A_\mathbb{T}\boldsymbol{\pi}, \\
    & \text{subject to} \quad
        \begin{cases}
            \sum_{i = 1}^N\pi_iT_i\leq\alpha, \\
            \sum_{i = 1}^N\pi_i = 1, \\
            0\leq\pi_i\leq m^{-1}, \quad 1\leq i\leq N.
        \end{cases}
\end{align}

where $m$ is used to control the maximum selection.

The best selection of K is determined by the strategy $\pi$ which serves as the solution to the above optimization problem.

\section{Additional Experiments}
\label{sec:more_experiments}
\subsection{Long-term forecasting experiment expansion}

In the long-term forecasting experiments, we compare the performance of different backbone models on the SWE benchmark, evaluating the relative L2 error for three variables (U, V, and H). Our setup inputs 5 frames and predicts 50 frames. For the SimVP-v2 model, using \method{} reduces the relative L2 error for SWE (u) from 0.0187 to 0.0154, SWE (v) from 0.0387 to 0.0342, and SWE (h) from 0.0443 to 0.0397. We visualize SWE (h) in 3D as shown in Figure~\ref{fig:case} [\textcolor{red}{I}]. For the ConvLSTM model, applying \method{} reduces the relative L2 error for SWE (u) from 0.0487 to 0.0321, SWE (v) from 0.0673 to 0.0351, and SWE (h) from 0.0762 to 0.0432. For the FNO model, using \method{} reduces the relative L2 error for SWE (u) from 0.0571 to 0.0502, SWE (v) from 0.0832 to 0.0653, and SWE (h) from 0.0981 to 0.0911. Overall, \method{} significantly improves the long-term forecasting accuracy of different backbone models.

\begin{figure*}[h]
    \centering
    \includegraphics[width=\textwidth]{image/casestudy.pdf}
    \caption{
    \textcolor{red}{I.} 3D visualization of the SWE(h), showing Ground-truth, SimVP-V2+BeamVQ predictions, and Error at T=1, 10, 20, 30, 40, 50. The first row shows Ground-truth, the second SimVP-V2+BeamVQ predictions, and the third Error. \textcolor{red}{II.} A case study. Building fire simulation with ventilation settings added to Wu's Prometheus~\cite{wu2024prometheus}. (a) Layout and HRR growth. (b) Comparison of physical metrics for different methods. (c) Ground-truth, ResNet+BeamVQ, and ResNet predictions.
    }
    \label{fig:case} 
\end{figure*}


\subsection{Experiment Statistical Significance}
\label{sec:significance}
To measure the statistical significance of our main experiment results, we choose three backbones to train on two datasets to run 5 times. 
Table~\ref{tab:significance} records the average and standard deviation of the test MSE loss.
The results prove that our method is statistically significant to outperform the baselines
because our confidence interval is always upper than the confidence interval of the baselines. 
Due to limited computation resources, we do not cover all ten backbones and five datasets, 
but we believe these results have shown that our method has consistent advantages.


\begin{table}[h]
\label{tab:significance}
\centering
\begin{scriptsize}
    \begin{sc}
    \caption{ The average and standard deviation of MSE in 5 runs}
    \label{tab:significance}
    \centering
        \renewcommand{\multirowsetup}{\centering}
        \setlength{\tabcolsep}{10pt}
        \begin{tabular}{l|cc|cc}
            \toprule
            
            \multirow{4}{*}{Model} & \multicolumn{4}{c}{Benchmarks}  \\
            \cmidrule(lr){2-5}
            & \multicolumn{2}{c}{NSE} &   \multicolumn{2}{c}{SEVIR}   \\
            \cmidrule(lr){2-5}
           & Ori & + BeamVQ & Ori & + BeamVQ  \\
            \midrule
            ConvLSTM &0.4092$\pm$0.0002 &\textbf{0.1277$\pm$0.0001}  & 0.1762 0.0007  & \textbf{0.1279$\pm$0.0009}  \\
            FNO &  0.2227$\pm$0.0003 &\textbf{0.1007 $\pm$0.0002}& 0.0787$\pm$0.0012 & \textbf{ 0.0437$\pm$0.0013} \\
            CNO & 0.2192 $\pm$0.0008 &\textbf{ 0.1492$\pm$0.0011}& 0.0057$\pm$0.0005 & \textbf{ 0.0053$\pm$0.0006} \\
            \bottomrule
        \end{tabular}
    \end{sc}

\end{scriptsize}
\end{table}
 
\end{document}
\endinput
%%
%% End of file `sample-sigconf.tex'.
