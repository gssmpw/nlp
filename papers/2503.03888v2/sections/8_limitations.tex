\section{Limitations}
\label{sec:limits}

We note several potential limitations to our approach. First, the use of machine learning may be seen to remove the intrinsic value of volunteer-based efforts that require close engagement with historical sources.  We acknowledge this trade-off, but we also note that the scale of deed records can make exclusively volunteer-based approaches impossible as a solution to documenting racial covenants across the country.  Moreover, we note that there are promising ways for citizens to engage with AI systems and model outputs -- such as through the interactive map that we provide -- that do not require them to function as a labeling workforce \citep{gray2019ghost}. Without needing volunteers to scale manual review, AI assistance can hence make room for distinct types of community engagement, such as around exploration of results, connection with other historical sources, and reflections on implications. 

Second, while our performance evaluation demonstrates remarkably strong results, wherein the AI system appears to spot racial covenants even in the face of serious OCR errors that can fool humans (see, e.g., Figure~\ref{fig:ocr-challenges}), our system does not have perfect recall. It may miss some racial covenants. It is, however, not clear that, relative to near perfect precision and 99.4\% recall, human performance would be any better. Human annotators can differ in quality, get tired after reading many documents, and miss the proverbial needle in the haystack.  

Third, our model may miss some covenants that are more subtle in nature. As \citet{rose2023general} notes, in \emph{Schulte v.\ Starks},\footnote{213 N.W.\ 102 (Mich.\ 1927).} the Michigan Supreme Court upheld the interpretation of a Detroit covenant that prohibited purchasers who ``would be injurious to the locality'' -- without explicit mention of race -- as barring African Americans. Humans may, of course, also miss these, and AI offers us a chance to prompt (via few-shot learning~\citep{wang2020generalizing}) for such boundary covenants. 

Fourth, and related, California civil rights law prohibits discrimination not just on the basis of race, but on the basis of many other protected attributes, including age, religion, sex, gender identity, familial status, disability, veteran or military status, national origin, genetic information, or source of income. We focused here on the predominant concern animating AB 1466, namely racial covenants, but county recorders may also need to assess for the presence of other restrictive covenants. While we do not explore that here, our general approach facilitates the discovery of such rare provisions through few-shot learning or further fine-tuning \citep{wang2020generalizing}. Indeed, our research has already uncovered instances of covenants plausibly based on family status, showing the potential path for a broader sweep to surface non-racial covenants.\footnote{For instance, one covenant indicates that any building is ``to be occupied by only one family.''} %

Last, some might object to the efforts to systematically redact racial covenants. Resources could be put to better use for affirmative housing reform, as these covenants cannot be enforced. Other commentators favor putting the onus on homeowners to redact deed records. Carol Rose, for instance, takes issue with proposals to redact deed records en masse, stating,  ``If these things are taken out of the record books, [they're] gone. It’s like Stalin pushing a button and saying ‘delete Bukharin.’''\footnote{Carol M.\ Rose, ``De-racing Property: Earl Dickerson and the Struggle Against Racially Restrictive Covenants,'' Dickerson Conference: Business Person and Movement Lawyer, University of Chicago Law School, Oct.\ 30, 2020, \url{https://www.youtube.com/watch?v=4FAIiLGCllo&t}, at 1:09:52.} We believe this critique is entirely apt for proposals to simply redact records. But critical to our approach -- and part of AB 1466 -- is the retention of nonredacted instances, making systematic and efficient documentation possible.\footnote{We believe Rose would agree, as she notes, 
``If you erase them from the records, you can’t have anything like these fabulous mapping projects'' (id.).} Technology here, if anything, enabled the understanding and mapping of racial covenants in Santa Clara County. AB 1466 could have mired local governments in compliance initiatives and technology frees up resources to focus on other goals. More generally, California's AB 1466 was a reaction against the perceived inefficacy of the law that required homeowners to take the initiative.  And if such statewide initiatives force greater systemic transparency around the historical use of racial covenants -- when deed records are otherwise only available for purchase on a one-off basis -- we view this as a compelling benefit. 