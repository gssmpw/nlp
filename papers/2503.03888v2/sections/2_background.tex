\section{Background}
\label{sec:background}

\subsection{Racial Covenants}
Racially restrictive covenants were legal clauses embedded in property deeds that prohibited the sale, lease, or occupation of land by individuals based on race. These covenants became a widespread tool for enforcing residential segregation in the United States during the first half of the 20th century. Generally, covenants ``\emph{run with the land},'' which meant that restrictions affect not only current but also all future owners of the real property~\citep{brooksrose2013saving}. While African Americans were the primary targets of racial covenants, other groups, such as Asians, Latinos, Jews, and Southern and Eastern Europeans, were also excluded from certain neighborhoods through the use of these discriminatory binding clauses~\citep{brooksrose2013saving}. Racial covenants were designed to maintain racially homogenous, white-majority neighborhoods by barring minority groups from settling in specific areas, and they were actively supported by real estate boards, developers, homeowner associations, and governmental institutions \citep{brooksrose2013saving}.

Racial covenants originated in the mid-19th century, but became more prevalent after  the Supreme Court held racial zoning unconstitutional in \emph{Buchanan v.\ Warley}.\footnote{245 U.S.\ 60 (1917). See also \citet[78]{rothstein2017color}.} During this period, white homeowners viewed racial covenants as a means to protect property values and maintain racial homogeneity within their communities. Integrating neighborhoods with non-white residents, particularly African Americans, was perceived as leading to economic decline and social instability.\footnote{\href{https://fraser.stlouisfed.org/files/docs/publications/fha/1936apr_fha_underwritingmanual.pdf}{The April 1936 edition of the \emph{Underwriting Manual} of the Federal Housing Administration} explicitly stated this as a policy as follows: 
``If a neighborhood is to retain stability it is necessary that properties shall continue to be occupied by the same social and racial classes. A change in social or racial occupancy generally leads to instability and a reduction in values. The protection offered against adverse changes should be found adequate before a high rating is given to this feature. Once the character of a neighborhood has been established it is usually impossible to induce a higher social class than those already in the neighborhood to purchase and occupy properties in its various locations.'' (Part II, para. 233)} Consequently, racial covenants were often marketed as desirable features in new suburban developments, which promoted ``restricted'' neighborhoods as more valuable, secure, and exclusive~\citep{santucci2020documenting,rothstein2017color}.

A typical restrictive covenant had the following language: 

{\setlength{\leftmargini}{2em} %
 \begin{quote}
 \footnotesize{
       ``No part of said property shall be sold, let, or leased, transferred, or assigned to, or occupied by any person not of the Caucasian race, or to be used by any other than a person of the Caucasian race.''
       }
 \end{quote}
}
But significant variation in the precise language is known to exist. 

Real estate institutions like the National Association of Real Estate Boards (NAREB) played a crucial role in institutionalizing racial covenants. From 1924 to 1950, NAREB’s code of ethics required realtors to engage in racial steering provisions, effectively ensuring that minority buyers were not introduced into white neighborhoods~\citep{brooksrose2013saving}.\footnote{The Article 34 of Part III of the 1924 NAREB's Realtors' Code of Ethics stated: ``A Realtor should never be instrumental in introducing into a neighborhood a character of property or occupancy, members of any race or nationality, or any individuals whose presence will clearly be detrimental to property values in that neighborhood.'' (as quoted in~\citep{brooksrose2013saving}.)} 
Violating this code could result in expulsion, further promoting racial segregation within the real estate industry \citep{santucci2020documenting}. Federal programs such as the Federal Housing Administration further entrenched these practices by making racial covenants a condition for mortgage insurance approval, thus embedding racial segregation in housing markets across the country~\citep{brooksrose2013saving}.

In 1948, the Supreme Court found racial covenants to be unenforceable in \emph{Shelley v.\ Kraemer}. The federal Fair Housing Act of 1968 prohibited the use of racial covenants. But because  covenants run with the land, such provisions have remained on the books, even if unenforceable. Much debate exists around the persistent impact of racial covenants. One perspective is that covenants institutionalized segregation in the housing market, contributing to enduring racial disparities in wealth accumulation, homeownership, and access to essential resources such as education and public services~\citep{MappingPrejudice2022_RRCs}. \citet{brooksrose2013saving} argue that racial covenants continued to have effect post-\emph{Shelley} as \emph{signaling} devices for the kind of community associated with the property. They provide an account stemming largely from litigated cases and the history in Chicago, and argue, based on game theory, that racial covenants were most widely deployed in ``loosely knit'' communities requiring a signaling device (id.). Using neighborhood data from Chicago, \citet{brooks2011covenants} estimates that racial covenants had effects lasting past \emph{Shelley}, consistent with signaling. Yet because a comprehensive register of racial covenants is so difficult to compile, studies have been limited in their ability to examine or test these accounts with quantitative evidence about the prevalence, dynamics, and geographic correlates of racial covenants.\footnote{\citet{brooks2011covenants}, for instance, provides one of the few quantitative analyses, but had to rely on covenant data at the neighborhood level.} 


\begin{figure}[t]
    \centering
    \includegraphics[width=\linewidth]{images/RRCBriefTimeline-v4.png}
    \caption{Brief overview of legal developments that impacted California's housing market in the 20th century. The Rumford Act was overturned by Proposition 14, which was in turn found unconstitutional by the California Supreme Court in \emph{Mulkey v.\ Reitman}, 64 Cal. 2d 529 (1966).}
    \label{fig:overall}
\end{figure}


\subsection{California Legislation}

In recent years, there has been growing desire to address the ongoing presence of racial covenants in property records. Numerous states, including California, Washington, Minnesota, and Texas, have passed laws allowing property owners to remove racial covenants from their deeds.  However, these laws typically place the responsibility on individual homeowners, resulting in a piecemeal approach. Between 1999-2021, California, for instance,  maintained a process by which homeowners could petition with the County Recorder to modify a racial covenant on their property. \citet{brooksrose2013saving} note that a homeowner would have to be quite ``dedicated'' to pursue that process, as the disclosure of the process is given to purchasers along with all other home disclosures.  As in other jurisdictions that put the onus on individual homeowners \citep{yalereport}, Santa Clara County had only received a handful of such requests prior to 2021.\footnote{Cal. Gov. Code \S\S~12956.1-12956.2.} One state legislator opined, ``the present system
is underutilized and public awareness on the issue is low.'' \citep{howard2021california} 

In recognition of this limitation, California adopted a more proactive and  comprehensive approach with the passage of Assembly Bill 1466 \citep{california_ab1466}. Instead of relying on homeowner initiative, AB 1466 mandates that all 58 counties in California establish programs to identify and redact racial covenants from property records. In addition, AB 1466 mandates the retention of each ``nonredacted record for future reference and public request needs.''\footnote{Cal.\ Gov.\ Code \S 12956.3(c).} In short, counties must redact unlawful discriminatory language in active property records, while retaining historical deed records.\footnote{Though, there is some leniency for false positives or false negatives: ``The failure of a county recorder to identify or redact illegal restrictive covenants, as required by this section, or the county recorder’s identification or redaction of any restrictive covenants that are later determined not to be illegal, shall not result in any liability against the county recorder or the county.'' Cal.\ Gov.\ Code \S 12956.3(g).}

The implementation has posed serious logistical challenges. A prior proposal faced resistance by county recorders for ``creat[ing] an enormous
workload'' and posing a ``potential near shut-down of county recorder offices.''\footnote{Prior proposals (AB 2204 and AB 985) would have placed the responsibility on title companies \citep{howard2021california}.} 
First, because these records span all county properties and all historical transactions, the sheer record volume is large. In counties such as Santa Clara -- with some 24 million of property records -- purely manual review is impractical. Second, the language used to identify racial groups and prohibitions can vary substantially over time and place.  Third, AB 1466 requires review by county counsel to formally record any amendment, making the process organizationally challenging to navigate. Fourth, while AB 1466 included a fee provision that allocated \$2 of recording fees per specified document for funding the implementation of AB 1466, such fees may not cover the full costs.\footnote{This fee provision would lapse by December 31, 2027.} In Santa Clara County, this would fund up to three positions, excluding costs for time for review by the county counsel, digitization of records, and any software to aid in the process. 
Some counties have turned to third-party private vendors to expedite the process. Los Angeles, for instance, hired a private firm in an \$8 million contract to carry out the process over seven years.\footnote{Jaclyn Cosgrove, Racist History Lives on in Millions of Housing records. L.A. County is about to fix that, \emph{L.A.\ Times}, Feb.\ 6, 2024,  \url{https://www.latimes.com/california/story/2024-02-06/l-a-county-will-remove-racist-restrictive-covenant-language-from-millions-of-documents}.} But resources across counties can vary widely. In late 2022, Santa Clara piloted a manual review process with  two staff members manually reviewing 89,000 pages of deeds, finding roughly 400 racial covenants. At an average of a minute per page, performing a manual review of the entire collection of 84 million pages of records would require approximately 1.4 million staff-hours, amounting to nearly 160 years of continuous work for a single individual.\footnote{Please refer to Section~\ref{sec:resource-cost-comparison} in the Appendix for additional details about our assumptions and calculations.} As \citet{howard2021california} notes, the big open question is whether ``fifty-eight different county recorders
in California are able to develop and maintain redaction
procedures that are consistent, predictable, effective,
efficient, and easily implemented.''

\subsection{Existing Efforts}

Outside of California, there are also increasing efforts to identify and address racial covenants. 

First, efforts at the national level have grappled  with the resource-intensive nature to identify racial covenants. The Uniform Law Commission proposed model legislation vesting principal responsibility in individual homeowners, mimicking California's pre-2021 process.\footnote{\url{https://www.uniformlaws.org/committees/community-home?CommunityKey=b1ed931f-d4c2-4078-867d-018a850ef303}} On the other hand, the federally proposed Mapping Housing Discrimination Act would provide grants to educational institutions to ``analyze, digitize, and map historic housing discrimination'' with a goal of a national database of racial covenants.\footnote{\url{https://www.congress.gov/bill/117th-congress/senate-bill/2549}.} The question of who bears responsibility hinges critically on understanding scalable approaches to sifting through deed records, as well as ongoing California efforts. 

Second, academic and grassroot initiatives have crowd-sourced efforts through large numbers of volunteers \citep{bakelmun2019open}. 
The University of Minnesota's \href{https://mappingprejudice.umn.edu}{\textit{Mapping Prejudice}} project, for instance, was one of the earliest  initiatives and relied on thousands of community volunteers to manually sift through digitized property deeds in Hennepin County, Minnesota, home to the city of Minneapolis. The \href{https://www.chicagocovenants.com/}{\textit{Chicago Covenants Project}} took a metropolitan-wide approach to documenting a range of historical housing practices, including racial covenants.  The \href{https://www.justiceindeedmi.org/}{\textit{Justice InDeed}} project similarly focused on identifying racial covenants and collaborating with community stakeholders to raise awareness and pursue local solutions in Washtenaw County, Michigan. 
These participatory approaches are laudable for improving community understanding of the local history of housing discrimination. The reliance on volunteers to sift through an immense volume of records, however, can make this approach infeasible for all jurisdictions.\footnote{A network of these initiatives was formed as the ``\href{https://www.nationalcovenantsresearchcoalition.com/}{National Covenants Research Coalition}.'' 
} 

Third, other projects have been state-initiated. Over a dozen states have enacted legislation to address racial covenants \citep{yalereport}. In 2022, Washington state, for instance, mandated a review of property records across the state to identify racial covenants.\footnote{Concerning Review and Property Owner Notification of Recorded Documents with Unlawful Racial Restrictions, SHB 1335 (2021), codified at Wash.\ Rev.\ Code Ann.\ \S~49.60.525 (2021).} This led to the establishment of the \href{https://inside.ewu.edu/racial-covenants-project/}{\textit{Eastern Washington Racial Covenants Research Project}}, supported by universities and state agencies. While more institutionalized, the manual nature of the review process still remains resource-intensive.\footnote{See Samantha Wohlfeil, ``As EWU readies to share maps of racial covenants in Eastern Washington, a Spokane title company is helping homeowners disavow the racist documents,'' \emph{Inlander}, April 25, 2024 (``[P]rofessors and student employees traveled to auditors' offices to dig through deed books by hand.'').}

Each of these efforts represents an important initiative in understanding the practice and history of housing discrimination. These initiatives have mobilized public interest, enlisted volunteers, and raised grassroots awareness of racial covenants. Yet the core shared approach of relying on purely human review can slow the rate at which jurisdictions can learn about these records and their impact on local histories and be infeasible for many other jurisdictions. Numerous California counties, for instance, have aimed to complete a scan of racial covenants by 2027 and simply do not have scores of volunteers or staff to peruse records. The legal obligation is placed in the offices of the county recorder and counsel, requiring a comprehensive, systematic, and scalable solution to prioritize limited human resources. As far as we are aware, no prior efforts have explored the power of large language models to assist in this process.\footnote{\textit{Mapping Prejudice} has used forms of automation to transcribe and analyze documents, but not modern machine learning in the search process itself.} 
