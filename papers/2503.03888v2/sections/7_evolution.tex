
\section{The Evolution of Racial Covenants in Santa Clara County} \label{sec:evolution}


Our analysis of Santa Clara County deeds offers a detailed examination of racial covenants over space and time. As we note in \Cref{sec:evaluation}, our final review sample contained a small amount of false positives and we suspect that there may be a small number of additional RRCs of unusual construction below the 75\% confidence threshold. Taking into account the relatively small error rate, we characterize some trends that we identified. 
From 1907 to 1974, we identified  roughly 7,500 deeds that the County recorded containing racial covenants. We first present results on the geographic distribution of these covenants (\S~\ref{subsec:map}). Second, we show how our data enables us to identify the small number of developers disproportionately responsible for racial covenants (\S~\ref{subsec:developers}). Third, we 
provide a characterization of the historical evolution of racial covenants that are distinctly informed by our new Santa Clara County data (\S~\ref{subsec:periods}). Overall, our findings illustrate how machine learning can support the implementation of redactions, while unearthing historical discrimination in a more granular fashion. 

\subsection{Geographic Distribution}
\label{subsec:map}

\begin{figure}
    \centering
    \includegraphics[width=\linewidth]{images/map_fig.pdf}
    \caption{\textbf{Top:} Clusters of racial covenants on a map of modern-day Santa Clara County. Some of the largest and most notable racially restricted developments -- discussed in this section -- are shown in red. \textbf{Bottom left:} Racial covenants in south Palo Alto and Mountain View. \textbf{Bottom right:} Racial covenants in downtown San Jose. Dots represent individual subdivisions and are scaled in proportion to the number of racial covenants within the subdivision.}
    \label{fig:scc_rich_map}
\end{figure}

Our geolocation pipeline, described in \S~\ref{sec:geo}, is able to identify the tract-level locations of 79.0\% of properties with racial covenants. Figure \ref{fig:scc_rich_map} plots the locations of properties subject to racial covenants, with dots proportional to the number of covenants in each tract. Figure \ref{fig:scc_map_faceted} depicts the spread of racial covenants throughout the County over the first half of 20th century.

Prior to 1920, racial covenants were a highly concentrated phenomenon in Santa Clara County, driven by a small number of large developers. In this early period, the overwhelming majority of covenants were attached to properties located in modern day Palo Alto and San Jose. The two largest racially restricted developments in the County at this time were the University Park and Stanford University Villa tracts, both developed with the creation of the Stanford university (lower left panel of Figure~\ref{fig:scc_rich_map}).\footnote{See, e.g., Stanford University: Some of the Building to be Opened for the Next Fall Term, \emph{The Evening News}, Feb.\ 8, 1889, at 3.}

As Figures \ref{fig:scc_map_faceted} and \ref{fig:covenant-distributions} show, racial covenants became dramatically more common across the County during the 1920s and 1930s. During this period, racially restricted properties became more diffuse, and it became increasingly common for individual sellers to insert discriminatory provisions on sale of their property. 1925 also saw the first use of a racial covenant at San Jose's Oak Hill Cemetery -- then publicly owned by the City of San Jose~\citep{rhoads_cemetery} -- which would go on to sell at least 50 burial plots for the sole use of ``the human dead of the Caucasian race'' (lower right panel of Figure~\ref{fig:scc_rich_map}).\footnote{In 1933, the Oak Hill Cemetery was sold to a private party, though it continued to sell racially restricted burial plots for more than a decade thereafter. The cemetery, now known as the Oak Hill Memorial Park, continues to operate to this day~\citep{rhoads_cemetery}.} The fact that the city of San Jose owned the cemetery, with racial covenants on burial deeds, complicates conventional historical accounts of racial covenants as marking the shift from discrimination by state action to by private action.\footnote{Since the late 1800s, the city had chartered and contracted with the Oak Hill Improvement Company and the Oak Hill Cemetery Association to operate the cemetery. Burial deeds were hence sold from the Association, while the city retained control of the land until the 1933 sale. See Cemetery Affairs: City May Sue Oak Hill Improvement, \emph{The Evening News},  Jan.\ 22, 1900, at 1. This anticipates questions of when a private business is sufficiently connected to government to be deemed a state actor. See, e.g., Burton v.\ Wilmington Parking Authority, 365 U.S.\ 715 (1961). Notably, in 1900, several Chinese associations purchased land adjacent to Oak Hill, where Chinese were not allowed to be buried, to create a Chinese Cemetery. See \url{https://thebpog.org/chinese-american-cemetery/}.} 

Notably, a rural community in the Santa Cruz Mountains known as Redwood Estates accounts for 796 covenants, more than 10\% of all found in the County. With a 1940 population of less than 4,200 individuals, it is likely that the entire town was covered by racial restrictions.\footnote{According to the 1940 decennial Census, the population of ``Redwood Township'', the enumeration district which included Redwood Estates, was 7,822. Excluding the town of Los Gatos, the remainder of the township's population was 4,225. While a population estimate for Redwood Estates alone is not available, several other rural communities existed within the township (including 96 racial covenants filed outside Redwood Estates), implying an upper bound of less than 4,200 people \citep{us_census_bureau_1940_nodate}.}

After 1940, we find that use of racial covenants steadily declined, first with the onset of World War II (potentially due to a  decline in new construction) and later with \emph{Shelley}. However, racial covenants did continue to proliferate in smaller numbers throughout the County, especially in areas of new development such as the city of Mountain View.

\subsection{The Role of Developers}
\label{subsec:developers}

Our analysis also shows that a small number of developers appear responsible for the vast majority of racial covenants. We calculate that merely ten developers appear responsible for roughly a third of covenants.  Part of the explanation here lies in the growth of the Santa Clara County population during this period, nearly tripling in population from 1920--1950. Developers played a prominent role in the construction of Santa Clara County (and, in turn, Silicon Valley), converting agricultural land into residential neighborhoods. 

Thomas A. Herschbach, for instance, was a prominent developer and builder of numerous subdivisions around San Jose, described laudably in historical volumes \citep{halberstadt}. When the Stone Church of Willow Glen Presbyterian ran out of funds, Herschbach, the builder, donated the roof (id.).  Yet not told in these volumes is the fact that Herschbach was single-handedly responsible for 161 deed records with racial covenants. In the new Palm Haven subdivision that he developed in the years before \emph{Buchanan},  advertisements bearing his name described the development as ``the most beautiful homeplace in San Jose,'' with a picture of six evidently white children playing in a sandbox, and touting a price lower than ``other restricted districts'' (Figure~\ref{fig:palmhaven}).\footnote{\emph{Sunday Mercury and Herald}, May 4, 1913, at 8.}  In the years of development, \emph{Buchanan} would strike down such restrictive racial zoning, and hence Herschbach developments turned to racial covenants.  Scores of local newspaper advertisements continue to advertise the ``restricted'' nature of properties in Santa Clara County post-\emph{Buchanan}, marking the linguistic shift from government-sanctioned to privately-enabled ``restrictions.'' As one property advertisement in Willow Glen, a neighborhood Herschbach developed, put it in 1946: ``Willow Glen's finest and most attractive subdivision \dots High building and \emph{racial restrictions}.''\footnote{\emph{San Jose Mercury Herald}, June 8, 1946, at 11 (emphasis added).}

Herschbach was far from alone, and we found that several other prominent members of California society were prolific spreaders of racial covenants. For example, Virginia M.\ Spinks, who served as an elector for Woodrow Wilson in the 1916 presidential election\footnote{Complete Totals General Election, \emph{San Jose Mercury Herald}, Nov.\ 21, 1916.} and later in his administration\footnote{Executive Order 2745, Authorizing Appointment of Virginia M.\ Spinks to Position in Department of Labor Without Regard to Civil Service Rules, Nov.\ 1, 1917.}, sold at least 87 properties with racial restrictions attached. The disproportionate role of a small number of sellers also challenges the notion that racial covenants were primarily a signaling device of more ``loosely knit'' communities, at least in California~\citep{brooksrose2013saving}.\footnote{Our evidence is not necessarily inconsistent, but the setting in Santa Clara County raises the question of how to define whether a community is ``loose-knit.'' At the relevant historical period, the County is (a) relatively homogeneous (with very few non-white residents), and (b) covenants are imposed by a small number of developers for new communities. The relevant conception may be about the homogeneity of potential purchasers, which is difficult to measure.}

Our findings also point to the critical role of \emph{agency} exercised by and responsibility of individual developers, a small number of whom developed large subdivisions that converted agricultural to residential land. 
Historical accounts on this diverge, with conventional narratives noting that ``crusaders'' for racial equality would lose business \citep{taeuber1961privately}. Others, however, have pointed to the role of Joseph Eichler, who developed many Palo Alto divisions and refused to adopt racial covenants. Eichler aimed to quietly demonstrate that integration could be good business, and came to exert influence on fair housing policy \citep{howell}.  Our evidence suggests that in an emerging housing market, where a small number of developers, such as Eichler and Herschbach, were market movers, and where racial covenants were still in the minority of overall deed records post-\emph{Buchanan}, agency may have been possible \citep{redford}.

\begin{figure}[pt]
    \centering
    \includegraphics[width=3.75in]{images/PalmHaven.jpg}
    \caption[]{1913 housing advertisement for the Palm Haven neighborhood in San Jose. 1913 is several years before \emph{Buchanan} found ``restricted districts'' based on race unconstitutional, and the advertisement emphasizes the ``restricted district[].'' Palm Haven construction dates straddled \emph{Buchanan}. It was developed by Thomas Herschbach who came to be responsible for 161 racial covenants in the County.}
    \label{fig:palmhaven}
\end{figure}

\subsection{Historical Evolution}
\label{subsec:periods}


\begin{figure}[ht]
    \centering
    \includegraphics[width=\linewidth]{images/tractmap_faceted.pdf}
    \caption[]{Density of properties with racial covenants in modern-day Census tracts in 1905, 1920, 1935, and 1950. Racial covenants are plotted cumulatively. Racial covenants are initially concentrated in modern-day Palo Alto and San Jose, but spread throughout the county between 1920 and 1950. Some tracts in the south and east of the County are omitted here for space reasons; a plot of the full County can be found in Appendix~\ref{appendix:full_faceted_map}.\vspace{0.5em}}
    \label{fig:scc_map_faceted}
\end{figure}


\paragraph{1. Emergence (1907--1916).}

The earliest instances of racial covenants date to 1907, with explicit language excluding multiple ethnic groups: ``the party of the second part also agrees for himself, his heirs, executors, administrators, or assigns that he will not sell; nor lease nor permit the premises to be occupied by Italians, Portuguese, colored people or Spaniards.'' Covenants codified racial hierarchies into private law. 

This early period coincided with the rise of nativist movements across the United States, fueled by fears that non-white and non-Anglo-Saxon groups would ``dilute'' white communities. In Santa Clara County, these anxieties were reflected in demographic shifts, as seen in Table~\ref{table:scc_population}, where the Black population declined from 989 in 1890 to 262 in 1910, and the Chinese population fell from 2,723 to 1,064. At the same time, the Japanese population saw a sharp increase, rising from just 27 in 1890 to 2,299 by 1910. These population dynamics likely amplified the perceived need among white property owners to impose restrictions on land ownership and occupancy.

The language of these early covenants reveals the racial hierarchies prevalent at the time, with African Americans, Asians, Latinos, and Mediterranean immigrants often singled out as ``undesirable.'' These groups were considered culturally incompatible with the aspirations of white, middle-class neighborhoods, which sought to maintain racial and cultural homogeneity. 

\begin{table}[!t]
\centering
{\renewcommand{\arraystretch}{1.1}
\scalebox{0.95}{
\begin{tabular}{l  r r r r r r r}
\toprule
 & \textbf{Total} & & & \textbf{Native} \\
\textbf{Year}  & \textbf{Population} & \textbf{White}     & \textbf{Black}    & \textbf{American}  & \textbf{Chinese}  & \textbf{Japanese} & \textbf{Others} \\
\midrule
1890  & 48,005  & 44,247  & 989    & 19     & 2,723   & 27      & 0      \\
1900  & 60,216  & 57,934  & 251    & 9      & 1,738   & 284     & 0      \\
1910  & 83,539  & 79,849  & 262    & 16     & 1,064   & 2,299   & 49     \\
1920  & 100,676 & 96,471  & 335    & 4      & 839     & 2,981   & 46     \\
1930  & 145,118 & 138,589 & 536    & 45     & 761     & 4,320   & 867    \\
1940  & 174,949 & 168,921 & 730    & 74     & 555     & 4,049   & 620    \\
1950  & 290,547 & 280,429 & 1,718  & 144    & 685     & 5,986   & 1,585  \\
1960  & 642,315 & 621,625 & 4,187  & 705    & 2,394   & 10,432  & 2,972  \\
1970  & 1,064,714 & 1,003,898 & 18,090 & 4,048  & 7,817   & 16,644  & 14,217 \\
1980  & 1,295,071 & 1,030,659 & 42,835 & 10,011 & 22,745  & 22,262  & 166,559 \\
\bottomrule
\end{tabular}
}
}
\vspace{0.5em}
\caption{Racial demographics of Santa Clara County from 1890 to 1980, based on U.S. Decennial Census data. The table shows population of various racial and ethnic groups over the period, reflecting significant changes in the county's racial composition as the population grew from approximately 48,000 in 1890 to nearly 1.3 million by 1980. The white population dominated the total population, increasing from 44,247 in 1890 to over 1 million in 1980, though its share of the total population declined as the County became more diverse. The African American population saw substantial growth, particularly after 1950, expanding from 989 in 1890 to over 42,000 by 1980. The Native American -- originally referred to as Indian in the Census Data -- population remained small but increased from just 19 individuals in 1890 to over 10,000 in 1980. Significant growth is observed in Asian populations, particularly among the Chinese and Japanese communities. By 1980, the Chinese population had grown to 22,745, while the Japanese reached 22,262. The ``Others'' category, which includes individuals not listed in the defined racial groups, shows considerable growth. The spike in 1980 likely reflects changes in the Census treatment of race as distinct from ethnicity (Hispanic), following a 1977 OMB directive.}
\label{table:scc_population}
\end{table}

\paragraph{2. Post-\emph{Buchanan} Growth (1917--1926).} The use of racial covenants expanded significantly following \emph{Buchanan}. The case shifted state racial zoning into private restrictive covenants -- seen as out of reach of the Fourteenth Amendment for lack of state action -- as an alternative means to maintain racial boundaries in housing. 

In Santa Clara County, the frequency of racial covenants surged during this period, with annual occurrences rising from 62 in 1917 to over 400 by 1926. This six-fold increase reflects the broader national trend of private actors taking on the role of enforcers of segregation. The language of the covenants also became more specific, particularly targeting African Americans, Japanese, Chinese, and other non-white groups. This shift aligned with the growing anti-immigrant sentiment that culminated in the passage of the Immigration Act of 1924, which severely restricted immigration from Asia.

This  period also coincided with the first Great Migration, leading some to describe the use of racial covenants as ``Jim Crow of the North.''\footnote{PBS, Minnesota Experience: Jim Crow of the North, \url{https://www.pbs.org/video/jim-crow-of-the-north-stijws/}}  So too, we find, in the West. Covenants from this era often included broad racial terms such as ``Mongolian'' and ``Negro,'' as well as references to specific ethnic groups like the Japanese and Chinese, who were viewed as economic competitors in industries like agriculture.

The rise in racial covenants also reflect a growing formalization of discriminatory practices within the real estate industry. Real estate boards, developers, and homeowners increasingly viewed covenants  as essential tools for protecting property values and maintaining racial homogeneity. This was particularly true in suburban developments, where new housing tracts were often marketed as ``restricted'' communities, promising potential white buyers that racial minorities such as African and Asian Americans would be barred from purchasing homes.  

\paragraph{3. Peak Period  (1927--1938).} The 1927--1938 period represents the height of racial covenant usage in the County, spurred by \emph{Corrigan v.\ Buckley}, which legitimized the use of racial covenants as private contractual agreements lacking state action required for constitutional coverage. White property developers, homeowners, and real estate boards escalated their reliance on these covenants as a primary tool for maintaining racially homogeneous neighborhoods. In 1928, Santa Clara County recorded over 600 covenants.  

The language in these covenants became more targeted and explicit. Deed records reveal widespread exclusion of specific ethnic groups, including African Americans, Chinese, Japanese, and other non-Caucasian communities. Terms such as ``Negro,'' ``Mongolian,'' and ``colored'' were commonly employed to delineate the racial boundaries of acceptable property owners and tenants. 

The rise of racial covenants during the late 1920s and early 1930s must also be viewed within the broader context of the post-World War I social climate and the Great Migration. The arrival of African Americans and other minority groups in northern and western cities created heightened racial anxieties among white homeowners, who sought to safeguard their communities through these legally sanctioned racial barriers. What is striking is that the rate at which racial covenants explicitly excluded Black purchasers was at the same  rate as that of Asian exclusion, despite much lower presence of Black residents in the county and the Chinese exclusionary period. This pattern is particularly striking, and corroborates other historical accounts that note that integrating a tract with Asian Americans was a ``minor issue,'' but that selling to African Americans was ``much more controversial and potentially damaging'' \citep{howell}.  As Section~\ref{subsec:developers} illustrated, restrictions via racial covenants were also increasingly marketed as a method of preserving property values, with the implicit understanding that racial segregation would prevent economic decline in white neighborhoods. This perceived linkage between racial homogeneity and property value preservation was deeply embedded in the logic of racial covenants during this era.

The Federal Housing Administration (FHA), known more broadly for promoting redlining, actively promoted the use of racial covenants as a condition for insuring mortgages. The FHA’s underwriting policies explicitly tied the stability of neighborhoods to racial homogeneity, further embedding the use of covenants in the development of new housing projects.

The FHA’s endorsement of racial covenants gave them an air of official legitimacy, encouraging real estate developers to incorporate these restrictions into the blueprints of suburban developments across the country, including in Santa Clara County. During this period, racial covenants not only persisted in existing neighborhoods but also proliferated in new developments as the housing market began to recover from the Depression. The covenants from this era reflect a deepening of racial and socioeconomic exclusions, as terms targeting minority groups became more codified and entrenched within the fabric of real estate transactions.



\paragraph{4. World War II Era Fluctuations (1939--1947).}

World War II brought about significant changes in racial dynamics. The wartime demand for labor, combined with executive orders promoting fair employment practices, allowed African Americans to gain greater access to jobs in defense industries and other sectors.  At the same time, the internment of Japanese Americans forcibly uprooted Japanese American communities across the West Coast. 

The post-war period saw a sharp resurgence of racial covenants. The return of soldiers and the end of wartime economic controls created a surge in demand for housing, and white homeowners and developers once again turned to racial covenants as a means of protecting their neighborhoods from racial integration.


\begin{figure}
    \centering
    \includegraphics[width=\linewidth]{images/race_analysis.pdf}
    \includegraphics[width=\linewidth]{images/excluded_groups_over_time_bar.pdf}
    \caption[]{\textbf{Top:} Number of property deeds with restrictive covenants from 1905--1974, divided by whether specific racial groups were excluded or only white/Caucasian individuals were permitted. Most pre-1915 covenants specifically exclude Black and Asian individuals, but the vast majority of later covenants are white-only. The small number of restrictive covenants matched after 1970 consists largely of older deeds filed for reference, rather than new restrictive covenants being introduced. \textbf{Bottom:} The number of occurrences of specific racial groups in covenants that exclude specific groups. East Asian and Black were by far the most commonly excluded demographics, but some covenants targeted other groups, such as Italian, Portuguese, Indian, and Mexican individuals.}
    \label{fig:covenant-distributions}
\end{figure}

\paragraph{5. Post-Shelley Decline (1948--1967).} \emph{Shelley} was a pivotal moment for racial covenants, as is reflected in Santa Clara County records. Racial covenants significantly and nearly immediately drop, with a near 75\% decrease in prevalence. %
While \emph{Shelley} made racial covenants unenforceable, \citet{brooksrose2013saving} argue that such covenants continued to play an important signaling role. Consistent with their account, homeowners and developers in Santa Clara County continued to include such covenants through the 1950s and 1960s. 

The Fair Housing Act (FHA) of 1968 prohibited racial discrimination in the sale, rental, and financing of housing, making racial covenants illegal, not just unenforceable. 








