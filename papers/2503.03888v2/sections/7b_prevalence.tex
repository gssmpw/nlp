\section{Prevalence}
\label{sec:prevalence}

So far, we have focused on deeds as the unit of analysis. A single deed record, however, may apply to multiple units, including tracts and neighborhoods. Focusing on the deed record makes sense for implementation under AB 1466, as only a single deed record needs to go through the redaction process. But the raw count of 7,500 deeds may significantly underestimate the affected number of properties. The single covenant document for Palo Alto's Southgate neighborhood, for instance, covers the entire subdivision of 196 homes depicted in Figure~\ref{fig:southgate}.

\begin{figure}[pt]
    \centering
    \includegraphics[width=4.25in]{images/Southgate.png}
    \caption[]{Assessor's map for Southgate neighborhood in Palo Alto from 1923. The entire neighborhood of then-196 homes is covered by one racial covenant from the Palo Alto Development Company, which would go on to sell individual properties.}
    \label{fig:southgate}
\end{figure}

To estimate the total number of properties covered by racial covenants, we design the following workflow. First, with a combination of keyword heuristics and few-shot LLM classification, we identify pre-1950 deed covenants which appear to cover entire tracts / neighborhoods.\footnote{To provide some intuition on this detection, neighborhood covenants are often referred to as ``Declarations,'' as opposed to ``Deed Records.''}
We find 412 such neighborhood covenants, each of which reflects a large tract being subdivided for sale to individuals.\footnote{Because of their disproportionate impact on our estimates, we manually confirmed that all of these matches contained racial covenants and applied to an entire tract.} Second, we employ the same map matching strategy described in \S~\ref{sec:geo} to identify surveyor's maps representing those tracts. Consistent with our results above, we are able to automatically match and geolocate 79\% of neighborhood covenants to specific maps. Third, for the 86 unmatched deeds, we review the deed record to identify the corresponding surveyor's map (and associated geographic location). Fourth, we inspect the original scans of each map to count the number of lots, reflecting individual housing units, in the tract. This yields a total of 18,871 lots covered by only 412 neighborhood-wide covenants.

Next, we use a language model (\texttt{gpt-4o-mini}) to identify pre-1950 deeds that apply to more than one lot, but do not cover an entire tract.\footnote{The model extracted out a list of lots mentioned for each deed. We verified the accuracy of this labeling with a random sample of twenty units.} This adds another 5,354 individual lots, associated with 1,293 deeds. An additional 5,612 covenants apply to only a single lot.

Last, we deduplicate our counts on a property level to isolate the number of affected properties. Multiple deed records for the same property may reflect sales of the same property, but our aim is to convert deeds into properties affected, regardless of the number of sale events during our observation period. To ensure that the same property is not double-counted, we deduplicate on the combination of (tract, block number, lot number) and filter individual covenants filed within tracts already covered by a community-wide covenant. This process removes 5,315 lots from our count.

Overall, we then have a count of 24,522 lots that were subject to racial covenants. 

To understand the relative magnitude of these restrictions, we use the Decennial Census, which report a total number of 56,406 dwelling units in 1940 and 92,315 in 1950.\footnote{\url{https://www2.census.gov/library/publications/decennial/1950/hc-1/hc-1-48.pdf}} This provides us with the following estimate: in 1950, \emph{one in four} properties were covered by a racial covenant. In 1940, nearly 30\% of all properties were covered by a racial covenant.\footnote{This is based on a count of 16,553 pre-1940 deeds.} This confirms that the identification of 7,500 deeds vastly understates the true impact of racial covenants in Santa Clara County. Because the county tripled in population from 1920 to 1950 (see Table~\ref{table:scc_population}) and the housing stock doubled from 1940 to 1950 -- at precisely the peak usage of racial covenants -- vast portions of the county were covered by racial covenants.  

We note that this estimate is based on a set of assumptions. It provides the first population-level estimate of what fraction of the housing stock was encumbered by racial covenants, and it is subject to uncertainty in resolving deeds into properties.  Factors that may bias this estimate downward are: (a) failure to detect neighborhood-wide covenant language; and (b) subdivision of lots into multiple dwelling units (e.g., into apartments). Factors that may bias this estimate upward are: (a) improper deduplication, (b) failure to develop a subdivided lot, and (c) the fact that a small percentage of covenants were written to expire after some number of decades.  Other factors that have unclear effects on the magnitude are potential inaccuracies in our map and deed matching process. Our best assessment is that the factors leading to undercounting predominate, making the estimate that one in four properties is affected closer to a lower bound. 

The bottom line, however, is simple: racial covenants were pervasively used across Santa Clara County.\footnote{Appendix~\ref{appendix:lot-coverage} provides several visualizations of the lot-level coverage of racial covenants in the County.}