

\section{Introduction}

Legal reform is complex. When a court declares a statutory provision unconstitutional, a legislature prohibits a certain practice, an agency initiates regulatory review, or the public adopts a referendum, such changes can ripple through thousands of code provisions, a thicket of regulations, and millions of administrative records. 
Armies of lawyers and clerks can spend thousands of hours to identify legal dependencies to implement such changes. Because this process is so resource-intensive, many outdated legal provisions can persist in official documents for decades.

One prominent example of this issue is the persistence of racially restrictive covenants in real property deeds. Examined extensively by lawyers and social scientists \citep{brooks2011covenants, brooksrose2013saving,gonda2015unjust,gotham2000urban,jones2000origins,ming1949racial,roisman2022stumbling,rothstein2017color,rose2016racial, rose2022property,rose2023general, vose1967caucasians,yalereport}, racial covenants are discriminatory clauses that prohibited the purchase, lease, or occupation of land based on race.\footnote{Discriminatory restrictive covenants may also apply to other attributes, such as religion, family status, and national origin, but we focus on the principal case of racial covenants here.} Although declared \emph{unenforceable} by the United States Supreme Court in \emph{Shelley v.\ Kraemer} (1948)\footnote{334 U.S.\ 1.} and \emph{illegal} under the Fair Housing Act (1968),\footnote{42 U.S.C.\ \S~3600, et seq.} such covenants continue to exist in the pages of real property records across the United States. The sheer volume makes identifying and redacting racial covenants both monumental and resource-intensive \citep{howard2021california}. Over a dozen  jurisdictions have enacted legislation to address racial covenants \citep{yalereport}. The typical approach has been to 
enable individual homeowners to petition for legal review to redact these records to limited effect.\footnote{\citet[][noting from interviews with county clerks that ``only a few people have used these provisions'']{yalereport}.}

But change is afoot. In 2021, California enacted Assembly Bill 1466 \citep[AB 1466;][]{california_ab1466},\footnote{AB 1466 is codified at Cal.\ Gov.\ Code \S~12956.3.} which mandates that all 58  counties develop programs to affirmatively identify and redact racial covenants from property records.\footnote{Under California's Fair Employment and Housing Act, unlawful restrictions are also ones based on age, race, color, religion, sex, gender, gender identity, gender expression, sexual orientation, familial status, marital status, disability, veteran or military status, national origin, ancestry, genetic information, or source of income. Cal.\ Gov. Code \S~12955.}  While AB 1466 is seen as an important step to recognizing and mitigating the remnants of institutionalized housing discrimination, its implementation presents significant challenges. Santa Clara County, for example, has more than 24 million property records, spanning over 84 million pages, including some that date back to the 1850s.\footnote{Santa Clara County Clerk-Recorder's Office, Restrictive Covenant Modification Program Implementation Plan, 
\url{https://clerkrecorder.sccgov.org/unlawfully-discriminatory-restrictive-covenant-modification-program-assembly-bill-1466}}

The complexity and scale of these historical documents, some of which are handwritten or stored on decades-old microfiche cards, render manual review infeasible. Given available resources for manual review, it could take a single county about 160 years and over \$22 million to complete a scan of all 24 million records.\footnote{At California's 2024 minimum wage of \$16.00 per hour, the projected cost of hiring human reviewers to manually examine the Santa Clara County's entire real property records would be about \$22.4 million.}


In a unique multiyear partnership between Stanford RegLab and the Santa Clara County Clerk-Recorder's Office, along with a collaborator at Princeton University, we prototyped, developed, and operationally integrated a machine learning-based pipeline to identify and map racial covenants at scale. 
Our system is capable of processing millions of documents in a single day. It offers an efficient and reliable solution that drastically reduces the time and labor required for manual review. As shown in Table~\ref{tab:cost_comparison}, our machine-learning pipeline has saved over 86,500 hours of manual human labor, costing less than 0.02\% of a full manual human review and under 2\% of a comparable off-the-shelf-model such as OpenAI's ChatGPT-3.5. Our solution offers an accurate, fast, and cost-effective  path for Santa Clara County -- and other jurisdictions -- to best utilize limited human resources, meet legislative requirements, and preserve important historical records for further study.

This article presents three contributions stemming from this effort. \emph{First}, we present our machine-learning pipeline for identifying and mapping racial covenants. Our pipeline processes images of historical property deeds, converts them into text, and then leverages a state-of-the-art, finetuned language model to accurately detect racial covenants. If unlawful language is found in a deed, the system highlights the content and extracts the property address. Both the highlighted documents and their corresponding address are then sent to Santa Clara County for legal review and final confirmation. Remarkably, our model achieves extremely strong performance, with a precision of 1.0 and a recall of 0.99 on an evaluation suite of real property deeds. 

We show that this AI-based approach offers significant advantages over traditional methods, such as keyword-based searches, which are prone to substantial false-positive rates. Scanning artifacts, such as poor OCR quality in older deeds, and ambiguous terms like ``white'' (which could refer to a person's name or a street) contribute to the inaccuracies of lexical search techniques. In contrast, our system analyzes the full semantic context of each document, enabling it to detect racial covenants that have atypical language structures or obscure phrasing, some of which had previously gone unnoticed by manual reviewers.

\emph{Second}, we discuss how we integrated the model into a responsible operational process that includes thorough legal review and the creation of a historical registry of the removed racial covenants. By retaining a historical record, we ensure that this dark chapter of housing discrimination is not erased from public memory, but preserved and understood. We also make our models, results, and web interface for reviewing records available to assist the hundreds of jurisdictions engaged in similar efforts to identify these unenforceable legal provisions.\footnote{These will be made available at \url{https://reglab.github.io/racialcovenants/}.} 

\begin{figure}[t]
    \centering
    \includegraphics[width=\linewidth]{images/Example_v2.pdf}
    \caption{Although racially restrictive covenants are no longer legally enforceable and are considered illegal under the Fair Housing Act today, they still exist in thousands, possibly even millions, of historical property records in California. One such example, found in a 1940 real property deed from Santa Clara County's archives, contains the following discriminatory language: ``{No persons not of the Caucasian Race shall be allowed to occupy, except as servants of residents, said real property or any part thereof.}'' The deed further specifies that ``[t]hese covenants are to run with the land and shall be binding on all parties,'' thereby affecting not only the tenants at the time but also the potential future owners of the land.}
    \label{fig:example}
\end{figure}

\emph{Third}, our findings shed light on the history of racial exclusion in the California housing market. The racial covenants identified by our system reveal distinct patterns of racial categorization, usage across time, and geographical clustering, adding to an important body of scholarship on housing discrimination and racial covenants. Our large-scale dataset enables researchers to understand and test for different accounts of racial covenants. Consistent with existing accounts \citep{rose2016racial}, early racial covenants in California specifically focused on Asian groups, but the number of covenants expressly barring black homeowners was at the same rate in the early 20th century, even when Asian residents far outnumbered black residents. We observe a drop in racial covenants after \emph{Shelley}, but consistent with \citet{brooksrose2013saving}, racial covenants persist well after 1948. We also find that a state actor (the city of San Jose) owned land subject to a substantial number of burial deeds with racial covenants (e.g., burial plots exclusively for ``Caucasian race''), complicating conventional accounts of covenants as a private substitute for public state action (racial zoning) found unconstitutional in 1917. Just ten developers appear responsible for nearly a third of racial covenants in the County, suggesting more \emph{agency} in the construction of what became Silicon Valley \citep{howell,redford, taeuber1961privately}.  We provide an estimate that \emph{one in four} properties the county were covered by a racial covenant in 1950. 

Overall, our project demonstrates the power of machine learning and large language models to play a substantial role in scalable legal reform and the public sector \citep{engstrom_government_2020}. At core, we show how AI can meaningfully assist to unearth, preserve, and shine a light on housing discrimination in a way that was obscured by previously inaccessible deed records.\footnote{As we articulate below, California deed records are not publicly available at scale, despite being public records.} 

\textbf{Organization}. Our paper proceeds as follows. Section~\ref{sec:background} provides background on racial covenants, California's reform efforts, and existing efforts to map and redact deeds. %
Section~\ref{sec:data} discusses the data processing steps to digitize, augment, and label deed records for machine learning. Section~\ref{sec:pipeline} discusses the AI-based detection pipeline, which shows that large language models enable substantial improvements over keyword-based searches, and the geolocation of records. Section~\ref{sec:evaluation} presents results, which show remarkable improvements, such as the reduction in the false positive rate from 28.9\% with keyword searches to 0\% with a fine-tuned open-source language model. Section~\ref{ref:integration} discusses how we integrated the AI system to preserve legal review of each redacted provision, but with dramatically lower search costs. Section~\ref{sec:evolution} shows how this comprehensive effort enables us to unearth rich historical facts about the evolution of racial covenants in Santa Clara County. Section~\ref{sec:prevalence} presents an estimate of the proportion of 1950 housing stock that was covered by racial covenants. Section~\ref{sec:limits} discusses limitations of the approach and Section~\ref{sec:conclude} concludes. 

\begin{table}[!t]
\centering
\begin{tabular}{l l l}
\toprule
\textbf{Method} & \textbf{Time} & \textbf{Monetary Cost} \\
\midrule
\textbf{Manual Review} {\footnotesize(One Staff Member)} & 9.89 years & \$1,400,000   \\
\textbf{Off-the-Shelf LM} {\footnotesize(GPT-3.5)} & 3.63 days & \$13,634   \\
\textbf{Off-the-Shelf LM} {\footnotesize(GPT-4 Turbo)} & 3.63 days & \$47,944   \\
\textbf{Our Custom LM} {\footnotesize(Finetuned Mistral)}  & 6 days & \$258 \\
\bottomrule
\end{tabular}
\vspace{0.2em}
\caption{Resource cost comparison for identifying racially restrictive covenants in Santa Clara County’s 5.2 million pages of property records from 1907 to 1978. Our custom finetuned Mistral model stands out as the most scalable and economical solution, completing the full review in just six days for \$258, a small fraction of the cost and time required for manual review, which would cost over \$1.4 million and take years to finish. For additional details, please refer to Section~\ref{sec:resource-cost-comparison} in the Appendix. %
}
\label{tab:cost_comparison}
\end{table}
