\section{Proofs for matrix trickle-down process}
\label{appendix-mtd}
In this section, we use the language of simplicial complexes as in \Cref{sec:covariance}.
That is, for a weighted edge coloring instance on a tree $T$ with $\beta$-extra
color losts $\+L$ and distribution $\mu$, we consider the colorings on a broom $K$ 
as a weighted simplicial complex $(\+C, \pi_{|K|})$ such that $\pi_{|K|} = \mu_K$.

Throughout this section, we consider $K$ as an edge set.
For a pinning $\tau\in \+C_{i}$ with $0\le i\le|K|-2$, we define
$K_\tau = \set{e\in K\mid e\notin \tau}$ and $\!{col}(\tau)=\set{c \mid \exists e\in K, \tau(e)=c}$. Let $\!{Id}_\tau \in \bb{R}^{\+C_1\times \+C_1}$ be the identity matrix restricted on $\+C_{\tau,1}$. Define 
$\!{Adj}_\tau \in \bb{R}^{\+C_1\times \+C_1}$ such that $\!{Adj}_\tau(ec,fc)=1$ if $ec,fc\in \+C_{\tau,1}$ and all other entries are $0$.
\subsection{Matrix trickle-down theorem}
The following proposition is the main tool we use in this section.
\begin{proposition}[Theorem 1.3 in \cite{ALOG21}]
	\label{thm:mtd-inductive}
	Given a pure $d$-dimensional weighted simplicial complex $(\+C,\pi_d)$, if there exists a family of matrices $\set{M_\tau\in\bb R^{\+C_1\times \+C_1}}$ satisfying
	\begin{itemize}
		\item For every $\tau\in\+C_{d-2}$, 
		\[
			\Pi_\tau P_\tau -2 \pi_\tau \pi_\tau^\top \mle M_\tau \mle \frac{1}{5}\Pi_\tau;
		\]
		\item For every face $\tau\in \+C_{d-k}$ with $k\geq 3$, one of the following two conditions hold:
		\begin{enumerate}
			\item 
			\[
				M_\tau \mle \frac{k-1}{3k-1}\Pi_\tau\quad\mbox{ and }\quad\E[x\sim\pi_\tau]{M_{\tau\cup\set{x}}}\mle M_\tau -\frac{k-1}{k-2}M_\tau\Pi^{-1}_\tau M_\tau
			\]
			\item $(\+C_\tau,\pi_{\tau,k})$ is the product of $M$ pure weighted simplicial complexes $(\+C^{(1)},\pi^{(1)}), \dots (\+C^{(M)},\pi^{(M)})$ of dimension $n_1,\dots,n_M$ respectively and
			\[
				M_\tau = \sum_{i\in [M]\colon n_i\ge 2} \frac{n_i(n_i-1)}{k(k-1)}\cdot M_{\tau\cup \eta_{-i}}
			\]
			where $\eta_{-i} = \eta\setminus \+C^{(i)}_1$ for an arbitrary $\eta\in \+C_{\tau,k}$.
			\end{enumerate}
	\end{itemize}
	Then for every face $\tau\in\+C_{d-k}$ with $k\geq 2$, it holds that
	\[
		\Pi_\tau P_\tau -\frac{k}{k-1}\pi_\tau \pi_\tau^\top \mle M_\tau \mle \frac{k-1}{3k-1}\Pi_\tau.
	\]
	In particular, $\lambda_2(P_\tau)\le \lambda_1(\Pi_\tau^{-1}M_\tau)$.
\end{proposition}

\subsection{Base case}\label{sec:base-case}
\newcommand{\revp}{\tilde{\*p}}
We do matrix trickle down on $\mu_K(\tau)$.
Consider the base case: Assume the free edges in $K$ are $e,f$ and other edges are pinned with assignment $\tau$. 
Let $\*p_{ef}=(p_{e,c},p_{f,c})_{c\in \+L^\tau(e)\cup \+L^\tau(f)}$ and $\*p_{fe}=(p_{f,c},p_{e,c})_{c\in \+L^\tau(e)\cup \+L^\tau(f)}$. We omit the superscript $\tau$ in $q_e^\tau$ and $q_{e,f}^\tau$ in base case part since it is clear in the context.
Moreover, let $\revp_e=(0, p_{e, c})_{c\in \+L^\tau(e)\cup \+L^\tau(f)}, \revp_f = (p_{f, c}, 0)_{c\in \+L^\tau(e)\cup \+L^\tau(f)}$,
and define $\bb 1_e, \bb 1_f\in \bb R^{\+C_\tau, 1}$ such that $\bb 1_e(i, c) = \1{i=e},\bb 1_f(i, c) = \1{i=f}$.
Finally, we define $\*1_{ef}\in \bb R^{\+C_{\tau, 1}\times \+C_{\tau, 1}}$ such that $\*1_{ef}((i, c_1), (j, c_2) = \1{c_1=c_2\wedge i\neq j}$.
\begin{align*}
    &\phantom{{}={}}\Pi_\tau P_\tau- 2\pi_\tau\pi_\tau^\top 
    \\
    &= \frac{1}{2(q_eq_f-q_{e,f})^2}\-{diag}(\*p_{ef})\tp{(q_eq_f-q_{e,f})(\bb{1}_e\bb{1}_f^\top + \bb{1}_f\bb{1}_e^\top-\*1_{ef})-\frac{(q_e\bb{1}_f+q_f\bb{1}_e)(q_e\bb{1}_f+q_f\bb{1}_e)^\top}{2}+\*p_{fe}\*p_{fe}^\top} \-{diag}(\*p_{ef})\\
    &\mle \frac{1}{2(q_eq_f-q_{e,f})^2}\-{diag}(\*p_{ef})\tp{\*p_{fe}\*p_{fe}^\top-(q_eq_f-q_{e,f})\*1_{ef})} \-{diag}(\*p_{ef})\\
    &\mle \frac{1}{2(q_eq_f-q_{e,f})^2}\-{diag}(\*p_{ef})\tp{2(\revp_{e}\revp_{e}^\top+\revp_{f}\revp_{f}^\top)-(q_eq_f-q_{e,f})\*1_{ef})} \-{diag}(\*p_{ef})
\end{align*}
We do a row summation to bound the first term. Firstly,
\begin{align*}
    \revp_{e}\revp_{e}^\top+\revp_{f}\revp_{f}^\top \mle q_e\-{diag}(\revp_e) + q_f\-{diag}(\revp_f).
\end{align*}
Then,
\begin{align*}
    \frac{1}{(q_eq_f-q_{e,f})^2}\-{diag}(\*p_{ef})(\revp_{e}\revp_{e}^\top+\revp_{f}\revp_{f}^\top)\-{diag}(\*p_{ef})
    \mle \frac{1}{(q_eq_f-q_{e,f})^2}\-{diag}(\*p_{ef})(q_e\-{diag}(\revp_e) + q_f\-{diag}(\revp_f))\-{diag}(\*p_{ef}),
\end{align*}
which is a diagonal matrix.
On the entry $(ec, ec)$, it equals
\begin{align*}
    \frac{q_fp_{ec}}{(q_eq_f-q_{e,f})^2}p_{fc}p_{ec}
   &=       \frac{2q_fq_e}{q_fq_e}
      \cdot \frac{p_{ec}}{q_e}
      \cdot \frac{q_fq_e}{q_fq_e-q_{fe}}
      \cdot \frac{(q_f-p_{fc})p_{ec}}{2(q_fq_e-q_{fe})}
      \cdot \frac{p_{fc}}{q_f-p_{fc}}
    \\&\le
            2
      \cdot \frac1\beta
      \cdot \frac{\beta}{\beta-1}
      \cdot \pi_\tau(ec)
      \cdot \frac{1}{\beta-1}
      & \text{\big(\Cref{lem:marginal-bound-weighted-1,lem:marginal-bound-weighted-2}\big)}
    \\&=\frac{2}{(\beta-1)^2}\pi_\tau(ec).
\end{align*}
Applying the same argument to $(fc, fc)$, we have
\begin{align*}
    \Pi_\tau P_\tau- 2\pi_\tau\pi_\tau^\top 
    \mle
    -\frac{1}{2(q_eq_f-q_{e,f})^2}\-{diag}(\*p_{ef})\tp{(q_eq_f-q_{e,f})\*1_{ef})} \-{diag}(\*p_{ef})
    +
    \frac{2}{(\beta-1)^2}\Pi_\tau.
\end{align*}
Let $M_\tau$ be a block diagonal matrix with blocks $M_\tau^c$:
\begin{equation}\label{eqn:base-case-def-ol}
    M_\tau^c=\frac{p_{e,c}p_{f,c}}{2(q_e q_f-q_{e,f})}\tp{\begin{matrix}
    0 & -1\\
    -1 & 0
\end{matrix}}
+
\frac{2}{(\beta-1)^2}\Pi^c_\tau.
\end{equation}
Then we have the base case inequality
\begin{align*}
    \Pi_\tau P_\tau- 2\pi_\tau\pi_\tau^\top \mle \-{diag}(M_\tau^c).
\end{align*}

\subsection{Induction}
The induction step in \Cref{thm:mtd-inductive} is to show that for every $\tau$ with $\!{codim}(\tau) = k>2$ and connected $G_\tau$,
\begin{equation}\label{eqn:induction-main}
	\E[x\sim\pi_\tau]{M_{\tau\cup \set{x}}}\mle M_\tau - \frac{k-1}{k-2}M_\tau \Pi_\tau^{-1} M_\tau.
\end{equation}
For every $\tau$ and $c\in [q]$, we will define a matrix $M_\tau^c \in \bb R^{K^c\times K^c}$ and let $M_\tau$ be the block diagonal matrix with block $M_\tau^c$ for every $c\in [q]$. 
It is not hard to see that we only require 
\begin{equation}\label{eqn:induction-main-c}
	\E[x\sim\pi_\tau]{M_{\tau\cup \set{x}}^c}\mle M_\tau^c - \frac{k-1}{k-2}M_\tau^c (\Pi_\tau^c)^{-1} M_\tau^c
\end{equation}
to hold for every $c$ and $\tau$, where $(\Pi_\tau^c)^{-1}$ is $\Pi_\tau^{-1}$ restricted on $K^c\times K^c$.
We now describe our construction of $M^c_\tau$ for a fixed color $c$.
We write $M_\tau^c$ into the sum of a diagonal matrix and an off-diagonal matrix, i.e.,
\begin{equation}\label{eqn:N-decompose}
	M_\tau^c = \frac{1}{k-1}(A_\tau^c+\Pi_\tau^c B_\tau^c),
\end{equation}
where $A_\tau^c$ is an off-diagonal matrix and $B_\tau^c$ is a diagonal matrix.

From now on, when $c$ is clear from the context, we will omit the superscript $c$ for matrices.
For example, we will write $M_\tau$, $\Pi_\tau$, $\!{Adj}_\tau$, $\!{Id}_\tau$, $A_\tau$, $B_\tau$, $\dots$ instead of
                           $M^c_\tau$, $\Pi^c_\tau$, $\!{Adj}_\tau^{c}$, $\!{Id}_\tau^{c}$, $A_\tau^c$, $B_\tau^c$, $\dots$ respectively.
Plugging the above construction of $M_\tau$ into \eqref{eqn:induction-main-c} and remembering that the superscript $c$ has been omitted, we obtain
 \[
 (k-1)\cdot\E[x\sim\pi_\tau]{A_{\tau\cup\set{x}}+\Pi_{\tau\cup\set{x}}B_{\tau\cup\set{x}}} \mle (k-2)\cdot\tp{A_\tau+\Pi_\tau B_\tau} -\tp{A_\tau+\Pi_\tau B_\tau}\Pi_\tau^{-1}\tp{A_\tau+\Pi_\tau B_\tau}.
 \]
Here we need the following inequality of matrices.
\begin{lemma}[Corollary 12 in \cite{WZZ24}]
	\label{lem:matrix-squared-sum}
Let $A_1,\dots,A_n$ be a collection of symmetric matrices and $\Pi\mge 0$. Then
\[
	\tp{\sum_{i=1}^n A_i}\Pi \tp{\sum_{i=1}^n A_i} \mle n\sum_{i=1}^n A_i\Pi A_i.
\]	
\end{lemma}
 It follows from \Cref{lem:matrix-squared-sum} that
 \[
 \tp{A_\tau+\Pi_\tau B_\tau}\Pi_\tau^{-1}\tp{A_\tau+\Pi_\tau B_\tau}\mle 2A_\tau\Pi_\tau^{-1}A_\tau + 2\Pi_\tau B_\tau^2.
 \]
 As a result, in order for \eqref{eqn:induction-main} to hold, we only need to design $A_\tau$ and $B_\tau$  satisfying
\begin{equation}\label{eqn:condition-main}
 	(k-1)\cdot \E[x\sim\pi_\tau]{A_{\tau\cup\set{x}}} - (k-2) \cdot A_\tau+2A_\tau \Pi_\tau^{-1} A_\tau  \mle (k-2)\Pi_\tau B_\tau -(k-1)\cdot \E[x\sim\pi_\tau]{\Pi_{\tau\cup\set{x}} B_{\tau\cup\set{x}}}-2\Pi_\tau\tp{B_\tau}^2.
\end{equation}

\subsection{Construction of $A_\tau^i$}\label{sss:Atau}
We define
\begin{equation}\label{eqn:A-def}
	A_\tau = a_{k}\cdot (k-1)\cdot \E[\sigma\sim \pi_{\tau, k-2}]{A_{\tau\cup\sigma}}.
\end{equation}
Then we can deduce the following relation between $A_\tau$'s whose co-dimensions differ by one.
\begin{lemma}
	\[
		\E[x\sim \pi_\tau]{A_{\tau\cup\set{x}}} = \frac{k-2}{k-1}\cdot \frac{a_{k-1}}{a_{k}} A_\tau(ec, fc).
	\]
	where $h = h_\tau \geq 1$.
\end{lemma}
\begin{proof}
For any $ec, fc \in K_\tau$,
	\begin{align*}
		\E[x\sim \pi_\tau]{A_{\tau\cup\set{x}}}(ec, fc)
        &= (k-2)\sum_{x\in \+C_{\tau,1}} \frac{1}{k} \mu_{\ol K_x}^\tau(x) a_{k-1} \sum_{\sigma\in \+C_{\tau\cup\set{x}, k-3}} \frac{2}{(k-1)(k-2)}\mu^{\tau\cup\set{x}}_{\ol K_\sigma}(\sigma) A_{\tau\cup\set{x}\cup\sigma}(ec, fc)\\
		&= \frac{2}{k(k-1)} \sum_{x\in \+C_{\tau,1}}\sum_{\sigma\in \+C_{\tau\cup\set{x}, k-3}}\mu^{\tau}_{\ol K_x}(x) \mu^{\tau\cup\set{x}}_{\ol K_\sigma}(\sigma)a_{k-1}A_{\tau\cup\set{x}\cup\sigma}(ec, fc)\\
		&= \frac{2}{k(k-1)} \sum_{\sigma'\in \+C_{\tau,k-2}} \mu^\tau_{\ol K_{\sigma'}}(\sigma') A_{\tau\cup\sigma'}(uc, vc) (k-2)a_{k-1}\\
        &= (k-2)\E[\sigma'\sim \pi_{\tau, k-2}]{A_{\tau\cup\sigma'}(ec, fc)}\\
		&= \frac{k-2}{k-1}\cdot \frac{a_{k-1}}{a_{k}} A_\tau(ec, fc).
	\end{align*}
\end{proof}

It follows from the definition that $A_\tau$ is proportional to the expectation of the base cases when the boundary is drawn from $\pi_{\tau,k-2}$. For some technical reasons, we would like to isolate those boundaries containing the color $c$. This leads us to the following lemma.

\begin{lemma}~\label{lem:A-tau-i}
    \newcommand{\taur}{\tau\cup\omega|_{K_\tau\setminus\set{e, f}}}
	\[
		A_\tau = \frac{2a_k}{k}\sum_{\substack{\omega\in \+C_{\tau,k}\\ c\notin\!{col}(\omega)}}\pi_{\tau,k}(\omega)A^{i,\omega}_\tau,
	\]
	where $A^{i,\omega}_\tau$ is the matrix supported on $K_\tau^c\times K_\tau^c$ such that
	\[
		A_\tau^{\omega}(uc,vc) = -\frac{p^{\taur}_{e, c}p^{\taur}_{f, c}}{\ol q^{\taur}_{e}\ol q^{\taur}_{f} - \ol q^{\taur}_{ef}}.
	\]
	and $\ol q^{\cdot}_{e}\defeq q^\cdot_{e} - p^\cdot_{ec}$,
    $\ol q^{\cdot}_{f}\defeq q^\cdot_{f} - p^\cdot_{fc}$, and
    $\ol q^{\cdot}_{ef}\defeq q^\cdot_{ef} - p^\cdot_{ec}p^\cdot_{fc}$.
\end{lemma}

\begin{proof}
    \newcommand{\taur}{\tau\cup\omega|_{K_\tau\setminus\set{e, f}}}
    In the proof we use $c\notin \cdot$ as a shortcut for $c\notin \!{col}(\cdot)$.
    By the definition of $A_\tau$,
    \begin{align*}
        A_\tau(ec, fc)
        &=  (k-1)a_k\frac{2}{k(k-1)}
              \sum_{\substack{\sigma\in \+C_{\tau, k-2}\\ c\notin\sigma}}
              \mu^\tau_{K_\tau\setminus\set{e, f}}(\sigma)
              \frac{p^{\tau\cup\sigma}_{ec}p^{\tau\cup\sigma}_{fc}}
              {q^{\tau\cup\sigma}_{e}q^{\tau\cup\sigma}_f - q^{\tau\cup\sigma}_{ef}}
      \\&=  \frac{2a_k}{k}
              \sum_{\substack{\sigma\in \+C_{\tau, k-2}\\ c\notin\sigma}}
              \Big(
              \sum_{\xi\in \+C_{\tau\cup\sigma, 2}}
              \mu^{\tau\cup\sigma}_{\set{e, f}}(\xi)
              \Big)
              \mu^\tau_{K_\tau\setminus\set{e, f}}(\sigma)
              \frac{p^{\tau\cup\sigma}_{ec}p^{\tau\cup\sigma}_{fc}}
              {q^{\tau\cup\sigma}_{e}q^{\tau\cup\sigma}_f - q^{\tau\cup\sigma}_{ef}}.
    \end{align*}
    The equality is becase if $c\in \sigma$, the $p^{\tau\cup\sigma}_{\cdot, c}$ terms vanish.
    Notice that for any $\sigma\in \+C_{\tau, k-2}$ and $\xi\in\+C_{\tau\cup\sigma, 2}$,
    \begin{align*}
        1 &= 
        \sum_{\xi\in \+C_{\tau\cup\sigma, 2}}
        \mu^{\tau\cup\sigma}_{\set{e, f}}(\xi)
        \\&=
        \frac{
        \sum_{\xi\in \+C_{\tau\cup\sigma, 2}}
        \mu^{\tau\cup\sigma}_{\set{e, f}}(\xi)
        }{
        \sum_{\substack{\xi\in \+C_{\tau\cup\sigma, 2}\\ c\notin \xi}}
        \mu^{\tau\cup\sigma}_{\set{e, f}}(\xi)
        }
        \sum_{\substack{\xi\in \+C_{\tau\cup\sigma, 2} \\ c\notin \xi}}
        \mu^{\tau\cup\sigma}_{\set{e, f}}(\xi)
        \\&=
        \frac{
        {q^{\tau\cup\sigma}_{e}q^{\tau\cup\sigma}_f - q^{\tau\cup\sigma}_{ef}}
        }{
        {\ol q^{\tau\cup\sigma}_{e}\ol q^{\tau\cup\sigma}_f - \ol q^{\tau\cup\sigma}_{ef}}
        }
        \sum_{\substack{\xi\in \+C_{\tau\cup\sigma, 2}\\ c\notin \xi}}
        \mu^{\tau\cup\sigma}_{\set{e, f}}(\xi).
    \end{align*}
    Multiply this expression in the former equation, we have
    \begin{align*}
        A_\tau(ec, fc)
        &=  \frac{2a_k}{k}
              \sum_{\substack{\sigma\in \+C_{\tau, k-2}\\ c\notin\sigma}}
              \mu^\tau_{K_\tau\setminus\set{e, f}}(\sigma)
              \frac{
              {q^{\tau\cup\sigma}_{e}q^{\tau\cup\sigma}_f - q^{\tau\cup\sigma}_{ef}}
              }{
              {\ol q^{\tau\cup\sigma}_{e}\ol q^{\tau\cup\sigma}_f - \ol q^{\tau\cup\sigma}_{ef}}
              }
              \sum_{\substack{\xi\in \+C_{\tau\cup\sigma, 2}\\ c\notin \xi}}
              \mu^{\tau\cup\sigma}_{\set{e, f}}(\xi)
                    \frac{p^{\tau\cup\sigma}_{ec}p^{\tau\cup\sigma}_{fc}}
                    {q^{\tau\cup\sigma}_{e}q^{\tau\cup\sigma}_f - q^{\tau\cup\sigma}_{ef}}
      \\&=  \frac{2a_k}{k}
              \sum_{\substack{\sigma\in \+C_{\tau, k-2}\\ c\notin\sigma}}
              \mu^\tau_{K_\tau\setminus\set{e, f}}(\sigma)
              \sum_{\substack{\xi\in \+C_{\tau\cup\sigma, 2}\\ c\notin \xi}}
              \mu^{\tau\cup\sigma}_{\set{e, f}}(\xi)
                    \frac{p^{\tau\cup\sigma}_{ec}p^{\tau\cup\sigma}_{fc}}
                         {\ol q^{\tau\cup\sigma}_{e}\ol q^{\tau\cup\sigma}_f - \ol q^{\tau\cup\sigma}_{ef}}
      \\&=  \frac{2a_k}{k}
              \sum_{\substack{\omega\in \+C_{\tau, k}\\ c\notin\sigma}}
              \mu^\tau_{K_\tau}(\omega)
                    \frac{p^{\taur}_{ec}p^{\taur}_{fc}}
                         {\ol q^{\taur}_{e}\ol q^{\taur}_f - \ol q^{\taur}_{ef}},
    \end{align*}
    and the lemma follows.
\end{proof}

\subsection{Spectral analysis of $A_\tau$}

% In the following lemma, we show that each matrix $A^{i,\omega}_\tau$ can be written as the sum of two matrices which we call the \emph{main term} and the \emph{remainder} respectively. The main term only depends on the adjacency matrix $\-{Adj}_\tau$ and $p, q$ terms under various boundary conditions and is irrelevant to $L_{uv}$ terms. All the effects of $L_{uv}$ terms are collected in the remainder.

For every $\omega\in\+C_{\tau,k}$ such that $c\not\in \!{col}((\omega\cup\tau)|_{K_\tau})$,
define $\Xi_\tau^{\omega} \in \bb R^{K^c\times K^c}$ as the diagonal matrix such that for every $ec \in K_\tau^{c}$:
\[
	\Xi_\tau^{\omega}(ec,ec) =
    \frac{p^{\tau\cup\omega|_{K\setminus \set e}}_{ec}}
         {\ol q^{\tau\cup\omega|_{K\setminus \set e}}_e}.
\]
\begin{lemma}\label{lem:A-tau-i-omega}
	\[
	A_\tau^{\omega} = \Xi_\tau^{\omega} (-\!{Adj}_\tau + \+R_\tau^{\omega}) \Xi_{\tau}^{\omega} ,
	\]
	where $\rho(\+R_\tau^{i,\omega}) \le  \frac{2(k-1)}{\beta - 1}$.
\end{lemma}
\begin{proof}
	Let $ec,fc\in K_\tau^{c}$.
    To ease the notation, when $\tau$ and $\omega$ are clear from the context,
    we use $\Gamma(e)$ and $\Gamma(e,f)$ to denote the partial coloring $(\tau\cup\omega)|_{V\setminus\set{e}}$ and $(\tau\cup\omega)|_{V\setminus\set{e,f}}$ respectively.
	Using our new notations, we have
	\[
	\Xi_\tau(ec,ec) =
    \frac{p^{\Gamma(e)}_{ec}}
         {\ol q^{\Gamma(e, f)}_e}.
	\]
	Observing that since $c\not\in\!{col}\tp{(\tau\cup \omega)|_{K_\tau}}$, we have
	\begin{align*}
        p^{\Gamma(e)}_{e, c} = p^{\Gamma(e, f)}_{e, c},
      \\p^{\Gamma(f)}_{f, c} = p^{\Gamma(e, f)}_{e, c}.
	\end{align*}
	So we can write $A_\tau^{i,\omega}$ as
	\begin{align*}
		A_\tau^{i,\omega}(uc,vc)
		&=-\frac{p^{\Gamma(e)}_{e, c}p^{\Gamma(f)}_{f, c}}
        {\ol q^{\Gamma(e,f)}_e \ol q^{\Gamma(e,f)}_f - \ol q^{\Gamma(e,f)}_{ef}}
	  \\&=\frac{p^{\Gamma(e)}_{e, c}p^{\Gamma(f)}_{f, c}}
        {\ol q^{\Gamma(e)}_e \ol q^{\Gamma(f)}_f}
        \Bigg(
        -1 + \+R_\tau^{\omega}(ec,fc) 
        \Bigg)
	\end{align*}
	where 
	\begin{equation}\label{eqn:eta-remain}
		\+R_\tau^{i,\omega}(ec,fc) =
        \frac{
        -{\ol q^{\Gamma(e)}_e \ol q^{\Gamma(f)}_f}
        +
        {\ol q^{\Gamma(e,f)}_e \ol q^{\Gamma(e,f)}_f - \ol q^{\Gamma(e,f)}_{ef}}
        }{
        {\ol q^{\Gamma(e,f)}_e \ol q^{\Gamma(e,f)}_f - \ol q^{\Gamma(e,f)}_{ef}}
        }.
	\end{equation}
    Notice that
    \begin{align*}
    \abs{\ol q^{\Gamma(e,f)}_e \ol q^{\Gamma(e,f)}_f - \ol q^{\Gamma(e)}_e \ol q^{\Gamma(f)}_f}
    &=
    \abs{\ol q^{\Gamma(e,f)}_e p^{\Gamma(e, f)}_{f, \omega(e)} +
         \ol q^{\Gamma(e,f)}_f p^{\Gamma(e, f)}_{e, \omega(f)} -
         p^{\Gamma(e, f)}_{f, \omega(e)} p^{\Gamma(e, f)}_{e, \omega(f)} }
    \\&\le 
         \ol q^{\Gamma(e,f)}_e p^{\Gamma(e, f)}_{f, \omega(e)} +
         \ol q^{\Gamma(e,f)}_f p^{\Gamma(e, f)}_{e, \omega(f)}.
    \end{align*}
    Therefore,
	\begin{align*}
		\abs{\+R_\tau^{i,\omega}(ec,fc)} \le
        \frac{
         \ol q^{\Gamma(e,f)}_e p^{\Gamma(e, f)}_{f, \omega(e)} +
         \ol q^{\Gamma(e,f)}_f p^{\Gamma(e, f)}_{e, \omega(f)}
        }{
        \ol q^{\Gamma(e,f)}_e \ol q^{\Gamma(e,f)}_f - \ol q^{\Gamma(e,f)}_{ef}
        },
	\end{align*}
    which equals 
    $\mu^{\Gamma(e, f)}_{\set e;\+L'}(\omega(e)) + \mu^{\Gamma(e, f)}_{\set f;\+L'}(\omega(f))$,
    defining $\+L'$ be the color lists obtained by removing $c$ from the color lists of $e, f$.
    By \Cref{lem:marginal-bound-weighted-1}, this is bounded by $\frac{2}{\beta-1}$
    since the modified coloring istance is $(\beta-1)$-extra.
    The lemma then follows by doing a row summation to $\+R_\tau^\omega$.
\end{proof}

\begin{proposition}\label{prop:matrix-sq-coeff}
    Consider non-zero coefficients $\lambda_i$, $i\in [N]$, matrix $A = \sum_{i\in [N]}\lambda_i A_i$ and 
    $\Sigma\mle 0$, and all matrices are square, of the same size and symemtric. Then
    \[
    A\Sigma A \mle
    \big(\sum_{i\in [N]} \lambda_i\big)
    \bigg( \sum_{i\in [N]}\lambda_i A_i\Sigma A_i \bigg).
    \]
\end{proposition}
\begin{proof}
    \begin{align*}
    A\Sigma A &= \sum_{i, j\in [N], i < j} \lambda_i\lambda_j \big( A_i\Sigma A_j + A_i\Sigma A_j \big)
               + \sum_{i\in [N]}\lambda_i^2 A_i\Sigma A_i
    \\& \mle \sum_{i, j\in [N]} \lambda_i\lambda_j\big( A_i\Sigma A_i + A_j\Sigma A_j \big)
               + \sum_{i\in [N]}\lambda_i^2 A_i\Sigma A_i
    \\& = 
    \big(\sum_{i\in [N]} \lambda_i\big)
    \bigg( \sum_{i\in [N]}\lambda_i A_i\Sigma A_i \bigg).
    \end{align*}
\end{proof}

Define $C_\tau\defeq \sum_{\substack{\omega\in \+C_\tau} \\ c\notin \omega}\pi_\tau(\omega)$,
and $\+C'_{\tau, k}\defeq \set{\omega\in \+C_{\tau, k} \mid c\notin\omega}$,
then the following lemma holds.
\begin{lemma}\label{lem:bound-A-tau-i}
 $A_\tau\Pi_\tau^{-1}A_\tau
 \mle
 \frac{4a_k^2C_\tau}{k^2} \sum_{\omega\in\+C'_{\tau, k}} \pi_\tau(\omega) A_\tau^{\omega}\Pi^{-1}_\tau A_\tau^{\omega}
 \mle
 \frac{4a_k^2}{k^2} \sum_{\omega\in\+C'_{\tau, k}} \pi_\tau(\omega) A_\tau^{\omega}\Pi^{-1}_\tau A_\tau^{\omega}$.
\end{lemma}
\begin{proof}
    By \Cref{lem:A-tau-i-omega} and \Cref{prop:matrix-sq-coeff}.
\end{proof}

In the following discussion,
let $\gamma=\bigg( 1+ \frac{\Delta-1}{\beta-1} \bigg)^3\frac{1}{\beta-1}$.
\begin{lemma}\label{lem:bound-A-tau-i-omega-square}
	$ A_\tau^{\omega} \Pi_\tau^{-1} A_\tau^{\omega}
    \mle \gamma k\cdot \Xi_\tau^{\omega}\tp{\tp{\!{Adj}_\tau}^2
    +
    \frac{4(k-1)^2}{(\beta-1)^2}\!{Id}_\tau}\Xi_\tau^{i,\omega}$.
\end{lemma}
\begin{proof}
	By \Cref{lem:marginal-lower-weighted,lem:marginal-upper-weighted,lem:marginal-bound-weighted-1} for any $ec \in K_\tau^{c}$,
	\begin{align*}
			\Xi_\tau^{\omega}\Pi_\tau^{-1}(uc, uc)\leq
            k \frac{(\beta+k-2)(\beta+\Delta-2)}{(\beta-1)^2}\frac{\ell_u - k + 1}{\ell_u-k-\Delta + 1}
            &\le
            k \frac{(\beta+\Delta-2)^2}{(\beta-1)^2}\frac{\beta + \Delta-2}{\beta-1}
            \\&\le k\bigg( 1+ \frac{\Delta-1}{\beta-1} \bigg)^3.
	\end{align*}
	Then it follows from \Cref{lem:A-tau-i-omega} and $\Xi_\tau^{i,\omega}\mle \frac{1}{\beta-1}\cdot \!{Id}_{\tau}$ that
	\begin{align*}
	 A_\tau^{\omega} \Pi_\tau^{-1} A_\tau^{\omega}
     \mle
     2 k\gamma \Xi_\tau^{i,\omega}\tp{\tp{\!{Adj}_\tau}^2  + \frac{4(k-1)^2}{(\beta-1)^2}\!{Id}_\tau}\Xi_\tau^{\omega}.
	\end{align*}

\end{proof}

\begin{lemma}\label{lem:bound-Xi-Pi}
$ \frac{1}{k}\sum_{\omega\in \+C_{\tau, k}}\pi_{\tau, k}(\omega)\Xi^\omega_\tau\Pi^{-1}_\tau = \!{Id}_\tau$.
\end{lemma}
\begin{proof}
    At entry $(ec, ec)$ such that $\pi_\tau(ec)\neq 0$, the LHS is
	\begin{align*}
        \frac{1}{k} \sum_{\omega\in \+C_{\tau, k}}\mu^\tau_{K_\tau}(\omega) \pi^{-1}_\tau(ec)
                      \frac{p^{\tau\cup\omega|_{K\setminus \set e}}_{e, c}}
                       {\ol q^{\tau\cup\omega|_{K\setminus \set e}}_{e   }}
        &= 
        \frac{1}{k} \sum_{\omega'\in \+C_{\tau, K_\tau}}\mu^\tau_{K_\tau}(\omega') \pi^{-1}_\tau(ec)
                      \frac{p^{\tau\cup\omega'}_{e, c}}
                       {\ol q^{\tau\cup\omega'}_{e   }}
      \\&= 
        \frac{1}{k} \sum_{\omega'\in \+C_{\tau, K_\tau}}\mu^\tau_{K_\tau\setminus\set e}(\omega')
                      \mu^{\tau\cup\omega'}_{\set e}(c)
                    \pi^{-1}_\tau(ec)
      \\&= 
        \frac{1}{k} \sum_{\omega'\in \+C_{\tau, K_\tau}}\mu^\tau_{K_\tau}(\omega\cup\set{ec})
                    \pi^{-1}_\tau(ec)
         = 
        \pi^{-1}_\tau(ec) \pi_\tau(ec)
        =1.
	\end{align*}
\end{proof}

We are now ready to bound $\tp{(k-1)\cdot \E[x\sim\pi_\tau]{A_{\tau\cup\set{x}}} - (k-2) \cdot A_\tau+4A_\tau \Pi_\tau^{-1} A_\tau}$ in the LHS of \Cref{eqn:condition-main}.
\begin{lemma}
	\label{lem:bound-A-tau}
	There exists a sequence of non-negative numbers $\set{a_h}_{0\le h\le \Delta}$ such that
	\[
		\Pi_\tau^{-\frac{1}{2}} \tp{(k-1)\cdot \E[x\sim\pi_\tau]{A_{\tau\cup\set{x}}}
        - (k-2) \cdot A_\tau+2A_\tau \Pi_\tau^{-1} A_\tau}
        \Pi_\tau^{-\frac{1}{2}} \mle
        \frac{8\gamma(k-1)}{\beta-1}\Big(1 + \frac{\Delta}{\beta-1}\big(1+\frac{2}{\beta-1}\big)\Big).
	\]
\end{lemma}
\begin{proof}
	Applying \Cref{lem:A-tau-i} and \Cref{lem:bound-A-tau-i}, we obtain
\[
	\!{LHS} \mle 
    \Pi_\tau^{-\frac{1}{2}}
    \tp{
    \frac{2(k-2)}{k}
    (a_{k-1}-a_k)
    \sum_{\omega\in\+C'_{\tau,k}}
      A_\tau^{\omega}
    +
    \frac{8 a_k^2}{k^2}\cdot 
    \sum_{\omega\in \+C'_{\tau,k}}
      A_\tau^{\omega}\Pi^{-1}_\tau A_\tau^{\omega}
    }\Pi_\tau^{-\frac{1}{2}}.
\]
Then by \Cref{lem:A-tau-i-omega} and \Cref{lem:bound-A-tau-i-omega-square}, we can bound above by
\begin{align}
	\label{eq:lem-bound-A-tau-1}
	\!{LHS} \mle\frac{2}{k}
    \Pi_\tau^{-\frac{1}{2}}
    \Xi_\tau^{\omega}
    \Bigg(
    &(k-2)
    (a_{k-1}-a_k)
    \sum_{\omega\in\+C'_{\tau,k}}
      \pi_\tau(\omega)
      (-\!{Adj}_\tau + \frac{2(k-1)}{\beta-1}\!{Id}_\tau)
    \\+&
    4\gamma a_k^2 \cdot
    \sum_{\omega\in \+C'_{\tau,k}}
      \pi_\tau(\omega)
      \Big(
      \!{Adj}^2_\tau + \frac{4(k-1)^2}{(\beta-1)^2}\!{Id}_\tau
      \Big)
    \Bigg)
    \Xi_{\tau}^{\omega}
    \Pi_\tau^{-\frac{1}{2}},
\end{align}
where $\gamma = \bigg( 1+ \frac{\Delta-1}{\beta-1} \bigg)^3\frac{1}{\beta-1}$.

We want to find a sequence of $\set{a_k}$
so that the spectral radius of the following matrices
$\tilde A_k$ appearing in the non-remainder terms in \Cref{eq:lem-bound-A-tau-1} is small:
\[
	\tilde A_h\defeq -(k-2)(a_{k}-a_{k-1})\!{Adj}_\tau + 4a_k^2\gamma\tp{\!{Adj}_\tau}^2.
\]
Since the spectrum of $\!{Adj}_\tau$ is $\set{-1, (k-1)}$, the spectrum of $\tilde A_k$ is
\[
	\set{
		(k-2)(a_{k-1}-a_k) + 4\gamma a_k^2, \;
		-(k-1)(k-2)(a_{k-1}-a_k) + 4\gamma (k-1)^2 a_k^2
	}.
\]
Define
\[
	a_k = \frac{1}{1+4\gamma(k-2)} (2\leq k \leq \Delta).
\]
Then we have
\begin{equation}\label{eqn:spectral-radius-of-A}
	\rho(\tilde A_k)\le \frac{4 \gamma  (1+4\gamma(k-2)) h}{(4 \gamma  (k-3)+1) (4 \gamma  (k-2)+1)^2}\le 4\gamma (k-1)
\end{equation}
for $k \geq 3$. In particular, when $k=2$, $\rho(\tilde A_h)\leq 4a_k^2 \gamma (k-1)^2 = 4\gamma$,
which is consistent with the above bound.
So we have $\rho(\tilde A_k)\leq 4\gamma (k-1)$ for $h\geq 1$.
Note that $\Xi_\tau^{\omega}\mle \frac{1}{\beta-1}\cdot \!{Id}_{\tau}$,
it then follows from \Cref{eqn:spectral-radius-of-A} and \Cref{lem:bound-Xi-Pi} that 
\begin{align}\label{eqn:bound-adj}
	&\phantom{{}={}}
	\frac{2}{k}
    \Pi_\tau^{-\frac{1}{2}}
    \Xi_\tau^{\omega}
    \Bigg(
    -(k-2)
    (a_{k-1}-a_k)
    \sum_{\omega\in\+C'_{\tau,k}}
      \pi_\tau(\omega)
      \!{Adj}_\tau
    +
    4\gamma a_k^2
    \sum_{\omega\in \+C'_{\tau,k}}
      \!{Adj}^2_\tau
    \Bigg)
    \Xi_{\tau}^{\omega}
    \Pi_\tau^{-\frac{1}{2}}
    \notag
    \\&\le \frac{8\gamma (k-1)}{\beta-1} \!{Id}_\tau.
\end{align}
A direct calculation yields that
\begin{align}\label{eqn:bound-remainder}
    &\phantom{{}={}}
	\frac{2}{k}
    \Pi_\tau^{-\frac{1}{2}}
    \Xi_\tau^{\omega}
    \Bigg(
    (k-2)
    (a_{k-1}-a_k)
    \sum_{\omega\in\+C'_{\tau,k}}
      \pi_\tau(\omega)
      \frac{2(k-1)}{\beta-1}\!{Id}_\tau
    +
    4\gamma a_k^2 \cdot
    \sum_{\omega\in \+C'_{\tau,k}}
      \pi_\tau(\omega)
      \frac{4(k-1)^2}{(\beta-1)^2}\!{Id}_\tau
    \Bigg)
    \Xi_{\tau}^{\omega}
    \Pi_\tau^{-\frac{1}{2}}
    \notag
    \\&\mle
    \frac{8\gamma(k-1)^2}{(\beta-1)^2}\Big(\frac{2}{\beta-1} + 1\Big)\!{Id}_\tau
    \mle
    \frac{8\gamma(k-1)\Delta}{(\beta-1)^2}\Big(\frac{2}{\beta-1} + 1\Big)\!{Id}_\tau.
\end{align}
Combining \Cref{eqn:bound-adj} and \Cref{eqn:bound-remainder} finishes the proof.

\end{proof}

\subsection{Construction of $B_\tau$}
\label{sss:Btau}
For $\tau$ of co-dimension $k > 2$,
we introduce coefficients $\set{b_k}_{3 \leq h \leq \Delta}$
whose values will be determined later, and define $B_\tau$ as follows:
\begin{equation} \label{eqn:B-def}
	B_\tau(ec,ec) = b_{k}
\end{equation}
for any $e \in K_\tau$ and all other entries are $0$
When $k=2$, this is exactly the base case considered in \Cref{sec:base-case}.
According to~\eqref{eqn:base-case-def-ol}, we have $b_1=\frac1{\tp{\beta-1}^2}$.

Notice that if $\!{codim}(\tau) = k\ge 3$, then
\begin{align}
\E[x\sim\pi_\tau]{\Pi_{\tau\cup\set{x}} b_k}
=
\pi_\tau(ec)^{-1}\sum_{x\in \+C_\tau}{\pi_\tau(x)\pi_{\tau\cup \set{x}}(ec)} b_{k-1}
=
\sum_{x\in \+C_\tau}{\pi_{\tau\cup \set{ec}}(x) b_{k-1}} = b_{k-1}.
\label{eqn:contraint-B-entry}
\end{align}
where the second equality follows from the fact that $\pi_\tau(x)\pi_{\tau\cup \set{x}}(vc)=\pi_{\tau\cup\set{vc}}(x)\pi_\tau(vc)$. %Note that the value of $B_\tau(vc,vc)$ only depends on the graph $G_\tau$ (irre)

By the discussion in the last section and the above definition, now \cref{eqn:condition-main} 
becomes
\[
(k-2)b_k - (k-1)b_{k-1} - 2b_k^2 \ge 
    \frac{8\gamma(k-1)\Delta}{(\beta-1)^2}\Big(\frac{2}{\beta-1} + 1\Big)
\]
for all $3\le k \le \Delta$.

Assume $\beta \geq 11$. Since 
\begin{align}\label{eqn:A-tau-i-upper-bound}
\begin{split}
		\Pi_\tau^{-1/2}A_\tau\Pi_\tau^{-1/2}
		&=  \frac{2a_{k}}{k}\cdot \Pi_\tau^{-1/2} \sum_{\omega\in \+C'_{\tau,k}}\pi_\tau(\omega) A^{\omega}_\tau \Pi_\tau^{-1/2} \\
		&=  \frac{2a_{k}}{k}\cdot \Pi_\tau^{-1/2} \sum_{\omega\in \+C_{\tau,k}'}\pi_\tau(\omega) \Xi_\tau^{\omega}(-\!{Adj}_\tau +\+R_\tau^{\omega})\Xi_\tau^{\omega}\Pi_\tau^{-1/2}\\
		&\mle \frac{2a_{k}}{(\beta-1)}\tp{1+\frac{2(k-1)}{\beta-1}}\!{Id}_\tau\\
		&\mle \frac{1}{(\beta-1)}\frac{1+\frac{2(k-1)}{\beta-1}}{1+\frac{4(k-2) \tp{1+\frac{\Delta-1}{\beta-1}}^3} {\beta-1}}\!{Id}_\tau\\
        &\mle \frac{1}{\beta-1} \!{Id}_\tau\\
		&\mle \frac{1}{10}\!{Id}_\tau,
\end{split}
\end{align}
%\ytodo{$\!{Id}_{V_\tau}\rightarrow \!{Id}_\tau$}
we have $M_\tau=\frac{\sum_i A_\tau + \Pi_\tau B_\tau}{k-1} \mle \frac{k-1}{3k-1}\Pi_\tau$ as long as $B_\tau \mle \tp{\frac{(k-1)^2}{3k-1}-\frac{1}{10}}\!{Id}_\tau$.
We strengthen this constraint to $B_\tau \mle \tp{\frac{1}{5}-\frac{1}{10}}\!{Id}_\tau=\frac{1}{10}\!{Id}_\tau$.

%Let $\iota(\Delta) = 1 + 0.1\log(\Delta)$ and additionally assume $\beta\ge 11$.
For brevity, we denote ${8\gamma\Delta}\Big(\frac{2}{\beta-1} + 1\Big)$ by $C(\Delta)$ in the following calculation. Therefore, our constraints for $\set{b_k}_{1\leq k\leq \Delta}$ are
\begin{equation}
	\label{eqn:constraint-B}\tag{$\blacktriangle$}
	\begin{cases}
		(k-2) b_k - (k-1)b_{k-2} \ge 2b_k^2+\frac{C(\Delta)(k-1)}{(\beta-1)^2},& 3\le k\le \Delta; \\
		b_k \le \frac1{10}, & 2\le k\le \Delta.
	\end{cases}
\end{equation}
It follows from Lemma 26 in~\cite{WZZ24} that there is a feasible solution of \cref{eqn:constraint-B}
as long as $\beta\ge c\sqrt\Delta\log^2\Delta + 2c$, where $c = \sqrt{20(1+2C(\Delta))}$. And the solution $b_k \leq \frac{1}{(\beta-1)^2}\tp{1+(6+16 C(\Delta))\Delta \log^2 \Delta}$.
Notice that if $\beta - 1\ge \max\set{\Delta, 10}$, then $C(\Delta)\le \frac{10}{\Delta-1}$,
and the solution exists if $\beta - 1\ge 20\log^2\Delta+2\sqrt{\frac{200}{\Delta-1}}$.

Putting all constraints to $\beta$ together, we have
\[\beta-1\ge \max\set{\Delta, 10, 20\log^2\Delta+2\sqrt{\frac{200}{\Delta-1}}},\]
which can be unified to a single bound that $\beta\ge \Delta + 50$.

\subsection{Proof of \Cref{lem:PiP-bound}}
\begin{proof}[Proof of \Cref{lem:PiP-bound}]
For any $\tau \in \+C_{d-k}$ with $k \geq 2$, we construct the matrices $A_\tau$ and $B_\tau$ as \Cref{sss:Atau} and \Cref{sss:Btau}. Then we have
    \[ \rho(\Pi_\tau^{-1/2}A_\tau\Pi_\tau^{-1/2})+ \rho(B_\tau) \le \frac{1}{\beta-1}+\frac{1}{(\beta-1)^2} + \frac{(6+16 C(\Delta))\Delta \log^2\Delta}{(\beta-1)^2}.
    \]
When $\beta \geq \Delta + 50$, the above term is upper bounded by $\eta_\Delta \defeq \frac{1+(6+\frac{160}{\Delta-1})\log^2 \Delta}{\Delta}+\frac{1}{\Delta^2}$.
Applying \Cref{thm:mtd-inductive}, we have
\[
\rho( P_\tau -\frac{k}{k-1}\*{1} \pi_\tau^\top) \le \rho(\Pi_\tau^{-1} M_\tau) \le \frac{\eta_{\Delta}}{k-1}.
\]
Taking $\tau = \emptyset$, we obtain that
\[
\Pi P - \frac{d}{d-1} \pi_\tau \pi_\tau^\top \mle \frac{\eta_{\Delta}}{d-1}\Pi.
\]
\end{proof}
