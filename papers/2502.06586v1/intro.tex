
\section{Introduction}

Sampling from the uniform distribution of proper edge colorings received lots of attention recently, with the advent of new tools in analyzing high-dimensional distributions~\cite{DHP20,ALOG21,WZZ24,CCFV25}. A proper edge coloring of an undirected graph with maximum degree $\Delta$ is an assignment of each edge with one of $q$ colors so each adjacent edges receive different color. Clearly a proper \emph{edge} coloring can be viewed as a proper \emph{vertex} coloring on its line graph.The sampling problem is to draw a proper edge coloring uniformly. Most of previous work focuses on the mixing time of Glauber dynamics. The work of~\cite{WZZ24} established the spectral independence property, a notion to measure the correlation in high-dimensional distributions~\cite{ALO21}, whenever $q>(2+o(1))\Delta$ for general graphs and the work of~\cite{CCFV25} establish approximate tensorization of variance on trees when $q\ge \Delta+1$. Note that Glauber dynamics is known to be reducible when $q<2\Delta$ on general graphs~\cite{MJNP19} and therefore the $q>(2+o(1))\Delta$ condition for Glauber dynamics is asymptotically tight. However, it is still open whether $q\approx 2\Delta$ is the threshold for efficient sampling proper edge colorings and there are some recent attempts to design other sampling algorithm under better conditions~\cite{DKLP25}. A (almost) uniformly sampling algorithm can be turned into fully polynomial-time randomized scheme (\textbf{FPRAS}) for counting the number of proper colorings using standard reduction~\cite{JVV86}. 

In this work we examine some other aspects for the correlation of the uniform distribution on edge colorings. We first established the \emph{coupling independence} property on general graphs when $q\ge 3 \Delta$. Coupling independence~\cite{CZ23} is a notion stronger than spectral independence, and thanks to recent work of~\cite{CFGZZ24}, building on the machinery of Moitra~\cite{Moitra19}, it (together with some other properties) implies a fully polynomial-time approximate scheme (\textbf{FPTAS}) for counting proper edge colorings. 
\begin{theorem}[Informal] \label{thm:FPTAS-informal}
    If $q\geq 3\Delta$, then there exists an \textbf{FPTAS} for counting the number of proper $q$-edge colorings on any graph $G$ with maximum degree $\Delta$.
\end{theorem}

Before our work, there is no similar results tailored for counting edge coloring. The best \textbf{FPTAS} for counting edge coloring is the same as the one for general vertex coloring, which requires $q>3.618\Delta$~\cite{CV25,CFGZZ24} for sufficiently large $\Delta$.

We then study the strong spatial mixing (SSM) property for edge colorings on \emph{trees}, an important notion to measure the correlation between sites in Gibbs distributions whose definition is in \Cref{sec:prelim-decay}.

% \begin{definition}[Strong spatial mixing]\label{def:WSM}\ctodo{Not CLMM23, this is a classic notion}
%     The distribution $\mu$ over the list edge coloring instance $(G = (V,E), \+L)$ satisfies \textit{strong spatial mixing (SSM)} with exponential decay rate $1 - \delta$ and constant $C = C(q,\Delta)$ if for any $e \in E$, every subset $\Lambda \subseteq E\setminus \{e\}$ and every pair of feasible pinning $\tau_1,\tau_2$ on $\Lambda$ which differ on $\partial_{\tau_1,\tau_2} = \set{e\in \Lambda\mid \tau_1(e)\neq \tau_2(e)}$, we have that
%     $$
%         \|\mu_{e}^\sigma - \mu_e^\tau\|_{\-{TV}} \leq C(1 - \delta)^K
%     $$
%     where $K = \min_{e'\in \partial_{\tau_1,\tau_2}}\-{dist}_{G}(e,e')$.
% \end{definition}

\begin{theorem}[Informal]\label{thm:SSM-informal}
    Let $T$ be a tree with maximum degree $\Delta$. If $q\geq (3+o(1))\Delta$, then the uniform distribution over $q$-colorings on $T$ exhibits strong spatial mixing with exponential decay rate.
\end{theorem}

Similar strong spatial mixing bounds on trees have been thoroughly studied for vertex colorings (e.g.~\cite{EGHSV19,CLMM23}). It is conjectured that SSM holds whenever $q\ge \Delta+1$ and in~\cite{CLMM23}, a $q\ge \Delta+3$ condition was established, almost resolving the conjecture. However, much less is known for edge colorings. 
The $q>(3+o(1))\Delta$ bound established in this work is by no means tight. We also discuss the limit of our approach and possible further improvement. On the other hand, we show that one cannot expect the strong spatial mixing property to hold when $q<2\Delta$, as we prove that $2\Delta-1$ is the threshold for \emph{weak spatial mixing}. Therefore, we conjecture that SSM holds for edge colorings on graphs with maximum degree $\Delta$ whenever $q\ge 2\Delta+\gamma$ for certain constant $\gamma$. We also show that the best bound one can expect using the analysis in this paper cannot be better than $q\approx 2.618\Delta$.

\begin{theorem}[Informal]\label{thm:WSM-informal}
    If $q\geq 2\Delta-1$, then the uniform distribution over $q$-colorings on any tree with maximum degree $\Delta$ exhibits weak spatial mixing with exponential decay rate. Otherwise, there exists a tree with maximum degree $\Delta$ such that the uniform distribution over $q$-coloring on it does not satisfy the weak spatial mixing property.
\end{theorem}
In the following, we give an overview with our technique, with an emphasis on the novelty.

\subsection{Technical contribution}

\paragraph{A new coupling strategy}

The coupling independence property is established via a new local coupling for edge colorings. Our coupling can somehow be viewed as a multi-spin version of Chen and Gu's coupling~\cite{CG24} for Holant problems with boolean domain. Their coupling, using the problem of $b$-matching as an example, begins with two instances differing at one pendant edge, or equivalently, two instances with a single constraint discrepancy. During the coupling process, the number of discrepancies can never increase but has a nonzero probability of decreasing to zero. Therefore, the coupling process terminates in expected constant number of steps. 

In the problem of edge coloring, we can design a local coupling starting from a single discrepancy so that the number of discrepancies can either increase to two, decrease to zero, or remain unchanged. We then use marginal probability bounds to control the probability of discrepancy increasing, while ensuring that the number of discrepancies decreases in expectation. 

\paragraph{Dimension reduction}

We establish the strong spatial mixing property on trees by analyzing marginal recursions, which is similar to~\cite{CLMM23}. However, unlike previous work for spin systems where the marginal on a single site is considered, we study the recursion for marginals on a ``broom'', namely all edges incident to the root. For each partial coloring on the broom, we can represent its marginal as a function of marginal probabilities of partial colorings in subtrees. However, the Jacobian matrix of this system can be as large as $q^\Delta \times \Delta q^\Delta$, and is technically very hard to analyze. Our key observation is that the Jacobian matrix is of low rank, and therefore we can apply a trace trick to bound its $2$-norm by the norm of a much smaller matrix. We call this step \emph{dimension reduction}.

\paragraph{From spectral independence to strong spatial mixing}

It is still challenging to directly bound the $2$-norm of the small matrix. We then observe that it can be written as the product of certain covariance matrices of marginal distributions on brooms. Therefore, ideally we can apply the known bounds for these covariance matrices, or equivalently the spectral independence bound for these marginals. However, these marginal probabilities are from the distribution of certain ``weighted edge colorings'' and one cannot directly apply previous spectral independence result for edge colorings. As a result, we apply the machinery of matrix trickle-down developed in~\cite{ALOG21} in the way of~\cite{WZZ24} to establish the desired spectral independence result. 

\paragraph{Top-down analysis of recursion}

There is another subtle technical issue in the above approach. When analyzing the contraction of marginal recursion, one needs to analyze the gradient / Jacobian at certain ``midpoint'' between two boundary conditions due to the application of fundamental theorem of calculus or mean-value theorem in the analysis. These midpoints, however, are not necessarily probabilities because the recursion may involve a potential function\footnote{Theses quantities are referred to as ``subdistributions'' in~\cite{CLMM23}}. In previous work, only certain marginal bounds are used to prove the contraction, and these bounds are also satisfied by the midpoints. However, in our case, we require these midpoints to satisfy refined properties, such as spectral independence, which does not hold in general. Therefore, we cannot apply the recursion in the traditional bottom-to-top manner, where one fixes the boundary value at level $L$, analyze the contraction at level $L-1$, then fixes boundary value at level $L-1$ and analyze the contraction at level $L-2$, and so on. Instead, we only fix boundary value at the leaves and analyze the composition of the recursion at each level as a whole. Therefore, we need to take ``midpoints'' only at the leaves, which defines our ``weighted edge coloring'' instance. Its spectral independence property can be established by the matrix trickle-down method, as discussed earlier.


\subsection{Organization of the paper}

After introducing the necessary preliminaries in \Cref{sec:prelim}, we give the marginal recursions and prove useful marginal bounds in \Cref{sec:marginals}. Then we present our proof of coupling independence, which implies the \textbf{FPTAS}, in \Cref{sec:FPTAS}. Strong spatial mixing on trees is proved in \Cref{sec:ssm} and the results of weak spatial mixing are proved in \Cref{sec:wsm}. In \Cref{sec:covariance}, we prove the spectral independence property for weighted edge coloring that will be used in the proof of strong spatial mixing.

