\section{Tight bound for weak spatial mixing}\label{sec:wsm}
% \zjtodo{Can we prove for general graphs instead of trees?}
In this section, we prove the tight bound for weak spatial mixing  of $q$-edge coloring instance on trees.
Note that the strong spatial mixing of $q$-edge coloring instance is a special case of spatial mixing of list instance (for list coloring instance, weak spatial mixing is equivalent to strong spatial mixing).
Therefore the theorem stated in this section is a weak version of spatial mixing conclusions and thus we can give a tight bound for trees. 

%We first introduce the definition of weak spatial mixing.
% \begin{definition}[Weak spatial mixing]\label{def:WSM}
%     The distribution $\mu$ over the proper list edge coloring instance $(G = (V,E), \+L)$ satisfies \textit{weak spatial mixing (WSM)} with exponential decay rate $1 - \delta$ and constant $C = C(q,\Delta)$ if for any $e \in E$, every subset $\Lambda \subseteq E\setminus \{e\}$ and every pair of feasible pinning $\sigma,\tau$ on $\Lambda$, we have that
%     $$
%         \|\mu_{e}^\sigma - \mu_e^\tau\|_{\-{TV}} \leq C(1 - \delta)^K
%     $$
%     where $K = \min_{e'\in \Lambda}\-{dist}_{G}(e,e')$.
% \end{definition}
The main theorem for weak spatial mixing on trees is as follows.
\begin{theorem}\label{thm:tight_WSM}
    Given a tree $T = (V,E)$ with $n$ vertices, $m$ edges and maximum degree $\Delta$.
    Suppose that instance $(T,\+L)$ is a $q$-edge coloring instance (i.e. for any $e\in E$, $\+L(e) = [q]$). 
    Then we have that
    \begin{enumerate}
        \item If $q \geq 2\Delta - 1$, the edge coloring instance satisfies weak spatial mixing with rate $1 - \frac{1 - \eps}{2\Delta - (1+\eps)}$, where $\eps = \max\{\frac{\Delta - 1}{\Delta}, \frac {e - 1}e\}$.
        \item If $q \leq 2\Delta - 2$, there exists an instance that does not satisfy weak spatial mixing.
    \end{enumerate}
\end{theorem}
Consider the following example which simply shows that hardness of weak spatial mixing:
\begin{example}\label{eg:hardness_WSM}
    Consider a $(\Delta - 1)$-regular tree with depth $d$ (the depth of the root $r$ is $0$) and $d$ is an even number. Let $\Lambda$ be the edges between vertices of depth $d$ and $d-1$. Let $\tau_1$ be the pinning over $\Lambda$ which only uses $1,2,\dots,\Delta-1$ and $\tau_2$ be the pinning over $\Lambda$ which only uses $\Delta,\dots,2\Delta - 2$. It is easy to verify that for any $\{u,v\}\in E\setminus \Lambda$ and $\-{dep}(u) = \-{dep}(v) + 1$,
    $$
        \forall \sigma \in \Omega^{\tau_1}, , \sigma(\{u,v\}) \in 
        \begin{cases}
            \{1,\dots,\Delta-1\} &, 2\ |\ \-{dep}(u)
            \\\{\Delta,\dots,2\Delta-2\} &, 2 \nmid \-{dep}(u)
        \end{cases}
    $$
    $$
        \forall \sigma \in \Omega^{\tau_2}, \sigma(\{u,v\}) \in 
        \begin{cases}
            \{\Delta,\dots,2\Delta-2\} &, 2\ |\ \-{dep}(u)
            \\\{1,\dots,\Delta-1\} &, 2 \nmid \-{dep}(u)
        \end{cases}
    $$
    where $\-{dep}(u)$ is the depth of $u$. Therefore, for any $e\in E\setminus \Lambda$, we have that
    $$
        \|\mu_e^\sigma - \mu_e^\tau\|_{\-{TV}} = 1
    $$
    which demonstrates that the weak spatial mixing does not hold.
\end{example}
The proof scheme is also using the idea of correlation decay and we use the uniform distribution as bridge to prove the weak spatial mixing property. And we use another recursion which is different from that in strong spatial mixing.
Instead of considering a broom of edges, we specify the marginal probability of a pendant edge and then generalize to every edge.
Since the lists of feasible colors are clear, we use $\Pr[T]{\cdot}$ to denote $\Pr[T,\+L]{\cdot}$ for simplicity.
\begin{lemma}[One-step contraction]\label{lem:WSM_contraction}
Suppose $(T = (V,E), \+L)$ is a $q$-edge coloring instance, where $T$ is a tree with a pendant edge $e = \{r',r\}$ on its root $r$ (that is, $\deg(r')=1$) and $\tau$ is the pinning over a set of edges $\Lambda$ whose edges are incident to leaf vertices.  
If for any $e_i = \{v_i,r\}\in E$, the subtree $T_i$ with pendant edge $e_i$ satisfies that 
$$
    \forall c\in[q], \abs{\Pr[T_i\cup \{e_i\}]{c(e_i) = c\mid c(\Lambda) = \tau} - \frac 1q} \leq \delta
$$
where $\delta < \frac 1q$ is a universal constant, then we have that
$$
    \forall c\in [q], \abs{\Pr[T]{c(e) = c\mid c(\Lambda) = \tau} - \frac 1q} \leq \frac{2\Delta - 2}{q(1 - \delta\abs{q-2\Delta + 2})} \delta.
$$
\end{lemma}
% \zjtodo{may be able to improve to general graphs?}
\begin{proof}[Proof of \Cref{lem:WSM_contraction}]
    Assume that there are $d$ children of $r$.
    Let $P_{r,c}$ denote $\Pr[T]{c(e) = c\mid c(\Lambda) = \tau}$ and $P_{i,c}$ denote $\Pr[T_i\cup\{e_i\}]{c(e_i) = c\mid c(\Lambda) = \tau}$.
    Then we have the following recursion for any $c\in [q]$.
    $$
        P_{r,c} = \frac{\displaystyle \sum_{A\in C_r, c\notin A}\prod_{i=1}^d P_{i,A_i}}{(q-d) \displaystyle \sum_{A\in C_r}\prod_{i=1}^d P_{i,A_i}}
    $$
    $\delta < \frac 1q$ implies that $|C_r| = q^{\underline{d}}$. Therefore, the following should be true for any $c\in[q]$.
    \begin{align}
        \nonumber \frac 1{P_{r,c}} &= q - d + \sum_{i=1}^d P_{i,c}\frac{\displaystyle (q-d)\sum_{A\in C_r, A_i = c}\prod_{j\neq i}P_{j,A_j}}{\displaystyle \sum_{A\in C_r, c\notin A}\prod_{j=1}^d P_{j,A_j}}
        \\\nonumber &= q - d + \sum_{i=1}^d P_{i,c}\frac{\displaystyle \sum_{A\in C_r, c\notin A}\prod_{j\neq i}P_{j,A_j}}{\displaystyle \sum_{A\in C_r, c\notin A}P_{i,A_i}\prod_{j\neq i} P_{j,A_j}}
        \\\nonumber &\leq q - d + \sum_{i=1}^d (\frac 1q + \delta)\frac{\displaystyle \sum_{A\in C_r, c\notin A}\prod_{j\neq i}P_{j,A_j}}{\displaystyle (\frac 1q - \delta)\sum_{A\in C_r, c\notin A}\prod_{j\neq i} P_{j,A_j}}
        \\ \label{eq:left_side_WSM_contraction} &= q + \frac{2dq\delta}{1-q\delta}.
    \end{align}
    In the same way, we can show that $\frac 1{P_{r,c}} \geq q - \frac{2dq\delta}{1 + q\delta}$. 
    Combine this inequality and \Cref{eq:left_side_WSM_contraction}, we get
    $$
        \frac{-2d\delta}{q-\delta q(q-2d)}\leq P_{r,c} - \frac 1q \leq \frac{2d\delta}{q+\delta q(q-2d)}
        \implies |P_{r,c} - \frac 1q| \leq \frac{2d}{q(1 - \delta\abs{q-2d})} \delta.
    $$
    When $q > 2d$, $\frac{2d}{q(1 - \delta(q-2d))} \delta$ is monotone increasing with respect to $d$. Then $d\leq \Delta - 1$ implies that
    $$
        \abs{P_{r,c} - \frac 1q} \leq \frac{2\Delta - 2}{q(1 - \delta\abs{q-2\Delta + 2})} \delta.
    $$
\end{proof}
% We introduce the following marginal bound before proving \Cref{thm:tight_WSM}.
% \begin{lemma}[Lemma 3 in \cite{gamarnik2015strong}]\label{lem:gamanik_lower_bound}
%     Consider a vertex list coloring instance $(G,\+L)$ on a graph $G$ where $q_v \geq d_v + \gamma$. Then for all $c \in \+L(v)$, we have that
%     $$
%         \Pr[G]{c(v) = c} \geq\frac 1{q}\tp{1 - \frac 1{\gamma}}^{d_v} 
%     $$
% \end{lemma}
\Cref{lem:marginal-bound-gkm} shows that the $q$-edge coloring instance admits the marginal lower bound, which is a start point of recursive contraction.
% \begin{corollary}[Corollary of Lemma 3 in \cite{gamarnik2015strong}]\label{cor:base_case_WSM}
%     Suppose $(T = (V,E), \+L)$ is a $q$-edge coloring instance, where $T$ is a tree, and $e = \{r',r\}$ is a pendant edge in $E$. For any $\Lambda \subseteq E$ which admits $\min_{e'\in \Lambda} \-{dist}_{T}(e',e) = 2$ and any feasible pinning $\tau$ over $\Lambda$, we have that
%     $$
%         \frac{1 - o_\Delta(1)}{eq}\leq \Pr[T]{c(e) = c\mid c(\Lambda) = \tau} \leq \frac 1{q-\Delta + 1}
%     $$
% \end{corollary}
Now we can prove \Cref{thm:tight_WSM}.
\begin{proof}[Proof of \Cref{thm:tight_WSM}]
    We prove weak spatial mixing for pendant edges first.
    Let $d$ denote $\min_{e'\in \Lambda} \-{dist}_{T}(e',e)$.
    \Cref{lem:marginal-bound-gkm} implies that if $d = 2$, 
    $$
        \frac{1}{eq}\leq \Pr[T]{c(e) = c\mid c(\Lambda) = \tau} \leq \frac 1{q-\Delta + 1}.
    $$
    The right hand side inequality trivially follows from the recursion in the proof of \Cref{lem:WSM_contraction}.
    Therefore, we have that
    \begin{align}\label{eq:base_case_WSM}
        \forall c\in [q], \abs{\Pr[T]{c(e) = c\mid c(\Lambda) = \tau} - \frac 1q} \leq \max\set{\frac{\Delta - 1}{\Delta}, \frac {e - 1}e}\frac 1q < \frac 1q
    \end{align}
    which serves as the base case of recursive contraction of the marginal probability.
    Let $\eps = \max\set{\frac{\Delta - 1}{\Delta}, \frac {e - 1}e}$.
    Therefore, Plugging \Cref{eq:base_case_WSM} and \Cref{lem:WSM_contraction}, we get that for $d\geq 2$,
    \begin{align}\label{eq:WSM_decay}
        \forall c\in [q], \abs{\Pr[T]{c(e) = c\mid c(\Lambda) = \tau} - \frac 1q} \leq \frac{\eps}{q} \tp{1 - \frac{1 - \eps}{2\Delta - (1+\eps)}}^{d-2}.
    \end{align}
    % {\color{red}
    %     Directly apply \Cref{lem:WSM_contraction} with $\delta = \frac \eps q$, We have that
    %     \begin{align*}
    %         \frac{2\Delta - 2}{q(1 - \delta(q-2\Delta + 2))}
    %         &= \frac{2\Delta - 2}{q - \eps(q-2\Delta + 2)}
    %         \\&= \frac{2\Delta - 2}{(1 - \eps) q + \eps(2\Delta -2)}
    %         = \frac{1}{(1-\eps)\frac{q}{2\Delta - 2} + \eps}
    %         \\&\leq \frac{2\Delta - 2}{(1 - \eps) (2\Delta - 1) + \eps(2\Delta -2)}
    %         \\&= \frac{2\Delta - 2}{2\Delta - 2 + 1 - \eps}
    %         \\&= 1 - \frac{1 - \eps}{2\Delta - 2 + 1 - \eps}
    %     \end{align*}
    % }
    % We pick $C = 2q\max\{\frac{\eps}{q}(1 - \frac{1 - \eps}{2\Delta - (1+\eps)})^{-2}, (1 - \frac{1 - \eps}{2\Delta - (1+\eps)})^{-1} \}$ to finish the proof for pendant edges.

% \yltodo{$\frac{\eps}{q}\tp{1-\frac{1-\eps}{\frac{q}{q-2\Delta+2}-\eps}}^{d-2}$}
    For general edge $e = \{u,v\}$, we split $e$ into two pendant edges $e_1 = \{u,w\}$ and $e_2 = \{w,v\}$ by adding a new vertex $w$ to $V$ and $\+L(e_1) = \+L(e_2) = \+L(e)$.
    Let $T'$ denote the new graph after splitting $e$. 
    Then we have that for any $c\in [q]$,
    \begin{align*}
        \Pr[T]{c(e) = c\mid c(\Lambda) = \tau} &= \Pr[T']{c(e_1) = c\mid c(e_1) = c(e_2), c(\Lambda) = \tau}
        \\&=\frac{\Pr[T']{c(e_1) = c(e_2) = c\mid c(\Lambda) = \tau}}{\Pr[T']{c(e_1) = c(e_2)\mid c(\Lambda) = \tau}}
        \\&= \frac{\Pr[T']{c(e_1) = c\mid c(\Lambda) = \tau}\Pr[T']{c(e_2) = c\mid c(\Lambda) = \tau}}{\sum_{c'\in[q]}\Pr[T']{c(e_1) = c'\mid c(\Lambda) = \tau}\Pr[T']{c(e_2) = c'\mid c(\Lambda) = \tau}}.
    \end{align*}
    The last equation follows from the disconnection between $e_1$ and $e_2$.
    For $d\geq 2$, let $\eta = \frac{\eps}{q} \tp{1 - \frac{1 - \eps}{2\Delta - (1+\eps)}}^{d-2}$ for simplicity.
    By \Cref{eq:WSM_decay}, we have that
    \begin{align*}
        \frac{(\frac 1q - \eta)^2}{q(\frac 1q + \eta)^2} \leq \Pr[T]{c(e) = c\mid c(\Lambda) = \tau} \leq \frac{(\frac 1q + \eta)^2}{q(\frac 1q - \eta)^2}.
    \end{align*}
    Therefore, for $d\geq 2$, plugging $\eta \leq \frac{\eps}{q}$ and the above inequality implies that 
    $$
        |\Pr[T]{c(e) = c\mid c(\Lambda) = \tau} - \frac 1q| \leq \frac{4\eta}{(1 - q\eta)^2}\leq \frac{4\eps}{q(1-\eps)^2} \tp{1 - \frac{1 - \eps}{2\Delta - (1+\eps)}}^{d-2}.
    $$
    We pick $C = \max\set{\frac{8\eps}{(1-\eps)^2}(1 - \frac{1 - \eps}{2\Delta - (1+\eps)})^{-2}, \tp{1 - \frac{1 - \eps}{2\Delta - (1+\eps)}}^{-1} }$ to finish the first part of the proof.

    For the second part, it is trivial after applying \Cref{eg:hardness_WSM}.
\end{proof}

