
\newcommand{\ma}{\mu_{E-i}^{i\pin a}}
\newcommand{\mb}{\mu_{E-i}^{i\pin b}}
\newcommand{\mab}{\mu_{E-i}^{\substack{i\pin a\\j\pin b}}}
\newcommand{\mba}{\mu_{E-i}^{\substack{i\pin b\\j\pin a}}}
\newcommand{\man}{\mu_{E-i}^{\substack{i\pin a\\b\notin N}}}
\newcommand{\mbn}{\mu_{E-i}^{\substack{i\pin b\\a\notin N}}}
\renewcommand{\pab}{\mu_{E-i}^{i\pin a}(j\pin b)}
\newcommand{\pba}{\mu_{E-i}^{i\pin b}(j\pin a)}
\newcommand{\pan}{\mu_{E-i}^{i\pin a}(b\notin N)}
\newcommand{\pbn}{\mu_{E-i}^{i\pin b}(a\notin N)}
\newcommand{\sjn}{\sum_{j\in N}}
\newcommand{\sj}{\sum_{j}}

\newcommand{\zzz}{1+\sum_k(\gamma_k\vee\delta_k)}



\section{FPTAS for counting proper edge colorings on general graphs $q\ge 3\Delta$}\label{sec:FPTAS}


In this section, we prove the following main algorithmic result, which is a formal version of \Cref{thm:FPTAS-informal}.
\begin{theorem}\label{thm:FPTAS}
    Assume $\Delta\geq 4$. There exists a deterministic algorithm that outputs $\hat{Z}$ satisfying $(1-\delta)Z_{G,\+L}\leq \hat{Z}\leq (1+\delta)Z_{G,\+L}$ for any $(\Delta + 2)$-extra edge coloring instance $(G,\+L)$ with maximum degree $\Delta$ and given error bound $0<\delta <1$ in time $\tp{\frac{n}{\delta}}^{C(\Delta)}$,  where $n$ is the number of edges in $G$ and $C(\Delta)=\+O\tp{\Delta^{\Delta\log \Delta}\log \Delta}$
    is a universal constant only depends on $\Delta$.
\end{theorem}

Our key contribution is the following coupling independence result.

\begin{theorem}\label{thm:coupling-independence}
    Let $(G=(V, E), \+L)$ be a $((1+\eps)\Delta+1)$-extra list-edge-coloring instance.  Then $\mu_E$ is $\tp{1+\frac{2}{\eps}}$-coupling independent. That is, for any $i\in E$, $a, b\in \+L(i)$,
    \[
    \W{\mu_E^{i\pin a}, \mu_E^{i\pin b}} \le 1 + \frac{2}{\eps}.
    \]
\end{theorem}

We first set up our terminologies to argue about the Wasserstein distance. We define some upper bounds for the $\+W_1$ distance between $\lambda\Delta$-extra list-colorings on $\Delta$-degree graphs with $s$ edges with respect to one different pinning. We will construct recursion on these upper bounds.
An edge is \emph{pendant} if one of its endpoints has degree exactly $1$.
\begin{definition}[Universal upper bounds for coupling independence]\label{def:kappa}
Define
\[
\kappa_{s, \Delta, \lambda} \defeq 
\sup_{\substack{(G=(V, E), \+L)\\ i\in E\colon i \text{ is pendant} \\ a, b\in \+L(i), a\neq b}}
\W{
\mu_{E-i}^{i\pin a}, \mu_{E-i}^{i\pin b}
}
\]
where $(G, \+L)$ is taken over
\begin{enumerate}
    \item all graph $G=(V, E)$ such that $\deg(G)\le \Delta$, $|E|\le s$;
    \item all color lists $\+L$ such that $\forall i\in E: |\+L(i)|\ge \deg(i) + \lambda \Delta + 1$.
\end{enumerate}
\end{definition}
\begin{remark}
    It is clear from the definition that $\kappa_{s+1, \Delta, \lambda}\ge \kappa_{s, \Delta, \lambda}$
    and $\kappa_{1, \Delta, \lambda}=0$.
\end{remark}

Note that we only pin color on the edge $i$ in \Cref{def:kappa}. If we need other pinnings, we can simply consider the pinnings as deleting the pinned edges and remove the pinned color from the lists of their adjacent edges.


The main lemma of this section is a recursion for $\kappa_{s, \Delta, \lambda}$
and leads to \Cref{thm:coupling-independence} immediately.

\begin{lemma}\label{lem:kappa-recursion}
    Let $\lambda=1+\eps$ for some $\eps>0$ and $s\ge 2$. Then
    \[
    \kappa_{s, \Delta, \lambda} \le \frac{2}{2+\eps}\tp{\kappa_{s-1, \Delta, \lambda} + \frac12}.
    \]
\end{lemma}

The proof of \Cref{lem:kappa-recursion} is based on a greedy one-step coupling of two marginal distributions with different pinning on a single edge. We describe the coupling in \Cref{sec:coupling}. The proofs of \Cref{lem:kappa-recursion}, \Cref{thm:coupling-independence} and \Cref{thm:FPTAS} are in \Cref{sec:CI-proof}.


\subsection{Decomposition of Wasserstein distance}\label{sec:coupling}


The following lemma shows how we can go from $\kappa_{s, \Delta, \lambda}$ to $\kappa_{s-1, \Delta, \lambda}$
by one extra pinning.
\begin{lemma}\label{lem:s-to-s-1}
    For the instance $(G=(V, E), \+L)$, the pendant edge $i=\set{u, v}\in E$, and the colors $a, b\in \+L(i)$
    that fit into the definition of $\kappa_{s, \Delta, \lambda}$. 
    Suppose $\deg(u)=1$ and $\deg(v)\ge 2$, and $j\in N(i)$, $a, b\in \+L(j)$.
    Then
    \begin{enumerate}
        \item $\W{\mab, \mba} \le 1 + \kappa_{s-1, \Delta, \lambda}$,
        \item $\W{\mab, \mb}, \W{\mba, \ma} \le 1 + 2\kappa_{s-1, \Delta, \lambda}$,
        \item $\W{\man, \mbn} = 0$.
    \end{enumerate}
\end{lemma}
\begin{proof}
%    \zhtodo{I think this proof works better with graphs drawn out.}
%    \ctodo{In case we have time to draw.}
    Assume $j=\set{v,w}$. 
    \begin{enumerate}
    \item We define a new instance $\tp{G'=(V', E'=E-i), \+L'}$ by removing $i$, disconnecting $j$ from $v$ and delete $a, b$ in the color lists of edges in $N(v)$. Then $j$ becomes to a pendant edge. We have that
    \begin{align*}
        \W{\mu_{E-i; (G, \+L)}^{\substack{i\pin a\\ j\pin b}}, \mu_{E-i;(G, \+L)}^{\substack{i\pin b\\ j\pin a}}}
        &= 1 + \W{\mu_{E-i-j; (G, \+L)}^{j\pin b}, \mu_{E-i-j; (G, \+L)}^{j\pin a}}\\
        &= 1 + \W{\mu_{E'-j;(G', \+L')}^{j\pin b}, \mu_{E'-j;(G', \+L')}^{j\pin a}}.
    \end{align*}
    After the deletion of $i$ and the removal of $a$ or $b$ from the color lists of $N(v)$, the number of extra colors of each edge remains unchanged. So by \Cref{def:kappa}, the  Wasserstein distance
    $\W{\mu_{E'-j;(G', \+L')}^{j\pin b}, \mu_{E'-j;(G', \+L')}^{j\pin a}}$
    is bounded by $\kappa_{s-1, \Delta, \lambda}$, so we have
    \begin{align*}
    \W{\mu_{E-i; (G, \+L)}^{\substack{i\pin a\\ j\pin b}}, \mu_{E-i;(G, \+L)}^{\substack{i\pin b\\ j\pin a}}}
       \le 1+\kappa_{s-1, \Delta, \lambda}.
    \end{align*}

    \item By the law of total probability,
    \begin{align*}
    \mu_{E-i; (G, \+L)}^{\substack{i\pin a\\ j\pin b}} - \mu_{E-i;(G, \+L)}^{i\pin b}
    &=\mu_{E-i; (G, \+L)}^{\substack{i\pin a\\ j\pin b}}
    - \sum_{c\in\+L(j)-b}\mu_{E-i;(G, \+L)}^{i\pin b}(j\pin c)
    \mu_{E-i;(G, \+L)}^{\substack{i\pin b\\j\pin c}}\\
    &=\sum_{c\in\+L(j)-b}
    \mu_{E-i;(G, \+L)}^{i\pin b}(j\pin c)
        \tp{\mu_{E-i; (G, \+L)}^{\substack{i\pin a\\ j\pin b}}
    -
    \mu_{E-i;(G, \+L)}^{\substack{i\pin b\\j\pin c}}}.
    \end{align*}
    By \cref{prop:coupling-convex-decomposition} this means
    \begin{align}
    \W{\mu_{E-i; (G, \+L)}^{\substack{i\pin a\\ j\pin b}} , \mu_{E-i;(G, \+L)}^{i\pin b}}
    &\le
    \sum_{c\in\+L(j)-b}
    \mu_{E-i;(G, \+L)}^{i\pin b}(j\pin c)
    \W{
    \mu_{E-i; (G, \+L)}^{\substack{i\pin a\\ j\pin b}}
    ,
    \mu_{E-i;(G, \+L)}^{\substack{i\pin b\\j\pin c}}
    }
    \notag
    \\&\le
    1+
    \sum_{c\in\+L(j)-b}
    \mu_{E-i;(G, \+L)}^{i\pin b}(j\pin c)
    \W{
    \mu_{E-i-j; (G, \+L)}^{\substack{i\pin a\\ j\pin b}}
    ,
    \mu_{E-i-j;(G, \+L)}^{\substack{i\pin b\\j\pin c}}
    }.
    \label{eq:sum-abac}
    \end{align}
    For each $c\in\+L(j)-b$, we construct a new list-edge-coloring instance $(G'=(V, E'), \+L')$
    By removing $j$, appending a new edge $j'$ to $w$, and removing $b$ from the color lists of edges in $N(v)$. Then we have the following identities since the color constraints of each pair of edges
    are the same.
    \begin{align*}
    \mu_{E-i-j; (G, \+L)}^{\substack{i\pin a\\ j\pin b}}
    =
    \mu_{E'-i-j'; (G', \+L')}^{\substack{i\pin a\\ j'\pin b}}
    ,\quad
    \mu_{E-i-j; (G, \+L)}^{\substack{i\pin b\\ j\pin c}}
    =
    \mu_{E'-i-j'; (G', \+L')}^{\substack{i\pin c\\ j'\pin c}}.
    \end{align*}
    Applying these identities and triangle inequality to~\eqref{eq:sum-abac}, we have
    \begin{align*}
    \W{\mu_{E-i; (G, \+L)}^{\substack{i\pin a\\ j\pin b}} , \mu_{E-i;(G, \+L)}^{i\pin b}}
    &\le
    1+
    \sum_{c\in\+L(j)-b}
    \mu_{E-i;(G, \+L)}^{i\pin b}(j\pin c)
    \W{
    \mu_{E'-i-j'; (G', \+L')}^{\substack{i\pin a\\ j'\pin b}}
    ,
    \mu_{E'-i-j'; (G', \+L')}^{\substack{i\pin c\\ j'\pin c}}.
    }
    \\&\le
    1+
    \sum_{c\in\+L(j)-b}
    \mu_{E-i;(G, \+L)}^{i\pin b}(j\pin c)
    \Bigg(
    \W{
    \mu_{E'-i-j'; (G', \+L')}^{\substack{i\pin a\\ j'\pin b}}
    ,
    \mu_{E'-i-j'; (G', \+L')}^{\substack{i\pin a\\ j'\pin c}}.
    }
    \\&\;+
    \W{
    \mu_{E'-i-j'; (G', \+L')}^{\substack{i\pin a\\ j'\pin c}}
    ,
    \mu_{E'-i-j'; (G', \+L')}^{\substack{i\pin c\\ j'\pin c}}.
    }
    \Bigg).
    \end{align*}
    Since we may remove the edge with the same pinning from the graph and remain the distribution unchanged,
    the two $\+W_1$ distances are both bounded by $\kappa_{s-1, \Delta, \lambda}$, proving the second part of 
    the lemma.

    \item For the third part, notice that the available colors of all edges in $E-i$ are 
    exactly the same, so ${\mu_{E-i}^{\substack{i\pin a\\b\notin N}}}={\mu_{E-i}^{\substack{i\pin b\\a\notin N}}}$
    and 
    \begin{align*}
        \W{\mu_{E}^{\substack{i\pin a\\b\notin N}},\mu_{E}^{\substack{i\pin b\\a\notin N}}}
        =0.
    \end{align*}
    \end{enumerate}
\end{proof}

Let $(G=(V, E), \+L)$ be a list-edge-coloring instance that fits into the constraints of $\kappa_{s, \Delta, \lambda}$ in \Cref{def:kappa}, $i\in E$ be a pendant edge, and $a, b\in \+L(i)$ be colors. We do a one-step coupling to reduce the $W_1$ distance between graphs with $s$ edges to that between graphs with $s-1$ edges using \Cref{prop:coupling-convex-decomposition}.

Denoting the two endpoints of $i$ by $u, v$, we may assume $u$ is the pendant vertex, that is $\deg(u) = 1$ without loss of generality.
    
%If $\deg(v)=1$, then the pinning of $i$ does not affect other edges, and $\W{\mu_{E}^{i\pin a}, \mu_{E}^{i\pin b}} = 1\le 1 + \frac{2}{2+\eps}\kappa_{s-1, \Delta,\lambda}$ holds naturally.
%\ctodo{The notation $E-i$.}
We denote the non-empty set $E(v)-i$ by $N$, and use integers from $1$ to denote the edges in $N$ so that $N = \set{1, \ldots, d}$. For every edge $j\in N$, define
\begin{align*}
    \gamma_j\defeq \frac{\pab}{\pan}, \quad
    \delta_j\defeq \frac{\pba}{\pbn}.
\end{align*}
The following lemma describes the greedy coupling we use.
\begin{lemma}
    \begin{align*}
        \ma-\mb
          =&\sum_{j\in N}\frac{\gamma_j\wedge\delta_j}{\zzz}\Bigg(\mab - \mba\Bigg)
        \\&+\sum_{j\in N}\frac{(\gamma_j-\delta_j)\vee 0}{\zzz}\Bigg(\mab-\mb\Bigg)
           +\sum_{j\in N}\frac{(\delta_j-\gamma_j)\vee 0}{\zzz}\Bigg(\ma-\mba\Bigg)
        \\&+\frac{1}{\zzz}\Bigg(\man-\mbn\Bigg).
    \end{align*}
\end{lemma}
\begin{proof}
    The criterion of decomposing $\ma-\mb$ is whether
    $a$ or $b$ appears in $N$.
    Without loss of generality, we assume $\pan\le\pbn$,
    and denote $\pan/\pbn$ by $\alpha$.
    By the law of total probability, we have
    \begin{align}
        \ma = \sjn\pab\mab + \pan\man.
    \end{align}
    It is also clear that
    \begin{align*}
        \mb
          &= \alpha \mb + (1-\alpha) \mb
        \\&= \alpha \Bigg(\sjn\pba\mba + \pbn\mbn\Bigg)
         + (1-\alpha) \mb
        \\&= \alpha \sjn\pba\mba + \pan\mbn + (1-\alpha)\mb.
    \end{align*}
    Then
    \begin{align*}
        &\ma - \mb
        \\=&\sjn \pab\mab + \pan \man -\sjn\pba\mba-\pbn\mbn.
    \end{align*}
    The point of the multiplier $\alpha$ is to align the coefficients
    of $\man$ and $\mbn$, so that they don't pair with other distributions,
    and will not introduce the pinning $\cdot\notin N$, which reduces the 
    number of extra colors, into the recursion.
    By the above two decompositions,
    \begin{align}
           \ma - \mb
          =&\sjn\pab\mab - \alpha\sjn\pba\mba
            \notag
        \\&-(1-\alpha)\mb
            \notag
        \\&+\pan\Bigg(\man-\mbn\Bigg).
        \label{eq:u-v-fresh-coeff}
    \end{align}
    In general $\pab$ and $\pba$ do not equal, and we need to analyze them carefully.
    Then we can express $\pab,\pan,\pba,\pbn$ by them.
    \begin{align*}
          \pab &= \frac{\pab}{\pan+\sum_{k\in N} \ma(k\pin b)} = \frac{\gamma_j}{1+\sum_k\gamma_{k}}.
    \end{align*}
    We omitted the range of $k$ for simplicity.
    Similarly,
    \begin{align*}
             \pba = \frac{\delta_j}{1+\sum_{k}\delta_k},
       \quad \pan = \frac{1}{1+\sum_k\gamma_k},
       \quad \pbn = \frac{1}{1+\sum_k\delta_k},
       \quad \alpha = \frac{1+\sum_k\delta_k}{1+\sum_k\gamma_k}.
    \end{align*}
    Then we plug them into \cref{eq:u-v-fresh-coeff}.
    \newcommand{\zz}{1+\sum_k\gamma_k}
    \begin{align*}
           \ma - \mb
          =&\sj\frac{\gamma_j}{\zz}\mab - \sj\frac{\delta_j}{\zz}\mba
        \\&+\frac{\sum_k(\delta_k-\gamma_k)}{ 1 + \sum_k\gamma_k}\mb
           +\frac{1}{ 1 + \sum_k\gamma_k}\Bigg(\man-\mbn\Bigg)
        \\=&\sj\frac{\gamma_j\wedge\delta_j}{\zz}\Bigg(\mab - \mba\Bigg)
        \\&+\sum_j\frac{(\gamma_j-\delta_j)\vee 0}{\zz}\mab
           -\sum_j\frac{(\delta_j-\gamma_j)\vee 0}{\zz}\mba
        \\&+\frac{\sum_k(\delta_k-\gamma_k)}{\zz}\mb
           +\frac{1}{\zz}\Bigg(\man-\mbn\Bigg).
    \end{align*}
    By adding $\frac{\sum_k (\delta_k-\gamma_k)\vee 0}{\zz}\Big(\ma-\mb\Big)$ on both sides, we get
    \begin{align*}
           \frac{1+\sum_k(\gamma_k\vee\delta_k)}{\zz}\Big(\ma - \mb\Big)
          =&\sj\frac{\gamma_j\wedge\delta_j}{\zz}\Bigg(\mab - \mba\Bigg)
        \\&+\sum_j\frac{(\gamma_j-\delta_j)\vee 0}{\zz}\mab
           -\sum_j\frac{(\delta_j-\gamma_j)\vee 0}{\zz}\mba
        \\&+\sj\frac{(\delta_j-\gamma_j)\vee 0}{\zz}\ma
           {\color{purple}-\sj\frac{(\delta_j-\gamma_j)\vee 0}{\zz}\mb}
        \\&{\color{purple}+\frac{\sum_k(\delta_k-\gamma_k)}{\zz}\mb}
           +\frac{1}{\zz}\Bigg(\man-\mbn\Bigg).
    \end{align*}
    The highlighted terms sums to $-\sj\frac{(\gamma_j-\delta_j)\vee 0}{\zz}\mb$.
    So we can pair the terms to get
    \begin{align*}
           \frac{1+\sum_k(\gamma_k\vee\delta_k)}{\zz}\Big(\ma - \mb\Big)
          =&\sj\frac{\gamma_j\wedge\delta_j}{\zz}\Bigg(\mab - \mba\Bigg)
        \\&+\sum_j\frac{(\gamma_j-\delta_j)\vee 0}{\zz}\Bigg(\mab-\mb\Bigg)
           +\sum_j\frac{(\delta_j-\gamma_j)\vee 0}{\zz}\Bigg(\ma-\mba\Bigg)
        \\&+\frac{1}{\zz}\Bigg(\man-\mbn\Bigg).
    \end{align*}
    Finally, by dividing $\frac{\zzz}{\zz}$, 
    we get the decomposition
    \begin{align*}
        \ma-\mb
          =&\sj\frac{\gamma_j\wedge\delta_j}{\zzz}\Bigg(\mab - \mba\Bigg)
        \\&+\sum_j\frac{(\gamma_j-\delta_j)\vee 0}{\zzz}\Bigg(\mab-\mb\Bigg)
           +\sum_j\frac{(\delta_j-\gamma_j)\vee 0}{\zzz}\Bigg(\ma-\mba\Bigg)
        \\&+\frac{1}{\zzz}\Bigg(\man-\mbn\Bigg).
    \end{align*}
  
\end{proof}

\subsection{Proof of main theorems}\label{sec:CI-proof}

Now we prove \Cref{lem:kappa-recursion}, which provides a recursion for $\kappa_{s,\Delta,\lambda}$.

\begin{proof}[Proof of \Cref{lem:kappa-recursion}]
    We consider every $\lambda\Delta+1$-extra color instance $\tp{G=(V,E),\+L}$ with a pendant edge $i=\set{u,v}$ such that $\deg(G)\le \Delta$, $\abs{E}\le s$. Suppose $\deg(u)=1$. If $\deg(v)=1$, then $\W{\ma,\mb}=0$. In the following we assume $\deg(v)\ge 2$. 

An application of \Cref{prop:coupling-convex-decomposition} gives
\begin{align*}
    \W{\ma, \mb}
      \leq&\sj\frac{\gamma_j\wedge\delta_j}{\zzz}\W{\mab, \mba}
    \\&+\sum_j\frac{(\gamma_j-\delta_j)\vee 0}{\zzz}\W{\mab, \mb}
       +\sum_j\frac{(\delta_j-\gamma_j)\vee 0}{\zzz}\W{\ma, \mba}
    \\&+\frac{1}{\zzz}\W{\man, \mbn}.
\end{align*}

By \Cref{lem:s-to-s-1},
\begin{align*}
    \W{\ma, \mb}&\le 1-\frac{1}{\zzz} + \frac{\sum_k(\gamma_k\wedge\delta_k) + 2|\gamma_j-\delta_j|}{\zzz}\kappa_{s-1,\Delta,\lambda}
    \\&\le 1-\frac{1}{\zzz} + \frac{\sum_k2(\gamma_k\vee\delta_k)}{\zzz}\kappa_{s-1,\Delta, \lambda}.
\end{align*}
Finally, the bound $\forall j\in N : \gamma_j,\delta_j\le \frac{1}{\lambda\Delta}$ from \Cref{cor:marginal-bound-gamma-delta} gives
\begin{align*}
    \W{\ma, \mb} \le \frac{1}{1+\lambda} + \frac{2/\lambda}{1+1/\lambda}\kappa_{s-1,\Delta, \lambda}
    \le \frac{2}{2+\eps}\Big(\frac12 + \kappa_{s-1,\Delta,\lambda}\Big).
\end{align*}
Taking the supremum as in \Cref{def:kappa} proves the lemma.
\end{proof}

\begin{proof}[Proof of \Cref{thm:coupling-independence}]
    \Cref{lem:kappa-recursion} shows that $\sup_{s}\kappa_{s, \Delta, \lambda}\le \frac{1}{2+\eps}(1+2/\eps) = 1/\eps$.
    And
    \begin{align*}
        \W{\mu_{E}^{i\pin a}, \mu_{E}^{i\pin b}} \le 1 + \W{\mu_{E-i}^{i\pin a}, \mu_{E-i}^{i\pin b}}.
    \end{align*}
    Since we may break $i$ into two pendant edges, the theorem follows from the triangle inequality.
\end{proof}

\Cref{thm:coupling-independence} shows that any $((1+\eps)\Delta+1)$-extra list-edge-coloring instance is $\tp{1+ \frac{2}{\eps}}$-coupling independent. This is because adding additional pinnings can be viewed as generating a new $((1+\eps)\Delta+1)$-extra list-edge-coloring instance by deleting the pinned edges and removing the corresponding colors from the lists of their adjacent edges. 

The work of~\cite{CFGZZ24} designs an \textbf{FPTAS} for counting the partition function of any Gibbs distribution of permissive spin systems that is marginally bounded and coupling independent. A spin system is specified by a $4$-tuple $S=(G = (V,E),q, A_E, A_V)$ where the state space is $[q]^V$ and the weight of a configuration is characterized by the matrices $A_E \in \bb{R}_{\geq 0}^{q\times q}$ and $A_V \in \bb{R}_{\geq 0}^q$. The Gibbs distribution is defined by:
\[
\mu(\sigma)\propto w(\sigma):=\prod_{u,v\in E} A_E(\sigma(u),\sigma(v))\prod_{v\in V} A_V(\sigma(v)).
\]
The normalizing factor of $\mu$ is called the partition function $Z:=\sum_{\sigma\in [q]^V}w(\sigma)$.

We say $S$ is permissive if for any partial configuration $\tau \in [q]^\Lambda$ with $\Lambda \subseteq V$, the conditional partition function $Z^\tau = \sum_{\sigma:\tau\subset \sigma} w(\sigma) >0$. For $\tau\in [q]^\Lambda$ with $\Lambda \subset V$, let $\mu_v^\tau$ be the marginal distribution on $v\in V\setminus \Lambda$ conditional on the partial configuration $\tau$. We say $\mu$ is $b$-marginally bounded if for any partial configuration $\tau \in [q]^\Lambda$ with $\Lambda \subseteq V$, any vertex $v \in V\setminus \Lambda$ and $c\in [q]$ such that $\mu_v^\tau>0$, we have $\mu_v^\tau\geq b$.

The main result of~\cite{CFGZZ24} that we will use is as follows.
\begin{theorem}[\cite{CFGZZ24}]\label{thm:ci-FPTAS}
    Let $q \geq 2, b>0, C>0, \Delta \geq 3$ be constants. There exists a deterministic algorithm such that given a permissive spin system $\mathcal{S}=\left(G, q, A_E, A_V\right)$ and error bound $0<\varepsilon<1$, if the Gibbs distribution of $\mathcal{S}$ is b-marginally bounded and satisfies $C$-coupling independence, and the maximum degree of $G$ is at most $\Delta$, then it returns $\hat{Z}$ satisfying $(1-\varepsilon) Z \leq \hat{Z} \leq(1+\varepsilon) Z$ in time $\left(\frac{n}{\varepsilon}\right)^{f(q, b, C, \Delta)}$, where $f(q, b, C, \Delta)=\Delta^{\+O\left(C\left(\log b^{-1}+\log C+\log \log \Delta\right)\right)} \log q$ is a constant.
\end{theorem}
In our setting, the spin system is list-edge-coloring instance. Be careful with the parameters since the spins are on edges of the graph. In order to prove \Cref{thm:FPTAS}, we need the marginal lower bound from \cite{GKM15}.
\begin{lemma}[Corollary of Lemma 3 in \cite{GKM15}] \label{lem:marginal-bound-gkm}
Fix a $\beta$-extra edge coloring instances $(G,\+L)$ of maximum degree $\Delta$ with Gibbs distribution $\mu$. For any partial coloring $\tau$ on $\Lambda \subseteq E$, any $e \in E\setminus \Lambda$, and $c \in \+L^\tau(e)$, it holds that
	\[
		\mu^{\tau}_e(c) \geq \frac{\tp{1-\frac{1}{\abs{\+L^\tau(e)}-\deg^\tau(e)}}^{\deg^\tau(e)}}{\abs{\+L^\tau(e)}}\ge \frac{(1-\frac{1}{\beta})^{2\Delta-2}}{\beta+2\Delta-2} .
	\]
 \end{lemma}
Now we give the proof of our main theorem in this section.
\begin{proof}[Proof of \Cref{thm:FPTAS}]
It is easy to verify that any $\beta$-extra edge coloring instance with $\beta \geq 1$ is permissive. From \Cref{thm:coupling-independence} and \Cref{lem:marginal-bound-gkm}, we know that the Gibbs distribution $\mu$ of a $(\Delta + 2)$-extra edge coloring instance $(G,\+L)$ is $\tp{1+ 2\Delta}$-coupling independent and $b$-marginally bounded where $b=\frac{(1-\frac{1}{\Delta + 2})^{2\Delta-2}}{3\Delta}$ is $\Omega\tp{\frac{1}{\Delta}}$. Then by \Cref{thm:ci-FPTAS}, there exists a deterministic algorithm that outputs $\hat{Z}$ satisfying $(1-\delta)Z_{G,\+L}\leq \hat{Z}\leq (1+\delta)Z_{G,\+L}$ in time $\tp{\frac{n}{\delta}}^{C(\Delta)}$ where $n$ is the number of edges in $G$ and $C(\Delta)=\+O\tp{\Delta^{\Delta\log \Delta}\log \Delta}$.
\end{proof}

