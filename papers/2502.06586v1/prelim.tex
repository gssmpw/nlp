\newcommand{\dist}{\-{dist}}

\section{Preliminaries}\label{sec:prelim}
We use the following notations. For any $a, b\in \mathbb R$, let $
    a\wedge b\defeq \min\set{a, b}$, $
    a\vee b\defeq \max\set{a, b}$. For any two non-negative integers $a\ge b$, let $a^{\underline{b}}$ be the falling factorial, i.e. $a^{\underline{b}}=\prod_{i=a-b+1}^a i$.
Let $\!{Id}$ denote the identity matrix. For a function $f\colon \Omega\to\bb R$ defined on a finite domain $\Omega$, we use $\Big[f(x)\Big]_{x\in\Omega}$ to denote the corresponding (column) vector in $\bb R^{\Omega}$. For any set $S$ and an element $x\in S$, we write $S-x$ for $S\setminus \set{x}$.

For two probability measures $\mu, \nu$ on the same probability space $\Omega$, we define $\|\mu-\nu\|_{\-{TV}}=\frac{1}{2} \sum_{\omega \in \Omega}|\mu(\omega)-\nu(\omega)|=\sup _{A \subseteq \Omega}|\mu(A)-\nu(A)|$ for the total variation distance between $\mu, \nu$. If $\mu, \nu$ are two probability distributions on finite state spaces $\Omega_1, \Omega_2$ respectively, then we say $\omega$ is a coupling of $\mu, \nu$ when it is a joint distribution on $\Omega_1 \times \Omega_2$ with $\mu, \nu$ as its marginals, i.e. $\mu(x)=\sum_{y \in \Omega_2} \omega(x, y)$ and $\nu(y)=\sum_{x \in \Omega_1} \omega(x, y)$ for every $x \in \Omega_1, y \in \Omega_2$.%\ctodo{I changed dist to \textbackslash dist}

Given a graph $G=(V,E)$, for any vertex $v\in V$, let $E(v)= \set{e\in E \mid  v\in e}$ and $\deg(v)=\abs{E(v)}$ be the degree of $v$; for any edge $e \in E$, let $\deg(e)$ be the degree of $e$. Moreover, we write $\deg(G)$ for the maximum (vertex) degree in $G$.
For  two edges $e,e'\in E$, we write $\dist_G(e, e')$ for the length of the shortest path between them in $G$ (not containing $e,e'$) and $\dist_G(e, e') = \infty$ if $e$ and $e'$ is disconnected. 
Similarly, for vertices $v,v'\in V$, $\dist_G(v,v')$ is the shortest path between them in $G$ and $\dist_G(v, v') = \infty$ if $v$ and $v'$ is disconnected.

\subsection{List edge coloring}
Fix a color set $[q]=\set{1, 2, \dots, q}$ where $q\in\bb N$. Let $G=(V,E)$ be an undirected graph and $\+L=\set{\+{L}(e)\subseteq[q]\cmid e\in E}$ be a collection of color lists associated with each edge in $E$. The pair $(G, \+L)$  is an instance of list edge coloring. 

If $\+L(e)=[q]$ for any $e\in E$, we say $(G,\+L)$ is a $q$-edge coloring instance.  If $\abs{\+L(e)}\geq \deg(e)+\beta$ for any $e\in E$, we say $(G,\+L)$ is a $\beta$-extra edge coloring instance. We say $\sigma:E\rightarrow [q]$ is a proper edge coloring if $\sigma(e) \in \+L(e)$ for any $e\in E$ and $\sigma(e_1)\neq \sigma(e_2)$ for any $e_1\cap e_2 \neq \emptyset$. Let $\Omega$ denote the set of all proper edge colorings and $\mu$ be the uniform distribution on $\Omega$. 

Let $\Lambda\subseteq E$ and $\tau\in [q]^\Lambda$. We say $\tau$ is a proper partial edge coloring on $\Lambda$ if it is a proper coloring on $(G[\Lambda],\+L|_\Lambda)$ where $G[\Lambda]$ is the subgraph of $G$ induced by $\Lambda$ and $\+L|_{\Lambda}=\set{\+L(e)\in \+L\cmid E\in \Lambda}$. Let $\Omega^\tau$ be the set of all proper edge colorings on $E$ that is compatible with $\tau$, i.e. $\Omega^\tau = \set{\sigma \in \Omega\mid \tau \subset \sigma}$ . We also define $\mu^\tau$ on $\Omega$ which is supported on $\Omega^\tau$ as $\mu^\tau(\cdot) = \Pr[\sigma \sim \mu]{\sigma=\cdot \mid \tau \subset \sigma}$.
For a subset $S\subseteq E\setminus \Lambda$ and a partial coloring $\omega$ on $S$, define $\Omega^\tau_S$ as the set of all proper partial edge colorings on $S$ that is compatible with
$\tau$ and $\mu^\tau_S(\omega)=\Pr[\sigma\sim \mu]{\omega \subset \sigma \mid \tau \subset \sigma}$. Especially, when $\Lambda=\set{i}$ and $\tau(i)=c$, we write the conditional distribution and the conditional marginal distribution by $\mu^{i\leftarrow c}$ and $\mu^{i\leftarrow c}_S$. Besides, we define the color lists after pinning  $\tau$ by $\+L^\tau$ such that for any $e\in E\setminus\Lambda$, $\+L^\tau(e)=\set{c\in \+L(e)\mid \mu^\tau_e(c)>0}$ and the degree after pinning by $\deg^\tau(e) = \abs{\set{e\cup f \neq \emptyset\mid f \in E\setminus \Lambda}}$.%\ctodo{The notation $\Omega^\tau$.}

For a given list edge coloring instance $(G,\+L)$, let $Z_{G,\+L}(M)$ denote the number of proper colorings with the condition $M$ satisfied, (or event $M$ happens) and $\Pr[G,\+L]{M}$ denote the probability that the condition $M$ is satisfied when a proper coloring is drawn uniformly at random. For an edge set $F\subseteq E$, we usually use $c(F)$ to denote the partial coloring on $F$. With a little abuse of notation, $c(F)$ is sometimes referred to as the set of colors used on $F$.
For a color $a$, we write $a\in F, a\notin F$ as shortcuts for $a\in c(F), a\notin c(F)$ respectively.

\subsection{The Wasserstein distance}
In this work, we restrict our discussions and terminologies to finite probability spaces without invoking general measure theory. 

\begin{definition}[Wasserstein distance]
    Let $\mu, \nu$ be two distributions defined on the same finite set $\Omega$ equipped with a metric $d(\cdot, \cdot)$. We define $\Gamma(\mu, \nu)$ as the set of couplings of $\mu$ and $\nu$. Then the \emph{Wasserstein ($1$-)distance} is defined by
    \[
    \W{\mu, \nu} \defeq
    \sup_{\tau\in \Gamma(\mu, \nu)} \E[(x, y)\sim \tau]{d(x, y)}.
    \]
\end{definition}
%\ctodo{changed the def using coupling directly}

%Joint distributions in $\Gamma(\mu, \nu)$ is also referred as a \emph{couplings}.
In this paper, our metric $d$ is always the Hamming distance. For two configurations $\sigma, \tau$ on $[q]^V$, their Hamming distance is defined as $d(\sigma, \tau)=\abs{\set{v \in V \mid \sigma(v)\neq \tau(v)}}$. We define the notion of coupling independence for Gibbs distribution here.
\begin{definition}[Coupling independence]
    We say a Gibbs distribution $\mu$ over $[q]^V$ satisfies $C$-coupling independence if for any two partial configurations $\sigma, \tau \in [q]^\Lambda$ on $\Lambda \subseteq V$ such that $d(\sigma, \tau)=1$,
    $$
\W{\mu^\sigma, \mu^\tau} \leq C
$$
where $\mu^\sigma$ and $\mu^\tau$ denote the Gibbs distribution conditional on $\sigma$ and $\tau$, respectively.
\end{definition}

We will use the following inequality w.r.t Wasserstein distance in later proof.
\begin{proposition}\label{prop:coupling-convex-decomposition}
    Let $\mu, \nu$ be arbitrary distributions on a common finite metric space $(\Omega, d)$.
    If there exists non-negative constants $\lambda^i, 1\le i\le k$
    and distributions $\set{\mu_i}_{1\le i\le k}$, $\set{\nu_i}_{1\le i\le k}$ on $\Omega$ such that
    \[
    \mu - \nu = \sum_{i=1}^k\lambda^i(\mu_i - \nu_i),
    \]
    where we regard both sides as functions on $\Omega$.
    Then
    \[
    \W{\mu, \nu} \le \sum_{i=1}^k \lambda^i{\W{\mu_i, \nu_i}}.
    \]
\end{proposition}
\begin{proof}
   The Kantotrovich-Rubinstein duality theorem (see Theorem 1.14 in \cite{villani2021topics}
   for the proof) states a equivalent form of Wasserstein distance:
   \begin{align*}
    \W{\mu, \nu} = \sup_{f\in L^1(\Omega)}\inner{f}{\mu - \nu},
   \end{align*}
   where $\inner{f}{\mu-\nu}\defeq\sum_{x\in \Omega} f(x)(\mu(x)-\nu(x))$ and
   $L^1(\Omega)\defeq\set{f:\Omega\to \bb R\mid \forall x, y\in \Omega : f(x)-f(y)\le d(x, y)}$
   is the space of $1$-Lipschiz functions.
   Then
   \begin{align*}
    \W{\mu, \nu} 
    =\sup_{f\in L^1(\Omega)}\inner{f}{\mu - \nu}
    =\sup_{f\in L^1(\Omega)}\inner{f}{\sum_{i=1}^k\lambda^i(\mu_i - \nu_i)}
    \le \sum_{i=1}^k\lambda^i\sup_{f\in L^1(\Omega)}\inner{f}{\sum_{i=1}^k\lambda^i(\mu_i - \nu_i)}
    =\sum_{i=1}^k\lambda^i\W{\mu_i, \nu_i}.
   \end{align*}
\end{proof}

\subsection{Correlation Decay}\label{sec:prelim-decay}
Correlation decay refers to the phenomenon that the correlation between the color assignments of edges diminishes as their distance in the graph increases. Specifically, there are two primary notions of correlation decay: strong spatial mixing and weak spatial mixing. These two notions differ in how they measure the ``distance'' over which the correlation should decay.
\begin{definition}[Strong spatial mixing]\label{def:SSM}
    The Gibbs distribution $\mu$ of the list edge coloring instance $(G = (V,E), \+L)$ satisfies \textit{strong spatial mixing (SSM)} with exponential decay rate $1 - \delta$ and constant $C = C(q,\Delta)$ if for any $e \in E$, every subset $\Lambda \subseteq E\setminus \{e\}$ and every pair of feasible pinning $\tau_1,\tau_2$ on $\Lambda$ which differ on $\partial_{\tau_1,\tau_2} = \set{e\in \Lambda\mid \tau_1(e)\neq \tau_2(e)}$, we have that
    $$
        \|\mu_{e}^\sigma - \mu_e^\tau\|_{\-{TV}} \leq C(1 - \delta)^K
    $$
    where $K = \min_{e'\in \partial_{\tau_1,\tau_2}}\-{dist}_{G}(e,e')$.
\end{definition}
\begin{definition}[Weak spatial mixing]\label{def:WSM}
    The Gibbs distribution $\mu$ of the list edge coloring instance $(G = (V,E), \+L)$ satisfies \textit{weak spatial mixing (WSM)} with exponential decay rate $1 - \delta$ and constant $C = C(q,\Delta)$ if for any $e \in E$, every subset $\Lambda \subseteq E\setminus \{e\}$ and every pair of feasible pinning $\sigma,\tau$ on $\Lambda$, we have that
    $$
        \|\mu_{e}^\sigma - \mu_e^\tau\|_{\-{TV}} \leq C(1 - \delta)^K
    $$
    where $K = \min_{e'\in \Lambda}\-{dist}_{G}(e,e')$.
\end{definition}