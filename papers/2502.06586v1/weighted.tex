
\section{Covariance matrix on brooms}\label{sec:covariance}
We will give an upper bound for the covariance matrix defined in \Cref{def:covariance} in this section.
\begin{lemma}\label{lem:SI-broom}
    Given a tree $T$ rooted with $r$ of depth $\ell+1$ and weight functions $\*p_v:E_{T_v}(v)\rightarrow \bb{R}_{\geq 0}$ for any $v\in B_T(r,\ell)$. If $\ell\geq 2$ and $\beta \geq \Delta + 50$, then $\*p_r\defeq  f^{\ell\rightarrow 0}(\tp{\*p_v}_{v\in B(r,\ell)})$ is $(1+\eta_\Delta)$-spectrally independent, i.e.
    \[
    \!{Cov}(\*p_r)\mle (1+\eta_{\Delta})\Pi(\*p_r).
    \]
    where $\eta_{\Delta}$ is $O\tp{\frac{\log^2 \Delta}{\Delta}}$.
\end{lemma}
\subsection{Weighted edge coloring instance}
\label{sec:weight-edge-coloring}
Given a tree $T$ and associated $\beta$-extra color lists $\+L$ for all edges in $T$, an weighted edge coloring instance is a distribution over proper list colorings $(T, \+L)$.
There are non-intersecting connected subgraphs $\set{K_i}_{i\in [N]}$ in $T$
called \emph{boundaries} and associated \emph{weight functions} $\set{w_i}_{i\in [N]}$
such that $w_i : C_{v_i} \to \mathbb R_{\ge 0}$.
\begin{definition}[Distribution of weighted list edge coloring]\label{def:weighted-coloring}    
    Given a list edge coloring instance $(T, \+L)$,
    boundaries $\set{K_i}_{i\in [N]}$
    and weighted functions $\set{w_i}_{i\in [N]}$
    as above, define the associated weighted edge coloring instance
    be the distribution $\mu$ on $\Omega$ such that
    \begin{align*}
        \mu(\sigma) \defeq \frac1Z \prod_{i\in [N]} w_i(\sigma|_{K_i}),
    \end{align*}
    where $Z\defeq \sum_{\sigma\in \Omega}\prod_{i\in [N]} w_i(\sigma|_{K_i})$
    is the partition function.
\end{definition}

Notice that, if we pick an $\omega_i\in\Omega_{K_i}$ for each $i\in [N]$,
then
\[\mu^{\omega_1\cup\cdots\cup \omega_N}(\sigma)
\propto
\begin{cases}
 \frac{1}{Z}   \prod_{i\in [N]} w_i(\sigma|_{K_i}),  \forall i\in [N], \sigma|_{K_i} = \omega_i,
 \\ 0, & \text{otherwise},
\end{cases}
\]
which means $\mu$ is a distribution over uniform edge colorings conditioned on a 
pinning on all $K_i$. This leads to the following lemma.
\begin{lemma}\label{lem:weighted-decomposition-to-unweighted}
    We have
    \[
        \mu = \sum_{\omega_1\in \Omega_{K_1},\cdots, \omega_N\in \Omega_{K_N}}
        \mu_{K_1\cup\cdots\cup K_N}(\omega_1\cup\cdots\cup\omega_{N}) \mu^{\omega_1\cup\cdots\cup\omega_N}.
    \]
    That is, the distribution on weighted coloring instance in \Cref{def:weighted-coloring}
    can be expressed as a mixture of uniform distributions over edge colorings.
\end{lemma}
\begin{proof}
    This is just an application of the total probability formula.
\end{proof}

With \Cref{lem:weighted-decomposition-to-unweighted}, we can prove
weighted edge colorings inherit the marginal bound \Cref{lem:claw-marginal-generalized}
on non-boundary edges.
\begin{lemma}[Marginal upper bound - weighted version]\label{lem:marginal-upper-weighted}
    Consider a weighted edge coloring instance on a tree $T=(V, E)$
    as defined in \Cref{def:weighted-coloring},
    whose distribution is denoted by $\mu$, and a pinning $\xi\in \Omega_{S}$ for some subset $S\subset E$.
    Then for any $i\in V$, $F\subseteq E_i\setminus \tp{\bigcup_{i\in [N]}K_i\cup S}$ and color $a$:
    denoting $\beta\defeq\min_{e\in F}\set{\abs{\+L^\xi(e)}-\deg(e)}$, we have
    \[\mu^\xi(a\in c(F)) \le \frac{|F|}{\beta - 1 + |F|}.\]
\end{lemma}
\begin{proof}
    We write $\mu^\xi$ into a mixture of uniform distributions on edge colorings
    by \Cref{lem:weighted-decomposition-to-unweighted}:
    \begin{align*}
        \mu^\xi = \sum_{\omega_1\in \Omega_{K_1},\cdots, \omega_N\in \Omega_{K_N}}
        \mu^\xi_{K_1\cup\cdots\cup K_N}(\omega_1\cup\cdots\cup\omega_{N}) \mu^{\xi\cup\omega_1\cup\cdots\cup\omega_N}.
    \end{align*}
    It suffices to prove the bound for any $\mu^{\xi\cup\omega_1\cup\cdots\cup\omega_N}$.
    Notice that $\mu^{\xi\cup\omega_1\cup\cdots\cup\omega_N}$ are uniform distributions 
    on proper colorings defined on $T\setminus\tp{\bigcup_{i\in [N]}K_i\cup S}$, with $\beta$-extra colors 
    on all edges in $F$,
    so \Cref{lem:claw-marginal-generalized} applies and proves the lemma.
\end{proof}

We also need the following lower bound in the matrix trickle-down.
\begin{lemma}\label{lem:marginal-lower-weighted}
    Consider a weighted edge coloring instance on a tree $T$ with $\beta$-extra
    color lists $\+L$.
    If $K=E(v)$ for some vertex $v$ in $T$ and none of the neighbours of $K$ is in the boundary,
    $\tau\in \Omega_S$ is a pinning on some subset $S \subset E$, and the number of 
    unpinned edges in $K$ is $k$. Then for any unpinned $e\in K$ and $c\in \+L^\tau(e)$,
    \[
    \mu^{\tau}_e(c) \ge \frac{(\beta-1)^2}{(\beta+k-2)(\beta+\Delta-2)}\frac{1}{\ell_e^\tau-k+1}.
    \]
    where $\ell^\tau_e\defeq|\+L^\tau(e)|$.
\end{lemma}
\begin{proof}
    We assume $e = \set{u, v}$, where $K=E(v)$ and $L\defeq E(v)$.
    By \Cref{lem:weighted-decomposition-to-unweighted},
    it suffices to prove the bound for any $\mu^{\sigma\cup\omega_1\cup\cdots\cup\omega_N}$.
    Denoting $\xi$ as the shortcut for $\sigma\cup\omega_1\cup\cdots\cup\omega_N$,
    we can do a further decomposition:
    \begin{align*}
        \mu^{\xi}_e(c) =
        \sum_{\substack{\sigma_1\in\Omega^{\xi            }_{K\setminus\set e}\\ c\notin \sigma_1}}
        \mu^{\xi                        }_{K\setminus\set e}(\sigma_1)\cdot
        \sum_{\substack{\sigma_2\in\Omega^{\xi\cup\sigma_1}_{L\setminus\set e}\\ c\notin \sigma_2}}
        \mu^{\xi\cup\sigma_1            }_{L\setminus\set e}(\sigma_2)\cdot
        \mu^{\xi\cup\sigma_1\cup\sigma_2}_e(c).
    \end{align*}
    By assumption, $K$ and $L$ are not in the boundary, then \Cref{lem:marginal-upper-weighted}
    gives%\ctodo{one more comma?}
    \begin{align*}
       &\sum_{\substack{\sigma_1\in\Omega^{\xi            }_{K\setminus\set e}\\ c\notin \sigma_1}}
        \mu^{\xi                        }_{K\setminus\set e}(\sigma_1)
        \ge \frac{\beta-1}{\beta + k - 2}
     \\&\sum_{\substack{\sigma_2\in\Omega^{\xi\cup\sigma_1}_{L\setminus\set e}\\ c\notin \sigma_2}}
        \mu^{\xi\cup\sigma_1            }_{L\setminus\set e}(\sigma_2)
        \ge \frac{\beta-1}{\beta + \Delta - 2}.
    \end{align*}
    Moreover, since $\mu^{\xi\cup\sigma_1\cup\sigma_2}_e$ is the uniform distribution
    over $\+L^{\xi\cup\sigma_1\cup\sigma_2}(e)$, whose size is at most $\ell_e^{\tau}-k+1$, it is
    lower bounded by $1/(\ell_e^{\tau}-k+1)$.

    Combining the three bounds, we have
    \begin{align*}
        \mu^{\tau}_e(c) \ge \frac{(\beta-1)^2}{(\beta+k-2)(\beta+\Delta-2)}\frac{1}{\ell_e^\tau-k+1}.
    \end{align*}
\end{proof}

After introducing the concept of weighted edge coloring instances, we now turn our attention to the marginal distribution on a broom. For simplicity we run the matrix trickle down theorem on one broom $K$ in $T$,
that does not intersect with the boundaries
(actually on the quotient simplicial complex on $K$).
Then it is necessary to look at the marginal distribution of $\mu$
on $K$ under some pinning $\xi \in \Omega_F$ on a subset of edges $F$.
We have
\begin{align}\label{eq:broom-marginal-1}
\mu^\xi_K(\tau)
&\propto \sum_{\substack{\sigma:\sigma|_K=\tau \\ \sigma|_F=\xi}}
 \prod_{i\in [N]} w_i(\sigma|_{K_i}).
\end{align}

The following lemma demonstrates the relation between the weighted coloring instance and the tree recursion.
\begin{lemma}\label{lem:p=mu}
    Assume the tree $T$ with root $r$ is of depth $\ell+1$. Let the boundaries be all brooms on $\ell+1$-th level.
    For any $v\in B(r,\ell)$, the weight function on $E_{T_v}(v)$ is just $\*p_v$, and $\*p=(\*p_v)_{v\in B(r,\ell)}$. Choose $K=E(r)$. Then for the simplicial complex defined as above, we have  \[\*p_r\defeq  f^{\ell\rightarrow 0}(\*p)= \mu_K .\]
\end{lemma}
\begin{proof} 
    For simplicity, we assume that all leaves in $T$ are at the same level. It can be generalized to any tree of depth $\ell+1$. We write the entries of $\*p_r$ explicitly:
    \begin{align*} 
    \*p_r(\pi) &\propto \prod_{v^1\in N(r)} \sum_{\substack{\sigma^1 \in  C_{v^1}\\\pi((r,v^1))\notin \sigma^1}} \*p_{v^1}(\sigma^1)\\
    &\propto \prod_{v^1\in N(r)}
    \sum_{\substack{\sigma^1 \in C_{v^1}\\\pi((r,v^1))\notin \sigma^1}} \cdots
    \prod_{v^\ell\in N(v^{\ell-1})}
    \sum_{\substack{\sigma^\ell \in C_{v^{\ell-1}}\\\sigma^{\ell-1}((v^{\ell-1},v^\ell))\notin \sigma^\ell} \*p_{v^\ell}}(\sigma^\ell)\\
    & = \sum_{\sigma|_K=\pi}\prod_{v \in B(v,\ell)} \*p_v(\sigma|_{E_{T_v}(v)}).
    \end{align*}
    Therefore, $\*p_r = \mu_K$.
\end{proof}

Suppose $K = \set{e_i}_{i\in [d]}$.
Then $T\setminus K$ is composed of $d$ disconnected subgraphs denoted by $T_i$
such that $T_i$ is adjacent to $K$ in $T$ (we define $T_i = \emptyset$
if $e_i$ is pendant).
Moreover, we define $\+K_i\defeq\set{K_j\mid K_j\subseteq T_i}$.
Since both $\set{T_i}$ and $\set{\+K_i}$ contain disjoint subgraphs and 
the $\sigma$ in the summation in \cref{eq:broom-marginal-1}
is determined by partial colorings $\sigma|_{T_i}$,
we can define $\+S^\xi_{i, c}\subseteq \Omega^\xi_{T_i}$ by the proper colorings 
on $T_i$ such that is compatible with $e_i$ being colored $c$.
\begin{align*}
\mu^\xi_K(\tau)
&\propto
  \sum_{\sigma_1\in \+S^\xi_{1, \tau(e_1)}}
  \sum_{\sigma_2\in \+S^\xi_{2, \tau(e_2)}}\dots
  \sum_{\sigma_N\in \+S^\xi_{N, \tau(e_N)}}
  \prod_{i\in [d]} \prod_{j : K_j\in \+K_i} w_j(\sigma_i|_{K_j})\\
&= \prod_{i\in [d]} \sum_{\sigma_i\in \+S^\xi_{i, \tau(e_i)}}
     \prod_{j : K_j\in \+K_i} w_j(\sigma_i|_{K_i}).
\end{align*}
Define 
\[
p^\xi_{e_i,c}=\sum_{\sigma_i\in \+S^\xi_{i, c}}
     \prod_{j : K_j\in \+K_i} w_j(\sigma_i|_{K_i}).
\]
Then we have $\mu^\xi_K(\tau)\propto\prod_{i\in [d]} p^\xi_{e_i,\tau(e_i)}$.
This proves the following lemma.
\begin{lemma}[Marginal distribution on a broom]
    Consider a list edge coloring instance $(T, \+L)$,
    boundaries $\set{K_i}_{i\in [N]}$
    weighted functions $\set{w_i}_{i\in [N]}$,
    and a pinning $\xi\in \Omega_{F}$ for some subset $F\subset E$.
    Then for a broom $K$ disjoint from boundaries
    there exists constants $p^\xi_{e, c}$ for $e\in K$, $c\in \+L^\xi(e)$ such that
    \[
    \mu^\xi_K(\tau)\propto\prod_{i\in [d]} p^\xi_{e_i,\tau(e_i)}.
    \]
\end{lemma}

Let $q^\xi_e=\sum_{c\in \+L^\xi(e)} p^\xi_{e,c}$
and $q^\xi_{e,f}=\sum_{c\in \+L^\xi(e)\cap \+L^\xi(f)} p^\xi_{e,c}p^\xi_{f,c}$.

Next we present some bounds on the quantities $p^\xi_{e, c}$ and $q^\xi_\cdot$
\begin{lemma}\label{lem:marginal-bound-weighted-1}
For any $e_i\in K, \xi\in \Omega_F$ for some subset $F \subset E$,
\[
\frac{p^\xi_{e_i,c}}{q^\xi_{e_i}}\le \frac{1}{\ell^\xi_{e_i}}
\]
\end{lemma}
\begin{proof}
    Denote the subtree induced by $T_i$ and $e_i$ by $\tilde T_i$.
    We consider the weighted coloring instance on $(\tilde T_i, \+L)$ with boundaries
    $\+K_i$ and weighted functions $\set{w_j}_{K_j\in \+K_i}$
    and denote the associated distribution by $\nu$.
    Then by \Cref{def:weighted-coloring},
    \[
    \nu^\xi_{e}(c) = \frac{p^\xi_{e_i, c}}{q^\xi_{e_i, c}}.
    \]
    Then
    \[
    \frac{p^\xi_{e_i,c}}{q^\xi_{e_i}}
    =
    \sum_{\sigma\in C_{T_i, \+L}} \nu^{\sigma\cup\xi}_e(c)\nu^\xi_{T_i}(\sigma).
    \]
By \Cref{lem:marginal-upper-weighted},
$\nu^{\sigma\cup\xi}_e(c)\leq \frac{1}{\beta}$.
On the other hand, $\sum_{\sigma\in C_{T_i, \+L}} \nu^\xi_{T_i}(\sigma)=1$.
So $\frac{p_{e,c}}{q_e}\leq \frac{1}{\beta}$.
\end{proof}
\begin{lemma}\label{lem:marginal-bound-weighted-2}
For $e,f\in K$ and $\xi\in \Omega_F$ for some subgraph $F$,
    \[
    q^\xi_{e,f}\leq \frac{q^\xi_e q^\xi_f}{\beta}
    \]
\end{lemma}
\begin{proof}
\begin{align*}
q^\xi_{e,f}&=\sum_{c\in \+L(e)\cap \+L(f)} p^\xi_{e,c}p^\xi_{f,c}\\
&\leq \tp{\sum_{c\in \+L(e)} \tp{p^\xi_{e,c}}^2}^{1/2} \tp{\sum_{c\in \+L(f)} (p^\xi_{f,c})^2}^{1/2}\\
&\leq \frac{\sqrt{q^\xi_e q^\xi_f}}{\beta}\tp{\sum_{c\in \+L^\xi(e)} p^\xi_{e,c}}^{1/2} \tp{\sum_{c\in \+L^\xi(f)} p^\xi_{f,c}}^{1/2}\\
&=\frac{q^\xi_e q^\xi_f}{\beta}.
\end{align*}
\end{proof}

\subsection{Simplicial complexes}
First we introduce simplicial complexes to encode the edge coloring instance. Given a universe $U$, a \emph{simplicial complex} $\+C\subseteq 2^U$ is a collection of subsets of $U$ that is downward close, which means that if $\sigma\in \+C$ and $\sigma'\subseteq\sigma$, then $\sigma'\in \+C$. Every element $\sigma\in \+C$ is called a \emph{face}, and a face that is not a proper subset of any other face is called a \emph{maximal face} or a \emph{facet}. The dimension of a face $\sigma$ is $\!{dim}(\sigma)\defeq \abs{\sigma}$, namely the size of $\sigma$. For every $k\ge 0$, let $\+C_k\defeq \set{\sigma\in\+C\cmid \abs{\sigma}=k}$ be the set of faces of dimension $k$. Specifically, $\+C_0=\set{\emptyset}$. The dimension of $\+C$ is the maximum dimension of faces in $\+C$. 

Besides, we say $\+C$ is a \emph{pure} $n$-dimensional simplicial complex if all maximal faces in $\+C$ are of dimension $n$. In this work we only deal with pure simplicial complexes. In a pure simplicial complexe, we define the co-dimension of a face $\sigma$ as $\!{codim}(\sigma)\defeq n-\!{dim}(\sigma)$.

Let $\pi_n$ be a distribution over the maximal faces $\+C_n$. We use the pair $(\+C,\pi_n)$ to denote a \emph{weighted simplicial complex} where for each $1\le k < n$, the distribution $\pi_n$ induces a distribution $\pi_k$ over $\+C_k$. Formally, for every $1\le k<n$ and every $\sigma'\in \+C_k$, $\pi_k(\sigma')$ is proportional to the sum of weights of maximal faces containing $\sigma$. Formally,
\begin{align*}
    \pi_k(\sigma') \defeq \frac{1}{\binom n k}\sum_{\sigma\in \+C_n\cmid\sigma' \subset \sigma} \pi_{n}(\sigma).
\end{align*}
It can be easily verified that $\pi_k$ is a distribution on $\+C_k$. Sometimes, we omit the subscript when $k=1$, i.e., we write $\pi$ for $\pi_1$.

Also we define the simplicial complexes generated by pinning a face in $\+C$. For a face $\tau \in \+C$ of dimension $k$, we define its \emph{link} as
\begin{align*}
    \+C_\tau=\set{\sigma\setminus \tau \cmid \sigma \in \+C \land \tau \subseteq \sigma }.
\end{align*}
Obviously, $\+C_\tau$ is a pure $(n-k)$-dimensional simplicial complex. Similarly, for every $1\le j \le n-k$, $\+C_{\tau,j}$ is the set of faces in $\+C_\tau$ of dimension $j$. We also use $\pi_{\tau,j}$ to denote the \emph{marginal distribution} on $\+C_{\tau,j}$. Formally, for every $\sigma\in \+C_{\tau,j}$,
\begin{align*}
    \pi_{\tau,j}(\sigma)\defeq \Pr[\alpha \sim \pi_{k+j}]{\alpha=\tau \cup \sigma \cmid \tau\subseteq \alpha}=\frac{\pi_{k+j}(\tau \cup \sigma)}{\binom{k+j}{k} \cdot \pi_{k}(\tau)}.
\end{align*}
We also drop the subscript when $j=1$, i.e., we write $\pi_\tau$ for $\pi_{\tau,1}$. We define a random walk $P_\tau$ with stationary distribution $\pi_\tau$ as
\[
P_\tau(x,y)=\frac{\pi_{\tau,2}(\set{x,y})}{2\pi_\tau(x)}.
\]

% Let $(\+C^{(1)},\pi^{(1)})$, $(\+C^{(2)},\pi^{(2)})$ be two pure weighted simplicial complexes of dimension $d_1$ and $d_2$ respectively. We define another pure weighted simplicial complex $(\+C,\pi)$ of dimension $d_1+d_2$ whose maximal faces are the disjoint union of maximal faces of $\+C^{(1)}$ and $\+C^{(2)}$ and $\pi$ is the product measure $\pi^{(1)} \times \pi^{(2)}$. $(\+C,\pi)$ is also called the \emph{product} of $(\+C^{(1)},\pi^{(1)})$ and $(\+C^{(2)},\pi^{(2)})$. This definition can be naturally generalized to the products of three or more weighted simplicial complexes.


\subsection{Matrix trickle down on a broom}

Now we define the corresponding simplicial complex to the weighted edge coloring instance defined in \Cref{sec:weight-edge-coloring}.
Recall that we are dealing with a weighted edge coloring instance on a tree $T=(V, E)$
with $\beta$-extra color lists $\+L$. $K = E_i$ for some $i\in V$ and $|K| = d$.
We are going to show that the distribution of weighted colorings on $K$ can be
represented by a weighted simplicial complex.

Since any proper edge coloring  $\sigma:E\rightarrow [q]$ could be regarded as a set of pairs of edge and color, namely $\set{(e,c)\in E\times [q]\cmid \sigma(e)=c}$, the weighted edge-coloring instance restricted on $K$ can be naturally represented as a pure weighted simplicial complex $(\+C,\pi_{d})$ where $\+C$ consists of all proper partial colorings in $(K,\+L|_K)$ and $\pi_{d}=\mu_K$.

Before introducing the matrix trickle-down theorem, we define notations for matrices related to $\pi_\tau$. Define $\Pi_\tau \in \bb R^{\+C_1\times\+C_1}$ as $\Pi_\tau \defeq \operatorname{diag}(\pi_\tau)$ supported on $\+C_{\tau,1}\times \+C_{\tau,1}$,
and $\pi_\tau\defeq[\pi_\tau(x)]_{x\in \+C_1}$ be a vector supported on $C_{\tau, 1}$

For convenience, define the pseudo inverse $\Pi_\tau^{-1} \in \bb R^{\+C_1\times\+C_1}$ of $\Pi_\tau$ as $\Pi_\tau^{-1}(x,x)=\pi_\tau(x)^{-1}$ for $x\in \+C_{\tau,1}$ and $0$ otherwise. Similarly, the pseudo inverse square root $\Pi_\tau^{-1/2} \in \bb R^{\+C_1\times\+C_1}$is defined as $\Pi_\tau^{-1/2}(x,x)=\pi_\tau(x)^{-1/2}$ for $x\in \+C_{\tau,1}$ and $0$ otherwise. When $\tau=\emptyset$, we omit the subscript.

Recalling \Cref{def:covariance} and \Cref{def:diag_mean}, 
the following lemma relates the covariance of $\mu_K$ and the matrices defined in this section.
\begin{proposition}\label{lem:mtd-to-si}
    \[
    \Pi(\mu_K)=d\Pi
    \]
    and
    \[
    \!{Cov}(\mu_K) =d \tp{(d-1) \tp{\Pi P -\frac{d}{d-1}\pi \pi^\top} + \Pi}
    \]
    in the sense of padding with zeros.
\end{proposition}
The proof is by direct calculation.
We apply the matrix trickle-down theorem on $(\+C,\pi_d)$ to prove the following lemma. The main idea of the proof is almost the same as that in \cite{WZZ24} except for substituting the number of feasible colors to the weights of feasible colors w.r.t $\set{p_{e,c}}$ after pinning. And the construction of matrix upper bound is simpler since the line graph of $K$ is a clique. We include the details in \Cref{appendix-mtd}.
\begin{lemma}\label{lem:PiP-bound}
    If $K$ is not adjacent to boundaries, i.e. $\set{K_i}$ and $\beta \geq \Delta + 50$, then the simplicial complex $(\+C,\pi_d)$ defined as above satisfying
    \[
    \Pi P -\frac{d}{d-1}\pi \pi^\top \mle \frac{\eta_{\Delta}}{d-1} \Pi.
    \]
    where $\eta_{\Delta}$ is $\+O\tp{\frac{\log^2 \Delta}{\Delta}}$.
\end{lemma}
Then \Cref{lem:SI-broom} is derived directly from \Cref{lem:PiP-bound} and \Cref{lem:mtd-to-si}.
