\section{Data Availability}
We have made our replication package available, which contains all the code and datasets available here ~\citep {reproduce}.


% \subsection{Prompt Templates}


\begin{figure}
\subfigure[Assertion Generation ~(b) Bugs2Fix ~ (c) Code Suggestion]
    {\includegraphics[width=1\linewidth]{figures/datapre.pdf}}
    \caption{The illustration of different RAG coding tasks alongside their respective prompt templates.}
    % \sw{move to appendix if we need more space.}
    \vspace{-0.3cm}
\end{figure}\label{fig:template}

\section{Token Taxonomy of Code}\label{sec:category}
(i) \textbf{Symbol}: Tokens representing operators, delimiters, and other symbolic elements that define the structure of the code (e.g., {\tt =, \{, ;,}). 
    
\noindent(ii) \textbf{Signature}: Tokens defining the declaration and parameters of methods, critical for understanding the interface and functionality of code components (e.g., {\tt calculate(int x)}).
    
\noindent(iii) \textbf{Invocation}: Tokens related to function or method calls, capturing interactions and dependencies within the code.
    
\noindent(iv) \textbf{Identifier}: Tokens that serve as variable names, class names, or other user-defined labels, are essential for understanding program semantics.

\noindent(v) \textbf{Structure}:  Tokens associated to loops, conditional, and other flow-control statements, which dictate the logical behavior of the program (e.g., {\tt if, for, class}). 


\section{Statistics of the Code Compression Dataset for Compressor Training}\label{sec:stat}
The constructed dataset includes original code examples paired with compressed code examples, with $\tau_{code}$ ranging from 0.1 to 0.9. As shown in Table~\ref{tab:stat}, the total number of training samples is nine times the number of parsable code examples in the knowledge base, reflecting the nine distinct $\tau_{code}$ values. The dataset is split into training, validation, and test sets in an 8:1:1 ratio.
\begin{table}[]
\centering
\caption{Statistics of the Code Compression Dataset for Compressor Training.}
\label{tab:stat}
\resizebox{0.5\textwidth}{!}{%
\begin{tabular}{ccc}
\hline
Task       & Total Samples & Split(Training/Test/Validation) \\ \hline
Assertion  & 70433*9       & 80\%/10\%/10\%                  \\
Bugs2Fix   & 48903*9       & 80\%/10\%/10\%                  \\
Suggestion & 89014*9       & 80\%/10\%/10\%                  \\ \hline
\end{tabular}
}
\end{table}

\section{More Results of the Out-of-Domain Capabilities of \ourtool}\label{sec:out}

\begin{table*}[]
\scriptsize
\centering
\caption{Cross-task Results: \textbf{Bold} font indicates the in-task scenario.}
\begin{tabular}{lllcccccc}
\hline
\multicolumn{3}{l}{\textbf{Compressor}} & \multicolumn{2}{c}{(a): \taskone}      & \multicolumn{2}{c}{(b): \tasktwo}   & \multicolumn{2}{c}{(c): \taskthree} \\ \hline
(a)    & (b)    & (c)    & \textbf{$\tau_{code}(\%)$} & \textbf{Exact Match(\%)} & \textbf{$\tau_{code}(\%)$} & \textbf{CodeBleu(\%)} & $\tau_{code}(\%)$      & \textbf{CodeBleu(\%)}   \\ \hline
           \Checkmark       &          &          & \textbf{31.5}       & \textbf{42.1}            & 33.2                & 50.3                  & 19.8            & 13.4                    \\
                  &    \Checkmark        &       & 28.4                & 41.9                     & \textbf{30.0}       & \textbf{61.9}         & 25.1            & 15.9                    \\
                  &          &    \Checkmark      & 34.2                & 34.1                     & 39.9                & 43.6                  & \textbf{32.2}   & \textbf{23.7}           \\ \hline
\end{tabular}
\label{tab:cross}
\end{table*}

To evaluate the out-of-domain effectiveness of our compressors, we performed cross-task experiments using $\mathcal{LM_C}$ trained on individual downstream tasks and tested on the other two out-of-domain tasks. Notably, task-specific special tokens (e.g., <ASSERTION>) were not used in these experiments. Table \ref{tab:cross} summarizes the results of these cross-task evaluations.

For example, in the first row, compressors trained on the Assertion Generation task are applied to compress code examples from Bugs2Fix and Code Suggestion. The Assertion Generation in-task compressor achieves a CodeBleu score of 50.3\% with a $\tau_{code}$ of 33.2\% on Bugs2Fix, compared to its in-task performance of 61.9\% CodeBleu and a $\tau_{code}$ of 30.0\% . While the compressor demonstrates slightly less precision in achieving the desired compression ratio and exhibits degradation in performance, it remains competitive against other baselines, such as the best-performing LLMlingua2 with a CodeBleu score of 48.1\%.




\section{More Results of Impact of the $\tau_{code}$ on the Effectiveness of \ourtool}\label{sec:appendix_ratio}



\begin{table*}[]
\scriptsize
\centering
\caption{Results with varying numbers of code snippets and $\tau_{\text{code}}$ settings of the compressor on the studied tasks. The overall compression ratio $\tau$ is achieved by compressing code snippets with $\tau_{\text{code}}$. }
% \sw{why oracle cannot control very precisely? for instance, compression ratio 0.5, oracle for assertion generation is 0.54 and 0.53?,fixed}}
\label{tab:setting}
\begin{tabular}{lccccccccc}
\hline
\multicolumn{1}{c}{\textbf{}}                       & \multicolumn{3}{c}{\textbf{Assertion Generation}}                          & \multicolumn{3}{c}{\textbf{\tasktwo}}                       & \multicolumn{3}{c}{\textbf{\taskthree}}                     \\
\multicolumn{1}{c}{\textbf{Approach}}               & \textbf{$\tau_{code}(\%)$}   & \textbf{$\tau(\%)$}  & \textbf{Exact Match(\%)} & \textbf{$\tau_{code}(\%)$}   & \textbf{$\tau(\%)$}  & \textbf{CodeBleu(\%)}    & \textbf{$\tau_{code}(\%)$}   & \textbf{$\tau(\%)$}  & \textbf{CodeBleu(\%)}    \\ \hline
\multicolumn{10}{l}{{\color[HTML]{000000} \textit{\textbf{1-shot, Setting $\tau_{code}$-\textcolor{blue}{0.1}}}}}                                                                                                                                                                                            \\
\ourtool w/o Copy                    & \textcolor{blue}{14.2} & 6.2                  & 48.3                 & \textcolor{blue}{29.5} & 25.9                 & 59.0                 & \textcolor{blue}{8.1}  & 5.2                  & 23.4                 \\
\ourtool                             & \textcolor{blue}{13.3} & 7.2                  & 50.1                 & \textcolor{blue}{11.5} & 9.1                  & 68.4                 & \textcolor{blue}{9.9}  & 8.9                  & 24.4                 \\
Oracle                                              & \textcolor{blue}{10.0} & 8.8                  & 49.8                 & \textcolor{blue}{10.0} & 8.1                  & 78.5                 & \textcolor{blue}{10.0} & 8.4                  & 24.5                 \\ \hline
\multicolumn{10}{l}{\textit{\textbf{1-shot, Setting $\tau_{code}$-\textcolor{blue}{0.3}}}}                                                                                                                                                                                                                   \\
\ourtool w/o Copy                    & \textcolor{blue}{33.4} & 28.5                 & 40.9                 & \textcolor{blue}{28.3} & 26.2                 & 56.7                 & \textcolor{blue}{24.7} & 21.7                 & 20.5                 \\
\ourtool                             & \textcolor{blue}{31.5} & 29.7                 & 42.1                 & \textcolor{blue}{30.0}   & 27.4                 & 61.9                 & \textcolor{blue}{32.2} & 27.5                 & 23.7                 \\
Oracle                                              & \textcolor{blue}{30.0} & 27.4                 & 46.2                 & \textcolor{blue}{30.0} & 20.2                 & 66.8                 & \textcolor{blue}{30.0} & 26.1                 & 23.8                 \\ \hline
\multicolumn{10}{l}{\textit{\textbf{1-shot, Setting $\tau_{code}$-\textcolor{blue}{0.5}}}}                                                                                                                                                                                                                   \\
\ourtool w/o Copy                    & \textcolor{blue}{44.1} & 38.3                 & 40.3                 & \textcolor{blue}{31.0} & 21.1                 & 56.5                 & \textcolor{blue}{42.9} & 34.4                 & 23.5                 \\
\ourtool                             & \textcolor{blue}{49.4} & 44.9                 & 45.1                 & \textcolor{blue}{47.8} & 41.4                 & 68.7                 & \textcolor{blue}{57.5} & 42.9                 & 22.1                 \\
Oracle                                              & \textcolor{blue}{50.0} & 45.2                 & 42.1                 & \textcolor{blue}{50.0} & 43.9                 & 67.1                 & \textcolor{blue}{50.0} & 41.9                 & 23.1                 \\ \hline
\textit{\textbf{1-shot}}                            &                              &                      &                      &                              &                      &                      &                              &                      &                      \\
w/o compression                                     & 0.0                          & 0.0                  & 50.5                 & 0.0                          & 0.0                  & 81.4                 & 0.0                          & 0.0                  & 24.7                 \\ \hline
\multicolumn{10}{l}{\textit{\textbf{3-shot, Setting $\tau_{code}$-\textcolor{blue}{0.5}}}}                                                                                                                                                                                                                   \\
\ourtool w/o Copy                    & \textcolor{blue}{44.2} & 39.4                 & 47.2                 & \textcolor{blue}{33.2} & 24.5                 & 67.2                 & \textcolor{blue}{41.2} & 39.1                 & 23.5                 \\
\ourtool                             & \textcolor{blue}{49.3} & 43.8                 & 48.9                 & \textcolor{blue}{50.9} & 42.3                 & 68.0                 & \textcolor{blue}{57.3} & 42.3                 & 24.4                 \\
Oracle                                              & \textcolor{blue}{50.0} & 41.1                 & 52.6                 & \textcolor{blue}{50.0} & 41.9                 & 68.9                 & \textcolor{blue}{50.0} & 42.5                 & 24.6                 \\ \hline
\textit{\textbf{3-shot}}                            & \multicolumn{1}{l}{}         & \multicolumn{1}{l}{} & \multicolumn{1}{l}{} & \multicolumn{1}{l}{}         & \multicolumn{1}{l}{} & \multicolumn{1}{l}{} & \multicolumn{1}{l}{}         & \multicolumn{1}{l}{} & \multicolumn{1}{l}{} \\
w/o compression                                     & 0.0                          & 0.0                  & 55.6                 & 0.0                          & 0.0                  & 85.2                 & 0.0                          & 0.0                  & 24.9                 \\ \hline
\textit{\textbf{5-shot, Setting $\tau_{code}$-\textcolor{blue}{0.5}}} &                              &                      &                      &                              &                      &                      &                              &                      &                      \\
\ourtool w/o Copy                    & \textcolor{blue}{42.9} & 41.3                 & 39.8                 & \textcolor{blue}{30.9} & 28.5                 & 62.4                 & \textcolor{blue}{38.5} & 49.9                 & 24.1                 \\
\ourtool                             & \textcolor{blue}{48.7} & 45.2                 & 49.9                 & \textcolor{blue}{52.1} & 45.1                 & 70.9                 & \textcolor{blue}{49.9} & 41.3                 & 25.1                 \\
Oracle                                              & \textcolor{blue}{50.0} & 43.9                 & 55.2                 & \textcolor{blue}{50.0} & 42.2                 & 74.4                 & \textcolor{blue}{50.0} & 41.6                 & 25.3                 \\ \hline
\textit{\textbf{5-shot}}                            & \multicolumn{1}{l}{}         & \multicolumn{1}{l}{} & \multicolumn{1}{l}{} & \multicolumn{1}{l}{}         & \multicolumn{1}{l}{} & \multicolumn{1}{l}{} & \multicolumn{1}{l}{}         & \multicolumn{1}{l}{} & \multicolumn{1}{l}{} \\
w/o compression                                     & 0.0                          & 0.0                  & 57.8                 & 0.0                          & 0.0                  & 86.4                 & 0.0                          & 0.0                  & 25.6                 \\ \hline
\end{tabular}
\end{table*}







% We aim to explore the trade-off between incorporating a greater number of highly compressed code examples vs. fewer but more detailed examples. The Oracle represents the performance on downstream tasks using the referenced compressed prompt derived from the training set constructed in Sec \ref{sec:trainingset}. 
% \sw{I think this should be a individual RQ} 
In Table~\ref{tab:setting}, we present results with varying $\tau_{code}$ settings. For compression ratio control, our framework enables the configuration of $\tau_{code}$ as an input to $\mathcal{LM_C}$, allowing it to adaptively compress code examples to match the specified ratio and thereby control the overall $\tau$ of the prompt. However, when using the same configuration and setup with the original CodeT5 structure (w/o copy), the framework struggles to effectively learn the token removal priority. Consequently, the output occasionally deviates from the specified ratio settings, as observed in the \tasktwo 1-shot experiment with $\tau_{code}=0.1$. This underscores the critical role of the copy mechanism in ensuring compliance with ratio settings.

% \sw{ I guess in general, keeping more tokens suggests better performance as figure 5 indicates. to discuss the trade-off is more interesting. so I would remove this to appendix and focus on discussing the trader-off between higher compression rate with fewer examples vs lower compression rate but with more examples (presented in the next paragraph), which is more interesting.}
Regarding the quality of generation in the end tasks, the performance varies with different $\tau_{code}$ when a single retrieved example is included. A lower $\tau_{code}$ of 0.1 achieves the highest quality, approximating complete examples. However, lower $\tau_{code}$ values do not always yield better results. For example, in the 1-shot setting on \taskone, $\tau_{code}=0.3$ achieves an exact match rate of 42.1\%, which is lower than the 45.1\% obtained with $\tau_{code}=0.5$ on the same task. 



\section{More Results of Impact of the Number of Shots on the Effectiveness of \ourtool}\label{sec:appendix_shot}

We also present results under varying shot settings. The findings indicate that increasing the number of compressed examples improves performance, consistent with prior observations for complete examples \citep{he2024}. Notably, this improvement extends to compressed examples. For instance, in \taskthree, increasing the number of $\tau_{code}=0.5$ compressed examples from 1 to 5 raises the CodeBleu score from 22.1 to 25.1, highlighting the advantage of incorporating multiple compressed examples into the prompt.

\begin{table*}[t]
\centering
\caption{A case study examining \model's response to a query about ``\emph{graders}'' in OpenAI's 12-day edu    cational series demonstrates its video retrieval capabilities and knowledge synthesis.}
\label{tab:case study}
\small
\begin{tabularx}{\textwidth}{X X}

\toprule
\textbf{Query:} Explain the purpose and functionality of 'graders' in the context of reinforcement fine-tuning.\\

\midrule
\textbf{Input Videos:} \href{https://www.youtube.com/playlist?list=PLOXw6I10VTv9lin5AzsHAHCTrC7BdVdEM}{12 Days of OpenAI} (A total of 12 videos) \\

\midrule

\textbf{\model\ Response:} 

\texttt{\#\#\# Understanding Gradation in Reinforcement Fine-Tuning}

\texttt{**Purpose of Graders**}

\texttt{Graders are integral components in the process of reinforcement fine-tuning, mainly utilized to evaluate and improve the performance of machine learning models...} (Omitted for brevity)

\texttt{**Functionality of Graders**}

\texttt{\colorbox{lightgreen}{The grading system operates by comparing a model's outputs against those of}}

\texttt{\colorbox{lightgreen}{verified correct responses. The mechanics of this assessment involve assigning}}

\texttt{\colorbox{lightgreen}{scores to the model's outputs based on their correctness. These scores range}}

\texttt{\colorbox{lightgreen}{from zero to one, where:}}


\texttt{\colorbox{lightblue}{- A score of **0** indicates that the model's answer is completely incorrect.}}

\texttt{\colorbox{lightblue}{- A score of **1** signals that the answer is fully correct.}}

\texttt{- Scores can also be fractional, allowing for partial credit. For instance, \colorbox{lightpink}{a score of **0.7** may denote that the model's suggestion was }}
\texttt{\colorbox{lightpink}{correct but not the highest in a ranked list of possible answers.}} 

\\

\texttt{The process includes:}

\texttt{1. **Input Comparison**: The grader takes both the model’s output and the correct answer, analyzing them to derive a score.}

\texttt{2. **Graded Outputs**: This interaction allows for a dynamic reflection of the model's performance, where the scores directly influence how the model is fine-tuned in subsequent training iterations.}

\texttt{3. **Feedback Loop**: The scores obtained from graders provide feedback that can be used to adjust learning parameters and improve the model’s future responses. This highlights the feedback mechanism that is crucial for reinforcement learning.}

\\

(The remaining content are omitted for brevity.)

\\

\midrule
\textbf{Retrieved Sources by \model:}

\model\ accurately retrieved four continuous video segments from the OpenAI's 12-day show, ranging from 10:00 to 12:00 in Day 2 of the video "\href{https://www.youtube.com/watch?v=yCIYS9fx56U}{Reinforcement Fine-Tuning}." Here, we highlight key moments relevant to the detailed content in the answer. From left to right, these are retrieved moments at timestamps \colorbox{lightgreen}{10:35}, \colorbox{lightblue}{10:39}, and \colorbox{lightpink}{11:10}, which provide informative insights that help \model\ give a comprehensive answer to the query.

\\

\begin{tabular}{ccc}
    {\includegraphics[width=0.3\textwidth]{figs/openai-1.png}} &
    {\includegraphics[width=0.3\textwidth]{figs/openai-2.png}} &
    {\includegraphics[width=0.3\textwidth]{figs/openai-3.png}} \\
\end{tabular}

\\

\bottomrule

\end{tabularx}
\vspace{-0.2in}
\end{table*}


% \section{Dataset Split}
% We follow the original split of the dataset into Train, Validation, and Test partitions. The training partition functions as our knowledge base for example retrieval. Due to computational resource constraints, we randomly sample 2,000 instances from both the validation and test sets for our experiments. The sampled validation set is used to study example removal priority, while the sampled test set serves to evaluate our framework's performance.

% \begin{table}[]
% \begin{tabular}{c|ccc}
% \hline
% Task                      & Knowledge Base & Test & Val  \\ \hline
% \textbackslash{}taskone   & 282, 932       & 2000 & 2000 \\
% \textbackslash{}tasktwo   & 273, 856       & 2000 & 2000 \\
% \textbackslash{}taskthree & 721, 021       & 2000 & 2000 \\ \hline
% \end{tabular}
% \end{table}

% \begin{table*}[ht]
% \centering
% \begin{tabular}{lccc}
% \hline
% \textbf{Task}             & \textbf{Knowledge Base (Parsable)} & \textbf{Test} & \textbf{Val} \\ \hline
% \textbf{\taskone}          & 144,112 (70,433)                  & 18,027            & 18,816                  \\
% \textbf{\tasktwo}          & 52,364 (48,903)                   & 6,545             & 6,546                   \\
% \textbf{\taskthree}        & 128,724 (89,014)                  & 10,727            & 5,149                   \\ \hline
% \end{tabular}
% \caption{Dataset statistics for different tasks.}
% \label{tab:task_statistics}
% \end{table*}