\subsection{RQ2: Trade-off between $\tau_{code}$ and Shots}\label{sec:rq2}

The objective of prompt compression is to minimize the number of tokens fed to the $\mathcal{BLM}$, while preserving acceptable generation quality. Given a fixed token budget, a trade-off arises between including fewer, less-compressed examples and more highly-compressed ones. Figure \ref{fig:tradeoff} illustrates this balance. \textbf{In general, appending fewer examples, with each example allocated more tokens, achieves better performance than increasing the number of shots while allocating fewer tokens per shot.} For instance, in \taskone, with a token budget of 500, a single example compressed at $\tau_{code}=0.1$ outperforms three examples compressed at $\tau_{code}=0.7$. Additionally, to achieve a fixed performance level, choosing fewer shots with a lower $\tau_{code}$ is more cost-effective, balancing token efficiency and performance.

% Please add the following required packages to your document preamble:
% \usepackage[table,xcdraw]{xcolor}
% Beamer presentation requires \usepackage{colortbl} instead of \usepackage[table,xcdraw]{xcolor}


\begin{figure}
    \includegraphics[width=1\linewidth]{figures/tf.pdf}
    \vspace{-0.1in}
    \caption{Trade-off between keeping more tokens in a single example or including more examples.}
    \label{fig:tradeoff}
    \vspace{-0.3cm}
\end{figure}




\subsection{RQ3: Compression Ratio Control}\label{sec:rq2}


Our LM-based compressors utilize an extended vocabulary and accept $\tau_{code}$ as input, enabling adaptive compression of code examples to meet the desired ratio. Figure \ref{fig:copyT5} illustrates the relation between the specified $\tau_{code}$ and the actual achieved values. The dotted line (Oracle) represents the standard outcome, and \ourtool closely aligns with this benchmark.
In contrast, compressors based on the original CodeT5 architecture (w/o the copy) struggle to produce outputs that match the desired ratio. Table~\ref{tab:setting} also provides specific results under varying $\tau_{code}$ configurations, further demonstrating the effectiveness of \ourtool and the critical role of the copy mechanism in achieving accurate compression ratio control.

\begin{figure}
\subfigure[Assertion Generation ~(b) Bugs2Fix ~ (c) Code Suggestion]
    {\includegraphics[width=1\linewidth]{figures/ratiocontrol.pdf}}
    \vspace{-0.2in}
    \caption{Compression ratio control.}\label{fig:template}
    \vspace{-0.2cm}
\end{figure}
% This discrepancy is evident in the \tasktwo 1-shot experiment with $\tau_{code}=0.1$, 
