\section{Related Work}
There has been a lot of similar research already in this field. With the introduction of VAR which uses different cameras at different angles to draw parallel lines is one of the advanced and state-of-the-art tech. I came up with a novel idea to implement the Offside detection. I have majorly referred two papers while working on this project and became my inspiration for the new idea. Vision based offline Vision Based Dynamic Offside Line Marker for Soccer Games; \cite{muthuraman2018vision} explains the step by step implementation of using Image processing techniques to detect lines on the Soccer pitch and then drawing parallel lines in 3D space. Although, this implementation is based on the FIFA Game dataset which is a kind of simulation dataset. Along with this, the authors used the players "Bodies" converting into blobs to draw parallel lines. Since the offside rule is very strict about which body parts are considered as "Offside", this paper does not use those specific body parts. Computer Vision based Offside Detection in soccer; \cite{10.1145/3422844.3423055}, uses the Keypose estimation to detect the only relevant body parts that are used in the Offside rule. This paper implements the offside detection on the body parts of the players on a real dataset for Images. These 2 papers inspired me to come up with a new idea of doing the same Offside detection on Real Video dataset of a broadcasted game and implementing keypose detection to make it accurate.