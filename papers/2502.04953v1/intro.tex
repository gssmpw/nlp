\section{Introduction}
\label{sec:intro}
Since 2017, more and more vulnerabilities have been reported every year on the U.S. National Vulnerability Database. The severity of the vulnerabilities also becomes higher, such as the Remote Code Execution (RCE) vulnerability Log4Shell (CVE-2021-44228)\footnote{\url{https://nvd.nist.gov/vuln/detail/CVE-2021-44228}} in Apache Log4j2, causing a huge interruption for the web services in the whole internet.% the Equifax data breach~\footnote{\url{}}

Due to this fact, researchers and practitioners propose novel techniques and develop automated tools to deal with a large volume of vulnerabilities. Security vulnerability research has recently gained ground not only in the academic venues for security but also in software engineering and artificial intelligence. This research area is devoted to the community with a plethora of techniques that can help with vulnerability detection~\cite{scandariato2014predicting, li2018vuldeepecker, chakraborty2021deep, fu2022linevul}, vulnerability assessment~\cite{le2021deepcva, upadhyay2020scada, han2017learning, du2019leopard}, vulnerability exploitation~\cite{iannone2021toward, avgerinos2014automatic, brumley2008automatic}, and vulnerability repair~\cite{pearce2023examining, fu2022vulrepair, bui2024apr4vul}.

In contrast with other techniques applied to vulnerabilities, vulnerability exploitation and vulnerability repair problems are much harder to solve. Vulnerability exploitation often requires more sophisticated techniques to reach the exploitable state of the vulnerability to synthesize a confident ``exploit'' or ``Proof of Concept''~\cite{avgerinos2014automatic}. Vulnerability repair often applies code transformation or code modification strategies and tries to eliminate the vulnerabilities while maintaining functional features. Repair techniques often require an oracle (e.g., test cases or exploits) to check the presence of the vulnerabilities against the patch suggestions from the repair tool~\cite{monperrus2018automatic}. However, in practice, vulnerability exploits are often not publicly available and require security expertise to create. Obtaining them from the vulnerability exploitation techniques can be another potential option.

The existing literature survey studies up to now~\cite{shahriar2009automatic, felderer2016model, felderer2016security, aydos2022security, sommer2023survey, ahsan2024systematic} have focused on a specific subset of exploitation techniques, for example, Felderer et al.~\cite{felderer2016security} conducted a survey on the state-of-the-art techniques for security testing. Moreover, they have not been performed in a systematic way. From this point, we look at the literature and systematically survey the techniques that can generate exploits for security vulnerabilities, broadening from automated exploiting generation, security testing, fuzzing, and other approaches.

\textbf{The scope of our survey.} We aim to search and review the academic works (peer-reviewed) that introduce novel techniques targeting generating exploits for security vulnerabilities. We do not focus on a specific family of techniques. For example, we may not cover all the primary fuzzing works in the literature. They are known for finding vulnerabilities, however many of them are not designed to find specific vulnerabilities but rather to find ``crashes''. We also focus only on the vulnerabilities at the software application level. Hardware vulnerabilities, network malware, and other categories of vulnerabilities are outside of the scope of this work.

\textbf{The contributions of this study include the following:}
\begin{itemize}
	\item A list of academic works on the exploit generation for software vulnerabilities
    \item An analysis of characteristics of exploit generation techniques and their availability and usability in practice
\end{itemize}

\textbf{Paper Outline.}
In Section \ref{sec:background}, we present essential terminology (\ref{subsec:definitions}) and related works (\ref{subsec:related}). 
Section \ref{sec:methodology} describes the methodology we adopted to conduct this literature review. 
We start from the goal of the survey and the research questions we considered (\ref{sec:research-questions}), then we move on to the collection of papers \ref{sec:paper-collection}; 
then, we list the inclusion and exclusion criteria we employed to select relevant papers (\ref{sec:relevance-criteria}) and, finally, how we reviewed and categorized papers (\ref{sec:paper-organization}).
In Section \ref{sec:results}, we present the results of our survey and discuss the answers to each of our research questions.
Section \ref{sec:limitations} mentions possible limitations of our work.
Finally, in Section \ref{sec:conclusions}, we draw our conclusion about the work done.