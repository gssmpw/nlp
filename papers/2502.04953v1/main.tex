\PassOptionsToPackage{dvipsnames}{xcolor}
\documentclass[manuscript,screen]{acmart}

%% \BibTeX command to typeset BibTeX logo in the docs
\AtBeginDocument{%
  \providecommand\BibTeX{{%
    Bib\TeX}}}

\setcopyright{acmlicensed}
\copyrightyear{2018}
\acmYear{2018}
\acmDOI{XXXXXXX.XXXXXXX}

%%%%%%%%%%%%%%%%%%%%%%%%%%%%%%%%%%%%%%%%%%%%%%%%%%%%%%%%%%%%%%%%%%% BEGIN PACKAGES
\usepackage{paralist}
\usepackage{xspace}
\usepackage{balance}
\usepackage{lipsum}
\usepackage{hyperref}
\usepackage{listings}

\usepackage{tikz}
\usetikzlibrary{arrows,shapes,positioning,shadows,trees}

\usepackage[dvipsnames]{xcolor}
\usepackage{pifont}
\usepackage{multirow}
\usepackage{hyperref}
\usepackage{ifthen}
%%%%%%%%%%%%%%%%%%%%%%%%%%%%%%%%%%%%%%%%%%%%%%%%%%%%%%%%%%%%%%%%%%% END PACKAGES

%%%%%%%%%%%%%%%%%%%%%%%%%%%%%%%%%%%%%%%%%%%%%%%%%%%%%%%%%%%%%%%%%%% BEGIN COMMAND DEFINITIONS
\newcommand{\cuong}[1]{\colorbox{OrangeRed}{{\scriptsize\bfseries\color{white}BQC:}} {\color{OrangeRed}\bfseries{#1}}}
\newcommand{\ema}[1]{\colorbox{OrangeRed}{{\scriptsize\bfseries\color{white}EI:}} {\color{OrangeRed}\bfseries{#1}}}
\newcommand{\mc}[1]{\colorbox{Blue}{{\scriptsize\bfseries\color{white}MC:}} {\color{Blue}\bfseries{#1}}}

\long\def\longcaption#1#2{\centering\begin{minipage}{#1}\vspace{-0.7\baselineskip}\scriptsize\noindent\emph{#2}\end{minipage}}
\long\def\longcaptionfig#1#2{\centering\begin{minipage}{#1}\vspace{-0.1\baselineskip}\scriptsize\noindent\emph{#2}\end{minipage}}

\newboolean{deliverable}
\setboolean{deliverable}{true}
%%%%%%%%%%%%%%%%%%%%%%%%%%%%%%%%%%%%%%%%%%%%%%%%%%%%%%%%%%%%%%%%%%% END COMMAND DEFINITIONS


%%%%%%%%%%%%%%%%%%%%%%%%%%%%%%%%%%%%%%%%%%%%%%%%%%%%%%%%%%%%%%%%%%% BEGIN DATA DEFINITIONS
\tikzset{ % Configuration for Taxonomy figure
	basic/.style  = {draw, text width=2cm, drop shadow, font=\sffamily, rectangle},
	root/.style   = {basic, rounded corners=2pt, thin, align=center,
		fill=green!0},
	level 2/.style = {basic, rounded corners=6pt, thin,align=center, fill=green!0,
		text width=8em},
	level 3/.style = {basic, thin, align=left, fill=pink!0, text width=6.5em}
}
%%%%%%%%%%%%%%%%%%%%%%%%%%%%%%%%%%%%%%%%%%%%%%%%%%%%%%%%%%%%%%%%%%% END DATA DEFINITIONS

\begin{document}

\title{A Systematic Literature Review on Automated Exploit\\and Security Test Generation}

\author{Quang-Cuong Bui}
\orcid{0000-0001-6072-9213}
\affiliation{%
  \institution{Hamburg University of Technology}
  \city{Hamburg}
  \country{Germany}
}
\email{cuong.bui@tuhh.de}

\author{Emanuele Iannone}
\orcid{0000-0001-7489-9969}
\affiliation{%
  \institution{Hamburg University of Technology}
  \city{Hamburg}
  \country{Germany}
}
\email{emanuele.iannone@tuhh.de}

\author{Maria Camporese}
\orcid{0009-0009-1178-0210}
\affiliation{%
  \institution{University of Trento}
  \city{Trento}
  \country{Italy}
}
\email{maria.camporese@unitn.it}

\author{Torge Hinrichs}
\orcid{0000-0001-7489-3540}
\affiliation{%
  \institution{Hamburg University of Technology}
  \city{Hamburg}
  \country{Germany}
}
\email{torge.hinrichs@tuhh.de}

\author{Catherine Tony}
\orcid{0000-0002-9916-4456}
\affiliation{%
  \institution{Hamburg University of Technology}
  \city{Hamburg}
  \country{Germany}
}
\email{catherine.tony@tuhh.de}

\author{László Tóth}
\affiliation{%
  \institution{University of Szeged}
  \city{Szeged}
  \country{Hungary}
}
\email{premissa@inf.u-szeged.hu}

\author{Fabio Palomba}
\orcid{0000-0001-9337-5116}
\affiliation{%
  \institution{University of Salerno}
  \city{Salerno}
  \country{Italy}
}
\email{fpalomba@unisa.it}

\author{Péter Hegedűs}
\orcid{0000-0003-4592-6504}
\affiliation{%
  \institution{University of Szeged, FrontEndART Ltd.}
  \city{Szeged}
  \country{Hungary}
}
\email{peter.hegedus@frontendart.com}

\author{Fabio Massacci}
\orcid{0000-0002-1091-8486}
\affiliation{%
  \institution{University of Trento}
  \city{Trento}
  \country{Italy}
}
\email{fabio.massacci@unitn.it}

\author{Riccardo Scandariato}
\orcid{0000-0003-3591-7671}
\affiliation{%
  \institution{Hamburg University of Technology}
  \city{Hamburg}
  \country{Germany}
}
\email{riccardo.scandariato@tuhh.de}

\renewcommand{\shortauthors}{Bui et al.}

\begin{abstract}
The exploit or the Proof of Concept of the vulnerability plays an important role in developing superior vulnerability repair techniques, as it can be used as an oracle to verify the correctness of the patches generated by the tools. However, the vulnerability exploits are often unavailable and require time and expert knowledge to craft. Obtaining them from the exploit generation techniques is another potential solution.
The goal of this survey is to aid the researchers and practitioners in understanding the existing techniques for exploit generation through the analysis of their characteristics and their usability in practice. We identify a list of exploit generation techniques from literature and group them into four categories: automated exploit generation, security testing, fuzzing, and other techniques. Most of the techniques focus on the memory-based vulnerabilities in C/C++ programs and web-based injection vulnerabilities in PHP and Java applications. We found only a few studies that publicly provided usable tools associated with their techniques.
\end{abstract}

%% Code below generated by http://dl.acm.org/ccs.cfm.
\begin{CCSXML}
<ccs2012>
   <concept>
       <concept_id>10002978.10003022.10003026</concept_id>
       <concept_desc>Security and privacy~Web application security</concept_desc>
       <concept_significance>500</concept_significance>
       </concept>
   <concept>
       <concept_id>10002978.10003022.10003023</concept_id>
       <concept_desc>Security and privacy~Software security engineering</concept_desc>
       <concept_significance>500</concept_significance>
       </concept>
   <concept>
       <concept_id>10011007.10011074.10011099.10011102</concept_id>
       <concept_desc>Software and its engineering~Software defect analysis</concept_desc>
       <concept_significance>500</concept_significance>
       </concept>
 </ccs2012>
\end{CCSXML}

\ccsdesc[500]{Security and privacy~Web application security}
\ccsdesc[500]{Security and privacy~Software security engineering}
\ccsdesc[500]{Software and its engineering~Software defect analysis}

\keywords{Software vulnerability, Exploit generation}

%\received{20 February 2007}
%\received[revised]{12 March 2009}
%\received[accepted]{5 June 2009}

\maketitle

\section{Introduction}


\begin{figure}[t]
\centering
\includegraphics[width=0.6\columnwidth]{figures/evaluation_desiderata_V5.pdf}
\vspace{-0.5cm}
\caption{\systemName is a platform for conducting realistic evaluations of code LLMs, collecting human preferences of coding models with real users, real tasks, and in realistic environments, aimed at addressing the limitations of existing evaluations.
}
\label{fig:motivation}
\end{figure}

\begin{figure*}[t]
\centering
\includegraphics[width=\textwidth]{figures/system_design_v2.png}
\caption{We introduce \systemName, a VSCode extension to collect human preferences of code directly in a developer's IDE. \systemName enables developers to use code completions from various models. The system comprises a) the interface in the user's IDE which presents paired completions to users (left), b) a sampling strategy that picks model pairs to reduce latency (right, top), and c) a prompting scheme that allows diverse LLMs to perform code completions with high fidelity.
Users can select between the top completion (green box) using \texttt{tab} or the bottom completion (blue box) using \texttt{shift+tab}.}
\label{fig:overview}
\end{figure*}

As model capabilities improve, large language models (LLMs) are increasingly integrated into user environments and workflows.
For example, software developers code with AI in integrated developer environments (IDEs)~\citep{peng2023impact}, doctors rely on notes generated through ambient listening~\citep{oberst2024science}, and lawyers consider case evidence identified by electronic discovery systems~\citep{yang2024beyond}.
Increasing deployment of models in productivity tools demands evaluation that more closely reflects real-world circumstances~\citep{hutchinson2022evaluation, saxon2024benchmarks, kapoor2024ai}.
While newer benchmarks and live platforms incorporate human feedback to capture real-world usage, they almost exclusively focus on evaluating LLMs in chat conversations~\citep{zheng2023judging,dubois2023alpacafarm,chiang2024chatbot, kirk2024the}.
Model evaluation must move beyond chat-based interactions and into specialized user environments.



 

In this work, we focus on evaluating LLM-based coding assistants. 
Despite the popularity of these tools---millions of developers use Github Copilot~\citep{Copilot}---existing
evaluations of the coding capabilities of new models exhibit multiple limitations (Figure~\ref{fig:motivation}, bottom).
Traditional ML benchmarks evaluate LLM capabilities by measuring how well a model can complete static, interview-style coding tasks~\citep{chen2021evaluating,austin2021program,jain2024livecodebench, white2024livebench} and lack \emph{real users}. 
User studies recruit real users to evaluate the effectiveness of LLMs as coding assistants, but are often limited to simple programming tasks as opposed to \emph{real tasks}~\citep{vaithilingam2022expectation,ross2023programmer, mozannar2024realhumaneval}.
Recent efforts to collect human feedback such as Chatbot Arena~\citep{chiang2024chatbot} are still removed from a \emph{realistic environment}, resulting in users and data that deviate from typical software development processes.
We introduce \systemName to address these limitations (Figure~\ref{fig:motivation}, top), and we describe our three main contributions below.


\textbf{We deploy \systemName in-the-wild to collect human preferences on code.} 
\systemName is a Visual Studio Code extension, collecting preferences directly in a developer's IDE within their actual workflow (Figure~\ref{fig:overview}).
\systemName provides developers with code completions, akin to the type of support provided by Github Copilot~\citep{Copilot}. 
Over the past 3 months, \systemName has served over~\completions suggestions from 10 state-of-the-art LLMs, 
gathering \sampleCount~votes from \userCount~users.
To collect user preferences,
\systemName presents a novel interface that shows users paired code completions from two different LLMs, which are determined based on a sampling strategy that aims to 
mitigate latency while preserving coverage across model comparisons.
Additionally, we devise a prompting scheme that allows a diverse set of models to perform code completions with high fidelity.
See Section~\ref{sec:system} and Section~\ref{sec:deployment} for details about system design and deployment respectively.



\textbf{We construct a leaderboard of user preferences and find notable differences from existing static benchmarks and human preference leaderboards.}
In general, we observe that smaller models seem to overperform in static benchmarks compared to our leaderboard, while performance among larger models is mixed (Section~\ref{sec:leaderboard_calculation}).
We attribute these differences to the fact that \systemName is exposed to users and tasks that differ drastically from code evaluations in the past. 
Our data spans 103 programming languages and 24 natural languages as well as a variety of real-world applications and code structures, while static benchmarks tend to focus on a specific programming and natural language and task (e.g. coding competition problems).
Additionally, while all of \systemName interactions contain code contexts and the majority involve infilling tasks, a much smaller fraction of Chatbot Arena's coding tasks contain code context, with infilling tasks appearing even more rarely. 
We analyze our data in depth in Section~\ref{subsec:comparison}.



\textbf{We derive new insights into user preferences of code by analyzing \systemName's diverse and distinct data distribution.}
We compare user preferences across different stratifications of input data (e.g., common versus rare languages) and observe which affect observed preferences most (Section~\ref{sec:analysis}).
For example, while user preferences stay relatively consistent across various programming languages, they differ drastically between different task categories (e.g. frontend/backend versus algorithm design).
We also observe variations in user preference due to different features related to code structure 
(e.g., context length and completion patterns).
We open-source \systemName and release a curated subset of code contexts.
Altogether, our results highlight the necessity of model evaluation in realistic and domain-specific settings.






Our work draws heavily from the literature on semiparametric inference and double machine learning~\citep{robins1994estimation,robins1995semiparametric,tsiatis2006semiparametric,chernozhukov2018double}. In particular, our estimator is an optimal combination of several Augmented Inverse Probability Weighting~(\aipw) estimators, whose outcome regressions are replaced with foundation models. Importantly, the standard $\aipw$ estimator, which relies on an outcome regression estimated using experimental data alone, is also included in the combination. This approach allows \ours~to significantly reduce finite sample (and potentially asymptotic) variance while attaining the semiparametric \emph{efficiency bound}---the smallest asymptotic variance among all consistent and asymptotically normal estimators of the average treatment effect---even when the foundation models are arbitrarily biased.


\paragraph{Integrating foundation models}
Prediction-powered inference~(\ppi)~\citep{angelopoulos2023prediction} is a statistical framework that constructs valid confidence intervals using a small labeled dataset and a large unlabeled dataset imputed by a foundation model. $\ppi$ has been applied in various domains, including generalization of causal inferences~\citep{demirel24prediction}, large language model evaluation~\citep{fisch2024stratified,dorner2024limitsscalableevaluationfrontier}, and improving the efficiency of social science experiments~\citep{broskamixed,egami2024using}. However, unlike our approach, $\ppi$ requires access to an additional unlabeled dataset from the same distribution as the experimental sample, which may be as costly as labeled data. Recent work by \citet{poulet2025prediction} introduces 
Prediction-powered inference for clinical trials ($\ppct$), an adaptation of $\ppi$ to estimate  average treatment effects in randomized experiments without any additional  external data. $\ppct$ combines the difference in means estimator with an 
$\aipw$ estimator that integrates the same foundation model as the outcome regression for both treatment and control groups. However, our work differs in two key aspects:
(i) $\ppct$ integrates a single foundation model, and (ii) $\ppct$ does not include the standard $\aipw$ estimator with the outcome regression estimated from experimental data. As a result, $\ppct$ cannot achieve the efficiency bound unless the foundation model is almost surely equal to the underlying outcome regression. 


 



\paragraph{Integrating observational data} There is growing interest in augmenting randomized experiments with data from observational studies to improve statistical precision. One approach involves first testing whether the observational data is compatible with the experimental data~\citep{dahabreh2024using}---for instance, using a statistical test to assess if the mean of the outcome conditional on the covariates is invariant across studies \cite{luedtke2019omnibus,hussain2023falsification,de2024detecting}—and then combining the datasets to improve precision, if the test does not reject. These tests, however, have low statistical power, especially when the experimental sample size is small, which is precisely when leveraging observational data would be most beneficial. To overcome this, a recent line of work integrates a prognostic score estimated from observational data as a covariate when estimating the outcome regression~\citep{schuler2022increasing,liao2023prognostic}. However, increasing the dimensionality of the problem---by adding an additional covariate---can increase estimation error and inflate the finite sample variance. Finally, the work most closely related to ours is \citet{karlsson2024robust}, that integrates an outcome regression estimated from observational data into the \aipw~estimator. In contrast, our approach is not constrained by the availability of well-structured observational data, since it leverages black-box foundation models trained on external data sources.

\section{Research Methodology}~\label{sec:Methodology}

In this section, we discuss the process of conducting our systematic review, e.g., our search strategy for data extraction of relevant studies, based on the guidelines of Kitchenham et al.~\cite{kitchenham2022segress} to conduct SLRs and Petersen et al.~\cite{PETERSEN20151} to conduct systematic mapping studies (SMSs) in Software Engineering. In this systematic review, we divide our work into a four-stage procedure, including planning, conducting, building a taxonomy, and reporting the review, illustrated in Fig.~\ref{fig:search}. The four stages are as follows: (1) the \emph{planning} stage involved identifying research questions (RQs) and specifying the detailed research plan for the study; (2) the \emph{conducting} stage involved analyzing and synthesizing the existing primary studies to answer the research questions; (3) the \emph{taxonomy} stage was introduced to optimize the data extraction results and consolidate a taxonomy schema for REDAST methodology; (4) the \emph{reporting} stage involved the reviewing, concluding and reporting the final result of our study.

\begin{figure}[!t]
    \centering
    \includegraphics[width=1\linewidth]{fig/methodology/searching-process.drawio.pdf}
    \caption{Systematic Literature Review Process}
    \label{fig:search}
\end{figure}

\subsection{Research Questions}
In this study, we developed five research questions (RQs) to identify the input and output, analyze technologies, evaluate metrics, identify challenges, and identify potential opportunities. 

\textbf{RQ1. What are the input configurations, formats, and notations used in the requirements in requirements-driven
automated software testing?} In requirements-driven testing, the input is some form of requirements specification -- which can vary significantly. RQ1 maps the input for REDAST and reports on the comparison among different formats for requirements specification.

\textbf{RQ2. What are the frameworks, tools, processing methods, and transformation techniques used in requirements-driven automated software testing studies?} RQ2 explores the technical solutions from requirements to generated artifacts, e.g., rule-based transformation applying natural language processing (NLP) pipelines and deep learning (DL) techniques, where we additionally discuss the potential intermediate representation and additional input for the transformation process.

\textbf{RQ3. What are the test formats and coverage criteria used in the requirements-driven automated software
testing process?} RQ3 focuses on identifying the formulation of generated artifacts (i.e., the final output). We map the adopted test formats and analyze their characteristics in the REDAST process.

\textbf{RQ4. How do existing studies evaluate the generated test artifacts in the requirements-driven automated software testing process?} RQ4 identifies the evaluation datasets, metrics, and case study methodologies in the selected papers. This aims to understand how researchers assess the effectiveness, accuracy, and practical applicability of the generated test artifacts.

\textbf{RQ5. What are the limitations and challenges of existing requirements-driven automated software testing methods in the current era?} RQ5 addresses the limitations and challenges of existing studies while exploring future directions in the current era of technology development. %It particularly highlights the potential benefits of advanced LLMs and examines their capacity to meet the high expectations placed on these cutting-edge language modeling technologies. %\textcolor{blue}{CA: Do we really need to focus on LLMs? TBD.} \textcolor{orange}{FW: About LLMs, I removed the direct emphase in RQ5 but kept the discussion in RQ5 and the solution section. I think that would be more appropriate.}

\subsection{Searching Strategy}

The overview of the search process is exhibited in Fig. \ref{fig:papers}, which includes all the details of our search steps.
\begin{table}[!ht]
\caption{List of Search Terms}
\label{table:search_term}
\begin{tabularx}{\textwidth}{lX}
\hline
\textbf{Terms Group} & \textbf{Terms} \\ \hline
Test Group & test* \\
Requirement Group & requirement* OR use case* OR user stor* OR specification* \\
Software Group & software* OR system* \\
Method Group & generat* OR deriv* OR map* OR creat* OR extract* OR design* OR priorit* OR construct* OR transform* \\ \hline
\end{tabularx}
\end{table}

\begin{figure}
    \centering
    \includegraphics[width=1\linewidth]{fig/methodology/search-papers.drawio.pdf}
    \caption{Study Search Process}
    \label{fig:papers}
\end{figure}

\subsubsection{Search String Formulation}
Our research questions (RQs) guided the identification of the main search terms. We designed our search string with generic keywords to avoid missing out on any related papers, where four groups of search terms are included, namely ``test group'', ``requirement group'', ``software group'', and ``method group''. In order to capture all the expressions of the search terms, we use wildcards to match the appendix of the word, e.g., ``test*'' can capture ``testing'', ``tests'' and so on. The search terms are listed in Table~\ref{table:search_term}, decided after iterative discussion and refinement among all the authors. As a result, we finally formed the search string as follows:


\hangindent=1.5em
 \textbf{ON ABSTRACT} ((``test*'') \textbf{AND} (``requirement*'' \textbf{OR} ``use case*'' \textbf{OR} ``user stor*'' \textbf{OR} ``specifications'') \textbf{AND} (``software*'' \textbf{OR} ``system*'') \textbf{AND} (``generat*'' \textbf{OR} ``deriv*'' \textbf{OR} ``map*'' \textbf{OR} ``creat*'' \textbf{OR} ``extract*'' \textbf{OR} ``design*'' \textbf{OR} ``priorit*'' \textbf{OR} ``construct*'' \textbf{OR} ``transform*''))

The search process was conducted in September 2024, and therefore, the search results reflect studies available up to that date. We conducted the search process on six online databases: IEEE Xplore, ACM Digital Library, Wiley, Scopus, Web of Science, and Science Direct. However, some databases were incompatible with our default search string in the following situations: (1) unsupported for searching within abstract, such as Scopus, and (2) limited search terms, such as ScienceDirect. Here, for (1) situation, we searched within the title, keyword, and abstract, and for (2) situation, we separately executed the search and removed the duplicate papers in the merging process. 

\subsubsection{Automated Searching and Duplicate Removal}
We used advanced search to execute our search string within our selected databases, following our designed selection criteria in Table \ref{table:selection}. The first search returned 27,333 papers. Specifically for the duplicate removal, we used a Python script to remove (1) overlapped search results among multiple databases and (2) conference or workshop papers, also found with the same title and authors in the other journals. After duplicate removal, we obtained 21,652 papers for further filtering.

\begin{table*}[]
\caption{Selection Criteria}
\label{table:selection}
\begin{tabularx}{\textwidth}{lX}
\hline
\textbf{Criterion ID} & \textbf{Criterion Description} \\ \hline
S01          & Papers written in English. \\
S02-1        & Papers in the subjects of "Computer Science" or "Software Engineering". \\
S02-2        & Papers published on software testing-related issues. \\
S03          & Papers published from 1991 to the present. \\ 
S04          & Papers with accessible full text. \\ \hline
\end{tabularx}
\end{table*}

\begin{table*}[]
\small
\caption{Inclusion and Exclusion Criteria}
\label{table:criteria}
\begin{tabularx}{\textwidth}{lX}
\hline
\textbf{ID}  & \textbf{Description} \\ \hline
\multicolumn{2}{l}{\textbf{Inclusion Criteria}} \\ \hline
I01 & Papers about requirements-driven automated system testing or acceptance testing generation, or studies that generate system-testing-related artifacts. \\
I02 & Peer-reviewed studies that have been used in academia with references from literature. \\ \hline
\multicolumn{2}{l}{\textbf{Exclusion Criteria}} \\ \hline
E01 & Studies that only support automated code generation, but not test-artifact generation. \\
E02 & Studies that do not use requirements-related information as an input. \\
E03 & Papers with fewer than 5 pages (1-4 pages). \\
E04 & Non-primary studies (secondary or tertiary studies). \\
E05 & Vision papers and grey literature (unpublished work), books (chapters), posters, discussions, opinions, keynotes, magazine articles, experience, and comparison papers. \\ \hline
\end{tabularx}
\end{table*}

\subsubsection{Filtering Process}

In this step, we filtered a total of 21,652 papers using the inclusion and exclusion criteria outlined in Table \ref{table:criteria}. This process was primarily carried out by the first and second authors. Our criteria are structured at different levels, facilitating a multi-step filtering process. This approach involves applying various criteria in three distinct phases. We employed a cross-verification method involving (1) the first and second authors and (2) the other authors. Initially, the filtering was conducted separately by the first and second authors. After cross-verifying their results, the results were then reviewed and discussed further by the other authors for final decision-making. We widely adopted this verification strategy within the filtering stages. During the filtering process, we managed our paper list using a BibTeX file and categorized the papers with color-coding through BibTeX management software\footnote{\url{https://bibdesk.sourceforge.io/}}, i.e., “red” for irrelevant papers, “yellow” for potentially relevant papers, and “blue” for relevant papers. This color-coding system facilitated the organization and review of papers according to their relevance.

The screening process is shown below,
\begin{itemize}
    \item \textbf{1st-round Filtering} was based on the title and abstract, using the criteria I01 and E01. At this stage, the number of papers was reduced from 21,652 to 9,071.
    \item \textbf{2nd-round Filtering}. We attempted to include requirements-related papers based on E02 on the title and abstract level, which resulted from 9,071 to 4,071 papers. We excluded all the papers that did not focus on requirements-related information as an input or only mentioned the term ``requirements'' but did not refer to the requirements specification.
    \item \textbf{3rd-round Filtering}. We selectively reviewed the content of papers identified as potentially relevant to requirements-driven automated test generation. This process resulted in 162 papers for further analysis.
\end{itemize}
Note that, especially for third-round filtering, we aimed to include as many relevant papers as possible, even borderline cases, according to our criteria. The results were then discussed iteratively among all the authors to reach a consensus.

\subsubsection{Snowballing}

Snowballing is necessary for identifying papers that may have been missed during the automated search. Following the guidelines by Wohlin~\cite{wohlin2014guidelines}, we conducted both forward and backward snowballing. As a result, we identified 24 additional papers through this process.

\subsubsection{Data Extraction}

Based on the formulated research questions (RQs), we designed 38 data extraction questions\footnote{\url{https://drive.google.com/file/d/1yjy-59Juu9L3WHaOPu-XQo-j-HHGTbx_/view?usp=sharing}} and created a Google Form to collect the required information from the relevant papers. The questions included 30 short-answer questions, six checkbox questions, and two selection questions. The data extraction was organized into five sections: (1) basic information: fundamental details such as title, author, venue, etc.; (2) open information: insights on motivation, limitations, challenges, etc.; (3) requirements: requirements format, notation, and related aspects; (4) methodology: details, including immediate representation and technique support; (5) test-related information: test format(s), coverage, and related elements. Similar to the filtering process, the first and second authors conducted the data extraction and then forwarded the results to the other authors to initiate the review meeting.

\subsubsection{Quality Assessment}

During the data extraction process, we encountered papers with insufficient information. To address this, we conducted a quality assessment in parallel to ensure the relevance of the papers to our objectives. This approach, also adopted in previous secondary studies~\cite{shamsujjoha2021developing, naveed2024model}, involved designing a set of assessment questions based on guidelines by Kitchenham et al.~\cite{kitchenham2022segress}. The quality assessment questions in our study are shown below:
\begin{itemize}
    \item \textbf{QA1}. Does this study clearly state \emph{how} requirements drive automated test generation?
    \item \textbf{QA2}. Does this study clearly state the \emph{aim} of REDAST?
    \item \textbf{QA3}. Does this study enable \emph{automation} in test generation?
    \item \textbf{QA4}. Does this study demonstrate the usability of the method from the perspective of methodology explanation, discussion, case examples, and experiments?
\end{itemize}
QA4 originates from an open perspective in the review process, where we focused on evaluation, discussion, and explanation. Our review also examined the study’s overall structure, including the methodology description, case studies, experiments, and analyses. The detailed results of the quality assessment are provided in the Appendix. Following this assessment, the final data extraction was based on 156 papers.

% \begin{table}[]
% \begin{tabular}{ll}
% \hline
% QA ID & QA Questions                                             \\ \hline
% Q01   & Does this study clearly state its aims?                  \\
% Q02   & Does this study clearly describe its methodology?        \\
% Q03   & Does this study involve automated test generation?       \\
% Q04   & Does this study include a promising evaluation?          \\
% Q05   & Does this study demonstrate the usability of the method? \\ \hline
% \end{tabular}%
% \caption{Questions for Quality Assessment}
% \label{table:qa}
% \end{table}

% automated quality assessment

% \textcolor{blue}{CA: Our search strategy focused on identifying requirements types first. We covered several sources, e.g., ~\cite{Pohl:11,wagner2019status} to identify different formats and notations of specifying requirements. However, this came out to be a long list, e.g., free-form NL requirements, semi-formal UML models, free-from textual use case models, UML class diagrams, UML activity diagrams, and so on. In this paper, we attempted to primarily focus on requirements-related aspects and not design-level information. Hence, we generalised our search string to include generic keywords, e.g., requirement*, use case*, and user stor*. We did so to avoid missing out on any papers, bringing too restrictive in our search strategy, and not creating a too-generic search string with all the aforementioned formats to avoid getting results beyond our review's scope.}


%% Use \subsection commands to start a subsection.



%\subsection{Study Selection}

% In this step, we further looked into the content of searched papers using our search strategy and applied our inclusion and exclusion criteria. Our filtering strategy aimed to pinpoint studies focused on requirements-driven system-level testing. Recognizing the presence of irrelevant papers in our search results, we established detailed selection criteria for preliminary inclusion and exclusion, as shown in Table \ref{table: criteria}. Specifically, we further developed the taxonomy schema to exclude two types of studies that did not meet the requirements for system-level testing: (1) studies supporting specification-driven test generation, such as UML-driven test generation, rather than requirements-driven testing, and (2) studies focusing on code-based test generation, such as requirement-driven code generation for unit testing.





\section{Review Results}
\label{sec:results}
This section will answer all the formulated research questions based on the selected studies. The answers to the defined research questions are given in each subsection.

\input{RQ1_Results.tex}
\subsection{RQ2. Techniques Used for AEG}

Table~\ref{tbl:rq2-results-aeg} summarizes the selected exploit generation studies in our survey and the main techniques employed by them.
Note that we report the studies with their IDs, which are labeled after we perform paper grouping in Section~\ref{sec:paper-organization}.
These techniques can be classified into four technique families as we mentioned in our taxonomy (Figure~\ref{fig:taxonomy}), including AEG, Security Testing, Fuzzing, and other kinds of techniques.

\begin{table}
\centering
	\caption{Main techniques used among the reviewed AEG studies.}
	\label{tbl:rq2-results-aeg}
        \footnotesize
        \resizebox{0.98\linewidth}{!}{%
	\begin{tabular}{|lll|}
		\toprule
		\textbf{Technique Family} & \textbf{Main Technique}                          & \textbf{Studies} \\
		\midrule
		 & Control-flow Hijacking  & S01, S03-09, S54, S55 \\
		{\textit{AEG}} & Crash Analysis-based  & S10 \\
		& Data-oriented  & S11 \\
		\midrule
		 &  Combinational Testing & S26 \\
		& Grammar-based Testing &  S27-29\\
		& Heuristic-based Testing & S30-31 \ifthenelse{\boolean{deliverable}}
{}{\mc{Not in the big Table 5}} \\
		& Hybrid Testing &  S32\\
		& Learning-based Testing &  S33, S34\\
		& Logic programming-based Testing & S35\\
		\textit{Security Testing} & Model-based Testing & S36-39 \\
		& Mutation-based Testing & S40-42\\
		& Pentesting & S43\\
		& Robustness Testing & S44\\
		& Search-based Testing & S45, S46 \\
		& Symbolic Execution-based Testing & S47 \\
		& Threat model-based Testing & S08, S48-50 \\
		\midrule
		& Coverage-based Greybox Fuzzing & S04, S13-16 \\
		& Directed Fuzzing & S17\\
		\multirow{0}{*}[0.15cm]{\textit{Fuzzing}} & Knowledge-based Fuzzing & S18\\
		& Mutation-based Fuzzing & S19 \\
		& Stateful Fuzzing & S20\\
		& Whitebox Fuzzing & S21, S22\\
		\midrule
		& Dynamic Taint Analysis & S60-63\\
		& Dynamic Symbolic Execution & S53, S55-58 \\
		\textit{Others} & Static Symbolic Execution & S59 \\
		& State model-based String Analysis & S27\\
		& Static String Analysis & S52\\
		\bottomrule     
	\end{tabular}
 }
\end{table}

\subsubsection{AEG}
In our 63 selected studies, twelve are dedicated to the AEG. The main approaches can be divided into three groups, i.e., Control-flow hijacking AEG, Crash Analysis-based AEG, and Data-oriented AEG. Control-flow Hijacking is the dominant technique with 10/12 AEG studies in our survey while we found only one for each of Crash Analysis-based and Data-oriented AEG. The popular approaches implementing Control-flow Hijacking AEG often require the program binaries as the inputs and consist four steps: (i) identify the vulnerability, (ii) obtain runtime information, (iii) generate the exploits, and (iv) verify the exploits. Meanwhile, Crash Analysis-based AEG (S10~\cite{Liu202271}) requires the crash information as the extra inputs, then this technique extract the execution trace tries and reproduce the crash under symbolic mode. If an exploitable state is found, a solver is utilized to resolve the path contraints to find malicious inputs and synthesize the exploit. Data-oriented AEG (S11~\cite{Pewny2019111}), on the other hand, tries manipulate the data control path instead of the execution control flow of the targeted program. This technique generate the data-oriented programming (DOP) attacks in the form of a high-level language, then compile them into concrete exploits for each kind of compilers.

\subsubsection{Security Testing}
Security testing has the majority of studies in our survey with a very wide range of testing techniques. They expand in multiple classes of testing techniques: from model-based testing, logic programming-based testing, heuristic-based testing, grammar-based testing, search-based testing to hybrid testing, learning-based testing, and pentesting. Xu et al. contribute to the threat model based security testing with their four works (S08, S48-S50~\cite{marback2013threat, xu2011tool, xu2012automated, Xu2015247}). They first manually contruct the PetriNets as the threat models, from which they generate the attack paths, and lastly the security test cases. Their approach can be applied to generate security tests in cross-language projects such as PHP and C/C++.
In the work of Zech et al. (S35~\cite{Zech201488}), they generates negative security tests based on a answer set programming that they collect from the developers. Chaleshtari et al. (S31~\cite{Chaleshtari20233430}) employed heuristic-based testing, where they define of 120 Metamorphic relations with domain-specific language for specifying security properties in Java code and then use them for generate security test cases. In S27~\cite{Mohammadi201678}, Mohammadi et al. used unit testing to generate Cross-side scripting (XSS) tests for Java Jakarta Server Pages (JSP). The inputs for the tests are provided by a grammar-based generator, this techniques were used detect zero-day XSS vulnerabilities. Del Grosso et al. (S35~\cite{DelGrosso20083125}) tried to generate test input to detect buffer overflow in C/C++ programs by leveraging search-based testing. The genetic algorithm they used gives rewards to the test inputs reaching the vulnerable statements. In this way, they can find the inputs that can trigger the vulnerabilities. Del Grosso et al. (S34~\cite{Liu2020286}) applied recent advances in deep learning for natural language processing to learn the semantic knowledge from code snippets vulnerable to SQL Injection attacks to generate the malicious inputs and then translate these inputs into security test cases.

\subsubsection{Fuzzing}
We acknowledge that there are a plethora of fuzzing techniques in the literature. However, as mentioned in the Section~\ref{sec:intro} and Section~\ref{sec:background}, we set our scope to the techniques that are specific designed to find vulnerabilities at the first place. So many of the fuzzing techniques (do not focus on security vulnerabilities, yet can be used to discover security bugs) are of our scope except for the ones in Table~\ref{tbl:rq2-results-aeg}. Coverage-based greybox fuzzing are employed by the most fuzzing studies (S04, S13-16~\cite{Bohme2019489, Liu2018, Wuestholz20201398, Gong2022374}) in our survey. S13~\cite{Bohme2019489} is one of the well-known work which are built on American Fuzzy Lop (AFL) tool framework and leverages the Markov chain as modeling a systematic exploration of the state space for finding inputs. Other fuzzing techniques can be named for finding vulnerabilities such as Knowledge-based fuzzing, Directed fuzzing, Mutation-based fuzzing, Stateful fuzzing, and Whitebox fuzzing.

\subsubsection{Others}
Some other techniques utilized for generating vulnerability exploits that we found are also well-known such as Taint analysis, Symbolic execution, and String analysis. Note that these techniques are frequently employed as parts of the other exploit generation studies, however, we only report the primary techniques. In this family, we list the studies that are not eligible to be classified into the big three technique families AEG, Security testing, and Fuzzing.
\subsection{RQ3. Targets of AEG Techniques}

\subsubsection{Targeted vulnerability types and programming languages}
As reported in Table~\ref{tbl:rqs-g1}, the majority of vulnerabilities that the AEG techniques focus on are memory-based vulnerabilities in C/C++ programs such as buffer overflow~\cite{Xu2022, shahriar2009automatic, DelGrosso20083125, Dixit2021}, heap overflow~\cite{Huang2019}, while the security testing techniques focus on web-based injection vulnerabilities such as XSS and SQL Injection~\cite{Bozic202020, zhang2010d, aydin2014automated, Huang2013208}, which are common in PHP- and Java-based applications.
Besides, fuzzing techniques such as~\cite{Bohme2019489, Yu2022, KallingalJoshy2021540}, try to explore the input space to trigger the crash states of both C/C++ and Java programs.
Also, some other techniques focus on specific types of vulnerabilities; for example, Tang et al.~\cite{Tang2017492} and Garcia et al.~\cite{Garcia2017661} tried to generate exploits for Android ICC vulnerabilities, Atlidakis et al.~\cite{Atlidakis2020387} tried to generate the request inputs that can trigger one of four RESTful security violations that they defined themselves.

\subsubsection{Targeted inputs}
Most of the techniques work on the binaries (usually the AEG and fuzzing techniques such as~\cite{Liu2018705, Liu202271, Gong2022374, Zhang201946}) or the source code (security testing techniques and the techniques that require performing symbolic execution or taint analysis such as~\cite{Chen20201580, DelGrosso20083125, avancini2012security, dao2011security, wu2018fuze}) of the targeted programs. Meanwhile, some other techniques focus on web-based vulnerabilities require a running instance of the applications with their APIs~\cite{Appelt2014259, Jan201612, Atlidakis2020387} or UIs~\cite{Simos2019122, Liu2016123, Bozic2020115, Huang2013208} exposed. Some specific techniques, such as model-based security testing, take the special inputs as the program models to find and exploit the vulnerabilities at abstract level~\cite{Pretschner2008338, Lebeau2013445, Khamaiseh2017534}.
\subsection{RQ4. Output of AEG techniques}

As shown in Table~\ref{tbl:rqs-g1}, we categorize the outputs of the tools into groups: (1) exploit, which is either a code snippet or a complete program that can be executed to exploit the vulnerabilities, (2) test cases, which are usually associated and can be executed with a testing framework such as \textsc{JUnit} or \textsc{Selenium} to find evidence of the vulnerability's presences in the project, and (3) input/payload which consists of crafted input values or objects to trigger the vulnerable states of the asset.

It can be foreseen that most AEG techniques (studies S01 -- S11) target generating working exploits that find not only the vulnerable states but also the exploitable states of the target programs.
\ifthenelse{\boolean{deliverable}}
{}
{\ema{We need to clarify (earlier) the difference between exploitable and vulnerable states because it's interesting.}}
However, Do et al.~\cite{Do2015401} proposed an AEG technique that generates an exploit in the form of a test case, which can be executed via the Java \textsc{JUnit} testing framework. In contrast to AEG, fuzzing techniques tend to generate the inputs or payloads that can aim to trigger the only vulnerable state without further exploitation actions.
This is true except for some fuzzing work where they also put more effort into reaching the exploitable state of the vulnerabilities~\cite{Bohme2019489}.
Most security testing techniques aim to generate security test cases to find vulnerabilities or security weaknesses in a more methodical way compared to AEG and Fuzzing approaches. Many testing techniques also aim to generate the malicious inputs~\cite{Bozic202020, Bozic2020115, Liu2020286, Appelt2014259} for triggering web-based vulnerabilities and exploits for memory-based vulnerabilities in program binaries~\cite{Nurmukhametov202137}.
\subsection{RQ5. Automation of AEG techniques}

As shown in Table~\ref{tbl:rqs-g1}, more than 80\% (54 out of 66) of the selected studies claimed in their papers that their tools can be fully automatically executed without any human interventions. 
We cannot verify the automation level of these techniques by ourselves as we do not have access to their tools (this will be discussed more in the results of RQ7 in Section~\ref{sub:rq7-results}).

Note that some techniques may require extra inputs, which must be provided before the tool is launched. At the same time, other tools require additional manual help from humans in between to operate the tools. For example, Zech et al.~\cite{Zech201488} collect feedback from developers to construct their logic programming answer set for generating security tests. In the studies focusing model-based~\cite{Pretschner2008338, Lebeau2013445, Khamaiseh2017534} and threat model-based~\cite{xu2011tool, xu2012automated, Xu2015247, marback2013threat} security testing, the program model creation and modification are also required to be done manually, which assist the automated process of generating test cases based on the models.

% or semi-automatically with a few manual steps in between. Some techniques require an interactive process for their executions such as xxx.
\subsection{RQ6. Evaluation of AEG techniques}

\ifthenelse{\boolean{deliverable}}
{}{\cuong{This section is not finished}}

Table~\ref{tbl:rq6-results} shows the types of datasets used for the evaluation of the techniques in the studies. Most of the techniques were evaluated on real-world applications with real-world CVE vulnerabilities. While others were experienced with artificial vulnerabilities, which are created for the purposes of security research or security competitions such as Capture-The-Flag competitions. More interestingly, some techniques were evaluated on both artificial and real-world datasets.

\begin{table}
\centering
	\caption{Datasets used to evaluate AEG techniques.}
	\label{tbl:rq6-results}
        \resizebox{0.98\linewidth}{!}{%
	\begin{tabular}{lp{12cm}}
		\toprule
		\textbf{Dataset type}      & \textbf{Studies} \\
		\midrule
		Artificial &  S04, S05, S09, S15, S24, S25, S27, S29, S30, S36, S38, S55, S58 \ifthenelse{\boolean{deliverable}}{}
{\mc{?}} \\
		Real-world &  S03, S04, S08, S13, S14, S16-22, S27, S28, S31, S33, S34, S39-42, S44, S45, S47-50, S52, S53-57, S59-62 \ifthenelse{\boolean{deliverable}}
{}{\mc{?}} \\
		Toy/PoC & S11, S35, S37 \\
		Artificial + Real-world & S01, S06, S08, S10, S32, S43 \\
		\bottomrule     
	\end{tabular}
        }
\end{table}
\subsection{RQ7. Availability and Usability of AEG techniques}
\label{sub:rq7-results}

Most of the studies propose concrete techniques for finding exploits or generating test cases; however, they do not come with the available tools. This can be explained by the security concerns of publicly providing these tools, which can aid malicious users in arming their weapons to find and exploit vulnerabilities in the real world.
Several techniques can be strictly shared for research purposes (i.e., the authors authorize other researchers and practitioners to access their tools and replication artifacts upon a proper request)~\cite{Appelt2014259, Xu2022, Garcia2017661}.
Another explanation might be the prototypical nature of the approaches presented. The authors could not feel comfortable releasing a not-ready tool or one that was too difficult to use without the original authors' support.
%
Table~\ref{tbl:rq7-results} shows the list of tools we found in the selected studies that are publicly available for everyone to use. Most of them are fuzzers or AEG tools.

\begin{table}
\centering
	\caption{Catalog of available AEG tools.}
	\label{tbl:rq7-results}
        \tiny
        \resizebox{0.98\linewidth}{!}{%
	\begin{tabular}{llll}
		\toprule
		\textbf{Study} & \textbf{Pub.} & \textbf{Tool name}  & \textbf{Tool link} \\
		\midrule
		S15 & \cite{Alshmrany202185}  & \textsc{FuSeBMC} & \url{https://github.com/kaled-alshmrany/FuSeBMC} \\ 
	    S24 & \cite{shoshitaishvili2018mechanical}  & \textsc{Mechaphish} & \url{https://github.com/mechaphish} \\
		S58 & \cite{Dixit2021}  & \textsc{AngErza} & \url{https://github.com/rudyerudite/AngErza} \\
		S06 & \cite{Wei2019120152}  & \textsc{AutoROP} & \url{https://github.com/wy666444/auto_rop} \\
		S07 & \cite{Brizendine202277}  & \textsc{JOPRocket} & \url{https://github.com/Bw3ll/JOP_ROCKET/} \\
		S04 & \cite{Liu2018}  & \textsc{CAFA} & \url{https://github.com/CAFA1/CAFA} \\
		S13 & \cite{Bohme2019489}  & \textsc{AFLFast} & \url{https://github.com/mboehme/aflfast} \\
		S \ifthenelse{\boolean{deliverable}}
{}{\mc{?}} & \cite{Chaleshtari20233430}  & \textsc{SMRL} & \url{https://sntsvv.github.io/SMRL/} \\
		S34 & \cite{Liu2020286}  & \textsc{DeepSQLi} & \url{https://github.com/COLA-Laboratory/issta2020} \\
		S39 & \cite{Nurmukhametov202137}  & \textsc{Majorca} & \url{https://github.com/ispras/rop-benchmark} \\
		S57 & \cite{wu2018fuze}  & \textsc{FUZE} & \url{https://github.com/ww9210/Linux_kernel_exploits} \\
		
		\bottomrule     
	\end{tabular}
        }
\end{table}



%\subsection{RQ8. Workflow of AEG techniques}
%\subsection{RQ9. Security knowledge required for using AEG techniques}

\section{Limitation}
The use of 3D-printed PLA for structural components improves improving ease of assembly and reduces weight and cost, yet it causes deformation under heavy load, which can diminish end-effector precision. Using metal, such as aluminum, would remedy this problem. Additionally, \robot relies on integrated joint relative encoders, requiring manual initialization in a fixed joint configuration each time the system is powered on. Using absolute joint encoders could significantly improve accuracy and ease of use, although it would increase the overall cost. 

%Reliance on commercially available actuators simplifies integration but imposes constraints on control frequency and customization, further limiting the potential for tailored performance improvements.

% The 6 DoF configuration provides sufficient mobility for most tasks; however, certain bimanual operations could benefit from an additional degree of freedom to handle complex joint constraints more effectively. Furthermore, the limited torque density of commercially available proprioceptive actuators restricts the payload and torque output, making the system less suitability for handling heavier loads or high-torque applications. 

The 6 DoF configuration of the arm provides sufficient mobility for single-arm manipulation tasks, yet it shows a limitation in certain bimanual manipulation problems. Specifically, when \robot holds onto a rigid object with both hands, each arm loses 1 DoF because the hands are fixed to the object during grasping. This leads to an underactuated kinematic chain which has a limited mobility in 3D space. We can achieve more mobility by letting the object slip inside the grippers, yet this renders the grasp less robust and simulation difficult. Therefore, we anticipate that designing a lightweight 3 DoF wrist in place of the current 2 DoF wrist allows a more diverse repertoire of manipulation in bimanual tasks.

Finally, the limited torque density of commercially available proprioceptive actuators restricts the performance. Currently, all of our actuators feature a 1:10 gear ratio, so \robot can handle up to 2.5 kg of payload. To handle a heavier object and manipulate it with higher torque, we expect the actuator to have 1:20$\sim$30 gear ratio, but it is difficult to find an off-the-shelf product that meets our requirements. Customizing the actuator to increase the torque density while minimizing the weight will enable \robot to move faster and handle more diverse objects.

%These constraints highlight opportunities for improvement in future iterations, including alternative materials for enhanced rigidity, custom actuator designs for higher control precision and torque density, the adoption of absolute joint encoders, and optimized configurations to balance dexterity and weight.



%\section{Discussion}
%\subsection{Open research challenges and future directions}

\section{Conclusion}
In this work, we propose a simple yet effective approach, called SMILE, for graph few-shot learning with fewer tasks. Specifically, we introduce a novel dual-level mixup strategy, including within-task and across-task mixup, for enriching the diversity of nodes within each task and the diversity of tasks. Also, we incorporate the degree-based prior information to learn expressive node embeddings. Theoretically, we prove that SMILE effectively enhances the model's generalization performance. Empirically, we conduct extensive experiments on multiple benchmarks and the results suggest that SMILE significantly outperforms other baselines, including both in-domain and cross-domain few-shot settings.

\begin{acks}
This work was partially supported by EU-funded project Sec4AI4Sec (grant no. 101120393).
\end{acks}

\bibliographystyle{ACM-Reference-Format}
\bibliography{references}

\end{document}
