\PassOptionsToPackage{dvipsnames}{xcolor}
\documentclass[manuscript,screen]{acmart}

%% \BibTeX command to typeset BibTeX logo in the docs
\AtBeginDocument{%
  \providecommand\BibTeX{{%
    Bib\TeX}}}

\setcopyright{acmlicensed}
\copyrightyear{2018}
\acmYear{2018}
\acmDOI{XXXXXXX.XXXXXXX}

%%%%%%%%%%%%%%%%%%%%%%%%%%%%%%%%%%%%%%%%%%%%%%%%%%%%%%%%%%%%%%%%%%% BEGIN PACKAGES
\usepackage{paralist}
\usepackage{xspace}
\usepackage{balance}
\usepackage{lipsum}
\usepackage{hyperref}
\usepackage{listings}

\usepackage{tikz}
\usetikzlibrary{arrows,shapes,positioning,shadows,trees}

\usepackage[dvipsnames]{xcolor}
\usepackage{pifont}
\usepackage{multirow}
\usepackage{hyperref}
\usepackage{ifthen}
%%%%%%%%%%%%%%%%%%%%%%%%%%%%%%%%%%%%%%%%%%%%%%%%%%%%%%%%%%%%%%%%%%% END PACKAGES

%%%%%%%%%%%%%%%%%%%%%%%%%%%%%%%%%%%%%%%%%%%%%%%%%%%%%%%%%%%%%%%%%%% BEGIN COMMAND DEFINITIONS
\newcommand{\cuong}[1]{\colorbox{OrangeRed}{{\scriptsize\bfseries\color{white}BQC:}} {\color{OrangeRed}\bfseries{#1}}}
\newcommand{\ema}[1]{\colorbox{OrangeRed}{{\scriptsize\bfseries\color{white}EI:}} {\color{OrangeRed}\bfseries{#1}}}
\newcommand{\mc}[1]{\colorbox{Blue}{{\scriptsize\bfseries\color{white}MC:}} {\color{Blue}\bfseries{#1}}}

\long\def\longcaption#1#2{\centering\begin{minipage}{#1}\vspace{-0.7\baselineskip}\scriptsize\noindent\emph{#2}\end{minipage}}
\long\def\longcaptionfig#1#2{\centering\begin{minipage}{#1}\vspace{-0.1\baselineskip}\scriptsize\noindent\emph{#2}\end{minipage}}

\newboolean{deliverable}
\setboolean{deliverable}{true}
%%%%%%%%%%%%%%%%%%%%%%%%%%%%%%%%%%%%%%%%%%%%%%%%%%%%%%%%%%%%%%%%%%% END COMMAND DEFINITIONS


%%%%%%%%%%%%%%%%%%%%%%%%%%%%%%%%%%%%%%%%%%%%%%%%%%%%%%%%%%%%%%%%%%% BEGIN DATA DEFINITIONS
\tikzset{ % Configuration for Taxonomy figure
	basic/.style  = {draw, text width=2cm, drop shadow, font=\sffamily, rectangle},
	root/.style   = {basic, rounded corners=2pt, thin, align=center,
		fill=green!0},
	level 2/.style = {basic, rounded corners=6pt, thin,align=center, fill=green!0,
		text width=8em},
	level 3/.style = {basic, thin, align=left, fill=pink!0, text width=6.5em}
}
%%%%%%%%%%%%%%%%%%%%%%%%%%%%%%%%%%%%%%%%%%%%%%%%%%%%%%%%%%%%%%%%%%% END DATA DEFINITIONS

\begin{document}

\title{A Systematic Literature Review on Automated Exploit\\and Security Test Generation}

\author{Quang-Cuong Bui}
\orcid{0000-0001-6072-9213}
\affiliation{%
  \institution{Hamburg University of Technology}
  \city{Hamburg}
  \country{Germany}
}
\email{cuong.bui@tuhh.de}

\author{Emanuele Iannone}
\orcid{0000-0001-7489-9969}
\affiliation{%
  \institution{Hamburg University of Technology}
  \city{Hamburg}
  \country{Germany}
}
\email{emanuele.iannone@tuhh.de}

\author{Maria Camporese}
\orcid{0009-0009-1178-0210}
\affiliation{%
  \institution{University of Trento}
  \city{Trento}
  \country{Italy}
}
\email{maria.camporese@unitn.it}

\author{Torge Hinrichs}
\orcid{0000-0001-7489-3540}
\affiliation{%
  \institution{Hamburg University of Technology}
  \city{Hamburg}
  \country{Germany}
}
\email{torge.hinrichs@tuhh.de}

\author{Catherine Tony}
\orcid{0000-0002-9916-4456}
\affiliation{%
  \institution{Hamburg University of Technology}
  \city{Hamburg}
  \country{Germany}
}
\email{catherine.tony@tuhh.de}

\author{László Tóth}
\affiliation{%
  \institution{University of Szeged}
  \city{Szeged}
  \country{Hungary}
}
\email{premissa@inf.u-szeged.hu}

\author{Fabio Palomba}
\orcid{0000-0001-9337-5116}
\affiliation{%
  \institution{University of Salerno}
  \city{Salerno}
  \country{Italy}
}
\email{fpalomba@unisa.it}

\author{Péter Hegedűs}
\orcid{0000-0003-4592-6504}
\affiliation{%
  \institution{University of Szeged, FrontEndART Ltd.}
  \city{Szeged}
  \country{Hungary}
}
\email{peter.hegedus@frontendart.com}

\author{Fabio Massacci}
\orcid{0000-0002-1091-8486}
\affiliation{%
  \institution{University of Trento}
  \city{Trento}
  \country{Italy}
}
\email{fabio.massacci@unitn.it}

\author{Riccardo Scandariato}
\orcid{0000-0003-3591-7671}
\affiliation{%
  \institution{Hamburg University of Technology}
  \city{Hamburg}
  \country{Germany}
}
\email{riccardo.scandariato@tuhh.de}

\renewcommand{\shortauthors}{Bui et al.}

\begin{abstract}
The exploit or the Proof of Concept of the vulnerability plays an important role in developing superior vulnerability repair techniques, as it can be used as an oracle to verify the correctness of the patches generated by the tools. However, the vulnerability exploits are often unavailable and require time and expert knowledge to craft. Obtaining them from the exploit generation techniques is another potential solution.
The goal of this survey is to aid the researchers and practitioners in understanding the existing techniques for exploit generation through the analysis of their characteristics and their usability in practice. We identify a list of exploit generation techniques from literature and group them into four categories: automated exploit generation, security testing, fuzzing, and other techniques. Most of the techniques focus on the memory-based vulnerabilities in C/C++ programs and web-based injection vulnerabilities in PHP and Java applications. We found only a few studies that publicly provided usable tools associated with their techniques.
\end{abstract}

%% Code below generated by http://dl.acm.org/ccs.cfm.
\begin{CCSXML}
<ccs2012>
   <concept>
       <concept_id>10002978.10003022.10003026</concept_id>
       <concept_desc>Security and privacy~Web application security</concept_desc>
       <concept_significance>500</concept_significance>
       </concept>
   <concept>
       <concept_id>10002978.10003022.10003023</concept_id>
       <concept_desc>Security and privacy~Software security engineering</concept_desc>
       <concept_significance>500</concept_significance>
       </concept>
   <concept>
       <concept_id>10011007.10011074.10011099.10011102</concept_id>
       <concept_desc>Software and its engineering~Software defect analysis</concept_desc>
       <concept_significance>500</concept_significance>
       </concept>
 </ccs2012>
\end{CCSXML}

\ccsdesc[500]{Security and privacy~Web application security}
\ccsdesc[500]{Security and privacy~Software security engineering}
\ccsdesc[500]{Software and its engineering~Software defect analysis}

\keywords{Software vulnerability, Exploit generation}

%\received{20 February 2007}
%\received[revised]{12 March 2009}
%\received[accepted]{5 June 2009}

\maketitle

\section{Introduction}
\label{sec:intro}

\begin{figure*}[tb]
    \centering
    \includegraphics[width=0.848\linewidth]{figs/circuitnn.pdf} 
    \caption{Illustration of differentiable CircuitNN. CircuitNN is designed based on differentiable NAND gates. After DAS is guided by PI and PO pairs of the truth table, CircuitNN can get the precise circuit architecture logic equivalent to the truth table.}
    \label{fig:circuitnn}
\end{figure*}

% 1. Describe the importance of logic synthesis
% 2. Existing Problems
% (a) Neural Architecture Search: Unstable, Predefined Setting, etc.
% (b) Circuit Generation: Probabilistic Model, Logic Equivalence

With the rapid advancement of technology, the scale of integrated circuits (ICs) has expanded exponentially. 
This expansion has introduced significant challenges in chip manufacturing, particularly concerning power and area metrics.
A primary objective in IC design is achieving the same circuit function with fewer transistors, thereby reducing power usage and area occupancy.

Logic synthesis~\cite{hachtel2005logicsynth}, a critical step in electronic design automation (EDA), transforms behavioral-level circuit designs into optimized gate-level circuits, ultimately yielding the final IC layout. 
The primary goal of logic synthesis is to identify the physical implementation with the fewest gates for a given circuit function. 
This task constitutes a challenging NP-hard combinatorial optimization problem. 
Current logic synthesis tools~\cite{brayton2010abc, wolf2013yosys} rely on human-designed heuristics, often leading to sub-optimal outcomes.

Differentiable architecture search (DAS) techniques~\cite{liu2018darts, chu2020darts} offer novel perspectives on addressing challenges in this problem.
Circuit functions can be represented through truth tables, which map binary inputs to their corresponding outputs. 
Truth tables provide a precise representation of input-output relationships, ensuring the design of functionally equivalent circuits.
Inspired by this, researchers~\cite{deepmind2024ai4sys, wang2024tnet} have begun exploring the application of DAS to synthesize circuits directly from truth tables.
Specifically, \citet{deepmind2024ai4sys} proposed CircuitNN, a framework that learns differentiable connection structures with logic gates, enabling the automatic generation of logic circuits from truth tables.
This approach significantly reduces the complexity of traditional circuit generation. 
Building on this, \citet{wang2024tnet} introduced T-Net, a triangle-shaped variant of CircuitNN, incorporating regularization techniques to enhance the efficiency of DAS.

Despite these advancements, several challenges remain. 
The computational complexity of DAS grows quadratically with the number of gates, posing scalability issues.
Although triangle-shaped architecture~\cite{wang2024tnet} partially mitigates this problem, redundancy persists. 
%Additionally, DAS is susceptible to converging to local optima, limiting the ability to search architectures that satisfy the given truth tables~\cite{liu2018darts}. 
%Furthermore, hyperparameters (network depth and layer width) require extensive searches, introducing complexity and prolonging the synthesis process. 
Additionally, DAS is susceptible to converging to local optima~\cite{liu2018darts} and hyperparameters (network depth and layer width) require extensive searches. 
The challenges arise from the vast search space in DAS. 
% Even with predefined settings for CircuitNN, finding a configuration that meets the truth table requires extensive trial and error during the DAS process. 
Intuitively, limiting the search space through predefined parameters (network depth, gates per layer, and connection probabilities) can significantly reduce the complexity.

Recent advances~\cite{openai2023gpt4, abramson2024alphafold3, esser2024sd3, li2024mar} in conditional generative models have demonstrated remarkable performance across language, vision, and graph generation tasks. 
Motivated by these developments, we propose a novel approach to circuit generation that generates preliminary circuit structures to guide DAS in generating refined circuits matching specified truth tables. 
Firstly, we introduce CircuitVQ, a tokenizer with a discrete codebook for circuit tokenization. 
Built upon our Circuit AutoEncoder framework~\cite{hou2022graphmae,li2023maskgae,wu2025mgvga}, CircuitVQ is trained through a circuit reconstruction task. 
Specifically, the CircuitVQ encoder encodes input circuits into discrete tokens using a learnable codebook, while the decoder reconstructs the circuit adjacency matrix based on these tokens.
Subsequently, the CircuitVQ encoder serves as a circuit tokenizer for CircuitAR pretraining, which employs a masked autoregressive modeling paradigm~\cite{chang2022maskgit, li2023mage}. 
In this process, the discrete codes function as supervision signals. 
After training, CircuitAR can generate discrete tokens progressively, which can be decoded into initial circuit structures by the decoder of the CircuitVQ. 
These prior insights can guide DAS in producing refined circuits that match the target truth tables precisely.

Our key contributions can be summarized as follows:
\begin{itemize}
\item We introduce CircuitVQ, a circuit tokenizer that facilitates graph autoregressive modeling for circuit generation, based on our Circuit AutoEncoder framework;
\item Develop CircuitAR, a model trained using masked autoregressive modeling, which generates initial circuit structures conditioned on given truth tables;
\item Propose a refinement framework that integrates differentiable architecture search to produce functionally equivalent circuits guided by target truth tables;
\item Comprehensive experiments demonstrating the scalability and capability emergence of our CircuitAR and the superior performance of the proposed circuit generation approach.
\end{itemize}

% Motivation
% (a) Diffusion (Vision, Graph), Autoregressive (Language, Vision)
% (b) Circuit Generation for Predefined Setting
% (c) Neural Architecture Search for Strict Logic Equivalence

% Contribution
% (a) Circuit Tokenizer (new transformer arch, training strategy)
% (b) CircuitAR (train and gen strategies, post-ar strategy)
% (c) Extensive Evaluation including BitD (Bit Distance) for Scalability


\section{Related Work}\label{sec:relatedwork}

Internet of Things (IoT) has seen rapid advancements in recent years, becoming an integral part of various domains, such as smart industries and homes, and serving as a key enabler in modern society.
However, despite its growth, IoT continues to face numerous security challenges, prompting significant research efforts aimed at improving IoT security.
With the rise of artificial intelligence (AI), machine learning (ML) and deep learning (DL)-based approaches have become increasingly popular in designing defense mechanisms for IoT devices, including malicious traffic classification~\cite{luo2022transformer,shafiq2020corrauc}, malware detection~\cite{vasan2020mthael,chaganti2022deep,aung2022atlas}, vulnerability discovery~\cite{neshenko2019demystifying}, and others~\cite{al2020survey,otoum2022dl,tambe2019detection}.

More recently, inspired by the success of large language models (LLMs), researchers have begun exploring the potential of LLMs to enhance IoT-related security tasks.
For instance, LLMs have been applied to existing IoT security challenges such as threat detection and fuzzing. Ferrag \etal~\cite{sokiotllm} introduced a BERT-based model, SecurityBERT, to achieve better cyber threat detection accuracy over traditional ML and DL-based methods. 
Similarly, Ma \etal~\cite{ma} and Wang \etal~\cite{llmiotfuz} proposed LLM-assisted fuzzing methods to uncover hidden bugs in IoT devices, enabling the detection of complex vulnerabilities that traditional techniques might miss.
Additionally, Yang \etal~\cite{yang2023iot} combined LLMs with static code analysis using prompt engineering to create a cost-effective solution for IoT vulnerability detection.
\cite{ji2024sevenllm} collected cybersecurity raw texts to train cybersecurity LLM to augment the analysis of cybersecurity events, and \cite{llmtikg} made use of a larger LLM to build knowledge graphs from public threat intelligence and use GPT to create datasets to fine-tune a smaller LLM to extract entities and TTPs from attack description.
Ferraris \etal~\cite{ferraris2024ici} proposed utilizing ChatGPT to enhance IoT trust semantics, aligning with W3C Web of Things (WoT) recommendations\footnote{\scriptsize \url{https://www.w3.org/WoT/}}.
This work extends the TrUStAPIS framework~\cite{ferraris2020trustapis}.

Beyond the above tasks, LLMs have been employed in other IoT challenges.
Meyuhas \etal~\cite{meyuhas2024iotlabel} used LLMs to address the problem of labeling previously unseen IoT devices.
\cite{llmiotcontrol,cui2024llmind} explored leveraging LLMs to control IoT devices and facilitate effective collaboration among them.
Mo \etal~\cite{mo2024iot} collected IoT sensor-natural language paired data and trained IoT-LM to interpret and interact with physical IoT sensors.
Xu \etal~\cite{xu2024penetrative} employed ChatGPT to interpret IoT sensor data and reason over tasks in the physical realm, introducing novel ways of integrating human knowledge into cyber-physical systems. 

Recently, Deldari \etal~\cite{deldari2024auditnet} proposed AuditNet, a conversational AI-based security assistant, which is most similar to \chatiot\ and also augmented by external knowledge.
However, AuditNet focused on standards, policies, and regulations of portable document format (PDF), and aimed to reduce the manual effort of security experts involved in compliance checks of IoT. 
On the other hand, we integrate IoT threat intelligence of various sources into \chatiot\ and can assist multiple kinds of users. Besides, we provide an end-to-end toolkit to process data in various formats, not limited to PDF. 

Together, these studies indicate that LLMs have great potential to improve the security of IoT systems in various domains, from vulnerability discovery to trustworthiness management. 
By integrating LLMs with IoT-specific threat intelligence, these models can be guided to meet the unique challenges posed by the IoT ecosystem.
Moreover, the continuous advancements in the LLM community, combined with increasingly accessible IoT datasets, are likely to further drive the adoption of LLMs in IoT-related research and practical applications.



\section{\label{sec:method}Methodology}

Each SIEM system uses its own RDL to define threat detection rules, and each RDL has its own schema.
For example, the Splunk SIEM uses the SPL to define its threat detection rules.
The task of understanding threat detection rules and recommending relevant MITRE ATT\&CK techniques (or sub-techniques) requires complex reasoning skills.
In the case of LLMs, this can be achieved with a technique called prompt chaining in which each task is divided into multiple sub-tasks in order to understand the complex reasoning behind the task.
Therefore, we employ a multi-phase architecture based on prompt chaining that leverages the power of LLMs to take a SIEM rule defined in any RDL and map it to relevant MITRE ATT\&CK techniques using the power of LLMs.
Our approach is based on the following intuitions:
\begin{itemize}[nosep,leftmargin=*]
    \item \textit{LLMs' implicit knowledge}: LLMs possess deep understanding of diverse RDLs. This enables them to interpret any rule, regardless of the RDL it is defined in, and convert it into comprehensible natural language text.
    \item \textit{LLMs' similarity comparison capability}: LLMs are adept at analyzing and comparing textual descriptions. 
    They can intelligently assess the similarity between two textual inputs to establish a meaningful connection.
\end{itemize}

\methodName has two main phases: (1) the rule to text translation phase, and (2) the MITRE ATT\&CK techniques recommendation phase.
These two phases in the pipeline include six key steps to determine relevant TTPs, as illustrated in Figure~\ref{fig:r2t}.

Although LLMs excel at translating SIEM rules into natural language, they often lack critical domain-specific contextual information related to IoCs in the rules.
To overcome this limitation, the \textit{rule to text translation} phase includes three steps: IoC extraction, contextual information retrieval, and natural language translation.

The workflow begins with the extraction of IoCs from the rules (for example, processes, log source, event codes, and file names) that the rule searches for in the logs (step (1)).In the next sstep a web search agent performs the task of obtaining additional contextual information about the IoCs discovered ((step 2)).
By incorporating this additional domain-specific information, the pipeline enhances the language translation, resulting in a more accurate and meaningful interpretation of SIEM rules.
The rule itself and the IoCs' contextual additional information from the previous stage are then used to translate the rule from RDL to natural language (step (3)).

The \textit{MITRE ATT\&CK techniques} recommendation phase of the pipeline includes the following three steps.
The rule in processed in data source identification step in which the probable origin of the data is identified (step (4)).
The description of the rule is then used to determine probable MITRE ATT\&CK techniques based on the implicit knowledge of the LLM (step (5)).
Finally, using chain-of-thought~\cite{wei2022chain} prompting, the most relevant techniques are extracted from the list of probable techniques (step (6)).
Each step of our method is further described in detail below.


% [bb=0 0 1440 900,width=1.43\linewidth,height=0.9\textwidth]
\begin{figure*}[htbp]
   \includegraphics[width=\textwidth]{Images/stages.jpg}
    
   \caption{An illustration of the different steps in \methodName.}
   \label{fig:stages}
\end{figure*} 

\subsection{IoC Extraction}
The context associated with a SIEM detection rule is crucial for its accurate interpretation and effective application. 
Obtaining this contextual understanding requires comprehensive analysis of the embedded IoCs in the SIEM rule.
In the first step, \methodName systematically identifies and extracts all IoCs, identifying the types of IoCs and their corresponding values that form the foundational elements of the detection rules. 
Leveraging the LLM's inherent understanding of rule structures and IoCs, we employ a zero-shot prompting approach for this task. 
Zero-shot prompting enables the direct extraction of IoCs from the rules without requiring extensive pre-training on specific datasets.

\noindent The result of this stage is a dictionary structure, where:
\begin{itemize}[nosep,leftmargin=*]
    \item Keys represent types of IoC, such as processes, files, IP addresses, and log sources.
    \item Values are lists containing specific IoC details, such as process names, file names, IP addresses, and log source identifiers.
\end{itemize}

In the example depicted in Figure~\ref{fig:stages}(a), the pipeline processes a rule for which relevant MITRE ATT\&CK techniques need to be recommended. 
The IoC extractor LLM produces a dictionary structure as output, organizing the IoCs in a structured format to support subsequent stages in the analysis pipeline. 



\subsection{Contextual Information Retrieval}
In this step, an LLM agent is employed to retrieve relevant information pertaining to the IoCs extracted from the rule.
A REACT agent~\cite{react} was used in this case to generate both reasoning traces and task-specific actions in an interleaved manner.
REACT agents interact with external tools to retrieve additional information that leads to more factual and reliable responses.
The LLM agent conducts a systematic search across web resources to gather additional contextual information for each IoC value present in the rule. 
This step addresses LLMS' lack of up-to-date knowledge or specialized domain expertise (which is critical to understanding the role and significance of the IoCs in the rule), without the need for retraining or fine-tuning.
Figure~\ref{fig:stages}(b) presents an example in which the rule includes the process name \texttt{soaphound.exe} as an IoC.
As can be seen, the web search results indicate that \texttt{soaphound.exe} is being used for active directory (AD) enumeration, which is important for the understanding of the attack. 

\subsection{Natural Language Translation}

The translation of detection rules into natural language textual descriptions fulfills three key objectives:
\begin{enumerate}[nosep,leftmargin=*]
    \item \textbf{Ensures that \methodName is format-agnostic}: It converts rules defined in various RDL formats into a generic, unstructured text format, ensuring compatibility with different SIEM systems, regardless of the specific rule format.
    \item \textbf{Provides contextual explanation}: It includes all relevant contextual information to produce a concise and comprehensible explanation of the rule.
    \item \textbf{Enhances the comprehension for LLMs}: It enables LLMs to more effectively compare the translated rule with descriptions in the MITRE ATT\&CK framework by providing a unified textual representation.
\end{enumerate}
To achieve these objectives, a zero-shot prompting technique is employed. 
The input to the LLM comprises two components:
\begin{itemize}
    \item \textbf{Syntactical information}: The rule itself, providing the structural and operational details.
    \item \textbf{Contextual information}: Details of the IoCs extracted from the rule, providing semantic insights into the rule's intent and function.
\end{itemize}
The LLM utilizes these inputs to generate a natural language textual description of the rule. 
This transformation not only ensures a more interpretable representation but also facilitates further steps of analysis and comparison, particularly in aligning the rule with MITRE ATT\&CK techniques and sub-techniques.



\subsection{Data Source or Mitigation Identification}
Identifying the most relevant data component or mitigation associated with the rule description in this step is critical for filtering out irrelevant MITRE ATT\&CK techniques (or sub-techniques) in subsequent steps of the pipeline.
In the MITRE ATT\&CK framework, data sources represent various categories of information that can be gathered from sensors or logs. 
These data sources include \textit{data components}, which are specific attributes or properties within a data source that are directly relevant to detecting a particular technique or sub-technique~. 
For example, in the context of the rule described in Figure~\ref{fig:stages}(a), the term \texttt{Endpoint.Processes} indicates that the activity is happening on an endpoint. 
Presence of the terms such as, \texttt{soaphound.exe}, \texttt{--buildcache}, \texttt{--certdump} and etc. indicate that the rule searches for command line execution of an executable named \texttt{soaphound.exe} with specific parameters. 
Therefore, the appropriate data source in this example is \textit{Command}, with the corresponding data component being \textit{Command Execution}.
Additionally, \textit{mitigations} are defined as categories of technologies or strategies that can prevent or reduce the impact of specific techniques or sub-techniques. 
The MITRE ATT\&CK framework explicitly establishes relationships between data components, mitigations, and techniques (or sub-techniques), enabling a systematic approach for identifying relevant elements.

To identify the most relevant data component or mitigation associated with a given rule description, we utilize agentic retrieval augmented generation (RAG), which incorporates an AI Agent-based implementation of the RAG framework.
Data from the MITRE ATT\&CK framework, specifically related to data components and mitigations, is stored in a vector database (e.g., ChromaDB). 
The process begins with the rule description from the previous stage, which serves as the input to the AI Agent. 
The LLM-powered agent automatically generates a search query tailored to retrieve relevant information from the RAG database.

For each query, the system retrieves the five most similar documents from the database, each containing contextual information about data components or mitigations. 
These documents are then utilized by the LLM agent to contextualize the rule description. 
By comparing the content of these retrieved documents with the rule description, the LLM agent determines and outputs the most relevant data component or mitigation along with a chain-of-thought as to why the data component or mitigation is related to the rule.


\subsection{Probable Technique Recommendation}

In this step, an LM agent is utilized to propose probable MITRE ATT\&CK techniques (and sub-techniques) that may be relevant to the description of the provided rule.
We used a REACT agent in this step as well to utilize both implicit and explicit knowledge during reasoning.
For explicit knowledge, the agent searches the MITRE ATT\&CK framework to obtain the list of probable techniques (and sub-techniques).
The natural language description of the rule from the previous step serves as input to the LLM agent.
The output of this stage consists of a list of JSON objects, each containing the MITRE technique ID, technique name, and technique description as seen in Figure~\ref{fig:stages}(c).

Throughout our experiments, we observed that as the number of recommendations ($k$) increases, both the framework's average recall and precision initially improve, however beyond a certain threshold of $k$, the %average 
precision begins to decline.
Based on these observations(please refer Table~\ref{tab:results3}), we selected a $k$-value of 11 to ensure a high recall.



\subsection{Relevant Technique Extraction}
In this step, \methodName refines the set of probable MITRE ATT\&CK techniques identified in the previous stage by eliminating irrelevant entries.
This step in the pipeline serves two primary purposes: (1) to enhance precision while maintaining recall achieved in previous step, and (2) to provide a clear rationale for the selection of the labels, ensuring transparency and interpretability of the mapping process.
This refinement process is grounded in the assumption that LLMs are effective for text similarity matching tasks.

The process comprises two key steps:
\begin{itemize}
    \item \textit{Rule-technique comparison}: The description of each technique in the set of probable techniques is compared with the rule description. 
    A chain-of-thought technique is then applied to elucidate the reasoning behind the association of each technique with the rule.
    \item \textit{Confidence calculation}: The generated chain-of-thought rationale for each technique (or sub-technique) is compared with the rule description to compute a relevance (or confidence) score, as done in prior work~\cite{freitas2024ai}.
    % \item \textbf{Reasoning}: \new{Add here the reasoning that it provides - explaining in NLP why it was selected...}
\end{itemize}

Techniques with higher confidence scores are deemed more relevant to the rule. 
Conversely, techniques with scores falling below a predefined threshold are excluded.
The techniques retained after this filtering step represent the most relevant techniques corresponding to the given rule's description. 


The chain-of-thought (CoT) rationale generated during the comparison of each rule to its probable technique is also provided as an output in this step.
This rationale offers a detailed natural language explanation, articulating why a particular technique is relevant to the given rule. 
Such explanations are highly valuable for security analysts, as they provide clear and transparent reasoning behind the mapping, enabling analysts to better understand and validate the association between the rule and the technique.
Other classification models proposed in previous works within this domain also suffer from the limitation of being black-box models, which lack the ability to provide clear reasoning or explanations. 
Unlike \methodName, these models fail to generate transparent, CoT rationales that explain why a particular rule is mapped to a specific technique, making them less interpretable and less useful for security analysts.

\section{Review Results}
\label{sec:results}
This section will answer all the formulated research questions based on the selected studies. The answers to the defined research questions are given in each subsection.

\input{RQ1_Results.tex}
\subsection{RQ2. Techniques Used for AEG}

Table~\ref{tbl:rq2-results-aeg} summarizes the selected exploit generation studies in our survey and the main techniques employed by them.
Note that we report the studies with their IDs, which are labeled after we perform paper grouping in Section~\ref{sec:paper-organization}.
These techniques can be classified into four technique families as we mentioned in our taxonomy (Figure~\ref{fig:taxonomy}), including AEG, Security Testing, Fuzzing, and other kinds of techniques.

\begin{table}
\centering
	\caption{Main techniques used among the reviewed AEG studies.}
	\label{tbl:rq2-results-aeg}
        \footnotesize
        \resizebox{0.98\linewidth}{!}{%
	\begin{tabular}{|lll|}
		\toprule
		\textbf{Technique Family} & \textbf{Main Technique}                          & \textbf{Studies} \\
		\midrule
		 & Control-flow Hijacking  & S01, S03-09, S54, S55 \\
		{\textit{AEG}} & Crash Analysis-based  & S10 \\
		& Data-oriented  & S11 \\
		\midrule
		 &  Combinational Testing & S26 \\
		& Grammar-based Testing &  S27-29\\
		& Heuristic-based Testing & S30-31 \ifthenelse{\boolean{deliverable}}
{}{\mc{Not in the big Table 5}} \\
		& Hybrid Testing &  S32\\
		& Learning-based Testing &  S33, S34\\
		& Logic programming-based Testing & S35\\
		\textit{Security Testing} & Model-based Testing & S36-39 \\
		& Mutation-based Testing & S40-42\\
		& Pentesting & S43\\
		& Robustness Testing & S44\\
		& Search-based Testing & S45, S46 \\
		& Symbolic Execution-based Testing & S47 \\
		& Threat model-based Testing & S08, S48-50 \\
		\midrule
		& Coverage-based Greybox Fuzzing & S04, S13-16 \\
		& Directed Fuzzing & S17\\
		\multirow{0}{*}[0.15cm]{\textit{Fuzzing}} & Knowledge-based Fuzzing & S18\\
		& Mutation-based Fuzzing & S19 \\
		& Stateful Fuzzing & S20\\
		& Whitebox Fuzzing & S21, S22\\
		\midrule
		& Dynamic Taint Analysis & S60-63\\
		& Dynamic Symbolic Execution & S53, S55-58 \\
		\textit{Others} & Static Symbolic Execution & S59 \\
		& State model-based String Analysis & S27\\
		& Static String Analysis & S52\\
		\bottomrule     
	\end{tabular}
 }
\end{table}

\subsubsection{AEG}
In our 63 selected studies, twelve are dedicated to the AEG. The main approaches can be divided into three groups, i.e., Control-flow hijacking AEG, Crash Analysis-based AEG, and Data-oriented AEG. Control-flow Hijacking is the dominant technique with 10/12 AEG studies in our survey while we found only one for each of Crash Analysis-based and Data-oriented AEG. The popular approaches implementing Control-flow Hijacking AEG often require the program binaries as the inputs and consist four steps: (i) identify the vulnerability, (ii) obtain runtime information, (iii) generate the exploits, and (iv) verify the exploits. Meanwhile, Crash Analysis-based AEG (S10~\cite{Liu202271}) requires the crash information as the extra inputs, then this technique extract the execution trace tries and reproduce the crash under symbolic mode. If an exploitable state is found, a solver is utilized to resolve the path contraints to find malicious inputs and synthesize the exploit. Data-oriented AEG (S11~\cite{Pewny2019111}), on the other hand, tries manipulate the data control path instead of the execution control flow of the targeted program. This technique generate the data-oriented programming (DOP) attacks in the form of a high-level language, then compile them into concrete exploits for each kind of compilers.

\subsubsection{Security Testing}
Security testing has the majority of studies in our survey with a very wide range of testing techniques. They expand in multiple classes of testing techniques: from model-based testing, logic programming-based testing, heuristic-based testing, grammar-based testing, search-based testing to hybrid testing, learning-based testing, and pentesting. Xu et al. contribute to the threat model based security testing with their four works (S08, S48-S50~\cite{marback2013threat, xu2011tool, xu2012automated, Xu2015247}). They first manually contruct the PetriNets as the threat models, from which they generate the attack paths, and lastly the security test cases. Their approach can be applied to generate security tests in cross-language projects such as PHP and C/C++.
In the work of Zech et al. (S35~\cite{Zech201488}), they generates negative security tests based on a answer set programming that they collect from the developers. Chaleshtari et al. (S31~\cite{Chaleshtari20233430}) employed heuristic-based testing, where they define of 120 Metamorphic relations with domain-specific language for specifying security properties in Java code and then use them for generate security test cases. In S27~\cite{Mohammadi201678}, Mohammadi et al. used unit testing to generate Cross-side scripting (XSS) tests for Java Jakarta Server Pages (JSP). The inputs for the tests are provided by a grammar-based generator, this techniques were used detect zero-day XSS vulnerabilities. Del Grosso et al. (S35~\cite{DelGrosso20083125}) tried to generate test input to detect buffer overflow in C/C++ programs by leveraging search-based testing. The genetic algorithm they used gives rewards to the test inputs reaching the vulnerable statements. In this way, they can find the inputs that can trigger the vulnerabilities. Del Grosso et al. (S34~\cite{Liu2020286}) applied recent advances in deep learning for natural language processing to learn the semantic knowledge from code snippets vulnerable to SQL Injection attacks to generate the malicious inputs and then translate these inputs into security test cases.

\subsubsection{Fuzzing}
We acknowledge that there are a plethora of fuzzing techniques in the literature. However, as mentioned in the Section~\ref{sec:intro} and Section~\ref{sec:background}, we set our scope to the techniques that are specific designed to find vulnerabilities at the first place. So many of the fuzzing techniques (do not focus on security vulnerabilities, yet can be used to discover security bugs) are of our scope except for the ones in Table~\ref{tbl:rq2-results-aeg}. Coverage-based greybox fuzzing are employed by the most fuzzing studies (S04, S13-16~\cite{Bohme2019489, Liu2018, Wuestholz20201398, Gong2022374}) in our survey. S13~\cite{Bohme2019489} is one of the well-known work which are built on American Fuzzy Lop (AFL) tool framework and leverages the Markov chain as modeling a systematic exploration of the state space for finding inputs. Other fuzzing techniques can be named for finding vulnerabilities such as Knowledge-based fuzzing, Directed fuzzing, Mutation-based fuzzing, Stateful fuzzing, and Whitebox fuzzing.

\subsubsection{Others}
Some other techniques utilized for generating vulnerability exploits that we found are also well-known such as Taint analysis, Symbolic execution, and String analysis. Note that these techniques are frequently employed as parts of the other exploit generation studies, however, we only report the primary techniques. In this family, we list the studies that are not eligible to be classified into the big three technique families AEG, Security testing, and Fuzzing.
\subsection{RQ3. Targets of AEG Techniques}

\subsubsection{Targeted vulnerability types and programming languages}
As reported in Table~\ref{tbl:rqs-g1}, the majority of vulnerabilities that the AEG techniques focus on are memory-based vulnerabilities in C/C++ programs such as buffer overflow~\cite{Xu2022, shahriar2009automatic, DelGrosso20083125, Dixit2021}, heap overflow~\cite{Huang2019}, while the security testing techniques focus on web-based injection vulnerabilities such as XSS and SQL Injection~\cite{Bozic202020, zhang2010d, aydin2014automated, Huang2013208}, which are common in PHP- and Java-based applications.
Besides, fuzzing techniques such as~\cite{Bohme2019489, Yu2022, KallingalJoshy2021540}, try to explore the input space to trigger the crash states of both C/C++ and Java programs.
Also, some other techniques focus on specific types of vulnerabilities; for example, Tang et al.~\cite{Tang2017492} and Garcia et al.~\cite{Garcia2017661} tried to generate exploits for Android ICC vulnerabilities, Atlidakis et al.~\cite{Atlidakis2020387} tried to generate the request inputs that can trigger one of four RESTful security violations that they defined themselves.

\subsubsection{Targeted inputs}
Most of the techniques work on the binaries (usually the AEG and fuzzing techniques such as~\cite{Liu2018705, Liu202271, Gong2022374, Zhang201946}) or the source code (security testing techniques and the techniques that require performing symbolic execution or taint analysis such as~\cite{Chen20201580, DelGrosso20083125, avancini2012security, dao2011security, wu2018fuze}) of the targeted programs. Meanwhile, some other techniques focus on web-based vulnerabilities require a running instance of the applications with their APIs~\cite{Appelt2014259, Jan201612, Atlidakis2020387} or UIs~\cite{Simos2019122, Liu2016123, Bozic2020115, Huang2013208} exposed. Some specific techniques, such as model-based security testing, take the special inputs as the program models to find and exploit the vulnerabilities at abstract level~\cite{Pretschner2008338, Lebeau2013445, Khamaiseh2017534}.
\subsection{RQ4. Output of AEG techniques}

As shown in Table~\ref{tbl:rqs-g1}, we categorize the outputs of the tools into groups: (1) exploit, which is either a code snippet or a complete program that can be executed to exploit the vulnerabilities, (2) test cases, which are usually associated and can be executed with a testing framework such as \textsc{JUnit} or \textsc{Selenium} to find evidence of the vulnerability's presences in the project, and (3) input/payload which consists of crafted input values or objects to trigger the vulnerable states of the asset.

It can be foreseen that most AEG techniques (studies S01 -- S11) target generating working exploits that find not only the vulnerable states but also the exploitable states of the target programs.
\ifthenelse{\boolean{deliverable}}
{}
{\ema{We need to clarify (earlier) the difference between exploitable and vulnerable states because it's interesting.}}
However, Do et al.~\cite{Do2015401} proposed an AEG technique that generates an exploit in the form of a test case, which can be executed via the Java \textsc{JUnit} testing framework. In contrast to AEG, fuzzing techniques tend to generate the inputs or payloads that can aim to trigger the only vulnerable state without further exploitation actions.
This is true except for some fuzzing work where they also put more effort into reaching the exploitable state of the vulnerabilities~\cite{Bohme2019489}.
Most security testing techniques aim to generate security test cases to find vulnerabilities or security weaknesses in a more methodical way compared to AEG and Fuzzing approaches. Many testing techniques also aim to generate the malicious inputs~\cite{Bozic202020, Bozic2020115, Liu2020286, Appelt2014259} for triggering web-based vulnerabilities and exploits for memory-based vulnerabilities in program binaries~\cite{Nurmukhametov202137}.
\subsection{RQ5. Automation of AEG techniques}

As shown in Table~\ref{tbl:rqs-g1}, more than 80\% (54 out of 66) of the selected studies claimed in their papers that their tools can be fully automatically executed without any human interventions. 
We cannot verify the automation level of these techniques by ourselves as we do not have access to their tools (this will be discussed more in the results of RQ7 in Section~\ref{sub:rq7-results}).

Note that some techniques may require extra inputs, which must be provided before the tool is launched. At the same time, other tools require additional manual help from humans in between to operate the tools. For example, Zech et al.~\cite{Zech201488} collect feedback from developers to construct their logic programming answer set for generating security tests. In the studies focusing model-based~\cite{Pretschner2008338, Lebeau2013445, Khamaiseh2017534} and threat model-based~\cite{xu2011tool, xu2012automated, Xu2015247, marback2013threat} security testing, the program model creation and modification are also required to be done manually, which assist the automated process of generating test cases based on the models.

% or semi-automatically with a few manual steps in between. Some techniques require an interactive process for their executions such as xxx.
\subsection{RQ6. Evaluation of AEG techniques}

\ifthenelse{\boolean{deliverable}}
{}{\cuong{This section is not finished}}

Table~\ref{tbl:rq6-results} shows the types of datasets used for the evaluation of the techniques in the studies. Most of the techniques were evaluated on real-world applications with real-world CVE vulnerabilities. While others were experienced with artificial vulnerabilities, which are created for the purposes of security research or security competitions such as Capture-The-Flag competitions. More interestingly, some techniques were evaluated on both artificial and real-world datasets.

\begin{table}
\centering
	\caption{Datasets used to evaluate AEG techniques.}
	\label{tbl:rq6-results}
        \resizebox{0.98\linewidth}{!}{%
	\begin{tabular}{lp{12cm}}
		\toprule
		\textbf{Dataset type}      & \textbf{Studies} \\
		\midrule
		Artificial &  S04, S05, S09, S15, S24, S25, S27, S29, S30, S36, S38, S55, S58 \ifthenelse{\boolean{deliverable}}{}
{\mc{?}} \\
		Real-world &  S03, S04, S08, S13, S14, S16-22, S27, S28, S31, S33, S34, S39-42, S44, S45, S47-50, S52, S53-57, S59-62 \ifthenelse{\boolean{deliverable}}
{}{\mc{?}} \\
		Toy/PoC & S11, S35, S37 \\
		Artificial + Real-world & S01, S06, S08, S10, S32, S43 \\
		\bottomrule     
	\end{tabular}
        }
\end{table}
\subsection{RQ7. Availability and Usability of AEG techniques}
\label{sub:rq7-results}

Most of the studies propose concrete techniques for finding exploits or generating test cases; however, they do not come with the available tools. This can be explained by the security concerns of publicly providing these tools, which can aid malicious users in arming their weapons to find and exploit vulnerabilities in the real world.
Several techniques can be strictly shared for research purposes (i.e., the authors authorize other researchers and practitioners to access their tools and replication artifacts upon a proper request)~\cite{Appelt2014259, Xu2022, Garcia2017661}.
Another explanation might be the prototypical nature of the approaches presented. The authors could not feel comfortable releasing a not-ready tool or one that was too difficult to use without the original authors' support.
%
Table~\ref{tbl:rq7-results} shows the list of tools we found in the selected studies that are publicly available for everyone to use. Most of them are fuzzers or AEG tools.

\begin{table}
\centering
	\caption{Catalog of available AEG tools.}
	\label{tbl:rq7-results}
        \tiny
        \resizebox{0.98\linewidth}{!}{%
	\begin{tabular}{llll}
		\toprule
		\textbf{Study} & \textbf{Pub.} & \textbf{Tool name}  & \textbf{Tool link} \\
		\midrule
		S15 & \cite{Alshmrany202185}  & \textsc{FuSeBMC} & \url{https://github.com/kaled-alshmrany/FuSeBMC} \\ 
	    S24 & \cite{shoshitaishvili2018mechanical}  & \textsc{Mechaphish} & \url{https://github.com/mechaphish} \\
		S58 & \cite{Dixit2021}  & \textsc{AngErza} & \url{https://github.com/rudyerudite/AngErza} \\
		S06 & \cite{Wei2019120152}  & \textsc{AutoROP} & \url{https://github.com/wy666444/auto_rop} \\
		S07 & \cite{Brizendine202277}  & \textsc{JOPRocket} & \url{https://github.com/Bw3ll/JOP_ROCKET/} \\
		S04 & \cite{Liu2018}  & \textsc{CAFA} & \url{https://github.com/CAFA1/CAFA} \\
		S13 & \cite{Bohme2019489}  & \textsc{AFLFast} & \url{https://github.com/mboehme/aflfast} \\
		S \ifthenelse{\boolean{deliverable}}
{}{\mc{?}} & \cite{Chaleshtari20233430}  & \textsc{SMRL} & \url{https://sntsvv.github.io/SMRL/} \\
		S34 & \cite{Liu2020286}  & \textsc{DeepSQLi} & \url{https://github.com/COLA-Laboratory/issta2020} \\
		S39 & \cite{Nurmukhametov202137}  & \textsc{Majorca} & \url{https://github.com/ispras/rop-benchmark} \\
		S57 & \cite{wu2018fuze}  & \textsc{FUZE} & \url{https://github.com/ww9210/Linux_kernel_exploits} \\
		
		\bottomrule     
	\end{tabular}
        }
\end{table}



%\subsection{RQ8. Workflow of AEG techniques}
%\subsection{RQ9. Security knowledge required for using AEG techniques}

\section{Limitations} 

In this work, we compared the effectiveness and interplay of SFT and RL-based methods, under fixed data constraints. In particular, we chose offline methods like DPO and KTO as the baseline implementation of the RL method because it eliminates the need for reward modeling or iterative finetuning. This means that the process of development is limited to collecting an offline dataset and fientuning it - making it the most fair comparable to SFT in terms of implementation effort, compute costs and annotation efforts. Since this baseline RL method shows optimal performance over SFT, we hope that this motivates future work to study more complex RL-based methods and their interplay with SFT. In addition, we used GPT4o annotation for synthetic data generation, and also for evaluating Summarization and Helpfulness, which could include potential biases inherited from the model. 

In addition, we limited the size of the model to under 10 Billion parameters, to keep the finetuning cost low enough to ignore as compared to the data annotation costs. In addition, it would be extremely compute resource intensive to run thousands of finetuning runs with larger model sizes like 70B parameters. We hope that future work would study the scaling trends of RL-based methods against different model sizes, and also study the compute-data trade-off in-depth.


%\section{Discussion}
%\subsection{Open research challenges and future directions}

\section*{Conclusion}
This paper aims to enhance our understanding of the computational complexity of computing various Shapley value variants. We found that for various ML models --- including decision trees, regression tree ensembles, weighted automata, and linear regression --- both local and global interventional and baseline SHAP can be computed in polynomial time under HMM modeled distributions. This extends popular algorithms, such as TreeSHAP, beyond their empirical distributional scope. We also establish strict complexity gaps between the various SHAP variants (baseline, interventional, and conditional) and prove the intractability of computing SHAP for tree ensembles and neural networks in simplified scenarios. Overall, we present SHAP as a versatile framework whose complexity depends on four key factors: \begin{inparaenum}[(i)] \item model type, \item SHAP variant, \item distribution modeling approach, \item and local vs. global explanations\end{inparaenum}. We believe this perspective provides deeper insight into the computational complexity of SHAP, paving the way for future work.




%We believe that our framework provides a more intricate understanding of SHAP computation complexity across different models, distributions, and variants, paving the way for further research.

Our work opens promising directions for future research. First, expanding our computational analysis to other SHAP-related metrics, such as asymmetric SHAP~\citep{frye20} and SAGE~\citep{covert2020understanding}, would be valuable. Additionally, we aim to explore more expressive distribution classes and relaxed assumptions beyond those in Section \ref{sec:tractable} while maintaining tractable SHAP computation. Finally, when exact computation is intractable (Section \ref{sec:intractable}), investigating the approximability of SHAP metrics through approximation and parameterized complexity theory~\citep{downey2012parameterized} is an important direction.

%Our work opens several promising avenues for future research on the computational properties of explainable AI methods, with a particular focus on SHAP. First, it would be interesting to broaden the computational analysis conducted in this work to include other popular SHAP-related metrics in the literature, such as asymmetric SHAP \cite{frye20} and SAGE \cite{covert2020understanding}. Also, in the future, we aim to explore more expressive distribution classes and relaxed distributional assumptions—extending beyond those examined in Section \ref{sec:tractable} —that still yield tractable SHAP computation. Finally, when exact computation proves intractable (Section \ref{sec:intractable}), it is worthwhile to theoretically investigate the question of the approximability of computing the SHAP metrics across various configurations, through the lens of approximation and parametrized complexity theory \cite{arora2009computational}.

%This paper aims to deepen our understanding of the computational complexity involved in obtaining different Shapley value variants. We found that for a variety of ML models, including decision trees, tree ensembles for regression, weighted automata, and linear regression models — computing both local and global interventional and baseline SHAP can be done in polynomial time when distributions are modeled by HMMs. This extends the distributional scope of popular algorithms like TreeSHAP, which is limited to empirical distributions. Additionally, we demonstrate a strict complexity gap between SHAP variants, showing that interventional and baseline SHAP can be strictly easier to compute than conditional SHAP. Despite these positive results, we uncovered intractability for various SHAP variants in neural networks and tree ensembles. Finally, we provided generalized complexity relations across SHAP variants. We believe that our framework offers a deeper understanding of the complexity involved in computing SHAP across various variants, models, distributions, as well as in both local and global computations, laying the groundwork for future research.

\begin{acks}
This work was partially supported by EU-funded project Sec4AI4Sec (grant no. 101120393).
\end{acks}

\bibliographystyle{ACM-Reference-Format}
\bibliography{references}

\end{document}
