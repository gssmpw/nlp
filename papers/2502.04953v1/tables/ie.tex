\begin{table}
\centering
	\caption{Exclusion criteria for selecting papers during the relevance assessment process. The initial set of criteria before starting the ``calibration'' are marked with \ding{51}.}
	\label{tbl:relevance-criteria-final}
        \resizebox{0.98\linewidth}{!}{%
	\begin{tabular}{|l|c|l|p{8cm}|}
        \hline
        \textbf{ID} & \textbf{Initial}          & \textbf{Name}                           & \textbf{Explanation}                                                                                      \\ \hline
        E1          & \ding{51} & Vulnerability detection only            & Vulnerability detection technique that does not generate exploits \\
        E2          & \ding{51} & Manual approach                         & Approach that is purely manual to perform                                                                 \\
        E3          &                            & Targeting known vulnerabilities         & Tool re-using available payloads or scripts to find known CVEs, such as Metasploit \cite{kennedy2011metasploit}                     \\
        E4          & \ding{51} & Targeting client side                   & Technique generating exploits at client side, such as SIEGE \cite{iannone2021toward} \\
        E5          & \ding{51} & Targeting hardware-level solutions      & Approach targeting hardware-level solutions such as Automotive, SmartGrid, CSP, IoT, etc.                   \\
        E6          &                            & Malware or network related              & Technique targeting malware, or network-related vulnerabilities                                          \\
        E7          &                            & Survey or evaluation                    & Study conducting a survey, or evaluation, but does not introduce any novel technique                   \\
        E8          &                            & Not applied to security vulnerabilities & Study not clearly applicable to security vulnerabilities \\
        E9          &                            & Testing for security-specific software  & Technique targeting security-specific software, e.g. testing for firewall software                       \\
        E10         & \ding{51} & Collection tool                         & Tool-chain or framework aggregating multiple other already existing tools                                      \\
        E11         &                            & Out of scope                            & Technique relative to other fields other than computer science, e.g. chemical, biology, etc.                          \\
        E12         & \ding{51} & Others                                  & Other reasons                                                                                             \\ \hline
        \end{tabular}
    }
\end{table}