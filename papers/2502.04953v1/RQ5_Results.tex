\subsection{RQ5. Automation of AEG techniques}

As shown in Table~\ref{tbl:rqs-g1}, more than 80\% (54 out of 66) of the selected studies claimed in their papers that their tools can be fully automatically executed without any human interventions. 
We cannot verify the automation level of these techniques by ourselves as we do not have access to their tools (this will be discussed more in the results of RQ7 in Section~\ref{sub:rq7-results}).

Note that some techniques may require extra inputs, which must be provided before the tool is launched. At the same time, other tools require additional manual help from humans in between to operate the tools. For example, Zech et al.~\cite{Zech201488} collect feedback from developers to construct their logic programming answer set for generating security tests. In the studies focusing model-based~\cite{Pretschner2008338, Lebeau2013445, Khamaiseh2017534} and threat model-based~\cite{xu2011tool, xu2012automated, Xu2015247, marback2013threat} security testing, the program model creation and modification are also required to be done manually, which assist the automated process of generating test cases based on the models.

% or semi-automatically with a few manual steps in between. Some techniques require an interactive process for their executions such as xxx.