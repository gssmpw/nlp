\subsection{RQ7. Availability and Usability of AEG techniques}
\label{sub:rq7-results}

Most of the studies propose concrete techniques for finding exploits or generating test cases; however, they do not come with the available tools. This can be explained by the security concerns of publicly providing these tools, which can aid malicious users in arming their weapons to find and exploit vulnerabilities in the real world.
Several techniques can be strictly shared for research purposes (i.e., the authors authorize other researchers and practitioners to access their tools and replication artifacts upon a proper request)~\cite{Appelt2014259, Xu2022, Garcia2017661}.
Another explanation might be the prototypical nature of the approaches presented. The authors could not feel comfortable releasing a not-ready tool or one that was too difficult to use without the original authors' support.
%
Table~\ref{tbl:rq7-results} shows the list of tools we found in the selected studies that are publicly available for everyone to use. Most of them are fuzzers or AEG tools.

\begin{table}
\centering
	\caption{Catalog of available AEG tools.}
	\label{tbl:rq7-results}
        \tiny
        \resizebox{0.98\linewidth}{!}{%
	\begin{tabular}{llll}
		\toprule
		\textbf{Study} & \textbf{Pub.} & \textbf{Tool name}  & \textbf{Tool link} \\
		\midrule
		S15 & \cite{Alshmrany202185}  & \textsc{FuSeBMC} & \url{https://github.com/kaled-alshmrany/FuSeBMC} \\ 
	    S24 & \cite{shoshitaishvili2018mechanical}  & \textsc{Mechaphish} & \url{https://github.com/mechaphish} \\
		S58 & \cite{Dixit2021}  & \textsc{AngErza} & \url{https://github.com/rudyerudite/AngErza} \\
		S06 & \cite{Wei2019120152}  & \textsc{AutoROP} & \url{https://github.com/wy666444/auto_rop} \\
		S07 & \cite{Brizendine202277}  & \textsc{JOPRocket} & \url{https://github.com/Bw3ll/JOP_ROCKET/} \\
		S04 & \cite{Liu2018}  & \textsc{CAFA} & \url{https://github.com/CAFA1/CAFA} \\
		S13 & \cite{Bohme2019489}  & \textsc{AFLFast} & \url{https://github.com/mboehme/aflfast} \\
		S \ifthenelse{\boolean{deliverable}}
{}{\mc{?}} & \cite{Chaleshtari20233430}  & \textsc{SMRL} & \url{https://sntsvv.github.io/SMRL/} \\
		S34 & \cite{Liu2020286}  & \textsc{DeepSQLi} & \url{https://github.com/COLA-Laboratory/issta2020} \\
		S39 & \cite{Nurmukhametov202137}  & \textsc{Majorca} & \url{https://github.com/ispras/rop-benchmark} \\
		S57 & \cite{wu2018fuze}  & \textsc{FUZE} & \url{https://github.com/ww9210/Linux_kernel_exploits} \\
		
		\bottomrule     
	\end{tabular}
        }
\end{table}

