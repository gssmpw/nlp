\section{Background and Related Work}
\label{sec:background}
In this section, we first introduce the definitions of the common terms we will use hereafter in this paper. Then, we summarize the other survey works that partially share the relevant interests of our study and discuss the differences between them and ours.

\subsection{Base Definitions}
\label{subsec:definitions}
\noindent\textbf{Vulnerability.} We adhere to the definition of the US National Institute of Standards and Technology (NIST), which describes a vulnerability as \textit{``a weakness in an information system, system security procedures, internal controls, or implementation that could be exploited or triggered by a threat source''}.\footnote{Definition of vulnerability according to NIST: \url{https://csrc.nist.gov/glossary/term/vulnerability}}

\noindent\textbf{Exploit.} There is no standardized definition for an exploit. Hence, we define it as a code snippet, program, user input, or test case designed and crafted to take advantage of a software application or system security vulnerability. Our definition is in line with the one given by Cisco.\footnote{Cisco's definition of an exploit: \url{https://www.cisco.com/c/en/us/products/security/advanced-malware-protection/what-is-exploit.html}}

\noindent\textbf{Automated Exploit Generation (AEG).} AEG refers to the class of techniques that aims to generate a working exploit for a given vulnerability in an automated manner.
The concept has gained popularity since it was used by Avgerinos et al.____.
An AEG approach can employ other automated program analysis and testing techniques like symbolic execution or genetic algorithms.
\ifthenelse{\boolean{deliverable}}
{}
{\ema{Need some references for SE-based and GA-based AEG.}}

%\subsection{Scope}
%\cuong{Declare a clear scope: AEG and Security Testing, not to focus to much on Fuzzing}

\subsection{Related Secondary Studies}
\label{subsec:related}

Over the years, several secondary studies concerning exploit generation and security testing in general have been published.
%
Table~\ref{tbl:related-surveys} highlights the main differences between six secondary studies related to our work. The first three columns in Table~\ref{tbl:related-surveys} indicate the references, published year, and the type of secondary studies.
In this respect, we mapped four types:
\begin{itemize}
    \item \textbf{Comparison} study, comparing the main characteristics of several tools and techniques, analyzing what they can do and what they cannot.
    \item \textbf{State-of-the-art (SoTA)} summary, recapping the main facts on the matter, like explaining the various security testing activities.
    \item \textbf{Mapping study}, presenting the key characteristics and findings found in research papers on the matter.
    \item \textbf{Systematic literature review (SLR)}, presenting a fully-fledged SLR and answering specific research questions to generate new knowledge.
\end{itemize}

\putsec{related}{Related Work}

\noindent \textbf{Efficient Radiance Field Rendering.}
%
The introduction of Neural Radiance Fields (NeRF)~\cite{mil:sri20} has
generated significant interest in efficient 3D scene representation and
rendering for radiance fields.
%
Over the past years, there has been a large amount of research aimed at
accelerating NeRFs through algorithmic or software
optimizations~\cite{mul:eva22,fri:yu22,che:fun23,sun:sun22}, and the
development of hardware
accelerators~\cite{lee:cho23,li:li23,son:wen23,mub:kan23,fen:liu24}.
%
The state-of-the-art method, 3D Gaussian splatting~\cite{ker:kop23}, has
further fueled interest in accelerating radiance field
rendering~\cite{rad:ste24,lee:lee24,nie:stu24,lee:rho24,ham:mel24} as it
employs rasterization primitives that can be rendered much faster than NeRFs.
%
However, previous research focused on software graphics rendering on
programmable cores or building dedicated hardware accelerators. In contrast,
\name{} investigates the potential of efficient radiance field rendering while
utilizing fixed-function units in graphics hardware.
%
To our knowledge, this is the first work that assesses the performance
implications of rendering Gaussian-based radiance fields on the hardware
graphics pipeline with software and hardware optimizations.

%%%%%%%%%%%%%%%%%%%%%%%%%%%%%%%%%%%%%%%%%%%%%%%%%%%%%%%%%%%%%%%%%%%%%%%%%%
\myparagraph{Enhancing Graphics Rendering Hardware.}
%
The performance advantage of executing graphics rendering on either
programmable shader cores or fixed-function units varies depending on the
rendering methods and hardware designs.
%
Previous studies have explored the performance implication of graphics hardware
design by developing simulation infrastructures for graphics
workloads~\cite{bar:gon06,gub:aam19,tin:sax23,arn:par13}.
%
Additionally, several studies have aimed to improve the performance of
special-purpose hardware such as ray tracing units in graphics
hardware~\cite{cho:now23,liu:cha21} and proposed hardware accelerators for
graphics applications~\cite{lu:hua17,ram:gri09}.
%
In contrast to these works, which primarily evaluate traditional graphics
workloads, our work focuses on improving the performance of volume rendering
workloads, such as Gaussian splatting, which require blending a huge number of
fragments per pixel.

%%%%%%%%%%%%%%%%%%%%%%%%%%%%%%%%%%%%%%%%%%%%%%%%%%%%%%%%%%%%%%%%%%%%%%%%%%
%
In the context of multi-sample anti-aliasing, prior work proposed reducing the
amount of redundant shading by merging fragments from adjacent triangles in a
mesh at the quad granularity~\cite{fat:bou10}.
%
While both our work and quad-fragment merging (QFM)~\cite{fat:bou10} aim to
reduce operations by merging quads, our proposed technique differs from QFM in
many aspects.
%
Our method aims to blend \emph{overlapping primitives} along the depth
direction and applies to quads from any primitive. In contrast, QFM merges quad
fragments from small (e.g., pixel-sized) triangles that \emph{share} an edge
(i.e., \emph{connected}, \emph{non-overlapping} triangles).
%
As such, QFM is not applicable to the scenes consisting of a number of
unconnected transparent triangles, such as those in 3D Gaussian splatting.
%
In addition, our method computes the \emph{exact} color for each pixel by
offloading blending operations from ROPs to shader units, whereas QFM
\emph{approximates} pixel colors by using the color from one triangle when
multiple triangles are merged into a single quad.



The next two columns show whether the work has systematically surveyed the literature and whether the study raised and answered some research questions.
The symbol ``\ding{51}'' indicates the full application of that aspect, while the symbol ``\ding{109}'' indicates no application at all.
Furthermore, the symbol ``\ding{119}'' indicates partial application.
The last three columns show if the study covered the main domains onto which we focus in this study: \textit{AEG}, \textit{Security Testing}, and \textit{Fuzzing}.
\ifthenelse{\boolean{deliverable}}
{}
{\ema{We need to clarify the meaning of these categories. In particular, what is AEG here? Besides, also "security testing" is rather general.}}

As seen in Table~\ref{tbl:related-surveys}, most of the existing surveys follow a systematic approach, except for the works of Shahriar et al.____, and Felderer et al.____, which do not explain the process they followed.
Only two works derived an explicit set of research questions, aiming to answer them and generate new knowledge on the matter____.
\ifthenelse{\boolean{deliverable}}
{}
{\ema{Add a more detailed explanation of what was done in such works. Also, the SLR on search-based security testing published recently should be added.}}
Regarding the domains, all related studies we found focus on security testing and do not include any works from AEG or Fuzzing for review. 
%
In this respect, this study aims to fill the gaps in the coverage of domains for AEG and Security testing techniques.
\ifthenelse{\boolean{deliverable}}
{}
{\ema{This explanation about fuzzing should be moved in Section 3 and mentioned partially in Section 1. I don't see here fitting that much}}
Despite fuzzing itself was not in our scope at first as it can be considered as a descendant of security testing; however, we still group a number of the techniques that we found in our study as fuzzing techniques (denote with the symbol ``\ding{119}'' in the Table~\ref{tbl:related-surveys}, meaning \textit{``partially cover''}).