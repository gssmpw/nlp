\section{Survey Method}
\label{sec:methodology}
This paper follows the guideline for conducting systematic reviews in software engineering introduced by Kitchenham and Charters~\cite{keele2007guidelines}, which suggests performing a systematic literature review in three phases: planning the review, conducting the review, and reporting the review results.
To this end, we first identify the need for our review and formulate the research questions.
In the second phase, we define the search string and inclusion and exclusion criteria to design our strategy for finding and selecting the primary studies of automated exploit generation.
Lastly, we review the selected papers in-depth and report our review results.
%
In this section, we first introduce the set of research questions underlying our survey (Section~\ref{sec:research-questions}). Next, we discuss how the papers from the literature are collected (Section~\ref{sec:paper-collection}) and the criteria that we derived and used to select the studies for the inclusion of our survey (Section~\ref{sec:relevance-criteria}). After all the papers had been collected, they were organized, categorized, and reviewed (Section~\ref{sec:paper-organization}).
\ifthenelse{\boolean{deliverable}}
{}
{\ema{We should also define a quality assessment form.}}

\subsection{Survey Goal and Research Questions}
\label{sec:research-questions}

The \textit{goal} of this study is to survey the literature on techniques and approaches to automatically generate exploits for software vulnerabilities.
The \textit{purpose} of this is to understand which techniques and approaches exist, their peculiarities, and under which circumstances they can be used.
This will provide security practitioners and researchers with a catalog of usable tools and elements for possible advancements in AEG.

Specifically, this study aims to achieve two main research objectives ($RO$):
\begin{itemize}
    \item \textbf{$RO_1$.} Analyze the research on AEG made so far and their key characteristics.
    \item \textbf{$RO_2$.} Evaluate the maturity of existing AEG techniques and their usability.
\end{itemize}

Therefore, we formulated seven research questions, shown in Table~\ref{tbl:research-questions}.
$RQ_1$--$RQ_5$ address $RO_1$, $RQ_6$--$RQ_7$ contribute to $RO_2$.
\ifthenelse{\boolean{deliverable}}
{}
{\ema{To revise better and provide the ratio for all objectives we tackle.}}

\begin{table}
\centering
\caption{Study research questions and their related objectives.}
%\longcaption{\columnwidth}{\textit{G1} = fundamental research questions, \textit{G2} = evaluation and maturity assessment research questions.}% \textit{G3} = open research questions.}
\label{tbl:research-questions}
\begin{tabular}{|c|l|}
   \toprule
   \textbf{Objective} & \textbf{Research Question} \\
   \midrule
   {\multirow{5}{*}{$RO_1$}} & RQ1. What is the publication trend concerning AEG? \\
   & RQ2. What kind of AEG techniques exist? \\
   & RQ3. What are the targets of AEG techniques? \\
   & RQ4. What is the form of the generated output of AEG techniques? \\% and how are the generated output of AEG techniques assessed? \\
    & RQ5. What is the degree of automation of AEG techniques? \\
   \midrule
   {\multirow{2}{*}{$RO_2$}} & RQ6. How were AEG techniques evaluated? \\
   & RQ7. Are the known AEG techniques available/usable? \\
%   \midrule
%   {\multirow{0}{*}[0.15cm]{\textit{G3}}} & RQ8. Can a standard workflow by applied for all the AEG techniques? \\
%   & RQ9. What level of security is required for the users for running AEG techniques? \\
   \bottomrule 
\end{tabular}
\end{table}

\subsection{Paper Collection}
\label{sec:paper-collection}
%The collection process of the papers in our study follows the procedure (name an active searching here~\cite{}), which includes two main steps: \textit{Active Searching} and \textit{Snowballing}.
To collect the studies for our survey, we designed the search string as \texttt{``( automat* AND software AND ( secur* OR vulnerab* ) AND ( test OR exploit ) AND generat* )''}.
%
With this, we aimed to capture papers concerning \textbf{automated techniques that can generate exploits or tests for software vulnerabilities}.
We maximize the chances to hit more papers by using the asterisk operator, for example, the search keyword \textit{``vulnerab*''} can retain the papers which contain the word \textit{``vulnerable''} or \textit{``vulnerability''} or \textit{``vulnerabilities''}.
The query was run only on the title, abstract, and keyword content, which is generally sufficient to capture the most relevant results.

\ifthenelse{\boolean{deliverable}}
{We performed our search on the \textsc{Scopus} database.\footnote{\textsc{Scopus} website: \url{www.scopus.com}} 
\textsc{Scopus} was chosen as the main database for surveying the literature as it covers the publications from all other popular choices, such as IEEE Xplore,\footnote{IEEE Xplore website: \url{https://ieeexplore.ieee.org/}} ACM Digital Library,\footnote{ACM Digital Library website: \url{https://dl.acm.org/}} and Web of Science.\footnote{Web of Science website: \url{https://clarivate.com/products/web-of-science/}}
We do not claim to have all the relevant papers in this area; however, we believe that the main results are covered to conduct an adequate survey of AEG techniques in the literature.}
{\ema{Present all sources here. Despite Scopus being the key, we also include IEEE, ACM, and WoS to further add things that Scopus might miss.}}

\ifthenelse{\boolean{deliverable}}
{}
{\ema{This should be updated to June/July 2024 at least...}}

The search retrieved an initial list of 1,206 papers up to June 2023.
After applying some screening filters (as described in Table~\ref{tbl:screening-criteria}), 608 papers were retained for the next steps.
Such filters aim to remove poorly relevant candidates and provide a reasonable inspection workload.

\ifthenelse{\boolean{deliverable}}
{}
{\ema{After we finalize all the numbers in the end, we need a figure}}

\subsection{Inclusion/Exclusion Criteria}
\label{sec:relevance-criteria}

\ifthenelse{\boolean{deliverable}}
{}
{\ema{How many researchers in the end?}}

\begin{table}[t]
\centering
	\caption{Screening criteria for filtering papers obtained after running the search query.}
	\label{tbl:screening-criteria}
 \tiny
        \resizebox{0.75\linewidth}{!}{%
    \begin{tabular}{|l|p{5cm}|}
    \hline
    \textbf{ID} & \textbf{Filter}                                                                                                      \\ \hline
    F1          & The paper is published between 2005 -- 2023                                                                           \\
    F2          & The paper is fully written in English                                                                                \\
    F3          & The paper is peer-reviewed (e.g. technical reports are excluded)                                                     \\
    F4          & The paper length is $\geq$ 6 pages                                                                                   \\
    F5          & The paper is not a duplicate or extension of an article already selected                                             \\
    F6          & The paper is a concrete research publication (e.g. chapters of books or collections of academic papers are excluded) \\ \hline
    \end{tabular}
    }
\end{table}

We performed the paper selection with four researchers, so we believe a clear list of inclusion/exclusion criteria will aid the researchers significantly with this assessment task.
To distill the right set of criteria, we ran a ``calibration'' phase on a sample of 101 studies selected from the 608 screened papers.
Specifically, we selected the most cited papers having at least $\geq$ 28 citations according to the number provided by \textsc{Scopus}.

\ifthenelse{\boolean{deliverable}}
{}
{\ema{Why 28 citations?}}

\begin{table}
\centering
	\caption{Exclusion criteria for selecting papers during the relevance assessment process. The initial set of criteria before starting the ``calibration'' are marked with \ding{51}.}
	\label{tbl:relevance-criteria-final}
        \resizebox{0.98\linewidth}{!}{%
	\begin{tabular}{|l|c|l|p{8cm}|}
        \hline
        \textbf{ID} & \textbf{Initial}          & \textbf{Name}                           & \textbf{Explanation}                                                                                      \\ \hline
        E1          & \ding{51} & Vulnerability detection only            & Vulnerability detection technique that does not generate exploits \\
        E2          & \ding{51} & Manual approach                         & Approach that is purely manual to perform                                                                 \\
        E3          &                            & Targeting known vulnerabilities         & Tool re-using available payloads or scripts to find known CVEs, such as Metasploit \cite{kennedy2011metasploit}                     \\
        E4          & \ding{51} & Targeting client side                   & Technique generating exploits at client side, such as SIEGE \cite{iannone2021toward} \\
        E5          & \ding{51} & Targeting hardware-level solutions      & Approach targeting hardware-level solutions such as Automotive, SmartGrid, CSP, IoT, etc.                   \\
        E6          &                            & Malware or network related              & Technique targeting malware, or network-related vulnerabilities                                          \\
        E7          &                            & Survey or evaluation                    & Study conducting a survey, or evaluation, but does not introduce any novel technique                   \\
        E8          &                            & Not applied to security vulnerabilities & Study not clearly applicable to security vulnerabilities \\
        E9          &                            & Testing for security-specific software  & Technique targeting security-specific software, e.g. testing for firewall software                       \\
        E10         & \ding{51} & Collection tool                         & Tool-chain or framework aggregating multiple other already existing tools                                      \\
        E11         &                            & Out of scope                            & Technique relative to other fields other than computer science, e.g. chemical, biology, etc.                          \\
        E12         & \ding{51} & Others                                  & Other reasons                                                                                             \\ \hline
        \end{tabular}
    }
\end{table}
To this end, we created a starting set of six exclusion criteria (as described in Table~\ref{tbl:screening-criteria}).
Then we asked each researcher to read the paper summary (e.g., Title, Abstract, Keywords) and independently make a decision for the paper: \textit{Included}, \textit{Excluded}, or \textit{Not Sure}. During the selection process, the researchers clarified which criteria the paper failed to meet. If the criteria did not exist, it was written down explicitly.
Ultimately, we merged the results from all the researchers. If there were any conflicts, we applied the voting system to decide the inclusion and exclusion of the paper. If there was no winner, we discussed it until we reached a consensus. There were not any papers marked with \textit{Not Sure} by all four researchers; therefore, they were either included or excluded at the end of our discussion.
As a result, we included 18 papers after the calibration phase. As an outcome of this process, we defined six more exclusion criteria for filtering the relevance of the papers. In total, we had twelve exclusion criteria (shown in Table~\ref{tbl:relevance-criteria-final}) used to guide the selection of the rest of the papers in our survey.

We then divide the rest 507 papers (\textit{= 608 - 101}) into smaller batches and assign them to the four researchers. Since we have a clear list of exclusion criteria this time, each paper was reviewed by only one researcher, and there was no conflict resolution afterward. Ultimately, we included 48 more papers, admitting 66 for the review phase.

\subsection{Paper Review and Categorization}
\label{sec:paper-organization}

\begin{table}[t]
\centering
\caption{Papers that went under review to answer our research questions. ``Study'' is an incremental ID we assigned for each paper. PL indicates the targeted Programming Language, and AL the Automation Level of the technique, which can either be from FA (Fully Automated), SA (Semi-Automated) or Interactive (Inter.).}
\label{tbl:rqs-g1}
\footnotesize
\resizebox{0.98\linewidth}{!}{%
\begin{tabular}{|lp{0.5cm}p{0.5cm}lllllll|}
\toprule
\textbf{Study}  &     \textbf{Pub.} &     \textbf{Year} &     \textbf{Input} &     \textbf{Output} &     \textbf{Vulnerability Types} &      \textbf{PL} &     \textbf{Targeted Asset} & \textbf{Oracle Assess.} &     \textbf{AL} \\
\midrule
S01 & \cite{Huang2014270}  &  2014  & Binary &  Exploit  &   Memory   &  C/C++   &  General programs  &  Implicit assertions  &   FA   \\
S02 & \cite{Do2015401}  &  2015  & Source code &  Test cases  &   Information flow   &  Java   &  General Java method  &  Automated assertions  &   FA   \\
S03 & \cite{Garmany2018300}  &  2018  & Binary &  Exploit  &   Crash   &  C/C++   &  Web browsers  &  Implicit assertions  &   Inter.   \\
S04 & \cite{Liu2018705}  &  2018  & Binary &  Exploit  &   Memory   &  C/C++   &  General programs  &  Implicit assertions  &   FA   \\
S05 & \cite{Huang2019}  &  2019  & Binary &  Exploit  &   Memory (heap)  &  C/C++   &  General programs  &  Implicit assertions  &   FA   \\
S06 & \cite{Wei2019120152}  &  2019  & Source code &  Exploit  &   Memory (ROP)  &  C/C++   &  General programs  &  Implicit assertions  &   FA   \\
S07 & \cite{Brizendine202277}  &  2022  & Binary &  Exploit  &   Memory (JOP)  &  C/C++   &  General programs  &  Implicit assertions  &   FA   \\
S08 & \cite{Xu2022}  &  2022  & Binary &  Exploit  &   Memory (bof)  &  C/C++   &  General programs  &  Implicit assertions  &   FA   \\
S09 & \cite{Wang2023}  &  2023  & Source code &  Input/Payload  &   Memory  &  C/C++   &  General programs  &  Implicit assertions  &   Inter.   \\
S10 & \cite{Liu202271}  &  2022  & Binary &  Exploit  &   Memory  &  C/C++   &  General programs  &  Implicit assertions  &   FA   \\
S11 & \cite{Pewny2019111}  &  2018  & Binary &  Exploit  &   Memory  &  C/C++   &  General programs  &  Implicit assertions  &   FA   \\

\midrule

S12 & \cite{Liu2018}  &  2018  & Binary &  Input/Payload  &   Crash  &  C/C++   &  General programs  &  Implicit assertions  &   FA   \\
S13 & \cite{Bohme2019489}  &  2019  & Binary &  Command  &   Crash  &  C/C++   &  General programs  &  Implicit assertions  &   FA   \\
S14 & \cite{Wuestholz20201398}  &  2020  & Source code &  Input/Payload  &   Crash  &  EVM Bytecode   &  Smart contracts  &  Implicit assertions  &   FA   \\
S15 & \cite{Alshmrany202185}  &  2021  & Source code &  Input/Payload  &   Multiple  &  C/C++   &  General programs  &  Implicit assertions  &   FA   \\
S16 & \cite{Gong2022374}  &  2022  & Binary &  Input/Payload  &   Crash  &  Binary-based   &  General programs  &  Implicit assertions  &   FA   \\
S17 & \cite{Yu2022}  &  2022  & Binary &  Input/Payload  &   Crash  &  C/C++  &  General programs  &  Implicit assertions  &   FA   \\
S18 & \cite{wu2011fuzzing}  &  2011  & Binary &  Input/Payload  &   Multiple  &  Binary-based   &  Windows apps  &  Implicit assertions  &   FA   \\
S19 & \cite{Kargen2015782}  &  2015  & Binary &  Input/Payload  &   Multiple  &  Binary-based  &  General programs  &  Implicit assertions  &   Inter.   \\
S20 & \cite{Atlidakis2020387}  &  2020  & Running Web API &  Input/Payload  &   RestAPI security violations  &  Web-based   &  Web applications  &  Implicit assertions  &   FA   \\
S21 & \cite{mahmood2012whitebox}  &  2012  & Binary &  Input/Payload  &   Crash  &  Java  &  Android apps  &  Implicit assertions  &   FA   \\
S22 & \cite{KallingalJoshy2021540}  &  2021  & Source code &  Test case  &   Memory  &  C/C++  &  General programs  &  Automated assertions  &   FA   \\
S23 & \cite{Ren2018391}  &  2018  & Binary &  Test case  &   File format &  C/C++  &  General programs  &  Automated assertions  &   FA   \\
S24 & \cite{shoshitaishvili2018mechanical}  &  2018  & Source code &  Input/Payload  &   Memory &  C/C++  &  General programs  &  Implicit assertions  &   FA   \\
S25 & \cite{Zhang201946}  &  2019  & Binary &   Input/Payload  &   Mutiple &  C/C++  &  General programs  &  Implicit assertions  &   FA   \\

\midrule

S26 & \cite{Simos2019122}  &  2019  & Running Web UI &   Test case  &   XSS &  PHP  &  Web apps  &  Automated assertions  &   FA   \\
S27 & \cite{Mohammadi201678}  &  2016  & Running Web UI &   Test case  &   XSS &  Java  &  Web apps  &  Automated assertions  &   FA   \\
S28 & \cite{Bozic202020}  &  2020  & Running Web UI &   Input/Payload  &   XSS, SQL injection &  Web-based  &  Web apps  &  Implicit assertions  &   FA   \\
S29 & \cite{Bozic2020115}  &  2020  & Running Web API &   Input/Payload  &   XSS &  Web-based  &  Web apps  &  Implicit assertions  &   FA   \\
S30 & \cite{zhang2010d}  &  2010  & Running Web API &   Input/Payload  &   XSS, SQL injection &  Web-based  &  Web apps  &  User-supplied rules  &   FA   \\
S31 & \cite{Chaleshtari20233430}  &  2020  & Running Web UI &   Test case  &   Multiple &  Web-based  &  Web apps  &  Automated assertions  &   FA   \\
S32 & \cite{Chen20201580}  &  2020  & Source code &   Input/Payload  &   Multiple &  C/C++  &  General programs  & Implicit assertions  &   FA   \\
S33 & \cite{Tang2017492}  &  2017  & Binary &   Test case  &   Android ICC vulnerabilities &  Java  &  Android apps  & Automated assertions  &   FA   \\
S34 & \cite{Liu2020286}  &  2020  & Running Web UI &   Input/Payload  &   SQL Injection &  Java  &  Web apps  & Implicit assertions  &   FA   \\
S35 & \cite{Zech201488}  &  2014  & Source code &   Test case  &   Multiple &  Multiple  &  General programs  & Implicit assertions  &   SA   \\
S36 & \cite{Pretschner2008338}  &  2008  & Program model &   Test case  &   RBAC vulnerabilities &  Multiple  &  General programs  & Manually  &   SA   \\
S37 & \cite{Lebeau2013445}  &  2013  & Program model &   Test case  &   XSS,  SQL Injection &  Java  &  General programs  & Manually &   SA   \\
S38 & \cite{Siavashi2018301}  &  2018  & Source code &   Test case  &   Multiple &  Python  &  General programs  & Automated assertions &   FA   \\
S39 & \cite{Nurmukhametov202137}  &  2021  & Binary &   Exploit  &   Memory (ROP) &  Binary-based  &  General programs  & Implicit assertions &   FA   \\
S40 & \cite{shahriar2009automatic}  &  2010  & Source code &   Test case  &   Memory (bof) &  C/C++  &  General programs  & Automated assertions &   FA   \\
S41 & \cite{Appelt2014259}  &  2014  & Running Web API &   Input/Payload  &   SQL Injection &  PHP  &  Web apps  &  ML-based detector  &   FA   \\
S42 & \cite{Jan201612}  &  2016  & Running Web API &   Input/Payload  &   XML Injection &  Web-based  &  Web apps  &  User-supplied rules  &   FA   \\
S43 & \cite{Liu2016123}  &  2016  & Running Web UI &   Test case  &   SQL Injection &  Web-based  &  Web apps  &  Automated assertions  &   FA   \\
S44 & \cite{Cotroneo2013125}  &  2013  & Binary &   Test case  &  Robustness vulnerabilities &  Binary-based  &  Operating System/Kernel  &  Automated assertions  &   FA   \\
S45 & \cite{DelGrosso20083125}  &  2015  & Source code &  Input/Payload  &   Memory (bof)  &  C/C++  &  General programs  &  Implicit assertions  &   FA   \\
S46 & \cite{Aziz2016183}  &  2016  & Running Web UI &   Test case  &   SQL Injection &  PHP  &  Web apps  &  Automated assertions  &   FA   \\
S47 & \cite{babic2011statically}  &  2011  & Binary &  Input/Payload  &   Memory   &  Binary-based   &  General programs  &  Implicit assertions  &   FA   \\
S48 & \cite{marback2013threat, xu2011tool, xu2012automated, Xu2015247}  &  2011--2013, 2015  & Running Web UI &   Test case  &   Multiple &  Web based  &  Web apps  &  Manually &   SA   \\
S49 & \cite{avancini2012security}  &  2012  & Source code &  Security oracle  &   Injection vulnerabilities   &  PHP  &  Web apps  &  Manually  &   SA   \\
S50 & \cite{Khamaiseh2017534}  &  2017  & Program model &  Test case  &   Multiple &  Multiple  &  General programs  &  Automated assertions  &   SA   \\

\midrule

S51 & \cite{Mohammadi201678}  &  2016  & Source code &   Test case  &   XSS &  Java  &  Web apps  & Automated assertions &   FA   \\
S52 & \cite{aydin2014automated}  &  2014  & Running Web UI &   Input/Payload  &   XSS, SQL Injection &  PHP  &  Web apps  & Implicit assertions &   FA   \\
S53 & \cite{godefroid2008grammar}  &  2008  & Source code &   Input/Payload  &   Crash &  C/C++  & General programs  & Implicit assertions &   FA   \\
S54 & \cite{Huang2013208}  &  2013  & Running Web UI &   Input/Payload  &   XSS, SQL Injection &  Web-based  &  Web apps  & Implicit assertions &   FA   \\
S55 & \cite{Do2015401}  &  2015  & Source code &   Input/Payload  &   Memory &  C/C++  &  General programs  & Implicit assertions &   FA   \\
S56 & \cite{feist2016finding}  &  2016  & Binary &   Input/Payload  &   Memory &  C/C++  &  General programs  & Implicit assertions &   FA   \\
S57 & \cite{wu2018fuze}  &  2018  & Source code &   Input/Payload  &   Memory &  C/C++  &  Operating systems/Kernel  & Implicit assertions &   FA   \\
S58 & \cite{Dixit2021}  &  2021  & Binary &   Input/Payload  &   Memory (bof) &  C/C++  &  General programs  & Implicit assertions &   FA   \\
S59 & \cite{Garcia2017661}  &  2017  & Binary &   Exploit  &   Android ICC vulnerabilities &  Java  &  Android apps  & Implicit assertions  &   FA   \\
S60 & \cite{kieyzun2009automatic}  &  2009  & Source code &  Input/Payload  &   SQL Injection  &  PHP  &  Web apps  &  Implicit assertions  &   SA   \\
S61 & \cite{dao2011security}  &  2011  & Source code &  Test case  &   SQL Injection  &  PHP  &  Web apps  &  Automated assertions  &   SA   \\
S62 & \cite{Wang2013216}  &  2013  & Binary &   Exploit  &  Memory &  Binary-based  &  Windows apps  & Implicit assertions  &   FA   \\
S63 & \cite{Hough2020284}  &  2020  & Source code &   Exploit  &  Injection vulnerabilities &  Java  &  Web apps  & Implicit assertions  &   FA   \\
\bottomrule
\end{tabular}
}
\end{table}

After we had selected papers for reviewing, we grouped papers that were (i) part of the same study or (ii) presented the development or evolution of the same approach.
%, and later, one was extended to the exploitation techniques proposed in the prior one
We treated each group as a single body of work in our reviewing process.
The merge was guided by the paper's author list and their actual content.
\ifthenelse{\boolean{deliverable}}
{}
{\ema{I link this grouping of things, but we might need to define more objective and less-ambiguous criteria.}}
This step identified four papers that could be grouped together, in which the authors developed a security testing technique based on threat modeling~\cite{xu2011tool,xu2012automated,marback2013threat,Xu2015247}.

Based on our aimed scope of this survey study and the selection results of the paper list, we defined four main groups of techniques of the relevant areas of automated exploit generation and security testing shown in Figure~\ref{fig:taxonomy}, that is, AEG, Security Testing, Fuzzing, and Others.
This helps us construct the taxonomy used to categorize the resulting papers.
We acknowledge that there can be other ways to categorize the studies in our survey.
\ifthenelse{\boolean{deliverable}}
{}
{\ema{As I said somewhere before, I feel these categories should be recreated. Not it's okay, but later. Then, we will also check if the taxonomy aligns with the one Cuong mentioned~\cite{pendleton2016survey}} \cuong{Add more points to convince that our taxonomy is good enough. See the same point in the two other CSUR papers}.}
However, we believe that our taxonomy aligns with security standards in practice~\cite{pendleton2016survey}.
Then, we read the whole text of all the papers to assign them to their right category.
\ifthenelse{\boolean{deliverable}}
{}
{\ema{We must explain here what we mean with each category (a table is fine)}}

\ifthenelse{\boolean{deliverable}}
{}
{\ema{We need to define the ``data collection form'', i.e., the aspects that we looker while reading the papers (i.e., the columns of the Excel sheet)}, so that we can motivate the columns in the big table}

The list of papers that went under review is reported in Table~\ref{tbl:rqs-g1}.
The information reported is based on the data extracted from papers to answer the defined research questions. 
%
The first three columns present the study IDs, associated publications (one study may span multiple publications), and the publishing years.
The next two columns show the techniques' required inputs and produced outputs. The next three columns describe information about the techniques' targets in terms of vulnerability types, programming languages, and targeted applications. The last two columns provide information on how the oracles generated from the tools were assessed and the automation level of the tools, which can be Fully Automated (FA), Semi-Automated (SA), or Interactive (Inter.).

\begin{figure}
\begin{tikzpicture}[
	level 1/.style={sibling distance=40mm},
	edge from parent/.style={->,draw},
	>=latex]

	% root of the the initial tree, level 1
	\node[root] {Our survey}
	% The first level, as children of the initial tree
	child {node[level 2] (c1) {AEG}}
	child {node[level 2] (c2) {Security Testing}}
	child {node[level 2] (c3) {Fuzzing}}
	child {node[level 2] (c4) {Others}};
	
	% The second level, relatively positioned nodes
	\begin{scope}[every node/.style={level 3}]
		\node [below of = c1, xshift=15pt] (c11) {Control-flow Hijacking};
		\node [below of = c11] (c12) {Data-oriented};
		
		\node [below of = c2, xshift=15pt] (c21) {Threat Model-based};
		\node [below of = c21] (c22) {Mutation-based};
		\node [below of = c22] (c23) {Grammar-based};
		\node [below of = c23] (c24) {Search-based};
		\node [below of = c24] (c25) {Heuristic-based};
		
		\node [below of = c3, xshift=15pt] (c31) {Whitebox};
		\node [below of = c31] (c32) {Coverage-based greybox};
		\node [below of = c32] (c33) {Stateful};
		\node [below of = c33] (c34) {Directed};
		\node [below of = c34] (c35) {Knowledge-based};
		
		\node [below of = c4, xshift=15pt] (c41) {Dynamic Symbolic Execution};
		\node [below of = c41] (c42) {Dynamic Taint Analysis};
		\node [below of = c42] (c43) {String Analysis};
	\end{scope}
	
	% lines from each level 1 node to every one of its ''children''
	\foreach \value in {1,2}
	\draw[->] (c1.195) |- (c1\value.west);
	
	\foreach \value in {1,...,5}
	\draw[->] (c2.195) |- (c2\value.west);
	
	\foreach \value in {1,...,5}
	\draw[->] (c3.195) |- (c3\value.west);
	
	\foreach \value in {1,...,3}
	\draw[->] (c4.195) |- (c4\value.west);
\end{tikzpicture}
\caption{Taxonomy of studies on Automated Exploit Generation.}
\label{fig:taxonomy}
\end{figure}

