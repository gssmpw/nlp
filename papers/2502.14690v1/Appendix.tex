\appendix

\section{Types of Distributional Constraints Studied in the Literature}\label{app:constraints}

\citet{aziz22} talk about the following types of distributional constraints:

\begin{itemize}
    \item \textbf{Lower quotas}: each college may have a lower quota, i.e., minimal number of students to admit. Our model has only upper quotas. Thus, RRC do not imply any model with lower quotas, and vise-versa.

    \item \textbf{Regional constraints}: our model does not have hard upper quotas for number of students admitted within a region, only for amount of corresponding resources distributed within a region. Our constraints become regional with disjoint regions if there is only one non-empty resource, and any student needs a unit of this resource in order to be admitted anywhere. Thus, our model implies the model with disjoint regions and regional quotas \citep{kamada15}. Therefore, all corresponding negative results carry over to our model (e.g., non-existence of a stable matching, and NP-hardness of checking whether a fair and non-wasteful matching exists \citep{aziz24}).

    \item \textbf{Diversity constraints}: models with type-specific quotas at each college, where any student possesses one or more types. There is no affirmative action agenda in our model. Thus, our model does not imply any model with diversity constraints, and vise-versa.

    \item \textbf{Multi-dimensional capacity constraints}: models with one or more types of resources, such that, each college has some amount of each resource to distribute among admitted students, while each student demands some fixed amount of each resource in order to study anywhere \citep{delacretaz23,nguyen21}. In our model, region possesses resources, and a student's demand for resources depends on the college of interest. Thus, our model does not imply any model with multi-dimensional capacity constraints, and vise-versa.

    \item \textbf{Matroidal constraints}: given a finite set $\Contracts$, let $\mathcal{F}\in 2^{\Contracts}$. A pair $(\Contracts,\mathcal{F})$ is a \textbf{matroid} if it satisfies the following three conditions: (i) $\emptyset\in \mathcal{F}$; (ii) if $X'\in\mathcal{F}$ and $X''\subseteq X'$, then $X''\in\mathcal{F}$; (iii) if $X',X''\in\mathcal{F}_r$ and $|X''|<|X'|$, then there exists some $x\in X'\backslash X''$, such that, $X''\cup\{x\}\in \mathcal{F}$. In the context of matchings, $\Contracts$ is a set of pairs of a student and a resource, and $\mathcal{F}_c$ is a set of sets of such pairs that a given college $c$ can admit. Thus, set of admissible sets of pairs is a matroid for each college. A matching is feasible if, first, any student has at most one match, and, second, any college $c$ is matched to some set from $\mathcal{F}_c$. RRC constraints are not matroidal (see \Cref{Ex1}).

    \begin{example}\label{Ex1}
        Consider a market with two students, two colleges, and one non-empty resource. Both colleges are in one region. Each college has exactly one seat to distribute. The region has exactly one unit of resource to distribute. Each student needs a unit of resource to be admitted.

         Consider two feasible matchings: $\mu_1 =\{(s_1,c_1,r)\}$ and $\mu_2 =\{(s_2,c_2,r)\}$. It implies that $\{\emptyset,\{(s_1,r)\}\}\subseteq\mathcal{F}_{c_1}$ and $\{\emptyset,\{(s_2,r)\}\}\subseteq\mathcal{F}_{c_2}$. Thus $\mu = \{(s_1,c_1,r),(s_2,c_2,r)\}$ should be also feasible, which is not true. Thus, RCC constraints are not matroidal.
    \end{example}

    \item \textbf{Hereditary constraints}: models where the feasibility of a matching is monotone in the number of students matched to the colleges. The heredity property requires that an empty matching to be feasible, and if a matching is feasible, then any matching in which the number of students matched to each college weakly decreases, is feasible as well. RCC constraints do not satisfy heredity (see \Cref{Ex2}).

    \begin{example}\label{Ex2}
        Consider a market with five students, one college, and one non-empty resource. The college has exactly three seats to distribute. The region containing the only college has exactly one unit of resource to distribute. Students $s_1$, $s_2$, and $s_3$ do not need a unit of a non-empty resource $r$ for admission, while students $s_4$ and $s_5$ can be admitted only with a unit of $r$.

         Consider $X'=\{(s_1,c_1,r_0),(s_2,c_1,r_0),(s_3,c_1,r_0)\}$ and $X''=\{(s_4,c_1,r),(s_5,c_1,r)\}$. First, $|X''|<|X'|$, second, $X'$ is feasible, while $X''$ is not. Thus, RCC constraints do not satisfy heredity.
    \end{example}

%    \begin{remark}
%    Hereditary and (i)\&(ii) in matroidal constraints, where $\Contracts$ is a set of all possible contracts, and $\mathcal{F}$ is a set of all feasible matchings, are fundamentally different and their relationship must be carefully established. Matroidal constraints work on the matching space while hereditary constraints work on $\mathbb{R}^{|{\Colleges}|}$ with vectors representing the number of students allocated to each college. For allocation problems such as classical college admissions, an equivalence between these two objects exists as matchings are transversal matroids and can be equivalently seen as polymatroids. However, this is not necessarily the case for more complex allocation problems.
%    \end{remark}

    \item \textbf{General upper-bound (intra-institution heredity constraints)} \citep{kamada24}: each college $c$ has a set of feasible sets of contracts (containing $c$) $\mathcal{F}_c$, such that, if $X'\in \mathcal{F}_c$ and $X''\subseteq X'$, then $X''\in \mathcal{F}_c$. A set of contracts $X\subseteq \Contracts$ is feasible if, and only if, each student is matched to at most one college, and each college is matched to one of its feasible sets of contracts. RRC constraints are not general upper-bound (see \Cref{Ex1}).
    
\end{itemize}

There is another type of distributional constraints studied in the literature \cite{imamura24a}:

\begin{itemize}
    \item \textbf{G-matroidal constraints (M$^\natural$-convex family)}: given a non-empty finite set $\Contracts$, let $\mathcal{F}\subseteq 2^{\Contracts}$. A pair $(\Contracts,\mathcal{F})$ is a \textbf{g-matroid} if for any $X',X''\in \mathcal{F}$ and $x'\in X'\backslash X''$, it holds that either (i) $X'\backslash\{x'\}\in \mathcal{F}$ and $X''\cup\{x'\}\in \mathcal{F}$, or (ii) there exists $x''\in X''\backslash X'$, such that, $(X'\backslash \{x'\})\cup\{x''\}\in \mathcal{F}$ and $(X''\cup\{x'\})\backslash\{x''\}\in \mathcal{F}$. In the context of matchings, $\Contracts$ is a set of pairs of a student and a resource, and $\mathcal{F}_c$ is a set of sets of such pairs that a given college $c$ can admit. Thus, set of admissible sets of pairs is a g-matroid for each college. A matching is feasible if, first, any student has at most one match, and, second, any college $c$ is matched to some set from $\mathcal{F}_c$. RRC constraints are not g-matroidal (see \Cref{Ex1}).
\end{itemize}

 We propose a new type of constraints:

\begin{itemize}
    \item \textbf{Set-hereditary constraints}: The set-heredity property requires that if a matching is feasible, then any subset of this matching is feasible as well, given that an empty matching is feasible, i.e., (i) and (ii) from matroidal constraints, where $\Contracts$ is a set of all possible contracts, and $\mathcal{F}$ is a set of all feasible matchings. RCC constraints obviously satisfy set-heredity. Note, that any problem under heredity satisfies set-heredity.

    \begin{remark}
    Hereditary and set-hereditary constraints are fundamentally different and their relationship must be carefully established. Set-hereditary constraints work on the matching space while hereditary constraints work on $\mathbb{R}^{|{\Colleges}|}$ with vectors representing the number of students allocated to each college. For allocation problems such as classical college admissions, an equivalence between these two objects exists as matchings are transversal matroids and can be equivalently seen as polymatroids. However, this is not the case anymore when considering resources as in our model.
    \end{remark}
\end{itemize}


\section{Numerical Results: Balanced Market with intermediate number of resources}\label{app:balanced_market_with_5_resources}

To complement the results presented in \Cref{sec:empirics}, \Cref{tab:blocking_contracts_five_resources} contains the average number of blocking contracts of the five mechanisms for balanced markets with $5$ resources. Similarly to the already observe results, DMC remains the most stable mechanism on average over the simulated markets. Moreover, DMC and CSD become stable whenever colleges present preferences alignment (as for 1 and 2 resources).

\begin{table}[ht]
\centering
\small{
\caption{Average numbers of blocking contracts produced by mechanisms presented in \Cref{section:mechanisms} over $100$ balanced markets with $100$ students, $10$ colleges, and $5$ resources. Mechanisms presenting zero blocking contracts were omitted from the corresponding kind of market.}
\begin{tabular}{cccccc}
\toprule
\multicolumn{6}{c}{\textbf{Horizontal Markets - Five resources ($\Resources_0 = \{r_0,r_1,...,r_4\}$)}}\\
\toprule
& Resource   & Waste   & Direct-Envy   & Indirect-Envy       & Total    \\
\hline
DRC & 0.39$\pm$0.615  & 0.7$\pm$1.253      & 0.0$\pm$0.0       & 8.15$\pm$5.656 & 9.24$\pm$5.774     \\
DMC & 0.07$\pm$0.255  & 0.74$\pm$1.11      & 0.0$\pm$0.0       & 2.42$\pm$2.187 & 3.23$\pm$2.336     \\
DUC & 0.0$\pm$0.0     & 171.85$\pm$137.178 & 0.0$\pm$0.0       & 0.0$\pm$0.0    & 171.85$\pm$137.178 \\
RSD & 0.0$\pm$0.0     & 0.0$\pm$0.0        & 48.68$\pm$20.564  & 9.03$\pm$5.39  & 57.71$\pm$22.899   \\
CSD & 0.0$\pm$0.0     & 0.0$\pm$0.0        & 18.68$\pm$11.723  & 3.44$\pm$2.434 & 22.12$\pm$13.028   \\
\toprule
\multicolumn{6}{c}{\textbf{Student-vertical Markets - Five resources ($\Resources_0 = \{r_0,r_1,...,r_4\}$)}}\\
\toprule
& Resource   & Waste   & Direct-Envy   & Indirect-Envy    & Total \\
\hline
DRC & 2.14$\pm$2.429  & 1.39$\pm$2.044     & 0.0$\pm$0.0       & 14.71$\pm$5.691  & 18.24$\pm$7.514    \\
DMC & 0.0$\pm$0.0     & 1.55$\pm$2.002     & 0.0$\pm$0.0       & 6.07$\pm$3.296   & 7.62$\pm$3.646     \\
DUC & 0.0$\pm$0.0     & 1153.75$\pm$265.45 & 0.0$\pm$0.0       & 0.0$\pm$0.0      & 1153.75$\pm$265.45 \\
RSD & 0.0$\pm$0.0     & 0.0$\pm$0.0        & 480.72$\pm$58.317 & 154.96$\pm$32.74 & 635.68$\pm$73.511  \\
CSD & 0.0$\pm$0.0     & 0.0$\pm$0.0        & 24.51$\pm$9.742   & 8.05$\pm$4.486   & 32.56$\pm$12.353   \\
\toprule
\multicolumn{6}{c}{\textbf{College-vertical Markets - Five resources ($\Resources_0 = \{r_0,r_1,...,r_4\}$)}}\\
\toprule
& Resource   & Waste   & Direct-Envy   & Indirect-Envy    & Total       \\
\hline
DRC & 0.45$\pm$0.669  & 0.18$\pm$0.623     & 0.0$\pm$0.0       & 5.89$\pm$3.76  & 6.52$\pm$4.124     \\
% DMC & 0.0$\pm$0.0     & 0.0$\pm$0.0        & 0.0$\pm$0.0       & 0.0$\pm$0.0    & 0.0$\pm$0.0        \\
DUC & 0.0$\pm$0.0     & 129.77$\pm$118.558 & 0.0$\pm$0.0       & 0.0$\pm$0.0    & 129.77$\pm$118.558 \\
RSD & 0.0$\pm$0.0     & 0.0$\pm$0.0        & 50.06$\pm$20.772  & 9.39$\pm$5.366 & 59.45$\pm$22.297   \\
% CSD & 0.0$\pm$0.0     & 0.0$\pm$0.0        & 0.0$\pm$0.0       & 0.0$\pm$0.0    & 0.0$\pm$0.0        \\
\toprule
\multicolumn{6}{c}{\textbf{Fully-vertical Markets - Five resources ($\Resources_0 = \{r_0,r_1,...,r_4\}$)}}\\
\toprule
& Resource   & Waste   & Direct-Envy   & Indirect-Envy    & Total       \\
\hline
DRC & 4.47$\pm$4.381  & 0.49$\pm$1.229     & 0.0$\pm$0.0       & 31.08$\pm$12.226  & 36.04$\pm$15.152   \\
DUC & 0.0$\pm$0.0     & 969.11$\pm$221.374 & 0.0$\pm$0.0       & 0.0$\pm$0.0       & 969.11$\pm$221.374 \\
RSD & 0.0$\pm$0.0     & 0.0$\pm$0.0        & 470.1$\pm$70.057  & 151.12$\pm$33.585 & 621.22$\pm$79.418  \\
\toprule
\end{tabular}
\label{tab:blocking_contracts_five_resources}
}
\end{table}

\section{Numerical Results: Unbalanced Markets with General Number of Regions}\label{Appendix:Unbalanced markets}

We also study the stability of the mechanisms introduced in \Cref{section:mechanisms} empirically under unbalanced markets presenting the same four kinds of preferences alignments introduced in \Cref{sec:empirics}. For each market, we fix the number of students to $100$, the number of colleges to $10$, and considered two cases for $|{\Resources_0}| \in \{2,5\}$. The region partitions were done randomly (both the number of regions per resource and the quota of each region). In addition, we added unbalance to the markets in two ways:
\begin{itemize}
    \item \textbf{Colleges Up/Down}: The sum of colleges' capacities is higher/lower than the total number of students,
    \item \textbf{Resources Up/Down}: The sum of resources' quotas is higher/lower than the total number of students.
\end{itemize}
For the Up markets we consider numbers equal to twice the number of students, e.g., for a Colleges Up market, the sum of capacities of all colleges sum up 2 times the number of available students on the market. For the Down versions, we consider numbers equal to the half of the available students. By mixing both, we are able to simulate four different kinds of unbalanced markets. We present the most representative results in the following. For the whole list of tables, please check at the code available with this article.

\begin{table}[ht]
\centering
\small{
\caption{Average numbers of blocking contracts produced by mechanisms presented in \Cref{section:mechanisms} over $100$ horizontal unbalanced markets with $100$ students, $10$ colleges, and $2$ resources. }
\begin{tabular}{cccccc}
\toprule
\multicolumn{6}{c}{\textbf{Horizontal Markets - Two resources ($\Resources_0 = \{r_0,r_1\}$)}}\\
\toprule
\multicolumn{6}{c}{\textbf{Colleges Up - Resources Up}} \\
\hline
& Resource   & Waste   & Direct-Envy   & Indirect-Envy       & Total    \\
\hline
DRC & 0.0±0.0     & 5.02±6.07      & 0.0±0.0       & 0.06±0.443 & 5.08±6.177     \\
DMC & 0.0±0.0     & 4.91±6.073     & 0.0±0.0       & 0.07±0.515 & 4.98±6.211     \\
DUC & 0.0±0.0     & 101.27±148.698 & 0.0±0.0       & 0.0±0.0    & 101.27±148.698 \\
RSD & 0.0±0.0     & 0.0±0.0        & 31.98±38.203  & 0.0±0.0    & 31.98±38.203   \\
CSD & 0.0±0.0     & 0.0±0.0        & 5.63±6.971    & 0.0±0.0    & 5.63±6.971     \\
\hline
\multicolumn{6}{c}{\textbf{Colleges Down - Resources Up}} \\
\hline
& Resource   & Waste   & Direct-Envy   & Indirect-Envy       & Total    \\
\hline
DRC & 1.42±1.55   & 0.0±0.0       & 0.0±0.0       & 2.47±2.398 & 3.89±3.693    \\
DMC & 0.0±0.0     & 0.0±0.0       & 0.0±0.0       & 0.23±0.676 & 0.23±0.676    \\
DUC & 0.0±0.0     & 62.61±132.137 & 0.0±0.0       & 0.0±0.0    & 62.61±132.137 \\
RSD & 0.0±0.0     & 0.0±0.0       & 377.21±50.627 & 5.26±8.818 & 382.47±50.476 \\
CSD & 0.0±0.0     & 0.0±0.0       & 76.2±21.532   & 1.17±1.945 & 77.37±22.221  \\
\hline
\multicolumn{6}{c}{\textbf{Colleges Up - Resources Down}} \\
\hline
& Resource   & Waste   & Direct-Envy   & Indirect-Envy       & Total    \\
\hline
DRC & 0.0±0.0     & 9.43±5.894   & 0.0±0.0       & 0.06±0.276 & 9.49±5.908   \\
DMC & 0.0±0.0     & 9.69±5.949   & 0.0±0.0       & 0.07±0.292 & 9.76±5.973   \\
DUC & 0.0±0.0     & 360.9±87.576 & 0.0±0.0       & 0.0±0.0    & 360.9±87.576 \\
RSD & 0.0±0.0     & 0.0±0.0      & 80.05±27.472  & 0.0±0.0    & 80.05±27.472 \\
CSD & 0.0±0.0     & 0.0±0.0      & 10.91±8.972   & 0.0±0.0    & 10.91±8.972  \\
\hline
\multicolumn{6}{c}{\textbf{Colleges Down - Resources Down}} \\
\hline
& Resource   & Waste   & Direct-Envy   & Indirect-Envy       & Total    \\
\hline
DRC & 0.41±0.895  & 0.0±0.0        & 0.0±0.0       & 1.23±1.264 & 1.64±1.863     \\
DMC & 0.0±0.0     & 0.0±0.0        & 0.0±0.0       & 0.69±1.036 & 0.69±1.036     \\
DUC & 0.0±0.0     & 228.61±137.069 & 0.0±0.0       & 0.0±0.0    & 228.61±137.069 \\
RSD & 0.0±0.0     & 0.0±0.0        & 363.83±71.876 & 2.03±5.747 & 365.86±72.514  \\
CSD & 0.0±0.0     & 0.0±0.0        & 59.4±18.344   & 0.4±1.086  & 59.8±18.639    \\
% \toprule
% \multicolumn{6}{c}{\textbf{Student-vertical Markets - Two Resources ($\Resources_0 = \{r_0,r_1\}$)}}\\
% \toprule
% \multicolumn{6}{c}{\textbf{Colleges Up - Resources Up}} \\
% \hline
% & Resources   & Waste         & Direct-Envy   & Indirect-Envy       & Total         \\
% \hline
% DRC & 1.0±1.449   & 1.39±3.066     & 0.0±0.0       & 3.26±3.091 & 5.65±5.062     \\
% DMC & 0.0±0.0     & 1.41±3.076     & 0.0±0.0       & 1.11±2.063 & 2.52±4.286     \\
% DUC & 0.0±0.0     & 106.73±162.503 & 0.0±0.0       & 0.0±0.0    & 106.73±162.503 \\
% RSD & 0.0±0.0     & 0.0±0.0        & 111.46±28.823 & 0.52±1.345 & 111.98±28.688  \\
% CSD & 0.0±0.0     & 0.0±0.0        & 4.57±5.942    & 0.03±0.171 & 4.6±5.958      \\
% \hline
% \toprule
% \multicolumn{6}{c}{\textbf{College-vertical Markets - Two Resources ($\Resources_0 = \{r_0,r_1\}$)}}\\
% \toprule
% \multicolumn{6}{c}{\textbf{Colleges Up - Resources Up}} \\
% \hline
% & Resources   & Waste         & Direct-Envy   & Indirect-Envy       & Total         \\
% \hline
% DRC & 0.0±0.0     & 0.62±1.488    & 0.0±0.0       & 0.03±0.222 & 0.65±1.532    \\
% % DMC & 0.0±0.0     & 0.0±0.0       & 0.0±0.0       & 0.0±0.0    & 0.0±0.0       \\
% DUC & 0.0±0.0     & 113.3±149.996 & 0.0±0.0       & 0.0±0.0    & 113.3±149.996 \\
% RSD & 0.0±0.0     & 0.0±0.0       & 36.12±40.797  & 0.0±0.0    & 36.12±40.797  \\
% % CSD & 0.0±0.0     & 0.0±0.0       & 0.0±0.0       & 0.0±0.0    & 0.0±0.0       \\
% \hline
\toprule
\end{tabular}
\label{tab:blocking_contracts_two_resources_unbalanced_horizontal_market}
}
\end{table}

Unlike the results observed for balanced horizontal markets with two resources, now all mechanisms produce blocking contracts. Waste is clearly affected by colleges capacities, as for colleges down markets the mechanisms manage to occupy most of their seats. The serial dictatorship mechanisms seem to benefit from colleges up markets as they reduce their indirect envy to zero. Indeed, as colleges have larger quotas, the serial dictatorship mechanisms can allocate to the students to their most preferred colleges. DMC remains the most stable mechanism in average, although for Colleges Up market, its advantage decreases considerably with respect to DRC and CSD.

For Student-vertical markets and fully-vertical markets, the overall performance of the mechanisms remains quite similar to the ones in \Cref{tab:blocking_contracts_two_resources_unbalanced_horizontal_market}, so we prefer to omit them. For College-vertical markets, however, DMC seems to become fully stable, as illustrated in \Cref{tab:blocking_contracts_two_resources_unbalanced_college_vertical_market}. Recall whenever a mechanism presents zero blocking contract we omit it from the table.

Experiments with five resources support the finding that DMC, being not stable overall, remains extremely good as a minimizer of the total number of blocking contracts. \Cref{tab:blocking_contracts_DMC_five_resources_unbalanced_markets} presents some of the few cases where DMC produces blocking contracts. Also, \Cref{tab:stability_unbalanced_markets} shows that DMC and DRC produce roughly the same amount of indirect envy if there is no shortage of resources (Resources Up markets), while DMC eliminates much more indirect envy if resources are scarce (Resources Down markets).

However, unlike \citet{ashlagi17} model, it is not clear that unbalanced markets help in achieving more stable matchings under RRC.

\begin{table}[ht]
\centering
\small{
\caption{Average numbers of blocking contracts produced by mechanisms presented in \Cref{section:mechanisms} over $100$ College-vertical unbalanced markets with $100$ students, $10$ colleges, and $2$ resources. Mechanisms presenting zero blocking contracts were omitted from the corresponding kind of market.}
\begin{tabular}{cccccc}
\toprule
\multicolumn{6}{c}{\textbf{College-vertical Markets - Two resources ($\Resources_0 = \{r_0,r_1\}$)}}\\
\toprule
\multicolumn{6}{c}{\textbf{Colleges Up - Resources Up}} \\
\hline
& Resource   & Waste   & Direct-Envy   & Indirect-Envy       & Total    \\
\hline
DRC & 0.0±0.0     & 0.62±1.488    & 0.0±0.0       & 0.03±0.222 & 0.65±1.532    \\
% DMC & 0.0±0.0     & 0.0±0.0       & 0.0±0.0       & 0.0±0.0    & 0.0±0.0       \\
DUC & 0.0±0.0     & 113.3±149.996 & 0.0±0.0       & 0.0±0.0    & 113.3±149.996 \\
RSD & 0.0±0.0     & 0.0±0.0       & 36.12±40.797  & 0.0±0.0    & 36.12±40.797  \\
% CSD & 0.0±0.0     & 0.0±0.0       & 0.0±0.0       & 0.0±0.0    & 0.0±0.0       \\
\hline
\multicolumn{6}{c}{\textbf{Colleges Down - Resources Up}} \\
\hline
& Resource   & Waste   & Direct-Envy   & Indirect-Envy       & Total    \\
\hline
DRC & 1.92±2.067  & 0.0±0.0       & 0.0±0.0       & 3.89±3.561 & 5.81±5.259    \\
% DMC & 0.0±0.0     & 0.0±0.0       & 0.0±0.0       & 0.0±0.0    & 0.0±0.0       \\
DUC & 0.0±0.0     & 58.48±123.348 & 0.0±0.0       & 0.0±0.0    & 58.48±123.348 \\
RSD & 0.0±0.0     & 0.0±0.0       & 382.29±53.147 & 6.27±7.992 & 388.56±54.15  \\
% CSD & 0.0±0.0     & 0.0±0.0       & 0.0±0.0       & 0.0±0.0    & 0.0±0.0       \\
\hline
\multicolumn{6}{c}{\textbf{Colleges Up - Resources Down}} \\
\hline
& Resource   & Waste   & Direct-Envy   & Indirect-Envy       & Total    \\
\hline
DRC & 0.01±0.099  & 0.76±0.96      & 0.0±0.0       & 0.07±0.604 & 0.84±1.146     \\
% DMC & 0.0±0.0     & 0.0±0.0        & 0.0±0.0       & 0.0±0.0    & 0.0±0.0        \\
DUC & 0.0±0.0     & 325.33±100.094 & 0.0±0.0       & 0.0±0.0    & 325.33±100.094 \\
RSD & 0.0±0.0     & 0.0±0.0        & 80.41±28.19   & 0.01±0.099 & 80.42±28.195   \\
% CSD & 0.0±0.0     & 0.0±0.0        & 0.0±0.0       & 0.0±0.0    & 0.0±0.0        \\
\hline
\multicolumn{6}{c}{\textbf{Colleges Down - Resources Down}} \\
\hline
& Resource   & Waste   & Direct-Envy   & Indirect-Envy       & Total    \\
\hline
DRC & 0.65±1.337  & 0.0±0.0        & 0.0±0.0       & 1.83±2.328 & 2.48±3.416     \\
% DMC & 0.0±0.0     & 0.0±0.0        & 0.0±0.0       & 0.0±0.0    & 0.0±0.0        \\
DUC & 0.0±0.0     & 226.57±146.552 & 0.0±0.0       & 0.0±0.0    & 226.57±146.552 \\
RSD & 0.0±0.0     & 0.0±0.0        & 368.74±71.727 & 1.19±3.682 & 369.93±71.828  \\
% CSD & 0.0±0.0     & 0.0±0.0        & 0.0±0.0       & 0.0±0.0    & 0.0±0.0        \\
\toprule
\end{tabular}
\label{tab:blocking_contracts_two_resources_unbalanced_college_vertical_market}
}
\end{table}



\begin{table}[ht]
\centering
\small{
\caption{Average numbers of blocking contracts produced by the DMC mechanism over $100$ unbalanced markets with $100$ students, $10$ colleges, and $5$ resources.}
\begin{tabular}{cccccc}
\toprule
\multicolumn{6}{c}{\textbf{Decreasing Maximal Cutoffs}}\\
\toprule
\multicolumn{6}{c}{\textbf{Horizontal Markets - Five resources ($\Resources_0 = \{r_0,r_1,...,r_4\}$)}}\\
\toprule
Market & Resource   & Waste   & Direct-Envy   & Indirect-Envy       & Total    \\
\hline
Colleges Down - Resources Up & 0.0±0.0     & 0.0±0.0        & 0.0±0.0       & 0.36±0.855     & 0.36±0.855     \\
\toprule
\multicolumn{6}{c}{\textbf{College-vertical Markets - Five resources ($\Resources_0 = \{r_0,r_1,...,r_4\}$)}}\\
\toprule
Market & Resource   & Waste   & Direct-Envy   & Indirect-Envy       & Total    \\
\hline
Colleges Up - Resources Up & 0.0±0.0     & 0.01±0.099     & 0.0±0.0        & 0.0±0.0   & 0.01±0.099     \\
\hline
Colleges Up - Resources Down & 0.0±0.0     & 0.06±0.237     & 0.0±0.0       & 0.0±0.0   & 0.06±0.237     \\
\hline
Colleges Down - Resources Down & 0.0±0.0     & 0.0±0.0        & 0.0±0.0        & 0.03±0.171   & 0.03±0.171     \\
\toprule
\multicolumn{6}{c}{\textbf{Student-vertical Markets - Five resources ($\Resources_0 = \{r_0,r_1,...,r_4\}$)}}\\
\toprule
Market & Resource   & Waste   & Direct-Envy   & Indirect-Envy       & Total    \\
\hline
Colleges Down - Resources Up & 0.0±0.0     & 0.0±0.0        & 0.0±0.0        & 0.31±0.643     & 0.31±0.643      \\
\toprule
\end{tabular}
\label{tab:blocking_contracts_DMC_five_resources_unbalanced_markets}
}
\end{table}


\begin{table}[ht]
\caption{Indirect and direct envy blocking contracts of the five mechanisms presented in \Cref{section:mechanisms} under unbalanced markets with five resources. A market A-B indicates Colleges A and Resources B. We omit the fully-vertical markets case as it obtained very similar results to the college-vertical markets.}
\centering
\begin{tabular}{cccccc}
\toprule
\multicolumn{6}{c}{\textbf{Horizontal Markets}} \\
\toprule
 & Blocking contract & Up-Up & Up-Down & Down-Up & Down-Down \\
\hline
DRC & Indirect-Envy & 0.16±0.524 &  9.28±6.735 & 0.24±0.873 & 2.33±1.97\\
\hline
DMC & Indirect-Envy & 0.13±0.462 &  0.36±0.855 & 0.17±0.601 & 0.96±0.958 \\
\hline
\multirow{2}{*}{RSD} & Indirect-Envy &  0.02±0.14  &  246.61±162.603 & 0.0±0.0 & 14.85±26.48 \\
\cline{2-6}
& Direct-Envy & 104.64±128.738 & 588.32±97.153  & 269.65±85.735 & 523.9±140.148\\
\hline
\multirow{2}{*}{CSD} & Indirect-Envy &  0.0±0.0 & 56.01±41.45 & 0.0±0.0 & 2.94±5.46\\
\cline{2-6}
& Direct-Envy & 22.65±29.896 & 111.77±42.184 & 33.85±26.066 & 64.97±29.648  \\
\toprule
\multicolumn{6}{c}{\textbf{Student-vertical Markets}} \\
\toprule
 & Blocking contract & Up-Up & Up-Down & Down-Up & Down-Down \\
\hline
DRC & Indirect-Envy & 7.64±5.677 & 7.65±5.583  & 3.71±2.692 & 1.92±2.134\\
\hline
DMC & Indirect-Envy & 4.48±3.288  &  0.31±0.643 & 3.86±2.642  & 0.67±0.775\\
\hline
\multirow{2}{*}{RSD} & Indirect-Envy & 23.99±31.548 & 363.97±226.451 & 0.52±3.523 & 26.84±48.04 \\
\cline{2-6}
& Direct-Envy & 275.32±75.458 & 645.18±104.457  & 288.95±74.089 & 598.9±120.894 \\
\hline
\multirow{2}{*}{CSD} & Indirect-Envy &  1.64±2.841  & 15.66±14.124 &  0.05±0.296 & 0.97±2.047\\
\cline{2-6}
& Direct-Envy &20.03±11.413 & 20.41±11.236 & 15.77±11.437 & 17.87±8.515\\
\toprule
\multicolumn{6}{c}{\textbf{College-vertical Markets}} \\
\toprule
 & Blocking contract & Up-Up & Up-Down & Down-Up & Down-Down \\
\hline
DRC & Indirect-Envy & 0.05±0.26 & 16.1±10.301  &  0.02±0.14 & 4.33±5.891 \\
\hline
DMC & Indirect-Envy & 0.0±0.0 & 0.0±0.0 & 0.0±0.0 &  0.03±0.171\\
\hline
\multirow{2}{*}{RSD} & Indirect-Envy & 0.06±0.42  &  243.62±158.935  &0.0±0.0 & 10.71±23.138\\
\cline{2-6}
& Direct-Envy & 115.91±130.028 & 598.49±85.734 & 256.0±86.427  & 532.67±135.066\\
\toprule
\end{tabular}
\label{tab:stability_unbalanced_markets}
\end{table}