% This must be in the first 5 lines to tell arXiv to use pdfLaTeX, which is strongly recommended.
\pdfoutput=1
% In particular, the hyperref package requires pdfLaTeX in order to break URLs across lines.

\documentclass[11pt]{article}

% Change "review" to "final" to generate the final (sometimes called camera-ready) version.
% Change to "preprint" to generate a non-anonymous version with page numbers.
\usepackage[preprint]{acl}
\usepackage{booktabs} % 导言区
\usepackage{multirow} % 导言区
% \usepackage{adjustbox}

% Standard package includes
\usepackage{times}
\usepackage{latexsym}
\usepackage{array} % 确保加载 array 包
\usepackage{caption} % 支持表格标题
% For proper rendering and hyphenation of words containing Latin characters (including in bib files)
\usepackage[T1]{fontenc}
% For Vietnamese characters
% \usepackage[T5]{fontenc}
% See https://www.latex-project.org/help/documentation/encguide.pdf for other character sets

% This assumes your files are encoded as UTF8
\usepackage[utf8]{inputenc}

% This is not strictly necessary, and may be commented out,
% but it will improve the layout of the manuscript,
% and will typically save some space.
\usepackage{microtype}

% This is also not strictly necessary, and may be commented out.
% However, it will improve the aesthetics of text in
% the typewriter font.
\usepackage{inconsolata}

%Including images in your LaTeX document requires adding
%additional package(s)
\usepackage{graphicx}

% If the title and author information does not fit in the area allocated, uncomment the following
%
%\setlength\titlebox{<dim>}
%
% and set <dim> to something 5cm or larger.

\newcommand{\red}[1]{\textcolor{blue}{#1}}

% \title{A Framework for Low-Resource Language Pretraining Based on Shared Weights: Application to Chinese Minority Languages}
\title{Multilingual Encoder Knows more than You Realize:\\Shared Weights Pretraining for Extremely Low-Resource Languages}

% Author information can be set in various styles:
% For several authors from the same institution:
% \author{Author 1 \and ... \and Author n \\
%         Address line \\ ... \\ Address line}
% if the names do not fit well on one line use
%         Author 1 \\ {\bf Author 2} \\ ... \\ {\bf Author n} \\
% For authors from different institutions:
% \author{Author 1 \\ Address line \\  ... \\ Address line
%         \And  ... \And
%         Author n \\ Address line \\ ... \\ Address line}
% To start a separate ``row'' of authors use \AND, as in
% \author{Author 1 \\ Address line \\  ... \\ Address line
%         \AND
%         Author 2 \\ Address line \\ ... \\ Address line \And
%         Author 3 \\ Address line \\ ... \\ Address line}

\author{%
  Zeli Su\textsuperscript{1,2}   \
  Ziyin Zhang\textsuperscript{3}  \
  Guixian Xu\textsuperscript{1,2 $\dagger$}  \
  Jianing Liu\textsuperscript{2} \AND
  % Ziyin Zhang\textsuperscript{3}\thanks{Zeli ran all the experiments and drafted the paper. Guixian is Zeli's graduate school supervisor. Ziyin supervised this project and revised the paper.}
  XU Han\textsuperscript{1,2} \
  Ting Zhang\textsuperscript{1,2} \
  Yushuang Dong\textsuperscript{1,2}\
  \vspace{6pt}\\
  \textsuperscript{1}Key Laboratory of Ethnic Language Intelligent Analysis and Security Governance of MOE \\
  \textsuperscript{2}Minzu University of China \
  \textsuperscript{3}Shanghai Jiao Tong University \\
  \texttt{\{rickamorty,guixian\_xu,hanxu,jianing\_liu,yushuangdong\}@muc.edu.cn}\\
  \texttt{daenerystargaryen@sjtu.edu.cn} \
  \texttt{tozhangting@126.com} \\
   \textsuperscript{$\dagger$} Corresponding author
}


% \author{
%  \textbf{First Author\textsuperscript{1}},
%  \textbf{Second Author\textsuperscript{1,2}},
%  \textbf{Third T. Author\textsuperscript{1}},
%  \textbf{Fourth Author\textsuperscript{1}},
% \\
%  \textbf{Fifth Author\textsuperscript{1,2}},
%  \textbf{Sixth Author\textsuperscript{1}},
%  \textbf{Seventh Author\textsuperscript{1}},
%  \textbf{Eighth Author \textsuperscript{1,2,3,4}},
% \\
%  \textbf{Ninth Author\textsuperscript{1}},
%  \textbf{Tenth Author\textsuperscript{1}},
%  \textbf{Eleventh E. Author\textsuperscript{1,2,3,4,5}},
%  \textbf{Twelfth Author\textsuperscript{1}},
% \\
%  \textbf{Thirteenth Author\textsuperscript{3}},
%  \textbf{Fourteenth F. Author\textsuperscript{2,4}},
%  \textbf{Fifteenth Author\textsuperscript{1}},
%  \textbf{Sixteenth Author\textsuperscript{1}},
% \\
%  \textbf{Seventeenth S. Author\textsuperscript{4,5}},
%  \textbf{Eighteenth Author\textsuperscript{3,4}},
%  \textbf{Nineteenth N. Author\textsuperscript{2,5}},
%  \textbf{Twentieth Author\textsuperscript{1}}
% \\
% \\
%  \textsuperscript{1}Affiliation 1,
%  \textsuperscript{2}Affiliation 2,
%  \textsuperscript{3}Affiliation 3,
%  \textsuperscript{4}Affiliation 4,
%  \textsuperscript{5}Affiliation 5
% \\
%  \small{
%    \textbf{Correspondence:} \href{mailto:email@domain}{email@domain}
%  }
% }

\begin{document}
\maketitle
\begin{abstract}
While multilingual language models like XLM-R have advanced multilingualism in NLP, they still perform poorly in extremely low-resource languages. This situation is exacerbated by the fact that modern LLMs such as LLaMA and Qwen support far fewer languages than XLM-R, making text generation models non-existent for many languages in the world. To tackle this challenge, we propose a novel framework for adapting multilingual encoders to text generation in extremely low-resource languages. By reusing the weights between the encoder and the decoder, our framework allows the model to leverage the learned semantic space of the encoder, enabling efficient learning and effective generalization in low-resource languages. Applying this framework to four Chinese minority languages, we present XLM-SWCM, and demonstrate its superior performance on various downstream tasks even when compared with much larger models.

\end{abstract}

%  long abstract
% Unlike resource-rich languages that achieve remarkable performance in various natural language processing (NLP) tasks, low-resource languages often perform poorly in many tasks due to data scarcity and frequently lack effective model support. In recent years, the XLM-R model, as one of the benchmarks for multilingual pretraining, has led to the development of a series of related models. However, in the application to low-resource languages, especially in tasks such as text generation and understanding, despite some progress, their performance remains limited, particularly in data-scarce scenarios. To address this issue, we propose a novel pretraining framework. This framework, based on existing encoder-only models, utilizes a shared weight mechanism to reuse the parameters of the encoder layers, mapping them to the corresponding decoder layers. This enables the decoder to directly leverage the rich semantic space, avoiding the need to train from scratch. This innovative design allows the model to learn and generalize more efficiently in data-limited settings. Based on this architecture, we pretrained a model called XLM-SWCN for Chinese minority language tasks. Experimental results show that XLM-SWCN significantly outperforms traditional baseline models and demonstrates faster convergence during training.

%%%%%%%%%%%%%%%%%%%%%%%%%%%%%%%%%%%%%%%%
\section{Introduction}
\label{sec:intro}
%%%%%%%%%%%%%%%%%%%%%%%%%%%%%%%%%%%%%%%%

Deformable linear object (DLO) manipulation is an active area of robotics research that deals with a variety of challenging tasks, such as needle threading, lace tying, cable rearrangement, and lasso throwing~\cite{zhang2021robots, chi2024iterative, haiderbhai2024sim2real}. Although the geometry of a DLO is well understood, humans show dexterity in quickly estimating physical parameters, such as length and stiffness, from relatively few visual observations of object manipulation trajectories. This is often used to condition control policies within these estimations~\cite{kuroki2024gendom, zhang2024adaptigraph}, enabling efficient adaptive control. Achieving a similar level of robotic dexterity requires dealing with the inherent high dimensionality and nonlinearity of such tasks, challenges that are further exacerbated by the inherent noise in visual servoing~\cite{arriola2020modeling, yin2021modeling}.

Learning policies in simulation and then performing a \emph{Sim2Real}~\cite{liang2024real, haiderbhai2024sim2real} deployment follows the hypothesis that the numerous simulated iterations of a given task will robustify the learnt policy to parametric variations, as mentioned above~\cite{peng2018sim}. However, this approach requires one to overcome the ``reality gap'' of robotic simulators, which is particularly problematic for soft objects. 

In this work, we first aim to achieve a reliable \emph{Real2Sim}~\cite{mehta2021user, liang2020learning} calibration of our simulator's parameters $\boldsymbol{\theta}$ to the real-world object parameters. Then, aiming at robust deployment, we account for any uncertainty about these parameters by exposing our learning algorithm to different hypotheses of $\boldsymbol{\theta}$.

Likelihood-free inference (LFI) deals with solving the inverse problem of probabilistically mapping real-world sensor observations to the respective simulation parameters $\boldsymbol{\theta}$ that are most likely to account for the observations.

\begin{figure}[!t]
    \centering
    \includegraphics[width=1.0\columnwidth]{fig-real2sim2real--workflow.pdf}
    \caption{Overview of our Real2Sim2Real framework. We perform inference for the posterior distribution $\hat{p}$ over system parameters $\boldsymbol{\theta}$ (Real2Sim). We use $\hat{p}$ to perform domain randomisation while training a PPO agent to perform a visuomotor DLO control task. We deploy and evaluate our sim-trained policy in the real world (Sim2Real).}
    \label{fig:header-system-overview}
\end{figure}

This involves inferring the multimodal distribution from which we can sample sets of simulation parameters with a high probability of replicating the modelled real-world phenomenon~\cite{ramos2019bayessim}. Any uncertainty over $\boldsymbol{\theta}$ can be modelled as the variances associated with the modes.

This enables us to perform Domain Randomisation (DR), which aims to create a variety of simulated environments, each with randomised system parameters, and then to train a policy that works in all of them for a given control task objective. Assuming that the parameters of the real system is a sample in the distribution associated with the variations seen at training time, it is expected that a policy learnt in simulation under a DR regime will adapt to the dynamics of the real world environment.

Recent literature has demonstrated how distributional embeddings, such as reproducing kernel Hilbert spaces~\cite{muandet2017kernel}, of inferred keypoint trajectories combine low dimensionality and robustness to visual data noise~\cite{antonova2022bayesian}, thus enabling robust Real2Sim calibration. On the Sim2Real side, we have seen fruitful model-based Reinforcement Learning (RL) approaches~\cite{zeng2021transporter, seita2021learning} to train deformable object manipulation policies in simulation and deploy them in the real world. However, we have yet to see an end-to-end \emph{Real2Sim2Real} system which combines the expressiveness of Bayesian inference with the flexibility of model-free RL~\cite{schulman2017proximal}.

We propose an integrated framework addressing a number of different technical considerations (Fig.~\ref{fig:header-system-overview}, Alg.~\ref{alg:real2sim2real-bsim}). 
We use BayesSim~\cite{ramos2019bayessim} with distributional state embeddings to infer the posteriors over physical parameters $\boldsymbol{\theta}$ of a set of visually observed (real) DLOs of varied length and stiffness.
We use each inferred posterior $\hat{p}(\boldsymbol{\theta})$ to perform domain randomisation and train a Sim2Real transferable model-free RL policy to perform a visuomotor ``reaching'' task with a DLO. We deploy our sim-trained policies in the real world and compare their performance across our set of DLOs.

In this paper, we make the following contributions.
\begin{enumerate}
    \item We present an \textbf{end-to-end Real2Sim2Real framework} for robust \textbf{vision-based} DLO manipulation. 
    \item We examine the capacity of BayesSim to \textbf{finely classify} the different physical properties of a deformable object drawn from a parametric set.
    \item We study the implications of \textbf{different randomisation domains} for RL policy learning in simulation and how this translates to \textbf{real world performance}. 
\end{enumerate}
Our experiments show that for a parameterised set of DLOs, an integrated distributional treatment of parameter inference, policy training, and zero-shot deployment enables inferring fine differences in physical properties and adapting RL agent behaviour to them.

\section{Related Works}\label{sec:related-work}
\subsection{Multilingual Corpus}
The evolution of multilingual large language models (LLMs) has been enabled by the release of extensive multilingual corpora such as CC100, mC4, OSCAR, CulturaX, and Madlad-400 \cite{CC100,mc4,OSCAR,culturax,Madlad}. While these resources cover a selection of low-resource languages to some extend, there remains a recognized gap in the representation for China's minority languages, primarily due to significant differences in writing systems.
   
China’s minority languages often use different writing systems from the same language family used elsewhere in the world. For example, Uyghur is primarily written in the Arabic script (UEY—Uyghurche Ereb Yëziqi) in China, with the Latin script (ULY—Uyghurche Latin Yëziqi) used as a supplementary form. In contrast, Uyghur in Russia and Central Asia is written in the Cyrillic script (USY—Uyghurche Shilir Yëziqi). When collecting data for minority languages, the aforementioned multilingual corpora either do not distinguish between such different writing systems, or only contain data from one system, as shown in Figure~\ref{fig:1_plot}.
   
Recently, the release of the Multilingual Corpus of Minority Languages in China (MC2,~\citealp{mc2}) breaks the gap in the availability of Chinese minority language pretraining corpora, covering four underrepresented languages: Tibetan, Uyghur, Kazakh, and Mongolian. This dataset is used as the primary pretraining corpus in our work.

    
\subsection{Development of Multilingual Language Models}
In the past few years, multilingual variants of pretrained language models have been proposed in the NLP community, such as mBART~\citep{mbart} and mT5~\citep{mt5}, supporting up to 100 languages and demonstrating powerful cross-lingual transfer capabilities. More recently, the emergence of large language models (LLMs) has revolutionized multilingual natural language processing. Models like PaLM~\citep{palm} and BLOOM~\citep{bloom} have made significant strides in multilingual capabilities, while the LLaMA family~\citep{llama} and its multilingual variants have democratized access to multilingual LLMs. Some specialized models represented by XGLM and NLLB~\citep{xglm,nllb} have focused on expanding language coverage and improving cross-lingual transfer capabilities across hundreds of low-resource languages. However, few of these models support Chinese minority languages.

% pre-trained language models have achieved significant breakthroughs in natural language processing, represented by BERT, RoBERTa, and T5 \cite{bert,roberta-base,T5}. 

\subsection{NLP for Minority Languages in China}
To enhance the accessibility of minority languages in China, prior studies have primarily focused on curating annotated datasets for various NLP tasks. These efforts have mainly concentrated on three key task categories: text classification \cite{bo-classfy,bo-classfy-2,bo-classfy-3}, question answering \cite{boqa}, and machine translation \cite{bomt}. Prominent models specifically trained for these languages include CINO~\citep{cino}, MiLMo~\citep{milmo}, and TiBert~\citep{tibert}. However, despite such progress, none of these models have released their pre-training corpora, and there is still a notable gap in the availability of models capable of text generation in these languages.


    
% Despite these advancements, significant challenges persist in effectively adapting models to text generation tasks for Chinese minority languages, particularly in low-resource settings. Currently, there is no unified model that can effectively handle text generation while leveraging existing resources for these languages. To address this gap, we propose \textbf{XLM-SWCM} (XLM-Shared Weight Chinese Minority), a novel seq2seq architecture that utilizes weight-sharing transfer learning from existing encoder models, offering a more efficient approach to handling text generation tasks across Chinese minority languages.
%%%%%%%%%%%%%%%%%%%%%%%%%%%%%%%%%%%%%%%%
\section{Method}
\label{sec:method}
%%%%%%%%%%%%%%%%%%%%%%%%%%%%%%%%%%%%%%%%

%%%%%%%%%%
\subsection{Keypoint detection from segmentation images}
\label{subsec:real2sim2real-seg-img-n-kps}
%%%%%%%%%%

Our perception module receives an RGB image observation as input, in which it detects the environment objects of interest, i.e. the controlled DLO and the visuomotor control target. We use detection masks to create segmentation images, through which we track our task's features for parameter inference and policy learning.

We use segmentation images to efficiently learn an unsupervised keypoint detection model, using the \emph{transporter} method~\cite{kulkarni2019unsupervised}, as implemented by~\cite{li2020causal}.
The method's combination of keypoint-based bottleneck layer and downstream reconstruction task maintains the temporal consistency of the extracted set of keypoints. However, keypoints are inferred in every timeframe, which in practice, particularly when working with real-world image data of deformable objects, creates problems such as keypoint position noise and permutations~\cite{antonova2022bayesian}.

%%%%%%%%%%
\subsection{Real2Sim with likelihood-free inference}
\label{subsec:real2sim-bayessim}
%%%%%%%%%%

Our problem is that of manipulating a DLO for a visuomotor reaching task, by learning a policy in simulation and deploying in the real-world, without any further fine-tuning. The Real2Sim part of our work deals with calibrating physical parameters that induce the deformable object's behaviour, such as its dimensions and material behaviour, which are difficult to tune manually. Details such as camera and workspace object placement, controller stiffness parameters, etc. are beyond the scope of our work, although they can be incorporated as additional tunable parameters in extensions of our method.

For our inference and domain randomisation experiments, we define a physical parameter vector $\boldsymbol{\theta} = \langle l, E \rangle$, where $l$ denotes the length of our DLO and $E$ denotes its Young's modulus. Thus, we want to infer a joint posterior $\hat{p}$, which contains information on both the dimensions and the material properties of our deformable object. For this, we use the BayesSim-RKHS variant~\cite{antonova2022bayesian}.

Following Algorithm~\ref{alg:real2sim2real-bsim}, we begin by assuming a uniform proposal prior $\Tilde{p}(\boldsymbol{\theta})$, which we use to initialise our reference prior $p_0 = \Tilde{p}$. We then perform domain randomisation as $\boldsymbol{\theta} \sim p_0$, while training in simulation an initial policy $\pi_{\boldsymbol{\beta}_0}$ for our task. We execute a rollout of $\pi_{\boldsymbol{\beta}_0}$ in the real world environment, to collect a real trajectory $\mathbf{x}^r$ while manipulating a specific DLO. We perform LFI through \emph{multiple} BayesSim iterations~\cite{possas2020online, antonova2022bayesian}. In each inference iteration $i$, we use $\pi_{\boldsymbol{\beta}_0}$ as a data collection policy, running $N$ rollouts of $\pi_{\boldsymbol{\beta}_0}$ in simulation to collect a dataset $\{(\boldsymbol{\theta}, \mathbf{x}^s)\}^N, \boldsymbol{\theta} \sim \Tilde{p}$, which we use to train our BayesSim conditional density function $q_{\phi}$. We use our $q_{\phi}$ and $\mathbf{x}^r$ to compute the posterior $\hat{p}(\boldsymbol{\theta} \mid \mathbf{x} = \mathbf{x}^r)$. We then update our reference prior $p_i = \hat{p}$ and loop again.

We can now adapt our domain randomisation distribution to the latest inferred posterior $\hat{p}$, and we proceed to retrain our task policy $\pi_{\boldsymbol{\beta}_1}$. Our hypothesis is that by sampling $\boldsymbol{\theta}$ from our object-specific posterior, we will get faster $\pi_{\boldsymbol{\beta}_1}$ convergence to consistently successful behaviour, and that running a rollout of $\pi_{\boldsymbol{\beta}_1}$ in the real-world environment we will collect higher cumulative reward for the specific DLO within the task episode's horizon.

\begin{figure}[t]
    \centering
    \includegraphics[width=1.0\columnwidth]{fig-distr-state-embed-method.pdf}
    \caption{An overview of our policy rollout and trajectory perception method, with keypoint extraction from segmentation images, and keypoint trajectory cos-only kernel mean embeddings using the RKHS-net layer (BayesSim-RKHS), illustration inspired by~\cite{antonova2022bayesian}.}
    \label{fig:perc-to-rkhs}
\end{figure}

\subsection{Keypoint trajectories in RKHS}

We use an RKHS-net layer~\cite{antonova2022bayesian} to construct a distributional state representation (Fig.~\ref{fig:perc-to-rkhs}), which is provably robust to visual noise and uncertainty due to inferred keypoint permutations. Intuitively, through our kernel mean embeddings~\cite{muandet2017kernel} we pull data from the input space (keypoint trajectory) to the feature space (RKHS). The explicit locations of the pulled points are not necessarily known or might be uncertain, but the relative similarity (inner product) of the pulled data points is known in the feature space~\cite{ghojogh2021reproducing}. This gives us noise robustness and permutation invariance, and it generally means that functional representations that are similar to each other are embedded closer within the RKHS, and thus they are more likely to be classified as similar.

The input of the RKHS-net layer is a vector of samples of a distribution. In this work, we compute the distributional embedding for a noisy trajectory of keypoints $\mathbf{x}~=~(\mathbf{x_1},...,\mathbf{x_n})$, where each keypoint is defined in a $2$D RGB pixel space. 

%%%%%%%%%%
\subsection{Policy Learning and Sim2Real Deployment}
\label{subsec:real2sim-param-inference}
%%%%%%%%%%

We use Proximal Policy Optimisation (PPO)~\cite{schulman2017proximal} as our reference model-free \emph{on-policy} RL algorithm. PPO is designed to maximise the expected reward with a clipped surrogate objective to prevent large updates. It samples actions from a Gaussian distribution, which is parameterised during policy learning.
We perform policy learning in simulation, performing domain randomisation using our likelihood-free inference \emph{reference prior} $p$ as a task parameterisation sampler, where $p$ is either the default uniform distribution $\mathit{U}$, or an inferred MoG posterior (Alg.~\ref{alg:real2sim2real-bsim}, line~\ref{alg:r2s2r:line:p-update}). 

%%%%%%%%%%%%%%%%%%%%%%%%%%%%%%%%%%%%%%%%%%%%%%%%%%
\begin{algorithm}[t]
\caption{Real2Sim2Real for DLO manipulation}
\label{alg:real2sim2real-bsim}
\begin{algorithmic}[1]
    \State \textbf{Given:} $N_{\text{LFI}}$: inference iterations
    \State Assume uniform proposal prior $\Tilde{p}(\boldsymbol{\theta}) \approx \mathit{U}$
    \State Assign reference prior $p_0 \gets \Tilde{p}$
    \State Train initial policy $\pi_{\boldsymbol{\beta}_0}(\mathbf{a}_t \mid \mathbf{s}_t)$, $\boldsymbol{\theta} \sim p_0$
    \State Run $1$ $\pi_{\boldsymbol{\beta}_0}$ rollout in the real env to collect $\mathbf{x}^r$
    \State \textbf{// 1. Real2Sim DLO parameter inference}
    \State $i \gets 0$
    \While{$i < N_{\text{LFI}}$} \label{alg:r2s2r:line:lfi-iter}
        \State $\{(\boldsymbol{\theta}, \mathbf{x}^s)\}^N \gets$ Run $N$ $\pi_{\boldsymbol{\beta}_0}$ rollouts in sim, $\boldsymbol{\theta} \sim p_i$
        \State Train $q_{\phi}$ over $\{(\boldsymbol{\theta}, \mathbf{x}^s)\}^N$
        \State $\hat{p}(\boldsymbol{\theta} \mid \mathbf{x} = \mathbf{x}^r) \propto p_i(\boldsymbol{\theta}) \mathbin{/} \Tilde{p}(\boldsymbol{\theta}) q_{\phi}(\boldsymbol{\theta} \mid \mathbf{x} = \mathbf{x}^r)$
        \State Update reference prior $p_i \gets \hat{p}$ \label{alg:r2s2r:line:p-update}
        \State $i \gets i + 1$
    \EndWhile
    \State \textbf{// 2. Policy training in sim}
    \State Train policy $\pi_{\boldsymbol{\beta}_1}(\mathbf{a}_t \mid \mathbf{s}_t)$, $\boldsymbol{\theta} \sim \hat{p}$
    \State \textbf{// 3. Sim2Real policy deployment}
    \State Evaluate $\pi_{\boldsymbol{\beta}_1}$ in the real env by running $1$ $\pi_{\boldsymbol{\beta}_1}$ rollout
\end{algorithmic}
\end{algorithm}
%%%%%%%%%%%%%%%%%%%%%%%%%%%%%%%%%%%%%%%%%%%%%%%%%%

\section{Experiments}\label{sec4:experiments}
\subsection{Pretraining}
% For the pre-training phase, we conducted experiments using three variants of our model (small, base, and large) with carefully tuned hyperparameters.

\paragraph{Training Configuration}
The models are trained for 8 epochs with a peak learning rate of 1e-4, AdamW~\citep{2017AdamW} optimizer, global batch size 600, and a linear learning rate scheduler with a warmup proportion of 0.1. The maximum sequence length is set to 256 tokens, and mixed-precision is enabled to optimize memory usage and training efficiency. To ensure training stability, the norms of gradients are clipped to 1.0. The models are trained on two NVIDIA A800 GPUs, each with 80GB of memory, and the training process takes 92 hours.

\paragraph{Balanced Sampling Strategy}
To address the inherent data imbalance across different languages, we implemente a balanced sampling strategy similar to XLM-R. The sampling probability for each language is calculated as

\begin{equation}
    p_i = \frac{q_i^\alpha}{\sum_j q_j^\alpha},
\end{equation}

\noindent where $q_i$ represents the original proportion of language $i$ in the dataset, and $\alpha$ (set to 0.3) is a smoothing parameter that balances between uniform sampling and size-proportional sampling. This approach ensures that low-resource languages receive adequate representation in the training process while maintaining the influence of larger datasets.
    
    
\paragraph{Model Adaptations}
We extende the model's vocabulary with special language tokens (<bo>, <kk>, <mn>, <ug>, <zh>) to handle our target languages (Tibetan, Kazakh, Mongolian, Uyghur, and Chinese). These language identifiers are directly added after the bos token <s> in the model inputs. This modification ensures that the model can effectively process and distinguish between different languages during both pre-training and downstream task finetuning. The same approach is consistently applied in all subsequent experiments.

% \paragraph{Loss Computation}
% To handle memory constraints when processing large sequences, we implemented a chunked cross-entropy loss calculation method. This approach divides the computation into manageable chunks of 1,000,000 tokens, enabling efficient training even with limited GPU resources while maintaining numerical stability.

Based on the aforementioned settings, we trained a new seq2seq model - XLM-SWCM, utilizing CINO-base-v2 as the encoder, with 457 million parameters. The detailed architectural configuration is provided in Appendix \ref{sec:Training Detail}.




\subsection{Downstream Tasks}

\subsubsection{Experiment Setting}\label{sec:experiments-downstream-setting}

To evaluate the capabilities of XLM-SWCM, we conduct fine-tuning experiments on three downstream tasks in both low-resource and high-resource languages: Text Summarization, Machine Reading Comprehension (MRC), and Machine Translation. These tasks are chosen to cover diverse areas of text generation in NLP.

\paragraph{Single-Language Fine-tuning} Due to the scarcity of labeled data for low-resource languages, we focus primarily on Tibetan for single-language fine-tuning, which has several publicly available datasets:

- \textbf{Text Summarization}: For this task, we utilize the Ti-Sum dataset~\cite{ti-sum} with 20,000 pairs of titles and articles.

- \textbf{MRC}: We mainly use the TibetanQA dataset~\cite{tibetanqa} for this task, which claims to contain 20K examples. However, only 2K examples are publicly available. Thus we enrich it by integrating 5K examples from the TibetanSFT Corpus\footnote{\href{https://huggingface.co/datasets/shajiu/ParallelCorpusSFT}{https://huggingface.co/datasets/shajiu/ParallelCorpusSFT}} and 3K examples translated from a Chinese MRC dataset \cite{chinese-mrc} using Google Translate. This approach enables us to create a comprehensive dataset consisting of 10K examples.

- \textbf{Machine Translation}: For Machine Translation, we also use the TibetanSFT Corpus, which is cleaned to generate 50,000 parallel Chinese-Tibetan sentence pairs.

\paragraph{Cross-lingual Transfer} In addition to single-language fine-tuning, we also conduct cross-lingual transfer experiments to test XLM-SWCM’s ability to generalize across multiple low-resource languages. This experiment aims to assess the model's performance in Tibetan, Uyghur, Mongolian, and Kazakh after being fine-tuned on a high-resource language (Simplified Chinese) and a very small number of samples in the target languages.

- \textbf{Text Summarization}: For Mandarin Chinese, we use the publicly available LCSTS dataset~\citep{lcsts}, which contains 100K samples scraped from various Chinese portals. For the four minority languages, approximately 3K cleaned samples per language are scraped from language-specific news portals, using the news titles as their summarization.
    
- \textbf{MRC}: For Chinese, we employ the CMRC 2018 dataset~\citep{cmrc2018}, which consists of 10K samples. For Tibetan, we use 500 samples extracted from the publicly available TibetanQA dataset. For the other three minority languages (Uyghur, Mongolian, Kazakh), we utilize machine translation tools to translate and clean MRC data, ultimately selecting 500 samples per language.

\paragraph{Baseline Models}
We employ two baseline models to ensure broad coverage and robust performance in handling Chinese minority languages. The first model builds upon LLaMA2-Chinese and is fine-tuned on the MC2 dataset, resulting in the \textit{MC2-LLaMA-13B} model. The second baseline, referred to as \textit{mBART-CM}, is an adaptation of mBART-cc25. Its vocabulary is expanded to include tokens specific to our minority languages, followed by further pretraining on MC2.

\paragraph{Training settings}
Both XLM-SWCM and mBART-CM are sequence-to-sequence models that are fine-tuned using standard training configurations. Each of these models is trained for 50 epochs with a batch size of 200 samples to ensure comprehensive learning and optimal performance. MC2-LLaMA-13B model is trained using LoRA~\citep{lora} with a rank of 8 for 3 epochs.


\begin{table*}[ht]
  \centering
  \begin{tabular}{c c ccc ccc ccc}
    \toprule
    \multirow{2}{*}{\textbf{Model}} & \multirow{2}{*}{\textbf{Size}}
    & \multicolumn{3}{c}{\textbf{Sum}}
    & \multicolumn{3}{c}{\textbf{MRC}}
    & \multicolumn{3}{c}{\textbf{MT}} \\[0.5ex]
    \cmidrule(lr){3-5}\cmidrule(lr){6-8}\cmidrule(lr){9-11}
    & 
    & \textbf{F} & \textbf{P} & \textbf{R}
    & \textbf{F} & \textbf{P} & \textbf{R}
    & \textbf{F} & \textbf{P} & \textbf{R}\\[0.5ex]
    \midrule
    MC2-LLaMA-13B & 13B
    & 16.1 & 12.3 & 15.5
    & 13.2 & 11.7 & 13.1
    & 15.1 & 12.2 & 16.8 \\[0.5ex]
    mBART-CM & 611M
    & 8.6 & 11.2 & 15.2
    & 7.9  & 6.1 & 5.6
    & 11.5 & 7.3 & 9.3 \\[0.5ex]
    XLM-SWCM (ours) & 457M
    & \textbf{25.7} & \textbf{29.1} & \textbf{24.2}
    & \textbf{16.4} & \textbf{29.5} & \textbf{16.2}
    & \textbf{24.5} & \textbf{26.3} & \textbf{24.3} \\[0.5ex]
    \bottomrule
  \end{tabular}
  \caption{\label{single}
    Performance metrics of the baseline models, evaluated using three ROUGE-L sub metrics: 
    \textbf{F} (F1-score), \textbf{P} (precision), 
    and \textbf{R} (recall). Size refers to the number of parameters in each model.
  }
\end{table*}


\begin{table*}[ht]
    \centering
    \begin{tabular}{l cc cc cc cc cc}
        \toprule
        \multirow{2}{*}{\textbf{Model}} 
        & \multicolumn{2}{c}{\textbf{Zh}} 
        & \multicolumn{2}{c}{\textbf{Bo}}
        & \multicolumn{2}{c}{\textbf{Ug}} 
        & \multicolumn{2}{c}{\textbf{Mn}} 
        & \multicolumn{2}{c}{\textbf{Kk}} \\[0.5ex]
        \cmidrule(lr){2-3}\cmidrule(lr){4-5}\cmidrule(lr){6-7}\cmidrule(lr){8-9}\cmidrule(lr){10-11}
        & \textbf{Sum} & \textbf{MRC}
        & \textbf{Sum} & \textbf{MRC}
        & \textbf{Sum} & \textbf{MRC}
        & \textbf{Sum} & \textbf{MRC}
        & \textbf{Sum} & \textbf{MRC} \\[0.5ex]
        \midrule
        MC2-LLaMA-13B    & 47.1 & 43.5 & 9.5 & 6.1 & 3.5 & 2.4 & 3.7 & 2.2 & 2.6 & 3.9 \\[0.5ex]
        MC2-LLaMA-13B*   & \textbf{47.3} & \textbf{44.7} & 13.1 & \textbf{11.5} & 11.7 & 10.1 & 9.7 & \textbf{10.2} & 2.9 & 4.6 \\[0.5ex]
        mBART-CM     & 32.7 & 25.6 & 6.8 & 2.1 & 2.7 & 2.2 & 3.1 & 1.7 & 0.2 & 0.1 \\[0.5ex]
        XLM-SWCM (ours)     & 33.1 & 23.5 & \textbf{17.1} & 11.1 & \textbf{12.5} & \textbf{11.1} & \textbf{13.5} & 7.2 & \textbf{5.6} & \textbf{6.9} \\[0.5ex]
        \bottomrule
    \end{tabular}
    \caption{\label{fewshot}
     Cross-lingual Transfer performance of different models on Text Summarization (Sum) and Machine Reading Comprehension (MRC) tasks, evaluated using ROUGE-L. The best results for each task are highlighted. *~indicates explicitly prompting MC2-LLaMA-13B with the language to be used in the response during evaluation.
    }
\end{table*}



\subsubsection{Experimental Results}

As illustrated in Table~\ref{single}, XLM-SWCM consistently outperforms the baseline models across all three tasks. Despite having fewer parameters, XLM-SWCM demonstrates a substantial margin of superiority over mBART-CM and even surpasses the much larger MC2-LLaMA-13B.

Notably, XLM-SWCM achieves an impressive \textbf{198.8\% improvement in F1-score for Text Summarization} over mBART-CM, along with a significant \textbf{107.6\% F1 improvement in MRC}. These remarkable gains are a direct result of XLM-SWCM's efficient weight sharing framework to maximize the utilization of pre-trained encoder features in resource-constrained scenarios. Even under equivalent seq2seq structures and identical training corpora, XLM-SWCM demonstrates greater efficiency and learning capacity.

In comparison to MC2-LLaMA-13B, which benefits from richer pretraining corpora and larger-scale parameters, XLM-SWCM achieves a \textbf{59\% higher F1-score in Text Summarization}, a \textbf{24.1\% F1 improvement in MRC}, and a \textbf{62.3\% higher F1-score in MT}. These results underscore the effectiveness of XLM-SWCM's shared weight framework in resource-constrained environments, making it a superior choice for tasks involving Chinese minority languages.

Table~\ref{fewshot} highlights the performance of XLM-SWCM and baseline models in cross-lingual transfer settings. For the primary source language (Zh), the baseline models demonstrate better performance, which stems from their larger parameter sizes and more extensive pretraining corpora in Simplified Chinese. However, when it comes to \textbf{generalization to minority languages}, XLM-SWCM showcases exceptional adaptability, significantly outperforming the baseline models. mBART-CM, for instance, struggles to distinguish between languages and often defaults to outputs in the primary language (Zh), even when language-specific labels are present. Similarly, MC2-LLaMA-13B exhibits language-related errors, though its performance improves when explicitly informed of the current language type, as seen with MC2-LLaMA-13B*.

In Text Summarization, XLM-SWCM outperforms all baselines. Specifically, XLM-SWCM achieves significant improvements of \textbf{30.5\%, 6.8\%, and 39.1\%} for Tibetan (Bo), Uyghur (Ug), and Mongolian (Mn) respectively over MC2-LLaMA-13B*, the best-performing baseline. For MRC, XLM-SWCM also demonstrates competitive performance across most languages, being only slightly weaker than MC2-LLaMA-13B* for Tibetan and Mongolian. 

Overall, these experiments indicate that XLM-SWCM can effectively leverage the shared weight mechanism to maximally reuse the semantic space of the pre-trained encoder, demonstrating excellent performance in Chinese minority language applications with limited data and parameter size.

\begin{table}[ht]
  \centering
  \resizebox{\columnwidth}{!}{
  \begin{tabular}{c c c c c}
    \toprule
    \textbf{Removing Module} & \textbf{Sum} & \textbf{MRC} & \textbf{MT} \\[0.5ex]
    \midrule
    None (XLM-SWCM) & \textbf{25.7} & \textbf{16.4} & \textbf{24.5} \\[0.5ex]
    MT & 25.6 & 15.1 & 20.3 \\[0.5ex]
    DAE & 22.4  & 12.2 & 18.7 \\[0.5ex]
    WS & 17.1  & 11.7 & 18.2 \\[0.5ex]
    MT + DAE & 22.5 & 12.3 & 17.7 \\[0.5ex]
    MT + WS  & 17.5 & 11.3 & 18.4 \\[0.5ex]
    DAE + WS & 15.2  & 11.9  & 17.1 \\[0.5ex]
    MT + DAE + WS & 15.9 & 10.8 & 16.5 \\[0.5ex]
    \bottomrule
  \end{tabular}
  }
  \caption{\label{ablation-single}
    Objective ablation results, evaluated using ROUGE-L.
    The experiments involve removing different combinations of training components, such as Machine Translation (MT), DAE (Denoising Auto-Encoding), and Weight Sharing (WS).
  }
\end{table}
% \begin{table*}[ht]
%     \centering
%     \begin{tabular}{l cc cc cc cc cc}
%         \toprule
%         \multirow{2}{*}{\centering \textbf{Removing Module}} 
%         & \multicolumn{2}{c}{\textbf{Zh}} 
%         & \multicolumn{2}{c}{\textbf{Bo}}
%         & \multicolumn{2}{c}{\textbf{Ug}} 
%         & \multicolumn{2}{c}{\textbf{Mn}} 
%         & \multicolumn{2}{c}{\textbf{Kk}} \\[0.5ex]
%         \cmidrule(lr){2-3}\cmidrule(lr){4-5}\cmidrule(lr){6-7}\cmidrule(lr){8-9}\cmidrule(lr){10-11}
%         & \textbf{Sum} & \textbf{MRC}
%         & \textbf{Sum} & \textbf{MRC}
%         & \textbf{Sum} & \textbf{MRC}
%         & \textbf{Sum} & \textbf{MRC}
%         & \textbf{Sum} & \textbf{MRC} \\[0.5ex]
%         \midrule
%          -MT  & 31.9 & 22.3 & 16.5 & 11.2 & 12.4 & 10.8 & 13.5 & 7.2 & 5.6 & 6.9 \\[0.5ex]
%          -MLM & 29.1 & 21.2 & 17.1 & 11.1 & 12.5 & 11.1 & 13.5 & 7.2 & 5.6 & 6.9 \\[0.5ex]
%          -WS  & 32.1 & 23.2 & 17.1 & 11.1 & 12.5 & 11.1 & 13.5 & 7.2 & 5.6 & 6.9 \\[0.5ex]
%         -MT + MLM  & 32.1 & 23.2 & 17.1 & 11.1 & 12.5 & 11.1 & 13.5 & 7.2 & 5.6 & 6.9 \\[0.5ex]
%         -MT + WS & 32.1 & 23.2 & 17.1 & 11.1 & 12.5 & 11.1 & 13.5 & 7.2 & 5.6 & 6.9 \\[0.5ex]
%         -MLM + WS & 32.1 & 23.2 & 17.1 & 11.1 & 12.5 & 11.1 & 13.5 & 7.2 & 5.6 & 6.9 \\[0.5ex]
%         -MT + MLM + WS  & 32.1 & 23.2 & 17.1 & 11.1 & 12.5 & 11.1 & 13.5 & 7.2 & 5.6 & 6.9 \\[0.5ex]
%         XLM-SWCM (ours) & 33.1 & 23.5 & \textbf{17.1} & 11.1 & \textbf{12.5} & \textbf{11.1} & \textbf{13.5} & 7.2 & \textbf{5.6} & \textbf{6.9} \\[0.5ex]
%         \bottomrule
%     \end{tabular}
%     \caption{\label{fewshot}
%      Few-shot performance of different models on Text Summarization (Sum) and Machine Reading Comprehension (MRC) tasks, evaluated using ROUGE-L as the metric. The best results for each task are highlighted. The asterisk (*) indicates that MC2-LLaMA-13B was explicitly prompted with the language to be used in the response during evaluation.
%     }
% \end{table*}


\section{Ablation Studies}\label{sec:ablation}
In this section, we present a series of ablation experiments aimed at evaluating the impact of key components in our framework that play essential roles in enhancing the model’s multilingual capabilities and improving its generalization to low-resource languages. We perform ablation experiments on the Tibetan finetuning tasks, maintaining a consistent finetuning setting with Section~\ref{sec:experiments-downstream-setting}.

\subsection{Objective  Ablation}
We first focus on three critical aspects of the model: DAE pretraining, machine translation, and weight initialization by removing each and combinations of them. The results are shown in Table~\ref{ablation-single}. Removing any of the three components is detreimental to performance, specifically:

- Machine Translation (MT): Removing machine translation has a relatively small impact on performance across tasks, as shown by both individual removal (maintaining 25.6 in Sum) and combined removals (MT+DAE vs DAE showing similar scores);
    
- Denoising Auto-Encoding (DAE): The removal of DAE pretraining causes considerable performance drops across all three downstream tasks, and its impact becomes more pronounced in combined removals (DAE+WS), indicating its fundamental importance in establishing the model's basic text generation capabilities.

- Weight Sharing (WS): The removal of weight sharing demonstrates the most significant impact among all modules, showing the largest performance drops in individual removal and maintaining this substantial negative effect across all combined removal scenarios, establishing it as the most crucial component for the model's effectiveness in low-resource settings.

In short, while all three components contribute positively to the model's performance, weight sharing emerges as the most critical component. This finding highlights the importance of weight sharing as a key architectural choice for multilingual models, especially in resource-constrained scenarios.

% These experiments demonstrate that shared weights are particularly beneficial for low-resource language tasks, enabling better generalization from limited data. 


\subsection{Structure Ablation}
We also perform experiments to evaluate the impact of different structural components in our proposed framework. These experiments aim to understand how the initialization of decoder weights and the insertion of normal layers affect model performance.

\subsubsection{Impact of Weight Initialization}
Firstly, we train a baseline model called \textbf{Cino-Transformer}. Unlike XLM-SWCM, the decoder of this model is randomly initialized, and also matches the number of encoder layers. The model is pretrained using the same DAE and MT tasks as XLM-SWCM but without weight sharing, and then finetuned on downstream tasks in the same setting as XLM-SWCM.

\begin{table}[ht]
  \centering
  \resizebox{\columnwidth}{!}{
  \begin{tabular}{c c c c c}
    \toprule
    \textbf{Model} & \textbf{Sum} & \textbf{MRC} & \textbf{MT} \\[0.5ex]
    \midrule
    Cino-Transformer & 18.9 & 13.5 & 18.7 \\[0.5ex]
    XLM-SWCM (ours) & \textbf{25.7} & \textbf{16.4} & \textbf{24.5} \\[0.5ex]
    \bottomrule
  \end{tabular}
  }
 \caption{\label{ablation-structure}
   Performance metrics of the Ablation of Weight Initialization, evaluated using the ROUGE-L metric. 
}
\end{table}

\begin{table}[ht]
  \centering
  \resizebox{\columnwidth}{!}{
  \begin{tabular}{c c c c c}
    \toprule
    \textbf{Model} & \textbf{Sum} & \textbf{MRC} & \textbf{MT} \\[0.5ex]
    \midrule
    BASE-A & 13.7 & 10.3 & 15.7 \\[0.5ex]
    BASE-B & 16.3 & 14.1 & 21.1 \\[0.5ex]
    XLM-SWCM (ours) & \textbf{25.7} & \textbf{16.4} & \textbf{24.5} \\[0.5ex]
    \bottomrule
  \end{tabular}
  }
\caption{\label{ablation-Normal-Layers}
   Performance metrics of the Ablation of Normal Layers, evaluated using the ROUGE-L metric. 
   \textbf{BASE-A} has fewer layers and does not include any normal layers, while \textbf{BASE-B} maintains the same number of layers as XLM-SWCM but uses weight duplication instead of normal layers. 
}

\end{table}

The results in Table \ref{ablation-structure} demonstrate the effectiveness of our weight initialization scheme. By transferring weights from the encoder to the decoder, XLM-SWCM can be efficiently adapted to text generation with limited training data, outperforming Cino-Transformer on all tasks.


\subsubsection{Impact of Randomly Initialized Layers}

Secondly, we explore the impact of inserting normal layers among the custom layers in the decoder. To assess the effectiveness of this modification, we use two baseline models for comparison:

- \textbf{Baseline A (XLM-SWCM without normal layers)}: This model is identical to XLM-SWCM but without any normal layers inserted into the custom layer architecture. The absence of normal layers leads to a reduced total number of layers in the decoder.

- \textbf{Baseline B (Weight duplication model)}: Instead of inserting normal layers, this model simply copies the weights of the preceding layer to maintain consistency in the number of model parameters. This results in identical weights across consecutive layers, allowing us to isolate the impact of inserting randomly initialized normal layers.

The results in Table~\ref{ablation-Normal-Layers} demonstrate the significant impact of inserting normal layers into the decoder. BASE-A, which has fewer layers, performs the worst across all tasks. BASE-B, which maintain the same number of layers as XLM-SWCM but lacks randomly initialized weights, shows some improvement but still underperforms.

Overall, these findings indicate that randomly initialized normal layers is also crucial for adapting encoders to text generation.

\subsubsection{Impact of Insertion Frequency of Normal Layers}
\label{sec:5.3.2}
Thirdly, we thoroughly investigate the impact of insertion frequency of normal layers in the decoder, and how this interacts with varying dataset sizes. This experiment is designed along two dimensions:

- \textbf{Insertion Frequency of Normal Layers}: we explore values of \( X \) where a normal layer is inserted after every \( X \) custom layers, with \( X \) ranging from 1 to 6. All these models are pretrained in the same setting as XLM-SWCM.

\begin{figure}
    \centering
    \includegraphics[width=1\linewidth]{figure/line_plot.pdf}
    \caption{ROUGE-L scores on Tibetan summarization for different X-values (insertion frequency of normal layers). The three lines correspond to different dataset sizes.}
    \label{fig:enter-label}
\end{figure}

- \textbf{Effect of Finetuning Dataset Size}: we evaluate the model’s performance on datasets of varying sizes, including 10K, 20K, and 50K samples. As the existing Ti-SUM dataset only has 20K samples, we supplement it by crawling and cleaning 30K additional news articles from various major Chinese websites. This dimension allows us to examine the interaction between the amount of available data and the frequency of normal layers.

The results are plotted in Figure~\ref{fig:enter-label}:

- For the small dataset (10k), larger $X$ results in better performance, as smaller decoders generalize more effectively when data is limited. In contrast, smaller $X$ (i.e. larger decoders) leads to overfitting.

- For the medium dataset (20k), performance peaks at \( X = 3 \). This indicates that a moderate decoder size strikes a balance between capacity and data availability.

- For the large dataset (50k), smaller $X$ achieve the highest F1-scores, as the larger decoder capacity enables the model to fully exploit the larger dataset.

Overall, these results demonstrate the flexibility of our framework, where the insertion frequency of normal layers can be adjusted based on the task-specific dataset size. Larger $X$ (fewer layers) is better suited for small datasets, while smaller $X$ (more layers) performs best on larger datasets.
% For medium datasets, a balance can be achieved with moderate X-values. Based on the experimental results and considering the data scarcity typically found in low-resource languages, our model adopts \( X = 3 \). This configuration achieves a balanced performance across datasets of all sizes, making it a suitable choice for both small and large data scenarios.
\section{Conclusion}\label{sec:conclusion}
In this work, we proposed a novel pretraining framework tailored for low-resource languages, with a particular focus on Chinese minority languages. Our framework leverages a shared weight mechanism between the encoder and decoder, which allows for the efficient adaptation of multilingual encoders to generation tasks without the need to start from scratch. Experimental results demonstrate that our model XLM-SWCM significantly outperforms traditional baselines on various text generation tasks for Tibetan, Uyghur, Kazakh, and Mongolian, which have long been underserved in NLP research. Our approach opens up new possibilities for developing robust models for these extremely low-resource languages, and also provides a promising method for the integration of resources across similar languages.

% We envision extending the shared weight mechanism to a broader range of languages, refining the pretraining process, and adapting the framework to other NLP tasks. Such advancements will help bridge the gap between high- and low-resource languages, ultimately fostering the development of more inclusive, universal language models.

\section{Limitations}
Due to the availability of pretrained language models for Chinese minority languages and high-quality corpora, our study focused on only four minority languages. Our single-language finetuning experiments are further constrained to Tibetan given the lack of relevant datasets, limiting the scope of our exploration.

Thus, we hope that future work will put more focus on the development of high-quality datasets in these minority languages and beyond, enabling a more thorough exploration of underrepresented languages in the LLM era.
% As more data becomes available and the model's capabilities continue to improve, the exploration of these languages will become a key direction for future research.







% \begin{table*}[ht]
%   \centering
%   \begin{tabular}{c c c c c c}
%     \toprule
%     {\textbf{X Value}} & {\textbf{Decoder Layers}} & {\textbf{Dataset Size}} 
%     & \multicolumn{1}{c}{\textbf{Sum}} 
%     & \multicolumn{1}{c}{\textbf{Mrc}} 
%     & \multicolumn{1}{c}{\textbf{Mt}} \\[0.5ex]
%     \midrule
%     \textbf{1} & 24 & 10,000
%     & 16 
%     & 13 
%     & 15 \\[0.5ex]
%     \textbf{2} & 12 & 20,000
%     & 8 
%     & 7  
%     & 11 \\[0.5ex]
%     \textbf{3} & 16 & 50,000
%     & \textbf{25} 
%     & \textbf{16} 
%     & \textbf{24} \\[0.5ex]
%     \textbf{4} & 16 & 50,000
%     & \textbf{25} 
%     & \textbf{16} 
%     & \textbf{24} \\[0.5ex]
%     \textbf{6} & 16 & 50,000
%     & \textbf{25} 
%     & \textbf{16} 
%     & \textbf{24} \\[0.5ex]
%     \bottomrule
%   \end{tabular}
%   \caption{\label{single}
%     Performance metrics of the experiments with different X values, evaluated using the F1-score. 
%     The "Decoder Layers" column indicates the number of layers in each model’s decoder, and the "Dataset Size" column shows the number of training samples used.
%   }
% \end{table*}



\bibliography{reference}

\subsection{Lloyd-Max Algorithm}
\label{subsec:Lloyd-Max}
For a given quantization bitwidth $B$ and an operand $\bm{X}$, the Lloyd-Max algorithm finds $2^B$ quantization levels $\{\hat{x}_i\}_{i=1}^{2^B}$ such that quantizing $\bm{X}$ by rounding each scalar in $\bm{X}$ to the nearest quantization level minimizes the quantization MSE. 

The algorithm starts with an initial guess of quantization levels and then iteratively computes quantization thresholds $\{\tau_i\}_{i=1}^{2^B-1}$ and updates quantization levels $\{\hat{x}_i\}_{i=1}^{2^B}$. Specifically, at iteration $n$, thresholds are set to the midpoints of the previous iteration's levels:
\begin{align*}
    \tau_i^{(n)}=\frac{\hat{x}_i^{(n-1)}+\hat{x}_{i+1}^{(n-1)}}2 \text{ for } i=1\ldots 2^B-1
\end{align*}
Subsequently, the quantization levels are re-computed as conditional means of the data regions defined by the new thresholds:
\begin{align*}
    \hat{x}_i^{(n)}=\mathbb{E}\left[ \bm{X} \big| \bm{X}\in [\tau_{i-1}^{(n)},\tau_i^{(n)}] \right] \text{ for } i=1\ldots 2^B
\end{align*}
where to satisfy boundary conditions we have $\tau_0=-\infty$ and $\tau_{2^B}=\infty$. The algorithm iterates the above steps until convergence.

Figure \ref{fig:lm_quant} compares the quantization levels of a $7$-bit floating point (E3M3) quantizer (left) to a $7$-bit Lloyd-Max quantizer (right) when quantizing a layer of weights from the GPT3-126M model at a per-tensor granularity. As shown, the Lloyd-Max quantizer achieves substantially lower quantization MSE. Further, Table \ref{tab:FP7_vs_LM7} shows the superior perplexity achieved by Lloyd-Max quantizers for bitwidths of $7$, $6$ and $5$. The difference between the quantizers is clear at 5 bits, where per-tensor FP quantization incurs a drastic and unacceptable increase in perplexity, while Lloyd-Max quantization incurs a much smaller increase. Nevertheless, we note that even the optimal Lloyd-Max quantizer incurs a notable ($\sim 1.5$) increase in perplexity due to the coarse granularity of quantization. 

\begin{figure}[h]
  \centering
  \includegraphics[width=0.7\linewidth]{sections/figures/LM7_FP7.pdf}
  \caption{\small Quantization levels and the corresponding quantization MSE of Floating Point (left) vs Lloyd-Max (right) Quantizers for a layer of weights in the GPT3-126M model.}
  \label{fig:lm_quant}
\end{figure}

\begin{table}[h]\scriptsize
\begin{center}
\caption{\label{tab:FP7_vs_LM7} \small Comparing perplexity (lower is better) achieved by floating point quantizers and Lloyd-Max quantizers on a GPT3-126M model for the Wikitext-103 dataset.}
\begin{tabular}{c|cc|c}
\hline
 \multirow{2}{*}{\textbf{Bitwidth}} & \multicolumn{2}{|c|}{\textbf{Floating-Point Quantizer}} & \textbf{Lloyd-Max Quantizer} \\
 & Best Format & Wikitext-103 Perplexity & Wikitext-103 Perplexity \\
\hline
7 & E3M3 & 18.32 & 18.27 \\
6 & E3M2 & 19.07 & 18.51 \\
5 & E4M0 & 43.89 & 19.71 \\
\hline
\end{tabular}
\end{center}
\end{table}

\subsection{Proof of Local Optimality of LO-BCQ}
\label{subsec:lobcq_opt_proof}
For a given block $\bm{b}_j$, the quantization MSE during LO-BCQ can be empirically evaluated as $\frac{1}{L_b}\lVert \bm{b}_j- \bm{\hat{b}}_j\rVert^2_2$ where $\bm{\hat{b}}_j$ is computed from equation (\ref{eq:clustered_quantization_definition}) as $C_{f(\bm{b}_j)}(\bm{b}_j)$. Further, for a given block cluster $\mathcal{B}_i$, we compute the quantization MSE as $\frac{1}{|\mathcal{B}_{i}|}\sum_{\bm{b} \in \mathcal{B}_{i}} \frac{1}{L_b}\lVert \bm{b}- C_i^{(n)}(\bm{b})\rVert^2_2$. Therefore, at the end of iteration $n$, we evaluate the overall quantization MSE $J^{(n)}$ for a given operand $\bm{X}$ composed of $N_c$ block clusters as:
\begin{align*}
    \label{eq:mse_iter_n}
    J^{(n)} = \frac{1}{N_c} \sum_{i=1}^{N_c} \frac{1}{|\mathcal{B}_{i}^{(n)}|}\sum_{\bm{v} \in \mathcal{B}_{i}^{(n)}} \frac{1}{L_b}\lVert \bm{b}- B_i^{(n)}(\bm{b})\rVert^2_2
\end{align*}

At the end of iteration $n$, the codebooks are updated from $\mathcal{C}^{(n-1)}$ to $\mathcal{C}^{(n)}$. However, the mapping of a given vector $\bm{b}_j$ to quantizers $\mathcal{C}^{(n)}$ remains as  $f^{(n)}(\bm{b}_j)$. At the next iteration, during the vector clustering step, $f^{(n+1)}(\bm{b}_j)$ finds new mapping of $\bm{b}_j$ to updated codebooks $\mathcal{C}^{(n)}$ such that the quantization MSE over the candidate codebooks is minimized. Therefore, we obtain the following result for $\bm{b}_j$:
\begin{align*}
\frac{1}{L_b}\lVert \bm{b}_j - C_{f^{(n+1)}(\bm{b}_j)}^{(n)}(\bm{b}_j)\rVert^2_2 \le \frac{1}{L_b}\lVert \bm{b}_j - C_{f^{(n)}(\bm{b}_j)}^{(n)}(\bm{b}_j)\rVert^2_2
\end{align*}

That is, quantizing $\bm{b}_j$ at the end of the block clustering step of iteration $n+1$ results in lower quantization MSE compared to quantizing at the end of iteration $n$. Since this is true for all $\bm{b} \in \bm{X}$, we assert the following:
\begin{equation}
\begin{split}
\label{eq:mse_ineq_1}
    \tilde{J}^{(n+1)} &= \frac{1}{N_c} \sum_{i=1}^{N_c} \frac{1}{|\mathcal{B}_{i}^{(n+1)}|}\sum_{\bm{b} \in \mathcal{B}_{i}^{(n+1)}} \frac{1}{L_b}\lVert \bm{b} - C_i^{(n)}(b)\rVert^2_2 \le J^{(n)}
\end{split}
\end{equation}
where $\tilde{J}^{(n+1)}$ is the the quantization MSE after the vector clustering step at iteration $n+1$.

Next, during the codebook update step (\ref{eq:quantizers_update}) at iteration $n+1$, the per-cluster codebooks $\mathcal{C}^{(n)}$ are updated to $\mathcal{C}^{(n+1)}$ by invoking the Lloyd-Max algorithm \citep{Lloyd}. We know that for any given value distribution, the Lloyd-Max algorithm minimizes the quantization MSE. Therefore, for a given vector cluster $\mathcal{B}_i$ we obtain the following result:

\begin{equation}
    \frac{1}{|\mathcal{B}_{i}^{(n+1)}|}\sum_{\bm{b} \in \mathcal{B}_{i}^{(n+1)}} \frac{1}{L_b}\lVert \bm{b}- C_i^{(n+1)}(\bm{b})\rVert^2_2 \le \frac{1}{|\mathcal{B}_{i}^{(n+1)}|}\sum_{\bm{b} \in \mathcal{B}_{i}^{(n+1)}} \frac{1}{L_b}\lVert \bm{b}- C_i^{(n)}(\bm{b})\rVert^2_2
\end{equation}

The above equation states that quantizing the given block cluster $\mathcal{B}_i$ after updating the associated codebook from $C_i^{(n)}$ to $C_i^{(n+1)}$ results in lower quantization MSE. Since this is true for all the block clusters, we derive the following result: 
\begin{equation}
\begin{split}
\label{eq:mse_ineq_2}
     J^{(n+1)} &= \frac{1}{N_c} \sum_{i=1}^{N_c} \frac{1}{|\mathcal{B}_{i}^{(n+1)}|}\sum_{\bm{b} \in \mathcal{B}_{i}^{(n+1)}} \frac{1}{L_b}\lVert \bm{b}- C_i^{(n+1)}(\bm{b})\rVert^2_2  \le \tilde{J}^{(n+1)}   
\end{split}
\end{equation}

Following (\ref{eq:mse_ineq_1}) and (\ref{eq:mse_ineq_2}), we find that the quantization MSE is non-increasing for each iteration, that is, $J^{(1)} \ge J^{(2)} \ge J^{(3)} \ge \ldots \ge J^{(M)}$ where $M$ is the maximum number of iterations. 
%Therefore, we can say that if the algorithm converges, then it must be that it has converged to a local minimum. 
\hfill $\blacksquare$


\begin{figure}
    \begin{center}
    \includegraphics[width=0.5\textwidth]{sections//figures/mse_vs_iter.pdf}
    \end{center}
    \caption{\small NMSE vs iterations during LO-BCQ compared to other block quantization proposals}
    \label{fig:nmse_vs_iter}
\end{figure}

Figure \ref{fig:nmse_vs_iter} shows the empirical convergence of LO-BCQ across several block lengths and number of codebooks. Also, the MSE achieved by LO-BCQ is compared to baselines such as MXFP and VSQ. As shown, LO-BCQ converges to a lower MSE than the baselines. Further, we achieve better convergence for larger number of codebooks ($N_c$) and for a smaller block length ($L_b$), both of which increase the bitwidth of BCQ (see Eq \ref{eq:bitwidth_bcq}).


\subsection{Additional Accuracy Results}
%Table \ref{tab:lobcq_config} lists the various LOBCQ configurations and their corresponding bitwidths.
\begin{table}
\setlength{\tabcolsep}{4.75pt}
\begin{center}
\caption{\label{tab:lobcq_config} Various LO-BCQ configurations and their bitwidths.}
\begin{tabular}{|c||c|c|c|c||c|c||c|} 
\hline
 & \multicolumn{4}{|c||}{$L_b=8$} & \multicolumn{2}{|c||}{$L_b=4$} & $L_b=2$ \\
 \hline
 \backslashbox{$L_A$\kern-1em}{\kern-1em$N_c$} & 2 & 4 & 8 & 16 & 2 & 4 & 2 \\
 \hline
 64 & 4.25 & 4.375 & 4.5 & 4.625 & 4.375 & 4.625 & 4.625\\
 \hline
 32 & 4.375 & 4.5 & 4.625& 4.75 & 4.5 & 4.75 & 4.75 \\
 \hline
 16 & 4.625 & 4.75& 4.875 & 5 & 4.75 & 5 & 5 \\
 \hline
\end{tabular}
\end{center}
\end{table}

%\subsection{Perplexity achieved by various LO-BCQ configurations on Wikitext-103 dataset}

\begin{table} \centering
\begin{tabular}{|c||c|c|c|c||c|c||c|} 
\hline
 $L_b \rightarrow$& \multicolumn{4}{c||}{8} & \multicolumn{2}{c||}{4} & 2\\
 \hline
 \backslashbox{$L_A$\kern-1em}{\kern-1em$N_c$} & 2 & 4 & 8 & 16 & 2 & 4 & 2  \\
 %$N_c \rightarrow$ & 2 & 4 & 8 & 16 & 2 & 4 & 2 \\
 \hline
 \hline
 \multicolumn{8}{c}{GPT3-1.3B (FP32 PPL = 9.98)} \\ 
 \hline
 \hline
 64 & 10.40 & 10.23 & 10.17 & 10.15 &  10.28 & 10.18 & 10.19 \\
 \hline
 32 & 10.25 & 10.20 & 10.15 & 10.12 &  10.23 & 10.17 & 10.17 \\
 \hline
 16 & 10.22 & 10.16 & 10.10 & 10.09 &  10.21 & 10.14 & 10.16 \\
 \hline
  \hline
 \multicolumn{8}{c}{GPT3-8B (FP32 PPL = 7.38)} \\ 
 \hline
 \hline
 64 & 7.61 & 7.52 & 7.48 &  7.47 &  7.55 &  7.49 & 7.50 \\
 \hline
 32 & 7.52 & 7.50 & 7.46 &  7.45 &  7.52 &  7.48 & 7.48  \\
 \hline
 16 & 7.51 & 7.48 & 7.44 &  7.44 &  7.51 &  7.49 & 7.47  \\
 \hline
\end{tabular}
\caption{\label{tab:ppl_gpt3_abalation} Wikitext-103 perplexity across GPT3-1.3B and 8B models.}
\end{table}

\begin{table} \centering
\begin{tabular}{|c||c|c|c|c||} 
\hline
 $L_b \rightarrow$& \multicolumn{4}{c||}{8}\\
 \hline
 \backslashbox{$L_A$\kern-1em}{\kern-1em$N_c$} & 2 & 4 & 8 & 16 \\
 %$N_c \rightarrow$ & 2 & 4 & 8 & 16 & 2 & 4 & 2 \\
 \hline
 \hline
 \multicolumn{5}{|c|}{Llama2-7B (FP32 PPL = 5.06)} \\ 
 \hline
 \hline
 64 & 5.31 & 5.26 & 5.19 & 5.18  \\
 \hline
 32 & 5.23 & 5.25 & 5.18 & 5.15  \\
 \hline
 16 & 5.23 & 5.19 & 5.16 & 5.14  \\
 \hline
 \multicolumn{5}{|c|}{Nemotron4-15B (FP32 PPL = 5.87)} \\ 
 \hline
 \hline
 64  & 6.3 & 6.20 & 6.13 & 6.08  \\
 \hline
 32  & 6.24 & 6.12 & 6.07 & 6.03  \\
 \hline
 16  & 6.12 & 6.14 & 6.04 & 6.02  \\
 \hline
 \multicolumn{5}{|c|}{Nemotron4-340B (FP32 PPL = 3.48)} \\ 
 \hline
 \hline
 64 & 3.67 & 3.62 & 3.60 & 3.59 \\
 \hline
 32 & 3.63 & 3.61 & 3.59 & 3.56 \\
 \hline
 16 & 3.61 & 3.58 & 3.57 & 3.55 \\
 \hline
\end{tabular}
\caption{\label{tab:ppl_llama7B_nemo15B} Wikitext-103 perplexity compared to FP32 baseline in Llama2-7B and Nemotron4-15B, 340B models}
\end{table}

%\subsection{Perplexity achieved by various LO-BCQ configurations on MMLU dataset}


\begin{table} \centering
\begin{tabular}{|c||c|c|c|c||c|c|c|c|} 
\hline
 $L_b \rightarrow$& \multicolumn{4}{c||}{8} & \multicolumn{4}{c||}{8}\\
 \hline
 \backslashbox{$L_A$\kern-1em}{\kern-1em$N_c$} & 2 & 4 & 8 & 16 & 2 & 4 & 8 & 16  \\
 %$N_c \rightarrow$ & 2 & 4 & 8 & 16 & 2 & 4 & 2 \\
 \hline
 \hline
 \multicolumn{5}{|c|}{Llama2-7B (FP32 Accuracy = 45.8\%)} & \multicolumn{4}{|c|}{Llama2-70B (FP32 Accuracy = 69.12\%)} \\ 
 \hline
 \hline
 64 & 43.9 & 43.4 & 43.9 & 44.9 & 68.07 & 68.27 & 68.17 & 68.75 \\
 \hline
 32 & 44.5 & 43.8 & 44.9 & 44.5 & 68.37 & 68.51 & 68.35 & 68.27  \\
 \hline
 16 & 43.9 & 42.7 & 44.9 & 45 & 68.12 & 68.77 & 68.31 & 68.59  \\
 \hline
 \hline
 \multicolumn{5}{|c|}{GPT3-22B (FP32 Accuracy = 38.75\%)} & \multicolumn{4}{|c|}{Nemotron4-15B (FP32 Accuracy = 64.3\%)} \\ 
 \hline
 \hline
 64 & 36.71 & 38.85 & 38.13 & 38.92 & 63.17 & 62.36 & 63.72 & 64.09 \\
 \hline
 32 & 37.95 & 38.69 & 39.45 & 38.34 & 64.05 & 62.30 & 63.8 & 64.33  \\
 \hline
 16 & 38.88 & 38.80 & 38.31 & 38.92 & 63.22 & 63.51 & 63.93 & 64.43  \\
 \hline
\end{tabular}
\caption{\label{tab:mmlu_abalation} Accuracy on MMLU dataset across GPT3-22B, Llama2-7B, 70B and Nemotron4-15B models.}
\end{table}


%\subsection{Perplexity achieved by various LO-BCQ configurations on LM evaluation harness}

\begin{table} \centering
\begin{tabular}{|c||c|c|c|c||c|c|c|c|} 
\hline
 $L_b \rightarrow$& \multicolumn{4}{c||}{8} & \multicolumn{4}{c||}{8}\\
 \hline
 \backslashbox{$L_A$\kern-1em}{\kern-1em$N_c$} & 2 & 4 & 8 & 16 & 2 & 4 & 8 & 16  \\
 %$N_c \rightarrow$ & 2 & 4 & 8 & 16 & 2 & 4 & 2 \\
 \hline
 \hline
 \multicolumn{5}{|c|}{Race (FP32 Accuracy = 37.51\%)} & \multicolumn{4}{|c|}{Boolq (FP32 Accuracy = 64.62\%)} \\ 
 \hline
 \hline
 64 & 36.94 & 37.13 & 36.27 & 37.13 & 63.73 & 62.26 & 63.49 & 63.36 \\
 \hline
 32 & 37.03 & 36.36 & 36.08 & 37.03 & 62.54 & 63.51 & 63.49 & 63.55  \\
 \hline
 16 & 37.03 & 37.03 & 36.46 & 37.03 & 61.1 & 63.79 & 63.58 & 63.33  \\
 \hline
 \hline
 \multicolumn{5}{|c|}{Winogrande (FP32 Accuracy = 58.01\%)} & \multicolumn{4}{|c|}{Piqa (FP32 Accuracy = 74.21\%)} \\ 
 \hline
 \hline
 64 & 58.17 & 57.22 & 57.85 & 58.33 & 73.01 & 73.07 & 73.07 & 72.80 \\
 \hline
 32 & 59.12 & 58.09 & 57.85 & 58.41 & 73.01 & 73.94 & 72.74 & 73.18  \\
 \hline
 16 & 57.93 & 58.88 & 57.93 & 58.56 & 73.94 & 72.80 & 73.01 & 73.94  \\
 \hline
\end{tabular}
\caption{\label{tab:mmlu_abalation} Accuracy on LM evaluation harness tasks on GPT3-1.3B model.}
\end{table}

\begin{table} \centering
\begin{tabular}{|c||c|c|c|c||c|c|c|c|} 
\hline
 $L_b \rightarrow$& \multicolumn{4}{c||}{8} & \multicolumn{4}{c||}{8}\\
 \hline
 \backslashbox{$L_A$\kern-1em}{\kern-1em$N_c$} & 2 & 4 & 8 & 16 & 2 & 4 & 8 & 16  \\
 %$N_c \rightarrow$ & 2 & 4 & 8 & 16 & 2 & 4 & 2 \\
 \hline
 \hline
 \multicolumn{5}{|c|}{Race (FP32 Accuracy = 41.34\%)} & \multicolumn{4}{|c|}{Boolq (FP32 Accuracy = 68.32\%)} \\ 
 \hline
 \hline
 64 & 40.48 & 40.10 & 39.43 & 39.90 & 69.20 & 68.41 & 69.45 & 68.56 \\
 \hline
 32 & 39.52 & 39.52 & 40.77 & 39.62 & 68.32 & 67.43 & 68.17 & 69.30  \\
 \hline
 16 & 39.81 & 39.71 & 39.90 & 40.38 & 68.10 & 66.33 & 69.51 & 69.42  \\
 \hline
 \hline
 \multicolumn{5}{|c|}{Winogrande (FP32 Accuracy = 67.88\%)} & \multicolumn{4}{|c|}{Piqa (FP32 Accuracy = 78.78\%)} \\ 
 \hline
 \hline
 64 & 66.85 & 66.61 & 67.72 & 67.88 & 77.31 & 77.42 & 77.75 & 77.64 \\
 \hline
 32 & 67.25 & 67.72 & 67.72 & 67.00 & 77.31 & 77.04 & 77.80 & 77.37  \\
 \hline
 16 & 68.11 & 68.90 & 67.88 & 67.48 & 77.37 & 78.13 & 78.13 & 77.69  \\
 \hline
\end{tabular}
\caption{\label{tab:mmlu_abalation} Accuracy on LM evaluation harness tasks on GPT3-8B model.}
\end{table}

\begin{table} \centering
\begin{tabular}{|c||c|c|c|c||c|c|c|c|} 
\hline
 $L_b \rightarrow$& \multicolumn{4}{c||}{8} & \multicolumn{4}{c||}{8}\\
 \hline
 \backslashbox{$L_A$\kern-1em}{\kern-1em$N_c$} & 2 & 4 & 8 & 16 & 2 & 4 & 8 & 16  \\
 %$N_c \rightarrow$ & 2 & 4 & 8 & 16 & 2 & 4 & 2 \\
 \hline
 \hline
 \multicolumn{5}{|c|}{Race (FP32 Accuracy = 40.67\%)} & \multicolumn{4}{|c|}{Boolq (FP32 Accuracy = 76.54\%)} \\ 
 \hline
 \hline
 64 & 40.48 & 40.10 & 39.43 & 39.90 & 75.41 & 75.11 & 77.09 & 75.66 \\
 \hline
 32 & 39.52 & 39.52 & 40.77 & 39.62 & 76.02 & 76.02 & 75.96 & 75.35  \\
 \hline
 16 & 39.81 & 39.71 & 39.90 & 40.38 & 75.05 & 73.82 & 75.72 & 76.09  \\
 \hline
 \hline
 \multicolumn{5}{|c|}{Winogrande (FP32 Accuracy = 70.64\%)} & \multicolumn{4}{|c|}{Piqa (FP32 Accuracy = 79.16\%)} \\ 
 \hline
 \hline
 64 & 69.14 & 70.17 & 70.17 & 70.56 & 78.24 & 79.00 & 78.62 & 78.73 \\
 \hline
 32 & 70.96 & 69.69 & 71.27 & 69.30 & 78.56 & 79.49 & 79.16 & 78.89  \\
 \hline
 16 & 71.03 & 69.53 & 69.69 & 70.40 & 78.13 & 79.16 & 79.00 & 79.00  \\
 \hline
\end{tabular}
\caption{\label{tab:mmlu_abalation} Accuracy on LM evaluation harness tasks on GPT3-22B model.}
\end{table}

\begin{table} \centering
\begin{tabular}{|c||c|c|c|c||c|c|c|c|} 
\hline
 $L_b \rightarrow$& \multicolumn{4}{c||}{8} & \multicolumn{4}{c||}{8}\\
 \hline
 \backslashbox{$L_A$\kern-1em}{\kern-1em$N_c$} & 2 & 4 & 8 & 16 & 2 & 4 & 8 & 16  \\
 %$N_c \rightarrow$ & 2 & 4 & 8 & 16 & 2 & 4 & 2 \\
 \hline
 \hline
 \multicolumn{5}{|c|}{Race (FP32 Accuracy = 44.4\%)} & \multicolumn{4}{|c|}{Boolq (FP32 Accuracy = 79.29\%)} \\ 
 \hline
 \hline
 64 & 42.49 & 42.51 & 42.58 & 43.45 & 77.58 & 77.37 & 77.43 & 78.1 \\
 \hline
 32 & 43.35 & 42.49 & 43.64 & 43.73 & 77.86 & 75.32 & 77.28 & 77.86  \\
 \hline
 16 & 44.21 & 44.21 & 43.64 & 42.97 & 78.65 & 77 & 76.94 & 77.98  \\
 \hline
 \hline
 \multicolumn{5}{|c|}{Winogrande (FP32 Accuracy = 69.38\%)} & \multicolumn{4}{|c|}{Piqa (FP32 Accuracy = 78.07\%)} \\ 
 \hline
 \hline
 64 & 68.9 & 68.43 & 69.77 & 68.19 & 77.09 & 76.82 & 77.09 & 77.86 \\
 \hline
 32 & 69.38 & 68.51 & 68.82 & 68.90 & 78.07 & 76.71 & 78.07 & 77.86  \\
 \hline
 16 & 69.53 & 67.09 & 69.38 & 68.90 & 77.37 & 77.8 & 77.91 & 77.69  \\
 \hline
\end{tabular}
\caption{\label{tab:mmlu_abalation} Accuracy on LM evaluation harness tasks on Llama2-7B model.}
\end{table}

\begin{table} \centering
\begin{tabular}{|c||c|c|c|c||c|c|c|c|} 
\hline
 $L_b \rightarrow$& \multicolumn{4}{c||}{8} & \multicolumn{4}{c||}{8}\\
 \hline
 \backslashbox{$L_A$\kern-1em}{\kern-1em$N_c$} & 2 & 4 & 8 & 16 & 2 & 4 & 8 & 16  \\
 %$N_c \rightarrow$ & 2 & 4 & 8 & 16 & 2 & 4 & 2 \\
 \hline
 \hline
 \multicolumn{5}{|c|}{Race (FP32 Accuracy = 48.8\%)} & \multicolumn{4}{|c|}{Boolq (FP32 Accuracy = 85.23\%)} \\ 
 \hline
 \hline
 64 & 49.00 & 49.00 & 49.28 & 48.71 & 82.82 & 84.28 & 84.03 & 84.25 \\
 \hline
 32 & 49.57 & 48.52 & 48.33 & 49.28 & 83.85 & 84.46 & 84.31 & 84.93  \\
 \hline
 16 & 49.85 & 49.09 & 49.28 & 48.99 & 85.11 & 84.46 & 84.61 & 83.94  \\
 \hline
 \hline
 \multicolumn{5}{|c|}{Winogrande (FP32 Accuracy = 79.95\%)} & \multicolumn{4}{|c|}{Piqa (FP32 Accuracy = 81.56\%)} \\ 
 \hline
 \hline
 64 & 78.77 & 78.45 & 78.37 & 79.16 & 81.45 & 80.69 & 81.45 & 81.5 \\
 \hline
 32 & 78.45 & 79.01 & 78.69 & 80.66 & 81.56 & 80.58 & 81.18 & 81.34  \\
 \hline
 16 & 79.95 & 79.56 & 79.79 & 79.72 & 81.28 & 81.66 & 81.28 & 80.96  \\
 \hline
\end{tabular}
\caption{\label{tab:mmlu_abalation} Accuracy on LM evaluation harness tasks on Llama2-70B model.}
\end{table}

%\section{MSE Studies}
%\textcolor{red}{TODO}


\subsection{Number Formats and Quantization Method}
\label{subsec:numFormats_quantMethod}
\subsubsection{Integer Format}
An $n$-bit signed integer (INT) is typically represented with a 2s-complement format \citep{yao2022zeroquant,xiao2023smoothquant,dai2021vsq}, where the most significant bit denotes the sign.

\subsubsection{Floating Point Format}
An $n$-bit signed floating point (FP) number $x$ comprises of a 1-bit sign ($x_{\mathrm{sign}}$), $B_m$-bit mantissa ($x_{\mathrm{mant}}$) and $B_e$-bit exponent ($x_{\mathrm{exp}}$) such that $B_m+B_e=n-1$. The associated constant exponent bias ($E_{\mathrm{bias}}$) is computed as $(2^{{B_e}-1}-1)$. We denote this format as $E_{B_e}M_{B_m}$.  

\subsubsection{Quantization Scheme}
\label{subsec:quant_method}
A quantization scheme dictates how a given unquantized tensor is converted to its quantized representation. We consider FP formats for the purpose of illustration. Given an unquantized tensor $\bm{X}$ and an FP format $E_{B_e}M_{B_m}$, we first, we compute the quantization scale factor $s_X$ that maps the maximum absolute value of $\bm{X}$ to the maximum quantization level of the $E_{B_e}M_{B_m}$ format as follows:
\begin{align}
\label{eq:sf}
    s_X = \frac{\mathrm{max}(|\bm{X}|)}{\mathrm{max}(E_{B_e}M_{B_m})}
\end{align}
In the above equation, $|\cdot|$ denotes the absolute value function.

Next, we scale $\bm{X}$ by $s_X$ and quantize it to $\hat{\bm{X}}$ by rounding it to the nearest quantization level of $E_{B_e}M_{B_m}$ as:

\begin{align}
\label{eq:tensor_quant}
    \hat{\bm{X}} = \text{round-to-nearest}\left(\frac{\bm{X}}{s_X}, E_{B_e}M_{B_m}\right)
\end{align}

We perform dynamic max-scaled quantization \citep{wu2020integer}, where the scale factor $s$ for activations is dynamically computed during runtime.

\subsection{Vector Scaled Quantization}
\begin{wrapfigure}{r}{0.35\linewidth}
  \centering
  \includegraphics[width=\linewidth]{sections/figures/vsquant.jpg}
  \caption{\small Vectorwise decomposition for per-vector scaled quantization (VSQ \citep{dai2021vsq}).}
  \label{fig:vsquant}
\end{wrapfigure}
During VSQ \citep{dai2021vsq}, the operand tensors are decomposed into 1D vectors in a hardware friendly manner as shown in Figure \ref{fig:vsquant}. Since the decomposed tensors are used as operands in matrix multiplications during inference, it is beneficial to perform this decomposition along the reduction dimension of the multiplication. The vectorwise quantization is performed similar to tensorwise quantization described in Equations \ref{eq:sf} and \ref{eq:tensor_quant}, where a scale factor $s_v$ is required for each vector $\bm{v}$ that maps the maximum absolute value of that vector to the maximum quantization level. While smaller vector lengths can lead to larger accuracy gains, the associated memory and computational overheads due to the per-vector scale factors increases. To alleviate these overheads, VSQ \citep{dai2021vsq} proposed a second level quantization of the per-vector scale factors to unsigned integers, while MX \citep{rouhani2023shared} quantizes them to integer powers of 2 (denoted as $2^{INT}$).

\subsubsection{MX Format}
The MX format proposed in \citep{rouhani2023microscaling} introduces the concept of sub-block shifting. For every two scalar elements of $b$-bits each, there is a shared exponent bit. The value of this exponent bit is determined through an empirical analysis that targets minimizing quantization MSE. We note that the FP format $E_{1}M_{b}$ is strictly better than MX from an accuracy perspective since it allocates a dedicated exponent bit to each scalar as opposed to sharing it across two scalars. Therefore, we conservatively bound the accuracy of a $b+2$-bit signed MX format with that of a $E_{1}M_{b}$ format in our comparisons. For instance, we use E1M2 format as a proxy for MX4.

\begin{figure}
    \centering
    \includegraphics[width=1\linewidth]{sections//figures/BlockFormats.pdf}
    \caption{\small Comparing LO-BCQ to MX format.}
    \label{fig:block_formats}
\end{figure}

Figure \ref{fig:block_formats} compares our $4$-bit LO-BCQ block format to MX \citep{rouhani2023microscaling}. As shown, both LO-BCQ and MX decompose a given operand tensor into block arrays and each block array into blocks. Similar to MX, we find that per-block quantization ($L_b < L_A$) leads to better accuracy due to increased flexibility. While MX achieves this through per-block $1$-bit micro-scales, we associate a dedicated codebook to each block through a per-block codebook selector. Further, MX quantizes the per-block array scale-factor to E8M0 format without per-tensor scaling. In contrast during LO-BCQ, we find that per-tensor scaling combined with quantization of per-block array scale-factor to E4M3 format results in superior inference accuracy across models. 



\end{document}
