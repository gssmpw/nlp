\section{Literature Review}
\subsection{2.1 Vertiport Location Selection}
Vertiports, which are essential infrastructure for UAM, should be installed in appropriate locations considering demand, economic feasibility, operational capability, and safety. Numerous studies employing various methodologies have been conducted to address these considerations.

The multi-criteria decision-making (MCDM) method involves making optimal choices by assessing multiple evaluation criteria across different options, employing techniques such as Analytic Hierarchy Process (AHP) and Focus Group Interviews (FGI). Fadhil (2018) analyzed the factors of UAM ground infrastructure placement and suitability using WLC and AHP techniques \cite{fadhil2018gis}. Lee et al. (2023) proposed an optimal Vertiport and UAM network by calculating various topographic, population and social data of Seoul and its weights through the FGI technique for pilots \cite{lee2023}. Recent studies have adequately considered whether the area is permissible for flight. 

From a data-driven perspective, recent studies on Vertiport location assessment have utilized data from four major categories—spatial, mobility, environmental, and socio-economic—for conducting multifaceted analyses. Straubinger et al. (2021) evaluated the UAM ecosystem in different urban spatial structures using population density data \cite{straubinger2021_employment}. Bulusu et al. (2021) analyzed feasible combinations of Vertiport locations using traffic data \cite{bulusu2021}. Kotwicz Herniczek et al. (2022) evaluated the impact of airspace restrictions on the feasibility of Vertiports and UAM routes using airspace data \cite{kotwicz_airspace}. EASA (2021) analyzed the social acceptance of UAM in Europe using surveys and noise assessments \cite{easa2021study}. Other related literature is summarized in Table \ref{tab:Major Data}.


\subsection{2.2 Integration of UAM with Existing Transportation Systems}

Since the emergence of the UAM concept, researchers have proposed various approaches for integrating Vertiports into the existing transportation system. Fadhil (2018) utilized GIS to select locations for UAM among ground infrastructure locations \cite{fadhil2018gis}. Rajendran et al. (2019) proposed Vertiport locations that enhance airport accessibility by analyzing taxi data in New York City \cite{rajendran2019}. Wang et al. (2023) suggested utilizing UAM to transfer passengers from suburban areas to airports, thereby facilitating access to conventional air transportation through an on-demand transfer service \cite{wang2023}. Those studies have focused on transfers between UAM and existing urban transportation networks, and the idea of linking highway infrastructure with UAM has not been proposed.


\begin{table}[H]
    \caption{List of Data Used for Vertiport Location Assessment}\label{tab:Major Data}
        \begin{center}
        \begin{tabular}{>{\raggedright\arraybackslash}m{0.2\linewidth}>{\raggedright\arraybackslash}m{0.22\linewidth}>{\raggedright\arraybackslash}m{0.5\linewidth}} \toprule[1.5pt]
            \multicolumn{1}{c}{\textbf{Category}} & \multicolumn{1}{c}{\textbf{Data}} & \multicolumn{1}{c}{\textbf{Application}} \\ 
            \midrule[1.5pt]
            \multirow{4}{*}[-0.5ex]{Spatial Data} & Population Density \cite{straubinger2021_employment, macias2023integrated} & Potential demand evaluation of Vertiport location based on population distribution and density Information \\ \cline{2-3}
             & Employment Density \cite{straubinger2021_employment, daskilewicz2018progress} & Evaluation of access to key CBDs based on employment density information \\ \cline{2-3}
             & Land Use \cite{yedavalli_land, vascik2019development} & Selection of locations based on the status of land use, such as commercial districts, residential districts, and industrial districts \\ \cline{2-3}
             & Height of Building \cite{lee2023, vascik2020geometric} & Evaluation of safety during take-off and landing through building data in the city center \\ \hline
            \multirow{4}{*}[-0.5ex]{Mobility Data} & Traffic \cite{bulusu2021, rajendran2023capacitated} & Evaluation of accessibility by road traffic, bus and subway passengers \\ \cline{2-3}
             & Commuting Pattern \cite{rimjha_commuter, murcca2021identification} & Identification of key user groups by analyzing commuting flow over time \\ \cline{2-3}
             & Public Transport \cite{wu2021_network, lim2019selection}& Evaluation of accessibility by analyzing bus, subway line and stop locations \\ \cline{2-3}
             & Traffic Congestion \cite{bulusu2021, jin2024robust} & Evaluation of traffic congestion level and to derive the optimal location \\ \hline
            \multirow{4}{*}[-0.5ex]{Environmental Data} & Air Space \cite{kotwicz_airspace, vascik2020geometric} & Selection of information-based locations related to aviation regulations, such as flight prohibitions/restrictions/control zones \\ \cline{2-3}
             & Nature Reserve \cite{lee2023, chen2022scalable} & Evaluation of environmental constraints such as Nature Reserve and Waterfront Areas \\ \cline{2-3}
             & Noise \cite{rimjha_noise, ison2023analysis} & Evaluation of the noise level from flying UAM vehicles and Identification of areas where the noise levels is higher than criteria. \\ \cline{2-3}
             & Weather \cite{bernyk2023aerodynamic, park2022comparison} & Analysis of historical wind data for UAM takeoff and landing direction assessment \\ \cline{2-3}
             & Cost of Construction \cite{taylor2020design, rimjha2021urban} & Economic feasibility assessment based on cost information for the Vertiport construction \\ \hline
            \multirow{4}{*}[-0.5ex]{\raggedright \makecell{Socio-Economic \\ Data}} & Cost of Operation \cite{tarafdar_cost, mendonca2022advanced} & Long-term operational feasibility assessment based on cost information required to operate Vertiport \\ \cline{2-3}
             & Estimation Earnings \cite{kai2022, murcca2021identification} & Evaluation of economic feasibility by predicting revenue from Vertiport operations \\ \cline{2-3}
             & Income Level \cite{cho2022uam, daskilewicz2018progress} & Economic feasibility assessment of UAM services through income distribution information \\ \cline{2-3}
             & Social Acceptance \cite{easa2021study, ison2024consumer}& Evaluation of social impact and acceptability based on survey or noise impact assessment \\ \cline{2-3}
             & Related Regulations \cite{perperidou_regul, vascik2020geometric} & Evaluation of regulations, policies, and restrictions for selecting Vertiport locations
             \\ 
             \bottomrule[1.5pt]
        \end{tabular}
    \end{center}
\end{table}