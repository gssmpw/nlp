%%%%%%%% ICML 2025 EXAMPLE LATEX SUBMISSION FILE %%%%%%%%%%%%%%%%%

\documentclass{article}

% Recommended, but optional, packages for figures and better typesetting:
\usepackage{microtype}
\usepackage{graphicx}
\usepackage{subfigure}
\usepackage{booktabs} % for 
\usepackage{hyperref}
% \usepackage{tocappendix}
\PassOptionsToPackage{sort}{natbib}

\usepackage{algorithm}
\usepackage{algorithmic}
\usepackage[final]{neurips_2022}

% % For theorems and such
% \usepackage{amsmath}
% \usepackage{amssymb}
% \usepackage{mathtools}
% \usepackage{amsthm}
%%%%% NEW MATH DEFINITIONS %%%%%

\usepackage{amsmath,amsfonts,bm}
\usepackage{derivative}
% Mark sections of captions for referring to divisions of figures
\newcommand{\figleft}{{\em (Left)}}
\newcommand{\figcenter}{{\em (Center)}}
\newcommand{\figright}{{\em (Right)}}
\newcommand{\figtop}{{\em (Top)}}
\newcommand{\figbottom}{{\em (Bottom)}}
\newcommand{\captiona}{{\em (a)}}
\newcommand{\captionb}{{\em (b)}}
\newcommand{\captionc}{{\em (c)}}
\newcommand{\captiond}{{\em (d)}}

% Highlight a newly defined term
\newcommand{\newterm}[1]{{\bf #1}}

% Derivative d 
\newcommand{\deriv}{{\mathrm{d}}}

% Figure reference, lower-case.
\def\figref#1{figure~\ref{#1}}
% Figure reference, capital. For start of sentence
\def\Figref#1{Figure~\ref{#1}}
\def\twofigref#1#2{figures \ref{#1} and \ref{#2}}
\def\quadfigref#1#2#3#4{figures \ref{#1}, \ref{#2}, \ref{#3} and \ref{#4}}
% Section reference, lower-case.
\def\secref#1{section~\ref{#1}}
% Section reference, capital.
\def\Secref#1{Section~\ref{#1}}
% Reference to two sections.
\def\twosecrefs#1#2{sections \ref{#1} and \ref{#2}}
% Reference to three sections.
\def\secrefs#1#2#3{sections \ref{#1}, \ref{#2} and \ref{#3}}
% Reference to an equation, lower-case.
\def\eqref#1{equation~\ref{#1}}
% Reference to an equation, upper case
\def\Eqref#1{Equation~\ref{#1}}
% A raw reference to an equation---avoid using if possible
\def\plaineqref#1{\ref{#1}}
% Reference to a chapter, lower-case.
\def\chapref#1{chapter~\ref{#1}}
% Reference to an equation, upper case.
\def\Chapref#1{Chapter~\ref{#1}}
% Reference to a range of chapters
\def\rangechapref#1#2{chapters\ref{#1}--\ref{#2}}
% Reference to an algorithm, lower-case.
\def\algref#1{algorithm~\ref{#1}}
% Reference to an algorithm, upper case.
\def\Algref#1{Algorithm~\ref{#1}}
\def\twoalgref#1#2{algorithms \ref{#1} and \ref{#2}}
\def\Twoalgref#1#2{Algorithms \ref{#1} and \ref{#2}}
% Reference to a part, lower case
\def\partref#1{part~\ref{#1}}
% Reference to a part, upper case
\def\Partref#1{Part~\ref{#1}}
\def\twopartref#1#2{parts \ref{#1} and \ref{#2}}

\def\ceil#1{\lceil #1 \rceil}
\def\floor#1{\lfloor #1 \rfloor}
\def\1{\bm{1}}
\newcommand{\train}{\mathcal{D}}
\newcommand{\valid}{\mathcal{D_{\mathrm{valid}}}}
\newcommand{\test}{\mathcal{D_{\mathrm{test}}}}

\def\eps{{\epsilon}}


% Random variables
\def\reta{{\textnormal{$\eta$}}}
\def\ra{{\textnormal{a}}}
\def\rb{{\textnormal{b}}}
\def\rc{{\textnormal{c}}}
\def\rd{{\textnormal{d}}}
\def\re{{\textnormal{e}}}
\def\rf{{\textnormal{f}}}
\def\rg{{\textnormal{g}}}
\def\rh{{\textnormal{h}}}
\def\ri{{\textnormal{i}}}
\def\rj{{\textnormal{j}}}
\def\rk{{\textnormal{k}}}
\def\rl{{\textnormal{l}}}
% rm is already a command, just don't name any random variables m
\def\rn{{\textnormal{n}}}
\def\ro{{\textnormal{o}}}
\def\rp{{\textnormal{p}}}
\def\rq{{\textnormal{q}}}
\def\rr{{\textnormal{r}}}
\def\rs{{\textnormal{s}}}
\def\rt{{\textnormal{t}}}
\def\ru{{\textnormal{u}}}
\def\rv{{\textnormal{v}}}
\def\rw{{\textnormal{w}}}
\def\rx{{\textnormal{x}}}
\def\ry{{\textnormal{y}}}
\def\rz{{\textnormal{z}}}

% Random vectors
\def\rvepsilon{{\mathbf{\epsilon}}}
\def\rvphi{{\mathbf{\phi}}}
\def\rvtheta{{\mathbf{\theta}}}
\def\rva{{\mathbf{a}}}
\def\rvb{{\mathbf{b}}}
\def\rvc{{\mathbf{c}}}
\def\rvd{{\mathbf{d}}}
\def\rve{{\mathbf{e}}}
\def\rvf{{\mathbf{f}}}
\def\rvg{{\mathbf{g}}}
\def\rvh{{\mathbf{h}}}
\def\rvu{{\mathbf{i}}}
\def\rvj{{\mathbf{j}}}
\def\rvk{{\mathbf{k}}}
\def\rvl{{\mathbf{l}}}
\def\rvm{{\mathbf{m}}}
\def\rvn{{\mathbf{n}}}
\def\rvo{{\mathbf{o}}}
\def\rvp{{\mathbf{p}}}
\def\rvq{{\mathbf{q}}}
\def\rvr{{\mathbf{r}}}
\def\rvs{{\mathbf{s}}}
\def\rvt{{\mathbf{t}}}
\def\rvu{{\mathbf{u}}}
\def\rvv{{\mathbf{v}}}
\def\rvw{{\mathbf{w}}}
\def\rvx{{\mathbf{x}}}
\def\rvy{{\mathbf{y}}}
\def\rvz{{\mathbf{z}}}

% Elements of random vectors
\def\erva{{\textnormal{a}}}
\def\ervb{{\textnormal{b}}}
\def\ervc{{\textnormal{c}}}
\def\ervd{{\textnormal{d}}}
\def\erve{{\textnormal{e}}}
\def\ervf{{\textnormal{f}}}
\def\ervg{{\textnormal{g}}}
\def\ervh{{\textnormal{h}}}
\def\ervi{{\textnormal{i}}}
\def\ervj{{\textnormal{j}}}
\def\ervk{{\textnormal{k}}}
\def\ervl{{\textnormal{l}}}
\def\ervm{{\textnormal{m}}}
\def\ervn{{\textnormal{n}}}
\def\ervo{{\textnormal{o}}}
\def\ervp{{\textnormal{p}}}
\def\ervq{{\textnormal{q}}}
\def\ervr{{\textnormal{r}}}
\def\ervs{{\textnormal{s}}}
\def\ervt{{\textnormal{t}}}
\def\ervu{{\textnormal{u}}}
\def\ervv{{\textnormal{v}}}
\def\ervw{{\textnormal{w}}}
\def\ervx{{\textnormal{x}}}
\def\ervy{{\textnormal{y}}}
\def\ervz{{\textnormal{z}}}

% Random matrices
\def\rmA{{\mathbf{A}}}
\def\rmB{{\mathbf{B}}}
\def\rmC{{\mathbf{C}}}
\def\rmD{{\mathbf{D}}}
\def\rmE{{\mathbf{E}}}
\def\rmF{{\mathbf{F}}}
\def\rmG{{\mathbf{G}}}
\def\rmH{{\mathbf{H}}}
\def\rmI{{\mathbf{I}}}
\def\rmJ{{\mathbf{J}}}
\def\rmK{{\mathbf{K}}}
\def\rmL{{\mathbf{L}}}
\def\rmM{{\mathbf{M}}}
\def\rmN{{\mathbf{N}}}
\def\rmO{{\mathbf{O}}}
\def\rmP{{\mathbf{P}}}
\def\rmQ{{\mathbf{Q}}}
\def\rmR{{\mathbf{R}}}
\def\rmS{{\mathbf{S}}}
\def\rmT{{\mathbf{T}}}
\def\rmU{{\mathbf{U}}}
\def\rmV{{\mathbf{V}}}
\def\rmW{{\mathbf{W}}}
\def\rmX{{\mathbf{X}}}
\def\rmY{{\mathbf{Y}}}
\def\rmZ{{\mathbf{Z}}}

% Elements of random matrices
\def\ermA{{\textnormal{A}}}
\def\ermB{{\textnormal{B}}}
\def\ermC{{\textnormal{C}}}
\def\ermD{{\textnormal{D}}}
\def\ermE{{\textnormal{E}}}
\def\ermF{{\textnormal{F}}}
\def\ermG{{\textnormal{G}}}
\def\ermH{{\textnormal{H}}}
\def\ermI{{\textnormal{I}}}
\def\ermJ{{\textnormal{J}}}
\def\ermK{{\textnormal{K}}}
\def\ermL{{\textnormal{L}}}
\def\ermM{{\textnormal{M}}}
\def\ermN{{\textnormal{N}}}
\def\ermO{{\textnormal{O}}}
\def\ermP{{\textnormal{P}}}
\def\ermQ{{\textnormal{Q}}}
\def\ermR{{\textnormal{R}}}
\def\ermS{{\textnormal{S}}}
\def\ermT{{\textnormal{T}}}
\def\ermU{{\textnormal{U}}}
\def\ermV{{\textnormal{V}}}
\def\ermW{{\textnormal{W}}}
\def\ermX{{\textnormal{X}}}
\def\ermY{{\textnormal{Y}}}
\def\ermZ{{\textnormal{Z}}}

% Vectors
\def\vzero{{\bm{0}}}
\def\vone{{\bm{1}}}
\def\vmu{{\bm{\mu}}}
\def\vtheta{{\bm{\theta}}}
\def\vphi{{\bm{\phi}}}
\def\va{{\bm{a}}}
\def\vb{{\bm{b}}}
\def\vc{{\bm{c}}}
\def\vd{{\bm{d}}}
\def\ve{{\bm{e}}}
\def\vf{{\bm{f}}}
\def\vg{{\bm{g}}}
\def\vh{{\bm{h}}}
\def\vi{{\bm{i}}}
\def\vj{{\bm{j}}}
\def\vk{{\bm{k}}}
\def\vl{{\bm{l}}}
\def\vm{{\bm{m}}}
\def\vn{{\bm{n}}}
\def\vo{{\bm{o}}}
\def\vp{{\bm{p}}}
\def\vq{{\bm{q}}}
\def\vr{{\bm{r}}}
\def\vs{{\bm{s}}}
\def\vt{{\bm{t}}}
\def\vu{{\bm{u}}}
\def\vv{{\bm{v}}}
\def\vw{{\bm{w}}}
\def\vx{{\bm{x}}}
\def\vy{{\bm{y}}}
\def\vz{{\bm{z}}}

% Elements of vectors
\def\evalpha{{\alpha}}
\def\evbeta{{\beta}}
\def\evepsilon{{\epsilon}}
\def\evlambda{{\lambda}}
\def\evomega{{\omega}}
\def\evmu{{\mu}}
\def\evpsi{{\psi}}
\def\evsigma{{\sigma}}
\def\evtheta{{\theta}}
\def\eva{{a}}
\def\evb{{b}}
\def\evc{{c}}
\def\evd{{d}}
\def\eve{{e}}
\def\evf{{f}}
\def\evg{{g}}
\def\evh{{h}}
\def\evi{{i}}
\def\evj{{j}}
\def\evk{{k}}
\def\evl{{l}}
\def\evm{{m}}
\def\evn{{n}}
\def\evo{{o}}
\def\evp{{p}}
\def\evq{{q}}
\def\evr{{r}}
\def\evs{{s}}
\def\evt{{t}}
\def\evu{{u}}
\def\evv{{v}}
\def\evw{{w}}
\def\evx{{x}}
\def\evy{{y}}
\def\evz{{z}}

% Matrix
\def\mA{{\bm{A}}}
\def\mB{{\bm{B}}}
\def\mC{{\bm{C}}}
\def\mD{{\bm{D}}}
\def\mE{{\bm{E}}}
\def\mF{{\bm{F}}}
\def\mG{{\bm{G}}}
\def\mH{{\bm{H}}}
\def\mI{{\bm{I}}}
\def\mJ{{\bm{J}}}
\def\mK{{\bm{K}}}
\def\mL{{\bm{L}}}
\def\mM{{\bm{M}}}
\def\mN{{\bm{N}}}
\def\mO{{\bm{O}}}
\def\mP{{\bm{P}}}
\def\mQ{{\bm{Q}}}
\def\mR{{\bm{R}}}
\def\mS{{\bm{S}}}
\def\mT{{\bm{T}}}
\def\mU{{\bm{U}}}
\def\mV{{\bm{V}}}
\def\mW{{\bm{W}}}
\def\mX{{\bm{X}}}
\def\mY{{\bm{Y}}}
\def\mZ{{\bm{Z}}}
\def\mBeta{{\bm{\beta}}}
\def\mPhi{{\bm{\Phi}}}
\def\mLambda{{\bm{\Lambda}}}
\def\mSigma{{\bm{\Sigma}}}

% Tensor
\DeclareMathAlphabet{\mathsfit}{\encodingdefault}{\sfdefault}{m}{sl}
\SetMathAlphabet{\mathsfit}{bold}{\encodingdefault}{\sfdefault}{bx}{n}
\newcommand{\tens}[1]{\bm{\mathsfit{#1}}}
\def\tA{{\tens{A}}}
\def\tB{{\tens{B}}}
\def\tC{{\tens{C}}}
\def\tD{{\tens{D}}}
\def\tE{{\tens{E}}}
\def\tF{{\tens{F}}}
\def\tG{{\tens{G}}}
\def\tH{{\tens{H}}}
\def\tI{{\tens{I}}}
\def\tJ{{\tens{J}}}
\def\tK{{\tens{K}}}
\def\tL{{\tens{L}}}
\def\tM{{\tens{M}}}
\def\tN{{\tens{N}}}
\def\tO{{\tens{O}}}
\def\tP{{\tens{P}}}
\def\tQ{{\tens{Q}}}
\def\tR{{\tens{R}}}
\def\tS{{\tens{S}}}
\def\tT{{\tens{T}}}
\def\tU{{\tens{U}}}
\def\tV{{\tens{V}}}
\def\tW{{\tens{W}}}
\def\tX{{\tens{X}}}
\def\tY{{\tens{Y}}}
\def\tZ{{\tens{Z}}}


% Graph
\def\gA{{\mathcal{A}}}
\def\gB{{\mathcal{B}}}
\def\gC{{\mathcal{C}}}
\def\gD{{\mathcal{D}}}
\def\gE{{\mathcal{E}}}
\def\gF{{\mathcal{F}}}
\def\gG{{\mathcal{G}}}
\def\gH{{\mathcal{H}}}
\def\gI{{\mathcal{I}}}
\def\gJ{{\mathcal{J}}}
\def\gK{{\mathcal{K}}}
\def\gL{{\mathcal{L}}}
\def\gM{{\mathcal{M}}}
\def\gN{{\mathcal{N}}}
\def\gO{{\mathcal{O}}}
\def\gP{{\mathcal{P}}}
\def\gQ{{\mathcal{Q}}}
\def\gR{{\mathcal{R}}}
\def\gS{{\mathcal{S}}}
\def\gT{{\mathcal{T}}}
\def\gU{{\mathcal{U}}}
\def\gV{{\mathcal{V}}}
\def\gW{{\mathcal{W}}}
\def\gX{{\mathcal{X}}}
\def\gY{{\mathcal{Y}}}
\def\gZ{{\mathcal{Z}}}

% Sets
\def\sA{{\mathbb{A}}}
\def\sB{{\mathbb{B}}}
\def\sC{{\mathbb{C}}}
\def\sD{{\mathbb{D}}}
% Don't use a set called E, because this would be the same as our symbol
% for expectation.
\def\sF{{\mathbb{F}}}
\def\sG{{\mathbb{G}}}
\def\sH{{\mathbb{H}}}
\def\sI{{\mathbb{I}}}
\def\sJ{{\mathbb{J}}}
\def\sK{{\mathbb{K}}}
\def\sL{{\mathbb{L}}}
\def\sM{{\mathbb{M}}}
\def\sN{{\mathbb{N}}}
\def\sO{{\mathbb{O}}}
\def\sP{{\mathbb{P}}}
\def\sQ{{\mathbb{Q}}}
\def\sR{{\mathbb{R}}}
\def\sS{{\mathbb{S}}}
\def\sT{{\mathbb{T}}}
\def\sU{{\mathbb{U}}}
\def\sV{{\mathbb{V}}}
\def\sW{{\mathbb{W}}}
\def\sX{{\mathbb{X}}}
\def\sY{{\mathbb{Y}}}
\def\sZ{{\mathbb{Z}}}

% Entries of a matrix
\def\emLambda{{\Lambda}}
\def\emA{{A}}
\def\emB{{B}}
\def\emC{{C}}
\def\emD{{D}}
\def\emE{{E}}
\def\emF{{F}}
\def\emG{{G}}
\def\emH{{H}}
\def\emI{{I}}
\def\emJ{{J}}
\def\emK{{K}}
\def\emL{{L}}
\def\emM{{M}}
\def\emN{{N}}
\def\emO{{O}}
\def\emP{{P}}
\def\emQ{{Q}}
\def\emR{{R}}
\def\emS{{S}}
\def\emT{{T}}
\def\emU{{U}}
\def\emV{{V}}
\def\emW{{W}}
\def\emX{{X}}
\def\emY{{Y}}
\def\emZ{{Z}}
\def\emSigma{{\Sigma}}

% entries of a tensor
% Same font as tensor, without \bm wrapper
\newcommand{\etens}[1]{\mathsfit{#1}}
\def\etLambda{{\etens{\Lambda}}}
\def\etA{{\etens{A}}}
\def\etB{{\etens{B}}}
\def\etC{{\etens{C}}}
\def\etD{{\etens{D}}}
\def\etE{{\etens{E}}}
\def\etF{{\etens{F}}}
\def\etG{{\etens{G}}}
\def\etH{{\etens{H}}}
\def\etI{{\etens{I}}}
\def\etJ{{\etens{J}}}
\def\etK{{\etens{K}}}
\def\etL{{\etens{L}}}
\def\etM{{\etens{M}}}
\def\etN{{\etens{N}}}
\def\etO{{\etens{O}}}
\def\etP{{\etens{P}}}
\def\etQ{{\etens{Q}}}
\def\etR{{\etens{R}}}
\def\etS{{\etens{S}}}
\def\etT{{\etens{T}}}
\def\etU{{\etens{U}}}
\def\etV{{\etens{V}}}
\def\etW{{\etens{W}}}
\def\etX{{\etens{X}}}
\def\etY{{\etens{Y}}}
\def\etZ{{\etens{Z}}}

% The true underlying data generating distribution
\newcommand{\pdata}{p_{\rm{data}}}
\newcommand{\ptarget}{p_{\rm{target}}}
\newcommand{\pprior}{p_{\rm{prior}}}
\newcommand{\pbase}{p_{\rm{base}}}
\newcommand{\pref}{p_{\rm{ref}}}

% The empirical distribution defined by the training set
\newcommand{\ptrain}{\hat{p}_{\rm{data}}}
\newcommand{\Ptrain}{\hat{P}_{\rm{data}}}
% The model distribution
\newcommand{\pmodel}{p_{\rm{model}}}
\newcommand{\Pmodel}{P_{\rm{model}}}
\newcommand{\ptildemodel}{\tilde{p}_{\rm{model}}}
% Stochastic autoencoder distributions
\newcommand{\pencode}{p_{\rm{encoder}}}
\newcommand{\pdecode}{p_{\rm{decoder}}}
\newcommand{\precons}{p_{\rm{reconstruct}}}

\newcommand{\laplace}{\mathrm{Laplace}} % Laplace distribution

\newcommand{\E}{\mathbb{E}}
\newcommand{\Ls}{\mathcal{L}}
\newcommand{\R}{\mathbb{R}}
\newcommand{\emp}{\tilde{p}}
\newcommand{\lr}{\alpha}
\newcommand{\reg}{\lambda}
\newcommand{\rect}{\mathrm{rectifier}}
\newcommand{\softmax}{\mathrm{softmax}}
\newcommand{\sigmoid}{\sigma}
\newcommand{\softplus}{\zeta}
\newcommand{\KL}{D_{\mathrm{KL}}}
\newcommand{\Var}{\mathrm{Var}}
\newcommand{\standarderror}{\mathrm{SE}}
\newcommand{\Cov}{\mathrm{Cov}}
% Wolfram Mathworld says $L^2$ is for function spaces and $\ell^2$ is for vectors
% But then they seem to use $L^2$ for vectors throughout the site, and so does
% wikipedia.
\newcommand{\normlzero}{L^0}
\newcommand{\normlone}{L^1}
\newcommand{\normltwo}{L^2}
\newcommand{\normlp}{L^p}
\newcommand{\normmax}{L^\infty}

\newcommand{\parents}{Pa} % See usage in notation.tex. Chosen to match Daphne's book.

\DeclareMathOperator*{\argmax}{arg\,max}
\DeclareMathOperator*{\argmin}{arg\,min}

\DeclareMathOperator{\sign}{sign}
\DeclareMathOperator{\Tr}{Tr}
\let\ab\allowbreak


\usepackage{ulem}
\usepackage{pifont}

% if you use cleveref..


\theoremstyle{plain}
\newtheorem{theorem}{Theorem}[section]
\newtheorem{proposition}{Proposition}
\newtheorem{lemma}{Lemma}
\newtheorem{corollary}[theorem]{Corollary}
\theoremstyle{definition}
\newtheorem{definition}[theorem]{Definition}
\newtheorem{assumption}[theorem]{Assumption}
\theoremstyle{remark}
\newtheorem{remark}[theorem]{Remark}

\usepackage{amsmath}


\usepackage[textsize=tiny]{todonotes}



% \usepackage{bm}

\newglossaryentry{AI}{
    name=AI: Artificial Intelligence,
    description={The simulation of human intelligence}
}

\newglossaryentry{LLM}{
    name=LLM: Large Language Model,
    description={A type of artificial intelligence designed to understand and generate human-like text}
}

\newglossaryentry{API}{
    name=API: Application Programming Interface,
    description={A software interface for offering a service to other pieces of software}
}

\newglossaryentry{AST}{
    name=AST: Abstract Syntax Tree,
    description={A tree representation of the abstract syntactic structure of source code written in a programming language}
}

\newglossaryentry{ReACC}{
    name=ReACC: Retrieval-Augmented Code Completion,
    description={A framework that enhances code completion by leveraging external context from a large codebase}
}

\newglossaryentry{CMSIS}{
    name=CMSIS: Cortex Microcontroller Software Interface Standard,
    description={A hardware abstraction layer independent of vendor for the Cortex-M processor series}
}

\newglossaryentry{HAL}{
    name=HAL: Hardware Abstraction Layer,
    description={A layer of programming that allows a computer operating system to interact with a hardware device at an abstract level}
}

\newglossaryentry{RAG}{
    name=RAG: Retrieval-Augmented Generation,
    description={A process that enhances large language models by allowing them to respond to prompts using a specified set of documents}
}

\newglossaryentry{STM32F407}{
    name=STM32F407: High-\allowbreak performance Microcontroller,
    description={A microcontroller that offers the performance of the Cortex-M4 core}
}

\newglossaryentry{AURIX TC334}{
    name=AURIX TC334: 32-bit Microcontroller from Infineon,
    description={A microcontroller designed for automotive and industrial applications, featuring a 32-bit TriCore-\allowbreak architecture}
}

\newglossaryentry{LED}{
    name=LED: Light-Emitting Diode,
    description={A semiconductor light source that emits light when current flows through it}
}

\newglossaryentry{CortexM4}{
    name=Cortex-M4: 32-bit processor design from ARM,
    description={A 32-bit processor design optimized for real-time applications with low power consumption}
}

\newglossaryentry{GPIO}{
    name=GPIO: General-Purpose Input/Output Pin,
    description={A versatile pin on a microcontroller that can be configured as either an input or an output. As an \textbf{input}, it can read external signals such as button presses. As an \textbf{output}, it can control devices}
}

\newglossaryentry{Offset}{
    name=Offset: Relative distance of a specific register,
    description={The relative distance or position of a specific register or memory location within a hardware block, measured from a base address}
}

\newglossaryentry{Clock}{
    name=Clock: Synchronization signal for operations,
    description={The signal used to synchronize operations within a microcontroller or hardware system, ensuring consistent timing and execution of tasks}
}

\newglossaryentry{GPT4oMini}{
    name=GPT-4o Mini: Large Language Model from OpenAI,
    description={A compact variant of the GPT-4 language model designed for cost-efficient and versatile tasks}
}

\newglossaryentry{FAISS}{
    name=FAISS: Facebook AI Similarity Search,
    description={An open-source library for efficient similarity search and clustering of high-dimensional vectors}
}


\usepackage[capitalize,noabbrev]{cleveref}
% \newcommand{\theHalgorithm}{\arabic{algorithm}}

\renewcommand{\algorithmiccomment}[1]{\hfill\(\triangleright\){\textcolor{blue}{\textit{#1}}}}

\newcommand{\TopComment}[1]{%
    \Statex \textcolor{blue}{\textit{#1}}
}

% \crefname{proposition}{proposition}{propositions}
% \Crefname{proposition}{Proposition}{Propositions}




\title{Towards Efficient Optimizer Design for LLM via Structured Fisher Approximation with a Low-Rank Extension}

\author{%
Wenbo Gong$^*$ \\
Microsoft Research \\
\texttt{wenbogong@microsoft.com} \\
\And
Meyer Scetbon$^*$ \\
Microsoft Research \\
\texttt{t-mscetbon@microsoft.com} \\
\And
Chao Ma$^*$ \\
Microsoft Research \\
\texttt{chao.ma@microsoft.com} \\
\And
Edward Meeds \\
Microsoft Research \\
\texttt{edward.meeds@microsoft.com} \\
}


\crefname{equation}{Eq.}{Eqs.}
\Crefname{equation}{Eq.}{Eqs.}
\crefname{appendix}{App.}{Apps.}   % For lowercase \cref
\Crefname{appendix}{App.}{Apps.}   % For capitalized \Cref

\crefname{figure}{Fig.}{Figs.}     % For lowercase \cref
\Crefname{figure}{Fig.}{Figs.}     % For capitalized \Cref
\crefname{section}{Sec.}{Secs.}
\Crefname{section}{Sec.}{Secs.}









\begin{document}

\maketitle


\def\thefootnote{*}\footnotetext{These authors contributed equally to this work.}
% It is OKAY to include author information, even for blind
% submissions: the style file will automatically remove it for you
% unless you've provided the [accepted] option to the icml2025
% package.

% List of affiliations: The first argument should be a (short)
% identifier you will use later to specify author affiliations
% Academic affiliations should list Department, University, City, Region, Country
% Industry affiliations should list Company, City, Region, Country

% You can specify symbols, otherwise they are numbered in order.
% Ideally, you should not use this facility. Affiliations will be numbered
% in order of appearance and this is the preferred way.



\begin{abstract}
Designing efficient optimizers for large language models (LLMs) with low-memory requirements and fast convergence is an important and challenging problem. 
This paper makes a step towards the systematic design of such optimizers through the lens of structured \gls{fim} approximation. We show that many state-of-the-art efficient optimizers can be viewed as solutions to \gls{fim} approximation (under the Frobenius norm) with specific structural assumptions. Building on these insights, we propose two design recommendations of practical efficient optimizers for LLMs, involving the careful selection of structural assumptions to balance generality and efficiency, and enhancing memory efficiency of optimizers with general structures through a novel low-rank extension framework. We demonstrate how to use each design approach by deriving new memory-efficient optimizers: \gls{ssgd} and \gls{alice}. Experiments on LLaMA pre-training (up to 1B parameters) validate the effectiveness, showing faster and better convergence than existing memory-efficient baselines and Adam with little memory overhead. Notably, \gls{alice} achieves better than $2\times$ faster convergence over Adam, while \gls{ssgd} delivers strong performance on the 1B model with SGD-like memory.
\end{abstract}

\section{Introduction}
\section{Introduction}
\label{sec:introduction}
The business processes of organizations are experiencing ever-increasing complexity due to the large amount of data, high number of users, and high-tech devices involved \cite{martin2021pmopportunitieschallenges, beerepoot2023biggestbpmproblems}. This complexity may cause business processes to deviate from normal control flow due to unforeseen and disruptive anomalies \cite{adams2023proceddsriftdetection}. These control-flow anomalies manifest as unknown, skipped, and wrongly-ordered activities in the traces of event logs monitored from the execution of business processes \cite{ko2023adsystematicreview}. For the sake of clarity, let us consider an illustrative example of such anomalies. Figure \ref{FP_ANOMALIES} shows a so-called event log footprint, which captures the control flow relations of four activities of a hypothetical event log. In particular, this footprint captures the control-flow relations between activities \texttt{a}, \texttt{b}, \texttt{c} and \texttt{d}. These are the causal ($\rightarrow$) relation, concurrent ($\parallel$) relation, and other ($\#$) relations such as exclusivity or non-local dependency \cite{aalst2022pmhandbook}. In addition, on the right are six traces, of which five exhibit skipped, wrongly-ordered and unknown control-flow anomalies. For example, $\langle$\texttt{a b d}$\rangle$ has a skipped activity, which is \texttt{c}. Because of this skipped activity, the control-flow relation \texttt{b}$\,\#\,$\texttt{d} is violated, since \texttt{d} directly follows \texttt{b} in the anomalous trace.
\begin{figure}[!t]
\centering
\includegraphics[width=0.9\columnwidth]{images/FP_ANOMALIES.png}
\caption{An example event log footprint with six traces, of which five exhibit control-flow anomalies.}
\label{FP_ANOMALIES}
\end{figure}

\subsection{Control-flow anomaly detection}
Control-flow anomaly detection techniques aim to characterize the normal control flow from event logs and verify whether these deviations occur in new event logs \cite{ko2023adsystematicreview}. To develop control-flow anomaly detection techniques, \revision{process mining} has seen widespread adoption owing to process discovery and \revision{conformance checking}. On the one hand, process discovery is a set of algorithms that encode control-flow relations as a set of model elements and constraints according to a given modeling formalism \cite{aalst2022pmhandbook}; hereafter, we refer to the Petri net, a widespread modeling formalism. On the other hand, \revision{conformance checking} is an explainable set of algorithms that allows linking any deviations with the reference Petri net and providing the fitness measure, namely a measure of how much the Petri net fits the new event log \cite{aalst2022pmhandbook}. Many control-flow anomaly detection techniques based on \revision{conformance checking} (hereafter, \revision{conformance checking}-based techniques) use the fitness measure to determine whether an event log is anomalous \cite{bezerra2009pmad, bezerra2013adlogspais, myers2018icsadpm, pecchia2020applicationfailuresanalysispm}. 

The scientific literature also includes many \revision{conformance checking}-independent techniques for control-flow anomaly detection that combine specific types of trace encodings with machine/deep learning \cite{ko2023adsystematicreview, tavares2023pmtraceencoding}. Whereas these techniques are very effective, their explainability is challenging due to both the type of trace encoding employed and the machine/deep learning model used \cite{rawal2022trustworthyaiadvances,li2023explainablead}. Hence, in the following, we focus on the shortcomings of \revision{conformance checking}-based techniques to investigate whether it is possible to support the development of competitive control-flow anomaly detection techniques while maintaining the explainable nature of \revision{conformance checking}.
\begin{figure}[!t]
\centering
\includegraphics[width=\columnwidth]{images/HIGH_LEVEL_VIEW.png}
\caption{A high-level view of the proposed framework for combining \revision{process mining}-based feature extraction with dimensionality reduction for control-flow anomaly detection.}
\label{HIGH_LEVEL_VIEW}
\end{figure}

\subsection{Shortcomings of \revision{conformance checking}-based techniques}
Unfortunately, the detection effectiveness of \revision{conformance checking}-based techniques is affected by noisy data and low-quality Petri nets, which may be due to human errors in the modeling process or representational bias of process discovery algorithms \cite{bezerra2013adlogspais, pecchia2020applicationfailuresanalysispm, aalst2016pm}. Specifically, on the one hand, noisy data may introduce infrequent and deceptive control-flow relations that may result in inconsistent fitness measures, whereas, on the other hand, checking event logs against a low-quality Petri net could lead to an unreliable distribution of fitness measures. Nonetheless, such Petri nets can still be used as references to obtain insightful information for \revision{process mining}-based feature extraction, supporting the development of competitive and explainable \revision{conformance checking}-based techniques for control-flow anomaly detection despite the problems above. For example, a few works outline that token-based \revision{conformance checking} can be used for \revision{process mining}-based feature extraction to build tabular data and develop effective \revision{conformance checking}-based techniques for control-flow anomaly detection \cite{singh2022lapmsh, debenedictis2023dtadiiot}. However, to the best of our knowledge, the scientific literature lacks a structured proposal for \revision{process mining}-based feature extraction using the state-of-the-art \revision{conformance checking} variant, namely alignment-based \revision{conformance checking}.

\subsection{Contributions}
We propose a novel \revision{process mining}-based feature extraction approach with alignment-based \revision{conformance checking}. This variant aligns the deviating control flow with a reference Petri net; the resulting alignment can be inspected to extract additional statistics such as the number of times a given activity caused mismatches \cite{aalst2022pmhandbook}. We integrate this approach into a flexible and explainable framework for developing techniques for control-flow anomaly detection. The framework combines \revision{process mining}-based feature extraction and dimensionality reduction to handle high-dimensional feature sets, achieve detection effectiveness, and support explainability. Notably, in addition to our proposed \revision{process mining}-based feature extraction approach, the framework allows employing other approaches, enabling a fair comparison of multiple \revision{conformance checking}-based and \revision{conformance checking}-independent techniques for control-flow anomaly detection. Figure \ref{HIGH_LEVEL_VIEW} shows a high-level view of the framework. Business processes are monitored, and event logs obtained from the database of information systems. Subsequently, \revision{process mining}-based feature extraction is applied to these event logs and tabular data input to dimensionality reduction to identify control-flow anomalies. We apply several \revision{conformance checking}-based and \revision{conformance checking}-independent framework techniques to publicly available datasets, simulated data of a case study from railways, and real-world data of a case study from healthcare. We show that the framework techniques implementing our approach outperform the baseline \revision{conformance checking}-based techniques while maintaining the explainable nature of \revision{conformance checking}.

In summary, the contributions of this paper are as follows.
\begin{itemize}
    \item{
        A novel \revision{process mining}-based feature extraction approach to support the development of competitive and explainable \revision{conformance checking}-based techniques for control-flow anomaly detection.
    }
    \item{
        A flexible and explainable framework for developing techniques for control-flow anomaly detection using \revision{process mining}-based feature extraction and dimensionality reduction.
    }
    \item{
        Application to synthetic and real-world datasets of several \revision{conformance checking}-based and \revision{conformance checking}-independent framework techniques, evaluating their detection effectiveness and explainability.
    }
\end{itemize}

The rest of the paper is organized as follows.
\begin{itemize}
    \item Section \ref{sec:related_work} reviews the existing techniques for control-flow anomaly detection, categorizing them into \revision{conformance checking}-based and \revision{conformance checking}-independent techniques.
    \item Section \ref{sec:abccfe} provides the preliminaries of \revision{process mining} to establish the notation used throughout the paper, and delves into the details of the proposed \revision{process mining}-based feature extraction approach with alignment-based \revision{conformance checking}.
    \item Section \ref{sec:framework} describes the framework for developing \revision{conformance checking}-based and \revision{conformance checking}-independent techniques for control-flow anomaly detection that combine \revision{process mining}-based feature extraction and dimensionality reduction.
    \item Section \ref{sec:evaluation} presents the experiments conducted with multiple framework and baseline techniques using data from publicly available datasets and case studies.
    \item Section \ref{sec:conclusions} draws the conclusions and presents future work.
\end{itemize}
\section{Preliminaries}
\label{sec: preliminary}
\section{Summary of Mathematical Notations}
\label{sec:notations}

We summarize the main mathematical notations used in the main paper in Table \ref{table:notations}.

\begin{table*}[h]
	\centering
	\caption{Summary of main mathematical notations.}
	%\resizebox{1\columnwidth}{!}{
		\begin{tabular}{c|l}
			\toprule
			Notation  & Description  \\
			\hline
            $ \mathcal{G} $ & a document graph \\
			$ \mathcal{D} $ & a corpus of documents, $ \mathcal{D}=\{d_i\}_{i=1}^{N} $ \\
			$ N $ & number of documents in the corpus, $ N=|\mathcal{D}| $ \\
			$ d_i $ & document $ i $ containing a sequence of words, $ d_i=\{w_{i,v}\}_{v=1}^{|d_i|}\subset\mathcal{V} $ \\
			$ \mathcal{V} $ & vocabulary \\
			$ |d_i| $ & number of words in document $ i $ \\
			$ \mathcal{E} $ & a set of graph edges connecting documents, $ \mathcal{E}=\{e_{ij}\} $ \\
			$ \mathcal{N}(i) $ & the neighbor set of document $ i $ \\
            $ \Bbb H^{n,K} $ & Hyperboloid model with dimension $ n $ and curvature $ -1/K $ \\
			$ \mathcal{T}_{\textbf{x}}\Bbb H^{n,K} $ & tangent (Euclidean) space around hyperbolic vector $ x\in\Bbb H^{n,K} $ \\
			$ \exp_{\textbf{x}}^K(\textbf{v}) $ & exponential map, projecting tangent vector $ \textbf{v} $ to hyperbolic space \\
			$ \log_{\textbf{x}}^K(\textbf{y}) $ & logarithmic map, projecting hyperbolic vector $ \textbf{y} $ to $ \textbf{x} $'s tangent space \\
			$ d_{\mathcal{L}}^K(\textbf{x},\textbf{y}) $ & hyperbolic distance between hyperbolic vectors $ \textbf{x} $ and $ \textbf{y} $ \\
			$ \text{PT}_{\textbf{x}\rightarrow\textbf{y}}^K(\textbf{v}) $ & parallel transport, transporting $ \textbf{v} $ from $ \textbf{x} $'s tangent space to $ \textbf{y} $'s \\
			$ H $ & length of a path on topic tree \\
            $ \sigma(t,i) $ & similarity between topic $ t $ and document $ i $ \\
            $ \bm{\pi}_i $ & path distribution of document $ i $ over topic tree \\
            $ \textbf{z}_{t,p} $ & hyperbolic ancestral hidden state of topic $ t $ \\
            $ \textbf{z}_{t,s} $ & hyperbolic fraternal hidden state of topic $ t $ \\
			$ \textbf{z}_t $ & hyperbolic hidden state of topic $ t $ \\
            $ \sigma(h,i) $ & similarity between topic $ t $ and document $ i $ \\
            $ \textbf{z}_h $ & hyperbolic hidden state of level $ h $ \\
			$ \bm{\delta}_i $ & level distribution of document $ i $ over topic tree \\
            $ \bm{\theta}_i $ & topic distribution of document $ i $ over topic tree \\
            $ \textbf{e}_i $ & hierarchical tree embedding of document $ i $ \\
            $ T $ & number of topics on topic tree \\
            $ \textbf{g}_i $ & hierarchical graph embedding of document $ i $ \\
			$ \textbf{U} $ & a matrix of word embeddings, $ \textbf{U}\in\Bbb R^{|\mathcal{V}|\times(n+1)} $ \\
			$ \bm{\beta} $ & topic-word distribution $ \bm{\beta}\in\Bbb R^{T\times |\mathcal{V}|} $ \\
			\bottomrule
		\end{tabular}
	%}
	%\vspace{-0.2cm}
	\label{table:notations}
\end{table*}
\subsection{Fisher information and natural gradient descent}
\label{subsec: fisher info}
Fisher information is a fundamental concept in statistics that measures the amount of information a random variable carries about a parameter of interest. In this paper, we ignore dependence between layers and treat each layer independently. 
Under the context of LLMs, with the vectorized mini-batched gradient of one layer $\vecg$, we define the empirical \gls{fim} for that layer as $\mF = \E[\vecg\vecg^T]$.
\Gls{ngd} leverages the inverse of \gls{fim} to smooth the local geometry to improve convergence and stabilize training \citep{martens2020new}. 
In practice, the square-root inverse of \gls{fim} may be preferred due to its properties and improved performance on certain optimization problems \citep{yang2008principal, lin2024can, loshchilov2016sgdr, bergstra2012random, choi2019empirical}.  
The corresponding update of $\mW$ is:
\begin{align}
    \mW\leftarrow \mW-\lambda \devect(\mF^{-\frac{1}{2}}\nabla_{\theta}\mathcal{L}).
    \label{eq: square root ngd}
\end{align}
Due to the large size of $\mF\in\R^{mn\times mn}$, computing its inverse is a significant impediment to applying this update rule in practice. One way to address this is to enforce certain structures to approximate \gls{fim}. In this paper, we consider two main classes of structures: block diagonal and Kronecker product matrices. They possess favorable properties that can significantly reduce computational complexity. \Cref{subapp: kronecker product and block diagonals} provides a brief introduction and summary of their key properties. We also include a comprehensive background in \cref{app: background} to be more self-contained.
\section{Structural Approximation to \gls{fim}}
%!TEX root = ../main.tex

\section{Outline}
\label{sec:outline}

\begin{enumerate}
    \item Introduction and Related work: 
    \begin{enumerate}
        \item Introduce contact-rich problems in robotics: hard and important 
        \item General problem formulation: $\lambda$ can be contact modes or contact forces in MPCC 
        \item Four lines of previous researches:
        \begin{enumerate}
            \item Fixed-mode sequence: hybrid-MPC 
            \item Mixed-integer programming (QP, nonconvex)
            \item Contact implicit: MPCC 
            \item GCS: an extension of mixed-integer convex programming (e.g. MI-SDP)
        \end{enumerate}
        \item Ours:
        \begin{enumerate}
            \item Formulate contact-rich problems as a general polynomial optimization (POP): stronger modelling capacity
            \item Explore various level of sparsity inside the Moment-SOS Hierarchy: (1) From "variable" to "term": CS and TS; (2) From "manually-find" to "automatically-generate": MF and MD; (3) Robotics-specific sparsity: Markov, mechanics, geometry... 
            \item Result: more general, faster, tighter 
            \item Others (can be omitted): iterative tightening, robust minimizer extraction, beyond TSSOS (ts_eq_mode) ... 
        \end{enumerate}
    \end{enumerate}

    \item Multi-level sparsity pattern: CS-TS (Core)
    \begin{enumerate}
        \item A reminder: STROM: only CS, handled by hand
        \item Toy example: double integrator with soft wall
        \item Level 1: Automatic CS pattern generation: MAX, MF, MD
        \begin{enumerate}
            \item Able to find undiscovered CS pattern in STROM
        \end{enumerate}
        \item Level 2: Automatic TS pattern generation: MAX, MF, MD 
        \begin{enumerate}
            \item Able to separate entangled contact modes 
        \end{enumerate}
    \end{enumerate}

    \item Robotics-specific sparsity pattern 
    \begin{enumerate}
        \item Observation 1: for complex robotics problems, automatic pattern generation fails to detect underlying Markov structure in OCP 
        \item Observation 2: for multi-body control problems, variables in each time-step naturally groups in each body 
        \item Observation 3: variables for kinematics, dynamics, contact, geometric collision naturally form different cliques 
        \item Based on the automatic generated patterns, we further refine the CS-TS cliques, even when the resulting graph fails RIP
        \item Result: faster, tighter 
    \end{enumerate}

    \item Experiments 
    \begin{enumerate}
        \item Simulation: 
        \begin{enumerate}
            \item Push Bot 
            \item Push Box 
            \item Push T 
            \item Push Box with obstacles: one, two, many obstacles 
            \item Planar Hand 
        \end{enumerate}
        \item Real-world: push T 
        \item Simulation metrics:
        \begin{enumerate}
            \item Conversion speed compared to TSSOS: one table. (1) column: (CS, TS) = (MF, MAX), (MF, MF), (MF, MD) ... (2) row: Problem class (only need to test one example with long time horizon in each problem class)
            \item For each problem class, run 10 random initial states, collect the following statistics: (1) Mosek solving time; (2) max KKT residual; (3) Local solver success rate; (4) suboptimal gap among the success ones.
            \item For each problem class, illustrate one simulation result.
        \end{enumerate}
        \item Real-world metrics: success rate 
    \end{enumerate}
\end{enumerate}

\section{Toy example: double integrator with soft wall}

Denote system's state as $(x, v)$, control input as $u$, two wall's force as $(\lam{1}, \lam{2})$, the dynamics is:
\begin{align}
    & \x[k+1] - \x[k] = \dt \cdot v_k \\
    & v_{k+1} - v_k = \frac{\dt}{m} \cdot (u_k + \lam{1}[k] - \lam{2}[k]) \\
    & u_{\max}^2 - u_k^2 \ge 0 \\
    & \lam{1}[k] \ge 0 \\
    & \frac{\lam{1}[k] }{k_1} + d_1 + \x[k] = 0 \\
    & \lam{1}[k] \left( \frac{\lam{1}[k] }{k_1} + d_1 + \x[k] \right) \ge 0 \\
    & \lam{2}[k] \ge 0 \\
    & \frac{\lam{2}[k] }{k_2} + d_2 - \x[k] = 0 \\
    & \lam{2}[k] \left( \frac{\lam{2}[k] }{k_2} + d_2 - \x[k] \right) \ge 0
\end{align} 
where $\dt$, $k_1$, $k_2$, $d_1$, $d_2$ all constants -- we can set them as $1$ here, since we don't actually solve the problem. Objective:
\begin{align}
    \sum_{k=0}^{N-1} u_k^2 
\end{align}
For this example, $N = 3$ or $4$ is enough.

\subsection{Adam: purely diagonal structure}
\label{subsec: existing optimizer}
There have been many work arguing that Adam's second moment aims to approximate the diagonal of \gls{fim} \citep{kingma2014adam, hwang2024fadam, sun2021connection}. Although it is easy to prove that this is, in fact, optimal approximation under \cref{eq: UFE equation}, we include the following proposition for completeness. 
% When Adam was originally proposed, it was argued that the purpose of the second moment was to approximate the diagonal elements of \gls{fim} \citep{kingma2014adam, hwang2024fadam}. With a purely diagonal structural assumption, this is, in fact, the optimal solution under \cref{eq: UFE equation}:
\begin{proposition}[diagonal approximation]
    Assuming $\family =\{\diagv(\vv); v_i>0\}$, then \cref{eq: UFE equation} has analytic solution 
    \begin{equation}
        \Ft^* = \diagv(\E[\vecg^2])
    \end{equation}
where $\vecg^2$ indicates the element-wise square of $\vecg=\vect(\mG)$. 
\label{prop: adam solution}
\end{proposition}
It is trivial to show the resulting square-root \gls{ngd} recovers Adam's second moment when using the \gls{ema} to estimate $\E$. Together with the first moment, one can recover Adam.

\subsection{Shampoo: Kronecker product structure}
\label{subsec: shampoo}
Although previous work \citep{morwani2024new} has demonstrated the connection of Shampoo \citep{gupta2018shampoo} to the Kronecker product approximation to \gls{fim} through power iteration algorithm, 
we will make its connection to \cref{eq: UFE equation} more explicit and provide an alternative view: Shampoo's pre-conditioner can be derived by minimizing an upper bound of \cref{eq: UFE equation} with this structural assumption: 
\begin{align*}
    \family=\{\Rnr \otimes \Lmr; \mR_n\in\Rnn, \mL_m\in\Rmm\}
\end{align*}
where $\mR_n$ and $\mL_m$ are \gls{spd} matrices.
\begin{theorem}[Shampoo pre-conditioner]
    Assume the above structural assumption, then we have an upper bound of \cref{eq: UFE equation}
    \begin{align}
        \Fnorm{\Ft-\mF} &\leq 3(mn\Vert\mA^2-\mC^2\Vert_F\Vert\mB^2-\mC^2\Vert_F\nonumber\\
        &+ \sqrt{mn}\Fnorm{\mC}(\Vert\mA^2-\mC^2\Vert_F+\Vert\mB^2-\mC^2\Vert_F))
        \label{eq: shampoo upper bound}
    \end{align}
    where $\mA = \Rnr\otimes \mI_m$, $\mB = \mI_n \otimes \Lmr$, $\mC=\E[\vecg\vecg^T]^{\frac{1}{2}}$. Minimizing the upper-bound admits
    \begin{align*}
        \mR_n^* = \frac{1}{m}\E[\mG^T\mG],\;\;\; \mL_m^* = \frac{1}{n}\E[\mG\mG^T]
    \end{align*}
    \label{thm: optimal shampoo}
\end{theorem}
\cref{subapp: Shampoo update formula} shows that the corresponding square-root \gls{ngd} leads to the Shampoo's update formula. Therefore, the structure behind Shampoo is the Kronecker product of two square-root \gls{spd} matrices.

\subsection{Normalization and Whitening: Simple block diagonal structures}
\label{subsec: sve}
For an input $\mG$, the whitening and normalization operator are defined as 
\begin{align*}
    \whiten(\mG) =& ~(\mG\mG^T)^{-\frac{1}{2}}\mG\\
    \normalize(\mG) =& ~\mG\mS^{-\frac{1}{2}}
    \label{eq: whiten and normal operator}
\end{align*}
where $(\cdot)^{-\frac{1}{2}}$ denotes square root inverse, and $\diag(\mS)$ contains the squared column $l_2$ norm of $\mG$. Next we provide an new interpretation of these operators and show that they are the square-root \gls{ngd} updates under the following structural assumptions:
\begin{align}
    \family =& \{\mI_n\otimes \mM\} \;\;\; (\text{Whitening}) \text{\footnotemark}  \\
    \family =& \{\mS\otimes \mI_m; S_{ii}>0,\; \forall i\} \;\;\; (\text{Normalization}) 
\end{align}
\footnotetext{Note that this structure has also been proposed and discussed  \cite{duvvuricombining} under a slightly different setting.}
where $\mM\in\Rmm$ is \gls{spd} and $\mS\in\Rnn$ is a positive diagonal matrix. Given those structural assumptions, one can show:


\begin{proposition}[Normalization and whitening]
    Assuming $\family = \{\mI_n\otimes \mM\}$, minimizing \cref{eq: UFE equation} admits
    \begin{align}
        \mM^* = \frac{1}{n}\E[\mG\mG^T]
        \label{eq: generalization whitening}
    \end{align}
    If one assumes $\family = \{\mS\otimes \mI_m; S_{ii}>0,\; \forall i\}$, then the corresponding solution is 
    \begin{align}
        \mS^* = \frac{1}{m}\E[\diagv(\vg_1^T\vg_1,\ldots,\vg_n^T\vg_n)]
    \label{eq: generalization to normalization}
    \end{align}
    \label{coro: generalization to whitening and normalization}
\end{proposition}
The proof can be found in \cref{subapp: proof normalization}, where we prove a much more general solution (\cref{thm: generalization to normal and whiten}), and \cref{coro: generalization to whitening and normalization} is a special case. The corresponding square-root \gls{ngd} update with \cref{eq: generalization whitening} is $\devect(\Ft^{-\frac{1}{2}}\vecg)= \sqrt{n}\E[\mG\mG^T]^{-\frac{1}{2}}\mG$ (refer to \cref{subapp: update of generalization of whitening}). Therefore, we can view $\whiten(\cdot)$ as a special case with one-sample estimate for $\E$. Similarly, normalization is the square-root \gls{ngd} update ($\devect(\Ft^{-\frac{1}{2}}\vecg)= \sqrt{m}\mG\mS^{*-\frac{1}{2}}$) with \cref{eq: generalization to normalization} and one-sample estimate of $\E$.

Many recently proposed optimizers, such as Muon, SWAN, LARS and LAMB \citep{jordan2024muon, ma2024swan, you2017lars, you2019lamb}, rely on normalization and/or whitening. These gradient operators improve convergence \citep{you2017lars, jordan2024muon} and can replace Adam’s internal states \citep{ma2024swan}. See \cref{subapp: connection to existing optimizers} for a detailed discussion.

\subsection{\gls{alicec}: Generalization to Adam with eigenspace rotation}
\label{subsec: alicec}
\label{subsubsec: alice-c optimizer}
All structures considered above are simple and do not strictly generalize Adam's purely diagonal structure. In this and the following sections, we consider two structures that strictly generalize Adam, normalization, and whitening. Here, we first consider a block diagonal matrix with a shared eigen-space. 

Precisely, we propose the following structure that generalizes Adam\footnote{In \cref{subapp: solution to general block diagonal}, we propose an even more general block diagonal structure.}:
\begin{align}
    \family = \{\diagb(\mM_1, \ldots, \mM_n); \mM_i =\mUf\mD_i\mUf^T\}
    \label{eq: alicec structure}
\end{align} 
where $\mUf$ defines a shared \textbf{full-rank} eigen-space, and $\mD_i$ is a positive eigenvalue matrix. Adam is a special case by constraining $\mUf=\mI$. Additionally, the structures in \cref{subsec: sve} are also special cases. Whitening is obtained by a shared $\mD$ (i.e.~$\mD_i=\mD$); and normalization is by $\mUf=\mI$, $\mD_i=s_i\mI$. 
However, this structure does not directly lead to an analytic solution for \cref{eq: UFE equation}. 
Instead, we propose to approximate the solution by solving 1-iteration alternating optimization: 

\begin{theorem}[1-iteration refinement]
    For the structure in \cref{eq: alicec structure}, we consider the following 1-iteration alternating optimization of objective (\ref{eq: UFE equation}): (i) constrain $\mD_i=\mD$ to be equal, and solve $\mUf^* = \arg \min_{\mUf,\mD} \Fnorm{\diagb(\mUf\mD_1\mUf^T, \ldots, \mUf\mD_n \mUf^T) -\bm{F}} $; (ii) fix the $\mUf^*$ and find $\{\mD_i^*\} = \arg \min_{\{\mD_i\}} \Fnorm{\diagb(\mUf^*\mD_1{\mUf^*}^T, \ldots, \mUf^*\mD_n {\mUf^*}^T) -\bm{F}}$. Then, (i) and (ii) admits the following analytic solution:
    \begin{equation}
        \mUf^* = \eig(\E[\mG\mG^T]).
    \end{equation}
    where $\eig$ is the eigenvalue decomposition; and: 
    \begin{equation}
        \tilde{\mD^*} = \diagm(\E[({\mUf^*}^T\mG)\elesquare])
    \end{equation}
    where $\tilde{\mD}^* = \diagb(\mD_1^*,\ldots,\mD_n^*)$. 
    \label{thm: alicec 1 step refinement}
\end{theorem}
Based on this result, we can derive the corresponding square-root \gls{ngd} with given $\mU$ (refer to \cref{subapp: update of generlized adam}):
\begin{equation}
    \devect(\Ft^{-\frac{1}{2}}\vecg) = \mUf\frac{\mUf^T\mG}{\sqrt{\E[(\mUf^T\mG)\elesquare]}}.
    \label{eq: alicec update}
\end{equation}
This can be viewed as applying Adam's update on a space ``rotated" by eigen-matrix $\mUf$.
Consequently, we propose an optimizer, called \gls{alicec}, with the following update procedures:
\begin{align}
    &\vm_{t} = \beta_1\vm_{t-1}+(1-\beta_1)\mG_t\;\;\; (\text{first moment})\nonumber\\
    &\mQ_{t} = \beta_3\mQ_{t-1}+(1-\beta_3)\mG_t\mG_t^T\;\;\;(\text{\gls{ema} for }\E[\mG_t\mG_t^T])\nonumber\\
    & \mU_{f,t} = \eig(\mQ_{t})\nonumber\\
    & \vv_t = \beta_2\vv_{t-1}+(1-\beta_2)(\mU_{f,t}^T\mG_t)\elesquare\;\;\;(\text{second moment})\nonumber\\
    &\Delta_t = \mU_{f,t}\frac{\mU_{f,t}^T\vm_{t}}{\sqrt{\vv_t}}
\label{eq: practical alicec equations}    
\end{align}
\Cref{alg: alicec optimizer} in \cref{subapp: AdaDiag} summarizes the practical procedure. 
In fact, the above procedures closely relates to two related works: AdaDiag and one-sided SOAP \citep{anonymous2024improving, vyas2024soap}, which are heuristic memory-efficient variants of the full algorithms (i.e.~AdaDiag++ and SOAP). Before we discuss their connections, next we first show that the SOAP optimizer can also be reformulated as FIM approximation under a specific structural assumption.


\subsection{SOAP: Combination of Kronecker product with block diagonal structure}
\label{subsec: new opt combination}
All previous structures, apart from the one behind Shampoo, are under the class of block diagonal structures. However, such block-diagonal structure does not takes into account the off-diagonal part of \gls{fim}. Structure under Kronecker product, like the one behind Shampoo, can go beyond this. Therefore, we can consider combining the structure of \gls{alicec} with Shampoo, to obtain a more general structural assumption. We show this exactly recovers SOAP \citep{vyas2024soap}.

Specifically, we consider the following structural assumption: 
\begin{equation}
    \family = \{(\mU_R\otimes \mU_L)\tilde{\mD}(\mU_R\otimes \mU_L)^T\}
    \label{eq: SOAP structure}
\end{equation}
where $\mU_R\in\Rnn$, $\mU_L\in\Rmm$ are orthonormal matrix, and $\tilde{\mD}\in\mathbb{R}^{mn\times mn}$ is a diagonal matrix with positive values. We can easily show that structure behind \gls{alicec} is a special case by constraining $\mU_R=\mI_n$; and Shampoo is also a special case by constraining $\tilde{\mD}$ to be decomposed by Kronecker product (refer to \cref{subapp: connections between structural assumptions}). 

Similar to \gls{alicec}, it is hard to directly minimizing \cref{eq: UFE equation} with this assumption. We can approximate the solution by a similar 1-iteration alternating optimization procedure as \gls{alicec}. 

\begin{theorem}[SOAP as 1-iteration alternating optimization of \Cref{eq: UFE equation}]
    Assuming the above structural assumptions.
    Consider the following 1-iteration aternating optimization of \Cref{eq: UFE equation}: (i) assuming $\tilde{\mD}$ can be decomposed as Kronecker product of two diagonal matrix, then solve for $\mU_R^*$, $\mU_L^* = \arg \min_{\mU_R, \mU_L, \tilde{\mD}} \Fnorm{(\mU_R\otimes \mU_L)\tilde{\mD}(\mU_R\otimes \mU_L)^T -\bm{F}} $; (ii) fix $\mU_R^*$, $\mU_L^*$, solve for $\tilde{\mD}^* = \arg \min_{\tilde{\mD}} \Fnorm{(\mU_R^*\otimes \mU_L^*)\tilde{\mD}(\mU_R^*\otimes \mU_L^*)^T -\bm{F}}$ without Kronecker product assumption of $\tilde{\mD}$. Then,
    (i) admits analytic solution when minimizing the upper bound of \cref{eq: UFE equation} (i.e.~\cref{eq: shampoo upper bound}):
    \begin{align*}
        \mU_R^* = \eig(\E[\mG^T\mG]),\;\;\; \mU_L^* = \eig(\E[\mG\mG^T]).
    \end{align*} Step (2) admits an analytic solution for \cref{eq: UFE equation}:
    \begin{align*}
        \tilde{\mD^*} = \diagm(\E[({\mU_L^*}^T\mG{\mU_R}^*)\elesquare])
    \end{align*}. 
    \label{thm: optimal asham}
\end{theorem}
The proof is a straightforward combination of 
\cref{thm: optimal shampoo} and \cref{thm: alicec 1 step refinement}, and can be found in \cref{subapp: proof asham}. One can show that the corresponding square-root \gls{ngd} update associated with the above result exactly recovers the update rules in SOAP optimizer (refer to \cref{subapp: update formula for SOAP} for details). 



\paragraph{Connections to \gls{alicec}} Compared to \gls{alicec}, SOAP follows a more general structural assumption. However, from the FIM approximation perspective, SOAP does not exactly solve the 1-iteration alternating optimization problem. Instead, when solving for $\mU_R^*, \mU_L^* = \arg \min_{\mU_R, \mU_L, \tilde{\mD}} \Fnorm{(\mU_R\otimes \mU_L)\tilde{\mD}(\mU_R\otimes \mU_L)^T -\bm{F}}$, SOAP minimizes the upper bound instead. On the contrary, \gls{alicec} exactly solves the 1-iteration refinement problem. In addition, the structures behind \gls{alicec} and SOAP are different, and \gls{alicec} should not be viewed as a simple variant of SOAP. 



% Despite the structural generality of \gls{alicec}, it only focuses on the block diagonals of \gls{fim}. To go beyond this, one can combine the structure of \gls{alicec} with Shampoo's Kronecker product to obtain a structure that generalizes all the others in this paper. Interestingly, this turns out to be SOAP/AdaDiag++. Refer to \cref{subsec: new opt combination}. 



% \subsection{\gls{alicec} and SOAP: Two generalizations of Adam}
% \label{subsec: alicec}
% \label{subsubsec: alice-c optimizer}
% All structures considered above are simple and do not strictly generalize Adam's purely diagonal structure. In this section we consider two structures that strictly generalize Adam, normalization, and whitening: a block diagonal matrix with a shared eigen-space; and a combination of a block-diagonal matrix and Kronecker product.

% First, consider a block-diagonal structure that generalizes Adam\footnote{In \cref{subapp: solution to general block diagonal}, we propose an even more general block diagonal structure.}:
% \begin{align}
%     \family = \{\diagb(\mM_1, \ldots, \mM_n); \mM_i =\mUf\mD_i\mUf^T\}
%     \label{eq: alicec structure}
% \end{align} 
% where $\mUf$ defines a shared \textbf{full-rank} eigen-space, and $\mD_i$ is a positive eigenvalue matrix. Adam is a special case by constraining $\mUf=\mI$. Additionally, the structures in \cref{subsec: sve} are also special cases by setting $\mD_i=\mD$ (whitening); and $\mUf=\mI$, $\mD_i=s_i\mI$ (normalization). 
% However, this structure does not directly lead to an analytic solution for \cref{eq: UFE equation}. 
% Instead, we propose to approximate the solution by solving a 1-step refinement. 

% \begin{theorem}[1-step refinement]
%     For the structure in \cref{eq: alicec structure}, we consider the following 1-step refinement: (i) assuming all $\mD_i$ are equal, and find $\mUf$; (ii) fix the $\mUf$ and find $\mD_i$. Then, step (i) admits the analytic solution:
%     \begin{equation}
%         \mUf = \eig(\E[\mG\mG^T]).
%     \end{equation}
%     where $\eig$ is the eigenvalue decomposition; step (ii) admits the analytic solution: 
%     \begin{equation}
%         \tilde{\mD} = \diagm(\E[(\mUf^T\mG)\elesquare])
%     \end{equation}
%     where $\tilde{\mD} = \diagb(\mD_1,\ldots,\mD_n)$. 
%     \label{thm: alicec 1 step refinement}
% \end{theorem}
% Based on this result, we can derive the corresponding square-root \gls{ngd} with given $\mU$ (refer to \cref{subapp: update of generlized adam}):
% \begin{equation}
%     \devect(\Ft^{-\frac{1}{2}}\vecg) = \mUf\frac{\mUf^T\mG}{\sqrt{\E[(\mUf^T\mG)\elesquare]}}.
%     \label{eq: alicec update}
% \end{equation}
% This can be viewed as applying Adam's update on a space ``rotated" by $\mUf$.
% Consequently, we propose an optimizer, called \gls{alicec}, with the following update procedures:
% \begin{align}
%     &\vm_{t} = \beta_1\vm_{t-1}+(1-\beta_1)\mG_t\;\;\; (\text{first moment})\nonumber\\
%     &\mQ_{t} = \beta_3\mQ_{t-1}+(1-\beta_3)\mG_t\mG_t^T\;\;\;(\text{\gls{ema} for }\E[\mG_t\mG_t^T])\nonumber\\
%     & \mU_{f,t} = \eig(\mQ_{t})\nonumber\\
%     & \vv_t = \beta_2\vv_{t-1}+(1-\beta_2)(\mU_{f,t}^T\mG_t)\elesquare\;\;\;(\text{second moment})\nonumber\\
%     &\Delta_t = \mU_{f,t}\frac{\mU_{f,t}^T\vm_{t}}{\sqrt{\vv_t}}
% \label{eq: practical alicec equations}    
% \end{align}
% \Cref{alg: alicec optimizer} in \cref{subapp: AdaDiag} summarizes the practical procedure. 
% In fact, the above procedures coincide with two concurrent works: AdaDiag and one-sided SOAP \citep{anonymous2024improving, vyas2024soap}, which are heuristic memory-efficient variants of their main algorithms (i.e.~AdaDiag++ and SOAP).
% In the following, we show \gls{alicec} should not be classified as a sinmple variant of SOAP/AdaDiag++, since they have different underlying structures, and should be given a different name. 



% \paragraph{SOAP/AdaDiag++} Despite the structural generality of \gls{alicec}, it only focuses on the block diagonals of \gls{fim}. To go beyond this, one can combine the structure of \gls{alicec} with Shampoo's Kronecker product to obtain a structure that generalizes all the others in this paper. Interestingly, this turns out to be SOAP/AdaDiag++. Refer to \cref{subsec: new opt combination}. 
\section{\gls{ssgd}: memory-efficient optimizer from a carefully selected structure}
\label{sec: ssgd}
The structured \gls{fim} approximation reveals two important insights: there exists a correspondence between structural assumption and optimizers, and structural generality often comes at the cost of practical efficiency. For example, while the structures of \gls{alicec} and SOAP offer more accurate \gls{fim} approximations than a simple structure like gradient normalization, they require expensive computation and memory consumption (\cref{tab: summary table}), making them impractical for training LLMs. Building on this, our first design recommendation is to \textbf{select structures that balance generality and practical efficiency.} 

To demonstrate this, we select a structure that generalizes gradient normalization, which scales both the rows and columns simultaneously:
\begin{align}
    \family = \{\mS \otimes \mQ\}
    \label{eq: ssgd structure}
\end{align}
where $\mS\in\Rnn$, $\mQ\in\Rmm$ are positive diagonal matrices. The idea of diagonal approximation has also been leveraged in previous work under different setups \citep{zhao2024adapprox, shazeer2018adafactor, li2018preconditioner}. 
The optimal solution of \cref{eq: UFE equation} can be solved by a fixed-point iterative procedure:

\begin{proposition}[Two-sided scaling]
Assuming the structure of \cref{eq: ssgd structure}, and $\E[\mG\elesquare]$ contains only positive values, solving \cref{eq: UFE equation} admits an iterative fixed point procedure:
\begin{align}
    \vs = \frac{\diag\left(\E[\mG^T\mQ\mG]\right)}{\Vert\mQ\Vert_F^2},\;\;\;
    \vq =\frac{\diag\left(\E[\mG\mS\mG^T]\right)}{\Vert\mS\Vert_F^2}.
    \label{eq: iterative procedure double scaling}
\end{align}
where $\vs=\diag(\mS)$, $\vq=\diag(\mQ)$.
Additionally, the fixed point solution $\vs$, $\vq$ converges to the right and left principal singular vector of $\E[\mG\elesquare]$ up to a scaling factor with unique $\mS^*\otimes \mQ^*$.
\label{prop: two sided scaling}
\end{proposition}
In practice, we find initializing $\vq=\bm{1}$ and use 1-sample estimate of $\E[\cdot]$ gives good performance. 
Interestingly, \citet{morwani2024new} also connects Shampoo to 1-step power iteration. Here, the \cref{eq: iterative procedure double scaling} can also be viewed as a power iteration algorithm. The main difference is that \citet{morwani2024new} assumes full \gls{spd} matrix $\mS$ and $\mQ$, but our structural constraint is positive diagonal matrix. Consequently, our procedure is computationally efficient and allows for multiple iterations.

The corresponding square-root \gls{ngd} update scales both rows and columns through $\mS$ and $\mQ$ (i.e.~$\devect(\Ft^{-\frac{1}{2}}\vecg) = \mQ^{-\frac{1}{2}}\mG\mS^{-\frac{1}{2}}$).
We name this optimizer, \glsreset{ssgd}\gls{ssgd} (\cref{alg: ssgd optimizer}). Although analytic solutions of $\vs$, $\vq$ exist, we perform $5$ steps of \cref{eq: iterative procedure double scaling} as an efficient approximation. To further stabilize training, we also incorporate the norm-growth limiter used in \citet{chen2024fira}. \gls{ssgd} is highly memory efficient since it only needs the storage of two diagonal matrices $\mS$ and $\mQ$ and a scalar for the limiter, consuming $m+n+1$ memory. In \cref{subapp: connection to existing optimizers} we discuss  connections to Adapprox, Apollo and Adafactor \citep{zhao2024adapprox, zhu2024apollo, shazeer2018adafactor}.
\begin{algorithm}
    \caption{\gls{ssgd}}
    \label{alg: ssgd optimizer}
    \begin{algorithmic}[1]
        \STATE {\bfseries Input:} learning rate $\lambda$, $\beta$, scale $\alpha$, limiter threshold $\gamma$, optimization steps $T$.
        \STATE $\vs_0 = 0$; $\vq_0=0$; $\phi_0=0$
        \FOR{$t=1,\ldots,T$}
            \STATE $\mG_t=\nabla_{\mW_t}\mathcal{L}$
            \STATE Obtain $\mS_t$ and $\mQ_t$ by \cref{eq: iterative procedure double scaling}
            \STATE $\vs_t=\beta\vs_{t-1}+(1-\beta)\diag(\mS_t)$; 
            \STATE $\vq_t=\beta\vq_{t-1}+(1-\beta)\diag(\mQ_t)$
            \STATE$\tilde{\mG}_t= \diagv(\vq_t)^{-\frac{1}{2}}\mG\diagv(\vs_t)^{-\frac{1}{2}}$
            % \STATE \textcolor{blue}{\%\textit{Norm-growth limiter}}
            \STATE $\eta =\gamma/\max\{\frac{\Vert\tilde{\mG}_t\Vert}{\phi_{t-1}}, \gamma\}$ if $t>1$ else $1$ 
            \STATE $\phi_t = \eta\Vert\tilde{\mG}_t\Vert$
            \STATE $\mW_{t+1}=\mW_t -\lambda\eta \alpha\tilde{\mG}_t$
        \ENDFOR
    \end{algorithmic}
\end{algorithm}

This design recommendation has its limitations. Finding such a structure with a balanced trade-off may not always be easy, and the resulting structure tends to be simple, offering less accurate approximation to \gls{fim} compared to the general ones. Since the main bottleneck of more general optimizers is their practical efficiency, our second design recommendation is to: \textbf{improve their efficiency by converting full-rank optimizers into low-rank counterparts using a novel low-rank extension framework}.


\section{\gls{alice}: memory-efficient optimizer from low-rank extension framework}
\label{sec: memory efficient opt}

In this section, we propose a novel low-rank framework consisting of three steps, low-rank \textbf{tracking}, subspace \textbf{switching}; and \textbf{compensation}. Tracking aims to reduce the memory cost of \gls{ema} states, whereas switching and compensation are designed to correct the potential issues caused by tracking and the limitations of low-rank projections. We demonstrate this procedure by converting \gls{alicec} to its low-rank version, \gls{alice}. While the procedure could be applied to SOAP in a similar manner, we leave this for future work. 

\paragraph{Reduce computational cost}
To improve the computational efficiency, we make two modifications to \gls{alicec}. First, we propose to use 1-step subspace iteration algorithm as an efficient scheme to find leading eigenvectors (\cref{alg: subspace iteration} in \cref{app: background}). Second, we only update projection $\mU$ every $K$ steps, effectively partitioning the training into time blocks with size $K$, and amortizing the cost. Consequently, $\mU$ is fixed within a time block. 

\subsection{Tracking: low-rank projections to reduce memory}
By carefully examining the \gls{alicec}'s procedure (\cref{eq: alicec update}), we notice all internal states are connected through the projection $\mU_{f,t}$. To reduce the memory cost, we can obtain a low-rank $\mU_t$ by keeping only the top $r$ eigenvectors, and denote the remaining $m-r$ basis as $\mU_{c,t}$ (i.e.~$\mU_{f,t}=[\mU_t, \mU_{c,t}]$). For tracking state $\mQ_t$, we can also apply low-rank approximation and only track the projected states. We call it \textbf{low-rank tracking}:
\begin{align}
    \bm{\sigma}_t = \mU_t^T\mG_t;\;\;\;\widetilde{\mQ}_{t+1} = \beta_3\widetilde{\mQ}_{t} + (1-\beta_3)\bm{\sigma}_{t}\bm{\sigma}_t^T
    \label{eq: tracking step}
\end{align}
where $\bm{\sigma}_t$ is the projected gradient, and $\widetilde{\mQ}_t$ is the low-rank tracking state. One can easily reconstruct back $\mQ_t \approx \mU_t\widetilde{\mQ}_t\mU_t^T$ when needed. This reduces the memory from $m^2$ to $r^2$. 

However, this low-rank projection comes with two major consequences: (1) the reconstructed state $\mQ_t$ is no longer accurate; (2) the resulting parameter update $\Delta$ in \cref{eq: alicec update} ignores the information within $\mU_{c,t}$ due to low-rank $\mU_t$. Next, we propose two additional steps, switching and compensation, rooted in theoretical insights to address these two problems, respectively.


\subsection{Switching: mixing leading basis with the complements}
\label{subsec: switching}
We omit the subscript $t$ in $\mU$ and $\mU_c$ in the following since they are fixed within a time block. 
Since the projected gradient $\bm{\sigma}_t$ only maintains the information in the space spanned by $\mU$, the low-rank tracking state $\widetilde{\mQ}_t$ necessarily discards information in $\mU_{c}$. Therefore, even if those directions should become the leading basis at the time we update the $\mU$, $\widetilde{\mQ}_t$ will ignore their contributions, causing the stability of the previous leading eigenvectors and preventing the exploration of other spaces. We prove that this is possible by showing that the true tracking state $\mQ_t$ can be decomposed into low-rank tracking reconstruction and a residual term quantifying the importance of $\mU_{c}$:
\begin{proposition}[Subspace switching]
    Assuming the setup mentioned above and all assumptions of \gls{alicec} are satisfied. We further assume the low-rank $\mU\in\Rmr$ is obtained at the beginning of $i+1$ time block by $\eig(\mQ^*_{ik},r)$ where $\mQ^*_{ik}$ is the true tracking state. Further, we assume the stability of the eigen-basis such that gradient $\mG_t$ during $i+1$ time block shares the same eigen-basis as $\mQ^*_{ik}$. Then, the true tracking state at the end of $i+1$ block, $\mQ^*_{(i+1)k}$, can be decomposed into:
    \begin{align}
        \mQ^*_{(i+1)k}=\sum_{t=ik+1}^{(i+1)k} \mG_t\mG_t^T = \sum_{t=ik+1}^{(i+1)k} \widetilde{\mG}_t\widetilde{\mG}_t^T+\mU_c\bm{\Sigma}_t\mU_c^T
        \label{eq: what is happening with low-rank U}
    \end{align}
    where $\widetilde{\mG}_t=\mU\bm{\sigma}_t$ is the low rank reconstructed gradients, $\bm{\Sigma}_t\in\mathbb{R}^{(m-r)\times (m-r)}$ is a diagonal matrix with positive values, and $\mU_c$ is the remaining eigen-basis such that $[\mU,\mU_c]$ will form the complete eigen-basis of $\mQ^*_{ik}$. 
    \label{prop: subspace switching}
\end{proposition}
When some entries in $\bm{\Sigma}_t$ become dominant, the corresponding basis in $\mU_c$ will form the new leading eigen-basis when updating the projection $\mU$ through $\eig(\mQ_{(i+1)k}^*)$. On the other hand, if $\mU$ is updated with low-rank reconstructed state alone, it is equivalent to setting $\bm{\Sigma}_t=0$, ignoring the contributions of $\mU_c$, and resulting in the stability of the previous leading basis. 
Inspired by this insight, we propose a practical scheme to update $\mU$, instead of relying on \gls{evd} of low-rank reconstructed state. We call it subspace switching (\cref{alg: subspace switching}). Intuitively, we force the new projection $\mU$ to mix the leading eigen-basis with randomly sampled eigenvectors from the remaining basis $\mU_c$, allowing the optimizer to explore other spaces. Although the true $\mU_c$ is hard to obtain in practice, we propose to approximate it by the QR decomposition of $\mU$. 

\begin{algorithm}
    \caption{Subspace switching}
    \label{alg: subspace switching}
    \begin{algorithmic}[1]
        \STATE {\bfseries Input:} Reconstructed state $\mQ$, rank $r$, leading basis number $l$, previous low-rank projection $\mU_{t-1}$
        \STATE $\mU_t'=\text{Subspace iteration}(\mQ, \mU_{t-1})$
        \STATE $\mU'_{t,1}\leftarrow \text{Keep top $l$ eigenvectors of }\mU'_t$ 
        \STATE Uniformly sample $r-l$ basis from the complement $\mU'_{c,t} = \qr(\mU'_{t})$ as $\mU'_{t,2}$
        \STATE Combine basis $\mU = [\mU'_{t,1},\mU'_{t,2}]$
        \STATE {\bfseries Return} $\mU$
    \end{algorithmic}
\end{algorithm}

\subsection{Compensation: convert low-rank update to be full-rank}
\label{subsec: compensation}
Another problem with low-rank projection $\mU$ is the information loss in the resulting parameter update $\Delta$ at each step. The goal of compensation step is to compensate for this information loss with minimal memory overhead. Firstly, we need to know which information has been discarded. We show that the update $\Delta$ with full-rank $\mU_{f}$ in \cref{eq: alicec update} can be decomposed into the low-rank update with $\mU$ and complement update controlled by $\mU_{c}$ (proof in \cref{subapp: discussion low-rank}):
\begin{align}
    \Delta =\mU\frac{\mU^T\mG}{\sqrt{\E[(\mU^T\mG)\elesquare]}}
    + \underbrace{\mU_{c}\frac{\mU_{c}^T\mG}{\sqrt{\E[(\mU_{c}^T\mG)\elesquare]}}}_{\devect(\Ft_{c}^{-\frac{1}{2}}\vecg)}
    \label{eq: alice update decomposition}
\end{align}
where $\Ft_{c}=\diagb(\mU_{c}\mD_{c,1}\mU_{c}^T,\ldots, \mU_{c}\mD_{c,n}\mU_{c}^T)$ is the approximated \gls{fim} corresponding to the complement basis $\mU_c$, $\mD_{c,i}=\diagv(\E[(\mU_c^T\vg_i)^2])$ and $^{-\frac{1}{2}}$ is the square-root pseudo-inverse. 

We notice that the discarded information, $\devect(\Ft_c^{\frac{1}{2}}\vecg)$, has the same format as the square-root \gls{ngd} with \gls{fim} $\Ft_c$. From the proposed \gls{fim} view point, the design of compensation term becomes the selection of a structure to approximate $\Ft_c$, and the application of the corresponding square-root \gls{ngd}. Considering the trade-off between structural generality and practical efficiency, one structural choice is the diagonal structure of normalization operator, which simply scales the columns of gradient matrix and is highly memory efficient. In addition, we only want to focus on the discarded information $\mU_c\mU_c^T\mG$ for compensation, rather than the entire gradient $\mG$. We propose the following compensation at each step $t$:
\begin{align}
    \mC_t= \devect((\mS\otimes\mU_c\mU_c^T)\vecg_t)=\mU_c\mU_c^T\mG_t\mS
    \label{eq: compensation term}
\end{align}
where $\mS$ is a positive diagonal matrix, $\mU_c\mU_c^T\mG=(\mG-\mU\mU^T\mG)$ is the gradient information within the remaining basis. We show that such design choice admits an optimal solution of $\mS$ to the \gls{fim} approximation problem.
\begin{theorem}[Optimal compensation]
    Assume that the conditions of \gls{alicec} are satisfied. With the proposed form of compensation (\cref{eq: compensation term}), minimizing \gls{fim} reconstruction loss
    \begin{align*}
        \Fnorm{(\mS_t^{-2}\otimes \mU_c\mU_c^T)-\Ft_{c}}
    \end{align*}
    admits analytic solution:
    \begin{equation}
        \diag(\mS_t) = \frac{\sqrt{m-r}}{\sqrt{\E[\bm{1}_m^T\mG_t\elesquare-\bm{1}_r^T(\mU^T\mG_t)\elesquare}]}
         \label{eq: optimal D for compensation}
    \end{equation}
    where $\bm{1}_m\in\mathbb{R}^m$, $\bm{1}_r\in\mathbb{R}^r$ are the column vectors with element $1$.
    \label{thm: optimal compensation}
\end{theorem}
\cref{alg: compensation} summarizes the compensation step.

\begin{algorithm}
    \caption{Compensation}
    \label{alg: compensation}
    \begin{algorithmic}[1]
        \STATE {\bfseries Input:} $\mG_t$, projection $\mU$, previous norm $\vp$, limiter norm $\bm{\phi}$, limiter threshold $\gamma$, decay rate $\beta$
        \STATE $\vp \leftarrow \beta\vp+(1-\beta)(\bm{1}_m^T\mG_t\elesquare-\bm{1}_r^T(\mU^T\mG_t)\elesquare)$
        \STATE $\mC_t\leftarrow\sqrt{m-r}(\mG_t-\mU\mU^T\mG_t)\diagv(\vp)^{-\frac{1}{2}}$
        \STATE $\eta=\gamma\slash \max\{\frac{\Vert\mC_t\Vert}{\phi},\gamma\}$ if $\phi>0$ else $1$        
        \STATE $\phi = \Vert\eta\mC_t\Vert$      
        \STATE $\mC_t\leftarrow \eta \mC_t$
        \STATE {\bfseries Return} $\mC_t$, $\vp$, $\phi$
    \end{algorithmic}
\end{algorithm}



\subsection{\Gls{alice} optimizer}
\label{subsec: alice optimizer}
By combining \gls{alicec} with low-rank $\mU$, tracking, switching and compensation, we obtain \gls{alice}, a novel low-rank optimizer. One can also design a simple variant, \gls{alicez}, by disabling the tracking for better memory efficiency.
\paragraph{Connections to GaLore}
Interestingly, GaLore, in fact, is an approximation to \gls{alice} without tracking, switching and compensation. Based on the connection of \gls{alice} to \gls{alicec}, we reveal that GaLore is a simple low-rank extension of \gls{alicec}, a more general optimizer than Adam. This also reflects the advantage of the \gls{fim} view point, which provides a deeper understanding and an explanation on its better performance than Adam under certain scenarios \citep{zhao2024galore}.
% Our derivation of \gls{alice} offers a connection of GaLore to \gls{alicec}, revealing that it is a low-rank version of a more powerful optimizer than Adam, and providing an explanation of its effectiveness. 


\begin{algorithm}
\caption{\gls{alice}/\gls{alicez} optimizer}
\label{alg: alice optimizer}
    \begin{algorithmic}[1]
        \STATE {\bfseries Input:} learning rate $\lambda$, scale $\alpha$, compensation scale $\alpha_c$, update interval $k$, $\beta_1$, $\beta_2$, $\beta_3$ ($\beta_3=0$ for \gls{alicez}),  optimization step $T$, rank $r$, loss function $\mathcal{L}$, limiter threshold $\gamma$, leading basis number $l$ 
        \STATE $\widetilde{\mQ}_0 = 0$, $\mU_0=0$, $\vp_0=0$, $\bm{\phi}=0$, $\vm_0=0$, $\vv_0=0$
        \FOR{$t=1\ldots, T$}
            \STATE $\mG_t=\nabla_{\mW_t}\mathcal{L}$
            \IF{$t==1$ or $(t\mod K)==0$}
                \STATE 
                $\mQ_t=\beta_3\mU_t\widetilde{\mQ}_{t-1}\mU_t^T+(1-\beta_3)\mG_t\mG_t^T$ 
                \STATE $\mU_t=Switch(\mQ_t, r, l, \mU_{t-1})$
            \ELSE
                \STATE $\mU_t=\mU_{t-1}$
            \ENDIF
            \STATE $\bm{\sigma}_t = \mU_t^T\mG_t$ 
            \STATE $\widetilde{\mQ}_t = \beta_3 \widetilde{\mQ}_{t-1} +(1-\beta_3)\bm{\sigma}_t\bm{\sigma}_t^T$ 
            \STATE $\vm_t = \beta_1 \vm_{t-1}+ (1-\beta_1)\bm{\sigma}_t$
            \STATE $\vv_t = \beta_2 \vv_{t-1} + (1-\beta_2) \bm{\sigma}_t\elesquare$
            \STATE $\bm{\omega} = \frac{\vm_t}{\sqrt{\vv_t}}$
            \STATE $\Delta_c, \vp_t, \bm{\phi}_t = Compensation(\mG_t, \mU_t, \vp_{t-1}, \bm{\phi}_{t-1}, \gamma, \beta_1)$ 
            \STATE $\mW_{t+1}=\mW_t-\lambda\alpha (\mU\bm{\omega}+\alpha_c\Delta_c)$
        \ENDFOR
    \end{algorithmic}
\end{algorithm}


\putsec{related}{Related Work}

\noindent \textbf{Efficient Radiance Field Rendering.}
%
The introduction of Neural Radiance Fields (NeRF)~\cite{mil:sri20} has
generated significant interest in efficient 3D scene representation and
rendering for radiance fields.
%
Over the past years, there has been a large amount of research aimed at
accelerating NeRFs through algorithmic or software
optimizations~\cite{mul:eva22,fri:yu22,che:fun23,sun:sun22}, and the
development of hardware
accelerators~\cite{lee:cho23,li:li23,son:wen23,mub:kan23,fen:liu24}.
%
The state-of-the-art method, 3D Gaussian splatting~\cite{ker:kop23}, has
further fueled interest in accelerating radiance field
rendering~\cite{rad:ste24,lee:lee24,nie:stu24,lee:rho24,ham:mel24} as it
employs rasterization primitives that can be rendered much faster than NeRFs.
%
However, previous research focused on software graphics rendering on
programmable cores or building dedicated hardware accelerators. In contrast,
\name{} investigates the potential of efficient radiance field rendering while
utilizing fixed-function units in graphics hardware.
%
To our knowledge, this is the first work that assesses the performance
implications of rendering Gaussian-based radiance fields on the hardware
graphics pipeline with software and hardware optimizations.

%%%%%%%%%%%%%%%%%%%%%%%%%%%%%%%%%%%%%%%%%%%%%%%%%%%%%%%%%%%%%%%%%%%%%%%%%%
\myparagraph{Enhancing Graphics Rendering Hardware.}
%
The performance advantage of executing graphics rendering on either
programmable shader cores or fixed-function units varies depending on the
rendering methods and hardware designs.
%
Previous studies have explored the performance implication of graphics hardware
design by developing simulation infrastructures for graphics
workloads~\cite{bar:gon06,gub:aam19,tin:sax23,arn:par13}.
%
Additionally, several studies have aimed to improve the performance of
special-purpose hardware such as ray tracing units in graphics
hardware~\cite{cho:now23,liu:cha21} and proposed hardware accelerators for
graphics applications~\cite{lu:hua17,ram:gri09}.
%
In contrast to these works, which primarily evaluate traditional graphics
workloads, our work focuses on improving the performance of volume rendering
workloads, such as Gaussian splatting, which require blending a huge number of
fragments per pixel.

%%%%%%%%%%%%%%%%%%%%%%%%%%%%%%%%%%%%%%%%%%%%%%%%%%%%%%%%%%%%%%%%%%%%%%%%%%
%
In the context of multi-sample anti-aliasing, prior work proposed reducing the
amount of redundant shading by merging fragments from adjacent triangles in a
mesh at the quad granularity~\cite{fat:bou10}.
%
While both our work and quad-fragment merging (QFM)~\cite{fat:bou10} aim to
reduce operations by merging quads, our proposed technique differs from QFM in
many aspects.
%
Our method aims to blend \emph{overlapping primitives} along the depth
direction and applies to quads from any primitive. In contrast, QFM merges quad
fragments from small (e.g., pixel-sized) triangles that \emph{share} an edge
(i.e., \emph{connected}, \emph{non-overlapping} triangles).
%
As such, QFM is not applicable to the scenes consisting of a number of
unconnected transparent triangles, such as those in 3D Gaussian splatting.
%
In addition, our method computes the \emph{exact} color for each pixel by
offloading blending operations from ROPs to shader units, whereas QFM
\emph{approximates} pixel colors by using the color from one triangle when
multiple triangles are merged into a single quad.



\section{Experiments}
\section{Experiments}
\label{sec:experiments}
The experiments are designed to address two key research questions.
First, \textbf{RQ1} evaluates whether the average $L_2$-norm of the counterfactual perturbation vectors ($\overline{||\perturb||}$) decreases as the model overfits the data, thereby providing further empirical validation for our hypothesis.
Second, \textbf{RQ2} evaluates the ability of the proposed counterfactual regularized loss, as defined in (\ref{eq:regularized_loss2}), to mitigate overfitting when compared to existing regularization techniques.

% The experiments are designed to address three key research questions. First, \textbf{RQ1} investigates whether the mean perturbation vector norm decreases as the model overfits the data, aiming to further validate our intuition. Second, \textbf{RQ2} explores whether the mean perturbation vector norm can be effectively leveraged as a regularization term during training, offering insights into its potential role in mitigating overfitting. Finally, \textbf{RQ3} examines whether our counterfactual regularizer enables the model to achieve superior performance compared to existing regularization methods, thus highlighting its practical advantage.

\subsection{Experimental Setup}
\textbf{\textit{Datasets, Models, and Tasks.}}
The experiments are conducted on three datasets: \textit{Water Potability}~\cite{kadiwal2020waterpotability}, \textit{Phomene}~\cite{phomene}, and \textit{CIFAR-10}~\cite{krizhevsky2009learning}. For \textit{Water Potability} and \textit{Phomene}, we randomly select $80\%$ of the samples for the training set, and the remaining $20\%$ for the test set, \textit{CIFAR-10} comes already split. Furthermore, we consider the following models: Logistic Regression, Multi-Layer Perceptron (MLP) with 100 and 30 neurons on each hidden layer, and PreactResNet-18~\cite{he2016cvecvv} as a Convolutional Neural Network (CNN) architecture.
We focus on binary classification tasks and leave the extension to multiclass scenarios for future work. However, for datasets that are inherently multiclass, we transform the problem into a binary classification task by selecting two classes, aligning with our assumption.

\smallskip
\noindent\textbf{\textit{Evaluation Measures.}} To characterize the degree of overfitting, we use the test loss, as it serves as a reliable indicator of the model's generalization capability to unseen data. Additionally, we evaluate the predictive performance of each model using the test accuracy.

\smallskip
\noindent\textbf{\textit{Baselines.}} We compare CF-Reg with the following regularization techniques: L1 (``Lasso''), L2 (``Ridge''), and Dropout.

\smallskip
\noindent\textbf{\textit{Configurations.}}
For each model, we adopt specific configurations as follows.
\begin{itemize}
\item \textit{Logistic Regression:} To induce overfitting in the model, we artificially increase the dimensionality of the data beyond the number of training samples by applying a polynomial feature expansion. This approach ensures that the model has enough capacity to overfit the training data, allowing us to analyze the impact of our counterfactual regularizer. The degree of the polynomial is chosen as the smallest degree that makes the number of features greater than the number of data.
\item \textit{Neural Networks (MLP and CNN):} To take advantage of the closed-form solution for computing the optimal perturbation vector as defined in (\ref{eq:opt-delta}), we use a local linear approximation of the neural network models. Hence, given an instance $\inst_i$, we consider the (optimal) counterfactual not with respect to $\model$ but with respect to:
\begin{equation}
\label{eq:taylor}
    \model^{lin}(\inst) = \model(\inst_i) + \nabla_{\inst}\model(\inst_i)(\inst - \inst_i),
\end{equation}
where $\model^{lin}$ represents the first-order Taylor approximation of $\model$ at $\inst_i$.
Note that this step is unnecessary for Logistic Regression, as it is inherently a linear model.
\end{itemize}

\smallskip
\noindent \textbf{\textit{Implementation Details.}} We run all experiments on a machine equipped with an AMD Ryzen 9 7900 12-Core Processor and an NVIDIA GeForce RTX 4090 GPU. Our implementation is based on the PyTorch Lightning framework. We use stochastic gradient descent as the optimizer with a learning rate of $\eta = 0.001$ and no weight decay. We use a batch size of $128$. The training and test steps are conducted for $6000$ epochs on the \textit{Water Potability} and \textit{Phoneme} datasets, while for the \textit{CIFAR-10} dataset, they are performed for $200$ epochs.
Finally, the contribution $w_i^{\varepsilon}$ of each training point $\inst_i$ is uniformly set as $w_i^{\varepsilon} = 1~\forall i\in \{1,\ldots,m\}$.

The source code implementation for our experiments is available at the following GitHub repository: \url{https://anonymous.4open.science/r/COCE-80B4/README.md} 

\subsection{RQ1: Counterfactual Perturbation vs. Overfitting}
To address \textbf{RQ1}, we analyze the relationship between the test loss and the average $L_2$-norm of the counterfactual perturbation vectors ($\overline{||\perturb||}$) over training epochs.

In particular, Figure~\ref{fig:delta_loss_epochs} depicts the evolution of $\overline{||\perturb||}$ alongside the test loss for an MLP trained \textit{without} regularization on the \textit{Water Potability} dataset. 
\begin{figure}[ht]
    \centering
    \includegraphics[width=0.85\linewidth]{img/delta_loss_epochs.png}
    \caption{The average counterfactual perturbation vector $\overline{||\perturb||}$ (left $y$-axis) and the cross-entropy test loss (right $y$-axis) over training epochs ($x$-axis) for an MLP trained on the \textit{Water Potability} dataset \textit{without} regularization.}
    \label{fig:delta_loss_epochs}
\end{figure}

The plot shows a clear trend as the model starts to overfit the data (evidenced by an increase in test loss). 
Notably, $\overline{||\perturb||}$ begins to decrease, which aligns with the hypothesis that the average distance to the optimal counterfactual example gets smaller as the model's decision boundary becomes increasingly adherent to the training data.

It is worth noting that this trend is heavily influenced by the choice of the counterfactual generator model. In particular, the relationship between $\overline{||\perturb||}$ and the degree of overfitting may become even more pronounced when leveraging more accurate counterfactual generators. However, these models often come at the cost of higher computational complexity, and their exploration is left to future work.

Nonetheless, we expect that $\overline{||\perturb||}$ will eventually stabilize at a plateau, as the average $L_2$-norm of the optimal counterfactual perturbations cannot vanish to zero.

% Additionally, the choice of employing the score-based counterfactual explanation framework to generate counterfactuals was driven to promote computational efficiency.

% Future enhancements to the framework may involve adopting models capable of generating more precise counterfactuals. While such approaches may yield to performance improvements, they are likely to come at the cost of increased computational complexity.


\subsection{RQ2: Counterfactual Regularization Performance}
To answer \textbf{RQ2}, we evaluate the effectiveness of the proposed counterfactual regularization (CF-Reg) by comparing its performance against existing baselines: unregularized training loss (No-Reg), L1 regularization (L1-Reg), L2 regularization (L2-Reg), and Dropout.
Specifically, for each model and dataset combination, Table~\ref{tab:regularization_comparison} presents the mean value and standard deviation of test accuracy achieved by each method across 5 random initialization. 

The table illustrates that our regularization technique consistently delivers better results than existing methods across all evaluated scenarios, except for one case -- i.e., Logistic Regression on the \textit{Phomene} dataset. 
However, this setting exhibits an unusual pattern, as the highest model accuracy is achieved without any regularization. Even in this case, CF-Reg still surpasses other regularization baselines.

From the results above, we derive the following key insights. First, CF-Reg proves to be effective across various model types, ranging from simple linear models (Logistic Regression) to deep architectures like MLPs and CNNs, and across diverse datasets, including both tabular and image data. 
Second, CF-Reg's strong performance on the \textit{Water} dataset with Logistic Regression suggests that its benefits may be more pronounced when applied to simpler models. However, the unexpected outcome on the \textit{Phoneme} dataset calls for further investigation into this phenomenon.


\begin{table*}[h!]
    \centering
    \caption{Mean value and standard deviation of test accuracy across 5 random initializations for different model, dataset, and regularization method. The best results are highlighted in \textbf{bold}.}
    \label{tab:regularization_comparison}
    \begin{tabular}{|c|c|c|c|c|c|c|}
        \hline
        \textbf{Model} & \textbf{Dataset} & \textbf{No-Reg} & \textbf{L1-Reg} & \textbf{L2-Reg} & \textbf{Dropout} & \textbf{CF-Reg (ours)} \\ \hline
        Logistic Regression   & \textit{Water}   & $0.6595 \pm 0.0038$   & $0.6729 \pm 0.0056$   & $0.6756 \pm 0.0046$  & N/A    & $\mathbf{0.6918 \pm 0.0036}$                     \\ \hline
        MLP   & \textit{Water}   & $0.6756 \pm 0.0042$   & $0.6790 \pm 0.0058$   & $0.6790 \pm 0.0023$  & $0.6750 \pm 0.0036$    & $\mathbf{0.6802 \pm 0.0046}$                    \\ \hline
%        MLP   & \textit{Adult}   & $0.8404 \pm 0.0010$   & $\mathbf{0.8495 \pm 0.0007}$   & $0.8489 \pm 0.0014$  & $\mathbf{0.8495 \pm 0.0016}$     & $0.8449 \pm 0.0019$                    \\ \hline
        Logistic Regression   & \textit{Phomene}   & $\mathbf{0.8148 \pm 0.0020}$   & $0.8041 \pm 0.0028$   & $0.7835 \pm 0.0176$  & N/A    & $0.8098 \pm 0.0055$                     \\ \hline
        MLP   & \textit{Phomene}   & $0.8677 \pm 0.0033$   & $0.8374 \pm 0.0080$   & $0.8673 \pm 0.0045$  & $0.8672 \pm 0.0042$     & $\mathbf{0.8718 \pm 0.0040}$                    \\ \hline
        CNN   & \textit{CIFAR-10} & $0.6670 \pm 0.0233$   & $0.6229 \pm 0.0850$   & $0.7348 \pm 0.0365$   & N/A    & $\mathbf{0.7427 \pm 0.0571}$                     \\ \hline
    \end{tabular}
\end{table*}

\begin{table*}[htb!]
    \centering
    \caption{Hyperparameter configurations utilized for the generation of Table \ref{tab:regularization_comparison}. For our regularization the hyperparameters are reported as $\mathbf{\alpha/\beta}$.}
    \label{tab:performance_parameters}
    \begin{tabular}{|c|c|c|c|c|c|c|}
        \hline
        \textbf{Model} & \textbf{Dataset} & \textbf{No-Reg} & \textbf{L1-Reg} & \textbf{L2-Reg} & \textbf{Dropout} & \textbf{CF-Reg (ours)} \\ \hline
        Logistic Regression   & \textit{Water}   & N/A   & $0.0093$   & $0.6927$  & N/A    & $0.3791/1.0355$                     \\ \hline
        MLP   & \textit{Water}   & N/A   & $0.0007$   & $0.0022$  & $0.0002$    & $0.2567/1.9775$                    \\ \hline
        Logistic Regression   &
        \textit{Phomene}   & N/A   & $0.0097$   & $0.7979$  & N/A    & $0.0571/1.8516$                     \\ \hline
        MLP   & \textit{Phomene}   & N/A   & $0.0007$   & $4.24\cdot10^{-5}$  & $0.0015$    & $0.0516/2.2700$                    \\ \hline
       % MLP   & \textit{Adult}   & N/A   & $0.0018$   & $0.0018$  & $0.0601$     & $0.0764/2.2068$                    \\ \hline
        CNN   & \textit{CIFAR-10} & N/A   & $0.0050$   & $0.0864$ & N/A    & $0.3018/
        2.1502$                     \\ \hline
    \end{tabular}
\end{table*}

\begin{table*}[htb!]
    \centering
    \caption{Mean value and standard deviation of training time across 5 different runs. The reported time (in seconds) corresponds to the generation of each entry in Table \ref{tab:regularization_comparison}. Times are }
    \label{tab:times}
    \begin{tabular}{|c|c|c|c|c|c|c|}
        \hline
        \textbf{Model} & \textbf{Dataset} & \textbf{No-Reg} & \textbf{L1-Reg} & \textbf{L2-Reg} & \textbf{Dropout} & \textbf{CF-Reg (ours)} \\ \hline
        Logistic Regression   & \textit{Water}   & $222.98 \pm 1.07$   & $239.94 \pm 2.59$   & $241.60 \pm 1.88$  & N/A    & $251.50 \pm 1.93$                     \\ \hline
        MLP   & \textit{Water}   & $225.71 \pm 3.85$   & $250.13 \pm 4.44$   & $255.78 \pm 2.38$  & $237.83 \pm 3.45$    & $266.48 \pm 3.46$                    \\ \hline
        Logistic Regression   & \textit{Phomene}   & $266.39 \pm 0.82$ & $367.52 \pm 6.85$   & $361.69 \pm 4.04$  & N/A   & $310.48 \pm 0.76$                    \\ \hline
        MLP   &
        \textit{Phomene} & $335.62 \pm 1.77$   & $390.86 \pm 2.11$   & $393.96 \pm 1.95$ & $363.51 \pm 5.07$    & $403.14 \pm 1.92$                     \\ \hline
       % MLP   & \textit{Adult}   & N/A   & $0.0018$   & $0.0018$  & $0.0601$     & $0.0764/2.2068$                    \\ \hline
        CNN   & \textit{CIFAR-10} & $370.09 \pm 0.18$   & $395.71 \pm 0.55$   & $401.38 \pm 0.16$ & N/A    & $1287.8 \pm 0.26$                     \\ \hline
    \end{tabular}
\end{table*}

\subsection{Feasibility of our Method}
A crucial requirement for any regularization technique is that it should impose minimal impact on the overall training process.
In this respect, CF-Reg introduces an overhead that depends on the time required to find the optimal counterfactual example for each training instance. 
As such, the more sophisticated the counterfactual generator model probed during training the higher would be the time required. However, a more advanced counterfactual generator might provide a more effective regularization. We discuss this trade-off in more details in Section~\ref{sec:discussion}.

Table~\ref{tab:times} presents the average training time ($\pm$ standard deviation) for each model and dataset combination listed in Table~\ref{tab:regularization_comparison}.
We can observe that the higher accuracy achieved by CF-Reg using the score-based counterfactual generator comes with only minimal overhead. However, when applied to deep neural networks with many hidden layers, such as \textit{PreactResNet-18}, the forward derivative computation required for the linearization of the network introduces a more noticeable computational cost, explaining the longer training times in the table.

\subsection{Hyperparameter Sensitivity Analysis}
The proposed counterfactual regularization technique relies on two key hyperparameters: $\alpha$ and $\beta$. The former is intrinsic to the loss formulation defined in (\ref{eq:cf-train}), while the latter is closely tied to the choice of the score-based counterfactual explanation method used.

Figure~\ref{fig:test_alpha_beta} illustrates how the test accuracy of an MLP trained on the \textit{Water Potability} dataset changes for different combinations of $\alpha$ and $\beta$.

\begin{figure}[ht]
    \centering
    \includegraphics[width=0.85\linewidth]{img/test_acc_alpha_beta.png}
    \caption{The test accuracy of an MLP trained on the \textit{Water Potability} dataset, evaluated while varying the weight of our counterfactual regularizer ($\alpha$) for different values of $\beta$.}
    \label{fig:test_alpha_beta}
\end{figure}

We observe that, for a fixed $\beta$, increasing the weight of our counterfactual regularizer ($\alpha$) can slightly improve test accuracy until a sudden drop is noticed for $\alpha > 0.1$.
This behavior was expected, as the impact of our penalty, like any regularization term, can be disruptive if not properly controlled.

Moreover, this finding further demonstrates that our regularization method, CF-Reg, is inherently data-driven. Therefore, it requires specific fine-tuning based on the combination of the model and dataset at hand.
\section{Conclusion}
In this work, we propose a simple yet effective approach, called SMILE, for graph few-shot learning with fewer tasks. Specifically, we introduce a novel dual-level mixup strategy, including within-task and across-task mixup, for enriching the diversity of nodes within each task and the diversity of tasks. Also, we incorporate the degree-based prior information to learn expressive node embeddings. Theoretically, we prove that SMILE effectively enhances the model's generalization performance. Empirically, we conduct extensive experiments on multiple benchmarks and the results suggest that SMILE significantly outperforms other baselines, including both in-domain and cross-domain few-shot settings.




% In the unusual situation where you want a paper to appear in the
% references without citing it in the main text, use \nocite
% \nocite{langley00}

\clearpage
\newpage
% \section*{Impact Statement}
\system advances cost-efficient AI by demonstrating how small on-device language models can collaborate with powerful cloud-hosted models to perform data-intensive reasoning. By reducing reliance on expensive remote inference, \system makes advanced AI more accessible and sustainable. This has broad societal implications, including lowering barriers to AI adoption and enhancing data privacy by keeping more computations local. However, careful consideration must be given to potential biases in small models and the security risks of local code execution. 
\bibliography{example_paper}
\bibliographystyle{plainnat}


%%%%%%%%%%%%%%%%%%%%%%%%%%%%%%%%%%%%%%%%%%%%%%%%%%%%%%%%%%%%%%%%%%%%%%%%%%%%%%%
%%%%%%%%%%%%%%%%%%%%%%%%%%%%%%%%%%%%%%%%%%%%%%%%%%%%%%%%%%%%%%%%%%%%%%%%%%%%%%%
% APPENDIX
%%%%%%%%%%%%%%%%%%%%%%%%%%%%%%%%%%%%%%%%%%%%%%%%%%%%%%%%%%%%%%%%%%%%%%%%%%%%%%%
%%%%%%%%%%%%%%%%%%%%%%%%%%%%%%%%%%%%%%%%%%%%%%%%%%%%%%%%%%%%%%%%%%%%%%%%%%%%%%%
\newpage
% \appendix
% \onecolumn
% \subsection{Lloyd-Max Algorithm}
\label{subsec:Lloyd-Max}
For a given quantization bitwidth $B$ and an operand $\bm{X}$, the Lloyd-Max algorithm finds $2^B$ quantization levels $\{\hat{x}_i\}_{i=1}^{2^B}$ such that quantizing $\bm{X}$ by rounding each scalar in $\bm{X}$ to the nearest quantization level minimizes the quantization MSE. 

The algorithm starts with an initial guess of quantization levels and then iteratively computes quantization thresholds $\{\tau_i\}_{i=1}^{2^B-1}$ and updates quantization levels $\{\hat{x}_i\}_{i=1}^{2^B}$. Specifically, at iteration $n$, thresholds are set to the midpoints of the previous iteration's levels:
\begin{align*}
    \tau_i^{(n)}=\frac{\hat{x}_i^{(n-1)}+\hat{x}_{i+1}^{(n-1)}}2 \text{ for } i=1\ldots 2^B-1
\end{align*}
Subsequently, the quantization levels are re-computed as conditional means of the data regions defined by the new thresholds:
\begin{align*}
    \hat{x}_i^{(n)}=\mathbb{E}\left[ \bm{X} \big| \bm{X}\in [\tau_{i-1}^{(n)},\tau_i^{(n)}] \right] \text{ for } i=1\ldots 2^B
\end{align*}
where to satisfy boundary conditions we have $\tau_0=-\infty$ and $\tau_{2^B}=\infty$. The algorithm iterates the above steps until convergence.

Figure \ref{fig:lm_quant} compares the quantization levels of a $7$-bit floating point (E3M3) quantizer (left) to a $7$-bit Lloyd-Max quantizer (right) when quantizing a layer of weights from the GPT3-126M model at a per-tensor granularity. As shown, the Lloyd-Max quantizer achieves substantially lower quantization MSE. Further, Table \ref{tab:FP7_vs_LM7} shows the superior perplexity achieved by Lloyd-Max quantizers for bitwidths of $7$, $6$ and $5$. The difference between the quantizers is clear at 5 bits, where per-tensor FP quantization incurs a drastic and unacceptable increase in perplexity, while Lloyd-Max quantization incurs a much smaller increase. Nevertheless, we note that even the optimal Lloyd-Max quantizer incurs a notable ($\sim 1.5$) increase in perplexity due to the coarse granularity of quantization. 

\begin{figure}[h]
  \centering
  \includegraphics[width=0.7\linewidth]{sections/figures/LM7_FP7.pdf}
  \caption{\small Quantization levels and the corresponding quantization MSE of Floating Point (left) vs Lloyd-Max (right) Quantizers for a layer of weights in the GPT3-126M model.}
  \label{fig:lm_quant}
\end{figure}

\begin{table}[h]\scriptsize
\begin{center}
\caption{\label{tab:FP7_vs_LM7} \small Comparing perplexity (lower is better) achieved by floating point quantizers and Lloyd-Max quantizers on a GPT3-126M model for the Wikitext-103 dataset.}
\begin{tabular}{c|cc|c}
\hline
 \multirow{2}{*}{\textbf{Bitwidth}} & \multicolumn{2}{|c|}{\textbf{Floating-Point Quantizer}} & \textbf{Lloyd-Max Quantizer} \\
 & Best Format & Wikitext-103 Perplexity & Wikitext-103 Perplexity \\
\hline
7 & E3M3 & 18.32 & 18.27 \\
6 & E3M2 & 19.07 & 18.51 \\
5 & E4M0 & 43.89 & 19.71 \\
\hline
\end{tabular}
\end{center}
\end{table}

\subsection{Proof of Local Optimality of LO-BCQ}
\label{subsec:lobcq_opt_proof}
For a given block $\bm{b}_j$, the quantization MSE during LO-BCQ can be empirically evaluated as $\frac{1}{L_b}\lVert \bm{b}_j- \bm{\hat{b}}_j\rVert^2_2$ where $\bm{\hat{b}}_j$ is computed from equation (\ref{eq:clustered_quantization_definition}) as $C_{f(\bm{b}_j)}(\bm{b}_j)$. Further, for a given block cluster $\mathcal{B}_i$, we compute the quantization MSE as $\frac{1}{|\mathcal{B}_{i}|}\sum_{\bm{b} \in \mathcal{B}_{i}} \frac{1}{L_b}\lVert \bm{b}- C_i^{(n)}(\bm{b})\rVert^2_2$. Therefore, at the end of iteration $n$, we evaluate the overall quantization MSE $J^{(n)}$ for a given operand $\bm{X}$ composed of $N_c$ block clusters as:
\begin{align*}
    \label{eq:mse_iter_n}
    J^{(n)} = \frac{1}{N_c} \sum_{i=1}^{N_c} \frac{1}{|\mathcal{B}_{i}^{(n)}|}\sum_{\bm{v} \in \mathcal{B}_{i}^{(n)}} \frac{1}{L_b}\lVert \bm{b}- B_i^{(n)}(\bm{b})\rVert^2_2
\end{align*}

At the end of iteration $n$, the codebooks are updated from $\mathcal{C}^{(n-1)}$ to $\mathcal{C}^{(n)}$. However, the mapping of a given vector $\bm{b}_j$ to quantizers $\mathcal{C}^{(n)}$ remains as  $f^{(n)}(\bm{b}_j)$. At the next iteration, during the vector clustering step, $f^{(n+1)}(\bm{b}_j)$ finds new mapping of $\bm{b}_j$ to updated codebooks $\mathcal{C}^{(n)}$ such that the quantization MSE over the candidate codebooks is minimized. Therefore, we obtain the following result for $\bm{b}_j$:
\begin{align*}
\frac{1}{L_b}\lVert \bm{b}_j - C_{f^{(n+1)}(\bm{b}_j)}^{(n)}(\bm{b}_j)\rVert^2_2 \le \frac{1}{L_b}\lVert \bm{b}_j - C_{f^{(n)}(\bm{b}_j)}^{(n)}(\bm{b}_j)\rVert^2_2
\end{align*}

That is, quantizing $\bm{b}_j$ at the end of the block clustering step of iteration $n+1$ results in lower quantization MSE compared to quantizing at the end of iteration $n$. Since this is true for all $\bm{b} \in \bm{X}$, we assert the following:
\begin{equation}
\begin{split}
\label{eq:mse_ineq_1}
    \tilde{J}^{(n+1)} &= \frac{1}{N_c} \sum_{i=1}^{N_c} \frac{1}{|\mathcal{B}_{i}^{(n+1)}|}\sum_{\bm{b} \in \mathcal{B}_{i}^{(n+1)}} \frac{1}{L_b}\lVert \bm{b} - C_i^{(n)}(b)\rVert^2_2 \le J^{(n)}
\end{split}
\end{equation}
where $\tilde{J}^{(n+1)}$ is the the quantization MSE after the vector clustering step at iteration $n+1$.

Next, during the codebook update step (\ref{eq:quantizers_update}) at iteration $n+1$, the per-cluster codebooks $\mathcal{C}^{(n)}$ are updated to $\mathcal{C}^{(n+1)}$ by invoking the Lloyd-Max algorithm \citep{Lloyd}. We know that for any given value distribution, the Lloyd-Max algorithm minimizes the quantization MSE. Therefore, for a given vector cluster $\mathcal{B}_i$ we obtain the following result:

\begin{equation}
    \frac{1}{|\mathcal{B}_{i}^{(n+1)}|}\sum_{\bm{b} \in \mathcal{B}_{i}^{(n+1)}} \frac{1}{L_b}\lVert \bm{b}- C_i^{(n+1)}(\bm{b})\rVert^2_2 \le \frac{1}{|\mathcal{B}_{i}^{(n+1)}|}\sum_{\bm{b} \in \mathcal{B}_{i}^{(n+1)}} \frac{1}{L_b}\lVert \bm{b}- C_i^{(n)}(\bm{b})\rVert^2_2
\end{equation}

The above equation states that quantizing the given block cluster $\mathcal{B}_i$ after updating the associated codebook from $C_i^{(n)}$ to $C_i^{(n+1)}$ results in lower quantization MSE. Since this is true for all the block clusters, we derive the following result: 
\begin{equation}
\begin{split}
\label{eq:mse_ineq_2}
     J^{(n+1)} &= \frac{1}{N_c} \sum_{i=1}^{N_c} \frac{1}{|\mathcal{B}_{i}^{(n+1)}|}\sum_{\bm{b} \in \mathcal{B}_{i}^{(n+1)}} \frac{1}{L_b}\lVert \bm{b}- C_i^{(n+1)}(\bm{b})\rVert^2_2  \le \tilde{J}^{(n+1)}   
\end{split}
\end{equation}

Following (\ref{eq:mse_ineq_1}) and (\ref{eq:mse_ineq_2}), we find that the quantization MSE is non-increasing for each iteration, that is, $J^{(1)} \ge J^{(2)} \ge J^{(3)} \ge \ldots \ge J^{(M)}$ where $M$ is the maximum number of iterations. 
%Therefore, we can say that if the algorithm converges, then it must be that it has converged to a local minimum. 
\hfill $\blacksquare$


\begin{figure}
    \begin{center}
    \includegraphics[width=0.5\textwidth]{sections//figures/mse_vs_iter.pdf}
    \end{center}
    \caption{\small NMSE vs iterations during LO-BCQ compared to other block quantization proposals}
    \label{fig:nmse_vs_iter}
\end{figure}

Figure \ref{fig:nmse_vs_iter} shows the empirical convergence of LO-BCQ across several block lengths and number of codebooks. Also, the MSE achieved by LO-BCQ is compared to baselines such as MXFP and VSQ. As shown, LO-BCQ converges to a lower MSE than the baselines. Further, we achieve better convergence for larger number of codebooks ($N_c$) and for a smaller block length ($L_b$), both of which increase the bitwidth of BCQ (see Eq \ref{eq:bitwidth_bcq}).


\subsection{Additional Accuracy Results}
%Table \ref{tab:lobcq_config} lists the various LOBCQ configurations and their corresponding bitwidths.
\begin{table}
\setlength{\tabcolsep}{4.75pt}
\begin{center}
\caption{\label{tab:lobcq_config} Various LO-BCQ configurations and their bitwidths.}
\begin{tabular}{|c||c|c|c|c||c|c||c|} 
\hline
 & \multicolumn{4}{|c||}{$L_b=8$} & \multicolumn{2}{|c||}{$L_b=4$} & $L_b=2$ \\
 \hline
 \backslashbox{$L_A$\kern-1em}{\kern-1em$N_c$} & 2 & 4 & 8 & 16 & 2 & 4 & 2 \\
 \hline
 64 & 4.25 & 4.375 & 4.5 & 4.625 & 4.375 & 4.625 & 4.625\\
 \hline
 32 & 4.375 & 4.5 & 4.625& 4.75 & 4.5 & 4.75 & 4.75 \\
 \hline
 16 & 4.625 & 4.75& 4.875 & 5 & 4.75 & 5 & 5 \\
 \hline
\end{tabular}
\end{center}
\end{table}

%\subsection{Perplexity achieved by various LO-BCQ configurations on Wikitext-103 dataset}

\begin{table} \centering
\begin{tabular}{|c||c|c|c|c||c|c||c|} 
\hline
 $L_b \rightarrow$& \multicolumn{4}{c||}{8} & \multicolumn{2}{c||}{4} & 2\\
 \hline
 \backslashbox{$L_A$\kern-1em}{\kern-1em$N_c$} & 2 & 4 & 8 & 16 & 2 & 4 & 2  \\
 %$N_c \rightarrow$ & 2 & 4 & 8 & 16 & 2 & 4 & 2 \\
 \hline
 \hline
 \multicolumn{8}{c}{GPT3-1.3B (FP32 PPL = 9.98)} \\ 
 \hline
 \hline
 64 & 10.40 & 10.23 & 10.17 & 10.15 &  10.28 & 10.18 & 10.19 \\
 \hline
 32 & 10.25 & 10.20 & 10.15 & 10.12 &  10.23 & 10.17 & 10.17 \\
 \hline
 16 & 10.22 & 10.16 & 10.10 & 10.09 &  10.21 & 10.14 & 10.16 \\
 \hline
  \hline
 \multicolumn{8}{c}{GPT3-8B (FP32 PPL = 7.38)} \\ 
 \hline
 \hline
 64 & 7.61 & 7.52 & 7.48 &  7.47 &  7.55 &  7.49 & 7.50 \\
 \hline
 32 & 7.52 & 7.50 & 7.46 &  7.45 &  7.52 &  7.48 & 7.48  \\
 \hline
 16 & 7.51 & 7.48 & 7.44 &  7.44 &  7.51 &  7.49 & 7.47  \\
 \hline
\end{tabular}
\caption{\label{tab:ppl_gpt3_abalation} Wikitext-103 perplexity across GPT3-1.3B and 8B models.}
\end{table}

\begin{table} \centering
\begin{tabular}{|c||c|c|c|c||} 
\hline
 $L_b \rightarrow$& \multicolumn{4}{c||}{8}\\
 \hline
 \backslashbox{$L_A$\kern-1em}{\kern-1em$N_c$} & 2 & 4 & 8 & 16 \\
 %$N_c \rightarrow$ & 2 & 4 & 8 & 16 & 2 & 4 & 2 \\
 \hline
 \hline
 \multicolumn{5}{|c|}{Llama2-7B (FP32 PPL = 5.06)} \\ 
 \hline
 \hline
 64 & 5.31 & 5.26 & 5.19 & 5.18  \\
 \hline
 32 & 5.23 & 5.25 & 5.18 & 5.15  \\
 \hline
 16 & 5.23 & 5.19 & 5.16 & 5.14  \\
 \hline
 \multicolumn{5}{|c|}{Nemotron4-15B (FP32 PPL = 5.87)} \\ 
 \hline
 \hline
 64  & 6.3 & 6.20 & 6.13 & 6.08  \\
 \hline
 32  & 6.24 & 6.12 & 6.07 & 6.03  \\
 \hline
 16  & 6.12 & 6.14 & 6.04 & 6.02  \\
 \hline
 \multicolumn{5}{|c|}{Nemotron4-340B (FP32 PPL = 3.48)} \\ 
 \hline
 \hline
 64 & 3.67 & 3.62 & 3.60 & 3.59 \\
 \hline
 32 & 3.63 & 3.61 & 3.59 & 3.56 \\
 \hline
 16 & 3.61 & 3.58 & 3.57 & 3.55 \\
 \hline
\end{tabular}
\caption{\label{tab:ppl_llama7B_nemo15B} Wikitext-103 perplexity compared to FP32 baseline in Llama2-7B and Nemotron4-15B, 340B models}
\end{table}

%\subsection{Perplexity achieved by various LO-BCQ configurations on MMLU dataset}


\begin{table} \centering
\begin{tabular}{|c||c|c|c|c||c|c|c|c|} 
\hline
 $L_b \rightarrow$& \multicolumn{4}{c||}{8} & \multicolumn{4}{c||}{8}\\
 \hline
 \backslashbox{$L_A$\kern-1em}{\kern-1em$N_c$} & 2 & 4 & 8 & 16 & 2 & 4 & 8 & 16  \\
 %$N_c \rightarrow$ & 2 & 4 & 8 & 16 & 2 & 4 & 2 \\
 \hline
 \hline
 \multicolumn{5}{|c|}{Llama2-7B (FP32 Accuracy = 45.8\%)} & \multicolumn{4}{|c|}{Llama2-70B (FP32 Accuracy = 69.12\%)} \\ 
 \hline
 \hline
 64 & 43.9 & 43.4 & 43.9 & 44.9 & 68.07 & 68.27 & 68.17 & 68.75 \\
 \hline
 32 & 44.5 & 43.8 & 44.9 & 44.5 & 68.37 & 68.51 & 68.35 & 68.27  \\
 \hline
 16 & 43.9 & 42.7 & 44.9 & 45 & 68.12 & 68.77 & 68.31 & 68.59  \\
 \hline
 \hline
 \multicolumn{5}{|c|}{GPT3-22B (FP32 Accuracy = 38.75\%)} & \multicolumn{4}{|c|}{Nemotron4-15B (FP32 Accuracy = 64.3\%)} \\ 
 \hline
 \hline
 64 & 36.71 & 38.85 & 38.13 & 38.92 & 63.17 & 62.36 & 63.72 & 64.09 \\
 \hline
 32 & 37.95 & 38.69 & 39.45 & 38.34 & 64.05 & 62.30 & 63.8 & 64.33  \\
 \hline
 16 & 38.88 & 38.80 & 38.31 & 38.92 & 63.22 & 63.51 & 63.93 & 64.43  \\
 \hline
\end{tabular}
\caption{\label{tab:mmlu_abalation} Accuracy on MMLU dataset across GPT3-22B, Llama2-7B, 70B and Nemotron4-15B models.}
\end{table}


%\subsection{Perplexity achieved by various LO-BCQ configurations on LM evaluation harness}

\begin{table} \centering
\begin{tabular}{|c||c|c|c|c||c|c|c|c|} 
\hline
 $L_b \rightarrow$& \multicolumn{4}{c||}{8} & \multicolumn{4}{c||}{8}\\
 \hline
 \backslashbox{$L_A$\kern-1em}{\kern-1em$N_c$} & 2 & 4 & 8 & 16 & 2 & 4 & 8 & 16  \\
 %$N_c \rightarrow$ & 2 & 4 & 8 & 16 & 2 & 4 & 2 \\
 \hline
 \hline
 \multicolumn{5}{|c|}{Race (FP32 Accuracy = 37.51\%)} & \multicolumn{4}{|c|}{Boolq (FP32 Accuracy = 64.62\%)} \\ 
 \hline
 \hline
 64 & 36.94 & 37.13 & 36.27 & 37.13 & 63.73 & 62.26 & 63.49 & 63.36 \\
 \hline
 32 & 37.03 & 36.36 & 36.08 & 37.03 & 62.54 & 63.51 & 63.49 & 63.55  \\
 \hline
 16 & 37.03 & 37.03 & 36.46 & 37.03 & 61.1 & 63.79 & 63.58 & 63.33  \\
 \hline
 \hline
 \multicolumn{5}{|c|}{Winogrande (FP32 Accuracy = 58.01\%)} & \multicolumn{4}{|c|}{Piqa (FP32 Accuracy = 74.21\%)} \\ 
 \hline
 \hline
 64 & 58.17 & 57.22 & 57.85 & 58.33 & 73.01 & 73.07 & 73.07 & 72.80 \\
 \hline
 32 & 59.12 & 58.09 & 57.85 & 58.41 & 73.01 & 73.94 & 72.74 & 73.18  \\
 \hline
 16 & 57.93 & 58.88 & 57.93 & 58.56 & 73.94 & 72.80 & 73.01 & 73.94  \\
 \hline
\end{tabular}
\caption{\label{tab:mmlu_abalation} Accuracy on LM evaluation harness tasks on GPT3-1.3B model.}
\end{table}

\begin{table} \centering
\begin{tabular}{|c||c|c|c|c||c|c|c|c|} 
\hline
 $L_b \rightarrow$& \multicolumn{4}{c||}{8} & \multicolumn{4}{c||}{8}\\
 \hline
 \backslashbox{$L_A$\kern-1em}{\kern-1em$N_c$} & 2 & 4 & 8 & 16 & 2 & 4 & 8 & 16  \\
 %$N_c \rightarrow$ & 2 & 4 & 8 & 16 & 2 & 4 & 2 \\
 \hline
 \hline
 \multicolumn{5}{|c|}{Race (FP32 Accuracy = 41.34\%)} & \multicolumn{4}{|c|}{Boolq (FP32 Accuracy = 68.32\%)} \\ 
 \hline
 \hline
 64 & 40.48 & 40.10 & 39.43 & 39.90 & 69.20 & 68.41 & 69.45 & 68.56 \\
 \hline
 32 & 39.52 & 39.52 & 40.77 & 39.62 & 68.32 & 67.43 & 68.17 & 69.30  \\
 \hline
 16 & 39.81 & 39.71 & 39.90 & 40.38 & 68.10 & 66.33 & 69.51 & 69.42  \\
 \hline
 \hline
 \multicolumn{5}{|c|}{Winogrande (FP32 Accuracy = 67.88\%)} & \multicolumn{4}{|c|}{Piqa (FP32 Accuracy = 78.78\%)} \\ 
 \hline
 \hline
 64 & 66.85 & 66.61 & 67.72 & 67.88 & 77.31 & 77.42 & 77.75 & 77.64 \\
 \hline
 32 & 67.25 & 67.72 & 67.72 & 67.00 & 77.31 & 77.04 & 77.80 & 77.37  \\
 \hline
 16 & 68.11 & 68.90 & 67.88 & 67.48 & 77.37 & 78.13 & 78.13 & 77.69  \\
 \hline
\end{tabular}
\caption{\label{tab:mmlu_abalation} Accuracy on LM evaluation harness tasks on GPT3-8B model.}
\end{table}

\begin{table} \centering
\begin{tabular}{|c||c|c|c|c||c|c|c|c|} 
\hline
 $L_b \rightarrow$& \multicolumn{4}{c||}{8} & \multicolumn{4}{c||}{8}\\
 \hline
 \backslashbox{$L_A$\kern-1em}{\kern-1em$N_c$} & 2 & 4 & 8 & 16 & 2 & 4 & 8 & 16  \\
 %$N_c \rightarrow$ & 2 & 4 & 8 & 16 & 2 & 4 & 2 \\
 \hline
 \hline
 \multicolumn{5}{|c|}{Race (FP32 Accuracy = 40.67\%)} & \multicolumn{4}{|c|}{Boolq (FP32 Accuracy = 76.54\%)} \\ 
 \hline
 \hline
 64 & 40.48 & 40.10 & 39.43 & 39.90 & 75.41 & 75.11 & 77.09 & 75.66 \\
 \hline
 32 & 39.52 & 39.52 & 40.77 & 39.62 & 76.02 & 76.02 & 75.96 & 75.35  \\
 \hline
 16 & 39.81 & 39.71 & 39.90 & 40.38 & 75.05 & 73.82 & 75.72 & 76.09  \\
 \hline
 \hline
 \multicolumn{5}{|c|}{Winogrande (FP32 Accuracy = 70.64\%)} & \multicolumn{4}{|c|}{Piqa (FP32 Accuracy = 79.16\%)} \\ 
 \hline
 \hline
 64 & 69.14 & 70.17 & 70.17 & 70.56 & 78.24 & 79.00 & 78.62 & 78.73 \\
 \hline
 32 & 70.96 & 69.69 & 71.27 & 69.30 & 78.56 & 79.49 & 79.16 & 78.89  \\
 \hline
 16 & 71.03 & 69.53 & 69.69 & 70.40 & 78.13 & 79.16 & 79.00 & 79.00  \\
 \hline
\end{tabular}
\caption{\label{tab:mmlu_abalation} Accuracy on LM evaluation harness tasks on GPT3-22B model.}
\end{table}

\begin{table} \centering
\begin{tabular}{|c||c|c|c|c||c|c|c|c|} 
\hline
 $L_b \rightarrow$& \multicolumn{4}{c||}{8} & \multicolumn{4}{c||}{8}\\
 \hline
 \backslashbox{$L_A$\kern-1em}{\kern-1em$N_c$} & 2 & 4 & 8 & 16 & 2 & 4 & 8 & 16  \\
 %$N_c \rightarrow$ & 2 & 4 & 8 & 16 & 2 & 4 & 2 \\
 \hline
 \hline
 \multicolumn{5}{|c|}{Race (FP32 Accuracy = 44.4\%)} & \multicolumn{4}{|c|}{Boolq (FP32 Accuracy = 79.29\%)} \\ 
 \hline
 \hline
 64 & 42.49 & 42.51 & 42.58 & 43.45 & 77.58 & 77.37 & 77.43 & 78.1 \\
 \hline
 32 & 43.35 & 42.49 & 43.64 & 43.73 & 77.86 & 75.32 & 77.28 & 77.86  \\
 \hline
 16 & 44.21 & 44.21 & 43.64 & 42.97 & 78.65 & 77 & 76.94 & 77.98  \\
 \hline
 \hline
 \multicolumn{5}{|c|}{Winogrande (FP32 Accuracy = 69.38\%)} & \multicolumn{4}{|c|}{Piqa (FP32 Accuracy = 78.07\%)} \\ 
 \hline
 \hline
 64 & 68.9 & 68.43 & 69.77 & 68.19 & 77.09 & 76.82 & 77.09 & 77.86 \\
 \hline
 32 & 69.38 & 68.51 & 68.82 & 68.90 & 78.07 & 76.71 & 78.07 & 77.86  \\
 \hline
 16 & 69.53 & 67.09 & 69.38 & 68.90 & 77.37 & 77.8 & 77.91 & 77.69  \\
 \hline
\end{tabular}
\caption{\label{tab:mmlu_abalation} Accuracy on LM evaluation harness tasks on Llama2-7B model.}
\end{table}

\begin{table} \centering
\begin{tabular}{|c||c|c|c|c||c|c|c|c|} 
\hline
 $L_b \rightarrow$& \multicolumn{4}{c||}{8} & \multicolumn{4}{c||}{8}\\
 \hline
 \backslashbox{$L_A$\kern-1em}{\kern-1em$N_c$} & 2 & 4 & 8 & 16 & 2 & 4 & 8 & 16  \\
 %$N_c \rightarrow$ & 2 & 4 & 8 & 16 & 2 & 4 & 2 \\
 \hline
 \hline
 \multicolumn{5}{|c|}{Race (FP32 Accuracy = 48.8\%)} & \multicolumn{4}{|c|}{Boolq (FP32 Accuracy = 85.23\%)} \\ 
 \hline
 \hline
 64 & 49.00 & 49.00 & 49.28 & 48.71 & 82.82 & 84.28 & 84.03 & 84.25 \\
 \hline
 32 & 49.57 & 48.52 & 48.33 & 49.28 & 83.85 & 84.46 & 84.31 & 84.93  \\
 \hline
 16 & 49.85 & 49.09 & 49.28 & 48.99 & 85.11 & 84.46 & 84.61 & 83.94  \\
 \hline
 \hline
 \multicolumn{5}{|c|}{Winogrande (FP32 Accuracy = 79.95\%)} & \multicolumn{4}{|c|}{Piqa (FP32 Accuracy = 81.56\%)} \\ 
 \hline
 \hline
 64 & 78.77 & 78.45 & 78.37 & 79.16 & 81.45 & 80.69 & 81.45 & 81.5 \\
 \hline
 32 & 78.45 & 79.01 & 78.69 & 80.66 & 81.56 & 80.58 & 81.18 & 81.34  \\
 \hline
 16 & 79.95 & 79.56 & 79.79 & 79.72 & 81.28 & 81.66 & 81.28 & 80.96  \\
 \hline
\end{tabular}
\caption{\label{tab:mmlu_abalation} Accuracy on LM evaluation harness tasks on Llama2-70B model.}
\end{table}

%\section{MSE Studies}
%\textcolor{red}{TODO}


\subsection{Number Formats and Quantization Method}
\label{subsec:numFormats_quantMethod}
\subsubsection{Integer Format}
An $n$-bit signed integer (INT) is typically represented with a 2s-complement format \citep{yao2022zeroquant,xiao2023smoothquant,dai2021vsq}, where the most significant bit denotes the sign.

\subsubsection{Floating Point Format}
An $n$-bit signed floating point (FP) number $x$ comprises of a 1-bit sign ($x_{\mathrm{sign}}$), $B_m$-bit mantissa ($x_{\mathrm{mant}}$) and $B_e$-bit exponent ($x_{\mathrm{exp}}$) such that $B_m+B_e=n-1$. The associated constant exponent bias ($E_{\mathrm{bias}}$) is computed as $(2^{{B_e}-1}-1)$. We denote this format as $E_{B_e}M_{B_m}$.  

\subsubsection{Quantization Scheme}
\label{subsec:quant_method}
A quantization scheme dictates how a given unquantized tensor is converted to its quantized representation. We consider FP formats for the purpose of illustration. Given an unquantized tensor $\bm{X}$ and an FP format $E_{B_e}M_{B_m}$, we first, we compute the quantization scale factor $s_X$ that maps the maximum absolute value of $\bm{X}$ to the maximum quantization level of the $E_{B_e}M_{B_m}$ format as follows:
\begin{align}
\label{eq:sf}
    s_X = \frac{\mathrm{max}(|\bm{X}|)}{\mathrm{max}(E_{B_e}M_{B_m})}
\end{align}
In the above equation, $|\cdot|$ denotes the absolute value function.

Next, we scale $\bm{X}$ by $s_X$ and quantize it to $\hat{\bm{X}}$ by rounding it to the nearest quantization level of $E_{B_e}M_{B_m}$ as:

\begin{align}
\label{eq:tensor_quant}
    \hat{\bm{X}} = \text{round-to-nearest}\left(\frac{\bm{X}}{s_X}, E_{B_e}M_{B_m}\right)
\end{align}

We perform dynamic max-scaled quantization \citep{wu2020integer}, where the scale factor $s$ for activations is dynamically computed during runtime.

\subsection{Vector Scaled Quantization}
\begin{wrapfigure}{r}{0.35\linewidth}
  \centering
  \includegraphics[width=\linewidth]{sections/figures/vsquant.jpg}
  \caption{\small Vectorwise decomposition for per-vector scaled quantization (VSQ \citep{dai2021vsq}).}
  \label{fig:vsquant}
\end{wrapfigure}
During VSQ \citep{dai2021vsq}, the operand tensors are decomposed into 1D vectors in a hardware friendly manner as shown in Figure \ref{fig:vsquant}. Since the decomposed tensors are used as operands in matrix multiplications during inference, it is beneficial to perform this decomposition along the reduction dimension of the multiplication. The vectorwise quantization is performed similar to tensorwise quantization described in Equations \ref{eq:sf} and \ref{eq:tensor_quant}, where a scale factor $s_v$ is required for each vector $\bm{v}$ that maps the maximum absolute value of that vector to the maximum quantization level. While smaller vector lengths can lead to larger accuracy gains, the associated memory and computational overheads due to the per-vector scale factors increases. To alleviate these overheads, VSQ \citep{dai2021vsq} proposed a second level quantization of the per-vector scale factors to unsigned integers, while MX \citep{rouhani2023shared} quantizes them to integer powers of 2 (denoted as $2^{INT}$).

\subsubsection{MX Format}
The MX format proposed in \citep{rouhani2023microscaling} introduces the concept of sub-block shifting. For every two scalar elements of $b$-bits each, there is a shared exponent bit. The value of this exponent bit is determined through an empirical analysis that targets minimizing quantization MSE. We note that the FP format $E_{1}M_{b}$ is strictly better than MX from an accuracy perspective since it allocates a dedicated exponent bit to each scalar as opposed to sharing it across two scalars. Therefore, we conservatively bound the accuracy of a $b+2$-bit signed MX format with that of a $E_{1}M_{b}$ format in our comparisons. For instance, we use E1M2 format as a proxy for MX4.

\begin{figure}
    \centering
    \includegraphics[width=1\linewidth]{sections//figures/BlockFormats.pdf}
    \caption{\small Comparing LO-BCQ to MX format.}
    \label{fig:block_formats}
\end{figure}

Figure \ref{fig:block_formats} compares our $4$-bit LO-BCQ block format to MX \citep{rouhani2023microscaling}. As shown, both LO-BCQ and MX decompose a given operand tensor into block arrays and each block array into blocks. Similar to MX, we find that per-block quantization ($L_b < L_A$) leads to better accuracy due to increased flexibility. While MX achieves this through per-block $1$-bit micro-scales, we associate a dedicated codebook to each block through a per-block codebook selector. Further, MX quantizes the per-block array scale-factor to E8M0 format without per-tensor scaling. In contrast during LO-BCQ, we find that per-tensor scaling combined with quantization of per-block array scale-factor to E4M3 format results in superior inference accuracy across models. 


\appendix
\onecolumn

\newglossaryentry{AI}{
    name=AI: Artificial Intelligence,
    description={The simulation of human intelligence}
}

\newglossaryentry{LLM}{
    name=LLM: Large Language Model,
    description={A type of artificial intelligence designed to understand and generate human-like text}
}

\newglossaryentry{API}{
    name=API: Application Programming Interface,
    description={A software interface for offering a service to other pieces of software}
}

\newglossaryentry{AST}{
    name=AST: Abstract Syntax Tree,
    description={A tree representation of the abstract syntactic structure of source code written in a programming language}
}

\newglossaryentry{ReACC}{
    name=ReACC: Retrieval-Augmented Code Completion,
    description={A framework that enhances code completion by leveraging external context from a large codebase}
}

\newglossaryentry{CMSIS}{
    name=CMSIS: Cortex Microcontroller Software Interface Standard,
    description={A hardware abstraction layer independent of vendor for the Cortex-M processor series}
}

\newglossaryentry{HAL}{
    name=HAL: Hardware Abstraction Layer,
    description={A layer of programming that allows a computer operating system to interact with a hardware device at an abstract level}
}

\newglossaryentry{RAG}{
    name=RAG: Retrieval-Augmented Generation,
    description={A process that enhances large language models by allowing them to respond to prompts using a specified set of documents}
}

\newglossaryentry{STM32F407}{
    name=STM32F407: High-\allowbreak performance Microcontroller,
    description={A microcontroller that offers the performance of the Cortex-M4 core}
}

\newglossaryentry{AURIX TC334}{
    name=AURIX TC334: 32-bit Microcontroller from Infineon,
    description={A microcontroller designed for automotive and industrial applications, featuring a 32-bit TriCore-\allowbreak architecture}
}

\newglossaryentry{LED}{
    name=LED: Light-Emitting Diode,
    description={A semiconductor light source that emits light when current flows through it}
}

\newglossaryentry{CortexM4}{
    name=Cortex-M4: 32-bit processor design from ARM,
    description={A 32-bit processor design optimized for real-time applications with low power consumption}
}

\newglossaryentry{GPIO}{
    name=GPIO: General-Purpose Input/Output Pin,
    description={A versatile pin on a microcontroller that can be configured as either an input or an output. As an \textbf{input}, it can read external signals such as button presses. As an \textbf{output}, it can control devices}
}

\newglossaryentry{Offset}{
    name=Offset: Relative distance of a specific register,
    description={The relative distance or position of a specific register or memory location within a hardware block, measured from a base address}
}

\newglossaryentry{Clock}{
    name=Clock: Synchronization signal for operations,
    description={The signal used to synchronize operations within a microcontroller or hardware system, ensuring consistent timing and execution of tasks}
}

\newglossaryentry{GPT4oMini}{
    name=GPT-4o Mini: Large Language Model from OpenAI,
    description={A compact variant of the GPT-4 language model designed for cost-efficient and versatile tasks}
}

\newglossaryentry{FAISS}{
    name=FAISS: Facebook AI Similarity Search,
    description={An open-source library for efficient similarity search and clustering of high-dimensional vectors}
}

\section{Related Work}\label{sec:related_works}
\gls{bp} estimation from \gls{ecg} and \gls{ppg} waveforms has received significant attention due to its potential for continuous, unobtrusive monitoring. Earlier work relied on classical machine learning with handcrafted features, but deep learning methods have since emerged as more robust alternatives. Convolutional or recurrent architectures designed for \gls{ecg}/\gls{ppg} have shown strong performance, including ResUNet with self-attention~\cite{Jamil}, U-Net variants~\cite{Mahmud_2022}, and hybrid \gls{cnn}--\gls{rnn} models~\cite{Paviglianiti2021ACO}. These architectures often outperform traditional feature-engineering approaches, particularly when both \gls{ecg} and \gls{ppg} signals are used~\cite{Paviglianiti2021ACO}.

Nevertheless, many existing methods train solely on \gls{ecg}/\gls{ppg} data, which, while plentiful~\cite{mimiciii,vitaldb,ptb-xl}, often exhibit significant variability in signal quality and patient-specific characteristics. This variability poses challenges for achieving robust generalization across populations. Recent work has explored transfer learning to overcome these issues; for example, Yang \emph{et~al.}~\cite{yang2023cross} studied the transfer of \gls{eeg} knowledge to \gls{ecg} classification tasks, achieving improved performance and reduced training costs. Joshi \emph{et~al.}~\cite{joshi2021deep} also explored the transfer of \gls{eeg} knowledge using a deep knowledge distillation framework to enhance single-lead \gls{ecg}-based sleep staging. However, these studies have largely focused on within-modality or narrow domain adaptations, leaving open the broader question of whether an \gls{eeg}-based foundation model can serve as a versatile starting point for generalized biosignal analysis.

\gls{eeg} has become an attractive candidate for pre-training large models not only because of the availability of large-scale \gls{eeg} repositories~\cite{TUEG} but also due to its rich multi-channel, temporal, and spectral dynamics~\cite{jiang2024large}. While many time-series modalities (for example, voice) also exhibit rich temporal structure, \gls{eeg}, \gls{ecg}, and \gls{ppg} share common physiological origins and similar noise characteristics, which facilitate the transfer of temporal pattern recognition capabilities. In other words, our hypothesis is that the underlying statistical properties and multi-dimensional dynamics in \gls{eeg} make it particularly well-suited for learning robust representations that can be effectively adapted to \gls{ecg}/\gls{ppg} tasks. Our work is the first to validate the feasibility of fine-tuning a transformer-based model initially trained on EEG (CEReBrO~\cite{CEReBrO}) for arterial \gls{bp} estimation using \gls{ecg} and \gls{ppg} data.

Beyond accuracy, real-world deployment of \gls{bp} estimation models calls for efficient inference. Traditional deep networks can be computationally expensive, motivating recent interest in quantization and other compression techniques~\cite{nagel2021whitepaperneuralnetwork}. Few studies have combined large-scale pre-training with post-training quantization for \gls{bp} monitoring. Hence, our method integrates these two aspects: leveraging a potent \gls{eeg}-based foundation model and applying quantization for a compact, high-accuracy cuffless \gls{bp} solution.
\section{Derivation of update formula}
\label{app: derivation of update formula}
In this section, we will explicitly show how to connect the solution from minimizing reconstruction loss of \gls{fim} (\cref{eq: UFE equation}) to corresponding update rule. 

\subsection{Shampoo's update formula}
\label{subapp: Shampoo update formula}
The key update formula of Shampoo is 
\begin{align*}
    \mW_t = \mW_{t-1} + \lambda \mL_{m,t}^{-\frac{1}{4}}\mG_t\mR_{n,t}^{-\frac{1}{4}}
\end{align*}
\begin{proof}
    From \cref{thm: optimal shampoo}, we simply apply the properties of Kronecker product to square-root version of natural gradient descent:
    \begin{align*}
        &\devect\left(\Ft^{-\frac{1}{2}}\vecg\right)\\
        =&\devect\left((\mR_{n}^{\frac{1}{2}}\otimes \mL_{m}^{\frac{1}{2}})^{-\frac{1}{2}}\vecg\right)\\
        =& \devect\left(\vect\left(\mL_{m}^{-\frac{1}{4}}\mG\mR_{n}^{-\frac{1}{4}}\right)\right)\\
        =& \mL_{m}^{-\frac{1}{4}}\mG\mR_{n}^{-\frac{1}{4}}
    \end{align*}
\end{proof}

\subsection{Generalization to whitening and normalization}
\label{subapp: update of generalization of whitening}
The square-root \gls{ngd} update with $\Ft$ in \cref{eq: generalization whitening} in \cref{coro: generalization to whitening and normalization} is 
\begin{equation}
    \devect\left(\Ft^{-\frac{1}{2}}\vecg\right) = \sqrt{n}\E[\mG\mG^T]^{-\frac{1}{2}}\mG
\end{equation}
\begin{proof}
    From the solution in \cref{eq: generalization whitening}, we can simply apply the properties of Kronecker product as in the derivation of Shampoo's update:
    \begin{align*}
        &\devect\left(\Ft^{-\frac{1}{2}}\vecg\right)\\
        =&\devect\left((\mI_n\otimes \mM)^{-\frac{1}{2}}\vecg\right)\\
        =&\devect\left(\vect\left(\sqrt{n}\mM^{-\frac{1}{2}}\mG\right)\right)\\
        =& \sqrt{n}\E[\mG\mG^T]^{-\frac{1}{2}}\mG
    \end{align*}
\end{proof}
Similarly, the square-root \gls{ngd} update with $\Ft=\mS\otimes \mI_m$ is 
\begin{equation}
    \devect\left(\Ft^{-\frac{1}{2}}\vecg\right) = \sqrt{m}\mG\mS^{-\frac{1}{2}}
\end{equation}
\begin{proof}
    This is trivial by applying the property of Kronecker product:
    \begin{align*}
        &\devect((\mS\otimes\mI_m)^{-\frac{1}{2}}\vecg)\\
    =& \devect(\vect(\mG\mS^{-\frac{1}{2}}))\\
    =&\mG\mS^{-\frac{1}{2}}
    \end{align*}
\end{proof}
\subsection{Update formula for \gls{alicec}}
\label{subapp: update of generlized adam}
\begin{proof}
    From the \cref{thm: alicec 1 step refinement}, we can apply the properties of block diagonal and Kronecker product with a full-rank $\mU$:
    \begin{align*}
        &\devect(\Ft^{-\frac{1}{2}}\vecg)\\
        =& \devect\left(\diagb\left(\mM_1^{-\frac{1}{}2},\ldots, \mM_n^{-\frac{1}{2}}\right)\vecg\right)\\
        =&\devect\left(\diagb\left(\mU\mD_1^{-\frac{1}{2}}\mU^T,\ldots,\mU\mD_{n}^{-\frac{1}{2}\mU^T}\right)\vecg\right)\\
        =&\devect\left((\mI_n\otimes \mU)\diagb(\sqrt{\mD_1},\ldots,\sqrt{\mD_n})(\mI\otimes \mU^T)\vecg\right)\\
        =&\devect\left((\mI_n\otimes \mU)\diagb(\sqrt{\mD_1},\ldots,\sqrt{\mD_n})\vect\left(\mU^T\mG\right)\right)\\
        =&\devect\left((\mI_n\otimes \mU)\vect\left(\frac{\mU^T\mG}{\sqrt{\E[(\mU^T\mG)\elesquare]}}\right)\right)\\
        =&\devect\left(\vect\left(\mU\frac{\mU^T\mG}{\sqrt{\E[(\mU^T\mG)\elesquare]}}\right)\right)\\
        =& \mU\frac{\mU^T\mG}{\sqrt{\E[(\mU^T\mG)\elesquare]}}
    \end{align*}
\end{proof}

\subsection{Update formula for SOAP}
\label{subapp: update formula for SOAP}
Based on the \cref{thm: optimal asham}, we can derive the update formula of the corresponding square-root \gls{ngd} following the same procedure as \gls{alicec}:
\begin{align*}
    \devect\left(\Ft^{-\frac{1}{2}}\vecg\right) = \mU_L\frac{\mU_L^T\mG\mU_R}{\sqrt{\E[(\mU_L^T\mG\mU_R)\elesquare]}}\mU_R^T.
\end{align*}
\begin{proof}
    \begin{align*}
        &\devect\left(\Ft^{-\frac{1}{2}}\vecg\right)\\
        =&\devect\left((\mU_R\otimes\mU_L)\diagm(\E[(\mU_L^T\mG\mU_R)\elesquare])^{-\frac{1}{2}}(\mU_R\otimes\mU_L)^T\vecg\right)\\
        =&\devect\left((\mU_R\otimes\mU_L)\diagm(\E[(\mU_L^T\mG\mU_R)\elesquare])^{-\frac{1}{2}}\vect\left(\mU_L^T\mG\mU_R\right)\right)\\
        =&\devect\left((\mU_R\otimes\mU_L)\vect\left(\frac{\mU_L^T\mG\mU_R}{\sqrt{\E[(\mU_L^T\mG\mU_R)\elesquare]}}\right)\right)\\
        =&\devect\left(\vect\left(\mU_L\frac{\mU_L^T\mG\mU_R}{\sqrt{\E[(\mU_L^T\mG\mU_R)\elesquare]}}\mU_R^T\right)\right)\\
        =&\mU_L\frac{\mU_L^T\mG\mU_R}{\sqrt{\E[(\mU_L^T\mG\mU_R)\elesquare]}}\mU_R^T
    \end{align*}
\end{proof}
Therefore, one can design the optimizer based on this update formula and exactly recovers the SOAP's procedure (\cref{eq: practical soap updates} in \cref{subapp: SOAP}).


\section{Theory and proof}
\label{app: theory and proof}
To prove the results, we need to first introduce some useful lemmas and inequalities.

\begin{lemma}
    Assume $\Ft$ is a block diagonal matrix with $n$ squared block matrix $\mM_i\in\Rmm$, then 
    \begin{equation}
        \min_{\Ft} \Fnorm{\Ft-\mF} = \min_{\{\mM_i\}_{i=1}^n} \sum_{i=1}^n \Fnorm{\mM_i} - 2\tr(\mM_i^T\E[\vg_i\vg_i^T]) + C
    \end{equation}
where $\vg_i$ is the $i^{\text{th}}$ column of gradient $\mG$, $C$ is a constant that is idenpendent of $\Ft$, and $\mF$ is the \gls{fim}. 
\label{lemma: block diagonal simplification}
\end{lemma}
\begin{proof}
    This is straightforward by expanding the \gls{f-norm}. 
    \begin{align*}
        &\Fnorm{\Ft-\mF}\\
        =& \tr\left((\Ft-\mF)^T(\Ft-\mF)\right)\\
        =&\Fnorm{\Ft} - 2\tr\left(\Ft^T\mF\right) + C\\
        =& \sum_{l=1}^{mn} \Ft_{l,:}^T\Ft_{:,l} - \Ft_{l,:}^T \mF_{:,l} + C\\
        =& \sum_{i=1}^n \Fnorm{\mM_i}-2\tr\left(\mM_i^T\E[\vg_i\vg_i^T]\right) + C
    \end{align*}
where $\Ft_{l,:}\in \mathbb{R}^{mn}$ indicates the $l^{\text{th}}$ row vector of $\Ft$ and $\Ft_{:,l}$ is the $l^{\text{th}}$ column vector. The last equation is obtained by the fact that $\Ft$ is a block diagonal matrix. So only the values of $\mF$ at the position of non-zero values $\Ft$ contributes to the trace, which is exactly the outer product: $\E[\vg_i\vg_i^T]$. 
\end{proof}

\begin{lemma}[Powers-Stormer inequality]
    For positive semi-definite operator $\mA$, $\mB$, we have the following inequality
    \begin{equation}
        \tr((\mA-\mB)^T(\mA-\mB)) \leq \Vert\mA^2 -\mB^2\Vert_1
        \label{eq: powers stormer inequality}
    \end{equation}
    \label{lemma: powers stormer inequality}
    where $\Vert\cdot\Vert_1$ is the trace norm.
\end{lemma}

\subsection{Proof of \cref{prop: adam solution}}
\label{subapp: proof of adam}
\begin{proof}
    From \cref{lemma: block diagonal simplification}, we have
    \begin{align*}
        &\Fnorm{\Ft-\mF}\\
        =& \sum_{i=1}^n \Fnorm{\mM_i} - 2\tr\left(\mM_i^T\E[\vg_i\vg_i^T]\right)\\
        =& \sum_{i=1}^n\sum_{j=1}^m M_{i,jj}^2 - 2M_{i,jj}\E[g_{i,j}^2]
    \end{align*}
    By taking the derivative w.r.t $M_{i,jj}$, we have
    \begin{align*}
        M_{i,jj} = \E[g_{i,j}^2]
    \end{align*}
    Thus, we have $\Ft = \diag(\E[\vecg^2])$.
\end{proof}
\subsection{Proof of \cref{thm: optimal shampoo}}
\label{subapp: proof of shampoo optimiality}

To prove this theorem, we need to leverage the \cref{coro: generalization to whitening and normalization} for generalized whitening (\cref{eq: generalization whitening}) in \cref{subsec: sve}. This is proved in \cref{subapp: proof normalization}. But in the following, we will provide an alternative proof for completeness.

\begin{proof}
    From \cref{lemma: block diagonal simplification}, we have
    \begin{align*}
        &\Fnorm{\Ft-\mF} \\
        =& \sum_{i=1}^n \Fnorm{\mM} - 2\tr(\mM^T\E[\vg_i\vg_i^T]) + C\\
        =& n\Fnorm{\mM} - 2\tr(\mM^T\E[\sum_{i=1}^n \vg_i\vg_i^T]) + C\\
        =& n\Fnorm{\mM} - 2\tr(\mM^T\E[\mG\mG^T]) + C \\
    \end{align*}
    To minimize this, we take the derivative w.r.t. $\mM$, we have
    \begin{align*}
        2n\mM - 2\E[\mG\mG^T] = 0 \Rightarrow \mM = \frac{1}{n} \E[\mG\mG^T]
    \end{align*}
\end{proof}
Next, we prove another proposition that is "symmetric" to the whitening results in \cref{coro: generalization to whitening and normalization}.
\begin{proposition}
    Assume $\family = \{\mR_n \otimes \mI_m\}$, where $\mR_n\in \Rnn$ is \gls{spd} matrix, then \cref{eq: UFE equation} can be analytically solved with the optimal solution as 
    \begin{equation}
        \mR_n^* = \frac{1}{m} \E[\mG^T\mG]
        \label{eq: optimal shampoo right}
    \end{equation}
    \label{prop: optimal shampoo right}
\end{proposition}
\begin{proof}
    Since $\mR_n\otimes \mI_m$ does not have a nice block diagonal structure like the previous proposition, we need to analyze it a bit more. First, we have
    \begin{align*}
        &\Fnorm{\mR_n \otimes \mI_m - \mF}\\
        =& \Fnorm{\mR_n\otimes \mI_m} - 2\tr\left(\underbrace{(\mR_n\otimes \mI_m)^T\E[\vecg\vecg^T]}_{\mZ}\right) + C\\
    \end{align*}
Since we only care about the diagonal of $\mZ$, therefore, we only inspect the block diagonal of $\mZ$ with each block $\mZ_i$ of size $\Rmm$, and $i=1,\ldots, n$. By basic algebra, we have
\begin{align*}
    \mZ_i = \sum_{k=1}^n R_{ik} \vg_k\vg_i^T
\end{align*}
where $\vg_k$ is the $k^{\text{th}}$ column of $\mG$. Therefore, we can simplify the trace of $\mZ$ as 
\begin{align*}
    \tr(\mZ) &= \sum_{i=1}^n\tr(\mZ_i)\\
    =& \tr(\sum_{i=1}^n\sum_{k=1}^n R_{ij}\vg_k\vg_i^T)\\
    =& \sum_{i=1}^n\sum_{k=1}^n\sum_{j=1}^m R_{ik}[\mG]_{ji}[\mG^T]_{kj}
\end{align*}
where $[\mG]_{ji}$ is the element of $\mG$ at $j^{\text{th}}$ row and $i^{\text{th}}$ column. 

Now, let's perform the same analysis of the following quantity
\begin{align*}
    \tr\left((\mI_m \otimes \mR_n)\E[\vecgt\vecgt^T]\right)
\end{align*}
where $\vecgt$ is the vectorized transposed gradient $\mG^T$. Namely, it now stacks the rows of $\mG$ instead of columns of $\mG$ like $\vecg$. This object is simple to treat due to its block diagonal structure, by algebric manipulation, we have
\begin{align*}
    \tr\left((\mI_m\otimes \mR_n)\E[\vecgt\vecgt^T]\right) &= \underbrace{\sum_{k=1}^m}_{\text{over blocks}}\tr(R_{ij}\underbrace{[\mG^T]_k}_{\text{kth column of }\mG^T}[\mG^T]_k^T)\\
    =&\sum_{k=1}^m \sum_{i=1}^n \sum_{j=1}^n R_{ij} [\mG^T]_{jk}[\mG]_{ki}
\end{align*}
Now, let's change the variable $i=i$, $j=k$ and $k=j$, the above becomes
\begin{align}
    &\tr\left((\mI_m\otimes \mR_n)\E[\vecgt\vecgt^T]\right) \nonumber\\
    =& \sum_{j=1}^m \sum_{i=1}^n \sum_{k=1}^n R_{ik} [\mG^T]_{kj}[\mG]_{ji} \nonumber\\
    =& \tr(\mZ) \label{eq: proof NI=IN}
\end{align}
We should also note that
\begin{align*}
    &\Fnorm{\mR_n\otimes \mI_m}\\
    =& \tr\left((\mR_n\otimes \mI_m)^T(\mR_n\otimes \mI_m)\right)\\
    =& \tr\left((\mR_n^T\mR_n)\otimes \mI_m\right)\\
    =& \tr(\mR_n^T\mR_n)\tr(\mI_m)\\
    =&\tr\left((\mI_m\otimes \mR_n)^T(\mI_m\otimes \mR_n)\right)\\
    =& \Fnorm{(\mI_m\otimes \mR_n)}
\end{align*}
Therefore, by using the above equation and \cref{eq: proof NI=IN}, the original minimization problem is translated to 
\begin{align*}
    \argmin_{\mR_n} \Fnorm{\mR_n\otimes \mI_m -\mF} = \argmin_{\mR_n}\Fnorm{\mI_m \otimes \mR_n -\E[\vecgt\vecgt^T]}
\end{align*}
Thus, we can leverage \cref{coro: generalization to whitening and normalization} to obtain the optimal solution
\begin{align*}
    \mR_n^* = \frac{1}{m} \E[\mG^T\mG]
\end{align*}
\end{proof}

With the above two propositions, we can start to prove \cref{thm: optimal shampoo}.
\begin{proof}
    First, we note that
    \begin{align*}
        &\Fnorm{\Rnr\otimes \Lmr - \mF}\\
        =& \Fnorm{\underbrace{(\mR_n\otimes \mI_m)^{\frac{1}{2}}}_{\mA}\underbrace{(\mI_n\otimes \mL_m)^{\frac{1}{2}}}_{\mB}-\underbrace{\E[\vecg\vecg^T]^{\frac{1}{2}}}_{\mC}\E[\vecg\vecg^T]^{\frac{1}{2}}}\\
        =& \Fnorm{\mA\mB - \mC\mC}
    \end{align*}
    Next, we will upper bound this quantity.
    First, we have
    \begin{align*}
        \mA\mB-\mC\mC = \mA(\mB-\mC) + (\mA-\mC)\mC
    \end{align*}
    By triangular inequality, we have
    \begin{align*}
        &\Vert\mA\mB-\mC\mC\Vert_F\\
        &\leq \Vert\mA(\mB-\mC)\Vert_F+\Vert(\mA-\mC)\mC\Vert_F \\
        &\leq \Vert\mA\Vert_F\Vert\mB-\mC\Vert_F+ \Vert\mA-\mC\Vert_F\Vert\mC\Vert_F\\
        &\leq (\Vert\mA-\mC\Vert_F+\Vert\mC\Vert_F)\Vert\mB-\mC\Vert_F+ \Vert\mA-\mC\Vert_F\Vert\mC\Vert_F\\
        &= \Vert\mA-\mC\Vert_F\Vert\mB-\mC\Vert_F+\Vert\mC\Vert_F(\Vert\mB-\mC\Vert_F+ \Vert\mA-\mC\Vert_F)
    \end{align*}
    Now, the squared norm can be upper bounded by 
    \begin{align}
        \Fnorm{\mA\mB-\mC\mC} \leq& 3\left(\Fnorm{\mA-\mC}\Fnorm{\mB-\mC}+\Fnorm{\mC}\Fnorm{\mA-\mC}+\Fnorm{\mC}\Fnorm{\mB-\mC}\right)\nonumber\\
        \leq&3\left(mn\Vert\mA^2-\mC^2\Vert_F\Vert\mB^2-\mC^2\Vert_F+\sqrt{mn}\Fnorm{\mC}\Vert\mA^2-\mC^2\Vert_F+\sqrt{mn}\Fnorm{\mC}\Vert\mB^2-\mC^2\Vert_F\right)
        \label{eq: proof upper bound shampoo}
    \end{align}
The first inequality is obtained by the fact that for any three matrix $\mP$, $\mQ$ and $\mH$, we have
\begin{align*}
    \Fnorm{\mP+\mQ+\mH}\leq& \left(\Vert\mP\Vert_F+\Vert\mQ\Vert_F+\Vert\mH\Vert_F\right)^2\\
    =& \Fnorm{\mP}+\Fnorm{\mQ}+\Fnorm{\mH} + 2\Vert\mP\Vert_F\Vert\mQ\Vert_F + 2\Vert\mP\Vert_F\Vert\mH\Vert_F+2\Vert\mQ\Vert_F\Vert\mH\Vert_F\\
    \leq& 3\left(\Fnorm{\mP}+\Fnorm{\mQ}+\Fnorm{\mH}\right)
\end{align*}
The second inequality is obtained by directly applying Powers-Stormer's inequality and Holder's inequality. For completeness, we will show how to upper-bound $\Fnorm{\mA-\mC}$, the rest can be bounded in the same way. 
From \cref{lemma: powers stormer inequality} and both $\mA$, $\mC$ are \gls{spd} matrix, we have
\begin{align*}
    \Fnorm{\mA-\mC}\leq \Vert\mA^2-\mC^2\Vert_1
\end{align*}
Then, we can select $p=q=2$ for Holder's inequaity and obtain
\begin{align*}
    \Vert\mA^2-\mC^2\Vert_1\leq \sqrt{mn}\Vert\mA^2-\mC^2\Vert_F
\end{align*}
where $\sqrt{mn}$ comes from the $\Vert\mI_{mn}\Vert_F$ in Holder's inequality. By substitute it back, we obtain the upper bound.

We can see that minimizing the upper bound \cref{eq: proof upper bound shampoo} is equivalent to minimize each $\Vert\mA^2-\mC^2\Vert_F$, $\Vert\mB^2-\mC^2\Vert_F$ individually, and 
\begin{align*}
    \Vert\mA^2-\mC^2\Vert_F &= \Vert\mR_n\otimes \mI_m - \mF\Vert_F\\
    \Vert\mB^2-\mC^2\Vert_F &= \Vert\mI_n\otimes \mL_m-\mF\Vert_F
\end{align*}
Thus, from \cref{coro: generalization to whitening and normalization} and \cref{prop: optimal shampoo right}, we prove the theorem. 
\end{proof}

\subsection{Proof of \cref{coro: generalization to whitening and normalization} and \cref{prop: two sided scaling}}
\label{subapp: proof normalization}
Instead of proving the \cref{coro: generalization to whitening and normalization}, we propose a generalization to those gradient operations, where \cref{coro: generalization to whitening and normalization} is a special case.

\paragraph{Structure assumption} We consider $\family=\{\mS\otimes \mM\}$ with identical \gls{spd} $\mM\in\Rmm$ and positive diagonal $\mS\in\Rnn$. 
The following theorem proves that the optimal solution can be solved by a fixed-point iteration. 
\begin{theorem}
    Assuming $\family=\{\mS\otimes \mM\}$ with positive diagonal $\mS\in\Rnn$ and \gls{spd} $\mM\in\Rmm$, and $\E_{GG'}[(\mG^T\mG')\elesquare]$ contains positive values, solving \cref{eq: UFE equation} admits a fixed point procedure:
    \begin{align}
        \diag(\mS) = \frac{\diag(\E[\mG^T\mM\mG])}{\Fnorm{\mM}},\;\;\;
        \mM=\frac{\E[\mG\mS\mG^T]}{\Fnorm{\mS}}.
        \label{eq: optimal S and M}
    \end{align}
    The solution $\diag(\mS^*)$ converges to the principal eigenvector of $\E[(\mG^T\mG')\elesquare]$ up to a scaling with unique $\mS^*\otimes \mM^*$. 
    \label{thm: generalization to normal and whiten}
\end{theorem}

To prove \cref{thm: generalization to normal and whiten}, we first introduce some classic results. 
\begin{theorem}[Perron-Frobenius theorem]
For a matrix $\mA\in\Rnn$ with positive entries, the principal eigenvalue $r$ is positive, called Perron-Frobenius eigenvalue. The corresponding eigenvector $\vv$ of $\mA$ is called Perron vector and only contains positive components: $\mA\vv=r\vv$ with $v_i>0$. In addition, there are no other positive eigenvectors of $\mA$. 
\label{thm: Perron-Frobenius theorem}
\end{theorem}

\begin{definition}[Hilbert projective metric]
    For any given vectors $\vv$, $\vw$ in $C\slash \{0\}$ where $C$ is a closed convex pointed non-negative cone $C$, i.e.~$C\cap (-C)=\{0\}$, the Hilbert projective metric is defined as 
    \begin{align*}
        d_H(\vv,\vw) = \log \left(\max_i \frac{v_i}{w_i}\right) - \log \left(\min_i \frac{v_i}{w_i}\right)
    \end{align*}
\end{definition}
This is a pseudo metric since it has a scaling invariance property: $d_H(\vv,\alpha\vm)=d_H(\vv,\vm)$ for $\alpha>0$. This means $d_H(\vv,\vm)=0$ does not mean $\vv=\vm$ but $\vv=\alpha\vm$ with some positive scaling $\alpha$. However, this is a metric on the space of rays inside the cone. 

\begin{theorem}[Birkhoff-Hopf theorem]
Let $\mP\in\Rnn$ be a positive matrix and let
\begin{align*}
    \kappa(\mP) = \inf\left\{\alpha\geq 0: d_H(\mP\vx,\mP\vy)\leq \alpha d_H(\vx,\vy), \forall \vx,\vy \in C_+, \vx \sim \vy \right\}
\end{align*}
where $C_+$ is the cone that each element is non-negative and $\sim$ is the induced equivalence relation. Namely, if $\vx\sim\vy$, there exists $\alpha,\beta>0$ such that $\alpha\vx<\vy<\beta\vx$, and $\vx<\vy$ means $y-x\in C_+$. Then, it holds
\begin{align*}
    \kappa(\mP) = \tanh{\frac{1}{4}\Delta(\mP)}\;\;\;\text{with}\;\Delta(\mP) = \max_{i,j,k,l}\frac{P_{ij}P_{kl}}{P_{il}P_{kj}}
\end{align*}
\label{thm: Birkhoff-Hopf}
\end{theorem}
This theorem suggests that when $\mP$ is a positive matrix, the corresponding linear mapping is contractive since $\tanh(\cdot)\leq 1$ under Hilbert projective metric. 

Now, let's prove the \cref{thm: generalization to normal and whiten}.
\begin{proof}
    First, we can simplify the \cref{eq: UFE equation} using \cref{lemma: block diagonal simplification}:
    \begin{align*}
        &\Fnorm{\mS\otimes\mM - \mF}\\
        =& \sum_{i=1}^n S_i^2\Fnorm{\mF} - 2\tr(S_i\mM\E[\vg_i\vg_i^T]) +C
    \end{align*}
    Then, we simply take its derivative w.r.t. $s_i$, and obtain
    \begin{align*}
        &2S_i\Fnorm{\mM} = 2\tr(\mM\E[\vg_i\vg_i^T])\\
        \Longrightarrow& S_i = \frac{\tr(\mM\E[\vg_i\vg_i^T])}{\Fnorm{\mM}}\\
        \Longrightarrow& \diag(\mS) = \frac{\diag\left(\E[\mG^T\mM\mG]\right)}{\Fnorm{\mM}}
    \end{align*}
    Similarly, we have
    \begin{align*}
        \mM =& \frac{\sum_{i=1}^nS_i\E[\vg_i\vg_i^T]}{\Fnorm{\mS}}\\
        =& \frac{\E[\mG\mS\mG^T]}{\Fnorm{\mS}}
    \end{align*}
These define an iterative procedure. Next, we will show it converges.
Let's substitute $\mM$ into $\mS$, and obtain
\begin{align*}
    \mS =& \diag\left(\E_{G}\left[\mG^T\E_{G'}[\mG'\mS\mG^{'T}]\mG\right]\right)\alpha(\mS)\\
    =& \diag\left(\E_{GG'}\left[\underbrace{\mG\mG^{'T}}_{\mH}\mS\mG^{'T}\mG\right]\right)
\end{align*}
where $\alpha(\mS)$ is the scaling term. In the following, we use $\E$ as $\E_{GG'}$.
Since we can show
\begin{align*}
    S_i = \E\left[\sum_{j}^nS_j\right],
\end{align*}
we can write $\mS$ in its vector format:
\begin{align*}
    \vs = \underbrace{\E\left[\mH\elesquare\right]}_{\mP}\vs.
\end{align*}
From the assumption, we know $\mP$ contains only positive values, let's define a quotient space for positive vectors $\vs$ and $\vq$ under the equivalence relation $\vs\sim \vs'$ if $\vs =\alpha \vs'$ for some positive scaling $\alpha$. Namely, we define a space of rays inside the positive cone. Therefore, the Hilbert projective metric becomes a real metric inside the quotient space. 

From the \cref{thm: Birkhoff-Hopf}, we know the linear mapping associated with $\mP$ is contractive. Therefore, we can follow the proof of Banach fixed point theorem on the previously defined quotient space with Hilbert projective metric to show the convergence of this fixed point iteration on $\vs$.  

Now, we show the solution $\vs^*$ is always positive. Since it is converging, therefor, the solution satisfies 
\begin{align*}
    \vs^* = \alpha(\vs^*) \mP\vs^*
\end{align*}
This is equivalent to finding the eigenvectors of $\mP$. By leveraging Perron-Frobenius theorem (\cref{thm: Perron-Frobenius theorem}), we know $\vs^*$ is the principal eigenvector of $\mP$, and only contain positive values. It is also easy to verify that this fixed point converges upto a positive scaling factor (this is expected since the contractive mapping holds true for the quotient space with Hilbert metric, that is invariant to scaling.)

Although $\vs^*$ is not unique, but $\mS\otimes \mM$ is, since for arbitrary positive scaling $\beta$
\begin{align*}
    &\vs^{'*}=\beta\vs^* \Longrightarrow \mM^{'*} = \frac{1}{\beta}\frac{\E[\mG\mS^*\mG^T]}{\Vert\vs\Vert_2^2}\\
    \Longrightarrow& \mS^{'*}\otimes\mM^{'*} = \mS^*\otimes \mM^*
\end{align*}
\end{proof}

Therefore, \cref{coro: generalization to whitening and normalization} is a direct consequence by substituting $\Ft = \mI_n\otimes\mM$ and $\Ft = \mS\otimes \mI_m$ into \cref{eq: optimal D for compensation}.


Next, we prove \cref{prop: two sided scaling}.
\begin{proof}
    From the \cref{thm: generalization to normal and whiten}, the iterative procedure for $\mQ$ can be simply obtained by taking the diagonals of $\mM$:
    \begin{align*}
        \mQ = \frac{\diag\left(\E\left[\mG\mS\mG^T\right]\right)}{\Fnorm{\mS}}.
    \end{align*}
    Following the same proof strategy of \cref{thm: generalization to normal and whiten}, we substitute $\mQ$ into the update of $\mS$ and re-write it into the vector format. First, let's rewrite the update of $\mS$
    \begin{align*}
        &S_i \propto \E[\sum_{j=1}G_{ji}^2 Q_j]\\
        \Longrightarrow& \vs = \frac{\mP^T \vq}{\Vert\vq\Vert_2^2}
    \end{align*}
    where $\mP=\E[\mG\elesquare]$. Similarly, $\vq = \frac{\mP\vs}{\Vert\vs\Vert_2^2}$.
    Thus,
    \begin{align*}
        \vs = \alpha(\vs) \mP^T\mP \vs
    \end{align*}
    From the assumption $\mP$ contains only positive values, we can follow the exact same argument made in \cref{thm: generalization to normal and whiten} to show the convergence of this fixed point update and the positivity of the final solution $\vs^*$. Precisely, $\vs^*$ and $\vq^*$ are the right and left principal singular vectors of $\mP$, respectively, and $\mS^*\otimes \mQ^*$ are unique. 
\end{proof}
\subsection{Proofs of \gls{alicec}}
\label{subapp: proofs of alicec}
% proof of general UDU under kernel PCA
% proof of the 1-step refinement
\subsubsection{Proof of \cref{thm: alicec 1 step refinement}}

\begin{proof}
    For simplicity, we omit the subscript $f$ in $\mU_{f}$. If we assume all $\mD_i$ are equal and only contain positive values, then each block $\mU\mD_i\mU^T$ are the same for all $i$, and it is \gls{spd} matrix. Then, to minimize the the loss \cref{eq: UFE equation}, we can directly leverage the whitening results in \cref{coro: generalization to whitening and normalization}, and obtain $\mM^*=\E[\mG\mG^T]$. Due to the structure of $\mU\mD\mU^T$, the optimal $\mU^*$ is exactly the eigen-matrix of $\mM^*$. 

    Next, we prove for any fixed $\mU$, we can find the corresponding optimal $\mD_i$. 
    From the block diagonal structure and \cref{lemma: block diagonal simplification}, we have
    \begin{align*}
        &\Fnorm{\Ft-\mF}\\
        =&\sum_{i=1}^n \Fnorm{\mU\mD_i\mU^T} - 2\tr\left(\mU^T\E[\vg_i\vg_i^T]\mU \mD_i\right) + C\\
        =& \sum_{i=1}^n \Fnorm{\mD_i} - 2\tr\left(\mU^T\underbrace{\E[\vg_i\vg_i^T]}_{\mH_i}\mU\mD_i\right) + C\\
        =&\sum_{i=1}^n\sum_{j=1}^m D_{i,jj}^2 - 2\sum_{i=1}^n\sum_{j=1}^m D_{i,jj}\vu_j^T\mH_i\vu_j + C
    \end{align*}
Taking the derivative w.r.t. $D_{i,jj}$, we can find the optimal $D_{i,jj}$ is 
\begin{align*}
    D^*_{i,jj} =& \vu_j^T\mH_i\vu_j\\
    =& \E[(\vu_j^T\vg_i)^2]
\end{align*}
Now, by simple algebra manipulation, we have
\begin{align*}
    \mD^*_i = \diagv(\E[(\mU^T\vg_i)^2])
\end{align*}
where $\diag_M$ is to expand the vector to a diagonal matrix. 
Finally, for $\tilde{\mD}$, we have
\begin{equation}
    \tilde{\mD} = \diagm(\E[(\mU^T\mG)\elesquare])
\end{equation}

The optimality of $\widetilde{\mD}$ can also be obtained by leveraging the Lemma 1 in \citep{george2018fast}, and set the eigenbasis as $\mI_n\otimes \mU$. 
\end{proof}
% \input{Appendix/Asham_proof}
\subsection{Proof of \cref{prop: subspace switching}}
\label{subapp: proof subspace switching}
\begin{proof}
    Within the time block $i+1$ with low-rank mapping $\mU$, the gradient at each step can be decomposed as 
    \begin{align*}
        \mG_t = \underbrace{\mU\mU^T\mG_t}_{\widetilde{\mG_t}} + \underbrace{(\mG_t - \mU\mU^T\mG_t)}_{\mR_t}
    \end{align*}
    Therefore, the true state $\mQ^*_{(i+1)k}$ can be simplified as 
    \begin{align*}
        & \mQ_{(i+1)k} = \sum_{t=ik+1}^{(i+1)k}\mG_t\mG_t^T\\
        =& \sum_{t=ik+1}^{(i+1)k} (\widetilde{\mG}_t+\mR_t)(\widetilde{\mG}_t+\mR_t)^T\\
        =& \sum_{t=ik+1}^{(i+1)k} \widetilde{\mG}_t\widetilde{\mG}_t^T+ \underbrace{\widetilde{\mG}_t\mR_t^T}_{0}+ \underbrace{\mR_t\widetilde{\mG}_t^T}_{0} + \mR_t\mR_t^T
    \end{align*}
The third equality is obtained because we assume $\mG_t\mG_t^T$ shares the same eigen-basis as $\mQ^*_{ik}$. Namely, 
\begin{align*}
\mG_t\mG_t^T=&[\mU,\mU_c]\left[\begin{array}{cc}
    \mA_{t} & 0 \\
   0  & \bm{\Sigma}_t
\end{array}\right]\left[\begin{array}{c}
     \mU^T  \\
     \mU_c^T
\end{array}\right]\\
=& \mU\mA_t\mU^T + \mU_c\bm{\Sigma}_t\mU_c^T
\end{align*}
where $\mA_t$ and $\bm{\Sigma}_t$ are diagonal matrix. Then, we have
\begin{align*}
    &\widetilde{\mG}_t\mR_t^T\\
    =& \mU\mU^T\mG_t(\mU_c\mU_c^T\mG_t)^T\\
    =& \mU\mU^T(\mU\mA_t\mU^T+\mU_c\bm{\Sigma}_t\mU_c^T)\mU_c\mU_c^T\\
    =& \mU\mA_t\underbrace{\mU^T\mU_c}_{0}\mU_c^T + \mU\underbrace{\mU^T\mU_c}_{0}\bm{\Sigma}_t\mU_c^T\mU_c\mU_c^T\\
    =&\bm{0}
\end{align*}
In addition, we can also simplify
\begin{align*}
    &\mR_t\mR_t^T\\
    =& \mU_c\mU_c^T\mG_t\mG_t^T\mU_c\mU_c^T\\
    =& \mU_c\mU_c^T(\mU\mA_t\mU^T+\mU_c\bm{\Sigma}_t\mU_c^T)\mU_c\mU_c^T\\
    =&\mU_c\bm{\Sigma}_t\mU_c^T
\end{align*}
Therefore, 
\begin{align*}
    \mQ^*_{(i+1)k} = \sum_{t=ik+1}^{(i+1)k} \widetilde{\mG}_t\widetilde{\mG}_t^T + \mU_c\bm{\Sigma}_t\mU_c^T
\end{align*}
\end{proof}


\subsection{Proof of \cref{thm: optimal compensation}}
\label{subapp: proof optimal compensation}

\begin{proof}
For simplicity, we ignore the subscript $t$ for the following proof.
% First, we note that $(\mI-\mU\mU^T) = \mU_c\mU_c^T$.
First, we let $\mO=\mS^{-2}$
    Then, the loss function can be written as 
    \begin{align*}
        &\Fnorm{\mO\otimes \mU_c\mU_c^T-\Ft_{c}}\\
        =& \sum_{i=1}^n \Fnorm{O_{ii} \mU_c\mU_c^T} - 2\tr((O_{ii}\mU_c\mU_c^T)^T(\mU_c\mM_i\mU_c^T))\\
        =&\sum_{i=1}^n \Fnorm{O_{ii} \mU_c\mU_c^T} - 2\tr(O_{ii}\mM_i)\\
        =& \sum_{i=1}^n\sum_{k=1}^{m-r}\sum_{j=1}^m O_{ii}^2U_{c,jk}^2 - 2\tr(O_{ii}\mM_i)
    \end{align*}
    where $\mM_i=\diag(\E[(\mU_c^T\vg_i)^2])$, $\vg_i$ is the $i^{\text{th}}$ column of $\mG$, and $U_{c, jk}$ is the element in $j^{\text{th}}$ row, $k^{\text{th}}$ column of $\mU_c$. 
    Then, we take the derivative w.r.t. $O_{ii}$, and we have
    \begin{align*}
        &2O_{ii} \sum_{k=1}^{m-r}\sum_{j=1}^m U_{c,jk}^2 = 2\sum_{k=1}^{m-r} \E[(\mU_c^T\vg_i)^2_{k}]\\
        \Longrightarrow& O_{ii} = \frac{\E[\sum_{k=1}^{m-r}(\mU_c^T\vg_i)_k^2]}{m-r}
    \end{align*}
    This form still requires the access to $\mU_c$. Next, let's simplify it. 
    First, let $\tilde{\mU} = [\mU, \mU_c]$ to be the complete basis, we can show
    \begin{align*}
        &\sum_{k=1}^m (\tilde{\mU}^T\vg_i)^2_k \\
        =&\tr((\tilde{\mU}^T\vg_i)^T(\tilde{\mU}^T\vg_i))\\
        =&\tr(\vg_i^T\underbrace{\tilde{\mU}\tilde{\mU}^T}_{\mI}\vg_i)\\
        =&\vg_i^T\vg_i\\
    \end{align*}
    Now, let's re-write the above in a different format:
    \begin{align*}
    &\sum_{k=1}^m (\tilde{\mU}^T\vg_i)^2_k \\
        =&\tr(\vg_i^T\tilde{\mU}\tilde{\mU}^T\vg_i)\\
        =&\tr(\tilde{\mU}^T\vg_i\vg_i^T\tilde{\mU})\\
        =&\tr\left(\left[\begin{array}{c}
             \mU^T  \\
             \mU_c^T
        \end{array}\right]\vg_i\vg_i^T[\mU,\mU_c]\right)\\
        =& \tr(\mU\mU^T(\vg_i\vg_i^T)+ \mU_c\mU_c^T(\vg_i\vg_i^T))\\
        =& \tr((\mU^T\vg_i)^T(\mU^T\vg_i)) + \tr((\mU_c^T\vg_i)^T(\mU_c^T\vg_i))\\
        =& \sum_{k=1}^r(\mU^T\vg_i)_k^2 + \sum_{k=1}^{m-r} (\mU_c^T\vg_i)^2_k
    \end{align*}
    Therefore, we have
    \begin{align*}
        \E[\sum_{k=1}^{m-r}(\mU_c^T\vg_i)_k^2] = \E[\vg_i^T\vg_i-\sum_{k=1}^r (\mU^T\vg_i)_k^2]
    \end{align*}
    So, we have 
    \begin{align*}
        \diag(\mO) = \frac{\E[\bm{1}_m^T\mG\elesquare - \bm{1}_r^T(\mU^T\mG)\elesquare]}{m-r}
    \end{align*}
    and 
    \begin{align*}
        \diag(\mD) = \frac{\sqrt{m-r}}{\sqrt{\E[\bm{1}_m^T\mG\elesquare - \bm{1}_r^T(\mU^T\mG)\elesquare]}} 
    \end{align*}
    
    
\end{proof}
\subsection{Proof of \cref{thm: optimal asham}}
\label{subapp: proof asham}

\begin{proof}
    The proof strategy is a straightforward combination of \cref{thm: optimal shampoo} and \cref{thm: alicec 1 step refinement}. First, when we assume $\tilde{\mD}$ has the Kronecker product structure, one can easily write 
    \begin{align*}
        &(\mU_R\otimes \mU_L)(\mS_R\otimes \mS_L)(\mU_R\otimes \mU_L)^T\\
        =& (\mU_R\otimes \mU_L)\left[(\mS_R\mU_R^T)\otimes (\mS_L\mU_L^T)\right]\\
        =& \underbrace{(\mU_R\mS_R\mU_R^T)}_{\mA}\otimes \underbrace{(\mU_L\mS_L\mU_L^T)}_{\mB}
    \end{align*}
    Therefore the loss (\cref{eq: UFE equation}) becomes
    \begin{align*}
        \Fnorm{\mA\mB-\mC\mC}
    \end{align*}
    where $\mC = \E[\vecg\vecg^T]^{\frac{1}{2}}$. 
    This is exactly the formulation used in \cref{thm: optimal shampoo} with $\mR_n^{\frac{1}{2}} = \mU_R\mS_R\mU_R^T$ and $\mL_m^{\frac{1}{2}} = \mU_L\mS_L\mU_L^T$. 

    Thus, by directly utilizing \cref{thm: optimal shampoo}, we can see the optimal solution
    \begin{align*}
        \mU_R\mS_R^2\mU_R^T &= \E[\mG^T\mG]\\
        \mU_L\mS_L^2\mU_L^T &= \E[\mG\mG^T]
    \end{align*}
    Due to the structural assumption of $\mU_R$, $\mU_L$, $\mS_R$, $\mS_L$, their corresponding optimal solution can directly obtained using eigenvalue decomposition. 

    Now, let's prove the optimal $\tilde{\mD}$ with any fixed $\mU_R$, $\mU_L$. This is also straightforward by applying the same technique as \cref{thm: alicec 1 step refinement}. 
    The loss can be written as 
    \begin{align*}
        &\Fnorm{\underbrace{(\mU_R\otimes \mU_L)}_{\Pi}\tilde{\mD}\underbrace{(\mU_R\otimes \mU_L)^T}_{\Pi^T}-\mF}\\
        =& \Fnorm{\Pi\tilde{\mD}\Pi^T} - 2\tr\left(\Pi^T\E[\vecg\vecg^T]\Pi\tilde{\mD}\right) + C
    \end{align*}
    Since it is easy to verify orthonormality of $\Pi$, i.e.$\Pi^T\Pi=\mI$, the above is simplified to
    \begin{align*}
        &\Fnorm{\tilde{\mD}}-2\tr\left(\Pi^T\E[\vecg\vecg^T]\Pi\tilde{\mD}\right)\\
        &=\sum_{i=1}^{mn} D_{ii}^2 -2 \sum_{i=1}^{mn} D_{ii}[\Pi]_{i}^T\E[\vecg\vecg^T][\Pi]_{i}
    \end{align*}
    where $[\Pi]_i$ is the $i^{\text{th}}$ column of matrix $\Pi$. Then, by taking the derivative, the optimal $D_{ii}$:
    \begin{align*}
        D^*_{ii} &= [\Pi]_i^T\E[\vecg\vecg^T][\Pi]_i\\
        =&\E[([\Pi]_i^T\vecg)^2]
    \end{align*}
    Therefore, 
    \begin{align*}
        \diag(\tilde{\mD}^*) &= \E[(\Pi^T\vecg)^2]\\
        =& \E[((\mU_R^T\otimes \mU_L^T)\vecg)^2]\\
        =&\vect((\E[(\mU_L^T\mG\mU_R)\elesquare]))\\
    \end{align*}
    \begin{align*}
        \Rightarrow \tilde{\mD}^* = \diag_M((\E[(\mU_L^T\mG\mU_R)\elesquare]))
    \end{align*}
    
\end{proof}
\section{Discussion of Assumptions}\label{sec:discussion}
In this paper, we have made several assumptions for the sake of clarity and simplicity. In this section, we discuss the rationale behind these assumptions, the extent to which these assumptions hold in practice, and the consequences for our protocol when these assumptions hold.

\subsection{Assumptions on the Demand}

There are two simplifying assumptions we make about the demand. First, we assume the demand at any time is relatively small compared to the channel capacities. Second, we take the demand to be constant over time. We elaborate upon both these points below.

\paragraph{Small demands} The assumption that demands are small relative to channel capacities is made precise in \eqref{eq:large_capacity_assumption}. This assumption simplifies two major aspects of our protocol. First, it largely removes congestion from consideration. In \eqref{eq:primal_problem}, there is no constraint ensuring that total flow in both directions stays below capacity--this is always met. Consequently, there is no Lagrange multiplier for congestion and no congestion pricing; only imbalance penalties apply. In contrast, protocols in \cite{sivaraman2020high, varma2021throughput, wang2024fence} include congestion fees due to explicit congestion constraints. Second, the bound \eqref{eq:large_capacity_assumption} ensures that as long as channels remain balanced, the network can always meet demand, no matter how the demand is routed. Since channels can rebalance when necessary, they never drop transactions. This allows prices and flows to adjust as per the equations in \eqref{eq:algorithm}, which makes it easier to prove the protocol's convergence guarantees. This also preserves the key property that a channel's price remains proportional to net money flow through it.

In practice, payment channel networks are used most often for micro-payments, for which on-chain transactions are prohibitively expensive; large transactions typically take place directly on the blockchain. For example, according to \cite{river2023lightning}, the average channel capacity is roughly $0.1$ BTC ($5,000$ BTC distributed over $50,000$ channels), while the average transaction amount is less than $0.0004$ BTC ($44.7k$ satoshis). Thus, the small demand assumption is not too unrealistic. Additionally, the occasional large transaction can be treated as a sequence of smaller transactions by breaking it into packets and executing each packet serially (as done by \cite{sivaraman2020high}).
Lastly, a good path discovery process that favors large capacity channels over small capacity ones can help ensure that the bound in \eqref{eq:large_capacity_assumption} holds.

\paragraph{Constant demands} 
In this work, we assume that any transacting pair of nodes have a steady transaction demand between them (see Section \ref{sec:transaction_requests}). Making this assumption is necessary to obtain the kind of guarantees that we have presented in this paper. Unless the demand is steady, it is unreasonable to expect that the flows converge to a steady value. Weaker assumptions on the demand lead to weaker guarantees. For example, with the more general setting of stochastic, but i.i.d. demand between any two nodes, \cite{varma2021throughput} shows that the channel queue lengths are bounded in expectation. If the demand can be arbitrary, then it is very hard to get any meaningful performance guarantees; \cite{wang2024fence} shows that even for a single bidirectional channel, the competitive ratio is infinite. Indeed, because a PCN is a decentralized system and decisions must be made based on local information alone, it is difficult for the network to find the optimal detailed balance flow at every time step with a time-varying demand.  With a steady demand, the network can discover the optimal flows in a reasonably short time, as our work shows.

We view the constant demand assumption as an approximation for a more general demand process that could be piece-wise constant, stochastic, or both (see simulations in Figure \ref{fig:five_nodes_variable_demand}).
We believe it should be possible to merge ideas from our work and \cite{varma2021throughput} to provide guarantees in a setting with random demands with arbitrary means. We leave this for future work. In addition, our work suggests that a reasonable method of handling stochastic demands is to queue the transaction requests \textit{at the source node} itself. This queuing action should be viewed in conjunction with flow-control. Indeed, a temporarily high unidirectional demand would raise prices for the sender, incentivizing the sender to stop sending the transactions. If the sender queues the transactions, they can send them later when prices drop. This form of queuing does not require any overhaul of the basic PCN infrastructure and is therefore simpler to implement than per-channel queues as suggested by \cite{sivaraman2020high} and \cite{varma2021throughput}.

\subsection{The Incentive of Channels}
The actions of the channels as prescribed by the DEBT control protocol can be summarized as follows. Channels adjust their prices in proportion to the net flow through them. They rebalance themselves whenever necessary and execute any transaction request that has been made of them. We discuss both these aspects below.

\paragraph{On Prices}
In this work, the exclusive role of channel prices is to ensure that the flows through each channel remains balanced. In practice, it would be important to include other components in a channel's price/fee as well: a congestion price  and an incentive price. The congestion price, as suggested by \cite{varma2021throughput}, would depend on the total flow of transactions through the channel, and would incentivize nodes to balance the load over different paths. The incentive price, which is commonly used in practice \cite{river2023lightning}, is necessary to provide channels with an incentive to serve as an intermediary for different channels. In practice, we expect both these components to be smaller than the imbalance price. Consequently, we expect the behavior of our protocol to be similar to our theoretical results even with these additional prices.

A key aspect of our protocol is that channel fees are allowed to be negative. Although the original Lightning network whitepaper \cite{poon2016bitcoin} suggests that negative channel prices may be a good solution to promote rebalancing, the idea of negative prices in not very popular in the literature. To our knowledge, the only prior work with this feature is \cite{varma2021throughput}. Indeed, in papers such as \cite{van2021merchant} and \cite{wang2024fence}, the price function is explicitly modified such that the channel price is never negative. The results of our paper show the benefits of negative prices. For one, in steady state, equal flows in both directions ensure that a channel doesn't loose any money (the other price components mentioned above ensure that the channel will only gain money). More importantly, negative prices are important to ensure that the protocol selectively stifles acyclic flows while allowing circulations to flow. Indeed, in the example of Section \ref{sec:flow_control_example}, the flows between nodes $A$ and $C$ are left on only because the large positive price over one channel is canceled by the corresponding negative price over the other channel, leading to a net zero price.

Lastly, observe that in the DEBT control protocol, the price charged by a channel does not depend on its capacity. This is a natural consequence of the price being the Lagrange multiplier for the net-zero flow constraint, which also does not depend on the channel capacity. In contrast, in many other works, the imbalance price is normalized by the channel capacity \cite{ren2018optimal, lin2020funds, wang2024fence}; this is shown to work well in practice. The rationale for such a price structure is explained well in \cite{wang2024fence}, where this fee is derived with the aim of always maintaining some balance (liquidity) at each end of every channel. This is a reasonable aim if a channel is to never rebalance itself; the experiments of the aforementioned papers are conducted in such a regime. In this work, however, we allow the channels to rebalance themselves a few times in order to settle on a detailed balance flow. This is because our focus is on the long-term steady state performance of the protocol. This difference in perspective also shows up in how the price depends on the channel imbalance. \cite{lin2020funds} and \cite{wang2024fence} advocate for strictly convex prices whereas this work and \cite{varma2021throughput} propose linear prices.

\paragraph{On Rebalancing} 
Recall that the DEBT control protocol ensures that the flows in the network converge to a detailed balance flow, which can be sustained perpetually without any rebalancing. However, during the transient phase (before convergence), channels may have to perform on-chain rebalancing a few times. Since rebalancing is an expensive operation, it is worthwhile discussing methods by which channels can reduce the extent of rebalancing. One option for the channels to reduce the extent of rebalancing is to increase their capacity; however, this comes at the cost of locking in more capital. Each channel can decide for itself the optimum amount of capital to lock in. Another option, which we discuss in Section \ref{sec:five_node}, is for channels to increase the rate $\gamma$ at which they adjust prices. 

Ultimately, whether or not it is beneficial for a channel to rebalance depends on the time-horizon under consideration. Our protocol is based on the assumption that the demand remains steady for a long period of time. If this is indeed the case, it would be worthwhile for a channel to rebalance itself as it can make up this cost through the incentive fees gained from the flow of transactions through it in steady state. If a channel chooses not to rebalance itself, however, there is a risk of being trapped in a deadlock, which is suboptimal for not only the nodes but also the channel.

\section{Conclusion}
This work presents DEBT control: a protocol for payment channel networks that uses source routing and flow control based on channel prices. The protocol is derived by posing a network utility maximization problem and analyzing its dual minimization. It is shown that under steady demands, the protocol guides the network to an optimal, sustainable point. Simulations show its robustness to demand variations. The work demonstrates that simple protocols with strong theoretical guarantees are possible for PCNs and we hope it inspires further theoretical research in this direction.
\section{Experimental Details}
\label{appdx:experiment}
Unless specified, we obtain a stream of datasets for all our experiments by simply sampling from the assumed probabilistic model, where the number of observations $n$ is sampled uniformly in the range $[64, 128]$. For efficient mini-batching over datasets with different cardinalities, we sample datasets with maximum cardinality $(128)$ and implement different cardinalities by masking out different numbers of observations for different datasets whenever required. 
% For all our experiments on supervised setups, we sample $\vx_i \sim \mathcal{N}(\mathbf{0}, \mathbf{I})$ for simplicity, but it is possible to explore other proposal distributions (e.g., heavy-tailed distributions) too. 
% In our Bayesian Neural Networks experiments, we considered a single-layered neural network with $\mathrm{Tanh}$ activation function and $32$ hidden dimensions. We considered the likelihood function as either a Gaussian or a categorical distribution using the logits, depending on regression and classification.

To evaluate both our proposed approach and the baselines, we compute an average of the predictive performances across $25$ different posterior samples for each of the $100$ fixed test datasets for all our experiments. 
That means for our proposed approach, we sample $25$ different parameter vectors from the approximate posterior that we obtain. For MCMC, we rely on $25$ MCMC samples, and for optimization, we train $25$ different parameter vectors where the randomness comes from initialization. 
For the optimization baseline, we perform a quick hyperparameter search over the space $\{0.01, 003, 0.001, 0.0003, 0.0001, 0.00003\}$ to pick the best learning rate that works for all of the test datasets and then use it to train for $1000$ iterations using the Adam optimizer~\citep{kingma2014adam}. For the MCMC baseline, we use the open-sourced implementation of Langevin-based MCMC sampling\footnote{\href{https://github.com/alisiahkoohi/Langevin-dynamics}{https://github.com/alisiahkoohi/Langevin-dynamics}} where we leave a chunk of the starting samples as burn-in and then start accepting samples after a regular interval (to not make them correlated). The details about the burn-in time and the regular interval for acceptance are provided in the corresponding experiments' sections below.

For our proposed approach of amortized inference, we do not consider explicit hyperparameter optimization and simply use a learning rate of $1\mathrm{e}\text{-}4$ with the Adam optimizer. For all experiments, we used linear scaling of the KL term in the training objectives as described in~\citep{higgins2017betavae}, which we refer to as warmup. Furthermore, training details for each experiment can be found below. 

\subsection{Fixed-Dim}
\label{appdx:details_fixed_dim}
In this section, we provide the experimental details relevant to reproducing the results of Section~\ref{sec:experiments}. All the models are trained with streaming data from the underlying probabilistic model, such that every iteration of training sees a new set of datasets. Training is done with a batch size of $128$, representing the number of datasets seen during one optimization step. Evaluations are done with $25$ samples and we ensure that the test datasets used for each probabilistic model are the same across all the compared methods, i.e., baselines, forward KL, and reverse KL. We train the amortized inference model and the forward KL baselines for the following different probabilistic models:

\textbf{Mean of Gaussian (GM):} We train the amortization models over $20,000$ iterations for both the $2$-dimensional as well as the $100$-dimensional setup. We use a linear warmup with $5000$ iterations over which the weight of the KL term in our proposed approach scales linearly from $0$ to $1$. We use an identity covariance matrix for the data-generating process, but it can be easily extended to the case of correlated or diagonal covariance-based Gaussian distributions.

\textbf{Gaussian Mixture Model (GMM):} We train the mixture model setup for $200,000$ iterations with $50,000$ iterations of warmup. We mainly experiment with $2$-dimensional and $5$-dimensional mixture models, with $2$ and $5$ mixture components for each setup. While we do use an identity covariance matrix for the data-generating process, again, it can be easily extended to other cases.
% For all our experiments, we compute the average over 25 different samples (either from the approximate posterior, or 25 different optimization runs, etc.) to report the downstream metrics. For the optimization baseline, we perform a quick hyperparameter search for each dataset over the space of $\{\}$

\textbf{Linear Regression (LR):} The amortization models for this setup are trained for $50,000$ iterations with $12,500$ iterations of warmup. The feature dimensions considered for this task are $1$ and $100$ dimensions, and the predictive variance $\sigma^2$ is assumed to be known and set as $0.25$.

\textbf{Nonlinear Regression (NLR):} We train the setup for $100,000$ iterations with $25,000$ iterations consisting of warmup. The feature dimensionalities considered are $1$-dimensional and $25$-dimensional, and training is done with a known predictive variance similar to the LR setup. For the probabilistic model, we consider both a $1$-layered and a $2$-layered multi-layer perceptron (MLP) network with 32 hidden units in each, and either a \textsc{relu} or \textsc{tanh} activation function.

\textbf{Linear Classification (LC):} We experiment with $2$-dimensional and $100$-dimensional setups with training done for $50,000$ iterations, out of which $12,500$ are used for warmup. Further, we train for both binary classification as well as a $5$-class classification setup.

\textbf{Nonlinear Classification (NLC):} We experiment with $2$-dimensional and $25$-dimensional setups with training done for $100,000$ iterations, out of which $2,5000$ are used for warmup. Further, we train for both binary classification as well as a $5$-class classification setup. For the probabilistic model, we consider both a $1$-layered and a $2$-layered multi-layer perceptron (MLP) network with 32 hidden units in each, and either a \textsc{relu} or \textsc{tanh} activation function.

\begin{table*}[t]
    \centering
    % \small
    \footnotesize	    
    \def\arraystretch{1.25}
    \setlength{\tabcolsep}{5pt}
    \begin{tabular}{lcr ccc cccc}
        \cmidrule[\heavyrulewidth]{1-9}
         &  &  & \multicolumn{6}{c}{\textit{$L_2$ Loss} ($\downarrow$)} \\
        \cmidrule(lr){4-9}
        \textbf{Objective} & $q_\varphi$ & \textbf{Model} & \multicolumn{3}{c}{\textbf{Linear Model $|$ MLP-TanH Data}} & \multicolumn{3}{c}{\textbf{MLP-TanH Model $|$ Linear Data}} & $\leftarrow\chi_{real}$ \\
        \cmidrule(lr){4-6}\cmidrule(lr){7-9}
        & & & \textit{LR} & \textit{NLR} & \textit{GP} & \textit{LR} & \textit{NLR} & \textit{GP} & $\leftarrow\chi_{sim}$ \\
        \cmidrule{1-9}
\multirow{4}{*}{Baseline} & - & Random & - & $17.761$\sstd{$0.074$}  & -  & $17.847$\sstd{$0.355$} & -  & -  \\
& - & Optimization & - & $1.213$\sstd{$0.000$} & -  & $0.360$\sstd{$0.001$} & -  & -  \\
& - & Langevin & - & $1.218$\sstd{$0.002$} & -  & $0.288$\sstd{$0.001$} & -  & -  \\
& - & HMC & - & $1.216$\sstd{$0.002$} & -  & $0.275$\sstd{$0.001$} & -  & -  \\
\cmidrule{2-9}
\multirow{3}{*}{Fwd-KL} & \multirow{6}{*}{\rotatebox[origin=c]{90}{Gaussian}} & GRU &$2.415$\sstd{$0.269$} & -  & -  & -  & $15.632$\sstd{$0.283$} & -  \\
& & DeepSets &$1.402$\sstd{$0.017$} & -  & -  & -  & $16.046$\sstd{$0.393$} & -  \\
& & Transformer &$2.216$\sstd{$0.097$} & -  & -  & -  & $15.454$\sstd{$0.246$} & -  \\
\cmidrule{3-9}
\multirow{3}{*}{Rev-KL}& & GRU &$1.766$\sstd{$0.044$} & $1.216$\sstd{$0.001$} & $4.566$\sstd{$0.199$} & $0.375$\sstd{$0.001$} & $0.386$\sstd{$0.002$} & $0.524$\sstd{$0.019$} \\
& & DeepSets &$1.237$\sstd{$0.006$} & $1.216$\sstd{$0.001$} & $3963.694$\sstd{$5602.411$} & $0.365$\sstd{$0.000$} & $0.377$\sstd{$0.003$} & $0.385$\sstd{$0.011$} \\
& & Transformer &$1.892$\sstd{$0.113$} & $1.226$\sstd{$0.001$} & $4.313$\sstd{$0.707$} & $0.367$\sstd{$0.006$} & $0.382$\sstd{$0.003$} & $0.458$\sstd{$0.048$} \\
\cmidrule{2-9}
\multirow{3}{*}{Fwd-KL} & \multirow{6}{*}{\rotatebox[origin=c]{90}{Flow}} & GRU &$2.180$\sstd{$0.024$} & -  & -  & -  & $9.800$\sstd{$0.473$} & -  \\
& & DeepSets &$1.713$\sstd{$0.244$} & -  & -  & -  & $15.253$\sstd{$0.403$} & -  \\
& & Transformer &$1.632$\sstd{$0.070$} & -  & -  & -  & $7.949$\sstd{$0.419$} & -  \\
\cmidrule{3-9}
\multirow{3}{*}{Rev-KL} & & GRU &$1.830$\sstd{$0.081$} & \highlight{$1.214$\sstd{$0.001$}} & $5.690$\sstd{$0.196$} & $0.346$\sstd{$0.004$} & $0.349$\sstd{$0.001$} & $0.520$\sstd{$0.015$} \\
& & DeepSets &$1.282$\sstd{$0.036$} & $1.218$\sstd{$0.001$} & $11.690$\sstd{$10.602$} & \highlight{$0.339$\sstd{$0.003$}} & $0.344$\sstd{$0.002$} & $0.397$\sstd{$0.026$} \\
& & Transformer &$1.471$\sstd{$0.016$} & $1.226$\sstd{$0.004$} & $5.194$\sstd{$0.320$} & $0.346$\sstd{$0.002$} & $0.347$\sstd{$0.001$} & $0.480$\sstd{$0.030$} \\
\cmidrule[\heavyrulewidth]{1-9}
    \end{tabular}
    \caption{\textbf{Model Misspecification}. Results for model misspecification under different training data $\chi_{sim}$, when evaluated under MLP-TanH and Linear Data ($\chi_{real}$), with the underlying model as a linear and MLP-TanH model respectively.}
    \vspace{-4mm}
    \label{tab:misspec_model}
\end{table*}
\begin{table*}[t]
    \centering
    % \small
    \footnotesize	    
    \def\arraystretch{1.25}
    \setlength{\tabcolsep}{5pt}
    \begin{tabular}{lcr ccc cccc}
        \cmidrule[\heavyrulewidth]{1-9}
         &  &  & \multicolumn{6}{c}{\textit{$L_2$ Loss} ($\downarrow$)} \\
        \cmidrule(lr){4-9}
        \textbf{Objective} & $q_\varphi$ & \textbf{Model} & \multicolumn{3}{c}{\textbf{Linear Model $|$ GP Data}} & \multicolumn{3}{c}{\textbf{MLP-TanH Model $|$ GP Data}} & $\leftarrow\chi_{real}$ \\
        \cmidrule(lr){4-6}\cmidrule(lr){7-9}
        & & & \textit{LR} & \textit{NLR} & \textit{GP} & \textit{LR} & \textit{NLR} & \textit{GP}  & $\leftarrow\chi_{sim}$ \\
        \cmidrule{1-9}
\multirow{4}{*}{Baseline} & - & Random & -  & -  & $2.681$\sstd{$0.089$} &  -  & -  & $16.236$\sstd{$0.381$} \\
& - & Optimization & -  & -  & $0.263$\sstd{$0.000$} & -  & -  & $0.007$\sstd{$0.000$} \\
& - & Langevin & -  & -  & $0.266$\sstd{$0.001$} & -  & -  & $0.022$\sstd{$0.001$} \\
& - & HMC & -  & -  & $0.266$\sstd{$0.000$} & -  & -  & $0.090$\sstd{$0.002$} \\
\cmidrule{2-9}
\multirow{3}{*}{Fwd-KL} & \multirow{6}{*}{\rotatebox[origin=c]{90}{Gaussian}} & GRU &$0.268$\sstd{$0.000$} & -  & -  & -  & $14.077$\sstd{$0.368$} & -  \\
& & DeepSets &$0.269$\sstd{$0.001$} & -  & -  & -  & $14.756$\sstd{$0.280$} & -  \\
& & Transformer &$0.270$\sstd{$0.001$} & -  & -  & -  & $14.733$\sstd{$0.513$} & -  \\
\cmidrule{3-9}
\multirow{3}{*}{Rev-KL} & & GRU &$0.268$\sstd{$0.000$} & $0.269$\sstd{$0.000$} & $0.266$\sstd{$0.000$} & $0.334$\sstd{$0.005$} & $0.157$\sstd{$0.003$} & $0.080$\sstd{$0.003$} \\
& & DeepSets &$0.269$\sstd{$0.000$} & $0.269$\sstd{$0.000$} & \highlight{$0.265$\sstd{$0.000$}} & $0.331$\sstd{$0.003$} & $0.146$\sstd{$0.002$} & $0.063$\sstd{$0.000$} \\
& & Transformer &$0.269$\sstd{$0.000$} & $0.269$\sstd{$0.000$} & $0.267$\sstd{$0.000$} & $0.310$\sstd{$0.013$} & $0.155$\sstd{$0.006$} & $0.066$\sstd{$0.004$} \\
\cmidrule{2-9}
\multirow{3}{*}{Fwd-KL} & \multirow{6}{*}{\rotatebox[origin=c]{90}{Flow}} & GRU &$0.268$\sstd{$0.000$} & -  & -  & -  & $9.756$\sstd{$0.192$} & -  \\
& & DeepSets &$0.269$\sstd{$0.001$} & -  & -  & -  & $14.345$\sstd{$0.628$} & -  \\
& & Transformer &$0.269$\sstd{$0.000$} & -  & -  & -  & $8.557$\sstd{$0.561$} & -  \\
\cmidrule{3-9}
\multirow{3}{*}{Rev-KL} & & GRU &$0.268$\sstd{$0.000$} & $0.270$\sstd{$0.001$} & $0.266$\sstd{$0.000$} & $0.289$\sstd{$0.011$} & $0.120$\sstd{$0.004$} & $0.059$\sstd{$0.003$} \\
& & DeepSets &$0.269$\sstd{$0.000$} & $0.269$\sstd{$0.001$} & $0.266$\sstd{$0.000$} & $0.270$\sstd{$0.008$} & $0.115$\sstd{$0.002$} & $0.059$\sstd{$0.002$} \\
& & Transformer &$0.269$\sstd{$0.001$} & $0.270$\sstd{$0.000$} & $0.267$\sstd{$0.000$} & $0.293$\sstd{$0.008$} & $0.120$\sstd{$0.005$} & \highlight{$0.055$\sstd{$0.002$}} \\
\cmidrule[\heavyrulewidth]{1-9}
    \end{tabular}
    \caption{\textbf{Model Misspecification}. Results for model misspecification under different training data $\chi_{sim}$, when evaluated under GP Data ($\chi_{real}$), with the underlying model as a linear and MLP-TanH model respectively.}
    \vspace{-4mm}
    \label{tab:misspec_gp}
\end{table*}
\subsection{Variable-Dim}
\label{appdx:details_max_dim}
In this section, we provide the experimental details relevant to reproducing the results of Section~\ref{sec:experiments}. All the models are trained with streaming data from the underlying probabilistic model, such that every iteration of training sees a new set of datasets. Training is done with a batch size of $128$, representing the number of datasets seen during one optimization step. Further, we ensure that the datasets sampled resemble a uniform distribution over the feature dimensions, ranging from $1$-dimensional to the maximal dimensional setup. Evaluations are done with $25$ samples and we ensure that the test datasets used for each probabilistic model are the same across all the compared methods, i.e., baselines, forward KL, and reverse KL. We train the amortized inference model and the forward KL baselines for the following different probabilistic models:

\textbf{Mean of Gaussian (GM):} We train the amortization models over $50,000$ iterations using a linear warmup with $12,5000$ iterations over which the weight of the KL term in our proposed approach scales linearly from $0$ to $1$. We use an identity covariance matrix for the data-generating process, but it can be easily extended to the case of correlated or diagonal covariance-based Gaussian distributions. In this setup, we consider a maximum of $100$ feature dimensions.

\textbf{Gaussian Mixture Model (GMM):} We train the mixture model setup for $500,000$ iterations with $125,000$ iterations of warmup. We set the maximal feature dimensions as $5$ and experiment with $2$ and $5$ mixture components. While we do use an identity covariance matrix for the data-generating process, again, it can be easily extended to other cases.
% For all our experiments, we compute the average over 25 different samples (either from the approximate posterior, or 25 different optimization runs, etc.) to report the downstream metrics. For the optimization baseline, we perform a quick hyperparameter search for each dataset over the space of $\{\}$

\textbf{Linear Regression (LR):} The amortization models for this setup are trained for $100,000$ iterations with $25,000$ iterations of warmup. The maximal feature dimension considered for this task is $100$-dimensional, and the predictive variance $\sigma^2$ is assumed to be known and set as $0.25$.

\textbf{Nonlinear Regression (NLR):} We train the setup for $250,000$ iterations with $62,500$ iterations consisting of warmup. The maximal feature dimension considered is $100$-dimensional, and training is done with a known predictive variance similar to the LR setup. For the probabilistic model, we consider both a $1$-layered and a $2$-layered multi-layer perceptron (MLP) network with 32 hidden units in each, and either a \textsc{relu} or \textsc{tanh} activation function.

\textbf{Linear Classification (LC):} We experiment with a maximal $100$-dimensional setup with training done for $100,000$ iterations, out of which $25,000$ are used for warmup. Further, we train for both binary classification as well as a $5$-class classification setup.

\textbf{Nonlinear Classification (NLC):} We experiment with a maximal $100$-dimensional setup with training done for $250,000$ iterations, out of which $62,500$ are used for warmup. Further, we train for both binary classification as well as a $5$-class classification setup. For the probabilistic model, we consider both a $1$-layered and a $2$-layered multi-layer perceptron (MLP) network with 32 hidden units in each, and either a \textsc{relu} or \textsc{tanh} activation function.

\begin{figure*}
    \centering
    \captionsetup[subfigure]{font=scriptsize}
    \includegraphics[width=\textwidth]{Draft/Plots/dimension_trends/KL.pdf}
    \vspace{-7mm}
    \caption{\textbf{Trends of Performance over different Dimensions in Variable Dimensionality Setup:} We see that our proposed reverse KL methodology outperforms the forward KL one.}
    \vspace{-5mm}
    \label{fig:dim_kl}
\end{figure*}

\begin{figure*}
    \centering
    \captionsetup[subfigure]{font=scriptsize}
    \includegraphics[width=\textwidth]{Draft/Plots/dimension_trends/Model.pdf}
    \vspace{-7mm}
    \caption{\textbf{Trends of Performance over different Dimensions in Variable Dimensionality Setup:} We see that transformer models generalize better to different dimensional inputs than DeepSets.}
    \vspace{-5mm}
    \label{fig:dim_kl}
\end{figure*}

\begin{figure*}
    \centering
    \captionsetup[subfigure]{font=scriptsize}
    \includegraphics[width=\textwidth]{Draft/Plots/dimension_trends/Variational_Approximation.pdf}
    \vspace{-7mm}
    \caption{\textbf{Trends of Performance over different Dimensions in Variable Dimensionality Setup:} We see that normalizing flows leads to similar performances than Gaussian based variational approximation.}
    \vspace{-5mm}
    \label{fig:dim_kl}
\end{figure*}
\subsection{Model Misspecification}
\label{appdx:details_misspecification}
In this section, we provide the experimental details relevant to reproducing the results of Section~\ref{sec:experiments}.
All models during this experiment are trained with streaming data from the currently used dataset-generating function $\chi$, such that every iteration of training sees a new batch of datasets. Training is done with a batch size of $128$, representing the number of datasets seen during one optimization step. Evaluation for all models is done with $10$ samples from each dataset-generator used in the respective experimental subsection and we ensure that the test datasets are the same across all compared methods, i.e., baselines, forward KL, and reverse KL.

\textbf{Linear Regression Model:} The linear regression amortization models are trained following the training setting for linear regression fixed dimensionality, that is, $50,000$ training iterations with $12,500$ iterations of warmup. The feature dimension considered for this task is $1$-dimension. The model is trained separately on datasets from three different generators $\chi$: linear regression, nonlinear regression, and Gaussian processes, and evaluated after training on test datasets from all of them.
For training with datasets from the linear regression probabilistic model, the predictive variance $\sigma^2$ is assumed to be known and set as $0.25$. 
The same variance is used for generating datasets from the nonlinear regression dataset generator with $1$ layer, $32$ hidden units, and \textsc{tanh} activation function. 
Lastly, datasets from the Gaussian process-based generator are sampled similarly, using the GPytorch library~\cite{gardner2018gpytorch}, where datasets are sampled of varying cardinality, ranging from $64$ to $128$. We use a zero-mean Gaussian Process (GP) with a unit lengthscale radial-basis function (RBF) kernel serving as the covariance matrix. Further, we use a very small noise of $\sigma^2 = 1\mathrm{e}^{-6}$ in the likelihood term of the GP.
Forward KL training in this experiment can only be done when the amortization model and the dataset-generating function are the same: when we train on datasets from the linear regression-based $\chi$. Table \ref{tab:misspec_model} provides a detailed overview of the results.


\textbf{Nonlinear Regression Models:} The nonlinear regression amortization models are trained following the training setting for nonlinear regression fixed dimensionality, that is, $100,000$ training iterations with $25,000$ iterations of warmup. Here, we consider two single-layer perceptions with 32 hidden units with a \textsc{tanh} activation function. The feature dimensionality considered is $1$ dimension.
We consider the same dataset-generating functions as in the misspecification experiment for a linear regression model above. However, the activation function used in the nonlinear regression dataset generator matches the activation function of the currently trained amortization model. In this case, forward KL training is possible in the two instances when trained on datasets from the corresponding nonlinear regression probabilistic model. A more detailed overview of the results can be found in Table \ref{tab:misspec_model} and \ref{tab:misspec_gp}.

\begin{figure}
    \centering
    \includegraphics[width=\textwidth]{Draft/Plots/real_world/linear_regression_Vanilla.pdf}
    \caption{\textbf{Tabular Experiments $|$ Linear Regression with Diagonal Gaussian}: For every regression dataset from the OpenML platform considered, we initialize the parameters of a linear regression-based probabilistic model with the amortized inference models which were trained with a diagonal Gaussian assumption. The parameters are then further trained with maximum-a-posteriori (MAP) estimate with gradient descent. Reverse and Forward KL denote initialization with the correspondingly trained amortized model. Prior refers to a MAP-based optimization baseline initialized from the prior $\gN(0, I)$, whereas Xavier refers to initialization from the Xavier initialization scheme.}
    \label{fig:regression_linear_vanilla}
\end{figure}

\subsection{Tabular Experiments}
\label{appdx:details_tabular}
For the tabular experiments, we train the amortized inference models for (non-)linear regression (NLR/LR) as well as (non-)linear classification (NLC/LC) with $\vx \sim \mathcal{N}(\mathbf{0}, \mathbf{I})$ as opposed to $\vx \sim \gU(-\mathbf{1}, \mathbf{1})$ in the dataset generating process $\chi$, with the rest of the settings the same as \textsc{maximum-dim} experiments. For the nonlinear setups, we only consider the \textsc{relu} case as it has seen predominant success in deep learning. Further, we only consider a 1-hidden layer neural network with 32 hidden dimensions in the probabilistic model. 

After having trained the amortized inference models, both for forward and reverse KL setups, we evaluate them on real-world tabular datasets. We first collect a subset of tabular datasets from the OpenML platform as outlined in Appendix~\ref{appdx:datasets}. Then, for each dataset, we perform a 5-fold cross-validation evaluation where the dataset is chunked into $5$ bins, of which, at any time, $4$ are used for training and one for evaluation. This procedure is repeated five times so that every chunk is used for evaluation once.

For each dataset, we normalize the observations and the targets so that they have zero mean and unit standard deviation. For the classification setups, we only normalize the inputs as the targets are categorical. For both forward KL and reverse KL amortization models, we initialize the probabilistic model from samples from the amortized model and then further finetune it via dataset-specific maximum a posteriori optimization. We repeat this setup over $25$ different samples from the inference model. In contrast, for the optimization baseline, we initialize the probabilistic models' parameters from $\gN(0, I)$, which is the prior that we consider, and then train 25 such models with maximum a posteriori objective using Adam optimizer. 

While we see that the amortization models, particularly the reverse KL model, lead to much better initialization and convergence, it is important to note that the benefits vanish if we initialize using the Xavier-init initialization scheme. However, we believe that this is not a fair comparison as it means that we are considering a different prior now, while the amortized models were trained with $\gN(0, I)$ prior. We defer the readers to the section below for additional discussion and experimental results.




\end{document}

