\section{Introduction}
Fair allocation of indivisible goods is a fundamental problem at the
intersection of economics and computer science. The goal is to allocate
a set of $m$ goods among a set of $n$ agents with heterogeneous
preferences. In contrast to the case of divisible goods (known as
cake-cutting) fundamental concepts of fairness—such as envy-freeness and proportionality— do not carry over when pivoting from the
continuous to the discrete domain. 

In this work, we investigate (approximate) Maximin Share fairness (MMS), a
well-studied concept introduced by
\cite{BudishMMS}. This notion can be seen as the discrete analogue of
proportionality in the context of indivisible goods.  The objective is to
allocate a bundle to each agent, ensuring that her value is at
least her {\em maximin share} (or a significant fraction of
it). Intuitively, an agent's maximin share is the maximum value she
could guarantee by proposing a partition of the items into
$n$ parts and then receiving the least desirable bundle from that
division. 

Although for two agents with additive valuations, MMS allocations always exist, \cite{KurokawaProcacciaWang18} showed that this is not always the case for three or more agents.
Studying approximate MMS fairness has led to a surge of research in the past
decade, for additive (e.g., \cite{AmanatidisMNS17,GhodsiHSSY21,GargTaki21,AkramiGarg24}) but also for more general valuation classes (e.g., \cite{BarmanKrishnamurthy20,GhodsiHSSY22,SeddighinSeddighin24}). 
The case of subadditive valuations is wide-open with a huge (polylogarithmic) gap between the best known upper and lower bounds. Closing this gap is considered a major open problem in this area.


 

