\section{$\boldsymbol{\alpha}$-MMS($\mathbf{d}$) for subadditive valuations}
\label{sec:Reductions}
In this section, we present a thorough study of conditions of existence (and non-existence) of $\boldsymbol{\alpha}$-MMS($\mathbf{d}$) allocations for various combinations of $\boldsymbol{\alpha}$ and $\mathbf{d}$. We provide two characterization results for three agents and several impossibility results for many agents. 

\subsection{Three Agents}
 In this section we consider three agents with subadditive valuations, and we fully characterize the conditions of existence of $1/2$-MMS$(\mathbf{d})$  (Theorem~\ref{thm:three-general}) and of $(1,1/2,1/2)$-MMS$(\mathbf{d})$ allocations (Theorem~\ref{thm:three-general-approximate}), with respect to any vector $\mathbf{d}$. We prove those results via a series of Lemmas and Corollaries in Sections~\ref{sec:CharacterizationLemmas}~and~\ref{sec:ManyAgentsImpossib}.

 We give a key claim (Claim~\ref{cl:disjointAllocations}) to be used as a technical tool in the follow up lemmas; if there exist two allocations that ``satisfy'' all but one agent, and those allocations are disjoint in one MMS bundle of the last agent, then one of the two allocations will leave items that the last agent values by at least $1/2$. Hence, that allocation can be extended to include the last agent that is guaranteed to receive at least $1/2$. 

 \begin{claim}
\label{cl:disjointAllocations}
    Consider any instance of $n$ agents, and any vectors ${\bf d}=(d_1,\ldots, d_n)$ and $\boldsymbol{\alpha}=(\alpha_1,\ldots,\alpha_n)$ with $\alpha_n =1/2$. If there exist two  allocations $A$ and $A'$, that are both $\boldsymbol{\alpha}_{-n}$-MMS$({\bf d}_{-n})$ for the first $n-1$ agents, such that there exists an MMS bundle $X$ of the last agent, where $X \cap \left(\bigcup_{Y\in A}Y\right)$ and $X \cap \left(\bigcup_{Y'\in A'}Y'\right)$ are disjoint,  
    then there exists a $\boldsymbol{\alpha}$-MMS$(\bf d)$ allocation for all $n$ agents.
\end{claim}


    
\begin{proof}
    The last agent values by at least $1/2$ either $X\cap \left(\bigcup_{Y\in A}Y\right)$ or $X\setminus \left(\bigcup_{Y\in A}Y\right) \subseteq X\cap \left(\bigcup_{Y'\in A'}Y'\right)$. W.l.o.g. suppose that the former holds. Then, the allocation where the last agent gets $X\cap \left(\bigcup_{Y\in A}Y\right) $ and the others get their allocated bundle in $A'$ is valid and it is $\boldsymbol{\alpha}$-MMS$(\bf d)$.
\end{proof}
 



\subsubsection{Characterizations of $1/2$-MMS($\mathbf{d}$) guarantees}
In this section we provide a complete characterization of results regarding $1/2$-MMS$(\mathbf{d})$ for three agents with subadditive valuations, for any vector $\mathbf{d}=(d_1,d_2,d_3)$. We summarize the results in the following theorem; note that we use the value $\sum_{i=1}^3d_i$ to distinguish among different $\mathbf{d}$, however, we do not claim that there is any strong correlation. 

\begin{theorem}\label{thm:three-general}
    A $1/2$-MMS$(\mathbf{d})$ allocation exists for three agents with subadditive valuation functions, when (i) $\mathbf{d}=(3,2,2)$ or (ii)  $\sum_{i=1}^3(d_i)\geq 8$ and $d_i=1$ for at most one agent $i$. In any other case, there exists an instance with no $1/2$-MMS$(\mathbf{d})$ allocation.
\end{theorem}
\begin{proof}
    The positive results are derived by using Observation~\ref{obs:simpleReduction} and showing that there is always a $1/2$-MMS$(3,2,2)$ allocation (Lemma~\ref{Lemma:threehalfs}), and a $(1,1/2,1/2)$-MMS$(\mathbf{d})$ allocation, for $\mathbf{d}=(5,2,1)$ (Lemma~\ref{lem:(5,2,1)}), $\mathbf{d}=(4,3,1)$ (Lemma~\ref{lem:(4,3,1)}), and $\mathbf{d}=(4,2,2)$  (Lemma~\ref{lemma:3Agents4}). 
    
    The impossibility results are derived by using Observation~\ref{obs:simpleReduction} and showing that for any of the following $\mathbf{d}$, there exists an instance that no $1/2$-MMS$(\mathbf{d})$ allocation exists. This is shown for $\mathbf{d}=(k,1,1)$, for any $k\geq 1$ (Corollary~\ref{cor:d_i<k}), $\mathbf{d}=(4,2,1)$ (Lemma~\ref{lem:imposs(4,2,1)}), $\mathbf{d}=(3,3,1)$ (Lemma~\ref{lem:imposs(3,3,1)}), and $\mathbf{d}=(2,2,2)$ (Lemma~\ref{lem:nagentsImp1}). 
    
    All Lemmas and Corollaries appear in Sections \ref{sec:CharacterizationLemmas}~and~\ref{sec:ManyAgentsImpossib}.
\end{proof}





\subsubsection{Characterizations of $(1,1/2,1/2)$-MMS$(\bf d)$ guarantees }
In this section we provide a complete characterization of results regarding $(1,1/2,1/2)$-MMS$(\bf d)$ for three agents with subadditive valuations.

\begin{theorem}\label{thm:three-general-approximate}
    A $(1,1/2,1/2)$-MMS$(\bf d)$ allocation exists for three agents with subadditive valuation functions, when $\sum_{i=1}^3(d_i)\geq 8$, when $d_i=1$ for at most one agent $i$, and $\max_i d_i \geq 4$. In any other case, there exists an instance with no $1/2$-MMS$(\mathbf{d})$ allocation.
\end{theorem}

\begin{proof}
    The positive results are derived by using Observation~\ref{obs:simpleReduction} and showing that there is always a  $(1,1/2,1/2)$-MMS$(\mathbf{d})$ allocation, for $\mathbf{d}=(5,2,1)$ (Lemma~\ref{lem:(5,2,1)}), $\mathbf{d}=(4,3,1)$ (Lemma~\ref{lem:(4,3,1)}) and $\mathbf{d}=(4,2,2)$ (Lemma~\ref{lemma:3Agents4}). 
    
    The impossibility results are derived by using Observation~\ref{obs:simpleReduction} and showing that for any of the following $\mathbf{d}$, there exists an instance that no $1/2$-MMS$(\mathbf{d})$ allocation exists. This is shown for $\mathbf{d}=(k,1,1)$, for any $k\geq 1$ (Corollary~\ref{cor:d_i<k}), $\mathbf{d}=(4,2,1)$ (Lemma~\ref{lem:imposs(4,2,1)}), and $\mathbf{d}=(3,3,3)$ (Lemma~\ref{lem:imposs(3,3,3)}).

    All Lemmas and Corollaries appear in Sections \ref{sec:CharacterizationLemmas}~and~\ref{sec:ManyAgentsImpossib}.
    \end{proof}

\subsubsection{$\alpha$-MMS$(\bf d)$ guarantees for three agents}
\label{sec:CharacterizationLemmas}

In this section we give a series of lemmas regarding the existence of $1/2$-MMS$(\bf d)$ and $(1,1/2,1/2)$-MMS$(\bf d)$ allocations, or their impossibilities, for three agents. Those lemmas are used in the characterizations of the existence of $1/2$-MMS$(\bf d)$ and $(1,1/2,1/2)$-MMS$(\bf d)$ allocations of the previous sections, focusing on results for three agents.


    \begin{lemma}
\label{lem:imposs(4,2,1)}
There exists an instance of three agents with subadditive valuation functions, where no $1/2$-MMS$(4,2,1)$ allocation exists.
\end{lemma}

\begin{proof}
    Consider an instance with $M=\{h,g_1,g_2,g_3\}$, where $v_1(S)=1$ for any $S\subseteq M$, $v_2(S)=1$, if $h\in S$, otherwise $v_2(S)=1/3 \cdot |S|$, and the last agent has an additive valuation function over $M$, with $v_3(h)=1/3$ and $v_3(g)=2/9$ for any $g\neq h$. Note that the first agent can partition $M$ into four bundles, each valued at $1$, the second agent can partition $M$ into $(\{h\},\{g_1,g_2,g_3\})$, valuing each bundle at $1$, and the third agent values $M$ at $1$.

    Suppose that there exists an allocation $A$ that is $1/2$-MMS$(4,2,1)$. If $h\in A_3$, then the third agent should receive at least one more item. Then, the second agent should receive two items other than $h$ to form a bundle with a value of at least $1/2$ to her; this leaves no items for the first agent. If $h\notin A_3$, the third agent must receive all remaining items to form a bundle with a value of at least $1/2$. This leaves one item for each of the first two agents. We conclude that no such allocation $A$ exists.  
\end{proof}

\begin{lemma} 
\label{lem:(5,2,1)}
A $(1,1/2,1/2)$-MMS$(5,2,1)$ allocation exists for three agents with subadditive valuation functions.
\end{lemma}
\begin{proof}
    We denote by $S,T,Q$ the three agents, and by $S_j,T_j,Q_j$ their $j$-th MMS bundle, respectively. 
    Consider all the cuts $C_{ijk}=\{S_i \cup S_j \cup S_k\}$ that we offer to $T$. We first show the following claim:
    \begin{claim}
        There exists a cut $C_{ijk}$, such that $v_T(T_1\cap C_{ijk})\geq 1/2$ and $v_T(T_2\cap C_{ijk})\geq 1/2$. 
    \end{claim}
    \begin{proof}
        Suppose on the contrary that there is no such cut. Then, for each $C_{ijk}$, agent $T$ has value at least $1/2$ for the intersection of $C_{ijk}$ and either $T_1$ or $T_2$ (but not both due to our assumption). The reason is that if $T$'s value was less than $1/2$ for both intersections, due to subadditivity any $C_{i'j'k'}$ for which $C_{ijk} \cup C_{i'j'k'}=M$ would contradict our assumption. 
        Let $\phi(C_{ijk})=1$ if $T$ has value at least $1/2$ for the intersection with $T_1$, otherwise, $\phi(C_{ijk})=2$.

        W.l.o.g. suppose that  $\phi(C_{123})=1$, then $\phi(C_{145})=2$, o.w. $v_T(C_{123}\cap T_2)+v_2(C_{145}\cap T_2)<1$, which is a contradiction to subadditivity, since the union of those two sets is $T_2$, and $v(T_2)\geq 1$. Similarly $\phi(C_{245})=2$ and $\phi(C_{345})=2$. 
        For the same reason, since $\phi(C_{145})=2$, it should be that $\phi(C_{234})=1$, and then $\phi(C_{125})=2$. Finally it should be $\phi(C_{345})=1$, but we argue above that $\phi(C_{345})=2$, which is a contradiction to our assumption. 
    \end{proof}  


        
    W.l.o.g. suppose that $C_{123}$ is the cut satisfying the statement of the above claim. Then, $T$ has value at least $1/2$ for both $T_1^*=T_1\cap C_{123}$ and $T_2^*=T_2\cap C_{123}$ (Figure \ref{fig:3agents521}(a)). Then, the allocations $A=(S_4,T_1^*)$ and  $A'=(S_5,T_2^*)$ for agents $S$ and $T$ are $(1,1/2)$-MMS$(5,2)$ and they are disjoint (Figures \ref{fig:3agents521}(b) and \ref{fig:3agents521}(c) resp.). By Claim~\ref{cl:disjointAllocations} the lemma follows. 
\end{proof}


\begin{center}
\begin{figure}[t] 
\begin{center}
    \begin{tabular}{ccc}
\begin{tikzpicture}[scale=0.45]
\draw[black, ultra thick](0,2) rectangle +(2,3);
\draw[pattern={north west lines},pattern color=red]
(0,2) rectangle +(2,3);
  \draw[step=1cm,] (0,0) grid (2,5);

\node[anchor=north] at (0.5,6.25) {$T_1$};
\node[anchor=north] at (1.5,6.25) {$T_2$};
\node[anchor=west] at (-1.5,0.5) {$S_5$};
\node[anchor=west] at (-1.5,1.5) {$S_4$};
\node[anchor=west] at (-1.5,2.5) {$S_3$};
\node[anchor=west] at (-1.5,3.5) {$S_2$};
\node[anchor=west] at (-1.5,4.5) {$S_1$};
\end{tikzpicture} & \begin{tikzpicture}[scale=0.45]
%\draw[black, ultra thick](0,2) rectangle +(2,3);
\draw[pattern={north west lines},pattern color=red]
(0,2) rectangle +(1,3);
\draw[pattern={horizontal lines},pattern color=blue](0,1) rectangle +(2,1);
  \draw[step=1cm,] (0,0) grid (2,5);

\node[anchor=north] at (0.5,6.25) {$T_1$};
\node[anchor=north] at (1.5,6.25) {$T_2$};
\node[anchor=west] at (-1.5,0.5) {$S_5$};
\node[anchor=west] at (-1.5,1.5) {$S_4$};
\node[anchor=west] at (-1.5,2.5) {$S_3$};
\node[anchor=west] at (-1.5,3.5) {$S_2$};
\node[anchor=west] at (-1.5,4.5) {$S_1$};
\end{tikzpicture} &
\begin{tikzpicture}[scale=0.45]
%\draw[black, ultra thick](0,2) rectangle +(2,3);
\draw[pattern={north west lines},pattern color=red]
(1,2) rectangle +(1,3);
\draw[pattern={horizontal lines},pattern color=blue](0,0) rectangle +(2,1);
  \draw[step=1cm,] (0,0) grid (2,5);

\node[anchor=north] at (0.5,6.25) {$T_1$};
\node[anchor=north] at (1.5,6.25) {$T_2$};
\node[anchor=west] at (-1.5,0.5) {$S_5$};
\node[anchor=west] at (-1.5,1.5) {$S_4$};
\node[anchor=west] at (-1.5,2.5) {$S_3$};
\node[anchor=west] at (-1.5,3.5) {$S_2$};
\node[anchor=west] at (-1.5,4.5) {$S_1$};
\end{tikzpicture}\\
(a) & (b) & (c) \\
    \end{tabular}
    \end{center}
\caption{In (a), with a thick line we show the cut $C=C_{123}$, and the red bundles correspond to $T_1^*,T_2^*$. In (b) and (c), we illustrate the allocations $A$ and $A'$, respectively; agent $S$ gets a whole bundle in each allocation (denoted with blue color).}
\label{fig:3agents521}
\end{figure}  
\end{center}



\begin{lemma} 
\label{lem:(4,3,1)}
A $(1,1/2,1/2)$-MMS$(4,3,1)$ allocation exists for three agents with subadditive valuation functions.  
\end{lemma}
\begin{center}
\begin{figure} 
\begin{center}
    \begin{tabular}{ccc}
\begin{tikzpicture}[scale=0.45]
%(0,0) rectangle +(1,2);
%\draw[pattern={north west lines},pattern color=red]
(0,0) rectangle +(1,2);
\draw[black, ultra thick](0,2) rectangle +(3,2);
\draw[pattern={north west lines},pattern color=red]
(0,2) rectangle +(2,2);
  \draw[step=1cm,] (0,0) grid (3,4);

\node[anchor=north] at (0.5,5.25) {$T_1$};
\node[anchor=north] at (1.5,5.25) {$T_2$};
\node[anchor=north] at (2.5,5.25) {$T_3$};
\node[anchor=west] at (-1.5,0.5) {$S_4$};
\node[anchor=west] at (-1.5,1.5) {$S_3$};
\node[anchor=west] at (-1.5,2.5) {$S_2$};
\node[anchor=west] at (-1.5,3.5) {$S_1$};
\end{tikzpicture} & \begin{tikzpicture}[scale=0.45]
%(0,0) rectangle +(1,2);
\draw[pattern={north west lines},pattern color=red]
(0,2) rectangle +(1,2);
\draw[pattern={horizontal lines},pattern color=blue](0,1) rectangle +(3,1);
  \draw[step=1cm,] (0,0) grid (3,4);

\node[anchor=north] at (0.5,5.25) {$T_1$};
\node[anchor=north] at (1.5,5.25) {$T_2$};
\node[anchor=north] at (2.5,5.25) {$T_3$};
\node[anchor=west] at (-1.5,0.5) {$S_4$};
\node[anchor=west] at (-1.5,1.5) {$S_3$};
\node[anchor=west] at (-1.5,2.5) {$S_2$};
\node[anchor=west] at (-1.5,3.5) {$S_1$};
\end{tikzpicture} &
\begin{tikzpicture}[scale=0.45]
\draw[pattern={north west lines},pattern color=red]
(1,2) rectangle +(1,2);
\draw[pattern={horizontal lines},pattern color=blue](0,0) rectangle +(3,1);
  \draw[step=1cm,] (0,0) grid (3,4);

\node[anchor=north] at (0.5,5.25) {$T_1$};
\node[anchor=north] at (1.5,5.25) {$T_2$};
\node[anchor=north] at (2.5,5.25) {$T_3$};
\node[anchor=west] at (-1.5,0.5) {$S_4$};
\node[anchor=west] at (-1.5,1.5) {$S_3$};
\node[anchor=west] at (-1.5,2.5) {$S_2$};
\node[anchor=west] at (-1.5,3.5) {$S_1$};
\end{tikzpicture}\\
(a) & (b) & (c) \\
    \end{tabular}
    \end{center}
\caption{In (a), with a thick line we show the cut $C=\{S_1 \cup S_2\}$. Red bundles correspond to $T_1^*,T_2^*$ (a). In (b) and (c), we illustrate allocations $A$ and $A'$, respectively; agent $S$ gets a whole bundle in each allocation (denoted with blue color).}
\label{fig:3agents431}
\end{figure}  
\end{center}

\begin{proof}
    We denote by $S,T,Q$ the three agents, and by $S_j,T_j,Q_j$ their $j$-th MMS bundle, respectively. Consider the cut $C={S_1\cup S_2}$ that we offer to $T$. W.l.o.g., suppose that
    $\mathcal{X}^*_T(C)=\mathcal{X}_T(C)$ and let $T_1^*,T_2^* \in \mathcal{X}^*_T(C)$ (Figure \ref{fig:3agents431} (a)). 
    Then, the allocations $A=(S_3,T_1^*)$ and  $A'=(S_4,T_2^*)$ (Figures \ref{fig:3agents431} (b) and \ref{fig:3agents431}(c) resp.) for agents $S$ and $T$ are $(1,1/2)$-MMS$(4,3)$ and they are disjoint. By Claim~\ref{cl:disjointAllocations} the lemma follows.
\end{proof}

\begin{lemma} \label{lemma:3Agents4}
A $(1,1/2,1/2)$-MMS$(4,2,2)$ allocation exists for three agents with subadditive valuation functions. 
\end{lemma}
\begin{proof}
We denote by $S,T,Q$ the three agents, and by $S_j,T_j,Q_j$ their $j$-th MMS bundle, respectively. We show the existence of two allocations for agents $T$ and $S$ that are disjoint on some $Q_j$,   and by Claim \ref{cl:disjointAllocations} the proof completes. 

We define the cuts $C_{ij}=\{S_i \cup S_j\}$ that we offer to $T$. W.l.o.g. assume that $v_T(T'_1)\geq 1/2$, for $T_1' =T_1\cap  C_{12}$ and $v_T(T_1'')\geq 1/2$ for $T_1''=T_1 \cap C_{23}$; if this is not the case we can rename the bundles of agent $S$.  Figures \ref{fig:3agents422}(a) and \ref{fig:3agents422}(b) illustrate the bundles and the cuts, respectively. 


Next we offer the cut $C=\{(Q_1 \setminus S_1) \cup S_3\}$ to $T$ (Figure \ref{fig:3agents422a}(a)). We consider the intersection of $T_2$ with $C$, and let $T_2^*=\max\{T_2\cap C, T_2\cap (M\setminus C)\}$. Note that $M\setminus C = (Q_2 \setminus S_3) \cup S_1$, so the cut provides some symmetry between $S_1$ and $S_3$; $S_1$ intersect with $T'_1$ but not $T''_1$, and for $S_3$ is the other way around. So, no matter which set is $T_2^*$, it does not intersect either $S_1$ or $S_3$, which in turns does not intersect with either $T'_1$ or $T''_1$. Therefore it is w.l.o.g. to assume that $T_2^*=T_2\cap C$ (Figure \ref{fig:3agents422a}(a)) and then, the allocations $A=(S_1,T_2^*)$ and  $A'=(S_4,T_1'')$ (Figures \ref{fig:3agents422a}(b) and \ref{fig:3agents422a}(c) resp.) for agents $S$ and $T$ are $(1,1/2)$-MMS$(4,2)$ and they are disjoint on $Q_2$. By Claim~\ref{cl:disjointAllocations} the lemma follows.
\end{proof}

\begin{center}

\begin{figure}[t]
\begin{center}

\begin{tabular}{cc}

\small{\begin{tikzpicture}[scale=0.45]
    \draw[yslant=0.5,xslant=-1,color=black,ultra thick] (6,3) rectangle +(-2,-2);    
    \draw[yslant=0.5,color=black,ultra thick] (3,-1) rectangle +(2,2);
    \draw[yslant=-0.5,color=black,ultra thick](3,4) rectangle +(-2,-2);
    \draw[yslant=0.5,xslant=-1,pattern={north west lines},pattern color=red](6,3) rectangle +(-2,-1);
    \draw[yslant=-0.5,pattern={north west lines},pattern color=red](2,4) rectangle +(-1,-2);
  \draw[yslant=-0.5] (1,0) grid (3,4);
  \draw[yslant=0.5] (3,-3) grid (5,1);
  \draw[yslant=0.5,xslant=-1] (4,1) grid (6,3);
\node[anchor=south west] at (-0.2,2.5,0) {$S_1$};
\node[anchor=south west] at (-0.2,1.5,0) {$S_2$};
\node[anchor=south west] at (-0.2,0.5,0) {$S_3$};
\node[anchor=south west] at (-0.2,-0.5,0) {$S_4$};
\node[anchor=north west] at (0.45,4.7,0) {$Q_1$};
\node[anchor=north west] at (1.5,5.3,0) {$Q_2$};
\node[anchor=north west] at (0.5,-0.8,0) {$T_1$};
\node[anchor=north west] at (1.5,-1.2,0) {$T_2$};
\end{tikzpicture}} &
\small{\begin{tikzpicture}[scale=0.45]
     %\draw[yslant=0.5,xslant=-1,color=black,ultra thick] (6,3) rectangle +(-2,-2);    
    \draw[yslant=0.5,color=black,ultra thick] (3,-2) rectangle +(2,2);
    \draw[yslant=-0.5,color=black,ultra thick](3,3) rectangle +(-2,-2);
    %\draw[yslant=0.5,xslant=-1,pattern={north west lines},pattern color=red](6,2) rectangle +(-2,-1);
    \draw[yslant=-0.5,pattern={north west lines},pattern color=red](2,3) rectangle +(-1,-2);
    
  \draw[yslant=-0.5] (1,0) grid (3,4);
  \draw[yslant=0.5] (3,-3) grid (5,1);
  \draw[yslant=0.5,xslant=-1] (4,1) grid (6,3);
\node[anchor=south west] at (-0.2,2.5,0) {$S_1$};
\node[anchor=south west] at (-0.2,1.5,0) {$S_2$};
\node[anchor=south west] at (-0.2,0.5,0) {$S_3$};
\node[anchor=south west] at (-0.2,-0.5,0) {$S_4$};
\node[anchor=north west] at (0.45,4.7,0) {$Q_1$};
\node[anchor=north west] at (1.5,5.3,0) {$Q_2$};
\node[anchor=north west] at (0.5,-0.8,0) {$T_1$};
\node[anchor=north west] at (1.5,-1.2,0) {$T_2$};
\end{tikzpicture}} \\
(a) & (b) \\
$T_1' \in \mathcal{X}_T(C_{12})$ &  $T_1'' \in \mathcal{X}_T(C_{23})$ \\
\end{tabular}
    
\end{center}
\caption{In (a) and (b), with a thick line we show the cut $C_{12}$ and $C_{23}$, respectively. Red bundles correspond to $T_1'$ in (a) and $T_1''$ in (b).}
\label{fig:3agents422}
\end{figure} 
\end{center}


\begin{center}

\begin{figure}[t]
\begin{center}

\begin{tabular}{ccc}
\small{\begin{tikzpicture}[scale=0.45]
      
    \draw[yslant=0.5,color=black,ultra thick] (3,-2) rectangle +(2,1);
    \draw[yslant=0.5,color=black,ultra thick] (3,-3) rectangle +(1,3);
    \draw[yslant=0.5,color=black,ultra thick,white] (3,-2) rectangle +(1,1);

    \draw[yslant=0.5,pattern={north west lines},pattern color=red] (3,-3) rectangle +(1,3);
    \draw[yslant=0.5,pattern={north west lines},pattern color=red] (3,-2) rectangle +(2,1);
    \draw[yslant=-0.5,pattern={north west lines},pattern color=red](3,3) rectangle +(-1,-3);
    
    \draw[yslant=-0.5,color=black,ultra thick](3,3) rectangle +(-2,-3);
  \draw[yslant=-0.5] (1,0) grid (3,4);
  \draw[yslant=0.5] (3,-3) grid (5,1);
  \draw[yslant=0.5,xslant=-1] (4,1) grid (6,3);
\node[anchor=south west] at (-0.2,2.5,0) {$S_1$};
\node[anchor=south west] at (-0.2,1.5,0) {$S_2$};
\node[anchor=south west] at (-0.2,0.5,0) {$S_3$};
\node[anchor=south west] at (-0.2,-0.5,0) {$S_4$};
\node[anchor=north west] at (0.45,4.7,0) {$Q_1$};
\node[anchor=north west] at (1.5,5.3,0) {$Q_2$};
\node[anchor=north west] at (0.5,-0.8,0) {$T_1$};
\node[anchor=north west] at (1.5,-1.2,0) {$T_2$};
\end{tikzpicture}}
 &
 \small{\begin{tikzpicture}[scale=0.45]
    
      
    

    \draw[yslant=0.5,pattern={north west lines},pattern color=red] (3,-3) rectangle +(1,3);
    \draw[yslant=0.5,pattern={north west lines},pattern color=red] (3,-2) rectangle +(2,1);
    \draw[yslant=-0.5,pattern={north west lines},pattern color=red](3,3) rectangle +(-1,-3);
    \draw[yslant=0.5,xslant=-1,pattern={horizontal lines},pattern color=blue](3,3) (6,3) rectangle +(-2,-2);  
    \draw[yslant=0.5,pattern={horizontal lines},pattern color=blue](3,0) rectangle +(2,1);     
    \draw[yslant=-0.5,pattern={horizontal lines},pattern color=blue](3,4) rectangle +(-2,-1);  
    
     \draw[yslant=-0.5] (1,0) grid (3,4);
  \draw[yslant=0.5] (3,-3) grid (5,1);
  \draw[yslant=0.5,xslant=-1] (4,1) grid (6,3);
\node[anchor=south west] at (-0.2,2.5,0) {$S_1$};
\node[anchor=south west] at (-0.2,1.5,0) {$S_2$};
\node[anchor=south west] at (-0.2,0.5,0) {$S_3$};
\node[anchor=south west] at (-0.2,-0.5,0) {$S_4$};
\node[anchor=north west] at (0.45,4.7,0) {$Q_1$};
\node[anchor=north west] at (1.5,5.3,0) {$Q_2$};
\node[anchor=north west] at (0.5,-0.8,0) {$T_1$};
\node[anchor=north west] at (1.5,-1.2,0) {$T_2$};
\end{tikzpicture}}
 &
 \small{\begin{tikzpicture}[scale=0.45]
\draw[yslant=-0.5,pattern={north west lines},pattern color=red](2,3) rectangle +(-1,-2);
\draw[yslant=0.5,pattern={horizontal lines},pattern color=blue](3,-3) rectangle +(2,1);     
\draw[yslant=-0.5,pattern={horizontal lines},pattern color=blue](3,1) rectangle +(-2,-1);  
  \draw[yslant=-0.5] (1,0) grid (3,4);
  \draw[yslant=0.5] (3,-3) grid (5,1);
  \draw[yslant=0.5,xslant=-1] (4,1) grid (6,3);
\node[anchor=south west] at (-0.2,2.5,0) {$S_1$};
\node[anchor=south west] at (-0.2,1.5,0) {$S_2$};
\node[anchor=south west] at (-0.2,0.5,0) {$S_3$};
\node[anchor=south west] at (-0.2,-0.5,0) {$S_4$};
\node[anchor=north west] at (0.45,4.7,0) {$Q_1$};
\node[anchor=north west] at (1.5,5.3,0) {$Q_2$};
\node[anchor=north west] at (0.5,-0.8,0) {$T_1$};
\node[anchor=north west] at (1.5,-1.2,0) {$T_2$};
\end{tikzpicture}}
\\
(a) & (b) & (c) \\
$T_2^* \in \mathcal{X}_T(C)$ &  $A = \{S_1,T_2^*\}$ & $A' = \{S_4,T_1''\}$  \\
\end{tabular}
    
\end{center}
\caption{For cut $C=\{(Q_1 \setminus S_1) \cup S_3\}$, let $T_2^* \in \mathcal{X}_T(C)$ (as shown in (a)). Then, the allocations $A=(S_1,T_2^*)$ (b) and  $A'=(S_4,T_1'')$ (c) are disjoint on $Q_2$. We use blue for agent $S$ and red for agent $T$'s bundles.}
\label{fig:3agents422a}
\end{figure} 
\end{center}

 






\begin{lemma} 
\label{lem:imposs(3,3,3)}
There exists an instance of three agents with subadditive valuation functions, where no $(1,1/2,1/2)$ -MMS allocation exists.
\end{lemma}

\begin{proof}
Consider an instance with items $M=\{g(i,j,k)\mid \mbox{ for } i,j,k \le 3\}$
and the following MMS bundles for the agents:
\begin{itemize}
    \item for agent $S$: $S_i=\{g(i,j,k)\mid \forall j,k\}$ for $i\in \{1,2,3\}$,
    \item for agent $T$: $T_j=\{g(i,j,k)\mid \forall i,k\}$ for $j\in \{1,2,3\}$,
    \item for agent $Q$: $Q_k=\{g(i,j,k)\mid \forall i,j\}$ for $k\in \{1,2,3\}$.
\end{itemize}
In the following, we make the convention that $i+1 = (i\mod 3) +1$, $i+2 = (i+1 \mod 3) +1$, and similarly for the other indices $j$ and $k$. 


We will construct the values of all agents symmetrically, such that for each agent $R\in \{S,Q,T\}$, her value for set $B \subseteq M$ will be $v_R(B)=\max_i\{v_R(B \cap R_i)\}$, $v_R(R_i)=1$ for any $i\in \{1,2,3\}$, and $v_R(B)\in\{1/2+\epsilon,1/2-\epsilon\}$ for $B\subset R_i$, for some $i\in\{1,2,3\}$ (where those values will be chosen appropriately so that the valuation functions are subadditive). 
Thus we only need to define the valuation functions only for $B \subset R_i$ for any $i\in \{1,2,3\}$.

For any $R\in \{S,Q,T\}$, $i\in \{1,2,3\}$, and $B \subset R_i$ with $\lvert B\rvert \notin \{4,5\}$, 
\begin{equation}
   \label{eq:422T}
       v_R(B)=\begin{cases}
      1/2 + \epsilon, & \text{if } \lvert B\rvert \geq 6\\
      1/2 - \epsilon, & \text{if } \lvert B\rvert \leq 3
      \end{cases}
   \end{equation} 

 

Consider now the bundles of the form $B^*=R_i\cap((X_j\cap Y_{k})\cup Y_{k+1})$, for any $i,j,k\in \{1,2,3\}$, where $X$ and $Y$ are the other two agents but $R$, with $X\neq Y$. Note that the above sets $B^*$ have cardinality $4$. For any $B \subset R_i$ with $\lvert B\rvert \in \{4,5\}$ we define

\begin{equation}
   \label{eq:422T}
       v_R(B)=\begin{cases}
      1/2 + \epsilon, & \text{if $\lvert B\rvert = 4$, and $B$ is some $B^*$ bundle}\\
      1/2 - \epsilon, & \text{if $\lvert B\rvert = 4$, and $B$ is different from all $B^*$ bundle} \\
      1-v_R(R_i\setminus B), & \text{if $\lvert B\rvert = 5$}
      \end{cases}
   \end{equation} 



Subadditivity is trivially guaranteed for those valuation functions. We will only show the following claim to verify that the valuations are valid monotone valuations. 

\begin{center}
\begin{figure}[t]
\begin{center}

\begin{tabular}{ccc}
\small{\begin{tikzpicture}[scale=0.475]


\draw[yslant=0.5,pattern={north west lines},pattern color=red] (4,-3) rectangle +(1,2);
\draw[yslant=0.5,pattern={north west lines},pattern color=red] (4,-3) rectangle +(3,1);
\draw[yslant=0.5,xslant=-1,pattern={north west lines},pattern color=red](3,-1) rectangle +(1,1);
\draw[yslant=-0.5,pattern={north west lines},pattern color=red](4,3) rectangle +(-1,-2);
    
\draw[yslant=0.5,pattern={horizontal lines},pattern color=blue](4,-4) rectangle +(3,1);     
\draw[yslant=-0.5,pattern={horizontal lines},pattern color=blue](4,1) rectangle +(-3,-1); 


\draw[yslant=-0.5] (1,0) grid (4,3);
\draw[yslant=0.5] (4,-4) grid (7,-1);
\draw[yslant=0.5,xslant=-1] (3,-1) grid (6,2);
\node[anchor=south west] at (-0.25,1.5,0) {$S_1$};
\node[anchor=south west] at (-0.25,0.5,0) {$S_2$};
\node[anchor=south west] at (-0.25,-0.5,0) {$S_3$};
\node[anchor=north west] at (0.3,-0.6,0) {$T_3$};
\node[anchor=north west] at (1.3,-1.1,0) {$T_2$};
\node[anchor=north west] at (2.3,-1.6,0) {$T_1$};
\node[anchor=north west] at (0.4,3.7,0) {$Q_1$};
\node[anchor=north west] at (1.4,4.2,0) {$Q_2$};
\node[anchor=north west] at (2.4,4.7,0) {$Q_3$};
\end{tikzpicture} }
 &
 \small{\begin{tikzpicture}[scale=0.45]


 \draw[yslant=0.5,xslant=-1,pattern={crosshatch dots},pattern color=green](3,-1) rectangle +(1,1);
    \draw[yslant=0.5,pattern={crosshatch dots},pattern color=green] (4,-3) rectangle +(1,2);
     \draw[yslant=-0.5,pattern={crosshatch dots},pattern color=green](4,2) rectangle +(-3,-1);  
\draw[yslant=-0.5,pattern={crosshatch dots},pattern color=green](4,3) rectangle +(-1,-1); 

\draw[yslant=0.5,pattern={horizontal lines},pattern color=blue](4,-4) rectangle +(3,1);     
\draw[yslant=-0.5,pattern={horizontal lines},pattern color=blue](4,1) rectangle +(-3,-1); 
\draw[yslant=-0.5] (1,0) grid (4,3);
\draw[yslant=0.5] (4,-4) grid (7,-1);
\draw[yslant=0.5,xslant=-1] (3,-1) grid (6,2);
\node[anchor=south west] at (-0.25,1.5,0) {$S_1$};
\node[anchor=south west] at (-0.25,0.5,0) {$S_2$};
\node[anchor=south west] at (-0.25,-0.5,0) {$S_3$};
\node[anchor=north west] at (0.3,-0.6,0) {$T_3$};
\node[anchor=north west] at (1.3,-1.1,0) {$T_2$};
\node[anchor=north west] at (2.3,-1.6,0) {$T_1$};
\node[anchor=north west] at (0.4,3.7,0) {$Q_1$};
\node[anchor=north west] at (1.4,4.2,0) {$Q_2$};
\node[anchor=north west] at (2.4,4.7,0) {$Q_3$};
\end{tikzpicture}}\\
(a) & (b)\\
\end{tabular}\end{center}\caption{The red bundle corresponds to $T_1 \cap ((Q_1 \cap S_2) \cup S_1)$ and the blue one to $S_3$, as shown in (a). Agent $Q$ values the remaining items at $1/2-\epsilon$. In (b), the green bundle corresponds to $Q_1 \cap ((T_1 \cap S_1) \cup S_2)$ and the blue one to $S_1$. Agent $T$ values the remaining items at $1/2-\epsilon$.}
\label{fig:3agents322a}
\end{figure}
\end{center}

\begin{claim}
If $S\subset R_i$ is some $B^*$ set for $R$, then $R_i\setminus S$ is not a superset of some other $B^*$ set for $R$.
\end{claim}
\begin{proof}
First note that for any $B^*=R_i\cap((X_j\cap Y_{k})\cup Y_{k+1})$, it is $R_i\setminus B^* = R_i \cap (((X_{j+1} \cup X_{j+2})\cap Y_k)\cup Y_{k+2}) $. So, it is clear that $R_i\setminus B^*$ is a superset of bundles of the form  $R_i\cap((X_{j'}\cap Y_{k'})\cup Y_{k'+2})$, which are not considered as $B^*$. So, if $S$ is some $B^*$ set for $R$, considering $R_i$, $R_i\setminus S$ cannot be a superset of some $B^*$ set for $R$.
\end{proof}




Assume for the sake of contradiction that there exists a $(1,1/2,1/2)$-MMS allocation, $A=(A_S,A_T,A_Q)$. 
Due to the definition of the valuation functions we may assume that $A_S$ is one MMS bundle for $S$, and $A_T,A_Q$ are  subsets of one MMS bundle of $T$ and $Q$, respectively. Note that the values are symmetric and we make no assumptions for agent $S$. As a result we can prove the inaproximability of $(1/2,1,1/2)$-MMS and $(1/2,1/2,1)$-MMS.


Due to the symmetric instance, suppose w.l.o.g. that $A_S=S_3$ and $A_T\subseteq T_1$. Figure~\ref{fig:3agents322a}(a) shows an example where $T$ receives a minimum possible cardinality set;  Figure~\ref{fig:3agents322a}(b) shows the case where we considered $Q$ instead of $T$, to illustrate the symmetry. We will next show that it should be that $A_T\supset T_1\cap S_2$. 

If $|A_T|\geq 6$, clearly $A_T=T_1\setminus S_3 \supset T_1\cap S_2$.
If $|A_T| = 5$, based on the definition of the valuations, it should be that $T_1\setminus A_T$ is not a $B^*$ bundle for $T$. Note that $T_1\setminus A_T $ is a superset of $ T_1\cap S_3$ and has cardinality $4$. Therefore, $T_1\setminus A_T = T_1\cap ((Q_k\cap S_1)\cup S_3)$, for some $k\in \{1,2,3\}$, otherwise (i.e., if we replace $S_1$ with $S_2$) it would be a $B^*$ set for $T$. As a result, $A_T \supset T_1\cap S_2$. If $|A_T| = 4$, then $A_T$ should be a $B^*$ set for $T$. Since $A_T\cap S_3 =\emptyset$, $A_T$ should be of the form $A_T = T_1\cap ((Q_k\cap S_1)\cup S_2)$, for some $k\in \{1,2,3\}$, which means again that  $A_T \supset T_1\cap S_2$. 

So, in any case $A_S\cup A_T \supset (T_1\cap S_2)\cup S_3$. This leaves $Q$ with some $A_Q\subseteq Q_k$, for some $k\in\{1,2,3\}$, with $|A_Q|\leq 5$. Note that $Q_k\setminus A_Q = Q_k\cap ((T_1\cap S_2)\cup S_3)$, which $Q$ values by more than $1/2$, therefore $v_Q(A_Q)<1/2$, by the valuation definition, which contradicts the assumption of the existence of a $(1,1/2,1/2)$-MMS allocation.
\end{proof}


\subsection{Many Agents}
\label{sec:ManyAgentsImpossib}
In this section we show impossibility results regarding many agents with subadditive valuations. More specifically, we show the existence of $1/2+\epsilon$ inapproximability for any vector $\boldsymbol{d}$ (Lemma ~\ref{Lemma:upperhalf}), inapproximability up to any factor $\epsilon$ when agents have less than $n$ MMS bundles (Lemma~\ref{lem:nagentsImp1}) or for MMS($n,\ldots,n,\left\lfloor\frac{n}{3}\right\rfloor$) bundles (Lemma~\ref{lem:imposs(3,3,1)}). Our last Lemma shows that any inapproximability result for $1/2$-MMS$(\mathbf{d})$ implies an inapproximability for $(1/3+\epsilon)$-MMS$(\mathbf{d})$ (Lemma~\ref{Lemma:upperBound}).

\begin{lemma}\label{Lemma:upperhalf}
    For any number of agents $n$ and any vector $\boldsymbol{d}=(d_1,\ldots, d_n)$ there exists an instance, where agents have subadditive valuations,  where no $(1/2+\epsilon)$-MMS$(\boldsymbol{d})$ allocation exists, for any $\epsilon>0$. 
\end{lemma}
\begin{proof}
    We will construct a counterexample in which every set $S \subseteq M$ has value either $1/2$ or $1$ for each agent $i$. Those valuations are carefully constructed so that if an agent receives more than $1/2$ value, it is impossible for any other agent to receive more than $1/2$.

    We consider a set of $\prod_{i=1}^n d_i$ items, that we denote as $M = \left\{g(r_1,\ldots,r_n),r_i \le d_i\right\}$.
    Let the bundle $B_{i,j}=\left\{g(r_1,\ldots,r_n):r_i=j\right\}$, for all $i\leq  n$ and $j \le d_i$. We construct the functions of each agent $i$ and set of items $S \subseteq M$ as follows:
    $$v_i\left(S\right) = \begin{cases}
        0 & \text{if } S = \emptyset\\
        1 & \text{if } \exists j: B_{i,j} \subseteq S \\
        1/2 & \text{otherwise}\\
    \end{cases}$$
Observe that the functions are subbaditive, and that for each agent $i$, $v_i(M)=1$, and there exists a partition of the items into $d_i$ bundles, namely $B_{i,1},\ldots,B_{i,d_i}$, where agent $i$ has value $1$ for each of them.

Consider any arbitrary pair of agents $i \ne i'$, then  $B_{i,j} \cap B_{i',j'} = \left\{g(r_1,\ldots,r_n):r_i=j,r_{i'}=j'\right\}$ for every pair $(j,j')$. Assume for the sake of contradiction that there exists an $(1/2+\epsilon)$-MMS$(\boldsymbol{d})$ allocation, in which bundles $A_i$ and $A_{i'}$ are allocated to agents $i$ and $i'$, respectively. Since $v_i\left(A_i\right) > 1/2$, there exists some $B_{i,j} \subseteq A_i$ for some $j$, and similarly since $v_{i'}\left(A_{i'}\right) > 1/2$, there exists some $B_{i',j'} \subseteq A_{i'}$ for some $j'$. Hence, for each item $g = g(r_1,\ldots,r_n):r_i=j,r_{i'}=j'$ we have $g \in A_i$ and also $g \in A_{i'}$ but this contradicts that bundles $A_i$ and $A_{i'}$ belong to a valid allocation. Therefore, in this instance there is no $(1/2+\epsilon)$-MMS$(\boldsymbol{d})$ allocation.
\end{proof}

\begin{lemma} 
\label{lem:nagentsImp1}
For any number of agents $n$ and any vector $\boldsymbol{d}=(d_1,\ldots, d_n)$, with $d_i< n$ for all $i$, there exists an instance, where agents have subadditive valuations, where no $\epsilon$-MMS($\boldsymbol{d}$) allocation exists, for any $\epsilon>0$.
\end{lemma}

\begin{proof}
    This is straight forward. Assume the instance with $n$ agents, $n-1$ items and $v_i\left(S\right) = 1, \forall S \subseteq M,i \in N$. Each agent $i$ can partition $M$ into $d_i< n$ bundles that they value each with $1$. However, in every possible allocation there exists at least one agent without allocated items and her value is equal to $0$.
\end{proof}

\begin{corollary}
\label{cor:d_i<k}
    For any number of agents $n$, and vector $\mathbf{d}=(d_1,\ldots, d_n)$, if there exists a subset of agents $N'\subseteq N$ such that $d_i< |N'|$ for all $i\in N'$,
    then there exists an instance, where agents have subadditive valuations, where no $\epsilon$-MMS$(\mathbf{d})$ allocation exists, for any $\epsilon>0$. 
\end{corollary}

\begin{lemma}
\label{lem:imposs(3,3,1)}
    For any number of agents $n$, there exists an instance, where agents have subadditive valuations, where no $1/2$-MMS$(n,\ldots,n,\left\lfloor \frac{n}{3} \right\rfloor)$ allocation exists, or even no $\boldsymbol{\alpha}$-MMS$(n,\ldots,n,\left\lfloor \frac{n}{3} \right\rfloor)$ allocation exists, with $\alpha_n\geq 1/2$ and $\alpha_i>0$, for all $i<n$.
\end{lemma}

\begin{proof}
    Consider an instance with $n$ items, $M=\{g_1,\ldots,g_n\}$, where for any agents $i<n$, $v_i(S)=1$, for any $S\subseteq M$. Regarding agent $n$, we define the bundles $B_k=\{g_{3k-2}, g_{3k-1},g_{3k}\}$, for $1\leq k\leq \left\lfloor \frac{n}{3} \right\rfloor$, and then for any $S\subseteq M$, we define $v_n(S)$ as,
$$v_n(S)=\frac{\max_{1\leq k \leq \left\lfloor \frac{n}{3} \right\rfloor} \lvert S\cap B_k\rvert}3 \,.$$

Clearly, there are $\left\lfloor \frac{n}{3} \right\rfloor$ disjoint bundles of $M$, $(B_1,\ldots,B_{\left\lfloor \frac{n}{3}\right\rfloor})$, for which agent $n$ has value $1$.     
Consider any allocation $A$ that is $\boldsymbol{\alpha}$-MMS$(n,\ldots,n,\left\lfloor \frac{n}{3} \right\rfloor)$, for any $\boldsymbol{\alpha}$ with  $\alpha_n\geq 1/2$ and $\alpha_i>0$, for all $i<n$. Then, it should be that $|A_n|\geq 2$ which leaves at most $n-2$ items for the rest $n-1$ agents. So one agent $i<n$ would not receive any item in $A$, which violates the fact that $\alpha_i>0$, for all $i<n$.
\end{proof}




We show in the following lemma that all impossibility results of the non-existence of $1/2$-MMS allocations, extend to non-existence of $(1/3+\epsilon)$-MMS allocation, for any $\epsilon > 0$.

\begin{lemma} \label{Lemma:upperBound}
    Given a vector $\boldsymbol{d}=(d_1,\ldots, d_n)$, if there exists an instance $I$ of $n$ agents with subadditive valuations over $m$ items where no $1/2$-MMS$(\boldsymbol{d})$ allocation exists, then there is also an instance $I'$ of $n$ agents with subadditive valuations over $m$ items where no $(1/3+\epsilon)$-MMS$(\boldsymbol{d})$ allocation exists, for any $\epsilon > 0$. 
\end{lemma}

\begin{proof}
    We will show that we can transform any instance $I=(N,M,\boldsymbol{v})$ into $I'=(N,M,\boldsymbol{v}')$ such that for every $S \subseteq M$ and $i \in N$, it holds that $v_i(S) \ge 1/2 \Leftrightarrow v'_i(S) \ge 2/3$ and also  $v_i(S) < 1/2 \Leftrightarrow  v'_i(S) \le 1/3$. Then, if there was an $(1/3+\epsilon)$-MMS allocation in $I'$, this allocation would also be $2/3$-MMS in $I'$, and therefore  $1/2$-MMS in $I$. However, by the lemma's assumption, this would be a contradiction, and therefore, there is no $(1/3+\epsilon)$-MMS allocation in $I'$.  

    
    We next show how to derive $\boldsymbol{v}'$ from $\boldsymbol{v}$. Given any subadditive valuation function $v_i$ of agent $i$, we construct the valuation function $v_i'$ as follows:
        $$v'_i\left(S\right) = \begin{cases}
        0 & \text{if } S = \emptyset\\
        1/3 & \text{if } 0 \le v_i(S) < 1/2\\
        2/3 & \text{if } 1/2 \le v_i(S) < 1\\
        1 & \text{if } v_i(S) \ge 1 \\
    \end{cases}$$
By definition, it hold that  $v_i(S) \ge 1/2 \Leftrightarrow v'_i(S) \ge 2/3$ and $v_i(S) < 1/2 \Leftrightarrow  v'_i(S) \le 1/3$. We will then show that $v'_i$ is a subadditive function. 

Consider any two sets of items, $S_1$ and $S_2$, and w.l.o.g. assume that $v'(S_1) \le v'(S_2)$. We will show that $v_i'(S_1)+v_i'(S_2) \geq v'_i(S_1 \cup S_2)$. This inequality holds if $S_1 = \emptyset$, since then $v_i'(S_1)+v_i'(S_2) = v_i'(S_2) = v'_i(S_1 \cup S_2)$. So, suppose that $S_1 \neq \emptyset$, which means that 
    $v_i'(S_1)\geq 1/3 $ and $ v_i'(S_2)\geq 1/3$. If $ v_i'(S_2)\geq 2/3$, then the subadditivity is not violated since $v_i'(S_1)+v_i'(S_2) \geq 1 \geq v'_i(S_1 \cup S_2)$, where the last inequality holds since by definition, $v_i'(S)\leq 1$, for any set $S$. So, at last suppose that $v_i'(S_1)=v_i'(S_2)= 1/3$. Then, $v_i(S_1)<1/2$ and $v_i(S_2)<1/2$, so by subbaditivity on $v_i$, $v_i(S_1 \cup S_2)<1$, which means that $v_i'(S_1 \cup S_2)\leq 2/3 = v_i'(S_1)+v_i'(S_2)$; so in this case the subadditivity is also preserved. Overall, $v_i'$ is subadditive and the lemma follows.  
\end{proof}

\begin{corollary}
    For any number of agents $n$, any number of goods $m$, and any vector $\boldsymbol{d}=(d_1,\ldots, d_n)$, if an $(1/3+\epsilon)$-MMS$(\boldsymbol{d})$ allocation, for some $\epsilon > 0$, is guaranteed to exist, for any subadditive valuation functions of the agents, then there is always an $1/2$-MMS$(\boldsymbol{d})$ allocation when agents have subadditive valuations.
\end{corollary}
    

