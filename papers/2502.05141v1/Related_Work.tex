\subsection{Further Related Work}


We refer the reader to the recent survey of \cite{AmanatidisABFLMVW23Survey} covering a wide variety of discrete fair division settings along with the main fairness notions and their properties. We focus on the concept of maximin share (MMS).

\paragraph{Maximin share and $\alpha$-MMS.} MMS has seen significant progress during the past years. Nevertheless, the case beyond additive utilities remains underexplored; we summarize the state-of-the-art bounds in Table~\ref{table:Results} for the most well-known superclasses of additive valuations. Additionally,  
 $1/2$-MMS allocations are known to exist for SPLC valuations \cite{ChekuriKKM24}. \cite{Hummel:HSS24} proved the same guarantee for the case of hereditary set systems, improving upon the $11/30$ guarantee of \cite{LiVetta21}. In sharp contrast, a surge of works has led to strong approximations for additive valuations \cite{KurokawaProcacciaWang18,AmanatidisMNS17,GhodsiHSSY21,GargTaki21,AkramiGST23}. Notably, \cite{AkramiGarg24} recently showed an approximation factor of $3/4+3/3836$. 



\begin{table}[]
    \centering

    \begin{tabular}{c c c}
        \toprule
        Submodular & Existence & Non-existence
        \\
        \midrule
        $n=2$ & $2/3^{\dagger}$ & $2/3^{\ddagger}$ \\
        $n=3$ & $10/27^{\dagger\dagger}$ & 
        \begin{tabular}{c}
             $3/4^{\mathparagraph}$, \\
             $\mathbf{\color{red}{2/3}}$\textcolor{red}{[Thm \ref{thm:SubmodUB}]}
        \end{tabular}
        \\
        
        %$n \ge 4$ & $10/27^{\dagger\dagger}$ & $3/4^{\mathparagraph}$ \\
        \midrule
        XOS \\
        \midrule
        $n \le 4$ &
            \begin{tabular}{c}
              $3/13^{\mathsection}$, \\
             $\mathbf{\color{red}{1/2}}$ \textcolor{red}{[Thm \ref{thm:mainTheorem}]}
        \end{tabular}
         
        
        & $1/2^{\mathparagraph}$ \\
        
        %$n \ge 5$ & $3/13^{\mathsection}$ & $1/2^{\mathparagraph}$\\
        \midrule
        Subadditive \\
        \midrule
        $n \le 4$ &
                \begin{tabular}{c}
             $\Omega(\frac{1}{\log n \log \log n})^{**}$, \\
$\mathbf{\color{red}{1/2}}$ \textcolor{red}{[Thm \ref{thm:mainTheorem}]}
        \end{tabular}
            & $1/2^{\mathparagraph}$ \\ 
        %$n \ge 5$ & $\Omega(\frac{1}{\log n \log \log n})^{**}$ & $1/2^{\mathparagraph}$\\
        \bottomrule
    \end{tabular}
    \caption{Best known MMS approximations. Our contributions appear in \textbf{bold red} color. $^{\dagger}$ \cite{ChristodoulouChristoforidis}; $^{\ddagger}$\cite{KKM23}; $^{\dagger\dagger}$\cite{UziahuFeigeMMS}; $^{\mathparagraph}$\cite{GhodsiHSSY22}; $^{\mathsection}$\cite{AkramiMehlhornSeddighinShahkarami23}; $^{**}$\cite{SeddighinSeddighin24}.}
    
    \label{table:Results}
\end{table}

\paragraph{1-out-of-$d$-MMS.} \cite{BudishMMS} introduced the notion and showed the existence of 1-out-of-$(n+1)$-MMS, albeit with excess goods. \cite{Aigner-HorevSH22} obtained a bound of $d=2n-2$ on the existence of 1-out-of-$d$ allocations under additive valuations, which was subsequently improved to 1-out-of-$\lceil 3n/2 \rceil$ by \cite{HosseiniS21}, 1-out-of-$\lfloor 3n/2 \rfloor$ by \cite{HosseiniSSH22}, and recently to 1-out-of-$4\lceil n/3 \rceil$ by \cite{AkramiGST24}.
\cite{BabaioffNT21} introduced the $\ell$-out-of-$d$ maximin share, which corresponds to the maximum value an agent can guarantee to herself if she partitions the items into $d$ bundles and  then being allocated the worst $\ell$ of them. 



\paragraph{Few Agents.} Settings with few agents have been previously studied in the fair division literature, particularly for MMS approximations involving three or four agents with additive valuations. The algorithm of \cite{KurokawaProcacciaWang18} provides a $3/4$-MMS guarantee for three or four agents. \cite{AmanatidisMNS17} subsequently established a bound of $\alpha \ge 7/8$ for the case of three agents, which was later improved to $8/9$ by \cite{GourvesMonnot19} and finally to $11/12$ by \cite{FeigeNorkinMMS}. \cite{GhodsiHSSY21} improved upon the previously known factor by showing a $4/5$ lower bound for four agents. 









