\section{Submodular Valuations}
\label{sec:Submod}

We present an improved upper bound for three agents with submodular valuation functions that rules out the existence of better than $2/3$-MMS allocations. The case of two agents has been previously studied in \cite{KKM23} and \cite{ChristodoulouChristoforidis}, establishing a tight answer of $2/3$ for the approximability of the maximin share. 



\begin{theorem}
\label{thm:SubmodUB}
    There exists an instance of $3$ agents with submodular valuations and $6$ items for which there is no $(2/3+\epsilon)$-MMS allocation, for any $\epsilon>0$.
\end{theorem}

\begin{proof}
    Let the set of three agents be $N=\{a_1,a_2,a_3\}$ and let the set of $6$ items be $M = \{g_1, G_1, g_2, G_2, g_3, G_3\}$. Two of the agents, $a_1$ and $a_2$, share the same valuation function.
    
    There are two types of items: small items, denoted by $g_i$, and large items, $G_i$. For each agent $a\in N$ and bundle $S\subset M$, with $\lvert S \rvert = 1$ we define 
    $$v_a(S) =   \begin{cases}
    2/3, &\mbox{ if } S=\{G_i\} \\
      1/3, & \mbox{ if } S=\{g_i\} \\
      \end{cases}$$
   i.e. each agent values small items with $1/3$ and large items with $2/3$. 
   
   
   
     Regarding the bundles of size two, each agent considers a bundle either as ``low'', for which they have value $2/3$, or as ``high'', for which they have value $1$. All agents consider the bundles of two small items as low, and of two large items as high. However, a bundle of a small and a large item may be considered differently by the agents. For each of the two agents that share the same valuation function, i.e., for $a \in \{a_1,a_2\}$, and for any bundle $S\subset M$, with $\lvert S \rvert = 2$, we define $$v_a(S) =   \begin{cases}
      1, & S=\{g_i,G_i\} \text{ or } S=\{G_i,G_j\} \quad \forall i,j \\
      2/3, & \text{otherwise} 
    \end{cases}$$
    For agent $a_3$, we shift the combinations of large and small items as follows: 
    
    $$  v_{a_3}(S) =   \begin{cases}
      1, & S=\{g_{(i\mod 3) +1},G_{i }\}  \text{ or } S=\{G_i,G_j\}\quad \forall i,j\\ 
      2/3, & \text{otherwise}
    \end{cases}$$
    
    The distinction between small and large bundles is illustrated in Figure \ref{fig:sizetwo} for all agents. Regarding the bundles with higher than $2$ cardinality, for any agent $a \in N$ and bundle $S\subseteq M,\lvert S\rvert \ge 3$ we define 
    $$v_a(S) =   \begin{cases}
    1, &\mbox{ if } |S|\geq 4 \mbox{ or } S=\{g_1,g_2,g_3\} \\
      \max_{S' \subseteq S,\lvert S' \rvert = 2}\left\{v_a\left(S\right)\right\}, & \mbox{ otherwise } 
      \end{cases}$$
    
    
    
    Observe that the MMS value of all agents is equal to $1$, because they can partition the items into three bundles, each of value $1$, by appropriately combining a small and a large item, i.e., $a_1$ and $a_2$ can form the bundles $\{g_1,G_1\},\{g_2,G_2\}$ and $\{g_3,G_3\}$, and $a_3$ can form the bundles $\{g_1,G_3\},\{g_2,G_1\}$ and $\{g_3,G_2\}$. By construction if a bundle has value $1$ for some agent then it  contains all small items, or two large items or it is one of her MMS bundles. %Note also that the MMS partition of agent $a_3$ is the MMS partition of the other agents shifted by $-1$.
    
    
    We show in the following claim that the valuation functions are submodular. 
    
    \begin{claim}
        The valuation functions defined above are submodular functions.
    \end{claim}
    \begin{proof}
    
    We first argue that the function is monotone. Indeed, the value for any set of cardinality $1$ is at most $2/3$, of cardinality $2$ is between $2/3$ and $1$, of cardinality $3$ is at most $1$ and cannot be less than the value of any of its subsets by definition, and for greater cardinality is $1$. 
    
    We proceed by showing that the valuation $v_{a_1}=v$ satisfies the submodularity property, i.e., for any two sets $S$ and $T$, such that $S \subset T \subset M$, and for any item $g\in M\setminus T$ it holds \begin{equation}
    v\left(S \cup \{g\}\right) - v(S) \ge v(T \cup \{g\}) - v(T)\,. 
    \label{eq:sub}
    \end{equation} 
    For $v_{a_3}$ the arguments are the same by shifting the indices of the small items. 
    %  from definition for every $\lvert S'\rvert \ge 3$ the function is monotone. If $\lvert S\rvert = 1$ and $\lvert S'\rvert = 2$ then $v_a(S) \le 2/3 \le v_a(S')$. Hence there must hold $v_a(S' \cup \{g\}) - v_a(S')>0$ and thus $S$ cannot contain more than $3$ items. As a result $S$ is empty, contains $1$ or $2$ items. 

    

Note that if $g$ is a small item, its marginal contribution to any set of items is either $0$, or $1/3$. If $g$ is a large item, its marginal contribution to any set of items may be $0$, or $1/3$, or $2/3$. Further, note that if $|T|\geq 4$, all items has a zero marginal contribution, and hence inequality \eqref{eq:sub} holds. So, we assume $|T|\leq 3$, and we distinguish between the different cases for the cardinality of $S$, which should be at most two. 

    {\bf Case 1: $|S|=0$.} In this case $v(S\cup \{g\})-v(S)=v(\{g\})$ which is the maximum possible contribution of any item,  and so, inequality \ref{eq:sub} holds.

    {\bf Case 2: $|S|=1$.} In this case $T$ contains at least two items and the marginal contribution of $g$ to $T$ is at most $1/3$, i.e., $v(T \cup \{g\}) - v(T) \le 1/3$. If $S$ contains a small item, then $v(S\cup\{g\})-v(S)\geq 2/3-1/3=1/3$, since $S\cup\{g\}$ contains two items and the value for those sets is at least $2/3$; so inequality \eqref{eq:sub} holds. If $S=\{G_i\}$, for some $i\in\{1,2,3\}$, then the only case that $v(S\cup\{g\})-v(S)=0$, is for $g=g_j$, for some $j\neq i$. 
    
    We will show that in that case $v(T \cup \{g\}) - v(T)=0$. Suppose on the contrary that $v(T \cup \{g\})=1$ and $v(T)=2/3$. Since $G_i\in T$, $T$ cannot contain any other large item or $g_i$. Moreover, $g\notin T$, where recall that $g=g_j$, for some $j\neq i$. Therefore, $T=\{G_i,g_k\}$, for the $k\notin\{i,j\}$. However, by definition of the valuations, $v(T \cup \{g\}) = v(\{G_i,g_j,g_k\})=2/3$, for $i,j,k$ being different pairwise. So,  $v(T \cup \{g\}) - v(T)=0$, meaning that inequality \eqref{eq:sub} holds in that case as well.   

    {\bf Case 2: $|S|=2$.} In this case, $v(T \cup \{g\})=1$, since $T \cup \{g\}$ has cardinality four, and also $v(S)\geq 2/3$, since $S$ has cardinality two. So, the only way that \eqref{eq:sub} is violated is if $v(T)=v(S)=v(S\cup\{g\}=2/3$. The only sets of cardinality three with value $2/3$ are of the form $\{G_i,g_j,g_k\}$, for $i,j,k$ being different pairwise. So, suppose that $T=\{G_i,g_j,g_k\}$ and $S$ is either $\{g_j,g_k\}$, or contains $G_i$ If $g=g_i$, then no matter what $S$ is, $v(S\cup\{g\})=1$. Similarly, if $g=G_j$, for any $j\neq i$, $v(S\cup\{g\})=1$. In any case, our assumption is violated and therefore inequality \eqref{eq:sub} always holds in that case as well.  
    \end{proof}
    
    
    Next we show that in this instance there is no $(2/3+\epsilon)$-MMS allocation, for any $\epsilon>0$. Assume for the sake of contradiction that there exists an $(2/3+\epsilon)$-MMS allocation. In this allocation, no agent may receive a single item, since each item has value at most $2/3$ for each agent. For the same reason, no agent may receive three or more items, since then some other agent must recieve at most one item that she values by at most $2/3$. As a result each agent gets exactly two items. If in the allocation some agent recieves two large items then some other agent receives only small items and her value is at most $2/3$. Thus, each agent gets a small and a large item. If those items do not correspond to one of their MMS bundles, they value their allocated set by at most $2/3$. Hence, each agent must receive one of their MMS bundles. Due to the construction we can allocate to at most two agents one of their MMS bundles, meaning that there is no $(2/3+\epsilon)$-MMS allocation in this instance.
\end{proof}

% \begin{figure}
% \centering
%     \begin{tabular}{c|c c c c c c}
%      & $g_1$ & $G_1$ & $g_2$ & $G_2$ & $g_3$ & $G_3$\\
%      \hline
%      $a_1$ & 1/3 & 2/3 & 1/3 & 2/3 & 1/3 & 2/3 \\
%      $a_2$ & 1/3 & 2/3 & 1/3 & 2/3 & 1/3 & 2/3 \\
%      $a_3$ & 1/3 & 2/3 & 1/3 & 2/3 & 1/3 & 2/3 \\
% \end{tabular}
%     \caption{Values of singletons. Items $g_1,g_2$ and $g_3$ are the small items and items $G_1,G_2$ and $G_3$ are the large.}
%     \label{fig:singletons}
% \end{figure}

\begin{figure}
\centering
\begin{tabular}{c|c c c c c c c c c}
     & $g_1G_1$ & $g_1G_2$ & $g_1G_3$ & $g_2G_2$ & $g_2G_3$ & $g_2G_1$ & $g_3G_3$ & $g_3G_1$ & $g_3G_2$ \\
     \hline
     $a_1$ & 1 &  2/3 &  2/3 & 1 &  2/3 &  2/3 & 1 &  2/3 &  2/3\\
     $a_2$ & 1 &  2/3 &  2/3 & 1 &  2/3 &  2/3 & 1 &  2/3 &  2/3 \\
     $a_3$ & 2/3 & 2/3 & 1 & 2/3 & 2/3 & 1 & 2/3 & 2/3 & 1 
     
\end{tabular}


\begin{tabular}{c|c c}
    & $g_ig_j$ & $G_iG_j$ \\
    \hline
    $a_1$ & 2/3 & 1 \\
    $a_2$ & 2/3 & 1 \\
    $a_3$ & 2/3 & 1
\end{tabular}
    \caption{If a bundle is large for some agent then either it contains two large items or it is one of her MMS bundles.}
    \label{fig:sizetwo}
\end{figure}





