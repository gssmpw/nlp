\subsection{Our contribution}

We study the existence of approximate MMS allocations, a predominant notion of fairness in settings with indivisible goods, under submodular
and subadditive valuations. We focus on settings with few agents. We are able to show the following results.

\begin{itemize}
    \item In our main technical result (Theorem~\ref{thm:mainTheorem}), we show the existence of $1/2$-MMS allocations for at most four agents with subadditive valuations. We note that this result improves the best known bounds for many other valuation classes, including OXS, Gross Substitutes, submodular and XOS. We emphasize that the guarantee of our theorem matches the best known impossibility result due to \cite{GhodsiHSSY22}, thus settling the case of four agents for both subbaditive and XOS valuations (see also Table~\ref{table:Results}). 

    \item We show the existence of $1/2$-MMS allocations for multiple agents when they have one of two admissible valuation functions (Theorem~\ref{thm:two-types}).

    
    \item On the way to prove Theorem~\ref{thm:mainTheorem} we develop a new model that is useful for inductive arguments. This model incorporates nicely previously studied MMS variants, and we believe that it is of independent interest. In Section~\ref{sec:Reductions}, we provide a thorough study of approximate guarantees for three agents and we are able to provide two complete characterizations: one for the case of $1/2$-MMS$(\mathbf{d})$ (Theorem~\ref{thm:three-general}) and one for the case of $(1,1/2,1/2)$-MMS$(\bf d)$ (Theorem~\ref{thm:three-general-approximate}). 

    \item We show an improved impossibility result for three agents with submodular valuation functions: we present an instance in which an $\alpha$-MMS allocation does not exist for $\alpha > 2/3$ (Theorem~\ref{thm:SubmodUB}). This improves the previous best known result of 3/4 due to \cite{GhodsiHSSY22}.

    
\end{itemize}


