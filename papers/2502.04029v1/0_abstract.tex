\begin{abstract}
Autistic students often face challenges in social interaction, which can hinder their educational and personal development. This study introduces Echo-Teddy, a Large Language Model (LLM)-based social robot designed to support autistic students in developing social and communication skills. Unlike previous chatbot-based solutions, Echo-Teddy leverages advanced LLM capabilities to provide more natural and adaptive interactions. The research addresses two key questions: (1) What are the design principles and initial prototype characteristics of an effective LLM-based social robot for autistic students? (2) What improvements can be made based on developer reflection-on-action and expert interviews?  The study employed a mixed-methods approach, combining prototype development with qualitative analysis of developer reflections and expert interviews. Key design principles identified include customizability, ethical considerations, and age-appropriate interactions. The initial prototype, built on a Raspberry Pi platform, features custom speech components and basic motor functions. Evaluation of the prototype revealed potential improvements in areas such as user interface, educational value, and practical implementation in educational settings. This research contributes to the growing field of AI-assisted special education by demonstrating the potential of LLM-based social robots in supporting autistic students. The findings provide valuable insights for future developments in accessible and effective social support tools for special education.

\keywords{large language model \and agentic flow \and social robot \and autism education}
\end{abstract}
