\section{Conclusion}

Expert interviews on the initial prototype of Echo-Teddy identified four critical areas for enhancing its effectiveness in supporting social communication for autistic students. First, minimizing response time and ensuring stable interaction were emphasized as key factors in maintaining engagement. Second, the robot’s physical form and nonverbal behavior modules required careful refinement to align with the sensory and social preferences of autistic children. Third, experts highlighted the need for tailored interaction scenarios and reinforcement strategies that adapt to specific learning goals and contextual factors. Finally, considerations for practical distribution and long-term maintenance were noted as essential for ensuring scalability and usability in real-world educational and home settings. These findings reinforced the importance of designing Echo-Teddy to extend everyday communication experiences beyond controlled environments, making it a useful companion in inclusive classrooms, therapy sessions, and home interactions.

Future research will focus on addressing these expert recommendations by enhancing Echo-Teddy’s engineering performance in key areas such as response time, motion control, and camera-based recognition. Further development will also explore interaction scenarios specifically tailored to the social and emotional characteristics of autistic students, ensuring that conversations and behavioral prompts align with their needs. Another priority is testing compatibility with Augmentative and Alternative Communication (AAC) input methods, allowing for multi-modal interactions that accommodate a broader range of communication preferences. The research team will also investigate modular design options, offering variations such as teddy bears or other animal forms to improve user preference and increase long-term engagement.

To evaluate Echo-Teddy’s real-world impact, longitudinal studies in inclusive classrooms will assess its role as a mediator between autistic and non-autistic students, examining how it facilitates peer interaction and social skill development over time. Through these refinements, Echo-Teddy aims to become a practical and scalable tool in special education, helping children and adolescents on the autism spectrum develop meaningful communication skills in both structured and naturalistic settings.