
\section{Methodology}

\subsection{Research Procedure}

This study employed a systematic design and evaluation process to develop Echo-Teddy, an LLM agent-based social robot for autistic students. The research proceeded in three phases: reviewing existing studies, developing the prototype, and conducting expert evaluations to refine its design and functionality. The first phase involved a comprehensive review of prior research on social interaction support for autistic students and robot-assisted interventions. This review informed the key design principles guiding the development of Echo-Teddy. The design emphasized context-aware conversational adaptability using an LLM, voice-based interaction for natural communication, non-verbal cues to enhance engagement, and a cost-effective hardware structure to ensure scalability.
Second, based on these principles, a prototype was developed using Raspberry Pi as the hardware platform, integrating LLM-based conversational software to facilitate interactive dialogue. The system was designed to process voice input and generate responses using synthesized speech. Additionally, simple non-verbal gestures were incorporated to support engagement, allowing the robot to communicate using both verbal and physical cues. Last, following the prototype development, reflective evaluation by the development team and expert interviews were conducted to assess Echo-Teddy’s usability and identify areas for improvement. Five special education experts participated in semi-structured interviews, evaluating aspects such as response speed, physical design, non-verbal communication, and educational applicability. The qualitative analysis of these evaluations identified key areas for refinement, informing subsequent improvements to the prototype. Insights from these findings shaped the next iteration of Echo-Teddy, ensuring better alignment with the needs of autistic students and practical classroom implementation.

\subsection{Research Tools}

\subsubsection{Hardware and Software}

Echo-Teddy integrates hardware and software components to create an interactive system designed specifically for autistic students. The hardware is built around a Raspberry Pi 5, which serves as the central processing unit. It is connected to a microphone for voice input, a speaker for audio output, and a motor for basic physical gestures. These components are enclosed within a child-friendly, soft teddy bear exterior, ensuring both a comforting design and practical durability. The use of affordable and widely available hardware components enhances cost-effectiveness and scalability, making Echo-Teddy a viable solution for broader implementation. Additional hardware specifications are detailed in Appendix 1.

The software architecture integrates cloud-based AI services to enhance interaction. Amazon Web Services (AWS) processes Speech-to-Text (STT), converting verbal input into text for further analysis. OpenAI’s GPT-4o-mini generates responses dynamically, allowing for context-aware and adaptive conversations that align with the user’s input and behavioral patterns. To accommodate Korean-speaking users, Naver Clova Voice is used for Text-to-Speech (TTS), ensuring natural and engaging auditory output. A prompt management module structures the interaction, refining the model’s responses to align with social appropriateness and individual student needs. The backend, built using FastAPI and hosted on AWS, facilitates efficient communication between the hardware and cloud-based AI services, ensuring scalability, low-latency processing, and reliable performance in real-time interactions.

\subsubsection{Interview Questionnaire}

The interview questionnaire for Echo-Teddy evaluation were designed to cover four key categories: \textit{Affordance}, \textit{Usability}, \textit{Instructional Design}, and \textit{Instructional Usefulness}. These categories were chosen to provide a comprehensive assessment of the tool from various perspectives. The \textit{Affordance} focuses on initial perceptions and understanding of the tool's functions. \textit{Usability} address the overall satisfaction, functionality, design, and interactive aspects such as sound and movement. The \textit{Instructional Design} delves into the educational context, exploring potential use scenarios, learning processes, and expected outcomes in terms of behavior, knowledge, skills, and attitudes. Finally, the \textit{Instructional Usefulness} examines the tool's effectiveness in achieving learning goals, its appropriateness for different learners, and potential benefits, both intended and unexpected. This structured approach to the interview questions ensures a thorough evaluation of Echo-Teddy, covering both its technical aspects and its educational value. By addressing these diverse areas, the interviews aim to gather comprehensive insights that will inform future improvements and adaptations of the tool, ultimately enhancing its effectiveness in supporting autistic students.

% Reflection
% 인터뷰 질문지
\begin{table*}[hbt!]
\centering
\scriptsize
\begin{tabular}{p{3cm}p{3cm}p{6cm}}
\hline
\multicolumn{1}{c}{\textbf{Category}} & \multicolumn{1}{c}{\textbf{Aspect}} & \multicolumn{1}{c}{\textbf{Question}} \\
\hline
\multirow{2}{*}{Affordance} & Use intention & What did you think you would want to do when you first saw the tool? \\
\cline{2-3}
 & Function understanding & Explicit vs. implicit function understanding? \\
\hline
\multirow{5}{*}{Usability} & Satisfaction & Overall satisfaction with the tool? \\
\cline{2-3}
 & Functionality & Is the functionality appropriate? \\
\cline{2-3}
 & Design & Is the visual design appropriate? \\
\cline{2-3}
 & Sound/Voice & Is the sound/voice of the tool appropriate? \\
\cline{2-3}
 & Movement & Is the movement of the tool appropriate? \\
\hline
\multirow{8}{*}{Instructional Design} & Use context & Where can the tool be used? \\
\cline{2-3}
 & Alternative methods & Other ways to achieve the learning goal? \\
\cline{2-3}
 & Learning process & What is the ideal learning process with the tool? \\
\cline{2-3}
 & Behavior outcome & What behavior change should occur? \\
\cline{2-3}
 & Knowledge outcome & What key knowledge should be learned? \\
\cline{2-3}
 & Skill outcome & What skills should be learned? \\
\cline{2-3}
 & Affective outcome & What attitudes or thoughts should change? \\
\cline{2-3}
 & Information format & How should the tool present information? \\
\hline
\multirow{11}{*}{Instructional Usefulness} & Learning process & What should the learning process look like? \\
\cline{2-3}
 & Behavior outcome & What behavior change will occur? \\
\cline{2-3}
 & Knowledge outcome & What knowledge will be learned? \\
\cline{2-3}
 & Skill outcome & What skills will be learned? \\
\cline{2-3}
 & Affective outcome & What attitudes or thoughts will change? \\
\cline{2-3}
 & Validity & Is the tool effective in achieving goals? \\
\cline{2-3}
 & Generality & Is the tool appropriate for all learners? \\
\cline{2-3}
 & \multirow{3}{*}{Usefulness} & Does the tool have any benefits? \\
 & & - What intended benefits are there? \\
 & & - What unexpected benefits are there? \\
\cline{2-3}
 & Improvements & What needs improvement and why? \\
\hline
\end{tabular}
\caption{Interview Questions for Echo-Teddy Evaluation}
\end{table*}