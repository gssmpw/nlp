\section{Introduction}

Autistic students often face significant challenges in social interaction, a core aspect of their educational experience \cite{rubin2004challenges}\cite{van2015higher}. These difficulties can lead to isolation and hinder their academic and personal development \cite{whitby2009academic}. Recognizing this issue, researchers have increasingly focused on the potential of social robots as companions to help autistic students navigate social interactions \cite{lorenzo2021action}\cite{kouroupa2022use}\cite{perez2024analysis}\cite{so2018using}.

While previous studies have explored various approaches to developing social robots for autistic students, these efforts have encountered several key limitations. Many existing solutions rely on rule-based or scripted chatbots \cite{bradford2020hear}\cite{halabieh2024computer}, which lack the flexibility to adpt the unique communication needs of autistic individuals. The design of these systems often fails to consider the varied ways autistic students process and respond to social cues, leading to unnatural and ineffective interactions \cite{xygkou2024can}\cite{gu2024technological}. Additionally, the robots used in previous research tend to be expensive and difficult to mass-produce or distribute widely, limiting their accessibility and potential impact \cite{alcorn2019educators}\cite{tennyson2016accessible}. High development and production costs remain a major obstacle, restricting the scalability and widespread adoption of social robots in educational settings.

Recent advancements in artificial intelligence (AI), particularly in Large Language Models (LLMs), have created new opportunities to enhance social robot interactions. While LLMs are commonly used in education as chatbots \cite{adeshola2024opportunities}\cite{lo2023impact}, there is a growing trend toward their use in more autonomous, adaptive, and decision-making roles \cite{lee23generative}. Unlike traditional chatbots, which rely on predefined scripts and responses, LLM agents can maintain conversational context, adapt dynamically to user input, and make decisions based on real-time interactions. This agentic approach is increasingly being integrated into robotics research, enabling social robots to engage in more natural, context-aware, and personalized interactions \cite{wang2024large}\cite{kim2024survey}\cite{driess2023palm}.

To address the limitations of existing research and leverage these advancements, we introduce Echo-Teddy, an LLM agent-based social robot designed specifically for autistic students. Unlike conventional chatbot-driven robots, Echo-Teddy integrates verbal and non-verbal communication cues, allowing for more responsive and meaningful interactions. By implementing LLM agents, Echo-Teddy maintains conversational context, adapts to individual student needs, and makes nuanced decisions in real-time. Additionally, we prioritize cost-effectiveness by utilizing affordable hardware such as Raspberry Pi, ensuring that Echo-Teddy remains a scalable and accessible solution in educational robotics.

Our study focuses on two primary research questions:
\begin{itemize}
    \item RQ1: What are the design principles and initial prototype characteristics of Echo-Teddy as an LLM agent-based social robot?
    \item RQ2: How can the initial prototype of Echo-Teddy be improved based on developer reflection-on-action and expert feedback?
\end{itemize}

\subsection{Contribution}

By exploring these questions, our research contributes to the development of more accessible, effective, and scalable social support tools for autistic students. Echo-Teddy not only addresses immediate limitations in social robot design but also lays the groundwork for future innovations at the intersection of artificial intelligence, robotics, and special education. In particular, our iterative evaluation process—combining developer insights and expert feedback—ensures that future versions of Echo-Teddy are better aligned with the needs of autistic students and the realities of special education environments.

The key contributions of this research are:
\begin{itemize}
    \item Development of Echo-Teddy, a novel LLM agent-based social robot specifically designed to address the limitations of previous chatbot-based solutions for autistic students.
    \item Exploration of LLM agent capabilities in educational robotics, demonstrating their potential for more adaptive, context-aware, and personalized interactions.
    \item Identification of key design principles for effective and scalable LLM agent-based social robots in special education.
    \item Implementation of a cost-effective prototype, leveraging affordable hardware like Raspberry Pi, to demonstrate the feasibility of accessible and scalable robotics solutions.
\end{itemize}


\begin{figure*}[hbt!]
    \centering
    \includegraphics[width=1\textwidth]{imgs/echo-teddy-arch.pdf}
    \caption{System architecture of echo teddy.}
    \label{fig:echo-teddy-arch}
\end{figure*}

