\newif\ifanon\anonfalse
\newif\iffullversion\fullversiontrue % fullversiontrue % 
\newif\ifcomments\commentsfalse % \commentstrue %

\newcounter{casestudy}

\documentclass[letterpaper,twocolumn,10pt]{article}
\usepackage{usenix}

%\documentclass{ecai}
\usepackage{graphicx}
\usepackage{latexsym}
\usepackage{hyperref}

\newcommand{\C}{{C}}
\renewcommand{\P}{\textsf{P}}
\newcommand{\NP}{\textsf{NP}}
\newcommand{\PSPACE}{\textsf{PSPACE}}

\usepackage{bbm}
\usepackage{mathtools}
\usepackage{tikz}
\usepackage{tabularx}
\usepackage{color,soul}
\usepackage{float} 
\usepackage{xspace}


\usetikzlibrary{arrows.meta}
\usetikzlibrary{calc,shapes.callouts,shapes.arrows}
\usetikzlibrary{positioning,chains,fit,shapes,calc}

\usepackage{amsmath,amssymb,amsthm}
\usepackage[capitalise]{cleveref}
\newtheorem{thm}{Theorem}
\newtheorem{definition}[thm]{Definition}
\newtheorem{lemma}[thm]{Lemma}
\newtheorem{algorithm}[thm]{Algorithm}
\newtheorem{mechanism}[thm]{Mechanism}
\newtheorem{contract}[thm]{Contract}
\newtheorem{attack}[thm]{Attack}
\newtheorem{remark}[thm]{Remark}

\crefname{thm}{Theorem}{Theorems}

\newcommand{\E}{\mathbb{E}}
\newcommand{\sign}{\textrm{sign}}
\newcommand{\negl}{\textsf{negl}}
\newcommand{\poly}{\textsf{poly}}

\ifcomments
    \newcommand{\sunoo}[1]{[\textcolor{magenta}{Sunoo: #1}]}
    \newcommand{\daji}[1]{[\textcolor{blue}{Daji: #1}]}
    \newcommand{\elettra}[1]{[\textcolor{violet}{Elettra: #1}]}
\else
    \newcommand{\sunoo}[1]{\ignorespaces}
    \newcommand{\daji}[1]{\ignorespaces}
    \newcommand{\elettra}[1]{\ignorespaces}
\fi

% bold-and-italic formatting
\newcommand{\bi}[1]{\textbf{\textit{#1}}}

% paragraph header with less spacing
\newcommand{\subh}[1]{\smallskip \noindent \textbf{{#1}}.}
\renewcommand{\paragraph}[1]{\subh{#1}}

%\iffullversion\else
    % reduces bullet point list spacing
    \usepackage[shortlabels]{enumitem}
    \setlength{\parskip}{0pt plus2pt minus2pt}
    \setlist[itemize]{noitemsep, topsep=0pt, leftmargin=.2in}
    \setlist[enumerate]{noitemsep, topsep=0pt, leftmargin=.2in}
%\fi

\newcommand{\qone}{How do interoperation and economic incentives clash
(or not)?\xspace}
\newcommand{\qtwo}{How do security arguments and economic incentives align (or not)?\xspace}
\newcommand{\qthree}{What needs to change in order for the company to support interoperation?\xspace}
\newcommand{\qfour}{What tension exists between security and other goals
both before and after?\xspace}

\newcommand{\question}[1]{\smallskip \noindent \textbf{{#1}}}


\begin{document}

% don't want date
\date{}


\title{\Large \bf {SoK}: ``Interoperability vs Security'' Arguments: A Technical Framework}


\ifanon
    \author{Anonymized for submission}
\else
    \author{
    {\rm Daji Landis}\\
    New York University
    \and
    {\rm Elettra Bietti}\\
    Northeastern University
    \and
    {\rm Sunoo Park}\\
    New York University
    } 
\fi

\maketitle

\begin{abstract}
Concerns about big tech's monopoly power have featured prominently in recent media and policy discourse, and regulators across the US, the EU, and beyond have ramped up efforts to promote healthier competition in the market. One of the favored approaches is to require certain kinds of \emph{interoperation} between platforms, to mitigate the current concentration of power in the biggest companies.
Unsurprisingly, interoperability initiatives have generally been met with vocal resistance by big tech companies. Perhaps more surprisingly, a significant part of that pushback has been in the name of \emph{security}---that is, arguing against interoperation on the basis that it will undermine security.

We conduct a detailed examination of ``security vs. interoperability'' arguments in the context of recent antitrust proceedings in the US and the EU. First, we propose a \emph{taxonomy} of such arguments. Second, we provide several detailed \emph{case studies}, which illustrate our taxonomy's utility in disentangling where security and interoperability are and are not in tension, where securing interoperable systems presents novel engineering challenges, and where ``security arguments'' against interoperability are really more about anti-competitive behavior than security. Third, we undertake a \emph{comparative analysis} that highlights key considerations around the interplay of economic incentives, market power, and security across diverse contexts where security and interoperability may appear to be in tension.
We believe systematically distinguishing cases and patterns within our taxonomy and analytical framework can be a valuable analytical tool for experts and non-experts alike in today's fast-paced regulatory landscape.


\end{abstract}

\section{Introduction}

Billions of users of phones and other digital devices today entrust vast amounts of personal data to their devices and to services that are controlled, in general, by a small number of very large companies. These companies have had huge discretion over whom to share this data with, and how to secure these vast data flows---in messaging, digital payments, tap-and-go technology, and more. Recent concerns around concentration of power in big tech have led to calls and legal mandates for platforms to support increased interoperation. The idea is that this would open the market to more smaller players, promoting healthier competition, and break away from existing large-scale business models involving expansive control over software and hardware ecosystems by a single market actor. Large tech platforms have every incentive to resist moves to open up their systems. In this context, companies seeking to resist interoperability mandates are increasingly arguing that the security of their systems requires keeping them closed and limiting competition. They also portray their systems' security as a key business proposition, and a key reason consumers are attracted to their business models. 

These kinds of statements tend to appear in public-facing communications and sometimes legal or regulatory filings. As such, they typically remain high level, and do not get into technical detail on the precise nature of the claimed tensions between security and interoperation. By way of illustration, in a public document responding to antitrust enforcement, Apple argues against interoperating with alternative app distributors, stating, ``[b]y requiring that all apps on iPhone be distributed through a single trusted
source, the App Store, we were able to accomplish our goal of protecting users more effectively than any other platform''~\cite{apple_white_paper}.

Such arguments are often framed as technical arguments about incurring unacceptable security risks, that courts and policymakers should be deferent to. These seemingly technical arguments may carry particular weight---for an audience of policymakers, regulators, and the general public---when invoked by companies that are already responsible for vast amounts of personal data that could be put at risk by an ill-advised regulation, especially those companies which have an established reputation for security and privacy. 

Yet the ``security vs. interoperability'' argument is an inherently interdisciplinary one, about integrating technical security goals with other policy goals. For non-technical stakeholders, it is not easy to get a detailed sense of the technical considerations underlying the stated security concerns. For the security community, it can be challenging to get a detailed sense of the socio-technical, economic, and policy considerations essential to a holistic evaluation of the security considerations in the relevant regulatory context. Focusing on only technical aspects or only economic or regulatory aspects yields an inherently incomplete picture; policymaking based on this incomplete picture could lead to missteps in the important and fast-moving area of big tech and antitrust. 

Thus, we believe it critical for the security community to
engage with these questions not by opining on isolated technical questions---a natural approach to staying within one's expertise---but by embracing the broader scope of the questions beyond security, especially when industry interests may not align with highlighting this broader scope.


This paper provides a framework for doing just that. Our framework captures the key economic considerations relevant to the regulatory context in which the ``security vs. interoperability'' discourse is rapidly unfolding. The framework is designed to be usable without detailed knowledge of antitrust law or the economics considerations in any given deployment context, and has sufficient generality to capture key aspects and patterns across all of the real-world case studies we present in this paper, which span diverse areas of security. Through analysis of these case studies, we illustrate our framework’s utility in disentangling where security and interoperability are and are not in tension, where securing interoperable systems presents novel engineering challenges, and where ``security arguments'' against interoperability may be a distraction from the real issue at hand: fair competition.

\smallskip
\paragraph{Beyond the security community}
The use and misuse of security arguments is an issue of natural interest to the security community.
In the present context, the broader implications lie as much in security outcomes as in competition policy outcomes, and the timeliness and potential breadth of impact of the ``security vs. interoperability’’ issue arises more from today’s competition policy context than from security.


The concentration of economic and political power in the hands of a few technology companies, alongside the novelty, complexity, and opacity of modern technology, has contributed to a climate in which companies strategically deploy (seemingly) technical arguments as devices to further their political and economic power and avoid regulation. For example, recent legal scholarship has highlighted the strategic use of 
privacy, content moderation, and security arguments to further companies’ economic and policy interests \cite{privacy_pretext,lochner_law_review, fidler_cybersecurity_2023,cohen_book, san_bern_law_review}. Notably,
\cite{privacy_pretext} discusses how privacy is used inconsistently as a pretext to further business aims, including not to allow competitors access to data, by companies that engage in expansive data selling and sharing of data when beneficial for business.  

The defining characteristic of such arguments is not that they are necessarily incorrect about security (or privacy), but that they appear to instrumentalize ``security'' into a label that is convenient for business ends. Although companies should, of course, be able to put forward the best case for their own interests in such regulatory contexts, this trend is concerning if and when seemingly technical arguments are treated with more deference than business or policy arguments, as more objective or factual, or as matters on which courts and regulators should defer to companies' expertise.


This paper focuses in particular on the security-related arguments that companies like Apple raise in the context of recent antitrust investigations such as the \emph{Epic v. Apple} case\footnote{Epic Games, Inc. v. Apple Inc., Case 4:20-cv-05640-YGR, Findings of Fact and Conclusions of Law Proposed by Epic Games, Inc. 2021.\label{fn:epic}} in the US or proceedings to enforce the European Union’s new \emph{Digital Markets Act}\footnote{REGULATION (EU) 2022/1925 on contestable and fair markets in the digital sector (Digital Markets Act).} (DMA) reform. In this context, tech companies raise a range of security-related justifications as a shield to avoid complying with competition and interoperability mandates that lawyers, courts and regulators are seeking to impose \cite{ll1, ll2, ll3, ll4, ll5}.

When companies raise security arguments in this context, in general, the criticality and nuances of the security issue cannot be expected to be transparent to lawyers, courts and policymakers, or to other public stakeholders. Further confusing matters, some of the very same companies have a history of well-founded warnings against serious security risks in other policy contexts (such as backdooring encryption \cite{fbi_case}).\footnote{These latter security-critical arguments may well have served the companies’ business interests as well.}

In legal and regulatory proceedings, companies seeking to defend themselves from competition claims may strategically bring in security experts; then, those seeking to impose competition obligations will, in turn, bring their own security experts. In the \emph{Epic v. Apple} case, companies' experts have asserted that a company’s systems might suffer serious security risks or degradation if the company is forced to comply with interoperability and competition obligations, and the opposing experts have argued that this is not true 
\cite{epic_apple_blog}.

Asked to adjudicate apparently conflicting technical evidence, courts, lawyers, and policymakers are often ill equipped to interrogate the nuances of the security arguments and trade-offs, to (efficiently) distinguish how critical a security argument is, and to determine which security arguments are more motivated by economic or policy ends than security goals. This can cause delays and raise costs whether in judicial or regulatory proceedings or legislative reform. 

The risks to society and to the integrity of the legal and policy process can be significant. Big tech companies can hire very large teams of lawyers, lobbyists and security experts. Their strategic use of security arguments in service of their business goals can be misleading and difficult to sort through. It can significantly slow down judicial and regulatory processes, and lead to bad decisions or policies that undermine competition, openness and innovation, harming competitors, small businesses and consumers. Such decisions can, in turn, have long term effects on the structure of digital markets, on the diversity of players in them and on the quality of digital services available to consumers.

We believe this paper’s focus on systematic evaluation of companies’ security arguments both in terms of technical security considerations and in the context of the broader economic and policy context of these proceedings can help the security and policy communities better analyze and conceptualize “security vs interoperability” tensions (and the lack thereof), leading to better policy decisions that will shape the future of technology markets and ecosystems.


\paragraph{Other related work}
That security engineering requires consideration of contextual sociotechnical factors is hardly a new observation~\cite{ross_book}, and the security economics literature has long highlighted and studied the idea that ``security failure is caused by bad
incentives at least as often as by bad design'' \cite{AM07,ross_econ}. Much of security economics considers how incentives influence security outcomes. Our inquiry takes a different angle, in a similar spirit: we ask how security arguments can be leveraged to serve other economic goals.

Digital competition policy and antitrust enforcement have been responding to digital markets trends long before the recent wave of interest in big tech and antitrust. Microsoft's bundling of Internet Explorer with Windows was the subject of a landmark 1998 antitrust case.\footnote{\emph{United States v. Microsoft Corporation}, 253 F.3d 34 (D.C. Cir. 2001).} Net neutrality regulations, which required internet service providers not to give preferred treatment to certain types of content or traffic for economic reasons, have been at the heart of political controversies since the 1990s \cite{nyt_net_neutrality, nn1, nn2, nn3, nn4, nn5, nn6}. The ``security vs. interoperability'' narrative, our focus in this paper, is a more recent development that very little prior work has considered. Since the DMA, a promising initial literature has begun to examine the complex technical considerations around securely  interoperating end-to-end encrypted messaging~\cite{dma_concerns_e2ee_ross,dma_how_to_e2ee}.

\smallskip
\paragraph{Summary of contributions}
This paper critically examines recent and ongoing ``security vs. interoperability'' arguments in the context of recent US and EU antitrust proceedings, including detailed technical analysis of the nature of claimed tensions between security and interoperability in diverse contexts. Our three key contributions are as follows.

\begin{enumerate}
    \item We systematize ``security vs. interoperability'' arguments into a \textbf{taxonomy} consisting of two broad categories---\emph{engineering} and \emph{vetting} concerns---and a third \emph{hybrid} category, and explain their relationship to the economic concepts of \emph{vertical} and \emph{horizontal} interoperability   (\S \ref{sec:framework}). 
    \item We present \textbf{seven case studies} across diverse contexts in which ``security vs. interoperability'' arguments arise: messaging, app sideloading, in-app purchases, app stores, browser engines, NFC for host card emulation, and physical device interoperation (\S\ref{sec:case-studies}). 
    \item By analyzing these case studies through the lens of our framework, we illuminate \textbf{key considerations around the interplay of economic incentives, market power, and security} in ``security vs. interoperability'' arguments, and highlight the \textbf{essential role of security trade-offs and existing policies in critically evaluating such arguments} in their broader economic context (\S\S\ref{sec:case-studies}--\ref{sec:big-picture}).
\end{enumerate}

\section{Our Framework}
\label{sec:framework}

\begin{table*}[ht]
\centering
\renewcommand{\arraystretch}{1.5} % Adjust row height for readability
%\setlength{\tabcolsep}{4pt} % Adjust column spacing
\begin{tabularx}{\textwidth}{|p{0.15\textwidth}|p{0.17\textwidth}|X|X|}

%\begin{tabular}{|c|c|c|}
\hline
\textbf{Type of concern} & 
\textbf{Starting point} &
\textbf{Relevant security status quo} &
\textbf{Mitigation}  \\
\hline

\bi{Engineering} &
No \emph{technical} \newline interoperability &
Technical security guarantees of \newline existing system design & Build something new (if possible; may require \emph{technical innovation})  \\ 
\hline
\bi{Vetting} &
No/onerous/costly \newline interoperation as a matter of \emph{policy} &
Security implications of existing \newline interoperation policy & Vet third parties and products as a \newline condition of interoperation (if possible; requires \emph{bargaining power})  \\

\hline

\end{tabularx}

\caption{Some Key Differences Between Security Engineering Concerns and Security Vetting Concerns}
\label{tab:security_arguments}
\end{table*}

\paragraph{``Security vs. interoperability'' arguments}
Our framework's focus is on contexts in which a company that has significant market power\footnote{What counts as a relevant degree of market power for antitrust action depends on the competition laws and competition authorities in different jurisdictions. Our framework and contributions are agnostic to the precise threshold of market power prescribed by any given jurisdiction.} 
invokes security-related concerns to argue for preventing, deterring, or making economically onerous third-party developers' efforts to interoperate with the company's platforms or systems. Our scope thus includes the invocation of security concerns to argue against legislation and regulation that encourages or mandates that such a company interoperate with third-party developers' products or services. We call these \emph{``security vs. interoperability'' arguments}.

We have found that ``security vs. interoperability'' arguments are often made in public statements, including developer- or consumer-facing material. They have also been used in the context of regulatory actions and lawsuits in some instances. In this paper, we focus on the arguments made in contexts where a formal legal or regulatory action has been taken, as they are comparatively well documented, and as the potential impact in such cases is more concrete. That said, our analytical framework generalizes to contexts not featuring any formal legal or regulatory actions.

\paragraph{Scope of ``interoperation''}
%\label{sec:framework:interop}
%
When products and services produced by one company (e.g., smartphones) support or facilitate user engagement with products and services developed by a third-party, (e.g., independent apps), we say they \emph{interoperate} and are \emph{interoperable}.

For our purposes, interoperation comprises both \emph{technical interoperability} (e.g., what are the requirements for an app to be technically capable of running on a particular platform?) and \emph{interoperation policy} (e.g., which of the technically interoperable apps does a platform permit as a matter of policy?). 
Thus, if a third-party developer is blocked from offering their app on a platform even though it is technically compatible, we consider this an issue of interoperation (but not of specifically technical interoperability). 

In many engineering contexts, technical interoperability would be the primary topic implicitly referenced when discussing interoperation. 
In many law and policy contexts, references to interoperability often encompass interoperation policy or interoperation more broadly.
As such, the broader scope is necessary to capture the full range of concerns that feature in ongoing ``security vs. interoperability'' discourse. 

\paragraph{Other terms} We write \emph{company} or \emph{platform} to refer to the developer of the main platform and \emph{third party} to refer to those developers looking to interoperate. For brevity, we may write \emph{products} to encompass both products and services.

\medskip

\subsection{Our Taxonomy}
\label{sec:framework:taxonomy}
We now introduce our main taxonomy, which distinguishes two broad types of ``security vs. interoperability'' arguments as raising either security \emph{engineering} concerns (\S \ref{sec:framework:taxonomy:eng}) or security \emph{vetting} concerns (\S \ref{sec:framework:taxonomy:vet}), summarized in Table~\ref{tab:security_arguments}, alongside a third hybrid category that raises concerns incorporating aspects both of the earlier two categories (\S \ref{sec:framework:taxonomy:hyb}).

Our taxonomy clarifies the different types of security concerns that may arise in response to the prospect of introducing interoperation between products or systems which did not previously interoperate. The technical, economic, and policy context in which these concerns are raised is critical to our categorization and analytical framework. In particular, the status quo of how the systems in question work \emph{before} interoperation is introduced, the nature of the technical security guarantees the systems already provide, and the security policies and trade-offs the companies already have in place, are essential considerations.


\subsubsection{Security Engineering Concerns}
\label{sec:framework:taxonomy:eng}

When considering interoperating systems that are currently not technically interoperable, there are contexts in which preserving security alongside interoperation requires that some aspect of the company's product be non-trivially changed. How to change the product to preserve the security guarantees currently offered by the company's product, while also supporting interoperation, may or may not be clear. 

In such contexts, companies may raise concerns about the costs and engineering challenges that would be required to preserve security while supporting interoperability.
We call these \emph{security engineering concerns}.

Security engineering concerns tend to arise when there are two (or more) parties trying to interoperate similar services while maintaining the technical security guarantees of their existing system designs. Generally, in such situations, all parties involved need to develop, agree on, and implement compatible protocols or systems, so the security engineering concern is shared among all the parties seeking to interoperate. Messaging services provide a clear example of this type of concern, as we detail in Section~\ref{sec:case-studies:engineering}.

\begin{center}
    \fbox{\parbox{8cm}{
    \textbf{Template of a Security \emph{Engineering} Concern} 
    \iffullversion
    \\ From a \emph{platform}'s perspective
    \fi
    \begin{enumerate}
        \item Interoperating would require a building new technical mechanism or architecture for our system to interface with other systems.
        \item Our system has existing technical security guarantees. Designing a new architecture to preserve these guarantees is an engineering task that is costly in time, effort, and coordination with others---and may require innovation, so success is not certain.
        \item Therefore, interoperating secure systems is more costly than interoperating systems in general, and if we don't get the engineering right, the security of billions of users could be undermined.
    \end{enumerate}
    }}
\end{center}

This template argument is logically coherent and highlights legitimate security concerns (as will the other templates that follow). It can thus be tempting, as a security expert asked to evaluate this argument, to pronounce it sound and move on. 

But our focus in this paper is not on the logical soundness of these arguments or whether they highlight legitimate security issues; they usually do both of those things.\footnote{Of course, ``security vs. interoperability'' arguments that did not meet these criteria would also be concerning and possibly misleading. We consider them out of scope not only because they are less prevalent, but also because they can be analyzed and debunked from a purely technical perspective.} Instead, we focus on the broader implications of how this type of argument can be and is in fact used in ``security vs. interoperability'' contexts. As we will see throughout the paper, analyzing this broader context illuminates how the different types of arguments in our taxonomy can be used to promote security and/or to push for other goals unrelated to security.

\subsubsection{Security Vetting Concerns}
\label{sec:framework:taxonomy:vet}

Security vetting concerns arise in contexts where there is already some level of interoperation supported by a system's technical architecture. Many devices and platforms support an ecosystem of apps, software, and connected devices that are integral parts of their functionality, but that are developed by third parties. Examples include platforms, browsers, or operating systems on which third-party software can run, and platforms that offer API access to third-party developers.

Once such an architecture is in place, there remains the question of which third parties a platform is willing to interoperate with.
Downloads and other connections can open a vector of attack for malware, spyware, and other undesirable outcomes for both the user and the device. Companies may vet and control which apps, systems, or software are permitted to interact with their devices with a view to preventing or mitigating such undesirable security outcomes (although these decisions might be motivated by other goals too).

In such contexts, when asked to support greater interoperation, companies may raise concerns that third-party software that interoperates with their systems may not be as secure as their own systems, and thus undermines the security of their users and/or systems. We call these \emph{security vetting concerns}.

\begin{center}
    \fbox{\parbox{8cm}{
    \textbf{Template of a Security \emph{Vetting} Concern} 
    \iffullversion
    \\ From a \emph{platform}'s perspective
    \fi
    \begin{enumerate}
        \item Interoperating would mean our users and systems interacting with third parties and third-party software.
        \item Third parties and third-party software may make different decisions about security than us. We cannot guarantee that third-party software would adhere to the same standards as software we have developed or carefully examined ourselves. 
        \item Therefore, allowing third-party software on our platform could undermine billions of users' security. 
    \end{enumerate}
    }}
\end{center}

The above template is so generic that it could apply to \emph{any} interoperation. As such, we find that vetting arguments often boil down to arguments directly against interoperation. 

Vetting third parties can be an important way to protect users, but also concentrates power and discretion of incumbent companies over ecosystems they control.  Companies in this position can use their role as security guarantor to charge developers for access or prevent competitors from reaching the market. Their interoperation policy choices strike a tradeoff between security, interoperation, and other business considerations that can be opaque, discretionary and (necessarily) not amenable to precise technical specification.

Due to the discretionary tradeoffs involved, arguments that security considerations \emph{necessitate} a particular form of vetting or security measures generally cannot be substantiated by purely technical reasoning; rather, they must refer to existing interoperation policies and the discretionary choices already present therein. This contrasts with security engineering concerns, which technical reasoning may suffice to substantiate.

Let us concretize the last point with a few familiar examples: phones, personal computers (PCs),\footnote{Throughout the paper, we use ``PC'' to refer personal computers. We do \emph{not} use the narrower definition of PC as a personal computer compatible with the original IBM PC. In particular, we consider Macs to be PCs.\label{fn:pc}} and video games. One clear strategy---arguably the only foolproof one---that would achieve the aim of protecting users from harms from third-party software would be to prevent phones from being able to download any software, and perhaps even bar them from the internet.  Such an approach would, of course, undermine the functionality of the product and be a bad economic move. Existing smartphone providers strike different trade-offs, but none of them take an absolute approach.


Whereas operating systems (OSs) for PCs  allow users to download almost anything, video-game consoles have much more stringent control over what is allowed on the machine.  This is not because of fundamental security distinctions between PCs and Xboxes, but rather because the respective developers of these technologies have chosen different trade-offs between security, interoperability, and other interests, taking into account the relevant application contexts.

PCs are key infrastructure for vast swathes of the economy and any exacting controls enacted by their manufacturers would likely either make the devices undesirable, if there is healthy competition, or negatively impact on the rest of the market, if there are monopoly conditions. 

In Section~\ref{sec:case-studies:vetting}, we provide four case studies of security vetting concerns: app sideloading, in-app purchases, alternative app stores, and WebKit.

\subsubsection{Hybrid Concerns}
\label{sec:framework:taxonomy:hyb}

Finally, there are situations featuring aspects of both security engineering concerns and security vetting concerns, in which interoperation would entail both the development of a secure system for interoperation and some sort of vetting of those developers looking to use that newly open system.

These instances tend to feature a more complex relationship between the company and the third-party.  In the security engineering setting, parties have to decide how to interoperate similar services and none of the parties in question necessarily has the upper hand in the decision making (although one company often has more power in practice). In the security vetting setting, the company that produces the platform is typically in power.  The third-parties looking to provide their product on that platform must convince the company that their product is secure. 
We discuss these relationships further, from an economic perspective, in Section~\ref{sec:framework:horiz-vertical}.

\begin{center}
    \fbox{\parbox{8cm}{
    \textbf{Template of a \emph{Hybrid} Concern} 
    \iffullversion
    \\ From a \emph{platform}'s perspective
    \fi
    \begin{enumerate}
        \item Interoperating would mean allowing third parties to access functionality reserved for only our use.
        \item Opening up those functionalities entails third parties newly interacting with security-sensitive parts of our OS. Designing a new system to secure such access is a costly engineering task that may require innovation.
        \item Therefore, third-party access to these functionalities carries inherent security risk, and could undermine the security of billions of users. 
    \end{enumerate}
    }}
\end{center}

The hybrid category combines both of these aspects. The third-parties must still demonstrate that their products meet the requirements of an interoperation policy, but there is also more engineering and collaboration needed to support technical interoperability. For example, in opening mobile wallet functionality to third parties, device/OS providers need to both add APIs to allow developers access to previously restricted hardware and vet which developers will have access. 

Section~\ref{sec:case-studies:hybrid} provides two case studies of hybrid concerns: NFC for mobile payments and physical device interoperability.

\subsubsection{Summary}

Our Templates are designed to capture just the essential elements of the argument patterns we have identified. Of course, they are oversimplifications with respect to concerns being raised in any real-world ``security vs. interoperability'' context. Unsurprisingly, the arguments made in practice are more complex, and may not neatly fall into a single category.

Our case studies and analysis in Section~\ref{sec:case-studies}--\ref{sec:big-picture} demonstrate the complexities of ``security vs. interoperability'' arguments in the wild, and show how recognizing the simple patterns identified in our taxonomy illuminates patterns that can inform critical evaluation of the security and economic aspects of real-world arguments in specific case study contexts.

\subsection{Horizontal and Vertical Interoperation}
\label{sec:framework:horiz-vertical}

Our distinction between engineering and vetting concerns is closely correlated to distinctions in the relationships of the interoperating products. 

To highlight this distinction, we define horizontal and vertical interoperation, generally in keeping with the use of these terms in \cite{hor_vert_interop}.  In a nutshell, by \emph{horizontal interoperation} we mean interoperation between products offered at the same level, 
%i.e., products that run \emph{with} each other, 
whereas \emph{vertical interoperation} is that between a platform or `higher layer'\footnote{We use `layers' in a general sense here, not specifically those from the OSI model, but generally to mean products that run on other products.} product and products that run \emph{on} that platform or higher layer product. 

Horizontal interoperation happens when similar products need to be made interoperable. Messaging is a good example, where previously competing products must collaborate on integrated functionality and security (Case Study \ref{sec:case-studies:messaging}). This happens when the products are \emph{substitutable}, a term used in economics to mean two products used for similar purposes that compete for the same user demand. %That does not necessarily mean that the products are completely the same -- one cannot yet use one messaging app to text users of a different messaging app. % Formally, two goods are substitutable if the demand for one increases when the price for another increases. For example, if one messaging app introduces a paywall, we could imagine user demand shifting to a similar service.

Vertical interoperation happens between products on different layers. Instances of vertical interoperation that arise naturally, as opposed to through regulatory or antitrust enforcement, tend to be between some platform and a \emph{complementary} product. Again using economic terminology, we say two products are \emph{complementary} if they are often used together, or, formally, if a price increase of one decreases the demand for the other. Apps are a particularly good example, and the success of smartphones is greatly dependent on the wide availability of apps and other complementary products that are developed by third parties.

Unlike with horizontal interoperability, there is an incentive to achieve good vertical interoperation between platforms and the apps that run on them, as these two products complement each other and increase customer utility in concert.  Rather, the competitive friction to vertical interoperation is either from the platform wanting to hold onto extractionary practices (e.g., by taking commission on in-app purchases; see \S\ref{sec:case-studies:vetting}) or from an effort to protect the platform's own complementary, higher layer product that is somehow privileged on the platform, which is called self-preferencing (e.g., payment apps having unique access to hardware functionality; see \S\ref{sec:case-studies:hybrid}).  This second case, where there is an in-house, higher layer, complementary product with exclusive access to platform functionality, is a hybrid case; the platform and the third-party product are complements, but the company's higher layer product is a substitute. 


\definecolor{myblue}{RGB}{80,80,160}
\definecolor{mygreen}{RGB}{160,80,80}
\begin{figure}[ht!]\centering
    \vspace{-20pt}
    \begin{tikzpicture}[thick,
      every node/.style={draw,circle},
      fsnode/.style={fill=myblue},
      ssnode/.style={fill=mygreen},
      every fit/.style={ellipse,draw,inner sep=-2pt,text width=2cm},shorten >= 3pt,shorten <= 3pt
    ]
    
    % the vertices of U
    \begin{scope}[start chain=going below,node distance=7mm]
    \node[fsnode,on chain] (cat1) [label=left: Engineering] {};
    \node[fsnode,on chain] (cat2) [label=left: Vetting] {};
    \node[fsnode,on chain] (cat3) [label=left: Hybrid] {};
    \end{scope}
    
    % the vertices of V
    \begin{scope}[xshift=3cm,yshift=-0.5cm,start chain=going below,node distance=7mm]
      \node[ssnode,on chain] (horiz) [label=right: Horizontal] {};
      \node[ssnode,on chain] (vert) [label=right: Vertical] {};
    \end{scope}
    
    % the set U
    %\node [myblue,fit=(f1) (f5),label=above:$U$] {};
    % the set V
    %\node [mygreen,fit=(s6) (s9),label=above:$V$] {};
    
    % the edges
    \draw[line width=2.5pt] (cat1) -- (horiz);
    \draw[dashed] (cat2) -- (horiz);
    \draw[dashed] (cat3) -- (horiz);
    \draw[dashed] (cat1) -- (vert);
    \draw[line width=2.5pt] (cat2) -- (vert);
    \draw[line width=2.5pt] (cat3) -- (vert);
    \end{tikzpicture}
    \vspace{-10pt}
    \caption{Our taxonomy and horizontal/vertical interoperation (strong/weak correlations in bold/dashed lines respectively)}
    \vspace{-10pt}
\end{figure}

\paragraph{Relation to taxonomy}
The economic context of horizontal or vertical interoperation may be useful to inform contextual analysis of ``security vs. interoperability'' arguments. Security engineering arguments are often related to horizontal interoperation, and security vetting arguments are often related to vertical interoperation, although this is not always the case.\footnote{Theoretically, both types of arguments can arise in both horizontal and vertical interoperation scenarios. In practice, many complex real-world scenarios exhibit aspects of both, but the observation we wish to highlight is that there is a stronger correlation between certain categories.}

It makes sense that vetting concerns and vertical interoperation tend to be seen together as they both arise when one company has a platform with which others interoperate. The hybrid cases from our taxonomy are typically also situations in which third parties are pursuing interoperability on a company's platform, so they are also more strongly associated with vertical interoperation.

\paragraph{Beyond our framework}
We emphasize that the tension between security and interoperability is not inherent or universal. Plenty of technologies have been designed to support open interoperation from the ground up, and many the concerns we raised in this section will not arise at all. Interoperation between email providers is a good (horizontal) example; third-party software on Linux is a good (vertical) example illustrating that interoperation and security can go hand-in-hand. 

\section{Case Studies and Analysis}
\label{sec:case-studies}

In this section, we examine examples from recent antitrust proceedings and discuss their details and how they fit into our framework.  The proceedings are at different stages; some are complete while others are just getting started. Our focus is not on the details of the legal proceedings, but rather on the ``security vs. interoperability'' considerations in each case.  

For each category in our taxonomy---engineering (\S\ref{sec:case-studies:engineering}), vetting (\S\ref{sec:case-studies:vetting}), and hybrid (\S\ref{sec:case-studies:hybrid})---we provide case studies and an analysis of patterns across the case studies, structured around four critical questions: (1) \qone (2) \qtwo (3) \qthree (4) \qfour

\subsection{Security Engineering Concerns}
\label{sec:case-studies:engineering}
This category is, in some sense, the simplest way to think about interoperation; here we have two or more  parties trying to build or retrofit their systems to work together. Picture dual-gauge trains with different wheel gauges able to operate (safely) on tracks of different width.  In our contexts, companies involved are generally competitors and the interoperation is generally horizontal.

This section, unlike the others, contains only one case study. Our case study contains two examples, iMessage and WhatsApp, which involve distinct considerations. We believe interoperable encrypted messaging is currently the most illustrative example in the category of security engineering concerns.

\refstepcounter{casestudy}
\paragraph{Case Study \thecasestudy: E2EE Messaging}
\label{sec:case-studies:messaging}
There have been efforts to mandate interoperation between end-to-end encrypted (E2EE) messaging services in both the US and the EU, with a focus on iMessage (Apple) and WhatsApp (Meta), respectively.\footnote{We focus on: United States v. Apple, Inc., No. 2:24-cv-04055 (D.N.J. filed Mar. 21, 2024) and CASE (EU) DMA.100024 Meta – number-independent
interpersonal communications services, 2023\label{fn:cases}} Interoperation would mean that users of one service could send an end-to-end encrypted (E2EE) message to the users of a different E2EE service, while preserving security\footnote{REGULATION (EU) 2022/1925 on contestable and fair markets in the digital sector (Digital Markets Act), Article 7(3) }. 
%Messaging, or more specifically, number-independent interpersonal communications services, is more multifaceted than just text messages, i.e. voice messages, images, and groups, and some more complicated features may be more complex than just messaging \cite{meta_interop}. 

Both WhatsApp and iMessage provide E2EE text messaging as a core part of their functionality. Like most modern messaging applications, they also offer other functionality: e.g., voice messages, voice calls, groups, and so on.

Before the actions of the antitrust authorities, WhatsApp users could only message other users on WhatsApp.  In contrast, iMessage users could already send and receive messages originating from Android phones; however this type of messaging is not end-to-end encrypted (unlike iMessage chats between iPhone users) and generally has less functionality.  The changes in response to the antitrust action are still materializing, but have mostly yet to be rolled out, so this reflects the status quo.   

Even when narrowing our focus to just E2EE text messaging,
it is clear there is substantial engineering work to be done to achieve secure interoperation. Meta based WhatsApp encryption on the Signal Protocol, which is developed by the non-profit Signal Foundation, and is considered the gold standard in the field \cite{meta_interop}. Meta will require that any third-party looking to interoperate uses the Signal Protocol or `a compatible protocol if they are able to demonstrate it offers the same security guarantees' \cite{meta_interop}. For many smaller providers, the possibility of interoperating with WhatsApp may provide ample incentive to do the engineering work to meet these requirements. For WhatsApp, secure interoperation will also require some engineering, including building secure connections with third-party servers and coordinating
message formats
 \cite{meta_interop}. This may take significant effort, but appears very tractable.

Conversely, iMessage does not use the Signal protocol, and is sufficiently dominant in the market that it is not incentivized to either switch to the Signal protocol or otherwise facilitate interoperation. Historically, protocols for communication have been agreed upon by whole communities of users and operators, as in the case of email. The Signal Protocol is an effort in that direction and we might eventually have a system of community standards for secure messaging that everyone adheres to and that allows anyone to participate.  Unlike email, messaging needs to be `retrofitted' to allow for interoperation. For now, the efforts for greater interoperation consist of a combination of companies retrofitting for interoperation on the condition of adherence to their standards, and community-led protocol development efforts. There are real technical and research challenges, and it seems possible that the community can develop much better ways of achieving interoperation securely \cite{dma_how_to_e2ee,dma_concerns_e2ee_ross}.

For the incumbents, interoperation entails allowing their services to send and receive information outside of their own networks, and thereby potentially helping competitors.  These incumbent services are currently sufficiently popular that not allowing greater interoperation has been a boon -- encouraging greater uptake -- rather than a deterrent for customers. With the example of Apple and iMessage in particular, the incompatibility of messaging may play a role in users switching to iPhones because iMessage is only available on iPhones \cite{imessage_doj}. 

\begin{remark}
One prominent point in the iMessage antitrust case\footnote{United States v. Apple (see footnote~\ref{fn:cases}).} has been the difference in appearance between messages sent between iPhones (blue bubbles) and those sent from other devices (green bubbles) \cite{imessage_doj} and the associated impact on consumer choice. This issue is not directly related to the technical feasibility of interoperation at all. In practice, the color differentiation signals a lack of interoperation, and that green messages are not end-to-end-encrypted \cite{ross_econ, apple_bubbles}. 
\end{remark}

\subsubsection{Analysis: Security Engineering Concerns}

\question{1. \qone}
The incumbent companies currently have strong network effects that outweigh any incentives to interoperate. Apple can also bundle goods in a way that incentivizes exclusivity. The fact that iMessage is only available on iPhones and is not fully interoperable with messaging from Android (and other) phones incentivizes users to switch to iPhones.  These incentives lead companies to simply avoid full interoperation, rather than to enact other punitive or exclusionary measures on competitors. Companies' outsized power in this context is enough to put them in positions of market dominance.

\question{2. \qtwo} Interoperation is generally not economically advantageous for the companies being asked to interoperate. For WhatsApp, opening up a closed system is not only an engineering challenge; but it also means losing existing network effects. 
In the case of iMessage, extending functionality, including encryption, will be an engineering task and will remove an incentive for Android users to switch to iPhones.  Here, however, there are also security arguments in the other direction: encrypting iMessage communications to/from Androids will better protect iMessage users. This is a interesting example of a company apparently eschewing security in favor of other economic incentives, yet at the same time framing some of their arguments for doing so as security arguments. This suggests that the utility and security of users is less important than exerting pressure on non-iPhone users to switch.


\question{3. \qthree}
Although Meta already uses the Signal Protocol for WhatsApp, there is still a substantial amount of engineering to make sure that interoperation is secure. The Signal protocol is a centralized server design, the predominant paradigm in encrypted messaging, and Meta has also made some modifications to the Signal protocol in WhatsApp \cite{meta_interop}. For Apple, some of the changes that are necessary for interoperation are simpler (such as the bubble color) whereas some are more complicated (such as encryption). In both cases, using community standards -- such as Signal or Messaging Layer Security (MLS) -- could achieve strong security for users as well as easier interoperation in the future.


\question{4. \qfour} 
End-to-end encryption is a well-researched goal, and there are standards bodies and academic researchers as well as companies looking to improve the existing technology. There are open questions around how best to achieve interoperation in E2EE, especially between existing systems that need to be `retrofitted' with interoperation capabilities~\cite{dma_how_to_e2ee}. At the same time, there are readily viable options using existing technology to set up interoperation that remains secure. Thus, security considerations are not an insurmountable obstacle to an interoperability mandate in this context; at the same time, research progress over the course of the next few years may reveal ways to do interoperable E2EE much better. In the case of iMessage, interoperating encryption for communications with Android users will ultimately better protect Apple's own users.  Thus, although initially difficult, security will be better served after the required changes.



\subsection{Security Vetting Concerns}
\label{sec:case-studies:vetting}

Our case studies in this section  relate to Apple's relationship to apps, so we begin with some background context. Apple has historically adhered to a `walled garden' approach for iPhones. Users are only able to download and use apps vetted by Apple according to their security, privacy, and quality standards. The stringent screening of apps on the App Store has furthered the iPhone's reputation as being more secure\cite{blog_android_apple_security}. 

This approach limits what users can do on their own devices. Such limited usability appears to be untenable on PCs.\footnote{Recall that we use ``PC'' broadly to include Macs; see footnote~\ref{fn:pc}.}  Apple's own approach to its Macs contrasts with its approach towards third-party software on its other devices. Apple has a notarization process for software intended for Macs, but this process is automated and 
much more limited than the process for apps on iOS \cite{apple_dev_mac, apple_app_review}. It is also cheaper---the cost to developers for Macs is also only the \$99 membership to the Apple Developer Program \cite{apple_dev_license_agree}, which iOS developers pay in addition to other substantial fees\footnote{See Case Study~\ref{sec:case-studies:iap} and the Appendix for more on fees.}---and easier---once notarized, this software can be distributed via project websites, not only on an Apple marketplace. These policies indicate Apple's discretionary choices about the trade-off between user choice and security vetting on its products. 

From the perspective of developers and antitrust officials, there is another dimension to this trade-off.  Apple's vetting of applications and its position as gatekeeper and security guarantor coincides with a position of market power.  Apple's control over its App Store allows it to determine what apps are offered and take a cut of transactions. 

Thus, to cast security vetting concerns as only a balancing of user choice against security, is to miss the economic incentives at play in the background.  The core reasons behind the push towards more open app distribution have been economic. Developers are not arguing for the freedom to distribute malware; rather they want to avoid the substantial rents that gatekeepers force upon them \footnote{Epic Games, Inc. v. Apple Inc., Case 4:20-cv-05640-YGR, Findings of Fact and Conclusions of Law Proposed by Epic Games, Inc. 2021}.
We now focus on four similar and overlapping examples of security vetting arguments.
All have to do with app distribution and use.

\refstepcounter{casestudy}
\paragraph{Case Study \thecasestudy: App Sideloading}
%\label{subsec:apps}
%\label{sec:case-studies:app-sideloading}
App sideloading refers to downloading an app onto a device via some route other than the official app store. This alternate route might be a website or alternative app store. % It is possible to circumvent this closed system by `jailbreaking' the device in question, but this is not sufficiently common for most app developers to rely on. 
Sideloading was not allowed on iPhones (or iPads) until the DMA, and is still not permitted outside the EU. Apple still requires that sideloaded apps be notarized, meaning the app is vetted and signed by Apple and only apps with such a signature can be downloaded onto a device.  This notarization procedure is not as thorough as the vetting for the App Store, but still includes all the security requirements in the App Store Guidelines \cite{apple_app_review}. A major reason why Apple says they need to notarize all apps, especially those with alternative distribution, is to maintain the security standards of apps on its App Store \cite{apple_white_paper}.

The notarization process is not necessarily cheaper than getting approved for the App Store, despite there being fewer requirements for Apple to check.  Developers that choose sideloading must pay a €0.50 `core technology fee' (CTF) for each first annual install over one million, where annual installs include updates \cite{apple_web_sideloading}. Apple allows free installs for a year by the same user, so within a year, the app can be updated multiple times for free, but this still means there is a charge to update each following year \cite{apple_ctf}. Updating apps to fix security issues is an important means of protecting users.  Disincentivizing proactive updates runs counter to Apple's stated goals of preserving security.  Depending on a developer's circumstances, this alternative pricing is not necessarily cheaper than the 30\% cut Apple takes from the App Store and IAP; see Appendix for an example calculation of fees.  Despite being designed to address concerns about Apple's gatekeeing role and rent-seeking behavior, the notarization fee scheme has not reduced the burden of fees for all developers. Apple justifies these continued high fees by invoking user security.


%Furthermore, the notarization criteria are not clearly less stringent than the App Store guidelines in terms of security. The notarization is less thorough -- its requirements are a subset of those used for App Store review -- but those requirements that are omitted are not security-relevant \cite{apple_app_review}.  If there are important guidelines missing from the notarization criteria, there is no reason that they cannot be the same as the App Store Guidelines, especially if the security of the phone is at risk. % It makes sense that Apple will reserve its guidelines that are more about quality control or performance for the App Store alone, as these add value, but it is not incentive compatible for Apple to allow potentially insecure apps to be notarized. This difference appears to be born out in the notarization guidelines \cite{apple_app_review}.  

\refstepcounter{casestudy}
\paragraph{Case Study \thecasestudy: Alternative App Stores}
\label{cs:app-stores}
%\label{sec:case-studies:alt-app-stores}
Apple must now allow alternative app stores, which will be able to distribute sideloaded apps. As discussed, the apps being sideloaded must be notarized by Apple and any in-app purchases would still pay the alternative payment commission (see Section \ref{sec:case-studies:iap}).  

The alternative app store itself must abide by additional criteria set by Apple, which Apple can check before allowing the alternative marketplace to operate. Alternative marketplaces must, for example, agree not to distribute apps that infringe on intellectual property rights or that are malicious as well as deal with requests for removal of apps by governments\cite{apple_alt_market}.  
Apple also points out that collecting data about (insecure) apps through its App Store is an important aspect of keeping users secure \cite{apple_white_paper}. There is no reason that interoperability regulations would prevent Apple from engaging in this type of data sharing with alternative app stores. Indeed, this type of data sharing would be beneficial even across existing marketplaces outside of Apple's control, e.g., Google Play. 

Thus far, these are vetting concerns, but there are also engineering considerations involved.
Apple created an API to allow these alternative marketplaces to be able to execute downloads \cite{apple_alt_market}.  There is also minor amount of extra work that needs to be done for developers distributing their apps via their website\cite{apple_app_distribution_website}.  Apple needs to set up these systems such that they interface with the OS correctly and securely. Though these are relatively lightweight engineering tasks, this provides a simple example illustrating a need for Apple to build secure architecture \emph{and} vet third parties to interoperate.

\refstepcounter{casestudy}
\paragraph{Case Study \thecasestudy: In-App Purchases}
\label{sec:case-studies:iap}
In-App Purchases (IAPs) are purchases of digital goods in apps on Apple devices, which are intermediated by Apple.  We use \emph{IAP} to refer specifically to Apple's in-app purchase system.  Apple generally takes a 30\% cut from these purchases \cite{apple_membership}.  Payment for physical goods within apps is treated differently, but the purchase of in-app goods, like coins in a game, are required to go through Apple and incur the fee \cite{apple_app_review}.
%Recent developments have allowed developers to allow goods purchased elsewhere to be used in the app in the EU \cite{apple_app_distribution}.  For example, a movie purchased on the desktop website could be used on the corresponding app.  Before Apple's efforts to comply with the DMA, all in-app digital good purchases were required to go via Apple's IAP \cite{apple_app_review}. %There are a few other regional exceptions, but this generally holds.  In the US, apps can facilitate transactions via `link outs,' where customers are directed out of the app to a website to finish a transaction, but Apple must be paid a 27\% commission \cite{apple_app_review}.
Securing payments is, of course important, and Apple's justification for requiring that apps on its devices take payments through Apple's system is to ensure the security of transactions\cite{apple_white_paper}.\footnote{It is unclear how this supports \emph{not} enforcing intermediated payments for physical goods purchases.}

Changes due to the DMA now allow developers to link out to websites or use alternative payment service providers for purchases of digital goods \cite{apple_alt_payment_eu}. 
Processing payments online is simple and usually results in much smaller fees; Epic reports a figure less that 5\%\footnote{Epic Games, INC. v. Apple INC., Case 4:20-cv-05640-YGR, Findings of Fact and Conclusions of Law Proposed by Epic Games, Inc. 2021, \S 454}. That said, developers who choose this option still have to pay a 10-17\% commission on sales of digital goods and provide Apple with transaction reports \cite{apple_alt_payment_eu}.


From the perspective of the user, the new, open system is not dissimilar from the general state of online payments or the case for physical goods purchased on apps, where each merchant must independently assure the user of the safety of the transaction. Developers can use other widely trusted methods of payment or opt to use Apple's IAPs. Thus, it falls to developers to vet their chosen payment service provider.% This is already the case for payments made for physical goods on iPhone apps.

\refstepcounter{casestudy}
\paragraph{Case Study \thecasestudy: WebKit}
%\label{sec:case-studies:webkit}
Prior to the DMA and the subsequent actions taken by the European Commission (henceforth `the Commission), all web browser apps on iOS had to use WebKit as their browser engine \cite{apple_webkit_requirement}.  Browser engines are part of web browser apps (and in-app browsing) that render the content of a website into a page ready for user interaction \cite{browser_sok}.  WebKit is open source, but belongs to Apple\cite{apple_webkit_owns}. As a result of Commission actions, Apple now allows alternative browser engines, only in the EU, both for browser apps and in-app browsing \cite{apple_alt_webkit}.  
Prior to these changes, Apple’s iOS was the only operating system that did not support other browser engines; other browser engines can be used on the operating systems of their competitors~\cite{team_gb_browser_engines}. In the developer documentation, the requirement that browsers use WebKit is not explicitly based on security, although there is notable community discussion around security motivations for this requirement~\cite{yes_we_cite_reddit}. 

Browser engines play a key role in preserving security for users online. Here, iOS users may be disadvantaged by the scarcity of options available \cite{browser_security}.  WebKit has not been demonstrated to be, and does not have a strong reputation for being, any more secure than other browser engines \cite{browser_security}. There have been bugs in WebKit and Safari \cite{browser_sok}, and not allowing developers (of browsers or those using in-app browsing) to use the tools they see fit could constrain their ability to make application-specific security choices.

Apple stipulates that any developer looking to use an alternative browser engine must meet certain requirements, including some related to security \cite{apple_alt_webkit}.  Any browser app would also have to abide by Apple's general app guidelines, including those regarding security \cite{apple_app_review}.  
%Developers are required to uphold security on a number of different points as well as address vulnerabilities in a timely manner \cite{apple_alt_webkit}. 
Any developers looking to implement an alternative browser engine must be approved by Apple, allowing Apple to vet potential alternatives and protect users from insufficiently secure options \cite{apple_alt_webkit}.

\subsubsection{Analysis: Security Vetting Concerns}
``Security vs. interoperability'' concerns in the vetting category exhibit a number of key patterns, as illustrated by the above case studies, and synthesized below.

\question{1. \qone}
The examples we have seen are similar in that Apple controls third-party developers' access to their potential customer base on the iPhone. The iPhone is a platform and Apple can exclude developers on grounds that their apps do not meet Apple's guidelines or because their services are not allowed in the ecosystem (e.g., alternative app stores). Apple's monopolistic position also allows it to charge developers more than might be expected in a more competitive setting. This type of power is markedly different from the market power discussed in the security engineering case study and in horizontal interoperation more generally. Here, Apple has the power to completely exclude a third party, whereas, in horizontal interoperation, it cannot cleave third-party developers from users.  

\question{2. \qtwo} 
Vetting apps to protect user security is a reasonable end, but it is also used as a means to extract rents. Keeping malware and other insecure apps away from users is also good for platforms, especially those that, like Apple, enjoy a reputation for security and good curation.  That said, it is hard to evaluate whether the price tag is justified, given the opacity of the system and limited competition. 

\question{3. \qthree}
It is clear in all these cases that some amount of vetting is important to preserve the security status quo. Apple's notarization system and CTF is Apple's suggested way forward. The notarization guidelines are less onerous for Apple to check than their status quo, and vetting new app stores will likely be a rare occurrence---but the new mechanisms are not necessarily cheaper for developers. There have been changes to Apple's pricing practices (and it is possible more changes will need to be made in the future).  Price changes might be impactful for the company's profits; they will not be difficult from an engineering standpoint. 


\question{4. \qfour} 
Apple has absolute control over third-parties' access to iPhone users. At the same time, they rely on those third-parties to develop apps for the iPhone, increasing the iPhone's utility and commercial success.
%
The ecosystem of iOS apps is massive, and preventing every possible `bad' outcome is too general and too far-reaching to be a viable reality.  Instead, Apple has policies (e.g., \cite{apple_app_review}) around how they protect users and devices.  Inherent in these policies are certain trade-offs around security, safety, privacy, and quality.  Guidelines that are too stringent might unnecessarily narrow the universe of available apps or take too many resources.  Guidelines that are too lenient might deeply undermine user trust and device integrity. 

The changes the Commission is enforcing will not necessarily impose new trade-offs when it comes to app vetting.  Apple itself decided (perhaps for economic reasons) to apply only a subset of app guidelines to those apps applying for notarization.  Alternative app payments and alternative app stores will require users to invest their trust in other third parties.  Thus, arriving at an acceptable trade-off falls to the users; they must choose which companies' security (and privacy) policies are trustworthy and/or worth the risk.

\subsection{Hybrid Concerns}
\label{sec:case-studies:hybrid}
The hybrid cases are those in which the company must engineer a system that can preserve security when opened up to third parties, as well as vet those third parties.  

%Note that in the case of messaging, the protocol needs to preserve security regardless of the source of incoming data and cannot vet the other party.  In the case of sideloading apps, vetting is the primary way to preserve security.  We now address cases that require both.  This occurs when a service is taking the place of an existing company service that is downstream of their main platform.  Interoperation requires both building an open and secure system that used to be closed and also vetting the security of third-party products.  

\refstepcounter{casestudy}
\paragraph{Case Study \thecasestudy: NFC for mobile payments}
%\label{sec:case-studies:nfc}
Near Field Communication (NFC) is a set of protocols that work at close range and allow two devices to communicate securely.  NFC is used for myriad applications linking smartphones to other devices, including reading tags on objects, serving as keys, or communicating credit card or identification information \cite{apple_security}. For an app to use NFC for mobile payments, it must have access to the NFC antenna in the phone. It must also access key information that is stored on the Secure Element, which holds sensitive information, including the card information necessary for host card emulation (HCE) \cite{apple_se}.  Apple had restricted third-parties from using NFC and the Secure Element for HCE, reserving this ability for Apple Wallet and Apple Pay. It did allow other usage of NFC via API access to the necessary hardware and software \cite{apple_hce}. 

EU antitrust authorities have identified this as monopolistic behavior in a case not based on the DMA\footnote{CASE (EU) AT.40452 –Apple – Mobile
Payments, Commission Decision, 2024 
%\url{https://ec.europa.eu/competition/antitrust/cases1/202441/AT_40452_10269725_10183_3.pdf}
} and Apple has opened up the HCE capability to third-party apps, subject to approval by Apple. The criteria for authorization require that the third-party developer commit to complying with laws and standards pertaining to privacy and security, including standard set by EMVCo, a standards body for card-based payments \cite{apple_hce}.  In addition to complying with existing standards, Apple requires that the third-party developer have certain policies around privacy and dealing with vulnerabilities in their product\cite{apple_hce}. This is a vetting concern, and screening those developers looking to use HCE is an important aspect of protecting security in the new, open system.

In order to `open up' NFC capabilities for other wallet applications, third-party developers must have access to NFC, which is already available, as well as access to the Secure Element \cite{apple_se_press_release}.  Access to the Secure Element is already separate from from the Secure Enclave, which is where the operating system stores other sensitive data \cite{apple_security}.  This new access must be built correctly in order to preserve the security of both the third-party data and the rest of the system, and the precise architecture should be carefully designed to this end\footnote{Android supports HCE both with and without a secure element, and allows third-party interoperation with the later method and developers cannot access the secure element \cite{android_hce}. Apple could potentially achieve interoperability without involving the secure element at all.}. 
The Secure Element has internal boundaries that could impede a malicious developer from accessing other sensitive data and none of this data needs to come into contact with the operating system or app -- the reply can go straight from the hardware to the payment processor \cite{evil}.  Thus, APIs built for third-party developers could grant access to only specific functionalities without compromising others.  There are also existing vulnerabilities \cite{evil}, which would not necessarily be exacerbated by interoperability and might benefit from greater attention from the community.  Thus, there is an engineering security concern that must be solved by Apple. 

Since opening up this functionality to third-party developers requires significant engineering \emph{and} vetting security concerns, we consider it a hybrid case under our framework.

\refstepcounter{casestudy}
\paragraph{Case Study \thecasestudy: Physical devices and P2P Wi-Fi}
%\label{sec:case-studies:os-integration}
There are a number of other hardware and software features of iOS that the DMA has stipulated should be made available to third-party developers\footnote{DMA.100203 – Article 6(7) – Apple – iOS – SP – Features for connected physical devices}.  As stated in the case, these measures relate to notifications on smartwatches, background execution, content casting and transfers, and paired device set up.  Apple will be required to provide interoperability with the full list of functionalities to third parties.  All of these functionalities are already available to Apple's own devices (including Apple Watch, AirPods, and other iPhones), so achieving interoperability is a question of opening up previously reserved functionality to third-parties, generally via APIs and SDKs.

Although the stipulations from the Commission are still evolving, Apple currently will evaluate interoperability requests on a case-by-case basis \cite{apple_white_paper_2}.  In their recent public-facing document, Apple mentions that Meta has made 15 requests for interoperation regarding various functionalities \cite{apple_white_paper_2}. While Apple may be required to interoperate with competitors, the regulators do not require that any interoperation request be granted in a way that would undermine the security of the device\footnote{REGULATION (EU) 2022/1925 on contestable and fair markets in the digital sector (Digital Markets Act), Article 6(7) }. Thus, Apple could deny interoperation requests from malicious actors and create bespoke and secure solutions for interoperation to reduce security risk.

One illustrative example is peer-to-peer (P2P) Wi-Fi connections, which can be used to transfer files, such as with Apple's AirDrop feature, and continuity services, such as Handoff or Universal Clipboard\cite{Darmstadt_21}. These functionalities use a Bluetooth Low Energy (BLE) connection between two devices to create an \emph{ad hoc} P2P Wi-Fi connection, which uses Apple's iCloud credentials as part of its certificate procedure \cite{apple_security}. Several aspects of these services, including BLE, have been shown to leak user information or otherwise have vulnerabilities \cite{BLE_not_private, ble_not_private_2, apple_bluetooth}.  Apple Wireless Direct Link (AWDL) is an key protocol used in P2P Wi-Fi, which is based on IEEE 802.11, a set of standards, but is currently proprietary \cite{Darmstadt_18}.  This protocol has been shown to have security vulnerabilities \cite{Darmstadt_18, Darmstadt_19, Darmstadt_21}. The fact that AWDL is proprietary is a barrier both to better study of its security and to interoperation \cite{Darmstadt_18}. 

To open P2P Wi-Fi up more broadly, the Commission proposes\footnote{DMA.100203 – Article 6(7) – Apple – iOS – SP – Features for connected physical devices} that Apple either make AWDL available to third-parties or that Apple facilitate third-party interoperation through Wi-Fi Aware,\footnote{Also called Neighbor Awareness Networking (NAN).} which is a public protocol that has similar functionality and is based on AWDL\cite{wifi_patent}. If Apple makes AWDL public, there is little engineering to be done, as long as the protocol is indeed secure.  If the security of AWDL is reliant upon its details remaining secret, i.e. \emph{security through obscurity}, Apple and its users might be better served by Wi-Fi Aware or an improved version of AWDL. 

%One key distinction between AWDL and Wi-Fi Aware is that the latter uses passwords as a means of protecting user security \cite{wifi_aware}, whereas AWDL relies on a user's iCloud to utilize certificates on the upper layer to secure communication \cite{apple_security, darm_21}.  This suggests that Wi-Fi Aware, or at least the password feature, confers more security than AWDL in this respect. In either case, AWDL would likely benefit from scrutiny from the community. Using a public protocol would not necessarily preclude Apple from vetting those developers with applications or devices that are allowed to pair to Apple products.

\subsubsection{Analysis: Hybrid Concerns}
The hybrid cases are more than just the sum of the other two; we see patterns specific to this more complicated case.  

\question{1. \qone}
The platform's market power is again a key feature of these cases. Since third-party products compete directly with those of the incumbent company (e.g., Apple Watch competing with third-party watches), the company has incentive to protect their current market share.
At the same time, the company has control over access to those key functionalities and can block access for competitors.
In contrast to vetting concerns, interoperation means that third parties gain access to functionality that was not previously available to third-party developers at all.  
Since the company must both build access to previously exclusive functionality and also vet who gets that access, the platform's control over the situation is even greater than in engineering or vetting cases.  Thus, the company's ability to block access is more robust in the hybrid case.

\question{2. \qtwo} The platform has a clear economic incentive to protect their exclusive products that are bundled or complementary to their main devices. Preventing third-party developers from offering the full functionality available on Apple's own products reduces the  value of those other products and puts theirs at an advantage, i.e., self-preferencing. Supporting interoperation would require security engineering in concert with third parties.  These incentives coalesce into significant motivation for a platform to keep systems closed. 

\question{3. \qthree} 
 Not only must the platform build secure APIs or other means of access for third-party developers, but many of these APIs may need to be bespoke for specific uses. This could be a significant engineering task. The platform can vet all proposals for interoperation; this enables them to enforce security standards, and  keep tabs on what third-party developers are trying to do. Interoperation in hybrid cases is likely to be subject to more scrutiny from the platform in control than in vetting cases, and to require more coordination between the platform and third parties. Although perhaps costly, this may make for better, more secure products.

\question{4. \qfour} 
Existing interoperation policy involves interoperation with third parties, despite some potential risks (such as with Bluetooth devices \cite{bluetooth_mitm}).
When opening interoperation, the platform's greater case-by-case control in hybrid cases may mean less difficulty in maintaining security standards than when having to vet millions of apps as in the pure vetting case.

\section{Comparative Analysis and Takeaways}
\label{sec:big-picture}

We now offer a \emph{comparative} analysis, identifying key considerations across our taxonomy and case studies. First, we consider \emph{economic incentives, market power, and security}, corresponding to Analysis Questions 1 and 2. Then we discuss \emph{security trade-offs}, corresponding to Analysis Questions 3 and 4. Finally, we offer big-picture takeaways. We also summarize our analysis in Table~\ref{tab:big_table} in the Appendix.

\paragraph{Economic incentives, market power, and security}
Market power is usually the purview of economists and lawyers, but recognizing the nature of a company's market power can be a key step in understanding the ways in which security can become subordinated to other business goals.

In our framework, we do not categorize cases by the nature of the incumbent company's market power, and yet different security concerns seem to correlate with different types of market power. For instance, we notice that cases with \bi{security engineering concerns} are also those in which the market power of the incumbent is just its size. The relative size of a company, or, more precisely, its market share, and the resultant network effects, can confer significant power without more complicated `gatekeeper' dynamics. As noted in Section \ref{sec:framework}, these instances are also often those in which the necessary interoperation is horizontal, suggesting that the parties in these cases might be on a more equal footing when it comes to making technical decisions.  

The nature of the security concerns and the discourse around them is, in some sense, the simplest for security engineering concerns: something needs to be built. That does not mean that these concerns are always as insurmountable or substantiated as the companies are incentivized to suggest. There is an engineering task, but the technology to interoperate securely often exists or can be developed. It is also fairly clear what `interoperating securely' will look like, in most cases, which stands in contrast to the other two categories.

This pattern has a counterpart in \bi{security vetting concerns}, which involve instances of vertical interoperation (with a platform), as this type of argument is made by platforms vetting other products. As noted in Section \ref{sec:case-studies}, vetting concerns involve company discretion regarding security decisions and policy choices, and rarely have clear-cut engineering answers. Thus, vetting concerns have the potential to be more malleable, and therefore more easily tailored to fit the company's economic needs. In addition to this malleability, the platform position gives the company outsized economic power in these scenarios, and these two aspects combine into significant power to impose punitive prices or exclude third-party developers.  Thus, this type of concern should be identified as particularly susceptible to a blurring of boundaries between security and economic incentives, compared to  engineering concerns.  

The \bi{hybrid case} is not simply a dilution of the dynamics of the other two cases.  The platform power manifests here too.  Rather than using platform power to extract fees, as with security vetting concerns, hybrid cases see companies using their platform power for self-preferencing their products, hobbling or sometimes entirely excluding potential competition.  In both vetting and hybrid cases, the company acts as the platform gatekeeper. However, the extensive control of the incumbent company to block competition through both vetting and engineering concerns is distinct to hybrid cases, meaning the platform power is thus particularly entrenched. 

The engineering demands in a hybrid case are no less significant than in a security engineering case and the platform power is bolstered by legitimate engineering concerns. This means there are two lines of defense to keep out both bad actors and competitors, compounding the effects we see in pure engineering or vetting cases.   Third-party competitors may be more completely shut out in this case as compared to the other two. Without interoperation, messaging apps can still function with their own users, and apps can be distributed on iPhones as long as they adhere to Apple's guidelines and fee schemes. But without access to the HCE, there can be no alternative to Apple Pay on the iPhone. Third-party smartwatches can function without access to all hardware functionalities, but their utility may be significantly limited. 

Thus, hybrid security concerns should be evaluated with careful consideration of these unusually strong market ramifications as compared to engineering or vetting concerns.

\paragraph{Security trade-offs}
Security engineering involves many practical trade-offs: between security and efficiency, usability, profit, and more. For example, achieving usability and security at once is well known to be challenging~\cite{usability_book}. Usability also interacts with companies' economic incentives: a phone with such tight security that it is unusable would be hard to sell.  
Economic interests can also motivate pressures against interoperation. Just as undermining usability through a narrow focus on security can ruin a product,  undermining competition through a narrow focus on security can ruin a market. A key difference is that the latter aligns with companies' interests, so could conceivably happen, whereas the former is against their interests, so would never happen in practice.

The simplest case is again the \bi{security engineering concern}. Inherent in the idea of an engineering concern is the idea that the problem can be solved by engineering. The trade-off, then, relates to how much to invest in solving the engineering problem and how much of a burden that is. In practice, of course, even the best, well-funded systems have bugs and vulnerabilities. Thus, companies must deploy systems that might be vulnerable.  This tension carries over into interoperation, where again systems might have vulnerabilities despite their best efforts. As such, we see using security arguments to push against interpretation as often ultimately about business interests more than security. 
We distinguish the preceding type of statement from the seemingly rarer type of security engineering concern that states that ``interoperation cannot be done securely \emph{at all} as a technical matter,’’ which is a different matter that must be evaluated from a technical security perspective.\footnote{The statement that E2EE messaging services cannot interoperate with unencrypted messaging services while maintaining their level of security would fall in this category. (Note: Regulators do not require such interoperation.)} We believe it is essential to systematically distinguish between these, as market actors are incentivized to obfuscate the distinction between these two types of security engineering concerns and evoke the higher security stakes of the latter in the former.


When companies raise \bi{security vetting concerns}, they may argue that security considerations necessitate limitations on interoperation, or perhaps preclude interoperation entirely. Recall that security vetting concerns arise from company policies that do not permit interoperation or make it costly or difficult, rather than technical interoperability issues. Policies generally involve discretionary trade-offs between security and other interests. The trade-offs made in existing policies are an essential reference to evaluate how much of a divergence from existing security practices, and how much of an economic burden, a proposed policy change would be. 

Consider the case of alternative app stores (Case Study~\ref{cs:app-stores}). The status quo involves a profitable fee structure by a gatekeeper platform. The aim of the proposed interoperation is primarily to allow developers to escape the fees, not existing vetting guidelines. 
Much like the App Store Guidelines, both the guidelines for notarization and the potential guidelines of any alternative app store will consist of policy decisions. The interoperability mandate does not suggest Apple permit app stores with policies that are worse for security than current App Store Guidelines. 
Apple's proposed solution involves preserving a fee structure profitable for Apple across all alternative app stores, the details of which would make operating on alternative app stores not worth it for many developers. One reason they give for the necessity of this fee structure is security. While security vetting can be important and does incur some costs, meaning some fee collection may be beneficial for security, the structure of this ``security vs. interoperability'' argument should highlight that it is not just about security, and raise questions around the (mis)alignment of economic incentives, security arguments, and interoperation.

The \bi{hybrid case} features all of the aforementioned types of trade-offs. That said, in some ways, the hybrid trade-offs are of a slightly different character than the engineering and vetting trade-offs. Those seeking to interoperate with OS functionality reserved for device manufacturer use are fewer than those seeking to make apps for smartphones, so vetting could be a project of a substantially smaller scale. Relatedly, interoperation could be bespoke, involve more monitoring, and require less scalability. 


\paragraph{Big picture}
Our framework, and our discussion of case studies through the lens of the framework, highlight how the following themes feature in ``security vs. interoperability'' discourse in impactful policy contexts in the wild.
\begin{enumerate}
    \item Interoperation and economic incentives often conflict.
    \item Security arguments and economic incentives often align.
    %Conversely, there is often an alignment between economic incentives and security arguments.
    \item Such security arguments and their implications can only be fully understood in light of the broader economic and policy context as well as the technical security context.
\end{enumerate}

In instances in which vetting is the primary type of security, incumbent companies necessarily make trade-offs and also have strong platform control that can be leveraged to extract fees. The `hybrid' combination of platform power and guarded access to functionality can grant incumbent companies even greater power to exclude their direct competition.

\section{Conclusion}
\label{sec:conc}

The security community has an important role to play in impactful ongoing competition policy debates due to the prevalence of security arguments deployed to argue against interoperation.
Security arguments should always be taken seriously, considering both their security implications and their economic and policy implications together. In this paper, we have presented our framework for analyzing ``security vs. interoperability'' arguments within their broader economic and policy context, along with case studies and analysis illustrating the versatility, utility, and takeaways of our framework.


%\clearpage

\iffullversion
    \section*{Acknowledgments}
    The authors would like to extend special thanks to Mallory Knodel for helpful discussion and many insights. We are also grateful for the generous help of Alexander Heinrich, who provided indispensable knowledge on technical matters.  Thanks also to Andrés Fábrega and Mathy Vanhoef.
    % special thanks to Mallory for extensive discussions about ...
    % Andrés provided tips on emulation
    % Mathy Vanhoef & Alexander Heinrich
\else
    \newpage
    \section*{Ethics Statement}
    The aim of this work is to systematize the perceived tensions between security and interoperability in recent antitrust proceedings. We hope our work will support better informed policy-making, which we consider a societally beneficial goal. This work did not involve experiments or disclosures, did not reveal or discuss previously unknown vulnerabilities, and relied solely on public information.
    
    \section*{Open Science Statement}
    We have no code or other research artifacts to which the open science policy might apply. All our sources are cited here.
\fi

{\footnotesize \bibliographystyle{acm}
\bibliography{refs}}
%\bibliography{refs}

% \newpage
%  \quad     
\newpage
\section*{Appendix}
\subsection*{Fee calculations comparing App Store Fee and CTF}
We briefly illustrate that the cost of paying the core technology fee (CTF) is not necessarily less expensive than the current 30\% commission fee. Suppose there is an app that costs €1 and has no in-app purchases available.  If this app sees 6 million downloads in one year, then the fee paid to Apple would be $6\cdot0.3=1.8$ million euros if it were distributed by the App Store, but would pay $(6-1)\cdot 0.5 =2.5$ million euros if it were distributed separately and paid the CTF (which is free for the first million downloads) \cite{apple_ctf}.  This constitutes a loss for the developer already in the first year.  If they were to then update the app, which would be freely available to the users, after one year, they would have to pay another €2.5 million while reaping no income. 

Apple also has specific requirements of the developers themselves that must be met for Apple to allow direct web distribution.  Web distribution, i.e. enabling users to directly download apps from the developer's website, is a simple way for developers to have direct access to their users. In order for a developer to be granted permission to do web distribution Apple requires that developers be members of Apple Developer Program for at least two years and have an app that had more than one million first annual installs in the previous year \cite{apple_web_sideloading}.  This would prevent new developers from offering their first apps in this manner.  Presumably, developers that have recently had an app with at least one million first annual installs are likely to follow-up with a similarly popular app.  As with all alternative distributions, the new app would incur the CTF for all installs above one million. Apple has some exceptions to their fees for small developers and NGOs \cite{apple_ctf}, but these groups are less likely to fulfill the eligibility requirements. Thus, while these requirements are ostensibly to control for responsible app developers, they track closely with criteria for selecting only those developers who will also have to pay substantial fees. 


\begin{table*}[ht]
\centering
\renewcommand{\arraystretch}{1.5} % Adjust row height for readability
%\setlength{\tabcolsep}{4pt} % Adjust column spacing
\begin{tabularx}{\textwidth}{|X|X|X|X|}

\hline
&\textbf{Engineering} & \textbf{Vetting} & \textbf{Hybrid} \\
\hline
\textbf{1. How do interoperation and economic incentives clash (or not)? } & Network effects are more important than interoperation and Apple can use its green bubbles to push iPhone uptake  & Forcing third parties through existing vetting means collecting fees opening that up could result in a loss of revenue & Preventing interoperation keeps other developers from offering products with those functionalities \\ 
\hline
\textbf{2. How do security arguments and economic incentives align (or not)?} & Interoperation could reduce uptake pressure and would be an engineering challenge & Being gatekeeper on security grounds coincides with extracting rent and keeping out competition & If only in-house products can be trusted with certain features, no competitors can offer similar products  \\
\hline
\textbf{3. What needs to change in order for the company to
interoperate? } & Building a system that allows for interoperation and/or moving to a standard protocol & Continuing existing vetting and developing vetting for newly opened avenues, changing pricing schemes & Building in interoperation functionality as well as vetting potential new comers  \\

\hline
\textbf{4. What tension exists between security and other goals
both before/after?} & No tension if you do it right! Getting the whole community involved might make a more secure product & Already allowing some less than perfect actors on the platform; new system could see a loss of revenue and less security if standards are lessened & Already using a vulnerable system; interoperation could improve security or keep it the same if engineering is done well  \\
\hline
\end{tabularx}

\caption{Summary of analysis results \sunoo{need to retitle and make sure there's a pointer to this table somewhere in the main body}}
\label{tab:big_table}
\end{table*}


\end{document}