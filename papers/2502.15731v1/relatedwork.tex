\section{Related Work}
The landscape of \ac{ai}-driven control in communication networks has seen significant advancements, particularly within the context of the Open Radio Access Network (O-RAN). The O-RAN Alliance has introduced the concept of the near-real-time RAN Intelligent Controller (near-RT RIC), which plays a pivotal role in enabling intelligent, software-defined management of \acp{ran} by supporting multi-party applications for enhanced performance and flexibility~\cite{polese2023understanding}. Lightweight control applications (i.e. xApps and rApps) have been developed for open-source \acp{ric} \cite{lee2020hosting, xavier2023machine}. Integration of \ac{ric}-based AI network controls with \ac{sdn} is also being explored \cite{xavier2024cross}. While the O-RAN architecture provides a standardized platform for \ac{ai}/\ac{ml} control in radio networks, there is no equivalent for transport and optical networks~\cite{lee2020hosting}. These domains use \ac{sdn} platforms like \ac{onos} to orchestrate network functions \cite{akinrintoyo2023reconfigurable}. In optical networks, \ac{ai} control focuses on offline \ac{qot} estimation, wavelength routing, and fault management \cite{mata2018artificial}. Extensions of \ac{onos} for online \ac{ai} and hierarchical control have been explored \cite{morro2018automated}. Applying \ac{ai} to radio and optical network control is relatively new, lacking accepted cross-domain frameworks, which will be essential for end-to-end AI controls. 

% Applying \ac{ai} to network control operations is relatively new, especially for radio and optical networks. Thus, standardized frameworks for cross-domain \ac{ai}-based control are lacking and highly beneficial.

% These domains typically utilize \ac{sdn} platforms like \ac{onos} to orchestrate network functions \cite{akinrintoyo2023reconfigurable}. The \ac{tip} further brings in open cell gateway functions to the environment \cite{grammel2018physical}. Much of the \ac{ai} control in optical networks focuses on offline \ac{qot} estimation, wavelength routing, and fault management \cite{mata2018artificial}. Extensions of \ac{onos} for online \ac{ai} and hierarchical control have also been explored \cite{morro2018automated}. The application of \ac{ai} methods to network control operations is a relatively new area of research, particularly in the context of radio and optical networks. Hence, there is a lack of standardized frameworks and tools for cross-domain \ac{ai}-based control and developing those will be highly beneficial.  

The evolution of the Cloud-RAN has also contributed to network performance enhancements. Defined by the \ac{ngnm} alliance, Cloud-RAN's architecture includes the \ac{cu}, \ac{du}, and \ac{ru}, connected via the Low Layer Split transport interface or fronthaul~\cite{das2019variable}. Traditional dedicated fiber fronthaul networks are being reconsidered in favor of \ac{pon} to reduce connectivity costs and support the increased small cell densification required for 5G and beyond. However, the challenge of latency caused by \ac{dba} remains a significant issue, particularly due to the lack of coordination between PON and DU upstream schedulers~\cite{slyne2024demonstration}.

The concept of "Cooperative DBA" was introduced to mitigate this latency by facilitating better coordination between optical and mobile networks. This concept has been further developed and standardized as the "Cooperative Transport Interface (CTI)," which allows for enhanced synchronization between PON and RAN upstream schedulers, resulting in more efficient resource allocation and reduced latency~\cite{nomura2017first}. Despite these advancements, the integration of AI capabilities into the \ac{pon} has not been fully explored.

Our work aims to bridge this gap by extending the principles of O-RAN near-RT RIC controllers into the optical domain, thereby promoting a unified AI-driven control framework across both wireless and optical networks. By adding AI capabilities to the \ac{pon}, we aim to achieve seamless integration and coordination between these domains, enabling more sophisticated and intelligent network management. This approach not only leverages the existing strengths of the O-RAN framework but also enhances it with AI-driven insights, facilitating the creation of intelligent, zero-touch networks that can self-configure, monitor, and repair across a broad range of network scenarios.

% In summary, while significant progress has been made in the development of AI-driven control frameworks for RAN, our research extends these principles to encompass the entire network, including the optical domain. By integrating AI capabilities into the CTI interface, we demonstrate how seamless coordination between different network segments can be achieved, paving the way for the development of advanced, AI-driven communication networks for the future.
% \vspace{-0.5em}