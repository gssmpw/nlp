\section{Related Work}
Revizor \cite{oleksenko2022revizor} uses Software-Hardware Contracts, the main difference being that Revizor uses a fuzzer with some speculative filters, whereas SpecRL uses an agent, which has the concept of sequential decision making. We believe this is a more intuitive approach to speculative execution leaks that require the "training" of a branch predictor. As we have shown, Revizor does not scale well with large program sizes, while SpecRL does.

% other related work.

AutoCAT \cite{luo2023autocat} was the first work to explore microarchitectural attacks using reinforcement learning. It only focused on cache timing attacks, however, and does not consider the branch predictor or any speculative aspects of the microarchitecture. MACTA \cite{cui2023macta} uses multiagent RL for attack and detection of cache side-channel, but it also only considers cache timing channel mainly on a simulators.