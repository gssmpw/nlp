\section{Related Work}
\label{sec:RelatedWork}

A related, but simpler, problem explored in previous works uses given disk centers.
That is, we are given two point sets: the assets $\points$ and the disk centers
$Y$, and a coverage function $\kappa$. Wanted are the radii of the disks
centered at $Y$ that meet the coverage requirement and minimize the sum of the
areas of the disks.  This is known as the non-uniform minimum-cost multi-cover
(MCMC) problem.
If $ \forall p \in \points, \kappa(p)=k $ it is referred to as the uniform MCMC.

The uniform MCMC, motivated by fault-tolerant sensor network design, was considered by Abu-Affash~et~al.~\cite{abu2011multi}, who presented an algorithm with a cost that is at most $23.02 + 63.95(\kappa_{\text{max}} - 1)$ times the cost of the optimal solution, where
$\kappa_{\text{max}}$ is the maximum coverage requirement.
The expected running time of the algorithm is $\bigO((n + m)\kappa_{\text{max}})$, where $n$ is the number of points of $\points$ and $m$  is the number of points of $Y$.

BarYehuda and Rawitz \cite{bar2013note} gave a $3^\alpha \kappa_{\text{max}}$-approximation algorithm for the non-uniform MCMC where the minimized cost is $\sum_{i=1}^{m} r_i^{\alpha}$.
The algorithm's expected running time   is polynomial.

Bhowmick~et~al.~\cite{bhowmick2013constant} also tackled the non-uniform MCMC problem.
The authors presented a polynomial-time algorithm achieving the first constant factor approximation for $\kappa(\points)>1$.
This work demonstrates that an approximation bound independent of $\kappa$ can be achieved.
The proposed algorithm recursively computes a $(\kappa - 1)$-cover and extends it to a $\kappa$-cover, relating the cost of this extension to the cost of a subset of disks referred to as the primary disks.
The algorithm's expected running time is polynomial.

Huang~et~al.~\cite{huang2021ptas,huang2024ptas} addressed the non-uniform MCMC problem.
The authors presented the first Polynomial Time Approximation Scheme (PTAS) for this problem for $\kappa(\points)>1$, providing a solution that can be made arbitrarily close to the optimal by choosing an appropriate $\epsilon$.
Their approach utilized techniques such as Balanced Recursive Realization and Bubble Charging, which allowed them to optimize the disks at a sub-disk level.
The expected running time of their $1+\epsilon$ approximation algorithm is $\bigO\left(n^{\bigO(1)} m^{\bigO(1/\epsilon) \bigO(d/\alpha)}\right)$, where the points are in $\mathbb{R}^d$ space and the minimized cost is $\sum_{i=1}^{m} r_i^{\alpha}$.
Instead of optimizing each disk as a whole, they show that it is possible to further approximate each disk with a set of sub-boxes and optimize them at the sub-disk level.
They first compute an approximate disk cover with minimum cost through dynamic programming, and then obtain the desired disk cover through a balanced recursive realization procedure.

Another line of closely related work examines placing the minimum number of unit disks to multi-cover a set of points.
As in our problem, the centers of these circles are arbitrary.
Unlike our problem, the radii are fixed and the number of unit disks is unbounded.
Gao~et~al.~\cite{gao2022fast} presented a 5-approximation algorithm with runtime $\bigO(n + \kappa_\text{max})$, and they introduced an improved 4-approximation algorithm with a higher time complexity of $\bigO(n^2)$. 
Filipov and Tomova tackled the problem of coverage with the minimum number of equal disks~\cite{Georgiev2023}.
They proposed a stochastic optimization algorithm of estimated time complexity $\bigO(n^2)$.