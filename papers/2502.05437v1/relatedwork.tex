\section{Related work}
There are a series of works on checking whether two given distributions are identical (e.g. \cite{CortesMR07,DoyenHR08,KieferMOWW11,BGMMPV24ICLR}), which can be viewed as the decision version of computing the TV-distance, i.e., checking whether it is zero. 



There are also a long line of works (e.g., see a survey in \cite{Canonne15}) studying the identical testing problem in access model, where the algorithm can only access the set of random samples from the distributions.
This setting is different from our setting, where we assume that all the parameters of the spin systems are given as the input to the algorithm.


A series of works~\cite{0001GMV20,ChenK14,Kiefer18,CanonneR14} studied the algorithm and the hardness of approximating the TV-distance with additive error, which is an easier problem than the relative-error approximation.


\cite{devroye2018total,ArbasAL23,kontorovich2024tensorization,kontorovich2024sharp} found \emph{closed-form formulas}, which approximates the TV-distance between two high-dimensional distributions with a \emph{fixed} relative error.
We study \emph{algorithms} achieving an \emph{arbitrary} $\epsilon$-relative error approximation for spin systems.


%