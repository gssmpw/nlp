\section{Parameter distance v.s. total variation distance}\label{sec:lower}
In this section, we prove \Cref{lem:TV-lower}. We first prove the result for the hardcore model in \Cref{sec:hardcore-proof}, and then prove the result for the soft-Ising model in \Cref{sec:Ising-proof}. 
\subsection{Analysis for the hardcore model}\label{sec:hardcore-proof}
Recall that our problem setting is: Let $G=(V,E)$ be a graph, and $(G,\lambda^\mu)$ and $(G,\lambda^\nu)$ are two hardcore models on the same graph, satisfying the uniqueness condition in~\eqref{eq:cond-hardcore}. Let $\mu$ and $\nu$ denote distributions of $(G,\lambda^\mu)$ and $(G,\lambda^\nu)$ respectively. The parameter distance $\dis(\mu,\nu)$ for hardcore model is defined by $\dis(\mu,\nu) \defeq \Vert \lambda^\mu - \lambda^\nu \Vert_{\infty}$.
%
Now we prove the first hardcore model part of \Cref{lem:TV-lower}: 
\[\DTV{\mu}{\nu} \geq \CC\cdot \dis(\mu,\nu), \quad\text{where } \CC = \frac{1}{5000}.\]

Let $i \in V$ be one vertex such that $|\lambda_i^\mu-\lambda_i^\nu|=\dis(\mu,\nu)$. Without loss of generality, we can assume that $\lambda_i^\mu-\lambda_i^\nu=\dis(\mu,\nu)$. Otherwise, we can flip the roles of $\mu$ and $\nu$ in the following proof.
Define two collections of independent sets of $V$:
\begin{align*}
    H_1:&=\{S\text{ is an independent set of }G\mid i\in S\},\\
    H_2:&=\{S \setminus \{i\} \mid S \in H_1 \}.
\end{align*}
In this proof, we use $\mu(S)$ to denote the probability of $\sigma \in \{\pm\}^V$ in $\mu$ such that $\sigma_i = +1$ if and only if $i\in S$.
Define $\mu(H_1) = \sum_{S \in H_1} \mu(S)$ and $\mu(H_2),\nu(H_1),\nu(H_2)$ in a similar way.
It is easy to verify that 
\begin{align*}
\frac{\mu(H_1)}{\mu(H_2)}=\lambda_i^\mu ,\quad \frac{\nu(H_1)}{\nu(H_2)}=\lambda_i^\nu.
\end{align*}

Then, we consider two cases depending on the value of $|\mu(H_2)-\nu(H_2)|$. The simple case is $|\mu(H_2)-\nu(H_2)|\geq \CC\cdot \dis(\mu,\nu)$, then $\DTV{\mu}{\nu} \geq|\mu(H_2)-\nu(H_2)|\geq \CC\cdot \dis(\mu,\nu)$.

The main case is $|\mu(H_2)-\nu(H_2)|< \CC\cdot \dis(\mu,\nu)$. In this case we show that $|\mu(H_1)-\nu(H_1)|\geq \CC\cdot \dis(\mu,\nu)$, which also implies $\DTV{\mu}{\nu} \geq|\mu(H_1)-\nu(H_1)|\geq \CC\cdot \dis(\mu,\nu)$.
We can lower bound the value of $|\mu(H_1)-\nu(H_1)|$ as follows:
\begin{align}\label{eq:mu-nu-diff}
\mu(H_1)-\nu(H_1)&=\lambda_i^\mu \cdot \mu(H_2)-\lambda_i^\nu \cdot \nu(H_2)\notag\\
&=(\lambda_i^\nu+\dis(\mu,\nu)) \mu(H_2)-\lambda_i^\nu \cdot \nu(H_2)\notag\\
&=\lambda_i^\nu(\mu(H_2)-\nu(H_2))+\dis(\mu,\nu) \cdot \mu(H_2)\notag\\
&>\dis(\mu,\nu) \cdot \mu(H_2)-\lambda_i^\nu \cdot \CC \cdot \dis(\mu,\nu).
\end{align}

Because $(G,\lambda^\mu)$ and $(G,\lambda^\nu)$ satisfy the uniqueness condition, we have for all vertex $ i\in V$, all  $x\in \{\mu,\nu\}$, $\lambda_i^x \leq \lambda_c(\Delta) = \frac{(\Delta-1)^{\Delta - 1}}{(\Delta-2)^\Delta}\leq 4$ because $\Delta \geq 3$.
Let $N(i)$ denotes the set of neighbors of $i$ in graph $G=(V,E)$. 
By definition of $H_2$, $\mu(H_2)$ is the probability of $j \notin S$ for all $j \in N(i) \cup \{i\}$ and $S \sim \mu$ is a random independent set from the hardcore model $(G,\lambda^\mu)$. Suppose there is a total ordering $<$ among all vertices in $V$. We have
%The independent set in $H_2$ is that: fix $i$ and $N(i)$ be zero, and other vertices belong to all independent sets in the induced graph $G[V-i-N(i)]$. Note that for any arbitrary independent set $S_0$, the partial configuration in $G[V-i-N(i)]$ is an independent set. We can compute that
\begin{align}\label{eq:mu-H2}
\mu(H_2)&= \Pr[S \sim \mu]{ i \notin S} \prod_{j \in N(i)}\Pr[S \sim \mu]{ j \notin S \mid (i \notin S) \land (\forall k \in N(i) \text{ with } k < j, k \notin S )} \notag\\
&\geq \frac{1}{1+\lambda_i^\mu}\prod_{j \in N(i)} \left(\frac{1}{1+\lambda_j^\mu}\right)\geq \left( \frac{1}{1+4/(\Delta-2)}\right)^{\Delta+1}\notag\\
&=\left( 1-\frac{4}{\Delta+2}\right)^{\Delta+1}> \frac{1}{1000}.
\end{align}
Also note that $\lambda_i^\nu \leq 4$ by the uniqueness condition. Recall $\CC = \frac{1}{5000}$. By~\eqref{eq:mu-nu-diff} and~\eqref{eq:mu-H2},  we have
\begin{align*}
\mu(H_1)-\nu(H_1)&>\dis(\mu,\nu) \cdot \mu(H_2)-\lambda_i^\nu \cdot \CC \cdot \dis(\mu,\nu)\\
&>\dis(\mu,\nu)\left(\frac{1}{1000}-\frac{4}{5000}\right)=\CC\cdot \dis(\mu,\nu).
\end{align*}

Next, we prove the second part of the hardcore model with $\CC = b^3$. Now, we do not have the uniqueness condition but we have a marginal lower bound $b$ and the the hardcore model is soft.
The proof is similar.
We only need to show how to lower bound the value in~\eqref{eq:mu-nu-diff}.
We first upper bound $\lambda^\nu_i$ in \eqref{eq:mu-nu-diff}. Consider the pinning that all neighbors of $i$ are fixed as $-1$, then the probability of $i$ taking $-1$ is $\frac{1}{1+\lambda^\nu_i} \geq b$. Thus, $\lambda_i^v \leq \frac{1-b}{b}$. Next, we lower bound the value of $\mu(H_2)$.
Let $N(i)$ denote all neighbors of $i$.
Note that $i$ takes value $+$ only if all neighbors of $i$ take the value $-$. Recall that we assume that $\lambda_i^\mu-\lambda_i^\nu=\dis(\mu,\nu)$ in the beginning of the proof. 
We can assume that $\dis(\mu,\nu) > 0$, otherwise $\DTV{\mu}{\nu} = 0$ and the lemma is trivial. 
We have $\lambda^\mu_i > 0$ so that $\mu_i(+) > 0$. 
By the marginal lower bound, $\mu_i(+) \geq b$.
The vertex $i$ takes value $+$ only if all neighbors take the value $-$. We have
\begin{align*}
    b\leq \mu_i(+) \leq \Pr[S\sim \mu]{ \forall j \in N(i), j \notin S}. 
\end{align*}
On the other hand, the value of $\mu(H_2)$ can be lower bound by
\begin{align*}
    \mu(H_2) &= \Pr[S \sim \mu]{\forall j \in N(i) \cup \{i\}, j \notin S}\\
    &= \Pr[S\sim \mu]{ \forall j \in N(i), j \notin S}\Pr[S \sim \mu]{i \notin S \mid \forall j \in N(i), j \notin S}\\
    &\geq b^2.
\end{align*}
We can set $\CC = b^3$ so that
\begin{align*}
\mu(H_1)-\nu(H_1)&>\dis(\mu,\nu) \cdot \mu(H_2)-\lambda_i^\nu \cdot \CC \cdot \dis(\mu,\nu)\\
&\geq\dis(\mu,\nu)\left(b^2-\frac{1-b}{b} \cdot b^3\right)=\CC\cdot \dis(\mu,\nu).
\end{align*}


%Now, we can lower bound the value of $\mu(H_2)$ such that 
%If there is no free neighbor of $i$, i.e., $\deg^{\text{free}}_i = 0$, then all neighbors of $i$ must take the value $-1$ and we have
%\begin{align*}
%    \mu(H_2) \geq \mu_v(-) \geq b > \frac{b}{2}.
%\end{align*}
%If there exists a free neighbor, then the free neighbor can take both values $\{\pm\}$, so that the marginal lower bound $b$ is at most $\frac{1}{2}$. Using \Cref{lem:const-degree},
%\begin{align*}
%    \mu(H_2) \geq b \cdot b^{\deg_i^{\text{free}}} \geq b^{1 + \frac{\ln (b)}{\ln (1-b)}}. %\geq b(1-b) \geq \frac{b}{2}.  
%\end{align*}
%We can set $\CC = b^{2 + \frac{\ln (b)}{\ln (1-b)}}$ so that
%\begin{align*}
%\mu(H_1)-\nu(H_1)&>\dis(\mu,\nu) \cdot \mu(H_2)-\lambda_i^\nu \cdot \CC \cdot \dis(\mu,\nu)\\
%&\geq\dis(\mu,\nu)\left(b^{1 + \frac{\ln (b)}{\ln (1-b)}}-\frac{1-b}{b} \cdot b^{2 + \frac{\ln (b)}{\ln (1-b)}}\right)=\CC\cdot \dis(\mu,\nu).
%\end{align*}

\subsection{Analysis for the soft-Ising model} \label{sec:Ising-proof}

Recall that our problem setting is: Let $G=(V,E)$ be a graph, 
and $(G,J^\mu,h^\mu)$ and $(G,J^\nu,h^\nu)$ are two soft-Ising models on the same graph.
Let $\mu$ and $\nu$ denote distributions of $(G,J^\mu,h^\mu)$ and $(G,J^\nu,h^\nu)$ respectively. 
The parameter distance $\dis(\mu,\nu)$ for soft-Ising model is defined by 
\[\dis(\mu,\nu) \defeq \max\left\{ \Vert J^\mu - J^\nu \Vert_{\max}, \max_v \frac{|h^\mu_v-h^\nu_v|}{\deg_v+1} \right\}.\]

Now we prove the soft-Ising model part of \Cref{lem:TV-lower}: 
if both $\mu$ and $\nu$ are $b$-marginally bounded for $0<b< 1$, then
\[\DTV{\mu}{\nu} \geq f(b) \cdot \dis(\mu,\nu), 
\quad\text{where } f(b) = \frac{1}{2}b^2.\]
%\todo{HL: I will fix parameter later}
Before proving the above, we first present the following lemma.
\begin{lemma}
\label{lem:local-global-dtv}
Let $\mu$ and $\nu$ be any two distributions on $\{\pm\}^V$, and $S\subseteq V$ be a subset of vertices.
Let $0<\delta<1$. 
If for any $\sigma\in \{\pm\}^S$, $\DTV{\mu^\sigma}{\nu^\sigma}\geq \delta$, then $\DTV{\mu}{\nu}\geq \delta/2$.
\end{lemma}

\begin{proof}
For any two distributions $p,q$ on space $\Omega$, by the definition of total variation distance,
\begin{align*}
\sum_{x\in \Omega}\min(p(x),q(x)) &= 1-\DTV{p}{q},\\
\sum_{x\in \Omega}\max(p(x),q(x)) &= 1+\DTV{p}{q}.
\end{align*}

Under the assumption of the lemma, we have
\begin{align*}
1-\DTV{\mu}{\nu}&=\sum_{\sigma\in \{\pm\}^V}\min(\mu(\sigma),\nu(\sigma))
\\
&=\sum_{\sigma\in \{\pm\}^S}\sum_{\tau\in \{\pm\}^{V\setminus S}}\min(\mu(\sigma)\mu(\tau\mid\sigma),\nu(\sigma)\nu(\tau\mid\sigma))
\\
&\leq \sum_{\sigma\in \{\pm\}^S}\max(\mu(\sigma),\nu(\sigma))\sum_{\tau\in \{\pm\}^{V\setminus S}}\min(\mu(\tau\mid\sigma),\nu(\tau\mid\sigma))
\\
(\text{since }\DTV{\mu^\sigma}{\nu^\sigma}\geq \delta) \quad 
&\leq \sum_{\sigma\in \{\pm\}^S}\max(\mu(\sigma),\nu(\sigma))(1-\delta) 
\\
&= (1+\DTV{\mu_S}{\nu_S})(1-\delta),
\\
&\leq 1-\delta+\DTV{\mu_S}{\nu_S}
\end{align*}    
which implies
\begin{align}
\label{eq:local-global-dtv-1}
\DTV{\mu}{\nu}\geq \delta - \DTV{\mu_S}{\nu_S}.
\end{align}
On the other hand, we have 
\begin{align}
\label{eq:local-global-dtv-2}
\DTV{\mu}{\nu} \geq \DTV{\mu_S}{\nu_S}.
\end{align}
Then \Cref{lem:local-global-dtv} follows by combining~\eqref{eq:local-global-dtv-1} and~\eqref{eq:local-global-dtv-2}.
\end{proof}

According to the definition of the parameter distance, there are $2$ cases:
\begin{itemize}
\item It exists $\{u,v\}\in E$ such that $|J^\mu_{u,v}-J^\nu_{u,v}|=\dis(\mu,\nu)$.
\item It exists $v\in V$ such that $|h_v^\mu-h_v^\nu|=\dis(\mu,\nu) \cdot (\deg_v+1)$.
\end{itemize}

For the first case, we will show that
\begin{align}
\label{eq:Ising-dtv-lowerbound-1}
%\forall \sigma \in \{\pm\}^{V\setminus\{u,v\}}, \DTV{\mu^\sigma}{\nu^\sigma}\geq b^2\cdot \dis(\mu,\nu) =  2f(b)\dis(\mu,\nu),
\forall \sigma \in \{\pm\}^{V\setminus\{u,v\}}, \DTV{\mu^\sigma}{\nu^\sigma}\geq 2f(b)\dis(\mu,\nu),
\end{align}
and for the second case, we will show that
\begin{align}
\label{eq:Ising-dtv-lowerbound-2}
%\forall \sigma \in \{\pm\}^{V\setminus\{v\}}, \DTV{\mu^\sigma}{\nu^\sigma}\geq b\cdot \dis(\mu,\nu) \geq 2f(b)\dis(\mu,\nu).
\forall \sigma \in \{\pm\}^{V\setminus\{v\}}, \DTV{\mu^\sigma}{\nu^\sigma} \geq 2f(b)\dis(\mu,\nu).
\end{align}

Assuming~\eqref{eq:Ising-dtv-lowerbound-1} and~\eqref{eq:Ising-dtv-lowerbound-2} hold, using \Cref{lem:local-global-dtv}, 
we can directly derive the result of the total variation distance lower bound for soft-Ising model claimed in \Cref{lem:TV-lower}.

Next, we will prove~\eqref{eq:Ising-dtv-lowerbound-1} and~\eqref{eq:Ising-dtv-lowerbound-2},
respectively to complete the entire proof.

\begin{proof}\ifthenelse{\boolean{conf}}{\textbf{of eq.\eqref{eq:Ising-dtv-lowerbound-1}}}{[Proof of eq.\eqref{eq:Ising-dtv-lowerbound-1}]}
Recall that $\mu,\nu$ are both $b$-marginal bounded which is defined in \Cref{def:marlow}. Thus for any $\sigma \in \{\pm\}^{V\setminus \{u,v\}}$, for any $\tau \in \{\pm\}^{\{u,v\}}$, we have $\mu^\sigma(u=\tau_u,v=\tau_v) = \mu^\sigma_u(\tau_u)\cdot \mu^{\sigma\wedge u\gets\tau_u}_v(\tau_v) \geq b^2$. The same local lower bound also holds for the distribution $\nu$.

Define $c^\mu_u=h^\mu_u+\sum_{\{u,w\}\in E, w\neq v}J^\mu_{u,w}\sigma_w$ and $c^\mu_v=h^\mu_v+\sum_{\{v,w\}\in E, w\neq u}J^\mu_{v,w}\sigma_w$ be the influence coefficient of the external field $\sigma$ on $u,v$ in the distribution $\mu$. Similarly, let $c^\nu_u$ and $c^\nu_v$ denote the corresponding influence coefficients in the distribution $\nu$. 

According to the definition of the Ising model, when the external field $\sigma$ is fixed, the local distribution of $u,v$ in the distribution $\mu$ is follows:
\begin{align*}
    \mu^\sigma(u=+,v=+) &= \exp(J^\mu_{u,v}+c^\mu_u+c^\mu_v-s^\mu),
    %\quad \nu^\sigma(u=+,v=+) = \exp(J^\nu_{u,v}+c^\nu_u+c^\nu_v-s^\nu)
    \\
    \mu^\sigma(u=+,v=-) &= \exp(-J^\mu_{u,v}+c^\mu_u-c^\mu_v-s^\mu),
    \\
    \mu^\sigma(u=-,v=+) &= \exp(-J^\mu_{u,v}-c^\mu_u+c^\mu_v-s^\mu),
    \\
    \mu^\sigma(u=-,v=-) &= \exp(J^\mu_{u,v}-c^\mu_u-c^\mu_v-s^\mu),
\end{align*}
where $s^\mu = \log(\exp(J^\mu_{u,v}+c^\mu_u+c^\mu_v)+\exp(-J^\mu_{u,v}+c^\mu_u-c^\mu_v))+\exp(-J^\mu_{u,v}-c^\mu_u+c^\mu_v)+\exp(J^\mu_{u,v}-c^\mu_u-c^\mu_v))$. 

Using the same method, we define $s^\nu$, and the local distribution of $u,v$ in the distribution $\nu$ can be expressed in terms of $c^\nu_u,c^\nu_v$ and $s^\nu$, as in the above equation.

Let, $p=J^\mu_{u,v}+c^\mu_u+c^\mu_v-s^\mu$ and $q=J^\nu_{u,v}+c^\nu_u+c^\nu_v-s^\nu$, then 
\begin{align}\label{eq:Ising-dtv-exp-ineq}
    \DTV{\mu^\sigma}{\nu^\sigma} &\geq |\mu^\sigma(u=+,v=+)-\nu^\sigma(u=+,v=+)|
    \nonumber \notag\\&= |\exp(p)-\exp(q)| =\int_{x=\min\{p,q\}}^{\max\{p,q\}} \exp(x)dx
    \nonumber \notag\\&\geq  \int_{x=\min\{p,q\}}^{\max\{p,q\}} \exp(p)dx =\exp(p)\cdot |p-q|
    \nonumber \notag\\{\text{(by marginal lower bound)}}\quad&\geq b^2\cdot |p-q|.
\end{align}

Based on this, we can derive the following lower bound on the total variation distance between $\mu^\sigma$ and $\nu^\sigma$:
\begin{align}
\label{eq:Ising-dtv-local-edge-1}
\DTV{\mu^\sigma}{\nu^\sigma} &\geq b^2 \cdot |(J^\mu_{u,v}+c^\mu_u+c^\mu_v-s^\mu)-(J^\nu_{u,v}+c^\nu_u+c^\nu_v-s^\nu)|.
\end{align}

Using the same method, consider the distribution differences in $\mu^\sigma$ and $\nu^\sigma$ for the remaining three cases $(u=+,v=-),(u=-,v=+)$ and $(u=-,v=-)$, we can also obtain the following lower bounds:
\begin{align}
\label{eq:Ising-dtv-local-edge-2}
\DTV{\mu^\sigma}{\nu^\sigma} &\geq b^2 \cdot |-(-J^\mu_{u,v}+c^\mu_u-c^\mu_v-s^\mu)+(-J^\nu_{u,v}+c^\nu_u-c^\nu_v-s^\nu)|,
\\
\label{eq:Ising-dtv-local-edge-3}
\DTV{\mu^\sigma}{\nu^\sigma} &\geq b^2 \cdot |-(-J^\mu_{u,v}-c^\mu_u+c^\mu_v-s^\mu)+(-J^\nu_{u,v}-c^\nu_u+c^\nu_v-s^\nu)|,
\\
\label{eq:Ising-dtv-local-edge-4}
\DTV{\mu^\sigma}{\nu^\sigma} &\geq b^2 \cdot |(J^\mu_{u,v}-c^\mu_u-c^\mu_v-s^\mu)-(J^\nu_{u,v}-c^\nu_u-c^\nu_v-s^\nu)|.
\end{align}

Note that the absolute value operation satisfies the triangle inequality, by combining inequalities \eqref{eq:Ising-dtv-local-edge-1}, \eqref{eq:Ising-dtv-local-edge-2}, \eqref{eq:Ising-dtv-local-edge-3} and \eqref{eq:Ising-dtv-local-edge-4}, we have
\begin{align*}
4\DTV{\mu^\sigma}{\nu^\sigma} \geq b^2\cdot 4|J^{\mu}_{u,v}-J^{\nu}_{u,v}|.
\end{align*}

Then for the case that $|J^\mu_{u,v}-J^\nu_{u,v}|=\dis(\mu,\nu)$, the lower bound of local total variation distance in \eqref{eq:Ising-dtv-lowerbound-1} is proven.
\end{proof}

\begin{proof}\ifthenelse{\boolean{conf}}{\textbf{of eq.\eqref{eq:Ising-dtv-lowerbound-2}}}{[Proof of eq.\eqref{eq:Ising-dtv-lowerbound-2}]}
For the case that $|h_v^\mu-h_v^\nu|=\dis(\mu,\nu) \cdot (\deg_v+1)$. Define $c^\mu=h^\mu_v+\sum_{\{u,v\}\in E}J^\mu_{u,v}\sigma_v$ and $c^\nu=h^\nu_v+\sum_{\{u,v\}\in E}J^\nu_{u,v}\sigma_v$ be the influence
coefficient of the external field $\sigma$ on $v$ in the distribution $\mu$ and $\nu$ respectively, note that for each $\{u,v\}\in E$, $|J^\mu_{u,v}-J^\nu_{u,v}|$ is bounded by $\dis(\mu,\nu)$, then 
\begin{align*}
|c^\mu-c^\nu| &\geq |h^\mu_v-h^\nu_v|-\sum_{\{u,v\}\in E}|J^\mu_{u,v}-J^\nu_{u,v}| \\&\geq (\deg_v+1)\dis(u,v)- \sum_{\{u,v\}\in E}\dis(\mu,\nu) = \dis(\mu,\nu).
\end{align*}


According to the definition of the Ising model, when the external field $\sigma$ is fixed, the local distribution on $v$ in the distribution $\mu$ is follows:
\begin{align*}
    \mu^\sigma(v=+) = \exp(c^\mu-s^\mu), \quad \mu^\sigma(v=-) = \exp(-c^\mu-s^\mu),
\end{align*}
where $s^\mu = \log (\exp(c^\mu)+\exp(-c^\mu))$. 

Using the same method, we define $s^\nu$, and the local distribution of $u,v$ in the distribution $\nu$ can be expressed in terms of $c^\nu$ and $s^\nu$, as in the above equation. 

Recall that $\mu,\nu$ are both $b$-marginal bounded, applying the same method in \eqref{eq:Ising-dtv-exp-ineq}, we can derive the following lowerbound on the total variation distance between $\mu^\sigma$ and $\nu^\sigma$:
\begin{align}
\label{eq:Ising-dtv-local-vertex-1}
\DTV{\mu^\sigma}{\nu^\sigma} &\geq b \cdot |(c^\mu-s^\mu)-(c^\nu-s^\nu)|,
\\
\label{eq:Ising-dtv-local-vertex-2}
\DTV{\mu^\sigma}{\nu^\sigma} &\geq b \cdot |-(-c^\mu-s^\mu)+(-c^\nu-s^\nu)|.
\end{align}

By combining the above inequalities, we have 
\begin{align*}
2\DTV{\mu^\sigma}{\nu^\sigma} \geq b \cdot 2|c^\mu-c^\nu| \geq 2b\cdot \dis(\mu^\sigma,\nu^\sigma).
\end{align*}
Note $0<b<1$, then the lower bound of local total variation distance in \eqref{eq:Ising-dtv-lowerbound-2} is proven.
\end{proof}

\section{Additive-error approximation algorithm}\label{sec:add}

\subsection{TV-distance between two Gibbs distributions}

We first present an algorithm can achieves the additive-error approximation to the total variation distance between two  \emph{general} Gibbs distributions, which covers Ising and hardcore models as special cases.
%Let $[q]=\{0,1,\ldots,q-1\}$ be a finite domain.
%
Let $\mu$ over $\{0,1\}^V$ be a Gibbs distribution over graph $G=(V,E)$. For any configuration $\sigma \in \{0,1\}^V$, $\mu(\sigma) = w_{\mu}(\sigma)/Z_\mu$, where $w_{\mu}(\cdot)$ is the weight function and $Z_\mu = \sum_{\tau \in \{0,1\}^V}w_\mu(\sigma)$.
%
Let $\TW_{G} \in \mathbb{N}$.
We say the Gibbs distribution $\mu$ admits a weight oracle with cost $\TW_{G}$ is given any $\sigma \in [q]^V$, it returns the exact weight $w_\mu(\sigma)$ in time $\TW_{G}$. 
%
Note that both Ising and hardcore models, as well as most Gibbs distributions, admit weight oracle with cost $\TW_{G} = O(|V|+|E|)$.
%
Recall the sampling and approximate counting oracles are defined in~\Cref{def:oracle}.

\begin{theorem}\label{thm:Approximate-Gibbs}
There exists an algorithm such that given two general Gibbs distributions $\mu$ and $\nu$ on the same graph $G=(V,E)$ and an error bound $\epsilon > 0$, if $\mu$ and $\nu$ both admit weight, sampling, and approximate counting oracles with cost $\TW_{G}$ and cost functions $\TS_G(\cdot)$ and $\TC_G(\cdot)$ respectively, then it returns a random $\hat{d}$ in time $O(\TC_G(\frac{\epsilon}{4})+ \frac{1}{\epsilon^2}(\TW_{G} + \TS_G(\frac{\epsilon}{4})))$ such that
\begin{align*}
    \Pr[]{\DTV{\mu}{\nu} - \epsilon \leq \hat{d} \leq \DTV{\mu}{\nu} + \epsilon} \geq \frac{2}{3}.
\end{align*}
\end{theorem}

%\todo{Compare with main results, no marginal lower bound, compare with previous results, weak assumption}

\begin{proof}
Define a random variable $X \in [0,1]$ such that $X = \max\left(0,1-\frac{\nu(\sigma)}{\mu(\sigma)}\right)$, where $\sigma \sim \mu$.
\begin{align*}
 \DTV{\mu}{\nu} &= \sum_{\sigma:\mu(\sigma)>\nu(\sigma)}|\mu(\sigma)-\nu(\sigma)| =   \sum_{\sigma:\mu(\sigma)>\nu(\sigma)}\mu(\sigma)\left|1-\frac{\nu(\sigma)}{\mu(\sigma)}\right|\\
 &=\sum_{\sigma}\mu(\sigma)\max\left(0,1-\frac{\nu(\sigma)}{\mu(\sigma)}\right) = \E[]{X}.
\end{align*}
By the definition of $X$, we have $0\leq X \leq 1$, so $\Var[]{X}\leq 1$.
Ideally, we want to draw independent samples of $X$ and take average to approximate $ \DTV{\mu}{\nu}$. However, the main issue is that given a $\sigma \sim \mu$, we cannot compute neither $\mu(\sigma)$ nor $\nu(\sigma)$ exactly. Alternatively, we will define another random variable $\hat{X} \in [0,1]$ that approximate the random variable $X$. 

%Let $Z_\mu$ and $Z_\nu$ denote the partition function of $\mu$ and $\nu$.
Call the approximate counting oracles of $\mu$ and $\nu$ to obtain $\hat{Z_\mu}$ and $\hat{Z_\nu}$ that approximate partition functions $Z_\mu$ and $Z_\nu$ with relative error bound $\frac{\epsilon}{4}$. 
We may assume both counting oracles succeed, which happens with probability at least 0.98.
Define the random variable $\hat{X} \in [0,1]$ by the following process.
\begin{enumerate}
    \item Call sampling oracle of $\mu$ to obtain one sample $\sigma \in \{\pm\}^V$ such that $\DTV{\sigma}{\mu}\leq \frac{\epsilon}{4}$.\label{step-I}
    \item Call weight oracles of both $\mu$ and $\nu$ to obtain exact weights $w_\mu(\sigma), w_\nu(\sigma)$. Compute $\hat{\mu}(\sigma)={w_\mu(\sigma)}/{\hat{Z_\mu}}$ and $\hat{\nu}(\sigma)={w_\nu(\sigma)}/{\hat{Z_\nu}}$.
    \item Define $\hat{X}=\max(0,1-{\hat{\nu}(\sigma)}/{\hat{\mu}(\sigma)})$, in particular, $\hat{X} = 0$ if $\hat\mu(\sigma) = 0$.
\end{enumerate}
Let $T = \frac{64}{\epsilon^2}$.
Our algorithm is the following simple process:
\begin{itemize}
    \item Draw $T$ independent samples $\hat{X}_1,\hat{X}_2,\ldots,\hat{X}_T$ of random variable $\hat{X}$.
    \item Output the average $\hat d = \frac{1}{T}\sum_{i=1}^T \hat{X}_i$.
\end{itemize}
It is easy to see the running time of our algorithm is $2\TC_G(\frac{\epsilon}{4})+T\cdot (\TS_G(\frac{\epsilon}{4}) + 2\TW_G +O(1))$. 


%Define $X(\mu):=\max(0,1-\frac{\nu(\sigma)}{\mu(\sigma)})$, we have:
%\begin{align*}
%    \DTV{\mu}{\nu} &= \sum_{\sigma:\mu(\sigma)>\nu(\sigma)}|\mu(\sigma)-\nu(\sigma)|\\
%    &= \sum_{\sigma:\mu(\sigma)>\nu(\sigma)}\mu(\sigma)|1-\frac{\nu(\sigma)}{\mu(\sigma)}|\\
    %&=\sum_{\sigma:\mu(\sigma)>0}\mu(\sigma)\max(0,1-\frac{\nu(\sigma)}{\mu(\sigma)})\\
    %&=\mathbb{E}_{\sigma \sim \mu}X(\mu).
%\end{align*}
%\begin{algorithm}[h]\label{alg:approx-tv1}
%    \caption{Approximate TV distance for Gibbs distributions.}
%    Let $S = 0$.

%    Call the approximate counting oracles of $\mu$ and $\nu$ to obtain approximation $\hat{Z_\mu}$ and $\hat{Z_\nu}$ with error bound $\frac{\epsilon}{4}$.
    
%    \For{$i=1,2,\cdots,T$}{
%    Call sampling oracle of $\mu$ to obtain one sample $\sigma$ of $\mu$ with error bound $\frac{\epsilon}{4}$.

%    Call weight oracles of both $\mu$ and $\nu$ to obtain exact weights $w_\mu(\sigma), w_\nu(\sigma)$. Compute $\hat{\mu}(\sigma)=\frac{w_\mu(\sigma)}{\hat{Z_\mu}}$ and $\hat{\nu}(\sigma)=\frac{w_\nu(\sigma)}{\hat{Z_\nu}}$.

%    Compute $S_i=\hat{X}(\sigma)=\max(0,1-\frac{\hat{\nu}(\sigma)}{\hat{\mu}(\sigma)})$, $S = S_i + S$.
    
%    }

%    Return $\hat{d}=\frac{S}{T}$.
%\end{algorithm}

To prove the correctness of our algorithm, we only need to show that
\begin{align}\label{eq:ex-error}
    \left\vert \E[]{\hat{X}} - \E[]{X} \right\vert = \left\vert \E[]{\hat{X}} - \DTV{\mu}{\nu} \right\vert \leq \frac{7\epsilon}{8}.
\end{align}
Note that $0 \leq \hat{X} \leq 1$ so that $\Var{\hat X} \leq 1$. By Hoeffding’s inequality, it is easy to show that with probability at least 0.9, $|\hat d - \mathbb{E}[\hat{X}] | \leq \epsilon/8$. Combining with~\eqref{eq:ex-error} proves the theorem. 

Now, we only need to verify~\eqref{eq:ex-error}.
We introduce a new random variable $X^*$ in analysis.
In the definition of $\hat{X}$,
assume we replace the sample $\sigma$ in \Cref{step-I} with a  perfect sampler of the distribution $\mu$. Let $X^*$ denote the resulting random variable. We first compare $\E[]{X^*}$ with $\E[]{X}$. The difference between $X^*$ and $X$ comes from the error of computing the ratio of $\mu(\sigma)$ and $\nu(\sigma)$.
Note that $\frac{\nu(\sigma)}{\mu(\sigma)} = \frac{w_\nu(\sigma)}{w_\mu(\sigma)}\cdot \frac{Z_\mu}{Z_\nu}$ and $\frac{\hat\nu(\sigma)}{\hat\mu(\sigma)} = \frac{w_\nu(\sigma)}{w_\mu(\sigma)}\cdot \frac{\hat Z_\mu}{\hat Z_\nu}$. By the definition of approximate counting oracle, for $\sigma$ with $\mu(\sigma) >0$,
\begin{align*}
 \left(1-\frac{5\epsilon}{8}\right)\frac{\nu(\sigma)}{\mu(\sigma)}   \leq  \frac{\hat{\nu}(\sigma)}{\hat{\mu}(\sigma)} \leq \left(1+\frac{5\epsilon}{8}\right)\frac{\nu(\sigma)}{\mu(\sigma)}.
\end{align*}
We can compute the expectation as 
\begin{align*}
    \E[]{X^*}&= \sum_{\sigma:\mu(\sigma)>0}\mu(\sigma)\max\left(0,1-\frac{\hat{\nu}(\sigma)}{\hat{\mu}(\sigma)}\right)  \leq\sum_{\sigma:\mu(\sigma)>0}\mu(\sigma)\max\left(0,1-\left(1-\frac{5\epsilon}{8}\right)\frac{\nu(\sigma)}{\mu(\sigma)}\right)\\
&\leq\sum_{\sigma:\mu(\sigma)>0}\mu(\sigma)\left(\max\left(0,1-\frac{\nu(\sigma)}{\mu(\sigma)}\right)+\frac{5\epsilon}{8}\frac{\nu(\sigma)}{\mu(\sigma)}\right)
\leq \DTV{\mu}{\nu}+\frac{5\epsilon}{8}\sum_{\sigma:\mu(\sigma)>0}\nu(\sigma)\\
    &\leq \DTV{\mu}{\nu}+\frac{5\epsilon}{8}.
\end{align*}
Using the same way, we could verify the other direction. We have 
\begin{align}\label{eq:ex-1}
|\E[]{X^*}-\E[]{X}| = |\mathbf{E}[X^*]-\DTV{\mu}{\nu}|\leq \frac{5\epsilon}{8}.    
\end{align}

Next, we compare $X^*$ to $\hat X$. The only difference is that they sample $\sigma$ from different distributions. Let $\mu'$ be the distribution defined by the approximate sampling oracle.
Define a function $f$ such that for any $\sigma \in \{\pm\}^V$, $f(\sigma) = \max(0,1-{\hat{\nu}(\sigma)}/{\hat{\mu}(\sigma)})$, where we set $f(\sigma) = 0$ if $\hat \mu (\sigma) = 0$.
We have $|\mathbf{E}(\hat X)-\mathbf{E}(X^*)|=\left | \mathbf{E}_{\sigma\sim \mu}[f(\sigma)]-\mathbf{E}_{\sigma\sim \mu'}[f(\sigma)] \right |$. 
Define $A = \{\sigma\mid \mu(\sigma)>\mu'(\sigma)\}$ and $B = \{\sigma \mid \mu(\sigma)<\mu'(\sigma)\}$.
We can write
\begin{align}\label{eq:ex2}
    |\mathbf{E}(\hat X)-\mathbf{E}(X^*)| &=
     \left | \sum_{\sigma \in A} (\mu'(\sigma)-\mu(\sigma))f(\sigma) + \sum_{\sigma \in B} (\mu'(\sigma)-\mu(\sigma))f(\sigma) \right|\notag\\
\text{($\star$)}\quad    &\leq \max \tp{\sum_{\sigma \in A} (\mu(\sigma)-\mu'(\sigma))f(\sigma),\sum_{\sigma \in B} (\mu'(\sigma)-\mu(\sigma))f(\sigma)}\notag\\
    %=\left | \sum_{\mu(\sigma)>0,\mu(\sigma)>\mu'(\sigma)} (\mu'(\sigma)-\mu(\sigma))\max(0,1-\frac{\hat{\nu}(\sigma)}{\hat{\mu(\sigma)}})+\sum_{\mu(\sigma)>0,\mu(\sigma)\leq\mu'(\sigma)} (\mu'(\sigma)-\mu(\sigma))\max(0,1-\frac{\hat{\nu}(\sigma)}{\hat{\mu(\sigma)}})   \right|\\
\text{(by $f(\sigma) \in [0,1]$)}\quad    &\leq \DTV{\mu}{\mu'}\leq \frac{\epsilon}{4},
\end{align}
where in inequality ($\star$), we use $\sum_{\sigma \in A} (\mu'(\sigma)-\mu(\sigma))f(\sigma) \leq 0$ and $\sum_{\sigma \in B} (\mu'(\sigma)-\mu(\sigma))f(\sigma) \geq 0$.
Finally,~\eqref{eq:ex-error} holds due to~\eqref{eq:ex-1} and \eqref{eq:ex2}.
\end{proof}

\Cref{thm:Approximate-Gibbs} implies the following corollary for concrete models.
\begin{corollary}\label{corollary:apps}
There exist an FPRAS for approximating TV-distances with additive error for following models: (1) Hardcore model satisfying uniqueness condition; (2) Ising model with spectral condition; (3) Ferromagnetic interaction with consistent field condition; and (4) Anti-ferromagnetic interaction at or within the uniqueness threshold.
\end{corollary}
%\begin{itemize}
%    \item Hardcore model satisfying uniqueness condition in~\eqref{eq:cond-hardcore};
%    \item Ising model with spectral condition   $\lambda_{\max}(J) - \lambda_{\min}(J) \leq 1 - \eta$ for some constant $\eta > 0$;
%    \item Ferromagnetic interaction with consistent field condition: $J_{uv} \geq 0$ for all edge $\{u,v\} \in E$ and $h_v \geq 0$ for all $v \in V$.
%    \item Anti-ferromagnetic interaction at or within the uniqueness threshold: $J_{u,v}=\beta\leq 0$ for all $\{u,v\} \in E$ and $\exp(2\beta) \geq \frac{\Delta-2}{\Delta} $.
%\end{itemize}
%\end{corollary}
The definitions of the conditions can be found in~\eqref{eq:cond-hardcore} and \Cref{cond:Ising}. 
In contrast to the relative-error approximation, the above corollary does not require a marginal lower bound for the Ising model.
Furthermore, since the proof of \Cref{thm:Approximate-Gibbs} does not use \Cref{lem:TV-lower}, \Cref{corollary:apps} holds for general (not necessarily soft) Ising models.
%Therefore, the $O(1)$ bounds for $\Vert h\Vert_\infty$, $\Vert J \Vert_{\max}$, and $\Delta$ in \Cref{cond:1} and \Cref{cond:2} have been removed. Note that without the $O(1)$ bounds for $\Vert h\Vert_\infty$, $\Vert J \Vert_{\max}$, and $\Delta$, Dobrushin's condition in \Cref{cond:2} implies the spectral condition in \Cref{cond:1}.









%First, we estimate the error of computing the ratio of $\mu(\sigma)$ and $\nu(\sigma)$.
%Note that $\frac{\nu(\sigma)}{\mu(\sigma)} = \frac{w_\nu(\sigma)}{w_\mu(\sigma)}\cdot \frac{Z_\mu}{Z_\nu}$ and $\frac{\hat\nu(\sigma)}{\hat\mu(\sigma)} = \frac{w_\nu(\sigma)}{w_\mu(\sigma)}\cdot \frac{\hat Z_\mu}{\hat Z_\nu}$. By the definition of approximate counting oracle, for $\sigma$ with $\mu(\sigma) >0$,
%\begin{align*}
% \left(1-\frac{5\epsilon}{8}\right)\frac{\nu(\sigma)}{\mu(\sigma)}   \leq  \frac{\hat{\nu}(\sigma)}{\hat{\mu}(\sigma)} \leq \left(1+\frac{5\epsilon}{8}\right)\frac{\nu(\sigma)}{\mu(\sigma)}.
%\end{align*}
%We do the following assumption in our analysis.



%If we use the perfect sampler to sample from $\mu$ rather than using sampling oracle in the algorithm, we could get $d^*=\frac{\sum_{i=1}^{T}S_i^*}{T}$. For each $S_i^*$, the expectation of $S_i^*$

%\begin{align*}
%    \frac{\hat{\nu}(\sigma)}{\hat{\mu}(\sigma)}&= \frac{w_\nu(\sigma)}{w_\mu(\sigma)}\frac{\hat{Z}_\mu}{\hat{Z}_\nu}\\
%    &\leq \frac{w_\nu(\sigma)}{w_\mu(\sigma)}\frac{Z_\mu(1+\frac{\epsilon}{4})}{Z_\nu(1-\frac{\epsilon}{4})}\\
%    &< (1+\frac{5\epsilon}{8})\frac{\nu(\sigma)}{\mu(\sigma)}.
%\end{align*}

%Similarly, we can get $\frac{\hat{\nu}(\sigma)}{\hat{\mu}(\sigma)}> (1-\frac{5\epsilon}{8})\frac{\nu(\sigma)}{\mu(\sigma)}$.

%Assume that there is a perfect sampler of the distribution $\mu$. If we use the perfect sampler to sample from $\mu$ rather than using sampling oracle in the algorithm, we could get $d^*=\frac{\sum_{i=1}^{T}S_i^*}{T}$. For each $S_i^*$, the expectation of $S_i^*$
%\begin{align*}
%    \mathbb{E}[S_i^*]&=\mathbb{E}_{\sigma \sim \mu}\hat{X}(\mu)\\
    %&=\sum_{\sigma:\mu(\sigma)>0}\mu(\sigma)\max(0,1-\frac{\hat{\nu}(\sigma)}{\hat{\mu}(\sigma)})\\
    %&\leq\sum_{\sigma:\mu(\sigma)>0}\mu(\sigma)\max(0,1-(1-\frac{5\epsilon}{8})\frac{\nu(\sigma)}{\mu(\sigma)})\\
    %&\leq\sum_{\sigma:\mu(\sigma)>0}\mu(\sigma)[\max(0,1-\frac{\nu(\sigma)}{\mu(\sigma)})+\frac{5\epsilon}{8}\frac{\nu(\sigma)}{\mu(\sigma)}]\\
    %&\leq \DTV{\mu}{\nu}+\frac{5\epsilon}{8}\sum_{\sigma:\mu(\sigma)>0}\nu(\sigma)\\
   % &\leq \DTV{\mu}{\nu}+\frac{5\epsilon}{8}.
%\end{align*}

%Using the same way, we could verify that: $|\mathbb{E}[S_i^*]-\DTV{\mu}{\nu}|\leq \frac{5\epsilon}{8}$. Next, consider the difference between perfect sampler and sampling oracle. Let the distribution of the sampling by the oracle is $\mu'$, then we have $\DTV{\mu}{\mu'}\leq \frac{\epsilon}{4}$.

%\begin{align*}
%    |\mathbb{E}(S_i)-\mathbb{E}(S_i^*)|&=|\sum_{\sigma:\mu(\sigma)>0}(\mu'(\sigma)-\mu(\sigma))\max(0,1-\frac{\hat{\nu}(\sigma)}{\hat{\mu}(\sigma)})|\\
 %   &\leq |\sum_{\sigma:\mu'(\sigma)>\mu(\sigma)}\mu'(\sigma)-\mu(\sigma)|\\
 %   &=\DTV{\mu}{\mu'}\leq \frac{\epsilon}{4}.
%\end{align*}

%Combining above result, we have $\mathbb{E}(d)=\mathbb{E}(S_i)$ satisfies: $|\mathbb{E}(S_i)-\DTV{\mu}{\nu}|\leq \frac{7\epsilon}{8}$. $0\leq S_i\leq 1$ for all i, so by Hoeffding's inequality, 
%\begin{align*}
%    \mathbb{P}(\hat{d}-\mathbb{E}[\hat{d}]>\frac{\epsilon}{8})&\leq \exp(-\frac{2T^2 (\frac{\epsilon}{8})^2}{\sum_{i=1}^{T}1^2})\\
 %   &=\exp(-\frac{T\epsilon^2}{32}).
%\end{align*}

%Let $T=\frac{64}{\epsilon^2}$ we have $\mathbb{P}(\hat{d}-\mathbb{E}[\hat{d}]>\frac{\epsilon}{8})\leq \frac{1}{6}$, then with another side $\mathbb{P}(\hat{d}-\mathbb{E}[\hat{d}]<-\frac{\epsilon}{8})\leq \frac{1}{6}$, we obtain that:
%\begin{align*}
%    \mathbb{P}(|\hat{d}-\mathbb{E}[\hat{d}]|<\frac{\epsilon}{8})\geq \frac{2}{3}.
%\end{align*}

%Combining with $|\mathbb{E}(S_i)-\DTV{\mu}{\nu}|\leq \frac{\epsilon}{8}$, our algorithm returns a random $\hat{d}$ in time $2\TC_G(\frac{\epsilon}{4})+ \left \lceil \frac{64}{\epsilon^2} \right \rceil (2\TW_{G} + \TS_G(\frac{\epsilon}{4}))$ such that $\Pr[]{\DTV{\mu}{\nu} - \epsilon \leq \hat{d} \leq \DTV{\mu}{\nu} + \epsilon} \geq \frac{2}{3}$.
   


\subsection{TV-distance between two marginal distributions}

In this subsection, we present an algorithm can achieves the additive-error approximation to the total variance distance between two marginal distributions. Let $\mu$ over $\{\pm\}^V$ be a Gibbs distribution over graph $G=(V,E)$, and $S$ is a subset of $V$. For any configuration $\sigma$, let $\sigma$ be a partial configuration over $\{\pm\}^S$. Recall that $\mu_S(\sigma)  \propto Z^{\sigma}$, where $Z^{\sigma}$ is the conditional partition function of $\sigma$ defined as
\begin{align*}
    Z^{\sigma}\defeq\sum_{\tau \in \{\pm\}^V: \tau_S = \sigma } w_\mu(\tau).
\end{align*}


\begin{definition}[approximate conditional counting oracle]\label{def:cond-count-oracle}
Let $\mathbb{S}$ be a spin system on graph G with Gibbs distribution $\mu$. 
%For any subset $S\subseteq V$, let 
Let $\TC_{G}:(0,1) \to \mathbb{N}$ be a function.
We say $\mathbb{S}$ admits a conditional counting oracle with cost function $\TC_{G}(\cdot)$ if given any $0<\epsilon<1$, and any partial configuration $\sigma \in \{\pm\}^S$ on a subset $S\subseteq V$, it returns a random number $\hat{Z}_\mu^{\sigma}$ in time $\TC_{G}(\epsilon)$ such that $Z_\mu^{\sigma}(1-\epsilon)\leq \hat{Z}_\mu^{\sigma}(\sigma) \leq Z_\mu^{\sigma}(1+\epsilon)$ with probability at least 0.99.
\end{definition}

The oracle above is stronger than the approximate counting oracle in \Cref{def:oracle}. The approximate counting oracle only answers the query for $S = \emptyset$, while the conditional counting oracle can answer the query for any subset $S\subseteq V$.


\begin{theorem}\label{thm:approx-margin-tv}
There exists an algorithm such that given two general Gibbs distributions $\mu$ and $\nu$ on the same graph $G=(V,E)$ with $n = |V|$, any subset $S \subseteq V$, and any $\epsilon > 0$, if $\mu$ and $\nu$ both admit sampling and conditional counting oracles with cost functions $\TS_G(\cdot)$ and $\TC_{G}(\cdot)$ respectively, then it returns a random number $\hat{d}$ in time $O(\frac{1}{\epsilon^2}\log\frac{1}{\epsilon})\cdot(\TC_{G}(\frac{\epsilon}{8})+\TS_{G}(\frac{\epsilon}{8}))$ such that \[\Pr{\big|\hat{d} - \DTV{\mu_S}{\nu_S}\big|\leq \epsilon} \geq \frac{2}{3}.\]
\end{theorem}

\begin{proof}
Define a random variable $Y\in [0,1]$ such that $Y=\max \left(0,1-\frac{\nu_S(\sigma)}{\mu_S(\sigma)}\right)$, where $\sigma \sim \mu_S$.

\begin{align*}
\DTV{\mu_S}{\mu_S}&=\sum_{\sigma \in \{\pm\}^S}\max\left( 0,\mu_S(\sigma)-\nu_S(\sigma) \right)\\
&=\sum_{\sigma \in \{\pm\}^S}\mu_S(\sigma)\max\left(0,1-\frac{\nu_S(\sigma)}{\mu_S(\sigma)} \right)=\E[]{Y}.
\end{align*}

The random variable $Y$ satisfies that $0\leq 1 \leq Y$ so $\Var{Y}\leq 1$. Similar to the proof of \Cref{thm:Approximate-Gibbs}, we want to draw independent sample of $Y$ and take average to approximate $\DTV{\mu_S}{\nu_S}$. However, here $\mu_S(\sigma) = Z^\sigma_\mu/Z_{\mu}$ and $\nu_S(\sigma) = Z^\sigma_\nu/Z_{\nu}$. An additional problem is that we cannot exactly compute the weight $Z^\sigma_\mu$ and $Z^\sigma_\nu$ for each partial configuration $\sigma\sim \mu_S$.

First note that we can boost the success probability of conditional counting oracle from $0.99$ to $1 - \delta$ by calling it independently for $O(\log \frac{1}{\delta})$ times and take the median. Let $\delta = \frac{\epsilon^2}{320}$.
Call conditional counting oracles with $S=V$ to obtain $\hat{Z_\mu}$ and $\hat{Z_\nu}$ that approximate partition functions $Z_\mu$ and $Z_\nu$ with relative error bound $\frac{\epsilon}{8}$. 
%Call the approximate counting oracles (The oracles could be seen as conditional counting oracles with $S=V$.) of $\mu$ and $\nu$ to obtain $\hat{Z_\mu}$ and $\hat{Z_\nu}$ that approximate partition functions $Z_\mu$ and $Z_\nu$ with relative error bound $\frac{\epsilon}{8}$. 
We may assume both counting oracles succeed, which happens with probability at least $1 - \delta$. Similarly, we first define the random variable $\hat{Y}$ by the following process:

\begin{enumerate}

\item Call sampling oracle of $\mu$ to obtain one sample $\sigma\in \{\pm\}^V$ such that $\DTV{\sigma}{\mu}\leq \frac{\epsilon}{8}$, and we use $\sigma_S \in \{\pm\}^S$ as our sample.

\item Call conditional counting oracles to obtain $\hat{Z}_\mu^{\sigma_S}$ and $\hat{Z}_\nu^{\sigma_S}$ with an error bound $\epsilon/8$ and success probability $1-\delta$. Compute $\hat{\mu}_S(\sigma_S)=\hat{Z}^{\sigma_S}_\mu/\hat{Z_\mu}$ and $\hat{\nu}_S(\sigma_S)=\hat{Z}^{\sigma_S}_\nu/\hat{Z_\nu}$.

\item Define $\hat{Y}=\max\left (0,1-\hat{\nu}_S(\sigma_S)/\hat{\mu}_S(\sigma_S) \right )$, and in particular, $\hat{Y}=0$ if $\hat{\mu}_S(\sigma)=0$.
\end{enumerate}
Let $T=\frac{64}{\epsilon^2}$, we present our algorithm by following process:

\begin{itemize}
\item Draw T samples $\hat{Y}_1,\hat{Y}_2,\dots,\hat{Y}_n$ of random variable $\hat{Y}$.

\item Output the average $\hat{d}=\frac{1}{T}\sum_{i=1}^{T}\hat{Y}_i$.
\end{itemize}

The running time of our algorithm is $(\TC_{G}(\frac{\epsilon}{8})+\TS_{G}(\frac{\epsilon}{8}))\cdot T \cdot O(\log \frac{1}{\delta})$.


We now analyze the approximation error. First note that the probability that all conditional counting oracles success is $(1-\frac{\epsilon^2}{320})^{2T+2}>0.98$ and $\DTV{\sigma}{\mu}\leq \frac{\epsilon}{8}$ implies $\DTV{\sigma_S}{\mu_S}\leq \frac{\epsilon}{8}$. 
Compared to the proof of \Cref{thm:Approximate-Gibbs}, the difference is that we can only compute $\hat{Z}_\mu^{\sigma_S}$ and $\hat{Z}_\nu^{\sigma_S}$ approximately. However, the error can still be bounded. 
We prove that: for $\sigma_S$ with $\mu_S(\sigma_S)>0$, $\left(1-\frac{5\epsilon}{8} \right)\frac{\nu_S(\sigma_S)}{\mu_S(\sigma_S)}\leq \frac{\hat{\nu}_S(\sigma_S)}{\hat{\mu}_S(\sigma_S)}\leq \left ( 1+\frac{5\epsilon}{8} \right ) \frac{\nu_S(\sigma_S)}{\mu_S(\sigma_S)}$. Since $\frac{\hat{\nu}_S(\sigma_S)}{\hat{\mu}_S(\sigma_S)}= \frac{\hat{Z}_\nu^{\sigma_S}}{\hat{Z}_\mu^{\sigma_S}}\frac{\hat{Z}_{\mu}}{\hat{Z}_{\nu}}$ and $\frac{\nu_S(\sigma_S)}{\mu_S(\sigma_S)}= \frac{Z_\nu^{\sigma_S}}{Z_\mu^{\sigma_S}}\frac{Z_{\mu}}{Z_{\nu}}$,
\begin{align*}
    \frac{\hat{\nu}_S(\sigma_S)}{\hat{\mu}_S(\sigma_S)}= \frac{\hat{Z}_\nu^{\sigma_S}}{\hat{Z}_\mu^{\sigma_S}}\frac{\hat{Z}_{\mu}}{\hat{Z}_{\nu}}
    \leq \frac{Z_\nu^{\sigma_S}(1+\frac{\epsilon}{8})}{Z_\mu^{\sigma_S}(1-\frac{\epsilon}{8})}\frac{Z_{\mu}(1+\frac{\epsilon}{8})}{Z_{\nu}(1-\frac{\epsilon}{8})}< \left(1+\frac{5\epsilon}{8}\right)\frac{\nu_S(\sigma_S)}{\mu_S(\sigma_S)}.
\end{align*}
The other side of the inequality can be proved similarly. The rest of the proof follows from the proof of \Cref{thm:Approximate-Gibbs}. 
\end{proof}


\Cref{thm:many-vertex-alg} is a simple corollary of \Cref{thm:approx-margin-tv}, which is proved in \cref{sec:marginthm}.

%\Cref{thm:many-vertex-alg} is a simple corollary of \Cref{thm:approx-margin-tv}.
%\begin{proof}[Proof of \Cref{thm:many-vertex-alg}]
%For hardcore model $(G,\lambda^\mu)$, $Z^\sigma_{\mu}$ where $\sigma \in \{\pm\}^S$ is the partition function of $(G[\Lambda],\lambda_\Lambda^\mu)$. The set $\Lambda$ is obtained from $V$ by removing all vertices in $S$ together with all neighbors $u$ of vertices $v \in S$ such that $\sigma_v = +1$. If $(G, \lambda^\mu)$ satisfies the uniqueness condition, then $(G[\Lambda], \lambda_\Lambda^\mu)$ also satisfies the uniqueness condition. Hence, by the previous results in~\cite{CFYZ22,CE22} and \cite{SVV09}, for both $\mu$ and $\nu$, the approximate conditional counting oracle with $\TC_G(\epsilon)=O(\frac{\Delta n^2}{\epsilon^2} \mathrm{polylog}\frac{n}{\epsilon})$ exists and the sampling oracle with $\TS_G(\epsilon)=O(\Delta n \mathrm{polylog}\frac{n}{\epsilon})$ exists. The theorem follows from \Cref{thm:approx-margin-tv}.
%\end{proof}


%\Cref{thm:many-vertex-alg} is a corollary of \Cref{thm:Approximate-Gibbs}, and we 







\section{The algorithm for instances with small parameter distance}\label{sec:alg-main}
Let $\mu$ and $\nu$ be two general Gibbs distributions (including hardcore and Ising models) on the same graph $G=(V,E)$ with $n = |V|$. Let $w_\mu(\sigma)$ and $w_\nu(\sigma)$ be the weights of $\mu$ and $\nu$ on configuration $\sigma \in \{\pm\}^V$. In this section, we focus on the case where the parameter distance $\dis(\mu,\nu)$ is small. We will first give a basic algorithm for instances satisfying \Cref{cond:meta}, and then verify the \Cref{cond:meta} for Ising models with small parameter distance. 
We next give a more advanced algorithm for hardcore models with small parameter distance.

\subsection{Basic algorithm}
Define random variable
\begin{align}\label{eq:defW}
    W \defeq \frac{w_\nu(\sigma)}{w_\mu(\sigma)}, \quad\text{where } \sigma \sim \mu.
\end{align}
%It is easy to verify that $\E[]{W} = \frac{Z_\nu}{Z_\mu}$, where $Z_\mu$ and $Z_\nu$ are the partition functions of $\mu$ and $\nu$. %The following lemma controls the variance of $W$. %Recall $\theta$ is the threshold parameter defined in~\eqref{eq:theta}, where $\theta = \frac{1}{100 n^2}$ for hardcore model and $\theta = \frac{1}{100 (n + m})$ for Ising model.
%We first give an algorithm for approximating the TV-distance between $\mu$ and $\nu$ under the following condition.
\begin{condition}\label{cond:meta}
Let $K,L\geq 1$ be two parameters.
Two Gibbs distributions $\mu$ and $\nu$ satisfy that
\begin{itemize}
    \item $\nu$ is absolutely continuous with respect to $\mu$: for all $\sigma \in \{\pm\}^V$, if $\mu(\sigma) = 0$, then $\nu(\sigma) = 0$; 
    \item $ \sqrt{\Var{W}} \leq K {\DTV{\mu}{\nu}}$;
    \item $\E[]{W} \geq \frac{1}{L}$. %{\color{blue} I Think we also need $\E[]{W}\geq \frac{1}{L}$.}
\end{itemize}
\end{condition}

\begin{theorem}\label{thm:alg-main}
    There exists an algorithm such that given two Gibbs distributions $\mu$ and $\nu$ on the same graph $G=(V,E)$, and any $0 < \epsilon <1$, if $\mu$ and $\nu$ satisfy \Cref{cond:meta} with $K$ and $L$, and both admit sampling and approximate counting oracles with cost functions $\TS_G(\cdot)$ and $\TC_G(\cdot)$, then it returns a random number $\hat{d}$ in time $O(\TC_G(\frac{\epsilon}{4}) + T \cdot \TS_G(\frac{1}{100T}))$, where $T = O(\frac{L^2K^2}{\epsilon^2})$, such that 
    \begin{align*}
        \Pr[]{ (1-\epsilon)\DTV{\mu}{\nu} \leq \hat{d} \leq (1+\epsilon)\DTV{\mu}{\nu} } \geq \frac{2}{3}.
    \end{align*}
    \end{theorem}

%Notice that 
%\begin{align*}
%\E[]{W}=\sum_{\sigma}\mu(\sigma)\frac{w_\nu(\sigma)}{w_\mu(\sigma)}=\sum_{\sigma}\frac{w_\mu(\sigma)}{Z_\mu}\frac{w_\nu(\sigma)}{w_\mu(\sigma)}=\frac{Z_\nu}{Z_\mu}.
%\end{align*}
\begin{proof}
    Since $\nu$ is absolutely continuous with respect to $\mu$ ($\nu \ll \mu$), we can compute that:
\begin{align}\label{eq:DTV}
\DTV{\mu}{\nu} &= \frac{1}{2}\sum_{\sigma\in \{\pm\}^V} \left \vert \mu(\sigma)-\nu(\sigma) \right\vert =\frac{1}{2}\sum_{\sigma\in \{\pm\}^V:\mu(\sigma)>0}\mu(\sigma)\left \vert 1-\frac{\nu(\sigma)}{\mu(\sigma)}\right \vert \notag \\
&=\frac{1}{2}\sum_{\sigma\in \{\pm\}^V:\mu(\sigma)>0}\mu(\sigma) \left \vert 1-\frac{w_\nu(\sigma)}{w_\mu(\sigma)}\frac{Z_\mu}{Z_\nu}\right \vert \notag =\frac{Z_\mu}{2Z_\nu}\sum_{\sigma\in \{\pm\}^V:\mu(\sigma)>0}\mu(\sigma)\left \vert \frac{Z_\nu}{Z_\mu}-\frac{w_\nu(\sigma)}{w_\mu(\sigma)} \right \vert \notag \\
& = \frac{Z_\mu}{2Z_\nu}\E[]{|\E[]{W}-W|},
\end{align}
where in the last step we use the fact that if $\nu \ll \mu$, then $\E[]{W}= \sum_{\sigma \in \{\pm\}^V:\mu(\sigma)>0} \frac{w_\mu(\sigma)}{Z_\mu} \frac{w_\nu(\sigma)}{w_\mu(\sigma)} = \frac{Z_\nu}{Z_\mu}$.
We next use sampling oracles to define a random variable $\hat{W}\in [0,+\infty)$, which serves as an approximation of the random variable $W$ in~\eqref{eq:defW}.
Define
\begin{align*}
   T \defeq \left \lceil \frac{10^4 L^2 K^2}{\epsilon^2} \right \rceil.
\end{align*}
\begin{enumerate}
\item Call the sampling oracle of $\mu$ to obtain a random $\sigma \in \{\pm\}^V$ such that $\DTV{\sigma}{\mu}\leq \frac{1}{100T}$. 

\item Compute $\hat{W}=\frac{w_\nu(\sigma)}{w_\mu(\sigma)}$. In particular, if $w_\mu(\sigma)=0$\footnote{This case can happen because our sampling oracle is approximate.}, then we set $\hat{W} = 0$.
\end{enumerate}
Given the random variable $\hat{W}$, our algorithm is given by the following processes:
\begin{tcolorbox}[colback=lightgray!20, colframe=lightgray!18, coltitle=black, title={\textbf{Basic algorithm for instances satisfying \Cref{cond:meta}}}]
    \begin{itemize}
        \item Call approximate counting oracles to obtain $\hat{Z}_\mu$ and $\hat{Z}_\nu$ with error $\frac{\epsilon}{4}$.
        \item Draw $T$ samples $\hat{W}_1,\dots, \hat{W}_T$ from $\hat{W}$ independently.
        \item Compute $\bar{W}=\frac{1}{T}\sum_{i=1}^T \hat{W}_i$.
        \item Compute $\bar{E}=\frac{1}{T}\sum_{i=1}^{T} |\hat{W}_i-\bar{W}|$.
        \item Return $\hat{d}=\frac{\hat{Z}_\mu}{2\hat{Z}_\nu}\bar{E}$.
        \end{itemize}
    \end{tcolorbox}
The total running time of the above algorithm is 
\begin{align*}
    2\TC_G\tp{\frac{\epsilon}{4}} + O\tp{T \cdot \TS_G\left(\frac{1}{100T}\right)}.
\end{align*}
We remark that if both sampling and approximate counting can be solved in polynomial time, i.e., for any $\delta \in (0,1)$, $\TC_G(\delta),\TS_G(\delta) = \mathrm{poly}(\frac{n}{\delta})$, where $n$ is the number of vertices in $G$, then the above running time is $\mathrm{poly}(\frac{nLK}{\epsilon})$.



We now prove the correctness of the algorithm.
First, due to the definition of approximate counting oracle, with probability at least $0.98$, we can bound the error from ${\hat{Z}_\mu}{\hat{Z}_\nu}$ as follows:
\begin{align}\label{eq:ErrorZ}
    \left(1-\frac{3\epsilon}{4}\right)\frac{Z_\mu}{Z_\nu}  \leq \frac{(1-\epsilon/4)Z_\mu}{(1+\epsilon/4)Z_\nu} \leq \frac{\hat{Z}_\mu}{\hat{Z}_\nu}\leq \frac{(1+\epsilon/4)Z_\mu}{(1-\epsilon/4)Z_\nu}<\left(1+\frac{3\epsilon}{4}\right)\frac{Z_\mu}{Z_\nu}.
\end{align}
%A similar argument gives a lower bound. We have
%\begin{align*}
%    \left(1-\frac{3\epsilon}{4}\right)\frac{Z_\mu}{Z_\nu}\leq \frac{\hat{Z}_\mu}{\hat{Z}_\nu} \leq \left(1+\frac{3\epsilon}{4}\right)\frac{Z_\mu}{Z_\nu}.
%\end{align*}
Suppose we can access a perfect sampler of $\mu$ and we draw perfect samples $W_1,\dots,W_T$ of $W$.
For each pair of $W_i$ and $\hat{W}_i$, there exists a coupling of $W_i,\hat{W}_i$ such that $\Pr[]{W_i \neq \hat{W}_i}\leq\frac{1}{100T}$. Then with probability at least $0.99$, $W_i = \hat{W}_i$ for all $1 \leq i \leq T$.
Consider an ideal algorithm that can use the perfect samples $W_1,\dots,W_T$.
Our real algorithm can be coupled successfully with the ideal algorithm with probability at least $0.99$. If we can show the ideal algorithm outputs correct result with probability at least $0.96$, then our real algorithm outputs correct result with probability at least $0.95 > 2/3$.

Now we assume all $W_i$ are perfect samples of $W$. We compute $\bar{W}=\frac{1}{T}\sum_{i=1}^{T}W_i$ and similarly $\bar{E}$ and $\hat{d}$. We only need to prove that $(1-\epsilon)d\leq \hat{d}\leq (1+\epsilon)d$ with probability at least 0.9, where $d=\DTV{\mu}{\nu}$. For every random variable $W_i$, by triangle inequality, we have
\begin{align*}
    \vert \E[]{W} - W_i \vert - \vert \bar{W} - \E[]{W} \vert \leq \vert \bar{W} - W_i \vert \leq \vert \E[]{W} - W_i \vert + \vert \bar{W} - \E[]{W} \vert.
\end{align*}
Note that $\bar{E} = \frac{1}{T}\sum_{i=1}^{T}|\bar{W}-W_i|$. We have
\begin{align}\label{eq:barE}
    \tp{\frac{1}{T}\sum_{i=1}^T\vert \E[]{W} - W_i \vert} - \vert \bar{W} - \E[]{W} \vert \leq \bar{E} \leq  \tp{\frac{1}{T}\sum_{i=1}^T\vert \E[]{W} - W_i \vert} + \vert \bar{W} - \E[]{W} \vert.
\end{align}
By definition of $\bar{W}$, we have $\E[]{\bar{W}} = \E[]{W}$ and $\Var[]{\bar{W}} = \frac{\Var[]{W}}{T}$. By Chebyshev's inequality,
%and we have $\E[]{\bar{W}} = \E[]{W}$ and $\Var[]{\bar{W}} = \frac{\Var[]{W}}{T}=\leq\frac{K^2 D^2}{T}$. We can compute 
\begin{align}\label{eq:barW}
    \Pr[]{|\bar{W}-\E[]{W}|\geq \frac{\epsilon d}{10L}}\leq \frac{100L^2\Var[]{\bar{W}}}{\epsilon^2 d^2}  = \frac{100L^2\Var[]{W}}{T\epsilon^2 d^2} \leq \frac{100L^2K^2d^2}{T \epsilon^2 d^2} \leq 0.01,
\end{align}
where the second inequality follows from \Cref{cond:meta}.
Next, consider the random variable
\begin{align*}
    R \defeq \vert \E[]{W} - W \vert.
\end{align*}
By the definition of $R$ and the variance bound in \Cref{cond:meta}, we know that 
\begin{align*}
    \Var[]{R} \leq \E[]{R^2} = \E[]{(\E[]{W}-W)^2} = \Var[]{W} \leq K^2 d^2.
\end{align*}
Note that $\frac{1}{T}\sum_{i=1}^T\vert \E[]{W} - W_i \vert$ is the average of $T$ i.i.d. random samples of $R$.
Denote $\bar{R} \defeq \frac{1}{T}\sum_{i=1}^T\vert \E[]{W} - W_i \vert$. Note that $\Var[]{\bar{R}} = \frac{\Var[]{R}}{T}$. By Chebyshev's inequality, we have
\begin{align}\label{eq:barR}
     \Pr[]{\left\vert \bar{R} - \E[]{R} \right\vert \geq \frac{\epsilon d}{10L}} \leq \frac{100L^2\Var[]{R}}{T \epsilon^2 d^2} \leq \frac{100L^2 K^2 d^2}{T \epsilon^2 d^2} \leq 0.01.
\end{align}
Combining~\eqref{eq:barE}, \eqref{eq:barW},~\eqref{eq:barR}, and a union bound, we have
\begin{align}\label{eq:barE-R}
    \Pr[]{\vert \bar{E} - \E[]{R} \vert \leq \frac{\epsilon d}{5L}} \geq 0.98.
\end{align}

Assume two good events in~\eqref{eq:barE-R} and~\eqref{eq:ErrorZ} both hold, which happens with probability at least $0.96$. The final output $\hat{d} = \frac{\hat{Z}_\mu}{2\hat{Z}_\nu}\bar{E}$ satisfies 
\begin{align*}
    \hat{d} &= \frac{\hat{Z}_\mu}{2\hat{Z}_\nu}\bar{E} \leq \frac{(1+3\epsilon/4)Z_\mu}{2Z_\nu} \cdot \tp{ \E[]{R} + \frac{\epsilon d}{5L} }\\
    &\leq \left(1+\frac{3\epsilon}{4} \right)\frac{Z_\mu}{2Z_\nu} \E[]{R} + \frac{Z_\mu}{Z_\nu L} \cdot \frac{\epsilon}{10} \cdot \left(1+\frac{3\epsilon}{4} \right) \cdot d.
\end{align*}
By~\eqref{eq:DTV}, we have $\frac{Z_\mu}{2Z_\nu} \E[]{R} = d$. By \Cref{cond:meta}, we have $\frac{Z_\mu}{Z_\nu L} = \frac{1}{L \E[]{W}}  \leq 1$. Therefore,
\begin{align*}
    \hat{d} \leq \left(1+\frac{3\epsilon}{4}\right)d + \frac{\epsilon}{10} \cdot \left(1+\frac{3\epsilon}{4} \right) \cdot d < (1+\epsilon)d.
\end{align*}
A similar argument gives a lower bound $\hat{d} \geq (1-\epsilon)d$.
\end{proof}

    %We first give the algorithm in \Cref{sec:alg-main-proof}. Then we apply the algorithm to hardcore model and Ising model separately. It is easy to verify \Cref{cond:meta} for Ising model. However, for hardcore model, we need to do some preprocessing to use the algorithm in \Cref{thm:alg-main}.





%\begin{lemma}\label{lem:W-variance}
%Let $\mu$ and $\nu$ be a pair of hardcore models or Ising models. If $\dis(\mu,\nu) \leq \theta$, then
%\begin{align*}
    %\sqrt{\Var{W}} \leq K {\dis(\mu,\nu)},
%\end{align*}
%where $K = 50n^2$ for hardcore model and $K = 4(n+m)$ for Ising model.
%\end{lemma}

%Given the variance bound of $W$, we can use the following algorithm to approximate the total variation distance between $\mu$ and $\nu$.

%\subsubsection{Applications to hardcore model without small external fields}\label{sec:hardcore-fast}
%Next, we verify \Cref{cond:meta} for the hardcore model in uniqueness regime without small external fields.
%We have the following lemma, which will be used to prove \Cref{thm:hardcore-2}.
%\begin{lemma}
%    Let $\mu$ and $\nu$ be two hardcore models on the same graph $G=(V,E)$. If $\mu$ and $\nu$ both satisfy the uniqueness condition in~\eqref{eq:cond-hardcore} and $\dis(\mu,\nu) < \frac{1}{10 n \Delta}$, and for any $v \in V$, $\lambda_v^\mu,\lambda_v^\nu = \Omega(1/\Delta)$, then $\mu$ and $\nu$ satisfy \Cref{cond:meta} with $K = O(\sqrt{n})$ and $L = 2$.
%\end{lemma}





\subsection{Advanced algorithm for hardcore model}\label{sec:var-main}

%Recall that $\theta$ is the threshold parameter for hardcore model in~\eqref{eq:theta}.
We give the following algorithm for hardcore models with small parameter distance.
\begin{theorem}\label{thm:hardcore-adv}
Let $\theta = 10^{-10}\frac{\epsilon^{1/4}}{n^{5/2}}$.
There exists an algorithm such that given two hardcore models $\mu$ and $\nu$ on the same graph $G=(V,E)$, and any $0 < \epsilon <1$, if $\mu$ and $\nu$ both satisfy uniqueness condition in~\eqref{eq:cond-hardcore} and $\dis(\mu,\nu) < \theta$, then it returns a random number $\hat{d}$ in time $\tilde{O}\left(\frac{n^7}{\epsilon^2}+\frac{n^{6.5}}{\epsilon^{9/4}}\right)$ such that $(1 - \epsilon)\DTV{\mu}{\nu} \leq \hat{d} \leq (1+\epsilon)\DTV{\mu}{\nu}$ with probability at least $2/3$.
\end{theorem}

Let $\lambda^\mu$ and $\lambda^\nu$ be external fields of two hardcore models $\mu$ and $\nu$, respectively. For simplicity of the notation, we denote 
\begin{align*}
    D &= \dis(\mu,\nu) = \Vert \lambda^\mu - \lambda^\nu \Vert_\infty < \theta = 10^{-10}\frac{\epsilon^{1/4}}{n^{5/2}},\\
    d &= \DTV{\mu}{\nu}.
\end{align*}  
%If $\lambda_v^\mu$ and $\lambda_v^\nu$ is small for all $v\in V$, the TV distance of $\mu$ and $\nu$ almost comes from $\frac{1}{2}(|\mu(\emptyset)-\nu(\emptyset)|+\sum_{v\in V}|\mu(\{v\}-\nu(\{v\}))|$, because the probability of the independent set with size larger than one is too small. Intuitively, we can use the term of independent sets of zero or one set size to approximate the TV distance. However, not all $\lambda_v$ is small. 
As discussed in the algorithm overview, we divide the vertices of $G$ into two parts: the "big" vertices and the "small" vertices. Define the threshold parameter
\begin{align*}
    \kappa \defeq 10^{-9}\frac{\epsilon^{1/4}}{n^{3/2}} = \Theta\left(\frac{\epsilon^{1/4}}{n^{3/2}}\right).
\end{align*}
Define two sets of vertices $B$ and $S$ in graph $G$:
\begin{align*}
    &B = \left\{v \in V \mid \min\{\lambda_v^\mu,\lambda_v^\nu\} \geq \kappa\right\},\\
    &S = V \setminus B = \left\{v \in V \mid \min\{\lambda_v^\mu,\lambda_v^\nu\} < \kappa\right\}.
\end{align*}
Recall $\mu_B$ and $\nu_B$ are the marginal distributions of $\mu$ and $\nu$ on $B$, respectively. Let $\Omega_B \subseteq \{\pm\}^B$ be the support of both $\mu_B$ and $\nu_B$. By the definition of $B$, for any $x\in \Omega_B$, all vertices $v \in B$ with $x_v = +1$ forms an independent set in $G$. For any $x\in \Omega_B$, let $\mu^x_S$ and $\nu^x_S$ be the marginal distributions of $\mu$ and $\nu$ on $S$ conditioned on $x$. The TV-distance between $\mu$ and $\nu$ can be represented by 
\begin{align}\label{eq:d-hardcore}
   d &= \frac{1}{2}\sum_{\sigma \in \{\pm\}^V} \left|\mu(\sigma)-{\nu(\sigma)}\right| = \frac{1}{2}\sum_{x \in \Omega_B}\sum_{y \in \{\pm\}^S} \left|\mu_B(x)\mu_S^x(y)-\nu_B(x)\nu_S^x(y)\right|\notag\\
    &= \frac{1}{2}\sum_{x \in \Omega_B}\mu_B(x) {\sum_{y \in \{\pm\}^S} \left|\frac{\nu_B(x)}{\mu_B(x)}\nu_S^x(y)-\mu_S^x(y)\right|}.
\end{align}
Define the function $f: \Omega_B \to \mathbb{R}$ as
\begin{align}\label{eq:def-f}
    f(x) \defeq \frac{1}{2}\sum_{y \in \{\pm\}^S} \left|\frac{\nu_B(x)}{\mu_B(x)}\nu_S^x(y)-\mu_S^x(y)\right|.
\end{align}
The calculation shows that $\DTV{\mu}{\nu} = \E[x \sim \mu_B]{f(x)}$. In a high-level view, our algorithm wants to draw i.i.d. samples $x\sim \mu_B$ and compute values $f(x)$ and then output the average value. Formally, we have the following lemmas.
Let $n$ denote the number of vertices in $G$.

\begin{lemma}\label{lem:hardcore-adv-var}
The variance $\Var[x \sim \mu_B]{f(x)} = O_\eta(d^2) \cdot (n^3 + n/\kappa)$, where $O_\eta$ holds a constant depending only on the gap $\eta$ in the uniqueness condition in~\eqref{eq:cond-hardcore}. %As a consequence, 
%\begin{align*}
% \sqrt{\Var[x \sim \mu_B]{f(x)}} \leq ? \cdot d.
%\end{align*}
\end{lemma}

\begin{lemma}\label{lem:hardcore-adv-1}
There exists a randomized data structure satisfies that 
\begin{itemize}
    \item the data structure can be constructed in time $\tilde{O}(\frac{n^7}{\epsilon^2}+\frac{n^{6.5}}{\epsilon^{9/4}})$ and the construction succeeds with probability at least 0.99;
    \item if the data structure is constructed successfully, then given any $x \in \Omega_B$, it deterministically answers an $\hat{f}(x) \geq 0$ in time $O(n^4)$ such that  
    \[\vert \hat{f}(x)-f(x)\vert \leq \frac{\epsilon}{50} \cdot d.\]
\end{itemize}
\end{lemma}
%There exists a randomized algorithm such that given any $x \in \Omega_B$ and any $\delta > 0$, it computes a random value $\hat{f}(x) \geq 0$ in time $?$ such that with probability at least $1-\delta$, 
%\[\vert \hat{f}(x)-f(x)\vert \leq \frac{\epsilon}{100} \cdot d.\]


\Cref{lem:hardcore-adv-var} can be used to control the variance of $f(x)$. \Cref{lem:hardcore-adv-1} is the main technical part of our algorithm.
We first assume both \Cref{lem:hardcore-adv-var} and \Cref{lem:hardcore-adv-1} hold and prove \Cref{thm:hardcore-adv}.
The proofs of \Cref{lem:hardcore-adv-var} and \Cref{lem:hardcore-adv-1} are given in Section \ref{sec:hardcore-adv-var} and \ref{sec:hardcore-adv-1}, respectively.
\ifthenelse{\boolean{conf}}{\begin{proof}\textbf{of \Cref{thm:hardcore-adv}}}{\begin{proof}[Proof of \Cref{thm:hardcore-adv}]}
Let $T= O(\frac{n^3 + n/\kappa}{\epsilon^2}) = O(\frac{n^3}{\epsilon^2}+\frac{n^{5/2}}{\epsilon^{9/4}})$ be large enough. 
    \begin{tcolorbox}[colback=lightgray!20, colframe=lightgray!18, coltitle=black, title={\textbf{The algorithm for hardcore model}}]
        \begin{itemize}
            \item Construct the data structure in \Cref{lem:hardcore-adv-1};
            \item Draw $T$ independent approximate samples $x_1,x_2,\ldots,x_T$ from the marginal distribution $\mu_B$ with $\DTV{\mu_B}{x_i} \leq \frac{1}{100T}$.
            \item Use the data structure to compute $\hat{f}(x_1),\hat{f}(x_2),\ldots,\hat{f}(x_T)$.
            \item Return $\hat{d}=\frac{1}{T}\sum_{i=1}^T \hat{f}(x_i)$.
            \end{itemize}
        \end{tcolorbox}

Consider an ideal algorithm $\+A^*$ that draw perfect samples $x_1,\ldots,x_T$ and exactly compute the values of $f(x_1),\ldots,f(x_T)$. Let $d^*$ denote the output $\frac{1}{T}\sum_{i=1}^Tf(x_i)$. We have $\E[]{d^*} = d$ and the variance of $d^*$ is $\frac{\Var[\mu_B]{f}}{T}$. By our choice of $T$ and the variance bound in \Cref{lem:hardcore-adv-var}, using Chebyshev's inequality, we have
\begin{align*}
    \Pr[]{|d^*-d| \geq \frac{\epsilon}{10}d} \leq 0.01.
\end{align*}

Consider our real algorithm $\+A$.
All the approximate samples $x_1,\ldots,x_T$ in $\+A$ can be coupled successfully with perfect samples with probability at least $1 - T \cdot \frac{1}{100T} = 0.99$.
Also that the data structure in $\+A$ is constructed successfully, which happen with  probability at least $0.99$.
By \Cref{lem:hardcore-adv-1}, there exists a coupling of $\+A$ and $\+A^*$ such that with probability at least 0.98, $|\hat{d}-d^*| \leq \frac{\epsilon}{50}d$. Hence, using a union bound, we have
\begin{align*}
    \Pr[]{(1-\epsilon)d \leq \hat{d} \leq (1+\epsilon)d } \geq 0.97 > \frac{2}{3}.
\end{align*}

%We may also assume that the samples $x_1,x_2,\ldots,x_T$ are perfect samples from $\mu_B$, which can be coupled with the real algorithm successfully with probability at least $1 - T \cdot \frac{1}{100T} = 0.99$.
%In the analysis, we consider the idealized version of $\+A$, which is denoted by $\+A^*$, such that given any $x \in \Omega_B$ and any $\delta > 0$, $\+A^*(x,\delta)$ returns the random variable $\+A(x,\delta)$ conditional on $\vert \hat{f}(x)-f(x)\vert \leq \frac{\epsilon}{100} \cdot d$.
%Consider the following modified algorithm for approximating the TV-distance.
%\begin{itemize}
%    \item In the first step, the algorithm draws perfect independent samples $x_1,x_2,\ldots,x_T$ from the marginal distribution $\mu_B$.
%    \item In the second step, the algorithm uses the algorithm $\+A^*$ to compute the value of all $\hat{f}({x_i})$.
%\end{itemize}
%We only use the modified algorithm in the analysis, we do not need to implement it.
 %We first show that this modified algorithm approximate the TV-distance with probability at least 0.99. We then show that our real algorithm (use the algorithm $\+A$ in second step) can be coupled successfully with the modified algorithm with probability at least 0.98. Finally, we show the real algorithm can approximate the TV-distance with probability at least 0.97  $> 2/3$.
%Fix $\delta = \frac{1}{100 T}$.
%For any $x \in \Omega_B$, by \Cref{lem:hardcore-adv-1}, we know $\+D(x) \geq 0$ and 
%\begin{align*}
%   \+D(x) \leq f(x) + \frac{\epsilon}{100} \cdot d \leq \left(\frac{30000n}{\kappa}+ \frac{\epsilon}{100} \right)d < \frac{10^4 n d}{\kappa}.
%\end{align*}
%As a consequence, it holds that $\sqrt{\Var[x \sim \mu_B]{\+D(x)}} < \frac{10^4 n d}{\kappa}.$
%Let $d^*$ denote the output of the modified algorithm.
%By choosing $T = O(\frac{n^2}{\kappa^2 \epsilon^2})$ large enough and using Chebyshev's inequality, we can show that 
%\begin{align*}
%   \Pr[]{\left\vert \hat{d} - \E[x \sim \mu_B]{\+D(x)} \right\vert \geq \frac{\epsilon}{100} \cdot d} \leq \frac{10^{12} n^2}{\epsilon^2 \kappa^2 T} < 0.01.
%\end{align*}
%For any $x \in \Omega_B$, $|\+D(x) - f(x)| \leq \frac{\epsilon}{100} \cdot d$. We can bound that 
%\begin{align*}
%\left\vert  \E[x \sim \mu_B]{\+D(x)} - \E[x \sim \mu_B]{f(x)} \right\vert = \left\vert \E[x \sim \mu_B]{\+D(x)} - d \right\vert \leq \frac{\epsilon}{100} \cdot d,
%\end{align*}
%where $\E[x \sim \mu_B]{f(x)} = d$ holds by~\eqref{eq:d-hardcore}. By a union bound, with probability at least $0.97$, the output $\hat{d}$ of the modified algorithm satisfies
%\begin{align*}
%    \left\vert \hat{d} - d \right\vert \leq \frac{\epsilon}{50} \cdot d \quad \implies \quad (1 - \epsilon)d \leq \hat{d} \leq (1+\epsilon) d.
%\end{align*}

%To show the output $\hat{d}$ of our real algorithm is correct, we can couple the modified algorithm and our real algorithm. 
%In the first step, we optimally couple the samples $x_1,x_2,\ldots,x_T$. Since $\DTV{\mu_B}{x_i} \leq \frac{1}{100T}$, with probability at least $0.99$, the samples $x_1,x_2,\ldots,x_T$ are the same in two algorithms.
%In the second step, we optimally couple the output $\+A(x_i,\delta)$ and $\+A^*(x_i,\delta)$ for all $i=1,2,\ldots,T$. By \Cref{lem:hardcore-adv-1}, the total variation distance between $\+A(x_i,\delta)$ and $\+A^*(x_i,\delta)$ is at most $\delta = \frac{1}{100 T}$. Hence, with probability at least $0.98$, the outputs $\hat{d} = d^*$. Putting everything together, we have
%\begin{align*}
%    \Pr[]{(1-\epsilon)d \leq \hat{d} \leq (1+\epsilon)d} \geq 0.97 > \frac{2}{3}.
%\end{align*}

The total running time is bounded by
\begin{align*}
 \tilde{O}\left(\frac{n^7}{\epsilon^2}+\frac{n^{6.5}}{\epsilon^{9/4}}\right) + O\left(n^4 \cdot T \right)   =  \tilde{O}\left(\frac{n^7}{\epsilon^2}+\frac{n^{6.5}}{\epsilon^{9/4}}\right). &\ifthenelse{\boolean{conf}}{}{\qedhere}
\end{align*}
%the first step cost time $\frac{n^4}{\kappa^2 \epsilon}$. 
%in the first step, we can use the Glauber dynamics to sample from the distribution $\mu_B$. The running time for generating each sample $x_i$ is $O_\eta(\Delta n \log (nT))$, where $\eta$ is the gap parameter in the uniqueness condition in~\eqref{eq:cond-hardcore}. The total running of the first step is $O_\eta(T\Delta n \log (nT))$. 
\end{proof}

\subsubsection{Analyze the variance of the estimator (Proof of \texorpdfstring{\Cref{lem:hardcore-adv-var})}{}}
\label{sec:hardcore-adv-var}
Before we prove the lemma, we first remark that one can show for any $x \in \Omega_B$, $|f(x) - 1| \leq O(\frac{n}{\kappa}d)$, which implies the $O(\frac{n^2}{\kappa^2} d^2) $ variance bound. 
This bound gives a polynomial-time algorithm but the degree of polynomial is higher. 
If we use this bound, then in the proof of \Cref{thm:hardcore-adv}, it requires $O(\frac{n^2}{\kappa^2 \epsilon^2})$ samples $x \sim \mu_B$ and \Cref{lem:hardcore-adv-1} computes each value of $\hat{f}(x)$ in a super-linear time.
Alternatively, we give a more technical analysis to achieve a better dependency on $\kappa$, which gives a better running time of our algorithm.

We bound the variance of $f(x)$ by bounding the second moment of $\E[\mu_B]{f^2} = \E[x \sim \mu_B]{f^2(x)}$. We first need the following upper bound on the value of $f(x)$:
\begin{align*}
        f(x) &= \frac{1}{2}\sum_{y \in \{\pm\}^S} \left|\frac{\nu_B(x)}{\mu_B(x)}\nu_S^x(y)-\mu_S^x(y)\right|\notag\\
\text{(by triangle inequality)}\quad       &\leq \frac{1}{2}\left\vert \frac{\nu_B(x)}{\mu_B(x)} -1 \right\vert \sum_{y \in \{\pm\}^S}\nu_S^x(y) + \frac{1}{2}\sum_{y \in \{\pm\}^S}\left|\nu_S^x(y)-\mu_S^x(y)\right|\notag\\
        &= \frac{1}{2}\left\vert \frac{\nu_B(x)}{\mu_B(x)} -1 \right\vert  + \DTV{\nu_S^x}{\mu_S^x}.
\end{align*}
We have the following upper bound on the $\DTV{\nu_S^x}{\mu_S^x}$.
\begin{lemma}\label{claim:tv-bound}
    for any $x \in \Omega_B$, it holds that $\DTV{\nu_S^x}{\mu_S^x} \leq 4 nD$.
\end{lemma}
The proof of \Cref{claim:tv-bound} will be given later.
Assume \Cref{claim:tv-bound} holds.
Since $f(x) \geq 0$, we can upper bound
\begin{align}\label{eq:varf-bd}
\E[\mu_B]{f^2} &\leq \frac{1}{4}\E[x \sim \mu_B]{\left(\frac{\nu_B(x)}{\mu_B(x)} -1 \right)^2 } + 4nD \E[x \sim \mu_B]{\left\vert \frac{\nu_B(x)}{\mu_B(x)} -1 \right\vert} + 16n^2D^2\notag\\
&=  \frac{1}{4}\E[x \sim \mu_B]{\left(\frac{\nu_B(x)}{\mu_B(x)} -1 \right)^2 } + 8 nD\cdot \DTV{\nu_B}{\mu_B} + 16n^2D^2\notag\\
&\leq  \frac{1}{4}\E[x \sim \mu_B]{\left(\frac{\nu_B(x)}{\mu_B(x)} -1 \right)^2 } + 4\cdot 10^8n^2d^2,
\end{align}
where the last inequality follows from the fact $d = \DTV{\mu}{\nu} > \DTV{\mu_B}{\nu_B}$ and $d \geq \frac{1}{5000}D$.

Define a function $h(x) = \frac{\nu_B(x)}{\mu_B(x)}$. We have $\E[\mu_B]h =1 $. Our task is reduced to bound the variance $\Var[\mu_B]{h} = \Var[x \sim \mu_B]{h(x)}$. We will use the following Poincar\'e inequality (i.e. the spectral gap of the Glauber dynamics) for the marginal distribution $\mu_B$.
For any subset $\Lambda \subseteq V$, recall that $\Omega_\Lambda$ denotes the support of $\mu_\Lambda$. 
\begin{lemma}[\text{\cite{ChenFYZ21}}]\label{lem:poincare}
Since the hardcore model $(G,\lambda^\mu)$ is in the uniqueness regime in~\eqref{eq:cond-hardcore} with constant gap $\eta>0$. For any function $g: \Omega_B \to \mathbb{R}$, it holds that
\begin{align*}
    \Var[\mu_B]{g} \leq C_\eta \sum_{v \in B} \sum_{\sigma \in \Omega_{B - v}} \mu_{B - v}(\sigma) \Var[\mu^\sigma_B]{g},
\end{align*}
where $B-v$ denote the set $B \setminus \{v\}$ and $C_\eta$ is a constant depending on $\eta$.
\end{lemma}
The Poincar\'e inequality in~\cite{ChenFYZ21} is stated from the entire Gibbs distribution $\mu$. One can lift it to marginal distribution $\mu_B$. For the completeness, we give a proof in \Cref{app:poin}.


%Note that our function $h$ is from $\Omega_B \subseteq \{\pm\}^B$ to $\mathbb{R}$. We can define a function $g:\Omega_\mu \to \mathbb{R}$ by $g(z) = \frac{\nu(z)}{\mu(z)}$. Using the law of total variance, we have
%\begin{align*}
%    \Var[\mu]{g} &= \Var[x \sim \mu_B]{\E[y \sim \mu_S^x]{g(x+y)}} + \E[x \sim \mu_B]{\Var[y \sim \mu_S^x]{g(x+y)}}\\
%    &= \Var[\mu_B]{h} + \E[x \sim \mu_B]{\Var[y \sim \mu_S^x]{g(x+y)}}\\
%    &\geq \Var[\mu_B]{h},
%\end{align*}
%where $x +y$ denote a full configuration in $\{\pm\}^V$ obtained by concatenating $x$ and $y$.
Fix a vertex $v \in B$ and a $\sigma \in \Omega_{B - v}$. 
Let $\sigma^{v_+}$ denote a configuration in $\{\pm\}^B$ obtained by extending $\sigma$ further by setting $v$ to $+1$. Define $\sigma^{v_-}$ similarly.
By the definition of variance and the definition of the function $h=\frac{\nu_B}{\mu_B}$, we can write
\begin{align*}
    \Var[\mu^\sigma_B]{h} &= \mu^\sigma_{v}(+1)\mu^\sigma_{v}(-1)(h(\sigma^{v_+}) -h(\sigma^{v_-}))^2\\
    &= \tp{ \frac{\nu_{B-v}(\sigma)}{ \mu_{B-v}(\sigma) }}^2 \mu^\sigma_{v}(+1)\mu^\sigma_{v}(-1)\left( \frac{\nu_{v}^{\sigma}(+1)}{\mu_{v}^{\sigma}(+1)} -  \frac{\nu_{v}^{\sigma}(-1)}{\mu_{v}^{\sigma}(-1)} \right)^2. 
\end{align*}
Note that $\Var[\mu^\sigma_B]{h} = 0$ if either $\mu^\sigma_v(+1) = 0$.
%Recall that $N_v(G)$ is the set of neighbors of $v$ in $G$.
We assume that $\mu^\sigma_v(+1) > 0$, then for every neighbor $u$ of vertex $v$, it must hold that if $u \in B$, then $\sigma_u = -1$.
We claim the following bounds.
\begin{claim}\label{claim:p-bound}
The following bounds hold
\begin{itemize}
    \item $\lambda^\mu_v/10 \leq \mu_v^\sigma(+1) \leq \lambda_v^\mu$;
    \item $|\mu^\sigma_v(+1)-\nu^\sigma_v(+1)| \leq 80nD \mu_v^\sigma(+1) + 4D$;
    \item $ \frac{\nu_{B-v}(\sigma)}{ \mu_{B-v}(\sigma) } \leq 2$.
\end{itemize}
\end{claim}
Assume the above claim holds, which will be proved later. Let $p = \mu_v^\sigma(+1)$. By our choices of parameters, since $D \leq \theta$ is sufficient small, $\mu^\sigma_v(+1) \geq \frac{\kappa}{10} \geq 100n D \mu_v^\sigma(+1) + 4D$. We have
\begin{align*}
 \left( \frac{\nu_{v}^{\sigma}(+1)}{\mu_{v}^{\sigma}(+1)} -  \frac{\nu_{v}^{\sigma}(-1)}{\mu_{v}^{\sigma}(-1)} \right)^2 \leq \left (\frac{4D + 80nD \mu^\sigma_v(+1)}{p(1-p)}\right )^2.    
\end{align*}
Using \Cref{claim:p-bound},
the variance can be bounded by
\begin{align*}
    \Var[\mu^\sigma_B]{h} &\leq 4 p(1-p) \frac{(80nDp +4D)^2}{p^2(1-p)^2} = O(D^2) \cdot \frac{(20np+1)^2}{p(1-p)} \leq O(D^2)\cdot(n^2p + n + 1/p)\\
    &\leq  O(D^2) \cdot \left(n^2 + \frac{1}{\kappa}\right).
\end{align*}
Using \Cref{lem:poincare} and the above upper bound, we have
\begin{align}\label{eq:varh}
    \Var[\mu_B]{h} &\leq C_\eta \sum_{v \in B} \sum_{\sigma \in \Omega_{B - v}} \mu_{B - v}(\sigma) \Var[\mu^\sigma_B]{h}\notag\\ &\leq O_\eta(D^2) \sum_{v \in B}\left(n^2 + \frac{1}{\kappa}\right)\notag\\
 (\text{by \Cref{lem:TV-lower} })\quad    &= O_\eta(d^2) \cdot \left(n^3 + \frac{n}{\kappa}\right).
\end{align}    
Hence, using \eqref{eq:varf-bd}, the variance of $f(x)$ is at most
\begin{align*}
\Var[\mu_B]{f} \leq \E[\mu_B]{f^2} \leq \Var[\mu_B]{h} +  O_\eta(n^2 d^2) = O_\eta(d^2) \cdot \left(n^3 + \frac{n}{\kappa}\right).
\end{align*}


\paragraph{Proofs of technical lemmas and claims}
We first give two general lemmas (\Cref{lem:Zcond-bound} and \Cref{lem:marginal-ratio}) that will be used in later proofs. We then prove all technical lemmas and claims appeared in the proof of \Cref{lem:hardcore-adv-var}.

Define the conditional partition function.
Fix a configuration $x \in \{\pm\}^B$. Define $Z_{S,\mu}^x$ as the conditional partition function defined by
\begin{align*}
    Z_{S,\mu}^x \defeq \sum_{y \in \{\pm\}^S: \mu(x+y) > 0} \prod_{v \in S: y_v = + 1}\lambda_v^\mu.
\end{align*}
Intuitively, $Z_{S,\mu}^x$ is the total weights of $y \in \{\pm\}^S$ such that $x+y$ is a valid configuration (forms an independent set in $G$), where $x+y \in \{\pm\}^V$ is the concatenation of $x$ and $y$.
Alternatively, $Z_{S,\mu}^x$ can be interpreted as follows. Let $N_G(v)$ denote the set of neighbors of $v$ in $G$. Given $x$, one can remove all vertices $v \in S$ from $S$ such that there exists $u \in N_G(v) \cap B$ with $x_u =+1$. Let $S^x \subseteq S$ denote the set of remaining vertices:
\begin{align}\label{eq:sx}
   S^x = S \setminus \{v \in S \mid \exists u \in N_G(v)\cap B \text{ s.t. } x_u = +1 \}. 
\end{align}
Then $Z_{S,\mu}^x$ is the partition function for the hardcore model in induced subgraph $G[S^x]$. 
The following property of the conditional partition function will be used in our proofs.
\begin{lemma}\label{lem:Zcond-bound}
Suppose $\kappa + \theta < 1/(10n)$.
For any $x \in \{\pm\}^B$, it holds that 
\begin{itemize}
    \item $1 \leq Z_{S,\mu}^x,Z_{S,\nu}^x < 2 $;
    \item $|Z_{S,\mu}^x - Z_{S,\nu}^x| \leq 2 n D$.
\end{itemize}
\end{lemma}
\begin{proof}
    Since the empty set contributes the weight $1$ to both $Z_{S,\mu}^x$, so $Z_{S,\mu}^x \geq 1$. For the upper bound, Let $\lambda_{\max} = \max_{v \in S} \max \{\lambda_v^\mu,\lambda_v^\nu\}$. By the definition of $S$, we have $\lambda_{\max} \leq \kappa + D < \kappa + \theta < \frac{1}{10n}$. Hence $Z_{S,\mu}^x \leq (1+1/(10n))^n < 2$. The same bound holds for $Z_{S,\nu}^x$.

    For any independent set $I$ in graph $G[S^x]$, the difference of the weights is $\left\vert \prod_{v\in I}\lambda_v^\nu-\prod_{v\in I} \lambda_v^\mu\right\vert$. We first show the difference of the weights of $I$ is at most $(\lambda_{\max}+D)^{|I|} - \lambda_{\max}^{|I|}$. Assume $\lambda_v^\nu = \lambda_v^\mu + \delta_v$, where $|\delta_v| \leq D$. Then 
    \begin{align}\label{eq:max-diff}
        \left\vert \prod_{v\in I}\lambda_v^\nu-\prod_{v\in I} \lambda_v^\mu\right\vert &= \left\vert \prod_{v\in I}(\lambda_v^\mu + \delta_v)-\prod_{v\in I} \lambda_v^\mu\right\vert = \left\vert \sum_{A \subseteq I: A \neq \emptyset} \prod_{v \in A}\delta_v \prod_{u \in I \setminus A} \lambda_u^\mu\right\vert\notag\\
        &\leq \sum_{A \subseteq I: A \neq \emptyset} \prod_{v \in A}D \prod_{u \in I \setminus A} \lambda_{\max}\notag\\
        &= (\lambda_{\max}+D)^{|I|} - \lambda_{\max}^{|I|}. 
    \end{align}
    The number of independent sets of size $k$ in $G[S^x]$ is at most $\binom{n}{k}$, where $n$ is the number of vertices in $G$. We have
    \begin{align}\label{eq:max-diff-sum}
      \vert Z_{S,\nu}^x-Z_{S,\mu}^x \vert &\leq \sum_{k=0}^n \binom{n}{k} (\lambda_{\max}+D)^k - \sum_{k=0}^n \binom{n}{k}\lambda_{\max}^k \notag\\
         &= \left(1+\lambda_{\max}+D\right)^n - \left(1+\lambda_{\max}\right)^n\notag\\
         &= (1+\lambda_{\max})^n\left( \left(1+\frac{D}{1+\lambda_{\max}}\right)^n -1 \right)\notag\\
    \text{by ($\lambda_{\max},D <1/(10n)$)}\quad &\leq 2nD.
    \end{align}
This proves the lemma.
\end{proof}

The second general lemma we will use is the following bound on the marginal ratio.
\begin{lemma}\label{lem:marginal-ratio}
    Suppose $\kappa + \theta < 1/(10n)$ and $\theta/\kappa < 1/(10n)$.
    For any $x \in \Omega_B$, it holds that
    \begin{align*}
        \left\vert\frac{\nu_B(x)}{\mu_B(x)} -1 \right\vert \leq \frac{10n D}{\kappa}.
    \end{align*}
\end{lemma}
\begin{proof}
    Define  $g(x)=\frac{\nu_B(x)}{\mu_B(x)}$. For $x \in \Omega_B \subseteq \{\pm\}^B$ and $y \in \{\pm\}^S$, we use $x + y$ to denote a full configuration in $\{\pm\}^V$ obtained by concatenating $x$ and $y$. We have
    \begin{align*}
        &\mu_B(x)=\sum_{y\in \{\pm\}^{S}}\mu(x+y)=\frac{\prod_{v\in B:x_v=1}\lambda_v^\mu \cdot Z_{S,\mu}^x}{Z_\mu},\text{ and}\\
        &\nu_B(x)=\sum_{y\in \{\pm\}^{S}}\nu(x+y)=\frac{\prod_{v\in B:x_v=1}\lambda_v^\nu \cdot Z_{S,\nu}^x}{Z_\nu}.
    \end{align*}
Then we have
\begin{align*}
    g(x)=\frac{\nu_B(x)}{\mu_B(x)}=\left(\prod_{v\in B:x_v=1} \frac{\lambda_v^\nu}{\lambda_v^\mu} \right) \cdot \frac{Z_{S,\nu}^x}{Z_{S,\mu}^x} \cdot \frac{Z_\mu}{Z_\nu}.
\end{align*}
Denote $\alpha=\prod_{v\in B:x_v=1} \frac{\lambda_v^\nu}{\lambda_v^\mu}$, $\beta=\frac{Z_{S,\nu}^x}{Z_{S,\mu}^x}$ and $\gamma =\frac{Z_\mu}{Z_\nu}$. We analyze each term one by one.

We have ${D}/{\kappa} \leq \theta / \kappa \leq 1/(10n)$ and  
\begin{align*}
    \alpha \leq \prod_{v\in B:x_v=1}\frac{\lambda_v^\mu+D}{\lambda_v^\mu}\leq \left(1+\frac{D}{\kappa}\right)^{|B|}\leq \left(1+\frac{D}{\kappa}\right)^{n}\leq 1+\frac{2nD}{\kappa}.
\end{align*}
A similar argument gives a lower bound $\alpha \geq 1-\frac{2nD}{\kappa}$. 

For the second term $\beta$, using \Cref{lem:Zcond-bound}, we have
\begin{align*}
    \vert \beta - 1 \vert = \left\vert \frac{Z_{S,\nu}^x}{Z_{S,\mu}^x} - 1 \right\vert = \frac{\vert Z_{S,\nu}^x-Z_{S,\mu}^x \vert}{Z_{S,\mu}^x} \leq \vert Z_{S,\nu}^x-Z_{S,\mu}^x \vert \leq 2nD.
\end{align*}

Finally, for $\gamma = \frac{Z_\mu}{Z_\nu}$, we have the following equivalent form
\begin{align*}
    \gamma = \frac{Z_\mu}{Z_\nu}=\frac{\sum_{x\in\Omega_B}(\prod_{v\in B:x_v=+1}\lambda_v^\mu \cdot Z_{S,\mu}^x)}{\sum_{x\in\Omega_B}(\prod_{v\in B:x_v=+1}\lambda_v^\nu \cdot Z_{S,\nu}^x)}.
\end{align*}
Using the bound for $\alpha$ and $\beta$, since $\theta/\kappa < 1/(10n)$ and $\theta+\kappa < 1/(10n)$, we have
\begin{align}\label{eq:upperZ/Z}
   1 -  \frac{4nD}{\kappa} < \gamma \leq \left( 1 + \frac{2nD}{\kappa} \right)(1+2nD) < 1 + \frac{4nD}{\kappa}.
\end{align}
Combining the bounds for $\alpha$, $\beta$ and $\gamma$, since $D/\kappa \leq \theta/\kappa < 1/(10n)$, we have
\begin{align*}
   1 -  \frac{10nD}{\kappa} < g(x) \leq \left( 1+\frac{2nD}{\kappa}\right)(1+2nD)\left( 1+\frac{4nD}{\kappa}\right) <  1 + \frac{10nD}{\kappa}. \ifthenelse{\boolean{conf}}{}{&\qedhere}
\end{align*}

\end{proof}

\ifthenelse{\boolean{conf}}{\begin{proof}\textbf{of \Cref{claim:tv-bound}}}{\begin{proof}[Proof of \Cref{claim:tv-bound}]}
    For any $y \in \{\pm\}^S$ such that $y+x$ forms an independent set, we can bound 
    \begin{align}\label{eq:bound-diffy}
        \left|\nu_S^x(y)-\mu_S^x(y)\right| &\leq \left\vert \frac{ \prod_{v \in S: y_v= +1}\lambda_v^\nu }{Z_{S,\nu}^x } - \frac{\prod_{v \in S: y_v= +1}\lambda_v^\mu }{Z_{S,\mu}^x } \right\vert\notag\\
    \text{(by $Z_{S,\mu}^x,Z_{S,\nu}^x \geq 1$)}\quad    &\leq \left\vert Z_{S,\mu}^x \prod_{v \in S:y_v=+1}\lambda_v^\nu - Z_{S,\nu}^x \prod_{v \in S:y_v=+1}\lambda_v^\mu \right\vert\notag\\
    \text{(triangle ineq.)}\quad  &\leq Z_{S,\mu}^x\left\vert \prod_{v \in S:y_v=+1}\lambda_v^\nu -  \prod_{v \in S:y_v=+1}\lambda_v^\mu\right\vert + \left\vert Z_{S,\mu}^x - Z_{S,\nu}^x \right\vert\prod_{v \in S:y_v=+1}\lambda_v^\mu .
    \end{align}
    We bound each term separately. 
    Recall that $\lambda_{\max} = \max_{v \in S} \max \{\lambda_v^\mu,\lambda_v^\nu\} < \kappa + \theta$.
    By \Cref{lem:Zcond-bound}, we have $Z^x_{S,\mu} < 2$. Using~\eqref{eq:max-diff}, we have
    \begin{align}\label{eq:bound-diffy-1}
        Z_{S,\mu}^x\left\vert \prod_{v \in S:y_v=+1}\lambda_v^\nu -  \prod_{v \in S:y_v=+1}\lambda_v^\mu\right\vert \leq 2 \left((\lambda_{\max}+D)^{\Vert y \Vert_+} - \lambda_{\max}^{\Vert y \Vert_+}\right),
    \end{align}
    where $\Vert y \Vert_+$ is the number of $+1$s in $y$. For the second term, using \Cref{lem:Zcond-bound}, we have
    \begin{align}\label{eq:bound-diffy-2}
        \prod_{v \in S:y_v=+1}\lambda_v^\mu\ \left\vert Z_{S,\mu}^x - Z_{S,\nu}^x \right\vert \leq 2nD \lambda_{\max}^{\Vert y \Vert_+}.
    \end{align}
    The number of independent sets of size $k$ is at most $\binom{n}{k}$. We have
    \begin{align*}
        \DTV{\nu_S^x}{\mu_S^x}&= \frac{1}{2}\sum_{y \in \{\pm\}^S}\left|\nu_S^x(y)-\mu_S^x(y)\right|\\
        &\leq \sum_{k=0}^n \binom{n}{k}  \left((\lambda_{\max}+D)^k - \lambda_{\max}^k\right) +   nD \sum_{k=0}^n \binom{n}{k} \lambda_{\max}^{k}\\
    \text{(by the same calculation as~\eqref{eq:max-diff-sum})}\quad &\leq 2nD +nD (1+\lambda_{\max})^n\\
    \text{(by $\lambda_{\max} < 1 / (10n)$)}\quad &< 4nD.
    \end{align*}
This proves the upper bound on TV distance.
    %Finally,~\eqref{eq:bound-f} can be further bounded by
    %\begin{align*}
    %    f(x) \leq \frac{5nD}{\kappa} + \DTV{\nu_S^x}{\mu_S^x} < \frac{5nD}{\kappa} + 4nD < \frac{6nD}{\kappa} \leq \frac{30000 n}{\kappa} \cdot d,
    %\end{align*}
    %where the last inequality follows from the fact $d \geq \frac{1}{5000}D$ in \Cref{lem:TV-lower}.
    %The lower bound $f(x) \geq 0$ holds trivially by the definition.
\end{proof}

\ifthenelse{\boolean{conf}}{\begin{proof}\textbf{of \Cref{claim:p-bound}}}{\begin{proof}[Proof of \Cref{claim:p-bound}]}
Recall that $\sigma$ is a configuration in $\Omega_{B-v}$, and $\sigma^{v_+},\sigma^{v_-}$ are configurations in $\{\pm\}^{B}$.
For simplicity of the notation, we use $Z_\mu^+$ to denote $Z_{S,\mu}^{\sigma^{v_+}}$ and $Z_\mu^-$ to denote $Z_{S,\mu}^{\sigma^{v_-}}$.
Similarly, we define $Z_\nu^+,Z_\nu^-$. We have
\begin{align*}
    \vert \mu^\sigma_v(+1) - \nu^\sigma_v(+1) \vert &= \left\vert \frac{\lambda_v^\mu Z_\mu^+}{Z_\mu^- + \lambda_v^\mu Z_\mu^+} - \frac{\lambda_v^\nu Z_\nu^+}{Z_\nu^- + \lambda_v^\nu Z_\nu^+} \right\vert = \left\vert \frac{\lambda_v^\mu Z_\mu^+Z_\nu^- - \lambda_v^\nu Z_\nu^+Z_\mu^-}{(Z_\mu^- + \lambda_v^\mu Z_\mu^+)(Z_\nu^- + \lambda_v^\nu Z_\nu^+)} \right\vert\\
    &\leq \left\vert \lambda_v^\mu Z_\mu^+Z_\nu^- - \lambda_v^\nu Z_\nu^+Z_\mu^- \right\vert\\
    &\leq Z_\nu^+Z_\mu^-|\lambda_v^\mu- \lambda_v^\nu|  + \lambda^\mu_v \vert Z_\mu^+Z_\nu^- - Z_\nu^+Z_\mu^-\vert.
\end{align*}
Using \Cref{lem:Zcond-bound}, we have $Z_\nu^+Z_\mu^-|\lambda_v^\mu- \lambda_v^\nu|\leq 4D$. We also know that $\vert Z_\mu^+ - Z_\mu^+ \vert \leq 2nD$ and $\vert Z_\mu^- - Z_\nu^- \vert \leq 2nD$. Therefore, by using the triangle inequality, we have
\begin{align*}
    \vert Z_\mu^+Z_\nu^- - Z_\nu^+Z_\mu^-\vert \leq Z_\mu^+|Z_\nu^- - Z_\mu^-| + Z_\mu^-|Z_\mu^+ - Z_\nu^+| \leq 8nD.   
\end{align*}
Therefore, we have
\begin{align*}
    \vert \mu^\sigma_v(+1) - \nu^\sigma_v(+1) \vert \leq 4D + 8nD \lambda_v^\mu.
\end{align*}


Next, we bound the value of $\mu^\sigma_v(+1)$. To sample from the distribution $\mu^\sigma_v$, one can first sample all the neighbors $u \in N_G(v) \setminus B$ of $v$, then sample the value of $v$ further conditional on the configuration of $N_G(v)$. Suppose with probability $q$, all vertices $u \in N_G(v) \setminus B$ are sampled to be $-1$. Note that $q \geq (\frac{1}{1+\lambda_{\max}})^n \geq (\frac{1}{1+\kappa+D})^n$. Since $\kappa + \theta < 1/(4n)$ and $D < \theta$, we have $q \geq \frac{1}{2}$. Conditional on this event, $v$ takes value $+1$ with probability $\frac{\lambda_v^\mu}{1+\lambda_v^\mu} \geq \frac{\lambda_v^\mu}{5}$, where we use the fact that $\lambda_v^\mu \leq \lambda_c(\Delta) \leq 4$. We have the  lower bound $\mu^\sigma_v(+1) \geq \frac{1}{10} \lambda_v^\mu$. 
On the other hand, we have the upper bound $ \mu^\sigma_v(+1) \leq \frac{\lambda_v^\mu}{1+\lambda_v^\mu} \leq  \lambda_v^\mu$. Combining together, we have
\begin{align*}
    \frac{\lambda_v^\mu}{10} \leq \mu^\sigma_v(+1) \leq  \lambda_v^\mu. 
\end{align*}
The above bound implies
\begin{align*}
    \vert \mu^\sigma_v(+1) - \nu^\sigma_v(+1) \vert \leq 4D + 80 nD \mu^\sigma_v(+1).
\end{align*}

For the last bound, using \Cref{lem:marginal-ratio}, we have
\begin{align*}
    \frac{ \nu_{B-v}(\sigma) }{\mu_{B-v}(\sigma)} = \frac{\nu_B(\sigma^{v_+})+\nu_B(\sigma^{v_-})}{\mu_B(\sigma^{v_+})+\mu_B(\sigma^{v_-})} \leq 1 + \frac{10n D}{\kappa} < 2,
\end{align*}
where the last inequality follows from the fact that $D/\kappa < 1/(10n)$.
\end{proof}

%To sample from the distribution $\mu^\sigma_v$, one can first sample all the neighbors $u \in N_G(v) \setminus B$ of $v$, then sample the value of $v$ further conditional on the configuration of $N_G(v)$. Suppose with probability $p$, all vertices $u \in N_G(v) \setminus B$ are sampled to be $-1$. We have
%\begin{align*}
%    \mu^\sigma_v(+1) = \frac{p\lambda_v^\mu}{1+\lambda^\mu_v}, \quad \mu^\sigma_v(-1) = \frac{p}{1+\lambda^\mu_v} + (1-p). %
%\end{align*}
%We can define the parameter $a$ such that 
%\begin{align*}
%    a = \frac{p\lambda_v^\mu}{p+ (1-p)(1+\lambda^\mu_v)}.
%\end{align*}



%Consider a similar procedure for sampling $\nu^\sigma_v$. Let $q$ denote the probability that all vertices $u \in N_G(v) \setminus B$ are sampled to be $-1$. We have
%\begin{align*}
%    \nu^\sigma_v(+1) = \frac{q\lambda_v^\nu}{1+\lambda^\nu_v}, \quad \nu^\sigma_v(-1) = \frac{q}{1+\lambda^\nu_v} + (1-q). 
%\end{align*}
%We can compute 
%\begin{align*}
%    \frac{\nu_{v}^{\sigma}(+1)}{\mu_{v}^{\sigma}(+1)} -  \frac{\nu_{v}^{\sigma}(-1)}{\mu_{v}^{\sigma}(-1)}  = \frac{q\lambda_v^\nu(1+\lambda^\mu_v)}{p\lambda_v^\mu(1+\lambda^\nu_v)} - .
%\end{align*}

%Since $\sigma$ is a configuration in $\{\pm\}^V$, if $\mu^\sigma_v(+1)$ and $\mu^\sigma_v(-1)$ are both positive, then $\mu^\sigma_v(+1)=\frac{\lambda_v^\mu}{1+\lambda^\mu_v}$ and $\mu^\sigma_v(-1)=\frac{1}{1+\lambda^\mu_v}$. We further have the following bound
%\begin{align*}
%    \Var[\mu^\sigma]{g}  = \tp{ \frac{\nu_{V-v}(\sigma_{V-v})}{ \mu_{V-v}(\sigma_{V-v}) }}^2 \frac{\lambda_v^\mu}{(1+\lambda_v^\mu)^2} \tp{\frac{\lambda_v^\nu(1+\lambda_v^\mu)}{\lambda_v^\mu(1+\lambda_v^\nu)} - \frac{1+\lambda_v^\mu}{1+\lambda_v^\nu}}^2.
%\end{align*}


%\begin{align*}
%    \Var[\mu_B]{h} =\Var[\mu]{g} &\leq C_\eta \sum_{v \in B} \sum_{\sigma \in \Omega^\mu_{V \setminus \set{v}}} \mu_{V \setminus \set{v}}(\sigma) \mu^\sigma_{v}(+1)\mu^\sigma_{v}(-1)(g(\sigma^{v_+}) -g(\sigma^{v_-}))^2\\
%    &\leq C_\eta \sum_{v \in B} \sum_{\tau \in \Omega^\mu_{B \setminus \set{v}}} \mu_{B \setminus \set{v}}(\tau) \mu^\tau_{v}(+1)\mu^\tau_{v}(-1)(g(\tau^{v_+}) -g(\tau^{v_-}))^2\\
%\end{align*}



%\todo{WF:edit end}


%\begin{align*}
%    \gamma &=\frac{Z_{S,\nu}^x}{Z_{S,\mu}^x} =1+\frac{Z_{S,\nu}^x-Z_{S,\mu}^x}{Z_{S,\mu}^x}\leq 1+|Z_{S,\nu}^x-Z_{S,\mu}^x|\leq 1+|(1+\max_{v\in S}(\lambda_v)+D)^n-(1+\max_{v\in S}(\lambda_v))^n|\\
%    &\leq 1+(1+\frac{\epsilon^5}{n^5})^n((1+\frac{D}{\epsilon^5/n^5})^n-1)
%    \leq 1+(1+\frac{2\epsilon^5}{n^4})\frac{1.9n^6D}{\epsilon^5}\leq 1+\frac{2n^6D}{\epsilon^5}.
%\end{align*}

%A similar argument gives a lower bound $\gamma \leq 1-\frac{3n^6D}{\epsilon^5}$. Because

%\begin{align*}
%    \frac{Z_\nu}{Z_\mu}=\frac{\sum_{x\in\{\pm\}^B}(\prod_{v\in B:x_v=1}\lambda_v^\mu \cdot Z_{S,\mu}^x)}{\sum_{x\in\{\pm\}^B}(\prod_{v\in B:x_v=1}\lambda_v^\nu \cdot Z_{S,\nu}^x)},
%\end{align*}
%we have $(1-\frac{2n^6D}{\epsilon^5})^2\leq\alpha\leq (1+\frac{2n^6D}{\epsilon^5})^2$. Finally, we have 
%\begin{align*}
%    1-\frac{10n^6D}{\epsilon^5} \leq (1-\frac{2n^6D}{\epsilon^5})^4\leq g(x)\leq (1+\frac{2n^6D}{\epsilon^5})^4\leq 1+\frac{10n^6D}{\epsilon^5}.
%\end{align*}




\subsubsection{Approximate the value of the estimator (Proof of \texorpdfstring{\Cref{lem:hardcore-adv-1}}{Lg})}\label{sec:hardcore-adv-1}
Define $\Vert y\Vert_+$ be the number of $+1$s in $y$. 
Let $t \geq 0$ be an integer.
The specific choice of $t$ will be fixed later.
Define the truncated function $f_t$ of $f$ defined by 
\begin{align*}
    f_t(x) \defeq \frac{1}{2}\sum_{y \in \{\pm\}^S: \Vert y\Vert_+\leq t} \left|\frac{\nu_B(x)}{\mu_B(x)}\nu_S^x(y)-\mu_S^x(y)\right|.
\end{align*}

Compared to $f$ in~\eqref{eq:def-f}, $f_t$ only includes the size at most $t$ independent sets of the induced subgraph $G[S^x]$, where $S^x$ is defined in~\eqref{eq:sx}. We have the following relation between $f_t$ and $f$.

\begin{lemma}\label{lem:appf}
  Suppose $\kappa + \theta < 1/(10n)$ and $\theta/\kappa < 1 / (10n)$.
  For any integer $t \geq 0$, any $x\in \Omega_B$,  
  \begin{align}\label{eq:error-t}
  0\leq f(x)-f_t(x)\leq 10^6\left(1+\frac{n}{10}\right)^{t+1} \kappa^t n^{t+2} \cdot d \defeq \eta(\kappa,t)\cdot d.
  \end{align}
\end{lemma}

\begin{proof} %\todo{check proof, $\kappa$}
First, by definition, $ f(x)-f_t(x)\geq 0$.
Recall that $g(x) = \frac{\nu_B(x)}{\mu_B(x)}$.
Let $\Omega_S^x \subseteq \{\pm\}^S$ be the set over all $y$ such that $x + y$ forms an independent set in $G$.
We have
\begin{align*}
f(x) - f_t(x) &= \frac{1}{2}\sum_{y \in \Omega_S^x: \Vert y\Vert_+ \geq t+1}\left|g(x)\nu_S^x(y) - \mu_S^x(y)\right|\\
\text{(by \Cref{lem:marginal-ratio})} \quad &\leq \frac{1}{2}\sum_{y \in \Omega_S^x: \Vert y\Vert_+ \geq t+1}\left|\nu_S^x(y) - \mu_S^x(y)\right| + \frac{5 nD}{\kappa} \cdot\sum_{y \in \Omega_S^x: \Vert y\Vert_+ \geq t+1}\nu_S^x(y)
\end{align*}
Recall that $\lambda_{\max} = \max_{v \in S} \max \{\lambda_v^\mu,\lambda_v^\nu\}$. Using the result in~\eqref{eq:bound-diffy},~\eqref{eq:bound-diffy-1} and~\eqref{eq:bound-diffy-2}, we have
\begin{align*}
    \sum_{y \in \Omega_S^x: \Vert y\Vert_+ \geq t+1}\left|\nu_S^x(y) - \mu_S^x(y)\right| \leq \sum_{k=t+1}^n \binom{n}{k} \left((\lambda_{\max}+D)^k - \lambda_{\max}^k\right) +   nD \sum_{k=t+1}^n \binom{n}{k} \lambda_{\max}^{k}.
\end{align*}
Let $\phi = \kappa + D$. Then $\lambda_{\max} \leq \phi$ and $(\lambda_{\max}+D)^k - \lambda_{\max}^k \leq (\phi+D)^k - \phi^k =\phi^k( (1+D/\phi)^k - 1 )$. Note  that $D/\phi < \theta/\kappa < 1/(4n)$. The last term is at most $\phi^k\cdot (2kD)/\phi \leq \phi^{k-1}\cdot (2nD)$. Then 
\begin{align*}
    f(x) - f_t(x)  &\leq \frac{nD}{\kappa}\sum_{k=t+1}^n \binom{n}{k} \phi^{k} +   \frac{nD}{2} \sum_{k=t+1}^n \binom{n}{k} \phi^k + \frac{5nD}{\kappa}\sum_{k=t+1}^n \binom{n}{k} \phi^k \leq \frac{8nD}{\kappa} \sum_{k=t+1}^n \binom{n}{k} \phi^k\\
    &\leq \frac{8nD}{\kappa}\sum_{k\geq t+1} (n\phi)^k
\end{align*}
Note that $\phi < \kappa + \theta < 1/(4n)$ and $\phi n < 1/10$. Also note that $\phi \leq (1+1/n)\kappa$. The last term is at most $ \frac{10nD}{\kappa} \left( 1 + 1/(10n) \right)^{t+1}\kappa^{t+1} n^{t+1}$. The lemma holds by using $D \leq 5000d$.
\end{proof}






%\begin{align*}
%&f(x)-\tilde{f}(x) = \frac{1}{2}\sum_{y \in \Omega_S^x: \Vert y\Vert_+ > 1}\left|g(x)\frac{ \prod_{v \in S: y_v= +1}\lambda_v^\nu }{Z_{S,\nu}^x } - \frac{\prod_{v \in S: y_v= +1}\lambda_v^\mu }{Z_{S,\mu}^x }\right|\notag\\
%\leq& \frac{1}{2}\sum_{y \in \Omega_S^x: \Vert y\Vert_+ > 1} \left | g(x)Z_{S,\mu}^x \prod_{v \in S: y_v= +1}\lambda_v^\nu - Z_{S,\nu}^x \prod_{v \in S: y_v= +1}\lambda_v^\mu   \right |\tag{by $Z_{S,\mu}^x, Z_{S,\nu}^x\geq 1$}\\
%\leq& \frac{1}{2}\sum_{y \in \Omega_S^x: \Vert y\Vert_+ > 1} \left( Z_{S,\nu}^x \left|\prod_{v \in S: y_v= +1}\lambda_v^\mu - \prod_{v \in S: y_v= +1}\lambda_v^\nu\right| 
%+\left|g(x)Z_{S,\mu}^x-Z_{S,\nu}^x\right|\prod_{v \in S: y_v= +1}\lambda_v^\nu \right ),
%\end{align*}
%where the last inequality holds from triangle inequality. By \Cref{lem:marginal-ratio}, $|g(x) - 1| \leq \frac{10n D}{\kappa}$.
%Using the result of \eqref{eq:max-diff-sum}, we have
%\begin{align*}
%    \left|g(x)Z_{S,\mu}^x-Z_{S,\nu}^x\right| \leq |1-g(x)|Z_{S,\mu}^x + |Z_{S,\mu}^x-Z_{S,\nu}^x|\leq Z_{S,\mu}^x\frac{10nD}{\kappa} + 2nD \leq \frac{30nD}{\kappa},
%\end{align*}
%where the last inequality holds because $Z^x_{S,\mu} \leq (1 + \theta + \kappa)^n < 2$.
%Similar to the analysis of \eqref{eq:max-diff}, recall that $\lambda_{\max} = \max_{v \in S} \max \{\lambda_v^\mu,\lambda_v^\nu\}$, we can compute that  
%\begin{align*}
%Z_{S,\nu}^x \left|\prod_{v \in S: y_v= +1}\lambda_v^\nu-\prod_{v \in S: y_v= +1}\lambda_v^\mu\right|+\left|g(x)Z_{S,\mu}^x-Z_{S,\nu}^x\right|\prod_{v \in S: y_v= +1}\lambda_v^\nu \\
%\leq\,& 2\left|(\lambda_{\max}+D)^{\Vert y\Vert_+}-\lambda_{\max}^{\Vert y\Vert_+}\right|+\frac{30nD}{\kappa}\lambda_{\max}^{\Vert y\Vert_+}\\
%(\text{by $\lambda_{\max} < \kappa + D$})\quad \leq\,& 2\left|(\kappa+2D)^{\Vert y\Vert_+}-(\kappa+D)^{\Vert y\Vert_+}\right|+\frac{30nD}{\kappa}(\kappa+D)^{\Vert y\Vert_+}\\
%(\text{by $\frac{D}{\kappa+D} \leq \frac{1}{4n}$})\quad < & \frac{50nD}{\kappa} (\kappa+D)^{\Vert y\Vert_+}.
%\end{align*}
%We  have
%\begin{align*}
%f(x)-\tilde{f}(x) &\leq \frac{1}{2} \sum_{y \in \Omega_S^x: \Vert y\Vert_+ > 2} \frac{50nD}{\kappa}(\kappa+D)^{\Vert y\Vert_+}\leq \frac{25nD}{\kappa} \sum_{k=3}^n \binom{n}{k} (\kappa+D)^{k} \\
%&\leq \frac{25nD}{\kappa} \sum_{k=3}^n (n(\kappa + \theta))^k \leq \frac{50nD}{\kappa} (n(\kappa+\theta))^3\\
%&= \frac{50n^4 (\kappa+\theta)^3}{\kappa} D.
%\end{align*}
%By our choice of $\theta$ and $\kappa$, it holds that $\theta \leq \kappa/(4n)$. Using \Cref{lem:TV-lower}, we have $f(x) - \tilde{f}(x)$ is at most $10^6 n^4 \kappa^2 d$. By our choice of $\kappa$, it is at most $\frac{\epsilon}{200} d$.



Next, we give our algorithm to approximate $f_t(x)$. We can expand $f_t$ as follows 
\begin{align}\label{eq:exp-ft}
f_t(x) &\defeq \frac{1}{2}\sum_{y \in \{\pm\}^S: \Vert y\Vert_+\leq t} \left|\frac{\nu_B(x)}{\mu_B(x)}  \nu_S^x(y)-\mu_S^x(y)\right|\notag\\
 &= \frac{1}{2}\sum_{y \in \{\pm\}^S: \Vert y\Vert_+\leq t} \left| {\left(\prod_{v\in B:x_v=1} \frac{\lambda_v^\nu}{\lambda_v^\mu} \right)} \cdot {\frac{Z_{S,\nu}^x}{Z_{S,\mu}^x}}\cdot {\frac{Z_\mu}{Z_\nu}} \cdot \nu_S^x(y)-\mu_S^x(y)\right|.
\end{align}

We now introduce the approximation of $Z_{S,\nu}^x,Z_{S,\mu}^x,\mu^x_S,\nu^x_S$ in the above formula. 
Let $\Omega^x \subseteq \{\pm\}^S$ be the set of all $y \in \{\pm\}^S$ such that $x + y$ forms an independent set in graph $G$
For any integer $t \geq 0$, let $\Omega_t^x$ be  all $y \in \Omega^x$ such that $\Vert y \Vert_+ \leq t$.
Define $\ux$ as the distribution $\mu_S^x$ restricted on $\Omega^x_t$. Formally,
\begin{align}\label{def:zx}
\forall y \in \Omega_{t}^x, \quad \ux(y) = \frac{\prod_{v \in S: y_v = +1}\lambda^\mu_v}{\Zux} \qquad\text{ and }\qquad \Zux=\sum_{\tau \in \Omega^x_{t}} \prod_{v \in S: \tau_v = +1}\lambda^\mu_v.
\end{align}
Similarly, we can define $\vx$ and $\Zvx$ from $\nu_S^x$. The following approximation lemmas hold.
\begin{lemma}\label{lem:appZ}
Suppose $\kappa + \theta < 1/(10n)$ and $\theta/\kappa < 1 / (10n)$.
For any integer $t \geq 0$ and any $x \in \Omega_B$, it holds that
\begin{align*}
    \left\vert \frac{Z_{S,\nu}^x}{Z_{S,\mu}^x} - \frac{\Zvx}{\Zux} \right\vert \leq \eta(\kappa,t)\cdot d \quad \text{and} \quad \left\vert \frac{Z_{S,\mu}^x}{Z_{S,\nu}^x} - \frac{\Zux}{\Zvx} \right\vert \leq \eta(\kappa,t) \cdot d,
\end{align*}
where $\eta(\kappa,t)$ is defined in~\eqref{eq:error-t}.
\end{lemma}
\begin{proof}
Due to the symmetry, we only consider the first term in our proof.
Define $\Omega^x_{> t}=\Omega^x \setminus\Omega^x_{t}$.
Because $Z_{S,\mu}^x,\Zux\geq 1$, similar to the proof of \Cref{lem:appf}, we have
\begin{align*}
    &\left\vert \frac{Z_{S,\nu}^x}{Z_{S,\mu}^x} - \frac{\Zvx}{\Zux} \right\vert \leq 
    \left\vert \Zux Z_{S,\nu}^x - Z_{S,\mu}^x \Zvx \right\vert\\
    &=\left\vert \ \sum_{\tau \in \Omega^x_{t}} \prod_{v \in S: \tau_v = +1}\lambda^\mu_v \sum_{\varphi \in \Omega^x} \prod_{u \in S: \varphi_u = +1}\lambda^\nu_u - \sum_{\tau \in \Omega^x_{ t}} \prod_{v \in S: \tau_v = +1}\lambda^\nu_v \sum_{\varphi \in \Omega^x} \prod_{u \in S: \varphi_u = +1}\lambda^\mu_u\right\vert\\
    &=\left\vert \sum_{\tau \in \Omega^x_{ t}}\sum_{\varphi \in \Omega^x_{> t}}\left(\prod_{v \in S: \tau_v = +1}\lambda^\mu_v \prod_{u \in S: \varphi_u = +1}\lambda^\nu_u-\prod_{v \in S: \tau_v = +1}\lambda^\nu_v\prod_{u \in S: \varphi_u = +1}\lambda^\mu_u \right ) \right\vert\\
    &\leq  \sum_{i=0}^{t}\sum_{j=t+1}^{n}\binom{n}{i}\binom{n}{j}\left((\lambda_{\max}+D)^{i+j}-\lambda_{\max}^{i+j} \right).
    %&\leq \sum_{k=t+1}^{n+t}\frac{n^k}{t!}\cdot 2nD\phi^{k-1} \quad \left(\binom{n}{j}\leq n^j/(t+1)!\text{ and at most $(t+1)$ pairs $(i,j)$ s.t. $i+j=k$.}\right)\\
    %&=\frac{nD}{\phi}\sum_{k=3}^{n+2}(n\phi)^k\leq \frac{\epsilon}{200}D.
\end{align*}
In the proof of \Cref{lem:appf}, we already show that  $(\lambda_{\max}+D)^k - \lambda_{\max}^k \leq \phi^{k-1}\cdot (2kD)$, where $\phi = \kappa + \theta$. Note that $\binom{n}{j}\leq n^j/(t+1)!$ and for any $k$, there are at most $(t+1)$ pairs $(i,j)$ s.t. $i+j=k$. We have
\begin{align*}
 \left\vert \frac{Z_{S,\nu}^x}{Z_{S,\mu}^x} - \frac{\Zvx}{\Zux} \right\vert  \leq \frac{2(n+t)D}{\phi t!}   \sum_{k=t+1}^{n+t}{n^k}\cdot \phi^{k} \leq \frac{4nD}{\kappa} \sum_{k \geq t+1}(n\phi)^t.  
\end{align*}
The last term is at most $\eta(\kappa,t) \cdot d$.
\end{proof}

\begin{lemma}\label{lem:sum-bound}
Suppose $\kappa + \theta < 1/(10n)$ and $\theta/\kappa < 1 / (10n)$. For any integer $t \geq 0$,    it holds that 
\begin{align*}
\sum_{y \in \Omega^x_t} \vert \ux(y) - \mu^x_S(y) \vert \leq \eta(\kappa,t) \cdot d \quad \text{and} \quad \sum_{y \in \Omega^x_t} \vert \vx(y) - \nu^x_S(y) \vert \leq \eta(\kappa,t) \cdot d.
\end{align*}
\end{lemma}

\begin{proof}
Due to the symmetry of $\mu$ and $\nu$, we only prove the first term here. We know that $\Zux\leq Z_{S,\mu}^x$,
\begin{align*}
    \sum_{y \in \Omega^x_t} \vert \ux(y) - \mu^x_S(y) \vert&=\sum_{y \in \Omega^x_t} \vert \frac{\prod_{v\in S:y_v=1}\lambda_v^\mu}{\Zux} - \frac{\prod_{v\in S:y_v=1}\lambda_v^\mu}{Z_{S,\mu}^x} \vert =\frac{Z_{S,\mu}^x-\Zux}{\Zux Z_{S,\mu}^x}\sum_{y \in \Omega^x_t}\prod_{v\in S:y_v=1}\lambda_v^\mu\\
    &=\frac{Z_{S,\mu}^x-\Zux}{\Zux Z_{S,\mu}^x}\Zux=\frac{Z_{S,\mu}^x-\Zux}{Z_{S,\mu}^x}\leq Z_{S,\mu}^x-\Zux\\
    &=\sum_{y\in \Omega_{>t}^x}\prod_{v\in S:y_v=1}\lambda_v^\mu\leq \sum_{y\in \Omega_{>t}^x}\lambda_{\max}^{\Vert y \Vert_+}\leq \sum_{k=t+1}^n\binom{n}{k}\phi^k\leq \frac{1}{(t+1)!}\sum_{k\geq t+1}(n\phi)^k.
\end{align*}
The last term is at most $\eta(\kappa,t) \cdot d$.
\end{proof}




Now, we are ready to give our algorithm.
First consider the ratio $R = \frac{Z_\nu}{Z_\mu}$ in~\eqref{eq:exp-ft}. We have the following algorithm to approximate $R$. 

\begin{lemma}\label{lem:alg-ratio}
If $\eta(\kappa,t)\leq \frac{\epsilon}{200}$, $\theta/\kappa < 1/(10n)$, and $\theta + \kappa < 1/(10n)$, where $\eta(\kappa,t)$ is defined in~\eqref{eq:error-t}, then there exists a randomized algorithm that computes a random number $\tilde{R}$ in time $T' \cdot \TS_G(T'/10^3) + T' \cdot O(n^t)$, where $T'=O(\frac{n^3 + n/\kappa}{\epsilon^2})$ and $\TS_G(\cdot)$ is the cost of sample oracle for $\mu$, such that with probability at least $0.99$,
\begin{align}\label{eq:tildeR}
   \left\vert \tilde{R} - \frac{Z_\nu}{Z_\mu} \right\vert \leq \frac{\epsilon}{100} d.
\end{align}
\end{lemma}

We first prove \Cref{lem:hardcore-adv-1} assuming \Cref{lem:alg-ratio}. \Cref{lem:alg-ratio} will be proved later.


\ifthenelse{\boolean{conf}}{\begin{proof}\textbf{of \Cref{lem:hardcore-adv-1}}}{\begin{proof}[Proof of \Cref{lem:hardcore-adv-1}]}
We fix the parameter $t = 4$. 
Recall that $\kappa = 10^{-9}\frac{\epsilon^{1/4}}{n^{3/2}}$ and $\theta = 10^{-10}\frac{\epsilon^{1/4}}{n^{5/2}}$.
We can verify that $\eta(\kappa,t) = 10^6\left(1+\frac{n}{10}\right)^{t+1} \kappa^t n^{t+2} \leq \frac{\epsilon}{200}$, $\theta + \kappa < 1/ (10n)$, and $\theta / \kappa < 1/(10n)$.

In the construction step, we use \Cref{lem:alg-ratio} to compute the random number $\tilde{R}$.
We say the construction step succeeds if~\eqref{eq:tildeR} is satisfied, which happens with probability at least $0.99$. 

In the query step, given any $x \in \Omega_B$, our data structure answers the following $\hat{f}(x)$:
\begin{align}\label{eq:hatf}
    \hat{f}(x) = \frac{1}{2}\sum_{y \in \Omega^x_t} \left| {\left(\prod_{v\in B:x_v=1} \frac{\lambda_v^\nu}{\lambda_v^\mu} \right)} \cdot \frac{\Zvx}{\Zux}\cdot \tilde{R} \cdot \vx(y)-\ux(y)\right|.
\end{align}

We bound the approximation error of $\hat{f}$. Define $A = \left(\prod_{v\in B:x_v=1} \frac{\lambda_v^\nu}{\lambda_v^\mu} \right) \cdot \frac{\Zvx}{\Zux}\cdot \tilde{R}$ and $B = \left(\prod_{v\in B:x_v=1} \frac{\lambda_v^\nu}{\lambda_v^\mu} \right) \cdot \frac{Z_{S,\nu}^x}{Z_{S,\mu}^x}\cdot \frac{Z_\nu}{Z_\mu}$. By \Cref{lem:Zcond-bound} and the analysis in the proof of \Cref{lem:marginal-ratio}, we have $\left(\prod_{v\in B:x_v=1} \frac{\lambda_v^\nu}{\lambda_v^\mu} \right) < (1+\theta/\kappa)^n < 2$, $\frac{Z_{S,\nu}^x}{Z_{S,\mu}^x}\leq 2$, and $\frac{Z_\nu}{Z_\mu}\leq 2$. Also note that $d \leq 1$. Using \Cref{lem:appZ} and our assumption on $\tilde{R}$, it holds that 
\begin{align*}
    |A-B| \leq \frac{\epsilon}{25}d.
\end{align*}
Using triangle inequality, we can bound
\begin{align*}
    |f^t(x) - \hat{f}(x)| \leq B\sum_{y \in \Omega^x_t}\vert \vx(y) - \nu^x_S(y) \vert + |A-B|\sum_{y \in \Omega^x_t}\vx(y) + \sum_{y \in \Omega^x_t}\vert \ux(y) - \mu^x_S(y) \vert.
\end{align*}
Using the fact $B\leq 8$ and \Cref{lem:sum-bound}, we have $ |f^t(x) - \hat{f}(x)| \leq \frac{\epsilon}{10}d$. By \Cref{lem:appf},
\begin{align*}
     |f(x) - \hat{f}(x)| \leq \frac{\epsilon}{8}d.
\end{align*}

The construction step takes time $T' \cdot \TS_G(T'/10^3) + T' \cdot O(n^4)$, where $T' = O(\frac{n^3}{\epsilon^2}+\frac{n^{5/2}}{\epsilon^{9/4}})$. Since the hardcore model satisfies the uniqueness condition in~\eqref{eq:cond-hardcore}, we have $\TS_G(T'/10^3) = O_\eta(\Delta n\log\frac{n}{\epsilon})$~\cite{CFYZ22,CE22}. 
The total construction time is $\tilde{O}(\frac{n^7}{\epsilon^2}+\frac{n^{6.5}}{\epsilon^{9/4}})$.
For each query, the running time is dominated by computing distributions $\ux$ and $\vx$, which is $O(n^t)=O(n^4)$.
\end{proof}






Finally, we prove \Cref{lem:alg-ratio}. 

\ifthenelse{\boolean{conf}}{\begin{proof}\textbf{of \Cref{lem:alg-ratio}}}{\begin{proof}[Proof of \Cref{lem:alg-ratio}]}
Recall our definition of $\mu_B$ and $\nu_B$ is that for any $x \in \Omega_B$,
\begin{align*}
    \mu_B(x)=\frac{\prod_{v\in B:x_v=1}\lambda_v^\mu Z_{S,\mu}^x}{Z_\mu},\quad \nu_B(x)=\frac{\prod_{v\in B:x_v=1}\lambda_v^\nu Z_{S,\nu}^x}{Z_\nu}.
\end{align*}
We can compute that 
\begin{align*}
    \frac{Z_\nu}{Z_\mu}=\sum_{x\in\Omega_B}\mu_B(x)\frac{\prod_{v\in B:x_v=1}\lambda_v^\nu Z_{S,\nu}^x}{\prod_{v\in B:x_v=1}\lambda_v^\mu Z_{S,\mu}^x}=\mathbf{E}_{x \sim \mu_B}\Big[{\underbrace{\frac{\prod_{v\in B:x_v=1}\lambda_v^\nu Z_{S,\nu}^x}{\prod_{v\in B:x_v=1}\lambda_v^\mu Z_{S,\mu}^x}}_{\defeq Q(x)}}\Big].
\end{align*}
We estimate $Z_\nu/Z_\mu$ by sampling $x$ from $\mu_B$, approximating $Q(x)$ and taking the average. We propose the algorithm as follow. Let $T' = O (\frac{n^3 + n/\kappa}{\epsilon^2})$ be a sufficiently large integer.
\begin{itemize}
    \item Draw $T'$ independent approximate samples $x_1,\cdots,x_{T'}$ from $\mu_B$ with $\DTV{\mu_B}{x_i}\leq \frac{1}{1000T'}$. 
    \item Compute $\tilde{Q}(x)=\frac{\prod_{v\in B:x_v=1}\lambda_v^\nu \Zvx}{\prod_{v\in B:x_v=1}\lambda_v^\mu \Zux}$ for $x=x_1,\cdots x_{T'}$, where $\Zux,\Zvx$ are defined in~\eqref{def:zx}.
    \item Compute $\tilde{R} = \frac{1}{T'}\sum_{i=1}^{T'}\tilde{Q}(x_i)$.
\end{itemize}

First, for any $x\in \Omega_B$,
by \Cref{lem:appZ}, since $\eta(\kappa,t)\leq \frac{\epsilon}{200}$ and $D/\kappa\leq \theta/\kappa\leq 1 / (10n)$,
\begin{align*}
    \left \vert Q(x)-\tilde{Q}(x)\right \vert&=\left \vert \frac{\prod_{v\in B:x_v=1}\lambda_v^\nu Z_{S,\nu}^x}{\prod_{v\in B:x_v=1}\lambda_v^\mu Z_{S,\mu}^x}-\frac{\prod_{v\in B:x_v=1}\lambda_v^\nu \Zvx}{\prod_{v\in B:x_v=1}\lambda_v^\mu \Zux} \right \vert=\left\vert \frac{Z_{S,\nu}^x}{Z_{S,\mu}^x} - \frac{\Zvx}{\Zux}\right \vert \prod_{v\in B:x_v=1}\frac{\lambda_v^\nu}{\lambda_v^\mu}\\
    &\leq \frac{\epsilon}{200}d \left(\frac{\kappa+D}{\kappa}\right)^n \leq \frac{\epsilon d}{150}.
\end{align*}
Recall $h(x)=\nu_B(x)/\mu_B(x)$, the variance of $Q(x)$ is
\begin{align*}
    \Var[\mu_B]{Q} = \Var[x\sim \mu_B]{\frac{\prod_{v\in B:x_v=1}\lambda_v^\nu Z_{S,\nu}^x}{\prod_{v\in B:x_v=1}\lambda_v^\mu Z_{S,\mu}^x}}=\Var[x\sim \mu_B]{\frac{Z_\nu \nu_B(x)}{Z_\mu \mu_B(x)}}=\frac{Z_\nu^2}{Z_\mu^2}\Var[\mu_B]{h}.
\end{align*}
In~\eqref{eq:upperZ/Z}, we showed that $\frac{Z_\mu}{Z_\nu}\leq 1+\frac{4nD}{\kappa}$, which implies $\frac{Z_\nu^2}{Z_\mu^2}\leq \left(1+\frac{4nD}{\kappa} \right )^2\leq e^{4/5}$. In~\eqref{eq:varh}, we also proved that $\Var[\mu_B]{h}\leq O_\eta(d^2) \cdot \left(n^3 + \frac{n}{\kappa}\right)$. We can conclude that 
\begin{align*}
    \Var[\mu_B]{Q} \leq O_\eta(d^2) \cdot \left(n^3 + \frac{n}{\kappa}\right).
\end{align*}


Assume that we have an ideal algorithm that draw perfect samples $x_1,\ldots,x_{T'}$ and exactly compute $Q(x_1),\cdots,Q(x_{T'})$ and compute $R^*=\frac{1}{T'}\sum_{i=1}^{T'}Q(x_i)$. Note $T' = O (\frac{n^3 + n/\kappa}{\epsilon^2})$. By Chebyshev's inequality, if $T'$ is sufficiently large, we have
\begin{align*}
    \Pr[]{|R^*-Z_\nu/Z_\mu|\geq \frac{\epsilon d}{300}}\leq 0.005.
\end{align*}
Note that  $\left \vert Q(x)-\tilde{Q}(x)\right \vert \leq \frac{\epsilon d}{150}$ for all $x \in \Omega_B$.
All the approximate samples $x_1,\cdots x_{T'}$ can be coupled successfully with perfect samples with probability at least $1-T'\cdot \frac{1}{1000T'}=0.999$.
We can couple our algorithm with ideal algorithm such that $|R^*-\tilde{R}|\leq \frac{\epsilon d}{150}$ with probability at least $0.99$. By a union bound, with probability at least 0.99,
\begin{align*}
    \left \vert \tilde{R}-\frac{Z_\nu}{Z_\mu} \right \vert \leq \frac{\epsilon d}{150}+\frac{\epsilon d}{300} = \frac{\epsilon d}{100}.
\end{align*}

The running time of our algorithm is $T' \cdot \TS_G(T'/10^3) + T' \cdot O(n^t)$ because each $\tilde{Q}(x)$ can be computed in time $O(n^t)$.
%Recall that 
%\begin {align*}
%    g(x)=\frac{\nu_B(x)}{\mu_B(x)}=  \underbrace{\left(\prod_{v\in B:x_v=1} \frac{\lambda_v^\nu}{\lambda_v^\mu} \right)}_{\alpha} \cdot \underbrace{\frac{Z_{S,\nu}^x}{Z_{S,\mu}^x}}_{\beta} \cdot \underbrace{\frac{Z_\mu}{Z_\nu}}_{\gamma}.
%\end{align*}
%The value of $\alpha$ can be computed in time $O(n)$. 
\end{proof}




%The total variation distance between $\ux$ and $\vx$ is
%\begin{align*}
%    \DTV{{\mu_S^x}'}{{\nu_S^x}'}=\frac{1}{2}\left (\left |\frac{1}{{Z_{S,\mu}^x}'}-\frac{1}{{Z_{S,\nu}^x}'} \right| +\sum_{v\in S^x}\left |\frac{\lambda_v^\mu}{{Z_{S,\mu}^x}'}-\frac{\lambda_v^\nu}{{Z_{S,\nu}^x}'} \right| \right),
%\end{align*}
%which can be directly computed. Now we can propose our algorithm as follow:


%\begin{itemize}
%    \item Input: $x\in \{\pm\}^B$.
%    \item Output: The estimation $\hat{f}(x)$.
%    \item Compute random $\hat{Z}_\mu$ and $\hat{Z}_\nu$ such that with probability at east $1 - \delta/10$, $\hat{Z}_\mu$ and $\hat{Z}_\nu$, approximate $Z_\mu$ and $Z_\nu$ respectively with relative-error $\epsilon_1$. Since both $\mu$ and $\nu$ satisfying uniqueness condition, this step can be solve in time $\frac{\Delta n}{\epsilon^2_1} \mathrm{polylog}$~\cite{CFYZ22,CE22,SVV09}.
%    \item  Compute ${Z_{S,\mu}^x}'$ and ${Z_{S,\mu}^x}'$.
%    \item Compute $\hat{g}(x)=\left(\prod_{v\in B:x_v=1}\frac{\lambda_v^\nu}{\lambda_v^\mu} \right)\cdot \frac{\hat{Z}_{S,\nu}^x}{\hat{Z}_{S,\mu}^x}\cdot \frac{\hat{Z}_\mu}{\hat{Z}_\nu}$.
%    \item Compute $\hat{f}(x)=\frac{1}{2}\left (\sum_{v\in S^x}|\hat{g}(x){\nu_S^x}'(v)-{\mu_S^x}'(v)|+|\hat{g}(x){\nu_S^x}'(\emptyset)-{\mu_S^x}'(\emptyset)| \right )$.
%\end{itemize}























\section{Proofs of algorithmic results}\label{sec:proof-main}

\subsection{The general algorithm (Proof of \texorpdfstring{\Cref{thm:Ising-1}}{Lg})}\label{sec:proof-gen}

\paragraph{Compute marginal lower bound}
In our main theorem, the input instance is promised to be $b$-marginally bounded for some parameter $b$. However, the specific value of $b$ is not given to the algorithm. The following algorithm computes the tight value of marginal lower bound.  
\begin{lemma}\label{lem:alg-b}
There exists an algorithm such that given any hardcore or Ising model $\mu$ in graph $G=(V,E)$, %if the model is $b$-marginally bounded (the value $b$ is not given to the algorithm), 
it returns a value $b$ in time at most $(1/b)^{O(1/b)} n$ for hardcore model and in time at most $O(n+m)$ for Ising model such that 
\begin{align}\label{eq:def-b}
    b = \min \{\mu^\sigma_v(c) \mid v\in V, \sigma \text{ is a feasible partial pinning, and } \mu^\sigma_v(c) > 0 \}.
\end{align}
\end{lemma}
The proof of \Cref{lem:alg-b} will be given later. With this lemma, we can assume that the value of $b$ is known to the algorithm. 


\paragraph{Pre-processing step}
For Ising model, we need the following pre-processing step to reduce the general Ising model to the soft-Ising model.
Recall that an Ising model $(G=(V,E),J,h)$ is said to be \emph{soft} if $h \in \mathbb{R}^V$ instead of $h \in (\mathbb{R} \cup \{\pm \infty\})^V$. There are three cases: 
\begin{enumerate}
    \item Case 1: If there exists $v$ such that ($h_v^\mu =  +\infty,h_v^\nu = - \infty$)  or ($h_v^\mu = -\infty, h_v^\nu = +\infty$), we can direct compute that $\DTV{\mu}{\nu}=1$.
    \item Case 2: If there exists $v$ such that ($h_v^\mu =  \pm \infty,h_v^\nu \neq \pm \infty$) or ($h_v^\mu \neq \pm \infty, h_v^\nu = \pm \infty$), without loss of generality, we consider the case ($h_v^\mu =  +\infty,h_v^\nu \neq \pm \infty$). We have 
    \begin{align*}
        \DTV{\mu}{\nu}\geq |\mu_v(-)-\nu_v(-)|=|0-\nu_v(-)|=\nu_v(-) \geq b,
    \end{align*} 
    where the last inequality due to the marginal lower bound.
    %Because $\nu$ is $b$-marginally bounded, for each $\sigma\in\{\pm\}^{V\setminus\{v\}}$, $\nu^\sigma(v=+)=\nu^\sigma(v=-)=0$ or $\nu^\sigma(v=-)\geq b$. $\nu^\sigma(v=-)\geq b\cdot \nu((V\setminus\{v\})=\sigma)$. Summing over all $\sigma\in\{\pm\}^{V\setminus\{v\}}$, we have $\DTV{\mu}{\nu}\geq \nu(v=-)\geq b$. 
    We use additive-error algorithm in \Cref{thm:Approximate-Gibbs} with additive error $b\epsilon$ in this case. The running time is 
    \begin{align}\label{eq:time-pre}
        O\tp{\TC_G\tp{\frac{b\epsilon}{4}}+ \frac{1}{b^2\epsilon^2}\tp{ \TW_{G} + \TS_G\tp{\frac{b\epsilon}{4}}}} =O\tp{\TC_G\tp{\frac{b\epsilon}{4}}+ \frac{1}{b^2\epsilon^2}\tp{\TS_G\tp{\frac{b\epsilon}{4}}}},
    \end{align}
    where the equation holds because $\TW_{G} = O(n+m)$ and we can assume $\TS_G(\cdot),\TC_G(\cdot)$ is at least $\Omega(n+m)$ since the algorithm needs to read all vertices and all edges.
    
    \item Case 3: For all $v\in V$, if $h_v^\nu=\pm \infty$ or $h_v^\mu=\pm \infty$, then $h_v^\nu=h_v^\mu$. These vertices are fixed to some value with probability 1. By the standard self-reducibility, one can remove all these vertices and change external fields of neighbors to obtain two soft-Ising models. Formally, one can go through all vertices $v \in V$ whose value is fixed as $c \in \{\pm\}$, for every free neighbor $u$ of $v$, update $h^\mu_u \gets h^\mu_u + J^\mu_{uv}c$ and $h^\nu_u \gets h^\nu_u + J^\nu_{uv}c$.

    
    We remark that (1) the new soft-Ising models also have the same marginal lower bound because we only remove vertices whose marginal lower bound is 1; (2) to sample from the soft-Ising model, one can call sampling oracle on original model and do a projection; to approximately count the partition function, one can use the approximating counting oracle on the original model, because two partition functions differ only by an easy-to-compute factor.
\end{enumerate}

We also do the pre-processing step for hardcore model. A hardcore model $(G,\lambda)$ is said to be soft if $\lambda \in \mathbb{R}_{>0}^V$ instead of $\lambda \in \mathbb{R}_{\geq 0}^V$. There two cases.
\begin{enumerate}
    \item Case 1: There exists $v$ such that $(\lambda^\mu_v = 0,\lambda^\nu_v > 0)$ or $(\lambda^\mu_v > 0,\lambda^\nu_v = 0)$, then $\DTV{\mu}{\nu} \geq b$. We use additive-error algorithm to solve the problem in time~\eqref{eq:time-pre}.
    \item Case 2: For all $v$, $\lambda^\mu_v = 0$ if and only if $\lambda^\nu_v = 0$. We can simply remove all such vertices and work on the soft-hardcore model on the remaining graph. Again, the marginal lower bound and sampling/approximate counting oracles also work for new soft-hardcore model.
\end{enumerate}

\paragraph{The main algorithm}
Since we work on soft models, $b\leq \frac{1}{2}$. 
Define the parameters
\begin{align*}
    \CC = \begin{cases}
         b^3 &\text{hardcore model},\\
         \frac{b^2}{2} &\text{Ising model}.
     \end{cases}, \text{ and } 
     \Thre = \begin{cases}
         \frac{b}{2(1-b)n} &\text{hardcore model},\\
         \frac{1}{2(n+3m)} &\text{Ising model}.
     \end{cases}
\end{align*}
The algorithm computes the parameter distance $\dis(\mu,\nu)$ in time $O(n + m)$.
If $\dis(\mu,\nu) \geq \theta$, then by \Cref{lem:TV-lower}, $\DTV{\mu}{\nu} \geq \theta \CC$, we use additive-error algorithm in \Cref{thm:Approximate-Gibbs} with additive error $\theta \CC \epsilon$. Similar to~\eqref{eq:time-pre}, the running time is 
\begin{align}\label{eq:time-add}
   O\tp{\TC_G\tp{\frac{\theta \CC \epsilon}{4}}+ \frac{1}{(\theta \CC \epsilon)^2}\tp{\TS_G\tp{\frac{\theta \CC \epsilon}{8}}}}.
\end{align}

Next, assume that $\dis(\mu,\nu) < \theta$. 
%We first verify \Cref{cond:meta} for the soft-Ising model, which gives the algorithm for soft-Ising model when parameter distance is small. Recall $\theta = \frac{1}{2(n+3m)}$ is the threshold parameter for soft-Ising model in \eqref{eq:theta}. Also recall $\CC$ is the parameter for soft-Ising model in \Cref{lem:TV-lower}.
We use the basic algorithm in \Cref{thm:alg-main}.
We have the following two lemmas for the soft-hardcore and soft-Ising models.

\begin{lemma}\label{lem:par-hardcore}
Let $\mu$ and $\nu$ be two soft-hardcore models satisfying $b$-marginal lower bound and $\dis(\mu,\nu) \leq \theta$. Then $\mu$ and $\nu$ satisfy \Cref{cond:meta} with $K = {4n}/{(b\CC)}$ and $L = 2$.
\end{lemma}

\begin{lemma}\label{lem:par-Ising}
Let $\mu$ and $\nu$ be two soft-Ising models satisfying $\dis(\mu,\nu) \leq \theta$ and $\DTV{\mu}{\nu} \geq \CC \dis(\mu,\nu)$. Then $\mu$ and $\nu$ satisfy \Cref{cond:meta} with $K = 4(n+m)/\CC$ and $L = 2$.
\end{lemma}

Assuming the above two lemmas, we can use \Cref{thm:alg-main} to solve the problem in time 
\begin{align}\label{eq:time-multi}
  O\tp{\TC_G\tp{\frac{\epsilon}{4}} + T \cdot \TS_G\tp{\frac{1}{100T}}}, \quad\text{where } T = O\tp{\frac{L^2K^2}{\epsilon^2}}.
\end{align}
The final running time of our algorithm is dominated by the maximum of~\eqref{eq:time-pre},~\eqref{eq:time-add},~\eqref{eq:time-multi}, and the running time in \Cref{lem:alg-b}. Since both $\TS_G,\TC_G$ are non-increasing functions, the running time of our algorithm is at most
\begin{align*}
    C_b \cdot \frac{N^2}{\epsilon^2} \TS_G \tp{ \frac{\epsilon^2}{C_b N^2}} + \TC_G\tp{\frac{\epsilon}{C_b N}} + C'_b N,
\end{align*}
where $C_b, C'_b \geq 1$ are parameters depending only on $b$ and for hardcore model, $N = n$; for Ising model, $N = n+ m$. 
For hardcore model
\begin{align*}
    C_b = \mathrm{poly}\tp{\frac{1}{b}}, \quad C'_b  = \tp{\frac{1}{b}}^{O(\frac{1}{b})};
\end{align*}
and for Ising model,
\begin{align*}
    C_b = \mathrm{poly}\tp{\frac{1}{b}}, \quad C'_b = O(1).
\end{align*}
The parameter $C'_b$ comes from the running time in \Cref{lem:alg-b}.
%This proves \Cref{thm:Ising-1}.

\begin{remark}\label{remark:b}
\Cref{thm:Ising-1} presents a simplified version that assumes the marginal lower bound \( b \) to be a constant. However, our algorithm applies to both the Ising and hardcore models with an arbitrary marginal lower bound \( b \), where \( b \) may depend on the size of the input. The running time of our algorithm is given by
\begin{align*}
    \mathrm{poly}\tp{\frac{1}{b}} \cdot \frac{N^2}{\epsilon^2} \TS_G \tp{ \frac{\mathrm{poly}(b) \epsilon^2}{N^2}} + \TC_G\tp{\frac{\mathrm{poly}(b) \epsilon}{ N}} + C'_b N,
\end{align*}    
\begin{itemize}
    \item \textbf{Ising model:} The parameter \( C'_b = O(1) \), so a polynomial-time reduction from TV-distance estimation to sampling and approximate counting exists for the Ising model with marginal lower bound \( b \geq \frac{1}{\mathrm{poly}(n)} \).
    \item \textbf{Hardcore model:} The parameter  \( C'_b = (1/b)^{O(1/b)} \), which comes from the running time in \Cref{lem:alg-b}. One can improve the last term \( C'_b N \) to \( \mathrm{poly}(n) \cdot \TC_G(\frac{1}{10}) \) by assuming a slightly stronger approximate counting oracle. In \Cref{thm:Ising-1}, we only assume approximate counting oracles for \( \mu \) and \( \nu \). If we further assume that approximate counting oracles work for all conditional distributions induced by \( \mu \) and \( \nu \), then by going through the proof of \Cref{lem:alg-b}, we can obtain an algorithm that computes \( b' \) in time \( \mathrm{poly}(n) \cdot \TC_G(\frac{1}{10}) \), such that with high probability, the value \( b' \) satisfies \( \frac{b}{2} \leq b' \leq b \) for \( b \) defined in~\eqref{eq:def-b}. This \( b' \) is also a marginal lower bound and provides a constant approximation to the true lower bound. We can then use this \( b' \) in the remainder of the algorithm. All the subsequent proofs follow for this $b'$.  Hence, given the stronger approximate counting oracle, the polynomial-time reduction exists for the hardcore model with marginal lower bound $b \geq \frac{1}{\mathrm{poly}(n)}$. 
\end{itemize}
\end{remark}

Finally, we give the proofs of technical lemmas.
We need the following property about the soft hardcore model.
\begin{lemma}\label{lem:const-degree}
Let $0 < b < 1$. Suppose a soft-hardcore model $(G,\lambda)$ is $b$-marginally bounded. For any vertex $v \in V$, let $\deg_v^{\text{free}}$ denote the number of free neighbors $u$ of $v$ such that $\lambda_u > 0$. For any $v \in V$, it holds that $\deg_v^{\text{free}} \leq \frac{\ln b}{\ln (1- b)}$.
\end{lemma}
\begin{proof}
Let $N^2(v)$ denote the set of vertices with distance 2 to vertex $v \in V$ in graph $G$. 
Let $\sigma$ be a pinning that fixes all vertices in $N^2(v)$ to the value $-1$. 
Let $N^{\text{free}}(v)=\{v_1,v_2,\ldots,v_\ell\}$ denote the set of free neighbors of $v$, where $\ell = \deg^{\text{free}}_v$. 
Let $\mu$ denote  the Gibbs distribution.
Conditional on $\sigma$, $v$ takes value $+$ only if all free neighbors of $v$ take the value $-$. 
Since the hardcore model is soft, $v$ takes $+$ with a positive probability so that the marginal lower bound appiles.
We have
\begin{align*}
    b\leq \mu_v^\sigma(+) \leq \prod_{j = 1}^\ell \mu_{v_j}(-\mid \sigma \text{ and } (\forall k < j, v_k \gets -)). 
\end{align*}
For any free neighbor $v_j$, conditional on $\sigma$ and $ v_k \gets -$ for all $k < j$, $v_j$ takes $+$ with a positive probability so that $v_j$ takes $+$ with probability at least $b$. Hence,
\begin{align*}
    b \leq (1-b)^{\deg^{\text{free}}_v}. 
\end{align*}
Note that $1 - b < 1$. This proves the upper bound of the free degree.
\end{proof}


\ifthenelse{\boolean{conf}}{\begin{proof}\textbf{of \Cref{lem:alg-b}}}{\begin{proof}[Proof of \Cref{lem:alg-b}]}
For hardcore model, if $\lambda_u=0$ for some $u\in V$, we fix $u=-$ and consider the remained subgraph. If the subgraph has no vertices, we just return b = 1. 

When $c=-$, for each any partial configuration $\tau$ on $\Lambda \subseteq V$ where $v \notin \Lambda$, consider the marginal probability $\mu_v^\tau(-)$.
Then $\mu_v^\tau(-)$ is a convex combination of $\mu_v^\sigma(-)$'s, where $\sigma$ is a partial configuration on $V  \setminus \{v\}$.
For any $\sigma$, $\mu^\sigma_v(-)$ is positive.
Due to the conditional independence, we only need to consider the worst pinning of on $N(v)$, where $N(v)$ is the set of all neighbors of the vertex $v$.
It is easy to see $\mu^\sigma_v(-) \geq \frac{1}{1+\lambda_v^\mu}$, and equality is achieved when $\sigma$ fixes the values of all neighbors to be $-$. 

Now we consider the case $c = +$ and fix a vertex $v$. Consider a partial configuration $\sigma\in \{\pm\}^\Lambda$ for subset $\Lambda \subseteq V\setminus \{v\}$. If there exists a vertex $u\in N(v)$ such that $u\in \Lambda$, to make $\mu_v^\sigma(+)$ nonzero, $\sigma(u)$ must be $-$. 
Assume $u \in N(v)$ and $\sigma_v = -$.
We consider another subset $\sigma'$ on $\Lambda' = \Lambda \setminus \{u\}$ such that $\sigma'_{\Lambda'} = \sigma_{\Lambda'}$. Define the notation
\begin{align*}
    w_\mu(\sigma,v = +) = \sum_{\tau \in \{\pm\}^V:\tau_\Lambda  = \sigma \land \tau_v = +} w_\mu(\tau).
\end{align*}
We have $w_\mu(\sigma,v = +) = w_\mu(\sigma',v = +)$, because $\tau_v = +$ forces all vertices in $N(v)$ to take the value $-$. On the other hand, $w_\mu(\sigma,v = -) \leq w_\mu(\sigma',v = -)$, because $u$ in $\sigma_0$ is free and it can either take $-$ or $+$.
Our goal is to find a condition such that $v$ takes $+$ with the minimum positive probability. We can assume $N(v) \cap \Lambda = \emptyset$.
Again, $\mu^\sigma_v(+)$ is a convex combination over all $\mu^\tau_v(+)$, where the feasible partial configuration $\tau \in \{\pm\}^{V\setminus (N(v)\cup v)}$ that fixes the value of all vertices except $N(v) \cup \{v\}$. We only need to consider the worst case of $\tau$. Note that
\begin{align*}
    \frac{w_\mu(\tau,v=+)}{w_\mu(\tau,v=-)} = \frac{\lambda_v^\mu}{\sum_{\rho \in \{\pm\}^{N(v)},w_\mu(\tau,\rho,-)>0 }\prod_{u\in N(v),\rho_u=+}\lambda_u^\mu}.
\end{align*}
It is easy to verify when $\tau = \tau_0$ such that for all $u \in {V\setminus (N(v)\cup v)}$, $\tau_0(u)= -$,
the above ratio obtains its minimum, because other $\tau$ may forbid some possible $\rho$ in the summation. Our algorithm for computing the value of $b$ is:
\begin{itemize}
    \item Compute $b_0=\frac{1}{1+\max_{v\in V}(\lambda_v^\mu)}$.
    \item For each $v\in V$, compute $m_v=\mu_{v}^{\tau_0}(+)$ by enumerating all independent sets of $N(v)$.
    \item Output $b=\min \{b_0,\min_{v\in V} \{m_v\}\}$.
\end{itemize}

By \cref{lem:const-degree}, because we already remove all vertices with zero $\lambda_v$, we have $N(v)\leq \frac{\ln(b)}{\ln (1-b)}$.
Let $k = \max_{v \in V}|N(v)| = O(\frac{1}{b}\log \frac{1}{b})$. 
The running time of the algorithm is 
\[O(n 2^k k^2) = \tp{\frac{1}{b}}^{O(\frac{1}{b})} n.\] 
The running time $O(k2^k)$ is for the exact computation of the $\mu^{\tau_0}_v(+)$. As stated in \Cref{remark:b}, given the approximate counting oracle for conditional distributions, we can compute an approximate value $p_v$ such that $\frac{1}{2} \mu^{\tau_0}_v(+) \leq p_v \leq \mu^{\tau_0}_v(+)$ with high probability.

For Ising model, similar to the pre-processing step, we can first remove all $v$ with $|h_v| = \infty$ and then change the external fields on all neighbors of $v$. After this step, we only need to consider a soft-Ising model  $(G,J,h)$ in the remaining graph $G$.

We also analyze when $\mu_v^\sigma(c)$ obtains the minimum. 
Since we deal with soft-Ising model, any $\sigma \in \{\pm\}^V$ appears with positive probability. Since $\mu_v^\sigma(c)$ is a convex combination of $\mu^\tau_v(c)$, where $\tau$ is a pinning on $V \setminus \{v\}$. Due to the conditional independence, if all $N(v)$ is fixed, then other vertices do not influence on the marginal distribution at $v$. Hence, to minimize $\mu_v^{\sigma}(c)$, we only need to consider $\sigma\in \{\pm\}^{N(v)}$. The marginal distribution can be written as
%For an arbitrary partial configuration $\sigma\in [q]^\Lambda$, where $\Lambda\subseteq V$, let $F = N(V)\setminus C$. We have
%\begin{align*}
%&\mu_v^{\sigma}(c)=\frac{\sum_{\tau\in\{\pm\}^F}w_{\mu}(\sigma,\tau,v=c) }{\sum_{\tau\in\{\pm\}^F}w_{\mu}(\sigma,\tau)} \text{, and}\\
%&\mu_v^{\sigma+\tau}(c) = \frac{w_{\mu}(\sigma,\tau,v=c)}{w_{\mu}(\sigma,\tau)}.
%\end{align*}
%To minimize $\mu_v^{\sigma}(c)$, we only need to find the minimum of $\mu_v^{\sigma+\tau}(c)$. If all $N(v)$ is fixed, then other vertices do not influence the value $\mu_v^{\sigma+\tau}(c)$. Our problem is simplified as minimizing $\mu_v^\sigma(c)$, where $\sigma\in \{\pm\}^{N(v)}$.
\begin{align*}
\frac{\mu_v^\sigma(c)}{\mu_v^\sigma(-c)}=\frac{\exp(\sum_{u\in N(v)}J_{vu}\sigma_u c+h_v c)}{\exp(-\sum_{u\in N(v)}J_{vu}\sigma_u c-h_v c)} =\exp\tp{2\sum_{u\in N(v)}J_{vu}\sigma_u c+2h_v c}.
\end{align*}
To find a $\sigma$ that minimizes $\sum_{u\in N(v)}J_{vu}^\mu\sigma_u c$, we greedily assign $\sigma_u \in \{\pm\}$ according to the sign of $J_{vu}$. The final result is for any $c \in \{\pm\}$,
\begin{align*}
\min_{\sigma\in \{\pm\}^{N(v)}} \mu_v^\sigma(c)=\frac{g(v,c)}{g(v,c)+1/g(v,c)}\text{, where } g(v,c)= \exp\tp{h_vc -\sum_{u\in N(v)}|J_{vu}|}.  
\end{align*}
Our algorithm is:
\begin{itemize}
    \item For each $v\in V$, compute $g(v) = \min_{c \in \{\pm\}}\frac{g(v,c)}{g(v,c)+1/g(v,c)}$.
    \item Output $b=\min_{v\in V} \{g(v)\}$.
\end{itemize}
The running time is $O(n+m)$.
\end{proof}


\ifthenelse{\boolean{conf}}{\begin{proof}\textbf{of \Cref{lem:par-hardcore}}}{\begin{proof}[Proof of \Cref{lem:par-hardcore}]}
For all $\sigma\in \{\pm\}^V$, if $\mu(\sigma)>0$, $\sigma$ corresponds to an independent set of $G$. Because $\nu$ is soft-hardcore, then $\nu(\sigma)>0$, so $\nu$ is absolutely continuous with respect to $\mu$. %Similarly, if $\nu(\sigma)>0$ for $\sigma\in \{\pm\}^V$, $\mu(\sigma)>0$.

For each $v\in V$, consider $\sigma = (-1)^{V\setminus v}$. Because $\mu$ and $\nu$ are both soft-hardcore models and satisfy the $b$-marginal lower bound, $\mu_{v}^\sigma(+),\mu_{v}^\sigma(-),\nu_{v}^\sigma(+),\nu_{v}^\sigma(-)\geq b$. We have 
$\frac{\lambda_u^x}{1+\lambda_u^x},\frac{1}{\lambda_u^x+1}\geq b$ for all $v\in V$, $x\in \{\mu,\nu\}$, which means
\begin{align*}
    \frac{b}{1-b}\leq \lambda_v^x\leq \frac{1-b}{b}.
\end{align*}
The above inequality means that $b\leq \frac{1}{2}$. We can compute the ratio of the weight
\begin{align*}
    \frac{w_\nu(\sigma)}{w_\mu(\sigma)}&=\prod_{v:\sigma(v)=+}\frac{\lambda_v^\nu}{\lambda_v^\mu}\leq \prod_{v:\sigma(v)=+} \frac{\lambda_v^\mu+\dis(\mu,\nu)}{\lambda_v^\mu}\\
    &\leq \left(\frac{\lambda_v^\mu+\dis(\mu,\nu)}{\lambda_v^\mu}\right)^n\leq \left(1+\frac{(1-b)\dis(\mu,\nu)}{b}\right)^n.
\end{align*}
For hardcore model, $\dis(\mu,\nu) \leq \theta=\frac{b}{2(1-b)n}$, so 
\begin{align*}
    \frac{w_\nu(\sigma)}{w_\mu(\sigma)}\leq 1+\frac{3n(1-b)\dis(\mu,\nu)}{b}.
\end{align*}
Similarly, $\frac{w_\nu(\sigma)}{w_\mu(\sigma)}\geq 1-\frac{n(1-b)\dis(\mu,\nu)}{b}$. By \Cref{lem:TV-lower}, $\DTV{\mu}{\nu} \geq \CC \dis(\mu,\nu)$, then
\begin{align*}
    \sqrt{\Var{W}}&\leq \frac{4n(1-b)\dis(\mu,\nu)}{b}\leq \frac{4n(1-b)}{b\CC}\DTV{\mu}{\nu} < \frac{4n}{b\CC}\DTV{\mu}{\nu} \text{, and}\\
    \E{W}&\geq \min_{\sigma} \frac{w_\nu(\sigma)}{w_\mu(\sigma)} \geq 1-\frac{1}{2}=\frac{1}{2}.
\end{align*}
This verifies \Cref{cond:meta}.
\end{proof}


\ifthenelse{\boolean{conf}}{\begin{proof} \textbf{of \Cref{lem:par-Ising}}}{\begin{proof}[Proof of \Cref{lem:par-Ising}]}
Since both $\mu$ and $\nu$ are soft-Ising models, the absolutely continuous condition $\nu \ll \mu$ holds.
    Let $J^\mu,h^\mu$ and $J^\nu,h^\nu$ be the interaction matrices and external field vectors of $\mu$ and $\nu$, respectively. Denote $D = \dis(\mu,\nu)$.
    For any $\sigma \in \{\pm\}^V$, we have
    \begin{align*}
        \frac{w_\nu(\sigma)}{w_\mu(\sigma)} = \exp\left( \sum_{u\in V} (h^\nu_u -h^\mu_u) \sigma_u + \sum_{\{u,v\}\in E} (J^\nu_{uv} - J^\mu_{uv})\sigma_u\sigma_v \right). 
    \end{align*}
    By the definition of parameter distance in \Cref{def:dis}, we have 

    \begin{align*}
        \left| \sum_{u\in V} (h^\nu_u -h^\mu_u) \sigma_u + \sum_{\{u,v\}\in E} (J^\nu_{uv} - J^\mu_{uv})\sigma_u\sigma_v \right| &\leq \sum_{u\in V}|h^\nu_u-h^\mu_u|+\sum_{\{u,v\}\in E} |J^\nu_{uv}-J^\mu_{uv}|
        \\
        &\leq \left(\sum_{u\in V} (\deg_u+1)+\sum_{\{u,v\}\in E}1 \right)D\\&=(n+3m)D,
    \end{align*}
    it implies
    %\todo{parameter $\theta$ changed}
    \begin{align*}
        \exp(-(n+3m)D) \leq \frac{w_\nu(\sigma)}{w_\mu(\sigma)} \leq \exp((n+3m)D).
    \end{align*}
    For soft-Ising model, $\theta = \frac{1}{2(n+3m)}$ and $D < \theta$, so that 
    \begin{align*}
       1 - (n+3m)D \leq \frac{w_\nu(\sigma)}{w_\mu(\sigma)} \leq 1 + 3(n+3m)D.
    \end{align*}
    Hence, $\sqrt{\Var{W}}\leq 4(n+3m)D \leq \frac{4(n+3m)}{\CC}\DTV{\mu}{\nu}$, and $\E[]{W} \geq 1 - \frac{1}{2} = \frac{1}{2}$.
\end{proof}



%Sketch of the proof.
%\begin{itemize}
%    \item Compute $b$;
%    \item Reduce general Ising to soft Ising; If there exists $v$ such that $h^\mu_v \neq h^\nu_v = \pm \infty$  then tv =1. If there exists $v$ such that $\lambda^\mu_v \pm \infty$ and $\lambda^\nu_v \neq \pm \infty$, then by marginal lower bound, $dtv \geq b$, using additive algorithm.
%    Otherwise, for all $v \in V$, if $\lambda^\mu_v = \pm \infty$ then $\lambda^\nu_v = \lambda^\mu_v$. One can remove all these vertices and change external fields of neighbors to obtain a soft Ising.
%    This soft Ising also have marginal lower bound $b$. We can use \Cref{lem:TV-lower}.
%    The sampling and counting oracle for input Ising also works for induced soft Ising. 
%\end{itemize}



\subsection{The improved algorithms for hardcore model in the uniqueness regime} \label{sec:hardcore-alg-proof}

\ifthenelse{\boolean{conf}}{\begin{proof}\textbf{of \Cref{thm:hardcore-1}}}{\begin{proof}[Proof of \Cref{thm:hardcore-1}]}
Consider a hardcore model $(G,\lambda)$, where $\lambda \in \mathbb{R}_{\geq 0}^V$, that satisfies the uniqueness condition.
Define threshold $\theta$ for hardcore model as 
\begin{align*}
  \theta = 10^{-10}\frac{\epsilon^{1/4}}{n^{5/2}} = \Theta\left(\frac{\epsilon^{1/4}}{n^{5/2}}\right). 
\end{align*}
If $\dis(\mu,\nu) > \theta$, then by \Cref{lem:TV-lower}, we know that $\DTV{\mu}{\nu} \geq \frac{\theta}{5000}$. 
we use the algorithm \Cref{thm:Approximate-Gibbs} to achieve the additive error $\frac{\epsilon D}{5000}$. 
For the hardcore model in the uniqueness regime, we have $\TS_G(\delta)=O_\eta(\Delta n \log \frac{n}{\delta})$ and $\TC_G(\delta) = \tilde{O}_\eta(\frac{\Delta n^2}{\delta^2})$. Note that $\TW_G = O(n)$. The running time for this case is at most 
\begin{align*}
 O\left(\TC_G\left(\frac{\epsilon D}{20000}\right)+ \frac{1}{\epsilon^2D^2}\left(n + \TS_G\left(\frac{\epsilon D}{20000}\right)\right)\right) = \tilde{O}_\eta\left( \frac{\Delta n^2}{\epsilon^2 \theta^2}\right) = \tilde{O}_\eta\left( \frac{\Delta n^7}{\epsilon^{5/2}}\right).
\end{align*}
If $\dis(\mu,\nu) \leq \theta$, we use \Cref{thm:hardcore-adv} with running time $\tilde{O}_\eta\left(\frac{n^7}{\epsilon^2}+\frac{n^{6.5}}{\epsilon^{9/4}}\right)$. The over all running time is the maximum of two cases, which is $\tilde{O}_\eta\left( \frac{\Delta n^7}{\epsilon^{5/2}}\right)$.
\end{proof}

The choice of $\theta$ is closely related to the choices of $t$ and $\kappa$ in the proof of \Cref{lem:hardcore-adv-1}. We choose the parameters to minimize the exponent on $n$ in the running time of \Cref{thm:hardcore-1}.


\ifthenelse{\boolean{conf}}{\begin{proof}\textbf{of \Cref{thm:hardcore-2}}}{\begin{proof}[Proof of \Cref{thm:hardcore-2}]}
    Now, we further assume that $\Delta = O(1)$ and $\lambda^\pi_v = \Omega(1)$ or $0$ for all $v\in V$ and $\pi \in \{\mu,\nu\}$.
    We do a similar pre-processing step as that in \Cref{sec:proof-gen}.
    Suppose there exists $v$ such that $(\lambda^\mu_v = \Omega(1),\lambda^\nu_v = 0)$ or $(\lambda^\mu_v = \Omega(1),\lambda^\nu_v = 0)$. Say we are in the first case. Then $\nu_v(+) = 0$ and $\mu_v(-) \geq \frac{\lambda^\mu_v}{1+\lambda^\mu_v}(\frac{1}{1+\lambda_c(\Delta)})^\Delta = \Omega(1)$. The total variation $\DTV{\mu}{\nu}=\Omega(1)$. We can use \Cref{thm:Approximate-Gibbs} to solve the problem in time $\tilde{O}(\frac{n^2}{\epsilon^2})$. For all $v \in V$ with $\lambda^\mu_v = \lambda^\nu_v=0$, we can remove $v$. Hence, we can assume $\Omega(1)=\lambda^\pi_v \leq (1 - \eta)\lambda_c(\Delta)$ for all $v \in V$ and $\pi \in \{\mu,\nu\}$.
    
    In this case, we use a different threshold $\theta_0 = \Theta(\frac{1}{\Delta n}) = \Theta(\frac{1}{n})$ because $\Delta=O(1)$. Suppose $\dis(\mu,\nu) \leq \theta_0$, then by~\eqref{eq:w-var}, we have $\Var[]{W} = O(n^2)$. It is easy to verify that $\frac{Z_\nu}{Z_\mu} = \Theta(1)$.  \Cref{thm:alg-main} gives an algorithm in time $\tilde{O}_\eta(n^3/\epsilon^2)$. Let us assume $D = \dis(\mu,\nu) > \theta_0$. If we directly apply \Cref{thm:Approximate-Gibbs} to achieve the additive error $\frac{\epsilon D}{5000}$, then the running time would be
    \begin{align*}
        O\left(\TC_G\left(\frac{\epsilon D}{20000}\right)+ \frac{1}{\epsilon^2D^2} \TS_G\left(\frac{\epsilon D}{20000}\right)\right).
    \end{align*}
    The second term $\frac{1}{\epsilon^2D^2} \TS_G\left(\frac{\epsilon D}{20000}\right)= \tilde{O}_\eta(n^3/\epsilon^2) $. But the bottleneck is the first term $\TC_G\left(\frac{\epsilon D}{20000}\right) = \tilde{O}_\eta(\frac{ n^2}{\epsilon^2D^2}) = \tilde{O}_\eta(\frac{ n^2}{\epsilon^2 \theta_0^2}) = \tilde{O}_\eta(n^4/\epsilon^2)$. However, we can improve the first term by noting that the algorithm in \Cref{thm:Approximate-Gibbs} only needs to approximate the ratio $\frac{Z_\nu}{Z_\mu}$ with relative-error $O(\epsilon D)$ and we show such ratio can be approximated in time $\tilde{O}_\eta(n^3/\epsilon^2)$. We can construct a sequence of $\lambda^{(0)},\lambda^{(1)},\cdots,\lambda^{(\ell)}$ such that  $\lambda^{(0)} = \lambda^\mu$, $\lambda^{(\ell)} = \lambda^{\nu}$ and other $\lambda^{(i)}$ are defined as follows. For any $v \in V$, let $\delta_v = \frac{\lambda^{\nu}_v}{\lambda^{\mu}_v}$. For any $1 \leq i \leq \ell$,  $\lambda^{(i)}_v$ is defined by $\lambda^{(i)}_v = \lambda^\mu_v \delta_v^{i/\ell}$.  We choose $\ell$ such that $\ell = \Theta(1+ nD)$. Note that $\delta_v  =1 \pm O(D)$. We have $\delta_v^{1/\ell} = 1 \pm O(\frac{1}{n})$.  Let $w_i$ be the weight function induced by $\lambda^{(i)}$. 
    Let $Z_i$ be the partition function induced by $w_i$. 
    Let $\mu_i$ be the Gibbs distribution induced by $w_i$.
    We further define $Z_{\ell + 1}$ by setting $\lambda^{(\ell + 1)}_v = \lambda^\mu_v \delta_v^{(\ell+1)/\ell}$. 
    Define random variable $W_i$ as 
    \begin{align*}
        W_i = \frac{w_{i}(X)}{w_{i-1}(X)}, \quad \text{where } X \sim \mu_{i-1}.
    \end{align*}
    Define $W \defeq \prod_{i=1}^{\ell}W_i$, where $W_i$'s are mutually independent. It is easy to verify that 
    \begin{align*}
    \E[]{W} = \frac{Z_\nu}{Z_\mu} = \frac{Z_\ell}{Z_0}, \quad\text{and } \Var[]{W} \leq \E[]{W^2} = \prod_{i=1}^{\ell}\E[]{W_i^2} = \frac{Z_{\ell + 1}Z_{\ell}}{Z_0Z_1}.
    \end{align*}
    We have the following bound
    \begin{align*}
     \frac{\Var[]{W}}{(\E[]{W})^2} \leq   \frac{\E[]{W^2}}{(\E[]{W})^2} & \leq \frac{Z_{\ell+1}}{Z_\ell} \cdot \frac{Z_0}{Z_1} = O(1).
    \end{align*}
    The last equality follows from the fact that $\delta_v^{1/\ell} = 1 \pm O(\frac{1}{n})$.
    Hence, to achieve $O(\epsilon D)$ relative-error, we can draw $O(\frac{1}{\epsilon^2 D^2})$ samples of $W$, each sample costs $\tilde{O}_\eta(n \ell) = \tilde{O}_\eta(n + n^2 D)$ time. The total running time is 
    \begin{align*}
        \tilde{O}_\eta\left(\frac{n + n^2 D}{\epsilon^2 D^2}\right) = \tilde{O}_\eta\left( \frac{n}{\epsilon^2 \theta_0^2} + \frac{n^2}{\epsilon^2 \theta_0} \right) = O_\eta\left( \frac{n^3}{\epsilon^2}\right). \ifthenelse{\boolean{conf}}{}{&\qedhere}&
    \end{align*}
    \end{proof}




\subsection{The algorithm for marginal distributions (Proof of \texorpdfstring{\Cref{thm:many-vertex-alg}}{Lg})}\label{sec:marginthm}

\Cref{thm:many-vertex-alg} is a simple corollary of \Cref{thm:approx-margin-tv}.
\ifthenelse{\boolean{conf}}{\begin{proof}\textbf{of \Cref{thm:many-vertex-alg}}}{\begin{proof}[Proof of \Cref{thm:many-vertex-alg}]}
For hardcore model $(G,\lambda^\mu)$, $Z^\sigma_{\mu}$ where $\sigma \in \{\pm\}^S$ is the partition function of $(G[\Lambda],\lambda_\Lambda^\mu)$. The set $\Lambda$ is obtained from $V$ by removing all vertices in $S$ together with all neighbors $u$ of vertices $v \in S$ such that $\sigma_v = +1$. If $(G, \lambda^\mu)$ satisfies the uniqueness condition, then $(G[\Lambda], \lambda_\Lambda^\mu)$ also satisfies the uniqueness condition. Hence, by the previous results in~\cite{CFYZ22,CE22} and \cite{SVV09}, for both $\mu$ and $\nu$, the approximate conditional counting oracle with $\TC_G(\epsilon)=O(\frac{\Delta n^2}{\epsilon^2} \mathrm{polylog}\frac{n}{\epsilon})$ exists and the sampling oracle with $\TS_G(\epsilon)=O(\Delta n \mathrm{polylog}\frac{n}{\epsilon})$ exists. The theorem follows from \Cref{thm:approx-margin-tv}.
\end{proof}

\section{Proofs of \#P-hardness results}\label{sec:hard}
In this section, we prove \Cref{thm:one-vertex} and \Cref{thm:many-vertex}. Our starting point is the \#P-hardness for exactly counting the number of independent sets in a graph.

\ifthenelse{\boolean{conf}}{\begin{proposition}[\text{\cite[Theorem 4.2]{DyerG00}}]}{\begin{proposition}[\text{\cite[Theorem 4.2]{DyerG00}}]}\label{prop:hard}
The following problem \#\textsf{Ind}(3) is \#P-complete.
\begin{itemize}
    \item Input: a graph $G=(V,E)$ with maximum degree $\Delta = 3$;
    \item Output: the exact number of independent sets in $G$.
\end{itemize}
\end{proposition}

%As a simple corollary, the following problem is  \#P-complete.
The above problem is exactly computing the partition function of $(G,\lambda)$ with $\lambda_v = 1$ for all $v \in V$.
%Also note that $1 < \lambda_c(3) = 4$ and the instance $(G,\lambda)$ satisfies the uniqueness condition in~\eqref{eq:cond-hardcore} with a constant gap $\eta = \frac{3}{4}$. We will show that if there exists a polynomial-time algorithm for problems in \Cref{thm:one-vertex} or \Cref{thm:many-vertex}, then the \#Ind(3) in \Cref{prop:hard} can be solved in polynomial time, which proves the hardness result. %We first prove \Cref{thm:one-vertex} and then extend the proof to \Cref{thm:many-vertex}.
%Let $G = (V,E)$ be a graph with maximum degree $\Delta = 3$. 
Let $n = |V|$ and $V = \{1,2,\ldots,n\}$. Let $\mu_{G,\bm{1}}$ denote the uniform distribution over all independent sets in $G$, which is the hardcore distribution in $G$ when $\lambda_v = 1$ for all $v \in V$. Define
\begin{align}\label{eq:def_pi}
    p_i = \Pr[X \sim \mu_{G,\bm{1}}]{X_i = 0 \mid \forall 1\leq j \leq i - 1, X_j = 0},
\end{align}
which is the probability that the vertex $i$ is not in a random independent set $X$ conditional on all $j < i$ not being in $X$. By definition, the total number of independent set is
\begin{align*}
    Z = \frac{1}{\mu_{G,\boldsymbol{1}}(\boldsymbol{0})} = \prod_{i=1}^n \frac{1}{p_i}.
\end{align*}
Suppose for any $i \in [n]$, we can compute $\hat{p}_i$ such that 
\begin{align}\label{eq:tar}
   (1 - 4^{-n})p_i \leq \hat{p}_i \leq (1+4^{-n})p_i.
\end{align}
Let $\hat{Z} = \prod_{i=1}^n \frac{1}{\hat{p}_i}$ and it holds that $(1-3^{-n})Z\leq \hat{Z} \leq (1+3^{-n})Z$. Note that $Z \leq 2^n$. We have
\begin{align*}
    |\hat{Z} - Z| \leq 3^{-n}Z \leq 1.5^{-n} < 0.01.
\end{align*}
We can round $\hat{Z}$ to the nearest integer to recover $Z$. Hence, $\#\text{Ind}(3)$ can be reduced to the following high-accuracy marginal estimation problem.
\begin{problem}\label{problem:mar} The high-accuracy marginal estimation problem is defined by
\begin{itemize}
    \item Input: a graph $G=(V,E)$ with $n$ vertices and maximum degree $\Delta = 3$;
    \item Output: $n$ numbers $(\hat{p}_i)_{i \in [n]}$ such that for all $ i \in [n]$, $(1 - 4^{-n})p_i \leq \hat{p}_i \leq (1+4^{-n})p_i$.
\end{itemize}
\end{problem}


%\subsection{Proof of \texorpdfstring{\Cref{thm:one-vertex}}{}}
\subsection{Hardness of approximating the TV-distance on a single vertex}
We first prove \Cref{thm:one-vertex}. Let $k(n) = 1$ for all $n \in \mathbb{N}$ be a constant function. We show that if there is a $\mathrm{poly}(n)$ time algorithm for \Cref{label:prob-mar} if the input error bound $\epsilon = \mathrm{poly}(n)$ and both two input hardcore models satisfy the uniqueness condition, then \Cref{problem:mar} can also be solved in $\mathrm{poly}(n)$ time. \Cref{thm:one-vertex} follows from \Cref{prop:hard}.

Fix an integer $i \in [n]$. Let $G_i$ denote the induced graph $G[S_i]$, where $S_i =\{j \in [n] \mid j \geq i\}$ is the set of vertices with label at least $i$. Let $\mu^{(i)}$ denote the uniform distribution over all independent set in graph $G_i$. In other words, $\mu^{(i)}$ is the Gibbs distribution of hardcore model $(G_i,\boldsymbol{1})$. Then $p_i$ in~\eqref{eq:def_pi} is the marginal distribution on vertex $i$ projected from $\mu^{(i)}$. If the maximum degree of $G_i$ is at most $2$, then $G_i$ is a set of disconnected lines or circles and $p_i$ can be computed exactly in polynomial time. We can assume the maximum degree of $G_i$ is $3$.  

Let $\alpha \geq 0$. Define vector $\lambda^\alpha \in \mathbb{R}^{S_i}$ by
\begin{align}\label{eq:def-lambda-alpha}
    \lambda_j^\alpha = \begin{cases}
        \frac{\alpha}{1 - \alpha} &\text{if } j = i; \\
        0 &\text{if } j \neq i.
    \end{cases}
\end{align}
Let $\nu^\alpha$ denote the Gibbs distribution of $(G_i,\lambda^\alpha)$. Note that $\lambda_c(3) = 4 > 1$. The following observation is easy to verify.
\begin{observation}\label{ob:uniq}
Both $\mu^{(i)}$ and $\nu^\alpha$ satisfies the uniqueness condition in~\eqref{eq:cond-hardcore} if $\alpha \leq \frac{1}{2}$.
\end{observation}


By the definition of $\nu^\alpha$, it is easy to see $\nu^\alpha_i(+1) = \alpha$  and
\begin{align*}
    \DTV{\mu^{(i)}_i}{\nu^\alpha_i} = \vert \mu^{(i)}_i(+1) - \alpha \vert = \vert p_i - \alpha \vert.
\end{align*}
%Since $\mu^{(i)}$ is a uniform distribution over all independent sets, we have $\mu^{(i)}_i(+1) \leq \frac{1}{2}$. 
Let $\+A(\alpha)$ be the algorithm such that given $\alpha \in [0,\frac{1}{2}]$, it returns a number $\hat{d}$ such that $ \frac{d_{\text{TV}}(\mu^{(i)}_i,\nu^\alpha_i)}{1+\epsilon}\leq \hat{d} \leq (1+\epsilon)d_{\text{TV}}(\mu^{(i)}_i,\nu^\alpha_i)$, where $\epsilon = \mathrm{poly}(n)$. By \Cref{ob:uniq}, if the polynomial-time algorithm for the problem in \Cref{thm:one-vertex} exists, then $\+A(\alpha)$ runs in $\mathrm{poly}(n)$ time.
We then can use the following algorithm to solve \Cref{label:prob-mar} for $p_i$ in $\mathrm{poly}(n)$ time. Thus, the hardness result in \Cref{thm:one-vertex} follows from \Cref{prop:hard}.

\ifthenelse{\boolean{conf}}{\begin{algorithm2e}[ht]}{
\begin{algorithm}[ht]
}\label{alg:high-TV}
    \caption{Algorithm for high-accuracy marginal estimation}
    Let $\alpha \gets \frac{1}{2}$ and $\epsilon = \mathrm{poly}(n)$ be the parameter assumed by algorithm $\+A$\;
    \For{$t$ from 1 to $50n (1+\epsilon)^2$}{
        $\hat{d} \gets \+A(\alpha)$\;
        $\alpha \gets \alpha - \hat{d}/(1+\epsilon)$\;
        if the bit length of $\alpha$ is more than $100n$, then round $\alpha$ up to the nearest number that has bit length at most $100n$\;
    }
    \Return $\hat{p}_i = \alpha$.
\ifthenelse{\boolean{conf}}{\end{algorithm2e}}{
\end{algorithm}
}
The above algorithm runs in $\mathrm{poly}(n)$ time. We show that the output $\hat{p}$ satisfies \eqref{eq:tar}.


Let $\alpha_t$ be the value of $\alpha$ after the $t$-th iteration. We first show that $p_i \leq \alpha_t$ for all $t$. 
 At the beginning, $\alpha_0 = 1/2$. Since $\mu^{(i)}$ is a uniform distribution over all independent sets, we have $\mu^{(i)}_i(+1) \leq \frac{1}{2}$. By the assumption of algorithm $\+A$, $\+A(\alpha_{t})/(1+\epsilon) \leq  d_{\text{TV}}(\mu^{(i)}_i,\nu^{\alpha_t}_i) = \alpha_t - p_i$. Hence, $\alpha_{t+1} \geq \alpha_t -  \+A(\alpha_{t})/(1+\epsilon) \geq p_i $. 
 
We next bound the value of $\alpha_t - p_i$. At the beginning, $\alpha_0 = \frac{1}{2}$ so that $\alpha_0 - p_i \leq \frac{1}{2}$. Note that $\alpha_{t+1} < \alpha_t -  \frac{\+A(\alpha_{t})}{1+\epsilon}+2^{-90n} \leq \alpha_t -  \frac{\alpha_t-p_i}{(1+\epsilon)^2}+2^{-90n} $, where $2^{-90n}$ is an upper bound of rounding error. The inequality implies that 
\begin{align*}
    \alpha_{t+1} - p_i \leq \left( 1 - \frac{1}{(1+\epsilon)^2} \right)(\alpha_t - p_i) + 2^{-90n}.
\end{align*}
Note that $\hat{p}_i = \alpha_{50n(1+\epsilon)^2}$.
Solving the recurrence implies that
\begin{align*}
    0 \leq \hat{p}_i - p_i \leq \exp\left( -\frac{50n(1+\epsilon)^2}{(1+\epsilon)^2} \right)\cdot \frac{1}{2} + (1+\epsilon)^2 \cdot 2^{-90n} < 2^{-40n}.
\end{align*}
Note that $p_i$ is at least $1/2^n$. Hence, the output $\hat{p}$ satisfies \eqref{eq:tar}.

\begin{remark*}
In the above proof, while the TV-distance between the two Gibbs distributions \(\mu^{(i)}\) and \(\nu^{\alpha}\) is large (because their parameter distance is 1), the TV distance between their marginal distributions at vertex \(i\) can be arbitrarily small. This highlights the distinction between the TV-distance of marginal distributions and the TV-distance of the entire distribution.
\end{remark*}


\subsection{Hardness of approximating the TV-distance on a subset of vertices}
We now prove \Cref{thm:many-vertex}. Let $G_i$ be the graph defined as above. 
Let $n_i = n - i + 1$ denote the number of vertices in $G_i$.
One can construct a graph $G_i'$ with by adding a set $\Lambda$ of $\ell$ isolated vertices to $G_i$.  Let $N = n_i + \ell$. Let $k(\cdot)$ be the function in \Cref{thm:many-vertex}. Note that $k(N) = N - \lceil N^\alpha \rceil$. Since $\alpha$ is a constant, one can set $\ell = n^{\Omega(1/\alpha)} = \mathrm{poly}(n)$ so that $k(N) \leq \ell$.
Hence, the size of $G_i'$ is a polynomial in $n$.

Let $\mu^{(i)}_{\text{new}}$ be the hardcore model on $G_i'$ such that the external fields on vertices in $G_i$ are 1 and the external fields on $\lambda$ are 0. Let $\nu^\alpha_{\text{new}}$ be the hardcore model on $G_i'$ such that the external fields on vertices in $G_i$ are $\lambda^\alpha$ and the external fields on $\Lambda$ are 0. Let $S$ be a subset of vertices containing vertex $i$ and $k(N) - 1$ vertices in $\Lambda$. It holds that 
\begin{align*}
 \DTV{\mu_{\text{new},S}^{(i)}}{\nu_{\text{new},S}^\alpha} = \DTV{\mu^{(i)}_i}{\nu^\alpha_i}.   
\end{align*}
In words, the total variation distance between marginal distributions on $S$ projected from $\mu^{(i)}_{\text{new}}$ and $\nu^\alpha_{\text{new}}$ is the same as the total variation distance between marginal distributions on vertex $i$ projected from $\mu^{(i)}$ and $\nu^\alpha$. Note that both $\mu^{(i)}_{\text{new}}$ and $\nu^\alpha_{\text{new}}$ satisfy the uniqueness condition. \Cref{thm:many-vertex} can be verified by going through the same reduction for \Cref{thm:one-vertex}.

\ifthenelse{\boolean{conf}}{}{
\section*{Acknowledgment}
Weiming Feng and Hongyang Liu gratefully acknowledge the support of ETH Z\"urich, where part of this work was conducted.  
Weiming Feng acknowledges the support of Dr.\ Max R\"ossler, the Walter Haefner Foundation, and the ETH Z\"urich Foundation during his affiliation with ETH Z\"urich.
}

