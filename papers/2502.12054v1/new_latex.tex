% This must be in the first 5 lines to tell arXiv to use pdfLaTeX, which is strongly recommended.
\pdfoutput=1
% In particular, the hyperref package requires pdfLaTeX in order to break URLs across lines.

\documentclass[11pt]{article}

% Change "review" to "final" to generate the final (sometimes called camera-ready) version.
% Change to "preprint" to generate a non-anonymous version with page numbers.
\usepackage[preprint]{acl}

% Standard package includes
\usepackage{times}
\usepackage{latexsym}
\usepackage{adjustbox}
\usepackage{enumitem}
\usepackage{booktabs}
% For proper rendering and hyphenation of words containing Latin characters (including in bib files)
\usepackage[T1]{fontenc}
% For Vietnamese characters
% \usepackage[T5]{fontenc}
% See https://www.latex-project.org/help/documentation/encguide.pdf for other character sets

% This assumes your files are encoded as UTF8
\usepackage[utf8]{inputenc}

% This is not strictly necessary, and may be commented out,
% but it will improve the layout of the manuscript,
% and will typically save some space.
\pdfoutput=1
\usepackage{microtype}
\usepackage{multicol}
\usepackage{multirow}
\usepackage{multirow}
\usepackage{makecell}
\usepackage{array}
\usepackage{booktabs}
\usepackage{algorithm}
\usepackage{algpseudocode}
\usepackage{amsmath}
\usepackage{colortbl}
% \usepackage{xcolor}
\usepackage{hhline}
% This is also not strictly necessary, and may be commented out.
% However, it will improve the aesthetics of text in
% the typewriter font.
\usepackage{inconsolata}

%Including images in your LaTeX document requires adding
%additional package(s)
\usepackage{graphicx}
\usepackage{amssymb}
\usepackage{pifont}
% \usepackage[table]{xcolor}
% If the title and author information does not fit in the area allocated, uncomment the following
%
%\setlength\titlebox{<dim>}
%
% and set <dim> to something 5cm or larger.

% \title{PhysReason: A Multi-step Physics Problem-Solving Benchmark with Structured Solution Assessment}
% \title{PhysReason: A Multi-step Physics Problem-Solving Benchmark with Step-level Solution Evaluation}
\title{PhysReason: A Comprehensive Benchmark towards \\ Physics-Based Reasoning}
% \title{PhysReason: How Far Are We from Physics-Based Reasoning \\ in Large Language Models?}

% Author information can be set in various styles:
% For several authors from the same institution:
% \author{Author 1 \and ... \and Author n \\
%         Address line \\ ... \\ Address line}
% if the names do not fit well on one line use
%         Author 1 \\ {\bf Author 2} \\ ... \\ {\bf Author n} \\
% For authors from different institutions:
% \author{Author 1 \\ Address line \\  ... \\ Address line
%         \And  ... \And
%         Author n \\ Address line \\ ... \\ Address line}
% To start a separate ``row'' of authors use \AND, as in
% \author{Author 1 \\ Address line \\  ... \\ Address line
%         \AND
%         Author 2 \\ Address line \\ ... \\ Address line \And
%         Author 3 \\ Address line \\ ... \\ Address line}
\author{Xinyu Zhang$^{1}$,\; Yuxuan Dong$^{1}$,\; Yanrui Wu$^{1}$,\; Jiaxing Huang$^{1}$,\;  Chengyou Jia $^{1}$,\; \\ \textbf{Basura Fernando} $^{3}$,\; \textbf{Mike Zheng Shou} $^{2}$,\;  \textbf{Lingling Zhang}$^{1}$\footnotemark[2],\; \textbf{Jun Liu}$^{1}$
% ffiliation / Address line 1 \\
\\
$^{1}$Xi’an Jiaotong University \; \\
$^{2}$Show Lab, National University of Singapore \; \\
$^{3}$Institute of High-Performance Computing, A*STAR \; \\
\texttt{zhang1393869716}@stu.xjtu.edu.cn
}
% \author{First Author \\
%   Affiliation / Address line 1 \\
%   Affiliation / Address line 2 \\
%   Affiliation / Address line 3 \\
%   \texttt{email@domain} \\\And
%   Second Author \\
%   Affiliation / Address line 1 \\
%   Affiliation / Address line 2 \\
%   Affiliation / Address line 3 \\
%   \texttt{email@domain} \\}

%\author{
%  \textbf{First Author\textsuperscript{1}},
%  \textbf{Second Author\textsuperscript{1,2}},
%  \textbf{Third T. Author\textsuperscript{1}},
%  \textbf{Fourth Author\textsuperscript{1}},
%\\
%  \textbf{Fifth Author\textsuperscript{1,2}},
%  \textbf{Sixth Author\textsuperscript{1}},
%  \textbf{Seventh Author\textsuperscript{1}},
%  \textbf{Eighth Author \textsuperscript{1,2,3,4}},
%\\
%  \textbf{Ninth Author\textsuperscript{1}},
%  \textbf{Tenth Author\textsuperscript{1}},
%  \textbf{Eleventh E. Author\textsuperscript{1,2,3,4,5}},
%  \textbf{Twelfth Author\textsuperscript{1}},
%\\
%  \textbf{Thirteenth Author\textsuperscript{3}},
%  \textbf{Fourteenth F. Author\textsuperscript{2,4}},
%  \textbf{Fifteenth Author\textsuperscript{1}},
%  \textbf{Sixteenth Author\textsuperscript{1}},
%\\
%  \textbf{Seventeenth S. Author\textsuperscript{4,5}},
%  \textbf{Eighteenth Author\textsuperscript{3,4}},
%  \textbf{Nineteenth N. Author\textsuperscript{2,5}},
%  \textbf{Twentieth Author\textsuperscript{1}}
%\\
%\\
%  \textsuperscript{1}Affiliation 1,
%  \textsuperscript{2}Affiliation 2,
%  \textsuperscript{3}Affiliation 3,
%  \textsuperscript{4}Affiliation 4,
%  \textsuperscript{5}Affiliation 5
%\\
%  \small{
%    \textbf{Correspondence:} \href{mailto:email@domain}{email@domain}
%  }
%}

\begin{document}
\maketitle
\begin{abstract}
Large language models demonstrate remarkable capabilities across various domains, especially mathematics and logic reasoning.
However, current evaluations overlook physics-based reasoning, a complex task requiring physics theorems and constraints.
We present PhysReason, a 1,200-problem benchmark comprising knowledge-based (25\%) and reasoning-based (75\%) problems, where the latter are divided into three difficulty levels (easy, medium, hard).
Notably, problems require an average of 8.1 solution steps, with hard problems requiring 15.6, reflecting the complexity of physics-based reasoning.
We propose the Physics Solution Auto Scoring Framework, incorporating efficient answer-level and comprehensive step-level evaluations.
Top-performing models like Deepseek-R1, Gemini-2.0-Flash-Thinking, and o3-mini-high achieve less than 60\% on answer-level evaluation, with performance dropping from knowledge questions (75.11\%) to hard problems (31.95\%).
Through step-level evaluation, we identify four key bottlenecks: Physics Theorem Application, Physics Process Understanding, Calculation, and Physics Condition Analysis.
These findings position PhysReason as a novel and comprehensive benchmark for evaluating physics-based reasoning capabilities in large language models.
Our code and data will be published at \url{https://dxzxy12138.github.io/PhysReason/}.
% Our evaluation shows even top models (Deepseek-R1, Gemini-2.0-Flash-Thinking, and o3-mini-high) score below 60\%.
% Furthermore, performance declines from knowledge (75.11\%) to hard problems (31.95\%), indicating performance degradation with increasing difficulty.
% We identify four key bottlenecks: Physics Theorem Application, Physics Process Understanding, Calculation, and Physics Condition Analysis. 
% We identify four key bottlenecks: Physics Theorem Application, Process Understanding, Calculation, and Condition Analysis. 
% These findings position PhysReason as a novel and robust benchmark for evaluating physics-based reasoning capabilities in large language models.
% We present PhysReason, a 1,200-problem benchmark comprising knowledge-based (25\%) and multi-level reasoning-based (75\%), and the latter are divided into three difficulty levels (easy, medium, hard). 
% Notably, problems require 8.06 solution steps on average, with hard-reasoning requiring 15.57, reflecting the complexity of physics-based reasoning.
% Notably, difficult-reasoning problems require 15.57 solution steps on average, reflecting physics-based reasoning complexity.
% We propose dual evaluation frameworks: an efficient answer-level evaluation and a comprehensive step-by-step reasoning process evaluation.
% Performance drops from knowledge-based (75.11\%) to difficult problems (31.95\%), showing degradation in deeper reasoning.
% Through systematic analysis, we identify four key bottlenecks (Physics Theorem Application, Physics Process Understanding, Calculation, and Physics Condition Analysis), establishing PhysReason as a new standard for evaluating physics-based reasoning capabilities in language models.
% Additionally, we propose dual evaluation frameworks: answer-level and step-level assessment. 
% The former provides efficient evaluation through answer comparison, while the latter enables comprehensive analysis through step-by-step reasoning verification.
% Our evaluation of 15 models on PhysReason reveals that even top performers (Deepseek-R1, Gemini-2.0-Flash-Thinking-0121, and o3-mini-high) achieve scores below 60\%. 
% Furthermore, model performance declines sharply from knowledge-based (75.11\%) to difficult-reasoning problems (31.95\%), indicating significant performance degradation with increasing reasoning depth.
% Through systematic analysis, we identify four key bottlenecks (Physics Theorem Application, Physics Process Understanding, Calculation, and Physics Condition Analysis), establishing PhysReason as a new standard for evaluating physics-based reasoning capabilities in language models.
\end{abstract}
% While Large Language Models (LLMs) have demonstrated remarkable capabilities in mathematical computation, code generation, and logical reasoning, these abilities alone cannot capture the complexity of real-world physics-based reasoning. 
% Unlike pure symbol manipulation tasks, physics-based reasoning demands a deeper understanding of natural phenomena, requiring the integration of observations, theoretical principles, and contextual factors through intricate chains of reasoning. 
% Large language models have demonstrated remarkable capabilities across domains, particularly in abstract reasoning, such as mathematics. 
% However, they exhibit significant limitations in real-world interactions due to incomplete physics understanding.
% While current evaluations predominantly focus on mathematical and logical reasoning, they often overlook physics-based reasoning - a complex process requiring the integration of multiple principles and real-world constraints.
% We introduce PhysReason, a novel benchmark specifically designed for physics-based reasoning evaluation.
% Unlike existing physics benchmarks that typically oversimplify physics phenomena into basic 3-4 step solutions and lack step-by-step reasoning evaluation, PhysReason offers substantial improvements.
% The benchmark comprises knowledge-based (25\%) and reasoning-based (75\%) problems, with the latter categorized into easy, medium, and difficult levels.
% Notably, difficult-reasoning problems require an average of 15.57 steps to solve, accurately reflecting real-world physics complexity.
% Additionally, we propose the Physics Solution Auto-Scoring (PSAS) framework, which enables step-by-step evaluation of reasoning processes and provides error analysis.
% Our evaluation reveals a significant performance disparity between knowledge-based problems (75.11\%) and difficult-reasoning problems (31.95\%).
% The analysis reveals substantial deficiencies in LLMs' abilities to comprehend physics scenarios and apply physics theorems, highlighting critical limitations in their physics-based reasoning capabilities.
% Large language models (LLMs) have demonstrated remarkable capabilities across domains, particularly in abstract reasoning. 
% However, they exhibit limitations in real-world interactions due to incomplete physics understanding. 
% While current evaluations predominantly focus on mathematical and logical reasoning, they overlook physics-based reasoning - a complex process that requires integrating multiple principles and real-world constraints.
% % However, existing physics benchmarks, while covering various knowledge levels, fail to assess the multi-step reasoning processes essential for real-world problem-solving, often reducing complex phenomena to oversimplified 3-4 step solutions. 
% We present PhysReason, a novel benchmark specifically designed to evaluate reasoning depth in physics. 
% However, existing physics benchmarks, while covering various knowledge levels, fail to assess the multi-step reasoning processes essential for real-world problem-solving, often reducing complex phenomena to oversimplified 3-4 step solutions. 
% Our benchmark features sophisticated problems requiring up to 44 solution steps, incorporates rich visual representations in 81\% of problems, and implements a systematic difficulty stratification across both knowledge-based (25\%) and reasoning-based (75\%) problems. 
% Through our innovative Physics Solution Auto-Scoring (PSAS) framework, we enable detailed analysis of model performance at each reasoning step. 
% Our evaluation reveals a striking performance gap between knowledge-based (75.11\%) and complex reasoning tasks (31.95\%), highlighting critical limitations in current LLMs' physics-based reasoning capabilities. 
% While Large Language Models (LLMs) excel in various reasoning tasks, their ability to perform complex, multi-step physics-based reasoning remains largely unexplored. 
% Current evaluations predominantly focus on mathematical, code, and logical reasoning, neglecting the unique demands of physics, which requires interpreting real-world phenomena, building physics models, and integrating multiple principles.
% Existing physics benchmarks, though spanning different knowledge levels, oversimplify the reasoning process and primarily assess final answers, hindering in-depth performance analysis.
% We introduce PhysReason, a comprehensive benchmark designed to address this gap. 
% It features a difficulty stratification, problems requiring up to 44 solution steps, rich visual information, and a novel Physics Solution Auto-Scoring (PSAS) framework for detailed, step-by-step analysis. 
% Our evaluation reveals significant performance disparities and identifies critical reasoning failure types, providing insights into LLMs' strengths and limitations in physics-based reasoning. 
% This work offers a new benchmark, a solution analysis framework and a quantitative comparison of models in physics-based reasoning.
% \end{abstract}
% Large Language Models show strong mathematical and coding abilities, yet their capacity to solve physics problems—combining conceptual and mathematical reasoning—remains poorly assessed.
% Large Language Models have demonstrated remarkable capabilities in mathematical reasoning and code generation, yet their ability to solve physics problems—which demand integration of conceptual understanding and mathematical reasoning—remains inadequately assessed by existing benchmarks.
% Large Language Models demonstrate remarkable capabilities in mathematical reasoning and code generation. 
% However, their ability to solve physics problems—which demand integration of conceptual understanding and mathematical reasoning—remains inadequately assessed by existing benchmarks.
% We present PhysReason, a comprehensive benchmark framework for evaluating multi-step physics problem-solving capabilities of these models. To address current limitations, PhysReason introduces four key innovations: 
% (1) a balanced distribution of knowledge-based (25\%) and reasoning-based (75\%) questions, (2) a structured step-by-step solution assessment framework capturing physics problem complexity through problems averaging 2.61 sub-questions,
% (3) extensive multimodal integration in 81\% of problems through diagrams, 
% and (4) a comprehensive error analysis system for reasoning failures. Our systematic evaluation reveals significant performance disparities between simple and complex problems (73.3\% vs 25.05\% accuracy), with o-like models demonstrating superior capabilities. 
% Our assessment reveals seven key reasoning failures in physics problem-solving, offering insights to improve future AI systems' physics capabilities.
% Through our assessment framework, we identify seven types of reasoning failures in integrating physics principles and maintaining solution paths. 
% These findings provide insights into current models' limitations and suggest directions for enhancing physics-based reasoning capabilities in future AI systems.
% \end{abstract}

\section{Introduction}
Large Language Models (LLMs) have demonstrated remarkable performance across various domains, such as math \cite{lightmanlet, cobbe2021training} and logical reasoning \cite{hendrycksmeasuring, xu2025large}. 
However, current evaluations often overlook physics-based reasoning, limiting their applications in scenarios such as robotics \cite{chow2025physbench} and autonomous driving \cite{huang2023applications}. 
This is because physics-based reasoning, integrating multiple theorems and physics constraints, is more closely aligned with practical applications than math and logical reasoning. 
Consequently, developing a comprehensive benchmark for evaluating LLMs' physics-based reasoning capabilities is crucial for discovering current limitations and guiding future improvements.
% Large Language Models (LLMs) have demonstrated remarkable performance across various domains, such as math \cite{lightmanlet, cobbe2021training} and logical reasoning \cite{hendrycksmeasuring, xu2025large}. 
% However, when applied to physics-world tasks like robotics \cite{gao2024physically, chow2025physbench}, and autonomous driving \cite{huang2023applications}, LLMs show significant limitations. 
% This gap emerges because physics-world scenarios require physics-based reasoning that integrates multiple theorems and physics constraints - a challenge distinct from mathematical or logical reasoning. 
% Consequently, developing a comprehensive benchmark for evaluating LLMs' physics-based reasoning capabilities is crucial for discovering current limitations and guiding future improvements.
% Large Language Models (LLMs) have demonstrated remarkable performance across various domains, such as math \cite{lightmanlet, cobbe2021training} and logical reasoning \cite{hendrycksmeasuring, xu2025large}. 
% However, LLMs currently struggle with applications in practical scenarios, such as robotics \cite{gao2024physically, chow2025physbench} and autonomous driving \cite{huang2023applications}. 
% This is because practical scenarios require physics-based reasoning that integrates multiple theorems and physics-world constraints, which is fundamentally different from math and logical reasoning. 
% Therefore, a benchmark to comprehensively evaluate LLM's physics-based reasoning capabilities is vital in improving its capabilities.
% Large Language Models (LLMs) have demonstrated remarkable performance across various domains, such as math \cite{lightmanlet, cobbe2021training} and logical reasoning \cite{hendrycksmeasuring, xu2025large}. 
% However, current evaluations often overlook physics-based reasoning, limiting their applications in practical scenarios such as robotics \cite{gao2024physically, chow2025physbench} and autonomous driving \cite{huang2023applications}. 
% This is because physics-based reasoning, integrating multiple theorems and physics-world constraints, is more closely aligned with practical applications than math and logical reasoning. 
% Therefore, a benchmark to comprehensively evaluate LLM's physics-based reasoning capabilities is vital in improving its capabilities.
% Large Language Models (LLMs) have demonstrated remarkable performance across various domains.
% However, current assessments of LLMs' reasoning capabilities primarily focus on math \cite{lightmanlet, cobbe2021training} and logical reasoning \cite{hendrycksmeasuring, xu2025large}, while overlooking physics-based reasoning.
% Physics-based reasoning capabilities are fundamental to unlocking LLMs' potential in practical applications, such as robotics \cite{gao2024physically, chow2025physbench} and autonomous driving \cite{huang2023applications}.
% This is because physics-based reasoning, integrating multiple theorems and physics-world constraints, is more closely aligned with practical applications than math and logical reasoning.
% Therefore, a benchmark to comprehensively evaluate LLM's physics-based reasoning capabilities is vital in improving its capabilities.
% Therefore, there is an urgent need to comprehensively evaluate LLMs' physics-based reasoning capabilities.
% Large Language Models (LLMs) have demonstrated remarkable capabilities across various domains \cite{jain2024livecodebench, jia2024chatgen}, yet they face significant limitations in robotics \cite{gao2024physically} and autonomous driving \cite{huang2023applications}. 
% This limitation stems from their incomplete understanding of physical principles and inability to perform reliable multi-step reasoning in real-world scenarios. 
% Enhanced physics-based reasoning capabilities would enable LLMs to better interact with the physical world and make safer decisions in dynamic environments.
% \par
% However, current LLM evaluations primarily focus on abstract reasoning, such as mathematics \cite{lightmanlet, cobbe2021training} and logic \cite{hendrycksmeasuring, xu2025large}, while overlooking physics-based reasoning.
% physics-based reasoning involves complex chains of inference that integrate multiple principles and consider real-world constraints, distinct from abstract reasoning.
\par
There are several pioneering physics benchmarks (K-12 level like ScienceQA \cite{lu2022learn}, college-level SciBench \cite{wangscibench}, and expert-level GPQA \cite{rein2024gpqa}) encompassing progressively advanced knowledge domains.
However, they exhibit two critical limitations: oversimplified reasoning processes and neglecting step-level evaluation.
These problems typically involve only 3-4 physics formulas, focusing solely on final answers to measure model performance.
Therefore, a benchmark featuring in-depth reasoning processes and step-level evaluation is urgently needed to measure LLMs' physics-based reasoning capabilities.
% Therefore, there is an urgent need for a benchmark featuring in-depth reasoning processes and step-level evaluation to measure LLMs' physics-based reasoning capabilities.
\par
\begin{figure*}[t]
\centering
\includegraphics[width=0.96\textwidth]{fig/example.pdf}
\vspace{-6pt}
\caption{An illustration of example from our PhysReason benchmark. Due to space constraints, only key components are shown. Please refer to Appendix \ref{Example} for complete annotations.}
\label{fig:0}
\vspace{-15pt}
\end{figure*}
\par
To address these limitations, we present PhysReason, a comprehensive benchmark comprising 1,200 problems designed to evaluate models' physics-based reasoning capabilities. 
As illustrated in Figure \ref{fig:0}, PhysReason features physics problems that require multi-step reasoning and precise application of physics theorems. 
The benchmark introduces several key characteristics:
\begin{enumerate}[leftmargin=*,itemsep=0pt,parsep=0pt,topsep=0pt]
\item \textbf{Stratified difficulty}: There are knowledge-based (25\%) and reasoning-based (75\%) problems, with reasoning problems categorized into easy, medium, and hard (25\% each).
% reasoning problems categorized into three progressive difficulty levels.
\item \textbf{Complex reasoning}: Solutions average 8.1 steps per problem, with hard problems reaching 15.6 steps, exceeding current physics benchmarks which typically only contain 3-4 steps.
\item \textbf{Multi-modal design}: 81\% of problems include diagrams, evaluating models' capabilities in comprehending visual and textual information.
% reflecting physics tasks that demand visual and textual reasoning.
\end{enumerate}
\par
% To comprehensively evaluate performance on the PhysReason benchmark, we propose the Physics Solution Auto Scoring Framework based on current LLMs' capabilities in information extraction and formula comparison. 
To evaluate performance on PhysReason comprehensively, we propose the Physics Solution Auto Scoring Framework (PSAS) based on current LLMs' capabilities in information extraction and formula comparison. 
% This framework encompasses two evaluation methods at the answer-level and step-level, named PSAS-A and PSAS-S respectively.
This framework encompasses two answer-level and step-level evaluation methods, PSAS-A and PSAS-S.
PSAS-A enables efficient evaluation through answer comparison, while PSAS-S facilitates comprehensive analysis through step-by-step reasoning verification. 
Experimental results demonstrate that PSAS significantly outperforms direct LLM-based evaluation approaches, achieving an evaluation accuracy exceeding 98\%.
% Experimental results demonstrate that PSAS significantly outperforms direct LLM-based evaluation approaches.
% PSAS significantly outperforms direct LLM-based evaluation approaches.
\par
% o3-mini-high \cite{o3_mini}, Gemini-2.0 Flash Thinking \cite{google_gemini_thinking}, Deepseek-R1 \cite{guo2025deepseek}
We evaluate seven non-O-like models and eight O-like models on the PhysReason benchmark. 
Results show that while Deepseek-R1 \cite{guo2025deepseek}, Gemini-2.0-Flash-Thinking-0121 \cite{google_gemini_thinking}, and o3-mini-high \cite{o3_mini} demonstrate superior performance, their average scores remain below 60\%. 
Moreover, models excel in basic physics concepts but consistently show performance degradation as problem difficulty and required solution steps increase (from 75.11\% to 31.95\%). 
This degradation stems from the models' inability to maintain accuracy across consecutive solution steps, so maintaining the reliability of the reasoning process is crucial. 
Through step-level evaluation, we identify four critical bottlenecks limiting model performance: Physics Theorem Application, Physics Process Understanding, Calculation Process, and Physics Condition Analysis.
% While Deepseek-R1 \cite{guo2025deepseek}, Gemini-2.0-Flash-Thinking-0121 \cite{google_gemini_thinking}, and o3-mini-high \cite{o3_mini} demonstrate superior performance, their average scores all remain below 60\%. 
% Our experiments reveal that models excel in basic physics concepts but consistently show performance degradation as problem difficulty and required solution steps increase (75.11\% to 31.95\%). 
% This degradation stems from models' inability to maintain accuracy across consecutive solution steps.
% This degradation stems from models' inability to maintain accuracy across consecutive solution steps, despite their strong initial reasoning capabilities.
% We identify four critical bottlenecks limiting model performance: Physics Theorem Application, Physics Process Understanding, Calculation Process, and Physics Condition Analysis. 
% In conclusion, PhysReason establishes new standards for evaluating and advancing AI models' physics-based reasoning capabilities.
% Models often initiate reasoning processes well but struggle to maintain accuracy across consecutive solution steps. 
% We select seven non-o-like models and eight o-like models for the experiment.
% The experiment results reveal significant performance disparities between knowledge and reasoning-difficult problems (75.11\% vs 31.95\% accuracy), highlighting the increasing challenges models face with complex reasoning tasks.
% Through our PSAS framework's automated error analysis, we identify Physics Theorem Application, Physics Process Understanding, and Calculation Process as the most frequent sources of errors, revealing their distinct limitations in physics-based reasoning.
% \begin{figure*}[t]
%     \centering
%     \includegraphics[width=0.98\textwidth]{fig/pipeline.pdf}
%     % \vspace{-10pt}
%     \caption{Illustration of the data collection pipeline.
% }
%     \label{fig:10}
% \vspace{-10pt}
% \end{figure*}
\section{Related Work}
\subsection{Large Language Model Evaluation}
LLMs have demonstrated remarkable performance across various domains, such as math reasoning \cite{jiang2024forward, li2024snapkv, imani2023mathprompter}, logical reasoning \cite{sun2024determlr, xu-etal-2024-symbol}, and text generation \cite{zhao2024docmath, liang2024controllable}.
However, they struggle with physics world interactions, limiting their adoption in areas like autonomous driving and robotics \cite{gao2024physically}.
Unlike mathematical and logical reasoning, physics-based reasoning requires the integration of multiple principles and physics-world constraints \cite{kline1981mathematics}.
Therefore, mastering physics-based reasoning is fundamental to unlocking LLMs' potential in practical applications \cite{lai2024vision}.
Current evaluations primarily focus on mathematical and logical reasoning, revealing a crucial gap in evaluating LLM capabilities based on physics-based reasoning.
% However, they struggle with physics world interactions, limiting their adoption in robot learning and autonomous driving.
% Unlike mathematical reasoning, physics-based reasoning requires integration of principles and physics-world constraints \cite{kline1981mathematics}.
% Mastering physics-based reasoning is key to unlocking LLMs' potential in practical applications \cite{lai2024vision}.
% Current evaluations focus on mathematical and logical reasoning, revealing a crucial gap in physics-based reasoning assessment of LLMs.
% This misalignment reveals a critical gap in LLM capability assessment.
% physics-based reasoning mastery is key to advancing LLMs in real-world applications \cite{lai2024vision}.
% Large Language Models (LLMs) have achieved remarkable success across various domains, particularly excelling in mathematical reasoning \cite{jiang2024forward, imani2023mathprompter}, logical reasoning \cite{sun2024determlr, xu-etal-2024-symbol}, text generation \cite{zhao2024docmath, liang2024controllable}.
% These models, while proficient in abstract thinking and instruction execution, face significant limitations in physics world interactions.
% physics-based reasoning demands multiple skills: interpreting data, building theoretical models, applying Physics Theorems, and considering contextual factors \cite{kline1981mathematics}.
% There exists a notable disparity between current LLM evaluation methods, which predominantly focus on abstract reasoning, and the comprehensive requirements of physics-based reasoning.
% This misalignment highlights a crucial gap in our assessment of LLM capabilities.
% Mastering physics-based reasoning, is fundamental to unlocking LLMs' potential across practical applications and bridging the gap between computational intelligence and real-world problem-solving \cite{lai2024vision}.
% \newcommand{\ccline}[2]{%
%   \cline{#1}\morecmidrules\cline{#2}%
% }
\subsection{Physics Benchmarks}
Existing physics benchmarks span three knowledge complexity levels: K-12 (ScienceQA \cite{lu2022learn}, E-EVAL \cite{hou-etal-2024-e}), college-level (MMLU \cite{hendrycksmeasuring}, AGIEval \cite{zhong2024agieval}, JEEBench \cite{arora2023have}, TheoremQA \cite{chen2023theoremqa}, EMMA \cite{hao2025can}, SciEval \cite{sun2024scieval}, C-Eval-STEM \cite{huang2024c}, SciBench \cite{wangscibench}), and expert-level (OlympiadBench\cite{he-etal-2024-olympiadbench}, GPQA \cite{rein2024gpqa}).
% (1) K-12 benchmarks like ScienceQA \cite{lu2022learn} and E-EVAL \cite{hou-etal-2024-e} focusing on basic concepts; (2) college-level benchmarks such as MMLU-STEM \cite{hendrycksmeasuring}, AGIEval \cite{zhong2024agieval}, JEEBench \cite{arora2023have}, TheoremQA \cite{chen2023theoremqa}, EMMA \cite{hao2025can} and C-Eval-STEM \cite{huang2024c}, covering multistep problems; and (3) expert-level benchmarks including SciBench \cite{wangscibench}, OlympiadBench\cite{he-etal-2024-olympiadbench} and GPQA \cite{rein2024gpqa} testing doctoral-level physics. 
While these benchmarks showcase LLMs' knowledge breadth, they simplify reasoning to 3-4 steps and emphasize only final answers.
PhysReason addresses these gaps through complex reasoning process and step-level evaluation.
\begin{table*}
\centering
\caption{Comparative analysis of our PhysReason with other physics-based reasoning benchmarks.
For \textbf{Knowledge}, COMP: Competition, COL: College, CEE: College Entrance Examination, K1-K12: Elementary and High School, PH.D: Doctor of Philosophy;
% For answer type, Num: Numeric value, Exp: Expression, Equ: Equation, Opt: Option;
For \textbf{question type}, OE: Open-ended, MC: Multiple-choice, Avg. T: Average Tokens;
For \textbf{solution type}, Avg. S: Average Steps.}
\vspace{-5pt}
\begin{adjustbox}{width=\textwidth}
\begin{tabular}{lcccccc@{\hspace{5pt}}ccc}
% \begin{tabular}{lccccccccc}
\hline
\multirow{2}{*}{Benchmark} & \multirow{2}{*}{Multi-modal} & \multirow{2}{*}{Size} & \multirow{2}{*}{Knowledge} & \multicolumn{2}{c}{Question} & \multicolumn{3}{c}{Solution} \\
\cmidrule(r){5-6} \cmidrule(l){7-9}
  &        &           &      & Type & Avg. T & Step-by-step & Avg. T  & Avg. S\\
\hline
JEEBench       & {\color{red}\ding{55}}  & 123  & CEE & OE,MC & 169.7 & -  & - & -\\
MMLU-Pro       &  {\color{red}\ding{55}} & 1299 & COL & MC    & 52.1  & -   & - & -\\
% AGIEval        &              & 200  & CEE & OE,MC & 100.38 & -   & - & -\\
GPQA           & {\color{red}\ding{55}} & 227  & PH.D. & OE    & 111.4 & {\color{red}\ding{55}} & 197.2 & 3.6\\
SciEval        & {\color{red}\ding{55}} & 1657 & - & OE,MC & 154.5 & -  & - & -\\
SciBench       & \color{green}\ding{51} & 295  & COL & OE    & 80.5  & {\color{red}\ding{55}} & 315.9 & 2.8\\
MMMU           & \color{green}\ding{51} & 443  & COL & OE,MC & 53.8  & - & - & -\\
ScienceQA      & \color{green}\ding{51} & 617  & K1-K12 & MC    & 13.3  & {\color{red}\ding{55}} & 63.0 & 2.4\\
OlympiadBench  & \color{green}\ding{51} & 2334 & COMP & OE    & 222.0 & {\color{red}\ding{55}} & 199.8 & 3.7\\
EMMA           & \color{green}\ding{51} & 156  & - & MC    & 109.5 & -  & - & -\\
\hline
Ours-Knowledge  & \color{green}\ding{51} & 300  & CEE+COMP & OE    & 163.7 & \color{green}\ding{51} & 196.5 & 3.3\\
Ours-Easy       & \color{green}\ding{51} & 300  & CEE+COMP & OE    & 171.2 & \color{green}\ding{51} & 241.5 & 5.0 \\
Ours-Medium     & \color{green}\ding{51} & 300  & CEE+COMP & OE    & 229.2 & \color{green}\ding{51} & 391.3 & 8.4\\
Ours-Hard  & \color{green}\ding{51} & 300  & CEE+COMP & OE    & 340.9 & \color{green}\ding{51} & 936.1 & 15.6\\
\hline
\rowcolor{gray!20} Ours-Full  & \color{green}\ding{51} & 1200  & CEE+COMP & OE    & 226.3 & \color{green}\ding{51} & 441.3 & 8.1\\
\hline
\end{tabular}
\end{adjustbox}
\label{tab:0}
\vspace{-15pt}
\end{table*}
% These benchmarks effectively demonstrate the expanding knowledge scope of LLMs.
% However, these benchmarks typically oversimplify reasoning to 3-4 steps and focus on answers, making it hard to diagnose models' reasoning failures.
% Our PhysReason benchmark addresses these limitations through comprehensive reasoning chains and step-by-step evaluation.
% \begin{table*}
% \centering
% \begin{adjustbox}{width=\textwidth}
% % \small
% \begin{tabular}{lcccccccccc}
% \hline
% \multirow{2}{*}{Benchmark} & Multi- & \multirow{2}{*}{Size} & \multirow{2}{*}{Konwledge} & \multicolumn{2}{c}{Question} & \multirow{2}{*}{Answer type} & \multicolumn{3}{c}{Solution} \\
% \cline{5-6} \cline{8-10}
% & modal & &  & Type & Avg. T &   & Type & Avg. T   & Avg. S\\
% \hline
% SciBench       & $\checkmark$ & 295  & COL & OE    & 80.51  & Num & WP & 315.85 & 2.79\\
% MMMU           & $\checkmark$ & 443  & COL & OE,MC & 53.82  & Num & - & - & -\\
% ScienceQA      & $\checkmark$ & 617  & K1-K12 & MC    & 13.31  & Opt & WP & 62.95 & 2.40\\
% SciEval        &              & 1657 & - & OE,MC & 154.47 & Num,Opt & -  & - & -\\
% JEEBench       &              & 123  & CEE & OE,MC & 169.69 & Num,Opt & -  & - & -\\
% % MMLU           &              & 411  & COL & MC    & 44.85  & Opt & -   & - & -\\
% MMLU-Pro       &              & 1299 & COL & MC    & 52.08  & Opt & -   & - & -\\
% AGIEval        &              & 200  & CEE & OE,MC & 100.38 & Num & -   & - & -\\
% OlympiadBench  & $\checkmark$ & 2334 & COMP & OE    & 222.03 & Num,Equ,Exp & WP & 199.82 & 3.72\\
% GPQA           &              & 227  & PH.D. & OE    & 111.41 & Num,Equ & WP & 197.17 & 3.58\\
% EMMA           & $\checkmark$ & 156  & - & MC    & 109.47 & Opt & -  & - & -\\
% \hline
% PhysReason-Knowledge  & $\checkmark$ & 300  & CEE+COMP & OE    & 163.68 & Num, Equ, Exp & SBS & 196.48 & 3.31 \\
% PhysReason-Easy       & $\checkmark$ & 300  & CEE+COMP & OE    & 171.21 & Num, Equ, Exp & SBS & 241.52 & 5.00 \\
% PhysReason-Medium     & $\checkmark$ & 300  & CEE+COMP & OE    & 229.19 & Num, Equ, Exp & SBS & 391.28 & 8.39\\
% PhysReason-hard  & $\checkmark$ & 300  & CEE+COMP & OE    & 340.94 & Num, Equ, Exp & SBS & 936.06 & 15.57\\
% \hline
% PhysReason  & $\checkmark$ & 1200  & CEE+COMP & OE    & 226.25 & Num, Equ, Exp & SBS & 441.34 & 8.06\\
% \hline
% \end{tabular}
% \end{adjustbox}
% \vspace{-5pt}
% \caption{Comparison of our benchmark with other physics benchmarks. 
% For difficulty level, COMP: Competition, COL: College, CEE: College Entrance Examination, K1-K12: Elementary and High School, PH.D: Doctor of Philosophy;
% For answer type, Num: Numeric value, Exp: Expression, Equ: Equation, Opt: Option;
% For question type, OE: Open-ended, MC: Multiple-choice;
% For solution type, WP: Whole Paragraph, SBS: Step by step, Avg. T means Average Tokens, Avg. S means Average Steps.}
% \label{tab:0}
% \vspace{-15pt}
% \end{table*}
% \vspace{-5pt}
\section{Benchmark}
\subsection{Collection}
\label{sec:3-1}
We describe our comprehensive data collection process that spans five key stages: \textbf{Acquisition}, \textbf{Standardization}, \textbf{Translation}, \textbf{Search Prevention}, and \textbf{Difficulty Classification}.
\par
\textbf{Acquisition:}
We collect public physics problems from global college entrance examinations, their associated practice tests, and international physics competitions. 
Our sources include Chinese, Indian, and Russian exams, as well as IPhO, APhO, EPhO, and so on.
This comprehensive benchmark derives from 1,254 PDFs containing over 20,000 unique problems, ensuring diverse difficulty levels.
\par
\textbf{Standardization:}
Using MinerU \cite{wang2024mineruopensourcesolutionprecise} framework, we parse the content of these PDFs into structured problem information.
Subsequently, all problems undergo rigorous deduplication, filtering, and formatting to ensure complete problem statements, precise physics terms, accurate expressions, and consistent presentation style.
\par
\textbf{Translation:} 
We implement a two-phase translation process utilizing translators for initial conversion.
Engineering Ph.D. candidates with physics expertise then verify the translations for accuracy and professionalism, especially physics terms.
\par
\textbf{Search Prevention:} 
Following \cite{rein2024gpqa}, we exclude problems whose solutions and answers can be found through a five-minute Google search to minimize potential data leakage.
\par
\textbf{Difficulty Classification:} 
Based on the time students typically need to solve problems and the theorems applied, questions are categorized into knowledge-based and reasoning-based types, with the latter subdivided into three difficulty levels (easy, medium, and hard). 
This classification enables the comprehension evaluation of physics concepts and physics-based reasoning capabilities.
\subsection{Annotation}
\label{sec:3-2}
As shown in Figure \ref{fig:0}, our annotation framework consists of 8 key elements: \textbf{Diagram}, \textbf{Context}, \textbf{Sub-questions}, \textbf{Solution}, \textbf{Step Analysis}, \textbf{Answer}, \textbf{Theorem}, and \textbf{Difficulty}.
\textbf{Context} presents the physics scenario and conditions.
\textbf{Diagram} visualizes the physics scenario with concise illustrations complementing the \textbf{Context}.
\textbf{Sub-questions} give questions to assess the understanding and application of the concept.
\textbf{Solution} provides a step-by-step reasoning process, and \textbf{Answer} gives the answer to each sub-question.
\textbf{Step Analysis} explains the physics theorem used in each step and the physics quantities obtained.
\textbf{Theorem} lists the physics theorems applied in the question, and \textbf{Difficulty} indicates the difficulty classification.
\begin{figure*}[t]
\centering
\includegraphics[width=0.98\textwidth]{fig/analysis_subplots_v2.png}
\vspace{-10pt}
\caption{Analysis of solution theorems, solution steps, and solution tokens across different problem categories,  with comparisons from SciBench, GPQA, and OlympiadBench.}
\label{fig:11}
\vspace{-15pt}
\end{figure*}
\subsection{Characteristics}
\label{sec:3-3}
PhysReason consists of 1,200 carefully curated physics problems as shown in Table \ref{tab:0}, with a strategic composition of 25\% knowledge-based and 75\% reasoning-based questions across various difficulty levels, collectively covering 147 physics theorems.
The problems span Classic Mechanics, Quantum Mechanics, Fluid Mechanics, Thermodynamics, Electromagnetics, Optics, and Relativity.
As shown in Figure \ref{fig:11}, three critical solution metrics (theorem, step, and token) correlate positively with problem difficulty levels, validating the rationality of our difficulty classification.
Notably, the medium and hard problems demonstrate higher complexity compared to existing benchmarks.
PhysReason distinguishes itself through three key characteristics:
\begin{enumerate}[leftmargin=*,itemsep=0pt,parsep=0pt,topsep=0pt]
\item \textbf{Stratified difficulty}: 
The benchmark maintains a carefully balanced composition of knowledge-based (25\%) and reasoning-based (75\%) problems. 
The reasoning-based problems are methodically distributed across three difficulty levels (easy, medium, and hard - 25\% each), enabling comprehensive capability evaluation.
\item \textbf{Complex reasoning}: 
Detailed step-by-step solution annotations accompany each problem.
These annotated solutions demonstrate complex reasoning chains averaging 8.1 steps per problem, with hard problems requiring up to 15.6 steps, significantly surpassing the complexity of existing physics-based reasoning benchmarks.
\item \textbf{Multi-modal design}: The benchmark features a high proportion (81\%) of problems with  diagrams, authentically replicating physics-based reasoning scenarios while effectively evaluating both textual and visual reasoning capabilities.
\end{enumerate}
\begin{algorithm*}
\small
\caption{Physics Solution Auto Scoring Framework-Step Level (PSAS-S)}
\begin{algorithmic}[1]
\State \textbf{Phase 1: Data Extraction} \Comment{Extract and normalize solution steps from model output}
    \State \textit{Input}: Model output $M$, Annotation solution steps $S = \{s_1, s_2, ..., s_n\}$, Annotation step formulas $F = \{f_1, f_2, ..., f_n\}$,
    Annotation step values $V = \{v_1, v_2, ..., v_n\}$ \Comment{Information needed for step-level evaluation}
    \For{$s_i \in S$}
        \State $E_i \gets\text{LLM}(\text{ExtractTemplate}(M, s_i))$ \Comment{E: extracted relevant steps}
    \EndFor
    \State Assert $|E| = |S|$ \Comment{Ensure one-to-one mapping between extracted and annotated steps}
\State \textbf{Phase 2: Scoring} \Comment{Evaluate formula application and numerical calculations}
    \For{$(e_i, s_i) \in (E, S)$}
        \State $\hat{f}_i \gets \text{ExtractFormula}(e_i)$ \Comment{Formula content}
        \State $\hat{v}_i \gets \text{ExtractValue}(e_i)$ \Comment{Calculation target}
        \State ${score}_i \gets 0.5 \times \text{ScoreFormula}(\hat{f}_i, f_i) + 0.5 \times \text{ScoreValue}(\hat{v}_i, v_i)$ \Comment{Get final score}
    \EndFor
\State ${final\_score} \gets \frac{\sum_{i=1}^n{{{score}_i}}}{n}$ \Comment{Get the final score with the step-level evaluation}
\State \textbf{Phase 3: First Error Step Detection} \Comment{Identify the earliest point of solution deviation}
    \State $first\_error\_step \gets \infty$ \Comment{Initialization}
    \For{$i \gets 1$ to $|S|$}
        \If{${score}_i < 1$}
            \State $error\_step \gets \text{FindOriStep}(M, e_i)$ with the relationship between $E$ and $M$\Comment{Find corresponding original step}
            \State $first\_error\_step \gets \text{min}(first\_error\_step, error\_step)$ \Comment{Get the minimum}
            % \State \textbf{break}
        \EndIf
    \EndFor
\State \textbf{Phase 4: Error Analysis} \Comment{Analyze the first error step}
    \State $\text{ErrorTypes } \mathcal{T} = \{\text{DAE}, \text{PTAE}, \text{PCAE}, \text{PPUE}, \text{VRE}, \text{CPE}, \text{BCAE} \}$ \Comment{Error categories}
    \If{$first\_error\_step < \infty$}
    \State $j \gets first\_error\_step$
        \State $error\_type \gets \text{LLM}(\text{ClassificationTemplate}(e_{j}, s_{j}, \mathcal{T}))$ \Comment{Identify error type}
        \State $error\_analysis \gets \text{LLM}(\text{AnalysisTemplate}(e_{j}, s_{j}))$ \Comment{Generate error analysis}
    \EndIf
    \State \textit{Output}: ${final\_score}$, $first\_error\_step$, $error\_type$, $error\_analysis$
\end{algorithmic}
\label{alg:scoring}
\end{algorithm*}
\section{Evaluation Framework}
\subsection{Why LLMs Can Evaluate?}
Unlike multiple-choice problems, PhysReason contains open-ended answers and steps with diverse expressions but consistent semantics.
Given that LLMs have demonstrated exceptional capabilities in both precise content extraction and formula consistency evaluation \cite{2023opencompass, gao2024omni}, they serve as practical tools for automated physics solution evaluation.
Therefore, we propose automated answer-level and step-level evaluations, achieving comprehensive evaluation and avoiding labor-intensive manual assessment.
\subsection{How Answer-Level Evaluation Works?}
\label{Answer-Level Evaluation}
We develop Physics Solution Auto Scoring Framework-Answer Level (PSAS-A), which evaluates based on sub-question answers.
Given a model's reasoning process $M$ for a problem with sub-questions $\{q_1, q_2, \ldots, q_n\}$, we first extract the model's answers $\hat{a}_{i}$ for each $q_{i}$ from $M$ with an LLM.
Then, we employ the LLM to verify if $\hat{a}_{i}$ is semantically consistent with the standard answer $a_{i}$ of sub-question $q_i$.
The comparison function $C(\hat{a}_{i}, a_{i})$ returns 1 if consistent and 0 otherwise.
Considering that the sub-questions with different steps should not carry equal weights in scoring, we use the length of annotation solution $s_i$ of sub-question $q_i$, i.e., $|(s_i)|$ as a weighting scalar.
The model's reasoning process $M$'s answer-level score for each problem is calculated as follows:
\begin{equation}
\text{Score}(M) = \frac{\sum_{q_i} |(s_i)| \times C(\hat{a}_{i}, a_{i})}{\sum_{q_i} |(s_i)|}
\end{equation}
\subsection{How Step-Level Evaluation Works}
\label{Step-Level Evaluation}
The current mainstream evaluation approach \cite{he-etal-2024-olympiadbench} with LLMs relies on answers, failing to reveal how and where models deviate from correct reasoning paths. 
To address this, we propose the Physics Solution Auto-Scoring Framework-Step Level (PSAS-S), which enables detailed assessment and analysis of each reasoning step.
The framework is divided into four phases: \textbf{Data Extraction}, \textbf{Scoring}, \textbf{First Error Step Detection}, \textbf{Error Analysis}, as detailed in Algorithm~\ref{alg:scoring}.
\par
\textbf{Data Extraction} phase leverages LLMs using \textit{Target} components from \textit{Step Analysis} annotations (Figure \ref{fig:0}) as prompts to locate and extract relevant content from model outputs for each annotated solution step $s_i$.
This phase effectively handles redundant thinking processes in LLM's reasoning process while maintaining semantic equivalence.
It obtains the mapping relationship between extracted relevant steps $E$ and annotated solution steps $S$.
\par
\textbf{Scoring} phase evaluates each step $s_i$ through two complementary components of theorem assessment $\textit{ScoreFormula}(\hat{f}_i, f_i)$ and result verification $\textit{ScoreValue}(\hat{v}_i, v_i)$, each with a weight of 0.5.
The final score is calculated as shown in Algorithm~\ref{alg:scoring}.
This ensures a balanced assessment of theorem application and computational accuracy.
\par
\begin{figure}[t]
\vspace{-10pt}
\centering
\includegraphics[width=0.48\textwidth]{fig/evaluation_case.pdf}
\vspace{-20pt}
\caption{Step-level evaluation example obtained from PSAS-S framework.}
\label{fig:20}
\vspace{-20pt}
\end{figure}
\par
\textbf{First Error Step Detection} phase identifies the earliest step of deviation from the correct solution path.
When any step is found with a score below $1$, \text{FindOriStep} function locates the corresponding original step in the model's raw output based on the mapping relationship between $E$ and $S$ obtained from the \textbf{Data Extraction} phase, and updates $first\_error\_step$ to maintain the earliest error position.
This enables precise identification of where the model's reasoning first goes wrong. 
\par
\textbf{Error Analysis} phase analyzes the first error step detected in the solution, with two components: error classification and error analysis. 
For error classification, PSAS-S considers seven types of common errors: Diagram Analysis Error (DAE), Physics Theorem Application Error (PTAE), Physics Condition Analysis Error (PCAE), Physics Process Understanding Error (PPUE), Variable Relationship Error (VRE), Calculation Process Error (CPE), and Boundary Condition Analysis Error (BCAE). 
Detailed error-type descriptions are available in the Appendix \ref{Error Type Details}.
LLMs use structured prompts to identify the error type for the first error step. 
Then, a comprehensive error analysis is generated to explain the reasoning behind the mistake. 
A simplified example is shown in Figure \ref{fig:20}.
\begin{table}[t]
\vspace{-10pt}
\caption{Comparison between PSAS framework and direct use of LLM evaluation, where Answer Acc. denotes the accuracy of answer-level evaluation and Step Acc. indicates the precision in identifying the initial error step in the reasoning process.}
\vspace{-5pt}
\begin{adjustbox}{width=\columnwidth}
% \small
\centering
\begin{tabular}{lcc}
\hline
Model & Answer Acc. & Step Acc. \\
\hline
Gemini-2.0-Flash & 87.81 & 33.18 \\
Deepseek-V3 & 89.78 & 34.45\\
Gemini-2.0-Flash-Thinking-0121 & 91.24 & 35.74 \\
Deepseek-R1 & 93.31 & 37.54\\
\hline
Our (Gemini-2.0-Flash) & 98.96 & 97.23 \\
Our (Deepseek-V3) & 99.35 & 98.04 \\
\hline
\end{tabular}
\end{adjustbox}
\label{tab:5}
\vspace{-15pt}
\end{table}
\subsection{Whether Evaluation Trustworthy?}
\label{Framework Accuracy Analysis}
To validate the reliability of both our PSAS-A and PSAS-S, we compare our PSAS against conventional direct LLM evaluation approaches at both answer-level and step-level, using the Chain-of-Thought reasoning strategy.
We implement experiments using Deepseek-V3 and Gemini-2.0-Flash as scoring models in the following experiments:
\begin{enumerate}[leftmargin=*,itemsep=0pt,parsep=0pt,topsep=0pt]
\item 
For answer-level evaluation, we employ scoring models to assess answer correctness by combining both model-generated outputs and annotation answers.
We then compare these results with the judgments obtained from PSAS-A.
\item For step-level evaluation, inspired by previous work \cite{zheng2024processbench}, we design the task of identifying the first error step in reasoning processes containing errors, where higher accuracy indicates a more precise evaluation of the reasoning process.
Then, we submit both model-generated and annotated reasoning processes to scoring models to determine the location of the first error step, comparing with PSAS-S. 
\end{enumerate}
\par
Then, we collect 8,400 reasoning processes generated from multiple advanced models, including {Deepseek-R1}, {Gemini-2.0-Flash}, {Gemini-2.0-Flash-Thinking-0121}, {GLM-Zero}, {o1-mini}, {QwQ-32B}, and {QvQ-72B}.
Subsequently, we randomly sample 1,000 reasoning processes and meticulously manually annotate them to determine the correctness of each answer and identify the location of the first error step.
The results presented in Table \ref{tab:5} demonstrate that our frameworks achieve superior performance compared to direct LLM evaluation, highlighting the accuracy and reliability of PSAS evaluation results on PhysReason.
\par
\section{Experiments}
\begin{table*}[t]
\centering
% \small
\caption{Model performance comparisons on the PhysReason benchmark using answer-level (left of /) and step-level (right of /) evaluations across different input combinations of Questions (Q), Images (I), and Image Captions (IC).
Gemini-2.0-T$^{\dagger}$ and $^{*}$ represent Gemini-2.0-Flash-Thinking-1206 and 0121.}
\vspace{-5pt}
\begin{adjustbox}{width=\textwidth}
\begin{tabular}{lcccccc}
\hline
Model & Input & Knowledge & Easy & Medium & Hard & Avg. \\
\hline
\rowcolor{gray!20} \multicolumn{7}{c}{\textbf{Non-O-like Models}} \\
Qwen2VL-72B & Q, I & 41.92/62.47 & 24.04/45.26 & 15.97/36.13 & 4.83/24.23 & 16.96/42.88\\
InternVL2.5-78B & Q, I & 28.34/64.71 & 24.16/50.69 & 17.72/38.56 & 9.71/25.95 & 19.98/45.89\\
GPT-4o & Q, I & 50.71/65.82 & 33.87/51.98 & 22.73/42.36 & 11.03/24.71 & 29.58/47.23\\
Deepseek-V3-671B & Q, IC & 55.86/66.14 & 40.06/52.77 & 26.63/44.02 & 13.73/26.87 & 34.07/48.42\\
Claude-3.5-Sonnet & Q, I & 54.14/66.45 & 41.35/55.85 & 28.14/44.86 & 15.11/28.51 & 34.69/49.88\\
Gemini-2.0-Flash & Q, I & 65.08/75.04 & 54.84/68.60 & 39.79/55.67 & 21.99/38.39 & 45.20/60.40 \\
Gemini-2.0-Pro & Q, I & 67.99/79.01 & 55.43/71.47 & 44.29/57.74 & 23.81/42.66 & 47.88/62.74 \\
\hline
\rowcolor{gray!20} \multicolumn{7}{c}{\textbf{O-like Models}} \\
o1-mini & Q, IC & 53.90/65.74 & 35.21/52.26 & 22.24/40.19 & 10.61/26.80 & 30.49/47.18\\
QvQ-72B & Q, I & 62.44/70.92 & 53.74/64.65 & 28.18/54.88 & 14.30/36.47 & 32.67/57.66\\
Gemini-2.0-T$^{\dagger}$ & Q, I & 65.35/77.20 & 51.89/67.49 & 44.43/58.95 & 27.14/45.48 & 47.20/63.07 \\
QwQ-32B & Q, IC & 62.03/76.28 & 54.92/71.08 & 43.64/62.14 & 22.99/42.19 & 45.89/63.87 \\
GLM-Zero & Q, IC & 64.95/80.36 & 54.11/71.54 & 41.32/63.67 & 23.04/47.46 & 46.52/65.76\\
o3-mini-high & Q, IC & 70.67/83.61 & 67.20/81.95 & 45.31/64.57 & 30.12/47.23 & 53.32/69.34\\
Gemini-2.0-T$^{*}$ & Q, I & 73.44/84.15 & 63.17/75.94 & 50.41/66.60 & 31.90/48.47 & 54.73/69.73 \\
Deepseek-R1 & Q, IC & 75.11/85.91 & 65.08/79.81 & 54.84/72.02 & 31.95/51.50 & 56.75/73.26 \\
\hline
\end{tabular}
\end{adjustbox}
\label{tab:3}
\vspace{-10pt}
\end{table*}
\begin{table}[t]
\caption{Comparison on PhysReason-mini with PSAS-A, where Gemini-2.0-T$^{\dagger}$ and $^{*}$ represent Gemini-2.0-Flash-Thinking-1206 and 0121. And K., E., M. and H. represent knowledge, easy, medium and hard.}
\vspace{-5pt}
\begin{adjustbox}{width=\columnwidth}
% \small
\begin{tabular}{lccccc}
\hline
Model & K. & E. & M. & H. & Avg. \\
\hline
o1-mini & 54.80 & 30.33 & 15.41 & 7.92 & 27.11 \\
QvQ-72B & 51.17 & 37.10 & 29.83 & 22.13 & 35.06 \\
QwQ-32B & 64.40 & 50.07 & 38.88 & 27.45 & 45.20 \\
Gemini-2.0-T$^{\dagger}$ & 71.47 & 49.97 & 36.83 & 22.97 & 45.42 \\
GLM-Zero & 72.70 & 50.17 & 43.42 & 24.70 & 47.75 \\
% GLM-Zero & 71.47 & 49.97 & 36.83 & 22.97 & 45.42 \\
% Gemini2.0 Thinking-1206 & 72.70 & 50.17 & 43.42 & 24.7 & 47.75 \\
o1 & 72.47 & 53.37 & 49.31 & 25.32 & 50.12 \\
o3-mini-high & 71.10 & 63.20 & 47.02 & 31.93 & 53.31\\
Gemini-2.0-T$^{*}$ & 76.33 & 56.87 & 51.85 & 32.61 & 54.42 \\
Deepseek-R1 & 85.17 & 60.77 & 47.24 & 33.23 & 56.60 \\
\hline
\end{tabular}
\end{adjustbox}
\label{tab:4}
\vspace{-20pt}
\end{table}
% \vspace{-10pt}
\subsection{Setting}
\textbf{Baselines:}
We evaluate current mainstream open-source and closed-source LLMs, VLMs, and several o-like models. 
For models that cannot accept visual inputs, we use Gemini-2.0-Flash to generate captions for each image as supplementary information.
We assess 15 advanced LLMs/VLMs under the zero-shot Chain-of-Thought (CoT) setting (encouraging models to think step by step), including 7 non-O-like models (Qwen2-VL-72B \cite{wang2024qwen2}, GPT-4o \cite{openai_gpt4}, Claude-3.5-Sonnet \cite{anthropic_claude}, InternVL2.5-78B \cite{chen2024expanding}, Deepseek-v3 \cite{deepseekai2024deepseekv3technicalreport}), Gemini-2.0-Flash \cite{google_gemini2}, Gemini-2.0-Pro \cite{google_gemini2_pro} and 8 O-like models  (QwQ-32B \cite{qwq-32b-preview}, QvQ-72B \cite{qvq-72b-preview}, o1-mini \cite{o1_mini}, o1 \cite{openai2024learning}, o3-mini-high \cite{o3_mini}, Gemini-2.0-Flash-Thinking \cite{google_gemini_thinking}, Deepseek-R1 \cite{guo2025deepseek}, GLM-Zero \cite{glm_zero}).
Note that Gemini-2.0-Flash-Thinking has two versions: 1206 and 0121.
Due to API limitations, we do not experiment with o1 on the entire dataset. 
All other models are evaluated on the complete benchmark.
\par
\textbf{Evaluation Workflow:}
We encourage models to generate reasoning processes step by step for all problems in PhysReason, with open-source models running on NVIDIA A800 GPUs.
Please refer to Appendix-\ref{evaluation_prompt} for the detail prompt template.
Then, we evaluate the models' performance with the PSAS framework at both the answer and step levels, as described in Sections \ref{Answer-Level Evaluation} and \ref{Step-Level Evaluation}.
Based on the experimental results in Section \ref{Framework Accuracy Analysis}, considering both efficiency and performance, we select Deepseek-V3 as the final scoring model.
\par
\textbf{PhysReason-mini:}
Considering that the complete PhysReason requires relative high evaluation costs, we create a balanced PhysReason subset - PhysReason-mini.
We randomly sample 200 questions from the whole benchmark (50 for each difficulty level), striving to achieve equal representation across categories wherever possible.
\subsection{Main Results}
As demonstrated in Tables \ref{tab:3} and \ref{tab:4}, the experimental results on the PhysReason and PhysReason-mini reveal several significant findings.
\par
\textbf{Model Categories:}
O-like models exceed non-O-like ones, with multiple O-like models surpassing 50\% answer-level accuracy compared to non-O-like models' peak of 47.88\%.
\par
\textbf{Difficulty Level Analysis:}
As the difficulty increases, the required solution steps also increase, while model performance severely declines, indicating that models still perform inadequately on physics problems requiring deep reasoning.
\par
\textbf{Step-level vs. Answer-level Evaluation:}
The two evaluation frameworks assess performance from different perspectives.
Step-level scores consistently surpass answer-level scores, indicating that models can achieve some correct steps despite failing to reach the correct final answer.
Moreover, the step-level score differences between models become more pronounced than those at the answer level as problem difficulty increases.
This demonstrates that step-level evaluation proves more discriminative in distinguishing model capabilities, particularly in highly challenging problems.
The distributions of these two evaluation methodologies exhibit non-perfect synchronization, indicating that step-level evaluation provides comprehensive insights to answer-level assessment.
\par
\textbf{Medium and Hard Problem Analysis:}
Performance on medium and hard reasoning problems can emerge as key differentiators of model physics-based reasoning ability.
Among these models, those achieving scores of 40/60 and 30/50 on answer-level and step-level evaluations respectively serve as critical reference points.
\par
\textbf{Knowledge-Reasoning Correlation Analysis:}
Results show a positive correlation between physics knowledge and reasoning capabilities, with Deepseek-R1 and Gemini-2.0-Flash-Thinking-0121 excelling in both aspects. 
Moreover, among models with similar scores on knowledge problems, O-like models tend to achieve higher scores on reasoning problems (as demonstrated by Gemini-2.0-Flash and Gemini-2.0-T$^{\dagger}$).
This suggests that reinforcement learning and training with thought chains help improve models' reasoning capabilities.
In conclusion, effective reasoning relies on knowledge capacity and model architecture.
\begin{table}[t]
\caption{Test-Time Compute Scaling Performance Comparisons on PhysReason-mini with PSAS-A, where Flash and Think denote Gemini-2.0-Flash and Gemini-2.0-Flash-Thinking-0121, and Tour. means Tournament.}
\vspace{-5pt}
\begin{adjustbox}{width=\columnwidth}
% \small
\centering
\begin{tabular}{lllcccc}
\toprule
{Base} & {Method} & {Reward} & N=1 & N=2 & N=4 & N=8 \\
\midrule
\multirow{4}{*}{Flash} 
    & \multirow{2}{*}{BoN} & Flash & 46.52 & 46.67 & 47.12 & 47.81\\
    & & Think & 46.52 & 47.37 & 48.87 & 50.94 \\
    \cmidrule{2-7}
    & \multirow{2}{*}{Tour.} & Flash & 46.52  & 45.87 & 47.36 & 49.58\\
    & & Think & 46.52 & 47.51 & 52.11 & 53.06\\
\midrule
\multirow{2}{*}{Think} 
    & BoN & Think & 54.42 & 52.27 & 54.78 & 55.13\\
    & Tour. & Think & 54.42 & 55.60 & 56.26 & 56.57\\
\bottomrule
\end{tabular}
\end{adjustbox}
\label{tab:10}
\vspace{-15pt}
\end{table}
\subsection{Results with Test-Time Compute Scaling}
We evaluate Best-of-N (BoN) and Tournament-Style selection \cite{snell2025scaling, yang2024qwen2} test-time compute scaling methods on PhysReason-mini. 
Using Gemini-2.0-Flash and Gemini-2.0-Flash-Thinking-0121 as base models, we test different reward model configurations: when Flash serves as base model, both itself and Thinking-0121 are evaluated as reward models, while Thinking-0121 uses self-reward due to its superior reasoning. 
Both methods \cite{cobbe2021training, lightmanlet, son2024varco} select optimal responses from multiple Chain-of-Thought candidates (N = 1, 2, 4, 8), as shown in Table \ref{tab:10}. 
These scaling methods demonstrate the potential to enhance model performance through strategic response selection and process reward modeling.
\begin{table}[t]
\caption{Performance Comparison with PSAS-A after Directly Concatenation (D. Acc) and Guided Error Localization (G. Acc) on PhysReason-mini, where Acc. means the original performance of the model, Gemini-2.0-T$^{*}$ represents Gemini-2.0-Flash-Thinking-0121.}
\vspace{-5pt}
\begin{adjustbox}{width=\columnwidth}
% \small
\centering
\begin{tabular}{lccc}
\hline
Model & Acc. & D. Acc. & G. Acc.\\
\hline
% GPT-4o & 27.22 & \\
Deepseek-V3 & 34.07 & 29.31 & 40.78\\
Gemini-2.0-Flash & 46.52 & 42.76 & 51.55\\
% o3-mini-high & 53.31 & \\
Gemini-2.0-T$^{*}$ & 54.42 & 50.66 & 56.82\\
Deepseek-R1 & 56.60 & 52.26 & 58.33\\
\hline
\end{tabular}
\end{adjustbox}
\label{tab:6}
\vspace{-10pt}
\end{table}
\subsection{Performance Improving with PSAS-S}
Given PSAS-S's capability to locate and analyze the first error step as presented in Section \ref{Step-Level Evaluation}, we conduct experiments on PhysReason-mini to explore whether models can correct errors after being informed.
The experiments are divided into \textit{Direct concatenation} and \textit{Guided error localization}.
The former (D. Acc.) combines questions with the previous reasoning process for a second attempt.
For the latter (G. Acc.), PSAS-S is used to locate and analyze the first error in the reasoning process, then combines the question, previous reasoning, and \textbf{the location and analysis of the first error} for a second attempt.
As shown in Table \ref{tab:6}, results show that direct concatenation decreased performance by 3-5\%, while guided error localization improved performance by 3-6\%.
This suggests that guiding LLMs to identify reasoning errors is crucial for enhancing their reasoning capabilities and also proves the effectiveness of our PSAS framework.
% \section{Analysis}
\begin{figure}[t]
    \centering
    \includegraphics[width=0.5\textwidth]{fig/error_types_count_selected.png}
    \vspace{-20pt}
    \caption{Error statistics with PSAS-S framwork in PhysReason-mini, where Gemini-T-1206 and Gemini-T-0121 denote Gemini-2.0-Flash-Thinking-1206 and Gemini-2.0-Flash-Thinking-0121. }
    \label{fig:3}
\vspace{-15pt}
\end{figure}
\subsection{Error Kind Distribution Analysis}
Discovering errors in reasoning processes is not equivalent to fully understanding them; it's also crucial to understand the causes of errors.
We analyze the error distributions of different models on PhysiReason-mini as shown in Figure \ref{fig:3}.
Four prevalent error types consistently challenge all models: Physics Theorem Application, Physics Process Understanding, Calculation Process, and Physics Condition Analysis.
This reveals models' limited intuitive physics understanding, highlighting the need for stronger physics-based reasoning capabilities.
% This reflect models' limitations in developing intuitive physics understanding, highlighting the importance of building foundational physics-based reasoning capabilities.
Notably, o1 and o3-mini-high show elevated Physics Process Understanding Errors but reduced Calculation Process Errors.
This maybe suggest a trade-off between conceptual comprehension and computational precision.
\begin{figure}[t]
    \centering
    \includegraphics[width=0.5\textwidth]{fig/model_performance_analysis_v2.png}
    \vspace{-20pt}
    \caption{Performance with PSAS-S framework in the hard problems from PhysReason-mini.}
    \label{fig:4}
\vspace{-12pt}
\end{figure}
% \vspace{-10pt}
\subsection{Hard Problem Analysis}
Our analysis of 50 hard reasoning problems from PhysReason-mini across 7 models reveals two key insights (Figure \ref{fig:4}).
Despite variations in overall performance, each model exhibits unique strengths in specific problem domains, demonstrating the diverse nature of their reasoning capabilities.
The models' achievement of some scores (below $1$) is notable, indicating their ability to initiate correct solution paths but failing to maintain this accuracy throughout the reasoning process.
These patterns suggest that while current models grasp basic physics concepts, they struggle to sustain accurate reasoning across extended solution steps.
\section{Conclusion}
We introduce PhysReason, a novel physics-based reasoning benchmark with stratified difficulty and Physics Solution Auto-Scoring Framework with answer and step level evaluation. 
Experimental results show a consistent decline in performance as reasoning depth increases. 
This benchmark establishes new standards for evaluating and improving AI models' physics-based reasoning abilities.

% \begin{table*}[t]
% \centering
% \begin{tabular}{cccccccc}
% \hline
%  Model & Input & O-like & Knowledge & Easy & Medium & Difficult & Avg. \\
% \hline
% Qwen2VL-72B & Q, I & $\times$ & & 31.05 & 17.59 & 3.77 & 16.75 \\
% % \hline
% InternVL2.5-78B & Q, I & $\times$ & & 33.65 & 19.11 & 8.14 & 19.5 \\
% % \hline
% GPT-4o & Q, I & $\times$ & & 43.1 & 25.95 & 6.21 & 24.17 \\
% % \hline
% Claude-3.5 Sonnet & & & & & & & \\
% % \hline
% o1-mini & Q, IC & $\checkmark$ & & 54.29 & 26.82 & 10.03 & 28.84 \\
% % \hline
% Deepseek v3-671B & Q, IC & $\times$ & & 54.89 & 28.16 & 7.99 & 28.87 \\
% % \hline
% QvQ-72B & Q, I & $\checkmark$ & & 56.09 & 33.63 & 13.27 & 33.1 \\
% % \hline
% gemini-2.0 & Q, I & $\times$ & & 61.29 & 48.32 & 16.01 & 41.28 \\
% % \hline
% QwQ-32B & Q, IC & $\checkmark$ & & 63.74 & 45.26 & 19.06 & 41.73 \\
% % \hline
% Deepseek-R1 & Q, IC & $\checkmark$ & & & & & \\
% % \hline
% GLM-Zero & Q, IC & $\checkmark$ & & & & & \\
% % \hline
% Gemini2.0 Thinking & Q, IC & $\checkmark$ & & 63.02 & 46.42 & 19.87 & 42.26 \\
% % \hline
% Gemini2.0 Thinking & Q, I & $\checkmark$ & & 68.02 & 44.09 & 21.76 & 43.33 \\
% \hline
% \end{tabular}
% \caption{Model Performance Comparison (Q: Question, I: Image, IC: Image Caption)}
% \label{tab:3}
% \end{table*}
% \begin{table*}
% \centering
% \begin{tabular}{lrrrrr}
% \hline
% \textbf{Statistics} & \textbf{Knowledge} & \textbf{Easy} & \textbf{Medium} & \textbf{Difficult} & \textbf{Average} \\
% \hline
% Avg Question Tokens & - & 207.80 & 279.43 & 380.66 & 293.71\\
% Max Question Tokens & - & 459 & 552 & 712 & -\\
% Avg Solution Tokens & - & 297.23 & 526.33 & 1016.95 & 627.40\\
% Max Solution Tokens & - & 618 & 1098 & 2850 & -\\
% Avg Sub-questions & - & 2.12 & 2.57 & 3.26 & 2.68\\
% Max Sub-questions & - & 3 & 5 & 6 & -\\
% All Sub-questions & - & 3 & 5 & 6 & -\\
% Avg Steps & - & 3.92 & 7.71 & 15.29 & 8.25\\
% Max Steps & - & 7 & 12 & 34 & -\\
% Avg Sub-question Steps & - & 1.86 & 3.00 & 4.69 & 2.97\\
% Max Sub-question Steps & - & 5 & 8 & 21 & -\\
% % & - & 3 & 5 & 6 & & -\\
% \hline
% \end{tabular}
% \caption{Statistics of our dataset}
% \end{table*}

% With approximately 1,800 problems and multimodal support, our dataset aligns with major benchmarks like SciEval and OlympiadBench while incorporating essential visual elements for physics problem representation. 

% In terms of question characteristics, we employ open-ended (OE) questions exclusively, diverging from the multiple-choice formats common in benchmarks like MMLU and ScienceQA. Question complexity, measured by token length, demonstrates a systematic progression from easy (207.80 tokens) through medium (279.43 tokens) to difficult levels (380.66 tokens). This progression reflects increasing conceptual complexity and problem sophistication, allowing for more nuanced evaluation of problem-solving capabilities.

% The dataset implements a comprehensive answer classification system encompassing numeric values, mathematical expressions, and equations. This tripartite classification provides more nuanced evaluation metrics compared to existing benchmarks that primarily utilize numeric answers or multiple-choice options. This diverse answer taxonomy better reflects the variety of solutions encountered in real-world physics problem-solving scenarios.

% A distinguishing feature is our Step-By-Step (SBS) solution format, which exhibits clear complexity scaling across difficulty levels. Easy problems average 3.92 steps (297.23 tokens), medium problems 7.71 steps (526.33 tokens), and difficult problems 15.29 steps (1016.95 tokens). This structured approach contrasts with the Whole Paragraph (WP) solutions prevalent in other benchmarks. The SBS format offers advantages in explicit problem-solving methodology, quantifiable solution complexity metrics, enhanced pedagogical value, and improved evaluation granularity.

% The dataset makes several methodological contributions to physics problem benchmarking through its integration of CEE and competition-level problems within a unified framework, multimodal support for comprehensive problem representation, and structured solution format enabling detailed analysis of problem-solving processes. The systematic design, characterized by carefully calibrated progression in problem complexity and comprehensive coverage of physics concepts, facilitates both algorithmic evaluation and human learning. This makes it particularly suitable for developing and testing advanced AI systems for physics problem-solving while maintaining its value as an educational resource. The combination of structured solutions, diverse answer types, and multimodal support distinguishes our dataset from existing benchmarks and provides a more robust framework for evaluating physics problem-solving capabilities.

% \begin{algorithm*}
% \caption{Error Step Generation and Selection}
% \begin{algorithmic}[1]

% \Function{GenerateErrorSteps}{$\mathcal{P}$} \Comment{$\mathcal{P}$: problem dataset}
%     \State Let $\mathcal{M} = \{m_1, m_2, m_3, m_4\}$ be the set of models
%     \State Let $\mathcal{C} = \{c_1, c_2, c_3, c_4, c_5\}$ be the scoring criteria
%     \State Let $E_p \gets \emptyset$ be error processes for problem $p$
%     \State Let $\omega_i$ be the weight for criterion $c_i$

%     \ForAll{$p \in \mathcal{P}$}
%         \State $E_p \gets \bigcup_{m \in \mathcal{M}} \{e_{p,m} : e_{p,m} = \Call{CollectErrors}{p, m}\}$
%     \EndFor

%     \State Let $F_p \gets \emptyset$ be first errors for problem $p$
%     \ForAll{$p \in \mathcal{P}$}
%         \State $F_p \gets \{f_{p,m} : f_{p,m} = \min(\Call{ErrorSteps}{e_{p,m}}), \forall m \in \mathcal{M}\}$
%     \EndFor

%     \State Let $S \gets \emptyset$ be selected errors
%     \ForAll{$p \in \mathcal{P}$}
%         \State $S_p \gets \Call{SelectBestError}{F_p}$
%     \EndFor
%     \State \Return $S$
% \EndFunction

% \Function{SelectBestError}{$F_p$}
%     \State $score_{max} \gets 0, e_{best} \gets \emptyset$
%     \ForAll{$e \in F_p$}
%         \State $\vec{s} \gets (s_1, s_2, s_3, s_4, s_5)$ where $s_i \in [1,5]$ \Comment{Score vector}
%         \State $s_i \gets \Call{EvaluateCriterion}{e, c_i}, \forall c_i \in \mathcal{C}$
%         \State $score_e \gets \frac{\sum_{i=1}^5 \omega_i s_i}{\sum_{i=1}^5 \omega_i}$ \Comment{Weighted average}
%         \If{$score_e > score_{max}$}
%             \State $score_{max} \gets score_e$
%             \State $e_{best} \gets e$
%         \EndIf
%     \EndFor
%     \State \Return $e_{best}$
% \EndFunction

% \State \textbf{Where:}
% \State $c_1$: importance, $c_2$: clarity, $c_3$: severity, $c_4$: relevance, $c_5$: correctability
% \State $\omega_i$: weights can be adjusted based on criteria priority
% \State $s_i$: individual criterion score where $s_i \in [1,5]$

% \end{algorithmic}
% \end{algorithm*}


% \begin{table}
% \begin{tabular}{lr}
% \hline
% \textbf{Basic Statistics} & \textbf{Count} \\
% \hline
% Total Problems & 241 \\
% Problems with Images & 212 (88\%) \\
% Problems with Solutions & 241 (100\%) \\
% \hline
% \end{tabular}
% \end{table}

% \begin{table}
% \begin{tabular}{lr}
% \hline
% \textbf{Problem Type Distribution} & \textbf{Count} \\
% \hline
% Knowledge Type & 0 \\
% Easy Reasoning Type & 66 \\
% Medium Reasoning Type & 96 \\
% Difficult Reasoning Type & 79 \\
% \hline
% \end{tabular}
% \end{table}



% \begin{algorithm*}
% \caption{Complete Error Step Analysis Framework}
% \begin{algorithmic}[1]
% \Require $\mathcal{P}$: Problem set, $\mathcal{M}$: LLM models, $\mathcal{S}{\text{ref}}$: Reference solution steps
% \Ensure $\mathcal{S}{\text{error}}$: Selected error steps for each problem

% % \Function{FindFirstError}{$M, S_{\text{ref}}$}
% % \For{$i \gets 1$ to $|S_{\text{ref}}|$}
% % \State $E_i \leftarrow \text{LLM_Extract}(M, S_{\text{ref},i})$
% % \State $\text{score} \leftarrow \text{ScoreStep}(E_i, S_{\text{ref},i})$
% % \If{$\text{score} < 1$}
% % \State $\text{error_type} \leftarrow \text{LLM}(\text{Template}(E_i, S_{\text{ref},i}))$
% % \State $\text{original_text} \leftarrow \text{FindOriginalStep}(M, E_i)$
% % \State \Return $(i, \text{original_text}, \text{error_type})$
% % \EndIf
% % \EndFor
% % \State \Return $\emptyset$
% % \EndFunction

% \Function{GenerateErrorSteps}{$\mathcal{P}$}
% \State $E, F, \mathcal{S}{\text{error}} \leftarrow \emptyset, \emptyset, \emptyset$
% \Comment{Error processes, First error steps, Selected errors}
% \ForAll{$p \in \mathcal{P}$}
% \State $E_p \leftarrow \bigcup\limits{m \in \mathcal{M}} \Call{CollectSteps}{p, m}$
% \State $F_p \leftarrow \emptyset$
% \ForAll{$m \in \mathcal{M}$}
% \State $\text{error_step} \leftarrow \Call{FindFirstError}{m(p), S_{\text{ref},p}}$
% \If{$\text{error_step} \neq \emptyset$}
% \State $F_p \leftarrow F_p \cup {\text{error_step}}$
% \EndIf
% \EndFor
% \State $\mathcal{S}{\text{error}} \leftarrow \mathcal{S}{\text{error}} \cup (p, \Call{SelectBestError}{F_p})$
% \EndFor
% \State \Return $\mathcal{S}_{\text{error}}$
% \EndFunction

% \Function{SelectBestError}{$F_p$}
% \State $\text{score}{\text{max}} \leftarrow 0$, $e{\text{best}} \leftarrow \emptyset$
% \ForAll{$e \in F_p$}
% \State $\text{score}e \leftarrow \frac{1}{4}\sum\limits{i=1}^4 s_i$ where $s_i \in [1,4]$
% \Comment{Importance, Clarity, Severity, Relevance}
% \If{$\text{score}e > \text{score}{\text{max}}$}
% \State $(\text{score}{\text{max}}, e{\text{best}}) \leftarrow (\text{score}e, e)$
% \EndIf
% \EndFor
% \State \Return $e{\text{best}}$
% \EndFunction

% \end{algorithmic}
% \end{algorithm*}


% \begin{algorithm*}
% \caption{Error Step Selection Algorithm}
% \begin{algorithmic}[1]

% \State \textit{Input}: $\mathcal{P}$: Problem set, $\mathcal{M}$: LLM models, $\mathcal{S}$: Reference solution steps
% \Function{GenerateErrorSteps}{$\mathcal{P}$}
% % \State $E, F, S \gets \emptyset$ \Comment{Error processes, First error steps, Selected errors}
% \ForAll{$p \in \mathcal{P}$}
% \State $E \gets \bigcup_{m \in \mathcal{M}} \Call{CollectSteps}{p, m}$
% \State $F \gets {e_{p,m} = \Call{GetFirstError}{E, S}}$
% \State $S_{err} \gets S \cup (p, \Call{SelectBestError}{F_p})$
% \EndFor
% \State \Return $S_{err}$
% \EndFunction

% \State \textit{Output}: $S_{err}$: Selected error steps for each problem

% \Function{GetFirstError}{$E$, $S$}
% \For{$i \gets 1$ to $|S|$}
% \State $e_i \gets \text{LLM\_Extract}(E, s_i)$
% \State $score \gets \text{Scoring}(e_i, s_i)$
% \If{$score < 1$}
% \State $original\_step \gets \text{LLM\_FindOriginalStep}(E, e_i)$
% \State \Return $original\_step$
% \EndIf
% \EndFor
% \State \Return None
% \EndFunction

% \Function{SelectBestError}{$F_p$}
% \State $score_{max} \gets 0, e_{best} \gets \emptyset$
% \ForAll{$e \in F_p$}
% % \State $\vec{s} \gets (s_1,s_2,s_3,s_4)$ \Comment{Importance, Clarity, Severity, Relevance}
% \State $score_e \gets \frac{1}{4}\sum_{i=1}^4 s_i$ where $s_i \in [1,4]$ \Comment{Importance, Clarity, Severity, Relevance}
% \If{$score_e > score_{max}$}
% \State $(score_{max}, e_{best}) \gets (score_e, e)$
% \EndIf
% \EndFor
% \State \Return $e_{best}$
% \EndFunction

% \end{algorithmic}
% \end{algorithm*}


% \begin{algorithm*}
% \caption{Error Injection Process}
% \begin{algorithmic}[1]
% \Procedure{InjectError}{$Problem$}
%     \State \textbf{Input:} Problem P
%     \State \textbf{Output:} Modified solution with optimal error
    
%     \State $ErrorSet \gets \emptyset$ 
%     \ForAll{$model \in \{deepseek, o1\_mini, gpt4, gemini\}$}
%         \State $error\_steps \gets GetFirstError(model(P))$
%         \State $ErrorSet \gets ErrorSet \cup \{error\_steps\}$
%     \EndFor
    
%     \State $best\_error \gets null$, $max\_score \gets -\infty$
%     \ForAll{$error \in ErrorSet$}
%         \State $score \gets(Score(error)$
%         \If{$score > max\_score$}
%             \State $best\_error \gets error$, $max\_score \gets score$
%         \EndIf
%     \EndFor
    
%     \State \Return ModifySolution(P, $best\_error$)
% \EndProcedure
% \end{algorithmic}
% \end{algorithm*}

% \begin{algorithm}
% \caption{Physics Problem Solution Auto-Scoring Process}
% \begin{algorithmic}[1]
% \State \textbf{Phase 1: Data Extraction}
%     \State \textit{Input}: Model output M, annotated steps $S = \{s_1, s_2, ..., s_n\}$
%     \For{$s_i \in S$}
%         \State $m_i$ \gets $LLM\_Extract(M, s_i)$ \Comment{Find matching step in $M$}
%         \State $E_i$ \gets $Normalize(m_i)$ \Comment{E: extracted steps set}
%     \EndFor
%     \State $A$ \gets $ExtractAnswer(M)$ \Comment{$A$: final answer}
%     \State Assert $|E| = |S|$ \Comment{Verify completeness}

% \State \textbf{Phase 2: Intermediate Value Processing}
%     \For{$(e_i, s_i) \in (E, S)$}
%         \State $F_i$ \gets $ExtractFormula(e_i)$ \Comment{F: formula content}
%         \State $R_i$ \gets $ExtractResult(e_i)$ \Comment{R: calculation result}
%         \State $score_i$ \gets $0.5 × ScoreFormula(F_i, s_i) + 0.5 × ScoreResult(R_i, s_i)$
%     \EndFor

% \State \textbf{Phase 3: Error Analysis} 
%     \State ErrorSet Δ \gets ∅
%     \For{(e_i, s_i) \in (E, S)}
%         \If{score_i < threshold_τ}
%             \State error_type \gets ClassifyError(e_i, s_i)
%             \State Δ \gets Δ ∪ \{(i, error_type)\}
%         \EndIf
%     \EndFor

% \State \textbf{Phase 4: Error Summary}
%     \State freq(ε) \gets ComputeFrequency(Δ) \Comment{ε: error types}
%     \State C \gets AnalyzeCauses(Δ) \Comment{C: cause analysis}
%     \State \textit{Output}: \{score_i\}_{i=1}^n, Δ, freq(ε), C

% \end{algorithmic}
% \end{algorithm}
% \begin{algorithm}
% \caption{Physics Problem Solution Auto-Scoring Process}
% \begin{algorithmic}[1]
% \State \textbf{Phase 1: Data Extraction}
%     \State \textit{Input}: Model output text
%     \If{regex\_match\_successful}
%         \State Extract steps and answers directly
%     \Else 
%         \State Use LLM to parse and normalize output
%     \EndIf
%     \State Ensure complete steps and final answer

% \State \textbf{Phase 2: Intermediate Value Processing}
%     \For{each step}
%         \State Extract using LLM:
%             \State formula\_content \Comment{50\% weight}
%             \State calculation\_result \Comment{50\% weight}
%         \State Score based on accuracy
%     \EndFor

% \State \textbf{Phase 3: Error Analysis} 
%     \For{each step}
%         \State Compare with expected solution
%         \If{error\_detected}
%             \State Classify error type
%             \State Record error cause
%         \EndIf
%     \EndFor

% \State \textbf{Phase 4: Error Summary}
%     \State Calculate error type frequencies
%     \State Analyze common error causes
%     \State \textit{Output}: Scoring report

% \end{algorithmic}
% \end{algorithm}
% \begin{table}
% % \small
% \begin{tabular}{lr}
% \hline
% Statistics & Number \\
% \hline
% Total Problems & 241 \\
% Problems with images & 212 (88\%) \\
% Problems with solutions & 241 (100\%) \\
% \hline
% Knowledge Type & 0 \\
% Easy Reasoning Type & 66 \\
% Medium Reasoning Type & 96 \\
% Difficult Reasoning Type & 79 \\
% \hline
% Average question tokens & 253 \\
% Max question tokens & 3,745 \\
% Average solution tokens & 347 \\
% Max solution tokens & 4,223 \\
% \hline
% \textbf{Konwledge} & \\
% Average question tokens & 207.80 \\
% Max question tokens & 459 \\
% Average solution tokens & 297.23 \\
% Max solution tokens & 618 \\
% Average sub\_questions & 3 \\
% Max sub\_questions & 2.12 \\
% \hline
% \textbf{Easy Reasoning} & \\
% Average question tokens & 207.80 \\
% Max question tokens & 459 \\
% Average solution tokens & 297.23 \\
% Max solution tokens & 618 \\
% Average sub\_questions & 3 \\
% Max sub\_questions & 2.12 \\
% \hline
% \textbf{Medium Reasoning} & \\
% Average question tokens & 279.43 \\
% Max question tokens & 552 \\
% Average solution tokens & 526.33 \\
% Max solution tokens & 1098 \\
% Average sub\_questions & 5 \\
% Max sub\_questions & 2.57 \\
% \hline
% \textbf{Difficult Reasoning} & \\
% Average question tokens & 380.66 \\
% Max question tokens & 712 \\
% Average solution tokens & 1016.95 \\
% Max solution tokens & 2850 \\
% Average sub\_questions & 6 \\
% Max sub\_questions & 3.26 \\
% \hline
% \end{tabular}
% \end{table}


% \begin{table*}
% \begin{tabular}{l>{\centering}p{1.2cm}cp{1.2cm}cp{1.3cm}cp{1.8cm}}
% \hline
% \multirow{2}{*}{Benchmark} & Multi- & \multirow{2}{*}{\makecell{Size}} & \multicolumn{2}{c}{Question} & \multicolumn{2}{c}{Solution} & \multirow{2}{*}{\makecell{Answer type}} \\
% & modal & & \multicolumn{2}{c}{\hrulefill} & \multicolumn{2}{c}{\hrulefill} & \\
% & & & Type & \makecell{Average tokens} & Type & \makecell{Average tokens} & \\
% \hline
% SciBench & $\checkmark$ & 295 & OE & 80.51 & \makecell{WP\\(12.7\%)} & 315.85 & Num \\
% MMMU & $\checkmark$ & 443 & OE,MC & 53.82 & - & - & Num \\
% ScienceQA & $\checkmark$ & 617 & MC & 13.31 & WP & 62.95 & Opt \\
% SciEval & & 1657 & OE,MC & 154.47 & - & - & Num,Opt \\
% JEEBench & & 123 & OE,MC & 169.69 & - & - & Num,Opt \\
% MMLU & & 548 & MC & 44.85 & - & - & Opt \\
% AGIEval & & 200 & OE,MC & 100.38 & - & - & Num \\
% OlympiadBench & $\checkmark$ & 2334 & OE & 222.03 & WP & 199.82 & \makecell{Num,Equ,\\Exp} \\
% GPhysReason & & 227 & OE & 111.41 & WP & 197.17 & Num,Equ \\
% EMMA & $\checkmark$ & 156 & MC & 109.47 & - & - & Opt \\
% \hline
% Our-Easy & $\checkmark$ & 66 & OE & 186.71 & SPS & 214.45 & \makecell{Num,Equ,\\Exp} \\
% Our-Medium & $\checkmark$ & 96 & OE & 254.41 & SPS & 396.07 & \makecell{Num,Equ,\\Exp} \\
% Our-Difficult & $\checkmark$ & 79 & OE & 350.04 & SPS & 791.04 & \makecell{Num,Equ,\\Exp} \\
% \hline
% \end{tabular}
% \end{table*}
% \begin{table*}
% \small
% \centering
% \begin{tabular}{ccccccccc}
% \hline
% \multirow{2}{*}{Benchmark} & Multi- & \multirow{2}{*}{Size} & \multicolumn{2}{c}{Question} & \multicolumn{2}{c}{Solution} & Answer \\
% \cline{4-5} \cline{6-7}
% & modal & & Type & Average tokens & Type & Average tokens & type \\
% \hline
% SciBench       & $\checkmark$ & 295  & OE    & 80.51  & WP(12.7\%) & 315.85 & Num\\
% MMMU           & $\checkmark$ & 443  & OE,MC & 53.82  & - & - &  Num\\
% ScienceQA      & $\checkmark$ & 617  & MC    & 13.31  & WP & 62.95 & Opt \\
% SciEval        &              & 1657 & OE,MC & 154.47 & -  & - &  Num,Opt\\
% JEEBench       &              & 123  & OE,MC & 169.69 & -  & - &  Num,Opt\\
% MMLU           &              & 548  & MC    & 44.85  & -   & - &  Opt\\
% AGIEval        &              & 200  & OE,MC & 100.38 & -   & - &  Num\\
% OlympiadBench  & $\checkmark$ & 2334 & OE    & 222.03 & WP & 199.82 &  Num,Equ,Exp\\
% GPQA           &              & 227  & OE    & 111.41 & WP & 197.17 &  Num,Equ\\
% EMMA           & $\checkmark$ & 156  & MC    & 109.47 &  -  & - &  Opt\\
% \hline
% Our-Easy       & $\checkmark$ & 66  & OE    & 186.71 & SPS & 214.45 &Num, Equ, Exp\\
% Our-Medium     & $\checkmark$ & 96  & OE    & 254.41 & SPS & 396.07 &Num, Equ, Exp\\
% Our-Difficult  & $\checkmark$ & 79  & OE    & 350.04 & SPS & 791.04 &Num, Equ, Exp\\
% \hline
% \end{tabular}
% \end{table*}

% \begin{table*}
% \begin{tabular}{ccccccccc}
% \hline
% \multirow{2}{*}{Benchmark} & \multirow{2}{*}{Multi-} & \multirow{2}{*}{Detailed} & \multirow{2}{*}{Difficulty} & \multicolumn{2}{c}{Size} & \multirow{2}{*}{Answer} & \multirow{2}{*}{Question} \\
%   & modal & solution & level & Physics & type & type \\
% \hline
% SciBench &  &    & COL & 217 & 295 & Num & EN & OE \\
% MMMU &  &  &  COL & 540 & 443 & Num & EN & MC,OE \\
% MathVista &  & &  - & 1,000 & & Num & EN & MC,OE \\
% ScienceQA & &   & H & & 617 & & EN & MC \\
% SciEval & & & - & & 1,657 & Num & EN & MC,FB,J \\
% JEEBench  & &  & CEE & 236 & 123 & Num & EN & MC,OE \\
% MMLU &  &   & COL & 948 & 548 & & EN & MC \\
% AGIEval &   & & CEE & 953 & 200 & Num & EN,ZH & MC,FB,OE \\
% GSM8K &  &  & E & 1,319 & & Num & EN & OE \\
% MATH &  & & COMP & 5,000 & & Num,Exp,Tup & EN & OE \\
% OlympiadBench &   &  & COMP & 6,142 & 2,334 & ALL & EN,ZH & OE \\
% \hline
% \end{tabular}
% \end{table*}

% To produce a PDF file, pdf\LaTeX{} is strongly recommended (over original \LaTeX{} plus dvips+ps2pdf or dvipdf). Xe\LaTeX{} also produces PDF files, and is especially suitable for text in non-Latin scripts.

% \section{Preamble}

% The first line of the file must be
% \begin{quote}
% \begin{verbatim}
% \documentclass[11pt]{article}
% \end{verbatim}
% \end{quote}

% To load the style file in the review version:
% \begin{quote}
% \begin{verbatim}
% \usepackage[review]{acl}
% \end{verbatim}
% \end{quote}
% For the final version, omit the \verb|review| option:
% \begin{quote}
% \begin{verbatim}
% \usepackage{acl}
% \end{verbatim}
% \end{quote}

% To use Times Roman, put the following in the preamble:
% \begin{quote}
% \begin{verbatim}
% \usepackage{times}
% \end{verbatim}
% \end{quote}
% (Alternatives like txfonts or newtx are also acceptable.)

% Please see the \LaTeX{} source of this document for comments on other packages that may be useful.

% Set the title and author using \verb|\title| and \verb|\author|. Within the author list, format multiple authors using \verb|\and| and \verb|\And| and \verb|\AND|; please see the \LaTeX{} source for examples.

% By default, the box containing the title and author names is set to the minimum of 5 cm. If you need more space, include the following in the preamble:
% \begin{quote}
% \begin{verbatim}
% \setlength\titlebox{<dim>}
% \end{verbatim}
% \end{quote}
% where \verb|<dim>| is replaced with a length. Do not set this length smaller than 5 cm.

% \section{Document Body}

% \subsection{Footnotes}

% Footnotes are inserted with the \verb|\footnote| command.\footnote{This is a footnote.}

% \subsection{Tables and figures}

% See Table~\ref{tab:accents} for an example of a table and its caption.
% \textbf{Do not override the default caption sizes.}

% \begin{table}
%   \centering
%   \begin{tabular}{lc}
%     \hline
%     \textbf{Command} & \textbf{Output} \\
%     \hline
%     \verb|{\"a}|     & {\"a}           \\
%     \verb|{\^e}|     & {\^e}           \\
%     \verb|{\`i}|     & {\`i}           \\
%     \verb|{\.I}|     & {\.I}           \\
%     \verb|{\o}|      & {\o}            \\
%     \verb|{\'u}|     & {\'u}           \\
%     \verb|{\aa}|     & {\aa}           \\\hline
%   \end{tabular}
%   \begin{tabular}{lc}
%     \hline
%     \textbf{Command} & \textbf{Output} \\
%     \hline
%     \verb|{\c c}|    & {\c c}          \\
%     \verb|{\u g}|    & {\u g}          \\
%     \verb|{\l}|      & {\l}            \\
%     \verb|{\~n}|     & {\~n}           \\
%     \verb|{\H o}|    & {\H o}          \\
%     \verb|{\v r}|    & {\v r}          \\
%     \verb|{\ss}|     & {\ss}           \\
%     \hline
%   \end{tabular}
%   \caption{Example commands for accented characters, to be used in, \emph{e.g.}, Bib\TeX{} entries.}
%   \label{tab:accents}
% \end{table}

% As much as possible, fonts in figures should conform
% to the document fonts. See Figure~\ref{fig:experiments} for an example of a figure and its caption.

% Using the \verb|graphicx| package graphics files can be included within figure
% environment at an appropriate point within the text.
% The \verb|graphicx| package supports various optional arguments to control the
% appearance of the figure.
% You must include it explicitly in the \LaTeX{} preamble (after the
% \verb|\documentclass| declaration and before \verb|\begin{document}|) using
% \verb|\usepackage{graphicx}|.

% \begin{figure}[t]
%   \includegraphics[width=\columnwidth]{example-image-golden}
%   \caption{A figure with a caption that runs for more than one line.
%     Example image is usually available through the \texttt{mwe} package
%     without even mentioning it in the preamble.}
%   \label{fig:experiments}
% \end{figure}

% \begin{figure*}[t]
%   \includegraphics[width=0.48\linewidth]{example-image-a} \hfill
%   \includegraphics[width=0.48\linewidth]{example-image-b}
%   \caption {A minimal working example to demonstrate how to place
%     two images side-by-side.}
% \end{figure*}

% \subsection{Hyperlinks}

% Users of older versions of \LaTeX{} may encounter the following error during compilation:
% \begin{quote}
% \verb|\pdfendlink| ended up in different nesting level than \verb|\pdfstartlink|.
% \end{quote}
% This happens when pdf\LaTeX{} is used and a citation splits across a page boundary. The best way to fix this is to upgrade \LaTeX{} to 2018-12-01 or later.

% \subsection{Citations}

% \begin{table*}
%   \centering
%   \begin{tabular}{lll}
%     \hline
%     \textbf{Output}           & \textbf{natbib command} & \textbf{ACL only command} \\
%     \hline
%     \citep{Gusfield:97}       & \verb|\citep|           &                           \\
%     \citealp{Gusfield:97}     & \verb|\citealp|         &                           \\
%     \citet{Gusfield:97}       & \verb|\citet|           &                           \\
%     \citeyearpar{Gusfield:97} & \verb|\citeyearpar|     &                           \\
%     \citeposs{Gusfield:97}    &                         & \verb|\citeposs|          \\
%     \hline
%   \end{tabular}
%   \caption{\label{citation-guide}
%     Citation commands supported by the style file.
%     The style is based on the natbib package and supports all natbib citation commands.
%     It also supports commands defined in previous ACL style files for compatibility.
%   }
% \end{table*}

% Table~\ref{citation-guide} shows the syntax supported by the style files.
% We encourage you to use the natbib styles.
% You can use the command \verb|\citet| (cite in text) to get ``author (year)'' citations, like this citation to a paper by \citet{Gusfield:97}.
% You can use the command \verb|\citep| (cite in parentheses) to get ``(author, year)'' citations \citep{Gusfield:97}.
% You can use the command \verb|\citealp| (alternative cite without parentheses) to get ``author, year'' citations, which is useful for using citations within parentheses (e.g. \citealp{Gusfield:97}).

% A possessive citation can be made with the command \verb|\citeposs|.
% This is not a standard natbib command, so it is generally not compatible
% with other style files.

% \subsection{References}

% \nocite{Ando2005,andrew2007scalable,rasooli-tetrault-2015}

% The \LaTeX{} and Bib\TeX{} style files provided roughly follow the American Psychological Association format.
% If your own bib file is named \texttt{custom.bib}, then placing the following before any appendices in your \LaTeX{} file will generate the references section for you:
% \begin{quote}
% \begin{verbatim}
% \bibliography{custom}
% \end{verbatim}
% \end{quote}

% You can obtain the complete ACL Anthology as a Bib\TeX{} file from \url{https://aclweb.org/anthology/anthology.bib.gz}.
% To include both the Anthology and your own .bib file, use the following instead of the above.
% \begin{quote}
% \begin{verbatim}
% \bibliography{anthology,custom}
% \end{verbatim}
% \end{quote}

% Please see Section~\ref{sec:bibtex} for information on preparing Bib\TeX{} files.

% \subsection{Equations}

% An example equation is shown below:
% \begin{equation}
%   \label{eq:example}
%   A = \pi r^2
% \end{equation}

% Labels for equation numbers, sections, subsections, figures and tables
% are all defined with the \verb|\label{label}| command and cross references
% to them are made with the \verb|\ref{label}| command.

% This an example cross-reference to Equation~\ref{eq:example}.

% \subsection{Appendices}

% Use \verb|\appendix| before any appendix section to switch the section numbering over to letters. See Appendix~\ref{sec:appendix} for an example.

% \section{Bib\TeX{} Files}
% \label{sec:bibtex}

% Unicode cannot be used in Bib\TeX{} entries, and some ways of typing special characters can disrupt Bib\TeX's alphabetization. The recommended way of typing special characters is shown in Table~\ref{tab:accents}.

% Please ensure that Bib\TeX{} records contain DOIs or URLs when possible, and for all the ACL materials that you reference.
% Use the \verb|doi| field for DOIs and the \verb|url| field for URLs.
% If a Bib\TeX{} entry has a URL or DOI field, the paper title in the references section will appear as a hyperlink to the paper, using the hyperref \LaTeX{} package.

% \section*{Acknowledgments}

% This document has been adapted
% by Steven Bethard, Ryan Cotterell and Rui Yan
% from the instructions for earlier ACL and NAACL proceedings, including those for
% ACL 2019 by Douwe Kiela and Ivan Vuli\'{c},
% NAACL 2019 by Stephanie Lukin and Alla Roskovskaya,
% ACL 2018 by Shay Cohen, Kevin Gimpel, and Wei Lu,
% NAACL 2018 by Margaret Mitchell and Stephanie Lukin,
% Bib\TeX{} suggestions for (NA)ACL 2017/2018 from Jason Eisner,
% ACL 2017 by Dan Gildea and Min-Yen Kan,
% NAACL 2017 by Margaret Mitchell,
% ACL 2012 by Maggie Li and Michael White,
% ACL 2010 by Jing-Shin Chang and Philipp Koehn,
% ACL 2008 by Johanna D. Moore, Simone Teufel, James Allan, and Sadaoki Furui,
% ACL 2005 by Hwee Tou Ng and Kemal Oflazer,
% ACL 2002 by Eugene Charniak and Dekang Lin,
% and earlier ACL and EACL formats written by several people, including
% John Chen, Henry S. Thompson and Donald Walker.
% Additional elements were taken from the formatting instructions of the \emph{International Joint Conference on Artificial Intelligence} and the \emph{Conference on Computer Vision and Pattern Recognition}.

% % Bibliography entries for the entire Anthology, followed by custom entries
% %\bibliography{anthology,custom}
% % Custom bibliography entries only
\newpage
\section*{Limitation}
Despite the comprehensive nature of our benchmark, two key limitations warrant discussion, concerning both benchmark construction and evaluation methodology.
First, they focus primarily on testing models' ability to apply and reason with physics theorems under idealized conditions, rather than fully reflecting real-world physics scenarios. 
However, it is worth noting that applying physics theorems under idealized conditions serves as the foundation for real-world physics scenarios, as the latter is more complex. 
However, current LLMs' performance even on idealized conditions remains unsatisfactory. 
Therefore, PhysReason remains valuable in evaluating models' ability to apply physics theorems for physics-based reasoning. 
Moreover, through data synthesis, many problems in PhysReason can be adapted to create real-world physics reasoning scenarios, which will be a direction for our future research.
Second, our evaluation framework, though achieving over 98\% accuracy using LLMs as assessment tools, is not without limitations. The PSAS-S framework, while demonstrating satisfactory performance, increases computational time for evaluation.
In future work, we will explore ways to optimize evaluation time while maintaining assessment accuracy.
\section*{Ethical Statement}
In developing PhysReason, we carefully considered and addressed potential implications and risks. 
Our benchmark, sourced exclusively from public official materials (IPhO, Gaokao, JEE, and authorized mock exams), undergoes rigorous data cleansing, deduplication, and standardization to ensure reliability while minimizing bias and data leakage. 
Committed to environmental sustainability, we publicly release complete datasets and accompanying scripts under appropriate licenses (MIT and CC BY-NC-SA) to cut down on unnecessary carbon footprint, while optimizing processing pipelines to reduce computational overhead. 
In all experiments, we strictly comply with all licenses for models and data. Our benchmark is an important resource that drives AGI's strength in scientific reasoning, maintaining high standards for data quality and ethical considerations.

\bibliography{custom}
\newpage
\appendix
% \begin{algorithm*}
% \caption{Physics Solution Error Analysis}
% \begin{algorithmic}[1]
% \Function{AnalyzePhysicsSolution}{problemData, solutionSteps}
%     \State errorTypes = [ \State \quad "DiagramAnalysisErrors", "PhysicsLawApplicationErrors", 
%         \State \quad "PhysicsConditionAnalysisErrors", "PhysicsProcessUnderstandingErrors"
%         \State \quad "VariableRelationshipErrors", "CalculationProcessErrors"
%     ]
    
%     \State identifiedErrors = []
%     \State explanations = []
    
%     \For{each errorType in errorTypes}
%         \State prompt = ConstructPrompt(problemData, solutionSteps, errorType)
%         \State response = QueryLLM(prompt)
        
%         \If{IsErrorFound(response)}
%             \State step, explanation = ExtractStepAndExplanation(response)
%             \State identifiedErrors.append(step)
%             \State explanations.append(explanation)
%         \EndIf
%     \EndFor
    
%     \If{identifiedErrors is empty}
%         \State fallbackPrompt = ConstructFallbackPrompt(problemData, solutionSteps)
%         \State response = QueryLLM(fallbackPrompt)
%         \State \Return ExtractStepAndExplanation(response)
%     \EndIf
    
%     \State earliestError = FindEarliestStep(identifiedErrors)
%     \State relevantExplanation = GetMatchingExplanation(earliestError, explanations)
    
%     \State \Return (relevantExplanation, earliestError)
% \EndFunction

% \Function{ProcessAllSolutions}{solutionDirectory}
%     \State correctCount = 0
%     \State totalProcessed = 0
    
%     \For{each solution in solutionDirectory}
%         \State wrongStepData = LoadWrongStepData(solution)
%         \State problemData = LoadProblemData(solution)
        
%         \State explanation, identifiedStep = AnalyzePhysicsSolution(problemData, wrongStepData)
%         \State actualWrongStep = wrongStepData.wrongStep
        
%         \State isCorrect = (identifiedStep == actualWrongStep)
%         \If{isCorrect}
%             \State correctCount += 1
%         \EndIf
        
%         \State SaveResults(solution, explanation, identifiedStep, actualWrongStep, isCorrect)
%         \State totalProcessed += 1
%     \EndFor
    
%     \State accuracy = correctCount / totalProcessed
%     \State \Return accuracy
% \EndFunction
% \end{algorithmic}
% \end{algorithm*}
% \section{Benchmark Details}
% \subsection{Data Source}
\section{Data Sources}
Our dataset is derived from four distinct sources, each representing different levels and approaches to physics education and assessment. These sources have been carefully selected to ensure comprehensive coverage of physics problems across various difficulty levels and cultural contexts. The diversity of these sources helps in creating a robust and well-rounded dataset that captures different pedagogical approaches and problem-solving methodologies.
\begin{itemize}
\item \textbf{International Physics Olympiad (IPhO) Problems}\\
The Physics Olympiad problems are globally recognized for their complexity and quality. These problems typically require multiple solution approaches and the integration of capabilities across mathematics and physics sub-disciplines. 
Participants in these competitions represent some of the world's strongest talent in physics logical reasoning. 
The problems often combine theoretical understanding with practical applications, requiring students to demonstrate both analytical and creative problem-solving skills. The international nature of these competitions ensures a diverse range of problem-solving approaches and cultural perspectives.
\item \textbf{Chinese National College Entrance Examination (Gaokao) Physics Questions}\\
The Gaokao physics questions represent a rigorous standardized assessment system that has been refined over decades. These questions are designed to test both fundamental understanding and advanced application of physics concepts at the high school level. 
The problems are carefully calibrated to discriminate between different levels of student ability while maintaining high reliability and validity. 
They often incorporate real-world scenarios and practical applications, making them particularly valuable for assessing applied physics knowledge.
\item \textbf{Chinese Mock Examinations at Various Levels}\\
Our collection includes a comprehensive range of mock examination questions from multiple administrative levels in China. This includes provincial-level mock exams, city-level assessment materials, and joint examination papers created through collaboration among multiple high schools. 
These diverse sources provide a rich spectrum of problem-solving scenarios and difficulty levels. 
The multi-tiered nature of these mock examinations reflects different regional interpretations of educational standards while maintaining alignment with national requirements. The variety in question sources ensures exposure to different testing styles and pedagogical approaches, making this dataset particularly valuable for understanding the breadth of physics education assessment in China.
\item \textbf{Indian Joint Entrance Examination (Advanced)}\\
This examination represents one of India's most prestigious and challenging engineering entrance tests. The exam structure, consisting of two papers with 50-60 questions each, provides a comprehensive assessment of physics knowledge alongside mathematics and chemistry. 
The questions are known for their analytical depth and often require multi-step problem-solving approaches. 
The exam's high stakes nature and competitive environment ensure that the problems are both challenging and discriminating, making them valuable additions to our dataset.
\item  \textbf{Others}\\
We also obtained some physics questions from non-Chinese and English sources on hugging-face, such as Russian \cite{phy_big}.
% We also obtained some physics questions from non-Chinese and English sources on hugging face, such as Russian [phy_big]. 
This dataset consists of a diverse collection of physics problems, categorized into different domains, including 1000 problems on Kinematics, 600 problems on Electricity and Circuits, and 500 problems on Thermodynamics. All data has been extracted from open sources, ensuring a wide variety of problem types and difficulty levels.
\end{itemize}
The PhysReason benchmark is derived from publicly available physics education materials including:
International Physics Olympiad problems (2008-2021), Chinese National College Entrance Examination physics questions (2010-2024), Indian Joint Entrance Examination Advanced physics problems (2010-2024), Chinese provincial and municipal mock examination questions (2015-2024).
We have collected more than 20,000 physics problems.
All problems were collected in accordance with fair use principles for educational and research purposes. 
The complete benchmark and associated code will be released under the MIT License for research use.
\par
The dataset contains no personally identifiable information. 
All problems are from standardized tests and competition materials with no individual student data.
This documentation ensures reproducibility and proper usage of the benchmark while protecting privacy and intellectual property rights.
% \label{sec:appendix}
\begin{figure*}[t]
    \centering
    \includegraphics[width=\textwidth]{fig/pipeline.pdf}
    % \vspace{-10pt}
    \caption{Illustration of the data collection pipeline.
}
    \label{fig:10}
\end{figure*}
\section{Benchmark}
\subsection{Collection}
\subsubsection{Data Acquisition}
We systematically collected, curated, and processed physics problems from diverse sources to ensure comprehensive coverage of physics concepts and problem-solving scenarios. 
% The dataset comprises over 26,846 pages of physics problems from 1,218 PDF files. 
Our dataset comprises 1,254 PDF documents totaling 27,874 pages, yielding over 20,000 unique problems. 
This extensive collection provides a rich foundation for developing a comprehensive physics problem benchmark.
\subsubsection{Data Standardization}
We implemented a systematic data processing pipeline utilizing MinerU \cite{wang2024mineruopensourcesolutionprecise} for PDF parsing. 
The standardization process encompasses several critical phases: initial format conversion, rigorous deduplication, and comprehensive formatting standardization.
Each question underwent a rigorous quality assessment process with specific evaluation criteria:
\begin{enumerate}
\item Complete problem statements with well-defined variables and conditions
\item Clear and unambiguous wording
\item Accurate expressions and units
% \item Proper figure references and annotations where applicable
\item Consistent formatting of equations and symbols
\end{enumerate}
% To ensure data quality, we established quantitative metrics:
% \begin{itemize}
% \item Format Consistency Index (FCI): Measuring standardization compliance ($\geq80\%$ required)
% \item Question Clarity Rating (QCR): Assessing problem statement clarity ($\geq80\%$ required)
% \end{itemize}

% \subsubsection{Translation and Quality Control}
% Since the collected data sets involve multiple languages, such as Chinese, English, Hindi, and Russian, we plan to translate them all into English to facilitate model understanding.
% We established a comprehensive three-phase translation and quality control process:
% \paragraph{Initial Translation}
% \begin{itemize}
% \item Strict adherence to standardized physics terminology guidelines
% \end{itemize}
% \paragraph{Expert Review}
% % Focus on:
% \begin{itemize}
% \item Technical accuracy of physics terms
% \item Mathematical expression consistency
% \item Semantic equivalence
% \item Context preservation
% \end{itemize}

% \paragraph{Quality Assurance}
% \begin{itemize}
% \item Delete problems that do not have clear answers and solutions
% \item Inter-translator agreement rate $\geq 80\%$ required
% \item Excludes datasets whose answers can be found through a five-minute Google search
% % \item 
% \end{itemize}

\subsubsection{Translation}
To standardize the multilingual dataset comprising Chinese, English, Hindi, and Russian content, we implement a two-phase process:

\paragraph{Phase I: Translation}
\begin{itemize}
\item Initial translation by translators
\item Strict adherence to standardized physics terminology
\item Consistent mathematical notation and expressions
\end{itemize}

\paragraph{Phase II: Verification}
\begin{itemize}
\item Validation by Engineering Ph.D. candidates
\item Verification of physics terminology accuracy
\item Confirmation of semantic equivalence
\item Review of mathematical expression consistency
\end{itemize}

\subsubsection{Search Prevention}
Following \cite{rein2024gpqa}, we exclude problems whose answers could be found through a five-minute Google search to minimize data leakage. This step ensures that model evaluation reflects genuine physics problem-solving capabilities rather than information retrieval abilities.

\subsubsection{Difficulty Classification:}
Questions were systematically categorized using a multi-dimensional classification framework:
\paragraph{Primary Classification}
\begin{itemize}
\item  Knowledge-based questions:
    \begin{itemize}
    \item Focus on fundamental physics concepts
    \item Clear-cut application of specific theorems
    \item Direct calculation or concept identification
    \end{itemize}
\item Reasoning-based questions:
    \begin{itemize}
        \item Multiple-theorem integration
        \item Multi-step problem-solving approaches
        \item Complex analytical thinking
    \end{itemize}
\end{itemize}
\paragraph{Difficulty Levels in Reasoning-based Questions}
\begin{itemize}
\item Easy:
    \begin{itemize}
    % \item 2-3 steps per sub-problem
    \item Total steps $\leq$ 5
    % \item Basic concept application
    \item Completion time: 0-5 minutes
    \end{itemize}
\end{itemize}

\begin{itemize}
\item Medium:
    \begin{itemize}
    % \item 4-6 steps per sub-problem
    \item Total steps $\leq$ 10
    % \item Integration of 2-3 concepts
    \item Completion time: 5-15 minutes
    \end{itemize}
\end{itemize}

\begin{itemize}
\item Hard:
    \begin{itemize}
    \item Total steps $>$ 10
    % \item Integration of multiple concepts
    % \item Complex problem decomposition
    \item Completion time: 15+ minutes
    \end{itemize}
\end{itemize}

% \paragraph{Classification Quality Control}
% \begin{itemize}
% \item Minimum three independent expert evaluators
% \item Inter-rater reliability (Cohen's Kappa) $\geq 0.8$
% \item Regular calibration meetings
% \item Systematic review of borderline cases
% \end{itemize}
% \begin{table}
% \small
% \centering
% \setlength{\tabcolsep}{5pt}
% \begin{tabular}{@{}cccccc@{}}
% \hline
% \textbf{Statistics} & \textbf{K} & \textbf{E} & \textbf{M} & \textbf{D} & \textbf{Avg.} \\
% \hline
% Avg. QT & 163.68 & 171.21 & 229.19 & 340.94 & 226.25\\
% Max. QT & 428    & 655    & 768    & 1213   & - \\
% Avg. ST & 196.48 & 241.52 & 391.28 & 936.06 & 441.34\\
% Max. ST & 502    & 756    & 1030   & 2977   & - \\
% Avg. SQ & 2.52   & 1.96   & 2.53   & 3.38   & 2.60\\
% Max. SQ & 4      & 4      & 5      & 15     & - \\
% Avg. SP & 3.31   & 5.00   & 8.39   & 15.57  & 8.07\\
% Max. SP & 5      & 6      & 10     & 44     & - \\
% Avg. SQP & 1.32  & 2.54   & 3.32   & 4.61   & 2.95\\
% Max. SQP & 3     & 6      & 10     & 21     & -\\
% \hline
% \end{tabular}
% \vspace{-5pt}
% \caption{Statistics of our benchmark. 
% For statistics, QT: Question tokens, ST: Soultion tokens, SQ: Sub-question number, SP: step number, SQP: Sub-question step number. 
% For classification, K: knowledge, E: easy, M: medium, D: difficult.}
% \label{tab:2}
% \vspace{-10pt}
% \end{table}
\subsection{Annotation}
\paragraph{Key Elements}
As shown in Figure \ref{fig:0}, our annotation framework consists of 7 key elements:
\begin{itemize}
    \item Context:
    \begin{itemize}
        \item Detailed physics scenario description: Describe the physics setup thoroughly, including objects, environment, and interactions. For example, specify angles, materials, initial conditions, and forces.
        \item Clear specification of conditions and constraints: Explicitly list all given conditions: initial conditions (e.g., initial velocity, position), boundary conditions, and constraints (e.g., inextensible string, frictionless surface).
        \item Standardized notation for physics quantities: Use consistent and standard symbols for physics quantities (e.g., $v$ for velocity, $a$ for acceleration, $m$ for mass) throughout the annotation.
    \end{itemize}
\end{itemize}

\begin{itemize}
    \item Sub-question:
    \begin{itemize}
        \item Hierarchical structure of related questions: Break down a complex problem into smaller, logically connected sub-questions. These should build upon each other.
        \item Clear progression of complexity: Sub-questions should increase in difficulty, guiding the learner from basic concepts to more advanced analysis.
    \end{itemize}
\end{itemize}

\begin{itemize}
    \item Solution:
    \begin{itemize}
        \item Detailed step-by-step reasoning process: Provide a comprehensive, step-by-step solution.  Do not skip any crucial reasoning steps.
        \item Each step contains at least one formula:  Each step in the solution should include at least one relevant physics formula (theorem, law, or derived equation).
        \item If the formula can be solved to a value, it should also have a value: If a step's formula yields a numerical result, provide that result.
    \end{itemize}
\end{itemize}

\begin{itemize}
    \item Step Analysis:
    \begin{itemize}
       \item Explicit theorem application rationale: Clearly state which theorem, law, or principle is applied in each step and why it's applicable. Example: Newton's Second Law.
        \item Physics quantity derivation explanation: Explain how unknown physics quantities are derived from known ones. Example: "$W = \Delta E_k$"
    \end{itemize}
\end{itemize}

\begin{itemize}
    \item Answer:
    \begin{itemize}
        \item Numerical results with appropriate units: Provide the correct numerical value and units for numerical answers (e.g., "$v = 5 m/s$").
        \item Formulaic results with appropriate symbols: For formulaic answers, use previously defined standard symbols and ensure the formula's correctness (e.g., "$v = \sqrt{2gh}$").
    \end{itemize}
\end{itemize}

\begin{itemize}
    \item Difficulty:
    \begin{itemize}
        \item Reasoning difficulty metrics: Use qualitative descriptions (e.g., "knowledge" "easy," "medium," "hard") 
        % \item Required knowledge level indication
        % \item Time estimation for solution
    \end{itemize}
\end{itemize}

\begin{itemize}
    \item Theorem:
    \begin{itemize}
        \item Comprehensive list of applicable theorems, laws, and formulas: Provide a complete list of all the specific physics theorems, laws, and equations that are relevant to solving the problem. Examples include: `Newton's Second Law', `Work-Energy Theorem', `Conservation of Momentum', `Kinematic Equations', etc.
        \item Core Concepts: Identify the fundamental physics principles and ideas that underpin the solution, even if they aren't expressed as a single equation. Examples include:  `Wave-Particle Duality'.
        % \item Prerequisites identification
    \end{itemize}
\end{itemize}

% \paragraph{Quality Control Measures}
% \begin{itemize}
% \item Double-blind annotation process
% \item Regular inter-annotator agreement assessment
% \item Systematic review of complex cases
% \item Periodic annotation guideline updates
% \end{itemize}

% This is an appendix.
\section{Error Type Details}
\label{Error Type Details}
% \subsection{Error Types}
The following is a summary of the error types, categorized and with expanded explanations:
\begin{enumerate}
    \item \textbf{Diagram Analysis Errors:}
    \begin{itemize}
        \item \textit{Description:} Errors related to the comprehension, plotting, analysis, or extraction of data from graphical representations. This encompasses any mistake made when working with diagrams, charts, or graphs.
        \item \textit{Examples:}
        \begin{itemize}
            \item Misreading the labels or units on the axes of a graph.
            \item Misinterpreting the trend of a curve (e.g., confusing a linear relationship with an exponential one).
            \item Failing to identify key data points or features on the graph (e.g., maxima, minima, intercepts).
            \item Incorrectly extrapolating or interpolating data from the graph.
            \item Drawing an inaccurate graph based on given data.
        \end{itemize}
    \end{itemize}

    \item \textbf{Physics Theorem Application Errors:}
    \begin{itemize}
        \item \textit{Description:} Errors arising from the incorrect application of physics theorems or principles, or using them in situations where they are not valid. This includes both misremembering the law itself and misapplying a correctly remembered law.
        \item \textit{Examples:}
        \begin{itemize}
            \item Applying Newton's Laws of Motion to a non-inertial reference frame without accounting for fictitious forces.
            \item Using the conservation of energy principle in a system where non-conservative forces (like friction) are doing significant work.
            \item Applying a formula outside of its valid range of applicability (e.g., using a small-angle approximation when the angle is large).
            \item Misunderstanding the conditions under which a particular law is valid.
        \end{itemize}
    \end{itemize}

    \item \textbf{Physics Condition Analysis Errors:}
    \begin{itemize}
        \item \textit{Description:} Errors related to the incorrect assessment of the physics system's boundaries, the forces acting on it, or its constituent components. This involves a misunderstanding of `what' is happening in the system.
        \item \textit{Examples:}
        \begin{itemize}
            \item Neglecting the force of friction in a situation where it is significant.
            \item Incorrectly identifying the system boundary, leading to errors in applying conservation laws.
            \item Misjudging whether a system is isolated (no external forces) or not.
            \item Failing to consider all relevant forces acting on an object.
            \item Misidentifying the components of a system that are interacting.
        \end{itemize}
    \end{itemize}

    \item \textbf{Physics Process Understanding Errors:}
    \begin{itemize}
        \item \textit{Description:} Errors stemming from a misunderstanding of how a physics phenomenon develops, how states change, or the causal relationships between events. This involves a misunderstanding of `how' things are happening.
        \item \textit{Examples:}
        \begin{itemize}
            \item Incorrectly analyzing the motion of a projectile, such as misunderstanding the independence of horizontal and vertical motion.
            \item Misunderstanding the mechanisms of energy transformation (e.g., confusing heat and temperature).
            \item Incorrectly predicting the direction of motion based on the forces involved.
            \item Having misconceptions about the nature of a physics process (e.g., believing that a continuous force is needed to maintain constant velocity).
        \end{itemize}
    \end{itemize}

    \item \textbf{Variable Relationship Errors:}
    \begin{itemize}
        \item \textit{Description:} Errors caused by misunderstanding the dependencies or functional relationships between different physics quantities. This involves incorrectly relating variables.
        \item \textit{Examples:}
        \begin{itemize}
            \item Incorrectly assuming that acceleration is directly proportional to velocity.
            \item Misunderstanding the relationship between force, mass, and acceleration (Newton's Second Law).
            \item Confusing the relationship between potential and kinetic energy.
            \item Failing to recognize an inverse relationship between two variables.
        \end{itemize}
    \end{itemize}

    \item \textbf{Calculation Process Errors:}
    \begin{itemize}
        \item \textit{Description:} Errors occurring during the mathematical manipulation of equations, the derivation of formulas, or the substitution of numerical values. These are purely mathematical mistakes.
        \item \textit{Examples:}
        \begin{itemize}
            \item Making algebraic errors when rearranging equations.
            \item Incorrectly performing unit conversions (e.g., mixing up meters and centimeters).
            \item Making arithmetic errors (e.g., simple addition or multiplication mistakes).
            \item Incorrectly substituting values into a formula.
            \item Errors in using a calculator.
        \end{itemize}
    \end{itemize}

    \item \textbf{Boundary Condition Analysis Errors:}
    \begin{itemize}
        \item \textit{Description:} Errors resulting from neglecting or mishandling special cases, limiting conditions, or the applicable ranges of variables or equations. This involves not considering the "edges" of the problem.
        \item \textit{Examples:}
        \begin{itemize}
            \item Failing to consider the behavior of a system at extremely high or low temperatures.
            \item Neglecting the effects of air resistance when analyzing projectile motion at high speeds.
            \item Not considering the limitations of a particular model or approximation.
            \item Applying a formula outside its range of validity.
            \item Ignoring initial conditions or other constraints.
        \end{itemize}
    \end{itemize}
\end{enumerate}
\begin{figure*}[t]
\centering
\includegraphics[width=\textwidth]{fig_sup/example_sup_2.pdf}
\caption{A knowledge example in our benchmark.}
\label{fig:sup_1}
% \vspace{-12pt}
\end{figure*}

\begin{figure*}[t]
\centering
\includegraphics[width=\textwidth]{fig_sup/example_sup.pdf}
\caption{An easy example in our benchmark.}
\label{fig:sup_0}
% \vspace{-12pt}
\end{figure*}

\begin{figure*}[t]
\centering
\includegraphics[width=\textwidth]{fig_sup/example_sup_3.pdf}
\caption{A medium example in our benchmark.}
\label{fig:sup_2}
% \vspace{-12pt}
\end{figure*}

\begin{figure*}[t]
\centering
\includegraphics[width=\textwidth]{fig_sup/example_sup_4.pdf}
\vspace{-20pt}
\caption{A hard example in our benchmark.}
\label{fig:sup_3}
\end{figure*}
\section{Example}
\label{Example}
We have provided a representative example for each of the four question difficulty levels—knowledge (Figure \ref{fig:sup_1}), easy (Figure \ref{fig:sup_0}), medium (Figure \ref{fig:sup_2}), and hard (Figure \ref{fig:sup_3}) to serve as a guide.
\par
The knowledge-level problem demonstrates the fundamental application of electromagnetic principles, requiring direct use of basic physics theorems without complex problem-solving steps.
This type of question focuses on testing models' understanding of core concepts and their ability to apply basic formulas.
\par
The easy-level problem involves a straightforward mechanical system with clear physics conditions. 
It requires models to apply basic conservation laws and Newton's laws in a sequential manner, with each step building logically on the previous one. 
The solution path is direct and requires minimal manipulation.
\par
The medium-level problem introduces multiple state changes and requires models to analyze a system under different configurations. 
It combines several physics principles and demands a more sophisticated understanding of how different variables interact.
The solution requires models to track system changes systematically while maintaining consistency in their physics-based reasoning.
\par
The hard-level problem presents a complex mechanical system with multiple connected components and sequential events. 
It requires models to analyze a series of interactions, apply multiple physics principles simultaneously, and consider various constraints throughout the problem-solving process. 
The solution demands both careful physics insight and mathematical rigor, testing models' ability to synthesize different concepts and handle multi-step calculations.
\par
These examples demonstrate the progressive complexity in physics problem-solving across different difficulty levels. From knowledge-level questions testing basic concept application, to hard problems requiring integration of multiple physics principles and sophisticated analysis, each level builds upon the previous one.
This hierarchical structure effectively assesses models' comprehension and problem-solving abilities, ranging from fundamental understanding to advanced physics-based reasoning and mathematical manipulation. The gradual increase in complexity helps evaluate models' mastery of both individual concepts and their ability to synthesize multiple physics principles in complex scenarios.
\begin{table*}[t]
\centering
% \small
\caption{Model performance comparisons on the PhysReason-mini benchmark using answer-level evaluation across different input combinations of Questions (Q), Images (I), and Image Captions (IC).
Gemini-2.0-T$^{\dagger}$ and $^{*}$ represent Gemini-2.0-Flash-Thinking-1206 and 0121.}
% \vspace{-5pt}
% \begin{adjustbox}{width=\textwidth}
\begin{tabular}{lcccccc}
\hline
Model & Input & Knowledge & Easy & Medium & Hard & Avg. \\
\hline
\textbf{Non-O-like Models} \\
Qwen2VL-72B & Q, I & 25.40 & 27.00 & 11.4 & 8.5 & 18.07\\
InternVL2.5-78B & Q, I & 37.90 & 20.60 & 18.14 & 7.97 & 21.15\\
GPT-4o & Q, I & 51.12 & 31.95 & 20.75 & 12.54 & 29.09\\
Claude-3.5-Sonnet & Q, I & 49.00 & 40.43 & 23.45 & 12.33 & 31.3\\
Deepseek-V3-671B & Q, IC & 56.60 & 40.97 & 22.22 & 14.61 & 33.6\\
Gemini-2.0-Flash & Q, I  & 67.80 & 52.10 & 40.00 & 23.19 & 46.52\\
Gemini-2.0-Pro & Q, I & 69.32 & 53.67 & 44.98 & 26.24 & 48.55 \\
\hline
\textbf{O-like Models} \\
o1-mini & Q, IC & 54.80 & 30.33 & 15.41 & 7.92 & 27.11 \\
QvQ-72B & Q, I & 51.17 & 37.10 & 29.83 & 22.13 & 35.06 \\
QwQ-32B  & Q, IC & 64.4  & 50.07 & 38.88 & 27.45 & 45.20 \\
Gemini-2.0-T$^{\dagger}$ & Q, I & 71.47 & 49.97 & 36.83 & 22.97 & 45.42 \\
GLM-Zero  & Q, IC & 72.70 & 50.17 & 43.42 & 24.70 & 47.75 \\
% GLM-Zero & 71.47 & 49.97 & 36.83 & 22.97 & 45.42 \\
% Gemini2.0 Thinking-1206 & 72.70 & 50.17 & 43.42 & 24.7 & 47.75 \\
o1  & Q, I & 72.47 & 53.37 & 49.31 & 25.32 & 50.12 \\
o3-mini-high  & Q, IC & 71.10 & 63.20 & 47.02 & 31.93 & 53.31\\
Gemini-2.0-T$^{*}$ & Q, I & 76.33 & 56.87 & 51.85 & 32.61 & 54.42 \\
Deepseek-R1 & Q, IC & 85.17 & 60.77 & 47.24 & 33.23 & 56.60 \\
\hline
\end{tabular}
% \end{adjustbox}
\label{tab:30}
% \vspace{-10pt}
\end{table*}

% \begin{table*}[t]
% \centering
% \caption{Model performance comparisons on the PhysReason-mini benchmark using answer-level evaluation across different input combinations of Questions (Q), Images (I), and Image Captions (IC).
% Gemini-2.0-T$^{*}$ representS Gemini-2.0-Flash-Thinking-0121.}
% % \vspace{-5pt}
% % \begin{adjustbox}{width=\textwidth}
% \begin{tabular}{lcccccc}
% \hline
% Model & Input & Knowledge & Easy & Medium & Hard & Avg. \\
% \hline
% GPT-4o & Q, IC & 50.35 & 30.82 & 17.91 & 11.65 & 27.78\\
% \rowcolor{gray!20} GPT-4o & Q, I & 51.12 & 31.95 & 20.75 & 12.54 & 29.09\\
% \hline
% Gemini-2.0-Flash & Q, IC  & 67.04 & 50.23 & 36.35 & 21.30 & 43.73\\
% \rowcolor{gray!20} Gemini-2.0-Flash & Q, I  & 67.80 & 52.10 & 40.00 & 23.19 & 46.52\\
% \hline
% Gemini-2.0-T$^{*}$ & Q, IC & 75.53 & 55.30 & 47.72 & 29.53 & 52.02 \\
% \rowcolor{gray!20} Gemini-2.0-T$^{*}$ & Q, I & 76.33 & 56.87 & 51.85 & 32.61 & 54.42 \\
% \hline
% \end{tabular}
% % \end{adjustbox}
% \label{tab:31}
% % \vspace{-10pt}
% \end{table*}
\section{Evaluation Prompt}
\label{evaluation_prompt}
% \subsection{Prompt Design}
To systematically evaluate models' mathematical reasoning capabilities, we designed a structured prompt template that follows the zero-shot Chain-of-Thought (CoT) paradigm. 
This template adopts a hierarchical structure comprising image information, problem context, and sequential sub-questions, requiring models to provide standardized step-by-step solutions. The prompt structure consists of the following key components:

\subsection{Input Components}
\begin{itemize}
    \item \textbf{Image Caption:} For models without direct image processing capabilities, we utilize Gemini-2.0-flash to generate image descriptions as supplementary information
    \item \textbf{Context:} Provides the overall background and fundamental information of the problem
    \item \textbf{Sub-questions:} Decomposes complex problems into progressive sub-questions
\end{itemize}

\subsection{Output Specifications}
The template requests a structured output format with the following requirements:
\begin{itemize}
    \item Step-by-step reasoning for each sub-question
    \item Continuous step numbering across sub-questions
    \item One formula and its solution process per step
    \item Mathematical formulas enclosed in LaTeX notation (\$)
\end{itemize}
\par
This design adheres to the zero-shot Chain-of-Thought paradigm, facilitating systematic thinking through explicit step division and standardized output format, which benefits both model reasoning and subsequent performance evaluation. 
The template's flexibility allows it to accommodate pjhysical problems of varying complexity, with adjustable numbers of sub-questions and solution steps based on specific problem requirements.
% \section{Wrong Reasoning Process Case}
\section{Details of Experimental Result}
We previously presented only partial model performance benchmarks on PhysReason-mini. 
And we provide a comprehensive performance evaluation across all models, as shown in Table \ref{tab:30}.
\par
% Additionally, given the substantial number of images in our benchmark, we conducted experiments on models capable of processing image inputs. We categorized the experiments into two groups: those using images as direct input and those using image captions as input. The experimental results are presented in Table \ref{tab:31}.
% The experimental results reveal several interesting patterns in how models handle image inputs versus image captions for physics reasoning tasks. The performance comparison between direct image input and image caption approaches shows that while image-based reasoning generally yields better results, the difference is not dramatically large. 
% This suggests that while visual features provide additional valuable information for solving physics problems, captions can still capture much of the essential problem information.
% \par
% In examining the performance across difficulty levels, we observe that the gap between image and caption-based approaches becomes more pronounced in medium and hard categories. This pattern indicates that visual information becomes increasingly crucial for more complex physics problems, where subtle visual cues and spatial relationships may be more challenging to fully capture in textual descriptions. 
% The knowledge and easy categories show relatively smaller differences between the two input modes, suggesting that simpler physics concepts can be effectively communicated through either medium.
% \par
% These findings have important implications for model development and application: direct image processing capabilities are beneficial.
% And the relatively strong performance of caption-based approaches suggests that text-only models remain viable for many physics reasoning tasks, particularly at lower difficulty levels.
\section{Details of Scientific Artifacts}
Our PhysReason benchmark dataset integrates problems from multiple sources: International Physics Olympiad (2008-2021), Chinese National College Entrance Examination (2010-2024), Indian Joint Entrance Examination Advanced (2010-2024), Chinese provincial and municipal mock examination questions (2015-2024), and additional physics problems from Russian sources, totaling over 20,000 unique physics problems from 1,254 PDF documents across 27,874 pages. 
The dataset has been carefully curated to ensure comprehensive coverage while respecting intellectual property rights - all problems are utilized under the CC BY-NC-SA and MIT licenses, and all materials were collected in accordance with fair use principles for educational and research purposes. 
We maintain strict privacy standards with no personally identifiable information included, as all problems are sourced from standardized tests and competition materials. 
The complete benchmark and associated code are made available for research use, requiring users to comply with both the MIT License terms for our implementation and the respective original licenses (CC BY-NC-SA) for the educational materials, thereby ensuring proper attribution and usage rights while promoting academic accessibility.
\section{Details of Computational Experiment}
Our computational experiments were conducted across multiple Large Language Models (LLMs), Vision Language Models (VLMs), and other specialized models. The infrastructure primarily consisted of NVIDIA A800 GPUs for running open-source models. 
For model specifications, we evaluated seventeen models in total, including Qwen2-VL-72B (72 billion parameters), QwQ-32B (32 billion parameters), QvQ-72B (72 billion parameters), InternVL2.5-78B (78 billion parameters), and various other commercial models like GPT-4, Claude-3.5-Sonnet, and Gemini series. 
All experiments were conducted under a zero-shot Chain-of-Thought (CoT) setting to encourage step-by-step reasoning. For the experimental setup, we utilized specific prompts (detailed in supplementary materials) to maintain consistency across all evaluations. The models processed the complete PhysReason benchmark dataset, with the exception of O1 due to API limitations. 
For performance evaluation, we employed both PSAS-A and PSAS-S frameworks, with Deepseek-V3 ultimately selected as the scoring model based on efficiency and performance considerations.
Regarding implementation details, models that couldn't process visual inputs were supplemented with image captions generated by Gemini-2.0-Flash. 
For reproducibility purposes, all prompt templates are provided in the supplementary materials.
Due to the computational cost of the PSAS-S framework, some experiments were conducted using only the PSAS-A framework to maintain efficiency.
\section{Details of human annotators}
For data annotation and evaluation, we engaged four graduate students (including both PhD and Master's students) from engineering disciplines who are also co-authors of this paper. 
All annotators possessed strong backgrounds in both high school and undergraduate physics, making them well-qualified for this task. 
Since the annotators were co-authors actively involved in the research, no formal recruitment process or compensation was required, and they were fully aware of how the data would be used in the study. 
The annotation process focused solely on physics content evaluation and did not involve collecting any personal identifying information or expose annotators to any risks.
As this research involved co-authors analyzing academic content rather than external human subjects, it was determined to be exempt from formal ethics review board approval. 
The annotation work was conducted as part of regular academic research activities within our institution.  
No protected or sensitive demographic information was collected or used in this research.
\section{Details of Ai Assistants In Research Or Writing}
We used Claude-3.5-Sonnet, o1, o3-mini-high, and Deepseek-R1 to help us write code and polish the paper.
% Despite our efforts, this work has one limitation. 
% % First, our dataset shows geographic concentration, with problems primarily sourced from Asian educational systems, which may not fully represent global physics education approaches. 
% Success on benchmark problems may not directly translate to real-world physics applications, as the evaluation occurs in idealized testing conditions rather than practical scenarios. 
% These limitations suggest valuable directions for future work in physics-based reasoning evaluation for AI systems.

% \begin{algorithm*}
% \small
% \caption{Physics Solution Auto-Scoring (PSAS)}
% \begin{algorithmic}[1]
% \State \textbf{Phase 1: Data Extraction}
%     \State \textit{Input}: Model output $M$, annotated steps $S = \{s_1, s_2, ..., s_n\}$
%     \For{$s_i \in S$}
%         \State $E_i \gets \text{Normalize}(\text{LLM\_Extract}(M, s_i))$ \Comment{E: extracted steps}
%     \EndFor
%     \State Assert $|E| = |S|$
% \State \textbf{Phase 2: Scoring}
%     \For{$(e_i, s_i) \in (E, S)$}
%         \State $F_i \gets \text{ExtractFormula}(e_i)$ \Comment{Formula content}
%         \State $R_i \gets \text{ExtractResult}(e_i)$ \Comment{Calculation result}
%         \State $\text{score}_i \gets 0.5 \times \text{ScoreFormula}(F_i, s_i) + 0.5 \times \text{ScoreResult}(R_i, s_i)$
%     \EndFor
% \State \textbf{Phase 3: Error Analysis} 
%     \State $\text{ErrorSet } \Delta \gets \emptyset$
%    \State $\text{ErrorTypes } \mathcal{T} = \{\text{DAE}, \text{PLAE}, \text{PCAE}, \text{PPUE}, \text{VRE}, \text{CPE}, \text{BCAE} \}$ \Comment{Error categories}
%     \For{$(e_i, s_i) \in (E, S)$}
%         \If{$\text{score}_i < 1$}
%             \State $\text{error\_types} \gets \text{LLM}(\text{Template}(e_i, s_i, \mathcal{T}))$ \Comment{Constraint can only have one error}
%             \State $\Delta \gets \Delta \cup \{(i, \text{error\_types})\}$
%         \EndIf
%     \EndFor
%     \State \textit{Output}: $\{\text{score}_i\}_{i=1}^n$, $\text{ErrorSet }\Delta$

% \end{algorithmic}
% \label{alg:scoring}
% \end{algorithm*}
% \begin{algorithm*}
% \small
% \caption{Physics Problem Solution Auto-Scoring Process}
% \begin{algorithmic}[1]
% \State \textbf{Phase 1: Data Extraction}
%     \State \textit{Input}: Model output $M$, annotated steps $S = \{s_1, s_2, ..., s_n\}$
%     \For{$s_i \in S$}
%         \State $E_i \gets \text{Validate}(\text{Normalize}(\text{LLM\_Extract}(M, s_i)))$ \Comment{E: extracted steps}
%     \EndFor
%     \State Assert $|E| = |S|$ \Comment{Ensure extraction completeness}
% \State \textbf{Phase 2: Scoring}
%     \For{$(e_i, s_i) \in (E, S)$}
%         \State $F_i \gets \text{ExtractFormula}(e_i)$ \Comment{Formula content}
%         \State $R_i \gets \text{ExtractResult}(e_i)$ \Comment{Calculation result}
%         \State $\text{score}_i \gets \alpha \times \text{ScoreFormula}(F_i, s_i) + (1-\alpha) \times \text{ScoreResult}(R_i, s_i)$ \Comment{$\alpha \in [0,1]$}
%     \EndFor
% \State \textbf{Phase 3: Error Analysis \& Summary} 
%     \State $\text{ErrorSet } \Delta \gets \emptyset$
%     \State $\mathcal{T} = \{\text{DAE}, \text{PLAE}, ..., \text{BCAE}\}$ \Comment{7 error categories}
%     \For{$(e_i, s_i) \in (E, S)$}
%         \If{$\text{score}_i < 1$}
%             \State $\text{error\_types} \gets \text{LLM}(\text{Template}(e_i, s_i, \mathcal{T}))$ \Comment{Constraint to one error}
%             \State $\Delta \gets \Delta \cup \{(i, \text{error\_types})\}$
%         \EndIf
%     \EndFor
%     \State $\text{ErrorFreq}(\tau) = \frac{|\{(i,t) \in \Delta: t = \tau\}|}{|\Delta|}, \forall \tau \in \mathcal{T}$
%     \State \textit{Output}: $\{\text{score}_i\}_{i=1}^n$, $\text{ErrorSet }\Delta$, $\{\text{ErrorFreq}(\tau)\}_{\tau \in \mathcal{T}}$
% \end{algorithmic}
% \label{alg:scoring}
% \end{algorithm*}

\end{document}
