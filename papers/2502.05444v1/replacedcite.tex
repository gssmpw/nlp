\section{Related Works}
With the advent of generative AI, several works have been proposed to generate synthetic colonoscopy images. The main focus of existing related works has been to perform synthetic polyp generation irrespective of the pathological characteristics of polyps and the imaging modality. The techniques adopted so far can broadly be divided into two categories, namely, GAN, and Diffusion Models.  

\subsubsection*{Generative Adversarial Network (GAN) based Techniques}
The initial frameworks for polyp generation are based on the adversarial concept and adopt different variants of GANs. For example, Shin et al.____ used a conditional-GAN approach to translate normal colonoscopy images to polyp images. This translation is achieved using an input-conditioned image which is a combination of an edge map and a polyp binary mask. A similar concept of converting normal frames to polyp frames is proposed in____. They utilized a conditional GAN architecture to produce polyps with varied characteristics by controlling the input-conditioned binary mask values. Such conditional translation is also reported by Fagereng et al.____. They developed a framework called PolypConnect which uses an EdgeConnect model to convert clean colon images to polyps when given an edge map and a polyp mask. Sasmal et al.____ performed polyp generation using DCGAN and used the obtained synthetic polyps to enhance classifier performance for differentiating adenoma and hyperplastic polyps. An identical augmentation approach is followed by Adjei et al.____ using synthetic polyps generated using a Pix2Pix model. Unlike the traditional GAN architecture, He et al.____ introduced an attacker in the framework to obtain false negative images. Sams and Shomee____ utilized a StyleGAN2-ada to generate random binary masks, which are combined with colon images. This integrated image is used as an input for a conditional GAN to obtain synthetic polyp images. The above methods focused on polyp generation irrespective of the imaging modalities. However, a few works have used GAN-based approaches to transfer styles between different imaging modalities, such as WLI and NBI. Golhar et al. ____ utilized the GAN inversion approach, which uses a latent representation of images to perform translation between NBI and WLI modalities. Following this technique, interpolation methods are used to change the polyp size. Similarly, Bhamre et al.____ used CycleGAN to convert WLI images to NBI images. Although these few existing works established the significance of NBI images over WLI images in polyp classification, the generation of new synthetic polyp images with different imaging modalities has not been explored in the literature.      


\subsubsection*{Diffusion Model based Techniques}
The related literature involves only a few works focused on polyp image generation. Machacek et al.____ used a conditional diffusion probabilistic model to produce synthetic polyp images using synthetic masks. They validated the effectiveness of generated data by utilizing it for training polyp segmentation models. Pishva et al.____ performed polyp generation using two diffusion models. The two models are fine-tuned on cropped-out polyps and clean colon images, respectively. This fine-tuning is followed by performing an inpainting using the latter model and cropped-out images. Du et al.____ proposed an adaptive refinement semantic diffusion model which considers the polyp and background ratio to adjust the diffusion loss. They also incorporated a pre-trained segmentation model that modifies the refinement loss depending on the difference between the predicted mask of the synthetic polyp and the binary mask used for its generation. The above-mentioned approaches followed a similar pattern of polyp generation using binary masks. The impact of text prompt based training and the generation of colonoscopy images with different imaging modalities still remains unexplored.