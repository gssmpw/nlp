\documentclass[11pt,letter]{article}

\usepackage[letterpaper,top=.13in,bottom=.17in,left=.57in,right=.57in,includefoot]{geometry}
\usepackage[natbib=true,style=nature,sorting=none]{biblatex} %Imports biblatex package
\addbibresource{sample.bib} %Import the bibliography file
\usepackage{graphicx}
\usepackage{booktabs}
\usepackage{fancyhdr}
\usepackage[fleqn]{amsmath}
\usepackage{epstopdf}
\usepackage{amsfonts}
\usepackage{setspace}
%\usepackage{iopams}
\usepackage{bbm}
\usepackage{verbatim}
\usepackage{rotating}
\usepackage{multirow}
\usepackage{txfonts}
\usepackage{blindtext}
\usepackage{multicol}
\setlength{\columnsep}{0.28in}
\setlength{\parindent}{0mm}
\usepackage{caption}
\captionsetup[table]{skip=0pt,singlelinecheck=off}
\AtBeginEnvironment{tabular}{\small}
\usepackage{enumitem}
\setlist[itemize]{leftmargin=*,topsep=0pt,itemsep=-1ex,partopsep=1ex,parsep=1ex}

\usepackage[compact]{titlesec}
\titleformat*{\section}{\MakeUppercase}
%\renewcommand\thesection{1.section}
\renewcommand\thesection{\arabic{section}}
\renewcommand\thesubsection{\rm\arabic{section}.\arabic{subsection}}
\renewcommand\thesubsubsection{\rm\arabic{section}.\arabic{subsection}.\arabic{subsubsection}}
\titleformat*{\subsection}{\it}
\titleformat*{\subsubsection}{\it}

\renewcommand\thetable{\arabic{table}}
\renewcommand\thefigure{\arabic{figure}}

\def\labelitemi{--}

%%%%%%%%%%%%%%%%%%%%%%%%%%%%%%
\begin{document}\pagestyle{empty}
\parbox[][1.3in][t]{\textwidth}{%
  {\LARGE Bridging Structural Dynamics and Biomechanics: Human Motion Estimation through Footstep-Induced Floor Vibrations\\~\\} 
  {\Large Yiwen Dong$^{1,2}$,  Jessica Rose$^2$ \& Hae Young Noh$^1$}\\
  \emph{\large $^1$ Civil and Environmental Engineering, Stanford University\\
  $^2$ Department of Orthopaedic Surgery, Stanford Medicine}\medskip
}

%%%%%%%%%%%%%%%%%%%%%%%%%%%%%%
\parbox[][2.9in][t]{\textwidth}{%
%This study aims to estimate lower-limb joint motion during walking through gait-induced floor vibrations. 
ABSTRACT: Quantitative estimation of human joint motion in daily living spaces is essential for early detection and rehabilitation tracking of neuromusculoskeletal disorders (e.g., Parkinson's) and mitigating trip and fall risks for older adults. Existing approaches involve monitoring devices such as cameras, wearables, and pressure mats, but have operational constraints such as direct line-of-sight, carrying devices, and dense deployment. To overcome these limitations, we leverage gait-induced floor vibration to estimate lower-limb joint motion (e.g., ankle, knee, and hip flexion angles), allowing non-intrusive and contactless gait health monitoring in people's living spaces. To overcome the high uncertainty in lower-limb movement given the limited information provided by the gait-induced floor vibrations, we formulate a physics-informed graph to integrate domain knowledge of gait biomechanics and structural dynamics into the model. Specifically, different types of nodes represent heterogeneous information from joint motions and floor vibrations; Their connecting edges represent the physiological relationships between joints and forces governed by gait biomechanics, as well as the relationships between forces and floor responses governed by the structural dynamics. As a result, our model poses physical constraints to reduce uncertainty while allowing information sharing between the body and the floor to make more accurate predictions. We evaluate our approach with 20 participants through a real-world walking experiment. We achieved an average of 3.7 degrees of mean absolute error in estimating 12 joint flexion angles (38\% error reduction from baseline), which is comparable to the performance of cameras and wearables in current medical practices.

%To overcome this challenge, we integrate 1) gait biomechanics, where each combination of joint motion angles result in unique footstep forces, and 2) structural dynamics, where these unique footstep forces generate distinctive dynamic floor responses. Specifically, we develop a physics-informed graph transformer model that represents 1) the physiological relationships among lower-limb joints and 2) the dynamics between footstep force and floor vibration through heterogeneous nodes and edges, allowing heterogeneous information sharing between joint motion and floor vibration. We evaluate our approach with 20 participants through a real-world walking experiment. Results show that our model has an average of 3.7 degrees of mean absolute error in estimating 12 joint flexion angles (38\% error reduction from a Long Short-Term Memory (LSTM) baseline), which is comparable to the performance of cameras and wearables in current medical practices.
}

\begin{multicols*}{2}

%%%%%%%%%%%%%%%%%%%%%%%%%%%%%%
\section{Introduction}
Quantitative estimation of joint motion during walking is essential for clinical detection and rehabilitation of gait disorders such as diabetes and Parkinson's and mitigation of trip and/or fall risks for older adults~\cite{schulte1993quantitative}. Existing approaches for joint motion estimation involve sensing devices such as motion capture (MoCap) systems, video cameras, wearables, and pressure/force sensors, but have operational constraints when used outside the laboratory settings. MoCap systems are commonly used in clinical usage~\cite{pfister2014comparative}, but they require marker installment and dedicated calibration. Pressure/force sensors capture ground reaction forces~\cite{stauffer1977force}, but they lack body motion information. Video cameras and wearables capture body motion~\cite{aggarwal1999human,tao2012gait}, but they can raise privacy concerns and/or cause discomfort when carrying devices all the time. 

In this study, we leverage human gait-induced floor vibration to infer the lower-limb joint motion, which has the benefits of being non-intrusive, wide-ranged, and contactless. The intuition of this approach is that each joint motion combination from the ankle, knee, and hip exerts a unique footstep force to the floor, which generates a unique floor vibration pattern. We capture these vibrations using geophone sensors mounted on the floor surfaces to infer the joint motion combination, allowing gait health monitoring in people's living spaces.

However, the main challenge is the high uncertainty in joint motion given the limited and indirect measurement provided by floor vibrations.
%, making it challenging for conventional data-driven models to infer the ankle, knee, and hip joint angles accurately. 
To overcome this challenge, we represent the indirect relationship between the floor vibration and the joint motions through a physics-informed graphical model. The model integrates 1) gait biomechanics, which describes the physiological relationships among joint motions through the connections of muscles and bones, and 2) structural dynamics, which describes the dynamic floor responses under footsteps governed by the structural dynamics equations. In our graph, different types of nodes represent heterogeneous information from joint motions and floor vibrations; Their connecting edges represent the physiological relationship between joints and forces, as well as the dynamics between forces and floor responses. The formulation of the graph poses physical constraints while allowing information sharing among heterogeneous data. 

The contributions of the study are that we:

\begin{enumerate}
    \item Develop a novel approach to estimate lower-limb joint motion for gait health monitoring using footstep-induced floor vibrations;

    \item Integrate structural dynamics and gait biomechanics to formulate a new human-structure interaction system and develop a physics-informed graphical model to reduce uncertainties in joint motion estimation for gait health monitoring;

    \item Evaluate our approach through a real-world experiment and obtain promising results.
\end{enumerate}


We evaluated our approach with 20 participants for 4 gait types commonly observed in clinics, through collaboration with the Lucile Packard Children's Hospital at Stanford. 
%Our approach achieved an average of 3.70 degrees of mean absolute error in estimating 12 critical joint flexion angles, which has reduced $38\%$ of the error from the baseline LSTM model (6.02 degrees). 
The accuracy is comparable to other sensing approaches used in medical practices such as cameras and wearables~\cite{majumder2020wearable,finkbiner2017video}.

\section{Bridging Structural Dynamics and Gait Biomechanics}\label{sec:2}
We formulate a new human-structure interaction (HSI) system by integrating structural dynamics and gait biomechanics through their common connection with the ground reaction forces (see Figure~\ref{fig:newHSI}). Our formulation closes the gap in existing work by inferring human posture for health~\cite{dong2024graphical}.
% identify the research gap in the existing human-structure interaction system and develop a new formulation by

% \subsection{Background of Human-Structure Interaction Systems}
% Existing human-structure interaction (HSI) systems leveraged structural dynamics equations to describe the characteristics of floor vibration induced by footstep forces, shown in Figure~\ref{fig:existingHSI}.

% \includegraphics[width=3.53in]{Figures/existingHSI.pdf}
%     \captionof{figure}{Diagram of the existing HSI systems.}\label{fig:existingHSI}

% This formulation considers both the human body and the floor structure as a lumped system with mass, stiffness, and viscous damping~\cite{caprani2016formulation}. The floor structure is typically modeled as a simply supported beam within the linear elastic range, which has multiple degrees of freedom system by discretizing it over the floor span. 

% Under these assumptions, the dynamic HSI system is represented as [8]:
% \begin{equation}
% \begin{aligned} 
% \begin{bmatrix} M_b & 0\\ 0 & m_p \end{bmatrix} 
% \begin{Bmatrix} \ddot{d} \\ \ddot{y} \end{Bmatrix}
% + \begin{bmatrix} C_b + c^* & -c_p N^T \\ -c_p N & c_p \end{bmatrix}
% \begin{Bmatrix} \dot{d} \\ \dot{y} \end{Bmatrix} \\
% + \begin{bmatrix} K_b + k^* & -k_p N^T \\ c_p v N_x k_p N & k_p \end{bmatrix}
% \begin{Bmatrix} d \\ y \end{Bmatrix}
% = \begin{Bmatrix} N^T G(t) \\ 0 \end{Bmatrix}
% \end{aligned}
% \end{equation}
% where $M_b, C_b, K_b$ represents the structural properties of the beam, $m_p, c_p, k_p$ represents the person's dynamic property when walking, $v$ is the walking speed, $N$ is the global shape function along the beam span, $G(t)$ is the footstep force, $d, y$ are the displacement of the structure and the person respectively.

% This formulation aims to assess the serviceability of the floor under pedestrian loading so it does not consider the gait health aspect of the person. As a result, the simplification made on the human body does not involve joint motion characteristics, making it unable to assess gait health during walking. 
% Since the footstep force is dominated by the vertical direction which is usually proportional to the body weight and approximately periodic, $F_l(t)$ is typically modeled as a periodical function defined by a Fourier series:
% \begin{equation}
%     F_l(t) = W\sum_{k=0}^{r}\eta_{k}cos(2\pi k f_p t + \varphi_k)
% \end{equation}
% where $W = m_p g$ is the weight of the pedestrian, $f_p$ is the pacing frequency, $\eta_k$ is the dynamic load factor of the $k_{th}$ harmonic, and $\varphi_k$ is the phase angle.
% This model

\includegraphics[width=3.53in]{Figures/concept_graph.pdf}
\captionof{figure}{Conceptual diagram of our new HSI system.}\label{fig:newHSI}

\vspace{0.1in}

% \subsection{Bridge between Gait Biomechanics and Structural Dynamics using Ground Reaction Forces}
%We develop a new formulation of the HSI system to infer human gait health from floor vibration. While there are existing HSI systems to assess the floor vibration under pedestrian loading for serviceability requirements, they do not consider the posture of the person, making them unable to assess gait health during walking. Our formulation integrates gait biomechanics with structural dynamics to bridge the gap between structural vibration sensing and gait health monitoring, illustrated in Figure~\ref{fig:newHSI}.

The physical insight of our HSI system can be divided into two parts. First, the human gait (represented by joint angles and body properties) exerts forces onto the floor, which is governed by gait biomechanics; Then, the ground reaction forces induce floor vibrations, which are affected by the floor properties and governed by structural dynamics equations. The details of these two parts are described below. %Our system bridges gait biomechanics and structural dynamics to establish the relationship between joint motion and floor vibration.

\textbf{Gait Biomechanics.} The inverse dynamics in gait biomechanics describe the relationship between joint angles and ground reaction forces, where each section of the leg is analyzed through a free-body diagram with forces and moments applied. The upper part of Figure~\ref{fig:newHSI} shows an example of the biomechanics of the shank. The complete analysis involves thigh, shank, and foot sections, where the hip, knee, and ankle joints are regarded as hinges for force/moment transmission. The above equations assume the body is symmetrical, the foot has negligible mass, and the motion in the frontal plane is negligible.

\textbf{Structural Dynamics.} The dynamics of the floor structures under footstep forces are typically represented through the equation of motion, as shown in the lower part of Figure~\ref{fig:newHSI}. The equation suggests that, as the foot exerts forces $F_1(t)$ to the floor, the resultant vibration which depends on the mass, stiffness, and damping of the floor, is captured by the sensor mounted on the floor surface as velocity $\dot{u}(t)$ or acceleration $\ddot{u}(t)$.

Our HSI formulation forms a complete chain of physical relations from human gait to floor vibrations, allowing formal analysis and inference of gait health information.

\begin{figure*}
\includegraphics[width=7in]{Figures/gaitgraph_emi_nomodel.pdf} 
\captionof{figure}{Diagram of our physics-informed graph (PIG) describing the relationships between hip, knee, ankle joint motions, floor vibrations, gait cycle time, and body properties. The nodes of the graph are represented as solid circles with various colors. The edges are represented as arrows. %Converting the conceptual model (right) to the physics-informed heterogeneous graph (left) to describe the relationships between hip, knee, and ankle joints and floor vibrations.
}\label{fig:graph}
\end{figure*}

\section{Joint Motion Estimation through Physics-Informed Graphical Model}

To infer joint motion from floor vibration for gait health assessments, we develop a physics-informed graphical (PIG) model based on our HSI formulation. We first model the gait and floor information through a heterogeneous graph with nodes and edges, and then design the information flow in the graph to pose physical constraints during model training to reduce uncertainty.

\subsection{Modeling Gait and Floor Information}
The graph consists of 4 types of nodes to represent the gait and floor information and 5 types of edges to model the physiological and structural dynamics relationships. The nodes are depicted as dots in Figure~\ref{fig:graph}, summarized below:
\begin{itemize}
    \item \textbf{Joint Nodes:} There are three types of joint nodes, each representing the hip, knee, and ankle joint motion. The information stored in each joint node is the magnitude of critical flexion/extension angles over each gait cycle, as described in Figure~\ref{fig:graph}. These are chosen because they are important indicators of gait abnormalities and inform potential trip/fall risks for doctors in gait clinics.
    \item \textbf{Time Nodes:} The time nodes contain important moments when critical joint angles happen in a gait cycle, including the foot strike time and foot off time that divides a gait cycle into the stance and swing phases.
    \item \textbf{Vibration Nodes:} The vibration nodes store the vibration generated on the floor recorded by sensors, which contains a vibration signal segment based on the beginning and the end of a gait cycle.
    \item \textbf{Body Nodes:} The body nodes describe the anthropometry of the walker, such as the body weight and leg lengths. These are important variables to determine the ground reaction forces, as discussed in Section~\ref{sec:2}.
\end{itemize}
\vspace{0.1in}

The relationships between various nodes are defined by edges. Figure~\ref{fig:graph} summarizes these relationships by presenting the relative position of joints and sensors in space (vertical direction), as well as the critical joint motions and vibration data in time (horizontal direction). These relationships are represented by various types of edges:
\begin{itemize}
    \item \textbf{Spatial Edges:} The spatial edges represent the physiological connection among hip, knee, and ankle joints. Since the change of motion in one joint affects the others as they are connected through muscles and bones, the spatial edges model such dependency and allow information sharing among various joints.
    \item \textbf{Temporal Edges:} The temporal edges connect within the same type of joint nodes, which describes the sequence of motion over a gait cycle such as the knee extension at the footstrike, flexion at loading time, extension at the foot off, and flexion during the swing phase~\cite{dong2024graphical}. 
    \item \textbf{Indirect Edges:} The indirect edges refer to the connection between the joint and vibration nodes. These edges encode the indirect relationship between the joint motion and the vibration data, which allows joint motion estimation through dynamic floor responses.
    \item \textbf{Time Constraint Edges:} The time constraint edges connect the time nodes with the joint nodes, representing the relationship between joint motion and the gait cycle, as described in our prior work~\cite{dong2023structure,dong2024ubiquitous}.
    \item \textbf{Body Dimension Constraint Edges:} These edges connect the joint nodes with body nodes that describe lower-limb lengths, allowing joint forces and moments to be estimated through gait biomechanics.
    \item \textbf{Force Constraint Edges:} This constraint bridges body weight and floor vibration through the ground reaction force. The main insight is that the ground reaction force is typically proportional to the body weight, resulting in larger vibration amplitudes.
\end{itemize}

The main benefit of this physics-informed graph is to incorporate complex dependencies and integrate heterogeneous information over time and space. By establishing the relationship among joints, vibrations, body properties, and gait cycle time, our model systematically reduces uncertainties for gait health monitoring.

\subsection{Posing Physical Constraints to Reduce Uncertainty}
In this section, we introduce the physics-informed graph (PIG) model, which allows training on a physics-informed graph for joint angle estimation. One main challenge in model training is the data requirement - due to the high complexity of graph formulation, it requires a large amount of walking data from each person, which is not practical for people with walking impairments. To overcome this challenge, %our model actively controls the information flows along the edges of our graph by enforcing physical equations in structural dynamics and gait biomechanics. 
we pose physical constraints to reduce uncertainty in data by enforcing equations in structural dynamics and gait biomechanics to control the data flow along the edges of our graph. This improves data efficiency and ``teaches'' our model to follow physical laws.


\includegraphics[width=3.53in]{Figures/graph_dynamics_emi.pdf}
\captionof{figure}{Information flow between the vibration nodes and force nodes to enforce structural dynamics equation.
}\label{fig:dynamics_flow}
% \subsubsection{Enforcing Structural Dynamics Through Multi-Node Aggregation}\label{sec:structmodel}
\vspace{0.1in}

First, we enforce structural dynamics by controlling how information aggregates from various vibration nodes to the force nodes, illustrated in Figure~\ref{fig:dynamics_flow}. Given that the relationship between ground reaction force $F_i(t)$ and floor vibration $u(t)$ is governed by the equation of motion $Mu(t) + C\dot{u}(t) + K\ddot{u}(t) = F_i(t)$, we first use the summation as our aggregation function. Then, we infer the typically unknown structural properties (mass, damping, and stiffness of the floor) in practical scenarios by developing ``structure property learners'' (represented as the long-short term memory (LSTM) modules in Figure~\ref{fig:dynamics_flow}) to implicitly extract the structural information. This controls information flow and enforces structural dynamics in our PIG model.

% \subsubsection{Modeling Gait Biomechanics Through Node-Level Transformation}\label{sec:biomodel}
Similarly, we enforce the gait biomechanics equations to reduce uncertainties in joint nodes. First, we transform the features at each joint to a space defined by $sin$ and $cos$ and body dimensions to the weights of leg sections and lengths of moment arms to approximate the biomechanics equation in Figure~\ref{fig:newHSI}. Then, we leverage the attention mechanism~\cite{brody2021attentive} to determine the importance of each transformed feature and create multiplied terms between joint and body nodes. Finally, we aggregate the information among the multiplied terms to pass on to the force node. Combining all the steps above, we enforce gait biomechanics in our PIG model.

% There are three main operators in the model: 1) Attention, which estimates the importance of each source node with respect to the target node; 2) Message Passing, which extracts the message using the node feature and pass the message to the connecting nodes; and 3) Aggregation, which combines the neighborhood message by the attention weight to estimate the target node for the next iteration. We bring in physical insights of gait biomechanics and structural dynamics through the design of these operations. 

% \subsubsection{Modeling Gait Biomechanics Through Node-Level Transformation}\label{sec:biomodel}
% We model gait biomechanics through feature transformations on the joint and body nodes. We encode the relationship between joint angles and ground reaction forces introduced in Section~\ref{sec:2} by transforming the node features during message passing and leveraging the attention mechanisms to automatically select the most relevant body dimension features with the resultant floor vibration. 

% First, we transform the joint nodes $\Theta_j$ to model the term $sin(\theta_1 \ddot{\theta_1})$ and/or $cos(\theta_1 \dot{\theta_1}^2)$ in gait biomechanics equations through message passing. Conventional message passing only considers the linear encoding, making it unable to capture the derivatives and the periodic functions in our problem. To address this issue, we model the derivatives as a non-linear transformation of the original angle and incorporating periodic functions through message passing:
% \begin{equation}\begin{aligned}
% \textbf{Msg}_{PIG} (\Theta_j) = W_0 sin(\sigma(W_1 H^{l-1}[\Theta_j])\cdot H^{l-1}[\Theta_j])
% \end{aligned}\end{equation}
% where $H^{l-1}[\Theta_j]$ represents the node feature from the previous iteration $l-1$, $W_0, W_1$ are vectors of linear coefficients, and $\sigma(\cdot)$ is the non-linear operator.

% Then, we transform the body nodes $P_j$ to infer latent variables such as the distance of center of mass $r_1$ and the moment of inertia $I_1$. Assuming that these variables can be estimated through the leg lengths and body weight, a neural network layer is used to transform these features:
% \begin{equation}
% \textbf{Msg}_{PIG} (P_j) = \sigma(W H^{l-1}[P_j])
% \end{equation}
% Next, we leverage mutual attention between body and joint nodes to determine the importance of each body property (e.g., choose between $r_1$ or $I_1$) when multiplying with the message passed from various joints:
% \begin{equation}
% \begin{aligned}
% & \textbf{Att}_{PIG}(P_i,\Theta_j) \\
% & = \underset{\forall i \in N(\Theta_j)}{Softmax} (W^{ATT}(\textbf{Msg}_{PIG} (P_i) || \textbf{Msg}_{PIG} (\Theta_j)))
% \end{aligned}
% \end{equation}
% where $W^{ATT}$ represents the importance of each body node with respect to the joint node by weight. This operation allows the model to estimate the terms that bridge joint angles with the ground reaction forces (e.g., $m_1 r_1 sin(\theta_1 \ddot{\theta_1})$ in Figure~\ref{fig:conceptgraph}).

% Finally, we combine the message from each joint node with the weight by a linear summation aggregation:
% \begin{equation}
% \textbf{Agg}_{PIG}(\cdot) = \sigma(\sum \textbf{Msg}_{PIG} (\Theta_j))
% \end{equation}

% This aggregation results in the ground reaction forces given by gait biomechanics equations.

% \subsubsection{Enforcing Structural Dynamics Through Multi-Node Aggregation}
% We enforce structural dynamics through the aggregation operation among various vibration nodes. Since the relationship between ground reaction force and floor vibration is a linear combination of $u(t), \dot{u}(t), \ddot{u}(t)$ in the equation of motion, we design the following aggregation to reflect structural dynamics:
% \begin{equation}
% \begin{aligned}
% & \textbf{Agg}_{PIG}(u_i(t), \dot{u}_i(t), \ddot{u}(t))  \\ &= w_{i1} H^{l-1}[u_i(t)] + w_{i2} H^{l-1}[\dot{u}_i(t)] + w_{i3} H^{l-1}[\ddot{u}(t)]
% \end{aligned}
% \end{equation}
% where $w_{i1}, w_{i2}, w_{i3}$ are vectors representing the stiffness, damping, and mass of the structure at the sensing location $i$. The aggregation of these nodes results in the ground reaction force, which is aligned with the results from the mutual attention aggregation of Section~\ref{sec:biomodel}.

% \subsubsection{Adapting Joint Dependencies to Various Gait Types}
% Our model can adapt to various types of gait by changing the spatial edges (which describes the dependencies among ankle, knee, and hip joint angles) according to the gait types while keeping the rest of the edges the same. This is based on the insight that the change of gait types does not change the equation forms in gait biomechanics or structural dynamics - it changes the dependency among various joints only. Therefore, we design adapters on top of the existing spatial edges, converting one type of joint dependencies to another:
% \begin{equation}
%     H^l(\Theta_j) = \sigma(\sum_{u\in N(\Theta_j)} [w_0^l \quad Adapt(g)]^T \frac{H^{l-1}_u}{|N(\Theta_j)|} )
% \end{equation}
% where $N(\Theta_j)$ denotes the neighborhood joint nodes connected with $\Theta_j$, $Adapt(g)$ is a neural network that convert one-hot vector $g$ into an embedding that describes the dependency between every two joints, and $w_0^l$ denotes the rest of the edge features that remain unchanged across different gait types. Compared with the existing data-driven models that requires domain adaptation or re-training when it comes to new gait types, the joint dependency adapters can be easily trained to accommodate new types of gait without a large amount of training data.

% rephrase
\section{Real-World Evaluation with Commonly Observed Gait Types}
We evaluate our approach through a 20-subject walking experiment by collaborating with medical doctors. The experiment setup and results are discussed in this section.

\subsection{Experiment Setup}
During the experiment, four commonly observed gait types are tested, including normal gait, toe-walking, flexed-knee, and gait with foot drag (see Figure~\ref{fig:exp}). Each participant first walks with their healthy natural gait for 20 trials, and then with ``simulated'' abnormal gait under the instruction of gait experts from Lucile Packard Children's Hospital for 10 trials each. All experiments are conducted following the approved IRB (IRB-55372).

\includegraphics[width=3.5in]{Figures/experiment_emi.pdf}
\captionof{figure}{Experiment setup with geophones and Motion Capture (MoCap) cameras for real-world walking.
}\label{fig:exp}
\vspace{0.1in}

The vibration data collection system includes four SM-24 geophone sensors mounted on the floor surface with a 500 Hz sampling frequency. The signals from the sensors are amplified by 500$\times$ to improve the signal-to-noise ratio. The ground truth joint motions are captured by a Vicon MoCap System with a frame rate of 100 fps. During the experiment, ten infrared cameras recorded 3-dimensional trajectories of lower-limb joints when walking. The joint angles are computed by the Vicon Plug-in Gait lower body model. A Vicon Lock Lab system is used to synchronize the vibration data with the lower-limb motion data.

\subsection{Results and Discussion}
Our approach has an average of 3.7 degrees of mean absolute error (MAE) in estimating 12 joint flexion angles on test data, which significantly outperforms the existing baseline (38\% error reduction) with 5.1 degrees MAE by using a fine-tuned LSTM model. The results breakdown for each motion segment is shown in Figure~\ref{fig:results}.


\includegraphics[width=3.53in]{Figures/results_emi.pdf}
\captionof{figure}{Results comparison between baseline model (LSTM) and our physics-informed graphical (PIG) model. 
}\label{fig:results}


\subsubsection{Comparison among Various Joints.} Among all the joint motion segments, our method has a relatively lower error on the ankle joint angles. This may be because the ankle motion directly influences the interaction between the foot and the floor, making it easier to infer from floor vibration. On the other hand, the swing phase joint angles have higher errors than the stance phase. This may be due to the lack of contact between the foot and the floor when the leg swings in the air. 

\subsubsection{Comparison among Various Gait Types}
Our results in joint angle estimation for normal walking (1.7 degree MAE) is significantly lower than that of the abnormal gait (4.4 degree MAE). This is because normal walking patterns are relatively consistent among various trials while the abnormal gait has significantly higher variance due to the instability of the posture.

\subsubsection{Comparison with Various Sensing Devices.} Our approach has comparable accuracy with the existing state-of-the-art sensing devices. For example, Majumder et al. evaluated the on-body wearables and reported a 2 to 3.4 degree of RMSE error~\cite{majumder2020wearable}. Finkbiner et al. developed video-based human pose estimation and reported 3.1 to 5.8 degrees of MAE for estimating joint angles~\cite{finkbiner2017video}. By comparison, our approach (3.7 degrees MAE) has a similar scale of accuracy while providing a non-intrusive and device-free user experience.

\section{Conclusion}
In this study, we estimate joint motions using footstep-induced floor vibrations by integrating structural dynamics and gait biomechanics. To overcome the high uncertainty challenge, we develop a physics-informed graphical model to enforce structural dynamics and gait biomechanics equations. Through a walking experiment with 20 people, we obtained 3.7 degrees of MAE in joint angle estimation, which is comparable to the existing portable devices.

\section*{References}
{\fontsize{9pt}{11pt}\selectfont
[1] Schulte, L., et al. (1993). A quantitative assessment of limited joint mobility in patients with diabetes. Arthritis \& Rheumatism: Of-ficial Journal of the American College of Rheumatology, 36(10), 1429-1443.\par
[2] Pfister, A., West, A. M., Bronner, S., \& Noah, J. A. (2014). Compara-tive abilities of Microsoft Kinect and Vicon 3D motion capture for gait analysis. Journal of medical engineering \& technology, 38(5), 274-280.\par
[3] Stauffer, R. N., Chao, E. Y., \& Brewster, R. C. (1977). Force and mo-tion analysis of the normal, diseased, and prosthetic ankle joint.  Clinical Orthopaedics and Related Research (1976-2007), 127, 189-196.\par
[4] Tao, W., Liu, T., Zheng, R., \& Feng, H. (2012). Gait analysis using wearable sensors. Sensors, 12(2), 2255-2283.\par
[5] Aggarwal, J. K., \& Cai, Q. (1999). Human motion analysis: A review. Computer vision and image understanding, 73(3), 428-440.\par
[6] Finkbiner, M. J., et al. (2017). Video movement analysis using smart- phones (ViMAS): a pilot study. Journal of Visualized Experiments, 121, e54659.\par
[7] Majumder, S., \& Deen, M. J. (2020). Wearable IMU-based system for  real-time monitoring of lower-limb joints. IEEE sensors journal, 21(6), 8267-8275.\par
[8] Dong, Y., \& Noh, H. Y. (2023). Structure-Agnostic Gait Cycle Segmen-tation for In-Home Gait Health Monitoring Through Footstep-Induced Structural Vibrations. Society for Experimental Mechan-ics Annual Conference and Exposition (65-74). Springer Nature. \par
[9] Dong, Y., \& Noh, H. Y. (2024). Ubiquitous gait analysis through footstep-induced floor vibrations. Sensors, 24(8), 2496. \par
[10] Dong, Y., Liu, J., Kim, S. E., Schadl, K., Rose, J., \& Noh, H. Y. (2024). Graphical Modeling of the Lower-Limb Joint Motion from the Dynamic Floor Responses Under Footstep Forces. In IMAC, A Conference and Exposition on Structural Dynamics (pp. 9-16). Springer Nature Switzerland. \par
[11] Brody, S., Alon, U., \& Yahav, E. (2021). How attentive are graph atten-tion networks? arXiv preprint arXiv:2105.14491.\par
}

%%%%%%%%%%%%%%%%%%%%%%%%%%%%%%
% \section{General Instructions}


% \subsection{General rules and submission details}
% Paper should not exceed 4 pages.  Name your file as follows: First three letters of the file name should be the first three letters of the last name of the first author, the second three letters should be the first letter of the first three words of the title of the paper (e.g. this paper: balpat.doc).


% \subsection{Type area}
% The text should fit exactly into the type area of 187 $\times$ 272 mm (7.36" $\times$ 10.71"). For correct settings of margins in the Page Setup dialog box (File menu) see Table \ref{tab:1} or simply use directly the settings of the provided word template. 

% ~
% \captionof{table}{Margin settings letter size paper (use letter size format).}
% \label{tab:1}
% \begin{tabular}{lllcll}\toprule
%     Setting	& \multicolumn{2}{l}{A4 size paper}	&& \multicolumn{2}{l}{Letter size paper} \\\cline{2-3}\cline{5-6}
%      & cm & inches && cm & inches\\\midrule
%     Top & 1.2 & 0.47" && 0.32 & 0.13"\\
%     Bottom &  1.3 & 0.51" && 0.42 & 0.17"\\
%     Left & 1.15 & 0.45" && 1.45 & 0.57"\\
%     Right & 1.15 & 0.45" && 1.45 & 0.57"\\
%     All other & 0.0 & 0.0" && 0.0 & 0.0"\\
%     Column width* & 9.0 & 3.54" && 9.0 & 3.54"\\
%     Column spacing*	& 0.7 & 0.28" && 0.7 & 0.28"\\
%     \bottomrule
% \end{tabular}\\
% *Column dialog box in Format menu.

% \subsection{Typefont, typesize, and spacing}
% Use Times New Roman 11 point size and 13 point line spacing (Normal;text tag). Use roman type except for the headings (Heading tags), parameters in mathematics (not for log, sin, cos, ln, max., d (in dx), etc), and the titles of journals and books which should all be in italics. You can use smaller font (10 points or 11 points) for tables (Table tags), figure captions (Figure caption tag) and the references (Reference text tag).

% Never use letterspacing and never use more than one space after each other.

% \subsection{Title, author and affiliation frame}
% Provide the paper title and the author/affiliation information in the top frame. Type the name of the first author (first the initials and then the last name) and if any of the co-authors have the same affiliation as the first author, add his/her name after an \& (or a comma if more names follow). Type the correct affiliation (Name of the institute, City, State/Province, Country). If there are authors linked to other institutes, type the name(s) of the author(s) and after a return the affiliation. Repeat this procedure until all affiliations have been typed. 
% All these texts fit in a frame which should not be changed (Width: Exactly 187 mm (7.36"); Height: Exactly 73 mm (2.87") from top margin; Lock anchor).

% \subsection{Title, author and affiliation frame}
% Provide an abstract of 100-200 words within the provided second frame. 

% \section{Layout of Text}
% \subsection{Text and indenting}
% Text is set in two columns of 9 cm (3.54") width each with 7 mm (0.28") spacing between the columns. All text should be typed in Times New Roman, 11 pt on 13 pt line spacing except for the paper title (18 pt on 20 pt), author(s) (14 pt on 16 pt), and the small text in tables, captions and references (10 pt on 11 pt). All line spacing is exact. Never add any space between lines or paragraphs. When a column has blank lines at the bottom of the page, add space above and below headings.


% First lines of paragraphs are indented 5 mm (0.2") except for paragraphs after a heading or a blank line (First paragraph tag).

% \subsection{Headings}
% Type primary headings in capital letters roman (Heading 1 tag) and secondary and tertiary headings in lower case italics (Headings 2 and 3 tags). Headings are set flush against the left margin. The tag will give two blank lines (26 pt) above and one (13 pt) beneath the primary headings, 1 1/4  blank lines (16 pt) above and a ½ blank line (6 pt) beneath the secondary headings and one blank line (13 pt) above the tertiary headings. Headings are not indented and neither are the first lines of text following the heading indented. If a primary heading is directly followed by a secondary heading, only a ½ blank line should be set between the two headings. 

% \subsection{Listing and numbering}
% When listing facts use either the style tag List signs or the style tag List numbers.

% \subsection{Equations}
% Use the equation editor of the selected word processing program. Equations are not indented (Formula tag). Number equations consecutively and place the number with the tab key at the end of the line, between parentheses. Refer to equations by these numbers. See for example Equation \ref{eq:1} below:

% From the above we note that $\sin \theta = (x + y) \times z$ or:
% \begin{equation}
%     K_t = \left(1-\frac{R^2\tau}{c_a + \nu\tan\delta} \right)^4 k_1 \label{eq:1}
% \end{equation}
% where $c_a =$ interface adhesion; $\nu=$ friction angle at interface; and $k_1 =$ shear stiffness number.
% For simple equations in the text always use superscript and subscript.

% \subsection{Tables}
% Locate tables close to the first reference to them in the text and number them consecutively. Avoid abbreviations in column headings. Indicate units in the line immediately below the heading. Type all text in tables in small type: 10 on 11 points (Table text tag). Type the caption above the table to the same width as the table (Table caption tag). See for example Table 1.

% \subsection{Figure captions}
% Always use the Figure caption style tag (10 points size on 11 points line space). Place the caption underneath the figure. Type as follows: ‘Figure 1. Caption.’ Leave one line of space between the figure caption and the text of the paper.

% \subsection{References}
% In the text, place the authors’ last names (without initials) and the date of publication in parentheses (see examples in Section 5). At the end of the paper, list all references in alphabetical order underneath the heading REFERENCES (Reference heading   tag). The references should be typed in small text (10 pt on 11 pt) and second and further lines should be indented 5.0 mm (0.2") (Reference text tag). If several works by the same author are cited, entries should be chronological:

% {\fontsize{9pt}{11pt}\selectfont
% Larch, A.A. 1996a. Development ...\par
% Larch, A.A. 1996b. Facilities ...\par
% Larch, A.A. 1997. Computer ...\par
% Larch, A.A. \& Jensen, M.C. 1996. Effects of ...\par
% Larch, A.A. \& Smith, B.P. 1993. Alpine ...\par}

% \subsubsection{Typography for references}

% {\fontsize{9pt}{11pt}\selectfont
% Last name, First name or Initials (ed.) year. Book title. City: Publisher.\par
% Last name, First name or Initials year. Title of article. Title of Journal (series number if necessary) volume number (issue number if necessary): page numbers.\par
% }

% \subsubsection{Examples}
% {\fontsize{9pt}{11pt}\selectfont
% Grove, A.T. 1980. Geomorphic evolution of the Sahara and the Nile. In M.A.J. Williams \& H. Faure (eds), The Sahara and the Nile: 21-35. Rotterdam: Balkema.\par
% Jappelli, R. \& Marconi, N. 1997. Recommendations and prejudices in the realm of foundation engineering in Italy: A historical review. In Carlo Viggiani (ed.), Geotechnical engineering for the preservation of monuments and historical sites; Proc. intern. symp., Napoli, 3-4 October 1996. Rotterdam: Balkema.\par
% Johnson, H.L. 1965. Artistic development in autistic children. Child Development 65(1): 13-16.\par
% Polhill, R.M. 1982. Crotalaria in Africa and Madagascar. Rotterdam: Balkema.%
% }

% \subsection{Notes}
% These should be avoided. Insert the information in the text. In tables the following reference marks should be used: *, **, etc. and the actual footnotes set directly underneath the table.

% \subsection{Conclusions}
% Conclusions should state concisely the most important propositions of the paper.

% \section{Photographs and figures}
% Number figures consecutively in the order in which reference is made to them in the text, making no distinction between diagrams and photographs. Figures should fit within the column width of 90 mm (3.54") or within the type area width of 187 mm (7.36"). Figures, etc. should not be centered, but placed against the left margin. Leave about one line of space between the actual text and figure (including caption). 	
% Never place any text next to a figure.


% \includegraphics[width=3.53in]{Picture1.png}
% \captionof{figure}{Caption of a typical figure.}\label{fig:1}



% \section{Preferences, symbols and units}
% Note the spacing, punctuation and caps in all the examples below.
% \begin{itemize}
%     \item \emph{References in the text:} Figure 1, Figures 2-4, 6, 8a, b (not abbreviated)
%     \item \emph{References between parentheses:} (Fig. 1), (Figs 2-4, 6, 8a, b) (abbreviated)
%     \item USA / UK / The Netherlands \emph{instead of} U.S.A. / U.K. / The Netherlands / the Netherlands
%     \item Author \& Author (1989) \emph{instead of} Author and Author (1989)
%     \item (Author 1989a, b, Author \& Author 1987) \emph{instead of} (Author, 1989a,b; Author and Author, 1987)
%     \item (Author et al. 1989) \emph{instead of} (Author, Author \& Author 1989)
%     \item \emph{Use the following style:} (Author, in press); (Author, in prep.); (Author, unpubl.); (Author, pers. comm.)
% \end{itemize}
% ~

% Use the official SI notations:
% \begin{itemize}
%     \item kg / m / kJ / mm \emph{instead of} kg. (Kg) / m. / kJ. (KJ) / mm.; 
%     \item 20°16'32''SW \emph{instead of} 20° 16' 32'' SW
%     \item 0.50 \emph{instead of} 0,50 (\emph{used in French text}); 9000 \emph{instead of} 9,000 \emph{but if more than} 10,000: 10,000 \emph{instead of} 10000
%     \item \textsuperscript{14}C \emph{instead of} C\textsuperscript{14} / C-14 \emph{and} BP / BC / AD \emph{instead of} of B.P. / B.C. / A.D.
%     \item × 20 \emph{instead of} ×20 / X20 / x 20; $4+5>7$ \emph{instead of} 4+5$>$7 \emph{but} –8 / +8 \emph{instead of} – 8 / + 8
%     \item e.g. / i.e. \emph{instead of} e.g., / i.e.,
% \end{itemize}

% \section{Submission of material} 
% Paper should not exceed 4 pages and should be e-mailed as a PDF file to Prof. Chatzis at:\\ 
% manolis.chatzis@eng.ox.ac.uk

% \section{Deadline}
% Deadline for submission is April 1, 2022. 





\end{multicols*}













\end{document}