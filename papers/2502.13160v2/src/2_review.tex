\section{Related Work}


\subsection{Information Asymmetry and Diffusion}


Information asymmetry \cite{b58,b35,b29} refers to the difference in information among parties in a transaction or interaction, where one party has more or better information than the other. 
% Information is often disclosed in different ways \cite{b13,b16}.
% From the sender's perspective, common disclosure methods can be divided into individual disclosure (company releases financial reports \cite{b30}, accountants release audit reports \cite{b13}), media disclosure (media report \cite{b9}, social media \cite{b29}), and interactive disclosure (investor conference calls \cite{b16}, interactive system design \cite{b7}) according to the main participants. 
There are two types of information asymmetry: asymmetric information, where one party is known but the other is not, and symmetric lack of information, where all parties are unknown \cite{b30}. 
% The distinction in information asymmetry affects how information spreads through channels and influences its dynamics.
Common information diffusion models, like the IC model \cite{b57} and SIR model \cite{b56}, use probabilistic approaches to simulate diffusion. 
While these models offer a structured framework, their reliance on mathematical constraints—such as individual activation probabilities and discrete states—limits their real-world applicability \cite{b55}.
In this paper, we use a simulation approach with LLM-based agents to explore complex social scenarios involving information asymmetry. 
By comparing our results with existing theoretical frameworks, we show that LLM-based agents exhibit behaviors similar to human information processing, validating the use of multi-agent simulations in such contexts.



\subsection{LLM-based Multi-Agent Social Simulation}


LLM-based Multi-Agent Social Simulation \cite{b59,m56} uses Multi-Agent System performance in a specific environment to explore social network \cite{m56}, economics \cite{m59}, psychology \cite{m61}, military \cite{m60} issues. 
%The agent's autonomy, pro-activeness, adaptability to various environments, and the LLM's capabilities for understanding and generating natural language together establish the technical and interactive foundation for social simulation.
MASS's research expands on social intelligence by considering the social capabilities of agents \cite{b1} from the perspective of information asymmetry. 
The study found that while LLMs are more likely to achieve social goals in omniscient scenarios, this does not reflect actual social interactions \cite{b47}. 
When agents actively share information in environments with unequal access to information, they assist in achieving objectives \cite{b12} and forming or changing relationships \cite{b59}. 
Common simulations of information asymmetry typically focus on fixed individual scenarios, lacking diverse information exchange.
In our work, we explore realistic social scenarios where agents must demonstrate heightened relational sensitivity, strategically allocate social attention, and maintain cognitive clarity in information processing, thereby enhancing agent capabilities in studying information diffusion.










%However, most current social simulations are based on fixed environments, but the actual society is an open environment, which means that information is variable, information sources and destinations are not fixed, and agents must act according to their wishes and context.
%So we built a simulation to learn how typical social structures dynamically spread and change in an open environment with asymmetric information.






