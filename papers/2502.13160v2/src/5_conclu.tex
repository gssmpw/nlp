\section{Conclusion}

In this paper, we employed the Dynamic Attention Algorithm to assist agents in processing information and tested information diffusion among multi-agents in 12 information-asymmetric open environments. At the macro level, we observed the agent's social identity, diffusion motivation, and the information gap changes. At the micro level, agents exhibited social behavior during diffusion, encountered the information cocoon, and leveraged social networks to accumulate social capital.





\section{Limitation}


% Through this work, we constructed and experimented with agents with human-like attention mechanisms in an open environment with asymmetric information, processing, understanding, and dissemination of multi-party information, as well as the dynamic changes in relationships. We proposed a two-tier simulation framework for learning open environments with asymmetric information, based on the agent attention mechanism of global workspace theory and Selective Attention Theory. In the experimental part, we designed three types of information content and four construction mechanisms for asymmetric environments, and explored the dynamic information dissemination process of micro-topologies with different centralities in dealing with various asymmetric environments. The experimental results can enhance the understanding of multi-agent information processing in an environment with asymmetric information, as well as discover the social behavior of agents.



\noindent \textbf{Ideal model and practical challenges} \quad In the experiment, we demonstrated that the addition of new agents triggered changes in the information circle within the group. 
Agents accumulated information resources for themselves by establishing and changing relationships. These phenomena are consistent with the description of social capital theory. 
In the open environment we have established, agents are free to add new members to the group at any time. However, the profile of each new member is customized by the agent. 
This ideal scenario does not reflect reality. In real life, resources and available personnel are often limited, which can lead to information asymmetry resulting from competition for those resources.
This will encourage research into the social abilities of agents, considering environmental variability and resource competitiveness, thus showcasing interactions and capabilities that better reflect social scenarios.