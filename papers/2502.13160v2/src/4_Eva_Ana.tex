




\begin{figure*}[ht]
    \centering
    \includegraphics[width=\linewidth]{pdf/data_gap.pdf}
    \caption{Information Gap (the blue bars) and Diffusion Gap (the red bars) for 12 asymmetric environments on four initial settings. Each simulation contains these two values. Differences between the two values represents the Diffusion Conversion Gap. The smaller the gap, the more individuals with known information tend to spread it, which means that the diffusion chain is relatively complete.}
    \label{fig:gap}
\end{figure*}



\section{Experiment and Analysis}
This section outlines the experimental design, data analysis, and findings. We examine how information asymmetry evolves within a multi-agent system by exploring various forms of asymmetric information and comparing the observed outcomes to established social science literature. This comparison demonstrates the viability of using LLM-based agents to simulate environments characterized by asymmetric information. Furthermore, we highlight novel insights made possible by multi-agent simulations—insights that are difficult to capture in traditional social science and communication studies, thereby illustrating new avenues for employing LLM-based agents in social science research.

\subsection{Experimental Settings}

To examine the formation of a dynamic information gap during agent information diffusion in an asymmetric open environment, we conducted a simulation and tested it in 12 different asymmetrical information environments.
The simulation is developed based on the SOTOPIA \cite{b53} library and employs the GPT-4o mini model \cite{b54} for the agent's decision-making process.
We randomly selected 5 agents from the 25 agents in Stanford Town \cite{m9} as the initial state group. 
Their profiles include gender, age, innateness, and occupation, and are evenly distributed. 
The group settings include the group's topology and initial relationship.



We jointly build an information asymmetry environment through \textit{information content} and \textit{distribution mechanism}. 
The main difference in the information content lies in its relevance to initial agents, and the distribution mechanism mainly affects the asymmetry generated directly at the source of information.
Based on the Construal Level Theory \cite{b69} in social psychology, we define five types of information content: other people's gossip (OG), public policy (PP), legal cases (LC), and stakeholder (SH). 
Furthermore, we define three distribution mechanisms: information broadcast (BC), information unicast (OA), and broadcast by round (BCR), creating asymmetry at the source of information.
BC represents the process of send the information to all five agents at the first round, while OA means only send information to one agent (agent 2 as the center in wheel and common node in circle).
BCR means send information to one agent each round until the initial five agents know the information.
Our information is generated by GPT-4o mini and is approximately 50 words long, as shown in table \ref{tab:content_analysis}.




We ran simulations three times for each topology corresponding to the information content and asymmetric mechanism, with the initial relationships between agents set to all positive or all negative.




\subsection{Macro-level Analysis}


We compare the information diffusion process of agent groups in different information asymmetric environments through the information gap, the diffusion gap, and information retention.


The information gap refers to the percentage of agents aware of the initial information compared to all agents. The diffusion gap indicates the percentage of agents who have shared the initial information within the group. Additionally, information retention measures how many rounds the initial information is maintained during the group's diffusion process.
Since the message sent by an agent is a short sentence rather than a long text, we use the Sentence-BERT method \cite{b70} to compare the similarity between the message and the initial information. If the similarity is greater than 80\%, it means the receiver is aware of the initial information, and the sender has successfully shared it.
It is important to note that an individual can only share information they already know, simulating the natural process of people sharing thoughts or observations they have recognized. The results are summarized as follows.



\begin{figure*}[ht]
    \centering
    \includegraphics[width=\linewidth]{pdf/data_retention.pdf}
    \caption{Information retention for 12 asymmetric environments under four initial settings. The larger the value, the longer the initial information is retained in the diffusion process and the harder it is to be submerged.}
    \label{fig:retention}
\end{figure*}




\begin{table*}[ht]
    \centering
    \small
    
    \scalebox{0.8}{
    \begin{tabular}{p{3cm}p{16cm}}
    \toprule
        cooperation & "Hey there! As a bartender and bar owner, ..., collaborate and share our ideas on making events special and enjoyable." \\
        \midrule
        \textcolor{bluee}{cooperation} & \textcolor{bluee}{"Hey! I came across a funding initiative that offers up to \$50,000 for innovative projects, ..., Let's explore our ideas together."} \\
        \midrule
        further discussion & "I'd love to discuss how we can challenge our preconceptions together and explore themes that resonate in both fields." \\
        \midrule
        \textcolor{bluee}{further discussion} & \textcolor{bluee}{"I've been reflecting on the recent events surrounding the disappearance of digital artwork, ..., What do you think this might symbolize in the context of our relationship with technology and existence?"} \\
        \midrule
        \textcolor{bluee}{altruism} & \textcolor{bluee}{"I came across an exciting funding initiative that offers up to \$50,000 for projects that, ..., this could really enhance your creative projects! Applications open next month, and I believe ..."} \\
        \midrule
        active inquiry & "I’d love to hear your thoughts on how we can use technology to enhance artistic expression." \\
        \midrule
        support & "I just want to share how much I appreciate your efforts in, ..., feel free to reach out to The Rose and Crown Pub. I'd love to support your initiatives!" \\
        \bottomrule
    \end{tabular}
    }
    \caption{Cases of agent's purpose for information diffusion. The diffusion of content can be related (colored red) or unrelated to the initial information and can serve a similar purpose.}
    \label{tab:behavior}
\end{table*}



\noindent \textbf{Agenda-Setting Theory \cite{b71}: Distributing information over time helps maintain relevant knowledge within a group, but is not effective for widespread sharing.} \quad
As shown in Figure \ref{fig:retention}, when information related to the initial five individuals (except the OG) is spread using BC and BCR, information retention is consistently higher with BC than with BCR. However, Figure \ref{fig:gap} shows that the diffusion gap values for BCR are always smaller than those for BC.
Although the initial five agents can know the initial information through BC and BCR, BCR delays the time. 
These two methods directly reflect the information source's control over the "agenda." The sooner information is introduced to the group (as in BC), the more likely it is to capture individuals' attention.
It can also be concluded that in the information diffusion process of LLM agents, slowing down the time it takes for agents to access information can reduce large-scale dissemination, thereby increasing information retention within the group.



\noindent \textbf{Social Identity Theory \cite{b72}: Individuals tend to spread highly relevant content.} \quad As shown in Figure 2, the largest difference between the information gap and the diffusion gap is concentrated in the OG information. This suggests that OG information is difficult to spread effectively within the current group. Individuals are more likely to share information that is highly relevant to themselves, whether for cooperation, understanding, in-depth discussion, or other purposes.






\subsection{Micro-level Analysis}





\begin{figure*}[ht]
    \centering
    \includegraphics[width=\linewidth]{pdf/case1.pdf}
    \caption{In this Social Capital Theory case study, agents 1, 3, 4, and 5 form relationships with new agents, creating distinct information circles within the growing group. Nodes represent individual agents, with colors indicating their lineage (the agent and its new recruits are in the same lineage). Node distance and edge color depth reflect the similarity between the current message and those of all agents. Agents who did not take action are not recorded.}
    \label{fig:capital}
\end{figure*}



In this section, we focus on the agent's action to understand the information diffusion in an open environment with information asymmetry.


\noindent \textbf{Social Behavior in information diffusion} \quad First, we observed that agents have social motivations when spreading information (shown in Table \ref{tab:behavior}), such as seeking cooperation, altruism, support, or discussion. When the initial information is sufficient to prompt an agent to spread it, the agent may seek cooperation based on this information. However, if the initial information is insufficient to influence the agent's spreading behavior, the agent may choose not to act in that round or may communicate based on other received information and the existing profiles of agents. This behavior reflects that, regardless of whether the content of the communication is related to the initial information, similar social motivations will still arise.



\begin{table}[ht]
    \centering
    
    \scalebox{0.8}{
    \begin{tabular}{ccccc}
    \toprule
        
                 (a) & positive
        46\%&  \textcolor{bluee}{negative
        54\%}&  & \\
        \midrule
                 (b) & circle
        50\%&  wheel
        50\%&  & \\
        \midrule
                 (c) & BC
        28\%&  \textcolor{bluee}{OA
        42\%}&  BCR
        30\%& \\
        \midrule
                 (d) & OG
        24\%&  PP
        26\%&  LC
        24\%& SH
        26\%\\
        \midrule
                 (e) & OG
        0\%&  PP
        12.5\%&  LC
        0\%& \textcolor{bluee}{SH
        87.5\%}\\
        \bottomrule
        
    \end{tabular}
    }
    \caption{The first four rows illustrate how different information asymmetry factors affect new agent diffusion. (a) and (b) represent the initial agent group settings, (c) and (d) show the external information asymmetry environment, and (e) is the proportion that new agents receive over 80\% similarity with the initial information.}
    \label{tab:newagent}
\end{table}



\begin{table}[!ht]
    \centering
    \footnotesize
    
    \begin{tabular}{p{0.6cm}p{1cm}p{1.3cm}p{0.5cm}p{0.5cm}p{1cm}}
         \toprule
         agent id&total received&average similarity $\uparrow$ &min&max& standard deviation $\downarrow$ \\
         \midrule
         2&  23&  0.41&  0&  0.99& 0.24\\
         \midrule
         3&  10&  0.67&  0.02&  0.99& 0.33\\
         \midrule
         \textcolor{bluee}{6}&  10&  \textcolor{bluee}{0.82}&  0.54&  1& \textcolor{bluee}{0.12}\\
         \midrule
         1&  5&  0.61&  0.30&  0.89& 0.23\\
         \midrule
         5&  4&  0.59&  0.42&  0.92& 0.16\\
         \midrule
         7&  3&  0.74&  0.67&  0.81& 0.06\\
         \midrule
         4&  2&  0.71&  0.71&  0.70& 0\\
         \midrule
         8&  2&  0.82&  0.82&  0.82& 0\\
         \bottomrule
    \end{tabular}
    
    \caption{Case study on Information Cocoon: In a simulation, higher pairwise similarity and lower standard deviation of agents' received information correlate with stronger cocoon formation.}
    \label{tab:cocoon}
\end{table}


\noindent \textbf{Diffusion in Open Environment} \quad The open environment enables agents to recruit new members in any round, with the profile of each new member determined by the recruiting agent. We summarized the results of 48 simulations, involving 304 instances of agents recruiting new members, as shown in Table \ref{tab:newagent}. The analysis examined the effects of two initial settings (relationship, topology) and two factors related to information asymmetry (information content, distribution mechanism) on the number of new agents recruited.
When the relationship is negative or when the outside world releases information in the form of OA, agents are more inclined to send message to new agents. Other conditions have little effect on the number of new agents. However, as seen in the proportions in (d) and (e), even though the content of the information has little effect on an agent's behavior in recruiting new members, it significantly influences whether the agent shares the initial information with the new recruit.
When the information is public and involves clear interests, agents are more likely to communicate it to new recruits. However, if the information pertains to private matters, it is not shared externally but discussed within the group.


\noindent \textbf{Social Capital Theory \cite{b62}} \quad According to Figure \ref{fig:capital}, we found that agent 4 continuously expanded its information circle in the group by establishing connections with new agents, and at the same time won the attention of agent 2. 
This illustrates how agents gather more information resources through social networks.
Gaining more public attention makes it easier to become an opinion leader in the group.



\noindent \textbf{Information Cocoon \cite{b63}} \quad Table \ref{tab:cocoon} shows that in this simulation, agent 6 received a total of 10 messages, with a pairwise similarity of 0.82 and a low standard deviation. 
This shows that even though agent 6 received a lot of information, it formed an information cocoon, making it more difficult to make decisions about dissemination actions based on diverse information.

