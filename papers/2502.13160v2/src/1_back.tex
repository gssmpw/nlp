
\section{Introduction}
Recent advances in large language models (LLMs) with strong reasoning and language understanding ability have established a robust foundation for developing agents that exhibit social intelligence \cite{b1}. Many studies have employed LLM-based agents to simulate human behavior, construct social networks, and explore various dimensions of social development and human conduct \cite{m56,b3}. For instance, researchers have investigated the social capabilities of these agents by modeling market competition \cite{m37}, economic flows \cite{m59}, international trade \cite{b2}, warfare \cite{m60}, and political party competition \cite{m63}, thereby providing insights and recommendations for real-world applications. However, these simulations often operate within fixed environments \cite{m9} or assume static channels for information transmission \cite{b55}. As a result, they often overlook the role of information opacity, \emph{i.e.}, the asymmetric distribution of information, which can profoundly influence actual human decision-making processes and, consequently, the validity of the simulation outcomes.

Real-world information is neither transparently nor equally distributed, leading to inherent information asymmetry \cite{b13}. Typically, individuals acquire information in a progressive, staged, and selective manner \cite{b37,b38}, with the effectiveness of this process depending on both the methods employed and the individual’s interpretive abilities. Consequently, organizations such as businesses \cite{b5}, prosecution agencies \cite{b10}, government systems \cite{b6,b8}, news media \cite{b9}, and software developers \cite{b7} have developed strategies to tailor the disclosure of information, thereby facilitating easier access. Moreover, interpersonal communication and the formation of social connections further enable individuals to obtain additional details \cite{b14}. Given the diversity of social networks, the nature and extent of the information that individuals receive are significantly shaped by their social interactions.

% In the real world, people usually receive information progressively, in stages, and selectively \cite{b37,b38}. 
% Information asymmetry \cite{b13} exists in any situation involving the exchange of information, meaning individuals often possess different amounts of information compared to one another. 
% People typically rely on actively seeking out information to understand events. 
% This approach can typically yield first-hand insights, and influenced by the methods used and the individual's ability to understand.
% So in real life, businesses \cite{b5}, prosecution agencies \cite{b10}, government systems \cite{b6,b8}, news reports \cite{b9}, and software developers \cite{b7} have many ways to customize the disclosure of information to make it easier for people to obtain information.
% In addition to learning information, people establish connections through communication and obtain more details \cite{b14}. 
% Due to the variety of social networks, the information individuals receive is influenced by their social connections, which differs from the information that reaches them directly.

%information diffusion process which often occur in real life,  but this cannot represent actual social interactions.
%Although research has explored agents' performance in social networks from the perspectives of scaling law \cite{m44}, information subject transparency \cite{b11}, task completion \cite{b12}, opinion dynamics \cite{b4}, and collective intelligence \cite{b3}, there is still a lack of discussion on the information diffusion process in the open environments with asymmetric information. 

In this project, we investigate the dynamics of information diffusion within an asymmetric open environment using a multi-agent simulation framework. An information asymmetry situation refers to a scenario where one party in a transaction or interaction possesses more, or higher quality, information than the other potentially due to varied information sources, evolving relationships, and differing contents of information. By comparing simulation outcomes with predictions derived from real-world information theory \cite{b62,b64}, we aim to understand how agents cope with asymmetric information and whether their behaviors mirror those of humans. We hope to enhance the validity of multi-agent social simulations under conditions of information asymmetry and to demonstrate that LLM-based agents can effectively simulate complex social dynamics.

To achieve this objective, we first introduce a two-tier general simulation framework designed to capture dynamic information diffusion. We also propose an agent attention mechanism \cite{b51,b49} that prioritizes critical information in a manner analogous to human information processing, enabling agents to handle multiple sources of information concurrently. We then examine the behaviors of agents under various external stimuli. Our study incorporates both macro-level and micro-level analyses to elucidate differences in information gaps, communication dynamics, and the evolving structure of information circles. We also explore the impact of integrating new agents and investigate various phenomena pertinent to psychology, communication, and sociology.


%Then, we build a network of agents to represent an initial social group. 

%Given external information stimuli, the initial group must receive information, decide how to allocate attention to this information, select or add new information receivers, and transmit messages. 
%Throughout this process, agents may alter their subjective relationships at any time, reflecting changes in their cognitive perceptions of these relationships.
%Additionally, we propose an agent attention mechanism to help the agent allocate attention to different information.
%This mechanism simulates human-like information processing to assist agents in handling multiple sources of information simultaneously.
%We collaboratively construct 12 information-asymmetric environments using various information contents and distribution mechanisms, allowing four initial groups of GPT-4o mini-based agents to run three simulations in each environment.
%We perform both macro-level and micro-level analyses to identify differences in information gaps, communication dynamics, the evolving nature of information circles, the importance of new agents joining, and various phenomena related to psychology, communication and sociology.



