\section{Appendix: VLM Prompts}
\label{prompts}

\subsection{Prompt for VML Dialog Games}
\label{game-prompts}

Prompt to Describer agent to answer questions about the image faithfully is the same for all datasets and domains:

\begin{lstlisting}
You are given an image and your task is to answer a given question about it. Be precise and accurate. Only answer the question, do not say anything else about the image.
Image: {image}
Question: {question}
Answer:
\end{lstlisting}

Prompt for Guesser for general images:

\begin{lstlisting}
You are given several images Image 1, Image 2, Image 3, Image 4 and image description.
This image description refers to only a single image, however, the image description might be incomplete.
You task is the following:
1) If the image description can only refer to a single image from the set of images (Image 1, Image 2, Image 3, Image 4) you should provide the answer in the format:
Answer: I know the answer, it is image X.
where X is the index of an image (1, 2, 3, 4).
Only provide a response in this format when you are absolutely certain to which image the image description refers to.
Never provide an answer in this format when the image description is empty.
2) If no image description is provided or the image description can refer to more than one image, your task is to ask additional question to narrow down the space of possible images from the set (Image 1, Image 2, Image 3, Image 4).
Ask any question that would help you to narrow the space of possible images.
Choose a question that would help you to maximise the information about the content of the target image.
Try to ask objective, factual questions that cover the content of the image, but not the deductions about the scene or any impressions about the image.
Follow the format:
Question: put your question here.
So, now given the image descriptions and 4 images, decide if you are going to make a guess (in that case produce an Answer) or ask a question (in that case produce a Question).
Image description: {image_description}
Image 1: {image1} Image 2: {image2} Image 3: {image3} Image 4: {image4}
\end{lstlisting}

In robotics experiments, the Guesser agent was prompted with instructions very similar to the general images, but mentioning that the images come from a robotics domain.
These instructions also direct the agents to focus on relevant visual features for robotic manipulation tasks as opposed to details in the background (e.g., people who are passing by, chairs moved):

\begin{lstlisting}
You are given two images (Image 1, Image 2) from a scene where robot is trying to {task} and image description.
This image description refers to only a single image, however, the image description might be incomplete.
You task is the following:
1) If the image description can only refer to a single image from the set of images (Image 1, Image 2) you should provide the answer in the format:
Answer: I know the answer, it is image X.
where X is the index of an image (1,2).
Only provide an response in this format when you are absolutely certain to which image the image description refers.
Never provide an answer in this format when the image description is empty.
2) If no image description is provided or the image description can refer to more than one image, your task is to ask additional question to narrow down the space of possible images from the set (Image 1, Image 2).
Try to ask objective, factual questions that cover the content of the image.
Choose a question that would help you to maximise the information about the content of the image.
NEVER ask questions about the background of the robotic scene (e.g., people in the background, scooters or chairs).
NEVER ask questions about the facts that are already known from the image description.
Follow the format:
Question: put your question here

So, now given the image descriptions and 2 images, decide if you are going to make a guess (in that case produce an Answer) or ask a question (in that case produce a Question).
Image description: {image_description}
Image 1: {image1} Image 2: {image2}
\end{lstlisting}

Prompt to Guesser agent to summarise the state of the existing dialog for all domains:

\begin{lstlisting}
You are given a short description of a scene and one question and answer about it.
Your task is to summarise the content of the scene in a short sentence or paragraph. Only provide a summary, do no output anything else.
Always include all the details 1) from the description, 2) from question-answer pair into your summary.
Description: {description}
Question: {question}
Answer: {answer}
Your summary: 
\end{lstlisting}

\subsection{Full Prompts for the Baseline Self-QA}
\label{qa-prompts}

To ask questions:

\begin{lstlisting}
You are given an image and your task is to ask a question about the content of this image.
Try to ask objective, factual questions that cover the content of the image, but not the deductions about the scene or any impressions about the image.
NEVER ask questions about the background of the robotic scene (e.g., people in the background, scooters or chairs).
Follow the format:
Question: put your question here.
So, now given the image, ask a question.
Image: {image}
Question:
\end{lstlisting}

To answer the questions:
\begin{lstlisting}
You are given an image and your task is to answer a given question about it. Be precise and accurate. Only answer the question, do not say anything else about the image.
If possible, ONLY answer with *yes* or *no*.
Image: {image}
Question: {question}
Answer:
\end{lstlisting}

\section{Appendix: Dialog Examples for Varying Number of Images}
\label{game-examples-n}

Figures~\ref{fig:dialog2}, \ref{fig:dialog4}, and \ref{fig:dialog8} illustrate dialog examples with a consistent target image (Image 1) but a varying number of distractor images (1, 3, and 7, respectively, corresponding to $N$ values of 2, 4, and 8).
These examples demonstrate the effect of $N$ on dialog length and complexity.
With only two images ($N=2$), the dialog is short, focusing on a single distinguishing feature.
With four images ($N=4$), the dialog becomes more complex, requiring two questions that progressively narrow down the possibilities.
However, with eight images ($N=8$), the Guesser is unable to identify the target image within the three-question limit. 
\begin{figure}[t]
    \centering
    \includegraphics[width=1\columnwidth]{assets/dialog2.pdf}
    \caption{An example of a dialog game with two images.}
    \label{fig:dialog2}
\end{figure}

\begin{figure}[t]
    \centering
    \includegraphics[width=1\columnwidth]{assets/dialog4.pdf}
    \caption{An example of a dialog game with four images.}
    \label{fig:dialog4}
\end{figure}

\begin{figure}[t]
    \centering
    \includegraphics[width=1\columnwidth]{assets/dialog8.pdf}
    \caption{An example of a dialog game with eight images.}
    \label{fig:dialog8}
\end{figure}

\section{Appendix: Question-Answers Generated by Dialog Games and Self-QA}
\label{sec:qa-examples}

Self-QA:
\begin{itemize}
    \item Question: Is there a yellow object in the image? Answer: yes
    \item Question: Is there a red object on the surface? Answer: yes
    \item Question: Are there any balls inside the basket? Answer: No
    \item Question: Is there a red ball in the image? Answer: yes
    \item Question: Are there two robotic arms in the image? Answer: yes
    \item Question: Is there a red apple in the image? Answer: yes
    \item Question: Is there a red object on the floor? Answer: yes
    \item Question: Is there a piece of fruit in the basket? Answer: yes
    \item Question: Is there a yellow triangle in the image? Answer: yes
\end{itemize}

VML Dialog Games:
\begin{itemize}
    \item Question: Are there any Lego blocks on the floor that are not in the bag? Answer: Yes.
    \item Question: Is the drawer in the image open? Answer: Yes.
    \item Question: Is the trash bin lid open or closed? Answer: Closed.
    \item Question: Is the bowl inside the drying rack? Answer: No.
    \item Question: Is the cheese in the basket? Answer: No.
    \item Question: Is the banana inside the drying rack? Answer: Yes.
    \item Question: Is the robot's gripper holding the belt? Answer: No.
    \item Question: Is there a basket in the image? Answer: No.
    \item Question: Is the drawer open? Answer: Yes.
\end{itemize}

\section{Appendix: LLM Inference and Training Details}
\label{sec:training-details}

We rely on Gemini 1.5 Flash (gemini-1.5-flash-002) model which is available for inference and fine-tuning through the Google Cloud Vertex API.
For generating dailog and evaluation, we sample with nucleus sampling selecting the top \num{0.8} probability mass of tokens.
For the evalaution, we use sampling temperature \num{0}.
Our batch size for SFT is \num{16}, we use Adam optimizer with learning rate \num{5e-07}.
To prevent overfitting to the small datasets from the dialog games, we use a small unrelated to the any of the tested tasks dataset of images with text and track token loss on it.
We select a checkpoint just before the loss starts increasing. 
This usually corresponds to approximately one epoch of fine-tuning.

