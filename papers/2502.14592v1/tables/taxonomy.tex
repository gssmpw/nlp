\begin{table*}
\begin{small}
\centering
\begin{tabular}{ M{0.1\textwidth} M{0.21\textwidth} M{0.31\textwidth} M{0.3\textwidth}}
%{ M{1.5cm} M{3.2cm} M{4.5cm} M{4.5cm}}
\toprule
 \centering{\textbf{Harm Category}} & \centering{\textbf{Required to Measure}} & \centering{\textbf{Domain-Agnostic Harms}\newline\textit{from existing taxonomies}} & \centering{\textbf{Education-Specific Harms}\newline\textit{from our interviews}} 
\tabularnewline
\midrule
\centering{Technical Harms} & 
Outputs of LLM-based systems &
Toxic content, biased content,
privacy violations,
hallucinations & 
Toxic content, biased content,
privacy violations,
hallucinations \\
\midrule
\centering{Human-LLM Interaction Harms} & 
Interactions between LLM-based systems and students and/or educators &
Malicious use (e.g., abusive content, misinformation), HCI harms (from anthropomorphization of LLMs) &  Academic dishonesty \\
\midrule
\centering{Harms From Broader Impacts} & 
Interactions between LLM-based systems and students, educators, and/or school systems &
Research opportunity cost, environmental harms, socioeconomic harms (e.g., labor impacts, reinforcement of inequitable power structures) &  Inhibiting student learning and social development, increasing educator workload, decreasing educator autonomy, exacerbating systemic inequalities in education \\
\bottomrule
\end{tabular}
\end{small}
\caption{Potential harms arising from the use of LLMs. Domain-agnostic harms are synthesized from \citet{bender_dangers_2021} and \citet{weidinger_taxonomy_2022}. Education-specific harms were raised by the edtech providers and educators that we interviewed. }
\label{t-taxonomy}
\Description{A table describing potential harms from LLMs, with the following columns: Harm Category, Required to Measure, Domain-Agnostic Harms, and Education-Specific Context. The first category is technical harms, which require the outputs of LLM-based systems to measure and include the domain-agnostic harms of toxic content, biased content, privacy violations, and hallucinations as well as the education-specific harm of large-scale privacy violations of minors. The second category is human-LLM interaction harms, which require interactions between LLM-based systems and students or educators to measure and include the domain-agnostic harms of malicious use (abusive content, misinformation) and HCI harms (from anthropomorphization of LLMs) as well as the education-specific harm of academic dishonesty. The third category is harms from broader impacts, which require interactions between LLM-based systems and students, educators, and school systems to measure and include the domain-agnostic harms of research opportunity cost, environmental harms, and socioeconomic harms (labor impacts, reinforcement of inequitable power structures) as well as the education-specific harms of inhibiting student learning and social development, increasing educator workload, decreasing educator autonomy, and exacerbating systemic inequalities in education.}
\end{table*}