\section{Methodology}
In this section, we present our proposed methodology for efficient encrypted network traffic classification. In \S \ref{subsec:netmatric}, we introduce \textit{NetMatrix}, the representation that captures the essential features of network traffic according to RFCs. Building upon this representation, in \S \ref{subsec:lim}, we explain the machine learning model selection for experimental evaluation. 

% A classifier which demonstrates that a lightweight machine learning model can achieve performance comparable to state-of-the-art methods. 
% Together, NetMatrix and LiM provide a practical and effective solution that addresses the limitations of existing approaches by emphasizing simplicity and efficiency without compromising classification accuracy.

\subsection{NetMatrix}
\label{subsec:netmatric}
NetMatrix is a concise tabular representation of network traffic designed to overcome the limitations of existing NTC methods. The proposed representation is built upon the following principles aligning to modern-day RFCs such as, RFC-8446, RFC-2328, RFC-4271, RFC-6247, RFC-791, RFC-9293, RFC-7323:

\textbf{Exclusion of Encrypted Payloads}: Recognizing that encrypted payload content lacks exploitable patterns as per TLS 1.3 specifications [RFC-8446], we entirely exclude it from NetMatrix to address ($C1$).

\textbf{Utilization of Encrypted Payload Length and Timing}: Acknowledging that TLS 1.3 highlights the susceptibility of encrypted packet lengths and timings to traffic analysis attacks [RFC-8446], we incorporate these attributes and solve ($C2$). Specifically, we use the total length field from the IP header and calculate the inter-arrival times between packets.

\textbf{Reduction of Representation Size}: By excluding noisy, session-specific header fields—such as IP Identification (IP-ID), checksums, sequence numbers, and TCP timestamps, we focus on stable, meaningful header attributes aligning to RFCs. This reduction minimizes noise and enhances classification performance by concentrating on relevant data, resolving ($C3$).

\textbf{Preservation of Inherent Network Traffic Schema}: Instead of transforming network traffic into text or images, which can obscure protocol structures, NetMatrix maintains the inherent schema by using a tabular format as a remedy for ($C4$). This representation preserves the relationships defined by network protocols.

\textbf{Reliance on Packets Containing Encrypted Data}: To ensure practicality, we base our classification on packets that contain encrypted payloads. This approach remains effective even when handshake packets are unavailable as a solution to ($C5$).

NetMatrix utilizes only three key attributes extracted from network packets to represent a session:

\textbf{Total Length (IP Header)}: The total length field in the IP header indicates the size of the entire IP packet, including both the header and the payload.

\textbf{Time-to-Live (TTL) (IP Header)}: The TTL field represents the maximum number of hops a packet can traverse before being discarded. While TTL values can vary due to routing changes, they are generally consistent for packets taking the same route between two endpoints. Routing protocols aim for minimal overhead, so minor route fluctuations have limited impact on the TTL upon arrival [RFC-2328, RFC-4271]. Therefore, TTL can provide useful information about network path characteristics.

\textbf{Inter-Arrival Time}: This attribute measures the time difference between two consecutive packets.

NetMatrix represents each network traffic session (i.e., a bidirectional flow between two endpoints) by extracting the aforementioned attributes from five consecutive packets containing encrypted payloads. For each packet, we collect: Total Length (2 bytes), TTL (1 byte), Inter-Arrival Time (calculated, 3 bytes). This results in a compact representation of 30 bytes per session, significantly reducing the data volume compared to existing methods. For instance, ET-BERT uses 620 bytes, and YaTC requires 1,600 bytes to represent a session. 

% The minimalistic approach of NetMatrix significantly reduces computational overhead, enabling faster processing and lower resource consumption. By excluding non-informative and noisy features, it focuses on attributes that meaningfully contribute to classification. The selected features align with TLS 1.3 and other relevant RFCs, ensuring that the methodology adheres to protocol specifications. Additionally, the tabular format preserves the inherent relationships within network traffic data, facilitating better understanding and interpretation by machine learning models.

\subsection{LiM}
\label{subsec:lim}

To complement the efficient representation provided by NetMatrix, we employ a vanilla XGBoost classifier \cite{xgboost}. As highlighted by \cite{xgboost_tabular_data}, tree-based models, such as XGBoost, are particularly adept at learning irregular patterns in the target function, and demonstrate superiority on tabular data. This insight justifies our selection of the XGBoost classifier in this study. Its inductive biases make it inherently well-suited for the structured nature of NetMatrix, enabling efficient and accurate classification without requiring the computational resources demanded by deep-learning models.



We refer to the integration of NetMatrix with the XGBoost classifier as \textit{LiM (Less is More)}. By utilizing LiM, we demonstrate that a simple machine learning model can achieve performance comparable to state-of-the-art deep learning approaches, such as ET-BERT and YaTC, without the need for complex architectures or extensive computational resources.