\section{Introduction}
The rapid growth of encrypted network traffic has significantly enhanced user privacy and security. However, it poses challenges for network traffic classification (NTC), a critical task for network management, security monitoring, and quality of service provisioning. Traditional classification methods that rely on inspecting packet payloads are rendered ineffective by encryption protocols like TLS 1.3 [RFC-8446], which are designed to prevent the extraction of meaningful patterns from encrypted payloads \cite{foundation_et_bert}.

Although many novel approaches are proposed for NTC, they often make several assumptions and employ methodologies that do not align with modern encryption specifications. Below, we present the resultant challenges (C) in NTC by comparing widespread assumptions with modern-day RFCs.

\textbf{\textit{C1.}Assumption of implicit patterns in encrypted payloads}: Many models operate under the premise that encrypted payloads contain exploitable patterns for classification \cite{foundation_et_bert, foundation_yatc_full}. This contradicts the TLS 1.3 specification [RFC-8446], which ensures that the same plaintext will always produce different ciphertexts due to the use of Authenticated Encryption with Associated Data (AEAD) ciphers and unique initialization vectors (IVs).

\textbf{\textit{C2.}Fixed-Size homogeneous representation of payloads}: Studies often truncate or pad payloads to create fixed-size inputs. This approach ignores TLS 1.3's acknowledgment that encrypted packet lengths and timing are susceptible to traffic analysis attacks [RFC-8446]. Padding or truncation does not effectively exploit this vulnerability and may obscure valuable size-related artifacts essential for classification.

\textbf{\textit{C3.}Inclusion of header without justification}: Including header fields such as IP Identification (IP-ID) [RFC-6274], IP header checksum [RFC-791], sequence and acknowledgment numbers [RFC-9293], and TCP options timestamps [RFC-7323] introduces noise. These fields often contain pseudo-random values initialized per session, adding unnecessary variability to the data.

\textbf{\textit{C4.}Representation of network traffic as text or images}: Transforming network traffic into textual or visual formats to leverage semantic, syntactic, or spatial relationships using  foundation models (e.g., BERT, vision transformers) can hinder the inherent structure of network protocols. Unlike natural language tokens or image pixels, network header attributes typically lack interdependencies, making such representations less effective.

\textbf{\textit{C5.}Reliance on TCP/TLS handshake packets}: Some classifiers \cite{foundation_yatc_full} depend on initial handshake packets for classification. However, mechanisms like TCP connection reuse [RFC-9293], TLS 1.3's 0-RTT data transmission, and session resumption via Pre-Shared Keys (PSKs) [RFC-8446] allow communication without frequent handshakes. Consequently, classifiers that rely on handshake packets may fail in these common scenarios. 

These challenges underscore the need for a practical and efficient approach to NTC. To bridge this gap, we propose \textit{NetMatrix}, a novel tabular representation that focuses on capturing meaningful features of network traffic leveraging RFC specifications (\S \ref{subsec:netmatric}). NetMatrix, paired with a vanilla XGBoost classifier, referred to as \textit{LiM} \footnote{Source code: https://github.com/nime-sha256/LiM} (Less is More) (\S \ref{subsec:lim}), achieves performance comparable to resource-intensive deep-learning models while maintaining significantly lower computational overhead (\S \ref{subsubsec:classification}). Moreover, our approach demonstrates the ability to handle high volumes of requests, adapt to new traffic classes within seconds, and operate with enhanced memory and energy efficiency (\S \ref{subsubsec:scalability}).

% These challenges highlight the need for a practical and efficient approach to NTC. In this study, we propose a solution to overcome these challenges with the following contributions:

% \textbf{NetMatrix representation}: A novel tabular representation that captures meaningful, non-noisy features of network traffic, facilitating efficient processing and preserving protocol structures.
% \textcolor{blue}{say that this si developed according to RFCs}

% \textbf{LiM (Less is More) a lightweight NTC model}: Demonstrate that a vanilla XGBoost classifier can achieve performance comparable to state-of-the-art methods.

% \textbf{Scalable and Practical Classifier}: Propose a classifier capable of handling high volumes of requests, training on new classes in seconds, and operating with enhanced memory and energy efficiency.

% \textbf{Alignment with Protocol Specifications}: Enhance the practicality and robustness of the approach by ensuring that our methodology complies with TLS 1.3 and other relevant RFCs. \\

% Through this work, we aim to shift the paradigm of encrypted network traffic classification towards simplicity and efficiency.