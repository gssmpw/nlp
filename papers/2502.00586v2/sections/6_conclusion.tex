\section{Conclusion and Future Work}
This paper addresses critical challenges in encrypted network traffic classification (NTC), including unrealistic assumptions about patterns in encrypted payloads, reliance on noisy or non-informative features, truncation/padding of payload data, and the computational inefficiencies of deep-learning-based methods. As a solution, we introduce NetMatrix, a streamlined tabular representation of network traffic, designed by leveraging modern RFC specifications to focus on meaningful and protocol-compliant attributes. Paired with a vanilla XGBoost classifier, this integration referred to as LiM (Less is More), demonstrates that simplicity can rival complexity.

Experimental evaluations show that LiM delivers competitive classification performance compared to state-of-the-art methods such as, ET-BERT and YaTC, while reducing computational overhead by orders of magnitude. These findings emphasize the importance of RFC-compliant traffic representations in tackling the complexities of encrypted NTC, offering a lightweight solution.

Despite its promise, this study is limited to the CSTNET-TLS1.3 dataset. Future work will expand the approach to a broader range of datasets, encompassing diverse traffic patterns, encryption protocols, and real-world network conditions. This will ensure LiM’s robustness across various network environments. Additionally, LiM will be evaluated against a wider array of state-of-the-art classifiers to refine its methodology and validate its efficiency further.

By emphasizing the utility of RFC-aligned representations and pursuing these directions, we aim to present LiM as a resource-efficient solution for encrypted NTC across diverse and evolving network landscapes.

% This paper addresses critical challenges in encrypted NTC, including, (1) unrealistic assumptions about implicit patterns in encrypted payloads, (2) reliance on noisy or non-informative features, (3) truncating/padding payload information, and (4) computational inefficiencies of deep-learning-based methods. As a solution, we introduces NetMatrix, a streamlined tabular representation of network traffic, designed by leveraging modern RFC specifications. By pairing NetMatrix with a vanilla XGBoost classifier, referred to as LiM, we demonstrate that Less is More. 

% Our evaluation demonstrates that LiM delivers competitive classification performance compared to state-of-the-art methods such as ET-BERT and YaTC, while significantly reducing computational overhead. These findings highlight the potential of RFC-complaint meaningful network traffic representations in addressing the complexities of encrypted NTC, providing a scalable and practical solution.

% However, this study is currently limited to the CSTNET-TLS1.3 dataset. To generalize its applicability, future work will extend the approach to a wider range of datasets encompassing diverse traffic patterns, encryption protocols, and different network conditions. This will ensure the robustness of LiM across various network landscapes and encryption contexts.

% Additionally, we aim to evaluate LiM against a broader set of state-of-the-art classifiers beyond ET-BERT and YaTC. This comparative analysis will help refine the methodology, identify areas for improvement, and further validate the scalability and efficiency of the proposed approach under varying conditions.

% By exploring these future directions, we aspire to establish LiM as a practical, scalable, and effective solution for encrypted network traffic classification across diverse network environments.

