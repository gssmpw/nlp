\section{Introduction}

In the era of information explosion, recommender systems~\cite{wu2022survey} have become an indispensable component of digital platforms, helping users navigate through massive amounts of content across e-commerce, social media, and entertainment domains. 
While traditional recommendation approaches~\cite{he2017neural} have achieved considerable success in providing personalized suggestions through analyzing user preferences and historical behaviors, they still face significant challenges in real-world applications, such as limited understanding of complex user intents, insufficient interaction capabilities, and the inability to provide interpretable recommendations~\cite{zhu2024recommender}.


Recent advancements in Large Language Models (LLMs)~\cite{achiam2023gpt} have sparked increasing interest in leveraging LLM-powered agents~\cite{wang2024survey} to address the aforementioned challenges in recommender systems. 
The integration of LLM-powered agents into recommender systems offers several compelling advantages over traditional approaches~\cite{zhu2024recommender}. 
First, LLM agents can understand complex user preferences and generate contextual recommendations through their sophisticated reasoning capabilities, enabling more nuanced decision-making beyond simple feature-based matching. 
Second, their natural language interaction abilities facilitate multi-turn conversations that proactively explore user interests and provide interpretable explanations, enhancing both recommendation accuracy and user experience. 
Third, these agents revolutionize user behavior simulation by generating more realistic user profiles that incorporate emotional states and temporal dynamics, enabling more effective system evaluation. Furthermore, the pre-trained knowledge and strong generalization capabilities of LLMs facilitate better knowledge transfer across domains, addressing persistent challenges such as cold-start~\cite{shu2024rah} with minimal additional training.



In this survey, we present a comprehensive review of LLM-powered agents for recommender systems. 
First, we introduce the background of traditional recommender systems and discuss their limitations in understanding complex user intents, interaction capabilities, and interpretability. 
We then systematically examine how LLM-powered agents address these challenges through three main paradigms: recommender-oriented (e.g., \cite{wang2024recmind,wang2024macrec}), interaction-oriented (e.g., \cite{zeng2024automated,friedman2023leveraging}), and simulation-oriented (e.g., ~\cite{yoon2024evaluating,guo2024knowledge}) approaches. 
Following that, we propose a unified agent architecture consisting of four core modules (Profile~\cite{cai2024flow,zhang2024agentcf}, Memory~\cite{shi2024large,fang2024multi}, Planning~\cite{wang2023drdt,shi2024large}, and Action~\cite{cshi,toolrec}) and analyze how existing methods implement these components. 
Furthermore, we compile comprehensive comparisons of datasets (including Amazon series, MovieLens, Steam, etc.) and evaluation methodologies, encompassing both standard recommendation metrics and novel evaluation approaches. 
Finally, we explore several promising future directions in this field.
%To enhance the recommendation performance with LLM-powered agents, in this survey, we thoroughly examine existing methods and highlight three dominant paradigms in the literature. Methods falling under strategic recommendation [BLLP, DRDT] focus on optimizing decision-making and long-term planning through multi-agent collaboration and sequential reasoning. Interactive recommendation approaches [AutoConcierge, RecLLM] leverage the natural language capabilities of LLMs to enable dialogue-based interactions and provide explanatory feedback, thereby enhancing user engagement and system transparency. Examining user simulation methods [Agent4Rec, AgentCF] aids in understanding user behaviors and preferences by constructing artificial users with profile, memory, and action modules, which facilitates system evaluation and optimization. Some hybrid frameworks [FLOW, MACRec] combine multiple paradigms to achieve comprehensive improvements. For instance, FLOW integrates strategic planning with user simulation through a feedback loop mechanism, while MACRec combines multi-agent decision-making with interactive recommendation capabilities.

\begin{figure*}
    \centering
    \includegraphics[width=0.97\linewidth]{pic/objective.pdf}
    \caption{Illustration of Different Method Objectives. We classify existing methods into the following three categories: (1) Recommender-oriented method; (2) Interaction-oriented method; (3) Simulation-oriented method.}
    \label{objective}
\end{figure*}

\begin{itemize}
    %\item We systematically categorize LLM-powered agent for recommender systems into three paradigms (recommender-oriented, interaction-oriented, and simulation-oriented).
    \item We propose  a systematic categorization of LLM-powered recommender agents, identifying three fundamental paradigms: recommender-oriented, interaction-oriented, and simulation-oriented approaches. This taxonomy provides a structured framework for understanding current research.

    %\item We analyze existing recommender systems through the lens of a unified agent architecture framework, which consists of four core modules: Profile, Memory, Planning, and Action. This analysis explores how different methods combine these components.
    \item We utilize a unified architectural framework for analyzing LLM-powered agent recommender, decomposing them into four essential modules: Profile Construction, Memory Management, Strategic Planning, and Action Execution. Through this framework, we systematically examine how existing methods integrate and implement these components.

    %\item We compile comprehensive comparisons of existing methods, datasets (including Amazon series, MovieLens, Steam, etc.), and evaluation methodologies (from standard recommendation metrics to novel evaluation approaches).
    \item We provide a comprehensive comparative analysis of existing methods, benchmark datasets, and evaluation methodologies, encompassing both traditional recommendation metrics and emerging evaluation approaches specifically designed for LLM-powered agent recommender.

    %\item We explore several promising  directions in LLM-powered agent for recommender systems for the future works.
\end{itemize}



\begin{figure}
    \centering
    \includegraphics[width=0.95\linewidth]{pic/component.pdf}
    \caption{Illustration of Agent Components and Corresponding Functions.}
    \label{component}
\end{figure}