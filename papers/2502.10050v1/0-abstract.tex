\begin{abstract}

% Traditional recommender systems, despite their widespread adoption in the digital era, continue to face fundamental challenges in understanding complex user intentions,enabling natural interactions, and providing explainable recommendations. 
% The emergence of Large Language Model (LLM)-powered agents has shown promising potential in addressing these limitations, leading to a new wave of research in recommender systems.
% This survey comprehensively examines the integration and development of LLM-powered agents in recommender systems.
% We systematically categorize existing research into three dominant paradigms: (1) Recommender-oriented method, focusing on utilizing different agents to tackle traditional recommendation tasks;  (2) Interaction-oriented method, focusing on enabling natural language interaction and enhancing recommendation interpretability through conversational engagement; and (3) Simulation-oriented method, focusing on employing various of agents to simulate user behaviors and item characteristics in recommender systems. 
% Besides, we provide an in-depth analysis of core components in LLM-powered agents for recommender systems, including: profile, memory, planning, and action modules.
% Then, we systematically review existing datasets and evaluation methodologies. 
% This survey not only presents the current landscape of LLM-powered agent recommender systems but also identifies key challenges and promising future research directions in this rapidly evolving field.

Recommender systems are essential components of many online platforms, yet traditional approaches still struggle with understanding complex user preferences and providing explainable recommendations.
The emergence of Large Language Model (LLM)-powered agents offers a promising approach by enabling natural language interactions and interpretable reasoning, potentially transforming research in recommender systems.
This survey provides a systematic review of the emerging applications of LLM-powered agents in recommender systems.
We identify and analyze three key paradigms in current research: (1) Recommender-oriented approaches, which leverage intelligent agents to enhance the fundamental recommendation mechanisms; (2) Interaction-oriented approaches, which facilitate dynamic user engagement through natural dialogue and interpretable suggestions; and (3) Simulation-oriented approaches, which employ multi-agent frameworks to model complex user-item interactions and system dynamics.
Beyond paradigm categorization, we analyze the architectural foundations of LLM-powered recommendation agents, examining their essential components: profile construction, memory management, strategic planning, and action execution. 
Our investigation extends to a comprehensive analysis of benchmark datasets and evaluation frameworks in this domain. 
This systematic examination not only illuminates the current state of LLM-powered agent recommender systems but also charts critical challenges and promising research directions in this transformative field.

\end{abstract}