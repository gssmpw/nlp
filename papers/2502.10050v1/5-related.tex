\section{Related Research Fields}

\paragraph{LLM-powered Recommender Systems} In recent years, recommender systems based on Large Language Models (LLMs) have attracted widespread attention. 
Such systems make full use of the powerful language understanding and generation capabilities of LLMs, bringing a new paradigm to traditional recommender systems.
Most existing methods are primarily designed for rating prediction~\cite{bao2023tallrec} and sequential recommendation~\cite{hou2024large,shao2024ulmrec,zheng2024adapting}.
CoLLM~\cite{zhang2023collm} captures and maps the collaborative information through external traditional models, forming collaborative embeddings used by LLMs. 
LlamaRec~\cite{llamarec} fine-tunes Llama-2-7b for list-wise ranking of the pre-selected items.
However, these methods would face significant limitations: the inability to simulate authentic user behaviors for enhanced personalization, the lack of effective memory mechanisms for long-term context awareness, and the rigid pipeline structure that prevents flexible task decomposition and seamless integration with external tools.

\paragraph{Conversational Recommender Systems}

Conversational recommender systems (CRS) have emerged as a significant research direction in recent years~\cite{jannach2021survey}, which are similar to the LLM-powered agent recommender systems. 
However, traditional methods~\cite{lei2020interactive} have two main drawbacks: attribute-based approaches are limited by rigid dialogue patterns, while generation-based methods suffer from restricted knowledge and poor generalization capabilities of small language models.


\section{Future Directions}

    %\item \textbf{Enhancement of Agent Capabilities}: Current LLM-powered recommendation agents exhibit limitations in behavioral understanding and long-term memory mechanisms. Future research should focus on advancing user intent comprehension, improving knowledge management systems, and enhancing task planning and decision-making capabilities to better accommodate dynamic user preferences.
\paragraph{Optimization of System Architecture} The integration between traditional recommendation methods and LLMs remains insufficient, with challenges in multi-agent collaboration and system interpretability. Future developments should explore flexible architectural designs, enhance agent cooperation efficiency, while ensuring transparency in recommendation.

\paragraph{Refinement of Evaluation Framework} There is a notable absence of unified and comprehensive evaluation standards for accurately measuring dialogue quality and recommendation effectiveness. Future research necessitates the establishment of robust evaluation frameworks, development of novel performance metrics, and consideration of privacy and security concerns in practical applications.

\paragraph{Security Recommender System} \cite{ning2024cheatagent} reveals the vulnerability of LLM-empowered recommender systems to adversarial attacks. In future,  the researchers could develop robust adversarial detection methods, investigate multi-agent defensive architectures, and integrating domain-specific security knowledge into defense mechanisms.