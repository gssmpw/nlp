\section{Limitations and Future Work}
Our work provides valuable insights and implications for the future design of co-creative systems for object generation in VR. 
In particular, we are confident that our technology-agnostic approach, which does not rely on current AI capabilities or technologies, will ensure that the results remain relevant for future systems, even if these are not constrained by the same technological limitations. %

However, we acknowledge that our experiment focused on \textbf{specific AI representation modes}, and other variants or visualizations may yield different results. 
We selected these specific modes of representation because of their importance for human collaboration to explore their impact on users' perceptions of the co-creative process and to stimulate further research and alternative implementations. 


Moreover, the \textbf{limited set of models} that participants were asked to modify was an extraneous variable, potentially affecting participants' experience with and perception of the system. Though we tried to minimize the effects of this variable by randomizing the order in which models were presented to each participant, we expect that with a small sample size such as ours, effects could not be avoided.

Also, the \textbf{limited number of possible AI modifications}, which remained constant during iterative modifications, could have led participants to the conclusion that their AI partner did not actually generate object parts but retrieved them from previously modeled parts. We hypothesize that this may have had an impact on the perception of AI and its results, as there was nothing about AI's capabilities and how its "generative" process changed between conditions.

Our study also did not explore the potential of a co-creative AI in \textbf{more complex, object-building scenarios}, such as simultaneous adjustments to multiple copies of the same object, adding and manipulating entire environments, or deformation and distortions of the object mesh (cf. \cite{Slim2024,Achlioptas2023,Angelis2024} on 3D shape editing and deforming).
We deliberately opted for a constrained set of predefined actions to focus our contribution on the fundamental questions of co-creative interaction and to provide a better understanding of co-creative tools in VR for object-building that are more than one-shot tools. While we are convinced that this provides a robust foundation for future work, we acknowledge that the constrained set of predefined actions might limit the generalizability of our findings, and future work in this domain is necessary.
Exploring how the AI's functionality scales with increased diversity and complexity of tasks is an important direction for future research, particularly for understanding the full potential of AI in co-creative VR systems.

Moreover, the efficiency of the co-creative tool itself was not explicitly evaluated in our study. 
The results presented thus may be correlated with how effectively the co-creative system executed its tasks, potentially influencing users' perceptions of the system's creativity. 
As generative AI becomes more advanced and prevalent in co-creative processes, future research should examine the role of AI efficiency and its impact on users' experiences and outcomes in co-creative tasks. Such investigations could provide a deeper understanding of how system performance and responsiveness affect the perceived value and usability of co-creative systems.




Since we performed the study as a controlled experiment, we acknowledge limitations in \textbf{external validity} in favor of establishing cause-effect relationships to recognize the impacts of individual representation modes. %


There are several interesting directions for \textbf{future work}:
One important area is exploring how different methods of emphasizing the AI's contribution process might positively impact a user's experience with a co-creative system, including whether such methods can enhance user navigation and coordination during collaboration. 
Additionally, investigating various ways to incrementally visualize the AI's generation of a 3D object could reveal how these visualizations affect users' perceptions of the AI and its creative process. 
There is also a need to examine the specific characteristics of avatars for co-creative AI that may cause discomfort to users, as well as how the perceived closeness to AI agents is affected by their embodiment and which other factors might influence this perception. 
Finally, future research should also explore whether manipulating the visual appearance of an AI's embodiment can alter people's understanding and evaluation of the AI and its creations.
