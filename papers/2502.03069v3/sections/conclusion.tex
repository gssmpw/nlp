\section{Conclusion}
Research on co-creative systems has predominantly focused on how humans participate in the co-creative process, often overlooking how AI contributions are represented and perceived by users. 
This paper addresses this gap by examining how an AI's contributions can be effectively represented for co-creative object generation in VR.
Through a Wizard-of-Oz study in a VR-based co-creative 3D object-generating environment, we investigated the effects on the co-creative collaboration experience for three key representation modes AI can use to communicate to the user: highlighting, incremental visualization, and embodiment of AI contributions.
Finally, we derive implications for the design of tools for co-creative object-building: First, designers should avoid combining highlighting with embodiment to increase user satisfaction. Second, while an embodiment increases perceived support, it also reduces ownership of the created objects. Third, designers should favor immediate visualization when the creation process is secondary.