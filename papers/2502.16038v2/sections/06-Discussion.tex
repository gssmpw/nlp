\section{Discussion and Future Work} 
This section provides a comprehensive discussion of the study’s key findings, highlights its limitations, and outlines promising directions for future research in the field.
% 本节全面讨论了该研究的主要发现,强调了其局限性,并概述了该领域未来研究的有希望的方向。

\subsection{Multimodal Integration and Progressive Relationship in Emotional Regulation} % 情绪调节中的多模态整合和渐进关系
This subsection explores the multifaceted role of emotional regulation in design, focusing on the integration of multimodal elements, the commonalities and differences across modalities, and the progressive relationship between emotional regulation and information communication.
% 本小节探讨了情绪调节在设计中的多方面作用,重点关注多模态元素的整合、模态之间的共性和差异,以及情绪调节与信息传递之间的递进关系。

% 通过文本、视觉、声音、交互来影响情绪效价、唤起度以及情绪激活系统
\textbf{The multidimensional mechanisms of emotional design: }By integrating text, visuals, sound, and interaction, design can finely tune users’ emotional valence, arousal and dominance, thereby effectively enhancing the reception and transmission of information. This multidimensional regulation not only works through emotional stimuli in a single modality but also strengthens users’ overall experience through the synergistic integration of modalities. For example, high-valence and high-arousal design strategies complement each other through positive narratives in text and dynamic colors in visuals, further enhancing emotional valence and user engagement with the support of sound and interaction. Particularly in memory formation, the narrative structure of text and the rhythmic continuity of sound jointly activate users’ memory mechanisms, while visual impact and interactive feedback reinforce memory encoding through immediate emotional triggers. These designs not only enhance emotional valence and arousal levels but also activate emotion-related neural, sensory-motor, motivational, and cognitive systems, constructing a comprehensive emotional experience from low-level automatic responses to higher-order cognitive processing. The rapid response of the nervous system ensures the immediacy of emotions, while the sensory-motor system enhances user immersion through dynamic interaction and tactile feedback. Simultaneously, the motivational system stimulates users’ willingness to act, and the cognitive system’s involvement enables deeper understanding and memory of information driven by emotions. This multidimensional regulation mechanism highlights the potential value of emotional regulation in information communication, emphasizing that design is not an isolated operation of modalities but an essential tool for achieving progressive emotional guidance and information communication through dynamic integration.
% 情感设计的多维作用机制:通过文本、视觉、声音和交互的综合运用,设计能够针对用户的情绪效价、唤起水平和支配感进行精细调节,从而有效促进信息的接收与传递。这种多维调控不仅通过单一模态的情感刺激发挥作用,更通过模态间的协同整合强化用户的整体体验。例如,高效价与高唤起的设计策略通过文本中的积极叙事与视觉中的动态色彩相辅相成,在声音与交互的支持下进一步增强情绪效价和用户参与度。特别是在记忆形成的过程中,文本的叙事结构与声音的节奏延续性共同激活用户的记忆机制,而视觉冲击和交互反馈则通过即时的情感触发强化记忆编码。这些设计不仅提升了情绪效价和唤起水平,还通过激活情绪相关的神经系统、感官运动系统、动机系统与认知系统,构建了从低层次自动反应到高阶认知加工的全方位情绪体验。神经系统的快速反应确保了情感的即时性,感官运动系统则通过动态交互和触觉反馈增强了用户的沉浸感。与此同时,动机系统激发了用户的行动意愿,而认知系统的参与使信息在情绪驱动下得到深层次的理解和记忆。这种多维调控机制凸显了情感调控对信息传播的潜在价值,强调设计并非孤立的模态操作,而是通过动态整合实现情感引导与信息传递递进过程的重要工具。

% 虽然文本、视觉、声音、交互都能影响情绪效价、唤起度,但不同设计在影响情绪过程中存在差异
\textbf{Cross-domain Design Commonalities and Differences:} Although text, visuals, sound, and interaction differ in how they regulate emotions, they exhibit strong commonalities in modulating emotional valence and arousal levels. All modalities enhance the efficient transmission of information by optimizing users’ perception, understanding, and memory of the content. For instance, design strategies for high-valence and high-arousal scenarios often emphasize immediate feedback and heightened emotional tension, which activate users’ motivational systems and accelerate the speed and scope of information communication. However, significant differences exist between modalities. Visual design conveys information through rapid sensory impact, while sound design excels at regulating emotions through continuity and rhythm. Text builds deeper emotional resonance through narrative structures, and interaction design strengthens the dominance through immediate feedback and active user participation. These differences indicate that in design practice, harmonizing the strengths of modalities and achieving consistent emotional guidance through integration are crucial for optimizing information communication. Especially at the arousal level, the mechanisms of control in different emotional scenarios—such as reinforcing a sense of mastery through immediate feedback in high-arousal contexts or achieving focus and stability through minimalistic design in low-arousal contexts—offer actionable guidance for future design strategies.
% 跨领域设计的共性与差异:尽管文本、视觉、声音和交互在调节情感方面的表现形式有所不同,它们在情绪效价与唤起水平的调控上表现出高度共性。所有模态的设计均通过优化用户对信息的感知、理解与记忆,促进了信息的高效传递。例如,高效价与高唤起场景的设计策略普遍强调即时反馈与情绪张力的增强,这种设计能够激活用户的动机系统,提升信息传播的速度与广度。然而,不同模态之间也存在显著差异。视觉设计通过快速的感官冲击传递信息,而声音设计擅长通过延续性和节奏调节情绪,文本则通过叙事结构构建更深层次的情感共鸣,交互设计则通过即时反馈和用户操作的主动参与强化支配感。这种差异表明,在设计实践中,如何协调模态间的优势,并通过整合实现一致的情感引导,是优化信息传递的重要方向。特别是在唤起度层面,支配感在不同情绪场景中的作用机制,例如在高唤起场景中通过即时反馈强化掌控感,在低唤起场景中通过简约设计实现专注与稳定——为未来的设计策略提供了更具操作性的指导。

% 效价、唤起在理解、记忆、分享3个阶段有显著作用。分为高效价+高唤起、高效价+地唤起、低效价+高唤起、低效价+低唤起 4种不同的差异。
\textbf{The Progressive Relationship Between Emotional Regulation and information communication:} 
The core goal of design is not only to regulate user emotions but also to drive effective information communication and positive behavioral transformation through emotional states. Emotional valence and arousal levels have significant impacts across the three stages of information communication: understanding, memory, and sharing. For instance, high-valence and high-arousal designs use intense emotional stimuli to capture users’ attention and exploratory motivation, facilitating rapid information communication. In contrast, high-valence and low-arousal designs provide a more stable psychological environment conducive to deep information internalization. Low-valence and high-arousal designs create tension and alertness, emphasizing the importance of information and prompting rapid user responses, such as the sound strategy in A Quiet Place. Conversely, low-valence and low-arousal designs use restrained expression to help users calmly process complex or unpleasant information. This demonstrates that emotional regulation not only affects users’ immediate emotional states but also progressively transforms information into long-term memory and behavioral impact. During this process, the full engagement of the four emotional activation systems amplifies the effectiveness of design. The combination of rapid neural responses and higher-order cognitive processing allows design to dynamically adapt to different dissemination needs while enhancing the effectiveness of information communication and the depth of emotional resonance.
% 情感调控与信息传达的递进关系:设计的核心目标不仅在于调控用户情绪,还在于通过情绪状态推动信息的有效传递与行为的积极转化。情绪效价和唤起水平在信息传递的理解、记忆和分享三个阶段中展现了显著影响力。例如,高效价与高唤起的设计通过强烈的情绪刺激激发用户的注意力和探索动机,推动信息的快速传递。相比之下,高效价与低唤起设计提供了更稳定的心理环境,有助于信息的深度内化。低效价与高唤起设计则通过制造紧张感与警觉性,强调信息的重要性并促使用户快速反应,如《寂静之地》的音效策略;而低效价与低唤起设计则通过克制的表达方式帮助用户冷静接受复杂或不愉快的信息。这表明情感调控不仅影响用户的即时情绪状态,还能够通过递进关系将信息转化为长期的记忆与行为影响。在这一过程中,情绪激活4系统的全面参与进一步放大了设计的有效性。快速的神经反应结合高阶认知加工,使设计能够动态适应不同传播需求,同时提升信息的传播效果与情感共鸣的深度。

\subsection{Limitations}
Although this study systematically explores the role of emotions in information communication and proposes a multimodal design framework, several significant limitations deserve attention. First, this study primarily relies on a review of existing literature, lacking direct experimental validation of emotional design in real-world application scenarios. This limitation may lead to a disconnect between theory and practice, especially when the applicability of multimodal design strategies across different dissemination platforms remains unclear. Second, the study focuses primarily on the effects of emotional valence (positive or negative) and arousal (emotional intensity) on information communication, overlooking the moderating roles of external factors such as cultural context, real-time interaction on social media, and the type of content being disseminated. Particularly in dynamic dissemination environments (e.g., real-time interactions on social media), emotional effects may fluctuate significantly over time and context, a phenomenon insufficiently addressed in the current research framework. Third, the proposed emotional design framework focuses mainly on the four dimensions of visual, textual, auditory, and interaction design. However, with the rapid development of technologies such as virtual reality (VR), augmented reality (AR), and multimodal sensory technologies (e.g., haptic and olfactory design), their roles in emotional dissemination merit further exploration. Finally, the universality of the design framework requires more cross-cultural validation studies, as preferences for emotional expression and reception can vary significantly across cultures, potentially influencing dissemination outcomes.
% 尽管本研究系统性地探索了情感在信息传播中的作用,并提出了多模态设计框架,但仍存在一些重要的局限性需要引起关注。首先,本研究的主要依据是现有文献的综述分析,缺乏直接针对情感设计在真实应用场景中的实验验证。这种局限可能导致理论与实践之间的脱节,特别是在多模态设计策略在不同传播平台上的适用性尚未明确的情况下。其次,本研究主要关注情感效价(正面或负面)与唤醒度(情感强度)对信息传播的影响,而忽视了情感传播过程中多种外部因素的调节作用,例如文化背景、社交媒体的实时互动以及传播信息的内容类型。尤其是在动态传播环境中(如社交媒体上的实时交互场景),情感效应可能随时间和情境的变化而显著波动,这在现有研究框架中未被充分探讨。第三,本研究提出的情感设计框架主要集中于视觉、文本、声音和交互设计四个维度,但随着虚拟现实(VR)、增强现实(AR)以及多模态感官技术(如触觉、嗅觉设计)的快速发展,这些新兴技术在情感传播中的作用值得进一步探索。最后,设计框架的普适性需要更多跨文化的验证研究,因为不同文化对情感表达和接受的偏好可能会显著影响传播效果。

\subsection{Opportunities For Future Research}
While significant progress has been made in designing for emotional regulation, many areas remain to be explored. First, the introduction of real-time emotional regulation technologies will further expand the possibilities for related design. For example, dynamically adjusting textual, visual, auditory, and interactive elements using artificial intelligence and biofeedback technologies can more precisely adapt to users’ immediate emotional states, thereby optimizing information communication efficiency. Additionally, optimizing multimodal integration is a critical direction. How to achieve consistency across modalities, avoid conflicts or interference, and leverage modal differences to create emotional tension presents new strategies for capturing user attention. Cultural adaptability is also a key area for future research. Significant differences exist in the acceptance and preferences for emotions in designs across cultural contexts, making emotional adaptation in global dissemination an important challenge. At the same time, designing for emotional regulation must consider its ethical boundaries. Designers need to balance the efficiency of information communication with user well-being, avoiding emotional manipulation that could lead to psychological burdens or polarization of societal emotions. Addressing these issues will provide stronger support for the application of related designs in social communication, education, entertainment, and public~services.
% 调节情感的设计虽已取得重要进展,但仍有许多值得探索的领域。首先,实时情感调控技术的引入将进一步丰富相关设计的可能性。例如,通过人工智能与生物反馈技术动态调整文本、视觉、声音与交互元素,可以更精准地适配用户的即时情绪状态,从而优化信息传递效率。此外,多模态整合的优化是重要方向。如何在模态间实现一致性,避免冲突或干扰,同时利用模态差异化制造情感张力,为吸引用户注意提供新的策略。文化适配性问题也是未来研究的重要领域,跨文化背景下设计对情感的接受度与偏好存在显著差异,如何实现全球传播中的情感适配将是重要挑战。同时,调节情感的设计需关注其伦理边界,设计师需平衡信息传递效率与用户福祉,避免情绪操控引发心理负担或社会情绪的极化。这些问题的解决将为相关设计在社会传播、教育、娱乐以及公共服务中的应用提供更强有力的支持。