\section{Conclusion} 
This study systematically examines the impact of emotions on information communication, analyzing the key roles of emotional valence, arousal and dominance across the three core stages: understanding, memory, and sharing. Through the four design dimensions of text, visuals, sound, and interaction, the study proposes a multimodal integrated framework for emotional regulation and explores how these designs activate emotion-related neural, sensory-motor, motivational, and cognitive systems. The study reveals the unique advantages and synergies of different modalities in emotional regulation, further highlighting the potential to optimize information communication efficiency and effectiveness by regulating emotions. These findings not only deepen the theoretical understanding of emotions in information communication but also provide systematic pathways and practical value for user experience optimization and innovative information design across multiple fields through multimodal design strategies.
% 本研究系统性地探讨了情感对信息传递的影响,从理解、记忆和分享三个核心环节入手,分析了情感效价、唤起水平与支配感在信息传播中的关键作用。通过文本、视觉、声音和交互四个设计维度,提出了一种多模态整合的情感调控框架,并进一步探讨了这些设计如何激活情绪相关的神经系统、感官运动系统、动机系统与认知系统。研究揭示了不同模态在情感调控中的独特优势及其协同作用,进一步明确了通过调节情绪优化信息传递效率与效果的潜力。这些发现不仅深化了情感在信息传播中的理论理解,还通过多模态设计策略为多领域的用户体验优化与信息设计创新提供了系统性路径和实用价值。