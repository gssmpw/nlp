\section{Why does emotion matter in information communication?} \label{sec:why} %为什么情感在信息传达中很重要?

Emotions play a crucial role in information communication as they directly affect the reception, processing, and dissemination of information \cite{ferrara2015measuring, berger2012makes}. Emotions not only influence individuals’ psychological experiences \cite{jimenez2012emotional, lopez2021translating, berger2012makes, stieglitz2013emotions}, decisions, and behaviors \cite{schreiner2021impact, de2021sadness}, but also shape public opinion, trigger collective action, and even drive social change through group interaction and social communication \cite{stieglitz2013emotions, ferrara2015measuring, son2022emotion}. Therefore, studying the impact of emotions on information communication holds not only theoretical value but also profound practical significance. This study examines the core role of emotions in information dissemination from five perspectives: communication studies \cite{tyng2017influences, schreiner2021impact, ferrara2015measuring, berger2012makes, vosoughi2018spread}, psychology \cite{mather2011arousal, hanson2014happy, jimenez2012emotional}, sociology \cite{sanford2004negative}, neurobiology \cite{phelps2004human, mcgaugh2015consolidating}, and humanism \cite{dykas2011attachment, stieglitz2013emotions, van2020seeking, dubey2020psychosocial}. It reveals how emotions enhance the effectiveness of information communication through mechanisms such as capturing attention \cite{egidi2012emotional, tyng2017influences}, promoting spread \cite{schreiner2021impact, ferrara2015measuring, berger2012makes, stieglitz2013emotions}, strengthening memory and understanding \cite{pawlowska2011influence, megalakaki2019effects, hanson2014happy, kensinger2007negative, kensinger2020retrieval, van2015good}, and stimulating group behavior \cite{ferrara2015measuring, son2022emotion, berger2012makes, de2021sadness}. At the same time, emotions humanize and energize the information, improving its acceptance and dissemination power \cite{dykas2011attachment, van2020seeking, dubey2020psychosocial}.
% 情感在信息传达中的作用至关重要,因为它直接影响信息的接受、处理和传播效果。情感不仅能影响个体的心理体验、决策和行为,还能通过群体互动与社会传播,塑造公共舆论、引发集体行动,甚至推动社会变革。因此,研究情感对信息传递的影响,不仅具有理论价值,也对实践应用有着深远意义。本研究从传播学、心理学、社会学、生物与神经科学以及人文主义五个角度,梳理了情感在信息传播中的核心作用,揭示了情感如何通过吸引注意、促进传播、强化记忆与理解、激发群体行为等多种方式,提升信息的传播效果。同时,情感也赋予信息人性化和感染力,提高了信息的接受度与传播力。

% In recent years, the impact of emotions on information communication has garnered significant attention in the fields of cognitive science and psychology. Emotions are not only psychological experiences of individuals but also profoundly influence behaviors and decisions, permeating the processes of information reception, processing, and dissemination. More importantly, emotions shape public opinion through group interactions and social communication, guide collective behaviors, and even drive social change \cite{egidi2012emotional}. Therefore, understanding the role of emotions in information communication is of great importance both theoretically and practically.
% %近年来,情感对信息传达的影响在认知科学和心理学领域备受关注。情绪不仅是个体的心理体验,还深刻影响行为和决策,贯穿信息的接受、处理和传播过程。更重要的是,情绪通过群体互动和社会传播塑造公共舆论,引导集体行为,甚至推动社会变革。因此,理解情感对信息传达的作用在理论和实践上都具有重要意义。

% To systematically explore the multidimensional impact of emotions on information communication, this study employed thematic analysis to code and categorize 167 relevant papers, distilling the core views on how emotions affect information communication. The research team read each paper individually, marking the role of emotions in information communication and conducting coding and categorization to identify different argumentative perspectives. Through repeated comparison and discussion, similar codes were integrated into higher-level themes. For instance, research indicates that emotive language enhances audience memory and understanding of news content \cite{moriya2023can}, thus this view was classified under the theme ``the role of emotion in news dissemination," falling into the ``application perspective."
% %为系统探讨情绪对信息传达的多维影响,本研究采用主题分析法,对167篇相关文献进行了编码和分类,以提炼情感对信息传达的核心观点。研究团队逐篇阅读文献,标记情感在信息传播中的作用,并进行编码和分类,以识别不同的论证角度。通过反复比较和讨论,相似的编码被整合为更高层次的主题。例如,有研究表明,情感化语言提高了受众对新闻内容的记忆和理解,因此该观点被归类为“在新闻传播中情感的作用”,并归入“应用角度”主题。

% Ultimately, this study elucidates the importance of emotion in information communication from five perspectives: communication, psychology, sociology, neuroscience, and humanism. First, from the perspective of communication, emotion effectively captures audience attention and drives the dissemination and network diffusion of information. Next, from a psychological perspective, emotion strengthens memory, facilitates understanding, and influences individual attitudes and behaviors, thereby enhancing the depth and effectiveness of information. At the sociological level, emotion not only promotes social interaction but also triggers group behavior, fostering information exchange and resonance among groups. From the perspective of neuroscience, emotional information is prioritized by the amygdala and reinforced through its interaction with the hippocampus, while the ventral striatum and insula enhance the dissemination of information. Finally, from a humanistic perspective, emotion prevents information from becoming indifferent, making it more humanized and impactful, thereby increasing its acceptance and dissemination~power.
% %最终,本研究从传播学、心理学、社会学、生物与神经科学以及人文主义五个角度来阐释情感在信息传达中的重要性。首先,从传播学角度看,情感能够有效吸引受众的注意力,并激发信息的传播与网络扩散。接着,从心理学角度,情感强化记忆、促进理解,并能影响个体的态度与行为,从而增强信息的深度和效果。在社会学层面,情感不仅能促进社会互动,还能激发群体行为,促进群体间的信息交换与共鸣。从生物与神经科学角度来看,情感信息通过杏仁核的优先处理和杏仁核与海马体的联动,强化了情感记忆,且腹侧纹状体与岛叶对信息的传播产生了增强作用。最后,从人文主义的角度,情感避免了信息的冷漠化,使其更具人性化与感染力,进而提升了信息的接受度与传播力。

\begin{figure*}[hbt!]
%\setlength{\abovecaptionskip}{-0.1mm}
\setlength{\intextsep}{10pt plus 2pt minus 2pt}
    \centering
    \includegraphics[width=18cm]{figs/Fig.1-Why.jpg}
    \caption{This figure illustrates the role of emotions in information communication from five perspectives: communication (P1), psychological (P2), neurobiology (P3),  sociology (P4), and humanism (P5).}
   \vspace{-2mm}
\label{fig:why}
\end{figure*}


\subsection{From the Perspective of Communication}%传播学角度
From the perspective of communication studies, the importance of emotion in information communication lies in its ability to effectively capture audience attention and drive rapid information diffusion.
%From the perspective of communication studies, the importance of emotion in information communication lies in its ability to effectively capture audience attention and drive rapid information diffusion. Especially in social media contexts, emotionalized content not only evokes strong emotional reactions but also stimulates the audience’s desire to share, thereby rapidly spreading across networks and expanding its~influence.
% 情感在信息传播中的重要性体现在其能够有效吸引受众注意力并推动信息的快速扩散。尤其在社交媒体环境中,情感化的内容不仅能够引发强烈的情绪反应,还能够激发受众的分享欲望,从而在网络中迅速传播并扩展其影响力。

\textbf{Emotion is a critical switch for capturing attention:} In information communication, particularly in social media contexts, emotions play a key role in drawing users’ attention. Studies have shown that emotionalized information, especially content with higher emotional arousal, is more likely to capture audience attention compared to neutral information \cite{egidi2012emotional, tyng2017influences}. Specifically, highly arousing emotional content such as anger and surprise tends to stimulate audience interest, achieve higher initial exposure, and rapidly capture viewers’ attention \cite{lopez2021translating, vosoughi2018spread}. These characteristics make emotional content a “switch for attention” in information communication, effectively capturing the audience’s focus. For example, Vosoughi et al. \cite{vosoughi2018spread} indicated that news capable of evoking anger or surprise often garners higher attention and spreads rapidly, often becoming “flashpoints” in online information communication. This effect is particularly pronounced on platforms like social media. Given the vast and rapidly changing information on these platforms, emotionalized content triggers intense emotional responses, gaining more exposure in a short time and rapidly spurring discussions and shares \cite{schreiner2021impact}.
% 情感是吸引注意力的关键开关:情感在信息传播中,尤其是在社交媒体环境下,扮演着吸引用户注意力的关键角色。研究表明,情感化的信息,尤其是那些情感唤起度较高的内容,比中性信息更容易引起观众的关注(Egidi & Nusbaum, 2012;Tyng et al., 2017)。特别是愤怒、惊讶等高唤起度的情感内容,更能激发受众的兴趣,获得更高的初始曝光率,并迅速吸引观众的注意力(Rojo López & Naranjo, 2021;Vosoughi et al., 2018)。这些特征使情感内容成为信息传播中的“关注开关”,能够有效抓住受众的眼球。例如,Vosoughi等(2018)指出,能够引发愤怒或惊讶的新闻往往吸引更高的关注度并迅速传播,通常成为网络信息传播的“引爆点”。在社交媒体等平台上,这一效应尤为明显。由于平台上的信息量巨大且瞬息万变,情感化的内容通过激发强烈的情绪反应,使信息在短时间内获得更多曝光,并迅速引发讨论和转发(Schreiner et al., 2021)。

\textbf{Emotion drives information communication and network diffusion:} Emotion-driven information demonstrates strong diffusion capability during dissemination. When information aligns with individuals’ emotional stance or values, audiences are more likely to actively share such content \cite{ferrara2015measuring}. This mechanism allows emotionalized content to overcome the challenge of information overload and achieve faster and broader dissemination on social media platforms, especially in emotionally resonant groups. Studies show that highly emotional news, whether true or false, often spreads more widely than rational and neutral news, highlighting the significant driving force of emotion in information communication \cite{berger2011arousal, stieglitz2013emotions}. Highly emotional content enhances its diffusion potential in social networks by eliciting emotional reactions from users, thereby promoting sharing and dissemination. When information evokes strong emotions such as anger, fear, or empathy, it often compels users to immediately share or comment, accelerating the spread of information. For example, news agencies often enhance dissemination by employing emotional narratives and visual designs, such as depicting victims’ emotions or urgent situations, to generate social concern and action \cite{lopez2021translating}. This increased willingness to disseminate helps emotionalized content achieve viral spread on social networks, creating greater diffusion effects \cite{de2021sadness, son2022emotion}.
% 情感推动信息传播与网络扩散:情感驱动的信息在传播过程中展现出强大的扩散能力。当信息与个体的情感立场或价值观相契合时,受众更倾向于主动分享这些内容(Ferrara & Yang, 2015)。这一机制使情感化的内容能够突破信息过载的困境,在社交媒体平台上实现更快速、更广泛的传播,特别是在情感共鸣强烈的群体中。研究表明,情感极为强烈的新闻(无论真假)往往比理性和中立的新闻传播得更广,显示了情感在信息传播中的巨大推动力(Berger & Milkman, 2012;Stieglitz & Dang-Xuan, 2013)。情感强烈的内容通过激发用户的情感反应,促进信息的分享和扩散,增强其在社交网络中的扩散潜力。当信息引发愤怒、恐惧或共情等强烈情感时,常促使用户立即转发或评论,进而加速信息的传播。例如,新闻机构在报道灾难事件时,常通过情感化的叙事和视觉设计(如展现受害者情感或紧急情况)提升传播效果,引发社会关注和行动(Rojo López & Naranjo, 2021)。这种传播意愿的提升,帮助情感化的信息在社交网络中实现病毒式传播,形成更大的传播效应(de León & Trilling, 2021;Son et al., 2022)。

Emotion is a core factor driving information communication, especially prominent in social media contexts. Intense emotional content (such as anger or surprise) can quickly capture audience attention, stimulate interest, and prompt sharing and discussion, leading to rapid information diffusion.
% 从传播学的角度看,情感是推动信息传播的核心因素,尤其在社交媒体环境中表现得尤为突出。强烈的情感内容(如愤怒或惊讶)能够迅速吸引受众的注意力,并通过激发兴趣促使他们主动分享和参与讨论,从而形成信息的快速扩散。


\subsection{From the Perspective of Psychological}%传播学角度
From the perspective of psychological emotion not only enhances the memorability of information but also facilitates understanding and triggers behavioral responses. Emotionalized information mobilizes cognitive resources, increases the depth of information processing, and helps the audience more easily remember and understand the content. Moreover, emotion plays a significant role in influencing audience attitudes and behaviors by eliciting emotional responses and prompting individuals to make corresponding behavioral changes.
% 从心理学的角度来看,情感不仅能增强信息的记忆效果,还能促进理解并引发行为反应。情感化的信息通过调动认知资源,提升信息加工的深度,帮助受众更容易记住和理解内容。同时,情感在影响受众态度和行为上的作用也尤为突出,能够激发情绪反应并促使个体作出相应的行为改变。

\textbf{Emotion enhances memory:} Emotion can strengthen memory in information communication by influencing individuals’ attention allocation and cognitive resources, thereby deepening and prolonging information processing. Psychological research indicates that emotional information is more likely to be encoded into long-term memory compared to neutral information, especially information eliciting strong emotional responses such as anger or sadness, which significantly increases memory prioritization \cite{mather2011arousal}. Emotion establishes a deep connection with individuals’ emotional experiences, promoting deeper information processing and enhancing its storage stability in long-term memory \cite{jimenez2012emotional}. For instance, in the field of education, the use of emotional narratives to help students master complex knowledge demonstrates the broad application of this characteristic of emotionalized information \cite{tyng2017influences}.
% 情感强化记忆:情感在信息传达中能够强化记忆力,这是因为它通过影响个体的注意力分配和认知资源,增强了信息的加工深度和持久性。心理学研究表明,情感性信息比中性信息更容易被编码进长期记忆,尤其是那些引发强烈情绪反应(如愤怒、悲伤)的信息,能够显著提升记忆优先级(Mather & Sutherland, 2011)。情感通过与个体的情感体验建立深度联系,促进了信息的深度加工,从而提升其在长期记忆中的存储稳定性(Jiménez-Ortega et al., 2012)。例如,在教育领域,通过情感叙事帮助学生掌握复杂知识,情感化信息的这种特性得到了广泛应用(Tyng et al., 2017)。

\textbf{Emotion facilitates comprehension:} Emotion can effectively enhance the comprehension of information during its transmission. Emotionalized information reduces cognitive load and increases contextual relevance, making complex content easier to comprehend. Emotionalized information enhances contextual relevance, helping audiences link the information to personal experiences and deepen understanding \cite{megalakaki2019effects, egidi2012emotional}. In practical applications, emotion focuses cognitive resources, enhancing the grasp of key information. For example, in health communication, patient stories instead of mere statistics can more intuitively convey health risks \cite{lopez2021translating}. Furthermore, in negative emotional states, audiences focus more on details and logic, thereby deepening their understanding of the information \cite{arfe2023effects}. Thus, emotion is a core factor in optimizing information processing and enhancing comprehension depth, widely validated in fields such as educational communication~\cite{megalakaki2019effects} and health promotion \cite{lopez2021translating}.
% 情感促进理解:情感在信息传达中能够有效提升信息的理解效果。情感化信息通过降低认知负担和增强情境关联性,使复杂内容更易被接受。情感化信息通过增加情境关联性,帮助受众将信息与个人经验联系,从而加深理解(Megalakaki et al., 2019,Egidi & Nusbaum, 2012)。在实际应用中,情感通过集中认知资源,强化对关键信息的把握。例如,在健康传播中,通过患者故事而非仅展示统计数据,能更直观地传达健康风险(Rojo López & Naranjo, 2021)。此外,负面情感状态下,受众对信息细节和逻辑的关注度显著增强,从而加深对信息的理解(Arfé et al., 2023)。因此,情感是优化信息处理、增强理解深度的核心因素,在教育传播(Megalakaki et al., 2019)、健康宣传(Rojo López & Naranjo, 2021)等领域得到了广泛验证。

\textbf{Emotion influences attitudes and behaviors:} From a psychological perspective, emotion significantly impacts audience attitudes and behavioral tendencies in information communication. Emotion enhances psychological connections with information by eliciting emotional responses (e.g., trust, anger), thereby increasing the persuasiveness and acceptance of the information \cite{lopez2021translating}. Positive emotions (e.g., hope or inspiration) often elicit a favorable attitude toward information, while negative emotions (e.g., fear or anger) may drive audiences to take specific actions or alter their behavior patterns \cite{berger2012makes, schreiner2021impact}. Thus, emotion deeply influences psychological processes, significantly enhancing the dissemination effectiveness and behavioral impact of~information.
% 情感影响态度与行为:从心理学角度来看,情感在信息传达中显著影响受众的态度和行为倾向。情感通过唤起受众的情绪反应(如信任、愤怒或同理心),加强他们对信息的心理联结,从而提升信息的说服力和接受程度(Rojo López & Naranjo, 2021)。正面情感(如希望或鼓舞)通常会激发受众对信息的积极态度,而负面情感(如恐惧或愤怒)则可能驱使受众采取具体行动或调整行为模式(Berger & Milkman, 2012;Schreiner et al., 2021)。因此,情感通过深刻影响受众的心理过程,显著增强了信息的传播效果与行为影响力。

Emotion enhances attention allocation and optimizes cognitive resources, making high-arousal emotional information (such as anger or sadness) more memorable to audiences, significantly improving memory retention \cite{mather2011arousal}. Moreover, emotionalized information increases contextual relevance, aiding audiences in better understanding complex content \cite{egidi2012emotional}. Additionally, emotional responses such as trust, anger, or empathy enhance the psychological connection with information, increasing its persuasiveness and behavioral conversion rate, thereby making it more impactful in dissemination \cite{berger2012makes}.
% 情感通过强化注意力分配和认知资源优化,使高唤醒的情感信息(如愤怒或悲伤)更容易被受众记住,显著提升信息的记忆效果(Mather & Sutherland, 2011)。此外,情感化信息通过增强情境关联性,帮助受众更好地理解复杂内容(Egidi & Nusbaum, 2012)。同时,情感唤起的信任、愤怒或共情反应,能够增强受众对信息的心理联结,从而提升信息的说服力和行为转化率,这使得信息在传播中更具影响力(Berger & Milkman, 2012)。



\subsection{From the Perspective of Neurobiology} %神经科学角度
From the perspective of neurobiology, emotion enhances the rapid processing and deep memory of emotional information by activating the amygdala and hippocampus, while the coordinated activity of the ventral striatum, insula, and empathy-related brain regions facilitates the understanding, resonance, and sharing of emotional information, thereby driving its communication.
% 情感通过激活杏仁核和海马体增强情感信息的快速处理与深度记忆,同时,腹侧纹状体、岛叶及共情相关脑区的协同作用促进了情感信息的理解、共鸣与分享,从而推动信息的传播。

% From the perspective of neurobiology, the importance of emotion in information communication stems from its direct impact on brain processing mechanisms. Emotional information activates specific brain regions, such as the amygdala and hippocampus, enabling rapid transmission and memory enhancement, while also leveraging reward systems and empathy mechanisms to promote dissemination and social diffusion. These neural mechanisms collectively explain why emotional information can quickly capture attention, deepen understanding, and achieve wide dissemination.
% % 从生物神经科学的角度来看,情感在信息传递中的重要性源于其对大脑处理机制的直接影响。情感信息通过激活特定的脑区,如杏仁核和海马体,快速传递并增强记忆,同时还通过奖励系统和共情机制,促进信息的传播与社会扩散。这些神经机制共同解释了情感信息为何能够迅速吸引注意、深化理解并广泛传播。

\textbf{Amygdala prioritizes emotional information:} Emotional information has a priority-processing characteristic, making it occupy a crucial role in information communication. Studies indicate that when information has emotional characteristics, the amygdala in the brain is quickly activated. As a core region for emotion processing, the amygdala can bypass complex cognitive evaluations to directly process emotional information \cite{ledoux2000emotion}. This fast pathway ensures that threats, rewards, or other emotion-related information can be rapidly attended to and trigger physiological responses. For instance, when hearing an emergency alarm or seeing an angry face, the amygdala reacts quickly, prompting rapid recognition and response to the situation \cite{phelps2004human}. This prioritization mechanism, in collaboration with the prefrontal cortex (PFC) and the thalamus, provides a neural basis for the rapid recognition and transmission of emotional information.
% 杏仁核优先处理情感信息:情感信息具有优先处理的特性,这使其在信息传递中占据重要位置。研究表明,当信息具有情感特征时,大脑中的杏仁核(amygdala)会迅速被激活,杏仁核作为情感加工的核心脑区,可以绕过复杂的认知评估直接处理情感性信息(LeDoux, 2000)。这一快捷通道确保了威胁、奖励或其他情感相关信息能够迅速被注意并引发生理反应。例如,当听到紧急警报或看到愤怒的面孔时,杏仁核快速反应,触发个体对情境的迅速识别与应对(Phelps & LeDoux, 2005)。这种优先机制通过与前额叶皮层(PFC)和丘脑的协作,为情感信息的快速识别和传递提供了神经基础。

\textbf{Amygdala and hippocampus strengthen emotional memory: }Emotion enhances memory encoding and storage by activating pathways between the amygdala and hippocampus \cite{cahill1998mechanisms}. This mechanism makes emotional events (e.g., inspiring speeches or catastrophic news) easier to remember than ordinary information. The amygdala tags the emotional intensity of information, while the hippocampus integrates these tags into long-term memory, ensuring the persistence of emotional information in individual memory \cite{phelps2004human}. Additionally, emotional tags not only influence initial memory but also reactivate related memory pathways during recall. Research shows that emotional events often leave deeper memory traces, making them more attractive in the dissemination process \cite{sakaki2014emotion}. Emotion also enhances the receiver’s understanding of information through the interaction between the sensory cortex and the prefrontal cortex.	The prefrontal cortex regulates emotional signals, assisting individuals in evaluating the importance and social significance of information, thereby further enhancing the impact of emotional information in dissemination.	

% 杏仁核与海马体强化情感记忆:情感通过激活杏仁核与海马体(hippocampus)之间的通路,强化信息的记忆编码与存储(Cahill & McGaugh, 1998)。这一机制使情感性事件(如激动人心的演讲或灾难性新闻)比普通信息更容易被记住。杏仁核负责标记信息的情感强度,海马体则将这些标记整合到长时记忆中,从而确保情感信息在个体记忆中的持久性(Phelps, 2004)。此外,情感标记不仅影响初始记忆,还会在回忆过程中重新激活相关记忆通路。研究发现,情感事件往往具有更深的记忆痕迹,因此在传播过程中更具吸引力(Sakaki et al., 2014)。情感还通过感知皮层与前额叶皮层的协作,加深接收者对信息的理解。前额叶皮层通过调控情感信号,帮助个体判断信息的重要性和社会意义,进一步优化情感信息在传播中的影响力。

\textbf{Ventral striatum and insula enhance information communication:} Emotion significantly enhances information sharing and dissemination by activating the brain’s reward system and social mechanisms. Research shows that the ventral striatum in the brain is activated when processing positive emotional information, directly enhancing individuals’ willingness to share \cite{haber2010reward}. Content with strong emotional tones (e.g., touching stories or angry news) is more likely to trigger sharing behavior, primarily due to the driving role of the reward system \cite{van2012reward}. The medial prefrontal cortex (mPFC) and the temporoparietal junction (TPJ) enhance receivers’ understanding and resonance with emotionalized information by eliciting empathy effects \cite{decety2008emotion}. This empathy effect facilitates information communication within groups, providing neural support for the social diffusion of information. Additionally, the emotional integration role of the brain’s insula adds stronger emotional dimensions to information, making it more attention-grabbing in group dissemination \cite{craig2009you}. These mechanisms collectively demonstrate that emotion-driven sharing behavior significantly enhances the efficiency of information communication and its social impact.
% 腹侧纹状体与岛叶增强信息传播:情感通过激活大脑的奖励系统和社会机制,显著增强信息的共享与传播。研究表明,大脑的腹侧纹状体(ventral striatum)在处理正向情感信息时会被激活,这种激活直接增强了个体的分享意愿(Haber & Knutson, 2010)。具有强烈情感色彩的内容(如感人故事或愤怒新闻)更容易触发分享行为,这种现象主要归因于奖励系统的驱动作用(Van Steenbergen et al., 2012)。内侧前额叶皮层(mPFC)和颞顶联合区(TPJ)通过引发共情效应,增强了接收者对情感化信息的理解与共鸣(Decety & Meyer, 2008)。这种共情效应促进了信息在群体间的传播,为信息的社会化扩散提供了神经支持。此外,大脑岛叶(insula)的情感整合作用能够赋予信息更强烈的情感维度,使其在群体传播中更加引人注目(Craig, 2009)。这些机制共同表明,情感驱动的共享行为大幅提升了信息的传播效率和社会影响力。

\subsection{From the Perspective of Sociology}% 社会学角度

From the Perspective of Sociology, the importance of emotion in information communication is not only reflected in its influence on individual behavior but also in its ability to evoke group emotions and enhance social cohesion, thereby facilitating social interaction and driving collective action.
%从社会学角度,情感在信息传达中的重要性不仅体现在对个体行为的影响上,还通过激发群体情绪和增强社会凝聚力,促进社会互动并推动集体行动。

\textbf{Emotional information promotes social interaction:} From a sociological perspective, emotion plays a critical role in enhancing social interaction in information communication. Emotional information stimulates group emotions and strengthens social cohesion, thereby facilitating communication and connection among individuals. When individuals share strongly emotional content on social media (e.g., joyful celebrations or touching stories), such emotions can quickly spread within groups, enhancing emotional consistency and a sense of identity among members \cite{stieglitz2013emotions, ferrara2015measuring}. This emotional resonance not only fosters interaction between individuals but also strengthens connections within communities and social groups. For instance, in public welfare activities, emotional information can inspire people to engage in discussions and donations, creating tighter social networks \cite{son2022emotion}. Emotional information thus becomes a vital tool for promoting social interaction and enhancing social~capital.
% 情感信息促进社会互动: 从社会学角度来看,情感在信息传达中起到了增强社会互动的关键作用。情感信息通过激发群体情绪和加强社会凝聚力,促进了人们之间的交流和联系。当个人在社交媒体上分享带有强烈情感的内容(如喜悦的庆祝、感人的故事),这种情感能够在群体中迅速传播,增强成员之间的情感一致性和认同感((Stieglitz & Dang-Xuan, 2013);(Ferrara & Yang, 2015))。这种情感共鸣不仅促进了个体之间的互动,还加强了社区和社会群体的连结。例如,在公益活动中,情感信息可以激励人们参与讨论和捐赠,形成更紧密的社会网络(Son et al., 2022)。情感信息因此成为促进社会互动和增强社会资本的重要工具。

\textbf{Emotion triggers group behavior:} Emotion in information communication not only facilitates social interaction but also induces changes in group behavior. High-arousal emotions (such as anger and enthusiasm) can stimulate collective actions within groups, leading to wider information communication. Studies show that highly emotional information is more readily accepted and disseminated by groups, thereby promoting discussions on social issues and driving social change \cite{berger2012makes}. For example, in social movements, emotionalized promotional content can mobilize the public to participate in protests, sign petitions, or support policies, demonstrating the importance of emotion in shaping social behavior \cite{de2021sadness}. Emotional information thus becomes a key factor in driving group behavior and facilitating social change.
% 情感引发群体行为: 情感在信息传达中不仅促进了社会互动,还能够引发群体行为的变化。高唤醒的情感(如愤怒、热情)能激发群体的共同行动,使信息传播更为广泛。研究表明,带有强烈情感的信息更容易被群体接受和传播,从而推动社会议题的讨论和社会变革(Berger & Milkman, 2012)。例如,在社会运动中,情感化的宣传内容可以动员公众参与抗议、签署请愿或支持某项政策,这些集体行动体现了情感在引导社会行为中的重要性(de León & Trilling, 2021)。情感信息因此成为驱动群体行为和促进社会变革的关键因素。

% Emotional information is both a vital link for fostering social interaction and a key driver of group behavior and social change. The social effects of emotional dissemination not only reveal the mechanisms of interaction in the digital society but also provide new perspectives for understanding social action and change.
% % 情感信息既是促进社会互动的重要纽带,也是驱动群体行为和社会变革的关键动力。这种情感传播的社会效应不仅展现了数字社会中的互动机制,还为理解社会行动和变革提供了新的视角。

\subsection{From the Perspective of Humanism} % 人文主义角度
\textbf{Emotion prevents information from becoming indifferent:} The importance of emotion in information communication lies in its ability to effectively prevent indifference, making the content more aligned with human-centered care. If information communication relies solely on rational appeals, it may overlook the audience’s emotional needs, thereby weakening the dissemination effect. As the philosopher Nietzsche \cite{nietzsche2017birth} stated in The Birth of Tragedy, humans understand the world not only through reason but also by establishing deeper meaning through emotion. Thus, emotion is the key factor in truly touching people’s hearts through information. Research shows that information lacking an emotional dimension (e.g., mere statistics or neutral language) may cause people to lose interest in the story behind the information and even appear mechanical and indifferent \cite{dykas2011attachment, stieglitz2013emotions}. For example, in the dissemination of breaking news or social issues, merely providing data and facts may make it difficult for audiences to resonate with the events. Conversely, emotionalized narratives (e.g., focusing on individual stories or group sentiments) can make information more concrete and emotional, avoiding the appearance of excessive indifference \cite{van2020seeking}. The integration of emotion makes information communication more humane. Scherer \cite{scherer2003vocal} pointed out that emotional information embedded in voice and tone can evoke audience resonance, preventing indifference toward the content. In health communication, negative emotional stories often evoke empathy for individual suffering and inspire deeper reflection \cite{dubey2020psychosocial}. This emotional engagement not only makes information communication a transfer of knowledge but also builds emotional bridges between people, allowing information to transcend indifference and truly touch hearts.
% 情感避免信息的冷漠化:情感在信息传播中的重要性在于它能有效避免信息的冷漠化,使传播内容更加贴近人性化关怀。信息传播若完全依赖理性诉求,可能忽视了受众对情感的需求,从而导致传播效果的削弱。正如哲学家尼采在《悲剧的诞生》中所言,人类不仅通过理性理解世界,还通过情感建立对意义的深层认知。因此,情感是信息真正打动人心的关键因素。研究表明,缺乏情感维度的信息(如单纯的统计数据或中性化的语言)可能让人们对信息背后的故事失去兴趣,甚至显得机械和冷漠(Dykas & Cassidy, 2011,Stieglitz & Dang-Xuan, 2013)。例如,在突发新闻或社会问题的传播中,仅仅提供数据和事实可能让受众难以与事件产生共鸣。相反,通过情感化叙述(如聚焦个体故事或群体感受),信息能够更加具象化和情感化,从而避免传播的内容显得过于冷漠(Van der Meer & Jin, 2020)。情感的融入使信息传播更加富有人性。Scherer(2003)指出,声音和语调中蕴含的情感信息能够激发听众的共鸣,避免受众对信息内容产生冷漠态度。在健康传播中,负面情感故事往往能够让受众对个体痛苦产生共情,并引发更深层次的思考(Dubey et al., 2020)。这种情感参与不仅让信息传播成为知识的传递,更架起了人与人之间的情感桥梁,使信息能够穿越冷漠,真正触及人心。
