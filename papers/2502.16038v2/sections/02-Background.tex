\section{Related Work}
This section will address three key aspects to better understand emotions and their role in information communication. First, it will introduce theories related to emotional models; second, it will examine the theoretical development of emotions in information dissemination; third, it will review interdisciplinary perspectives and methods in emotion studies. 
% 为了更好地理解情感及其在信息传播中的作用,本节将从三个方面进行阐述。首先,介绍情感模型的相关理论;其次,分析情感在信息传播中的理论演变;最后,综述情感研究的跨学科视角与方法。

\subsection{Emotions models} \label{sec:emotion_models}
Emotion plays a vital role in information communication, profoundly affecting the comprehension, reception, and dissemination of information across verbal\cite{bestelmeyer2017effects, gobl2003role, weinstein2018you}, written \cite{mar2011emotion, jaaskelainen2020neural, lekkas2022using, panksepp2012archeology}, and nonverbal channels \cite{jonauskaite2019color, kallabis2024investigating, thumfart2008modeling, wei2006image, mayer2014benefits, ferrara2015measuring, hanson2014happy, xie2017negative, van2015good, pfeuffer122measuring}. Academic research has introduced various emotion models to investigate the dimensional and compositional aspects of emotions, establishing a robust theoretical foundation for emotion studies. Emotion models are generally classified into two main categories: discrete \cite{plutchik1980general, izard1991psychology, ekman1992argument} and continuous models \cite{russell1980circumplex, mehrabian1974approach, osgood1952nature}.
% 情感在信息传达过程中具有关键意义,无论是口头交流、书面表达还是非语言信号的传递,情感的表达都显著影响信息的理解、接受与传播效果。学术研究通过提出多种情感模型,探索了情感的维度性与组合性,为情感研究提供了坚实的理论基础。情感模型主要分为两类:离散情感模型和连续情感模型。


\textbf{Discrete Models} suggest that emotions consist of distinct, independent categories, with variations in the number and characteristics of basic emotions across different theoretical frameworks. Plutchik \cite{plutchik1980general} introduced the Wheel of Emotions model, identifying eight fundamental emotions (e.g., joy, fear, anger, sadness) and highlighting their combinatorial nature and intensity variations. For instance, anxiety arises from a combination of fear and anger, while anger spans a spectrum from mild irritation to intense rage. Later, Izard \cite{izard1991psychology} identified ten fundamental emotions, including interest, shame, and guilt, thereby expanding the range of emotional categories and emphasizing the physiological foundations of emotions and their link to facial expressions. In contrast, Ekman \cite{ekman1992argument} identified six universal emotions—anger, fear, happiness, sadness, disgust, and surprise—that demonstrate substantial cross-cultural consistency and serve as the foundation for subsequent emotion recognition and cross-cultural~research.

% 离散情感模型:离散情感模型认为情感由若干独立的基本类别构成,各模型在基础情感的数量和特性上有所不同。Plutchik(1980)提出情感轮模型,定义了八种基本情感(如喜悦、恐惧、愤怒、悲伤等),并强调情感之间的组合性和强度变化,例如焦虑是恐惧与愤怒的结合,愤怒则从轻微不满到强烈愤怒逐渐递增。随后,Izard(1991)提出十种基本情感,包括兴趣、羞愧和内疚等,进一步丰富了情感类别,并突出了情感的生理基础及其与面部表情的关联。相比之下,Ekman(1992)提出的六种基本情感(愤怒、恐惧、快乐、悲伤、厌恶和惊讶)具有显著的跨文化一致性,并成为后续情感识别与跨文化研究的广泛应用基础。

\textbf{Continuous Models} conceptualize emotions as a dynamic and continuously evolving process. Russell \cite{russell1980circumplex} introduced the circumplex model of affect, characterizing emotional states along two dimensions: valence, which represents the degree of pleasantness, and arousal, which quantifies emotional intensity. For example, excitement is a high-valence, high-arousal emotion, while depression is a low-valence, low-arousal emotion. The PAD model proposed by Mehrabian and Russell \cite{mehrabian1974approach} builds on this by adding dominance, which describes the degree of control emotion exerts over behavior. For instance, anger is associated with high dominance, whereas fear is typically linked to a sense of powerlessness. Osgood’s \cite{osgood1952nature} three-dimensional emotion model is similar to Russell's, describing the dynamic changes in emotions through “evaluation” (corresponding to valence), “activation” (corresponding to arousal), and “potency” (corresponding to dominance).

% 连续情感模型:连续情感模型将情感视为连续变化的过程。Russell(1980)的情感环模型用效价(Valence)和唤起度(Arousal)两个维度描述情感状态:效价反映情感的愉悦程度,唤起度衡量情感强度。例如,兴奋属于高效价高唤起情感,而抑郁则为低效价低唤起情感。Mehrabian和Russell(1974)提出的PAD模型在此基础上增加了支配度(Dominance),用于描述情感对行为的控制程度,如愤怒具有较强支配性,而恐惧通常表现为无力感。Osgood(1952)的三维情感模型与Russell的模型相似,也通过“评价”(与效价对应)、“活跃”(与唤起度对应)和“力量”(与支配度对应)描述情感的动态变化。


Despite variations in classification frameworks and dimensional selection across emotion models, valence, arousal, and dominance remain universally significant in describing emotions. These dimensions, by regulating emotional attributes, significantly influence the understanding, memory, and dissemination of information, making them core elements in the study of information communication.
% 尽管情感模型在分类框架和维度选择上各具特点,效价、唤起度和支配度三维在情感描述中具有普适性和重要性。这些维度通过调节情感属性,显著影响信息的理解、记忆与传播效果,是信息传播研究中的核心要素。

% Emotional factors play a key role in the process of information communication. Whether in verbal communication\cite{bestelmeyer2017effects, gobl2003role, weinstein2018you}, written expression, \cite{mar2011emotion, jaaskelainen2020neural, lekkas2022using, panksepp2012archeology} or the transmission of nonverbal signals \cite{jonauskaite2019color, kallabis2024investigating, thumfart2008modeling, wei2006image, mayer2014benefits, ferrara2015measuring, hanson2014happy, xie2017negative, van2015good, pfeuffer122measuring}, the expression of emotions can significantly affect the understanding, acceptance, and communication of information. Scholars have advanced the field of emotional factors by proposing various emotion models, which are generally categorized into two types: discrete emotion models \cite{plutchik1980general, izard1991psychology, ekman1992argument} and continuous emotion models \cite{russell1980circumplex, mehrabian1974approach, osgood1952nature}.
% %情感因素在信息传达过程中发挥着关键作用,无论是在口头交流、书面表达还是非语言信号的传递中,情感的表达都能显著影响信息的理解、接受与传播效果。学者们通过提出多种情感模型,推动了情感因素领域的发展,尤其在情感的维度性和组合性方面,为相关研究提供了丰富的理论基础。情感模型通常可以分为两类:离散情感模型和连续情感模型。


% Discrete emotion models focus on the basic categories of emotions, positing that emotions consist of several independent, fundamental emotions. Ekman \cite{ekman1992argument} proposed that emotions are composed of a set of biologically universal basic emotions, which exhibit significant cross-cultural and cross-species consistency and are accompanied by distinct physiological and behavioral responses. Ekman \cite{ekman1992argument} believed that the six emotions of anger, fear, joy, sadness, disgust, and surprise are universally shared by humans, and each of these emotions can be expressed through unique signals such as facial expressions and bodily responses. However, as emotion research advanced, scholars challenged the applicability of this model, arguing that these six basic emotions are insufficient to fully capture the complexity of human emotions. To address this, Plutchik \cite{plutchik1980general} proposed the emotion wheel model, which expanded the scope of emotional classification. Researchers argued that emotions are not only discrete but also have relationships in terms of intensity and combinations. He classified emotions into eight basic types: joy, trust, fear, surprise, sadness, disgust, anger, and anticipation, and noted that these basic emotions can combine to form more complex emotions. For example, “anxiety” is a combination of anger and fear, while “excitement” is a blend of joy and anticipation. He also emphasized the variability of emotional intensity, suggesting that each basic emotion has different intensity levels, such as anger, which can range from mild irritation to intense rage. Additionally, Izard \cite{izard1991psychology} proposed a model of ten basic emotions, further enriching the emotional classification system, and emphasized the connection between emotional physiological responses and facial expressions, providing theoretical support for facial expression recognition technologies in emotion analysis.

% %离散情感模型侧重于情感的基本类别,认为情感是由若干个独立的基本情感组成的。 Ekman(1992)提出,情感是由一组生物学上普遍存在的基础情感构成,这些情感具有显著的跨文化和跨物种的一致性,并伴随独特的生理和行为反应。Ekman 认为,愤怒、恐惧、快乐、悲伤、厌恶和惊讶六种情感是人类共有的基本情感,每一种情感都可以通过面部表情、身体反应等独特信号进行表现。然而,随着情感研究的深入,学者们对这一模型的适用性提出了挑战,认为仅凭这六种基本情感无法全面捕捉人类情感的复杂性。为此,Plutchik(1980)提出了情感轮模型,扩展了情感分类的范围。研究者认为,情感不仅是离散的,而且情感之间存在强度和组合的关系。他将情感划分为八种基本情感:喜悦、信任、恐惧、惊讶、悲伤、厌恶、愤怒和期待,并指出这些基本情感能够通过组合形成更为复杂的情感。例如,“焦虑”是愤怒和恐惧的组合,“兴奋”是喜悦和期待的结合。他还强调了情感强度的变化,认为每种基本情感都有不同的强度表现,如愤怒从轻微的不满到激烈的愤怒都有不同的表现。此外,Izard(1991)提出了十种基本情感的模型,进一步丰富了情感的分类体系,并强调情感的生理反应与面部表情的联系,为情感分析中的表情识别技术提供了理论支持。

% Unlike discrete emotion models, continuous emotion models view emotions as a continuous process of change rather than isolated categories. One typical example of a continuous emotion model is Russell’s \cite{russell1980circumplex} circumplex model of affect. This model describes emotions through two main dimensions: valence and arousal. Valence reflects the positivity or negativity of an emotion, indicating whether the emotion is pleasant or unpleasant, while arousal measures the level of activation, indicating the intensity of the emotion. For example, emotions like joy and excitement may both have high arousal, whereas depression and fatigue are characterized by low arousal. Through these two dimensions, emotional states are presented on a two-dimensional plane, clearly depicting the continuous variations of emotions. However, this model still has certain limitations in capturing the complexity of emotions, particularly in terms of the blending and diversity of emotions. Building on this, Mehrabian \& Russell \cite{mehrabian1974approach} proposed the PAD model (pleasure, arousal, dominance). In this model, “Pleasure” has a similar definition to valence, both used to measure the positivity or negativity of an emotion. Building on valence and arousal, the PAD model introduces a third dimension—dominance, which reflects an individual’s dominance or dominance in an emotional state, indicating the degree of emotional control over behavior. For example, both anger and fear may have low dominance, but anger is typically associated with a stronger desire for control, while fear is characterized by a sense of helplessness. The PAD model provides an important theoretical foundation for the quantification and multidimensional modeling of emotions in emotion analysis. Additionally, Osgood’s \cite{osgood1952nature} three-dimensional emotion model shares many similarities with Russell’s \cite{russell1980circumplex} circumplex model of affect. Osgood’s model emphasizes three dimensions of emotion: ``evaluation,” ``activity,” and ``power.” These dimensions align with valence, arousal, and dominance in expressing the multidimensionality of emotions and the dynamic changes in emotional states. The “evaluation” dimension is similar to valence, representing the positivity or negativity of an emotion; the “activity” dimension corresponds to arousal, reflecting the intensity or activation level of the emotion; while the “power” dimension is analogous to dominance, describing the influence of emotion on an individual’s control over behavior.

% % 与离散情感模型不同,连续情感模型认为情感是一个连续变化的过程,而非孤立的类别。Russell(1980)提出的情感环模型就是典型的连续情感模型之一。该模型通过两个主要维度来描述情感:效价(Valence)和唤起度(Arousal)。效价反映情感的正负性,表示情感是愉快还是不愉快;而唤起度则衡量情感的激活程度,表示情感的强度。例如,愉快和兴奋的情绪可能都具有较高的唤起度,而抑郁和疲倦则表现为较低的唤起度。通过这两个维度,情感状态被呈现在一个二维平面上,能够清晰地表达情感的连续变化。然而,这一模型在捕捉情感的复杂性,尤其是情感的混合和多样性方面,仍存在一定的局限。在此基础上,Mehrabian和Russell(1974)提出了PAD模型(Pleasure, Arousal, Dominance)。在这个模型中,“Pleasure”与效价(Valence)具有相似的定义,均用于衡量情感的积极性和消极性。而在效价和唤起度的基础上,PAD模型增加了第三个维度——支配度(Dominance),该维度反映了个体在情感状态下的控制感或支配感,反映情感对行为的控制程度。例如,愤怒和恐惧都可能具有低支配度,但愤怒通常伴随更强的支配需求,而恐惧则表现为一种无力感。PAD模型为情感分析中的情绪量化和多维度建模提供了重要的理论基础。此外,Osgood(1952)提出的三维情感模型也与Russell(1980)的情感环模型有许多相似之处。Osgood的模型强调情感的“评价”(Evaluation)、“活跃”(Activity)和“力量”(Power)三个维度。这些维度与效价、唤起度和支配度在表达情感的多维性和情感状态的动态变化上具有相通之处。“评价”维度与效价类似,表示情感的正负性;“活跃”维度对应唤起度,反映情感的强度或激活水平;而“力量”维度则类似于支配度,描述情感对个体行为控制的影响。


% Although the classification frameworks of emotional models are diverse, encompassing various dimensions and complex emotional combinations, the three dimensions of valence, arousal, and dominance play a crucial role in the description and understanding of emotions. These dimensions play a key role in information communication, as they influence the comprehension, memory, and sharing of information by modulating the attributes of emotions.
% % 尽管情感模型的分类框架多样,涵盖了多种维度和复杂的情感组合,但效价、唤起度和支配度这三个维度在情感的描述和理解中发挥着重要作用。这些维度在信息传播中扮演着关键角色,它们通过调节情感的属性,影响信息的理解、记忆以及分享。


\subsection{Theoretical Evolution of Emotions in information~communication}
Emotions play a central role in information communication, a key focus of multidisciplinary research. Since the early days of psychological research, emotions have been regarded as critical variables influencing cognitive processing, memory, and behavior. Ebbinghaus’s \cite{ebbinghaus1913memory} memory experiments revealed that emotionally charged content is more memorable than neutral content, laying the foundation for studying emotions in information communication. As research progressed, emotions were found not only to regulate individual cognitive behavior but also to amplify the scope and speed of information communication through social interactions, becoming a key variable in understanding communication~patterns.
% 情感在信息传播中的作用一直是多学科研究的核心议题。自心理学早期研究起,情感被认为是影响认知加工、记忆和行为的重要变量。Ebbinghaus(1913)的记忆实验首次揭示,情感化内容相比中性内容更容易被记住,这一发现奠定了情感在信息传播研究中的基础地位。随着研究的深入,情感逐渐被发现不仅能够调节个体的认知行为,还通过社会交互放大信息传播的范围和速度,成为理解传播模式的关键变量。

The core mechanisms through which emotions influence information communication include attracting attention, modulating information processing, and enhancing memory encoding. Lazarus\cite{lazarus1991emotion} cognitive theory of emotion revealed that emotions shape the fundamental logic of communication by selectively regulating information processing. Further research demonstrated that information with high emotional valence and arousal significantly enhances attention and memory, directly promoting its communication. Berger \& Milkman’s\cite{berger2012makes} validated this phenomenon experimentally, finding that strongly emotion-driven information, such as anger or surprise, spreads particularly effectively on social networks.
% 情感影响信息传播的核心机制包括对注意力的吸引、信息加工的调节以及记忆编码的增强。Lazarus(1991)提出的情绪认知理论揭示,情感通过对信息加工的选择性调节作用,塑造了传播的基本逻辑。研究进一步表明,高情感效价和高唤起度的信息在吸引注意力和加深记忆方面具有显著优势,这种优势直接促进了信息的传播。Berger和Milkman(2012)通过实验验证了这一现象,发现愤怒、惊讶等强情绪驱动的信息在社交网络中扩散效果尤为显著。

In the era of social media, the role of emotions in driving information communication has been magnified. Research by Vosoughi et al. \cite{vosoughi2018spread} showed that emotionally charged information spreads wider and deeper than neutral content. Specifically, anger or fear-driven content rapidly spreads on social platforms, not only increasing communication efficiency but also exacerbating the spread of fake news and emotional polarization. This dual effect further highlights the importance and complexity of emotional~communication.
% 进入社交媒体时代,情感对信息传播的推动作用被放大。Vosoughi等(2018)的研究表明,带有强烈情感特征的信息传播范围和深度显著高于中性信息。特别是愤怒或恐惧驱动的内容在社交平台上的快速扩散,不仅提高了传播效率,还引发了假新闻扩散和情绪极化等负面效应。这种双重影响进一步凸显了情感传播的重要性与复杂性。

Recent advances in neuroscience have provided new perspectives on understanding the influence of emotions on information communication. Izard’s\cite{izard1993four} proposed that emotions exert their influence through the coordination of the nervous and sensory systems. For example, the activity of the amygdala and prefrontal cortex enhances the attention-capturing effects of emotional content, while sensory stimuli intensify the emotional experience. Additionally, Fredrickson’s \cite{fredrickson2001role} “Broaden-and-Build” theory revealed that positive emotions broaden cognitive perspectives and facilitate the integration of complex information, whereas negative emotions tend to enhance attention to detail. These studies provide a theoretical basis for understanding the deeper impacts of emotions on information communication.
% 近年来,神经科学的进展为理解情感对信息传播的影响提供了全新的视角。Izard(1993)提出,情感通过神经系统和感官系统的协作发挥作用。例如,杏仁核和前额叶皮层的活动增强了情感化信息对注意力的吸引力,而感官刺激强化了这种情感体验。此外,Fredrickson(2001)的“扩展与建构”理论揭示,积极情绪能够扩大认知视野,促进复杂信息的整合,而消极情绪则倾向于增强对细节的关注。这些研究为理解情感在信息传播中的深层次影响提供了理论依据。


\subsection{Interdisciplinary Perspectives and Approaches in Emotion~Research}% 情绪研究的跨学科视角与方法
The complexity of emotions has driven research beyond a single discipline, forming an interdisciplinary research framework. Psychology has laid the theoretical foundation for emotion research, while sociology and communication studies have uncovered the impact of emotions on group behavior, with computer science and design providing technical support and practical applications. The interdisciplinary perspective has not only enriched research methods on emotions but also promoted their application in information communication.
% 情感的复杂性促使其研究逐渐超越单一学科,形成了多领域交叉的研究框架。心理学为情感研究奠定了理论基础,社会学与传播学揭示了情感在群体行为中的影响力,计算机科学和设计学则提供了技术支持和实践应用。跨学科的视角不仅丰富了情感的研究方法,也推动了其在信息传播中的应用。

Psychology primarily explores how emotions regulate cognition and behavior.  Ekman’s \cite{ekman1992argument} proposed the basic emotion theory, which suggests that emotions such as anger, joy, and fear have cross-cultural consistency and are expressed through facial expressions and vocal tones. Lazarus’s \cite{lazarus1991emotion} cognitive theory of emotion further reveals how emotions influence attention allocation and information memory through appraisal mechanisms. These theories provide a core framework for understanding the role of emotions in communication.
% 心理学主要探讨情感对认知和行为的调节作用。Ekman(1992)提出的基本情绪理论指出,愤怒、喜悦、恐惧等基本情绪具有跨文化一致性,并通过面部表情和语音表现。Lazarus(1991)的情绪认知理论进一步揭示了情感如何通过评估机制影响注意力分配和信息记忆。这些理论为理解情感在传播中的作用提供了核心框架。

% Advancements in neuroscience have deepened our understanding of the mechanisms of emotion activation. Using brain imaging techniques such as functional magnetic resonance imaging (fMRI) and electroencephalography (EEG), researchers have unveiled the neural mechanisms underlying emotion activation. For example, the amygdala has been identified as a critical region for processing intense emotions, while the prefrontal cortex is closely associated with emotion regulation and decision-making behaviors \cite{ledoux2000emotion}. These findings provide physiological support for understanding the mechanisms by which emotions influence information communication. For instance, sensory stimuli such as color and sound can rapidly activate the nervous system, eliciting users’ attention and emotional responses. These studies offer direct neuroscientific evidence for designing emotionally engaging informational content.
% % 神经科学的发展深化了对情感激活机制的理解。通过脑成像技术,如功能磁共振成像(fMRI)和脑电图(EEG),研究者揭示了情感激活的神经机制。例如,杏仁核被发现是处理强烈情绪的关键区域,前额叶皮层则与情感调控和决策行为密切相关[1]。这些研究结果为情感对信息传播的机制提供了生理学支持,例如,通过颜色和声音等感官刺激,可以快速激活神经系统,引发用户的注意力和情感反应。这些研究为设计具有情感吸引力的信息内容提供了直接的神经科学依据。

Sociology and communication studies focus on the role of emotions in group behavior and social interaction. Emotions are regarded as one of the driving factors that regulate group behavior. For example, Collins \cite{collins2004interaction} proposed the interaction ritual chains theory, which suggests that emotions can amplify the impact of information through social interaction, shaping collective memory. In communication studies, emotions are used to explain the diffusion patterns of information on social media, such as how highly emotionally arousing content spreads within groups and triggers behavioral resonance\cite{berger2011arousal}.
% 社会学和传播学关注情感在群体行为和社会互动中的作用。情感被视为调节群体行为的驱动因素之一。例如,Collins(2004)提出的互动仪式链理论指出,情感可以通过社会互动放大信息的影响力,塑造集体记忆。在传播学中,情感被用来解释信息在社交媒体上的扩散模式,如高情绪唤起的信息如何在群体间传播并引发行为共鸣。

The advancement of computer science has significantly contributed to the technical foundations of emotion research. Picard\cite{picard2000affective} first proposed the concept of affective computing, aiming to recognize and simulate human emotions through algorithms. In recent years, advances in deep learning and natural language processing have enabled the application of emotion analysis in text, audio, and images. For example, analyzing emotional vocabulary in social media text predicts the scope and impact of information communication~\cite{mohammad2013crowdsourcing}.
% 在计算机科学领域,情感计算(Affective Computing)为情感的测量和建模提供了技术支持。Picard(2000)首次提出情感计算的概念,致力于通过算法识别和模拟人类情感。近年来,深度学习和自然语言处理技术的快速发展使得情感分析在文本、音频和图像中的应用成为可能。例如,通过分析社交媒体文本中的情感词汇,可以预测信息传播的范围和效果。这种技术不仅提高了情感研究的精确性,也为信息传播的优化提供了工具支持。

Design studies place emotions at the core of user experience design. Norman \cite{norman2007emotional} proposed in emotional design that design should evoke users’ emotional resonance rather than merely meet functional requirements. For instance, the use of narrative text\cite{mar2011emotion}, visual symbols \cite{mayer2014benefits}, and immersive virtual reality (VR) technologies \cite{slater2016enhancing} enhance the appeal and memorability of information communication through emotional design.
% 设计学将情感作为用户体验设计的核心。Norman(2004)在《情感化设计》中提出,设计需触发用户的情感共鸣,而不仅仅满足功能需求。叙事性文本、视觉符号及沉浸式虚拟现实技术(VR)等手段通过情感化设计增强了信息传播的吸引力和记忆力.

Despite expanding the scope of emotion research, interdisciplinary approaches have introduced challenges. On one hand, differences in the definition and measurement of emotions across disciplines hinder research integration; on the other hand, further exploration is needed to apply complex emotion models to various information communication scenarios.
% 尽管跨学科研究拓展了情感的研究视角,但也带来了挑战。一方面,不同学科在情感定义与测量上的差异阻碍了研究整合;另一方面,如何将复杂情感模型应用于不同的信息传播场景,仍然需要进一步探索。


% \subsection{The Current State and Challenges of Emotion-Driven information communication Research} % 情绪驱动信息传播研究现状与挑战
% In recent years, emotion-driven in  formation communication research has made significant progress, particularly in the fields of advertising, education, health communication, and political communication. Researchers have found that emotions not only affect the speed of information communication but also determine its lasting impact. For example, Schreiner et al. \cite{schreiner2021impact} pointed out that integrating emotional design strategies, such as narrative text and dynamic visuals, can significantly enhance information retention and communication effectiveness. Additionally, Dabbous \& Aoun Barakat \cite{dabbous2023influence} analyzed the role of emotions in driving information-sharing behavior on social media, further validating the potential of emotional design in digital communication.
% % 近年来,情感驱动的信息传播研究取得了显著进展,尤其是在广告、教育、健康传播和政治传播领域。研究者发现,情感不仅影响信息的传播速度,还决定了信息的持久影响力。例如,Schreiner 等(2021)指出,通过整合情感化设计策略(如叙事文本与动态视觉),可以显著提升信息记忆和传播效果。此外,Dabbous 和 Aoun Barakat(2023)分析了情感对社交媒体信息分享行为的驱动作用,进一步验证了情感化设计在数字传播中的潜力。

% However, emotion-driven information communication research also faces several challenges. First, existing research lacks a comprehensive review of the underlying mechanisms through which emotions influence information communication. In particular, further exploration is needed on how emotions are collaboratively activated across different emotional systems through multimodal design \cite{izard1993four}. Second, the specific effects of different emotion types and communication contexts on information propagation vary significantly. For example, the moderating role of cultural context on the acceptance of emotional valence and communication outcomes has not been thoroughly examined \cite{megalakaki2019effects}. Finally, the integrated application of multimodal emotional design remains limited in practice, particularly in the integration of affective computing technologies with traditional design~methods.
% % 然而,情感驱动的信息传播研究也面临若干挑战。首先,现有研究对情感影响信息传播的内在机制仍缺乏全面梳理。尤其是情感如何通过多模态设计协同激活不同情绪系统的研究仍待深入。其次,不同情感类型和传播背景对信息传播效果的具体影响存在显著差异。例如,文化背景对情感效价的接受度和传播效果的调节作用尚未被充分研究。最后,多模态情感化设计的整合应用在实践中仍存在一定局限性,尤其是在情感计算技术与传统设计方法的结合方面。

% These research developments and challenges indicate that emotion-driven information communication research requires not only further integration of interdisciplinary theories but also the combination of technology and practice to explore systematic design frameworks and application pathways.
% % 这些研究现状与挑战表明,情感驱动的信息传播研究不仅需要进一步整合多学科理论,还需要结合技术与实践,探索系统性的设计框架与应用路径。

