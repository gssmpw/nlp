\section{What emotions factors effects Information Communication?} \label{sec:what} %哪些情绪因素影响信息传播?
% 细分什么情感元素会影响Information Communication。然后把效价和唤起度拆分开。How写具体的设计模式。

% 1. 调研了哪些心理学的模型,再写这些模型把情绪划分成了唤起、效价、支配感。离散+连续。
% 2. 对于信息传达普遍是有较大影响的,影响情感的因素具体有哪些(唤起、效价、支配感),这3个是构成影响交流的最主要的因素,具体的定义(效价在哪里定义的,和其他模型相类似。)具体如何影响的。

The three key emotional factors—valence, arousal, and dominance \cite{mehrabian1974approach, osgood1952nature}—are crucial emotional elements that influence information communication. In \cref{sec:emotion_models}, we have already explored the theoretical foundations of these three emotional dimensions. This chapter will provide an in-depth analysis of how these three dimensions specifically affect the omprehension\cite{egidi2012emotional, megalakaki2019effects, mather2011arousal, tyng2017influences}, memory \cite{pawlowska2011influence, madan2019positive,megalakaki2019effects}, and sharing \cite{ferrara2015measuring, stieglitz2013emotions, vosoughi2018spread} of information, further revealing their key role in information dissemination.

% Emotions have a multi-dimensional impact on information communication, influencing comprehension, memory, and sharing.  This not only highlights the complexity of emotions but also emphasizes their central role in the communication process.  This section integrates existing research to explore the specific roles and manifestations of the three dimensions—valence, arousal, and dominance—within information comprehension, memory, and sharing.  It aims to provide a foundation for a deeper understanding of the pivotal role of emotions in information communication.
% % 情绪在信息传达中发挥着多层次的作用,其对理解、记忆与分享的影响不仅体现了情绪的复杂性,也反映了其在信息传播中的关键地位。本章首先综述了情绪的不同维度模型,并在此基础上梳理出效价、唤起度和支配度三大核心维度,作为研究情绪作用的理论框架。随后,结合已有研究,深入探讨这三个维度在信息理解、记忆和分享过程中的具体表现与作用,为全面理解情绪在信息传播中的关键角色奠定基础。

% 这里需要插入一个不同模型的图片
% first reviews various dimensional models of emotions and, based on this, outlines three core dimensions—valence, arousal, and dominance—as a theoretical framework for studying the role of emotions. Subsequently, 


\subsection{The Impact of Emotions on Comprehension}
Emotions play an important regulatory role in information comprehension. Different emotional characteristics influence attention allocation and cognitive resource utilization, shaping the depth and accuracy of an individual’s understanding. Valence regulates attention scope and cognitive resource allocation, affecting the depth of understanding \cite{egidi2012emotional, megalakaki2019effects}. Arousal determines information priority and resource concentration \cite{mather2011arousal, tyng2017influences}. Dominance influences active information processing and integration \cite{mehrabian1996pleasure, gross1998emerging}. These emotional factors underscore the pivotal role of emotions in understanding complex information by regulating processing patterns. They provide critical insights into emotional mechanisms underlying information~communication. 
%情绪在信息理解中发挥着重要调节作用,不同的情绪特征通过影响注意力分配和认知资源利用,塑造了个体的理解深度与准确性。效价通过调节注意力范围和认知资源分配影响理解的深度(Egidi & Nusbaum, 2012; Megalakaki et al., 2019),唤起度决定了信息优先级和资源集中程度(Mather & Sutherland, 2011; Tyng et al., 2017),而支配度则塑造了对信息的主动处理和整合方式(Mehrabian, 1996; Gross, 1998)。这些情绪因素通过调节信息处理模式,揭示了情绪在理解复杂信息中的核心作用,为探索信息传播中的情绪机制提供了重要视角。


\subsubsection{Emotional Valence}
Positive emotions can significantly enhance the understanding of information, but they may also increase cognitive load for individuals in certain situations. Firstly, positive emotions aid individuals in understanding complex information more deeply by enhancing motivation and attention. For instance, Egidi \& Nusbaum \cite{egidi2012emotional} found that individuals in a positive emotional state exhibit greater depth of understanding when processing emotional language information, as positive emotions enhance focus and motivation. Additionally, Megalakaki et al. \cite{megalakaki2019effects} and Zhang \cite{zhang2023effect} noted that positive emotions can expand cognitive resources, enabling individuals to analyze complex texts in greater detail, thereby enhancing comprehension outcomes. Tyng et al. \cite{tyng2017influences} further indicated that this emotional state broadens the range of attention, facilitating the processing of multiple complex pieces of information and thereby improving understanding capabilities. However, positive emotions may affect the accuracy of understanding due to increased cognitive load. Research by Jiménez-Ortega et al. \cite{jimenez2012emotional} showed that, although cognitive processing is deeper in a positive emotional state, individuals exhibit longer reaction times and higher error rates in semantic processing tasks. This may be because positive emotions lead individuals to focus on both core information and secondary details, thereby distracting attention and increasing cognitive~load.
%积极情绪能显著促进信息的理解,同时也可能在某些情况下增加个体的认知负担。首先,积极情绪通过增强动机和注意力,帮助个体更深入地理解复杂信息。例如,Egidi和Nusbaum(2012)发现,积极情绪的个体在处理情绪化语言信息时展现出更高的理解深度,这是因为积极情绪提高了专注度和动机。此外,Megalakaki et al.(2019)以及Zhang(2023)指出,积极情绪能够扩展认知资源,使个体能更细致地分析复杂文本,从而提高理解效果。Tyng等(2017)进一步指出,这种情绪状态扩大了注意力范围,有利于处理多重复杂信息,从而提升理解能力。然而,积极情绪可能因增加认知负担而影响理解的准确性。Jiménez-Ortega等(2012)的研究显示,在积极情绪状态下,尽管认知加工更深入,但在语义处理任务中个体的反应时间更长,错误率也更高。这可能是因为积极情绪导致个体同时关注核心信息和次要细节,从而分散了注意力,增加了认知负荷。

In contrast to positive emotions, negative emotions have a complex and multidimensional impact on information comprehension. Research indicates that negative emotions enhance attention to detail, potentially improving reasoning abilities, but they may also weaken overall comprehension \cite{egidi2012emotional, lang2007cognition}. For example, Lang et al. \cite{lang2007cognition} found that viewers in a negative emotional state paid more attention to negative details in news reports, neglecting the overall content. A study by Arfé et al. \cite{arfe2023effects} also showed that the negative emotion group had significantly longer initial fixation times during text reading compared to the neutral emotion group, indicating that this emotional state triggered deeper information processing. Additionally, negative emotions prompt individuals to engage in deeper reasoning and analysis, which is very beneficial for understanding complex information \cite{megalakaki2019effects}. However, negative emotions may also lead individuals to interpret information from a biased perspective, thereby reducing objectivity and rational judgment abilities, which affects the accurate understanding of information \cite{tyng2017influences}.
%相对于积极情绪,消极情绪对信息理解的影响复杂且多维。研究表明,消极情绪增强了对细节信息的关注,可能提升推理能力,但同时也可能减弱整体理解能力。例如,Lang等(2007)发现,消极情绪状态下的观众更多关注新闻中的负面细节,忽视整体内容。Arfé等(2023)的研究也显示,消极情绪组在文本阅读时的初次注视时间显著长于中性情绪组,表明这种情绪状态引发了更深入的信息处理。此外,消极情绪还促使个体进行更深入的推理和分析,这对理解复杂信息非常有利。然而,消极情绪也可能导致个体以偏见性视角解读信息,从而降低客观性和理性评判能力,影响信息的正确理解。

Finally, the impact of neutral emotions on information comprehension is relatively balanced and not as pronounced as that of positive or negative emotions. In a neutral emotional state, the lack of emotional bias leads to more stable information processing and a more balanced allocation of resources. For example, studies by Earles et al. \cite{earles2016memory} and Megalakaki et al. \cite{megalakaki2019effects} indicate that neutral emotions are conducive to maintaining objectivity and consistency in information processing. This emotional state does not significantly enhance or weaken comprehension abilities but has a particular advantage in tasks requiring high objectivity and accuracy. However, the lack of emotional cues in a neutral emotional state may hinder deep processing of information and long-term memory. Overall, neutral emotions exhibit a neutral role in information comprehension and are suitable for performing tasks that require high objectivity and stability.
%最后,中性情绪对信息理解的影响相对平衡,不如积极或消极情绪显著。在中性情绪下,由于缺乏情感偏向,信息处理更为稳定,资源分配更加均衡。例如,Earles等(2016)和Megalakaki等(2019)的研究表明,中性情绪有利于保持信息处理的客观性与一致性。这种情绪状态不会显著提升或削弱理解能力,但在需要高度客观性和准确性的任务中具有特别的优势。然而,由于缺乏情感线索,中性情绪可能不利于信息的深度加工和长期记忆。总的来说,中性情绪在信息理解中表现中立,适合执行需要高度客观和稳定的任务。

\subsubsection{Emotional Arousal}
The impact of emotional arousal on information comprehension depends on the intensity of arousal. High-arousal emotions, such as fear or excitement, can significantly alter cognitive states, enhancing attention concentration in a short period and prompting prioritized processing of key information \cite{marchewka2016arousal}. This concentrated allocation of cognitive resources aids quick responses in complex situations, but it may also lead to a ``tunnel effect," where individuals are prone to neglect non-core information and details. For example, individuals experiencing intense threatening emotions focus more on the source of the threat and coping strategies, while memories of other details may become vague~\cite{mather2011arousal}.
%情绪唤起度对信息理解的影响取决于唤起强度。高唤起度情绪,如恐惧或兴奋,能显著改变认知状态,短时间内提高注意力集中程度,促使优先处理关键信息。这种认知资源的集中分配有助于在复杂情境中快速反应,但同时也可能导致“隧道效应”,即个体容易忽略非核心信息和细节。例如,强烈威胁性情绪下的个体更专注于威胁来源和应对方式,而其他细节记忆则可能模糊。
Additionally, high-arousal emotional states may lead to increased cognitive load, thereby reducing comprehension accuracy \cite{jimenez2012emotional}. High-arousal emotions are often accompanied by tension or anxiety, which interfere with comprehensive and rational information analysis, increasing the likelihood of cognitive biases. In this state, individuals are more inclined to process information emotionally, a tendency that becomes more pronounced when dealing with negative emotions \cite{pessoa2009emotion}.
%此外,高唤起情绪状态还可能导致认知负荷增加,进而影响理解的精确性。这是因为高唤起情绪通常伴随紧张或焦虑,这些情绪干扰了全面和理性的信息分析,易于形成认知偏差。在这种状态下,个体还可能倾向于情绪化处理信息,尤其在处理负性情绪时,这一倾向更为明显。

In contrast, low-arousal emotions, such as mild pleasure or worry, have a more balanced impact on information comprehension. Under low-arousal emotional states, the allocation of cognitive resources is more balanced, making it easier to integrate multiple sources of information for comprehensive understanding. In this emotional state, individuals are less likely to be disturbed by emotions, allowing for effective integration of complex texts or multi-source information, thereby enhancing the accuracy and comprehensiveness of overall comprehension. Under low to moderate arousal emotions, individuals' cognitive flexibility is enhanced, which not only aids in processing complex information but also improves creative thinking, resulting in better performance across various comprehension tasks \cite{tyng2017influences}. 
%相较之下,低唤起度情绪,如轻微的愉悦或忧虑,对信息理解具有更平衡的影响。在低唤起度情绪状态下,个体的认知资源分配更均衡,更容易整合多个信息来源并进行全面理解。在这种情绪状态下,个体不容易受到情绪干扰,能够有效地整合复杂文本或多源信息,从而提升整体理解的准确性和全面性。低至中等唤起度情绪下,个体的认知灵活性得到增强,这不仅有助于处理复杂信息,还能提高创造性思维,使个体在多种理解任务中表现更佳。

In summary, emotional arousal influences information comprehension in different ways. High-arousal emotions facilitate the rapid understanding and processing of critical information in emergency situations but may limit comprehensive integration of complex information; whereas low-arousal emotions are more suitable for tasks requiring balanced and in-depth analysis, aiding in comprehensive and objective understanding of information.
%总结而言,情绪唤起度通过不同方式影响信息理解。高唤起度情绪有助于在紧急情况下快速理解和处理关键信息,但可能限制对复杂信息的全面性整合;而低唤起度情绪更适合于需要平衡与深入分析的任务,有助于全面且客观的信息理解。

% \subsubsection{The Neural Mechanisms of Emotions in Comprehension}
% Emotional activation during the comprehension process involves multiple key brain areas, which interact to influence the processing and understanding of information. Particularly, the amygdala plays a central role in processing emotional information \cite{ledoux2000emotion}, affecting cognitive responses and memory encoding through its coordination with areas like the prefrontal cortex and hippocampus \cite{phelps2004human}.
% % 情绪在理解过程中的神经激活涉及多个关键脑区,这些脑区通过相互作用共同影响信息的处理和理解。特别是杏仁核在处理情绪信息时起到核心作用,通过与前额叶皮层和海马体等区域的协调作用,影响认知反应和记忆编码。

% The valence of emotions, such as positive emotions (like happiness or satisfaction) and negative emotions (like fear or anger), differently affects information processing. During positive emotions, activity in the ventromedial prefrontal cortex (VMPFC) and the striatum, which are related to the reward and motivation systems \cite{haber2010reward}, helps individuals to make comprehensive assessments and judgments based on emotional responses \cite{bechara2000emotion}. In contrast, negative emotions make individuals focus more on potential threats, enhancing the depth of processing threat-related information but potentially increasing cognitive load and weakening the inhibitory function of the prefrontal cortex (especially the DLPFC), which interferes with rational thinking \cite{pessoa2009emotion, ochsner2005cognitive}.
% % 情绪的效价,如积极情绪(如快乐或满足)和消极情绪(如恐惧或愤怒),对信息处理的影响方式有所不同。积极情绪时,腹内侧前额叶皮层(VMPFC)和纹状体的活跃与奖励和动机系统相关,有助于促使个体基于情绪反应进行全面的评估和判断。相比之下,消极情绪使个体更专注于潜在的威胁,虽然能够增强对威胁信息的处理深度,但也可能增加认知负担,削弱前额叶皮层(尤其是DLPFC)的抑制功能,干扰理性思考。

% Subsequently, the arousal level of emotions also significantly impacts information processing. High arousal emotions, whether positive or negative, trigger physiological responses such as increased heart rate by activating the amygdala, enhancing alertness to information \cite{lang2010emotion}. While this response enhances the ability to process details, excessively high arousal may weaken cognitive control by the DLPFC, impacting rational analysis \cite{van2012reward}. To counteract this disruption, the anterior cingulate cortex (ACC) and the insula are activated to regulate emotional and cognitive conflicts, restoring cognitive balance \cite{etkin2011emotional}. The ACC helps refocus attention, reducing the negative impacts of emotions on cognition, while the insula integrates emotional and cognitive networks, adjusting information processing strategies \cite{craig2009you, singer2009common}. 
% %随后,情绪的唤起度也对信息处理产生重要影响。高唤起情绪,无论正性还是负性,都会通过激活杏仁核引发生理反应,如心跳加速,增强信息的警觉性。这种反应虽增强了信息处理的细节能力,但过高的唤起度可能削弱DLPFC的认知控制,影响理性分析。为应对这种干扰,前扣带回(ACC)和岛叶(Insula)被激活,调节情绪与认知冲突,恢复认知平衡。ACC帮助重新集中注意,减轻情绪对认知的负面影响,而岛叶整合情绪与认知网络,调整信息处理策略。

% Conversely, low-arousal emotions (such as calmness or mild satisfaction) facilitate focus and comprehensive integration of information through the regulation of the prefrontal cortex (PFC) and cingulate cortex. Due to lower arousal, individuals can process information stably and comprehensively in a calm state \cite{damasio1994descartes}, although this may reduce the ability to prioritize key information \cite{pessoa2008relationship}.
% %相反,低唤起度的情绪(如平静或轻度满足)通过前额叶皮层(PFC)和扣带回的调节,帮助个体保持专注,并促进信息的全面整合。由于唤起度较低,个体能在平静状态下进行稳定而全面的信息处理,尽管这可能会降低对关键信息的优先处理能力。

% In summary, the valence and arousal levels of emotions specifically affect the comprehension process by activating brain regions such as the amygdala, prefrontal cortex, cingulate cortex, and insula. Different combinations of emotional valence and arousal evoke varied responses in these neural networks, thereby facilitating or inhibiting the comprehension process. A deeper understanding of these mechanisms helps reveal the multifaceted roles of emotions in cognitive processes and their complex effects on information processing.
% %综上所述,情绪的效价和唤起度通过激活杏仁核、前额叶皮层、扣带回和岛叶等脑区,对理解过程产生具体的影响。不同情绪效价和唤起度的组合在这些神经网络中引发不同反应,从而促进或抑制理解过程。深入理解这些机制,有助于揭示情绪在认知过程中的多方面作用及其对信息处理的复杂影响。

% \subsubsection{Summary}
% Emotions have multiple and complex effects on information comprehension. Positive emotions can enhance attention and motivation, expanding an individual's cognitive resources, thus facilitating the integration of information and comprehension of complex texts; however, in some cases, the expanded scope of attention may lead to an increased cognitive load, thus affecting the accuracy of comprehension. Negative emotions tend to enhance attention to details and prompt deeper reasoning analysis, but the accompanying cognitive load may reduce overall comprehension efficiency. Although both positive and negative emotions can increase cognitive load, their specific impact mechanisms differ. The burden of positive emotions often stems from the dispersion of cognitive resources, leading to insufficient attention to key information, while the burden of negative emotions arises more from the intensity of the emotion and excessive focus on details, hindering information integration. In contrast, neutral emotions maintain a balance in information processing; although lacking intense emotional arousal, they help maintain cognitive objectivity and consistency, suitable for tasks requiring high objectivity and precision.
% %情绪对信息理解的影响具有多重和复杂的作用。积极情绪既可以通过增强注意力和动机,扩展个体的认知资源,从而促进信息整合和复杂文本的理解;但在某些情况下,可能由于扩展的注意力范围导致认知负荷增加,从而影响理解的准确性。消极情绪则倾向于增强对细节的关注,促使更深入的推理分析,但伴随的认知负担可能降低整体理解效率。虽然积极情绪和消极情绪都可能增加认知负担,但其具体影响机制不同。积极情绪的负担往往来源于认知资源的分散,导致对关键信息的注意不足,而消极情绪的负担更多来源于情绪的强烈性和对细节的过度关注,导致信息整合受阻。相比之下,中性情绪在信息处理中保持平衡,虽然缺乏强烈情感激发,但有助于维持认知的客观性和一致性,适用于需要高度客观性和精确性的任务。

% The neural mechanisms of emotion in information comprehension involve the collaboration of brain areas such as the amygdala, prefrontal cortex, cingulate cortex, and insula. Different combinations of emotional states and arousal levels affect the activation of these areas, thereby promoting or inhibiting the comprehension process. Additionally, the consistency between emotional states and the content of information significantly affects the smoothness and efficiency of information processing.
% %情绪对信息理解的神经机制涉及杏仁核、前额叶皮层、扣带回和岛叶等脑区的协作,不同情绪状态和唤起度的组合会影响这些区域的激活,从而对理解过程产生促进或抑制作用。此外,情绪状态与信息内容的一致性也显著影响信息处理的流畅性和效率。

\subsubsection{Emotional Dominance}
Emotional dominance refers to the dominance and dominance that an individual perceives in a given situation, and its level has a profound impact on information comprehension and cognitive processing. High dominance emotions are typically associated with confidence, a dominance, and positive emotional responses \cite{betella2016affective}. This state not only enhances the individual’s dominance over the environment but also promotes the efficient allocation of cognitive resources. Research indicates that individuals experiencing high dominance emotions are more likely to actively process situational information, prioritize key information, and adjust strategies based on needs, resulting in better performance in complex decision-making~\cite{mehrabian1996pleasure}.
% 情绪支配度是个体在情境中感知到的控制感和主导地位,其高低对信息理解和认知加工具有深远影响。高支配度情绪通常伴随自信心、控制感和积极的情绪反应(Betella & Verschure, 2016)。这种状态不仅提升了个体对环境的掌控感,还促进了认知资源的高效分配。研究表明,高支配度情绪个体更倾向于主动处理情境信息,优先关注关键信息,并根据需求调整策略,从而在复杂决策中表现更为出色(Mehrabian, 1996)。

In contrast, low dominance emotions are associated with negative emotions such as anxiety and helplessness, which suppress the effective utilization of cognitive resources \cite{kurth2010link}. Individuals with low dominance often struggle to focus their attention during information processing due to emotional disturbances, resulting in passive responses or even misinterpretation of information. Furthermore, research indicates that low dominance emotions may impair executive function, making it difficult for individuals to complete multitasking or integrate complex information \cite{gross1998emerging}.
% 相较之下,低支配度情绪与焦虑、无助等负面情绪相关,这会抑制认知资源的有效利用(Kurth et al., 2010)。低支配度个体在信息加工过程中常因情绪困扰难以集中注意力,表现为被动反应甚至信息误读。此外,研究指出,低支配度情绪可能削弱执行功能,导致个体难以完成多任务处理或复杂信息的整合(Gross, 1998)。

The role of emotional dominance in comprehension is also reflected in its impact on emotion regulation. Individuals experiencing high dominance emotions enhance information processing efficiency by optimizing emotional responses (such as reducing negative emotional interference), while low dominance emotions may exacerbate emotional fluctuations, further hindering the normal functioning of cognitive abilities \cite{gross1998emerging, kurth2010link}. This suggests that emotional dominance not only influences an individual’s emotional experience but also determines the depth and breadth of information comprehension, which in turn affects overall decision-making and behavior \cite{mehrabian1996pleasure, johnson2012dominance}.
% 情绪支配度在理解中的作用还体现在其对情绪调节的影响。高支配度情绪个体通过优化情绪反应(如减少负面情绪干扰)来提升信息加工的效率,而低支配度情绪则可能加剧情绪波动,进一步阻碍认知功能的正常发挥(Gross, 1998; Kurth et al., 2010)。这表明,情绪支配度不仅影响个体的情绪体验,还决定了其信息理解的深度和广度,进而影响整体决策与行为(Mehrabian, 1996; Johnson et al., 2012)。

\subsubsection{Summary}

The influence of emotions on information comprehension is multidimensional, with valence, arousal, and dominance collectively shaping individuals’ processing and integration of information. Positive emotions, by expanding cognitive resources and enhancing motivation, facilitate information integration and the understanding of complex texts but may affect accuracy due to attention dispersion \cite{egidi2012emotional}. Negative emotions enhance attention to details and improve reasoning ability but may limit overall efficiency due to increased cognitive load \cite{lang2007cognition}. Neutral emotions stabilize the allocation of cognitive resources, maintaining objectivity and consistency in comprehension, making them suitable for tasks requiring high precision \cite{earles2016memory}. Arousal has a particularly significant impact on comprehension: high-arousal emotions prioritize core information but may overlook minor details, while low-arousal emotions support comprehensive information integration, making understanding deeper and more flexible \cite{mather2011arousal}. Dominance further affects individuals’ dominance over information: high-dominance emotions enhance proactivity and processing efficiency, while low-dominance emotions may lead to distraction and comprehension difficulties \cite{mehrabian1996pleasure}. The interaction of emotional variables profoundly impacts information comprehension performance, revealing the complex relationship between emotions and cognition.
% 情绪对信息理解的影响是多维度的,其效价、唤起度和支配度共同塑造了个体对信息的加工与整合。积极情绪通过扩展认知资源和增强动机,有助于促进信息的整合与复杂文本的理解,但可能因注意力分散而影响准确性(Egidi & Nusbaum, 2012)。消极情绪增强了对细节的关注,提升推理能力,但可能因认知负担增加而限制整体效率(Lang et al., 2007)。中性情绪则通过稳定认知资源分配,保持理解的客观性和一致性,适用于高精确性任务(Earles et al., 2016)。唤起度对理解的作用尤为显著,高唤起情绪优先处理核心信息,但可能忽略次要细节;低唤起情绪支持全面信息整合,使理解更具深度与灵活性(Mather & Sutherland, 2011)。支配度进一步影响个体对信息的掌控感,高支配度情绪增强主动性与处理效率,而低支配度情绪可能导致注意力分散与理解障碍(Mehrabian, 1996)。情绪变量的交互作用深刻影响了信息理解的表现,揭示了情绪与认知之间的复杂关系。



\subsection{The Impact of Emotions on Memory}
Emotions play a significant role in the memory process, with valence, arousal, and dominance being key factors influencing memory depth and quality. Different emotional characteristics determine the selectivity, persistence, and accuracy of memory: valence determines the emotional orientation of memory content, arousal affects the intensity and clarity of memory, and dominance shapes individuals’ prioritization and integration of information in memory. These emotional factors influence the generation and retrieval of memory, directly affecting the organization and communication of information, thereby laying the foundation for a deeper understanding of how emotions influence information communication.
% 情绪在记忆过程中发挥重要作用,其效价、唤起度和支配度是影响记忆深度和质量的关键因素。不同情绪特征决定了记忆的选择性、持久性和准确性:效价决定了记忆内容的情感取向,唤起度影响记忆的强度和清晰度,而支配度则塑造了个体对信息的优先记忆和整合方式。这些情绪因素通过调节记忆的生成和提取,直接影响信息的组织和传达,为深入理解情绪如何影响信息传播奠定了基础。

\subsubsection{Emotional Valence}
Positive emotions play a significant facilitative role in the formation and retention of memory, particularly in influencing emotionally relevant information. Research has found that even in the case of incidental memory, positive emotions can deepen the retention of emotional vocabulary and related concepts, making individuals more likely to activate and recall these associated contents \cite{pawlowska2011influence, megalakaki2019effects}. Furthermore, individuals in a positive emotional state are more likely to remember information related to positive emotions, indicating that positive emotions significantly enhance the memory of associated information. Additionally, individuals experiencing pleasant emotions perform better in free recall and associative memory tests than those in a neutral emotional state \cite{lee1999effects, hanson2014happy}, allowing for more accurate recall of brand names and associated information \cite{madan2019positive}. This suggests that positive emotions not only aid in remembering specific information but also expand and integrate overall memory capabilities. Positive emotions also play a crucial role in the retention of long-term memory. 
%积极情绪在记忆的形成和保持过程中扮演着显著的促进角色,尤其是对情绪相关信息的影响更为突出。研究发现,即使在无意记忆的情况下,积极情绪也能加深情绪词汇和相关概念的记忆深度,个体更易于激活和回忆这些关联内容。同时,积极情绪状态下的人们更容易记住与积极情绪相关的信息,说明积极情绪对关联信息的记忆具有显著的促进作用。此外,愉快情绪的个体在自由回忆和联想记忆测试中的表现优于处于中性情绪状态的个体,能够更准确地回忆品牌名称及其相关信息,表明积极情绪不仅有助于记住特定信息,还能扩展和整合整体记忆能力。积极情绪在长期记忆保持方面同样发挥重要作用。

Numerous experimental studies have indicated that positive emotions positively influence both the formation and long-term retention of memory \cite{tyng2017influences}; for example, individuals in a pleasant emotional state retain memory of photographs for a longer duration, regardless of whether these photographs contain emotional content \cite{hanson2014happy}. Images associated with positive emotions are more easily recognized and remembered than neutral or negative images; thus, positive emotions facilitate the consolidation and storage of information, promoting the retention of memory over longer periods \cite{chainay2012emotional}.
%多项实验研究指出,积极情绪对记忆的形成和长期保持都有积极影响,例如在愉快情绪下的个体对照片的记忆保持时间更长,无论这些照片是否包含情绪内容。积极情绪图片比中性或消极情绪图片更容易被识别和记住,因此,积极情绪有助于巩固和存储信息,促进记忆在更长时间内的保持。

In contrast, the effects of negative emotions on memory are more complex, including enhanced memory accuracy, strengthened consolidation of long-term memory, as well as an increased likelihood of false memories. Negative emotions enhance individuals' attention to key details, which not only improves memory accuracy in free recall and cued recall tasks \cite{kensinger2007negative, kensinger2020retrieval, xie2017negative}, but also enables more accurate memory recall after longer periods \cite{van2015good}. This phenomenon may occur because negative emotions encourage individuals to engage in more in-depth processing of event details, thereby enhancing the durability of memories. However, negative emotions can also have side effects, such as an increase in false memories \cite{marchewka2016arousal}. This is because, while focusing on core details, individuals may overlook other information, necessitating the ``filling in" of these gaps during~recall \cite{brainerd2008does}.
% 与此相反,消极情绪对记忆的影响则更为复杂,其效果包括提升记忆的准确性、加强长期记忆的巩固,同时也增加了虚假记忆的出现概率。由于消极情绪增强了个体对关键细节的关注,这不仅提高了自由回忆和提示回忆任务中的记忆精确度, 而且还能在较长时间后提供更准确的记忆回忆。这种现象可能是因为消极情绪促使个体对事件细节进行更深入的处理,从而增强了记忆的持久性。然而,消极情绪也可能带来副作用,如虚假记忆的增加。这是因为在集中注意核心细节的同时,个体可能会忽略其他信息,导致在回忆时不得不“填补”这些遗漏的部分。

The impact of neutral emotions on memory is relatively weak, generally falling between that of positive and negative emotions. Neutral emotions lack significant emotional arousal, which leads to lower prioritization in encoding and consolidating memories of neutral emotional content \cite{earles2016memory, kensinger2020retrieval, megalakaki2019effects, lindstrom2011emotion}. Nevertheless, neutral emotions can enhance stability and consistency in performance for certain tasks. For example, maintaining a neutral emotional state during tasks requiring high precision can reduce emotional interference, thereby improving stability and consistency in task performance \cite{kensinger2006processing, storbeck2005sadness}. Additionally, individuals in a neutral emotional state are better able to sustain attention, facilitating more efficient processing and retention of information.
%中性情绪对记忆的影响相对较弱,通常介于积极和消极情绪之间。中性情绪缺乏显著的情感激发,因此个体对中性情绪材料的记忆编码和巩固优先级较低。尽管如此,中性情绪在某些任务中有助于提高执行的稳定性和一致性。例如,在执行高度依赖精确性的任务时,保持中性情绪有助于减少情绪干扰,从而提高任务的稳定性和一致性。此外,中性情绪状态下,个体的注意力更容易维持,从而确保信息能够得到更高效的加工和记忆。

\subsubsection{Emotional Arousal}
The impact of emotional arousal on memory is a complex and significant process. Research indicates that, compared to low-arousal or neutral events, high-arousal emotional events are typically easier to remember and play an important role in the processes of memory formation, consolidation, and retrieval. 
%情绪唤起度对记忆的影响是一个复杂且显著的过程。研究表明,相比于低唤起或中性事件,高唤起的情绪事件通常更容易被记住,并在记忆的形成、巩固和提取过程中发挥着重要作用。

Firstly, during the encoding process of memory, high-arousal emotional events can significantly enhance the level of detail in memory. This is because high-arousal emotions allocate more cognitive resources to relevant information, improving the accuracy and durability of memory \cite{mather2011arousal}. For instance, after experiencing high-arousal emotional events such as trauma or extreme excitement, individuals often vividly remember the details of the event. 
%首先,在记忆的编码过程中,高唤起情绪事件能显著增强记忆的详细程度。这是因为高唤起情绪使个体对相关信息分配更多的认知资源,提高了记忆的准确性和持久性。例如,在经历创伤或极度兴奋等高唤起情绪事件后,个体往往能清晰记住事件的细节。

During the consolidation phase of memory, high-arousal emotional events promote long-term storage of memories through the regulation of hormonal levels in the body. The release of these hormones strengthens memory traces, making them easier to recall accurately after extended periods \cite{mcgaugh2015consolidating}. For example, individuals who have experienced high-arousal emotional events such as significant accidents or personal achievements can vividly reconstruct specific scenarios from those events, even after a long time has passed. 
%在记忆的巩固阶段,高唤起的情绪事件通过体内激素水平的调节,促进记忆的长期存储。这些激素的释放强化了记忆痕迹,使其更易在长时间后被准确回忆。例如,个体在经历重大事故或个人成就等高唤起情绪事件后,即使过了很长时间,仍能生动地再现事件中的具体情景。

During the process of memory retrieval, the influence of high-arousal emotions is equally significant. High-arousal emotions often trigger vivid and specific recollections, prompting individuals to focus on core information and key details during retrieval. However, while high-arousal emotions can aid in recalling details, they may also increase recall biases and errors, especially when dealing with negative emotional events. Research shows that high-arousal negative emotions may lead to selective biases, where individuals focus on specific emotional content while neglecting other minor details, potentially resulting in false memories \cite{brainerd2008does}. 
% 在记忆提取的过程中,高唤起情绪的影响同样显著。这些事件由于其高唤起度,通常引发更生动、具体的回忆,使得个体在提取时更倾向于集中注意力于核心信息和重要细节。然而,尽管高唤起情绪有助于细节回忆,它也可能增加回忆的偏差和误差,尤其是在处理负面情绪事件时。研究表明,高唤起的负面情绪可能导致选择性偏差,即个体在专注于特定情绪内容时忽略其他次要细节,可能产生虚假记忆。


In summary, emotional arousal significantly influences individuals' memory performance for emotional events by enhancing the processes of encoding, consolidation, and retrieval. This enhancement effect makes high-arousal emotional events more memorable and leads to longer retention compared to low-arousal or neutral emotional events. However, high arousal can also increase the selectivity and bias of memory, thereby affecting its accuracy.
%总之,情绪唤起度通过增强记忆的编码、巩固和提取过程,显著影响个体对情绪性事件的记忆表现。这种增强效应使得高唤起情绪事件相比低唤起或中性情绪事件更易被记住且记忆维持时间更长。然而,高唤起度也可能增加记忆的选择性和偏差性,从而影响记忆的精确性。

% \subsubsection{The Neural Mechanisms of Emotions in Memory}
% The formation and consolidation of emotions depend on the coordinated activity of multiple brain regions. This section will explore the neural mechanisms by which emotions influence memory, particularly how emotions regulate the formation and consolidation of memories through specific neural pathways. This helps to understand why emotional events are more easily remembered than neutral events. 
% %情绪的形成与巩固依赖于多个大脑区域的协同作用。本节将探讨情绪对记忆的神经机制,特别是情绪如何通过特定的神经通路调节记忆的形成和巩固。这有助于理解为什么情绪性事件比中性事件更容易被记住。

% Emotional events typically activate the amygdala and hippocampus significantly, both of which are critical regions for memory formation. Activation of the amygdala enhances memory for emotional events, regardless of whether they involve high-arousal positive or negative emotions \cite{jaaskelainen2020neural, hamann2002ecstasy}. Research has found that the consistency between emotional states and the emotional content of information significantly affects memory encoding and consolidation: processing is smoother when consistent and more difficult when inconsistent, reflected by an increase in the N400 peak, highlighting the amygdala's crucial role in processing emotional information \cite{jimenez2012emotional, egidi2012emotional}. 
% %情绪性事件通常会显著激活大脑中的杏仁核和海马体,这两个区域是记忆形成的关键环节。无论是高唤醒的积极情绪还是消极情绪,杏仁核的激活都增强了情绪事件的记忆。研究发现,情绪状态与信息情感的一致性显著影响记忆的编码和巩固:一致时处理更顺畅,不一致时则增加处理难度,表现为N400峰值增大,突出显示了杏仁核在情绪信息处理中的关键作用。

% Firstly, the emotional valence has a significant impact on memory. Positive emotions typically activate the hippocampus via the dopamine system, enhancing the deep processing of information and thereby improving the efficiency of memory storage and retrieval \cite{alexander2021neuroscience, hamann2002ecstasy}. This emotional state also activates the left amygdala and brain areas associated with reward and emotion, such as the ventral striatum and ventromedial prefrontal cortex, facilitating memory encoding and retrieval \cite{hamann2002ecstasy}. In contrast, negative emotions enhance memory for details by activating the amygdala and visual processing areas, increasing the durability and vividness of memories \cite{kensinger2020retrieval}. For example, studies have found that negative emotions can significantly enhance memory performance in Alzheimer's patients and the~elderly \cite{kazui2000impact}. 
% %首先,情绪效价(即积极或消极情绪)对记忆产生显著影响。积极情绪通常通过多巴胺系统激活海马体,增强信息的深度处理,从而提高记忆的存储和提取效率。这种情绪还激活了左侧杏仁核及与奖励和情绪相关的脑区,如腹侧纹状体和腹内侧前额皮质,从而促进记忆的编码和提取。相反,消极情绪通过激活杏仁核和视觉处理区域,强化对细节的记忆能力,增加记忆的持久性和鲜明度。例如,研究发现,消极情绪可以显著增强阿尔茨海默病患者和老年人的记忆效果。

% Subsequently, emotional arousal levels also play a key role in the memory process. High-arousal emotions, whether positive or negative, promote the consolidation of long-term memories by activating the amygdala, thereby enhancing neural connectivity between the hippocampus and the prefrontal cortex \cite{mcgaugh2015consolidating}. In this state, detailed memories are strengthened \cite{marchewka2016arousal, mather2009disentangling}, particularly for high-priority information, while low-priority information may be suppressed \cite{sakaki2014emotion}. For example, in high-pressure environments such as exams, anxious emotions can help individuals more effectively retain key points while selectively disregarding less essential information. 
% %随后,情绪唤醒水平也在记忆过程中扮演关键角色。高唤醒情绪(无论是积极还是消极)通过激活杏仁核,增强海马体与前额叶皮层之间的神经连接,从而促进长期记忆的巩固。这种状态下,记忆细节得到加强,特别是对高优先级信息的记忆,而低优先级信息可能会被抑制。例如,在高压环境下,如考试,紧张的情绪可能使人更加记住考试的关键知识点,同时忽略掉次要内容。

\subsubsection{Emotional Dominance}
The role of emotional dominance in memory primarily manifests in memory selectivity and memory reconstruction. Research shows that in high dominance emotional states, individuals tend to prioritize remembering key information related to their own emotions. This memory selectivity enhances the depth of storage for important information, enabling individuals to quickly retrieve key content when needed \cite{ledoux2000emotion}. For example, in threatening situations, high dominance fear emotions significantly enhance memory for threat-related information through the activation of the amygdala, ensuring that individuals can respond quickly in similar situations \cite{anderson2001lesions}.
% 情绪支配度对记忆的作用主要体现在记忆选择性和记忆重建两个方面。研究表明,高支配度情绪状态下,个体倾向于优先记忆与自身情绪相关的关键信息。这种记忆选择性强化了重要信息的存储深度,使个体在需要时能够迅速提取关键内容(LeDoux (2000))。例如,在威胁情境中,高支配度的恐惧情绪通过杏仁核的激活显著增强了对威胁信息的记忆,确保个体在类似情境下能够快速反应(Anderson & Phelps, 2001)。

However, the effects of high dominance emotions are not limited to negative emotions. High dominance in positive emotions, such as excitement and joy, also enhances memory comprehensiveness. In this state, individuals not only remember core information but also effectively integrate relevant background information, forming a more comprehensive memory structure \cite{fredrickson2001role}. In contrast, low dominance emotional states generally limit an individual’s memory capacity. In this state, individuals are more likely to neglect non-emotional information, leading to fragmented or distorted memories \cite{schacter2007cognitive}.
% 然而,高支配度情绪的作用并非仅限于负性情绪。正性情绪的高支配度,例如兴奋和愉悦,同样有助于提升记忆的全面性。这种状态下,个体不仅记住了核心信息,还能有效整合相关背景信息,形成更加全面的记忆结构(Fredrickson, 2001)。与此形成对比,低支配度情绪状态通常限制了个体的记忆能力。这种状态下,个体更容易忽略非情绪性信息,导致记忆片段化或失真(Schacter & Addis, 2007)。

Furthermore, emotional dominance has a profound impact on memory reconstruction. Individuals with high dominance emotions tend to focus on positive or useful content during memory recall, thereby enhancing the functionality of memory \cite{fredrickson2001role, schacter2007cognitive}. In contrast, individuals with low dominance emotions may be biased towards retrieving negative memories due to emotional interference, which may further impair their coping and emotion regulation abilities \cite{anderson2001lesions, ledoux2000emotion}. These differences suggest that emotional dominance not only influences memory quality through memory selectivity but also shapes an individual’s cognitive patterns through the memory reconstruction process \cite{bower1981mood, schacter2007cognitive}.
% 此外,情绪支配度对记忆重建具有深远影响。高支配度情绪个体在记忆回忆时倾向于聚焦于积极或有用的内容,从而增强记忆的功能性(Fredrickson, 2001; Schacter & Addis, 2007)。而低支配度情绪个体则可能因情绪干扰而偏向于消极记忆的提取,这可能进一步削弱其应对能力和情绪调节能力(Anderson & Phelps, 2001; LeDoux, 2000)。这些差异表明,情绪支配度不仅通过记忆选择性影响记忆质量,还通过记忆重建过程塑造个体的认知模式(Bower, 1981; Schacter & Addis, 2007)。


\subsubsection{Summary}

The impact of emotions on memory is a complex multidimensional process, with valence, arousal, and dominance collectively shaping the formation and retrieval of memory. Positive emotions typically enhance the encoding and consolidation of associative information by expanding cognitive resources, improving the retention of long-term memory, but may interfere with focus on core information due to attention dispersion \cite{lee1999effects}. Negative emotions increase attention to detail, improving memory accuracy and durability, but neglect of non-core information may lead to the creation of false memories \cite{kensinger2007negative, brainerd2008does}. Neutral emotions provide balance and stability to the memory process, particularly for tasks requiring high objectivity and precision \cite{kensinger2006processing}. Arousal also plays a crucial role in memory: high-arousal emotions enhance the depth and durability of memory by focusing cognitive resources but may increase selective bias due to neglect of secondary information \cite{mather2011arousal}. Low-arousal emotions, on the other hand, support the comprehensive integration of information, making memory more holistic and flexible. Meanwhile, high-dominance emotions enhance the memory and recall of critical information, while low-dominance emotions may result in fragmented and distorted memories \cite{fredrickson2001role, schacter2007cognitive}. Overall, the profound influence of emotions on memory reveals the complex interactions between emotions and cognitive functions.
% 情绪对记忆的影响是一个复杂的多维过程,其效价、唤起度和支配度共同塑造了记忆的生成与再现。积极情绪通常通过扩展认知资源促进关联信息的编码与巩固,同时提升长期记忆的保持能力,但可能因注意力扩散干扰对核心信息的聚焦(Lee & Sternthal, 2014)。消极情绪增强对细节的关注,提高记忆的准确性和持久性,但对非核心信息的忽视可能导致虚假记忆的产生(Kensinger et al., 2007; Brainerd et al., 2008)。中性情绪则为记忆过程提供平衡和稳定性,尤其适用于需要高客观性和精确性的任务(Kensinger & Schacter, 2006)。唤起度在记忆中同样起到重要作用,高唤起度情绪通过集中认知资源,显著增强了记忆的深度和持久性,但可能因忽视次要信息而增加选择性偏差(Mather & Sutherland, 2011)。低唤起度情绪则支持信息的全面整合,使记忆更具整体性和灵活性。与此同时,高支配度情绪强化了对关键信息的记忆和回忆能力,而低支配度情绪则可能导致信息的片段化和失真(Fredrickson, 2001; Schacter & Addis, 2007)。整体而言,情绪对记忆的深远影响揭示了情绪与认知功能之间的复杂互动。

\subsection{The Impact of Emotions on Information Sharing}
Emotions play a core driving role in information sharing, from motivating sharing behaviors to shaping communication patterns, with different emotional characteristics showing significant differences in the effects of information diffusion. Positive emotions typically drive rapid information communication by enhancing appeal and triggering interactive behaviors, whereas negative emotions play a critical role in crises or significant events through their emotional resonance effects. Furthermore, the arousal and dominance of emotions further influence the depth and breadth of sharing, making emotions a critical variable in the process of information communication. This interaction between emotions and information communication reveals the complex psychological mechanisms underlying social behaviors.
% 情绪在信息分享中起着核心驱动作用,从激发分享动机到塑造传播模式,不同情绪特征对信息的扩散效果展现出显著差异。积极情绪通常通过吸引力提升和互动行为激发,推动信息的快速传播,而消极情绪则以其情感共鸣效应在危机或重大事件中扮演关键角色。此外,情绪的唤起度和支配度进一步影响分享的深度与广度,使得情绪成为信息传播过程中不可忽视的关键变量。这种情绪与信息传播的交互作用揭示了社交行为背后的复杂心理机制。



\subsubsection{Emotional Valence}
Positive emotions play a crucial role in promoting information sharing, but the complexity of their impact should not be overlooked. Research shows that individuals are more inclined to share content on social media under the influence of positive emotions. Ferrara \& Yang \cite{ferrara2015measuring} noted that positive emotions increase the appeal of social media content, thereby facilitating its widespread communication. Additionally, studies by Stieglitz \& Dang-Xuan \cite{stieglitz2013emotions} and Schreiner et al. \cite{schreiner2021impact} have found that content with positive emotions is more likely to be shared and commented on, indicating that positive emotions can significantly enhance the communication of information.
%积极情绪在信息分享中具有重要的推动作用,但其影响的复杂性不容忽视。研究显示,在积极情绪的驱动下,个体更倾向于在社交媒体上分享内容。Ferrara 和 Yang(2015)指出,积极情绪提升了社交媒体内容的吸引力,从而促进了信息的广泛传播。此外,Stieglitz 和 Dang-Xuan(2013)及Schreiner等(2021)的研究均发现,带有积极情绪的内容更易于被转发和评论,表明积极情绪能显著增强信息的传播效果。

However, the impact of positive emotions is not always beneficial. Dabbous \& Aoun Barakat \cite{dabbous2023influence} observed that while positive emotions enhance the appeal of information, they may also lead users to overlook its authenticity, thereby unintentionally contributing to the communication of inaccurate or false information. These findings suggest that although positive emotions can support the communication of information, their effects must be carefully managed to mitigate the spread of misleading~content.
%然而,积极情绪的影响并非总是积极的。Dabbous 和 Aoun Barakat(2023)的研究表明,积极情绪虽增加了信息的吸引力,但这也可能让用户忽视对信息真实性的审查,从而无意中帮助传播了不准确或虚假的信息。这说明,虽然积极情绪可以促进信息的传播,但在实际应用中需谨慎处理,以防止误导性信息的广泛传播。

In contrast, negative emotions also play a significant role in information communication. Studies show that negative emotions such as anger, sadness, or anxiety can stimulate users' willingness to share and spread information. Ferrara \& Yang \cite{ferrara2015measuring} noted that negative emotional content on social media is prone to emotional contagion, prompting users to seek emotional resonance and social support through sharing. Moreover, in crises and political events, the role of negative emotions is particularly prominent; they can encourage the public to focus more deeply on and understand relevant information in crisis situations, thereby accelerating the communication of information \cite{sanford2004negative}. Research by De León \& Trilling  \cite{de2021sadness} and Rojo López \& Naranjo \cite{lopez2021translating} has shown that negative emotions significantly increase the number of times news is shared and public attention. 
%相比之下,消极情绪同样在信息传播中扮演了显著的角色。研究表明,消极情绪如愤怒、悲伤或焦虑可激发用户分享和传播信息的意愿。Ferrara 和 Yang(2015)指出,社交媒体上的消极情绪内容易引起情绪传染,促使用户通过分享寻求情感共鸣和社会支持。此外,在危机和政治事件中,消极情绪的作用尤为显著,消极情绪能够促使公众在危机情境下更深入地关注和理解相关信息,从而加速了信息的传播。de León 和 Trilling(2021)及Rojo López 和 Naranjo(2021)的研究显示,消极情绪显著增加了新闻的分享次数和公众的关注度。

Neutral emotions, while having a relatively weaker influence on information sharing, still contribute to the process of information communication. Due to the absence of a strong emotional drive, content associated with neutral emotions typically exhibits lower attractiveness and spreadability on social media. However, research by Son et al. \cite{son2022emotion} highlights the unique value of neutral emotions, as individuals in a neutral emotional state are more likely to adopt a rational and objective perspective. This tendency supports the accurate conveyance and reception of information. Consequently, although neutral emotions are less effective than other emotions in amplifying the spread of information, they play an indispensable role in maintaining objectivity and accuracy in communication. 
%尽管中性情绪对信息分享的影响相对较弱,但其在信息传播中也起到一定作用。由于中性情绪缺乏强烈的情感驱动,其相关内容在社交媒体上的吸引力和传播性通常较低。然而,Son等(2022)的研究表明,中性情绪的内容具有其独特价值,因为在中性情绪状态下,人们往往表现出更中立和理性的态度,这有助于信息的客观传达和准确接收。因此,虽然中性情绪在提升信息传播性方面不如其他情绪有效,但其在保证信息传达的客观性和准确性方面具有不可替代的作用。

By integrating and examining the impact of positive, negative, and neutral emotions on information communication, we can better understand and utilize the complex role of emotions in information sharing, thereby effectively guiding the design and communication strategies of social media content.
%通过整合和审视积极、消极及中性情绪对信息传播的影响,我们可以更好地理解和利用情绪在信息分享中的复杂作用,从而有效地指导社交媒体内容的设计和传播策略。

\subsubsection{Emotional Arousal}
Emotional arousal plays an important role in information sharing on social media. Studies show that both positive and negative high-arousal emotions significantly increase the spread of information \cite{berger2011arousal}. High-arousal emotions such as awe, excitement, anger, or anxiety stimulate a sense of urgency and motivation in users, prompting them to share content more frequently on social media \cite{berger2012makes}. In crisis events or major news, high-arousal negative emotions such as anger or anxiety often trigger strong reactions in users, driving them to quickly share related information on social media \cite{lopez2021translating, sanford2004negative}. This sharing behavior is often motivated by the need to seek social support or express emotional resonance. Conversely, high-arousal positive emotions such as excitement and awe not only enhance users' sharing behavior but also expand the range of information communication. This indicates that users are more inclined to share content on social media when experiencing positive high-arousal emotions, seeking social validation and emotional resonance \cite{son2022emotion}. However, the communication effect of high-arousal emotions also poses risks. In highly emotional states, especially during crisis events or periods of social unrest, users may quickly share information without adequate verification, thereby increasing the risk of spreading fake news and misinformation \cite{dabbous2023influence}. 
% 情绪唤起度在社交媒体上的信息分享中扮演了重要角色。研究表明,无论是正面还是负面,高唤起情绪都显著增加了信息的传播性。高唤起的情绪如敬畏、兴奋、愤怒或焦虑激发用户的紧迫感和行动动机,这促使他们在社交媒体上更频繁地分享内容。在危机事件或重大新闻中,高唤起的负面情绪,如愤怒或焦虑,常引发用户的强烈反应,推动他们在社交媒体上迅速分享相关信息。这种分享行为往往是出于寻求社会支持或表达情感共鸣的需要。相反,高唤起的正面情绪如兴奋和敬畏不仅增强用户的分享行为,还能扩大信息的传播范围。这表明用户在体验到积极的高唤起情绪时,更倾向于在社交媒体上分享内容,寻求社会认同和情感共鸣。然而,高唤起情绪的传播效果也存在风险。在高度情绪化的状态下,尤其是在危机事件或社会动荡期间,用户可能在未经充分验证的情况下快速分享信息,从而增加假新闻和虚假信息的扩散风险。

Conversely, the communication effect of low-arousal emotions is relatively weak; the lack of strong stimulation usually results in users being calmer or more neutral in such emotional states, leading to a lower motivation to share \cite{son2022emotion}. However, in specific contexts, low-arousal positive emotions such as contentment or gratitude may also prompt users to share information with the aim of supporting or educating others \cite{stieglitz2013emotions}. Additionally, low-arousal negative emotions, such as sadness, may motivate users to reflect deeply and engage in discussions about major disasters or social issues. This process can promote information communication to seek social support or drive social change. \cite{de2021sadness}.
% 相对地,低唤起情绪的传播效果相对较弱,由于缺乏强烈的激发力,使得用户在这类情绪状态下通常表现出更多的冷静或中立,导致分享动机较低。然而,在特定情境下,低唤起的积极情绪如满足或感激也可能促使用户出于支持或教育他人的目的进行信息分享。此外,低唤起的负面情绪如悲伤在应对重大灾难或社会问题时,可能促使用户进行深度反思和讨论,推动信息传播以寻求社会支持或促进社会变革。

Therefore, emotional arousal is a key factor determining the effectiveness of information communication on social media, with high-arousal emotions typically amplifying both the spread of information and the motivation to share it, while low-arousal emotions exert a more subtle influence on user sharing behavior, particularly in specific contexts. These findings offer valuable guidance for optimizing social media strategies and enhancing crisis management approaches.
% 因此,情绪唤起度是决定社交媒体信息传播效果的关键因素,高唤起情绪通常能显著提升信息的传播性和分享动力,而低唤起情绪则在特定情境下以较为细微的方式影响用户分享行为。这些洞察对于制定有效的社交媒体运营和危机管理策略具有重要价值。

% \subsubsection{The Neural Mechanisms of Emotions in Sharing}
% Emotions play a crucial role in information sharing by activating multiple key areas of the brain \cite{ferrara2015measuring}. These areas include the amygdala, ventromedial prefrontal cortex (VMPFC), temporoparietal junction (TPJ), and the insula, which collectively regulate an individual's motivation and behavior for sharing. Initially, the amygdala plays a central role in processing emotional responses, such as anger or surprise. Activation of the amygdala enhances the salience of emotions, making individuals more aware of the significance of emotions, thereby increasing their motivation to express and share these emotions with others \cite{decety2008emotion}. For instance, high-arousal negative emotions such as anger or fear activate the amygdala, prompting individuals to perceive the importance of emotions and to tend towards sharing these emotions. This mechanism is particularly evident in social media, where the effect of emotional contagion operates through this pathway \cite{ferrara2015measuring}.
% % 情绪通过激活大脑多个关键区域,在信息分享中发挥重要作用。这些区域包括杏仁核、腹内侧前额叶皮层(VMPFC)、颞顶联合区(TPJ)和岛叶,它们共同调节个体的分享动机和行为。首先,杏仁核在处理情绪反应,如愤怒或惊讶时,起着核心作用。杏仁核的激活提升了情绪的显著性,使个体更加感知到情绪的重要性,从而增加了想他人表达和分享情绪的动机。例如,高唤起的负面情绪如愤怒或恐惧,使得杏仁核活跃,促使个体感知到情绪的重要性,并倾向于进行情绪分享。这种机制在社交媒体中尤为明显,情绪传染效应即通过这一途径发挥作用。

% Concurrently, the ventromedial prefrontal cortex (VMPFC) plays a crucial role in sharing positive emotions. When an individual experiences positive emotions, the VMPFC is activated and cooperates with the brain's reward system to evaluate the social and emotional benefits of sharing the emotional experience, enhancing the motivation to express emotions \cite{haber2010reward}. Additionally, when negative emotions are highly aroused, the VMPFC collaborates with the amygdala to assess the risks and rewards of expressing emotions, determining whether and how to share negative emotions \cite{bechara2000emotion}.
% %同时,腹内侧前额叶皮层(VMPFC)在积极情绪分享中扮演重要角色。当个体体验到积极情绪时,VMPFC被激活,并与大脑奖励系统协作,评估分享情绪体验的社交和情感收益,增加情绪表达的动力。此外,当消极情绪唤起度较高时,VMPFC也会与杏仁核协作,评估情绪表达的风险与回报,决定是否以及如何分享负面情绪。

% Activation of the temporoparietal junction (TPJ) enables individuals to better understand and predict others' emotional responses, adjusting their sharing strategies accordingly \cite{mitchell2009inferences}. For example, when individuals perceive that others are interested in or sympathetic to their emotional experiences, activation of the TPJ encourages them to express their emotions more openly to strengthen social connections.
% %颞顶联合区(TPJ)的激活则使个体能更好地理解和预测他人情绪反应,调整自己的分享策略。例如,当个体感知到对方对自己情绪体验感兴趣或表示同情时,TPJ的激活会促使其更开放地表达情绪,以增强社交联系。

% The insula also plays a key role when emotional experiences are accompanied by physiological responses. As a region that integrates internal bodily states with emotional experiences, the insula processes physiological signals such as increased heart rate and rapid breathing, thereby influencing the motivation to share. Craig \cite{craig2009you} notes that when emotional experiences are intense and accompanied by significant physiological reactions, high activation of the insula increases individuals' tendencies to seek resonance and support, making them more likely to share their emotions directly or intensely.
% %岛叶在情绪体验伴随生理反应时也起到关键作用。作为整合身体内部状态与情绪体验的区域,岛叶处理如心跳加快、呼吸急促等生理信号,进而影响分享动机。Craig(2009)指出,当情绪体验强烈且伴有明显的生理反应时,岛叶的高激活水平增加了个体寻求共鸣和支持的倾向,使其更可能直接或强烈地分享情绪。

% The valence (positive or negative) and arousal (high or low) of emotions jointly influence the activation patterns of the above brain regions, complexly regulating individuals' motivations and behaviors in the process of information sharing. High-arousal positive emotions typically enhance the perceived value and motivation to share positive emotions through the cooperation of the VMPFC and amygdala; whereas high-arousal negative emotions regulate the relationship between expressing emotions and social risks during sharing through interactions between the amygdala, insula, and other areas such as the ACC and DMPFC \cite{lang2010emotion}. In contrast, low-arousal emotions (whether positive or negative) typically result in less activation, and individuals rely more on the TPJ and DMPFC to decide whether to express emotions \cite{pessoa2009emotion}.
% % 情绪的效价(正性或负性)和唤起度(高或低)共同影响上述脑区的激活模式,从而复杂地调节个体在信息分享过程中的动机和行为。高唤起的正性情绪通常通过VMPFC和杏仁核的协作,增加正面情绪的价值感和分享动机;而高唤起的负性情绪则通过杏仁核、岛叶和其他区域(如ACC和DMPFC)的相互调节,使个体在分享过程中权衡表达情绪与社交风险的关系。相较之下,低唤起情绪(无论正性还是负性)通常激活较少,个体更依赖TPJ和DMPFC来判断是否表达情绪。


\subsubsection{Emotional Dominance}

Emotional dominance plays a decisive role in information sharing. Individuals in high dominance emotional states typically exhibit strong self-confidence and social dominance, making them more willing to actively share information. Such individuals tend to view sharing behavior as an important means of expressing self-worth and influence \cite{mehrabian1996pleasure}. Research indicates that this emotional state encourages individuals to participate more actively in information exchange and establish higher social status within groups \cite{fredrickson2001role}. Social cognitive theory \cite{bandura1989human} provides theoretical support for this phenomenon. Individuals with high dominance emotions often reinforce their self-efficacy through sharing behaviors and use positive feedback during the sharing process to further enhance their social skills and influence. In contrast, individuals with low dominance emotions perceive information sharing as a threat due to a lack of confidence, fearing negative evaluation from loss of control or questioning \cite{lebel2017moving}. This psychological drive leads individuals with low dominance emotions to adopt conservative or passive sharing strategies.
% 在信息分享中,情绪支配度起着决定性作用。高支配度情绪状态的个体通常伴随较强的自信心和社会主导感,从而更愿意主动分享信息。这种个体倾向将分享行为视为表达自我价值和影响力的重要手段(Mehrabian (1996))。研究表明,这种情绪状态会促使个体更加积极地参与信息交流,并在群体中建立更高的社会地位(Fredrickson (2001) )。社会认知理论(Bandura, 1989)为这种现象提供了理论支持。高支配度情绪个体往往通过分享行为强化自我效能感,并利用分享过程中的积极反馈进一步提升其社交能力和影响力。相比之下,低支配度情绪个体由于缺乏自信,通常将信息分享视为一种潜在威胁,担心因信息失控或被质疑而遭受负面评价(Lebel, 2017)。这种心理驱动促使低支配度个体选择保守或被动的分享方式。

Additionally, social comparison theory \cite{festinger1954theory} reveals the deeper motivations of emotional dominance in information sharing: individuals with high dominance emotions tend to engage in positive comparisons through sharing behaviors, thereby further solidifying their advantages within the group. For example, they are more inclined to share information related to achievements or successes to enhance others’ recognition of their abilities. In contrast, individuals with low dominance emotions tend to avoid sharing behaviors to prevent exposing weaknesses or losing control during comparisons~\cite{morrison2000organizational}.

% 此外,社会比较理论(Festinger, 1954)揭示了情绪支配度在信息分享中的深层动机:高支配度情绪个体倾向于通过分享行为进行积极比较,从而进一步巩固其在群体中的优势地位。例如,他们更愿意分享与成就或成功相关的信息,以增强他人对其能力的认可。而低支配度情绪个体则倾向于回避分享行为,以避免在比较过程中暴露弱点或失去控制权(Morrison & Milliken, 2000)。

In summary, emotional dominance not only determines the motivation and strategies for information sharing but also shapes individuals’ social roles and influence within groups \cite{mehrabian1996pleasure, bandura1989human}. High dominance emotions encourage individuals to achieve social goals through active sharing, while low dominance emotions limit their ability to disseminate information, potentially further diminishing their status and value within social networks \cite{fredrickson2001role, morrison2000organizational}.

% 总结而言,情绪支配度不仅决定了信息分享的动机和策略,还塑造了个体在群体中的社会角色与影响力(Mehrabian (1996); Bandura, 1989)。高支配度情绪促使个体通过积极分享实现社交目标,而低支配度情绪则限制了个体的信息传播能力,可能进一步削弱其社交网络中的地位和价值(Fredrickson (2001) ; Morrison & Milliken, 2000)

\subsubsection{Summary}
The influence of emotions on information sharing is multi-layered and complex. Positive emotions enhance content appeal and willingness to interact, promoting rapid information diffusion on social media \cite{ferrara2015measuring}; however, this facilitation may also lead users to overlook the authenticity of the information \cite{dabbous2023influence}. Negative emotions, with their strong emotional resonance effect, particularly in crises and political events, drive widespread communication of information \cite{lopez2021translating}. Neutral emotions, although less likely to promote communication, have unique advantages in ensuring the accuracy and objectivity of information \cite{son2022emotion}.
High-arousal emotions, whether positive or negative, significantly enhance sharing motivation and communication breadth, while low-arousal emotions are better suited for in-depth discussions and reflection \cite{berger2012makes}. Dominance affects information-sharing strategies: high-dominance emotions encourage individuals to actively share to enhance social status, while low-dominance emotions tend toward selective sharing or avoidance \cite{mehrabian1996pleasure, bandura1989human}. The combined effects of emotional valence, arousal, and dominance further reveal the multidimensionality of information communication, providing valuable insights for optimizing social media communication~strategies.


% 情绪对信息分享的影响多层次且复杂。积极情绪通过增强内容吸引力和互动意愿,促进信息在社交媒体上的快速扩散(Ferrara & Yang, 2015);然而,这种促进作用也可能导致用户忽略信息真实性(Dabbous & Barakat, 2023)。消极情绪以其强烈的情感共鸣效应,特别是在危机和政治事件中,推动信息的广泛传播(Rojo López & Naranjo, 2021)。中性情绪尽管传播性较弱,但在信息准确性和客观性上具有独特优势(Son et al., 2022)。高唤起情绪无论正负,均显著提升分享动力和传播广度,而低唤起情绪则在深度讨论和反思中表现更佳(Berger & Milkman, 2012)。支配度影响信息分享策略,高支配度情绪促使个体积极分享以强化社会地位,而低支配度情绪则倾向于选择性分享或回避分享(Mehrabian, 1996; Bandura, 1989)。情绪效价、唤起度和支配度的综合作用进一步揭示了信息传播的多维性,为优化社交媒体传播策略提供了重要启示。