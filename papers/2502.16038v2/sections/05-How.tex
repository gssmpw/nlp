\section{How to Promote Information Communication by Regulating Emotions?}\label{sec:how} % 如何通过调节情感促进信息传达

% 把若妍调研的4个系统放到How的部分,情感是如何激活的,

% 设计空间要把What 和 how 的东西都涵盖。

% 怎么控制情感要素出发。

This section explores how design strategies can regulate emotions to optimize the effectiveness of information delivery. This study proposes a comprehensive multidimensional design framework aimed at regulating emotional valence, arousal and dominance, thereby providing theoretical support and practical guidance for the understanding, memory, and communication of information. The section begins with the construction of the overall design space and theoretical foundations, introducing a multi-system model of emotional activation as the guiding framework. It then analyzes the specific roles of text, visuals, sound, and interaction in emotional regulation across four dimensions. Through these analyses, the section comprehensively illustrates how multimodal design can be flexibly applied across various contexts to achieve effective emotional regulation and information optimization.
% 本章探讨如何通过设计手段调节情感,以优化信息的传递效果。本研究提出了一个综合性的多维度设计框架,旨在通过调节情感效价、唤起水平和支配感,为信息的理解、记忆和传播提供理论支持与实践指导。本章内容将首先从整体设计空间的构建与理论基础出发,提出情绪激活的多系统模型作为指导框架,随后从文本、视觉、声音和交互四个维度分别分析其对情绪调控的具体作用。通过这些分析,本章全面阐明了多模态设计如何在多种情境中灵活应用,从而实现高效的情感调控与信息优化。


\subsection{Overview of the Design Space}
The construction of the design space is founded on the systematic integration of various design elements, aiming to precisely control emotional valence and arousal through their modulation to optimize information delivery. EExisting research identifies four core dimensions of user experience design: text, visuals, sound, and interaction. Mohamad Roseli \& Aziz \cite{roseli2023affective}, through thematic analysis, summarized eight key design elements—images, text, audio, color, layout, navigation, feedback, and reward mechanisms—which can be further classified into the aforementioned four core dimensions. This classification highlights the critical role of each design dimension in shaping users’ information reception and emotional responses. Therefore, incorporating these core design dimensions into the design space systematically maps the pathways of users’ emotional responses, providing targeted strategies for emotional regulation.
% 设计空间的构建以多种设计元素的系统整合为基础,旨在通过调节这些元素实现对情感效价与唤起度的精准控制,以优化信息传达效果。现有研究指出,用户体验中的核心设计层面可划分为文本、视觉、声音和交互四个维度。Mohamad Roseli 和 Aziz (2023)通过主题分析总结了图像、文本、音频、颜色、布局、导航、反馈与奖励机制等八大关键设计要素,这些要素可进一步归类至上述四个核心维度中。这种分类明确了各个设计层面在用户接收信息与情感反应中的重要作用。因此,将这些核心设计层面纳入设计空间中,能够系统覆盖用户情感反应的路径,为情感调控提供针对性的设计策略。

\begin{table}[htbp]
\centering
\setlength{\tabcolsep}{7pt}
\captionsetup{skip=10pt}
\begin{tabular}{|c|c|c|}
\hline
\multirow{4}{*}{\begin{tabular}[c]{@{}c@{}}\textbf{Text} \\ \textbf{Design}\end{tabular}} 
& Headline & \cite{kuiken2017effective, kim2016compete, kourogi2015identifying, blom2015click} \\ \cline{2-3} 
& Narrative structure & \cite{mar2011emotion, jaaskelainen2020neural, menninghaus2017emotional} \\ \cline{2-3} 
& Narrative content & \cite{lekkas2022using, leshner2018breast, campbell2008hero} \\ \cline{2-3} 
& Description & \cite{seo2019process,lee2020impact, ludwig2013more, slavova2019towards, nguyen2014affective} \\ \hline

\multirow{4}{*}{\begin{tabular}[c]{@{}c@{}}\textbf{Visual} \\ \textbf{Design}\end{tabular}} 
& Color & \cite{pazda2024colorfulness, jonauskaite2019color, kallabis2024investigating, weijs2023effects, lin2023effect, suk2010emotional, tarvainen2014content, wilms2018color} \\ \cline{2-3} 
& Shape & \cite{lu2012shape, etzi2016arousing, ebe2015emotion, wei2006image, mayer2014benefits, ferrara2015measuring, thumfart2008modeling} \\ \cline{2-3} 
& Layout & \cite{makin2012implicit, lu2017investigation, carretie2019emomadrid, wang2022innovation, mai2011rule, resnick2003design} \\ \cline{2-3} 
& Images & \cite{hanson2014happy, xie2017negative,van2015good, hou2024emotional, pfeuffer122measuring, kensinger2007negative,dudarev2024social, kuzinas2024creative} \\ \hline

\multirow{3}{*}{\begin{tabular}[c]{@{}c@{}}\textbf{Sound} \\ \textbf{Design}\end{tabular}} 
& Tone & \cite{schirmer2010mark, weinstein2018you, bestelmeyer2017effects,  gobl2003role} \\ \cline{2-3} 
& Sound effects & \cite{eerola2012timbre, parncutt2011consonance, schulte2001quality, ostendorf2020sounds, mazur2019effects} \\ \cline{2-3} 
&  Music & \cite{hofbauer2024background, moon2024investigating, shepherd2024investigating, thaut2015neurobiological, bernardi2006cardiovascular, juslin2008emotional,kabre2024predisposed, baltazaremotional} \\ \hline

\multirow{3}{*}{\begin{tabular}[c]{@{}c@{}}\textbf{Interaction} \\ \textbf{Design}\end{tabular}} 
& Interaction methods & \cite{sundar2014user, wodehouse2014exploring, wang2024enhancing,  olugbade2023touch, amoor2014designing } \\ \cline{2-3} 
& Motion effects & \cite{hanjalic2005affective, wollner2018slow, lockyer2012affective, yoo2005processing} \\ \cline{2-3} 
& Navigation design & \cite{wang2024enhancing, abdelaal2023accessibility, sheng2012effects, sundar2014user, amoor2014designing} \\ \hline
\end{tabular}

\caption{Design Dimensions and Emotional Aspects}
\label{table:design_dimensions}
\end{table}

After clarifying the design dimensions, the specific design elements within each dimension were further refined. For instance, in text design, narrative structure and wording style can evoke user emotions through plot tension and word choice \cite{egidi2012emotional}. Visual design relies on elements such as color, image, shapes, and layout to influence users’ emotional valence and arousal. Sound design elicits emotional responses through tone, music, and sound effects \cite{plass2014emotional}. Interaction design enhances users’ dominance through interactive methods, motion effects, and navigation designs \cite{shneiderman2010designing}. The integration of these design dimensions provides a structured framework for emotional regulation, enabling the development of tailored strategies to meet the demands of various communication contexts.
% 在明确设计维度后,进一步细化了各维度的设计元素。例如,在文本设计方面,叙事结构与措辞风格能够通过情节张力与词汇选择调动用户的情绪【Egidi & Nusbaum, 2012】;视觉设计依托颜色、图片、形状与布局等元素影响用户的情感效价与唤起度;声音设计通过语调、音乐与音效激发情感反应【Plass et al., 2014】;交互设计则通过交互方式、动效与导航功能强化用户的支配感【Shneiderman, 1998】。这些设计维度的整合为情感调控提供了结构化的框架,能够针对不同传播情境的需求制定相应的策略。


To further reveal how these design elements specifically function in emotional regulation, this study introduces the multi-system model of emotional activation \cite{izard1993four}. The model emphasizes that emotional activation is the result of multi-level and multi-mechanism collaboration, involving the coordination of the neural system, sensorimotor system, motivational system, and cognitive system.
% 为进一步揭示这些设计元素如何具体作用于情感调控,本研究引入了情绪激活的多系统模型(Izard, 1993)。该模型强调,情绪激活是多层次、多机制协作的结果,涉及神经系统、感官运动系统、动机系统和认知系统的协同参与。


The nervous system rapidly generates emotional responses through the activity of specific brain regions, such as the interaction between the amygdala and prefrontal cortex. For example, sensory stimuli such as color\cite{pazda2024colorfulness} or sound\cite{clewett2024emotional} can quickly capture attention and evoke emotional responses, thereby enhancing the appeal and impact of information. The sensorimotor system reinforces emotional experiences through physical sensations and motor responses (e.g., increased heart rate, muscle tension), which can be amplified by tactile feedback\cite{olugbade2023touch} or dynamic visual effects\cite{amoor2014designing}. In information communication design, these sensory-level feedback mechanisms help convey emotional information, engage users more deeply, and enhance the perception and effectiveness of information delivery. The motivational system triggers emotional responses by regulating goal-oriented behavior. For instance, commonly used reward mechanisms or time-limited promotional strategies in information design effectively stimulate motivational emotions, facilitating the dissemination and reception of information. The cognitive system generates emotional responses through the evaluation, reasoning, and judgment of information. Activation of this system is particularly crucial in information communication. For example, narrative texts \cite{mar2011emotion, lekkas2022using} or complex visual symbols \cite{ferrara2015measuring, mayer2014benefits} can guide users’ emotional evaluations, enhancing the depth and impact of the information~content.
% 神经系统通过大脑特定区域的活动(如杏仁核与前额叶皮层的交互)快速生成情绪反应。例如,通过颜色或声音的感官刺激可以迅速引发注意力与情绪反应,从而增强信息的吸引力与影响力【Jacobs & Mendl, 1999】。感官运动系统则通过身体的感知与运动反应(如心跳加快、肌肉紧张)进一步强化情绪体验,这些反应可通过触觉反馈或动态的视觉效果得到增强。在信息传达设计中,这些感官层面的反馈帮助传递情感信息,使用户更加投入,从而提升信息的感知度与传达效果。动机系统通过目标导向行为的调节引发情绪反应,例如信息设计中常用的奖励机制或限时促销策略能够有效激发动机情绪,促进信息的传播与接收。认知系统则通过对信息的评估、推理与判断生成情绪反应。这一系统的激活在信息传达中尤为重要,例如叙事性文本或复杂的视觉符号设计能够引导用户的情绪评估,增强信息内容的深度与感染力。


\begin{figure}[hbt!]
%\setlength{\abovecaptionskip}{-0.1mm}
\setlength{\intextsep}{10pt plus 2pt minus 2pt}
    \centering
    \includegraphics[width=0.5\textwidth]{figs/Multisystem_model.png}
    \caption{Multisystem Model of Emotion Activation: A conceptual framework illustrating the interaction between neural processes, sensorimotor processes, affective processes, cognitive processes, and emotional experience in the generation and regulation of emotions.}
   \vspace{-2mm}
\label{fig:What}
\end{figure}

The interactions among these emotional systems not only determine the generation and expression of emotions but also shape the specific pathways of information communication. By modulating the activation levels of these systems, design can effectively regulate emotions, thereby optimizing the effectiveness of information delivery. This integration of theory and practice provides a solid theoretical foundation and practical guidance for multidimensional design spaces.
% 这些情感系统的相互作用不仅决定了情绪的生成与表达,也塑造了信息传达的具体路径。通过调控这些系统的激活程度,设计可以实现对情感的有效调节,从而优化信息的传递效果。这种理论与实践的结合为多维设计空间提供了坚实的理论基础与实操指引。

\subsection{Text Design}
Text is not merely a means of conveying information; it also regulates readers’ emotional responses through the skillful arrangement of narrative structures and wording, thereby enhancing memory, comprehension, and information sharing. In text design, different headline, narrative structures, content, wording, and styles of expression each uniquely influence emotions, allowing regulation of the audience’s emotional valence and arousal from multiple perspectives. The following analyzes the impact of text design on emotions from these four dimensions. The structure and format of titles collectively influence readers’ emotional valence and arousal in various ways. Narrative structure shapes users’ emotional experiences through the progression of stories and emotional tone, while narrative content guides emotional responses through plot arrangements and character resonance. Meanwhile, wording and expression styles influence the intensity and direction of emotions through specific words and framing.

% 文本在信息传达中不仅仅是表达信息的手段,它还通过叙事结构和措辞的巧妙安排来调节读者的情感反应,从而提升文本的记忆、理解和信息分享。在文本设计中,不同的叙事结构、叙事内容以及措辞和表达风格对情感的影响各具特色,能够从多个角度调控受众的情绪效价和情绪唤起度。以下从这四个维度来分析文本设计对情感的影响。标题的结构和形式通过多种方式综合影响读者的情绪效价和情绪唤起度。叙事结构通过故事的展开方式和情感基调来塑造用户的情绪体验,叙事内容则通过情节安排和角色共鸣引导情绪反应; 而措辞和表达风格则通过具体的词语和框架影响情绪的强度和方向。


\renewcommand{\arraystretch}{1.8} % 调整行间距
\begin{table*}[ht]
\fontsize{8}{9}\selectfont
\centering

\begin{tabularx}{\textwidth}{|>{\centering\arraybackslash}m{0.5cm}|>{\centering\arraybackslash}m{1.4cm}|>{\centering\arraybackslash}m{3.4cm}|>{\centering\arraybackslash}m{3.4cm}|>{\centering\arraybackslash}m{3.4cm}|>{\centering\arraybackslash}m{3.4cm}|}
\hline
\rowcolor[HTML]{D9EAD3} 
\multicolumn{2}{|c|}{\textbf{Dimension}} & \textbf{Headline} & \textbf{Narrative Structure} & \textbf{Narrative Content} & \textbf{Wording} \\ 
\hline

% Emotional Dimensions Section
\multirow{3}{*}{\rotatebox{90}{\parbox{3cm}{\centering \textbf{Emotional \\ Dimensions}}}} & 
\cellcolor[HTML]{FDF6E8} \textbf{Valence} &    
Celebrity references evoke positive emotions and attract readers \cite{kim2016compete}.
Forward cues (e.g., “Here’s why”) increase positivity \cite{blom2015click}. & 
Positive tone (e.g., hope, success) enhances positive valence, while negative tone (e.g., tragedy) intensifies negative emotions \cite{mar2011emotion}. & 
Positive events (e.g., recovery, victory) evoke positive emotions, while negative events (e.g., loss) evoke negative emotions \cite{lekkas2022using}. & 
Gain framing (e.g., positive outcomes) boosts positive valence, while loss framing enhances negative valence \cite{seo2019process}. \\ \cline{2-6}

& \cellcolor[HTML]{FDF6E8} \textbf{Arousal} & 
Concise headlines quickly grab attention \cite{blom2015click}. 
Hot topics and suspense increase arousal \cite{kim2016compete}. & 
Suspenseful or high-peak narratives sustain high arousal levels \cite{leshner2018breast}. & 
Emotional content (e.g., loss or pain) enhances arousal. Resonance with characters amplifies engagement \cite{lekkas2022using}. & 
Emotional language (e.g., “pain”) heightens intensity; parallel structures boost appeal \cite{menninghaus2017emotional}. \\ \cline{2-6}

& \cellcolor[HTML]{FDF6E8} \textbf{Dominance} & 
Clear headlines enhance user control (e.g., “Solve it in one step”) \cite{kourogi2015identifying}. Threatening headlines reduce trust \cite{kim2016compete}. & 
Heroic narratives (e.g., success stories) enhance control; failure narratives reduce it \cite{jaaskelainen2020neural}. & 
Success stories increase control; alternating emotions in complex narratives enrich the experience \cite{lekkas2022using}. & 
Positive language (e.g., “happiness”) boosts control, while abrupt phonemes reduce it \cite{menninghaus2017emotional}. \\ \hline

% Multisystem Activation Section
\multirow{4}{*}{\rotatebox{90}{\parbox{3cm}{\centering \textbf{Multisystem \\ Activation}}}} 

& \cellcolor[HTML]{F0EFF7} \textbf{Neural Systems} & 
Urgent headlines (e.g., “Danger ahead”) activate the amygdala, triggering physiological responses like increased heart rate \cite{panksepp2012archeology}. & 
Tense narratives activate the amygdala, causing fear or excitement; complex stories engage the prefrontal cortex \cite{jaaskelainen2020neural}. & 
Descriptions of fear or anger activate the sympathetic nervous system, leading to adrenaline release \cite{panksepp2012archeology}. & 
Emotionally charged wording triggers the amygdala, inducing tension or excitement \cite{panksepp2012archeology}. \\ \cline{2-6}

& \cellcolor[HTML]{F0EFF7} \textbf{Sensorimotor Systems} & 
Urgent headlines (e.g., “Last chance”) trigger physical responses like muscle tension \cite{james1884mind}. & 
Climax or conflict scenes evoke bodily reactions, such as rapid breathing \cite{panksepp2012archeology}. & 
 Vivid descriptions of conflict or fear may cause muscle tension or mimicry (e.g., holding breath) \cite{james1884mind}. & 
Emotional words (e.g., “tragic”) may provoke physical reactions like frowning or fast breathing \cite{panksepp2012archeology}. \\ \cline{2-6}

& \cellcolor[HTML]{F0EFF7} \textbf{Cognitive Systems} & 
Headlines guide quick judgment (e.g., “A miracle happened” evokes optimism) \cite{lazarus1991emotion}. & 
Narratives help evaluate emotional contexts (e.g., hope in heroic stories, sadness in tragic ones) \cite{mar2011emotion}. & 
Readers assess character actions, evoking emotions like anger or empathy \cite{nguyen2014affective}. & 
Positive words (e.g., “success”) evoke optimism; negative words (e.g., “failure”) evoke pessimism \cite{seo2019process}. \\ \cline{2-6}

& \cellcolor[HTML]{F0EFF7} \textbf{Motivational Systems} & 
Positive headlines (e.g., “Act now”) stimulate motivation, while urgency headlines evoke avoidance \cite{calvo2010affect}. & 
Heroic narratives inspire achievement motivation; twists encourage exploration \cite{campbell2008hero}. & 
Adventure stories spark exploration; success stories foster pursuit motivation \cite{mar2011emotion}. & 
Positive wording inspires pursuit motivation; negative wording triggers avoidance or self-protection \cite{lazarus1991emotion}. \\  \hline

\end{tabularx}

\caption{Emotional Dimensions and Multisystem Activation in Text Design Elements}
\label{tab:text_design}
\end{table*}

% \begin{figure*}[hbt!]
% %\setlength{\abovecaptionskip}{-0.1mm}
% \setlength{\intextsep}{10pt plus 2pt minus 2pt}
%     \centering
%     \includegraphics[width=18cm]{figs/text_design.png}
%     \caption{Emotional Dimensions and Multisystem Activation in Text Design Elements.}
%    \vspace{-2mm}
% \label{fig:why}
% \end{figure*}


\subsubsection{Headline}

\begin{wrapfigure}{l}{0.06\textwidth}
  %\begin{center}
  % \vspace{-11pt} % 调整垂直位置
    \includegraphics[width=0.07\textwidth]{figs/icon/headline.png}
  %\end{center}
\end{wrapfigure} 

Headline, as the first point of contact for readers engaging with a text, significantly influence emotional valence and arousal levels. Studies indicate that forward-reference, a common headline design strategy, employs pronouns or phrases to hint at subsequent content (e.g., “This is why” or “What you would never expect”), creating an information gap that arouses readers’ curiosity about the unknown, prompting further reading \cite{blom2015click}. This technique, by leveraging logical incompleteness of information, stimulates exploratory interest and serves as an effective method to enhance emotional arousal. 
% 标题作为读者接触文本的第一入口,其结构和形式对情绪效价和唤起度具有显著影响。研究表明,前向引用作为一种常见的标题设计策略,通过使用代词或短语暗示后续内容(例如“这就是为什么”或“你绝对想不到的事情”),制造信息缺口,从而引发读者对未知信息的好奇心,促使其进一步阅读(Blom & Hansen, 2015)。这种技巧通过逻辑上的信息不完整性激发探索兴趣,是提升情绪唤起度的有效手段。

This process not only affects readers’ cognition but also activates the amygdala and sympathetic nervous system through the neural system, triggering physiological responses such as increased heart rate and pupil dilation, which further enhance tension and focus \cite{panksepp2012archeology, calvo2010affect}. Such physiological activation gives title design a stronger emotional driving force. Additionally, titles such as “Last Chance” or “Miss It, Lose It” may also trigger bodily alertness through the sensorimotor system, such as muscle tension or rapid breathing, aiding users in quickly focusing on information \cite{james1884mind}.
% 这一过程不仅对读者的认知产生影响,还通过神经系统激活杏仁核和交感神经系统,触发心跳加速、瞳孔扩张等生理反应,进一步增强紧张感和注意力集中(Panksepp & Biven, 2012,Calvo & D’Mello, 2010)。这种生理层面的激活使标题设计具有更强的情绪驱动力。此外,类似“最后机会”或“错过即失”之类的标题,还可能通过感官运动系统引发身体警觉反应,例如肌肉紧张或快速呼吸,帮助用户快速聚焦信息(James-Lange, 1884)。

In contrast, suspense-based strategies focus on creating emotional tension and anticipation within headlines, using phrases such as “You’ll regret not seeing this” or “Stunning discoveries shocked everyone” to emphasize importance or surprise and evoke emotional engagement. This design also significantly impacts the cognitive system by guiding users to quickly assess situations (e.g., “Miracle Happens” evokes positive emotions, while “Time is Running Out” evokes anxiety), prompting them to take action \cite{lazarus1991emotion}.
% 与之相比,悬念设置则更注重在标题中营造情绪上的紧张感和期待,例如使用“不看你会后悔”或“惊人的发现震撼了所有人”等语句,通过强调内容的重要性或意外性来刺激情绪参与感。这种设计对认知系统同样产生显著影响,通过引导用户快速判断情境(如“奇迹发生”激发正向情感,“时间紧迫”激发焦虑)促使用户采取行动(Lazarus, 1991)。

The simplicity of a headline and the use of symbols are equally crucial for eliciting readers’ emotional responses. Concise headlines reduce cognitive load, enabling readers to quickly grasp core information and thereby enhancing emotional arousal levels. Special symbols (e.g., parentheses or dashes) further emphasize key information, increasing the attractiveness of headlines \cite{kourogi2015identifying}. Additionally, celebrity effects and trending topics can significantly enhance emotional arousal levels. Incorporating celebrity names or referencing popular events in headlines helps capture attention, evoke emotional reactions, and increase click-through rates \cite{kim2016compete}.
% 标题的简洁性和符号使用对读者情绪反应同样至关重要。简短标题降低认知负荷,使读者能够快速理解核心信息,从而提升情绪唤起水平。而特殊符号(如括号或破折号)则进一步突出重点信息,提高标题的吸引力(Kourogi et al., 2015)。此外,名人效应和热点话题能够显著提升情绪唤起水平。在标题中使用名人姓名或提及热门事件,有助于吸引注意力并激发情感反应,同时提高点击率(Kim et al., 2016)。

However, the use of vague wording and interrogative sentences in title design must be approached with caution. Although they can enhance emotional arousal, excessive use may undermine content credibility and reduce emotional valence \cite{kuiken2017effective}. Therefore, by balancing simplicity, symbol usage, and information clarity, title design can achieve an optimal balance between attracting attention and maintaining trust.
%然而,标题设计中模糊措辞和疑问句的使用需谨慎。尽管它们可以增强情绪唤起,但若使用过多可能影响内容可信度,降低情绪效价(Kuiken et al., 2017)。因此,通过平衡简洁性、符号运用和信息明确性,标题设计能够在吸引注意力与保持信任度之间取得最佳效果。


\subsubsection{Narrative Structure}

\begin{wrapfigure}{l}{0.06\textwidth}
  %\begin{center}
  \vspace{-11pt} % 调整垂直位置
    \includegraphics[width=0.07\textwidth]{figs/icon/narrative_structure.png}
  %\end{center}
\end{wrapfigure} 

Narrative structure significantly influences readers’ emotional experiences by organizing information and plot arrangements. The core of narrative structure lies in designing the unfolding of a story, including setting emotional tone frameworks, controlling emotional rhythms, and employing tension and suspense to shape emotional valence and arousal levels. The emotional tone framework at the beginning lays the overall atmosphere of the story and is a critical component of the narrative structure. For instance, adopting a negative tone as the opening design can add a sense of oppression and tension to the narrative, making subsequent conflicts more striking; whereas a positive tone creates a relaxed and pleasant emotional atmosphere for readers, mitigating emotional tension in later developments \cite{mar2011emotion}. This emotional framework not only influences the story’s overall atmosphere but also modulates the degree of readers’ emotional engagement. Negative-toned scenarios may also activate the sensorimotor system by triggering reactions such as muscle tension or rapid breathing~\cite{james1884mind, panksepp2012archeology}.
% 叙事结构通过对信息和情节的组织安排显著影响读者的情绪体验。叙事结构的核心在于设计故事的展开方式,包括情感基调框架的设定、情绪节奏的控制以及紧张感和悬念的运用,从而塑造情绪效价和情绪唤起水平。开篇的情感基调框架为故事奠定整体氛围,是叙事结构的重要组成部分。例如,采用负面基调作为开篇设计能够为整个叙事增添压抑感和张力,使后续情节的冲突更加引人注目;而正面基调开篇则为读者营造轻松愉悦的情绪氛围,从而缓和后续情节中的情绪紧张感(Mar et al., 2011)。这种情感框架不仅影响故事的整体氛围,还能够调节读者的情感投入程度。负面基调情节还可能通过触发肌肉紧张或呼吸急促等反应,激活感官运动系统(James-Lange, 1884;Panksepp & Biven, 2012)。


Tension and suspense design are crucial methods for enhancing emotional arousal in narrative structures. By gradually revealing crisis situations and delaying information disclosure, narratives can evoke readers’ anticipation and emotional engagement. For example, during the climax or key turning points of a story, creating conflicting tensions can elevate emotional intensity, significantly enhancing the immersive experience of emotional engagement. Additionally, strategically placing high-emotion points (such as suspense or climax scenes) at the beginning, climax, and end of a narrative enriches the emotional rhythm and further strengthens emotional valence \cite{jaaskelainen2020neural}. Suspenseful design not only sustains emotional tension but may also activate the reader’s nervous system, leading to increased heart rate and pupil dilation, further strengthening emotional memory through amygdala activity~\cite{panksepp2012archeology, jaaskelainen2020neural} .
%紧张感与悬念设计是叙事结构提升情绪唤起的重要手段。通过逐步揭示危机情境和延迟信息披露,叙事能够引发读者的期待和情感投入。例如,在故事的高潮或关键转折点,通过制造矛盾冲突来提升情绪强度,可以显著增强情感体验的沉浸感(Jääskeläinen et al., 2020)。悬念设计不仅使情绪紧张感得以延续,还可能激活读者的神经系统,导致心跳加速和瞳孔扩张,并通过杏仁核的活动进一步强化情绪记忆(Panksepp & Biven, 2012;Jääskeläinen et al., 2020)。

Additionally, strategically placing high-emotion nodes (such as suspense or climactic scenes) at the beginning, climax, and conclusion of a narrative enriches the emotional rhythm and further enhances emotional valence \cite{jaaskelainen2020neural}. The control of emotional rhythm is another crucial factor by which narrative structure influences emotional experiences. Stories with high-frequency emotional fluctuations maintain tension by rapidly switching emotional states (e.g., alternating fear and hope), keeping readers highly engaged. However, excessively frequent fluctuations may lead to emotional fatigue, reducing readers’ emotional engagement \cite{lekkas2022using}. Conversely, narratives with more stable emotional rhythms are better suited to guiding audiences into sustained emotional states; for example, heartwarming stories maintain a consistent emotional tone, allowing readers to feel comfortable and relaxed throughout the reading process \cite{jaaskelainen2020neural}. This control of emotional rhythm not only enhances the narrative’s immersive quality but also prevents emotional overload or~fatigue.
% 此外,在叙事开端、高潮和结尾合理分布高情绪节点(如悬念或高潮情节),能够使情感节奏更为丰富,并进一步强化情绪效价。% 情绪节奏的控制是叙事结构影响情绪体验的另一重要因素。高频情绪波动的故事通过情感状态的快速切换(如恐惧与希望交替)来保持紧张感,使读者始终保持高度关注。然而,过于频繁的波动可能导致情绪疲劳,削弱读者的情感参与度(Lekkas et al., 2022)。相反,情绪节奏较为稳定的叙事结构更适合引导观众逐步进入持续的情感状态,例如温情类故事通过一致的情感基调使观众在整个阅读过程中始终感到舒适与放松(Jääskeläinen et al., 2020)。这种情绪节奏的控制不仅有助于增强叙事的沉浸感,还能够避免情绪体验的过载或疲惫。

% In contrast, non-narrative structures are more direct, rapidly eliciting emotions through linear information delivery, particularly in content involving negative emotions, which can quickly provoke tension or fear in readers. For instance, breaking news reports often focus on negative emotions, triggering defensive reactions or action motivations among the audience \cite{leshner2018breast}. Such defensive reactions not only prompt audiences to quickly evaluate risks and threats, thereby activating the cognitive system \cite{lazarus1991emotion}, but also trigger avoidance motivation by emphasizing the urgency of threats, leading users to take practical actions to avoid risks or address issues, thereby activating the motivational system \cite{calvo2010affect}.

%相比之下,非叙事结构则更加直接,通过信息的线性传达迅速激发情绪,特别是在涉及负面情绪内容时,这种结构能够快速引发读者的紧张或恐惧情绪。例如,突发新闻的直接报道常常以负面情感作为重点,激发受众的防御反应或行动动机(Leshner et al., 2018)。这种防御反应不仅促使受众快速评估风险与威胁,从而激活认知系统(Lazarus, 1991),还通过强调威胁的紧迫性触发规避动机,使用户倾向于采取规避风险或应对问题的实际行为,从而激活动机系统(Calvo & D’Mello, 2010)。




\subsubsection{Narrative Content}

\begin{wrapfigure}{l}{0.06\textwidth}
  %\begin{center}
  \vspace{-11pt} % 调整垂直位置
       \includegraphics[width=0.07\textwidth]{figs/icon/narrative_content.png}
  %\end{center}
\end{wrapfigure} 

Narrative content profoundly influences audience emotions through the selection of specific plot elements, character development, and emotional tension control. Different plot types directly shape emotional valence; for example, negative emotional content (such as crises, pain, or loss) often evokes sadness and sympathy, while positive emotional content (such as recovery, victory, or hope) enhances positive emotional valence, making the audience feel optimistic and reassured \cite{mar2011emotion, lekkas2022using}. This influence on emotional valence not only affects the cognitive level but may also trigger noticeable physiological reactions. For example, scenes depicting fear or tension often lead to bodily mimicry responses, such as muscle tension or rapid breathing, thereby activating the sensorimotor system \cite{james1884mind, panksepp2012archeology}.
% 叙事内容通过具体情节的选择、角色塑造和情感张力的控制,对受众的情绪产生深远影响。不同情节类型能够直接塑造情绪效价,例如,负面情绪内容(如危机、痛苦或失落)往往引发悲伤、同情等负面情绪;而正面情绪内容(如康复、胜利或希望)则增强积极情绪效价,使观众感到乐观和安心(Mar et al., 2011;Lekkas et al., 2022)。
% 这种对情绪效价的影响不仅作用于认知层面,还可能引发明显的生理反应,例如描述恐惧或紧张的场景往往会导致身体模仿反应,如肌肉紧张或呼吸急促,从而激活感官运动系统(James-Lange, 1884;Panksepp & Biven, 2012)。

Character development and emotional resonance are important means of deepening emotional experiences in narrative content. When emotional depictions of characters in a narrative are authentic and detailed, the audience is more likely to experience psychological resonance, enter the emotional world of the characters, and further enhance emotional arousal. For instance, plot designs in which characters encounter and gradually overcome challenges can lead the audience to empathize and form deep emotional identification with the characters’ experiences. Additionally, personal associations triggered by the plot content can intensify emotional responses. When the audience connects the story to their own experiences, this similarity can significantly amplify emotional valence, making the emotional experience more~profound \cite{mar2011emotion, nguyen2014affective}. This process activates the audience's cognitive system, which evaluates the reasonableness of the characters' actions within the context, further modulating emotional responses. %The formation of such emotional resonance relies on the cognitive system, where readers evaluate events by analyzing the rationality of characters’ behaviors and contexts \cite{lazarus1991emotion, }.
% 角色塑造和情感共鸣是叙事内容深化情绪体验的重要途径。当叙事中的角色情感描写真实且细腻时,受众更容易产生心理共鸣,进入角色的情感世界,进一步增强情绪唤起水平。例如,故事中的角色经历困境并逐步解决的情节设计,能够引导受众感同身受,并对角色的情感体验形成深刻认同。同时,情节内容引发的个人联想也会强化情绪反应。当受众将故事与自身经历联系起来时,这种相似性能够显著放大情绪效价,使情感体验更加深刻(Mar et al., 2011)。这种情感共鸣的形成依赖于认知系统的作用,读者在这一过程中通过分析角色行为与情境的合理性,对事件进行评估(Lazarus, 1991;Nguyen et al., 2014)。这一过程激活了受众的认知系统,通过评估角色行为与情境的合理性,进一步调节情绪反应。

The regulation of plot intensity and emotional tension directly influences the depth and duration of emotional arousal. For example, intense negative emotional scenarios (such as loss, fear, or crisis) often trigger high emotional arousal \cite{jaaskelainen2020neural}, activating the amygdala and the sympathetic nervous system in the neural network, leading to adrenaline secretion and maintaining the audience's focused attention \cite{panksepp2012archeology, jaaskelainen2020neural}. This physiological response not only intensifies the emotional experience but also extends the emotional memory, leaving a profound impression of the storyline on the audience. On this basis, the design of complex emotional plots (such as the intertwining of pain and hope) enhances the depth of emotional layers, allowing the audience to experience deeper resonance through the emotional highs and lows \cite{leshner2018breast}. This display of emotional complexity not only enables the audience to experience richer emotions but also stimulates achievement motivation, further influencing their behaviors and emotional responses \cite{mar2011emotion, campbell2008hero}.
% 情节强度和情感张力的调控则直接影响情绪唤起的深度和持续性。例如,强烈的负面情绪情节(如失去、恐惧或危机)通常会引发高情绪唤起,激活神经系统中的杏仁核和交感神经系统,引发肾上腺素分泌并保持受众的注意力集中(Panksepp & Biven, 2012;Jääskeläinen et al., 2020)。这种生理反应不仅增加情绪体验的强度,还能延续情绪的记忆,促使受众对故事情节保持深刻印象。在此基础上,复杂情绪的情节设计(如痛苦与希望交织)增强了情感的层次感,使观众在情感的起伏中获得更深刻的共鸣(Leshner et al., 2018)。这种情感复杂性的展示,不仅使观众体验更为丰富的情绪,还能激发成就动机,进一步影响受众的行为和情绪反应(Mar et al., 2011;Campbell, 1949)。


% The regulation of plot intensity and emotional tension directly influences the depth and persistence of emotional arousal. For instance, intense negative emotional scenarios (such as loss, fear, or crisis) often provoke high emotional arousal, activating the amygdala and sympathetic nervous system, triggering adrenaline secretion and maintaining the audience’s focus \cite{panksepp2012archeology, jaaskelainen2020neural}. This neural activation mechanism not only amplifies the intensity of emotional experiences but also prolongs the emotional afterglow, leaving the audience vividly recalling the content even after the story ends.
% 情节强度和情感张力的调控则直接影响情绪唤起的深度和持续性。例如,强烈的负面情绪情节(如失去、恐惧或危机)通常会引发高情绪唤起,激活神经系统中的杏仁核和交感神经系统,引发肾上腺素分泌并保持受众的注意力集中(Panksepp & Biven, 2012;Jääskeläinen et al., 2020)。这种神经系统的激活机制不仅增强了情绪体验的强度,还能够延续情绪余韵,使观众在故事结束后仍对内容记忆犹新。

% Meanwhile, the complex emotional design in narratives (such as the interplay of pain and hope) can further enhance emotional engagement through the characters’ efforts and successes \cite{leshner2018breast}. This expression of complex emotions not only enriches the audience’s emotional experience but also activates the motivational system, particularly achievement motivation, through the characters’ accomplishments. For instance, audiences may project the characters’ successes onto their own goals, thereby reinforcing their own behavioral drive \cite{mar2011emotion, campbell2008hero}. Through such design, the impact of narrative content transcends the text itself, creating a more enduring resonance in the audience’s emotions and~behaviors.
% % 与此同时,叙事中的复杂情感设计(如痛苦与希望交织)可以通过角色的努力与成功,进一步强化情感参与。这种复杂情绪的表达不仅能够丰富受众的情感体验,还能够通过角色的成就激发动机系统,尤其是成就动机。例如,观众可能将角色的成功投射到自身目标中,从而强化对自身行为的驱动力(Mar et al., 2011;Campbell, 1949)。通过这样的设计,叙事内容的影响得以超越文本本身,使其在受众的情绪和行为中产生更持久的共鸣。

\subsubsection{Wording} %措辞/表达风格
\begin{wrapfigure}{l}{0.06\textwidth}
  %\begin{center}
  \vspace{-11pt} % 调整垂直位置
    \includegraphics[width=0.07\textwidth]{figs/icon/wording.png}
  %\end{center}
\end{wrapfigure} 

Wording has a profound impact on emotional valence, arousal, and dominance in aspects such as information framing, word choice, and language style. Through careful wording design, the text can not only directly influence the audience's emotional responses but also further amplify the emotional experience through the nervous system, sensorimotor system, and cognitive system.
The choice of information framing is particularly important in shaping emotional valence. Gain framing (e.g., "Taking this measure can improve health") typically enhances positive emotions, making the audience feel optimistic and positive; whereas loss framing (e.g., "If no action is taken, health will deteriorate") is more likely to evoke negative emotions such as anxiety and tension \cite{seo2019process}. These changes in emotional valence not only affect the psychological level but may also involve physiological responses, such as physical tension, activating the sensorimotor system \cite{panksepp2012archeology}. Research shows that the arousal effect of negative emotions is more enduring; even after the emotional intensity subsides, negative impressions may persist for a long time; whereas positive emotions tend to fade quickly over time \cite{ludwig2013more}.
% 措辞在信息框架、词汇选择和语言风格等方面对情绪效价、唤起度和支配度有深刻影响。通过恰当的措辞设计,文本不仅能够直接影响受众的情绪反应,还能通过神经系统、感官运动系统和认知系统进一步强化情绪体验。信息框架的选择对情绪效价的塑造尤为重要。增益框架(如“采取这一措施可以改善健康”)通常提升正向情绪,令受众感到积极与乐观;而损失框架(如“如果不采取行动,健康状况将恶化”)则更容易引发焦虑、紧张等负面情绪(Seo & Dillard, 2019)。这种情绪效价的变化不仅停留在心理层面,还可能伴随身体紧张等生理反应,激活感官运动系统(Panksepp & Biven, 2012)。研究表明,负面情绪的唤起效果更持久,即使情绪强度减弱后,负面印象仍可能长期保留;而正面情绪则往往随时间快速衰减(Ludwig et al., 2013)。

The choice of words in phrasing directly affects emotional responses and the motivational system. Positive words (e.g., "happiness," "hope") can evoke positive emotions, making the audience feel pleasant and reassured, while also stimulating exploratory motivation, prompting them to take proactive actions; whereas negative words (e.g., "pain," "loss") intensify emotional tension, eliciting sadness or anxiety, and are more likely to trigger avoidance motivation, leading the audience to prefer risk avoidance or self-protection \cite{lee2020impact, calvo2010affect}. Additionally, mild words (e.g., "challenge," "support") can moderate negative emotional valence to some extent, preventing excessive emotionality; whereas intense words (e.g., "loneliness," "fear") amplify emotional arousal effects, significantly enhancing the emotional impact of the text \cite{nguyen2014affective}. In this process, the audience evaluates these words contextually through their cognitive system, thereby modulating their emotional responses and exhibiting different motivational tendencies in behavioral decision-making \cite{lazarus1991emotion}.
% 措辞中词汇的选择直接影响情绪反应与动机系统。正面词汇(如“幸福”“希望”)能够激发积极情绪,让受众感到愉悦与安心,同时激发探索动机,促使受众采取积极行动;而负面词汇(如“痛苦”“失落”)则增强情绪张力,引发悲伤或紧张情绪,并更容易触发规避动机,使受众倾向于规避风险或保护自身利益(Lee & Potter, 2020;Calvo & D’Mello, 2010)。此外,温和词汇(如“挑战”“支持”)能够在一定程度上缓解负面情绪效价,避免过度情绪化;而强烈词汇(如“孤独”“恐惧”)则通过放大情绪唤起效果,显著增强文本的情感冲击力(Nguyen et al., 2014)。在这一过程中,受众通过认知系统对这些词汇进行情境评估,从而调节自身的情绪反应,并在行为决策中表现出不同的动机倾向(Lazarus, 1991)。

Language style plays a crucial role in enhancing the depth of emotional expression. Parallel structures (such as repetition and symmetrical sentence patterns) can enhance the rhythm and aesthetic appeal of the text, making emotional expression more attractive and impactful, thereby enhancing positive emotional valence and prolonging emotional persistence \cite{menninghaus2017emotional}. Additionally, phonetic characteristics subtly influence emotional arousal. For instance, abrupt phonemes (such as "explosion" or "roar") tend to evoke tension and alertness, whereas smooth phonemes (such as "gentle" or "whisper") are better suited to conveying soothing and calm emotions \cite{slavova2019towards}. These linguistic features further enhance the audience's emotional experience by activating the sensorimotor and nervous systems.
% 语言风格在增强情感表达深度方面具有重要作用。平行结构(如重复和对称句式)能够强化文本的节奏感与美感,使情感表达更具吸引力和感染力,从而提升正向情绪效价,并延长情绪的持久性(Menninghaus et al., 2017)。此外,音素特性对情绪唤起也有潜移默化的影响。例如,突发型音素(如“爆炸”或“轰鸣”)易引发紧张和警觉情绪,而流畅型音素(如“轻柔”或“呢喃”)则更适合传递舒缓与平静的情感(Slavova, 2019)。这些语言特征通过激活感官运动系统和神经系统,进一步强化受众的情感体验。

Notably, the effectiveness of wording is directly related to the degree of alignment with the audience's cultural background in terms of emotional valence and arousal effects. When the language style of the text aligns with the audience's cultural practices and community norms, the audience is more likely to experience emotional resonance, thereby enhancing the text's appeal and credibility; conversely, language expressions that deviate from the cultural context may result in emotional detachment, weakening the text's emotional impact and resonance \cite{ludwig2013more}.
% 值得注意的是,措辞效果与受众文化背景的契合程度直接关系到情绪效价和唤起效果。当文本的语言风格与受众的文化习惯和社群规范相符时,受众更容易产生情感认同,从而增强文本的吸引力和可信度;相反,脱离文化语境的语言表达可能导致情感距离感,削弱文本的情绪感染力和共鸣效果(Ludwig et al., 2013)。

% \subsubsection{Summary}
% Text design demonstrates multidimensional influence in regulating emotional valence, arousal, and sense of dominance. Through the interplay of elements such as titles, narrative structure, narrative content, and wording, text design can flexibly adapt to the emotional needs of different communication contexts, guiding readers to experience target emotions during reading.
% % 文本设计在调节情绪效价、唤起度和支配感方面展现了多维度的影响力,借助标题、叙事结构、叙事内容和措辞等要素的交互作用,文本设计能够灵活适应不同传播情境下的情绪需求,并引导读者在阅读中产生目标情绪体验。

% In scenarios of \textbf{high valence and arousal}, text design often aims to evoke positive emotions and high energy. In the classic novel The Count of Monte Cristo, the climactic moment of the protagonist’s successful revenge portrays an exhilarating victory of justice over evil, and this narrative design effectively enhances emotional valence and arousal levels. Similarly, in inspirational speeches or public service campaigns, titles often employ suspenseful or call-to-action language, such as “Join us to create a better future” or “You too can change the world,” creating an information gap to stimulate readers’ exploratory motivation \cite{blom2015click}. The narrative content focuses on victories in adversity, telling stories of protagonists overcoming challenges and ultimately achieving success, evoking excitement and resonance in readers \cite{leshner2018breast, jaaskelainen2020neural}. In terms of wording, positive and emotionally powerful words such as “breakthrough,” “miracle,” and “achievement” are used, while incorporating a few negative words (e.g., “narrow victory” or “crisis”) to heighten emotional tension and enhance arousal levels \cite{seo2019process}. This design activates readers’ amygdala and reward centers, not only enhancing attention and emotional memory, but also increasing their sense of dominance, making them feel empowered and capable of taking action.
% % 在高情绪效价与高情绪唤起的场景中,文本设计通常以激发积极情绪和高能量为目标。经典小说《基督山伯爵》中,主人公复仇成功的高潮情节,展现了正义战胜邪恶的激动时刻,这种情节设计有效提升了情绪效价和唤起水平。同样地,在鼓舞人心的演讲稿或公益宣传中,标题通常采用悬念或号召性语言,如“加入我们,共创美好未来”或“你也可以改变世界”,通过制造信息缺口激发读者的探索动机(Blom & Hansen, 2015)。叙事内容则聚焦于逆境中的胜利,通过讲述主人公克服困难并最终取得成就的故事,让读者感受到兴奋和共鸣(Leshner et al., 2018;Jääskeläinen et al., 2020)。措辞方面使用积极且感染力强的正面词汇,如“突破”“奇迹”“成就”,同时加入少量负面词汇(如“险胜”或“危机”)增强情感张力,从而提高情绪唤起水平(Seo & Dillard, 2019)。这种设计通过激活读者的杏仁核和奖励中枢,不仅提高了注意力和情绪记忆,还增强了读者的支配感,使其感到自身具有掌控和行动的能力。

% In scenarios of \textbf{high valence and low arousal}, text design focuses on conveying calm positive emotions, avoiding emotional fluctuations. For example, in Pride and Prejudice, the ending where Elizabeth and Darcy overcome misunderstandings and live together, uses a stable narrative structure and gentle language to depict a happy daily life, bringing readers a sense of peaceful satisfaction. Titles are typically simple and clear, such as “A Happy Daily Life Begins Here” or “Experience the Beauty of Life,” avoiding high-arousal rhetoric (Lekkas et al., 2022). The narrative content focuses on depicting small moments of beauty in daily life, using delicate emotional descriptions to create a tranquil atmosphere \cite{lekkas2022using}. Wording includes warm and gentle words such as “warmth,” “comfort,” and “contentment,” avoiding any words that might induce tension or high arousal \cite{lee2020impact}. This design reduces the intensity of emotional arousal but enhances readers’ sense of dominance by creating a sense of affinity, allowing them to gain emotional satisfaction in a stable emotional state.

% %在高情绪效价与低情绪唤起的场景中,文本设计侧重于传递平和的正面情绪,避免引发情绪波动。例如,《傲慢与偏见》中伊丽莎白和达西最终消除误解、共同生活的结局,以稳定的叙事结构和温和的语言描绘幸福的日常,带给读者平静的满足感。标题通常简洁而明确,例如“幸福的日常从这里开始”或“感受生活的美好”,避免使用高唤起的修辞。叙事内容侧重于刻画日常生活中的微小美好,以细腻的情感描写营造宁静氛围(Lekkas et al., 2022)。措辞选用温暖、柔和的词汇,如“温馨”“舒适”“满足”,避免使用任何可能引发紧张或高唤起的词语(Lee & Potter, 2020)。这种设计降低了情绪唤起的强度,但通过营造亲和感提升了读者的支配感,使其在稳定的情绪状态中获得舒适的情感满足。

% In scenarios of \textbf{low valence and high arousal}, text design aims to heighten readers’ tension and focus, often used to convey crisis and unease. For example, in Mary Shelley’s Frankenstein, Victor Frankenstein’s remorse and fear after creating the monster run throughout the main storyline, particularly as the monster’s revenge against humanity unfolds, filling the narrative with tension and unease. Titles of such texts often use loss framing to emphasize potential risks, such as “Ignoring These Signals Could Lead to Irreversible Consequences” or “This Crisis Is Spreading” \cite{seo2019process}. Narrative content depicts crisis scenarios in detail, such as the spread of natural disasters or the worsening of social issues, keeping readers tense and evoking defensive reactions \cite{jaaskelainen2020neural}. The wording heavily employs negative vocabulary (e.g., “danger,” “out of control,” “predicament”), supplemented by abrupt phonemes (e.g., “alarm” or “thunderous collapse”) to intensify emotional arousal \cite{slavova2019towards}. Although this design results in overall low emotional valence, it effectively stimulates avoidance motivation, prompting the audience to take responsive actions.

% % 在低情绪效价与高情绪唤起的场景中,文本设计旨在强化读者的紧张感和注意力,常用于表达危机与不安。例如,在玛丽·雪莱的《弗兰肯斯坦》中,维克多·弗兰肯斯坦在创造出怪物后的悔恨与恐惧贯穿了故事主线,尤其是当怪物对人类的复仇逐步展开时,情节充满了紧张与不安。此类文本的标题通常使用损失框架,突出潜在风险,例如“忽视这些信号可能导致不可挽回的后果”或“这场危机正在蔓延”。叙事内容通过详细描绘危机情境,例如自然灾害的扩散或社会问题的恶化,让读者持续保持紧张感,并激发防御性反应(Jääskeläinen et al., 2020)。措辞中大量使用负面词汇(如“危险”“失控”“困境”),并辅以突发音素的表达(如“警报”或“轰然倒塌”)增强情绪唤起(Slavova, 2019)。这种设计虽然整体情绪效价较低,但能够激发规避动机,使受众采取应对行动。

% In scenarios of \textbf{low valence and arousal}, text design aims to maintain a subdued negative emotion, suitable for portraying inevitable realities. For example, in Tolstoy’s Anna Karenina, a calm tone is used to depict Anna’s lonely state of mind following the breakdown of her marriage, creating a low-arousal negative atmosphere. Titles typically convey negative emotions directly, such as “An Irreversible Regret” or “The Indifferent Truth.” Narrative content presents negative information in a straightforward manner, avoiding any emotional fluctuations, such as describing a regrettable social event or an unavoidable life situation \cite{mar2011emotion}. The wording employs mild negative words such as “regret,” “loss,” and “helplessness,” avoiding strong expressions that could provoke emotional fluctuations \cite{lee2020impact}. This design aims to help the audience understand the existence of the issue, but without using emotional tension to compel immediate action, thereby maintaining a low level of arousal.

% % 在低情绪效价与低情绪唤起的场景中,文本设计以维持低调的负面情绪为目标,适用于呈现无可奈何的现实状况。例如,托尔斯泰的《安娜·卡列尼娜》中,通过平静的语调描写安娜在婚姻破裂后的孤独心境,营造了一种低唤起的消极氛围。标题通常直接传递消极情绪,例如“一个无法改变的遗憾”或“冷漠的真相”。叙事内容以直述方式呈现负面信息,避免引入任何情绪波动,例如描述一件令人遗憾的社会事件或无奈的生活境况(Mar et al., 2011)。措辞中选用平和的负面词汇,如“遗憾”“失落”“无助”,避免使用可能引发情绪波动的强烈表达(Lee & Potter, 2020)。这种设计旨在让受众理解问题的存在,但不通过情绪张力强迫其采取立即行动,从而保持较低的唤起水平。

% By integrating elements of text design with the requirements of different emotional combinations, the balance of emotional valence, arousal, and dominance can be effectively regulated. This approach not only accommodates diverse communication needs, but also provides clear guidance for design practices.
% % 通过将文本设计的要素与不同情绪组合的需求相结合,可以有效调节情绪效价、唤起度和支配感的平衡。这种方法不仅能够适应多样化的传播需求,还为设计实践提供了清晰的指导方向。

\subsection{Visual Design}
Visual design is one of the most direct and rapid means of influencing user emotions. It can be broken down into elements such as color, imagery, shape, and layout. The choice of color directly impacts emotional valence and arousal. For instance, warm colors like red and orange often evoke positive emotions, while cool colors like blue and green help create a calm and relaxing atmosphere \cite{plass2014emotional}. Images can significantly influence viewers’ cognitive processes and memory by conveying explicit emotional information. The design of shapes and layouts also plays a critical role. Rounded and soft shapes are often associated with warm and friendly emotional experiences, whereas sharp geometric shapes may evoke tension or alertness \cite{mayer2014benefits}. By combining visual elements effectively, designers can swiftly regulate users’ emotions at the visual~level.
% 视觉设计是影响用户情绪最直观和快速的手段之一。细分为颜色、图片、形状和布局等元素。颜色的选择能够直接影响情感效价和唤起度,例如,暖色系如红色和橙色通常能引发积极情绪,而冷色系如蓝色和绿色则有助于营造平静和放松的氛围【Plass et al., 2014】。图片能够通过传递明确的情感信息显著影响观众的认知过程和记忆效果。形状和布局的设计也起到关键作用。圆润和柔和的形状往往与温暖、友好的情感体验相关,而尖锐的几何形状则可能激发紧张或警觉感【Mayer & Estrella, 2014】。通过合理的视觉元素组合,设计师能够在视觉层面快速调节用户的情绪。

\renewcommand{\arraystretch}{1.8} % 调整行间距
\begin{table*}[ht]
\fontsize{8}{9}\selectfont
\centering

\begin{tabularx}{\textwidth}{|>{\centering\arraybackslash}m{0.5cm}|>{\centering\arraybackslash}m{1.4cm}|>{\centering\arraybackslash}m{3.4cm}|>{\centering\arraybackslash}m{3.4cm}|>{\centering\arraybackslash}m{3.4cm}|>{\centering\arraybackslash}m{3.4cm}|}
\hline
\rowcolor[HTML]{D9EAD3} 
\multicolumn{2}{|c|}{\textbf{Dimension}} & \textbf{Color} & \textbf{Shape} & \textbf{Images} & \textbf{Layout} \\ \hline

% Emotional Dimensions Section
\multirow{3}{*}{\rotatebox{90}{\parbox{3cm}{\centering \textbf{Emotional \\ Dimensions}}}} & 
\cellcolor[HTML]{FDF6E8} \textbf{Valence} & 
\textbf{Warm colors} (e.g., red, yellow) evoke positive emotions and energy \cite{jonauskaite2019color,kallabis2024investigating}. \textbf{Cool colors} (e.g., blue, green) promote calmness \cite{wilms2018color}. & 
\textbf{Round shapes} convey friendliness and warmth \cite{wei2006image}. \textbf{Sharp shapes} evoke alertness \cite{thumfart2008modeling}. & 
\textbf{Positive images} (e.g., nature, smiling faces) enhance positive emotions \cite{hou2024emotional}. \textbf{Negative images} (e.g., disasters) amplify negative emotions \cite{pfeuffer122measuring}. & 
\textbf{Simple layouts} reduce distractions, enhancing positive emotions \cite{lu2017investigation}. \textbf{Symmetrical layouts} evoke balance and trust \cite{fiorini2024role}. \\ \cline{2-6}

& \cellcolor[HTML]{FDF6E8} \textbf{Arousal} & 
\textbf{Warm colors} increase arousal (excitement), while \textbf{cool colors} reduce it (relaxation) \cite{jonauskaite2019color}. & 
\textbf{Complex shapes} heighten arousal \cite{lu2012shape}. 
\textbf{Rounded shapes} are calming \cite{etzi2016arousing}. & 
\textbf{High-intensity images} (e.g., emergencies) increase arousal \cite{pfeuffer122measuring}. \textbf{Peaceful images} reduce arousal \cite{hao2024judging}. & 
\textbf{Complex layouts} boost exploration and arousal \cite{carretie2019emomadrid}. \textbf{Dynamic layouts} add vitality \cite{lu2020exploring}. \\ \cline{2-6}

& \cellcolor[HTML]{FDF6E8} \textbf{Dominance} & 
\textbf{Warm, bright colors} (e.g., light yellow) enhance control, while \textbf{low-brightness cool} colors reduce it \cite{weijs2023effects}. & 
\textbf{Circular shapes} enhance control and safety \cite{lu2012shape}. \textbf{Sharp shapes} decrease control \cite{wei2006image}. & 
\textbf{Wide images} (e.g., 16:9 ratio) enhance control \cite{kuzinas2024creative}. \textbf{Isolated subjects} reduce control \cite{lin2023effect}. & 
\textbf{Rule-of-thirds layouts} enhance control \cite{machajdik2010affective}. 
\textbf{Shallow depth layouts} focus attention and increase control \cite{datta2006studying}. \\ \hline

% Multisystem Activation Section
\multirow{4}{*}{\rotatebox{90}{\parbox{3cm}{\centering \textbf{Multisystem \\ Activation}}}} & 
\cellcolor[HTML]{F0EFF7} \textbf{Neural Systems} & 
\textbf{Colors} activate the sympathetic or parasympathetic systems \cite{ledoux1998emotional}. & 
\textbf{Rounded shapes} relax the nervous system, while \textbf{sharp shapes} induce alertness  \cite{ledoux1998emotional}. & 
\textbf{Emotional images }activate the amygdala, affecting emotions \cite{phelps1998specifying}. & 
\textbf{Simple layouts} induce calm; \textbf{complex layouts} activate stress responses \cite{ledoux1998emotional}. \\ \cline{2-6}

&\cellcolor[HTML]{F0EFF7} \textbf{Sensorimotor Systems} & 
\textbf{Color contrast }affects sensory responses (e.g., pupil dilation, muscle tension) \cite{zajonc1980feeling}. & 
\textbf{Rounded shapes} feel safe; \textbf{sharp shapes} induce tension \cite{lidwell2010universal}. & 
\textbf{Emotional images} trigger physical reactions (e.g., smiling or frowning)\cite{ekman1992argument}. & 
\textbf{Symmetrical layouts} evoke harmony; \textbf{asymmetrical} ones may cause unease \cite{machajdik2010affective}. \\ \cline{2-6}

& \cellcolor[HTML]{F0EFF7} \textbf{Cognitive Systems} & 
\textbf{Colors} influence cognitive assessment (e.g., red for danger, green for relaxation) \cite{zajonc1980feeling}. & 
Shapes impact safety perception (e.g., circles feel safe, sharp shapes signal danger) \cite{lidwell2010universal}. & 
Emotional expressions in images trigger empathy and cognitive responses \cite{hou2024emotional}. & 
\textbf{Symmetrical layouts} provide cognitive ease; \textbf{asymmetrical} ones may confuse \cite{carretie2019emomadrid}. \\ \cline{2-6}

&\cellcolor[HTML]{F0EFF7} \textbf{Motivational Systems} & 
\textbf{Warm colors} (e.g., red, yellow) stimulate urgency and action, while \textbf{cool colors} promote comfort and trust \cite{phelps1998specifying}. & 
\textbf{Rounded shapes }evoke trust; \textbf{sharp shapes} stimulate urgency or challenge \cite{lidwell2010universal}. & 
\textbf{Achievement-oriented images} (e.g., celebrations) inspire motivation \cite{pfeuffer122measuring}. & 
\textbf{Simple layouts} enhance task motivation;\textbf{ complex layouts} increase cognitive load \cite{lu2020exploring}. \\ \hline
\end{tabularx}

\caption{Emotional Dimensions and Multisystem Activation in Visual Design Elements} 
\label{tab:visual_design}
\end{table*}

% \begin{figure*}[hbt!]
% %\setlength{\abovecaptionskip}{-0.1mm}
% \setlength{\intextsep}{10pt plus 2pt minus 2pt}
%     \centering
%     \includegraphics[width=18cm]{figs/Visual_design.png}
%     \caption{Emotional Dimensions and Multisystem Activation in Visual Design Elements.}
%    \vspace{-2mm}
% \label{fig:why}
% \end{figure*}

\subsubsection{Color}
\begin{wrapfigure}{l}{0.06\textwidth}
  %\begin{center}
  \vspace{-11pt} % 调整垂直位置
        \includegraphics[width=0.07\textwidth]{figs/icon/color.png}
  %\end{center}
\end{wrapfigure} 

Color plays a critical role in visual design, with its impact on user emotions primarily reflected in three dimensions: hue, saturation, and brightness. Hue, as the fundamental attribute of color, corresponds to different emotional effects. Studies show that warm colors (e.g., red, orange), due to their high emotional valence and arousal, are commonly used in advertising and entertainment design to convey positive emotions such as joy and excitement \cite{jonauskaite2019color, kallabis2024investigating}. Cool colors (e.g., blue, green) tend to convey calm and comfortable emotions, making them suitable for settings such as healthcare and meditation that aim to soothe emotions \cite{ wilms2018color}. Long-wavelength colors (e.g., red) often enhance emotional arousal, while short-wavelength colors (e.g., blue) are better suited for conveying low-arousal emotional experiences~\cite{wilms2018color}.
Hue not only influences emotional valence but also shapes users’ emotional experiences and behavioral responses through the cognitive and motivational systems. Warm colors (e.g., red, orange), associated with a sense of urgency, are often used in warning signs or time-sensitive designs to quickly capture attention and elicit user alertness and action tendencies \cite{zajonc1980feeling, phelps1998specifying}. Cool colors (e.g., blue, green), associated with comfort and safety, are suitable for conveying trust and relaxation, such as in healthcare or financial interface design \cite{phelps1998specifying}. %This emotional association with hue not only enhances the emotional conveyance of the design but also directly impacts users’ emotional evaluation of the context.
% 颜色在视觉设计中扮演着重要角色,其对用户情绪的影响主要体现在色相、饱和度和亮度三个维度上。色相是颜色的基本属性,不同的色相对应不同的情感效应。研究表明,暖色系(如红色、橙色)因其高情绪效价和唤起度,常用于广告和娱乐设计以传递愉悦、兴奋等正向情绪(Jonauskaite et al., 2019;Kallabis et al., 2024)。冷色系(如蓝色、绿色)则倾向于传递平静与舒适的情绪,适合用于医疗、冥想等安抚情绪的场景(Beekmans & Braun, 2024,Wilms, L., & Oberfeld, D. (2018))。长波长颜色(如红色)常增强情绪唤起,而短波长颜色(如蓝色)更适合传递低唤起的情绪体验(Wilms, L., & Oberfeld, D. (2018))。色相不仅影响情绪效价,还通过认知系统和动机系统塑造用户的情绪体验和行为反应。暖色系(如红、橙)因其与紧迫感的联想,常用于警示标志或强调时效性的设计,能够迅速吸引注意力并激发用户的警觉和行动倾向(Zajonc, 1980;Phelps et al., 1998)。冷色系(如蓝、绿)则以其与舒适和安全感的关联,适合用于传递信任与放松的场景,如医疗或金融界面设计(Phelps et al., 1998)。这种色相的情感联结不仅强化了设计的情感传递效果,还直接影响用户对情境的情感评估。

Saturation describes the purity and intensity of color and significantly affects emotional valence and arousal. Highly saturated colors (e.g., vivid red, bright blue) have striking visual effects that enhance users’ emotional valence and arousal, making them suitable for attention-grabbing design scenarios \cite{lin2023effect}. For example, using highly saturated colors in advertisements or promotional pages can quickly capture users’ attention and stimulate the sensory-motor system, eliciting immediate responses such as pupil dilation or visual focus \cite{zajonc1980feeling}. In contrast, low-saturation colors (e.g., soft bluish-gray) are more inclined to convey calm and serene emotional experiences \cite{pazda2024colorfulness}. They are suitable for interfaces intended for prolonged use, such as reading platforms or educational interfaces \cite{wang2013interpretable}.
% 饱和度描述颜色的纯度和强度,对情绪效价和唤起度有显著影响。高饱和度的颜色(如鲜红、亮蓝)视觉效果鲜明,能够提升用户的情绪效价和唤起度,适用于吸引注意力的设计场景(Lin et al., 2023)。例如,在广告或宣传页面中使用高饱和度的颜色可以迅速吸引用户视线,并刺激感官运动系统产生即时反应,如瞳孔扩张或视觉聚焦(Zajonc, 1980)。相比之下,低饱和度颜色(如柔和的蓝灰色)更倾向于传递平静、冷静的情绪体验(Pazda et al., 2024),适合长时间使用的界面设计,如阅读界面或教育平台([wang et al, 2013])。

Brightness significantly influences the perception of emotional positivity or negativity. High-brightness colors (e.g., light yellow) typically make emotional valence more positive and are used to create a relaxed atmosphere \cite{jonauskaite2019color}. For example, in children’s education or recreational settings, light yellow can stimulate users’ sense of pleasure through the nervous system, making them feel relaxed and at ease \cite{ledoux2000emotion}. Low-brightness tones (e.g., dark gray, deep red) are often used to convey feelings of oppression or tension. For example, in suspense films or horror games, these tones activate the sympathetic nervous system, eliciting tension and high arousal~\cite{weijs2023effects}.
% 亮度显著影响情绪的正负向感知。高亮度颜色(如浅黄色)通常使情绪效价更积极,用于营造轻松氛围(Jonauskaite et al., 2019);例如,在儿童教育或休闲场景中,浅黄色能够通过神经系统激发用户的愉悦感,使其感到轻松和放松(LeDoux, 1996)。低亮度色调(如深灰、暗红)则常用于传递压抑或紧张情绪的场景,例如悬疑电影或恐怖游戏,通过激活交感神经系统引发紧张感和高唤起水平(Weijs et al., 2023)。

By comprehensively regulating hue, saturation, and brightness, designers can precisely adjust users’ emotional valence and arousal levels. Moreover, the physiological and cognitive effects of color, mediated by the sensory-motor and nervous systems, further drive users to take specific actions under the influence of the motivational system. This systematic approach to color design helps optimize user experience, aligning it more closely with design objectives.
% 通过对色相、饱和度和亮度的综合调控,设计师能够精准调节用户的情绪效价和唤起水平。此外,颜色的生理与认知影响通过感官运动系统和神经系统的介导作用,在动机系统的驱动下,进一步促使用户采取特定的行为。这种系统性的颜色设计有助于优化用户体验,使其更加符合设计目标。


\subsubsection{Image}
\begin{wrapfigure}{l}{0.06\textwidth}
  %\begin{center}
  \vspace{-11pt} % 调整垂直位置
        \includegraphics[width=0.07\textwidth]{figs/icon/image.png}
  %\end{center}
\end{wrapfigure} 
Images in visual design not only convey emotional information but also influence users’ cognition and memory \cite{hanson2014happy, xie2017negative, van2015good}. Designers regulate users’ emotional experiences and evoke multi-layered emotional responses through image content, social cues, and presentation methods.
% 图片在视觉设计中不仅传递情感信息,还影响用户的认知和记忆效果(Hanson & II, 2014; Xie & Zhang, 2017;Van Bergen et al., 2015)。设计师通过图像内容、社交线索和呈现方式,调控用户的情绪体验并激发多层次的情绪反应。

The content of an image determines the direction of emotional conveyance. Positive emotional images (e.g., natural landscapes, smiling faces) enhance users’ feelings of joy and relaxation. Commonly used in advertising and educational contexts, these images strengthen the valence of positive emotions through emotional connections \cite{hou2024emotional}. Highly arousing negative emotional images (e.g., disaster scenes, horror visuals) evoke users’ sense of alertness, increasing their focus on critical information, making them suitable for public service campaigns or news reporting \cite{pfeuffer122measuring}. Challenging images (e.g., extreme sports) stimulate physical tension and the motivational system for risk-taking, while tranquil natural landscapes are more suited to evoke relaxation and calm emotional states \cite{pfeuffer122measuring}.
These images not only trigger immediate physiological responses (e.g., increased heart rate or relaxation) through the sensory-motor system but also activate the nervous and cognitive systems, further enhancing users’ emotional evaluation and behavioral responses to the context \cite{phelps1998specifying, kensinger2007negative}.
% 图像内容决定了情感传递的方向。正面情绪图片(如自然风景、微笑人物)能够提升用户的愉悦与放松感。常见于广告和教育场景,其情感联结增强了积极情绪的效价(Hou & Wang, 2024)。高唤起的负面情绪图片(如灾害场景、恐怖画面)通过激发用户的警觉感,增加其对关键信息的关注度,适用于公益宣传或新闻报道(Pfeuffer et al., 2024)。挑战性图片(如极限运动)能够引发身体紧张和冒险的动机系统,而轻松的自然风景图像则更适合激发放松与平静的情感状态(Pfeuffer et al., 2024)。这些图像不仅通过感官运动系统引发即时的生理反应(如心跳加速或放松),还激活神经系统和认知系统,进一步强化用户对情境的情感评估和行为反应(Ekman, 1971;Phelps et al., 1998,Kensinger(2007))。

Social cues play an important regulatory role in the emotional transmission of images. Facial expressions, body language, and interactive contexts in images can significantly enhance users’ emotional resonance. For example, positive social cues such as smiling faces can partially mitigate users’ negative emotional responses even in adverse contexts, thereby enhancing emotional valence \cite{dudarev2024social}. In group images, the overall emotional valence often outweighs the influence of individual emotions. For instance, images where most group members display smiling expressions typically elicit stronger positive emotional reactions, while the influence of a single individual showing extreme negative emotions is relatively limited \cite{hao2024judging}. Facial expressions in images not only trigger users’ emotional evaluations of the context (e.g., sympathy or anger) through the cognitive system but also activate the motivational system. For instance, collective emotions in celebratory scenes can inspire users’ motivation to engage in social interactions \cite{hou2024emotional}. The use of such social cues makes images a powerful tool for enhancing emotional valence and the dominance.
% 社交线索在图片中的情感传递中起到重要调节作用。图片中的人物面孔、肢体语言和互动情境能够显著提升用户的情感共鸣。例如,微笑面孔等正面社交线索,即使在负面情境下也能够部分缓解用户的消极情绪反应,从而提高情绪效价【Dudarev et al., 2024】。在群体图像中,整体的情绪效价往往超越单个个体的情绪影响。例如,多数群体成员展现微笑表情的图像,通常会激发更高的积极情绪反应,而单个表现出极端负面情绪的个体影响相对有限(Hou & Wang, 2024)。图片中的面部表情不仅通过认知系统引发用户对情境的情感评估(如同情或愤怒),还能够激发动机系统,例如庆祝场景的集体情绪能够激发用户参与社交或互动的动机(Hou & Wang, 2024)。这种社交线索的使用,使图片成为强化情绪效价与支配感的有力工具。

Visual presentation methods further amplify the emotional impact of images. Research has shown that wide-format images (e.g., 16:9 aspect ratio) enhance users’ dominance and visual pleasure through their familiarity \cite{kuzinas2024creative}. Moreover, by adjusting the composition ratio and visual focus of an image, designers can highlight key information and reduce visual distractions. For instance, images with shallow depth of field emphasize the subject by blurring the background, which helps reduce visual distractions, enhances users’ focus on key information, and stimulates the cognitive system for faster information processing~\cite{datta2006studying}.
% % 视觉呈现方式进一步增强了图片的情感效果。研究发现,16:9 的纵横比因其广泛应用于现代媒体而成为用户的视觉惯性选择,这种比例的图片更易激发用户的熟悉感和认同感,从而提升情绪效价【Kuzinas et al., 2024】。此外,通过调整图像的构图比例和视觉焦点,设计师可以突出关键信息并减少视觉干扰。例如,低景深的图像通过模糊背景突出主体,帮助用户专注于关键信息,同时营造一种沉浸和平静的氛围。

By integrating content, social cues, and visual presentation methods, image design can precisely regulate users’ emotional valence, arousal, and dominance. Through multi-layered emotional systems, it can also stimulate specific behavioral motivations, playing a critical role in achieving design objectives.
% 通过内容、社交线索和视觉呈现方式的综合运用,图片设计能够精准调节用户的情绪效价、唤起度和支配感,并通过多层次的情绪系统激发特定的行为动机,使其在设计目标中发挥关键作用。

\subsubsection{Shape}
\begin{wrapfigure}{l}{0.06\textwidth}
  %\begin{center}
  \vspace{-11pt} % 调整垂直位置
        \includegraphics[width=0.07\textwidth]{figs/icon/shape.png}
  %\end{center}
\end{wrapfigure} 
Shape is a fundamental element in visual design, profoundly influencing users’ emotional valence, arousal, and dominance through its geometric characteristics.
% 形状是视觉设计中的核心元素,通过其几何特性对用户的情绪效价、唤起度和支配感产生深远影响。

The geometric properties of shapes play a critical role in emotional conveyance. Circular shapes and soft curves often convey feelings of friendliness and safety, enhancing positive valence and comfort \cite{lu2012shape}. For example, circular buttons and icons with soft edges in children’s education or healthcare interfaces can reduce anxiety and enhance trust through the nervous system \cite{ledoux2000emotion}. In contrast, sharp shapes, with their angular features, convey alertness and tension, amplifying negative valence and eliciting high emotional arousal. They are suitable for traffic warning signs or designs that require attention \cite{thumfart2008modeling}. Complex or irregular shapes enhance tension and psychological conflict through their instability, commonly used in thriller movie promotions or suspenseful content design. Their intense visual stimulation directly activates the sympathetic nervous system, inducing physiological tension responses \cite{ebe2015emotion}.
%  形状的几何特性对情感传递起到关键作用。圆形和柔和的曲线通常传递、友好与安全的情感,能够增强正效价和舒适感(Lu et al., 2012)。例如,儿童教育或医疗界面中的圆形按钮和柔和边缘的图标,能够通过神经系统降低焦虑感并增强信任感(LeDoux, 1996)。相比之下,尖锐形状[2]以其棱角分明的特性传递警觉性和紧张感,强化负面效价并引发较高的情绪唤起,适用于交通警示标志或需要吸引注意力的设计场景(Thumfart et al., 2008)。复杂或不规则形状则通过其不稳定性增强紧张感和心理冲突,在惊悚电影宣传或悬疑内容设计中尤为常见,其强烈的视觉刺激直接激活交感神经系统,引发生理紧张反应(Ebe & Umemuro, 2015)。

The combination of shapes and emotional symbols further enriches the emotional experience. Studies have shown that anthropomorphic circles (e.g., smiley icons) can quickly evoke users’ positive emotions and enhance interactive experiences, significantly boosting user engagement through positive emotional valence \cite{mayer2014benefits}. Conversely, frowning faces or sharp-shaped symbols are often used to convey risk or warnings, triggering users’ defensive mechanisms and increasing their attention to the information \cite{ferrara2015measuring}. The use of such symbols can influence users’ emotional evaluations of the context through the cognitive system while activating the motivational system to prompt specific behaviors.
% 形状与情感符号的结合进一步丰富了情绪体验。研究表明,拟人化的圆形(如微笑图标)能够迅速激发用户的积极情绪并增强互动体验,其正向情绪效价显著提升了用户的参与感(Mayer & Estrella, 2014)。相反,哭脸或尖锐的形状符号则常用于传递风险或警告,触发用户的防御机制并增强对信息的关注度(Ferrara & Yang, 2015)。这种符号的使用能够通过认知系统影响用户对情境的情感评估,同时激活动机系统,促使用户采取特定的行为。

% In contrast, sharp shapes with angular features convey alertness and tension, amplifying visual stimuli to elicit heightened emotional arousal. They are suited for scenarios requiring attention or communicating risk. For instance, traffic warning signs and safety labels trigger defensive responses rapidly \cite{thumfart2008modeling}. Complex shapes also play a distinct role in emotional regulation. Simple shapes convey reliability and low-arousal emotions, while complex or irregular shapes (e.g., polygons, dynamic patterns) evoke tension and unease. These visual characteristics are effective in thriller promotions or suspenseful content, simulating unstable effects to heighten tension \cite{ebe2015emotion}. By carefully selecting and combining geometric properties, designers can modulate emotional experiences to suit various design needs.
% %与此相对,尖锐形状以其棱角分明的特性传递警觉性和紧张感,通过增强视觉刺激,引发较高的情绪唤起,适用于需要吸引注意力或传递风险的场景,如交通警示标志和安全标识,这些设计能够迅速激发用户的防御反应【Thumfart et al., 2008】。此外,复杂形状也在情绪调节中发挥独特作用。简单规则的形状传递可靠性和低唤起情绪,而复杂或不规则形状(如多边形或动态图案)则容易引发紧张与不安情绪。这类视觉特性在惊悚电影的宣传设计或悬疑内容的表达中尤为有效,通过模拟不稳定的视觉效果增强用户的心理紧张感【Ebe & Umemuro, 2015】。通过合理选择和搭配形状的几何特性,设计师能够有效调控用户的情绪体验,以适应不同设计场景的需求。

% Moreover, integrating shapes with emotional symbols can further enhance emotional experiences. Studies indicate that endowing shapes with anthropomorphic traits (e.g., smiling circular icons) can quickly evoke positive emotions in users and increase their affinity for products or information. For example, smiley emojis on social media enhance users’ interaction rates by leveraging their positive emotional valence \cite{mayer2014benefits}. In contrast, negative symbols (e.g., crying faces) are employed to convey negative information, directing users’ attention to potential problems or risks.
% % 此外,将形状与情感符号结合能够进一步增强情感体验。研究表明,赋予形状拟人化特征(如微笑的圆形图标)可以快速激发用户的积极情绪,增加对产品或信息的好感。例如,社交媒体中的笑脸表情符号以其正向情绪效价提升了用户的互动率【Mayer & Estrella, 2014】,而负面符号(如哭脸)则用于传达消极信息,引导用户关注潜在问题或风险。

By effectively utilizing the geometric properties of shapes and emotional symbols, designers can holistically regulate users’ emotional valence, arousal, and dominance. This multi-layered design approach not only enhances users’ emotional experiences through feedback from the sensory-motor and nervous systems but also guides users’ behavioral choices via the cognitive and motivational systems. This ensures that the design better aligns with the requirements of its application~context.
% 通过合理运用形状的几何特性和情感符号,设计师能够综合调控用户的情绪效价、唤起度和支配感。这种多层次的设计方法不仅通过感官运动系统和神经系统的反馈增强用户的情绪体验,还通过认知系统和动机系统引导用户的行为选择,使设计更加符合应用场景的需求。

\subsubsection{Layout}
\begin{wrapfigure}{l}{0.06\textwidth}
  %\begin{center}
  \vspace{-11pt} % 调整垂直位置
        \includegraphics[width=0.07\textwidth]{figs/icon/layout.png}
  %\end{center}
\end{wrapfigure} 
Layout directly influences users’ emotional experiences in visual design through spatial arrangement and element organization. Key features such as symmetry, complexity, composition, and depth of field significantly regulate emotional valence and arousal~levels.
% 布局在视觉设计中通过空间安排和元素组织直接影响用户的情绪体验,其对称性、复杂性、构图方式、景深等关键特性能够显著调控情绪效价和情绪唤起水平。

Symmetrical layouts, with their sense of balance and order, reduce users’ cognitive load and create a stable and comfortable emotional experience \cite{makin2012implicit}. This design induces physical comfort responses, such as relaxed eye movements, through the sensory-motor system, while also generating physiological calming effects via the nervous system \cite{ledoux2000emotion}. The visual stability of symmetrical layouts is commonly used in financial or healthcare settings to enhance users’ sense of trust and security. In contrast, asymmetrical layouts, while increasing visual dynamism, may evoke slight feelings of unease, making them suitable for creative settings that aim to stimulate exploratory motivation \cite{carretie2019emomadrid}.
% 对称布局以其均衡感和有序性,降低了用户的认知负荷,营造出稳定与舒适的情绪体验。这种设计通过感官运动系统引发身体上的舒适反应,例如放松的眼部活动,同时通过神经系统产生生理上的平静效果(LeDoux, 1996)。对称布局的视觉稳定性常用于金融或医疗场景,以增强用户的信任感和安全感。相比之下,不对称布局虽然增加了视觉动态感,但可能引发轻微的不安情绪,适合用于创意性场景,激发探索动机。

Complex layouts, with diverse visual elements and rich layering, stimulate users’ curiosity and emotional engagement. This type of layout is commonly used in social media, entertainment apps, or creative display platforms, where intensified visual stimuli enhance emotional arousal levels. However, overly complex layouts may lead to information overload, diminishing users’ dominance and even inducing negative emotions \cite{carretie2019emomadrid}. Thus, complex layouts require a balance between richness and information clarity to avoid diminishing user experience. On the neurological level, complex layouts trigger stress responses, thereby enhancing users’ physiological alertness and focus.
% 复杂布局通过多样化的视觉元素和丰富的层次感,激发用户的探索欲望和情感参与感。这种布局常用于社交媒体、娱乐应用或创意展示平台,能够通过强化视觉刺激提升情绪唤起水平【Carretié et al., 2019】。然而,过度复杂的布局可能导致信息过载,削弱用户的支配感,甚至引发负面情绪。因此,复杂布局需要在丰富性与信息清晰性之间找到平衡,以避免用户体验的下降。在神经系统的层面上,复杂布局会触发应激反应,从而提升用户的生理警觉性和注意力集中程度。

Composition directly influences users’ emotional valence and dominance. The rule of thirds layout, by evenly dividing the frame, creates a harmonious visual effect, enhancing users’ comfort and positive emotional experiences. This composition helps users quickly focus on key information, improving the efficiency of information delivery \cite{mai2011rule}. Circular or curved layouts leverage visual enclosure and flow to capture users’ attention, making them suitable for branding and interactive designs. They enhance emotional engagement to increase user involvement \cite{resnick2003design}. For instance, slanted lines or curved guiding lines enhance the dynamism of the layout while increasing the appeal of key content.
% 构图方式直接影响用户的情绪效价与支配感。三分法布局通过将画面均匀分割,同样创造出了和谐的视觉效果,提升了用户的舒适感与积极情绪体验。这种构图在内容呈现中帮助用户快速聚焦关键信息,提高了信息传递的效率【Mai et al., 2011】。而圆形或弧形布局则利用视觉上的封闭性与流动性吸引用户注意,适用于品牌展示与互动设计,通过增强情感参与感来提高用户的投入度【Resnick, 2003】。例如,倾斜线条或曲线引导线能够提升布局的动态感,同时强化关键内容的吸引力。

Depth of field is an important method in layout design for emphasizing core information and reducing visual distractions. A shallow depth-of-field layout blurs the background and focuses on the subject, reducing interference from the sensory-motor system and enhancing users’ attention \cite{datta2006studying}. For example, shallow depth-of-field image designs on product pages can help users quickly identify core content while activating the nervous system to enhance information retention. Furthermore, this layout directs visual focus to trigger users’ execution motivation, making it suitable for action-oriented design scenarios.
% 景深在布局设计中通过模糊背景并突出主要对象,有助于减少视觉干扰,增强用户对关键信息的专注感。这种设计常见于强调核心内容的场景,如广告页面或产品展示,既提升了用户的理解效率,又增强了信息记忆【Datta et al., 2006】。

Layout design can achieve a dynamic balance among emotional valence, arousal, and dominance by adjusting symmetry, complexity, composition, and depth of field. This multidimensional design approach activates sensory-motor, nervous, cognitive, and motivational systems, satisfying users' emotional needs and driving behavioral responses.	
% 通过对对称性、复杂性、构图方式和景深的综合调控,布局设计能够在情绪效价、唤起度和支配感之间实现动态平衡。这种多维度的设计方法通过激活感官运动系统、神经系统、认知系统和动机系统,在满足用户情感需求的同时,推动用户的行为响应。

% \subsubsection{Summary}
% Visual design profoundly influences users’ emotional valence, arousal level, and sense of control through the integrated use of color, images, shapes, and layouts. In different contexts, design strategies need to be refined based on the target emotional state to achieve precise emotional delivery and effectively guide user behavior.
% % 视觉设计通过颜色、图片、形状和布局的综合运用,对用户的情绪效价、唤起度和支配感产生深远影响。在不同情境中,设计策略需根据目标情绪状态进行精细化调整,从而实现情感的精准传递与用户行为的有效引导。

% In scenarios of \textbf{high valence and high arousal}, visual design leverages the synergy of color, imagery, shapes, and layout to evoke positive emotions and high energy. For instance, in Delacroix’s iconic painting Liberty Leading the People, the vibrant red and golden yellow high-saturation warm tones significantly enhance emotional valence and arousal. These color choices intensify the scene’s tension and emotional engagement, allowing viewers to feel the passion of revolution and the hope of victory \cite{jonauskaite2019color, kallabis2024investigating}. The figure of “Liberty” in the painting, with her dynamic pose of raising a flag and expressive facial features, symbolizes courage and action. This design not only activates the viewer’s sensory-motor system but also strengthens emotional alignment with victory and justice through the cognitive system \cite{kensinger2007negative, hou2024emotional}. Furthermore, the sharp shapes and diagonal layout amplify the painting’s visual impact, with dynamic guidance and a sense of depth directing the viewer’s gaze to the focal area, thereby enhancing attention and emotional memory \cite{wei2006image}. This multi-layered visual design effectively conveys intense emotions and profound meaning by eliciting emotional resonance and cognitive reactions from the audience.
% % 在高情绪效价与高情绪唤起的场景中,视觉设计通过色彩、图像内容、形状和布局的协同作用来激发积极情绪与高能量。例如,德拉克洛瓦的经典画作《自由引导人民》中,鲜艳的红色和金黄色的高饱和度暖色调显著提升了情绪效价和唤起水平,这些色彩选择强化了场景的张力和观众的情感参与,使其感受到革命的激情与胜利的希望(Jonauskaite et al., 2019;Kallabis et al., 2024)。画面中的“自由女神”以高举旗帜的动态姿态和鲜明表情传递出勇气与行动的象征意义,这种设计不仅激发了观众的感官运动系统,还通过认知系统强化了对胜利和正义的情感认同(Kensinger, 2007;Hou & Wang, 2024)。此外,尖锐形状和对角线布局进一步增强了画面的视觉冲击力,动态引导和层次感使观众的视线集中在核心区域,强化了注意力与情感记忆({Wei-Ning et al., 2006,Image retrieval by emotional semantics: A study of emotional space and feature extraction)。这种多层次的视觉设计通过激发观众的情感共鸣与认知反应,有效传递了强烈的情感与深刻的意义。

% In scenarios of \textbf{high valence and low arousal}, visual design employs soft colors, serene imagery, rounded shapes, and symmetrical layouts to create a warm and tranquil emotional atmosphere, helping viewers or users achieve a relaxed emotional state. For instance, John Constable’s The Hay Wain features soft greens and blues as the primary hues, complemented by low-saturation and medium-to-high brightness warm tones, creating a calm and harmonious visual experience \cite{wilms2018color}. The layout of the painting forms a stable composition through the balanced distribution of the river, trees, and houses, reducing visual fatigue for viewers and enhancing the smoothness of the viewing experience \cite{lu2020exploring}. Rounded shapes and subtle gradations of light and shadow further diminish visual intensity; for example, the gentle curves of trees and grass naturally alleviate any tension in the scene \cite{lu2012shape}. These design elements not only regulate viewers’ neural system responses through low-arousal visual characteristics but also enhance peace and pleasure by reducing cognitive load and visual conflict. Additionally, this design strategy is frequently applied in medical interfaces and educational platforms. For example, using low-saturation cool tones such as blue and green with symmetrical layouts can provide users with a sense of trust and security, while rounded buttons and gentle curved icons further enhance comfort \cite{ledoux2000emotion}. In summary, through meticulous design of color, shape, and layout, visual design can achieve emotional goals of high valence in low-arousal states.
% % 在高情绪效价与低情绪唤起的场景中,视觉设计通过柔和的色彩、平静的图像内容、圆润的形状和对称布局,营造出温暖与平和的情感氛围,帮助观众或用户进入放松的情绪状态。例如,约翰·康斯特布尔的《干草车》以柔和的绿色和蓝色为主色调,辅以低饱和度和中高亮度的暖色系,营造出安静和谐的视觉体验(Wilms & Oberfeld, 2018)。画面的布局通过河流、树林和房屋的合理分布形成了稳定的构图,减少了观众的视觉疲劳,同时提升了观看过程的流畅感(lu et al., 2020)。圆润的形状和细腻的明暗过渡进一步降低了视觉刺激强度,例如树木和草地的柔和曲线自然消解了画面中的任何紧张元素(Lu et al., 2012)。这些设计元素不仅通过低唤起的视觉特性调节观众的神经系统反应,还通过降低认知负荷和视觉冲突增强了平和与愉悦感。此外,这种设计策略在实践中常用于医疗界面与教育平台。例如,使用冷色系如蓝色与绿色的低饱和度色调和对称布局,能够为用户提供信任感与安全感,同时通过圆润的按钮和柔和曲线图标进一步优化舒适体验(LeDoux, 1996)。综合来看,通过对色彩、形状和布局的精细设计,视觉设计能够在低唤起状态下实现高情绪效价的情感目标。

% In scenarios of \textbf{low valence and high arousal}, visual design utilizes strong color contrasts, intense imagery, sharp shapes, and asymmetrical layouts to create tension and alertness, thereby eliciting heightened attention and emotional reactions from viewers. For instance, in Edvard Munch’s The Scream, the intense contrast between highly saturated orange and deep blue conveys a strong atmosphere of unease, while the distorted figure and wavy lines further amplify feelings of fear and anxiety. This design directly triggers high arousal levels via the sympathetic nervous system and conveys negative emotional valence \cite{wilms2018color, lu2012shape}. Moreover, the asymmetrical composition and dramatic perspective create a sense of visual oppression, reinforcing the intense emotions conveyed by the artwork (Makin et al., 2012). Similar strategies are widely applied in warning signs and public campaigns, employing highly saturated red and black contrasts, sharp geometric shapes (such as triangles), and disaster imagery to convey a sense of threat and stimulate defensive reactions \cite{pfeuffer122measuring, slavova2019towards}. This visual design approach activates viewers’ cognitive and neural systems, not only increasing emotional arousal but also prompting rapid information processing and action-taking, making it especially effective in communication contexts emphasizing crises and risks.
% % 在低情绪效价与高情绪唤起的场景中,视觉设计通过强烈的色彩对比、激烈的图像内容、尖锐的形状和非对称布局,营造出紧张感和警觉性,从而激发观众的高度注意力和情绪反应。例如,爱德华·蒙克的《呐喊》中,通过高饱和度的橙色与深蓝色的剧烈对比,传递出强烈的不安氛围,同时画中人物扭曲的形状与波浪线条进一步增强了恐惧与焦虑感,这种设计直接通过交感神经系统引发高唤起水平并传递负面情绪效价(Wilms & Oberfeld, 2018;Lu et al., 2012)。此外,非对称的构图与剧烈的透视效果带来视觉上的压迫感,强化了画面传递的紧张情绪(Makin et al., 2012)。类似的策略广泛应用于警示标志和公益宣传中,例如通过高饱和度的红色与黑色对比、尖锐的几何形状(如三角形)和灾害场景照片,传递威胁感并刺激防御性反应(Pfeuffer et al., 2024;Slavova, 2019)。这种视觉设计方式通过激活观众的认知和神经系统,不仅提升了情绪唤起水平,还促使受众快速处理信息并采取行动,在强调危机和风险的传播情境中具有重要作用。

% In scenarios of \textbf{low valence and low arousal}, visual design tends to use low-saturation colors, bland imagery, simple shapes, and symmetrical or static layouts to convey an atmosphere of indifference and helplessness. This design aims to maintain low stimulation levels, creating a subdued yet serene visual experience. 
% For example, in Caspar David Friedrich’s The Monk by the Sea, the use of gray-blue tones and low brightness highlights a mood of solitude and desolation. The painting’s vast, empty landscape and simplified human figure evoke a sense of detachment and helplessness while reducing the intensity of emotional arousal \cite{pazda2024colorfulness}. The symmetrical layout and linear composition of the painting further reduce visual complexity and distractions, allowing viewers to experience a muted negative emotional valence in a low-stimulation context (Makin et al., 2012). This design principle is not only evident in specific artworks but also serves as a reference for communication scenarios involving low emotional valence and low arousal. For instance, shape design often employs simple, linear geometric elements, such as square interface frames and straight lines, to convey restrained and calm emotions \cite{lu2012shape}. These characteristics reduce stimulation of the nervous and sensorimotor systems, suppressing emotional fluctuations while fostering a detached emotional attitude. Such designs are particularly suitable for presenting unavoidable social issues or fatalistic themes, reducing visual complexity and emotional intensity while guiding viewers to process information in a rational and calm manner. This approach effectively maintains low emotional valence and low arousal levels by minimizing emotional intensity and visual complexity, ultimately providing viewers with a profound yet subdued emotional experience.

% % 在低情绪效价与低情绪唤起的场景中,视觉设计倾向于通过低饱和度的色彩、平淡的图像内容、简单的形状以及对称或静态布局来传递冷漠与无奈的情感氛围。这种设计旨在维持低刺激水平,营造出压抑但平和的视觉体验。例如,卡斯帕·大卫·弗里德里希的《海边的僧侣》中,通过使用灰蓝色调和低亮度的画面,突出孤独与荒凉的情绪。画作中的空旷场景和简化的人物形象让观众感受到疏离与无助,同时降低了情绪唤起的强度(Pazda et al., 2024)。画面的对称布局和直线构图进一步减少了视觉上的复杂性和干扰,使观众在低刺激的情境中体验到淡然的负面情绪效价(Makin et al., 2012)。这种设计原则不仅体现在具体作品中,也为低情绪效价与低唤起的传播情境提供了参考。例如,形状设计中常采用简单、直线的几何元素,如方形界面框架和平直线条,传递克制与冷静的情感(Lu et al., 2012)。这些特性通过减少神经系统和感官运动系统的刺激,在抑制情绪波动的同时,营造出冷漠的情感态度。这样的设计尤其适用于呈现无法避免的社会问题或宿命论主题,在降低视觉复杂性与情绪强度的同时,引导观众以理性和冷静的方式接纳信息。这种方法通过压缩情绪强度和视觉复杂性,有效维持了低情绪效价与低唤起水平,最终为观众带来一种深刻却平淡的情感体验。

% By dynamically adjusting colors, images, shapes, and layouts, visual design can achieve a balance tailored to the demands of various emotional valence, arousal, and dominance levels. This design strategy not only enhances visual expressiveness but also provides scientific evidence for emotional regulation and user behavior guidance, offering practical design guidelines for diverse applications.
% % 通过对颜色、图片、形状和布局的多维调控,视觉设计能够在不同情绪效价、唤起度和支配感的需求下实现动态平衡。这种设计策略不仅提升了视觉表现力,还为情绪调控和用户行为引导提供了科学依据,为多场景的应用提供了可行的设计指南。


\subsection{Sound Design}
Sound design plays a critical role in information communication and emotional regulation, influencing users’ emotional experiences through tone, music, and sound effects. Proper adjustments of sound elements, such as pitch modulation and music tempo, can significantly enhance memory and comprehension of information.
% 声音设计在信息传递和情感调节中起着至关重要的作用,通过语调、音乐和音效来影响用户的情绪体验。适当调整声音元素,如语调的高低和音乐的节奏,可以显著提高信息的记忆和理解效果。

\renewcommand{\arraystretch}{1.8} % 调整行间距
\captionsetup{font=small}

\begin{table*}[ht]
\fontsize{8}{9}\selectfont
\centering
\begin{tabularx}{\textwidth}{|>{\centering\arraybackslash}m{0.6cm}|>{\centering\arraybackslash}m{1.55cm}|>{\centering\arraybackslash}m{4.6cm}|>{\centering\arraybackslash}m{4.6cm}|>{\centering\arraybackslash}m{4.6cm}|} 
\hline
\rowcolor[HTML]{D9EAD3} 
\multicolumn{2}{|c|}{\textbf{Dimension}} & \textbf{Tone} & \textbf{Sound Effects} & \textbf{Music} \\ \hline

% Emotional Dimensions Section
\multirow{3}{*}{\rotatebox{90}{\parbox{3cm}{\centering \textbf{Emotional \\ Dimensions}}}} & 
\cellcolor[HTML]{FDF6E8} \textbf{Valence} & 
\textbf{Pleasant tone} enhances positive emotions, Low tone amplifies negative emotions \cite{schirmer2010mark}. 
\textbf{Supportive tone} conveys calmness, Controlling tone induces discomfort \cite{weinstein2018you}. & 
\textbf{Harmonious sound} effects evoke positive valence. \textbf{Dissonant sound} effects elicit negative emotions \cite{parncutt2011consonance}. & 
\textbf{Harmonious melodies} evoke joy \cite{hofbauer2024background}. \textbf{Dissonant melodies} evoke sadness \cite{kabre2024predisposed}. \\ \cline{2-5}

& \cellcolor[HTML]{FDF6E8} \textbf{Arousal} & 
\textbf{Rapid tone} changes increase arousal \cite{bestelmeyer2017effects}. \textbf{Soft tones} decrease arousal \cite{gobl2003role}. &  
\textbf{Fast-attack sound effects} increase arousal \cite{clewett2024emotional}. \textbf{Slow-attack sound effects} decrease arousal \cite{eerola2012timbre}. & 
\textbf{Fast-paced music} increases arousal \cite{hofbauer2024background}. \textbf{Slow-paced music} reduces arousal \cite{shepherd2024investigating}. \\ \cline{2-5}

& \cellcolor[HTML]{FDF6E8} \textbf{Dominance} & 
\textbf{Supportive tone} enhances control \cite{weinstein2018you}. \textbf{Controlling tone} diminishes control \cite{james1884mind}. & 
\textbf{Harmonious sound} effects enhance control \cite{james1884mind}. \textbf{Dissonant sound} effects reduce control \cite{james1884mind}. & 
\textbf{Upbeat melodies} enhance control \cite{moon2024investigating}. \textbf{Low-pitched melodies} reduce control \cite{shepherd2024investigating}. \\ \hline

% Multisystem Activation Section
\multirow{4}{*}{\rotatebox{90}{\parbox{3cm}{\centering \textbf{Multisystem \\ Activation}}}} & 
\cellcolor[HTML]{F0EFF7} \textbf{Neural System} & 
\textbf{Gentle tones} regulate the parasympathetic system \cite{gobl2003role}. \textbf{High-arousal tones} activate the sympathetic system \cite{bestelmeyer2017effects}. & 
\textbf{Sharp sound effects} activate the sympathetic system \cite{iversen2000emotional}. \textbf{Fast-attack sound effects} trigger physiological responses \cite{iversen2000emotional}. \textbf{Slow-attack sound effects} regulate the parasympathetic system \cite{iversen2000emotional}. & 
\textbf{Fast-paced music} activates the sympathetic system \cite{juslin2008emotional}. \textbf{Slow-paced music} regulates the parasympathetic system \cite{juslin2008emotional}. \\ \cline{2-5}

& \cellcolor[HTML]{F0EFF7} \textbf{Sensorimotor System} & 
\textbf{Cheerful tone} promotes relaxation \cite{frijda1986emotions}. \textbf{Low tone} induces tension \cite{james1884mind}. & 
\textbf{Sharp sound effects} trigger tension responses \cite{iversen2000emotional, plutchik1980general}. \textbf{Harmonious sound effects} promote physical relaxation \cite{james1884mind}. & 
\textbf{Fast-paced music} induces body movement \cite{thaut2015neurobiological}. \textbf{Slow-paced music} aids relaxation \cite{bernardi2006cardiovascular}. \\ \cline{2-5}

& \cellcolor[HTML]{F0EFF7} \textbf{Cognitive Systems} & 
\textbf{Rising tone} increases anticipation \cite{zajonc1980feeling}. \textbf{Falling tone} reduces cognitive load \cite{schirmer2010mark}. & 
\textbf{Harmonious sound effects} improve focus \cite{schulte2001quality}. \textbf{Sharp sound effects} convey threat \cite{schulte2001quality}. & 
\textbf{Harmonious music} enhances cognitive efficiency \cite{zajonc1980feeling}. \textbf{Chaotic music} distracts attention \cite{hofbauer2024background}. \\ \cline{2-5}

& \cellcolor[HTML]{F0EFF7} \textbf{Motivational Systems} & 
\textbf{Authoritative tone} triggers urgency \cite{fogg2009behavior}. \textbf{Supportive tone} enhances a sense of safety \cite{weinstein2018you}. & 
\textbf{Reward sound effects} boost achievement motivation \cite{fogg2009behavior}. \textbf{Alarm sound effects} trigger protective motivation \cite{mazur2019effects}. & 
\textbf{Fast-paced and harmonious music} stimulate achievement motivation \cite{juslin2008emotional}. \textbf{Low-pitched music} encourages introspection \cite{hofbauer2024background}. \\ \hline

\end{tabularx}
\caption{Emotional Dimensions and Multisystem Activation in Sound Design Elements} 
\label{tab:sound_design}
\end{table*}

% \begin{figure*}[hbt!]
% %\setlength{\abovecaptionskip}{-0.1mm}
% \setlength{\intextsep}{10pt plus 2pt minus 2pt}
%     \centering
%     \includegraphics[width=18cm]{figs/sound_design.png}
%     \caption{Emotional Dimensions and Multisystem Activation in Sound Design Elements.}
%    \vspace{-2mm}
% \label{fig:why}
% \end{figure*}

\subsubsection{Tone}
\begin{wrapfigure}{l}{0.06\textwidth}
  %\begin{center}
  \vspace{-11pt} % 调整垂直位置
        \includegraphics[width=0.07\textwidth]{figs/icon/tone.png}
  %\end{center}
\end{wrapfigure} 
Tone, as a critical feature of sound design, profoundly influences listeners’ emotional valence, arousal levels, and dominance through adjustments in pitch, timbre, speech rate, and volume. Research shows that the consistency between tone and the emotional content of words can significantly enhance listeners’ emotional experiences. For instance, a pleasant tone (lively and bright timbre) paired with positive words can amplify the words’ positive valence, while a deep or melancholic tone can intensify the emotional negativity of negative words, leaving a deeper emotional impression on listeners \cite{schirmer2010mark}. This emotional consistency effect not only strengthens the expression of emotional information but also leaves a lasting impression in listeners’ emotional memory.
% 语调作为声音设计的重要特征,通过音调、音色、语速和音量的多重调控,深刻影响听众的情绪效价、唤起水平和支配感。研究表明,语调与词汇情绪内容的一致性能够显著增强听众的情感体验。例如,愉悦语调(轻快且明亮的音质)与积极词汇结合时,能够提升词汇的正面效价;而低沉或悲伤的语调则会放大消极词汇的情绪负向性,使听众对这些内容产生更深刻的情感记忆(Schirmer, 2010)。这种情绪一致性效应不仅强化了情感信息的表达,还在听众的情感记忆中留下长期印象。

Supportive tones (e.g., soft volume, slow speech rate, gentle timbre) can convey positive and empathetic emotions, making listeners feel respected and accepted, thereby enhancing emotional valence and fostering a calm emotional atmosphere. In contrast, controlling or hurried tones (e.g., high volume, fast speech rate) often convey a sense of pressure, reducing emotional valence, and particularly in the context of negative content, they can provoke tension and discomfort in listeners \cite{weinstein2018you}. These tones act on the sensory-motor system through the auditory system, eliciting physical responses such as muscle relaxation or tension, thereby focusing listeners’ attention on the information conveyed by the tone \cite{james1884mind, frijda1986emotions} .
% 支持性语调(如轻柔的音量、缓慢的语速、柔和的音质)能够传递积极和理解的情感,使听众感受到尊重与接纳,从而提升情绪效价并营造平和的情绪氛围。相反,控制性或急促语调(高音量、快速语速)通常传递压迫感,削弱情绪效价,尤其在负面内容的表达中易引发听众的紧张感和不适情绪(Weinstein et al., 2018)。这些语调通过听觉系统作用于感官运动系统,引发身体反应,例如肌肉的放松或紧张,从而使听众更专注于语调传递的信息(James, 1884;Frijda, 1986)。

At the level of emotional arousal, variations in tone speed and intensity play a crucial role. High-arousal tones (e.g., anger or fear) activate auditory neural pathways and the sympathetic nervous system, triggering physiological responses such as increased heart rate and pupil dilation, thereby heightening listeners’ perception of urgent or high-risk information \cite{bestelmeyer2017effects}. In contrast, gentle tones (moderate volume, slow speech rate, steady pitch) reduce arousal levels, helping listeners alleviate anxiety and achieve a state of calm. Such tones are often used in relaxation settings, where parasympathetic nervous system regulation induces physical comfort, such as reduced blood pressure and slower heart rate \cite{gobl2003role}.
% 在情绪唤起层面,语调的快慢与强度变化扮演关键角色。高唤起语调(如愤怒或恐惧)会激活听觉神经通路和交感神经系统,引发心跳加速、瞳孔扩张等生理反应,强化听众对紧急或高风险信息的感知(Bestelmeyer et al., 2017)。相比之下,柔和语调(音量适中、语速缓慢、音调平稳)通过降低唤起水平,帮助听众缓解焦虑并进入平静状态。此类语调常用于放松场景,通过副交感神经系统的调节,带来血压降低和心率放缓的身体舒适感(Gobl & Chasaide, 2003)。

The emotional impact of tone is also deeply rooted in the cognitive system. Different tones provide semantic cues for information. For example, a rising interrogative tone implies incomplete intentions, enhancing the audience’s anticipation for subsequent content, while a descending declarative tone conveys certainty and trust, reducing cognitive load \cite{zajonc1980feeling}. Moreover, tone not only influences the immediate interpretation of information but also has a lasting impact on the emotional representation of words, which remains significant even after explicit memory fades \cite{schirmer2010mark}. Therefore, tone plays a critical role in enhancing the emotional value and memory retention of information for listeners.
% 语调的情绪影响还深植于认知系统中。不同语调为信息提供语义提示,例如,上扬的疑问语调暗示未完成的意图,增强听众对后续内容的期待;而下降的陈述语调则传递确定性与信任感,减轻认知负荷(Zajonc, 1980)。此外,语调不仅影响信息的即时解读,还对词汇的情绪表征产生长期作用,即使在显性记忆消退后,这种影响仍然显著(Schirmer, 2010)。因此,语调在提升听众对信息的情感价值和记忆持久性方面具有重要意义。

The activation of the motivational system is also a critical aspect of how tone influences emotions. Authoritative tone (fast speech rate and stable intonation) conveys seriousness and urgency, which can stimulate action motivation, prompting listeners to make quick decisions or take action \cite{fogg2009behavior}. For example, authoritative tone in navigation instructions can enhance users’ focus on tasks. Supportive tone (gentle and melodic voice) tends to evoke a sense of safety, making listeners more willing to accept information and establish emotional connections with it \cite{weinstein2018you}. This activation of the motivational system not only increases information acceptance but also enhances users’ trust and reliance on the design.
% 动机系统的激活也是语调影响情绪的重要方面。权威语调(语速较快、语调稳定)通过传递严肃和紧迫感,能够激发行动动机,促使听众迅速决策或采取行为(Fogg, 2009)。例如,导航指令中的权威语调能够提高用户对操作的专注力。而支持性语调(柔和且富有韵律的语音)则倾向于激发安全感动机,使听众更愿意接受信息并与之建立情感连接。[1]这种动机系统的触发不仅提升了信息的接受度,还增强了用户对设计的信任感和依赖性。

Overall, tone, through the synergistic effects of the sensory-motor system, nervous system, cognitive system, and motivational system, exerts a multi-level influence on the regulation of listeners’ emotions and the guidance of their behavior. In tone design, designers can precisely adjust tone characteristics to meet the needs of different scenarios, thereby achieving effective emotional shaping and positive behavioral guidance.
% 从整体来看,语调通过感官运动系统、神经系统、认知系统和动机系统的协同作用,在听众的情绪调节和行为引导中发挥了多层次的影响。设计师在语调设计中,可以针对不同场景的需求精准调控语调特性,从而实现情绪的有效塑造与行为的积极引导。

\subsubsection{Sound Effects}
\begin{wrapfigure}{l}{0.06\textwidth}
  %\begin{center}
   \vspace{-11pt} % 调整垂直位置
        \includegraphics[width=0.07\textwidth]{figs/icon/soundeffects.png}
  %\end{center}
\end{wrapfigure} 
Sound effects significantly influence emotional valence and arousal through spectral characteristics, harmonicity, and attack slopes \cite{eerola2012timbre}. Harmonious intervals (e.g., perfect fifth or perfect fourth) typically evoke positive valence, making listeners feel comfortable and relaxed. This sound characteristic is widely used in relaxation and meditation scenarios, such as melodies in background music simulating natural sounds (e.g., flowing water or birdsong), helping to create a peaceful atmosphere. In contrast, dissonant intervals (e.g., minor second or major second) often induce tension, unease, or negative emotions due to their high-frequency energy and acoustic roughness, making them suitable for portraying danger or suspense\cite{parncutt2011consonance}. This contrast manifests in the sensory-motor system as varying degrees of physical responses, such as muscle relaxation when hearing harmonious sounds, while dissonant sounds may lead to muscle tension or stress reactions like frowning \cite{james1884mind}.
% 音效通过频谱特征、和谐度、攻击斜率等方面对情绪效价和唤起度产生显著影响。[1]和谐音程(如纯五度或纯四度)通常带来积极效价,使听众感到舒适与放松。这种音效特性被广泛应用于放松和冥想场景,例如背景音乐中模拟自然声音(如流水或鸟鸣)的旋律,有助于创造平和的氛围。相对而言,不和谐音程(如小二度或大二度)因其高频能量和声学粗糙性,常引发紧张、不安或负面情绪,适用于表现危险或悬疑的场景(Parncutt & Hair, 2011)。[2]这种对比在感官运动系统中表现为不同程度的身体反应,例如听到和谐音效时肌肉放松,而不和谐音效可能导致肌肉紧绷或皱眉等应激反应(James, 1884)。

The attack slope of sound effects, which refers to the rate of change from silence to maximum amplitude, directly affects emotional arousal levels. Rapid-attack sound effects (e.g., abrupt alarm sounds) significantly heighten listeners’ alertness and tension \cite{eerola2012timbre}. Such high-arousal sound effects are commonly used in emergency scenarios, such as fire alarms or medical alert tones, to quickly capture attention and activate the sympathetic nervous system, eliciting physiological responses like increased heart rate and rapid breathing \cite{iversen2000emotional}. In contrast, slow-attack sound effects (e.g., gradually intensifying string music) reduce arousal levels through smooth volume changes, creating a calming emotional experience suitable for emotional regulation or meditation scenarios \cite{zajonc1980feeling}. This characteristic of sound effects is primarily regulated by the amygdala and hypothalamus within the nervous system, determining the intensity of emotional arousal and physiological response patterns \cite{iversen2000emotional}.
% 音效的攻击斜率,即声音从静止到最大振幅的变化速度,直接影响情绪唤起水平。快速攻击音效(如短促警报声)能够显著提升听众的警觉性和紧张感。这种高唤起音效常用于紧急情境,例如火警警报或医疗急救提示音,迅速吸引注意力并激活交感神经系统,引发心跳加速和呼吸急促等生理反应(Iversen et al., 2000)。相较之下,慢攻击音效(如渐强的弦乐)通过平缓的音量变化降低唤起水平,营造舒缓的情绪体验,适合用于情绪调节或冥想的场景(Zajonc, 1980)。音效的这一特性在神经系统中主要通过杏仁核和下丘脑的调节,决定情绪的唤起强度和生理反应模式。

From the perspective of the cognitive system, sound effects influence emotions not only through direct perception but also by shaping listeners’ emotional evaluations through contextual cues. For instance, harmonious background sound effects with gentle rhythms are often interpreted as positive and safe contextual cues, enhancing listeners’ sense of pleasure and focus. In contrast, sharp sound effects (e.g., metal scraping or piercing alarm sounds) convey signals of threat, prompting listeners to assess potential risks and heightening their vigilance and preparedness \cite{schulte2001quality}. The role of sound effects in the cognitive system is also evident in their auxiliary function in information processing. For example, prompt sounds within a soft soundscape can improve task performance, whereas complex or jarring sound effects may distract attention and impair cognitive performance~\cite{schulte2001quality}.

% 从认知系统的角度来看,音效不仅通过直接感知影响情绪,还通过情境暗示塑造听众的情感评估。例如,和谐的背景音效与轻柔的节奏常被解读为积极和安全的情境提示,能够增强听众的愉悦感和专注力。相反,尖锐音效(如金属摩擦声或刺耳警报声)传递威胁性信号,会激发听众对潜在风险的认知评估,使其更加警觉和防备(Schulte-Fortkamp, 2001)。音效在认知系统中的作用还表现在其对信息处理的辅助功能。例如,柔和音效背景下的提示音能够提升任务执行效率,而复杂或刺耳的音效可能分散注意力,降低认知表现(Ostendorf et al., 2020)。

Sound effects also have a significant impact on activating the motivational system. Reward sound effects (e.g., task completion prompts) reinforce positive feedback, stimulating listeners’ achievement motivation and encouraging greater engagement in target tasks \cite{fogg2009behavior}. For example, reward sound effects for gaining points in games increase player engagement and enhance satisfaction upon task completion. Conversely, alarm sounds convey urgent information that triggers protective motivation, prompting listeners to take swift risk-avoidance actions. This motivational trigger mechanism is widely used in affective computing and behavioral design to guide users in making quick decisions in specific contexts~\cite{mazur2019effects}.
% 音效对动机系统的激活同样显著。奖励音效(如任务完成提示音)通过强化积极反馈,能够激发听众的成就动机,使其更愿意投入到目标任务中(Fogg, 2009)。例如,游戏中的积分奖励音效不仅增加了玩家的参与感,还提升了任务完成的满足感。相反,警报音效通过传递紧急信息触发保护动机,使听众迅速采取规避风险的行动。这种动机触发机制在情感计算和行为设计中被广泛应用,用于引导用户在特定情境中做出快速反应(Mazur et al., 2019)。

\subsubsection{Music}
\begin{wrapfigure}{l}{0.06\textwidth}
  %\begin{center}
  %\vspace{-11pt} % 调整垂直位置
        \includegraphics[width=0.07\textwidth]{figs/icon/music.png}
  %\end{center}
\end{wrapfigure} 
Music exerts a multidimensional impact on emotional valence and arousal levels through features such as melody, rhythm, and emotional complexity. Music tempo is one of the key factors in emotional regulation. Fast-paced music (such as tempos above 130 beats per minute) induces higher emotional arousal \cite{hofbauer2024background}, making listeners feel excited and invigorated, and is widely used in settings like fitness or competitive sports \cite{moon2024investigating}. In contrast, slow-paced music (such as tempos between 60 and 90 beats per minute) helps alleviate anxiety and promotes relaxation\cite{hofbauer2024background}, making it a common choice in relaxing contexts such as meditation and psychotherapy \cite{shepherd2024investigating}. From the sensorimotor system perspective, fast-paced music often triggers bodily movements, such as involuntary swaying, while slow-paced music aids in relaxing the body by relaxing muscles and stabilizing breathing \cite{thaut2015neurobiological, bernardi2006cardiovascular}. On the neurological level, fast-paced music enhances excitation by activating the sympathetic nervous system, while slow-paced music regulates through the parasympathetic nervous system, lowering heart rate and blood pressure, thereby providing physiological comfort and relaxation \cite{juslin2008emotional}.
% 音乐通过旋律、节奏和情绪复杂性等特征对情绪效价和情绪唤起水平产生多维影响。音乐节奏是情绪调节的重要因素之一。快节奏音乐(如每分钟130次节拍以上)激发较高的情绪唤起,使听众感到兴奋与振奋,因而被广泛应用于健身或竞技等场景(Moon et al., 2024)。相反,慢节奏音乐(如每分钟60至90次节拍)有助于缓解焦虑并促进放松,这使其成为冥想、心理治疗等放松情境中的常用选择(Shepherd et al., 2024)。从感官运动系统来看,快节奏音乐常引发身体律动,如不自主摇摆,而慢节奏音乐则通过放松肌肉和稳定呼吸帮助身体放松(Thaut et al., 2015; Bernardi et al., 2006)。在神经系统层面,快节奏音乐通过激活交感神经系统增强兴奋感,慢节奏音乐则通过副交感神经系统调节,降低心率与血压,从而带来生理上的舒适与放松(Juslin & Västfjäll, 2008)。

The melodic characteristics of music also play a key role in shaping emotional valence. Harmonious melodies and bright tonalities typically convey positive emotions, helping to enhance focus and creativity, commonly found in work or social settings \cite{hofbauer2024background}. In contrast, low tonality and slow melodies convey sad or introspective emotions, widely used in emotional films or therapeutic settings \cite{kabre2024predisposed}. Dissonant melodies, due to their tense and chaotic nature, are often used to create tension or provoke alertness in scenes such as horror films or warning messages. Additionally, fast-paced music and harmonious melodies often stimulate achievement motivation and exploration motivation, helping listeners stay focused on tasks and improve performance; whereas low, somber music may trigger introspective motivation, encouraging listeners to engage in deep emotional thinking and reflection \cite{juslin2008emotional}.
% 音乐的旋律特性同样在情绪效价的塑造中起着关键作用。和谐旋律和明亮调性通常传递积极情绪,帮助提升专注力和创造力,常见于工作或社交场景(Hofbauer et al., 2024)。与之相对,低调性与缓慢旋律则传递悲伤或内省情绪,广泛应用于情感电影或疗愈场景(Kabre & Srivastava, 2024)。不和谐旋律因其紧张和混乱特性,常用于制造张力或引起警觉的场景,如恐怖电影或警示信息。此外,快节奏音乐和和谐旋律往往激发成就动机和探索动机,使听众更加专注于任务并提升执行力;而低沉的音乐则可能引发内省动机,促使听众进行深刻的情感思考和反省。

Music not only affects an individual's psychological state through emotional regulation but also significantly influences the cognitive system, thereby impacting task performance. Harmonious music helps improve the efficiency of cognitive tasks, such as enhancing reading speed and memory; whereas dissonant music may distract attention and reduce task efficiency \cite{zajonc1980feeling}. Positive emotional music can stimulate the willingness to perform tasks and promote active engagement; in contrast, negative emotional music may suppress initiative, making listeners more cautious \cite{hofbauer2024background}.
% 音乐不仅通过情绪调节影响个体的心理状态,还能够显著作用于认知系统,从而影响任务表现。和谐音乐有助于提升认知任务的效率,如增强阅读速度与记忆力;而不和谐的音乐则可能分散注意力,降低任务效率(Zajonc, 1980)。正面情绪音乐能够激发任务执行意愿,促进积极参与;相反,负面情绪音乐则可能抑制主动性,令听众更为谨慎(Hofbauer et al., 2024)。

The emotional complexity of music enhances the depth of emotional experiences. Music that blends positive and negative emotions (such as melodies combining sorrow and appreciation) can evoke a deeper level of emotional resonance, particularly in contexts with high emotional expression demands, such as film scores \cite{baltazaremotional}.
% 音乐的情感复杂性也增强了情绪体验的层次感。融合正面与负面情绪的音乐(如悲痛与欣赏并存的旋律)能引发更高层次的情感共鸣,尤其在影视配乐等情感表达需求较高的场景中(Baltazar et al., 2024)。


%\subsubsection{Summary}
% Sound design plays a crucial role in emotional regulation, combining tone, sound effects, and background music to create diverse emotional experiences across varying emotional valence and arousal levels. Whether in high-valence, high-arousal scenarios that evoke positive emotions or low-valence, low-arousal settings that convey solitude, sound design engages the listener’s emotions through sensory stimuli while also influencing emotional memory and behavioral responses via neural and cognitive mechanisms.
% % 声音设计在情绪调节中发挥着重要作用,通过语调、音效和背景音乐的结合,为不同情绪效价和唤起水平创造多样的情感体验。无论是激发积极情绪的高效价、高唤起场景,还是传递孤寂情感的低效价、低唤起情境,声音设计不仅通过感官刺激调动听众的情绪,还通过神经系统和认知机制影响情感记忆与行为反应。

% In scenarios of \textbf{high valence and high arousal}, sound design employs an intricate combination of invigorating tone, intense sound effects, and fast-paced background music, effectively evoking positive emotions and intense sensory experiences. For example, the iconic training montage in Rocky features the soundtrack Gonna Fly Now, characterized by fast tempo and high-pitched melodies, paired with pronounced drumbeats and stirring orchestration, with a tempo exceeding 130 beats per minute \cite{moon2024investigating}, generating a powerful and consistent rhythm that successfully induces listeners’ physical movement and excitement. The inclusion of boxing punches, rapid breathing, and footsteps in the soundtrack, with their sharp attack slopes, further intensifies tension and action motivation, drawing viewers’ focus on the protagonist’s relentless efforts \cite{iversen2000emotional}. Furthermore, the tone of dialogue conveys a sense of strength through moderate volume and steady rhythm, aligning seamlessly with the emotional tone of the background music to create an emotional climax \cite{schirmer2010mark}. This design synergizes music, tone, and sound effects, not only activating the auditory and sympathetic nervous systems but also enhancing emotional memory and a dominance, enabling viewers to deeply feel a positive drive. Such high-energy sound design is especially common in fitness settings, commercial advertisements, and entertainment content, as it strengthens listeners’ attention and emotional memory through sensory-motor system stimulation, while also activating the motivational system, making listeners feel empowered and capable of action, thereby significantly enhancing emotional experience and behavioral engagement.
% % 在高情绪效价与高情绪唤起的场景中,声音设计通过激昂的语调、强烈的音效和快节奏背景音乐的精妙结合,有效激发积极情绪和强烈的感官体验。例如,电影《洛奇》中经典的训练场景配乐《Gonna Fly Now》以快节奏和高音调的旋律为主,融合节奏鲜明的鼓点和激昂的管弦乐,其每分钟超过130次的节拍(Moon et al., 2024)强烈而一致,成功唤起听众的身体律动和兴奋感。配乐中穿插的拳击声、急促的呼吸声以及脚步声,通过快速攻击斜率进一步强化了紧张感和行动动机,吸引观众更加聚焦于主角的奋力拼搏(Iversen et al., 2000)。此外,对话中传递力量感的语调,以适中的音量和稳定的节奏表现出坚毅与自信,与背景音乐的情感基调完美契合,形成了情绪上的高潮(Schirmer, 2010)。这种设计通过音乐、语调和音效的协同作用,不仅激活了听觉和交感神经系统,还增强了情绪记忆和支配感,使观众深刻感受到正向的驱动力。这种高能量特性的声音设计在健身场景、商业广告和娱乐内容中尤为常见,通过感官运动系统的刺激强化听众的注意力和情感记忆,同时通过激活动机系统,使听众感到自身具有掌控力和行动能力,极大提升了情绪体验和行为投入。

% In scenarios of \textbf{high valence and low arousal}, sound design employs gentle tones, calming sound effects, and soothing background music to create a warm and stable emotional atmosphere. For example, in the movie Life is Beautiful, the protagonist’s father uses a soft and slightly humorous tone to explain the harsh environment of the concentration camp to his son, with warm vocal timbre and slow speech rate that not only reduce arousal levels but also convey profound care and hope \cite{schirmer2010mark}. This tonal design complements the emotional tone of the background music Life is Beautiful. Composed by Nicola Piovani, this score features soft string instrumentation and a slow melody, which enhances the warmth and affinity of the scene through harmonious intervals and lyrical melodies. This sound design reduces stimulation to the nervous system and activates the regulatory functions of the parasympathetic nervous system, helping to lower heart rate and blood pressure, thus providing listeners with physiological and psychological comfort \cite{juslin2008emotional}. The soothing rhythm of the music and the caring elements in the tone complement each other, jointly shaping a tranquil yet profound emotional experience. This design, through the seamless integration of tone, sound effects, and music, not only deepens the audience’s resonance with the emotional bonds between characters, but also imbues the narrative with a sense of stability and hope. Such sound design is widely used in therapeutic or emotional narrative contexts, where reducing sensory stimulation enhances the audience’s emotional resonance and internal calmness, while successfully elevating emotional valence in a low-arousal state.
% % 在高情绪效价与低情绪唤起的场景中,声音设计通过柔和的语调、平静的音效和舒缓的背景音乐,巧妙地营造出温暖而安定的情绪氛围。例如,电影《美丽人生》中,主角父亲在集中营中用轻柔且略带幽默的语调向儿子解释险恶的环境,其温暖的音色和缓慢的语速不仅降低了情绪唤起水平,还传递了深切的关怀与希望(Schirmer, 2010)。这一语调设计与背景音乐《Life is Beautiful》的情感基调相得益彰。这首由尼古拉·皮奥瓦尼创作的配乐,以柔和的弦乐和缓慢的旋律为主,通过和谐的音程与抒情的旋律增强了场景的亲和力和温暖感。这种声音设计通过减少对神经系统的刺激,激活副交感神经的调节作用,有助于降低心率和血压,为听众带来生理和心理上的舒适感(Juslin & Västfjäll, 2008)。音乐的舒缓节奏与语调中的关怀元素相辅相成,共同塑造了平静而深刻的情绪体验。这种设计通过语调、音效与音乐的紧密融合,不仅加深了观众对角色间情感纽带的共鸣,还为叙事注入了一种安定与希望的氛围。这类声音设计广泛应用于心理疗愈或情感叙事场景,通过降低感官刺激的强度,增强观众的情感共鸣与内在安定感,同时在低唤起的状态下成功提升情绪效价。

% In scenarios of \textbf{low valence and high arousal}, sound design employs tense tones, jarring sound effects, and high-energy background music to create intense unease and emotional tension. For example, in the movie A Quiet Place, Marco Beltrami’s background score What Is Safe uses deep strings and progressively intensifying rhythms, combined with dissonant intervals and sudden high-pitched insertions, to evoke sustained fear and a sense of oppression in the audience. This music design, through gradual volume crescendos and abrupt acoustic shocks, stimulates the auditory and sympathetic nervous systems, significantly increasing heart rate and alertness \cite{iversen2000emotional}. Simultaneously, environmental sound effects in the film, such as subtle footsteps and occasional object collisions, complement the tense atmosphere of the background music. These sound effects, with their rapid attack slopes and high-frequency energy, provoke the audience’s cognitive assessment of potential threats, further heightening emotional arousal levels. Complementing this sound and music design, the actors’ dialogue is condensed into brief whispers, with a repressed tone that aligns with the emotional tone of the scenes, enhancing the audience’s empathy and immersion in the characters’ situations. This integrated sound design keeps the audience in a state of sustained tension, while the overlap of emotional arousal and a sense of threat triggers strong defensive motivation, drawing greater focus on the film’s narrative progression. Such sound design is particularly prevalent in horror and thriller films, where the precise coordination of sensory stimuli effectively conveys tension and deep emotional experiences.

% % 在低情绪效价与高情绪唤起的场景中,声音设计通过紧张的语调、刺耳的音效和高能量的背景音乐,共同营造出强烈的不安感和高张力的情绪氛围。例如,电影《寂静之地》中,马可·贝尔特拉米创作的背景音乐《What Is Safe》以低沉的弦乐和逐渐强化的节奏为基础,结合不和谐的音程和突然的高音尖锐插入,使观众感受到持续的恐惧与压迫感。这种音乐设计通过缓慢的音量增长与突如其来的声学冲击,刺激了听觉神经系统和交感神经系统,显著提高了心率和警觉性(Iversen et al., 2000)。同时,电影中的环境音效,如轻微的脚步声和偶尔的物体碰撞声,与背景音乐的紧张氛围相辅相成。这些音效通过快速的攻击斜率和高频率能量,引发了观众对潜在威胁的强烈认知评估,进一步提升了情绪唤起水平。配合这一音效与音乐设计,演员台词被压缩为短促低语,语调中蕴含的压抑感与场景的情感基调相一致,增强了观众对角色情境的同理心与代入感。这种声音设计的综合运用,不仅让观众持续处于紧张状态,还通过情绪唤起和威胁感的叠加,触发了强烈的防御动机,使其更加专注于电影情节的发展。这类声音设计在恐怖电影和惊悚片中尤为常见,通过感官刺激的高度协调,有效传递了紧张情绪和深刻的情感体验。

% In scenarios of \textbf{low valence and low arousal}, sound design employs deep tones, calm sound effects, and background music emphasizing silence to convey feelings of indifference, loneliness, or fatalism. For example, in the movie The Wind That Shakes the Barley, low-volume, slow-paced dialogue and tone express the characters’ resignation towards life. This tone avoids emotional fluctuations, using restrained pacing and muted sound quality to convey the characters’ inner solitude to the audience \cite{schirmer2010mark}. 
% The background music Katyusha Theme, primarily featuring deep bass strings and sparse percussion, leaves significant pauses between notes, evoking a sense of time’s stagnation and environmental desolation for the audience \cite{wilms2018color}. Simultaneously, sound design minimizes high-frequency elements, retaining only natural effects like wind and raindrops, with low tones and slow rhythms that reduce auditory stimulation and emotional arousal \cite{wilms2018color}. This sound design reduces neural activation and sensory-motor system stimulation, guiding the audience into a passive and introspective emotional state. By compressing the dynamic range and volume variation of sounds, it sustains a low-arousal emotional atmosphere, allowing the audience to perceive negative emotions while remaining calm. This type of sound design is often used to portray fatalistic themes or depict helplessness in narrative settings, reducing sensory load to create a profound, calm yet tragically tinged emotional experience.
% % 在低情绪效价与低情绪唤起的场景中,声音设计通过低沉的语调、平缓的音效和寂静感突出的背景音乐,传递出冷漠、孤寂或宿命的情感氛围。例如,电影《风吹稻浪》中,用低音量、缓慢节奏的对话和语调表达出角色对生活无可奈何的态度。这种语调避免了任何情绪波动,通过压抑的语速和沉闷的音质,将角色的内心孤独传递给观众(Schirmer, 2010)。背景音乐《Katyusha Theme》以深沉的低音弦乐和少量背景打击乐为主,音符之间留有大量空白,使观众感受到时间的停滞感和环境的冷清(Wilms & Oberfeld, 2018)。同时,音效设计中减少了高频元素,仅保留风声、雨滴声等自然音效,低声调且节奏缓慢,通过削弱听觉刺激降低情绪唤起水平(Wilms & Oberfeld, 2018)。这种声音设计通过减少神经系统的激活和感官运动系统的刺激,引导观众进入一种被动且内敛的情绪状态。通过压缩声音的动态范围和音量变化,进一步维持低唤起的情绪氛围,使观众在感知到负面情绪的同时保持冷静。这种声音设计常用于表现宿命论主题或描绘无力感的叙事场景,通过降低感官负担,营造一种深沉、冷静却带有悲剧意味的情感体验。


\subsection{Interaction Design}
Interaction design aims to optimize the interaction process between humans and systems, enhancing user experience and emotional satisfaction. It encompasses key elements such as intuitive navigation, instant feedback mechanisms, dynamic elements, and haptic interactions. These components work together to not only improve operational fluency and efficiency but also significantly influence users’ emotional valence and arousal. By effectively integrating these design elements, interaction design achieves a dual enhancement of functionality and emotional value.
% 交互设计 旨在通过优化人与系统之间的交互过程,提升用户体验和情感满足。它包括直观的导航、即时反馈机制、动态元素和触觉交互等关键要素,这些共同作用,不仅提升操作流畅性和效率,还深刻影响用户的情绪效价和情绪唤起。通过合理整合这些设计要素,交互设计实现了功能性与情感价值的双重提升。

\renewcommand{\arraystretch}{1.6} % 调整行间距
\captionsetup{font=small}

\begin{table*}[ht]
\fontsize{8}{9}\selectfont
\centering
\begin{tabularx}{\textwidth}{|>{\centering\arraybackslash}m{0.6cm}|>{\centering\arraybackslash}m{1.55cm}|>{\centering\arraybackslash}m{4.6cm}|>{\centering\arraybackslash}m{4.6cm}|>{\centering\arraybackslash}m{4.6cm}|} 
\hline
\rowcolor[HTML]{D9EAD3} 
\multicolumn{2}{|c|}{\textbf{Dimension}} & \textbf{Interaction Method} & \textbf{Motion Effects} & \textbf{Navigation Design} \\ \hline

% Emotional Dimensions Section
\multirow{3}{*}{\rotatebox{90}{\parbox{3cm}{\centering \textbf{Emotional \\ Dimensions}}}} & 
\cellcolor[HTML]{FDF6E8} \textbf{Valence} & 
Intuitive interactions (e.g., tapping, swiping) reduce learning cost, increase comfort and satisfaction \cite{sundar2014user, amoor2014designing}.  Complex interactions (e.g., multi-finger gestures) increase frustration and negative emotions \cite{sundar2014user}. & 
Linear motion effects convey stability, enhancing calmness and attraction \cite{lockyer2012affective}. Complex curves may cause tension and unease \cite{lockyer2012affective}. & 
Simple navigation reduces cognitive load, enhancing positive emotions \cite{wang2024enhancing, abdelaal2023accessibility}. Complex navigation induces confusion \cite{sheng2012effects}.\\ \cline{2-5}

& \cellcolor[HTML]{FDF6E8} \textbf{Arousal} & 
Simple gestures (e.g., tapping, swiping) create a natural, smooth experience, controlling arousal \cite{wodehouse2014exploring}. Strong haptic feedback attracts attention but may cause tension \cite{olugbade2023touch}. &  
High-intensity effects (e.g., rapid cuts) increase attention and arousal \cite{hanjalic2005affective}. Slow motion creates a calming atmosphere \cite{wollner2018slow}. & 
Immediate feedback (e.g., visual or tactile cues) reduces anxiety, stabilizing emotions \cite{sundar2014user, wang2024enhancing}. Dynamic navigation enhances engagement and arousal \cite{amoor2014designing}. \\ \cline{2-5}

& \cellcolor[HTML]{FDF6E8} \textbf{Dominance} & 
Free exploratory interactions (e.g., drag and zoom) enhance user control and engagement \cite{amoor2014designing}.  Lengthy, complex tasks reduce control \cite{wang2024enhancing}. & 
Dynamic navigation (e.g., expandable menus) boosts user control \cite{amoor2014designing}. Lack of feedback undermines control \cite{wang2024enhancing}. & 
Consistent navigation aligns with user expectations, boosting trust and control \cite{abdelaal2023accessibility}. Errors (e.g., dead links) reduce control \cite{sheng2012effects}. \\ \hline

% Multisystem Activation Section
\multirow{4}{*}{\rotatebox{90}{\parbox{3cm}{\centering \textbf{Multisystem \\ Activation}}}} 

& \cellcolor[HTML]{F0EFF7} \textbf{Neural System} & 
Real-time feedback (e.g., vibration) stimulates sensory neurons, enhancing task awareness \cite{olugbade2023touch}. & 
Dynamic transitions activate the visual pathway, improving neural plasticity \cite{yoo2005processing}. High-intensity visuals (e.g., flashing) trigger short-term tension \cite{hanjalic2005affective}. & 
Dynamic navigation (e.g., auto-scrolling) strengthens neural responses through sensory input \cite{sundar2014user}. Immediate feedback (e.g., click confirmation) enhances achievement sensation \cite{wang2024enhancing}. \\ \cline{2-5}

& \cellcolor[HTML]{F0EFF7} \textbf{Sensorimotor System} & 
Intuitive interactions (e.g., swiping, tapping) activate touch and motion feedback, reinforcing motor learning \cite{sundar2014user}. Complex gestures may cause hand fatigue \cite{wodehouse2014exploring}. & 
Fast animations stimulate dynamic perception \cite{hanjalic2005affective}. Slow motion promotes detailed visual processing \cite{wollner2018slow}. & 
Simplified navigation paths reduce hand-eye coordination stress \cite{wang2024enhancing}.  Dynamic feedback enhances task completion awareness \cite{sundar2014user}. \\ \cline{2-5}

& \cellcolor[HTML]{F0EFF7} \textbf{Cognitive Systems} & 
Intuitive interactions reduce cognitive load, improving task efficiency \cite{wang2024enhancing}. Complex interactions may impair cognitive performance, especially for first-time users \cite{sundar2014user}. & 
Gradual animations optimize information flow, reducing interference \cite{lockyer2012affective}. Smooth animations enhance visual processing \cite{yoo2005processing}. & 
Logical navigation structure reduces search time, enhancing decision-making \cite{sheng2012effects}. Predictable navigation improves familiarity and confidence \cite{abdelaal2023accessibility}. \\ \cline{2-5}

& \cellcolor[HTML]{F0EFF7} \textbf{Motivational Systems} & 
Reward-based feedback (e.g., task completion sound) boosts achievement motivation \cite{amoor2014designing, wang2024enhancing}. Complex interactions may reduce motivation, especially for new users \cite{wodehouse2014exploring}. & 
Dynamic design (e.g., layered animations) fosters competition or challenge motivation \cite{lockyer2012affective}. Attention-grabbing effects (e.g., floating buttons) encourage exploration \cite{wollner2018slow}. & 
Interactive navigation (e.g., drag-based interfaces) stimulates exploratory motivation \cite{amoor2014designing}. Dynamic task maps encourage curiosity and learning \cite{sundar2014user}. \\ \hline

\end{tabularx}
\caption{Emotional Dimensions and Multisystem Activation in Interaction Design Elements}
\label{tab:interaction_design}
\end{table*}

% \begin{figure*}[hbt!]
% %\setlength{\abovecaptionskip}{-0.1mm}
% \setlength{\intextsep}{10pt plus 2pt minus 2pt}
%     \centering
%     \includegraphics[width=18cm]{figs/interaction_design.png}
%     \caption{Emotional Dimensions and Multisystem Activation in Interaction Design Elements.}
%    \vspace{-2mm}
% \label{fig:why}
% \end{figure*}

\subsubsection{Interaction Methods}
\begin{wrapfigure}{l}{0.06\textwidth}
  %\begin{center}
   \vspace{-11pt} % 调整垂直位置
        \includegraphics[width=0.07\textwidth]{figs/icon/interaction_methods.png}
  %\end{center}
\end{wrapfigure} 
Interaction methods, as a core element of user experience design, profoundly influence users’ emotional experiences, behavioral performance, and the overall effectiveness of system interactions. By optimizing the intuitiveness of interactions, feedback mechanisms, and tactile or gesture-based operations, significant improvements can be achieved in users’ emotional valence, arousal levels, and dominance, while also stimulating the coordinated engagement of sensory-motor systems, neural systems, cognitive systems, and motivational~systems.
% 交互方式作为用户体验设计中的核心要素,深刻影响用户的情绪体验、行为表现以及系统交互的整体效果。通过优化交互的直观性、反馈机制以及触觉与手势操作,可以显著改善用户的情绪效价、唤起度与控制感,同时激发感官运动系统、神经系统、认知系统和动机系统的协同作用。

Intuitive interaction methods, with their simplicity and low cognitive load, significantly enhance users’ positive emotional valence. For example, common interaction methods such as clicking and sliding reduce learning costs, enabling users to quickly adapt to interface logic and complete tasks smoothly, thereby enhancing operational fluidity and comfort \cite{sundar2014user}. Effortless and natural gestures, such as single-finger swiping and tapping, can activate the sensory-motor system, allowing users to perceive the smoothness and coordination of interactions, further enhancing the enjoyment of the experience \cite{wodehouse2014exploring}.
% 直观的交互方式,因其操作简便和认知负荷低,能够显著提升用户的正向情绪效价。例如,点击与滑动等常见交互方式,通过降低学习成本,使用户快速适应界面逻辑并顺利完成任务,增强了操作流畅性与舒适感(Sundar et al., 2014)。这些轻松自然的手势操作(如单指滑动、轻触)还能够激活感官运动系统,使用户在操作过程中感受到交互的顺畅性和身体感知的协调性,进一步提升交互体验的愉悦感(Wodehouse & Sheridan, 2014)。

In contrast, complex interaction methods (e.g., multi-finger gestures, cumbersome task flows) increase users’ operational burden and learning costs, significantly leading to frustration and anxiety. For instance, in mobile applications requiring complex gestures, users may feel frustrated due to difficulty in mastering operations quickly, which significantly reduces emotional valence and willingness to engage. Additionally, such high-complexity operations may cause unnecessary physical tension, such as muscle strain or hand fatigue, further diminishing users’ trust and satisfaction with the interface \cite{wodehouse2014exploring}. This negative emotional experience is particularly pronounced for first-time users of a new system, potentially leading to a negative overall evaluation of the interaction.
% 相比之下,复杂的交互方式(如多指手势、繁琐的任务流程)则会增加用户的操作负担和学习成本,导致挫败感与焦虑感的显著增加。例如,在需要复杂手势的移动应用中,用户可能因无法快速掌握操作方式而感到沮丧,显著降低情绪效价和参与意愿。此外,这种高复杂度操作还可能引发不必要的生理紧张状态,如肌肉紧张或手部疲劳,进一步削弱用户对界面的信任和满意度(Wodehouse & Sheridan, 2014[2])。尤其是首次接触新系统的用户,这种负面情绪体验尤为突出,容易对整体交互评价产生负面影响。

Real-time feedback mechanisms play a crucial role in interaction design. By utilizing visual (e.g., button highlights), auditory (e.g., notification sounds), or haptic (e.g., vibrations) feedback, users can immediately perceive the results of their actions, significantly reducing emotional stress caused by uncertainty \cite{wang2024enhancing}. Haptic feedback is particularly important; micro-vibrations or warm tactile sensations activate the sensory-motor system, enhancing users’ physical perception of operations while improving intuitiveness and comfort \cite{olugbade2023touch}. For example, in navigation applications, combining tactile vibrations upon confirmation with visual highlights helps users clearly perceive task completion, significantly enhancing their dominance and achievement.
% 即时反馈机制在交互设计中扮演了重要角色。通过视觉(如按钮高亮)、声音(如提示音)或触觉(如振动)反馈机制,用户能够及时感知操作结果,从而显著减少不确定性带来的情绪压力(Wang, 2024)。触觉反馈尤其重要,微振动或温暖的触感能够通过激活感官运动系统,增强用户对操作的身体感知,同时提升直观性和舒适感(Olugbade et al., 2023)。例如,在导航应用中,点击确认的触觉振动与视觉高亮结合,能够帮助用户明确任务的完成状态,从而显著提升控制感和成就感。

Exploratory interaction design, by offering freedom and reward feedback, enhances users’ emotional arousal and significantly stimulates their exploratory motivation. For example, features like map zooming and interface dragging allow users to freely adjust perspectives and explore system functionalities, enhancing their dominance over tasks and positive emotional experiences \cite{amoor2014designing}. Additionally, reward-based feedback (e.g., visual cues or sound effects upon task completion) activates the brain’s reward center, further enhancing users’ sense of achievement and engagement. This design is particularly effective in gamified interfaces or learning platforms, motivating users to continue interacting and enriching the depth and breadth of their emotional experience~\cite{wang2024enhancing}.
% 探索性交互设计通过提供自由度和奖励反馈,不仅提高了用户的情绪唤起水平,还显著激发了探索动机。例如,地图缩放和界面拖动等功能设计,能够让用户自由调整视角并探索系统功能,增强了对任务的掌控感和积极情绪体验(Amoor Pour, 2014)。此外,奖励式反馈(如任务完成后的视觉提示或音效)通过激活大脑的奖励中枢,进一步增强了用户的成就感和参与积极性。这种设计在游戏化界面或学习平台中尤为有效,通过激励用户持续操作,提升了情感体验的深度与广度(Wang, 2024)。

Interaction methods, by optimizing operational fluidity, feedback mechanisms, and emotional incentives, can significantly enhance users’ emotional experiences and behavioral performance across multiple dimensions. Designers, by refining interaction details, can effectively activate users’ sensory-motor systems, neural systems, cognitive systems, and motivational systems, achieving emotional regulation and behavioral guidance in various contexts. This user-centered design philosophy not only enhances the usability of the system but also provides a more systematic theoretical foundation and practical guidance for emotional design.
% 交互方式通过优化操作流畅性、反馈机制和情感激励,能够在多维度上显著提升用户的情绪体验和行为表现。设计师通过细化交互细节,可以有效激活用户的感官运动系统、神经系统、认知系统和动机系统,在不同情境中实现情绪调控和行为引导的目标。这种以用户为中心的设计理念,不仅增强了系统的使用价值,也为情感化设计提供了更加系统化的理论依据和实践指导。

\subsubsection{Motion Effects}
\begin{wrapfigure}{l}{0.06\textwidth}
  %\begin{center}
  %\vspace{-11pt} % 调整垂直位置
        \includegraphics[width=0.07\textwidth]{figs/icon/motion_effects.png}
  %\end{center}
\end{wrapfigure} 
Motion effects, as an indispensable element of interaction design, significantly influence users’ emotional experiences and behavioral responses through characteristics such as speed, path, direction, and complexity.
% 动效作为交互设计中不可或缺的元素,通过速度、路径、方向和复杂度等特性显著影响用户的情绪体验和行为反应。
The speed of motion effects is a crucial factor in influencing the level of emotional arousal. Fast motion effects (such as rapid page transitions or zooming) can stimulate the visual tracking system and the sensory-motor system, enhancing dynamic perception intensity, thereby increasing emotional arousal levels. These effects are suitable for attracting attention or creating a tense atmosphere \cite{hanjalic2005affective}. Conversely, slow motion effects (such as pages gradually unfolding or smooth loading) extend users’ emotional perception time, creating a calm or contemplative atmosphere. These low-dynamic effects can activate the parasympathetic nervous system, reducing users’ physiological tension levels, which helps enhance emotional valence in meditation or relaxation contexts \cite{wollner2018slow}.
% 动效的速度 是影响情绪唤起水平的重要因素。快速动效(如页面快速切换或缩放)能够激发视觉追踪系统和感官运动系统,提升动态感知强度,从而增强情绪唤起水平,适用于吸引注意力或营造紧张氛围的场景(Hanjalic & Xu, 2005)。相反,慢动作动效(如页面缓缓展开或平滑加载)通过延长用户的情绪感知时间,营造出平静或沉思的氛围。这类低动态动效能够通过激活副交感神经系统,降低用户的生理紧张水平,有助于在冥想或放松情境中提升情绪效价(Wöllner et al., 2018)。

The path and direction of motion effects profoundly influence user emotions. Linear motion conveys stability and reliability, enhancing users’ trust in the interface and their emotional valence. Conversely, curved or irregular paths may evoke tension and unease, making them suitable for warning or emergency designs \cite{lockyer2012affective}. Additionally, the design of motion direction also impacts users’ emotional perception. Inward motion typically conveys attraction and affinity, helping to enhance user engagement. Outward motion, on the other hand, often evokes a sense of detachment, making it suitable for undo notifications or exit~animations\cite{lockyer2012affective}.
% 动效的路径和方向 也深刻影响用户情绪。直线运动传递稳定性和可靠性,提升用户对界面的信任感和情绪效价;而曲折或不规则路径可能引发紧张和不安情绪,适合警示或紧急提示的设计(Lockyer & Bartram, 2012)。此外,运动方向的设计也会影响用户的情绪感知。向内运动通常带来吸引力和亲和感,有助于增强用户的参与度;而向外运动则容易引发疏离感,适用于撤销提示或结束动画。

Animation complexity plays a critical role in emotional regulation. Moderately complex animation designs can activate the cognitive system by capturing attention, optimize information processing efficiency, and enhance users’ emotional engagement \cite{yoo2005processing}. For example, progressively unfolding interface animations can optimize information flow and enhance memory retention. However, excessive complexity may increase cognitive load, leading to frustration and emotional stress, especially in multitasking scenarios where it can reduce emotional valence \cite{hanjalic2005affective}.
% 动画复杂度 在情绪调节中发挥关键作用。适度复杂的动画设计能通过吸引注意力激活认知系统,优化信息处理效率,并增强用户的情绪参与感(Yoo & Kim, 2005)。[5]例如,渐进式展开的界面动画能够优化信息流通并增强记忆效果。然而,过高的复杂度可能增加信息负荷,引发挫败感和情绪压力,尤其在多任务场景中容易降低情绪效价(Hanjalic & Xu, 2005)。

Animation effects can also activate the user’s motivation system, enhancing emotional engagement and behavioral intention. For example, high-motion designs (e.g., hover buttons or layered animations) stimulate exploratory and achievement motivation by enhancing dynamic effects, increasing user engagement and task completion rates \cite{wollner2018slow}. Dynamic navigation effects (e.g., auto-scrolling or page transitions) leverage instant feedback mechanisms to enhance users’ perception of task completion status, further increasing their dominance and~confidence.
% 动效还能够激发用户的动机系统,增强情绪参与感和行为意愿。例如,高动感设计(如悬浮按钮或层叠动画)通过增强动态效果激发探索动机和成就动机,提升用户参与意愿和任务完成度(Wöllner et al., 2018)。动态导航效果(如自动滚动或页面过渡)通过即时反馈机制,增强用户对任务完成状态的感知,进一步提高控制感和信心。

Animation design can optimize emotions for various scenarios through the adjustment of speed, path, direction, and complexity. For instance, high-dynamic effects are suitable for scenarios requiring heightened emotional arousal or high engagement, while low-dynamic effects and slow motion are better for conveying calmness and contemplative emotions. The adjustment of animation complexity requires a balance between emotional stimulation and information processing to maximize user acceptance and content retention. By designing animation characteristics appropriately, designers can effectively enhance the emotional dimensions of user experience.
% 动效设计通过速度、路径、方向和复杂度的调控,可以实现针对不同场景需求的情绪优化。例如,高动态效果适用于需要提升情绪唤起的紧张或高参与度场景,而低动态效果和慢动作更适合传递平静与沉思的情绪状态。动画复杂度的调节需要在情绪刺激和信息处理之间找到平衡点,以最大化用户的接受度和内容记忆效果。通过合理设计动效特性,设计师能够有效提升用户体验的情感维度。

\subsubsection{Navigation design}
\begin{wrapfigure}{l}{0.06\textwidth}
  %\begin{center}
  \vspace{-11pt} % 调整垂直位置
        \includegraphics[width=0.07\textwidth]{figs/icon/Navigation_design.png}
  %\end{center}
\end{wrapfigure} 
Navigation design profoundly impacts users’ emotional valence, emotional arousal, and motivation system activation by optimizing information architecture and path guidance. A streamlined navigation structure, dynamic navigation elements, and consistent design collectively contribute to smoother user experiences and stronger positive emotional~feedback.
% 导航设计通过优化信息架构和路径引导,对用户情绪效价、情绪唤起以及动机系统的激发产生深远影响。简洁的导航结构、动态导航元素和一致性设计共同作用,帮助用户获得更流畅的操作体验和更强的情绪正向反馈。

Simplified navigation is a key factor in enhancing users’ emotional valence. Clear menu hierarchies and intuitive structures help users quickly locate target information, reducing exploration time and operational difficulty, thus bringing joy and satisfaction \cite{wang2024enhancing}. For example, on e-commerce platforms, users can quickly find desired products through one-click filtering functions, significantly reducing operational burdens and enhancing trust and willingness to use the platform \cite{abdelaal2023accessibility}. In contrast, multi-layered nested menus or hidden navigation controls may increase users’ cognitive load and frustration, particularly when users struggle to adapt quickly, making such emotional reactions especially pronounced \cite{sheng2012effects}.
% 简洁导航 是提升用户情绪效价的关键因素。清晰的菜单层级和直观的结构能够帮助用户快速定位目标信息,减少探索时间和操作难度,从而带来愉悦和满足感(Wang, 2024)。例如,在电商平台中,用户通过一键筛选功能快速找到所需商品,可以显著降低操作负担,提高对平台的信任感和使用意愿(Abdelaal & Al-Thani, 2023)。与此相对,多层嵌套菜单或隐藏式导航控件可能增加用户的认知负荷和挫败感,尤其当用户无法快速适应时,这种情绪反应尤为显著(Sheng et al., 2012)。

Dynamic navigation design plays a significant role in enhancing user engagement and emotional arousal levels. For instance, the expansion effect of dynamic menus or sliding transitions in path selection enhance users’ perception of task status through the integration of visual and tactile elements \cite{sundar2014user}. These designs also activate the sensory-motor system, enhancing users’ dominance and interaction fluency through dynamic visual stimuli and tactile feedback, thereby making the operation process more enjoyable \cite{lockyer2012affective}. The advantage of dynamic navigation lies in its ability to balance visual appeal and functionality, aiding users in focusing on their task~objectives.
% 动态导航设计 在提升用户参与感和情绪唤起水平方面具有显著作用。例如,动态菜单的展开效果或路径选择中的滑动过渡,通过视觉与触觉的结合,增强了用户对任务状态的感知(Sundar et al., 2014)。这些设计还激活了感官运动系统,通过动态视觉刺激和触觉反馈,提升用户的控制感和交互流畅性,从而使操作过程更加愉悦(Lockyer & Bartram, 2012)。动态导航的优势在于其能够平衡视觉吸引力与功能性,帮助用户专注于任务目标。

Consistency design is the foundation for building a reliable navigation system. Navigation layouts and intuitive operational logic aligned with users’ cognitive models can significantly enhance their familiarity and trust in the system \cite{abdelaal2023accessibility}. For instance, common menu categorization and consistent navigation styles can reduce users’ learning costs, decrease uncertainty during operations, and enhance their dominance over the system. Conversely, navigation anomalies such as dead links or inactive buttons may undermine users’ trust in the interface, causing emotional unease and a sense of loss of control, thereby reducing engagement willingness \cite{sheng2012effects}.
% 一致性设计 是构建可靠导航系统的基础。符合用户认知模型的导航布局和直观的操作逻辑能够显著提升用户的熟悉感和信任感(Abdelaal & Al-Thani, 2023)。例如,常见的菜单分类和一致的导航风格能够降低用户的学习成本,减少操作中的不确定性,并提高用户对系统的控制感。反之,导航中的异常(如死链接或无效按钮)可能破坏用户对界面的信任,引发情绪上的不安和失控感,从而降低参与意愿(Sheng et al., 2012)。

Exploratory navigation design stimulates the activity of the motivational system. For example, navigation designs that allow users to drag, zoom, or expand multi-layer information maps create more opportunities for active participation and exploration. These designs enhance operational enjoyment and a sense of accomplishment, encouraging users to continue interacting \cite{amoor2014designing}. Additionally, the predictability and logic of navigation can boost users’ confidence in completing tasks, thereby enhancing operational enjoyment and motivation.
% 探索性导航设计 激发了动机系统的活跃。例如,允许用户自由拖动、缩放或展开多层信息地图的导航设计,为用户创造了更多主动参与和探索的机会。这类设计通过增加操作的趣味性和成就感,使用户更愿意继续互动(Amoor Pour, 2014)。此外,导航的可预测性和逻辑性也能增强用户对任务完成的信心,从而提升操作的愉悦感和动机。


% \subsubsection{Summary}
