\section{Introduction}

% 从science的论文开始写,通过大量的数据事实证明,情感对于信息的传播,无论是真实还是虚假的都有作用,但是情感到底怎么影响传播没有解决,过去10多年在研究这个主题,但是并没有人做一个清晰梳理,然后引出这篇论文

% 写作思路:
% 1. 研究表明,情感对于信息传播是非常关键的,但现有研究缺乏系统的梳理,情感到底如何影响信息传达的。我们通过情感增强或者削弱信息传达是学而未决的。我们通过大量的研究解决这些问题。
% 2. 因为我们要了解人类情感如何被激活,激活的机制里面,是否与设计相关。
% 3. 为此我们调研了先调研了 why(为什么情感在信息已传达很重要)?  情感因素怎样影响信息传达的。 情感的激活机制有哪些,这些机制如何与设计相关联,怎样通过设计促进或者削弱信息传达效果的目的。

%第一段: 介绍背景与问题,第二段:介绍现有解决方案,引出缺陷; 第三段:说明解决这些缺陷很难,面临的挑战有哪些 总结出来; 第四段:介绍技术方案 并说明为什么能够解决挑战; 第五段 总结贡献

In the era of informatization and globalization, the development of internet and mobile communication technologies has greatly accelerated the speed and broadened the reach of information communication. Effective information communication relies not only on technological factors but also on whether the information can be fully understood, remembered, and further propagated by the recipients. Emotions play a crucial role in this process. Research by Vosoughi et al. \cite{vosoughi2018spread} indicates that emotions significantly influence the patterns and scope of information communication. Their study found that emotional content spreads faster and further than neutral information. In particular, false news spreads significantly more than true news because it evokes strong emotions such as fear, surprise, and disgust, while true news tends to elicit emotions like trust, joy, and sadness. These findings suggest that emotions not only affect how information is received, but also significantly influence how it is remembered and communicated. Therefore, exploring the role of emotions in information communication—especially through design strategies that activate emotions to optimize communication effects—has become a crucial and urgent area of research.
% 在信息化和全球化的时代,互联网和移动通信技术的发展显著加速了信息的传播速度和覆盖范围。信息的有效传播不仅依赖于技术因素,还需要考虑信息是否能够被接收者充分理解、记忆并进一步传播。情感在这一过程中扮演了至关重要的角色。Vosoughi等人(2018)的研究表明,情感显著影响了信息传播的模式和范围。研究发现,与中性信息相比,情感化内容传播得更快、更广,其中假新闻因其引发恐惧、惊讶和厌恶等强烈情绪而显著超过真新闻的扩散范围和深度,而真新闻更多引发信任、喜悦和悲伤等情绪。这些现象反映出情感不仅改变了信息的接受模式,也对信息的记忆与扩散规律产生了重要影响。因此,系统性探讨情感在信息传播中的作用,尤其是通过设计激活情感以优化传播效果,成为一个亟待解决的重要课题。

Existing studies have explored the impact of emotions on information communication from various perspectives. Numerous studies indicate that emotion-driven information communication has broad applications in advertising\cite{bigne2023influence}, health campaigns \cite{montijn2021forgetting, kok2013positive}, political communication \cite{schreiner2021impact, de2021sadness, dabbous2023influence}, and education\cite{plass2014emotional}. Positive emotions such as joy and anticipation often enhance the appeal and communication of information, whereas negative emotions such as fear and anger may deepen cognitive processing but also increase the risk of spreading misinformation \cite{megalakaki2019effects, arfe2023effects}. Additionally, studies suggest that the design of text, visuals, sound, and interaction can significantly enhance memory retention of information and accelerate its communication on social media \cite{schreiner2021impact, dabbous2023influence}. This phenomenon is corroborated by Vosoughi et al. \cite{vosoughi2018spread}, who found that false news is more likely to be shared due to its novelty and emotional arousal. While these studies provide important insights into the role of emotions in information communication, a systematic framework to integrate these findings and guide practice remains lacking.

% 在现有研究中,学者们从多个角度探讨了情感对信息传播的影响。多项研究指出情感驱动的信息传播在广告营销、健康宣传(Montijn et al. 2021; Kok et al., 2013)、政治传播(Schreiner等(2021); de León, E., & Trilling, D. (2021).Dabbous 和 Aoun Barakat(2023))和教育(Plass et al. 2014)等领域中具有广泛的应用价值。正面情感如喜悦和期待通常能提升信息的吸引力和传播效果,而负面情感如恐惧和愤怒则可能加深对信息的认知加工,尽管同时也增加了错误信息传播的风险(Megalakaki et al.(2019),Arfé等(2023))。此外,研究还指出,通过文本、视觉、声音以及交互的设计可以显著增强信息的记忆效果,并加速其在社交媒体上的扩散(Schreiner等(2021); Dabbous 和 Aoun Barakat(2023)),这种现象也得到了Vosoughi等人(2018)的验证,假新闻因其更具新颖性和情绪激发性而更容易被分享。尽管这些研究为理解情感在信息传播中的作用提供了重要见解,但尚缺乏系统性框架来整合这些发现并指导实践。

Despite the progress made in emotional information communication research, several challenges remain. First, much of the existing research focuses primarily on describing the phenomenon of emotional content transmission, without deeply analyzing the mechanisms through which emotions influence communication.  Specifically, the ways in which emotions activate users’ cognition and behavior to enhance or diminish the effectiveness of communication have not been fully explored.  Second, current research tends to focus on emotional strategies within a single design dimension (e.g., text or visuals), lacking a systematic integration of multidimensional design approaches such as text, visuals, sound, and interaction.  These challenges limit the practical applicability of emotional design.
% 尽管情感化信息传播研究取得了许多进展,但仍面临以下几个显著挑战:(1)现有研究多聚焦于情感化内容传播的现象描述,而未深入分析情感影响信息传播的内在机制。尤其是情感如何通过激活用户的认知与行为,优化或削弱信息传播效果的逻辑尚未被充分阐明。(2)当前的研究多集中于单一设计维度(如文本或视觉)的情感化策略,而缺乏对文本、视觉、声音和交互等多维设计方法的系统整合。这些挑战使得情感化设计在实践中的指导价值有限。
        

This study addresses these challenges by focusing on three key questions:
(1) why are emotions so crucial in information communication?
(2) what emotions factors influence information communication?
(3) how can design activate emotions to optimize information communication?
In \cref{sec:why}, the study explains the importance of emotions in information communication from the perspectives of communication, psychological, neurobiology, sociology,  and humanism. In \cref{sec:what}, it summarizes the specific effects of emotional valence (positive or negative), arousal (emotional intensity), and dominance (emotional control) on information communication, uncovering the underlying logic of how emotions influence communication.

\enlargethispage{-1cm}


Finally, in \cref{sec:how}, the study integrates emotional strategies in text, visual, sound, and interaction design to propose a comprehensive design space.  It also introduces Izard’s \cite{izard1993four} “four-system” model of emotional activation—which includes the neural system, sensorimotor system, motivational system, and cognitive system—as a framework for studying emotional regulation. Ultimately, the study integrates theory and practice to develop design strategies tailored to various scenarios.
% 为了应对这些挑战,本研究围绕以下三个核心问题展开:(1) 为什么情感在信息传递中如此重要?(2) 哪些情感因素影响了信息传达?(3) 如何通过设计激活情感,以优化信息的传播效果?为此,本研究从“为什么(Why)”、“是什么(What)”和“如何(How)”三个层次进行系统探讨。在“Why”部分,本研究从传播学、心理学、社会学、神经科学以及人文主义角度阐述了情感对信息传播的关键性。在“What”部分,总结了情感效价(正面或负面)、唤起水平(情绪强度)和支配感(情绪控制感)对信息传播的具体影响,揭示了情感影响信息传播的内在逻辑。最后,从“如何(How)”的角度,本研究整合了文本、视觉、声音和交互设计的情感化策略,提出了一套设计空间,同时,引入了情感激活系统理论Izard(1993)提出的情绪激活“四系统”模型(包括神经系统、感官运动系统、动机系统与认知系统)为研究情感调控提供了系统框架,最终整合理论与实践,制定在不同场景下的设计策略。

The primary contributions of this study are threefold:
\begin{itemize}
    \item First, this study systematically reviews existing literature and deeply analyzes the critical role of emotions in information communication and its underlying mechanisms, providing an integrated and multidimensional theoretical framework for future research.
    \item Second, the study proposes an emotional design space based on four core dimensions—text, visuals, sound, and interaction—systematically summarizing the specific roles and practical strategies of each design dimension in emotional activation and communication optimization. This framework offers comprehensive theoretical guidance and practical support for designers.
    \item Finally, the study presents actionable solutions for emotional communication design tailored to diverse contexts, providing a solid foundation for both academic research and practical applications. These solutions encompass narrative structures, visual layouts, auditory elements, and interactive feedback.
\end{itemize}

% 本研究的主要贡献体现在三个方面。首先,本研究通过系统梳理现有文献,深入分析了情感对信息传播的重要作用及其背后的深层机制,为未来相关研究提供了整合性和多维度的理论框架。其次,本研究提出了一套基于文本、视觉、声音和交互四个核心维度的情感化设计空间,系统总结了各设计维度在情感激发和传播优化中的具体作用及实践策略。这一框架为设计者提供了全面的理论指导和实践支持。最后,本研究为情感化传播设计提出了具体可行的方案,为不同情境下的情感化传播提出了具体可行的设计方案,从叙事结构、视觉布局到声音背景和交互反馈,为学术研究与实际应用奠定了坚实基础。