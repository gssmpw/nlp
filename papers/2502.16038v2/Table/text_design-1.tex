% \captionsetup{font=small}
% \renewcommand{\arraystretch}{1.6} % 调整行间距

\begin{table}[ht]
\fontsize{8}{9}\selectfont
\centering
\rowcolors{2}{gray!10}{white}

\begin{tabularx}{\textwidth}{|>{\centering\arraybackslash}m{1.5cm}|>{\centering\arraybackslash}m{1.8cm}|>{\centering\arraybackslash}m{1.8cm}|>{\centering\arraybackslash}m{1.8cm}|>{\centering\arraybackslash}m{1.8cm}|>{\centering\arraybackslash}m{1.8cm}|>{\centering\arraybackslash}m{1.8cm}|>{\centering\arraybackslash}m{1.8cm}|}

\hline
\rowcolor[HTML]{D9EAD3} 
\textbf{Sub-Design Element} & \multicolumn{3}{c|}{\textbf{Emotional Dimensions}} & \multicolumn{4}{c|}{\textbf{Multisystem Activation}} \\ \cline{2-8}
\rowcolor[HTML]{D9EAD3} 
& \textbf{Valence} & \textbf{Arousal} & \textbf{Dominance} & \textbf{Neural Systems} & \textbf{Sensorimotor Systems} & \textbf{Motivational Systems} & \textbf{Cognitive Systems} \\ \hline

\textbf{Headline} & 
Familiar or celebrity references evoke positive emotions \cite{kim2016compete}. & 
Concise titles and symbolic elements enhance emotional arousal \cite{kourogi2015identifying}. & 
Clear, intuitive titles boost user control; threatening titles weaken dominance \cite{kim2016compete}. & 
Threatening titles activate the amygdala, triggering physiological responses \cite{panksepp2012archeology}. & 
Urgent titles may cause physical reactions, such as muscle tension or rapid breathing \cite{james1884mind}. & 
Positive titles evoke exploratory motivation; urgent titles stimulate avoidance \cite{calvo2010affect}. & 
Titles guide quick judgments about situations, evoking positive or negative emotions \cite{lazarus1991emotion}. \\ \hline

\textbf{Narrative Structure} & 
Positive narrative openings increase valence; negative ones reduce it \cite{mar2011emotion}. & 
Suspense and peak events intensify emotional fluctuations \cite{leshner2018breast}. & 
Heroic narratives boost control; failure endings weaken dominance \cite{mar2011emotion}. & 
Tense scenes activate the amygdala, triggering fear or excitement \cite{jaaskelainen2020neural}. & 
Conflict or climactic events may induce physical reactions, like body stiffness \cite{james1884mind}. & 
The hero’s journey evokes achievement motivation; plot twists trigger exploration \cite{campbell2008hero}. & 
Narratives help users evaluate emotional contexts, evoking hope or sadness \cite{lazarus1991emotion}. \\ \hline

\textbf{Narrative Content} & 
Positive events evoke joy or relief; negative events induce sadness or anxiety \cite{nguyen2014affective}. & 
Intense content heightens arousal \cite{lekkas2022using}. & 
Success stories boost control; alternating emotions enhance dominance \cite{nguyen2014affective}. & 
Descriptions of fear activate adrenaline via neural responses \cite{panksepp2012archeology}. & 
Vivid depictions (e.g., conflict or fear) trigger mimicry (e.g., holding breath) \cite{james1884mind}. & 
Adventurous stories evoke exploration, while achievements stimulate pursuit motivation \cite{mar2011emotion}. & 
Readers evaluate characters’ behaviors, eliciting anger or sympathy \cite{lazarus1991emotion}. \\ \hline

\textbf{Description} & 
Positive wording improves valence; loss frames intensify negative feelings \cite{seo2019process}. & 
Strong words (e.g., “pain”) increase arousal; parallel structures enhance appeal \cite{lee2020impact}. & 
Positive words (e.g., “happiness”) boost trust; abrupt phonemes reduce dominance \cite{menninghaus2017emotional}. & 
Highly emotional words trigger neural responses like tension or excitement \cite{calvo2010affect}. & 
Emotional words may evoke physical mimicry (e.g., frowning) \cite{panksepp2012archeology}. & 
Positive language promotes pursuit behavior; negative language triggers avoidance \cite{lazarus1991emotion}. & 
Wording influences user evaluation, leading to optimism or caution \cite{seo2019process}. \\ \hline

\end{tabularx}

\caption{Emotional Dimensions and Multisystem Activation in Text Design Elements}
\label{tab:enhanced_design}
\end{table}