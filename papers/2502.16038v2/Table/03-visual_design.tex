\renewcommand{\arraystretch}{1.8} % 调整行间距
\begin{table*}[ht]
\fontsize{8}{9}\selectfont
\centering

\begin{tabularx}{\textwidth}{|>{\centering\arraybackslash}m{0.5cm}|>{\centering\arraybackslash}m{1.4cm}|>{\centering\arraybackslash}m{3.4cm}|>{\centering\arraybackslash}m{3.4cm}|>{\centering\arraybackslash}m{3.4cm}|>{\centering\arraybackslash}m{3.4cm}|}
\hline
\rowcolor[HTML]{D9EAD3} 
\multicolumn{2}{|c|}{\textbf{Dimension}} & \textbf{Color} & \textbf{Shape} & \textbf{Images} & \textbf{Layout} \\ \hline

% Emotional Dimensions Section
\multirow{3}{*}{\rotatebox{90}{\parbox{3cm}{\centering \textbf{Emotional \\ Dimensions}}}} & 
\cellcolor[HTML]{FDF6E8} \textbf{Valence} & 
\textbf{Warm colors} (e.g., red, yellow) evoke positive emotions and energy \cite{jonauskaite2019color,kallabis2024investigating}. \textbf{Cool colors} (e.g., blue, green) promote calmness \cite{wilms2018color}. & 
\textbf{Round shapes} convey friendliness and warmth \cite{wei2006image}. \textbf{Sharp shapes} evoke alertness \cite{thumfart2008modeling}. & 
\textbf{Positive images} (e.g., nature, smiling faces) enhance positive emotions \cite{hou2024emotional}. \textbf{Negative images} (e.g., disasters) amplify negative emotions \cite{pfeuffer122measuring}. & 
\textbf{Simple layouts} reduce distractions, enhancing positive emotions \cite{lu2017investigation}. \textbf{Symmetrical layouts} evoke balance and trust \cite{fiorini2024role}. \\ \cline{2-6}

& \cellcolor[HTML]{FDF6E8} \textbf{Arousal} & 
\textbf{Warm colors} increase arousal (excitement), while \textbf{cool colors} reduce it (relaxation) \cite{jonauskaite2019color}. & 
\textbf{Complex shapes} heighten arousal \cite{lu2012shape}. 
\textbf{Rounded shapes} are calming \cite{etzi2016arousing}. & 
\textbf{High-intensity images} (e.g., emergencies) increase arousal \cite{pfeuffer122measuring}. \textbf{Peaceful images} reduce arousal \cite{hao2024judging}. & 
\textbf{Complex layouts} boost exploration and arousal \cite{carretie2019emomadrid}. \textbf{Dynamic layouts} add vitality \cite{lu2020exploring}. \\ \cline{2-6}

& \cellcolor[HTML]{FDF6E8} \textbf{Dominance} & 
\textbf{Warm, bright colors} (e.g., light yellow) enhance control, while \textbf{low-brightness cool} colors reduce it \cite{weijs2023effects}. & 
\textbf{Circular shapes} enhance control and safety \cite{lu2012shape}. \textbf{Sharp shapes} decrease control \cite{wei2006image}. & 
\textbf{Wide images} (e.g., 16:9 ratio) enhance control \cite{kuzinas2024creative}. \textbf{Isolated subjects} reduce control \cite{lin2023effect}. & 
\textbf{Rule-of-thirds layouts} enhance control \cite{machajdik2010affective}. 
\textbf{Shallow depth layouts} focus attention and increase control \cite{datta2006studying}. \\ \hline

% Multisystem Activation Section
\multirow{4}{*}{\rotatebox{90}{\parbox{3cm}{\centering \textbf{Multisystem \\ Activation}}}} & 
\cellcolor[HTML]{F0EFF7} \textbf{Neural Systems} & 
\textbf{Colors} activate the sympathetic or parasympathetic systems \cite{ledoux1998emotional}. & 
\textbf{Rounded shapes} relax the nervous system, while \textbf{sharp shapes} induce alertness  \cite{ledoux1998emotional}. & 
\textbf{Emotional images }activate the amygdala, affecting emotions \cite{phelps1998specifying}. & 
\textbf{Simple layouts} induce calm; \textbf{complex layouts} activate stress responses \cite{ledoux1998emotional}. \\ \cline{2-6}

&\cellcolor[HTML]{F0EFF7} \textbf{Sensorimotor Systems} & 
\textbf{Color contrast }affects sensory responses (e.g., pupil dilation, muscle tension) \cite{zajonc1980feeling}. & 
\textbf{Rounded shapes} feel safe; \textbf{sharp shapes} induce tension \cite{lidwell2010universal}. & 
\textbf{Emotional images} trigger physical reactions (e.g., smiling or frowning)\cite{ekman1992argument}. & 
\textbf{Symmetrical layouts} evoke harmony; \textbf{asymmetrical} ones may cause unease \cite{machajdik2010affective}. \\ \cline{2-6}

& \cellcolor[HTML]{F0EFF7} \textbf{Cognitive Systems} & 
\textbf{Colors} influence cognitive assessment (e.g., red for danger, green for relaxation) \cite{zajonc1980feeling}. & 
Shapes impact safety perception (e.g., circles feel safe, sharp shapes signal danger) \cite{lidwell2010universal}. & 
Emotional expressions in images trigger empathy and cognitive responses \cite{hou2024emotional}. & 
\textbf{Symmetrical layouts} provide cognitive ease; \textbf{asymmetrical} ones may confuse \cite{carretie2019emomadrid}. \\ \cline{2-6}

&\cellcolor[HTML]{F0EFF7} \textbf{Motivational Systems} & 
\textbf{Warm colors} (e.g., red, yellow) stimulate urgency and action, while \textbf{cool colors} promote comfort and trust \cite{phelps1998specifying}. & 
\textbf{Rounded shapes }evoke trust; \textbf{sharp shapes} stimulate urgency or challenge \cite{lidwell2010universal}. & 
\textbf{Achievement-oriented images} (e.g., celebrations) inspire motivation \cite{pfeuffer122measuring}. & 
\textbf{Simple layouts} enhance task motivation;\textbf{ complex layouts} increase cognitive load \cite{lu2020exploring}. \\ \hline
\end{tabularx}

\caption{Emotional Dimensions and Multisystem Activation in Visual Design Elements} 
\label{tab:visual_design}
\end{table*}