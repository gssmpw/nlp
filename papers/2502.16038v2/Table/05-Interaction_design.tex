\renewcommand{\arraystretch}{1.6} % 调整行间距
\captionsetup{font=small}

\begin{table*}[ht]
\fontsize{8}{9}\selectfont
\centering
\begin{tabularx}{\textwidth}{|>{\centering\arraybackslash}m{0.6cm}|>{\centering\arraybackslash}m{1.55cm}|>{\centering\arraybackslash}m{4.6cm}|>{\centering\arraybackslash}m{4.6cm}|>{\centering\arraybackslash}m{4.6cm}|} 
\hline
\rowcolor[HTML]{D9EAD3} 
\multicolumn{2}{|c|}{\textbf{Dimension}} & \textbf{Interaction Method} & \textbf{Motion Effects} & \textbf{Navigation Design} \\ \hline

% Emotional Dimensions Section
\multirow{3}{*}{\rotatebox{90}{\parbox{3cm}{\centering \textbf{Emotional \\ Dimensions}}}} & 
\cellcolor[HTML]{FDF6E8} \textbf{Valence} & 
Intuitive interactions (e.g., tapping, swiping) reduce learning cost, increase comfort and satisfaction \cite{sundar2014user, amoor2014designing}.  Complex interactions (e.g., multi-finger gestures) increase frustration and negative emotions \cite{sundar2014user}. & 
Linear motion effects convey stability, enhancing calmness and attraction \cite{lockyer2012affective}. Complex curves may cause tension and unease \cite{lockyer2012affective}. & 
Simple navigation reduces cognitive load, enhancing positive emotions \cite{wang2024enhancing, abdelaal2023accessibility}. Complex navigation induces confusion \cite{sheng2012effects}.\\ \cline{2-5}

& \cellcolor[HTML]{FDF6E8} \textbf{Arousal} & 
Simple gestures (e.g., tapping, swiping) create a natural, smooth experience, controlling arousal \cite{wodehouse2014exploring}. Strong haptic feedback attracts attention but may cause tension \cite{olugbade2023touch}. &  
High-intensity effects (e.g., rapid cuts) increase attention and arousal \cite{hanjalic2005affective}. Slow motion creates a calming atmosphere \cite{wollner2018slow}. & 
Immediate feedback (e.g., visual or tactile cues) reduces anxiety, stabilizing emotions \cite{sundar2014user, wang2024enhancing}. Dynamic navigation enhances engagement and arousal \cite{amoor2014designing}. \\ \cline{2-5}

& \cellcolor[HTML]{FDF6E8} \textbf{Dominance} & 
Free exploratory interactions (e.g., drag and zoom) enhance user control and engagement \cite{amoor2014designing}.  Lengthy, complex tasks reduce control \cite{wang2024enhancing}. & 
Dynamic navigation (e.g., expandable menus) boosts user control \cite{amoor2014designing}. Lack of feedback undermines control \cite{wang2024enhancing}. & 
Consistent navigation aligns with user expectations, boosting trust and control \cite{abdelaal2023accessibility}. Errors (e.g., dead links) reduce control \cite{sheng2012effects}. \\ \hline

% Multisystem Activation Section
\multirow{4}{*}{\rotatebox{90}{\parbox{3cm}{\centering \textbf{Multisystem \\ Activation}}}} 

& \cellcolor[HTML]{F0EFF7} \textbf{Neural System} & 
Real-time feedback (e.g., vibration) stimulates sensory neurons, enhancing task awareness \cite{olugbade2023touch}. & 
Dynamic transitions activate the visual pathway, improving neural plasticity \cite{yoo2005processing}. High-intensity visuals (e.g., flashing) trigger short-term tension \cite{hanjalic2005affective}. & 
Dynamic navigation (e.g., auto-scrolling) strengthens neural responses through sensory input \cite{sundar2014user}. Immediate feedback (e.g., click confirmation) enhances achievement sensation \cite{wang2024enhancing}. \\ \cline{2-5}

& \cellcolor[HTML]{F0EFF7} \textbf{Sensorimotor System} & 
Intuitive interactions (e.g., swiping, tapping) activate touch and motion feedback, reinforcing motor learning \cite{sundar2014user}. Complex gestures may cause hand fatigue \cite{wodehouse2014exploring}. & 
Fast animations stimulate dynamic perception \cite{hanjalic2005affective}. Slow motion promotes detailed visual processing \cite{wollner2018slow}. & 
Simplified navigation paths reduce hand-eye coordination stress \cite{wang2024enhancing}.  Dynamic feedback enhances task completion awareness \cite{sundar2014user}. \\ \cline{2-5}

& \cellcolor[HTML]{F0EFF7} \textbf{Cognitive Systems} & 
Intuitive interactions reduce cognitive load, improving task efficiency \cite{wang2024enhancing}. Complex interactions may impair cognitive performance, especially for first-time users \cite{sundar2014user}. & 
Gradual animations optimize information flow, reducing interference \cite{lockyer2012affective}. Smooth animations enhance visual processing \cite{yoo2005processing}. & 
Logical navigation structure reduces search time, enhancing decision-making \cite{sheng2012effects}. Predictable navigation improves familiarity and confidence \cite{abdelaal2023accessibility}. \\ \cline{2-5}

& \cellcolor[HTML]{F0EFF7} \textbf{Motivational Systems} & 
Reward-based feedback (e.g., task completion sound) boosts achievement motivation \cite{amoor2014designing, wang2024enhancing}. Complex interactions may reduce motivation, especially for new users \cite{wodehouse2014exploring}. & 
Dynamic design (e.g., layered animations) fosters competition or challenge motivation \cite{lockyer2012affective}. Attention-grabbing effects (e.g., floating buttons) encourage exploration \cite{wollner2018slow}. & 
Interactive navigation (e.g., drag-based interfaces) stimulates exploratory motivation \cite{amoor2014designing}. Dynamic task maps encourage curiosity and learning \cite{sundar2014user}. \\ \hline

\end{tabularx}
\caption{Emotional Dimensions and Multisystem Activation in Interaction Design Elements}
\label{tab:interaction_design}
\end{table*}