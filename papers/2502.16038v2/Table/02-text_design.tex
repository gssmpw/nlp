\renewcommand{\arraystretch}{1.8} % 调整行间距
\begin{table*}[ht]
\fontsize{8}{9}\selectfont
\centering

\begin{tabularx}{\textwidth}{|>{\centering\arraybackslash}m{0.5cm}|>{\centering\arraybackslash}m{1.4cm}|>{\centering\arraybackslash}m{3.4cm}|>{\centering\arraybackslash}m{3.4cm}|>{\centering\arraybackslash}m{3.4cm}|>{\centering\arraybackslash}m{3.4cm}|}
\hline
\rowcolor[HTML]{D9EAD3} 
\multicolumn{2}{|c|}{\textbf{Dimension}} & \textbf{Headline} & \textbf{Narrative Structure} & \textbf{Narrative Content} & \textbf{Wording} \\ 
\hline

% Emotional Dimensions Section
\multirow{3}{*}{\rotatebox{90}{\parbox{3cm}{\centering \textbf{Emotional \\ Dimensions}}}} & 
\cellcolor[HTML]{FDF6E8} \textbf{Valence} &    
Celebrity references evoke positive emotions and attract readers \cite{kim2016compete}.
Forward cues (e.g., “Here’s why”) increase positivity \cite{blom2015click}. & 
Positive tone (e.g., hope, success) enhances positive valence, while negative tone (e.g., tragedy) intensifies negative emotions \cite{mar2011emotion}. & 
Positive events (e.g., recovery, victory) evoke positive emotions, while negative events (e.g., loss) evoke negative emotions \cite{lekkas2022using}. & 
Gain framing (e.g., positive outcomes) boosts positive valence, while loss framing enhances negative valence \cite{seo2019process}. \\ \cline{2-6}

& \cellcolor[HTML]{FDF6E8} \textbf{Arousal} & 
Concise headlines quickly grab attention \cite{blom2015click}. 
Hot topics and suspense increase arousal \cite{kim2016compete}. & 
Suspenseful or high-peak narratives sustain high arousal levels \cite{leshner2018breast}. & 
Emotional content (e.g., loss or pain) enhances arousal. Resonance with characters amplifies engagement \cite{lekkas2022using}. & 
Emotional language (e.g., “pain”) heightens intensity; parallel structures boost appeal \cite{menninghaus2017emotional}. \\ \cline{2-6}

& \cellcolor[HTML]{FDF6E8} \textbf{Dominance} & 
Clear headlines enhance user control (e.g., “Solve it in one step”) \cite{kourogi2015identifying}. Threatening headlines reduce trust \cite{kim2016compete}. & 
Heroic narratives (e.g., success stories) enhance control; failure narratives reduce it \cite{jaaskelainen2020neural}. & 
Success stories increase control; alternating emotions in complex narratives enrich the experience \cite{lekkas2022using}. & 
Positive language (e.g., “happiness”) boosts control, while abrupt phonemes reduce it \cite{menninghaus2017emotional}. \\ \hline

% Multisystem Activation Section
\multirow{4}{*}{\rotatebox{90}{\parbox{3cm}{\centering \textbf{Multisystem \\ Activation}}}} 

& \cellcolor[HTML]{F0EFF7} \textbf{Neural Systems} & 
Urgent headlines (e.g., “Danger ahead”) activate the amygdala, triggering physiological responses like increased heart rate \cite{panksepp2012archeology}. & 
Tense narratives activate the amygdala, causing fear or excitement; complex stories engage the prefrontal cortex \cite{jaaskelainen2020neural}. & 
Descriptions of fear or anger activate the sympathetic nervous system, leading to adrenaline release \cite{panksepp2012archeology}. & 
Emotionally charged wording triggers the amygdala, inducing tension or excitement \cite{panksepp2012archeology}. \\ \cline{2-6}

& \cellcolor[HTML]{F0EFF7} \textbf{Sensorimotor Systems} & 
Urgent headlines (e.g., “Last chance”) trigger physical responses like muscle tension \cite{james1884mind}. & 
Climax or conflict scenes evoke bodily reactions, such as rapid breathing \cite{panksepp2012archeology}. & 
 Vivid descriptions of conflict or fear may cause muscle tension or mimicry (e.g., holding breath) \cite{james1884mind}. & 
Emotional words (e.g., “tragic”) may provoke physical reactions like frowning or fast breathing \cite{panksepp2012archeology}. \\ \cline{2-6}

& \cellcolor[HTML]{F0EFF7} \textbf{Cognitive Systems} & 
Headlines guide quick judgment (e.g., “A miracle happened” evokes optimism) \cite{lazarus1991emotion}. & 
Narratives help evaluate emotional contexts (e.g., hope in heroic stories, sadness in tragic ones) \cite{mar2011emotion}. & 
Readers assess character actions, evoking emotions like anger or empathy \cite{nguyen2014affective}. & 
Positive words (e.g., “success”) evoke optimism; negative words (e.g., “failure”) evoke pessimism \cite{seo2019process}. \\ \cline{2-6}

& \cellcolor[HTML]{F0EFF7} \textbf{Motivational Systems} & 
Positive headlines (e.g., “Act now”) stimulate motivation, while urgency headlines evoke avoidance \cite{calvo2010affect}. & 
Heroic narratives inspire achievement motivation; twists encourage exploration \cite{campbell2008hero}. & 
Adventure stories spark exploration; success stories foster pursuit motivation \cite{mar2011emotion}. & 
Positive wording inspires pursuit motivation; negative wording triggers avoidance or self-protection \cite{lazarus1991emotion}. \\  \hline

\end{tabularx}

\caption{Emotional Dimensions and Multisystem Activation in Text Design Elements}
\label{tab:text_design}
\end{table*}