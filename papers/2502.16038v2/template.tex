\documentclass[lettersize,journal]{IEEEtran}
\usepackage{amsmath,amsfonts}
\usepackage{algorithmic}
\pdfoutput=1
\usepackage{algorithm}
\usepackage{array}
\usepackage[caption=false,font=normalsize,labelfont=sf,textfont=sf]{subfig}
\usepackage{textcomp}
\usepackage{stfloats}
\usepackage{url}
\usepackage{verbatim}
\usepackage{graphicx}
\usepackage{tabularx}
\usepackage{xcolor}
\usepackage{colortbl}  % 解决表格颜色问题
\usepackage{cite}
\usepackage{wrapfig}
\usepackage{multirow}
\usepackage{caption}
\usepackage{cleveref}
\usepackage{placeins}
\usepackage{caption}

\hyphenation{op-tical net-works semi-conduc-tor IEEE-Xplore}
% updated with editorial comments 8/9/2021

\begin{document}

\author{Shixiong Cao, Nan Cao
        % <-this % stops a space
\thanks{Shixiong Cao and Nan Cao are co-first authors, and Nan Cao is the corresponding author, with the Intelligent Big Data Visualization Lab, Tongji University. Email: \{caoshixiong,nan.cao\}@tongji.edu.cn.}}% <-this % stops a space
%\thanks{Manuscript received April 20, 2025; revised August 16, 2021.}

% 页眉
\markboth{Journal of \LaTeX\ Class Files, February~2025}%
{Shell \MakeLowercase{\textit{et al.}}: A Sample Article Using IEEEtran.cls for IEEE Journals}

\IEEEpubid{0000--0000/00\$00.00~\copyright~2025 IEEE}

\flushbottom

%题目
\title{How Does Emotion Affect Information Communication}
\maketitle

%摘要
\begin{abstract}
In recent years, the pivotal role of emotions in information dissemination has attracted extensive attention. Research indicates that emotionally charged content significantly outperforms neutral content in dissemination speed and coverage due to its ability to evoke stronger emotional responses. Valence, arousal, and dominance are key factors influencing information retention, comprehension, and sharing. However, systematic analysis of the mechanisms by which emotions affect dissemination and the application of multidimensional emotional design remain limited. This study reviews 166 relevant articles, examining the role of emotions in understanding and sharing information, and proposes a multidimensional design framework based on text, visuals, sound, and interaction design. Incorporating the ``Four-System" model of emotion activation, it develops an emotion regulation framework to optimize the application of emotions in dissemination. This research offers theoretical support and practical guidance for emotion-driven dissemination, laying a foundation for advancing this field.
\end{abstract}

\begin{IEEEkeywords}
Emotion, Information communication, Comprehension, Memorization, Sharing, Design space
\end{IEEEkeywords}


% \section*{Main text}

% Total text of article 6,000–8,000 words (i.e. from the Introduction to the Outlook section; without including the abstract, figure captions or text boxes).



% \section*{Introduction}

% We recommend that you keep the introduction to 800 words maximum. The introduction should answer the following questions: what is the topic of the article? Why is this topic of importance? What recent developments make it timely to review this topic now? And how will this topic be reviewed? To answer the last question, we recommend adding a guiding paragraph at the end of your introduction that clearly states what will (and what won’t) be discussed in your Review, so readers and referees will know what to expect.

People engage in activities in online forums to exchange ideas and express diverse opinions. Such online activities can evolve and escalate into binary-style debates, pitting one person against another~\cite{sridhar_joint_2015}. Previous research has shown the potential benefits of debating in online forums such as enhancing deliberative democracy~\cite{habermas_theory_1984, semaan_designing_2015, baughan_someone_2021} and debaters' critical thinking skills~\cite{walton_dialogue_1989, tanprasert_debate_2024}. For example, people who hold conflicting stances can help each other rethink from a different perspective. However, research has also shown that such debates could result in people attacking each other using aggressive words, leading to depressive emotions~\cite{shuv-ami_new_2022}. Hatred could spread among various groups debating different topics~\cite{iandoli_impact_2021, nasim_investigating_2023, vasconcellos_analyzing_2023, qin_dismantling_2024}, such as politics, sports, and gender.

In recent years, people have integrated Generative AI (GenAI) into various writing tasks, such as summarizing~\cite{august_know_2024}, editing~\cite{li_value_2024}, creative writing~\cite{chakrabarty_help_2022, li_value_2024, yang_ai_2022, yuan_wordcraft_2022}, as well as constructing arguments~\cite{jakesch_co-writing_2023, li_value_2024} and assisting with online discussions~\cite{lin_case_2024}. This raises new concerns in online debates. For example, an internally synthesized algorithm of Large Language Models (LLMs) could produce hallucinations~\cite{fischer_generative_2023, razi_not_2024}, which may act as a catalyst for the spread of misinformation in online forums~\cite{fischer_generative_2023}. In addition, GenAI could introduce biased information to forum members~\cite{razi_not_2024}, which may intensify pre-existing debates. Moreover, integrating GenAI into various writing scenarios may also result in weak insights~\cite{hadan_great_2024}, raising concerns about the impact of GenAI on the ecology of online forum debates.

Given these concerns, this study aims to explore how people use GenAI to engage in debates in online forums. The integration of GenAI is not only reshaping everyday writing practices but also has the potential to redefine the online argument-making paradigm. Previous research has demonstrated the potential of co-writing with GenAI, focusing primarily on its influence on individual writing tasks~\cite{august_know_2024, chakrabarty_help_2022, jakesch_co-writing_2023, li_value_2024, yang_ai_2022}. However, the use of GenAI in the context of online debates, which combine elements of both confrontation and collaboration among remote members, remains underexplored. To explore it, we created an online forum for participants to engage in debates with the assistance of ChatGPT (GPT-4o) (\autoref{fig1}). This study enables us to closely observe how people make arguments and analyze their process data of using GenAI. We will examine three research questions to understand how the use of GenAI shapes debates in online forums: 

\begin{itemize}
\item{\textbf{RQ1}: How do people who participate in a debate on online forums collaborate with GenAI in making arguments?}

\item{\textbf{RQ2}: What patterns of arguments emerge when collaborating with GenAI to participate in a debate on online forums?}

\item{\textbf{RQ3}: How does the use of GenAI for making arguments change when a new member joins an existing debate in online forums?}
\end{itemize}

Given the universality and accessibility of debate topics, we chose one that is widely recognized and able to spark intense debates: soccer, which is regarded as the world's most popular sport~\cite{stolen_physiology_2005}. Building on this topic, we selected "Messi vs. Ronaldo: Who is better?" as the case for our study because it has been an enduring and heated debate among soccer fans. We created a small online forum as the platform for AI-mediated debates, particularly focusing on the debates among members and their interactions with ChatGPT. This approach enables more detailed observation and analysis of the entire process while fostering a nuanced understanding. The study consists of two parts: a one-on-one turn-based debate and a three-person free debate. In the first part, two participants, one supporting Messi and the other Ronaldo, took turns sharing their points of view to challenge each other through forum posts, mirroring the polarized debates that are omnipresent online. In the second part, a new participant joined the ongoing debate, and three participants were allowed to post freely without turn-based restriction, reflecting the spontaneous and unstructured nature of debates on social media. After the two-part study, semi-structured interviews were conducted to explore the participants' experiences. The researchers then applied content analysis and thematic analysis, triangulating the data from forum posts, ChatGPT records, and interview transcripts.

We found that participants prompted ChatGPT for aggressive responses, trying to tailor ChatGPT to fit the debate scenario. While ChatGPT provided participants with statistics and examples, it also led to the creation of similar posts. Furthermore, participants' posts contained logical fallacies such as hasty generalizations, straw man arguments, and ad hominem attacks. Participants reduced the use of ChatGPT to foster better human-human communication when a new member joined an ongoing debate midway. This work highlights the importance of examining how polarized forum members collaborated with GenAI to engage in online debates, aiming to inspire broader implications for socially oriented applications of GenAI.
\subsection{Compose Theory}
ComposeOn promotes people without a music background to compose their own music. We design the system to detect melody users put in and give suggestions for continuations based on basic music and harmony progression theory. "Western music written during Baroque, Classical, and Romantic periods (ca.1650-ca.1900) is called tonal music, which has a point of gravitation called tonic." \citep{laitz2008complete} The keys and scales gradually formed based on that and thus formed tonal hierarchy and harmony function theory.

"The music of the tonal era is almost exclusively tertian, which means being constructed of stacked 3rd." \citep{kostka2006materials} Chords are marked with their root notes (where the stack begins) using roman numerals. Functionally, they are basically divided into Tonic chords: I, Dominant chord: V, and predominant/subdominant chords: ii, iv, vi. The harmony often progresses as Tonic–Subdominant–Dominant–Tonic. Sometimes may use iii or Vii, and different 7th chords, but their function is various depending on the texture \citep{aldwell2010harmony}.

There are also uses out of this basic progression, such as sequence, modulation or transportation. We also considered those situations with limited possibilities within our database.

Based on the above theory, we suggest notes within the key and the possible harmony progressions. As to music phrases, we followed basic 2+2/4+4 (refer to measure numbers) to form the music phrase and thus sentence and sections. \citep{schoenberg1967fundamentals} We mainly focus on songwriting, so the intro, verse — chorus — verse — chorus —bridge — chorus — outro structure of song is also being considered. \citep{masterclass2021songwriting}

\subsection{Voice to MIDI Technology and Its Applications}

MIDI (Musical Instrument Digital Interface) is a technical standard that describes a protocol, digital interface, and connectors, allowing various electronic musical instruments, computers, and other related devices to connect and communicate with each other \citep{midi1996complete}. Voice to MIDI technology is the process of converting vocal or other audio signals into this MIDI data format, playing a crucial role in music production and analysis. This technology involves multiple steps, including pitch detection, note segmentation, quantization, and MIDI conversion. Pitch detection typically employs algorithms such as YIN \citep{decheveigne2002yin} or pYIN \citep{mauch2014pyin} to estimate the fundamental frequency of audio. Subsequently, the continuous pitch sequence is segmented into discrete notes, which are then time-aligned to a musical grid. Finally, the detected note information is converted into MIDI events. Voice-to-MIDI technology has found applications in various fields, such as quickly converting hummed melodies into MIDI for song composition, providing instant feedback to students in music education, and generating real-time music based on user voice input in interactive music systems. This technology not only simplifies the music creation process but also provides powerful tools for music analysis and education.

\subsection{Automated Melody Analysis}

Automated melody analysis is a significant branch of music information retrieval, aimed at extracting and analyzing melodic features from musical data. This process typically includes the analysis of notes, chords, and chord progressions. Note analysis involves extracting attributes such as pitch, duration, and velocity. Chord analysis focuses on identifying combinations of simultaneously sounding notes, often using algorithms like Chordino \citep{mauch2010simultaneous}. Chord progression analysis examines patterns in the series of chords, utilizing methods such as hidden Markov models \citep{rohrmeier2012comparing}. In this field, Musicpy, a powerful Python music programming library, provides extensive functional support. It not only has concise syntax for representing various musical elements but also incorporates a complete music theory system supporting advanced musical operations. Musicpy's core data structures include notes, keys, chords, scales, etc., offering various practical functions including chord identification, melody analysis, and automatic composition. Through Musicpy, researchers and music creators can conveniently achieve automated melody analysis, explore musical structures and characteristics. Notably, the ComposeOn project extensively utilizes Musicpy's powerful capabilities, particularly in chord extraction and chord progression analysis. ComposeOn employs Musicpy's algorithms to identify and analyze the progression patterns of these chords, thereby providing an important foundation for music analysis and creation. This application demonstrates the practicality and effectiveness of Musicpy in real-world music analysis projects.

\subsection{Automatic Accompaniment Generation}
Accompaniment generation is referred to as "the audio realization of a chord sequence"
by systems like MySong \cite{simon2008mysong}, which represents a significant advancement in the field of automatic accompaniment generation for vocal melodies. MySong allows users to input vocal melodies, which the system then inputs to a hidden Markov model to recommend chords. However, MySong's capabilities are limited to generating accompaniments, whereas our ComposeOn system empowers users to easily extend and develop their melodic ideas into complete compositions, providing a more comprehensive music creation and learning experience.

%\input{sections/03-Corpus Collection}
\section{Why does emotion matter in information communication?} \label{sec:why} %为什么情感在信息传达中很重要?

Emotions play a crucial role in information communication as they directly affect the reception, processing, and dissemination of information \cite{ferrara2015measuring, berger2012makes}. Emotions not only influence individuals’ psychological experiences \cite{jimenez2012emotional, lopez2021translating, berger2012makes, stieglitz2013emotions}, decisions, and behaviors \cite{schreiner2021impact, de2021sadness}, but also shape public opinion, trigger collective action, and even drive social change through group interaction and social communication \cite{stieglitz2013emotions, ferrara2015measuring, son2022emotion}. Therefore, studying the impact of emotions on information communication holds not only theoretical value but also profound practical significance. This study examines the core role of emotions in information dissemination from five perspectives: communication studies \cite{tyng2017influences, schreiner2021impact, ferrara2015measuring, berger2012makes, vosoughi2018spread}, psychology \cite{mather2011arousal, hanson2014happy, jimenez2012emotional}, sociology \cite{sanford2004negative}, neurobiology \cite{phelps2004human, mcgaugh2015consolidating}, and humanism \cite{dykas2011attachment, stieglitz2013emotions, van2020seeking, dubey2020psychosocial}. It reveals how emotions enhance the effectiveness of information communication through mechanisms such as capturing attention \cite{egidi2012emotional, tyng2017influences}, promoting spread \cite{schreiner2021impact, ferrara2015measuring, berger2012makes, stieglitz2013emotions}, strengthening memory and understanding \cite{pawlowska2011influence, megalakaki2019effects, hanson2014happy, kensinger2007negative, kensinger2020retrieval, van2015good}, and stimulating group behavior \cite{ferrara2015measuring, son2022emotion, berger2012makes, de2021sadness}. At the same time, emotions humanize and energize the information, improving its acceptance and dissemination power \cite{dykas2011attachment, van2020seeking, dubey2020psychosocial}.
% 情感在信息传达中的作用至关重要,因为它直接影响信息的接受、处理和传播效果。情感不仅能影响个体的心理体验、决策和行为,还能通过群体互动与社会传播,塑造公共舆论、引发集体行动,甚至推动社会变革。因此,研究情感对信息传递的影响,不仅具有理论价值,也对实践应用有着深远意义。本研究从传播学、心理学、社会学、生物与神经科学以及人文主义五个角度,梳理了情感在信息传播中的核心作用,揭示了情感如何通过吸引注意、促进传播、强化记忆与理解、激发群体行为等多种方式,提升信息的传播效果。同时,情感也赋予信息人性化和感染力,提高了信息的接受度与传播力。

% In recent years, the impact of emotions on information communication has garnered significant attention in the fields of cognitive science and psychology. Emotions are not only psychological experiences of individuals but also profoundly influence behaviors and decisions, permeating the processes of information reception, processing, and dissemination. More importantly, emotions shape public opinion through group interactions and social communication, guide collective behaviors, and even drive social change \cite{egidi2012emotional}. Therefore, understanding the role of emotions in information communication is of great importance both theoretically and practically.
% %近年来,情感对信息传达的影响在认知科学和心理学领域备受关注。情绪不仅是个体的心理体验,还深刻影响行为和决策,贯穿信息的接受、处理和传播过程。更重要的是,情绪通过群体互动和社会传播塑造公共舆论,引导集体行为,甚至推动社会变革。因此,理解情感对信息传达的作用在理论和实践上都具有重要意义。

% To systematically explore the multidimensional impact of emotions on information communication, this study employed thematic analysis to code and categorize 167 relevant papers, distilling the core views on how emotions affect information communication. The research team read each paper individually, marking the role of emotions in information communication and conducting coding and categorization to identify different argumentative perspectives. Through repeated comparison and discussion, similar codes were integrated into higher-level themes. For instance, research indicates that emotive language enhances audience memory and understanding of news content \cite{moriya2023can}, thus this view was classified under the theme ``the role of emotion in news dissemination," falling into the ``application perspective."
% %为系统探讨情绪对信息传达的多维影响,本研究采用主题分析法,对167篇相关文献进行了编码和分类,以提炼情感对信息传达的核心观点。研究团队逐篇阅读文献,标记情感在信息传播中的作用,并进行编码和分类,以识别不同的论证角度。通过反复比较和讨论,相似的编码被整合为更高层次的主题。例如,有研究表明,情感化语言提高了受众对新闻内容的记忆和理解,因此该观点被归类为“在新闻传播中情感的作用”,并归入“应用角度”主题。

% Ultimately, this study elucidates the importance of emotion in information communication from five perspectives: communication, psychology, sociology, neuroscience, and humanism. First, from the perspective of communication, emotion effectively captures audience attention and drives the dissemination and network diffusion of information. Next, from a psychological perspective, emotion strengthens memory, facilitates understanding, and influences individual attitudes and behaviors, thereby enhancing the depth and effectiveness of information. At the sociological level, emotion not only promotes social interaction but also triggers group behavior, fostering information exchange and resonance among groups. From the perspective of neuroscience, emotional information is prioritized by the amygdala and reinforced through its interaction with the hippocampus, while the ventral striatum and insula enhance the dissemination of information. Finally, from a humanistic perspective, emotion prevents information from becoming indifferent, making it more humanized and impactful, thereby increasing its acceptance and dissemination~power.
% %最终,本研究从传播学、心理学、社会学、生物与神经科学以及人文主义五个角度来阐释情感在信息传达中的重要性。首先,从传播学角度看,情感能够有效吸引受众的注意力,并激发信息的传播与网络扩散。接着,从心理学角度,情感强化记忆、促进理解,并能影响个体的态度与行为,从而增强信息的深度和效果。在社会学层面,情感不仅能促进社会互动,还能激发群体行为,促进群体间的信息交换与共鸣。从生物与神经科学角度来看,情感信息通过杏仁核的优先处理和杏仁核与海马体的联动,强化了情感记忆,且腹侧纹状体与岛叶对信息的传播产生了增强作用。最后,从人文主义的角度,情感避免了信息的冷漠化,使其更具人性化与感染力,进而提升了信息的接受度与传播力。

\begin{figure*}[hbt!]
%\setlength{\abovecaptionskip}{-0.1mm}
\setlength{\intextsep}{10pt plus 2pt minus 2pt}
    \centering
    \includegraphics[width=18cm]{figs/Fig.1-Why.jpg}
    \caption{This figure illustrates the role of emotions in information communication from five perspectives: communication (P1), psychological (P2), neurobiology (P3),  sociology (P4), and humanism (P5).}
   \vspace{-2mm}
\label{fig:why}
\end{figure*}


\subsection{From the Perspective of Communication}%传播学角度
From the perspective of communication studies, the importance of emotion in information communication lies in its ability to effectively capture audience attention and drive rapid information diffusion.
%From the perspective of communication studies, the importance of emotion in information communication lies in its ability to effectively capture audience attention and drive rapid information diffusion. Especially in social media contexts, emotionalized content not only evokes strong emotional reactions but also stimulates the audience’s desire to share, thereby rapidly spreading across networks and expanding its~influence.
% 情感在信息传播中的重要性体现在其能够有效吸引受众注意力并推动信息的快速扩散。尤其在社交媒体环境中,情感化的内容不仅能够引发强烈的情绪反应,还能够激发受众的分享欲望,从而在网络中迅速传播并扩展其影响力。

\textbf{Emotion is a critical switch for capturing attention:} In information communication, particularly in social media contexts, emotions play a key role in drawing users’ attention. Studies have shown that emotionalized information, especially content with higher emotional arousal, is more likely to capture audience attention compared to neutral information \cite{egidi2012emotional, tyng2017influences}. Specifically, highly arousing emotional content such as anger and surprise tends to stimulate audience interest, achieve higher initial exposure, and rapidly capture viewers’ attention \cite{lopez2021translating, vosoughi2018spread}. These characteristics make emotional content a “switch for attention” in information communication, effectively capturing the audience’s focus. For example, Vosoughi et al. \cite{vosoughi2018spread} indicated that news capable of evoking anger or surprise often garners higher attention and spreads rapidly, often becoming “flashpoints” in online information communication. This effect is particularly pronounced on platforms like social media. Given the vast and rapidly changing information on these platforms, emotionalized content triggers intense emotional responses, gaining more exposure in a short time and rapidly spurring discussions and shares \cite{schreiner2021impact}.
% 情感是吸引注意力的关键开关:情感在信息传播中,尤其是在社交媒体环境下,扮演着吸引用户注意力的关键角色。研究表明,情感化的信息,尤其是那些情感唤起度较高的内容,比中性信息更容易引起观众的关注(Egidi & Nusbaum, 2012;Tyng et al., 2017)。特别是愤怒、惊讶等高唤起度的情感内容,更能激发受众的兴趣,获得更高的初始曝光率,并迅速吸引观众的注意力(Rojo López & Naranjo, 2021;Vosoughi et al., 2018)。这些特征使情感内容成为信息传播中的“关注开关”,能够有效抓住受众的眼球。例如,Vosoughi等(2018)指出,能够引发愤怒或惊讶的新闻往往吸引更高的关注度并迅速传播,通常成为网络信息传播的“引爆点”。在社交媒体等平台上,这一效应尤为明显。由于平台上的信息量巨大且瞬息万变,情感化的内容通过激发强烈的情绪反应,使信息在短时间内获得更多曝光,并迅速引发讨论和转发(Schreiner et al., 2021)。

\textbf{Emotion drives information communication and network diffusion:} Emotion-driven information demonstrates strong diffusion capability during dissemination. When information aligns with individuals’ emotional stance or values, audiences are more likely to actively share such content \cite{ferrara2015measuring}. This mechanism allows emotionalized content to overcome the challenge of information overload and achieve faster and broader dissemination on social media platforms, especially in emotionally resonant groups. Studies show that highly emotional news, whether true or false, often spreads more widely than rational and neutral news, highlighting the significant driving force of emotion in information communication \cite{berger2011arousal, stieglitz2013emotions}. Highly emotional content enhances its diffusion potential in social networks by eliciting emotional reactions from users, thereby promoting sharing and dissemination. When information evokes strong emotions such as anger, fear, or empathy, it often compels users to immediately share or comment, accelerating the spread of information. For example, news agencies often enhance dissemination by employing emotional narratives and visual designs, such as depicting victims’ emotions or urgent situations, to generate social concern and action \cite{lopez2021translating}. This increased willingness to disseminate helps emotionalized content achieve viral spread on social networks, creating greater diffusion effects \cite{de2021sadness, son2022emotion}.
% 情感推动信息传播与网络扩散:情感驱动的信息在传播过程中展现出强大的扩散能力。当信息与个体的情感立场或价值观相契合时,受众更倾向于主动分享这些内容(Ferrara & Yang, 2015)。这一机制使情感化的内容能够突破信息过载的困境,在社交媒体平台上实现更快速、更广泛的传播,特别是在情感共鸣强烈的群体中。研究表明,情感极为强烈的新闻(无论真假)往往比理性和中立的新闻传播得更广,显示了情感在信息传播中的巨大推动力(Berger & Milkman, 2012;Stieglitz & Dang-Xuan, 2013)。情感强烈的内容通过激发用户的情感反应,促进信息的分享和扩散,增强其在社交网络中的扩散潜力。当信息引发愤怒、恐惧或共情等强烈情感时,常促使用户立即转发或评论,进而加速信息的传播。例如,新闻机构在报道灾难事件时,常通过情感化的叙事和视觉设计(如展现受害者情感或紧急情况)提升传播效果,引发社会关注和行动(Rojo López & Naranjo, 2021)。这种传播意愿的提升,帮助情感化的信息在社交网络中实现病毒式传播,形成更大的传播效应(de León & Trilling, 2021;Son et al., 2022)。

Emotion is a core factor driving information communication, especially prominent in social media contexts. Intense emotional content (such as anger or surprise) can quickly capture audience attention, stimulate interest, and prompt sharing and discussion, leading to rapid information diffusion.
% 从传播学的角度看,情感是推动信息传播的核心因素,尤其在社交媒体环境中表现得尤为突出。强烈的情感内容(如愤怒或惊讶)能够迅速吸引受众的注意力,并通过激发兴趣促使他们主动分享和参与讨论,从而形成信息的快速扩散。


\subsection{From the Perspective of Psychological}%传播学角度
From the perspective of psychological emotion not only enhances the memorability of information but also facilitates understanding and triggers behavioral responses. Emotionalized information mobilizes cognitive resources, increases the depth of information processing, and helps the audience more easily remember and understand the content. Moreover, emotion plays a significant role in influencing audience attitudes and behaviors by eliciting emotional responses and prompting individuals to make corresponding behavioral changes.
% 从心理学的角度来看,情感不仅能增强信息的记忆效果,还能促进理解并引发行为反应。情感化的信息通过调动认知资源,提升信息加工的深度,帮助受众更容易记住和理解内容。同时,情感在影响受众态度和行为上的作用也尤为突出,能够激发情绪反应并促使个体作出相应的行为改变。

\textbf{Emotion enhances memory:} Emotion can strengthen memory in information communication by influencing individuals’ attention allocation and cognitive resources, thereby deepening and prolonging information processing. Psychological research indicates that emotional information is more likely to be encoded into long-term memory compared to neutral information, especially information eliciting strong emotional responses such as anger or sadness, which significantly increases memory prioritization \cite{mather2011arousal}. Emotion establishes a deep connection with individuals’ emotional experiences, promoting deeper information processing and enhancing its storage stability in long-term memory \cite{jimenez2012emotional}. For instance, in the field of education, the use of emotional narratives to help students master complex knowledge demonstrates the broad application of this characteristic of emotionalized information \cite{tyng2017influences}.
% 情感强化记忆:情感在信息传达中能够强化记忆力,这是因为它通过影响个体的注意力分配和认知资源,增强了信息的加工深度和持久性。心理学研究表明,情感性信息比中性信息更容易被编码进长期记忆,尤其是那些引发强烈情绪反应(如愤怒、悲伤)的信息,能够显著提升记忆优先级(Mather & Sutherland, 2011)。情感通过与个体的情感体验建立深度联系,促进了信息的深度加工,从而提升其在长期记忆中的存储稳定性(Jiménez-Ortega et al., 2012)。例如,在教育领域,通过情感叙事帮助学生掌握复杂知识,情感化信息的这种特性得到了广泛应用(Tyng et al., 2017)。

\textbf{Emotion facilitates comprehension:} Emotion can effectively enhance the comprehension of information during its transmission. Emotionalized information reduces cognitive load and increases contextual relevance, making complex content easier to comprehend. Emotionalized information enhances contextual relevance, helping audiences link the information to personal experiences and deepen understanding \cite{megalakaki2019effects, egidi2012emotional}. In practical applications, emotion focuses cognitive resources, enhancing the grasp of key information. For example, in health communication, patient stories instead of mere statistics can more intuitively convey health risks \cite{lopez2021translating}. Furthermore, in negative emotional states, audiences focus more on details and logic, thereby deepening their understanding of the information \cite{arfe2023effects}. Thus, emotion is a core factor in optimizing information processing and enhancing comprehension depth, widely validated in fields such as educational communication~\cite{megalakaki2019effects} and health promotion \cite{lopez2021translating}.
% 情感促进理解:情感在信息传达中能够有效提升信息的理解效果。情感化信息通过降低认知负担和增强情境关联性,使复杂内容更易被接受。情感化信息通过增加情境关联性,帮助受众将信息与个人经验联系,从而加深理解(Megalakaki et al., 2019,Egidi & Nusbaum, 2012)。在实际应用中,情感通过集中认知资源,强化对关键信息的把握。例如,在健康传播中,通过患者故事而非仅展示统计数据,能更直观地传达健康风险(Rojo López & Naranjo, 2021)。此外,负面情感状态下,受众对信息细节和逻辑的关注度显著增强,从而加深对信息的理解(Arfé et al., 2023)。因此,情感是优化信息处理、增强理解深度的核心因素,在教育传播(Megalakaki et al., 2019)、健康宣传(Rojo López & Naranjo, 2021)等领域得到了广泛验证。

\textbf{Emotion influences attitudes and behaviors:} From a psychological perspective, emotion significantly impacts audience attitudes and behavioral tendencies in information communication. Emotion enhances psychological connections with information by eliciting emotional responses (e.g., trust, anger), thereby increasing the persuasiveness and acceptance of the information \cite{lopez2021translating}. Positive emotions (e.g., hope or inspiration) often elicit a favorable attitude toward information, while negative emotions (e.g., fear or anger) may drive audiences to take specific actions or alter their behavior patterns \cite{berger2012makes, schreiner2021impact}. Thus, emotion deeply influences psychological processes, significantly enhancing the dissemination effectiveness and behavioral impact of~information.
% 情感影响态度与行为:从心理学角度来看,情感在信息传达中显著影响受众的态度和行为倾向。情感通过唤起受众的情绪反应(如信任、愤怒或同理心),加强他们对信息的心理联结,从而提升信息的说服力和接受程度(Rojo López & Naranjo, 2021)。正面情感(如希望或鼓舞)通常会激发受众对信息的积极态度,而负面情感(如恐惧或愤怒)则可能驱使受众采取具体行动或调整行为模式(Berger & Milkman, 2012;Schreiner et al., 2021)。因此,情感通过深刻影响受众的心理过程,显著增强了信息的传播效果与行为影响力。

Emotion enhances attention allocation and optimizes cognitive resources, making high-arousal emotional information (such as anger or sadness) more memorable to audiences, significantly improving memory retention \cite{mather2011arousal}. Moreover, emotionalized information increases contextual relevance, aiding audiences in better understanding complex content \cite{egidi2012emotional}. Additionally, emotional responses such as trust, anger, or empathy enhance the psychological connection with information, increasing its persuasiveness and behavioral conversion rate, thereby making it more impactful in dissemination \cite{berger2012makes}.
% 情感通过强化注意力分配和认知资源优化,使高唤醒的情感信息(如愤怒或悲伤)更容易被受众记住,显著提升信息的记忆效果(Mather & Sutherland, 2011)。此外,情感化信息通过增强情境关联性,帮助受众更好地理解复杂内容(Egidi & Nusbaum, 2012)。同时,情感唤起的信任、愤怒或共情反应,能够增强受众对信息的心理联结,从而提升信息的说服力和行为转化率,这使得信息在传播中更具影响力(Berger & Milkman, 2012)。



\subsection{From the Perspective of Neurobiology} %神经科学角度
From the perspective of neurobiology, emotion enhances the rapid processing and deep memory of emotional information by activating the amygdala and hippocampus, while the coordinated activity of the ventral striatum, insula, and empathy-related brain regions facilitates the understanding, resonance, and sharing of emotional information, thereby driving its communication.
% 情感通过激活杏仁核和海马体增强情感信息的快速处理与深度记忆,同时,腹侧纹状体、岛叶及共情相关脑区的协同作用促进了情感信息的理解、共鸣与分享,从而推动信息的传播。

% From the perspective of neurobiology, the importance of emotion in information communication stems from its direct impact on brain processing mechanisms. Emotional information activates specific brain regions, such as the amygdala and hippocampus, enabling rapid transmission and memory enhancement, while also leveraging reward systems and empathy mechanisms to promote dissemination and social diffusion. These neural mechanisms collectively explain why emotional information can quickly capture attention, deepen understanding, and achieve wide dissemination.
% % 从生物神经科学的角度来看,情感在信息传递中的重要性源于其对大脑处理机制的直接影响。情感信息通过激活特定的脑区,如杏仁核和海马体,快速传递并增强记忆,同时还通过奖励系统和共情机制,促进信息的传播与社会扩散。这些神经机制共同解释了情感信息为何能够迅速吸引注意、深化理解并广泛传播。

\textbf{Amygdala prioritizes emotional information:} Emotional information has a priority-processing characteristic, making it occupy a crucial role in information communication. Studies indicate that when information has emotional characteristics, the amygdala in the brain is quickly activated. As a core region for emotion processing, the amygdala can bypass complex cognitive evaluations to directly process emotional information \cite{ledoux2000emotion}. This fast pathway ensures that threats, rewards, or other emotion-related information can be rapidly attended to and trigger physiological responses. For instance, when hearing an emergency alarm or seeing an angry face, the amygdala reacts quickly, prompting rapid recognition and response to the situation \cite{phelps2004human}. This prioritization mechanism, in collaboration with the prefrontal cortex (PFC) and the thalamus, provides a neural basis for the rapid recognition and transmission of emotional information.
% 杏仁核优先处理情感信息:情感信息具有优先处理的特性,这使其在信息传递中占据重要位置。研究表明,当信息具有情感特征时,大脑中的杏仁核(amygdala)会迅速被激活,杏仁核作为情感加工的核心脑区,可以绕过复杂的认知评估直接处理情感性信息(LeDoux, 2000)。这一快捷通道确保了威胁、奖励或其他情感相关信息能够迅速被注意并引发生理反应。例如,当听到紧急警报或看到愤怒的面孔时,杏仁核快速反应,触发个体对情境的迅速识别与应对(Phelps & LeDoux, 2005)。这种优先机制通过与前额叶皮层(PFC)和丘脑的协作,为情感信息的快速识别和传递提供了神经基础。

\textbf{Amygdala and hippocampus strengthen emotional memory: }Emotion enhances memory encoding and storage by activating pathways between the amygdala and hippocampus \cite{cahill1998mechanisms}. This mechanism makes emotional events (e.g., inspiring speeches or catastrophic news) easier to remember than ordinary information. The amygdala tags the emotional intensity of information, while the hippocampus integrates these tags into long-term memory, ensuring the persistence of emotional information in individual memory \cite{phelps2004human}. Additionally, emotional tags not only influence initial memory but also reactivate related memory pathways during recall. Research shows that emotional events often leave deeper memory traces, making them more attractive in the dissemination process \cite{sakaki2014emotion}. Emotion also enhances the receiver’s understanding of information through the interaction between the sensory cortex and the prefrontal cortex.	The prefrontal cortex regulates emotional signals, assisting individuals in evaluating the importance and social significance of information, thereby further enhancing the impact of emotional information in dissemination.	

% 杏仁核与海马体强化情感记忆:情感通过激活杏仁核与海马体(hippocampus)之间的通路,强化信息的记忆编码与存储(Cahill & McGaugh, 1998)。这一机制使情感性事件(如激动人心的演讲或灾难性新闻)比普通信息更容易被记住。杏仁核负责标记信息的情感强度,海马体则将这些标记整合到长时记忆中,从而确保情感信息在个体记忆中的持久性(Phelps, 2004)。此外,情感标记不仅影响初始记忆,还会在回忆过程中重新激活相关记忆通路。研究发现,情感事件往往具有更深的记忆痕迹,因此在传播过程中更具吸引力(Sakaki et al., 2014)。情感还通过感知皮层与前额叶皮层的协作,加深接收者对信息的理解。前额叶皮层通过调控情感信号,帮助个体判断信息的重要性和社会意义,进一步优化情感信息在传播中的影响力。

\textbf{Ventral striatum and insula enhance information communication:} Emotion significantly enhances information sharing and dissemination by activating the brain’s reward system and social mechanisms. Research shows that the ventral striatum in the brain is activated when processing positive emotional information, directly enhancing individuals’ willingness to share \cite{haber2010reward}. Content with strong emotional tones (e.g., touching stories or angry news) is more likely to trigger sharing behavior, primarily due to the driving role of the reward system \cite{van2012reward}. The medial prefrontal cortex (mPFC) and the temporoparietal junction (TPJ) enhance receivers’ understanding and resonance with emotionalized information by eliciting empathy effects \cite{decety2008emotion}. This empathy effect facilitates information communication within groups, providing neural support for the social diffusion of information. Additionally, the emotional integration role of the brain’s insula adds stronger emotional dimensions to information, making it more attention-grabbing in group dissemination \cite{craig2009you}. These mechanisms collectively demonstrate that emotion-driven sharing behavior significantly enhances the efficiency of information communication and its social impact.
% 腹侧纹状体与岛叶增强信息传播:情感通过激活大脑的奖励系统和社会机制,显著增强信息的共享与传播。研究表明,大脑的腹侧纹状体(ventral striatum)在处理正向情感信息时会被激活,这种激活直接增强了个体的分享意愿(Haber & Knutson, 2010)。具有强烈情感色彩的内容(如感人故事或愤怒新闻)更容易触发分享行为,这种现象主要归因于奖励系统的驱动作用(Van Steenbergen et al., 2012)。内侧前额叶皮层(mPFC)和颞顶联合区(TPJ)通过引发共情效应,增强了接收者对情感化信息的理解与共鸣(Decety & Meyer, 2008)。这种共情效应促进了信息在群体间的传播,为信息的社会化扩散提供了神经支持。此外,大脑岛叶(insula)的情感整合作用能够赋予信息更强烈的情感维度,使其在群体传播中更加引人注目(Craig, 2009)。这些机制共同表明,情感驱动的共享行为大幅提升了信息的传播效率和社会影响力。

\subsection{From the Perspective of Sociology}% 社会学角度

From the Perspective of Sociology, the importance of emotion in information communication is not only reflected in its influence on individual behavior but also in its ability to evoke group emotions and enhance social cohesion, thereby facilitating social interaction and driving collective action.
%从社会学角度,情感在信息传达中的重要性不仅体现在对个体行为的影响上,还通过激发群体情绪和增强社会凝聚力,促进社会互动并推动集体行动。

\textbf{Emotional information promotes social interaction:} From a sociological perspective, emotion plays a critical role in enhancing social interaction in information communication. Emotional information stimulates group emotions and strengthens social cohesion, thereby facilitating communication and connection among individuals. When individuals share strongly emotional content on social media (e.g., joyful celebrations or touching stories), such emotions can quickly spread within groups, enhancing emotional consistency and a sense of identity among members \cite{stieglitz2013emotions, ferrara2015measuring}. This emotional resonance not only fosters interaction between individuals but also strengthens connections within communities and social groups. For instance, in public welfare activities, emotional information can inspire people to engage in discussions and donations, creating tighter social networks \cite{son2022emotion}. Emotional information thus becomes a vital tool for promoting social interaction and enhancing social~capital.
% 情感信息促进社会互动: 从社会学角度来看,情感在信息传达中起到了增强社会互动的关键作用。情感信息通过激发群体情绪和加强社会凝聚力,促进了人们之间的交流和联系。当个人在社交媒体上分享带有强烈情感的内容(如喜悦的庆祝、感人的故事),这种情感能够在群体中迅速传播,增强成员之间的情感一致性和认同感((Stieglitz & Dang-Xuan, 2013);(Ferrara & Yang, 2015))。这种情感共鸣不仅促进了个体之间的互动,还加强了社区和社会群体的连结。例如,在公益活动中,情感信息可以激励人们参与讨论和捐赠,形成更紧密的社会网络(Son et al., 2022)。情感信息因此成为促进社会互动和增强社会资本的重要工具。

\textbf{Emotion triggers group behavior:} Emotion in information communication not only facilitates social interaction but also induces changes in group behavior. High-arousal emotions (such as anger and enthusiasm) can stimulate collective actions within groups, leading to wider information communication. Studies show that highly emotional information is more readily accepted and disseminated by groups, thereby promoting discussions on social issues and driving social change \cite{berger2012makes}. For example, in social movements, emotionalized promotional content can mobilize the public to participate in protests, sign petitions, or support policies, demonstrating the importance of emotion in shaping social behavior \cite{de2021sadness}. Emotional information thus becomes a key factor in driving group behavior and facilitating social change.
% 情感引发群体行为: 情感在信息传达中不仅促进了社会互动,还能够引发群体行为的变化。高唤醒的情感(如愤怒、热情)能激发群体的共同行动,使信息传播更为广泛。研究表明,带有强烈情感的信息更容易被群体接受和传播,从而推动社会议题的讨论和社会变革(Berger & Milkman, 2012)。例如,在社会运动中,情感化的宣传内容可以动员公众参与抗议、签署请愿或支持某项政策,这些集体行动体现了情感在引导社会行为中的重要性(de León & Trilling, 2021)。情感信息因此成为驱动群体行为和促进社会变革的关键因素。

% Emotional information is both a vital link for fostering social interaction and a key driver of group behavior and social change. The social effects of emotional dissemination not only reveal the mechanisms of interaction in the digital society but also provide new perspectives for understanding social action and change.
% % 情感信息既是促进社会互动的重要纽带,也是驱动群体行为和社会变革的关键动力。这种情感传播的社会效应不仅展现了数字社会中的互动机制,还为理解社会行动和变革提供了新的视角。

\subsection{From the Perspective of Humanism} % 人文主义角度
\textbf{Emotion prevents information from becoming indifferent:} The importance of emotion in information communication lies in its ability to effectively prevent indifference, making the content more aligned with human-centered care. If information communication relies solely on rational appeals, it may overlook the audience’s emotional needs, thereby weakening the dissemination effect. As the philosopher Nietzsche \cite{nietzsche2017birth} stated in The Birth of Tragedy, humans understand the world not only through reason but also by establishing deeper meaning through emotion. Thus, emotion is the key factor in truly touching people’s hearts through information. Research shows that information lacking an emotional dimension (e.g., mere statistics or neutral language) may cause people to lose interest in the story behind the information and even appear mechanical and indifferent \cite{dykas2011attachment, stieglitz2013emotions}. For example, in the dissemination of breaking news or social issues, merely providing data and facts may make it difficult for audiences to resonate with the events. Conversely, emotionalized narratives (e.g., focusing on individual stories or group sentiments) can make information more concrete and emotional, avoiding the appearance of excessive indifference \cite{van2020seeking}. The integration of emotion makes information communication more humane. Scherer \cite{scherer2003vocal} pointed out that emotional information embedded in voice and tone can evoke audience resonance, preventing indifference toward the content. In health communication, negative emotional stories often evoke empathy for individual suffering and inspire deeper reflection \cite{dubey2020psychosocial}. This emotional engagement not only makes information communication a transfer of knowledge but also builds emotional bridges between people, allowing information to transcend indifference and truly touch hearts.
% 情感避免信息的冷漠化:情感在信息传播中的重要性在于它能有效避免信息的冷漠化,使传播内容更加贴近人性化关怀。信息传播若完全依赖理性诉求,可能忽视了受众对情感的需求,从而导致传播效果的削弱。正如哲学家尼采在《悲剧的诞生》中所言,人类不仅通过理性理解世界,还通过情感建立对意义的深层认知。因此,情感是信息真正打动人心的关键因素。研究表明,缺乏情感维度的信息(如单纯的统计数据或中性化的语言)可能让人们对信息背后的故事失去兴趣,甚至显得机械和冷漠(Dykas & Cassidy, 2011,Stieglitz & Dang-Xuan, 2013)。例如,在突发新闻或社会问题的传播中,仅仅提供数据和事实可能让受众难以与事件产生共鸣。相反,通过情感化叙述(如聚焦个体故事或群体感受),信息能够更加具象化和情感化,从而避免传播的内容显得过于冷漠(Van der Meer & Jin, 2020)。情感的融入使信息传播更加富有人性。Scherer(2003)指出,声音和语调中蕴含的情感信息能够激发听众的共鸣,避免受众对信息内容产生冷漠态度。在健康传播中,负面情感故事往往能够让受众对个体痛苦产生共情,并引发更深层次的思考(Dubey et al., 2020)。这种情感参与不仅让信息传播成为知识的传递,更架起了人与人之间的情感桥梁,使信息能够穿越冷漠,真正触及人心。

\section{What emotions factors effects Information Communication?} \label{sec:what} %哪些情绪因素影响信息传播?
% 细分什么情感元素会影响Information Communication。然后把效价和唤起度拆分开。How写具体的设计模式。

% 1. 调研了哪些心理学的模型,再写这些模型把情绪划分成了唤起、效价、支配感。离散+连续。
% 2. 对于信息传达普遍是有较大影响的,影响情感的因素具体有哪些(唤起、效价、支配感),这3个是构成影响交流的最主要的因素,具体的定义(效价在哪里定义的,和其他模型相类似。)具体如何影响的。

The three key emotional factors—valence, arousal, and dominance \cite{mehrabian1974approach, osgood1952nature}—are crucial emotional elements that influence information communication. In \cref{sec:emotion_models}, we have already explored the theoretical foundations of these three emotional dimensions. This chapter will provide an in-depth analysis of how these three dimensions specifically affect the omprehension\cite{egidi2012emotional, megalakaki2019effects, mather2011arousal, tyng2017influences}, memory \cite{pawlowska2011influence, madan2019positive,megalakaki2019effects}, and sharing \cite{ferrara2015measuring, stieglitz2013emotions, vosoughi2018spread} of information, further revealing their key role in information dissemination.

% Emotions have a multi-dimensional impact on information communication, influencing comprehension, memory, and sharing.  This not only highlights the complexity of emotions but also emphasizes their central role in the communication process.  This section integrates existing research to explore the specific roles and manifestations of the three dimensions—valence, arousal, and dominance—within information comprehension, memory, and sharing.  It aims to provide a foundation for a deeper understanding of the pivotal role of emotions in information communication.
% % 情绪在信息传达中发挥着多层次的作用,其对理解、记忆与分享的影响不仅体现了情绪的复杂性,也反映了其在信息传播中的关键地位。本章首先综述了情绪的不同维度模型,并在此基础上梳理出效价、唤起度和支配度三大核心维度,作为研究情绪作用的理论框架。随后,结合已有研究,深入探讨这三个维度在信息理解、记忆和分享过程中的具体表现与作用,为全面理解情绪在信息传播中的关键角色奠定基础。

% 这里需要插入一个不同模型的图片
% first reviews various dimensional models of emotions and, based on this, outlines three core dimensions—valence, arousal, and dominance—as a theoretical framework for studying the role of emotions. Subsequently, 


\subsection{The Impact of Emotions on Comprehension}
Emotions play an important regulatory role in information comprehension. Different emotional characteristics influence attention allocation and cognitive resource utilization, shaping the depth and accuracy of an individual’s understanding. Valence regulates attention scope and cognitive resource allocation, affecting the depth of understanding \cite{egidi2012emotional, megalakaki2019effects}. Arousal determines information priority and resource concentration \cite{mather2011arousal, tyng2017influences}. Dominance influences active information processing and integration \cite{mehrabian1996pleasure, gross1998emerging}. These emotional factors underscore the pivotal role of emotions in understanding complex information by regulating processing patterns. They provide critical insights into emotional mechanisms underlying information~communication. 
%情绪在信息理解中发挥着重要调节作用,不同的情绪特征通过影响注意力分配和认知资源利用,塑造了个体的理解深度与准确性。效价通过调节注意力范围和认知资源分配影响理解的深度(Egidi & Nusbaum, 2012; Megalakaki et al., 2019),唤起度决定了信息优先级和资源集中程度(Mather & Sutherland, 2011; Tyng et al., 2017),而支配度则塑造了对信息的主动处理和整合方式(Mehrabian, 1996; Gross, 1998)。这些情绪因素通过调节信息处理模式,揭示了情绪在理解复杂信息中的核心作用,为探索信息传播中的情绪机制提供了重要视角。


\subsubsection{Emotional Valence}
Positive emotions can significantly enhance the understanding of information, but they may also increase cognitive load for individuals in certain situations. Firstly, positive emotions aid individuals in understanding complex information more deeply by enhancing motivation and attention. For instance, Egidi \& Nusbaum \cite{egidi2012emotional} found that individuals in a positive emotional state exhibit greater depth of understanding when processing emotional language information, as positive emotions enhance focus and motivation. Additionally, Megalakaki et al. \cite{megalakaki2019effects} and Zhang \cite{zhang2023effect} noted that positive emotions can expand cognitive resources, enabling individuals to analyze complex texts in greater detail, thereby enhancing comprehension outcomes. Tyng et al. \cite{tyng2017influences} further indicated that this emotional state broadens the range of attention, facilitating the processing of multiple complex pieces of information and thereby improving understanding capabilities. However, positive emotions may affect the accuracy of understanding due to increased cognitive load. Research by Jiménez-Ortega et al. \cite{jimenez2012emotional} showed that, although cognitive processing is deeper in a positive emotional state, individuals exhibit longer reaction times and higher error rates in semantic processing tasks. This may be because positive emotions lead individuals to focus on both core information and secondary details, thereby distracting attention and increasing cognitive~load.
%积极情绪能显著促进信息的理解,同时也可能在某些情况下增加个体的认知负担。首先,积极情绪通过增强动机和注意力,帮助个体更深入地理解复杂信息。例如,Egidi和Nusbaum(2012)发现,积极情绪的个体在处理情绪化语言信息时展现出更高的理解深度,这是因为积极情绪提高了专注度和动机。此外,Megalakaki et al.(2019)以及Zhang(2023)指出,积极情绪能够扩展认知资源,使个体能更细致地分析复杂文本,从而提高理解效果。Tyng等(2017)进一步指出,这种情绪状态扩大了注意力范围,有利于处理多重复杂信息,从而提升理解能力。然而,积极情绪可能因增加认知负担而影响理解的准确性。Jiménez-Ortega等(2012)的研究显示,在积极情绪状态下,尽管认知加工更深入,但在语义处理任务中个体的反应时间更长,错误率也更高。这可能是因为积极情绪导致个体同时关注核心信息和次要细节,从而分散了注意力,增加了认知负荷。

In contrast to positive emotions, negative emotions have a complex and multidimensional impact on information comprehension. Research indicates that negative emotions enhance attention to detail, potentially improving reasoning abilities, but they may also weaken overall comprehension \cite{egidi2012emotional, lang2007cognition}. For example, Lang et al. \cite{lang2007cognition} found that viewers in a negative emotional state paid more attention to negative details in news reports, neglecting the overall content. A study by Arfé et al. \cite{arfe2023effects} also showed that the negative emotion group had significantly longer initial fixation times during text reading compared to the neutral emotion group, indicating that this emotional state triggered deeper information processing. Additionally, negative emotions prompt individuals to engage in deeper reasoning and analysis, which is very beneficial for understanding complex information \cite{megalakaki2019effects}. However, negative emotions may also lead individuals to interpret information from a biased perspective, thereby reducing objectivity and rational judgment abilities, which affects the accurate understanding of information \cite{tyng2017influences}.
%相对于积极情绪,消极情绪对信息理解的影响复杂且多维。研究表明,消极情绪增强了对细节信息的关注,可能提升推理能力,但同时也可能减弱整体理解能力。例如,Lang等(2007)发现,消极情绪状态下的观众更多关注新闻中的负面细节,忽视整体内容。Arfé等(2023)的研究也显示,消极情绪组在文本阅读时的初次注视时间显著长于中性情绪组,表明这种情绪状态引发了更深入的信息处理。此外,消极情绪还促使个体进行更深入的推理和分析,这对理解复杂信息非常有利。然而,消极情绪也可能导致个体以偏见性视角解读信息,从而降低客观性和理性评判能力,影响信息的正确理解。

Finally, the impact of neutral emotions on information comprehension is relatively balanced and not as pronounced as that of positive or negative emotions. In a neutral emotional state, the lack of emotional bias leads to more stable information processing and a more balanced allocation of resources. For example, studies by Earles et al. \cite{earles2016memory} and Megalakaki et al. \cite{megalakaki2019effects} indicate that neutral emotions are conducive to maintaining objectivity and consistency in information processing. This emotional state does not significantly enhance or weaken comprehension abilities but has a particular advantage in tasks requiring high objectivity and accuracy. However, the lack of emotional cues in a neutral emotional state may hinder deep processing of information and long-term memory. Overall, neutral emotions exhibit a neutral role in information comprehension and are suitable for performing tasks that require high objectivity and stability.
%最后,中性情绪对信息理解的影响相对平衡,不如积极或消极情绪显著。在中性情绪下,由于缺乏情感偏向,信息处理更为稳定,资源分配更加均衡。例如,Earles等(2016)和Megalakaki等(2019)的研究表明,中性情绪有利于保持信息处理的客观性与一致性。这种情绪状态不会显著提升或削弱理解能力,但在需要高度客观性和准确性的任务中具有特别的优势。然而,由于缺乏情感线索,中性情绪可能不利于信息的深度加工和长期记忆。总的来说,中性情绪在信息理解中表现中立,适合执行需要高度客观和稳定的任务。

\subsubsection{Emotional Arousal}
The impact of emotional arousal on information comprehension depends on the intensity of arousal. High-arousal emotions, such as fear or excitement, can significantly alter cognitive states, enhancing attention concentration in a short period and prompting prioritized processing of key information \cite{marchewka2016arousal}. This concentrated allocation of cognitive resources aids quick responses in complex situations, but it may also lead to a ``tunnel effect," where individuals are prone to neglect non-core information and details. For example, individuals experiencing intense threatening emotions focus more on the source of the threat and coping strategies, while memories of other details may become vague~\cite{mather2011arousal}.
%情绪唤起度对信息理解的影响取决于唤起强度。高唤起度情绪,如恐惧或兴奋,能显著改变认知状态,短时间内提高注意力集中程度,促使优先处理关键信息。这种认知资源的集中分配有助于在复杂情境中快速反应,但同时也可能导致“隧道效应”,即个体容易忽略非核心信息和细节。例如,强烈威胁性情绪下的个体更专注于威胁来源和应对方式,而其他细节记忆则可能模糊。
Additionally, high-arousal emotional states may lead to increased cognitive load, thereby reducing comprehension accuracy \cite{jimenez2012emotional}. High-arousal emotions are often accompanied by tension or anxiety, which interfere with comprehensive and rational information analysis, increasing the likelihood of cognitive biases. In this state, individuals are more inclined to process information emotionally, a tendency that becomes more pronounced when dealing with negative emotions \cite{pessoa2009emotion}.
%此外,高唤起情绪状态还可能导致认知负荷增加,进而影响理解的精确性。这是因为高唤起情绪通常伴随紧张或焦虑,这些情绪干扰了全面和理性的信息分析,易于形成认知偏差。在这种状态下,个体还可能倾向于情绪化处理信息,尤其在处理负性情绪时,这一倾向更为明显。

In contrast, low-arousal emotions, such as mild pleasure or worry, have a more balanced impact on information comprehension. Under low-arousal emotional states, the allocation of cognitive resources is more balanced, making it easier to integrate multiple sources of information for comprehensive understanding. In this emotional state, individuals are less likely to be disturbed by emotions, allowing for effective integration of complex texts or multi-source information, thereby enhancing the accuracy and comprehensiveness of overall comprehension. Under low to moderate arousal emotions, individuals' cognitive flexibility is enhanced, which not only aids in processing complex information but also improves creative thinking, resulting in better performance across various comprehension tasks \cite{tyng2017influences}. 
%相较之下,低唤起度情绪,如轻微的愉悦或忧虑,对信息理解具有更平衡的影响。在低唤起度情绪状态下,个体的认知资源分配更均衡,更容易整合多个信息来源并进行全面理解。在这种情绪状态下,个体不容易受到情绪干扰,能够有效地整合复杂文本或多源信息,从而提升整体理解的准确性和全面性。低至中等唤起度情绪下,个体的认知灵活性得到增强,这不仅有助于处理复杂信息,还能提高创造性思维,使个体在多种理解任务中表现更佳。

In summary, emotional arousal influences information comprehension in different ways. High-arousal emotions facilitate the rapid understanding and processing of critical information in emergency situations but may limit comprehensive integration of complex information; whereas low-arousal emotions are more suitable for tasks requiring balanced and in-depth analysis, aiding in comprehensive and objective understanding of information.
%总结而言,情绪唤起度通过不同方式影响信息理解。高唤起度情绪有助于在紧急情况下快速理解和处理关键信息,但可能限制对复杂信息的全面性整合;而低唤起度情绪更适合于需要平衡与深入分析的任务,有助于全面且客观的信息理解。

% \subsubsection{The Neural Mechanisms of Emotions in Comprehension}
% Emotional activation during the comprehension process involves multiple key brain areas, which interact to influence the processing and understanding of information. Particularly, the amygdala plays a central role in processing emotional information \cite{ledoux2000emotion}, affecting cognitive responses and memory encoding through its coordination with areas like the prefrontal cortex and hippocampus \cite{phelps2004human}.
% % 情绪在理解过程中的神经激活涉及多个关键脑区,这些脑区通过相互作用共同影响信息的处理和理解。特别是杏仁核在处理情绪信息时起到核心作用,通过与前额叶皮层和海马体等区域的协调作用,影响认知反应和记忆编码。

% The valence of emotions, such as positive emotions (like happiness or satisfaction) and negative emotions (like fear or anger), differently affects information processing. During positive emotions, activity in the ventromedial prefrontal cortex (VMPFC) and the striatum, which are related to the reward and motivation systems \cite{haber2010reward}, helps individuals to make comprehensive assessments and judgments based on emotional responses \cite{bechara2000emotion}. In contrast, negative emotions make individuals focus more on potential threats, enhancing the depth of processing threat-related information but potentially increasing cognitive load and weakening the inhibitory function of the prefrontal cortex (especially the DLPFC), which interferes with rational thinking \cite{pessoa2009emotion, ochsner2005cognitive}.
% % 情绪的效价,如积极情绪(如快乐或满足)和消极情绪(如恐惧或愤怒),对信息处理的影响方式有所不同。积极情绪时,腹内侧前额叶皮层(VMPFC)和纹状体的活跃与奖励和动机系统相关,有助于促使个体基于情绪反应进行全面的评估和判断。相比之下,消极情绪使个体更专注于潜在的威胁,虽然能够增强对威胁信息的处理深度,但也可能增加认知负担,削弱前额叶皮层(尤其是DLPFC)的抑制功能,干扰理性思考。

% Subsequently, the arousal level of emotions also significantly impacts information processing. High arousal emotions, whether positive or negative, trigger physiological responses such as increased heart rate by activating the amygdala, enhancing alertness to information \cite{lang2010emotion}. While this response enhances the ability to process details, excessively high arousal may weaken cognitive control by the DLPFC, impacting rational analysis \cite{van2012reward}. To counteract this disruption, the anterior cingulate cortex (ACC) and the insula are activated to regulate emotional and cognitive conflicts, restoring cognitive balance \cite{etkin2011emotional}. The ACC helps refocus attention, reducing the negative impacts of emotions on cognition, while the insula integrates emotional and cognitive networks, adjusting information processing strategies \cite{craig2009you, singer2009common}. 
% %随后,情绪的唤起度也对信息处理产生重要影响。高唤起情绪,无论正性还是负性,都会通过激活杏仁核引发生理反应,如心跳加速,增强信息的警觉性。这种反应虽增强了信息处理的细节能力,但过高的唤起度可能削弱DLPFC的认知控制,影响理性分析。为应对这种干扰,前扣带回(ACC)和岛叶(Insula)被激活,调节情绪与认知冲突,恢复认知平衡。ACC帮助重新集中注意,减轻情绪对认知的负面影响,而岛叶整合情绪与认知网络,调整信息处理策略。

% Conversely, low-arousal emotions (such as calmness or mild satisfaction) facilitate focus and comprehensive integration of information through the regulation of the prefrontal cortex (PFC) and cingulate cortex. Due to lower arousal, individuals can process information stably and comprehensively in a calm state \cite{damasio1994descartes}, although this may reduce the ability to prioritize key information \cite{pessoa2008relationship}.
% %相反,低唤起度的情绪(如平静或轻度满足)通过前额叶皮层(PFC)和扣带回的调节,帮助个体保持专注,并促进信息的全面整合。由于唤起度较低,个体能在平静状态下进行稳定而全面的信息处理,尽管这可能会降低对关键信息的优先处理能力。

% In summary, the valence and arousal levels of emotions specifically affect the comprehension process by activating brain regions such as the amygdala, prefrontal cortex, cingulate cortex, and insula. Different combinations of emotional valence and arousal evoke varied responses in these neural networks, thereby facilitating or inhibiting the comprehension process. A deeper understanding of these mechanisms helps reveal the multifaceted roles of emotions in cognitive processes and their complex effects on information processing.
% %综上所述,情绪的效价和唤起度通过激活杏仁核、前额叶皮层、扣带回和岛叶等脑区,对理解过程产生具体的影响。不同情绪效价和唤起度的组合在这些神经网络中引发不同反应,从而促进或抑制理解过程。深入理解这些机制,有助于揭示情绪在认知过程中的多方面作用及其对信息处理的复杂影响。

% \subsubsection{Summary}
% Emotions have multiple and complex effects on information comprehension. Positive emotions can enhance attention and motivation, expanding an individual's cognitive resources, thus facilitating the integration of information and comprehension of complex texts; however, in some cases, the expanded scope of attention may lead to an increased cognitive load, thus affecting the accuracy of comprehension. Negative emotions tend to enhance attention to details and prompt deeper reasoning analysis, but the accompanying cognitive load may reduce overall comprehension efficiency. Although both positive and negative emotions can increase cognitive load, their specific impact mechanisms differ. The burden of positive emotions often stems from the dispersion of cognitive resources, leading to insufficient attention to key information, while the burden of negative emotions arises more from the intensity of the emotion and excessive focus on details, hindering information integration. In contrast, neutral emotions maintain a balance in information processing; although lacking intense emotional arousal, they help maintain cognitive objectivity and consistency, suitable for tasks requiring high objectivity and precision.
% %情绪对信息理解的影响具有多重和复杂的作用。积极情绪既可以通过增强注意力和动机,扩展个体的认知资源,从而促进信息整合和复杂文本的理解;但在某些情况下,可能由于扩展的注意力范围导致认知负荷增加,从而影响理解的准确性。消极情绪则倾向于增强对细节的关注,促使更深入的推理分析,但伴随的认知负担可能降低整体理解效率。虽然积极情绪和消极情绪都可能增加认知负担,但其具体影响机制不同。积极情绪的负担往往来源于认知资源的分散,导致对关键信息的注意不足,而消极情绪的负担更多来源于情绪的强烈性和对细节的过度关注,导致信息整合受阻。相比之下,中性情绪在信息处理中保持平衡,虽然缺乏强烈情感激发,但有助于维持认知的客观性和一致性,适用于需要高度客观性和精确性的任务。

% The neural mechanisms of emotion in information comprehension involve the collaboration of brain areas such as the amygdala, prefrontal cortex, cingulate cortex, and insula. Different combinations of emotional states and arousal levels affect the activation of these areas, thereby promoting or inhibiting the comprehension process. Additionally, the consistency between emotional states and the content of information significantly affects the smoothness and efficiency of information processing.
% %情绪对信息理解的神经机制涉及杏仁核、前额叶皮层、扣带回和岛叶等脑区的协作,不同情绪状态和唤起度的组合会影响这些区域的激活,从而对理解过程产生促进或抑制作用。此外,情绪状态与信息内容的一致性也显著影响信息处理的流畅性和效率。

\subsubsection{Emotional Dominance}
Emotional dominance refers to the dominance and dominance that an individual perceives in a given situation, and its level has a profound impact on information comprehension and cognitive processing. High dominance emotions are typically associated with confidence, a dominance, and positive emotional responses \cite{betella2016affective}. This state not only enhances the individual’s dominance over the environment but also promotes the efficient allocation of cognitive resources. Research indicates that individuals experiencing high dominance emotions are more likely to actively process situational information, prioritize key information, and adjust strategies based on needs, resulting in better performance in complex decision-making~\cite{mehrabian1996pleasure}.
% 情绪支配度是个体在情境中感知到的控制感和主导地位,其高低对信息理解和认知加工具有深远影响。高支配度情绪通常伴随自信心、控制感和积极的情绪反应(Betella & Verschure, 2016)。这种状态不仅提升了个体对环境的掌控感,还促进了认知资源的高效分配。研究表明,高支配度情绪个体更倾向于主动处理情境信息,优先关注关键信息,并根据需求调整策略,从而在复杂决策中表现更为出色(Mehrabian, 1996)。

In contrast, low dominance emotions are associated with negative emotions such as anxiety and helplessness, which suppress the effective utilization of cognitive resources \cite{kurth2010link}. Individuals with low dominance often struggle to focus their attention during information processing due to emotional disturbances, resulting in passive responses or even misinterpretation of information. Furthermore, research indicates that low dominance emotions may impair executive function, making it difficult for individuals to complete multitasking or integrate complex information \cite{gross1998emerging}.
% 相较之下,低支配度情绪与焦虑、无助等负面情绪相关,这会抑制认知资源的有效利用(Kurth et al., 2010)。低支配度个体在信息加工过程中常因情绪困扰难以集中注意力,表现为被动反应甚至信息误读。此外,研究指出,低支配度情绪可能削弱执行功能,导致个体难以完成多任务处理或复杂信息的整合(Gross, 1998)。

The role of emotional dominance in comprehension is also reflected in its impact on emotion regulation. Individuals experiencing high dominance emotions enhance information processing efficiency by optimizing emotional responses (such as reducing negative emotional interference), while low dominance emotions may exacerbate emotional fluctuations, further hindering the normal functioning of cognitive abilities \cite{gross1998emerging, kurth2010link}. This suggests that emotional dominance not only influences an individual’s emotional experience but also determines the depth and breadth of information comprehension, which in turn affects overall decision-making and behavior \cite{mehrabian1996pleasure, johnson2012dominance}.
% 情绪支配度在理解中的作用还体现在其对情绪调节的影响。高支配度情绪个体通过优化情绪反应(如减少负面情绪干扰)来提升信息加工的效率,而低支配度情绪则可能加剧情绪波动,进一步阻碍认知功能的正常发挥(Gross, 1998; Kurth et al., 2010)。这表明,情绪支配度不仅影响个体的情绪体验,还决定了其信息理解的深度和广度,进而影响整体决策与行为(Mehrabian, 1996; Johnson et al., 2012)。

\subsubsection{Summary}

The influence of emotions on information comprehension is multidimensional, with valence, arousal, and dominance collectively shaping individuals’ processing and integration of information. Positive emotions, by expanding cognitive resources and enhancing motivation, facilitate information integration and the understanding of complex texts but may affect accuracy due to attention dispersion \cite{egidi2012emotional}. Negative emotions enhance attention to details and improve reasoning ability but may limit overall efficiency due to increased cognitive load \cite{lang2007cognition}. Neutral emotions stabilize the allocation of cognitive resources, maintaining objectivity and consistency in comprehension, making them suitable for tasks requiring high precision \cite{earles2016memory}. Arousal has a particularly significant impact on comprehension: high-arousal emotions prioritize core information but may overlook minor details, while low-arousal emotions support comprehensive information integration, making understanding deeper and more flexible \cite{mather2011arousal}. Dominance further affects individuals’ dominance over information: high-dominance emotions enhance proactivity and processing efficiency, while low-dominance emotions may lead to distraction and comprehension difficulties \cite{mehrabian1996pleasure}. The interaction of emotional variables profoundly impacts information comprehension performance, revealing the complex relationship between emotions and cognition.
% 情绪对信息理解的影响是多维度的,其效价、唤起度和支配度共同塑造了个体对信息的加工与整合。积极情绪通过扩展认知资源和增强动机,有助于促进信息的整合与复杂文本的理解,但可能因注意力分散而影响准确性(Egidi & Nusbaum, 2012)。消极情绪增强了对细节的关注,提升推理能力,但可能因认知负担增加而限制整体效率(Lang et al., 2007)。中性情绪则通过稳定认知资源分配,保持理解的客观性和一致性,适用于高精确性任务(Earles et al., 2016)。唤起度对理解的作用尤为显著,高唤起情绪优先处理核心信息,但可能忽略次要细节;低唤起情绪支持全面信息整合,使理解更具深度与灵活性(Mather & Sutherland, 2011)。支配度进一步影响个体对信息的掌控感,高支配度情绪增强主动性与处理效率,而低支配度情绪可能导致注意力分散与理解障碍(Mehrabian, 1996)。情绪变量的交互作用深刻影响了信息理解的表现,揭示了情绪与认知之间的复杂关系。



\subsection{The Impact of Emotions on Memory}
Emotions play a significant role in the memory process, with valence, arousal, and dominance being key factors influencing memory depth and quality. Different emotional characteristics determine the selectivity, persistence, and accuracy of memory: valence determines the emotional orientation of memory content, arousal affects the intensity and clarity of memory, and dominance shapes individuals’ prioritization and integration of information in memory. These emotional factors influence the generation and retrieval of memory, directly affecting the organization and communication of information, thereby laying the foundation for a deeper understanding of how emotions influence information communication.
% 情绪在记忆过程中发挥重要作用,其效价、唤起度和支配度是影响记忆深度和质量的关键因素。不同情绪特征决定了记忆的选择性、持久性和准确性:效价决定了记忆内容的情感取向,唤起度影响记忆的强度和清晰度,而支配度则塑造了个体对信息的优先记忆和整合方式。这些情绪因素通过调节记忆的生成和提取,直接影响信息的组织和传达,为深入理解情绪如何影响信息传播奠定了基础。

\subsubsection{Emotional Valence}
Positive emotions play a significant facilitative role in the formation and retention of memory, particularly in influencing emotionally relevant information. Research has found that even in the case of incidental memory, positive emotions can deepen the retention of emotional vocabulary and related concepts, making individuals more likely to activate and recall these associated contents \cite{pawlowska2011influence, megalakaki2019effects}. Furthermore, individuals in a positive emotional state are more likely to remember information related to positive emotions, indicating that positive emotions significantly enhance the memory of associated information. Additionally, individuals experiencing pleasant emotions perform better in free recall and associative memory tests than those in a neutral emotional state \cite{lee1999effects, hanson2014happy}, allowing for more accurate recall of brand names and associated information \cite{madan2019positive}. This suggests that positive emotions not only aid in remembering specific information but also expand and integrate overall memory capabilities. Positive emotions also play a crucial role in the retention of long-term memory. 
%积极情绪在记忆的形成和保持过程中扮演着显著的促进角色,尤其是对情绪相关信息的影响更为突出。研究发现,即使在无意记忆的情况下,积极情绪也能加深情绪词汇和相关概念的记忆深度,个体更易于激活和回忆这些关联内容。同时,积极情绪状态下的人们更容易记住与积极情绪相关的信息,说明积极情绪对关联信息的记忆具有显著的促进作用。此外,愉快情绪的个体在自由回忆和联想记忆测试中的表现优于处于中性情绪状态的个体,能够更准确地回忆品牌名称及其相关信息,表明积极情绪不仅有助于记住特定信息,还能扩展和整合整体记忆能力。积极情绪在长期记忆保持方面同样发挥重要作用。

Numerous experimental studies have indicated that positive emotions positively influence both the formation and long-term retention of memory \cite{tyng2017influences}; for example, individuals in a pleasant emotional state retain memory of photographs for a longer duration, regardless of whether these photographs contain emotional content \cite{hanson2014happy}. Images associated with positive emotions are more easily recognized and remembered than neutral or negative images; thus, positive emotions facilitate the consolidation and storage of information, promoting the retention of memory over longer periods \cite{chainay2012emotional}.
%多项实验研究指出,积极情绪对记忆的形成和长期保持都有积极影响,例如在愉快情绪下的个体对照片的记忆保持时间更长,无论这些照片是否包含情绪内容。积极情绪图片比中性或消极情绪图片更容易被识别和记住,因此,积极情绪有助于巩固和存储信息,促进记忆在更长时间内的保持。

In contrast, the effects of negative emotions on memory are more complex, including enhanced memory accuracy, strengthened consolidation of long-term memory, as well as an increased likelihood of false memories. Negative emotions enhance individuals' attention to key details, which not only improves memory accuracy in free recall and cued recall tasks \cite{kensinger2007negative, kensinger2020retrieval, xie2017negative}, but also enables more accurate memory recall after longer periods \cite{van2015good}. This phenomenon may occur because negative emotions encourage individuals to engage in more in-depth processing of event details, thereby enhancing the durability of memories. However, negative emotions can also have side effects, such as an increase in false memories \cite{marchewka2016arousal}. This is because, while focusing on core details, individuals may overlook other information, necessitating the ``filling in" of these gaps during~recall \cite{brainerd2008does}.
% 与此相反,消极情绪对记忆的影响则更为复杂,其效果包括提升记忆的准确性、加强长期记忆的巩固,同时也增加了虚假记忆的出现概率。由于消极情绪增强了个体对关键细节的关注,这不仅提高了自由回忆和提示回忆任务中的记忆精确度, 而且还能在较长时间后提供更准确的记忆回忆。这种现象可能是因为消极情绪促使个体对事件细节进行更深入的处理,从而增强了记忆的持久性。然而,消极情绪也可能带来副作用,如虚假记忆的增加。这是因为在集中注意核心细节的同时,个体可能会忽略其他信息,导致在回忆时不得不“填补”这些遗漏的部分。

The impact of neutral emotions on memory is relatively weak, generally falling between that of positive and negative emotions. Neutral emotions lack significant emotional arousal, which leads to lower prioritization in encoding and consolidating memories of neutral emotional content \cite{earles2016memory, kensinger2020retrieval, megalakaki2019effects, lindstrom2011emotion}. Nevertheless, neutral emotions can enhance stability and consistency in performance for certain tasks. For example, maintaining a neutral emotional state during tasks requiring high precision can reduce emotional interference, thereby improving stability and consistency in task performance \cite{kensinger2006processing, storbeck2005sadness}. Additionally, individuals in a neutral emotional state are better able to sustain attention, facilitating more efficient processing and retention of information.
%中性情绪对记忆的影响相对较弱,通常介于积极和消极情绪之间。中性情绪缺乏显著的情感激发,因此个体对中性情绪材料的记忆编码和巩固优先级较低。尽管如此,中性情绪在某些任务中有助于提高执行的稳定性和一致性。例如,在执行高度依赖精确性的任务时,保持中性情绪有助于减少情绪干扰,从而提高任务的稳定性和一致性。此外,中性情绪状态下,个体的注意力更容易维持,从而确保信息能够得到更高效的加工和记忆。

\subsubsection{Emotional Arousal}
The impact of emotional arousal on memory is a complex and significant process. Research indicates that, compared to low-arousal or neutral events, high-arousal emotional events are typically easier to remember and play an important role in the processes of memory formation, consolidation, and retrieval. 
%情绪唤起度对记忆的影响是一个复杂且显著的过程。研究表明,相比于低唤起或中性事件,高唤起的情绪事件通常更容易被记住,并在记忆的形成、巩固和提取过程中发挥着重要作用。

Firstly, during the encoding process of memory, high-arousal emotional events can significantly enhance the level of detail in memory. This is because high-arousal emotions allocate more cognitive resources to relevant information, improving the accuracy and durability of memory \cite{mather2011arousal}. For instance, after experiencing high-arousal emotional events such as trauma or extreme excitement, individuals often vividly remember the details of the event. 
%首先,在记忆的编码过程中,高唤起情绪事件能显著增强记忆的详细程度。这是因为高唤起情绪使个体对相关信息分配更多的认知资源,提高了记忆的准确性和持久性。例如,在经历创伤或极度兴奋等高唤起情绪事件后,个体往往能清晰记住事件的细节。

During the consolidation phase of memory, high-arousal emotional events promote long-term storage of memories through the regulation of hormonal levels in the body. The release of these hormones strengthens memory traces, making them easier to recall accurately after extended periods \cite{mcgaugh2015consolidating}. For example, individuals who have experienced high-arousal emotional events such as significant accidents or personal achievements can vividly reconstruct specific scenarios from those events, even after a long time has passed. 
%在记忆的巩固阶段,高唤起的情绪事件通过体内激素水平的调节,促进记忆的长期存储。这些激素的释放强化了记忆痕迹,使其更易在长时间后被准确回忆。例如,个体在经历重大事故或个人成就等高唤起情绪事件后,即使过了很长时间,仍能生动地再现事件中的具体情景。

During the process of memory retrieval, the influence of high-arousal emotions is equally significant. High-arousal emotions often trigger vivid and specific recollections, prompting individuals to focus on core information and key details during retrieval. However, while high-arousal emotions can aid in recalling details, they may also increase recall biases and errors, especially when dealing with negative emotional events. Research shows that high-arousal negative emotions may lead to selective biases, where individuals focus on specific emotional content while neglecting other minor details, potentially resulting in false memories \cite{brainerd2008does}. 
% 在记忆提取的过程中,高唤起情绪的影响同样显著。这些事件由于其高唤起度,通常引发更生动、具体的回忆,使得个体在提取时更倾向于集中注意力于核心信息和重要细节。然而,尽管高唤起情绪有助于细节回忆,它也可能增加回忆的偏差和误差,尤其是在处理负面情绪事件时。研究表明,高唤起的负面情绪可能导致选择性偏差,即个体在专注于特定情绪内容时忽略其他次要细节,可能产生虚假记忆。


In summary, emotional arousal significantly influences individuals' memory performance for emotional events by enhancing the processes of encoding, consolidation, and retrieval. This enhancement effect makes high-arousal emotional events more memorable and leads to longer retention compared to low-arousal or neutral emotional events. However, high arousal can also increase the selectivity and bias of memory, thereby affecting its accuracy.
%总之,情绪唤起度通过增强记忆的编码、巩固和提取过程,显著影响个体对情绪性事件的记忆表现。这种增强效应使得高唤起情绪事件相比低唤起或中性情绪事件更易被记住且记忆维持时间更长。然而,高唤起度也可能增加记忆的选择性和偏差性,从而影响记忆的精确性。

% \subsubsection{The Neural Mechanisms of Emotions in Memory}
% The formation and consolidation of emotions depend on the coordinated activity of multiple brain regions. This section will explore the neural mechanisms by which emotions influence memory, particularly how emotions regulate the formation and consolidation of memories through specific neural pathways. This helps to understand why emotional events are more easily remembered than neutral events. 
% %情绪的形成与巩固依赖于多个大脑区域的协同作用。本节将探讨情绪对记忆的神经机制,特别是情绪如何通过特定的神经通路调节记忆的形成和巩固。这有助于理解为什么情绪性事件比中性事件更容易被记住。

% Emotional events typically activate the amygdala and hippocampus significantly, both of which are critical regions for memory formation. Activation of the amygdala enhances memory for emotional events, regardless of whether they involve high-arousal positive or negative emotions \cite{jaaskelainen2020neural, hamann2002ecstasy}. Research has found that the consistency between emotional states and the emotional content of information significantly affects memory encoding and consolidation: processing is smoother when consistent and more difficult when inconsistent, reflected by an increase in the N400 peak, highlighting the amygdala's crucial role in processing emotional information \cite{jimenez2012emotional, egidi2012emotional}. 
% %情绪性事件通常会显著激活大脑中的杏仁核和海马体,这两个区域是记忆形成的关键环节。无论是高唤醒的积极情绪还是消极情绪,杏仁核的激活都增强了情绪事件的记忆。研究发现,情绪状态与信息情感的一致性显著影响记忆的编码和巩固:一致时处理更顺畅,不一致时则增加处理难度,表现为N400峰值增大,突出显示了杏仁核在情绪信息处理中的关键作用。

% Firstly, the emotional valence has a significant impact on memory. Positive emotions typically activate the hippocampus via the dopamine system, enhancing the deep processing of information and thereby improving the efficiency of memory storage and retrieval \cite{alexander2021neuroscience, hamann2002ecstasy}. This emotional state also activates the left amygdala and brain areas associated with reward and emotion, such as the ventral striatum and ventromedial prefrontal cortex, facilitating memory encoding and retrieval \cite{hamann2002ecstasy}. In contrast, negative emotions enhance memory for details by activating the amygdala and visual processing areas, increasing the durability and vividness of memories \cite{kensinger2020retrieval}. For example, studies have found that negative emotions can significantly enhance memory performance in Alzheimer's patients and the~elderly \cite{kazui2000impact}. 
% %首先,情绪效价(即积极或消极情绪)对记忆产生显著影响。积极情绪通常通过多巴胺系统激活海马体,增强信息的深度处理,从而提高记忆的存储和提取效率。这种情绪还激活了左侧杏仁核及与奖励和情绪相关的脑区,如腹侧纹状体和腹内侧前额皮质,从而促进记忆的编码和提取。相反,消极情绪通过激活杏仁核和视觉处理区域,强化对细节的记忆能力,增加记忆的持久性和鲜明度。例如,研究发现,消极情绪可以显著增强阿尔茨海默病患者和老年人的记忆效果。

% Subsequently, emotional arousal levels also play a key role in the memory process. High-arousal emotions, whether positive or negative, promote the consolidation of long-term memories by activating the amygdala, thereby enhancing neural connectivity between the hippocampus and the prefrontal cortex \cite{mcgaugh2015consolidating}. In this state, detailed memories are strengthened \cite{marchewka2016arousal, mather2009disentangling}, particularly for high-priority information, while low-priority information may be suppressed \cite{sakaki2014emotion}. For example, in high-pressure environments such as exams, anxious emotions can help individuals more effectively retain key points while selectively disregarding less essential information. 
% %随后,情绪唤醒水平也在记忆过程中扮演关键角色。高唤醒情绪(无论是积极还是消极)通过激活杏仁核,增强海马体与前额叶皮层之间的神经连接,从而促进长期记忆的巩固。这种状态下,记忆细节得到加强,特别是对高优先级信息的记忆,而低优先级信息可能会被抑制。例如,在高压环境下,如考试,紧张的情绪可能使人更加记住考试的关键知识点,同时忽略掉次要内容。

\subsubsection{Emotional Dominance}
The role of emotional dominance in memory primarily manifests in memory selectivity and memory reconstruction. Research shows that in high dominance emotional states, individuals tend to prioritize remembering key information related to their own emotions. This memory selectivity enhances the depth of storage for important information, enabling individuals to quickly retrieve key content when needed \cite{ledoux2000emotion}. For example, in threatening situations, high dominance fear emotions significantly enhance memory for threat-related information through the activation of the amygdala, ensuring that individuals can respond quickly in similar situations \cite{anderson2001lesions}.
% 情绪支配度对记忆的作用主要体现在记忆选择性和记忆重建两个方面。研究表明,高支配度情绪状态下,个体倾向于优先记忆与自身情绪相关的关键信息。这种记忆选择性强化了重要信息的存储深度,使个体在需要时能够迅速提取关键内容(LeDoux (2000))。例如,在威胁情境中,高支配度的恐惧情绪通过杏仁核的激活显著增强了对威胁信息的记忆,确保个体在类似情境下能够快速反应(Anderson & Phelps, 2001)。

However, the effects of high dominance emotions are not limited to negative emotions. High dominance in positive emotions, such as excitement and joy, also enhances memory comprehensiveness. In this state, individuals not only remember core information but also effectively integrate relevant background information, forming a more comprehensive memory structure \cite{fredrickson2001role}. In contrast, low dominance emotional states generally limit an individual’s memory capacity. In this state, individuals are more likely to neglect non-emotional information, leading to fragmented or distorted memories \cite{schacter2007cognitive}.
% 然而,高支配度情绪的作用并非仅限于负性情绪。正性情绪的高支配度,例如兴奋和愉悦,同样有助于提升记忆的全面性。这种状态下,个体不仅记住了核心信息,还能有效整合相关背景信息,形成更加全面的记忆结构(Fredrickson, 2001)。与此形成对比,低支配度情绪状态通常限制了个体的记忆能力。这种状态下,个体更容易忽略非情绪性信息,导致记忆片段化或失真(Schacter & Addis, 2007)。

Furthermore, emotional dominance has a profound impact on memory reconstruction. Individuals with high dominance emotions tend to focus on positive or useful content during memory recall, thereby enhancing the functionality of memory \cite{fredrickson2001role, schacter2007cognitive}. In contrast, individuals with low dominance emotions may be biased towards retrieving negative memories due to emotional interference, which may further impair their coping and emotion regulation abilities \cite{anderson2001lesions, ledoux2000emotion}. These differences suggest that emotional dominance not only influences memory quality through memory selectivity but also shapes an individual’s cognitive patterns through the memory reconstruction process \cite{bower1981mood, schacter2007cognitive}.
% 此外,情绪支配度对记忆重建具有深远影响。高支配度情绪个体在记忆回忆时倾向于聚焦于积极或有用的内容,从而增强记忆的功能性(Fredrickson, 2001; Schacter & Addis, 2007)。而低支配度情绪个体则可能因情绪干扰而偏向于消极记忆的提取,这可能进一步削弱其应对能力和情绪调节能力(Anderson & Phelps, 2001; LeDoux, 2000)。这些差异表明,情绪支配度不仅通过记忆选择性影响记忆质量,还通过记忆重建过程塑造个体的认知模式(Bower, 1981; Schacter & Addis, 2007)。


\subsubsection{Summary}

The impact of emotions on memory is a complex multidimensional process, with valence, arousal, and dominance collectively shaping the formation and retrieval of memory. Positive emotions typically enhance the encoding and consolidation of associative information by expanding cognitive resources, improving the retention of long-term memory, but may interfere with focus on core information due to attention dispersion \cite{lee1999effects}. Negative emotions increase attention to detail, improving memory accuracy and durability, but neglect of non-core information may lead to the creation of false memories \cite{kensinger2007negative, brainerd2008does}. Neutral emotions provide balance and stability to the memory process, particularly for tasks requiring high objectivity and precision \cite{kensinger2006processing}. Arousal also plays a crucial role in memory: high-arousal emotions enhance the depth and durability of memory by focusing cognitive resources but may increase selective bias due to neglect of secondary information \cite{mather2011arousal}. Low-arousal emotions, on the other hand, support the comprehensive integration of information, making memory more holistic and flexible. Meanwhile, high-dominance emotions enhance the memory and recall of critical information, while low-dominance emotions may result in fragmented and distorted memories \cite{fredrickson2001role, schacter2007cognitive}. Overall, the profound influence of emotions on memory reveals the complex interactions between emotions and cognitive functions.
% 情绪对记忆的影响是一个复杂的多维过程,其效价、唤起度和支配度共同塑造了记忆的生成与再现。积极情绪通常通过扩展认知资源促进关联信息的编码与巩固,同时提升长期记忆的保持能力,但可能因注意力扩散干扰对核心信息的聚焦(Lee & Sternthal, 2014)。消极情绪增强对细节的关注,提高记忆的准确性和持久性,但对非核心信息的忽视可能导致虚假记忆的产生(Kensinger et al., 2007; Brainerd et al., 2008)。中性情绪则为记忆过程提供平衡和稳定性,尤其适用于需要高客观性和精确性的任务(Kensinger & Schacter, 2006)。唤起度在记忆中同样起到重要作用,高唤起度情绪通过集中认知资源,显著增强了记忆的深度和持久性,但可能因忽视次要信息而增加选择性偏差(Mather & Sutherland, 2011)。低唤起度情绪则支持信息的全面整合,使记忆更具整体性和灵活性。与此同时,高支配度情绪强化了对关键信息的记忆和回忆能力,而低支配度情绪则可能导致信息的片段化和失真(Fredrickson, 2001; Schacter & Addis, 2007)。整体而言,情绪对记忆的深远影响揭示了情绪与认知功能之间的复杂互动。

\subsection{The Impact of Emotions on Information Sharing}
Emotions play a core driving role in information sharing, from motivating sharing behaviors to shaping communication patterns, with different emotional characteristics showing significant differences in the effects of information diffusion. Positive emotions typically drive rapid information communication by enhancing appeal and triggering interactive behaviors, whereas negative emotions play a critical role in crises or significant events through their emotional resonance effects. Furthermore, the arousal and dominance of emotions further influence the depth and breadth of sharing, making emotions a critical variable in the process of information communication. This interaction between emotions and information communication reveals the complex psychological mechanisms underlying social behaviors.
% 情绪在信息分享中起着核心驱动作用,从激发分享动机到塑造传播模式,不同情绪特征对信息的扩散效果展现出显著差异。积极情绪通常通过吸引力提升和互动行为激发,推动信息的快速传播,而消极情绪则以其情感共鸣效应在危机或重大事件中扮演关键角色。此外,情绪的唤起度和支配度进一步影响分享的深度与广度,使得情绪成为信息传播过程中不可忽视的关键变量。这种情绪与信息传播的交互作用揭示了社交行为背后的复杂心理机制。



\subsubsection{Emotional Valence}
Positive emotions play a crucial role in promoting information sharing, but the complexity of their impact should not be overlooked. Research shows that individuals are more inclined to share content on social media under the influence of positive emotions. Ferrara \& Yang \cite{ferrara2015measuring} noted that positive emotions increase the appeal of social media content, thereby facilitating its widespread communication. Additionally, studies by Stieglitz \& Dang-Xuan \cite{stieglitz2013emotions} and Schreiner et al. \cite{schreiner2021impact} have found that content with positive emotions is more likely to be shared and commented on, indicating that positive emotions can significantly enhance the communication of information.
%积极情绪在信息分享中具有重要的推动作用,但其影响的复杂性不容忽视。研究显示,在积极情绪的驱动下,个体更倾向于在社交媒体上分享内容。Ferrara 和 Yang(2015)指出,积极情绪提升了社交媒体内容的吸引力,从而促进了信息的广泛传播。此外,Stieglitz 和 Dang-Xuan(2013)及Schreiner等(2021)的研究均发现,带有积极情绪的内容更易于被转发和评论,表明积极情绪能显著增强信息的传播效果。

However, the impact of positive emotions is not always beneficial. Dabbous \& Aoun Barakat \cite{dabbous2023influence} observed that while positive emotions enhance the appeal of information, they may also lead users to overlook its authenticity, thereby unintentionally contributing to the communication of inaccurate or false information. These findings suggest that although positive emotions can support the communication of information, their effects must be carefully managed to mitigate the spread of misleading~content.
%然而,积极情绪的影响并非总是积极的。Dabbous 和 Aoun Barakat(2023)的研究表明,积极情绪虽增加了信息的吸引力,但这也可能让用户忽视对信息真实性的审查,从而无意中帮助传播了不准确或虚假的信息。这说明,虽然积极情绪可以促进信息的传播,但在实际应用中需谨慎处理,以防止误导性信息的广泛传播。

In contrast, negative emotions also play a significant role in information communication. Studies show that negative emotions such as anger, sadness, or anxiety can stimulate users' willingness to share and spread information. Ferrara \& Yang \cite{ferrara2015measuring} noted that negative emotional content on social media is prone to emotional contagion, prompting users to seek emotional resonance and social support through sharing. Moreover, in crises and political events, the role of negative emotions is particularly prominent; they can encourage the public to focus more deeply on and understand relevant information in crisis situations, thereby accelerating the communication of information \cite{sanford2004negative}. Research by De León \& Trilling  \cite{de2021sadness} and Rojo López \& Naranjo \cite{lopez2021translating} has shown that negative emotions significantly increase the number of times news is shared and public attention. 
%相比之下,消极情绪同样在信息传播中扮演了显著的角色。研究表明,消极情绪如愤怒、悲伤或焦虑可激发用户分享和传播信息的意愿。Ferrara 和 Yang(2015)指出,社交媒体上的消极情绪内容易引起情绪传染,促使用户通过分享寻求情感共鸣和社会支持。此外,在危机和政治事件中,消极情绪的作用尤为显著,消极情绪能够促使公众在危机情境下更深入地关注和理解相关信息,从而加速了信息的传播。de León 和 Trilling(2021)及Rojo López 和 Naranjo(2021)的研究显示,消极情绪显著增加了新闻的分享次数和公众的关注度。

Neutral emotions, while having a relatively weaker influence on information sharing, still contribute to the process of information communication. Due to the absence of a strong emotional drive, content associated with neutral emotions typically exhibits lower attractiveness and spreadability on social media. However, research by Son et al. \cite{son2022emotion} highlights the unique value of neutral emotions, as individuals in a neutral emotional state are more likely to adopt a rational and objective perspective. This tendency supports the accurate conveyance and reception of information. Consequently, although neutral emotions are less effective than other emotions in amplifying the spread of information, they play an indispensable role in maintaining objectivity and accuracy in communication. 
%尽管中性情绪对信息分享的影响相对较弱,但其在信息传播中也起到一定作用。由于中性情绪缺乏强烈的情感驱动,其相关内容在社交媒体上的吸引力和传播性通常较低。然而,Son等(2022)的研究表明,中性情绪的内容具有其独特价值,因为在中性情绪状态下,人们往往表现出更中立和理性的态度,这有助于信息的客观传达和准确接收。因此,虽然中性情绪在提升信息传播性方面不如其他情绪有效,但其在保证信息传达的客观性和准确性方面具有不可替代的作用。

By integrating and examining the impact of positive, negative, and neutral emotions on information communication, we can better understand and utilize the complex role of emotions in information sharing, thereby effectively guiding the design and communication strategies of social media content.
%通过整合和审视积极、消极及中性情绪对信息传播的影响,我们可以更好地理解和利用情绪在信息分享中的复杂作用,从而有效地指导社交媒体内容的设计和传播策略。

\subsubsection{Emotional Arousal}
Emotional arousal plays an important role in information sharing on social media. Studies show that both positive and negative high-arousal emotions significantly increase the spread of information \cite{berger2011arousal}. High-arousal emotions such as awe, excitement, anger, or anxiety stimulate a sense of urgency and motivation in users, prompting them to share content more frequently on social media \cite{berger2012makes}. In crisis events or major news, high-arousal negative emotions such as anger or anxiety often trigger strong reactions in users, driving them to quickly share related information on social media \cite{lopez2021translating, sanford2004negative}. This sharing behavior is often motivated by the need to seek social support or express emotional resonance. Conversely, high-arousal positive emotions such as excitement and awe not only enhance users' sharing behavior but also expand the range of information communication. This indicates that users are more inclined to share content on social media when experiencing positive high-arousal emotions, seeking social validation and emotional resonance \cite{son2022emotion}. However, the communication effect of high-arousal emotions also poses risks. In highly emotional states, especially during crisis events or periods of social unrest, users may quickly share information without adequate verification, thereby increasing the risk of spreading fake news and misinformation \cite{dabbous2023influence}. 
% 情绪唤起度在社交媒体上的信息分享中扮演了重要角色。研究表明,无论是正面还是负面,高唤起情绪都显著增加了信息的传播性。高唤起的情绪如敬畏、兴奋、愤怒或焦虑激发用户的紧迫感和行动动机,这促使他们在社交媒体上更频繁地分享内容。在危机事件或重大新闻中,高唤起的负面情绪,如愤怒或焦虑,常引发用户的强烈反应,推动他们在社交媒体上迅速分享相关信息。这种分享行为往往是出于寻求社会支持或表达情感共鸣的需要。相反,高唤起的正面情绪如兴奋和敬畏不仅增强用户的分享行为,还能扩大信息的传播范围。这表明用户在体验到积极的高唤起情绪时,更倾向于在社交媒体上分享内容,寻求社会认同和情感共鸣。然而,高唤起情绪的传播效果也存在风险。在高度情绪化的状态下,尤其是在危机事件或社会动荡期间,用户可能在未经充分验证的情况下快速分享信息,从而增加假新闻和虚假信息的扩散风险。

Conversely, the communication effect of low-arousal emotions is relatively weak; the lack of strong stimulation usually results in users being calmer or more neutral in such emotional states, leading to a lower motivation to share \cite{son2022emotion}. However, in specific contexts, low-arousal positive emotions such as contentment or gratitude may also prompt users to share information with the aim of supporting or educating others \cite{stieglitz2013emotions}. Additionally, low-arousal negative emotions, such as sadness, may motivate users to reflect deeply and engage in discussions about major disasters or social issues. This process can promote information communication to seek social support or drive social change. \cite{de2021sadness}.
% 相对地,低唤起情绪的传播效果相对较弱,由于缺乏强烈的激发力,使得用户在这类情绪状态下通常表现出更多的冷静或中立,导致分享动机较低。然而,在特定情境下,低唤起的积极情绪如满足或感激也可能促使用户出于支持或教育他人的目的进行信息分享。此外,低唤起的负面情绪如悲伤在应对重大灾难或社会问题时,可能促使用户进行深度反思和讨论,推动信息传播以寻求社会支持或促进社会变革。

Therefore, emotional arousal is a key factor determining the effectiveness of information communication on social media, with high-arousal emotions typically amplifying both the spread of information and the motivation to share it, while low-arousal emotions exert a more subtle influence on user sharing behavior, particularly in specific contexts. These findings offer valuable guidance for optimizing social media strategies and enhancing crisis management approaches.
% 因此,情绪唤起度是决定社交媒体信息传播效果的关键因素,高唤起情绪通常能显著提升信息的传播性和分享动力,而低唤起情绪则在特定情境下以较为细微的方式影响用户分享行为。这些洞察对于制定有效的社交媒体运营和危机管理策略具有重要价值。

% \subsubsection{The Neural Mechanisms of Emotions in Sharing}
% Emotions play a crucial role in information sharing by activating multiple key areas of the brain \cite{ferrara2015measuring}. These areas include the amygdala, ventromedial prefrontal cortex (VMPFC), temporoparietal junction (TPJ), and the insula, which collectively regulate an individual's motivation and behavior for sharing. Initially, the amygdala plays a central role in processing emotional responses, such as anger or surprise. Activation of the amygdala enhances the salience of emotions, making individuals more aware of the significance of emotions, thereby increasing their motivation to express and share these emotions with others \cite{decety2008emotion}. For instance, high-arousal negative emotions such as anger or fear activate the amygdala, prompting individuals to perceive the importance of emotions and to tend towards sharing these emotions. This mechanism is particularly evident in social media, where the effect of emotional contagion operates through this pathway \cite{ferrara2015measuring}.
% % 情绪通过激活大脑多个关键区域,在信息分享中发挥重要作用。这些区域包括杏仁核、腹内侧前额叶皮层(VMPFC)、颞顶联合区(TPJ)和岛叶,它们共同调节个体的分享动机和行为。首先,杏仁核在处理情绪反应,如愤怒或惊讶时,起着核心作用。杏仁核的激活提升了情绪的显著性,使个体更加感知到情绪的重要性,从而增加了想他人表达和分享情绪的动机。例如,高唤起的负面情绪如愤怒或恐惧,使得杏仁核活跃,促使个体感知到情绪的重要性,并倾向于进行情绪分享。这种机制在社交媒体中尤为明显,情绪传染效应即通过这一途径发挥作用。

% Concurrently, the ventromedial prefrontal cortex (VMPFC) plays a crucial role in sharing positive emotions. When an individual experiences positive emotions, the VMPFC is activated and cooperates with the brain's reward system to evaluate the social and emotional benefits of sharing the emotional experience, enhancing the motivation to express emotions \cite{haber2010reward}. Additionally, when negative emotions are highly aroused, the VMPFC collaborates with the amygdala to assess the risks and rewards of expressing emotions, determining whether and how to share negative emotions \cite{bechara2000emotion}.
% %同时,腹内侧前额叶皮层(VMPFC)在积极情绪分享中扮演重要角色。当个体体验到积极情绪时,VMPFC被激活,并与大脑奖励系统协作,评估分享情绪体验的社交和情感收益,增加情绪表达的动力。此外,当消极情绪唤起度较高时,VMPFC也会与杏仁核协作,评估情绪表达的风险与回报,决定是否以及如何分享负面情绪。

% Activation of the temporoparietal junction (TPJ) enables individuals to better understand and predict others' emotional responses, adjusting their sharing strategies accordingly \cite{mitchell2009inferences}. For example, when individuals perceive that others are interested in or sympathetic to their emotional experiences, activation of the TPJ encourages them to express their emotions more openly to strengthen social connections.
% %颞顶联合区(TPJ)的激活则使个体能更好地理解和预测他人情绪反应,调整自己的分享策略。例如,当个体感知到对方对自己情绪体验感兴趣或表示同情时,TPJ的激活会促使其更开放地表达情绪,以增强社交联系。

% The insula also plays a key role when emotional experiences are accompanied by physiological responses. As a region that integrates internal bodily states with emotional experiences, the insula processes physiological signals such as increased heart rate and rapid breathing, thereby influencing the motivation to share. Craig \cite{craig2009you} notes that when emotional experiences are intense and accompanied by significant physiological reactions, high activation of the insula increases individuals' tendencies to seek resonance and support, making them more likely to share their emotions directly or intensely.
% %岛叶在情绪体验伴随生理反应时也起到关键作用。作为整合身体内部状态与情绪体验的区域,岛叶处理如心跳加快、呼吸急促等生理信号,进而影响分享动机。Craig(2009)指出,当情绪体验强烈且伴有明显的生理反应时,岛叶的高激活水平增加了个体寻求共鸣和支持的倾向,使其更可能直接或强烈地分享情绪。

% The valence (positive or negative) and arousal (high or low) of emotions jointly influence the activation patterns of the above brain regions, complexly regulating individuals' motivations and behaviors in the process of information sharing. High-arousal positive emotions typically enhance the perceived value and motivation to share positive emotions through the cooperation of the VMPFC and amygdala; whereas high-arousal negative emotions regulate the relationship between expressing emotions and social risks during sharing through interactions between the amygdala, insula, and other areas such as the ACC and DMPFC \cite{lang2010emotion}. In contrast, low-arousal emotions (whether positive or negative) typically result in less activation, and individuals rely more on the TPJ and DMPFC to decide whether to express emotions \cite{pessoa2009emotion}.
% % 情绪的效价(正性或负性)和唤起度(高或低)共同影响上述脑区的激活模式,从而复杂地调节个体在信息分享过程中的动机和行为。高唤起的正性情绪通常通过VMPFC和杏仁核的协作,增加正面情绪的价值感和分享动机;而高唤起的负性情绪则通过杏仁核、岛叶和其他区域(如ACC和DMPFC)的相互调节,使个体在分享过程中权衡表达情绪与社交风险的关系。相较之下,低唤起情绪(无论正性还是负性)通常激活较少,个体更依赖TPJ和DMPFC来判断是否表达情绪。


\subsubsection{Emotional Dominance}

Emotional dominance plays a decisive role in information sharing. Individuals in high dominance emotional states typically exhibit strong self-confidence and social dominance, making them more willing to actively share information. Such individuals tend to view sharing behavior as an important means of expressing self-worth and influence \cite{mehrabian1996pleasure}. Research indicates that this emotional state encourages individuals to participate more actively in information exchange and establish higher social status within groups \cite{fredrickson2001role}. Social cognitive theory \cite{bandura1989human} provides theoretical support for this phenomenon. Individuals with high dominance emotions often reinforce their self-efficacy through sharing behaviors and use positive feedback during the sharing process to further enhance their social skills and influence. In contrast, individuals with low dominance emotions perceive information sharing as a threat due to a lack of confidence, fearing negative evaluation from loss of control or questioning \cite{lebel2017moving}. This psychological drive leads individuals with low dominance emotions to adopt conservative or passive sharing strategies.
% 在信息分享中,情绪支配度起着决定性作用。高支配度情绪状态的个体通常伴随较强的自信心和社会主导感,从而更愿意主动分享信息。这种个体倾向将分享行为视为表达自我价值和影响力的重要手段(Mehrabian (1996))。研究表明,这种情绪状态会促使个体更加积极地参与信息交流,并在群体中建立更高的社会地位(Fredrickson (2001) )。社会认知理论(Bandura, 1989)为这种现象提供了理论支持。高支配度情绪个体往往通过分享行为强化自我效能感,并利用分享过程中的积极反馈进一步提升其社交能力和影响力。相比之下,低支配度情绪个体由于缺乏自信,通常将信息分享视为一种潜在威胁,担心因信息失控或被质疑而遭受负面评价(Lebel, 2017)。这种心理驱动促使低支配度个体选择保守或被动的分享方式。

Additionally, social comparison theory \cite{festinger1954theory} reveals the deeper motivations of emotional dominance in information sharing: individuals with high dominance emotions tend to engage in positive comparisons through sharing behaviors, thereby further solidifying their advantages within the group. For example, they are more inclined to share information related to achievements or successes to enhance others’ recognition of their abilities. In contrast, individuals with low dominance emotions tend to avoid sharing behaviors to prevent exposing weaknesses or losing control during comparisons~\cite{morrison2000organizational}.

% 此外,社会比较理论(Festinger, 1954)揭示了情绪支配度在信息分享中的深层动机:高支配度情绪个体倾向于通过分享行为进行积极比较,从而进一步巩固其在群体中的优势地位。例如,他们更愿意分享与成就或成功相关的信息,以增强他人对其能力的认可。而低支配度情绪个体则倾向于回避分享行为,以避免在比较过程中暴露弱点或失去控制权(Morrison & Milliken, 2000)。

In summary, emotional dominance not only determines the motivation and strategies for information sharing but also shapes individuals’ social roles and influence within groups \cite{mehrabian1996pleasure, bandura1989human}. High dominance emotions encourage individuals to achieve social goals through active sharing, while low dominance emotions limit their ability to disseminate information, potentially further diminishing their status and value within social networks \cite{fredrickson2001role, morrison2000organizational}.

% 总结而言,情绪支配度不仅决定了信息分享的动机和策略,还塑造了个体在群体中的社会角色与影响力(Mehrabian (1996); Bandura, 1989)。高支配度情绪促使个体通过积极分享实现社交目标,而低支配度情绪则限制了个体的信息传播能力,可能进一步削弱其社交网络中的地位和价值(Fredrickson (2001) ; Morrison & Milliken, 2000)

\subsubsection{Summary}
The influence of emotions on information sharing is multi-layered and complex. Positive emotions enhance content appeal and willingness to interact, promoting rapid information diffusion on social media \cite{ferrara2015measuring}; however, this facilitation may also lead users to overlook the authenticity of the information \cite{dabbous2023influence}. Negative emotions, with their strong emotional resonance effect, particularly in crises and political events, drive widespread communication of information \cite{lopez2021translating}. Neutral emotions, although less likely to promote communication, have unique advantages in ensuring the accuracy and objectivity of information \cite{son2022emotion}.
High-arousal emotions, whether positive or negative, significantly enhance sharing motivation and communication breadth, while low-arousal emotions are better suited for in-depth discussions and reflection \cite{berger2012makes}. Dominance affects information-sharing strategies: high-dominance emotions encourage individuals to actively share to enhance social status, while low-dominance emotions tend toward selective sharing or avoidance \cite{mehrabian1996pleasure, bandura1989human}. The combined effects of emotional valence, arousal, and dominance further reveal the multidimensionality of information communication, providing valuable insights for optimizing social media communication~strategies.


% 情绪对信息分享的影响多层次且复杂。积极情绪通过增强内容吸引力和互动意愿,促进信息在社交媒体上的快速扩散(Ferrara & Yang, 2015);然而,这种促进作用也可能导致用户忽略信息真实性(Dabbous & Barakat, 2023)。消极情绪以其强烈的情感共鸣效应,特别是在危机和政治事件中,推动信息的广泛传播(Rojo López & Naranjo, 2021)。中性情绪尽管传播性较弱,但在信息准确性和客观性上具有独特优势(Son et al., 2022)。高唤起情绪无论正负,均显著提升分享动力和传播广度,而低唤起情绪则在深度讨论和反思中表现更佳(Berger & Milkman, 2012)。支配度影响信息分享策略,高支配度情绪促使个体积极分享以强化社会地位,而低支配度情绪则倾向于选择性分享或回避分享(Mehrabian, 1996; Bandura, 1989)。情绪效价、唤起度和支配度的综合作用进一步揭示了信息传播的多维性,为优化社交媒体传播策略提供了重要启示。
\section{How to Promote Information Communication by Regulating Emotions?}\label{sec:how} % 如何通过调节情感促进信息传达

% 把若妍调研的4个系统放到How的部分,情感是如何激活的,

% 设计空间要把What 和 how 的东西都涵盖。

% 怎么控制情感要素出发。

This section explores how design strategies can regulate emotions to optimize the effectiveness of information delivery. This study proposes a comprehensive multidimensional design framework aimed at regulating emotional valence, arousal and dominance, thereby providing theoretical support and practical guidance for the understanding, memory, and communication of information. The section begins with the construction of the overall design space and theoretical foundations, introducing a multi-system model of emotional activation as the guiding framework. It then analyzes the specific roles of text, visuals, sound, and interaction in emotional regulation across four dimensions. Through these analyses, the section comprehensively illustrates how multimodal design can be flexibly applied across various contexts to achieve effective emotional regulation and information optimization.
% 本章探讨如何通过设计手段调节情感,以优化信息的传递效果。本研究提出了一个综合性的多维度设计框架,旨在通过调节情感效价、唤起水平和支配感,为信息的理解、记忆和传播提供理论支持与实践指导。本章内容将首先从整体设计空间的构建与理论基础出发,提出情绪激活的多系统模型作为指导框架,随后从文本、视觉、声音和交互四个维度分别分析其对情绪调控的具体作用。通过这些分析,本章全面阐明了多模态设计如何在多种情境中灵活应用,从而实现高效的情感调控与信息优化。


\subsection{Overview of the Design Space}
The construction of the design space is founded on the systematic integration of various design elements, aiming to precisely control emotional valence and arousal through their modulation to optimize information delivery. EExisting research identifies four core dimensions of user experience design: text, visuals, sound, and interaction. Mohamad Roseli \& Aziz \cite{roseli2023affective}, through thematic analysis, summarized eight key design elements—images, text, audio, color, layout, navigation, feedback, and reward mechanisms—which can be further classified into the aforementioned four core dimensions. This classification highlights the critical role of each design dimension in shaping users’ information reception and emotional responses. Therefore, incorporating these core design dimensions into the design space systematically maps the pathways of users’ emotional responses, providing targeted strategies for emotional regulation.
% 设计空间的构建以多种设计元素的系统整合为基础,旨在通过调节这些元素实现对情感效价与唤起度的精准控制,以优化信息传达效果。现有研究指出,用户体验中的核心设计层面可划分为文本、视觉、声音和交互四个维度。Mohamad Roseli 和 Aziz (2023)通过主题分析总结了图像、文本、音频、颜色、布局、导航、反馈与奖励机制等八大关键设计要素,这些要素可进一步归类至上述四个核心维度中。这种分类明确了各个设计层面在用户接收信息与情感反应中的重要作用。因此,将这些核心设计层面纳入设计空间中,能够系统覆盖用户情感反应的路径,为情感调控提供针对性的设计策略。

\begin{table}[htbp]
\centering
\setlength{\tabcolsep}{7pt}
\captionsetup{skip=10pt}
\begin{tabular}{|c|c|c|}
\hline
\multirow{4}{*}{\begin{tabular}[c]{@{}c@{}}\textbf{Text} \\ \textbf{Design}\end{tabular}} 
& Headline & \cite{kuiken2017effective, kim2016compete, kourogi2015identifying, blom2015click} \\ \cline{2-3} 
& Narrative structure & \cite{mar2011emotion, jaaskelainen2020neural, menninghaus2017emotional} \\ \cline{2-3} 
& Narrative content & \cite{lekkas2022using, leshner2018breast, campbell2008hero} \\ \cline{2-3} 
& Description & \cite{seo2019process,lee2020impact, ludwig2013more, slavova2019towards, nguyen2014affective} \\ \hline

\multirow{4}{*}{\begin{tabular}[c]{@{}c@{}}\textbf{Visual} \\ \textbf{Design}\end{tabular}} 
& Color & \cite{pazda2024colorfulness, jonauskaite2019color, kallabis2024investigating, weijs2023effects, lin2023effect, suk2010emotional, tarvainen2014content, wilms2018color} \\ \cline{2-3} 
& Shape & \cite{lu2012shape, etzi2016arousing, ebe2015emotion, wei2006image, mayer2014benefits, ferrara2015measuring, thumfart2008modeling} \\ \cline{2-3} 
& Layout & \cite{makin2012implicit, lu2017investigation, carretie2019emomadrid, wang2022innovation, mai2011rule, resnick2003design} \\ \cline{2-3} 
& Images & \cite{hanson2014happy, xie2017negative,van2015good, hou2024emotional, pfeuffer122measuring, kensinger2007negative,dudarev2024social, kuzinas2024creative} \\ \hline

\multirow{3}{*}{\begin{tabular}[c]{@{}c@{}}\textbf{Sound} \\ \textbf{Design}\end{tabular}} 
& Tone & \cite{schirmer2010mark, weinstein2018you, bestelmeyer2017effects,  gobl2003role} \\ \cline{2-3} 
& Sound effects & \cite{eerola2012timbre, parncutt2011consonance, schulte2001quality, ostendorf2020sounds, mazur2019effects} \\ \cline{2-3} 
&  Music & \cite{hofbauer2024background, moon2024investigating, shepherd2024investigating, thaut2015neurobiological, bernardi2006cardiovascular, juslin2008emotional,kabre2024predisposed, baltazaremotional} \\ \hline

\multirow{3}{*}{\begin{tabular}[c]{@{}c@{}}\textbf{Interaction} \\ \textbf{Design}\end{tabular}} 
& Interaction methods & \cite{sundar2014user, wodehouse2014exploring, wang2024enhancing,  olugbade2023touch, amoor2014designing } \\ \cline{2-3} 
& Motion effects & \cite{hanjalic2005affective, wollner2018slow, lockyer2012affective, yoo2005processing} \\ \cline{2-3} 
& Navigation design & \cite{wang2024enhancing, abdelaal2023accessibility, sheng2012effects, sundar2014user, amoor2014designing} \\ \hline
\end{tabular}

\caption{Design Dimensions and Emotional Aspects}
\label{table:design_dimensions}
\end{table}

After clarifying the design dimensions, the specific design elements within each dimension were further refined. For instance, in text design, narrative structure and wording style can evoke user emotions through plot tension and word choice \cite{egidi2012emotional}. Visual design relies on elements such as color, image, shapes, and layout to influence users’ emotional valence and arousal. Sound design elicits emotional responses through tone, music, and sound effects \cite{plass2014emotional}. Interaction design enhances users’ dominance through interactive methods, motion effects, and navigation designs \cite{shneiderman2010designing}. The integration of these design dimensions provides a structured framework for emotional regulation, enabling the development of tailored strategies to meet the demands of various communication contexts.
% 在明确设计维度后,进一步细化了各维度的设计元素。例如,在文本设计方面,叙事结构与措辞风格能够通过情节张力与词汇选择调动用户的情绪【Egidi & Nusbaum, 2012】;视觉设计依托颜色、图片、形状与布局等元素影响用户的情感效价与唤起度;声音设计通过语调、音乐与音效激发情感反应【Plass et al., 2014】;交互设计则通过交互方式、动效与导航功能强化用户的支配感【Shneiderman, 1998】。这些设计维度的整合为情感调控提供了结构化的框架,能够针对不同传播情境的需求制定相应的策略。


To further reveal how these design elements specifically function in emotional regulation, this study introduces the multi-system model of emotional activation \cite{izard1993four}. The model emphasizes that emotional activation is the result of multi-level and multi-mechanism collaboration, involving the coordination of the neural system, sensorimotor system, motivational system, and cognitive system.
% 为进一步揭示这些设计元素如何具体作用于情感调控,本研究引入了情绪激活的多系统模型(Izard, 1993)。该模型强调,情绪激活是多层次、多机制协作的结果,涉及神经系统、感官运动系统、动机系统和认知系统的协同参与。


The nervous system rapidly generates emotional responses through the activity of specific brain regions, such as the interaction between the amygdala and prefrontal cortex. For example, sensory stimuli such as color\cite{pazda2024colorfulness} or sound\cite{clewett2024emotional} can quickly capture attention and evoke emotional responses, thereby enhancing the appeal and impact of information. The sensorimotor system reinforces emotional experiences through physical sensations and motor responses (e.g., increased heart rate, muscle tension), which can be amplified by tactile feedback\cite{olugbade2023touch} or dynamic visual effects\cite{amoor2014designing}. In information communication design, these sensory-level feedback mechanisms help convey emotional information, engage users more deeply, and enhance the perception and effectiveness of information delivery. The motivational system triggers emotional responses by regulating goal-oriented behavior. For instance, commonly used reward mechanisms or time-limited promotional strategies in information design effectively stimulate motivational emotions, facilitating the dissemination and reception of information. The cognitive system generates emotional responses through the evaluation, reasoning, and judgment of information. Activation of this system is particularly crucial in information communication. For example, narrative texts \cite{mar2011emotion, lekkas2022using} or complex visual symbols \cite{ferrara2015measuring, mayer2014benefits} can guide users’ emotional evaluations, enhancing the depth and impact of the information~content.
% 神经系统通过大脑特定区域的活动(如杏仁核与前额叶皮层的交互)快速生成情绪反应。例如,通过颜色或声音的感官刺激可以迅速引发注意力与情绪反应,从而增强信息的吸引力与影响力【Jacobs & Mendl, 1999】。感官运动系统则通过身体的感知与运动反应(如心跳加快、肌肉紧张)进一步强化情绪体验,这些反应可通过触觉反馈或动态的视觉效果得到增强。在信息传达设计中,这些感官层面的反馈帮助传递情感信息,使用户更加投入,从而提升信息的感知度与传达效果。动机系统通过目标导向行为的调节引发情绪反应,例如信息设计中常用的奖励机制或限时促销策略能够有效激发动机情绪,促进信息的传播与接收。认知系统则通过对信息的评估、推理与判断生成情绪反应。这一系统的激活在信息传达中尤为重要,例如叙事性文本或复杂的视觉符号设计能够引导用户的情绪评估,增强信息内容的深度与感染力。


\begin{figure}[hbt!]
%\setlength{\abovecaptionskip}{-0.1mm}
\setlength{\intextsep}{10pt plus 2pt minus 2pt}
    \centering
    \includegraphics[width=0.5\textwidth]{figs/Multisystem_model.png}
    \caption{Multisystem Model of Emotion Activation: A conceptual framework illustrating the interaction between neural processes, sensorimotor processes, affective processes, cognitive processes, and emotional experience in the generation and regulation of emotions.}
   \vspace{-2mm}
\label{fig:What}
\end{figure}

The interactions among these emotional systems not only determine the generation and expression of emotions but also shape the specific pathways of information communication. By modulating the activation levels of these systems, design can effectively regulate emotions, thereby optimizing the effectiveness of information delivery. This integration of theory and practice provides a solid theoretical foundation and practical guidance for multidimensional design spaces.
% 这些情感系统的相互作用不仅决定了情绪的生成与表达,也塑造了信息传达的具体路径。通过调控这些系统的激活程度,设计可以实现对情感的有效调节,从而优化信息的传递效果。这种理论与实践的结合为多维设计空间提供了坚实的理论基础与实操指引。

\subsection{Text Design}
Text is not merely a means of conveying information; it also regulates readers’ emotional responses through the skillful arrangement of narrative structures and wording, thereby enhancing memory, comprehension, and information sharing. In text design, different headline, narrative structures, content, wording, and styles of expression each uniquely influence emotions, allowing regulation of the audience’s emotional valence and arousal from multiple perspectives. The following analyzes the impact of text design on emotions from these four dimensions. The structure and format of titles collectively influence readers’ emotional valence and arousal in various ways. Narrative structure shapes users’ emotional experiences through the progression of stories and emotional tone, while narrative content guides emotional responses through plot arrangements and character resonance. Meanwhile, wording and expression styles influence the intensity and direction of emotions through specific words and framing.

% 文本在信息传达中不仅仅是表达信息的手段,它还通过叙事结构和措辞的巧妙安排来调节读者的情感反应,从而提升文本的记忆、理解和信息分享。在文本设计中,不同的叙事结构、叙事内容以及措辞和表达风格对情感的影响各具特色,能够从多个角度调控受众的情绪效价和情绪唤起度。以下从这四个维度来分析文本设计对情感的影响。标题的结构和形式通过多种方式综合影响读者的情绪效价和情绪唤起度。叙事结构通过故事的展开方式和情感基调来塑造用户的情绪体验,叙事内容则通过情节安排和角色共鸣引导情绪反应; 而措辞和表达风格则通过具体的词语和框架影响情绪的强度和方向。


\renewcommand{\arraystretch}{1.8} % 调整行间距
\begin{table*}[ht]
\fontsize{8}{9}\selectfont
\centering

\begin{tabularx}{\textwidth}{|>{\centering\arraybackslash}m{0.5cm}|>{\centering\arraybackslash}m{1.4cm}|>{\centering\arraybackslash}m{3.4cm}|>{\centering\arraybackslash}m{3.4cm}|>{\centering\arraybackslash}m{3.4cm}|>{\centering\arraybackslash}m{3.4cm}|}
\hline
\rowcolor[HTML]{D9EAD3} 
\multicolumn{2}{|c|}{\textbf{Dimension}} & \textbf{Headline} & \textbf{Narrative Structure} & \textbf{Narrative Content} & \textbf{Wording} \\ 
\hline

% Emotional Dimensions Section
\multirow{3}{*}{\rotatebox{90}{\parbox{3cm}{\centering \textbf{Emotional \\ Dimensions}}}} & 
\cellcolor[HTML]{FDF6E8} \textbf{Valence} &    
Celebrity references evoke positive emotions and attract readers \cite{kim2016compete}.
Forward cues (e.g., “Here’s why”) increase positivity \cite{blom2015click}. & 
Positive tone (e.g., hope, success) enhances positive valence, while negative tone (e.g., tragedy) intensifies negative emotions \cite{mar2011emotion}. & 
Positive events (e.g., recovery, victory) evoke positive emotions, while negative events (e.g., loss) evoke negative emotions \cite{lekkas2022using}. & 
Gain framing (e.g., positive outcomes) boosts positive valence, while loss framing enhances negative valence \cite{seo2019process}. \\ \cline{2-6}

& \cellcolor[HTML]{FDF6E8} \textbf{Arousal} & 
Concise headlines quickly grab attention \cite{blom2015click}. 
Hot topics and suspense increase arousal \cite{kim2016compete}. & 
Suspenseful or high-peak narratives sustain high arousal levels \cite{leshner2018breast}. & 
Emotional content (e.g., loss or pain) enhances arousal. Resonance with characters amplifies engagement \cite{lekkas2022using}. & 
Emotional language (e.g., “pain”) heightens intensity; parallel structures boost appeal \cite{menninghaus2017emotional}. \\ \cline{2-6}

& \cellcolor[HTML]{FDF6E8} \textbf{Dominance} & 
Clear headlines enhance user control (e.g., “Solve it in one step”) \cite{kourogi2015identifying}. Threatening headlines reduce trust \cite{kim2016compete}. & 
Heroic narratives (e.g., success stories) enhance control; failure narratives reduce it \cite{jaaskelainen2020neural}. & 
Success stories increase control; alternating emotions in complex narratives enrich the experience \cite{lekkas2022using}. & 
Positive language (e.g., “happiness”) boosts control, while abrupt phonemes reduce it \cite{menninghaus2017emotional}. \\ \hline

% Multisystem Activation Section
\multirow{4}{*}{\rotatebox{90}{\parbox{3cm}{\centering \textbf{Multisystem \\ Activation}}}} 

& \cellcolor[HTML]{F0EFF7} \textbf{Neural Systems} & 
Urgent headlines (e.g., “Danger ahead”) activate the amygdala, triggering physiological responses like increased heart rate \cite{panksepp2012archeology}. & 
Tense narratives activate the amygdala, causing fear or excitement; complex stories engage the prefrontal cortex \cite{jaaskelainen2020neural}. & 
Descriptions of fear or anger activate the sympathetic nervous system, leading to adrenaline release \cite{panksepp2012archeology}. & 
Emotionally charged wording triggers the amygdala, inducing tension or excitement \cite{panksepp2012archeology}. \\ \cline{2-6}

& \cellcolor[HTML]{F0EFF7} \textbf{Sensorimotor Systems} & 
Urgent headlines (e.g., “Last chance”) trigger physical responses like muscle tension \cite{james1884mind}. & 
Climax or conflict scenes evoke bodily reactions, such as rapid breathing \cite{panksepp2012archeology}. & 
 Vivid descriptions of conflict or fear may cause muscle tension or mimicry (e.g., holding breath) \cite{james1884mind}. & 
Emotional words (e.g., “tragic”) may provoke physical reactions like frowning or fast breathing \cite{panksepp2012archeology}. \\ \cline{2-6}

& \cellcolor[HTML]{F0EFF7} \textbf{Cognitive Systems} & 
Headlines guide quick judgment (e.g., “A miracle happened” evokes optimism) \cite{lazarus1991emotion}. & 
Narratives help evaluate emotional contexts (e.g., hope in heroic stories, sadness in tragic ones) \cite{mar2011emotion}. & 
Readers assess character actions, evoking emotions like anger or empathy \cite{nguyen2014affective}. & 
Positive words (e.g., “success”) evoke optimism; negative words (e.g., “failure”) evoke pessimism \cite{seo2019process}. \\ \cline{2-6}

& \cellcolor[HTML]{F0EFF7} \textbf{Motivational Systems} & 
Positive headlines (e.g., “Act now”) stimulate motivation, while urgency headlines evoke avoidance \cite{calvo2010affect}. & 
Heroic narratives inspire achievement motivation; twists encourage exploration \cite{campbell2008hero}. & 
Adventure stories spark exploration; success stories foster pursuit motivation \cite{mar2011emotion}. & 
Positive wording inspires pursuit motivation; negative wording triggers avoidance or self-protection \cite{lazarus1991emotion}. \\  \hline

\end{tabularx}

\caption{Emotional Dimensions and Multisystem Activation in Text Design Elements}
\label{tab:text_design}
\end{table*}

% \begin{figure*}[hbt!]
% %\setlength{\abovecaptionskip}{-0.1mm}
% \setlength{\intextsep}{10pt plus 2pt minus 2pt}
%     \centering
%     \includegraphics[width=18cm]{figs/text_design.png}
%     \caption{Emotional Dimensions and Multisystem Activation in Text Design Elements.}
%    \vspace{-2mm}
% \label{fig:why}
% \end{figure*}


\subsubsection{Headline}

\begin{wrapfigure}{l}{0.06\textwidth}
  %\begin{center}
  % \vspace{-11pt} % 调整垂直位置
    \includegraphics[width=0.07\textwidth]{figs/icon/headline.png}
  %\end{center}
\end{wrapfigure} 

Headline, as the first point of contact for readers engaging with a text, significantly influence emotional valence and arousal levels. Studies indicate that forward-reference, a common headline design strategy, employs pronouns or phrases to hint at subsequent content (e.g., “This is why” or “What you would never expect”), creating an information gap that arouses readers’ curiosity about the unknown, prompting further reading \cite{blom2015click}. This technique, by leveraging logical incompleteness of information, stimulates exploratory interest and serves as an effective method to enhance emotional arousal. 
% 标题作为读者接触文本的第一入口,其结构和形式对情绪效价和唤起度具有显著影响。研究表明,前向引用作为一种常见的标题设计策略,通过使用代词或短语暗示后续内容(例如“这就是为什么”或“你绝对想不到的事情”),制造信息缺口,从而引发读者对未知信息的好奇心,促使其进一步阅读(Blom & Hansen, 2015)。这种技巧通过逻辑上的信息不完整性激发探索兴趣,是提升情绪唤起度的有效手段。

This process not only affects readers’ cognition but also activates the amygdala and sympathetic nervous system through the neural system, triggering physiological responses such as increased heart rate and pupil dilation, which further enhance tension and focus \cite{panksepp2012archeology, calvo2010affect}. Such physiological activation gives title design a stronger emotional driving force. Additionally, titles such as “Last Chance” or “Miss It, Lose It” may also trigger bodily alertness through the sensorimotor system, such as muscle tension or rapid breathing, aiding users in quickly focusing on information \cite{james1884mind}.
% 这一过程不仅对读者的认知产生影响,还通过神经系统激活杏仁核和交感神经系统,触发心跳加速、瞳孔扩张等生理反应,进一步增强紧张感和注意力集中(Panksepp & Biven, 2012,Calvo & D’Mello, 2010)。这种生理层面的激活使标题设计具有更强的情绪驱动力。此外,类似“最后机会”或“错过即失”之类的标题,还可能通过感官运动系统引发身体警觉反应,例如肌肉紧张或快速呼吸,帮助用户快速聚焦信息(James-Lange, 1884)。

In contrast, suspense-based strategies focus on creating emotional tension and anticipation within headlines, using phrases such as “You’ll regret not seeing this” or “Stunning discoveries shocked everyone” to emphasize importance or surprise and evoke emotional engagement. This design also significantly impacts the cognitive system by guiding users to quickly assess situations (e.g., “Miracle Happens” evokes positive emotions, while “Time is Running Out” evokes anxiety), prompting them to take action \cite{lazarus1991emotion}.
% 与之相比,悬念设置则更注重在标题中营造情绪上的紧张感和期待,例如使用“不看你会后悔”或“惊人的发现震撼了所有人”等语句,通过强调内容的重要性或意外性来刺激情绪参与感。这种设计对认知系统同样产生显著影响,通过引导用户快速判断情境(如“奇迹发生”激发正向情感,“时间紧迫”激发焦虑)促使用户采取行动(Lazarus, 1991)。

The simplicity of a headline and the use of symbols are equally crucial for eliciting readers’ emotional responses. Concise headlines reduce cognitive load, enabling readers to quickly grasp core information and thereby enhancing emotional arousal levels. Special symbols (e.g., parentheses or dashes) further emphasize key information, increasing the attractiveness of headlines \cite{kourogi2015identifying}. Additionally, celebrity effects and trending topics can significantly enhance emotional arousal levels. Incorporating celebrity names or referencing popular events in headlines helps capture attention, evoke emotional reactions, and increase click-through rates \cite{kim2016compete}.
% 标题的简洁性和符号使用对读者情绪反应同样至关重要。简短标题降低认知负荷,使读者能够快速理解核心信息,从而提升情绪唤起水平。而特殊符号(如括号或破折号)则进一步突出重点信息,提高标题的吸引力(Kourogi et al., 2015)。此外,名人效应和热点话题能够显著提升情绪唤起水平。在标题中使用名人姓名或提及热门事件,有助于吸引注意力并激发情感反应,同时提高点击率(Kim et al., 2016)。

However, the use of vague wording and interrogative sentences in title design must be approached with caution. Although they can enhance emotional arousal, excessive use may undermine content credibility and reduce emotional valence \cite{kuiken2017effective}. Therefore, by balancing simplicity, symbol usage, and information clarity, title design can achieve an optimal balance between attracting attention and maintaining trust.
%然而,标题设计中模糊措辞和疑问句的使用需谨慎。尽管它们可以增强情绪唤起,但若使用过多可能影响内容可信度,降低情绪效价(Kuiken et al., 2017)。因此,通过平衡简洁性、符号运用和信息明确性,标题设计能够在吸引注意力与保持信任度之间取得最佳效果。


\subsubsection{Narrative Structure}

\begin{wrapfigure}{l}{0.06\textwidth}
  %\begin{center}
  \vspace{-11pt} % 调整垂直位置
    \includegraphics[width=0.07\textwidth]{figs/icon/narrative_structure.png}
  %\end{center}
\end{wrapfigure} 

Narrative structure significantly influences readers’ emotional experiences by organizing information and plot arrangements. The core of narrative structure lies in designing the unfolding of a story, including setting emotional tone frameworks, controlling emotional rhythms, and employing tension and suspense to shape emotional valence and arousal levels. The emotional tone framework at the beginning lays the overall atmosphere of the story and is a critical component of the narrative structure. For instance, adopting a negative tone as the opening design can add a sense of oppression and tension to the narrative, making subsequent conflicts more striking; whereas a positive tone creates a relaxed and pleasant emotional atmosphere for readers, mitigating emotional tension in later developments \cite{mar2011emotion}. This emotional framework not only influences the story’s overall atmosphere but also modulates the degree of readers’ emotional engagement. Negative-toned scenarios may also activate the sensorimotor system by triggering reactions such as muscle tension or rapid breathing~\cite{james1884mind, panksepp2012archeology}.
% 叙事结构通过对信息和情节的组织安排显著影响读者的情绪体验。叙事结构的核心在于设计故事的展开方式,包括情感基调框架的设定、情绪节奏的控制以及紧张感和悬念的运用,从而塑造情绪效价和情绪唤起水平。开篇的情感基调框架为故事奠定整体氛围,是叙事结构的重要组成部分。例如,采用负面基调作为开篇设计能够为整个叙事增添压抑感和张力,使后续情节的冲突更加引人注目;而正面基调开篇则为读者营造轻松愉悦的情绪氛围,从而缓和后续情节中的情绪紧张感(Mar et al., 2011)。这种情感框架不仅影响故事的整体氛围,还能够调节读者的情感投入程度。负面基调情节还可能通过触发肌肉紧张或呼吸急促等反应,激活感官运动系统(James-Lange, 1884;Panksepp & Biven, 2012)。


Tension and suspense design are crucial methods for enhancing emotional arousal in narrative structures. By gradually revealing crisis situations and delaying information disclosure, narratives can evoke readers’ anticipation and emotional engagement. For example, during the climax or key turning points of a story, creating conflicting tensions can elevate emotional intensity, significantly enhancing the immersive experience of emotional engagement. Additionally, strategically placing high-emotion points (such as suspense or climax scenes) at the beginning, climax, and end of a narrative enriches the emotional rhythm and further strengthens emotional valence \cite{jaaskelainen2020neural}. Suspenseful design not only sustains emotional tension but may also activate the reader’s nervous system, leading to increased heart rate and pupil dilation, further strengthening emotional memory through amygdala activity~\cite{panksepp2012archeology, jaaskelainen2020neural} .
%紧张感与悬念设计是叙事结构提升情绪唤起的重要手段。通过逐步揭示危机情境和延迟信息披露,叙事能够引发读者的期待和情感投入。例如,在故事的高潮或关键转折点,通过制造矛盾冲突来提升情绪强度,可以显著增强情感体验的沉浸感(Jääskeläinen et al., 2020)。悬念设计不仅使情绪紧张感得以延续,还可能激活读者的神经系统,导致心跳加速和瞳孔扩张,并通过杏仁核的活动进一步强化情绪记忆(Panksepp & Biven, 2012;Jääskeläinen et al., 2020)。

Additionally, strategically placing high-emotion nodes (such as suspense or climactic scenes) at the beginning, climax, and conclusion of a narrative enriches the emotional rhythm and further enhances emotional valence \cite{jaaskelainen2020neural}. The control of emotional rhythm is another crucial factor by which narrative structure influences emotional experiences. Stories with high-frequency emotional fluctuations maintain tension by rapidly switching emotional states (e.g., alternating fear and hope), keeping readers highly engaged. However, excessively frequent fluctuations may lead to emotional fatigue, reducing readers’ emotional engagement \cite{lekkas2022using}. Conversely, narratives with more stable emotional rhythms are better suited to guiding audiences into sustained emotional states; for example, heartwarming stories maintain a consistent emotional tone, allowing readers to feel comfortable and relaxed throughout the reading process \cite{jaaskelainen2020neural}. This control of emotional rhythm not only enhances the narrative’s immersive quality but also prevents emotional overload or~fatigue.
% 此外,在叙事开端、高潮和结尾合理分布高情绪节点(如悬念或高潮情节),能够使情感节奏更为丰富,并进一步强化情绪效价。% 情绪节奏的控制是叙事结构影响情绪体验的另一重要因素。高频情绪波动的故事通过情感状态的快速切换(如恐惧与希望交替)来保持紧张感,使读者始终保持高度关注。然而,过于频繁的波动可能导致情绪疲劳,削弱读者的情感参与度(Lekkas et al., 2022)。相反,情绪节奏较为稳定的叙事结构更适合引导观众逐步进入持续的情感状态,例如温情类故事通过一致的情感基调使观众在整个阅读过程中始终感到舒适与放松(Jääskeläinen et al., 2020)。这种情绪节奏的控制不仅有助于增强叙事的沉浸感,还能够避免情绪体验的过载或疲惫。

% In contrast, non-narrative structures are more direct, rapidly eliciting emotions through linear information delivery, particularly in content involving negative emotions, which can quickly provoke tension or fear in readers. For instance, breaking news reports often focus on negative emotions, triggering defensive reactions or action motivations among the audience \cite{leshner2018breast}. Such defensive reactions not only prompt audiences to quickly evaluate risks and threats, thereby activating the cognitive system \cite{lazarus1991emotion}, but also trigger avoidance motivation by emphasizing the urgency of threats, leading users to take practical actions to avoid risks or address issues, thereby activating the motivational system \cite{calvo2010affect}.

%相比之下,非叙事结构则更加直接,通过信息的线性传达迅速激发情绪,特别是在涉及负面情绪内容时,这种结构能够快速引发读者的紧张或恐惧情绪。例如,突发新闻的直接报道常常以负面情感作为重点,激发受众的防御反应或行动动机(Leshner et al., 2018)。这种防御反应不仅促使受众快速评估风险与威胁,从而激活认知系统(Lazarus, 1991),还通过强调威胁的紧迫性触发规避动机,使用户倾向于采取规避风险或应对问题的实际行为,从而激活动机系统(Calvo & D’Mello, 2010)。




\subsubsection{Narrative Content}

\begin{wrapfigure}{l}{0.06\textwidth}
  %\begin{center}
  \vspace{-11pt} % 调整垂直位置
       \includegraphics[width=0.07\textwidth]{figs/icon/narrative_content.png}
  %\end{center}
\end{wrapfigure} 

Narrative content profoundly influences audience emotions through the selection of specific plot elements, character development, and emotional tension control. Different plot types directly shape emotional valence; for example, negative emotional content (such as crises, pain, or loss) often evokes sadness and sympathy, while positive emotional content (such as recovery, victory, or hope) enhances positive emotional valence, making the audience feel optimistic and reassured \cite{mar2011emotion, lekkas2022using}. This influence on emotional valence not only affects the cognitive level but may also trigger noticeable physiological reactions. For example, scenes depicting fear or tension often lead to bodily mimicry responses, such as muscle tension or rapid breathing, thereby activating the sensorimotor system \cite{james1884mind, panksepp2012archeology}.
% 叙事内容通过具体情节的选择、角色塑造和情感张力的控制,对受众的情绪产生深远影响。不同情节类型能够直接塑造情绪效价,例如,负面情绪内容(如危机、痛苦或失落)往往引发悲伤、同情等负面情绪;而正面情绪内容(如康复、胜利或希望)则增强积极情绪效价,使观众感到乐观和安心(Mar et al., 2011;Lekkas et al., 2022)。
% 这种对情绪效价的影响不仅作用于认知层面,还可能引发明显的生理反应,例如描述恐惧或紧张的场景往往会导致身体模仿反应,如肌肉紧张或呼吸急促,从而激活感官运动系统(James-Lange, 1884;Panksepp & Biven, 2012)。

Character development and emotional resonance are important means of deepening emotional experiences in narrative content. When emotional depictions of characters in a narrative are authentic and detailed, the audience is more likely to experience psychological resonance, enter the emotional world of the characters, and further enhance emotional arousal. For instance, plot designs in which characters encounter and gradually overcome challenges can lead the audience to empathize and form deep emotional identification with the characters’ experiences. Additionally, personal associations triggered by the plot content can intensify emotional responses. When the audience connects the story to their own experiences, this similarity can significantly amplify emotional valence, making the emotional experience more~profound \cite{mar2011emotion, nguyen2014affective}. This process activates the audience's cognitive system, which evaluates the reasonableness of the characters' actions within the context, further modulating emotional responses. %The formation of such emotional resonance relies on the cognitive system, where readers evaluate events by analyzing the rationality of characters’ behaviors and contexts \cite{lazarus1991emotion, }.
% 角色塑造和情感共鸣是叙事内容深化情绪体验的重要途径。当叙事中的角色情感描写真实且细腻时,受众更容易产生心理共鸣,进入角色的情感世界,进一步增强情绪唤起水平。例如,故事中的角色经历困境并逐步解决的情节设计,能够引导受众感同身受,并对角色的情感体验形成深刻认同。同时,情节内容引发的个人联想也会强化情绪反应。当受众将故事与自身经历联系起来时,这种相似性能够显著放大情绪效价,使情感体验更加深刻(Mar et al., 2011)。这种情感共鸣的形成依赖于认知系统的作用,读者在这一过程中通过分析角色行为与情境的合理性,对事件进行评估(Lazarus, 1991;Nguyen et al., 2014)。这一过程激活了受众的认知系统,通过评估角色行为与情境的合理性,进一步调节情绪反应。

The regulation of plot intensity and emotional tension directly influences the depth and duration of emotional arousal. For example, intense negative emotional scenarios (such as loss, fear, or crisis) often trigger high emotional arousal \cite{jaaskelainen2020neural}, activating the amygdala and the sympathetic nervous system in the neural network, leading to adrenaline secretion and maintaining the audience's focused attention \cite{panksepp2012archeology, jaaskelainen2020neural}. This physiological response not only intensifies the emotional experience but also extends the emotional memory, leaving a profound impression of the storyline on the audience. On this basis, the design of complex emotional plots (such as the intertwining of pain and hope) enhances the depth of emotional layers, allowing the audience to experience deeper resonance through the emotional highs and lows \cite{leshner2018breast}. This display of emotional complexity not only enables the audience to experience richer emotions but also stimulates achievement motivation, further influencing their behaviors and emotional responses \cite{mar2011emotion, campbell2008hero}.
% 情节强度和情感张力的调控则直接影响情绪唤起的深度和持续性。例如,强烈的负面情绪情节(如失去、恐惧或危机)通常会引发高情绪唤起,激活神经系统中的杏仁核和交感神经系统,引发肾上腺素分泌并保持受众的注意力集中(Panksepp & Biven, 2012;Jääskeläinen et al., 2020)。这种生理反应不仅增加情绪体验的强度,还能延续情绪的记忆,促使受众对故事情节保持深刻印象。在此基础上,复杂情绪的情节设计(如痛苦与希望交织)增强了情感的层次感,使观众在情感的起伏中获得更深刻的共鸣(Leshner et al., 2018)。这种情感复杂性的展示,不仅使观众体验更为丰富的情绪,还能激发成就动机,进一步影响受众的行为和情绪反应(Mar et al., 2011;Campbell, 1949)。


% The regulation of plot intensity and emotional tension directly influences the depth and persistence of emotional arousal. For instance, intense negative emotional scenarios (such as loss, fear, or crisis) often provoke high emotional arousal, activating the amygdala and sympathetic nervous system, triggering adrenaline secretion and maintaining the audience’s focus \cite{panksepp2012archeology, jaaskelainen2020neural}. This neural activation mechanism not only amplifies the intensity of emotional experiences but also prolongs the emotional afterglow, leaving the audience vividly recalling the content even after the story ends.
% 情节强度和情感张力的调控则直接影响情绪唤起的深度和持续性。例如,强烈的负面情绪情节(如失去、恐惧或危机)通常会引发高情绪唤起,激活神经系统中的杏仁核和交感神经系统,引发肾上腺素分泌并保持受众的注意力集中(Panksepp & Biven, 2012;Jääskeläinen et al., 2020)。这种神经系统的激活机制不仅增强了情绪体验的强度,还能够延续情绪余韵,使观众在故事结束后仍对内容记忆犹新。

% Meanwhile, the complex emotional design in narratives (such as the interplay of pain and hope) can further enhance emotional engagement through the characters’ efforts and successes \cite{leshner2018breast}. This expression of complex emotions not only enriches the audience’s emotional experience but also activates the motivational system, particularly achievement motivation, through the characters’ accomplishments. For instance, audiences may project the characters’ successes onto their own goals, thereby reinforcing their own behavioral drive \cite{mar2011emotion, campbell2008hero}. Through such design, the impact of narrative content transcends the text itself, creating a more enduring resonance in the audience’s emotions and~behaviors.
% % 与此同时,叙事中的复杂情感设计(如痛苦与希望交织)可以通过角色的努力与成功,进一步强化情感参与。这种复杂情绪的表达不仅能够丰富受众的情感体验,还能够通过角色的成就激发动机系统,尤其是成就动机。例如,观众可能将角色的成功投射到自身目标中,从而强化对自身行为的驱动力(Mar et al., 2011;Campbell, 1949)。通过这样的设计,叙事内容的影响得以超越文本本身,使其在受众的情绪和行为中产生更持久的共鸣。

\subsubsection{Wording} %措辞/表达风格
\begin{wrapfigure}{l}{0.06\textwidth}
  %\begin{center}
  \vspace{-11pt} % 调整垂直位置
    \includegraphics[width=0.07\textwidth]{figs/icon/wording.png}
  %\end{center}
\end{wrapfigure} 

Wording has a profound impact on emotional valence, arousal, and dominance in aspects such as information framing, word choice, and language style. Through careful wording design, the text can not only directly influence the audience's emotional responses but also further amplify the emotional experience through the nervous system, sensorimotor system, and cognitive system.
The choice of information framing is particularly important in shaping emotional valence. Gain framing (e.g., "Taking this measure can improve health") typically enhances positive emotions, making the audience feel optimistic and positive; whereas loss framing (e.g., "If no action is taken, health will deteriorate") is more likely to evoke negative emotions such as anxiety and tension \cite{seo2019process}. These changes in emotional valence not only affect the psychological level but may also involve physiological responses, such as physical tension, activating the sensorimotor system \cite{panksepp2012archeology}. Research shows that the arousal effect of negative emotions is more enduring; even after the emotional intensity subsides, negative impressions may persist for a long time; whereas positive emotions tend to fade quickly over time \cite{ludwig2013more}.
% 措辞在信息框架、词汇选择和语言风格等方面对情绪效价、唤起度和支配度有深刻影响。通过恰当的措辞设计,文本不仅能够直接影响受众的情绪反应,还能通过神经系统、感官运动系统和认知系统进一步强化情绪体验。信息框架的选择对情绪效价的塑造尤为重要。增益框架(如“采取这一措施可以改善健康”)通常提升正向情绪,令受众感到积极与乐观;而损失框架(如“如果不采取行动,健康状况将恶化”)则更容易引发焦虑、紧张等负面情绪(Seo & Dillard, 2019)。这种情绪效价的变化不仅停留在心理层面,还可能伴随身体紧张等生理反应,激活感官运动系统(Panksepp & Biven, 2012)。研究表明,负面情绪的唤起效果更持久,即使情绪强度减弱后,负面印象仍可能长期保留;而正面情绪则往往随时间快速衰减(Ludwig et al., 2013)。

The choice of words in phrasing directly affects emotional responses and the motivational system. Positive words (e.g., "happiness," "hope") can evoke positive emotions, making the audience feel pleasant and reassured, while also stimulating exploratory motivation, prompting them to take proactive actions; whereas negative words (e.g., "pain," "loss") intensify emotional tension, eliciting sadness or anxiety, and are more likely to trigger avoidance motivation, leading the audience to prefer risk avoidance or self-protection \cite{lee2020impact, calvo2010affect}. Additionally, mild words (e.g., "challenge," "support") can moderate negative emotional valence to some extent, preventing excessive emotionality; whereas intense words (e.g., "loneliness," "fear") amplify emotional arousal effects, significantly enhancing the emotional impact of the text \cite{nguyen2014affective}. In this process, the audience evaluates these words contextually through their cognitive system, thereby modulating their emotional responses and exhibiting different motivational tendencies in behavioral decision-making \cite{lazarus1991emotion}.
% 措辞中词汇的选择直接影响情绪反应与动机系统。正面词汇(如“幸福”“希望”)能够激发积极情绪,让受众感到愉悦与安心,同时激发探索动机,促使受众采取积极行动;而负面词汇(如“痛苦”“失落”)则增强情绪张力,引发悲伤或紧张情绪,并更容易触发规避动机,使受众倾向于规避风险或保护自身利益(Lee & Potter, 2020;Calvo & D’Mello, 2010)。此外,温和词汇(如“挑战”“支持”)能够在一定程度上缓解负面情绪效价,避免过度情绪化;而强烈词汇(如“孤独”“恐惧”)则通过放大情绪唤起效果,显著增强文本的情感冲击力(Nguyen et al., 2014)。在这一过程中,受众通过认知系统对这些词汇进行情境评估,从而调节自身的情绪反应,并在行为决策中表现出不同的动机倾向(Lazarus, 1991)。

Language style plays a crucial role in enhancing the depth of emotional expression. Parallel structures (such as repetition and symmetrical sentence patterns) can enhance the rhythm and aesthetic appeal of the text, making emotional expression more attractive and impactful, thereby enhancing positive emotional valence and prolonging emotional persistence \cite{menninghaus2017emotional}. Additionally, phonetic characteristics subtly influence emotional arousal. For instance, abrupt phonemes (such as "explosion" or "roar") tend to evoke tension and alertness, whereas smooth phonemes (such as "gentle" or "whisper") are better suited to conveying soothing and calm emotions \cite{slavova2019towards}. These linguistic features further enhance the audience's emotional experience by activating the sensorimotor and nervous systems.
% 语言风格在增强情感表达深度方面具有重要作用。平行结构(如重复和对称句式)能够强化文本的节奏感与美感,使情感表达更具吸引力和感染力,从而提升正向情绪效价,并延长情绪的持久性(Menninghaus et al., 2017)。此外,音素特性对情绪唤起也有潜移默化的影响。例如,突发型音素(如“爆炸”或“轰鸣”)易引发紧张和警觉情绪,而流畅型音素(如“轻柔”或“呢喃”)则更适合传递舒缓与平静的情感(Slavova, 2019)。这些语言特征通过激活感官运动系统和神经系统,进一步强化受众的情感体验。

Notably, the effectiveness of wording is directly related to the degree of alignment with the audience's cultural background in terms of emotional valence and arousal effects. When the language style of the text aligns with the audience's cultural practices and community norms, the audience is more likely to experience emotional resonance, thereby enhancing the text's appeal and credibility; conversely, language expressions that deviate from the cultural context may result in emotional detachment, weakening the text's emotional impact and resonance \cite{ludwig2013more}.
% 值得注意的是,措辞效果与受众文化背景的契合程度直接关系到情绪效价和唤起效果。当文本的语言风格与受众的文化习惯和社群规范相符时,受众更容易产生情感认同,从而增强文本的吸引力和可信度;相反,脱离文化语境的语言表达可能导致情感距离感,削弱文本的情绪感染力和共鸣效果(Ludwig et al., 2013)。

% \subsubsection{Summary}
% Text design demonstrates multidimensional influence in regulating emotional valence, arousal, and sense of dominance. Through the interplay of elements such as titles, narrative structure, narrative content, and wording, text design can flexibly adapt to the emotional needs of different communication contexts, guiding readers to experience target emotions during reading.
% % 文本设计在调节情绪效价、唤起度和支配感方面展现了多维度的影响力,借助标题、叙事结构、叙事内容和措辞等要素的交互作用,文本设计能够灵活适应不同传播情境下的情绪需求,并引导读者在阅读中产生目标情绪体验。

% In scenarios of \textbf{high valence and arousal}, text design often aims to evoke positive emotions and high energy. In the classic novel The Count of Monte Cristo, the climactic moment of the protagonist’s successful revenge portrays an exhilarating victory of justice over evil, and this narrative design effectively enhances emotional valence and arousal levels. Similarly, in inspirational speeches or public service campaigns, titles often employ suspenseful or call-to-action language, such as “Join us to create a better future” or “You too can change the world,” creating an information gap to stimulate readers’ exploratory motivation \cite{blom2015click}. The narrative content focuses on victories in adversity, telling stories of protagonists overcoming challenges and ultimately achieving success, evoking excitement and resonance in readers \cite{leshner2018breast, jaaskelainen2020neural}. In terms of wording, positive and emotionally powerful words such as “breakthrough,” “miracle,” and “achievement” are used, while incorporating a few negative words (e.g., “narrow victory” or “crisis”) to heighten emotional tension and enhance arousal levels \cite{seo2019process}. This design activates readers’ amygdala and reward centers, not only enhancing attention and emotional memory, but also increasing their sense of dominance, making them feel empowered and capable of taking action.
% % 在高情绪效价与高情绪唤起的场景中,文本设计通常以激发积极情绪和高能量为目标。经典小说《基督山伯爵》中,主人公复仇成功的高潮情节,展现了正义战胜邪恶的激动时刻,这种情节设计有效提升了情绪效价和唤起水平。同样地,在鼓舞人心的演讲稿或公益宣传中,标题通常采用悬念或号召性语言,如“加入我们,共创美好未来”或“你也可以改变世界”,通过制造信息缺口激发读者的探索动机(Blom & Hansen, 2015)。叙事内容则聚焦于逆境中的胜利,通过讲述主人公克服困难并最终取得成就的故事,让读者感受到兴奋和共鸣(Leshner et al., 2018;Jääskeläinen et al., 2020)。措辞方面使用积极且感染力强的正面词汇,如“突破”“奇迹”“成就”,同时加入少量负面词汇(如“险胜”或“危机”)增强情感张力,从而提高情绪唤起水平(Seo & Dillard, 2019)。这种设计通过激活读者的杏仁核和奖励中枢,不仅提高了注意力和情绪记忆,还增强了读者的支配感,使其感到自身具有掌控和行动的能力。

% In scenarios of \textbf{high valence and low arousal}, text design focuses on conveying calm positive emotions, avoiding emotional fluctuations. For example, in Pride and Prejudice, the ending where Elizabeth and Darcy overcome misunderstandings and live together, uses a stable narrative structure and gentle language to depict a happy daily life, bringing readers a sense of peaceful satisfaction. Titles are typically simple and clear, such as “A Happy Daily Life Begins Here” or “Experience the Beauty of Life,” avoiding high-arousal rhetoric (Lekkas et al., 2022). The narrative content focuses on depicting small moments of beauty in daily life, using delicate emotional descriptions to create a tranquil atmosphere \cite{lekkas2022using}. Wording includes warm and gentle words such as “warmth,” “comfort,” and “contentment,” avoiding any words that might induce tension or high arousal \cite{lee2020impact}. This design reduces the intensity of emotional arousal but enhances readers’ sense of dominance by creating a sense of affinity, allowing them to gain emotional satisfaction in a stable emotional state.

% %在高情绪效价与低情绪唤起的场景中,文本设计侧重于传递平和的正面情绪,避免引发情绪波动。例如,《傲慢与偏见》中伊丽莎白和达西最终消除误解、共同生活的结局,以稳定的叙事结构和温和的语言描绘幸福的日常,带给读者平静的满足感。标题通常简洁而明确,例如“幸福的日常从这里开始”或“感受生活的美好”,避免使用高唤起的修辞。叙事内容侧重于刻画日常生活中的微小美好,以细腻的情感描写营造宁静氛围(Lekkas et al., 2022)。措辞选用温暖、柔和的词汇,如“温馨”“舒适”“满足”,避免使用任何可能引发紧张或高唤起的词语(Lee & Potter, 2020)。这种设计降低了情绪唤起的强度,但通过营造亲和感提升了读者的支配感,使其在稳定的情绪状态中获得舒适的情感满足。

% In scenarios of \textbf{low valence and high arousal}, text design aims to heighten readers’ tension and focus, often used to convey crisis and unease. For example, in Mary Shelley’s Frankenstein, Victor Frankenstein’s remorse and fear after creating the monster run throughout the main storyline, particularly as the monster’s revenge against humanity unfolds, filling the narrative with tension and unease. Titles of such texts often use loss framing to emphasize potential risks, such as “Ignoring These Signals Could Lead to Irreversible Consequences” or “This Crisis Is Spreading” \cite{seo2019process}. Narrative content depicts crisis scenarios in detail, such as the spread of natural disasters or the worsening of social issues, keeping readers tense and evoking defensive reactions \cite{jaaskelainen2020neural}. The wording heavily employs negative vocabulary (e.g., “danger,” “out of control,” “predicament”), supplemented by abrupt phonemes (e.g., “alarm” or “thunderous collapse”) to intensify emotional arousal \cite{slavova2019towards}. Although this design results in overall low emotional valence, it effectively stimulates avoidance motivation, prompting the audience to take responsive actions.

% % 在低情绪效价与高情绪唤起的场景中,文本设计旨在强化读者的紧张感和注意力,常用于表达危机与不安。例如,在玛丽·雪莱的《弗兰肯斯坦》中,维克多·弗兰肯斯坦在创造出怪物后的悔恨与恐惧贯穿了故事主线,尤其是当怪物对人类的复仇逐步展开时,情节充满了紧张与不安。此类文本的标题通常使用损失框架,突出潜在风险,例如“忽视这些信号可能导致不可挽回的后果”或“这场危机正在蔓延”。叙事内容通过详细描绘危机情境,例如自然灾害的扩散或社会问题的恶化,让读者持续保持紧张感,并激发防御性反应(Jääskeläinen et al., 2020)。措辞中大量使用负面词汇(如“危险”“失控”“困境”),并辅以突发音素的表达(如“警报”或“轰然倒塌”)增强情绪唤起(Slavova, 2019)。这种设计虽然整体情绪效价较低,但能够激发规避动机,使受众采取应对行动。

% In scenarios of \textbf{low valence and arousal}, text design aims to maintain a subdued negative emotion, suitable for portraying inevitable realities. For example, in Tolstoy’s Anna Karenina, a calm tone is used to depict Anna’s lonely state of mind following the breakdown of her marriage, creating a low-arousal negative atmosphere. Titles typically convey negative emotions directly, such as “An Irreversible Regret” or “The Indifferent Truth.” Narrative content presents negative information in a straightforward manner, avoiding any emotional fluctuations, such as describing a regrettable social event or an unavoidable life situation \cite{mar2011emotion}. The wording employs mild negative words such as “regret,” “loss,” and “helplessness,” avoiding strong expressions that could provoke emotional fluctuations \cite{lee2020impact}. This design aims to help the audience understand the existence of the issue, but without using emotional tension to compel immediate action, thereby maintaining a low level of arousal.

% % 在低情绪效价与低情绪唤起的场景中,文本设计以维持低调的负面情绪为目标,适用于呈现无可奈何的现实状况。例如,托尔斯泰的《安娜·卡列尼娜》中,通过平静的语调描写安娜在婚姻破裂后的孤独心境,营造了一种低唤起的消极氛围。标题通常直接传递消极情绪,例如“一个无法改变的遗憾”或“冷漠的真相”。叙事内容以直述方式呈现负面信息,避免引入任何情绪波动,例如描述一件令人遗憾的社会事件或无奈的生活境况(Mar et al., 2011)。措辞中选用平和的负面词汇,如“遗憾”“失落”“无助”,避免使用可能引发情绪波动的强烈表达(Lee & Potter, 2020)。这种设计旨在让受众理解问题的存在,但不通过情绪张力强迫其采取立即行动,从而保持较低的唤起水平。

% By integrating elements of text design with the requirements of different emotional combinations, the balance of emotional valence, arousal, and dominance can be effectively regulated. This approach not only accommodates diverse communication needs, but also provides clear guidance for design practices.
% % 通过将文本设计的要素与不同情绪组合的需求相结合,可以有效调节情绪效价、唤起度和支配感的平衡。这种方法不仅能够适应多样化的传播需求,还为设计实践提供了清晰的指导方向。

\subsection{Visual Design}
Visual design is one of the most direct and rapid means of influencing user emotions. It can be broken down into elements such as color, imagery, shape, and layout. The choice of color directly impacts emotional valence and arousal. For instance, warm colors like red and orange often evoke positive emotions, while cool colors like blue and green help create a calm and relaxing atmosphere \cite{plass2014emotional}. Images can significantly influence viewers’ cognitive processes and memory by conveying explicit emotional information. The design of shapes and layouts also plays a critical role. Rounded and soft shapes are often associated with warm and friendly emotional experiences, whereas sharp geometric shapes may evoke tension or alertness \cite{mayer2014benefits}. By combining visual elements effectively, designers can swiftly regulate users’ emotions at the visual~level.
% 视觉设计是影响用户情绪最直观和快速的手段之一。细分为颜色、图片、形状和布局等元素。颜色的选择能够直接影响情感效价和唤起度,例如,暖色系如红色和橙色通常能引发积极情绪,而冷色系如蓝色和绿色则有助于营造平静和放松的氛围【Plass et al., 2014】。图片能够通过传递明确的情感信息显著影响观众的认知过程和记忆效果。形状和布局的设计也起到关键作用。圆润和柔和的形状往往与温暖、友好的情感体验相关,而尖锐的几何形状则可能激发紧张或警觉感【Mayer & Estrella, 2014】。通过合理的视觉元素组合,设计师能够在视觉层面快速调节用户的情绪。

\renewcommand{\arraystretch}{1.8} % 调整行间距
\begin{table*}[ht]
\fontsize{8}{9}\selectfont
\centering

\begin{tabularx}{\textwidth}{|>{\centering\arraybackslash}m{0.5cm}|>{\centering\arraybackslash}m{1.4cm}|>{\centering\arraybackslash}m{3.4cm}|>{\centering\arraybackslash}m{3.4cm}|>{\centering\arraybackslash}m{3.4cm}|>{\centering\arraybackslash}m{3.4cm}|}
\hline
\rowcolor[HTML]{D9EAD3} 
\multicolumn{2}{|c|}{\textbf{Dimension}} & \textbf{Color} & \textbf{Shape} & \textbf{Images} & \textbf{Layout} \\ \hline

% Emotional Dimensions Section
\multirow{3}{*}{\rotatebox{90}{\parbox{3cm}{\centering \textbf{Emotional \\ Dimensions}}}} & 
\cellcolor[HTML]{FDF6E8} \textbf{Valence} & 
\textbf{Warm colors} (e.g., red, yellow) evoke positive emotions and energy \cite{jonauskaite2019color,kallabis2024investigating}. \textbf{Cool colors} (e.g., blue, green) promote calmness \cite{wilms2018color}. & 
\textbf{Round shapes} convey friendliness and warmth \cite{wei2006image}. \textbf{Sharp shapes} evoke alertness \cite{thumfart2008modeling}. & 
\textbf{Positive images} (e.g., nature, smiling faces) enhance positive emotions \cite{hou2024emotional}. \textbf{Negative images} (e.g., disasters) amplify negative emotions \cite{pfeuffer122measuring}. & 
\textbf{Simple layouts} reduce distractions, enhancing positive emotions \cite{lu2017investigation}. \textbf{Symmetrical layouts} evoke balance and trust \cite{fiorini2024role}. \\ \cline{2-6}

& \cellcolor[HTML]{FDF6E8} \textbf{Arousal} & 
\textbf{Warm colors} increase arousal (excitement), while \textbf{cool colors} reduce it (relaxation) \cite{jonauskaite2019color}. & 
\textbf{Complex shapes} heighten arousal \cite{lu2012shape}. 
\textbf{Rounded shapes} are calming \cite{etzi2016arousing}. & 
\textbf{High-intensity images} (e.g., emergencies) increase arousal \cite{pfeuffer122measuring}. \textbf{Peaceful images} reduce arousal \cite{hao2024judging}. & 
\textbf{Complex layouts} boost exploration and arousal \cite{carretie2019emomadrid}. \textbf{Dynamic layouts} add vitality \cite{lu2020exploring}. \\ \cline{2-6}

& \cellcolor[HTML]{FDF6E8} \textbf{Dominance} & 
\textbf{Warm, bright colors} (e.g., light yellow) enhance control, while \textbf{low-brightness cool} colors reduce it \cite{weijs2023effects}. & 
\textbf{Circular shapes} enhance control and safety \cite{lu2012shape}. \textbf{Sharp shapes} decrease control \cite{wei2006image}. & 
\textbf{Wide images} (e.g., 16:9 ratio) enhance control \cite{kuzinas2024creative}. \textbf{Isolated subjects} reduce control \cite{lin2023effect}. & 
\textbf{Rule-of-thirds layouts} enhance control \cite{machajdik2010affective}. 
\textbf{Shallow depth layouts} focus attention and increase control \cite{datta2006studying}. \\ \hline

% Multisystem Activation Section
\multirow{4}{*}{\rotatebox{90}{\parbox{3cm}{\centering \textbf{Multisystem \\ Activation}}}} & 
\cellcolor[HTML]{F0EFF7} \textbf{Neural Systems} & 
\textbf{Colors} activate the sympathetic or parasympathetic systems \cite{ledoux1998emotional}. & 
\textbf{Rounded shapes} relax the nervous system, while \textbf{sharp shapes} induce alertness  \cite{ledoux1998emotional}. & 
\textbf{Emotional images }activate the amygdala, affecting emotions \cite{phelps1998specifying}. & 
\textbf{Simple layouts} induce calm; \textbf{complex layouts} activate stress responses \cite{ledoux1998emotional}. \\ \cline{2-6}

&\cellcolor[HTML]{F0EFF7} \textbf{Sensorimotor Systems} & 
\textbf{Color contrast }affects sensory responses (e.g., pupil dilation, muscle tension) \cite{zajonc1980feeling}. & 
\textbf{Rounded shapes} feel safe; \textbf{sharp shapes} induce tension \cite{lidwell2010universal}. & 
\textbf{Emotional images} trigger physical reactions (e.g., smiling or frowning)\cite{ekman1992argument}. & 
\textbf{Symmetrical layouts} evoke harmony; \textbf{asymmetrical} ones may cause unease \cite{machajdik2010affective}. \\ \cline{2-6}

& \cellcolor[HTML]{F0EFF7} \textbf{Cognitive Systems} & 
\textbf{Colors} influence cognitive assessment (e.g., red for danger, green for relaxation) \cite{zajonc1980feeling}. & 
Shapes impact safety perception (e.g., circles feel safe, sharp shapes signal danger) \cite{lidwell2010universal}. & 
Emotional expressions in images trigger empathy and cognitive responses \cite{hou2024emotional}. & 
\textbf{Symmetrical layouts} provide cognitive ease; \textbf{asymmetrical} ones may confuse \cite{carretie2019emomadrid}. \\ \cline{2-6}

&\cellcolor[HTML]{F0EFF7} \textbf{Motivational Systems} & 
\textbf{Warm colors} (e.g., red, yellow) stimulate urgency and action, while \textbf{cool colors} promote comfort and trust \cite{phelps1998specifying}. & 
\textbf{Rounded shapes }evoke trust; \textbf{sharp shapes} stimulate urgency or challenge \cite{lidwell2010universal}. & 
\textbf{Achievement-oriented images} (e.g., celebrations) inspire motivation \cite{pfeuffer122measuring}. & 
\textbf{Simple layouts} enhance task motivation;\textbf{ complex layouts} increase cognitive load \cite{lu2020exploring}. \\ \hline
\end{tabularx}

\caption{Emotional Dimensions and Multisystem Activation in Visual Design Elements} 
\label{tab:visual_design}
\end{table*}

% \begin{figure*}[hbt!]
% %\setlength{\abovecaptionskip}{-0.1mm}
% \setlength{\intextsep}{10pt plus 2pt minus 2pt}
%     \centering
%     \includegraphics[width=18cm]{figs/Visual_design.png}
%     \caption{Emotional Dimensions and Multisystem Activation in Visual Design Elements.}
%    \vspace{-2mm}
% \label{fig:why}
% \end{figure*}

\subsubsection{Color}
\begin{wrapfigure}{l}{0.06\textwidth}
  %\begin{center}
  \vspace{-11pt} % 调整垂直位置
        \includegraphics[width=0.07\textwidth]{figs/icon/color.png}
  %\end{center}
\end{wrapfigure} 

Color plays a critical role in visual design, with its impact on user emotions primarily reflected in three dimensions: hue, saturation, and brightness. Hue, as the fundamental attribute of color, corresponds to different emotional effects. Studies show that warm colors (e.g., red, orange), due to their high emotional valence and arousal, are commonly used in advertising and entertainment design to convey positive emotions such as joy and excitement \cite{jonauskaite2019color, kallabis2024investigating}. Cool colors (e.g., blue, green) tend to convey calm and comfortable emotions, making them suitable for settings such as healthcare and meditation that aim to soothe emotions \cite{ wilms2018color}. Long-wavelength colors (e.g., red) often enhance emotional arousal, while short-wavelength colors (e.g., blue) are better suited for conveying low-arousal emotional experiences~\cite{wilms2018color}.
Hue not only influences emotional valence but also shapes users’ emotional experiences and behavioral responses through the cognitive and motivational systems. Warm colors (e.g., red, orange), associated with a sense of urgency, are often used in warning signs or time-sensitive designs to quickly capture attention and elicit user alertness and action tendencies \cite{zajonc1980feeling, phelps1998specifying}. Cool colors (e.g., blue, green), associated with comfort and safety, are suitable for conveying trust and relaxation, such as in healthcare or financial interface design \cite{phelps1998specifying}. %This emotional association with hue not only enhances the emotional conveyance of the design but also directly impacts users’ emotional evaluation of the context.
% 颜色在视觉设计中扮演着重要角色,其对用户情绪的影响主要体现在色相、饱和度和亮度三个维度上。色相是颜色的基本属性,不同的色相对应不同的情感效应。研究表明,暖色系(如红色、橙色)因其高情绪效价和唤起度,常用于广告和娱乐设计以传递愉悦、兴奋等正向情绪(Jonauskaite et al., 2019;Kallabis et al., 2024)。冷色系(如蓝色、绿色)则倾向于传递平静与舒适的情绪,适合用于医疗、冥想等安抚情绪的场景(Beekmans & Braun, 2024,Wilms, L., & Oberfeld, D. (2018))。长波长颜色(如红色)常增强情绪唤起,而短波长颜色(如蓝色)更适合传递低唤起的情绪体验(Wilms, L., & Oberfeld, D. (2018))。色相不仅影响情绪效价,还通过认知系统和动机系统塑造用户的情绪体验和行为反应。暖色系(如红、橙)因其与紧迫感的联想,常用于警示标志或强调时效性的设计,能够迅速吸引注意力并激发用户的警觉和行动倾向(Zajonc, 1980;Phelps et al., 1998)。冷色系(如蓝、绿)则以其与舒适和安全感的关联,适合用于传递信任与放松的场景,如医疗或金融界面设计(Phelps et al., 1998)。这种色相的情感联结不仅强化了设计的情感传递效果,还直接影响用户对情境的情感评估。

Saturation describes the purity and intensity of color and significantly affects emotional valence and arousal. Highly saturated colors (e.g., vivid red, bright blue) have striking visual effects that enhance users’ emotional valence and arousal, making them suitable for attention-grabbing design scenarios \cite{lin2023effect}. For example, using highly saturated colors in advertisements or promotional pages can quickly capture users’ attention and stimulate the sensory-motor system, eliciting immediate responses such as pupil dilation or visual focus \cite{zajonc1980feeling}. In contrast, low-saturation colors (e.g., soft bluish-gray) are more inclined to convey calm and serene emotional experiences \cite{pazda2024colorfulness}. They are suitable for interfaces intended for prolonged use, such as reading platforms or educational interfaces \cite{wang2013interpretable}.
% 饱和度描述颜色的纯度和强度,对情绪效价和唤起度有显著影响。高饱和度的颜色(如鲜红、亮蓝)视觉效果鲜明,能够提升用户的情绪效价和唤起度,适用于吸引注意力的设计场景(Lin et al., 2023)。例如,在广告或宣传页面中使用高饱和度的颜色可以迅速吸引用户视线,并刺激感官运动系统产生即时反应,如瞳孔扩张或视觉聚焦(Zajonc, 1980)。相比之下,低饱和度颜色(如柔和的蓝灰色)更倾向于传递平静、冷静的情绪体验(Pazda et al., 2024),适合长时间使用的界面设计,如阅读界面或教育平台([wang et al, 2013])。

Brightness significantly influences the perception of emotional positivity or negativity. High-brightness colors (e.g., light yellow) typically make emotional valence more positive and are used to create a relaxed atmosphere \cite{jonauskaite2019color}. For example, in children’s education or recreational settings, light yellow can stimulate users’ sense of pleasure through the nervous system, making them feel relaxed and at ease \cite{ledoux2000emotion}. Low-brightness tones (e.g., dark gray, deep red) are often used to convey feelings of oppression or tension. For example, in suspense films or horror games, these tones activate the sympathetic nervous system, eliciting tension and high arousal~\cite{weijs2023effects}.
% 亮度显著影响情绪的正负向感知。高亮度颜色(如浅黄色)通常使情绪效价更积极,用于营造轻松氛围(Jonauskaite et al., 2019);例如,在儿童教育或休闲场景中,浅黄色能够通过神经系统激发用户的愉悦感,使其感到轻松和放松(LeDoux, 1996)。低亮度色调(如深灰、暗红)则常用于传递压抑或紧张情绪的场景,例如悬疑电影或恐怖游戏,通过激活交感神经系统引发紧张感和高唤起水平(Weijs et al., 2023)。

By comprehensively regulating hue, saturation, and brightness, designers can precisely adjust users’ emotional valence and arousal levels. Moreover, the physiological and cognitive effects of color, mediated by the sensory-motor and nervous systems, further drive users to take specific actions under the influence of the motivational system. This systematic approach to color design helps optimize user experience, aligning it more closely with design objectives.
% 通过对色相、饱和度和亮度的综合调控,设计师能够精准调节用户的情绪效价和唤起水平。此外,颜色的生理与认知影响通过感官运动系统和神经系统的介导作用,在动机系统的驱动下,进一步促使用户采取特定的行为。这种系统性的颜色设计有助于优化用户体验,使其更加符合设计目标。


\subsubsection{Image}
\begin{wrapfigure}{l}{0.06\textwidth}
  %\begin{center}
  \vspace{-11pt} % 调整垂直位置
        \includegraphics[width=0.07\textwidth]{figs/icon/image.png}
  %\end{center}
\end{wrapfigure} 
Images in visual design not only convey emotional information but also influence users’ cognition and memory \cite{hanson2014happy, xie2017negative, van2015good}. Designers regulate users’ emotional experiences and evoke multi-layered emotional responses through image content, social cues, and presentation methods.
% 图片在视觉设计中不仅传递情感信息,还影响用户的认知和记忆效果(Hanson & II, 2014; Xie & Zhang, 2017;Van Bergen et al., 2015)。设计师通过图像内容、社交线索和呈现方式,调控用户的情绪体验并激发多层次的情绪反应。

The content of an image determines the direction of emotional conveyance. Positive emotional images (e.g., natural landscapes, smiling faces) enhance users’ feelings of joy and relaxation. Commonly used in advertising and educational contexts, these images strengthen the valence of positive emotions through emotional connections \cite{hou2024emotional}. Highly arousing negative emotional images (e.g., disaster scenes, horror visuals) evoke users’ sense of alertness, increasing their focus on critical information, making them suitable for public service campaigns or news reporting \cite{pfeuffer122measuring}. Challenging images (e.g., extreme sports) stimulate physical tension and the motivational system for risk-taking, while tranquil natural landscapes are more suited to evoke relaxation and calm emotional states \cite{pfeuffer122measuring}.
These images not only trigger immediate physiological responses (e.g., increased heart rate or relaxation) through the sensory-motor system but also activate the nervous and cognitive systems, further enhancing users’ emotional evaluation and behavioral responses to the context \cite{phelps1998specifying, kensinger2007negative}.
% 图像内容决定了情感传递的方向。正面情绪图片(如自然风景、微笑人物)能够提升用户的愉悦与放松感。常见于广告和教育场景,其情感联结增强了积极情绪的效价(Hou & Wang, 2024)。高唤起的负面情绪图片(如灾害场景、恐怖画面)通过激发用户的警觉感,增加其对关键信息的关注度,适用于公益宣传或新闻报道(Pfeuffer et al., 2024)。挑战性图片(如极限运动)能够引发身体紧张和冒险的动机系统,而轻松的自然风景图像则更适合激发放松与平静的情感状态(Pfeuffer et al., 2024)。这些图像不仅通过感官运动系统引发即时的生理反应(如心跳加速或放松),还激活神经系统和认知系统,进一步强化用户对情境的情感评估和行为反应(Ekman, 1971;Phelps et al., 1998,Kensinger(2007))。

Social cues play an important regulatory role in the emotional transmission of images. Facial expressions, body language, and interactive contexts in images can significantly enhance users’ emotional resonance. For example, positive social cues such as smiling faces can partially mitigate users’ negative emotional responses even in adverse contexts, thereby enhancing emotional valence \cite{dudarev2024social}. In group images, the overall emotional valence often outweighs the influence of individual emotions. For instance, images where most group members display smiling expressions typically elicit stronger positive emotional reactions, while the influence of a single individual showing extreme negative emotions is relatively limited \cite{hao2024judging}. Facial expressions in images not only trigger users’ emotional evaluations of the context (e.g., sympathy or anger) through the cognitive system but also activate the motivational system. For instance, collective emotions in celebratory scenes can inspire users’ motivation to engage in social interactions \cite{hou2024emotional}. The use of such social cues makes images a powerful tool for enhancing emotional valence and the dominance.
% 社交线索在图片中的情感传递中起到重要调节作用。图片中的人物面孔、肢体语言和互动情境能够显著提升用户的情感共鸣。例如,微笑面孔等正面社交线索,即使在负面情境下也能够部分缓解用户的消极情绪反应,从而提高情绪效价【Dudarev et al., 2024】。在群体图像中,整体的情绪效价往往超越单个个体的情绪影响。例如,多数群体成员展现微笑表情的图像,通常会激发更高的积极情绪反应,而单个表现出极端负面情绪的个体影响相对有限(Hou & Wang, 2024)。图片中的面部表情不仅通过认知系统引发用户对情境的情感评估(如同情或愤怒),还能够激发动机系统,例如庆祝场景的集体情绪能够激发用户参与社交或互动的动机(Hou & Wang, 2024)。这种社交线索的使用,使图片成为强化情绪效价与支配感的有力工具。

Visual presentation methods further amplify the emotional impact of images. Research has shown that wide-format images (e.g., 16:9 aspect ratio) enhance users’ dominance and visual pleasure through their familiarity \cite{kuzinas2024creative}. Moreover, by adjusting the composition ratio and visual focus of an image, designers can highlight key information and reduce visual distractions. For instance, images with shallow depth of field emphasize the subject by blurring the background, which helps reduce visual distractions, enhances users’ focus on key information, and stimulates the cognitive system for faster information processing~\cite{datta2006studying}.
% % 视觉呈现方式进一步增强了图片的情感效果。研究发现,16:9 的纵横比因其广泛应用于现代媒体而成为用户的视觉惯性选择,这种比例的图片更易激发用户的熟悉感和认同感,从而提升情绪效价【Kuzinas et al., 2024】。此外,通过调整图像的构图比例和视觉焦点,设计师可以突出关键信息并减少视觉干扰。例如,低景深的图像通过模糊背景突出主体,帮助用户专注于关键信息,同时营造一种沉浸和平静的氛围。

By integrating content, social cues, and visual presentation methods, image design can precisely regulate users’ emotional valence, arousal, and dominance. Through multi-layered emotional systems, it can also stimulate specific behavioral motivations, playing a critical role in achieving design objectives.
% 通过内容、社交线索和视觉呈现方式的综合运用,图片设计能够精准调节用户的情绪效价、唤起度和支配感,并通过多层次的情绪系统激发特定的行为动机,使其在设计目标中发挥关键作用。

\subsubsection{Shape}
\begin{wrapfigure}{l}{0.06\textwidth}
  %\begin{center}
  \vspace{-11pt} % 调整垂直位置
        \includegraphics[width=0.07\textwidth]{figs/icon/shape.png}
  %\end{center}
\end{wrapfigure} 
Shape is a fundamental element in visual design, profoundly influencing users’ emotional valence, arousal, and dominance through its geometric characteristics.
% 形状是视觉设计中的核心元素,通过其几何特性对用户的情绪效价、唤起度和支配感产生深远影响。

The geometric properties of shapes play a critical role in emotional conveyance. Circular shapes and soft curves often convey feelings of friendliness and safety, enhancing positive valence and comfort \cite{lu2012shape}. For example, circular buttons and icons with soft edges in children’s education or healthcare interfaces can reduce anxiety and enhance trust through the nervous system \cite{ledoux2000emotion}. In contrast, sharp shapes, with their angular features, convey alertness and tension, amplifying negative valence and eliciting high emotional arousal. They are suitable for traffic warning signs or designs that require attention \cite{thumfart2008modeling}. Complex or irregular shapes enhance tension and psychological conflict through their instability, commonly used in thriller movie promotions or suspenseful content design. Their intense visual stimulation directly activates the sympathetic nervous system, inducing physiological tension responses \cite{ebe2015emotion}.
%  形状的几何特性对情感传递起到关键作用。圆形和柔和的曲线通常传递、友好与安全的情感,能够增强正效价和舒适感(Lu et al., 2012)。例如,儿童教育或医疗界面中的圆形按钮和柔和边缘的图标,能够通过神经系统降低焦虑感并增强信任感(LeDoux, 1996)。相比之下,尖锐形状[2]以其棱角分明的特性传递警觉性和紧张感,强化负面效价并引发较高的情绪唤起,适用于交通警示标志或需要吸引注意力的设计场景(Thumfart et al., 2008)。复杂或不规则形状则通过其不稳定性增强紧张感和心理冲突,在惊悚电影宣传或悬疑内容设计中尤为常见,其强烈的视觉刺激直接激活交感神经系统,引发生理紧张反应(Ebe & Umemuro, 2015)。

The combination of shapes and emotional symbols further enriches the emotional experience. Studies have shown that anthropomorphic circles (e.g., smiley icons) can quickly evoke users’ positive emotions and enhance interactive experiences, significantly boosting user engagement through positive emotional valence \cite{mayer2014benefits}. Conversely, frowning faces or sharp-shaped symbols are often used to convey risk or warnings, triggering users’ defensive mechanisms and increasing their attention to the information \cite{ferrara2015measuring}. The use of such symbols can influence users’ emotional evaluations of the context through the cognitive system while activating the motivational system to prompt specific behaviors.
% 形状与情感符号的结合进一步丰富了情绪体验。研究表明,拟人化的圆形(如微笑图标)能够迅速激发用户的积极情绪并增强互动体验,其正向情绪效价显著提升了用户的参与感(Mayer & Estrella, 2014)。相反,哭脸或尖锐的形状符号则常用于传递风险或警告,触发用户的防御机制并增强对信息的关注度(Ferrara & Yang, 2015)。这种符号的使用能够通过认知系统影响用户对情境的情感评估,同时激活动机系统,促使用户采取特定的行为。

% In contrast, sharp shapes with angular features convey alertness and tension, amplifying visual stimuli to elicit heightened emotional arousal. They are suited for scenarios requiring attention or communicating risk. For instance, traffic warning signs and safety labels trigger defensive responses rapidly \cite{thumfart2008modeling}. Complex shapes also play a distinct role in emotional regulation. Simple shapes convey reliability and low-arousal emotions, while complex or irregular shapes (e.g., polygons, dynamic patterns) evoke tension and unease. These visual characteristics are effective in thriller promotions or suspenseful content, simulating unstable effects to heighten tension \cite{ebe2015emotion}. By carefully selecting and combining geometric properties, designers can modulate emotional experiences to suit various design needs.
% %与此相对,尖锐形状以其棱角分明的特性传递警觉性和紧张感,通过增强视觉刺激,引发较高的情绪唤起,适用于需要吸引注意力或传递风险的场景,如交通警示标志和安全标识,这些设计能够迅速激发用户的防御反应【Thumfart et al., 2008】。此外,复杂形状也在情绪调节中发挥独特作用。简单规则的形状传递可靠性和低唤起情绪,而复杂或不规则形状(如多边形或动态图案)则容易引发紧张与不安情绪。这类视觉特性在惊悚电影的宣传设计或悬疑内容的表达中尤为有效,通过模拟不稳定的视觉效果增强用户的心理紧张感【Ebe & Umemuro, 2015】。通过合理选择和搭配形状的几何特性,设计师能够有效调控用户的情绪体验,以适应不同设计场景的需求。

% Moreover, integrating shapes with emotional symbols can further enhance emotional experiences. Studies indicate that endowing shapes with anthropomorphic traits (e.g., smiling circular icons) can quickly evoke positive emotions in users and increase their affinity for products or information. For example, smiley emojis on social media enhance users’ interaction rates by leveraging their positive emotional valence \cite{mayer2014benefits}. In contrast, negative symbols (e.g., crying faces) are employed to convey negative information, directing users’ attention to potential problems or risks.
% % 此外,将形状与情感符号结合能够进一步增强情感体验。研究表明,赋予形状拟人化特征(如微笑的圆形图标)可以快速激发用户的积极情绪,增加对产品或信息的好感。例如,社交媒体中的笑脸表情符号以其正向情绪效价提升了用户的互动率【Mayer & Estrella, 2014】,而负面符号(如哭脸)则用于传达消极信息,引导用户关注潜在问题或风险。

By effectively utilizing the geometric properties of shapes and emotional symbols, designers can holistically regulate users’ emotional valence, arousal, and dominance. This multi-layered design approach not only enhances users’ emotional experiences through feedback from the sensory-motor and nervous systems but also guides users’ behavioral choices via the cognitive and motivational systems. This ensures that the design better aligns with the requirements of its application~context.
% 通过合理运用形状的几何特性和情感符号,设计师能够综合调控用户的情绪效价、唤起度和支配感。这种多层次的设计方法不仅通过感官运动系统和神经系统的反馈增强用户的情绪体验,还通过认知系统和动机系统引导用户的行为选择,使设计更加符合应用场景的需求。

\subsubsection{Layout}
\begin{wrapfigure}{l}{0.06\textwidth}
  %\begin{center}
  \vspace{-11pt} % 调整垂直位置
        \includegraphics[width=0.07\textwidth]{figs/icon/layout.png}
  %\end{center}
\end{wrapfigure} 
Layout directly influences users’ emotional experiences in visual design through spatial arrangement and element organization. Key features such as symmetry, complexity, composition, and depth of field significantly regulate emotional valence and arousal~levels.
% 布局在视觉设计中通过空间安排和元素组织直接影响用户的情绪体验,其对称性、复杂性、构图方式、景深等关键特性能够显著调控情绪效价和情绪唤起水平。

Symmetrical layouts, with their sense of balance and order, reduce users’ cognitive load and create a stable and comfortable emotional experience \cite{makin2012implicit}. This design induces physical comfort responses, such as relaxed eye movements, through the sensory-motor system, while also generating physiological calming effects via the nervous system \cite{ledoux2000emotion}. The visual stability of symmetrical layouts is commonly used in financial or healthcare settings to enhance users’ sense of trust and security. In contrast, asymmetrical layouts, while increasing visual dynamism, may evoke slight feelings of unease, making them suitable for creative settings that aim to stimulate exploratory motivation \cite{carretie2019emomadrid}.
% 对称布局以其均衡感和有序性,降低了用户的认知负荷,营造出稳定与舒适的情绪体验。这种设计通过感官运动系统引发身体上的舒适反应,例如放松的眼部活动,同时通过神经系统产生生理上的平静效果(LeDoux, 1996)。对称布局的视觉稳定性常用于金融或医疗场景,以增强用户的信任感和安全感。相比之下,不对称布局虽然增加了视觉动态感,但可能引发轻微的不安情绪,适合用于创意性场景,激发探索动机。

Complex layouts, with diverse visual elements and rich layering, stimulate users’ curiosity and emotional engagement. This type of layout is commonly used in social media, entertainment apps, or creative display platforms, where intensified visual stimuli enhance emotional arousal levels. However, overly complex layouts may lead to information overload, diminishing users’ dominance and even inducing negative emotions \cite{carretie2019emomadrid}. Thus, complex layouts require a balance between richness and information clarity to avoid diminishing user experience. On the neurological level, complex layouts trigger stress responses, thereby enhancing users’ physiological alertness and focus.
% 复杂布局通过多样化的视觉元素和丰富的层次感,激发用户的探索欲望和情感参与感。这种布局常用于社交媒体、娱乐应用或创意展示平台,能够通过强化视觉刺激提升情绪唤起水平【Carretié et al., 2019】。然而,过度复杂的布局可能导致信息过载,削弱用户的支配感,甚至引发负面情绪。因此,复杂布局需要在丰富性与信息清晰性之间找到平衡,以避免用户体验的下降。在神经系统的层面上,复杂布局会触发应激反应,从而提升用户的生理警觉性和注意力集中程度。

Composition directly influences users’ emotional valence and dominance. The rule of thirds layout, by evenly dividing the frame, creates a harmonious visual effect, enhancing users’ comfort and positive emotional experiences. This composition helps users quickly focus on key information, improving the efficiency of information delivery \cite{mai2011rule}. Circular or curved layouts leverage visual enclosure and flow to capture users’ attention, making them suitable for branding and interactive designs. They enhance emotional engagement to increase user involvement \cite{resnick2003design}. For instance, slanted lines or curved guiding lines enhance the dynamism of the layout while increasing the appeal of key content.
% 构图方式直接影响用户的情绪效价与支配感。三分法布局通过将画面均匀分割,同样创造出了和谐的视觉效果,提升了用户的舒适感与积极情绪体验。这种构图在内容呈现中帮助用户快速聚焦关键信息,提高了信息传递的效率【Mai et al., 2011】。而圆形或弧形布局则利用视觉上的封闭性与流动性吸引用户注意,适用于品牌展示与互动设计,通过增强情感参与感来提高用户的投入度【Resnick, 2003】。例如,倾斜线条或曲线引导线能够提升布局的动态感,同时强化关键内容的吸引力。

Depth of field is an important method in layout design for emphasizing core information and reducing visual distractions. A shallow depth-of-field layout blurs the background and focuses on the subject, reducing interference from the sensory-motor system and enhancing users’ attention \cite{datta2006studying}. For example, shallow depth-of-field image designs on product pages can help users quickly identify core content while activating the nervous system to enhance information retention. Furthermore, this layout directs visual focus to trigger users’ execution motivation, making it suitable for action-oriented design scenarios.
% 景深在布局设计中通过模糊背景并突出主要对象,有助于减少视觉干扰,增强用户对关键信息的专注感。这种设计常见于强调核心内容的场景,如广告页面或产品展示,既提升了用户的理解效率,又增强了信息记忆【Datta et al., 2006】。

Layout design can achieve a dynamic balance among emotional valence, arousal, and dominance by adjusting symmetry, complexity, composition, and depth of field. This multidimensional design approach activates sensory-motor, nervous, cognitive, and motivational systems, satisfying users' emotional needs and driving behavioral responses.	
% 通过对对称性、复杂性、构图方式和景深的综合调控,布局设计能够在情绪效价、唤起度和支配感之间实现动态平衡。这种多维度的设计方法通过激活感官运动系统、神经系统、认知系统和动机系统,在满足用户情感需求的同时,推动用户的行为响应。

% \subsubsection{Summary}
% Visual design profoundly influences users’ emotional valence, arousal level, and sense of control through the integrated use of color, images, shapes, and layouts. In different contexts, design strategies need to be refined based on the target emotional state to achieve precise emotional delivery and effectively guide user behavior.
% % 视觉设计通过颜色、图片、形状和布局的综合运用,对用户的情绪效价、唤起度和支配感产生深远影响。在不同情境中,设计策略需根据目标情绪状态进行精细化调整,从而实现情感的精准传递与用户行为的有效引导。

% In scenarios of \textbf{high valence and high arousal}, visual design leverages the synergy of color, imagery, shapes, and layout to evoke positive emotions and high energy. For instance, in Delacroix’s iconic painting Liberty Leading the People, the vibrant red and golden yellow high-saturation warm tones significantly enhance emotional valence and arousal. These color choices intensify the scene’s tension and emotional engagement, allowing viewers to feel the passion of revolution and the hope of victory \cite{jonauskaite2019color, kallabis2024investigating}. The figure of “Liberty” in the painting, with her dynamic pose of raising a flag and expressive facial features, symbolizes courage and action. This design not only activates the viewer’s sensory-motor system but also strengthens emotional alignment with victory and justice through the cognitive system \cite{kensinger2007negative, hou2024emotional}. Furthermore, the sharp shapes and diagonal layout amplify the painting’s visual impact, with dynamic guidance and a sense of depth directing the viewer’s gaze to the focal area, thereby enhancing attention and emotional memory \cite{wei2006image}. This multi-layered visual design effectively conveys intense emotions and profound meaning by eliciting emotional resonance and cognitive reactions from the audience.
% % 在高情绪效价与高情绪唤起的场景中,视觉设计通过色彩、图像内容、形状和布局的协同作用来激发积极情绪与高能量。例如,德拉克洛瓦的经典画作《自由引导人民》中,鲜艳的红色和金黄色的高饱和度暖色调显著提升了情绪效价和唤起水平,这些色彩选择强化了场景的张力和观众的情感参与,使其感受到革命的激情与胜利的希望(Jonauskaite et al., 2019;Kallabis et al., 2024)。画面中的“自由女神”以高举旗帜的动态姿态和鲜明表情传递出勇气与行动的象征意义,这种设计不仅激发了观众的感官运动系统,还通过认知系统强化了对胜利和正义的情感认同(Kensinger, 2007;Hou & Wang, 2024)。此外,尖锐形状和对角线布局进一步增强了画面的视觉冲击力,动态引导和层次感使观众的视线集中在核心区域,强化了注意力与情感记忆({Wei-Ning et al., 2006,Image retrieval by emotional semantics: A study of emotional space and feature extraction)。这种多层次的视觉设计通过激发观众的情感共鸣与认知反应,有效传递了强烈的情感与深刻的意义。

% In scenarios of \textbf{high valence and low arousal}, visual design employs soft colors, serene imagery, rounded shapes, and symmetrical layouts to create a warm and tranquil emotional atmosphere, helping viewers or users achieve a relaxed emotional state. For instance, John Constable’s The Hay Wain features soft greens and blues as the primary hues, complemented by low-saturation and medium-to-high brightness warm tones, creating a calm and harmonious visual experience \cite{wilms2018color}. The layout of the painting forms a stable composition through the balanced distribution of the river, trees, and houses, reducing visual fatigue for viewers and enhancing the smoothness of the viewing experience \cite{lu2020exploring}. Rounded shapes and subtle gradations of light and shadow further diminish visual intensity; for example, the gentle curves of trees and grass naturally alleviate any tension in the scene \cite{lu2012shape}. These design elements not only regulate viewers’ neural system responses through low-arousal visual characteristics but also enhance peace and pleasure by reducing cognitive load and visual conflict. Additionally, this design strategy is frequently applied in medical interfaces and educational platforms. For example, using low-saturation cool tones such as blue and green with symmetrical layouts can provide users with a sense of trust and security, while rounded buttons and gentle curved icons further enhance comfort \cite{ledoux2000emotion}. In summary, through meticulous design of color, shape, and layout, visual design can achieve emotional goals of high valence in low-arousal states.
% % 在高情绪效价与低情绪唤起的场景中,视觉设计通过柔和的色彩、平静的图像内容、圆润的形状和对称布局,营造出温暖与平和的情感氛围,帮助观众或用户进入放松的情绪状态。例如,约翰·康斯特布尔的《干草车》以柔和的绿色和蓝色为主色调,辅以低饱和度和中高亮度的暖色系,营造出安静和谐的视觉体验(Wilms & Oberfeld, 2018)。画面的布局通过河流、树林和房屋的合理分布形成了稳定的构图,减少了观众的视觉疲劳,同时提升了观看过程的流畅感(lu et al., 2020)。圆润的形状和细腻的明暗过渡进一步降低了视觉刺激强度,例如树木和草地的柔和曲线自然消解了画面中的任何紧张元素(Lu et al., 2012)。这些设计元素不仅通过低唤起的视觉特性调节观众的神经系统反应,还通过降低认知负荷和视觉冲突增强了平和与愉悦感。此外,这种设计策略在实践中常用于医疗界面与教育平台。例如,使用冷色系如蓝色与绿色的低饱和度色调和对称布局,能够为用户提供信任感与安全感,同时通过圆润的按钮和柔和曲线图标进一步优化舒适体验(LeDoux, 1996)。综合来看,通过对色彩、形状和布局的精细设计,视觉设计能够在低唤起状态下实现高情绪效价的情感目标。

% In scenarios of \textbf{low valence and high arousal}, visual design utilizes strong color contrasts, intense imagery, sharp shapes, and asymmetrical layouts to create tension and alertness, thereby eliciting heightened attention and emotional reactions from viewers. For instance, in Edvard Munch’s The Scream, the intense contrast between highly saturated orange and deep blue conveys a strong atmosphere of unease, while the distorted figure and wavy lines further amplify feelings of fear and anxiety. This design directly triggers high arousal levels via the sympathetic nervous system and conveys negative emotional valence \cite{wilms2018color, lu2012shape}. Moreover, the asymmetrical composition and dramatic perspective create a sense of visual oppression, reinforcing the intense emotions conveyed by the artwork (Makin et al., 2012). Similar strategies are widely applied in warning signs and public campaigns, employing highly saturated red and black contrasts, sharp geometric shapes (such as triangles), and disaster imagery to convey a sense of threat and stimulate defensive reactions \cite{pfeuffer122measuring, slavova2019towards}. This visual design approach activates viewers’ cognitive and neural systems, not only increasing emotional arousal but also prompting rapid information processing and action-taking, making it especially effective in communication contexts emphasizing crises and risks.
% % 在低情绪效价与高情绪唤起的场景中,视觉设计通过强烈的色彩对比、激烈的图像内容、尖锐的形状和非对称布局,营造出紧张感和警觉性,从而激发观众的高度注意力和情绪反应。例如,爱德华·蒙克的《呐喊》中,通过高饱和度的橙色与深蓝色的剧烈对比,传递出强烈的不安氛围,同时画中人物扭曲的形状与波浪线条进一步增强了恐惧与焦虑感,这种设计直接通过交感神经系统引发高唤起水平并传递负面情绪效价(Wilms & Oberfeld, 2018;Lu et al., 2012)。此外,非对称的构图与剧烈的透视效果带来视觉上的压迫感,强化了画面传递的紧张情绪(Makin et al., 2012)。类似的策略广泛应用于警示标志和公益宣传中,例如通过高饱和度的红色与黑色对比、尖锐的几何形状(如三角形)和灾害场景照片,传递威胁感并刺激防御性反应(Pfeuffer et al., 2024;Slavova, 2019)。这种视觉设计方式通过激活观众的认知和神经系统,不仅提升了情绪唤起水平,还促使受众快速处理信息并采取行动,在强调危机和风险的传播情境中具有重要作用。

% In scenarios of \textbf{low valence and low arousal}, visual design tends to use low-saturation colors, bland imagery, simple shapes, and symmetrical or static layouts to convey an atmosphere of indifference and helplessness. This design aims to maintain low stimulation levels, creating a subdued yet serene visual experience. 
% For example, in Caspar David Friedrich’s The Monk by the Sea, the use of gray-blue tones and low brightness highlights a mood of solitude and desolation. The painting’s vast, empty landscape and simplified human figure evoke a sense of detachment and helplessness while reducing the intensity of emotional arousal \cite{pazda2024colorfulness}. The symmetrical layout and linear composition of the painting further reduce visual complexity and distractions, allowing viewers to experience a muted negative emotional valence in a low-stimulation context (Makin et al., 2012). This design principle is not only evident in specific artworks but also serves as a reference for communication scenarios involving low emotional valence and low arousal. For instance, shape design often employs simple, linear geometric elements, such as square interface frames and straight lines, to convey restrained and calm emotions \cite{lu2012shape}. These characteristics reduce stimulation of the nervous and sensorimotor systems, suppressing emotional fluctuations while fostering a detached emotional attitude. Such designs are particularly suitable for presenting unavoidable social issues or fatalistic themes, reducing visual complexity and emotional intensity while guiding viewers to process information in a rational and calm manner. This approach effectively maintains low emotional valence and low arousal levels by minimizing emotional intensity and visual complexity, ultimately providing viewers with a profound yet subdued emotional experience.

% % 在低情绪效价与低情绪唤起的场景中,视觉设计倾向于通过低饱和度的色彩、平淡的图像内容、简单的形状以及对称或静态布局来传递冷漠与无奈的情感氛围。这种设计旨在维持低刺激水平,营造出压抑但平和的视觉体验。例如,卡斯帕·大卫·弗里德里希的《海边的僧侣》中,通过使用灰蓝色调和低亮度的画面,突出孤独与荒凉的情绪。画作中的空旷场景和简化的人物形象让观众感受到疏离与无助,同时降低了情绪唤起的强度(Pazda et al., 2024)。画面的对称布局和直线构图进一步减少了视觉上的复杂性和干扰,使观众在低刺激的情境中体验到淡然的负面情绪效价(Makin et al., 2012)。这种设计原则不仅体现在具体作品中,也为低情绪效价与低唤起的传播情境提供了参考。例如,形状设计中常采用简单、直线的几何元素,如方形界面框架和平直线条,传递克制与冷静的情感(Lu et al., 2012)。这些特性通过减少神经系统和感官运动系统的刺激,在抑制情绪波动的同时,营造出冷漠的情感态度。这样的设计尤其适用于呈现无法避免的社会问题或宿命论主题,在降低视觉复杂性与情绪强度的同时,引导观众以理性和冷静的方式接纳信息。这种方法通过压缩情绪强度和视觉复杂性,有效维持了低情绪效价与低唤起水平,最终为观众带来一种深刻却平淡的情感体验。

% By dynamically adjusting colors, images, shapes, and layouts, visual design can achieve a balance tailored to the demands of various emotional valence, arousal, and dominance levels. This design strategy not only enhances visual expressiveness but also provides scientific evidence for emotional regulation and user behavior guidance, offering practical design guidelines for diverse applications.
% % 通过对颜色、图片、形状和布局的多维调控,视觉设计能够在不同情绪效价、唤起度和支配感的需求下实现动态平衡。这种设计策略不仅提升了视觉表现力,还为情绪调控和用户行为引导提供了科学依据,为多场景的应用提供了可行的设计指南。


\subsection{Sound Design}
Sound design plays a critical role in information communication and emotional regulation, influencing users’ emotional experiences through tone, music, and sound effects. Proper adjustments of sound elements, such as pitch modulation and music tempo, can significantly enhance memory and comprehension of information.
% 声音设计在信息传递和情感调节中起着至关重要的作用,通过语调、音乐和音效来影响用户的情绪体验。适当调整声音元素,如语调的高低和音乐的节奏,可以显著提高信息的记忆和理解效果。

\renewcommand{\arraystretch}{1.8} % 调整行间距
\captionsetup{font=small}

\begin{table*}[ht]
\fontsize{8}{9}\selectfont
\centering
\begin{tabularx}{\textwidth}{|>{\centering\arraybackslash}m{0.6cm}|>{\centering\arraybackslash}m{1.55cm}|>{\centering\arraybackslash}m{4.6cm}|>{\centering\arraybackslash}m{4.6cm}|>{\centering\arraybackslash}m{4.6cm}|} 
\hline
\rowcolor[HTML]{D9EAD3} 
\multicolumn{2}{|c|}{\textbf{Dimension}} & \textbf{Tone} & \textbf{Sound Effects} & \textbf{Music} \\ \hline

% Emotional Dimensions Section
\multirow{3}{*}{\rotatebox{90}{\parbox{3cm}{\centering \textbf{Emotional \\ Dimensions}}}} & 
\cellcolor[HTML]{FDF6E8} \textbf{Valence} & 
\textbf{Pleasant tone} enhances positive emotions, Low tone amplifies negative emotions \cite{schirmer2010mark}. 
\textbf{Supportive tone} conveys calmness, Controlling tone induces discomfort \cite{weinstein2018you}. & 
\textbf{Harmonious sound} effects evoke positive valence. \textbf{Dissonant sound} effects elicit negative emotions \cite{parncutt2011consonance}. & 
\textbf{Harmonious melodies} evoke joy \cite{hofbauer2024background}. \textbf{Dissonant melodies} evoke sadness \cite{kabre2024predisposed}. \\ \cline{2-5}

& \cellcolor[HTML]{FDF6E8} \textbf{Arousal} & 
\textbf{Rapid tone} changes increase arousal \cite{bestelmeyer2017effects}. \textbf{Soft tones} decrease arousal \cite{gobl2003role}. &  
\textbf{Fast-attack sound effects} increase arousal \cite{clewett2024emotional}. \textbf{Slow-attack sound effects} decrease arousal \cite{eerola2012timbre}. & 
\textbf{Fast-paced music} increases arousal \cite{hofbauer2024background}. \textbf{Slow-paced music} reduces arousal \cite{shepherd2024investigating}. \\ \cline{2-5}

& \cellcolor[HTML]{FDF6E8} \textbf{Dominance} & 
\textbf{Supportive tone} enhances control \cite{weinstein2018you}. \textbf{Controlling tone} diminishes control \cite{james1884mind}. & 
\textbf{Harmonious sound} effects enhance control \cite{james1884mind}. \textbf{Dissonant sound} effects reduce control \cite{james1884mind}. & 
\textbf{Upbeat melodies} enhance control \cite{moon2024investigating}. \textbf{Low-pitched melodies} reduce control \cite{shepherd2024investigating}. \\ \hline

% Multisystem Activation Section
\multirow{4}{*}{\rotatebox{90}{\parbox{3cm}{\centering \textbf{Multisystem \\ Activation}}}} & 
\cellcolor[HTML]{F0EFF7} \textbf{Neural System} & 
\textbf{Gentle tones} regulate the parasympathetic system \cite{gobl2003role}. \textbf{High-arousal tones} activate the sympathetic system \cite{bestelmeyer2017effects}. & 
\textbf{Sharp sound effects} activate the sympathetic system \cite{iversen2000emotional}. \textbf{Fast-attack sound effects} trigger physiological responses \cite{iversen2000emotional}. \textbf{Slow-attack sound effects} regulate the parasympathetic system \cite{iversen2000emotional}. & 
\textbf{Fast-paced music} activates the sympathetic system \cite{juslin2008emotional}. \textbf{Slow-paced music} regulates the parasympathetic system \cite{juslin2008emotional}. \\ \cline{2-5}

& \cellcolor[HTML]{F0EFF7} \textbf{Sensorimotor System} & 
\textbf{Cheerful tone} promotes relaxation \cite{frijda1986emotions}. \textbf{Low tone} induces tension \cite{james1884mind}. & 
\textbf{Sharp sound effects} trigger tension responses \cite{iversen2000emotional, plutchik1980general}. \textbf{Harmonious sound effects} promote physical relaxation \cite{james1884mind}. & 
\textbf{Fast-paced music} induces body movement \cite{thaut2015neurobiological}. \textbf{Slow-paced music} aids relaxation \cite{bernardi2006cardiovascular}. \\ \cline{2-5}

& \cellcolor[HTML]{F0EFF7} \textbf{Cognitive Systems} & 
\textbf{Rising tone} increases anticipation \cite{zajonc1980feeling}. \textbf{Falling tone} reduces cognitive load \cite{schirmer2010mark}. & 
\textbf{Harmonious sound effects} improve focus \cite{schulte2001quality}. \textbf{Sharp sound effects} convey threat \cite{schulte2001quality}. & 
\textbf{Harmonious music} enhances cognitive efficiency \cite{zajonc1980feeling}. \textbf{Chaotic music} distracts attention \cite{hofbauer2024background}. \\ \cline{2-5}

& \cellcolor[HTML]{F0EFF7} \textbf{Motivational Systems} & 
\textbf{Authoritative tone} triggers urgency \cite{fogg2009behavior}. \textbf{Supportive tone} enhances a sense of safety \cite{weinstein2018you}. & 
\textbf{Reward sound effects} boost achievement motivation \cite{fogg2009behavior}. \textbf{Alarm sound effects} trigger protective motivation \cite{mazur2019effects}. & 
\textbf{Fast-paced and harmonious music} stimulate achievement motivation \cite{juslin2008emotional}. \textbf{Low-pitched music} encourages introspection \cite{hofbauer2024background}. \\ \hline

\end{tabularx}
\caption{Emotional Dimensions and Multisystem Activation in Sound Design Elements} 
\label{tab:sound_design}
\end{table*}

% \begin{figure*}[hbt!]
% %\setlength{\abovecaptionskip}{-0.1mm}
% \setlength{\intextsep}{10pt plus 2pt minus 2pt}
%     \centering
%     \includegraphics[width=18cm]{figs/sound_design.png}
%     \caption{Emotional Dimensions and Multisystem Activation in Sound Design Elements.}
%    \vspace{-2mm}
% \label{fig:why}
% \end{figure*}

\subsubsection{Tone}
\begin{wrapfigure}{l}{0.06\textwidth}
  %\begin{center}
  \vspace{-11pt} % 调整垂直位置
        \includegraphics[width=0.07\textwidth]{figs/icon/tone.png}
  %\end{center}
\end{wrapfigure} 
Tone, as a critical feature of sound design, profoundly influences listeners’ emotional valence, arousal levels, and dominance through adjustments in pitch, timbre, speech rate, and volume. Research shows that the consistency between tone and the emotional content of words can significantly enhance listeners’ emotional experiences. For instance, a pleasant tone (lively and bright timbre) paired with positive words can amplify the words’ positive valence, while a deep or melancholic tone can intensify the emotional negativity of negative words, leaving a deeper emotional impression on listeners \cite{schirmer2010mark}. This emotional consistency effect not only strengthens the expression of emotional information but also leaves a lasting impression in listeners’ emotional memory.
% 语调作为声音设计的重要特征,通过音调、音色、语速和音量的多重调控,深刻影响听众的情绪效价、唤起水平和支配感。研究表明,语调与词汇情绪内容的一致性能够显著增强听众的情感体验。例如,愉悦语调(轻快且明亮的音质)与积极词汇结合时,能够提升词汇的正面效价;而低沉或悲伤的语调则会放大消极词汇的情绪负向性,使听众对这些内容产生更深刻的情感记忆(Schirmer, 2010)。这种情绪一致性效应不仅强化了情感信息的表达,还在听众的情感记忆中留下长期印象。

Supportive tones (e.g., soft volume, slow speech rate, gentle timbre) can convey positive and empathetic emotions, making listeners feel respected and accepted, thereby enhancing emotional valence and fostering a calm emotional atmosphere. In contrast, controlling or hurried tones (e.g., high volume, fast speech rate) often convey a sense of pressure, reducing emotional valence, and particularly in the context of negative content, they can provoke tension and discomfort in listeners \cite{weinstein2018you}. These tones act on the sensory-motor system through the auditory system, eliciting physical responses such as muscle relaxation or tension, thereby focusing listeners’ attention on the information conveyed by the tone \cite{james1884mind, frijda1986emotions} .
% 支持性语调(如轻柔的音量、缓慢的语速、柔和的音质)能够传递积极和理解的情感,使听众感受到尊重与接纳,从而提升情绪效价并营造平和的情绪氛围。相反,控制性或急促语调(高音量、快速语速)通常传递压迫感,削弱情绪效价,尤其在负面内容的表达中易引发听众的紧张感和不适情绪(Weinstein et al., 2018)。这些语调通过听觉系统作用于感官运动系统,引发身体反应,例如肌肉的放松或紧张,从而使听众更专注于语调传递的信息(James, 1884;Frijda, 1986)。

At the level of emotional arousal, variations in tone speed and intensity play a crucial role. High-arousal tones (e.g., anger or fear) activate auditory neural pathways and the sympathetic nervous system, triggering physiological responses such as increased heart rate and pupil dilation, thereby heightening listeners’ perception of urgent or high-risk information \cite{bestelmeyer2017effects}. In contrast, gentle tones (moderate volume, slow speech rate, steady pitch) reduce arousal levels, helping listeners alleviate anxiety and achieve a state of calm. Such tones are often used in relaxation settings, where parasympathetic nervous system regulation induces physical comfort, such as reduced blood pressure and slower heart rate \cite{gobl2003role}.
% 在情绪唤起层面,语调的快慢与强度变化扮演关键角色。高唤起语调(如愤怒或恐惧)会激活听觉神经通路和交感神经系统,引发心跳加速、瞳孔扩张等生理反应,强化听众对紧急或高风险信息的感知(Bestelmeyer et al., 2017)。相比之下,柔和语调(音量适中、语速缓慢、音调平稳)通过降低唤起水平,帮助听众缓解焦虑并进入平静状态。此类语调常用于放松场景,通过副交感神经系统的调节,带来血压降低和心率放缓的身体舒适感(Gobl & Chasaide, 2003)。

The emotional impact of tone is also deeply rooted in the cognitive system. Different tones provide semantic cues for information. For example, a rising interrogative tone implies incomplete intentions, enhancing the audience’s anticipation for subsequent content, while a descending declarative tone conveys certainty and trust, reducing cognitive load \cite{zajonc1980feeling}. Moreover, tone not only influences the immediate interpretation of information but also has a lasting impact on the emotional representation of words, which remains significant even after explicit memory fades \cite{schirmer2010mark}. Therefore, tone plays a critical role in enhancing the emotional value and memory retention of information for listeners.
% 语调的情绪影响还深植于认知系统中。不同语调为信息提供语义提示,例如,上扬的疑问语调暗示未完成的意图,增强听众对后续内容的期待;而下降的陈述语调则传递确定性与信任感,减轻认知负荷(Zajonc, 1980)。此外,语调不仅影响信息的即时解读,还对词汇的情绪表征产生长期作用,即使在显性记忆消退后,这种影响仍然显著(Schirmer, 2010)。因此,语调在提升听众对信息的情感价值和记忆持久性方面具有重要意义。

The activation of the motivational system is also a critical aspect of how tone influences emotions. Authoritative tone (fast speech rate and stable intonation) conveys seriousness and urgency, which can stimulate action motivation, prompting listeners to make quick decisions or take action \cite{fogg2009behavior}. For example, authoritative tone in navigation instructions can enhance users’ focus on tasks. Supportive tone (gentle and melodic voice) tends to evoke a sense of safety, making listeners more willing to accept information and establish emotional connections with it \cite{weinstein2018you}. This activation of the motivational system not only increases information acceptance but also enhances users’ trust and reliance on the design.
% 动机系统的激活也是语调影响情绪的重要方面。权威语调(语速较快、语调稳定)通过传递严肃和紧迫感,能够激发行动动机,促使听众迅速决策或采取行为(Fogg, 2009)。例如,导航指令中的权威语调能够提高用户对操作的专注力。而支持性语调(柔和且富有韵律的语音)则倾向于激发安全感动机,使听众更愿意接受信息并与之建立情感连接。[1]这种动机系统的触发不仅提升了信息的接受度,还增强了用户对设计的信任感和依赖性。

Overall, tone, through the synergistic effects of the sensory-motor system, nervous system, cognitive system, and motivational system, exerts a multi-level influence on the regulation of listeners’ emotions and the guidance of their behavior. In tone design, designers can precisely adjust tone characteristics to meet the needs of different scenarios, thereby achieving effective emotional shaping and positive behavioral guidance.
% 从整体来看,语调通过感官运动系统、神经系统、认知系统和动机系统的协同作用,在听众的情绪调节和行为引导中发挥了多层次的影响。设计师在语调设计中,可以针对不同场景的需求精准调控语调特性,从而实现情绪的有效塑造与行为的积极引导。

\subsubsection{Sound Effects}
\begin{wrapfigure}{l}{0.06\textwidth}
  %\begin{center}
   \vspace{-11pt} % 调整垂直位置
        \includegraphics[width=0.07\textwidth]{figs/icon/soundeffects.png}
  %\end{center}
\end{wrapfigure} 
Sound effects significantly influence emotional valence and arousal through spectral characteristics, harmonicity, and attack slopes \cite{eerola2012timbre}. Harmonious intervals (e.g., perfect fifth or perfect fourth) typically evoke positive valence, making listeners feel comfortable and relaxed. This sound characteristic is widely used in relaxation and meditation scenarios, such as melodies in background music simulating natural sounds (e.g., flowing water or birdsong), helping to create a peaceful atmosphere. In contrast, dissonant intervals (e.g., minor second or major second) often induce tension, unease, or negative emotions due to their high-frequency energy and acoustic roughness, making them suitable for portraying danger or suspense\cite{parncutt2011consonance}. This contrast manifests in the sensory-motor system as varying degrees of physical responses, such as muscle relaxation when hearing harmonious sounds, while dissonant sounds may lead to muscle tension or stress reactions like frowning \cite{james1884mind}.
% 音效通过频谱特征、和谐度、攻击斜率等方面对情绪效价和唤起度产生显著影响。[1]和谐音程(如纯五度或纯四度)通常带来积极效价,使听众感到舒适与放松。这种音效特性被广泛应用于放松和冥想场景,例如背景音乐中模拟自然声音(如流水或鸟鸣)的旋律,有助于创造平和的氛围。相对而言,不和谐音程(如小二度或大二度)因其高频能量和声学粗糙性,常引发紧张、不安或负面情绪,适用于表现危险或悬疑的场景(Parncutt & Hair, 2011)。[2]这种对比在感官运动系统中表现为不同程度的身体反应,例如听到和谐音效时肌肉放松,而不和谐音效可能导致肌肉紧绷或皱眉等应激反应(James, 1884)。

The attack slope of sound effects, which refers to the rate of change from silence to maximum amplitude, directly affects emotional arousal levels. Rapid-attack sound effects (e.g., abrupt alarm sounds) significantly heighten listeners’ alertness and tension \cite{eerola2012timbre}. Such high-arousal sound effects are commonly used in emergency scenarios, such as fire alarms or medical alert tones, to quickly capture attention and activate the sympathetic nervous system, eliciting physiological responses like increased heart rate and rapid breathing \cite{iversen2000emotional}. In contrast, slow-attack sound effects (e.g., gradually intensifying string music) reduce arousal levels through smooth volume changes, creating a calming emotional experience suitable for emotional regulation or meditation scenarios \cite{zajonc1980feeling}. This characteristic of sound effects is primarily regulated by the amygdala and hypothalamus within the nervous system, determining the intensity of emotional arousal and physiological response patterns \cite{iversen2000emotional}.
% 音效的攻击斜率,即声音从静止到最大振幅的变化速度,直接影响情绪唤起水平。快速攻击音效(如短促警报声)能够显著提升听众的警觉性和紧张感。这种高唤起音效常用于紧急情境,例如火警警报或医疗急救提示音,迅速吸引注意力并激活交感神经系统,引发心跳加速和呼吸急促等生理反应(Iversen et al., 2000)。相较之下,慢攻击音效(如渐强的弦乐)通过平缓的音量变化降低唤起水平,营造舒缓的情绪体验,适合用于情绪调节或冥想的场景(Zajonc, 1980)。音效的这一特性在神经系统中主要通过杏仁核和下丘脑的调节,决定情绪的唤起强度和生理反应模式。

From the perspective of the cognitive system, sound effects influence emotions not only through direct perception but also by shaping listeners’ emotional evaluations through contextual cues. For instance, harmonious background sound effects with gentle rhythms are often interpreted as positive and safe contextual cues, enhancing listeners’ sense of pleasure and focus. In contrast, sharp sound effects (e.g., metal scraping or piercing alarm sounds) convey signals of threat, prompting listeners to assess potential risks and heightening their vigilance and preparedness \cite{schulte2001quality}. The role of sound effects in the cognitive system is also evident in their auxiliary function in information processing. For example, prompt sounds within a soft soundscape can improve task performance, whereas complex or jarring sound effects may distract attention and impair cognitive performance~\cite{schulte2001quality}.

% 从认知系统的角度来看,音效不仅通过直接感知影响情绪,还通过情境暗示塑造听众的情感评估。例如,和谐的背景音效与轻柔的节奏常被解读为积极和安全的情境提示,能够增强听众的愉悦感和专注力。相反,尖锐音效(如金属摩擦声或刺耳警报声)传递威胁性信号,会激发听众对潜在风险的认知评估,使其更加警觉和防备(Schulte-Fortkamp, 2001)。音效在认知系统中的作用还表现在其对信息处理的辅助功能。例如,柔和音效背景下的提示音能够提升任务执行效率,而复杂或刺耳的音效可能分散注意力,降低认知表现(Ostendorf et al., 2020)。

Sound effects also have a significant impact on activating the motivational system. Reward sound effects (e.g., task completion prompts) reinforce positive feedback, stimulating listeners’ achievement motivation and encouraging greater engagement in target tasks \cite{fogg2009behavior}. For example, reward sound effects for gaining points in games increase player engagement and enhance satisfaction upon task completion. Conversely, alarm sounds convey urgent information that triggers protective motivation, prompting listeners to take swift risk-avoidance actions. This motivational trigger mechanism is widely used in affective computing and behavioral design to guide users in making quick decisions in specific contexts~\cite{mazur2019effects}.
% 音效对动机系统的激活同样显著。奖励音效(如任务完成提示音)通过强化积极反馈,能够激发听众的成就动机,使其更愿意投入到目标任务中(Fogg, 2009)。例如,游戏中的积分奖励音效不仅增加了玩家的参与感,还提升了任务完成的满足感。相反,警报音效通过传递紧急信息触发保护动机,使听众迅速采取规避风险的行动。这种动机触发机制在情感计算和行为设计中被广泛应用,用于引导用户在特定情境中做出快速反应(Mazur et al., 2019)。

\subsubsection{Music}
\begin{wrapfigure}{l}{0.06\textwidth}
  %\begin{center}
  %\vspace{-11pt} % 调整垂直位置
        \includegraphics[width=0.07\textwidth]{figs/icon/music.png}
  %\end{center}
\end{wrapfigure} 
Music exerts a multidimensional impact on emotional valence and arousal levels through features such as melody, rhythm, and emotional complexity. Music tempo is one of the key factors in emotional regulation. Fast-paced music (such as tempos above 130 beats per minute) induces higher emotional arousal \cite{hofbauer2024background}, making listeners feel excited and invigorated, and is widely used in settings like fitness or competitive sports \cite{moon2024investigating}. In contrast, slow-paced music (such as tempos between 60 and 90 beats per minute) helps alleviate anxiety and promotes relaxation\cite{hofbauer2024background}, making it a common choice in relaxing contexts such as meditation and psychotherapy \cite{shepherd2024investigating}. From the sensorimotor system perspective, fast-paced music often triggers bodily movements, such as involuntary swaying, while slow-paced music aids in relaxing the body by relaxing muscles and stabilizing breathing \cite{thaut2015neurobiological, bernardi2006cardiovascular}. On the neurological level, fast-paced music enhances excitation by activating the sympathetic nervous system, while slow-paced music regulates through the parasympathetic nervous system, lowering heart rate and blood pressure, thereby providing physiological comfort and relaxation \cite{juslin2008emotional}.
% 音乐通过旋律、节奏和情绪复杂性等特征对情绪效价和情绪唤起水平产生多维影响。音乐节奏是情绪调节的重要因素之一。快节奏音乐(如每分钟130次节拍以上)激发较高的情绪唤起,使听众感到兴奋与振奋,因而被广泛应用于健身或竞技等场景(Moon et al., 2024)。相反,慢节奏音乐(如每分钟60至90次节拍)有助于缓解焦虑并促进放松,这使其成为冥想、心理治疗等放松情境中的常用选择(Shepherd et al., 2024)。从感官运动系统来看,快节奏音乐常引发身体律动,如不自主摇摆,而慢节奏音乐则通过放松肌肉和稳定呼吸帮助身体放松(Thaut et al., 2015; Bernardi et al., 2006)。在神经系统层面,快节奏音乐通过激活交感神经系统增强兴奋感,慢节奏音乐则通过副交感神经系统调节,降低心率与血压,从而带来生理上的舒适与放松(Juslin & Västfjäll, 2008)。

The melodic characteristics of music also play a key role in shaping emotional valence. Harmonious melodies and bright tonalities typically convey positive emotions, helping to enhance focus and creativity, commonly found in work or social settings \cite{hofbauer2024background}. In contrast, low tonality and slow melodies convey sad or introspective emotions, widely used in emotional films or therapeutic settings \cite{kabre2024predisposed}. Dissonant melodies, due to their tense and chaotic nature, are often used to create tension or provoke alertness in scenes such as horror films or warning messages. Additionally, fast-paced music and harmonious melodies often stimulate achievement motivation and exploration motivation, helping listeners stay focused on tasks and improve performance; whereas low, somber music may trigger introspective motivation, encouraging listeners to engage in deep emotional thinking and reflection \cite{juslin2008emotional}.
% 音乐的旋律特性同样在情绪效价的塑造中起着关键作用。和谐旋律和明亮调性通常传递积极情绪,帮助提升专注力和创造力,常见于工作或社交场景(Hofbauer et al., 2024)。与之相对,低调性与缓慢旋律则传递悲伤或内省情绪,广泛应用于情感电影或疗愈场景(Kabre & Srivastava, 2024)。不和谐旋律因其紧张和混乱特性,常用于制造张力或引起警觉的场景,如恐怖电影或警示信息。此外,快节奏音乐和和谐旋律往往激发成就动机和探索动机,使听众更加专注于任务并提升执行力;而低沉的音乐则可能引发内省动机,促使听众进行深刻的情感思考和反省。

Music not only affects an individual's psychological state through emotional regulation but also significantly influences the cognitive system, thereby impacting task performance. Harmonious music helps improve the efficiency of cognitive tasks, such as enhancing reading speed and memory; whereas dissonant music may distract attention and reduce task efficiency \cite{zajonc1980feeling}. Positive emotional music can stimulate the willingness to perform tasks and promote active engagement; in contrast, negative emotional music may suppress initiative, making listeners more cautious \cite{hofbauer2024background}.
% 音乐不仅通过情绪调节影响个体的心理状态,还能够显著作用于认知系统,从而影响任务表现。和谐音乐有助于提升认知任务的效率,如增强阅读速度与记忆力;而不和谐的音乐则可能分散注意力,降低任务效率(Zajonc, 1980)。正面情绪音乐能够激发任务执行意愿,促进积极参与;相反,负面情绪音乐则可能抑制主动性,令听众更为谨慎(Hofbauer et al., 2024)。

The emotional complexity of music enhances the depth of emotional experiences. Music that blends positive and negative emotions (such as melodies combining sorrow and appreciation) can evoke a deeper level of emotional resonance, particularly in contexts with high emotional expression demands, such as film scores \cite{baltazaremotional}.
% 音乐的情感复杂性也增强了情绪体验的层次感。融合正面与负面情绪的音乐(如悲痛与欣赏并存的旋律)能引发更高层次的情感共鸣,尤其在影视配乐等情感表达需求较高的场景中(Baltazar et al., 2024)。


%\subsubsection{Summary}
% Sound design plays a crucial role in emotional regulation, combining tone, sound effects, and background music to create diverse emotional experiences across varying emotional valence and arousal levels. Whether in high-valence, high-arousal scenarios that evoke positive emotions or low-valence, low-arousal settings that convey solitude, sound design engages the listener’s emotions through sensory stimuli while also influencing emotional memory and behavioral responses via neural and cognitive mechanisms.
% % 声音设计在情绪调节中发挥着重要作用,通过语调、音效和背景音乐的结合,为不同情绪效价和唤起水平创造多样的情感体验。无论是激发积极情绪的高效价、高唤起场景,还是传递孤寂情感的低效价、低唤起情境,声音设计不仅通过感官刺激调动听众的情绪,还通过神经系统和认知机制影响情感记忆与行为反应。

% In scenarios of \textbf{high valence and high arousal}, sound design employs an intricate combination of invigorating tone, intense sound effects, and fast-paced background music, effectively evoking positive emotions and intense sensory experiences. For example, the iconic training montage in Rocky features the soundtrack Gonna Fly Now, characterized by fast tempo and high-pitched melodies, paired with pronounced drumbeats and stirring orchestration, with a tempo exceeding 130 beats per minute \cite{moon2024investigating}, generating a powerful and consistent rhythm that successfully induces listeners’ physical movement and excitement. The inclusion of boxing punches, rapid breathing, and footsteps in the soundtrack, with their sharp attack slopes, further intensifies tension and action motivation, drawing viewers’ focus on the protagonist’s relentless efforts \cite{iversen2000emotional}. Furthermore, the tone of dialogue conveys a sense of strength through moderate volume and steady rhythm, aligning seamlessly with the emotional tone of the background music to create an emotional climax \cite{schirmer2010mark}. This design synergizes music, tone, and sound effects, not only activating the auditory and sympathetic nervous systems but also enhancing emotional memory and a dominance, enabling viewers to deeply feel a positive drive. Such high-energy sound design is especially common in fitness settings, commercial advertisements, and entertainment content, as it strengthens listeners’ attention and emotional memory through sensory-motor system stimulation, while also activating the motivational system, making listeners feel empowered and capable of action, thereby significantly enhancing emotional experience and behavioral engagement.
% % 在高情绪效价与高情绪唤起的场景中,声音设计通过激昂的语调、强烈的音效和快节奏背景音乐的精妙结合,有效激发积极情绪和强烈的感官体验。例如,电影《洛奇》中经典的训练场景配乐《Gonna Fly Now》以快节奏和高音调的旋律为主,融合节奏鲜明的鼓点和激昂的管弦乐,其每分钟超过130次的节拍(Moon et al., 2024)强烈而一致,成功唤起听众的身体律动和兴奋感。配乐中穿插的拳击声、急促的呼吸声以及脚步声,通过快速攻击斜率进一步强化了紧张感和行动动机,吸引观众更加聚焦于主角的奋力拼搏(Iversen et al., 2000)。此外,对话中传递力量感的语调,以适中的音量和稳定的节奏表现出坚毅与自信,与背景音乐的情感基调完美契合,形成了情绪上的高潮(Schirmer, 2010)。这种设计通过音乐、语调和音效的协同作用,不仅激活了听觉和交感神经系统,还增强了情绪记忆和支配感,使观众深刻感受到正向的驱动力。这种高能量特性的声音设计在健身场景、商业广告和娱乐内容中尤为常见,通过感官运动系统的刺激强化听众的注意力和情感记忆,同时通过激活动机系统,使听众感到自身具有掌控力和行动能力,极大提升了情绪体验和行为投入。

% In scenarios of \textbf{high valence and low arousal}, sound design employs gentle tones, calming sound effects, and soothing background music to create a warm and stable emotional atmosphere. For example, in the movie Life is Beautiful, the protagonist’s father uses a soft and slightly humorous tone to explain the harsh environment of the concentration camp to his son, with warm vocal timbre and slow speech rate that not only reduce arousal levels but also convey profound care and hope \cite{schirmer2010mark}. This tonal design complements the emotional tone of the background music Life is Beautiful. Composed by Nicola Piovani, this score features soft string instrumentation and a slow melody, which enhances the warmth and affinity of the scene through harmonious intervals and lyrical melodies. This sound design reduces stimulation to the nervous system and activates the regulatory functions of the parasympathetic nervous system, helping to lower heart rate and blood pressure, thus providing listeners with physiological and psychological comfort \cite{juslin2008emotional}. The soothing rhythm of the music and the caring elements in the tone complement each other, jointly shaping a tranquil yet profound emotional experience. This design, through the seamless integration of tone, sound effects, and music, not only deepens the audience’s resonance with the emotional bonds between characters, but also imbues the narrative with a sense of stability and hope. Such sound design is widely used in therapeutic or emotional narrative contexts, where reducing sensory stimulation enhances the audience’s emotional resonance and internal calmness, while successfully elevating emotional valence in a low-arousal state.
% % 在高情绪效价与低情绪唤起的场景中,声音设计通过柔和的语调、平静的音效和舒缓的背景音乐,巧妙地营造出温暖而安定的情绪氛围。例如,电影《美丽人生》中,主角父亲在集中营中用轻柔且略带幽默的语调向儿子解释险恶的环境,其温暖的音色和缓慢的语速不仅降低了情绪唤起水平,还传递了深切的关怀与希望(Schirmer, 2010)。这一语调设计与背景音乐《Life is Beautiful》的情感基调相得益彰。这首由尼古拉·皮奥瓦尼创作的配乐,以柔和的弦乐和缓慢的旋律为主,通过和谐的音程与抒情的旋律增强了场景的亲和力和温暖感。这种声音设计通过减少对神经系统的刺激,激活副交感神经的调节作用,有助于降低心率和血压,为听众带来生理和心理上的舒适感(Juslin & Västfjäll, 2008)。音乐的舒缓节奏与语调中的关怀元素相辅相成,共同塑造了平静而深刻的情绪体验。这种设计通过语调、音效与音乐的紧密融合,不仅加深了观众对角色间情感纽带的共鸣,还为叙事注入了一种安定与希望的氛围。这类声音设计广泛应用于心理疗愈或情感叙事场景,通过降低感官刺激的强度,增强观众的情感共鸣与内在安定感,同时在低唤起的状态下成功提升情绪效价。

% In scenarios of \textbf{low valence and high arousal}, sound design employs tense tones, jarring sound effects, and high-energy background music to create intense unease and emotional tension. For example, in the movie A Quiet Place, Marco Beltrami’s background score What Is Safe uses deep strings and progressively intensifying rhythms, combined with dissonant intervals and sudden high-pitched insertions, to evoke sustained fear and a sense of oppression in the audience. This music design, through gradual volume crescendos and abrupt acoustic shocks, stimulates the auditory and sympathetic nervous systems, significantly increasing heart rate and alertness \cite{iversen2000emotional}. Simultaneously, environmental sound effects in the film, such as subtle footsteps and occasional object collisions, complement the tense atmosphere of the background music. These sound effects, with their rapid attack slopes and high-frequency energy, provoke the audience’s cognitive assessment of potential threats, further heightening emotional arousal levels. Complementing this sound and music design, the actors’ dialogue is condensed into brief whispers, with a repressed tone that aligns with the emotional tone of the scenes, enhancing the audience’s empathy and immersion in the characters’ situations. This integrated sound design keeps the audience in a state of sustained tension, while the overlap of emotional arousal and a sense of threat triggers strong defensive motivation, drawing greater focus on the film’s narrative progression. Such sound design is particularly prevalent in horror and thriller films, where the precise coordination of sensory stimuli effectively conveys tension and deep emotional experiences.

% % 在低情绪效价与高情绪唤起的场景中,声音设计通过紧张的语调、刺耳的音效和高能量的背景音乐,共同营造出强烈的不安感和高张力的情绪氛围。例如,电影《寂静之地》中,马可·贝尔特拉米创作的背景音乐《What Is Safe》以低沉的弦乐和逐渐强化的节奏为基础,结合不和谐的音程和突然的高音尖锐插入,使观众感受到持续的恐惧与压迫感。这种音乐设计通过缓慢的音量增长与突如其来的声学冲击,刺激了听觉神经系统和交感神经系统,显著提高了心率和警觉性(Iversen et al., 2000)。同时,电影中的环境音效,如轻微的脚步声和偶尔的物体碰撞声,与背景音乐的紧张氛围相辅相成。这些音效通过快速的攻击斜率和高频率能量,引发了观众对潜在威胁的强烈认知评估,进一步提升了情绪唤起水平。配合这一音效与音乐设计,演员台词被压缩为短促低语,语调中蕴含的压抑感与场景的情感基调相一致,增强了观众对角色情境的同理心与代入感。这种声音设计的综合运用,不仅让观众持续处于紧张状态,还通过情绪唤起和威胁感的叠加,触发了强烈的防御动机,使其更加专注于电影情节的发展。这类声音设计在恐怖电影和惊悚片中尤为常见,通过感官刺激的高度协调,有效传递了紧张情绪和深刻的情感体验。

% In scenarios of \textbf{low valence and low arousal}, sound design employs deep tones, calm sound effects, and background music emphasizing silence to convey feelings of indifference, loneliness, or fatalism. For example, in the movie The Wind That Shakes the Barley, low-volume, slow-paced dialogue and tone express the characters’ resignation towards life. This tone avoids emotional fluctuations, using restrained pacing and muted sound quality to convey the characters’ inner solitude to the audience \cite{schirmer2010mark}. 
% The background music Katyusha Theme, primarily featuring deep bass strings and sparse percussion, leaves significant pauses between notes, evoking a sense of time’s stagnation and environmental desolation for the audience \cite{wilms2018color}. Simultaneously, sound design minimizes high-frequency elements, retaining only natural effects like wind and raindrops, with low tones and slow rhythms that reduce auditory stimulation and emotional arousal \cite{wilms2018color}. This sound design reduces neural activation and sensory-motor system stimulation, guiding the audience into a passive and introspective emotional state. By compressing the dynamic range and volume variation of sounds, it sustains a low-arousal emotional atmosphere, allowing the audience to perceive negative emotions while remaining calm. This type of sound design is often used to portray fatalistic themes or depict helplessness in narrative settings, reducing sensory load to create a profound, calm yet tragically tinged emotional experience.
% % 在低情绪效价与低情绪唤起的场景中,声音设计通过低沉的语调、平缓的音效和寂静感突出的背景音乐,传递出冷漠、孤寂或宿命的情感氛围。例如,电影《风吹稻浪》中,用低音量、缓慢节奏的对话和语调表达出角色对生活无可奈何的态度。这种语调避免了任何情绪波动,通过压抑的语速和沉闷的音质,将角色的内心孤独传递给观众(Schirmer, 2010)。背景音乐《Katyusha Theme》以深沉的低音弦乐和少量背景打击乐为主,音符之间留有大量空白,使观众感受到时间的停滞感和环境的冷清(Wilms & Oberfeld, 2018)。同时,音效设计中减少了高频元素,仅保留风声、雨滴声等自然音效,低声调且节奏缓慢,通过削弱听觉刺激降低情绪唤起水平(Wilms & Oberfeld, 2018)。这种声音设计通过减少神经系统的激活和感官运动系统的刺激,引导观众进入一种被动且内敛的情绪状态。通过压缩声音的动态范围和音量变化,进一步维持低唤起的情绪氛围,使观众在感知到负面情绪的同时保持冷静。这种声音设计常用于表现宿命论主题或描绘无力感的叙事场景,通过降低感官负担,营造一种深沉、冷静却带有悲剧意味的情感体验。


\subsection{Interaction Design}
Interaction design aims to optimize the interaction process between humans and systems, enhancing user experience and emotional satisfaction. It encompasses key elements such as intuitive navigation, instant feedback mechanisms, dynamic elements, and haptic interactions. These components work together to not only improve operational fluency and efficiency but also significantly influence users’ emotional valence and arousal. By effectively integrating these design elements, interaction design achieves a dual enhancement of functionality and emotional value.
% 交互设计 旨在通过优化人与系统之间的交互过程,提升用户体验和情感满足。它包括直观的导航、即时反馈机制、动态元素和触觉交互等关键要素,这些共同作用,不仅提升操作流畅性和效率,还深刻影响用户的情绪效价和情绪唤起。通过合理整合这些设计要素,交互设计实现了功能性与情感价值的双重提升。

\renewcommand{\arraystretch}{1.6} % 调整行间距
\captionsetup{font=small}

\begin{table*}[ht]
\fontsize{8}{9}\selectfont
\centering
\begin{tabularx}{\textwidth}{|>{\centering\arraybackslash}m{0.6cm}|>{\centering\arraybackslash}m{1.55cm}|>{\centering\arraybackslash}m{4.6cm}|>{\centering\arraybackslash}m{4.6cm}|>{\centering\arraybackslash}m{4.6cm}|} 
\hline
\rowcolor[HTML]{D9EAD3} 
\multicolumn{2}{|c|}{\textbf{Dimension}} & \textbf{Interaction Method} & \textbf{Motion Effects} & \textbf{Navigation Design} \\ \hline

% Emotional Dimensions Section
\multirow{3}{*}{\rotatebox{90}{\parbox{3cm}{\centering \textbf{Emotional \\ Dimensions}}}} & 
\cellcolor[HTML]{FDF6E8} \textbf{Valence} & 
Intuitive interactions (e.g., tapping, swiping) reduce learning cost, increase comfort and satisfaction \cite{sundar2014user, amoor2014designing}.  Complex interactions (e.g., multi-finger gestures) increase frustration and negative emotions \cite{sundar2014user}. & 
Linear motion effects convey stability, enhancing calmness and attraction \cite{lockyer2012affective}. Complex curves may cause tension and unease \cite{lockyer2012affective}. & 
Simple navigation reduces cognitive load, enhancing positive emotions \cite{wang2024enhancing, abdelaal2023accessibility}. Complex navigation induces confusion \cite{sheng2012effects}.\\ \cline{2-5}

& \cellcolor[HTML]{FDF6E8} \textbf{Arousal} & 
Simple gestures (e.g., tapping, swiping) create a natural, smooth experience, controlling arousal \cite{wodehouse2014exploring}. Strong haptic feedback attracts attention but may cause tension \cite{olugbade2023touch}. &  
High-intensity effects (e.g., rapid cuts) increase attention and arousal \cite{hanjalic2005affective}. Slow motion creates a calming atmosphere \cite{wollner2018slow}. & 
Immediate feedback (e.g., visual or tactile cues) reduces anxiety, stabilizing emotions \cite{sundar2014user, wang2024enhancing}. Dynamic navigation enhances engagement and arousal \cite{amoor2014designing}. \\ \cline{2-5}

& \cellcolor[HTML]{FDF6E8} \textbf{Dominance} & 
Free exploratory interactions (e.g., drag and zoom) enhance user control and engagement \cite{amoor2014designing}.  Lengthy, complex tasks reduce control \cite{wang2024enhancing}. & 
Dynamic navigation (e.g., expandable menus) boosts user control \cite{amoor2014designing}. Lack of feedback undermines control \cite{wang2024enhancing}. & 
Consistent navigation aligns with user expectations, boosting trust and control \cite{abdelaal2023accessibility}. Errors (e.g., dead links) reduce control \cite{sheng2012effects}. \\ \hline

% Multisystem Activation Section
\multirow{4}{*}{\rotatebox{90}{\parbox{3cm}{\centering \textbf{Multisystem \\ Activation}}}} 

& \cellcolor[HTML]{F0EFF7} \textbf{Neural System} & 
Real-time feedback (e.g., vibration) stimulates sensory neurons, enhancing task awareness \cite{olugbade2023touch}. & 
Dynamic transitions activate the visual pathway, improving neural plasticity \cite{yoo2005processing}. High-intensity visuals (e.g., flashing) trigger short-term tension \cite{hanjalic2005affective}. & 
Dynamic navigation (e.g., auto-scrolling) strengthens neural responses through sensory input \cite{sundar2014user}. Immediate feedback (e.g., click confirmation) enhances achievement sensation \cite{wang2024enhancing}. \\ \cline{2-5}

& \cellcolor[HTML]{F0EFF7} \textbf{Sensorimotor System} & 
Intuitive interactions (e.g., swiping, tapping) activate touch and motion feedback, reinforcing motor learning \cite{sundar2014user}. Complex gestures may cause hand fatigue \cite{wodehouse2014exploring}. & 
Fast animations stimulate dynamic perception \cite{hanjalic2005affective}. Slow motion promotes detailed visual processing \cite{wollner2018slow}. & 
Simplified navigation paths reduce hand-eye coordination stress \cite{wang2024enhancing}.  Dynamic feedback enhances task completion awareness \cite{sundar2014user}. \\ \cline{2-5}

& \cellcolor[HTML]{F0EFF7} \textbf{Cognitive Systems} & 
Intuitive interactions reduce cognitive load, improving task efficiency \cite{wang2024enhancing}. Complex interactions may impair cognitive performance, especially for first-time users \cite{sundar2014user}. & 
Gradual animations optimize information flow, reducing interference \cite{lockyer2012affective}. Smooth animations enhance visual processing \cite{yoo2005processing}. & 
Logical navigation structure reduces search time, enhancing decision-making \cite{sheng2012effects}. Predictable navigation improves familiarity and confidence \cite{abdelaal2023accessibility}. \\ \cline{2-5}

& \cellcolor[HTML]{F0EFF7} \textbf{Motivational Systems} & 
Reward-based feedback (e.g., task completion sound) boosts achievement motivation \cite{amoor2014designing, wang2024enhancing}. Complex interactions may reduce motivation, especially for new users \cite{wodehouse2014exploring}. & 
Dynamic design (e.g., layered animations) fosters competition or challenge motivation \cite{lockyer2012affective}. Attention-grabbing effects (e.g., floating buttons) encourage exploration \cite{wollner2018slow}. & 
Interactive navigation (e.g., drag-based interfaces) stimulates exploratory motivation \cite{amoor2014designing}. Dynamic task maps encourage curiosity and learning \cite{sundar2014user}. \\ \hline

\end{tabularx}
\caption{Emotional Dimensions and Multisystem Activation in Interaction Design Elements}
\label{tab:interaction_design}
\end{table*}

% \begin{figure*}[hbt!]
% %\setlength{\abovecaptionskip}{-0.1mm}
% \setlength{\intextsep}{10pt plus 2pt minus 2pt}
%     \centering
%     \includegraphics[width=18cm]{figs/interaction_design.png}
%     \caption{Emotional Dimensions and Multisystem Activation in Interaction Design Elements.}
%    \vspace{-2mm}
% \label{fig:why}
% \end{figure*}

\subsubsection{Interaction Methods}
\begin{wrapfigure}{l}{0.06\textwidth}
  %\begin{center}
   \vspace{-11pt} % 调整垂直位置
        \includegraphics[width=0.07\textwidth]{figs/icon/interaction_methods.png}
  %\end{center}
\end{wrapfigure} 
Interaction methods, as a core element of user experience design, profoundly influence users’ emotional experiences, behavioral performance, and the overall effectiveness of system interactions. By optimizing the intuitiveness of interactions, feedback mechanisms, and tactile or gesture-based operations, significant improvements can be achieved in users’ emotional valence, arousal levels, and dominance, while also stimulating the coordinated engagement of sensory-motor systems, neural systems, cognitive systems, and motivational~systems.
% 交互方式作为用户体验设计中的核心要素,深刻影响用户的情绪体验、行为表现以及系统交互的整体效果。通过优化交互的直观性、反馈机制以及触觉与手势操作,可以显著改善用户的情绪效价、唤起度与控制感,同时激发感官运动系统、神经系统、认知系统和动机系统的协同作用。

Intuitive interaction methods, with their simplicity and low cognitive load, significantly enhance users’ positive emotional valence. For example, common interaction methods such as clicking and sliding reduce learning costs, enabling users to quickly adapt to interface logic and complete tasks smoothly, thereby enhancing operational fluidity and comfort \cite{sundar2014user}. Effortless and natural gestures, such as single-finger swiping and tapping, can activate the sensory-motor system, allowing users to perceive the smoothness and coordination of interactions, further enhancing the enjoyment of the experience \cite{wodehouse2014exploring}.
% 直观的交互方式,因其操作简便和认知负荷低,能够显著提升用户的正向情绪效价。例如,点击与滑动等常见交互方式,通过降低学习成本,使用户快速适应界面逻辑并顺利完成任务,增强了操作流畅性与舒适感(Sundar et al., 2014)。这些轻松自然的手势操作(如单指滑动、轻触)还能够激活感官运动系统,使用户在操作过程中感受到交互的顺畅性和身体感知的协调性,进一步提升交互体验的愉悦感(Wodehouse & Sheridan, 2014)。

In contrast, complex interaction methods (e.g., multi-finger gestures, cumbersome task flows) increase users’ operational burden and learning costs, significantly leading to frustration and anxiety. For instance, in mobile applications requiring complex gestures, users may feel frustrated due to difficulty in mastering operations quickly, which significantly reduces emotional valence and willingness to engage. Additionally, such high-complexity operations may cause unnecessary physical tension, such as muscle strain or hand fatigue, further diminishing users’ trust and satisfaction with the interface \cite{wodehouse2014exploring}. This negative emotional experience is particularly pronounced for first-time users of a new system, potentially leading to a negative overall evaluation of the interaction.
% 相比之下,复杂的交互方式(如多指手势、繁琐的任务流程)则会增加用户的操作负担和学习成本,导致挫败感与焦虑感的显著增加。例如,在需要复杂手势的移动应用中,用户可能因无法快速掌握操作方式而感到沮丧,显著降低情绪效价和参与意愿。此外,这种高复杂度操作还可能引发不必要的生理紧张状态,如肌肉紧张或手部疲劳,进一步削弱用户对界面的信任和满意度(Wodehouse & Sheridan, 2014[2])。尤其是首次接触新系统的用户,这种负面情绪体验尤为突出,容易对整体交互评价产生负面影响。

Real-time feedback mechanisms play a crucial role in interaction design. By utilizing visual (e.g., button highlights), auditory (e.g., notification sounds), or haptic (e.g., vibrations) feedback, users can immediately perceive the results of their actions, significantly reducing emotional stress caused by uncertainty \cite{wang2024enhancing}. Haptic feedback is particularly important; micro-vibrations or warm tactile sensations activate the sensory-motor system, enhancing users’ physical perception of operations while improving intuitiveness and comfort \cite{olugbade2023touch}. For example, in navigation applications, combining tactile vibrations upon confirmation with visual highlights helps users clearly perceive task completion, significantly enhancing their dominance and achievement.
% 即时反馈机制在交互设计中扮演了重要角色。通过视觉(如按钮高亮)、声音(如提示音)或触觉(如振动)反馈机制,用户能够及时感知操作结果,从而显著减少不确定性带来的情绪压力(Wang, 2024)。触觉反馈尤其重要,微振动或温暖的触感能够通过激活感官运动系统,增强用户对操作的身体感知,同时提升直观性和舒适感(Olugbade et al., 2023)。例如,在导航应用中,点击确认的触觉振动与视觉高亮结合,能够帮助用户明确任务的完成状态,从而显著提升控制感和成就感。

Exploratory interaction design, by offering freedom and reward feedback, enhances users’ emotional arousal and significantly stimulates their exploratory motivation. For example, features like map zooming and interface dragging allow users to freely adjust perspectives and explore system functionalities, enhancing their dominance over tasks and positive emotional experiences \cite{amoor2014designing}. Additionally, reward-based feedback (e.g., visual cues or sound effects upon task completion) activates the brain’s reward center, further enhancing users’ sense of achievement and engagement. This design is particularly effective in gamified interfaces or learning platforms, motivating users to continue interacting and enriching the depth and breadth of their emotional experience~\cite{wang2024enhancing}.
% 探索性交互设计通过提供自由度和奖励反馈,不仅提高了用户的情绪唤起水平,还显著激发了探索动机。例如,地图缩放和界面拖动等功能设计,能够让用户自由调整视角并探索系统功能,增强了对任务的掌控感和积极情绪体验(Amoor Pour, 2014)。此外,奖励式反馈(如任务完成后的视觉提示或音效)通过激活大脑的奖励中枢,进一步增强了用户的成就感和参与积极性。这种设计在游戏化界面或学习平台中尤为有效,通过激励用户持续操作,提升了情感体验的深度与广度(Wang, 2024)。

Interaction methods, by optimizing operational fluidity, feedback mechanisms, and emotional incentives, can significantly enhance users’ emotional experiences and behavioral performance across multiple dimensions. Designers, by refining interaction details, can effectively activate users’ sensory-motor systems, neural systems, cognitive systems, and motivational systems, achieving emotional regulation and behavioral guidance in various contexts. This user-centered design philosophy not only enhances the usability of the system but also provides a more systematic theoretical foundation and practical guidance for emotional design.
% 交互方式通过优化操作流畅性、反馈机制和情感激励,能够在多维度上显著提升用户的情绪体验和行为表现。设计师通过细化交互细节,可以有效激活用户的感官运动系统、神经系统、认知系统和动机系统,在不同情境中实现情绪调控和行为引导的目标。这种以用户为中心的设计理念,不仅增强了系统的使用价值,也为情感化设计提供了更加系统化的理论依据和实践指导。

\subsubsection{Motion Effects}
\begin{wrapfigure}{l}{0.06\textwidth}
  %\begin{center}
  %\vspace{-11pt} % 调整垂直位置
        \includegraphics[width=0.07\textwidth]{figs/icon/motion_effects.png}
  %\end{center}
\end{wrapfigure} 
Motion effects, as an indispensable element of interaction design, significantly influence users’ emotional experiences and behavioral responses through characteristics such as speed, path, direction, and complexity.
% 动效作为交互设计中不可或缺的元素,通过速度、路径、方向和复杂度等特性显著影响用户的情绪体验和行为反应。
The speed of motion effects is a crucial factor in influencing the level of emotional arousal. Fast motion effects (such as rapid page transitions or zooming) can stimulate the visual tracking system and the sensory-motor system, enhancing dynamic perception intensity, thereby increasing emotional arousal levels. These effects are suitable for attracting attention or creating a tense atmosphere \cite{hanjalic2005affective}. Conversely, slow motion effects (such as pages gradually unfolding or smooth loading) extend users’ emotional perception time, creating a calm or contemplative atmosphere. These low-dynamic effects can activate the parasympathetic nervous system, reducing users’ physiological tension levels, which helps enhance emotional valence in meditation or relaxation contexts \cite{wollner2018slow}.
% 动效的速度 是影响情绪唤起水平的重要因素。快速动效(如页面快速切换或缩放)能够激发视觉追踪系统和感官运动系统,提升动态感知强度,从而增强情绪唤起水平,适用于吸引注意力或营造紧张氛围的场景(Hanjalic & Xu, 2005)。相反,慢动作动效(如页面缓缓展开或平滑加载)通过延长用户的情绪感知时间,营造出平静或沉思的氛围。这类低动态动效能够通过激活副交感神经系统,降低用户的生理紧张水平,有助于在冥想或放松情境中提升情绪效价(Wöllner et al., 2018)。

The path and direction of motion effects profoundly influence user emotions. Linear motion conveys stability and reliability, enhancing users’ trust in the interface and their emotional valence. Conversely, curved or irregular paths may evoke tension and unease, making them suitable for warning or emergency designs \cite{lockyer2012affective}. Additionally, the design of motion direction also impacts users’ emotional perception. Inward motion typically conveys attraction and affinity, helping to enhance user engagement. Outward motion, on the other hand, often evokes a sense of detachment, making it suitable for undo notifications or exit~animations\cite{lockyer2012affective}.
% 动效的路径和方向 也深刻影响用户情绪。直线运动传递稳定性和可靠性,提升用户对界面的信任感和情绪效价;而曲折或不规则路径可能引发紧张和不安情绪,适合警示或紧急提示的设计(Lockyer & Bartram, 2012)。此外,运动方向的设计也会影响用户的情绪感知。向内运动通常带来吸引力和亲和感,有助于增强用户的参与度;而向外运动则容易引发疏离感,适用于撤销提示或结束动画。

Animation complexity plays a critical role in emotional regulation. Moderately complex animation designs can activate the cognitive system by capturing attention, optimize information processing efficiency, and enhance users’ emotional engagement \cite{yoo2005processing}. For example, progressively unfolding interface animations can optimize information flow and enhance memory retention. However, excessive complexity may increase cognitive load, leading to frustration and emotional stress, especially in multitasking scenarios where it can reduce emotional valence \cite{hanjalic2005affective}.
% 动画复杂度 在情绪调节中发挥关键作用。适度复杂的动画设计能通过吸引注意力激活认知系统,优化信息处理效率,并增强用户的情绪参与感(Yoo & Kim, 2005)。[5]例如,渐进式展开的界面动画能够优化信息流通并增强记忆效果。然而,过高的复杂度可能增加信息负荷,引发挫败感和情绪压力,尤其在多任务场景中容易降低情绪效价(Hanjalic & Xu, 2005)。

Animation effects can also activate the user’s motivation system, enhancing emotional engagement and behavioral intention. For example, high-motion designs (e.g., hover buttons or layered animations) stimulate exploratory and achievement motivation by enhancing dynamic effects, increasing user engagement and task completion rates \cite{wollner2018slow}. Dynamic navigation effects (e.g., auto-scrolling or page transitions) leverage instant feedback mechanisms to enhance users’ perception of task completion status, further increasing their dominance and~confidence.
% 动效还能够激发用户的动机系统,增强情绪参与感和行为意愿。例如,高动感设计(如悬浮按钮或层叠动画)通过增强动态效果激发探索动机和成就动机,提升用户参与意愿和任务完成度(Wöllner et al., 2018)。动态导航效果(如自动滚动或页面过渡)通过即时反馈机制,增强用户对任务完成状态的感知,进一步提高控制感和信心。

Animation design can optimize emotions for various scenarios through the adjustment of speed, path, direction, and complexity. For instance, high-dynamic effects are suitable for scenarios requiring heightened emotional arousal or high engagement, while low-dynamic effects and slow motion are better for conveying calmness and contemplative emotions. The adjustment of animation complexity requires a balance between emotional stimulation and information processing to maximize user acceptance and content retention. By designing animation characteristics appropriately, designers can effectively enhance the emotional dimensions of user experience.
% 动效设计通过速度、路径、方向和复杂度的调控,可以实现针对不同场景需求的情绪优化。例如,高动态效果适用于需要提升情绪唤起的紧张或高参与度场景,而低动态效果和慢动作更适合传递平静与沉思的情绪状态。动画复杂度的调节需要在情绪刺激和信息处理之间找到平衡点,以最大化用户的接受度和内容记忆效果。通过合理设计动效特性,设计师能够有效提升用户体验的情感维度。

\subsubsection{Navigation design}
\begin{wrapfigure}{l}{0.06\textwidth}
  %\begin{center}
  \vspace{-11pt} % 调整垂直位置
        \includegraphics[width=0.07\textwidth]{figs/icon/Navigation_design.png}
  %\end{center}
\end{wrapfigure} 
Navigation design profoundly impacts users’ emotional valence, emotional arousal, and motivation system activation by optimizing information architecture and path guidance. A streamlined navigation structure, dynamic navigation elements, and consistent design collectively contribute to smoother user experiences and stronger positive emotional~feedback.
% 导航设计通过优化信息架构和路径引导,对用户情绪效价、情绪唤起以及动机系统的激发产生深远影响。简洁的导航结构、动态导航元素和一致性设计共同作用,帮助用户获得更流畅的操作体验和更强的情绪正向反馈。

Simplified navigation is a key factor in enhancing users’ emotional valence. Clear menu hierarchies and intuitive structures help users quickly locate target information, reducing exploration time and operational difficulty, thus bringing joy and satisfaction \cite{wang2024enhancing}. For example, on e-commerce platforms, users can quickly find desired products through one-click filtering functions, significantly reducing operational burdens and enhancing trust and willingness to use the platform \cite{abdelaal2023accessibility}. In contrast, multi-layered nested menus or hidden navigation controls may increase users’ cognitive load and frustration, particularly when users struggle to adapt quickly, making such emotional reactions especially pronounced \cite{sheng2012effects}.
% 简洁导航 是提升用户情绪效价的关键因素。清晰的菜单层级和直观的结构能够帮助用户快速定位目标信息,减少探索时间和操作难度,从而带来愉悦和满足感(Wang, 2024)。例如,在电商平台中,用户通过一键筛选功能快速找到所需商品,可以显著降低操作负担,提高对平台的信任感和使用意愿(Abdelaal & Al-Thani, 2023)。与此相对,多层嵌套菜单或隐藏式导航控件可能增加用户的认知负荷和挫败感,尤其当用户无法快速适应时,这种情绪反应尤为显著(Sheng et al., 2012)。

Dynamic navigation design plays a significant role in enhancing user engagement and emotional arousal levels. For instance, the expansion effect of dynamic menus or sliding transitions in path selection enhance users’ perception of task status through the integration of visual and tactile elements \cite{sundar2014user}. These designs also activate the sensory-motor system, enhancing users’ dominance and interaction fluency through dynamic visual stimuli and tactile feedback, thereby making the operation process more enjoyable \cite{lockyer2012affective}. The advantage of dynamic navigation lies in its ability to balance visual appeal and functionality, aiding users in focusing on their task~objectives.
% 动态导航设计 在提升用户参与感和情绪唤起水平方面具有显著作用。例如,动态菜单的展开效果或路径选择中的滑动过渡,通过视觉与触觉的结合,增强了用户对任务状态的感知(Sundar et al., 2014)。这些设计还激活了感官运动系统,通过动态视觉刺激和触觉反馈,提升用户的控制感和交互流畅性,从而使操作过程更加愉悦(Lockyer & Bartram, 2012)。动态导航的优势在于其能够平衡视觉吸引力与功能性,帮助用户专注于任务目标。

Consistency design is the foundation for building a reliable navigation system. Navigation layouts and intuitive operational logic aligned with users’ cognitive models can significantly enhance their familiarity and trust in the system \cite{abdelaal2023accessibility}. For instance, common menu categorization and consistent navigation styles can reduce users’ learning costs, decrease uncertainty during operations, and enhance their dominance over the system. Conversely, navigation anomalies such as dead links or inactive buttons may undermine users’ trust in the interface, causing emotional unease and a sense of loss of control, thereby reducing engagement willingness \cite{sheng2012effects}.
% 一致性设计 是构建可靠导航系统的基础。符合用户认知模型的导航布局和直观的操作逻辑能够显著提升用户的熟悉感和信任感(Abdelaal & Al-Thani, 2023)。例如,常见的菜单分类和一致的导航风格能够降低用户的学习成本,减少操作中的不确定性,并提高用户对系统的控制感。反之,导航中的异常(如死链接或无效按钮)可能破坏用户对界面的信任,引发情绪上的不安和失控感,从而降低参与意愿(Sheng et al., 2012)。

Exploratory navigation design stimulates the activity of the motivational system. For example, navigation designs that allow users to drag, zoom, or expand multi-layer information maps create more opportunities for active participation and exploration. These designs enhance operational enjoyment and a sense of accomplishment, encouraging users to continue interacting \cite{amoor2014designing}. Additionally, the predictability and logic of navigation can boost users’ confidence in completing tasks, thereby enhancing operational enjoyment and motivation.
% 探索性导航设计 激发了动机系统的活跃。例如,允许用户自由拖动、缩放或展开多层信息地图的导航设计,为用户创造了更多主动参与和探索的机会。这类设计通过增加操作的趣味性和成就感,使用户更愿意继续互动(Amoor Pour, 2014)。此外,导航的可预测性和逻辑性也能增强用户对任务完成的信心,从而提升操作的愉悦感和动机。


% \subsubsection{Summary}

\subsection{General Discussion}
Our
study provided empirical evidence that the intervention improved some aspects of users' general media literacy, their behavioral intentions to use different strategies for checking misinformation and their performance in discriminating between fake and real news. However, it did not substantially enhance players’ prosuming skills, despite the mechanics of the game being tailored to support content generation. One possible explanation of this result would be that pre-survey data indicated that participants %already 
entered the study with high levels of confidence in their prosuming abilities. This "ceiling effect" likely constrained measurable gains, as participants had or perceived less room for improvement.
Another explanation is that high-level skills of critical prosuming require collaborative efforts and collective intelligence between participants \cite{lin2013understanding}, which was not possible to achieve in the PvP model of the game. Finally, the game only provided four rounds of content creation for each person, which could be sufficient to apply a critical perspective on the content, but not enough to
train creation effectiveness.

Another important finding is that, similarly to the previous studies of another game-based intervention \cite{leder2024feedback} we did not find significant effects from the intervention on the person's self-confidence in tackling misinformation.
Interestingly, game log data indicated that players' proficiency improved during gameplay; by the third and fourth rounds, they typically produced longer and more comprehensive messages compared to the initial rounds. Social Cognitive Theory \cite{bandura1997self}. may explain this discrepancy. According to it, enactive mastery experiences — ie., successful task completion— are the most influential sources of self-efficacy\cite{bandura1997self}. Positive experiences bolster self-efficacy, while repeated failures undermine it. In our game, success was determined by evaluations from LLM-simulated public opinion. The qualitative results revealed that the players found that no single strategy was effective for all characters, requiring frequent adjustments based on the unique characteristics of each character. This unpredictability made it difficult to achieve consistent success. While players developed greater proficiency during the game, the difficulty in achieving consistent success may have limited their perceived self-efficacy. However, this outcom can also be considered through the lens of the educational effects of the game. Previous studies about differential challenges of misinformation showed that young adults %might lack the critical thinking skills needed to assess information effectively and 
often overestimate their ability to assess information effectively\cite{papapicco2022adolescents,porat2018measuring}; in this context, the reaction of our participants, most of whom were young adults could be a positive signal that they became aware of the complexity of misinformation and the absence of one-size-fit-all solutions. 
%I deleted this explanation
%Furthermore, as character evaluations responded to the actions of both players, scores fluctuated frequently, with the gains of one player often neutralized by the moves of their opponent. This dynamic likely hindered the confidence of the players in their ability to discern misinformation successfully, even though their actual performance improved, as evidenced by significant gains in discriminative test results.Thus, while players developed greater proficiency during the game, the lack of sustained clear success may have limited their perceived self-efficacy. 

%For example, in our qualitative findings, participants reported scepticism toward information that purely relied on authoritative sources because they found their opponents used fabricated evidence from these sources to gain LLM-simulated characters’ trust and increase scores. This gameplay experience reminded them of real-life situations, where misinformation often exploits trust by citing credible authorities. As a result, players learned to examine the intent behind messages instead of automatically trusting authoritative sources.
%\textcolor{blue}{This evolution in how participants evaluated information aligns with the Elaboration Likelihood Model (ELM)\cite{petty1984source}, which describes how individuals process persuasive messages via central(critical evaluation of content) or peripheral routes(reliance on heuristic cues). In the gameplay, participants appeared to shift from peripheral processing (trusting authority as a shortcut) to central processing (evaluating the intent and content of the message) when exposed to the misuse of authority in gameplay. Notably, many participants suggested cross-checking sources as a practical solution for verifying information, demonstrating their enhanced critical thinking skills}.

Lastly, In our qualitative findings, participants reported skepticism toward information that purely relied on authoritative sources because they found their opponents used fabricated evidence from these sources to gain the trust of the LLM-simulated characters, with a view to increasing their game score. This gameplay experience reminded them of real-life situations, where misinformation often exploits trust by citing credible authorities. As a result, players learned to examine the intent behind messages instead of automatically trusting authoritative sources. This evolution in how participants evaluated information aligns with the Elaboration Likelihood Model \cite{petty1984source}, which describes how individuals process persuasive messages via central (critical evaluation of content) or peripheral routes (reliance on heuristic cues). In the gameplay, participants appeared to shift from peripheral processing (trusting authority as a shortcut) to central processing (evaluating the intent and content of the message) when exposed to the misuse of authority in gameplay.
Furthermore, our findings align with studies showing that gamified inoculation techniques for pre-bunking misinformation can trigger skepticism related to both false and real news\cite{hameleers2023intended,modirrousta2023gamified}. While this induced skepticism might seem limiting —potentially thus undermining trust in high-credibility sources— we believe it supports the goal of fostering critical media literacy. By encouraging players to evaluate the trustfulness of content, source and intent,  the game develops essential skills for navigating today's complex information landscape. Rather than promoting cynicism, this skepticism cultivates constructive inquiry that helps individuals better discern reliable information. Notably, many participants pointed to the need to cross-check sources as a practical solution for verifying information, demonstrating their enhanced critical thinking skills.  
%Additionally, recent game interventions have explored ways to improve individuals' ability to assess both real and false information. Strategies include teaching players to identify credible versus low-credibility media \cite{micallef2021fakey}, rewarding accurate judgments \cite{barzilai2023misinformation}, and incorporating feedback tests after gameplay \cite{leder2024feedback}. }

\subsection{Outcomes of the Game Mechanics}

\subsubsection{PvP model for Media-Literacy Game}
Unlike prior misinformation education game which broadly employ single-player mechanics \cite{roozenbeek2019fake,camCambridgeGame,harmonysquare,jeon2021chamberbreaker}; "Breaking the News" applies PvP mechanics. While in general, previous studies in serious games showed that PvP games are more engaging and motivating \cite{cagiltay2015effect}; at the same time in competitive environments, the motivation to “win” may overshadow educational goals.  In addition, in recent studies of another misinformation game it showed that some students may not find %gamification or
competition enjoyable or motivating\cite{axelsson2024bad}. 
%our results also indicate players learnd from the opponents. 
Our results demonstrated that, in general, our participants were highly motivated to play one against another. Moreover, they indicated that they learned from each other's strategies, and it helped them better understand the dynamic of misinformation. Therefore, we can conclude that, in our case, this approach was beneficial to fulfilling the intended purpose of the game. 
However, our study was conducted on East Asian participants, who are part of a collectivist-oriented rather than competitive culture \cite{chung1999social}. Previous studies have shown that cultural factors play an important role in the degree of competitiveness in gamified interventions \cite{oyibo2017investigation}. Therefore, it is also possible that in other cultural settings, the game's incentives can trigger more intense competition, which can negatively affect educational results. 


%These games frame the narrative as either misinformation creator or misinformation debunker and asks the player to adopt one of the roles, tasked by discerning misinformation using choice-based interaction. 


%\subsubsection{Free-form Input Generation}
%\textcolor{blue}{One notable feature of Breaking the News is its open world narrative and free-form response format, which contrasts with the linear choice-based formats commonly used in prior game intervention [REF]. 
%Previous works in the field of role-playing games showed that this approach makes game interaction more natural and increases user engagement \cite{csepregi2021effect,ashby2023personalized}.


%This format positively influences players’ learning outcomes by enhancing engagement and replayability. 

%In choice-based formats,interactions are typically brief, requiring players to passively select predetermined options. This structure can leads to "guesswork," where learners may choose correct answers without fully understanding the underlying concepts. In contrast, free-form responses force players to actively reason and articulate original ideas, fostering deeper engagement and critical thinking. 

%Additionally, by allowing players to develop their own narratives and gameplay experiences, each playthrough feels unique, encouraging them to return to the game.

%This variability not only boosts intrinsic motivation but also contributes to the game's educational value. When players repeatedly engage with core mechanics through varied narratives, they continue practicing essential skills, thereby reinforcing and enhancing their learning outcomes\cite{kucklich2004play}}.

%\textcolor{blue}{An additional strength of this format lies in the autonomy it provides. Players can craft personalized narratives that align with their individual learning styles, a factor proven to result in better learning outcomes in serious game design\cite{hwang2012development}. Furthermore, the freedom to explore and shape their own gameplay experiences not only deepens engagement but also excites players, acting as a powerful motivator for repeated play\cite{ravyse2017success}. By promoting intrinsic motivation and repeated practice, the free-form and open-world design demonstrate their potential to enrich both player experience and educational effectiveness. Given that inoculation interventions often show diminishing effectiveness over time, developing an engaging, replayable game that can consistently reinforce players' resistance to misinformation.}

%Free form rewriting: 
\subsubsection{Free-form Input Generation}
One notable feature of Breaking the News is its free-form response format, which contrasts with the linear choice-based formats commonly used in prior game intervention\cite{roozenbeek2019fake,camCambridgeGame,harmonysquare,jeon2021chamberbreaker,micallef2021fakey}. 
Previous works in the field of role-playing games showed that this approach makes game interaction more natural and increases user engagement \cite{csepregi2021effect,ashby2023personalized}.
Based on our data, we can say that this format positively influences players’ learning outcomes by enhancing engagement and replayability. 
By allowing players to develop narratives themselves, each playthrough feels unique, encouraging players to return to the game. This variability not only boosts intrinsic motivation but also contributes to the game's educational value. When players repeatedly engage with core mechanics through varied narratives, they continue practicing essential skills, thereby reinforcing and enhancing their learning outcomes\cite{kucklich2004play}. 

In an educational setting, previous works showed that in choice-based formats, interactions are typically brief, requiring students to select predetermined options. This structure could lead to "guesswork," where learners may choose correct answers without fully understanding the underlying concepts. In contrast, free-form responses force students to actively reason and articulate original ideas, fostering deeper engagement and critical thinking \cite{bryfczynski2012besocratic}. This autonomy allows players to craft responses based on their understanding of their roles, making this another motivating factor for returning to the game\cite{ravyse2017success}.
 
%\textcolor{blue}{An additional strength of this format lies in the autonomy it provides, allowing them to craft responses based on their own understanding of their roles. For example, during gameplay,  we observed that one player in the journalist-debunker role chose to mimic social media comments rather than rely on scientific strategies, crafting responses that resembled typical user comments. The freedom to explore and shape their own gameplay experiences not only increase engagement but also motivates players to return\cite{ravyse2017success}. Given the memory-strengthening effects of repeated learning, requiring people to recall what they have learned helps them relearn and reinforce these lessons\cite{nader2009single}.Therefore, developing an engaging, replayable game can consistently reinforce players' resistance to misinformation.}
%Players can craft personalized narratives that align with their individual learning styles, a factor proven to result in better learning outcomes in serious game design\cite{hwang2012development}. Furthermore, 


\subsubsection{LLM-Powered Feedback}
%\textcolor{blue}{Another innovation of Breaking the News lies in its interaction and feedback mechanisms. The game incorporates LLM-based role-play as evaluators, creating an interactive feedback for players. LLMs demonstrate the ability to simulate human behavior and reactions, consistent with findings from prior research[REF]. To the best of our knowledge, this is the first attempt to integrate LLM role play into a misinformation education game, opening new possibilities for interactive learning experiences in media literacy game interventions.} 

%\textcolor{blue}{In the game, the LLMs mimic human reactions and behaviors by leveraging demographic, personality, behavioral, and psychological features towards misinformation. This approach provides a personalized gameplay experience, fostering greater engagement and enhanced learning outcomes compared to binary feedback mechanisms (e.g., true or false responses). Players interact with LLM personas by crafting strategies based on their perceived characteristics. When players’ actions get the expected results(eg, persona's reaction towards on the side), they experience a sense of achievement. Conversely, when the outcomes deviate from expectations, players adapt by employing alternative persuasive strategies tailored to the persona. This dynamic interaction encourages active learning, problem solving, and critical thinking. Notably, players exhibit higher levels of engagement with personas they found relatable due to personal experiences, focusing more attentively on their feedback. This level of engagement is difficult to achieve with traditional binary feedback mechanisms, which provide limited insights beyond correctness.}

%\textcolor{blue}{Feedback plays an important role in shaping game-based learning outcomes(REF). In our case, users observed that personas react more obviously to emotional elements. This is because persona’s feedback begin with emotional feelings (such as panic), followed by behaviors (like sharing news with friends or taking no action). These emotional responses also reflect their trust level scores. To get better game performance, players used various tactics to trigger emotions or address these emotions in gameplay. As a results, most participants reported that they learned to detect emotional manipulation techniques. To further enhance learning outcomes, future games could adjust LLM outputs to teach diverse strategies more effectively. }
%Another innovation of Breaking the News is the interaction and feedback component. Compared to choice-based format where player present predetermined options and get binary and concise feedback(eg, true or false)Players tasks actively craft their responses and then receive evaluation by five different LLM-stimulated personas. These personas are based on their characteristics, offering unique feedback toward the player’s output. By engaging with these nuanced evaluations, players gain insights into why certain arguments might appear trustworthy or untrustworthy, encouraging them to refine their strategies dynamically. Furthermore, Since the game is last multiple rounds, players will get to know personas and adjust their misinformation narratives or debunking approaches based on the personas’ reactions. Thus, they learn and practice various tactics to identify and debunk misinformation.
%innovation in LLM as evaluator in game and other settings. 
%feedback in serious game's advantage and disadvantage

Another innovation of Breaking the News lies in incorporating LLM-based "characters" as evaluators, creating interactive feedback for players. Although there are other attempts to incorporate LLMs into helping users
learn about misinformation \cite{danry2023don, hsu2024enhancing} (including gamified attempts\cite{tang2024mystery}), in these approaches, AI was mostly the source of correcting information. While this approach has benefits, it is can also be criticised due to  the possibility of LLMs creating incorrect by plausible text \cite{kim2024can,agarwal2024faithfulness}. In this case, it is possible that the intervention will disinform people to an even greater extent. In contrast, in our study, we incorporate AI not as a source of information but as non-playable characters with their own opinions. Therefore, the educational part of the game becomes more robust to resist the negative effects of erroneous generations, as they can only affect the opinion of the "character" but not the main narrative of the game. In general, LLMs demonstrate the ability to simulate human behavior and reactions, consistent with findings from prior research\cite{1park2023generative}; we also find, that the system is capable of emulating the opinion of 5 different characters at the same time and still keeping them consistently different during the game.
 
One advantage of dynamic LLM feedback is that it achieves greater engagement compared to binary feedback (e.g., true or false). Based on our observation, players adapted by employing alternative persuasive strategies tailored to the character and concentrated on the character's feedback. Notably, players exhibit higher levels of engagement with characters they found relatable due to personal experiences, focusing more attentively on their feedback. This level of engagement is difficult to achieve with traditional binary feedback mechanisms, which provide limited insights beyond checking their correctness.%\textcolor{blue}{Feedback plays an important role in shaping game-based learning outcomes(REF). In our case, users observed that characters react more obviously to emotional elements. This is because character’s feedback begin with emotional feelings (such as panic), followed by behaviors (like sharing news with friends or taking no action). These emotional responses also reflect their trust level scores. To get better game performance, players used various tactics to trigger emotions or address these emotions in gameplay. As a results, most participants reported that they learned to detect emotional manipulation techniques.} %To further enhance learning outcomes, future games could adjust LLM outputs to teach diverse strategies more effectively.

%In the game, the LLMs mimic human reactions and behaviours by leveraging demographic, personality, behavioural, and psychological features towards misinformation. Based on our observation, players were able to catch these different clues and crafted strategies based on these characteristics; players adapted by employing alternative persuasive strategies tailored to the character and concentrated on the character's feedback. This level of engagement is difficult to achieve with traditional binary feedback mechanisms, which provide limited insights beyond correctness.

\subsection{Design Implications for the Further Development of Serious Games with LLM-components}
Based on our experience designing and testing the game, we summarize the following recommendations to practitioners working on similar projects.

\subsubsection {Balance Between Freedom and Guidance}
In our game, we sought to challenge the existing choice-based approach in misinformation education games by providing users with free-form input. We found this approach triggers reflection, which helps to build hands-on experience and make the game more enjoyable. Yet, we also found that it relies on players’ existing knowledge of misinformation. For example, players might incorporate unverified information they’ve encountered on social media into the game, which is specifically problematic for the debunker role. While we provided the players with comprehensible instructions to guide their role's actions (how to act as a debunker or an evil influencer), it would be better to incorporate more context-specific tips in each stage of the game to help users explore different ways of winning the game and deepening their learning. We suggest using the approach used in the free-input educational interventions (e.g. \cite{bryfczynski2012besocratic}) to build clear, understandable criteria for free-form answers. These will not stop creativity but help people tailor their answers to the context of the game. We also suggest adding a preliminary stage to the game in the form of a simple choice-based questionnaire, helping people understand the role and the guidelines for the role. For example, we can introduce to the individual playing the debunker role the Debunking Ethics Standards  \cite{edmo2024,afp2024,eeas2024}, and ask them a series of questions about understanding the content of the guidelines before starting the main game. In addition, we can incorporate an additional AI-based mechanic to check the quality of answers, not only in terms of its influentiality, but also of its coherence with debunking practices guidelines. Similarly, we can demonstrate a "Score of Manipulativeness" to the influencer role. This better shows us how the ideas players implement into the message can be judged, from the point of view of dealing with misinformation.
\subsubsection {Replayability and Feedback}
%LLM 
One of the critical challenges in serious games is maintaining replayability, as this is important to facilitate the learning process \cite{adetunji2024unlocking} and making interventions more sustainable \cite{silveira2016open}. Moreover, a lack of replayability in educational games can limit both educational and behavioral change \cite{epstein2021tabletop}. To address this, we incorporated elements designed to ensure the game can be played multiple times, such as offering two distinct player roles and providing a free-form input mechanism that allows participants to explore a broader decision space. However, we recommend that future interventions consider additional methods to further enhance replayability.

One such approach involves leveraging opportunities to introduce various characters to represent public opinion. In our game, we observed that players finds characters' feedback more engaging when this resonates with them personally. By introducing more characters, or by allowing players to customize characters to better reflect their own experience, the game could encourage players to return and interact with new characters. The result would be a more engaging experience. For instance, research has shown that debunking misinformation often occurs within families and can sometimes lead to conflict \cite{scott2023figured}. In such a scenario, players could customize a character based on their previous experiences with family members, thereby practicing their own debunking strategies in a risk-free environment.
%Character feedback can be designed to elicit deeper cognitive engagement and reflective thinking. For example, when a player attempts to spread rumors intended to evoke fear and anxiety, the character might respond with questions such as: “What evidence supports this claim?” “Why might it be spreading so rapidly?” or “What would happen if I believe this information?” Studies have shown that AI-framed questioning can stimulate user reasoning and enable independent information evaluation(REF).This approach can also motivate players to return to the game equipped with new strategies. Over multiple playthroughs, as players become more skilled in critical reasoning, they gain access to increasingly sophisticated dialogues and insights from the characters, further enhancing replayability.

While in-game feedback guides player behavior during play, post-game debriefing sessions help consolidate learning and improve future performance. Research has found that debriefing is a crucial opportunity for players to process and integrate their learning experiences\cite{crookall2014engaging,leder2024feedback,barzilai2024learning}. After gameplay, we suggest arranging debriefing sessions that allow players to review their strategies, assess their effectiveness, and receive constructive feedback, potentially improving learning outcomes. For instance, after a session focused on combating misinformation, a post-game review might present an ideal debunking response or a well-supported counterargument. Such structured reflection enables players to internalize lessons and increases the likelihood that they will re-enter the game with newly gained insights, thereby reinforcing both learning and replayability.
%For example, some characters aligned closely with players’ prior experiences, prompting players to focus on that particular feedback. In other cases, a character proved difficult to persuade, sparking the player’s interest in challenging that character and crafting more tailored responses, sometimes at the expense of others.

\subsection{Limitations and Future Work}
We acknowledge the following limitations. 
% I still think it's need cultural context in limiation
Our participants were mainly from an East Asian country, which can limit the generalizability of our findings. Previous studies have shown that cultural contexts influence individuals' perceptions of misinformation and interventions \cite{noman2024designing}. For example, study revealed that Mexican and Spanish users were more likely to trust -related misinformation compared to users from Ireland, the UK and the USA \cite{roozenbeek2020susceptibility}. At the same time, non-western participants demonstrated a significantly higher willingness to challenge misinformation. The effect of proposed misinformation design interventions was significantly greater than that seen in analysis of the behaviors and attitudes in the UK population\cite{noman2024designing}. 
Previous studies also suggest that certain populations may face greater challenges in being able to critically evaluate information. For example, a large-scale study observed that Asian individuals encounter more difficulties in assessing health information from social media compared to other populations \cite{chandrasekaran2024racial}. 
%This increased difficulty could be attributed to various factors, such as differing educational approaches[REF], access to resources[REF], or cultural norms surrounding information consumption. 
In addition, individuals from an Asian background are more likely to incorporate social media information into their health-related decisions, potentially increasing their susceptibility to misinformation \cite{chandrasekaran2024racial}. Therefore, future work should explore the cross-cultural applicability of our findings in different cultural contexts. However, as the cultural background of our participants can potentially make them more vulnerable to misinformation than other populations, we believe that our intervention is valuable as providing insight into the outlook, activity and the relevance of certain interventions for these populations.
%Serious games have the potential to address this challenge by geographic and cultural boundaries. By tailoring narrative elements and characters to resonate with specific cultural contexts, these games can enhance both engagement and learning outcomes. In our game, we incorporated traditional medicine into the narrative(News in the game), knowing that it is a topic that resonates strongly with East Asian participants. For example, 48.4\% of Hong Kong residents reported using traditional medicine before the COVID-19 pandemic\cite{lam2021public}. Similarly, a national survey in South Korea found a 74.8\% prevalence of traditional medicine use overall\cite{ock2009use}.In China, traditional medicine is formally integrated into the healthcare system as one of the most widely practiced modalities\cite{chung2023implementation}.This cultural relevance provided a familiar backdrop for the game's content, allowing players to engage more intuitively and craft responses based on their lived experiences. 

%Since the game's news scenarios are easily modifiable, future adaptations could incorporate culturally specific settings to enhance relatability and impact for other audiences. By customizing content to align with different cultural contexts, the game can potentially improve its effectiveness in media literacy and reducing susceptibility to misinformation across diverse populations.}

%In a large-scale study of US populations, Chandrasekaran et al. \cite{chandrasekaran2024racial} found that people of Asian origin are more likely to incorporate social media information into their health-related decisions, potentially increasing their susceptibility to misinformation\cite{chandrasekaran2024racial}. At the same time, previous work showed that non-western participants demonstrated a significantly higher willingness to challenge misinformation and that the effect of proposed misinformation design interventions was significantly greater compared to the UK population.
%Taking together, the further studies required to determine if the intervention will be equally efficient on other ethnical groups, however, the current results showed that the game can be used to help the populations potentially more vulnerable to misinformation.
 
Our sample was relatively homogeneous in age. A recent meta-analysis of articles about different intervention approaches showed that neither age nor gender significantly impacts the effectiveness of media literacy interventions \cite{lu2024can}. However, previous work has suggested that media literacy interventions designed for certain age groups (e.g., older adults and adolescents) achieved greater effects\cite{moore2022digital,hartwig2024adolescents}. Future work should determine if our approach is efficient in other age groups of users and, if necessary, tailor scenarios to suit the various needs of different age groups.

Thirdly, our study provided only a one-time intervention and observed immediate learning effects; 
%we are unsure how long these effects will persist.
Previous work showed that even a one-time interaction with an educational game can provide long-term improvement in misinformation recognition. For instance, Maertens et al. tested the game "Bad News" and found that inoculation effects lasted for at least 13 weeks. This suggests the potential for the long-term effectiveness of active inoculation interventions with regular assessment\cite{maertens2021long}. Still, future research should include multiple time points to assess the long-term effectiveness of our game intervention. There should also be comparisons between one-time and multiple play sessions, with explorations of the impacts of players assuming different roles within the games.
%Suppose we had incorporated a control group or compared different types of training. In that case, we might have identified elements that could be better addressed through lectures, potentially reducing gameplay time and preventing participant exhaustion.
%two players learning diff things: spread vs control, diff play flow so how can you assume they played the same game?
In this study, each player was limited to a single role, either a misinformation creator or a debunker.  This resulted in different learning experiences depending on their assigned role. %While participants reported learning from their opponents, the educational outcomes between roles may have varied. 
The primary reason for not having role-switching in our study was the length of the game and its cognitive demands, which we feared would lead to player exhaustion if roles were switched mid-game. In future iterations, we aim to improve the design by allowing players to save their progress and switch roles during subsequent sessions. This could offer a more immersive experience, as players would gain perspectives from both the misinformation creator and debunker roles. Additionally, we aim to introduce new modes, such as a family mode, where players interact with two LLM-simulated characters. This option would reduce cognitive load while maintaining engagement. 
%how is this representing range of possible misinformation? survey is not comprehensive, should be finding that they learned about one subtopic but not another that was not addressed.

The current game is also limited in its sole focus on a pandemic scenario. In reality, misinformation spans multiple domains, with health-related misinformation able to influence political events such as elections. Our game only addressed text-based misinformation, while visual and video-based misinformation pose even greater challenges and are harder to detect. Future work could include multimedia content, such as images and videos, to more accurately simulate the diverse forms of misinformation that exist in the real world.

The current game approach may unintentionally foster skepticism toward both true and false news, a common issue in misinformation pre-bunking interventions\cite{hameleers2023intended,modirrousta2023gamified}. %Recent efforts aim to improve individuals' ability to assess information credibility. These include teaching players to distinguish between credible and low-credibility media \cite{micallef2021fakey}, rewarding accurate judgments \cite{barzilai2023misinformation}, and using feedback tests after gameplay \cite{leder2024feedback}. 
While we believe that the benefits of promoting critical thinking towards sources are very important in prebunking interventions,
%discuss feedback mechanisms in the design implications, 
we further recommend incorporating features that clearly differentiate high- and low-credibility sources during gameplay.

While LLMs like ChatGPT-4o were used to simulate human reactions in the game, these models do not fully replicate the complexities of human behavior. Human reactions are often nuanced and influenced by multiple factors, including culture, history, and personal experience. LLM-generated characters may oversimplify human emotions and fail to grasp the full context of certain situations. Despite us using an advanced ChatGPT-4o model, it may have missed these subtleties, leading to interactions that feel artificial or incomplete. For example, in the gameplay, players employ strong emotional manipulation strategies to provoke specific responses. However, the LLM-generated characters do not react as expected, where participants reported feeling frustrated when their emotional manipulation strategies did not yield the anticipated reactions. This can limit the diversity and depth of the simulated interactions and detract from the realism and fairness of the experience for certain audiences.

LLMs are trained on large datasets that may contain intrinsic biases, which can manifest themselves in unintended ways during gameplay\cite{kasneci2023chatgpt}. Studies have shown that LLMs are prone to so-called "hallucinations" and can reflect stereotypes or skewed perspectives, which could affect how certain characters respond in the game\cite{xie2024can}. For instance, if biased training data influence a simulated character’s reaction, it may inadvertently reinforce player stereotypes about certain groups of people. 

%Finally, the game was primarily designed in English, which may have affected immersion and comprehension for participants who were not fluent in the language.

%how is this representing range of possible misinformation? survey is not comprehensive, should be finding that they learned about one subtopic but not another that was not addressed.

%LLM do not truly represent the public opinion, biased, misrepresented.

%two players learning diff things: spread vs control, diff play flow so how can you assume they played the same game?


%\subsection{Future Work}
%There are several directions for future work. One potential expansion is to incorporate more diverse topics of misinformation beyond pandemics and health, such as political misinformation or climate change, to enhance replayability. Although misinformation tactics share common features across different domains, each topic presents unique characteristics and challenges. We have already demonstrated that games are an effective medium for confronting misinformation, and the mechanics of \textit{Breaking the News} are designed to simulate the complexity of misinformation in the real world. By changing the in-game news topics, we could tailor the educational impact for different audiences relatively easily.

%Another direction is to allow players to switch roles, experiencing both the misinformation creator and debunker perspectives. This could provide a fuller, more immersive experience. To further enhance the educational value and replayability, we also plan to incorporate more tools to help players investigate evidence. For instance, as in MathE, which designed artificial search engines for its game, we aim to introduce investigative techniques like lateral reading and source evaluation to support both player roles.

%Additionally, future work could explore different formats of misinformation within the game, including multimedia content like images and videos. This would address the limitation of focusing mainly on text-based misinformation and would reflect the growing challenges posed by visual misinformation in real life. Allowing players to create or debunk multimedia misinformation would make the game more relevant to the evolving nature of misinformation.
\section{Conclusion}
RPA evaluation lacks consistency due to varying tasks, domains, and agent attributes. Our systematic review of $1,676$ papers reveals that task-specific requirements shape agent attributes, while both task characteristics and agent design influence evaluation metrics. By identifying these interdependencies, we propose guidelines to enhance RPA assessment reliability, contributing to a more structured and systematic evaluation framework. \\


\noindent\textbf{Acknowledgements}\\
We would like to thank Ruoyan Chen for her valuable contributions in creating the figures and assisting with other graphical elements of this paper. This work was supported by the National Key Research and Development Program of China (2023YFB3107100), and China Postdoctoral Science Foundation(2023M732674).\\



%\section*{References} 
%\bibliographystyle{naturemag-doi}
\bibliographystyle{IEEEtran}
\bibliography{reference}

% 150–200 references are suggested. All referenced work should be accepted for publication or on recognized Archive databases. References should be given superscript numbers and cited sequentially according to where they appear in the main text, followed by those in the text boxes, figure captions and then tables (i.e. the order they appear in this template).
% Journal abbreviation in italics, volume number in bold. If there are six or more authors for a reference, only the first author should be listed, followed by 'et al.'. Journal name abbreviations are followed by a full stop. Please include full page ranges. \\

% \noindent Please do not cite web sites in the reference list — these should be listed separately, with the URLs in the Related Links section (see below).  If you are unsure if the website is a ‘Related link’ or should go in the ‘References’ section then identify if there is a publication date on the website.  If there is a publication date, it is likely that the item will be in the ‘References’ section.  If there is no publication date, it is likely that the item will be in the ‘Related Links’ section.\\

% \noindent When citing a book, please indicate if you are citing a specific page range or chapter.  Otherwise, we presume that you are citing the entire book.  \\

% \noindent \textbf{Highlighted references (optional)} Please select 5–-10 key references and provide a single sentence for each, highlighting the significance of the work.\\



% \noindent\textbf{Author contributions}\\
% Please describe the contributions made by each author.  Please use the initials of the individual author to explain these contributions.  These contributions are also required when you upload the files to our submission website.\\

% \noindent\textbf{Competing interests}\\
% Nature Journals require authors to declare any competing interests in relation to the work described. Information on this policy is available \href{http://www.nature.com/authors/policies/competing.html}{here}. \\


% \noindent\textbf{Supplementary information (optional)}
% If your article requires supplementary information, please include these files for peer-review. Please note that supplementary information will not be edited.



\end{document}


