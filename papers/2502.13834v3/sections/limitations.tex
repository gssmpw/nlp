\section{Limitations and Future Work}

While \name has shown significant promise in generating formal proofs for Olympiad inequalities, several avenues remain open for enhancement and expansion.

{\bf Automating the Formalization of Tactics. } Our framework currently relies on a set of manually crafted tactics for scaling and rewriting inequalities, such as various forms of AM-GM inequality. This manual effort may impact scalability, given that the effectiveness of our approach is closely tied to the breadth of available tactics. Future work could focus on automating the discovery, formalization, and proof of new tactics to expand the tactic library. Developing methods for automatic tactic generation would reduce human effort and enhance the framework's scalability and adaptability.


{\bf Enhancing the Reasoning Capabilities of LLMs. } We leverage the mathematical insights learned by LLMs in our framework, and there is potential to further improve their reasoning performance. One promising direction is to collect or generate additional formal inequality problems and their corresponding proofs to create a richer dataset for fine-tuning LLMs specifically for this task. Some existing techniques~\citep{li2025neuro} may be useful for generating diverse and high-quality problems to enhance the LLMs' capabilities in handling inequalities, leading to better overall performance.


{\bf Broadening the Application Domain. } While our framework currently focuses on Olympiad-level elementary algebraic inequalities, extending it to more complex problems, such as concentration inequalities in machine learning theory, presents an exciting avenue for future research. This would involve improving the symbolic solver to handle inequality structures that consist of infinite variables and higher-order concepts like expectations or variances. Developing efficient algorithms and symbolic reasoning methods for these advanced mathematical constructs could significantly broaden the applicability of our neuro-symbolic paradigm. Extending our approach to other mathematical domains holds great potential and is a promising direction for future work.
