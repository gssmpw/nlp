\section{Conclusion}
\label{sec:conclusion}
In this paper, we introduce a neuro-symbolic framework for generating formal proofs that integrates the mathematical intuition learned by LLMs with domain-specific insights encoded by symbolic methods, specifically focusing on the domain of Olympiad inequalities. We categorize the tactics used in inequality proofs into two types: scaling and rewriting. Symbolic methods are employed to generate and filter scaling tactics by applying a set of lemmas through mechanical symbolic reasoning. LLMs are leveraged to generate rewriting tactics, implicitly pruning the infinite number of equivalent transformations to a manageable set. We further combine symbolic tools with LLMs to prune and rank subgoals, enhancing the efficiency of proof search. Experiments on challenging inequalities from three problem sets show that our neuro-symbolic inequality prover \name significantly outperforms both LLMs and symbolic methods, demonstrating the effectiveness of the neuro-symbolic integration and laying a solid foundation for its adoption in broader domains.


\newpage

\section*{Acknowledgment}
We appreciate anonymous reviewers for their valuable comments and engaging discussions. This work was partially conducted during Zenan’s and Wen's internships at MSRA. It is supported by the National Natural Science Foundation of China (Grants \#62025202) and the Frontier Technologies R\&D Program of Jiangsu (BF2024059). Zhaoyu Li and Xujie Si were also supported, in part, by Individual Discovery Grants from the Natural Sciences and Engineering Research Council of Canada, and the Canada CIFAR AI Chair Program. Xian Zhang (\texttt{zhxian@microsoft.com}), Kaiyu Yang (\texttt{kaiyuy@meta.com}), and Xiaoxing Ma (\texttt{xxm@nju.edu.cn}) are the corresponding authors. 
