\begin{tcolorbox}[breakable, enhanced, colback=black!5!white, colframe=black!75!white,  title=A Failed Case of \dsp]
{\small
\textbf{Prompt:} 
Draft an informal solution similar to below.  The informal solution will be used to sketch a formal Lean proof. Here are some examples:

Example:

Informal:

(* \#\#\# Problem \{Problem-1\} \#\#\# Solution \{Solution-1\} *)

Informal:

(* \#\#\# Problem \{Problem-2\} \#\#\# Solution \{Solution-2\} *)

Informal: 

(* \#\#\# Problem

if $a, b, c$ are positive real numbers, then
$$\frac{a}{b}+\frac{b}{c}+\frac{c}{a} \geq \sqrt{\frac{a^{2}+1}{b^{2}+1}}+\sqrt{\frac{b^{2}+1}{c^{2}+1}}+\sqrt{\frac{c^{2}+1}{a^{2}+1}} .$$

\tcbline

\textbf{Informal solution:}

\#\#\# Solution

We will use the AM-GM inequality which states that for non-negative real numbers $x$ and $y$, the arithmetic mean is greater than or equal to the geometric mean: $\frac{x+y}{2} \geq \sqrt{xy}$.

First, let's consider the terms on the left-hand side one by one and apply AM-GM to each term along with its corresponding term on the right-hand side.

For the term $\frac{a}{b}$ and its corresponding term $\sqrt{\frac{a^{2}+1}{b^{2}+1}}$, we apply AM-GM to $\frac{a}{b}$ and $\frac{a}{b}\cdot\frac{1}{\frac{a^{2}+1}{b^{2}+1}} = \frac{b^{2}+1}{a}$:

$$
\frac{\frac{a}{b} + \frac{b^{2}+1}{a}}{2} \geq \sqrt{\frac{a}{b} \cdot \frac{b^{2}+1}{a}} = \sqrt{\frac{a^{2}+1}{b^{2}+1}}.
$$

Repeating this process for $\frac{b}{c}$ with $\sqrt{\frac{b^{2}+1}{c^{2}+1}}$ and $\frac{c}{a}$ with $\sqrt{\frac{c^{2}+1}{a^{2}+1}}$, we obtain:

$$
\frac{\frac{b}{c} + \frac{c^{2}+1}{b}}{2} \geq \sqrt{\frac{b^{2}+1}{c^{2}+1}},
$$
and
$$
\frac{\frac{c}{a} + \frac{a^{2}+1}{c}}{2} \geq \sqrt{\frac{c^{2}+1}{a^{2}+1}}.
$$

Adding these three inequalities together, we get:

$$
\frac{1}{2} \left(\frac{a}{b} + \frac{b}{c} + \frac{c}{a} + \frac{b^{2}+1}{a} + \frac{c^{2}+1}{b} + \frac{a^{2}+1}{c}\right) \geq \sqrt{\frac{a^{2}+1}{b^{2}+1}} + \sqrt{\frac{b^{2}+1}{c^{2}+1}} + \sqrt{\frac{c^{2}+1}{a^{2}+1}}.
$$

{\color{red} Notice that the left-hand side contains the terms $\frac{a}{b} + \frac{b}{c} + \frac{c}{a}$ and additional terms. Since $\frac{b^{2}+1}{a} + \frac{c^{2}+1}{b} + \frac{a^{2}+1}{c} > 0$ (because all terms are positive), we can drop these additional terms to strengthen the inequality:

$$
{\color{red} \frac{a}{b} + \frac{b}{c} + \frac{c}{a} \geq \sqrt{\frac{a^{2}+1}{b^{2}+1}} + \sqrt{\frac{b^{2}+1}{c^{2}+1}} + \sqrt{\frac{c^{2}+1}{a^{2}+1}}.}
$$}

Thus, we have shown that the original inequality holds for any positive real numbers $a, b, c$.*)

\tcbline

\textbf{Prompt:}

Translate the informal solution into a sketch of the formal Lean 4 proof. Add `suggest\_tactics' in the sketch whenever possible. `suggest\_tactics` will be used to call the automated theorem prover. Here are some examples:

Example:

Informal: 

(* \#\#\# Problem 

\{informal\_problem\} 

\#\#\# Solution 

\{informal\_solution\}

Formal: 

\{formal\_problem\}  *)

Informal: 

(* \#\#\# Problem 

\{informal\_problem\} 

\#\#\# Solution 

\{informal\_solution\}

Formal:

\tcbline



\textbf{Formal solution:}

\begin{lstlisting}
import Mathlib.Data.Real.Basic
import Mathlib.Tactic.Linarith
import Mathlib.Tactic.SuggestTactics

theorem inequality_amgm (a b c : ℝ) (ha : 0 < a) (hb : 0 < b) (hc : 0 < c) :
  (a / b) + (b / c) + (c / a) ≥
    (sqrt ((a^2 + 1) / (b^2 + 1))) + (sqrt ((b^2 + 1) / (c^2 + 1))) + (sqrt ((c^2 + 1) / (a^2 + 1))) :=
by
  -- Apply the Arithmetic Mean-Geometric Mean Inequality (AM-GM)
  have am_gm_ab := real.am_gm (a / b) ((b^2 + 1) / a),
  have am_gm_bc := real.am_gm (b / c) ((c^2 + 1) / b),
  have am_gm_ca := real.am_gm (c / a) ((a^2 + 1) / c),

  -- The AM-GM inequality gives us the following:
  -- am_gm_ab : (a / b + (b^2 + 1) / a) / 2 ≥ sqrt ((a / b) * ((b^2 + 1) / a))
  -- am_gm_bc : (b / c + (c^2 + 1) / b) / 2 ≥ sqrt ((b / c) * ((c^2 + 1) / b))
  -- am_gm_ca : (c / a + (a^2 + 1) / c) / 2 ≥ sqrt ((c / a) * ((a^2 + 1) / c))

  -- Now, we sum the three inequalities:
  linarith [am_gm_ab, am_gm_bc, am_gm_ca],

  -- Simplify the right-hand side to match the terms of the original inequality
  -- `suggest_tactics` could assist in automating algebraic simplifications here:
  suggest\_tactics,

  -- Conclude the proof using positivity of additional terms
  -- Positivity of terms like (b^2 + 1) / a and others should be straightforward
  have pos_terms : (b^2 + 1) / a + (c^2 + 1) / b + (a^2 + 1) / c > 0,
  { apply add_pos, -- breaking it down into multiple positive parts
    all_goals { apply div_pos; linarith } },

  -- Dropping the additional positive terms from the left-hand side to strengthen the inequality
  linarith [pos_terms],
  
\end{lstlisting}

\#\#\# Explanation of the Steps:

1) Initial Setup and Imports: We start by importing necessary modules from Mathlib, such as basic real number operations and the linarith tactic for linear arithmetic.

2) Statement of the Theorem: The theorem inequality\_amgm is stated for three positive real numbers $a$, $b$, and $c$. The assumptions ha, hb, and  hc enforce the positivity conditions.

3) ......

4) ......

\#\#\# Conclusion

This sketch outlines the formal structure of the proof and highlights where automation can assist.

}
\end{tcolorbox}