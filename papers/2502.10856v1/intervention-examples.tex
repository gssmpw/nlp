\section{Illustrative Examples of Responsible Computing Interventions}
\label{app:examples}

\subsection{Individual Assignments or problems}

\Cref{tab:indv-prob-ex} gives examples of individual assignments/problems (which were discussed in \Cref{sec:indiv-assignment}).

\begin{table}[]
    \centering
    {\renewcommand{\arraystretch}{1.2}%
    \begin{tabular}{|P{0.1}|P{0.09}|P{0.47}|P{0.17}|P{0.17}|}
    \hline
    \textbf{Assignment} & \textbf{Relevant Course (topics)} &\textbf{Description}& \textbf{Link} &\textbf{Related}\\
    \hline
    SQL Injection attack &
Databases &
This assignment  incorporates privacy, ethics and security into introductory databases courses with an activity that lets students experiment with an SQL injection attack.  The goal of the activity is to help students understand how the attack occurs and how it can harm others.  The activity also includes discussion questions related to the ethics surrounding securing data and user privacy.
&
 Allegheny College. Created by \href{https://github.com/GatorEthics/privacy}{ Jordan Wilson, Oliver Bonham-Carter}&
Allegheny College has a collection of \href{https://csethics.allegheny.edu/}{responsible Computing activities}\\
\hline
Developers as Decision Makers &
CS 1 (conditionals) &
Students develop a scoring algorithm to determine which classmates are prioritized for housing on campus. Students use a human-centered design process to reflect on the ways in which different scoring algorithms can advantage or harm different groups of people.
& Bucknell University. Created by \href{https://ethicalcs.github.io/#decision-makers}{Evan Peck}&
Peck has a whole collection of ethical \href{https://ethicalcs.github.io/}{reflection modules for CS 1}\\ 
\hline
Elections&
Data Structures (Binary Tree and Heaps)&
These assignments focus on the U.S. elections process in the context of an introductory data structures course.&
Created by \href{https://responsibleproblemsolving.github.io/#elections}{Suresh Venkatasubramanian, Sorelle Friedler, Seny Kamara, and Kathi Fisler}&
The same group has \href{https://responsibleproblemsolving.github.io/}{other responsible computing assignments} for data structures courses\\
\hline
Thesis Advisor Allocation &
Algorithms (Stable Matching Problem)&
Assignment in this folder asks students to design a stable allocation of students to thesis advisors and asks them to reflect on the implications of the their design choices.&
Haverford College. Created by \href{https://github.com/responsibleproblemsolving/algorithmdesign/tree/master/Chapter1_Introduction_Some_Representative_Problems/thesis_advisor_allocation}{Sorelle Friedler}&
This \href{https://github.com/responsibleproblemsolving/algorithmdesign/tree/master}{github page has collection of other responsible algorithms problems}\\
\hline
College Admissions Algorithms&
CS 1 (lists and functions)&
Create an automated system to process college applications&
University of Colorado Boulder. Created by \href{https://www.internetruleslab.com/ethicsbased-computer-science-assignments}{Natalie Garrett and Casey Fiesler}&
There are other \href{https://www.internetruleslab.com/ethicsbased-computer-science-assignments}{CS 1 assignments} from the same group\\
\hline
Algorithmic Fairness&
Algorithms&
Ausitn's cake cutting algorithm as an example of an inherently fair algorithm. Cake cutting algorithms consider the fair division of a resource that can be divided without losing value. Austin's algorithm guarantees a fair outcome even if the parties try to cheat&
University of Washington in St. Louis. Created by \href{https://www.cse.wustl.edu/~cytron/RCS/}{Ron K. Cytron, Tomas Larsen, Russell Scharf}&
The same group has \href{https://www.cse.wustl.edu/~cytron/RCS/}{other activities}\\
\hline
    \end{tabular}
    }
    \caption{Examples of Individual Assignments or problems.}
    \label{tab:indv-prob-ex}
\end{table}

\subsection{Self-contained Lesson/Module}

\Cref{tab:self-cont-module-ex} gives examples of self-contained lessons/modules (which were discussed in \Cref{sec:self-contained-module}).

\begin{table}[]
    \centering
    {\renewcommand{\arraystretch}{1.2}%
    \begin{tabular}{|P{0.1}|P{0.09}|P{0.47}|P{0.17}|P{0.17}|}
    \hline
    \textbf{Module} & \textbf{Courses} &\textbf{Description}& \textbf{Link} &\textbf{Related}\\
    \hline
Speculative Ethics Classroom Exercises&
Multiple&
This page contains information and resources about the Black Mirror Writers Room and teaching exercises for ethical speculation in computing.&
University of Colorado Boulder. Created by \href{https://www.internetruleslab.com/black-mirror-writers-room}{Casey Fiesler}&
\\
\hline
Ethical Implications of the Adoption of Facial
Recognition Technology role play&
Multiple&
Introduces  social responsibility via an activity that simulates a city hall discussion about adopting facial recognition technology,  The overall goal is to encourage students to evaluate the values of decision makers vs. those affected in the community as it relates to the adoption of technology.  The materials for the role play include a teaching plan, case, and discussion prompts&
Miami Dade College. \href{https://www.mdc.edu/entec/downloads/rpg-facial-recognition.pdf}{Ethical Implications of the Adoption of Facial Recognition Technology. School of Engineering and Technology}&
\\
\hline
An Introduction to Data Ethics&
Data Science courses&
This introductory ethics module for data science courses includes a reading, homework assignments, and case studies, all designed to spark a conversation about ethical issues that students will face in their role as data practitioners
&
Markkula Center for Applied Ethics, Santa Clara University. Created by \href{https://www.scu.edu/ethics/focus-areas/technology-ethics/resources/an-introduction-to-data-ethics/}{Shannon Vallor.}&
Markkula Center has other \href{https://www.scu.edu/ethics/focus-areas/internet-ethics/teaching-modules/ and more general resources: https://www.scu.edu/ethics/focus-areas/technology-ethics/resources/embedding-ethics-into-computing-curricula-resources-and-suggestions/}{resources on tech and ethics}
\\
\hline
A City Decides On Self-Driving Buses&
Multiple&
Student do role play as different stakeholder in a decision at the city level on whether the city will allow self driving cars or not&
Georgia Institute of Technology. Created by \href{https://sites.gatech.edu/responsiblecomputerscience/overview/}{Ellen Zegura, Jason Borenstein, Benjamin Shapiro, Amanda Meng, and Emma Logevall}&
The same group \href{https://sites.gatech.edu/responsiblecomputerscience/}{other role playing activities}
\\
\hline
Making Computing Anti-Racist&
First year seminar&
First-year University at Buffalo students accepted the challenge to spend two weeks of their semester imagining what it would take to build a world in which computing could become anti-racist. Starting with the specific case of the use of predictive policing algorithms, they proposed computational and non-computational solutions to the problems exacerbated by technology in society&
University at Buffalo. Created by \href{https://www-student.cse.buffalo.edu/cseimpossibleproject/}{Kenneth Joseph and Dalia Muller}&
\\
\hline
Crypto and Cypherpunk Ethics&
Security/ Cryptography&
This module draws upon cypherpunk and cypherpunk-related ethical analyses of cryptography to explore the ongoing debates involving personal privacy, national security, system/device security and the meaning of an open society. Through reading, discussion and small group work, students will develop conceptual and practical knowledge about the ethics of cryptography&
\href{https://www.bemidjistate.edu/academics/departments/mathematics-computer-science/rcs/teaching-modules/crypto-and-cypherpunk-ethics/}{
Bemidji State University}&
The group has other material related to \href{https://www.bemidjistate.edu/academics/departments/mathematics-computer-science/rcs/teaching-modules/}{teaching responsible CS}
\\
\hline
Value Sensitive Design&
Multiple&
Value Sensitive Design, or VSD, is a framework for integrating ethics and values into the design of technological systems. VSD is the unifying foundation for a interdisciplinary, College-wide effort to integrate ethics into undergraduate education in the Khoury College. &
\href{https://vsd.ccs.neu.edu/}{Northeastern University}
&\\
\hline
    \end{tabular}
    }
    \caption{Examples of Self-contained Lesson/Module.}
    \label{tab:self-cont-module-ex}
\end{table}

\subsection{Integrated Lesson/Module}

\Cref{tab:int-module-ex} gives examples of integrated lessons/modules (which were discussed in \Cref{sec:integrated-module}).

\begin{table}[]
    \centering
    {\renewcommand{\arraystretch}{1.2}%
    \begin{tabular}{|P{0.1}|P{0.09}|P{0.47}|P{0.17}|P{0.17}|}
    \hline
    \textbf{Name} & \textbf{Courses} &\textbf{Description}& \textbf{Link} &\textbf{Related}\\
    \hline
Inclusive HCI&
Human Computer Interaction (HCI)&
The aim of this exercise is to dispel the myth of the universal user and increase students’ awareness of accessibility challenges (and opportunities!) in human-computer interactions. Consisting of an individual homework assignment and an in-class group activity, it is meant to enrich and activate students’ awareness of exclusion in design and help them think through important philosophical issues in inclusive design.&
Georgetown University. Created by \href{https://ethicslab.georgetown.edu/blog/responsible-cs-exercise-inclusive-hci-now-available}{Ethics Lab}
&Ethics Labs has other \href{https://ethicslab.georgetown.edu/mozilla-grant}{related resources}\\
\hline
Embedded Ethics&
Integrated Lesson/Module&
Embeds “philosophers directly into computer science courses” through a repository of modules that include cases, teaching plans for discussing relevant ethical issues in various CS courses.  The Embedded Ethics project provides SEP curricula for numerous CS courses and can be adopted to provide pedagogy on ethics through an undergraduate program&
Harvard University. Created by \href{https://embeddedethics.seas.harvard.edu/}{Embedded EthiCS}
&Other schools now have similar programs now, E.g. at \href{https://embeddedethics.stanford.edu/}{Stanford University} 
\\
\hline
Building ethical guardrails for Technological products/artifacts from a stakeholder perspective&
Multiple &
A project that builds on decolonial and feminist pedagogical practices. Works as a project or assignment for graduate and undergraduate students. Works best if students have been exposed to literature or examples of algorithmic bias, exclusionary system or interface design, and/or social impacts from computing technology.&
U. Mass Amherst. Created by Michelle D. Trim and Paige Gulley~\cite{10.1145/3615335.3623037}&
\\
\hline

\hline
    \end{tabular}
    }
    \caption{Examples of Integrated Lesson/Module.}
    \label{tab:int-module-ex}
\end{table}

\subsection{Responsible Computing Theme}

\Cref{tab:rc-theme-ex} gives examples of responsible computing themes (which were discussed in \Cref{sec:RC-theme}).

\begin{table}[]
    \centering
    {\renewcommand{\arraystretch}{1.2}%
    \begin{tabular}{|P{0.1}|P{0.09}|P{0.47}|P{0.22}|P{0.12}|}
    \hline
    \textbf{Name} & \textbf{Courses} &\textbf{Description}& \textbf{Link} &\textbf{Related}\\
    \hline
Human Context and Ethics Toolkit&
Multiple&
A set of concepts and methods from Science, Technology, and Society (STS) and History selected to build understanding of the datafied world, helping students to identify where human power structures and value choices get built into technical work, and empowering them to discover how, when, and where they can responsibly and effectively intervene.&
University at California, Berkeley. Created by \href{https://data.berkeley.edu/hce-toolkit}{Margarita Boenig-Liptsin, Cathryn Carson} 
&\href{https://data.berkeley.edu/human-contexts-and-ethics}{Human Contexts and Ethics Program at Berkeley}\\ 
\hline
Access to High Speed Internet&
Algorithms&
There are multiple activities/assignments in CSE 331 that all have the common thread of accessing high speed Internet.&
University at Buffalo. Created by \href{https://c4sg.cse.buffalo.edu/projects/Teaching%20Responsible%20Computing.html#algorithms-course-cse-331}{Atri Rudra, CSE 331 (Algorithms and Complexity)}.&
\\
\hline
    \end{tabular}
    }
    \caption{Examples of Responsible Computing Themes.}
    \label{tab:rc-theme-ex}
\end{table}

\subsection{Dedicated Course}

\Cref{tab:dedicated-ex} gives examples of responsible computing themes (which were discussed in \Cref{sec:dedicated}).

\begin{table}[]
    \centering
    {\renewcommand{\arraystretch}{1.2}%
    \begin{tabular}{|P{0.1}|P{0.1}|P{0.6}|P{0.2}|}
    \hline
    \textbf{Course} & \textbf{Required vs. Elective} &\textbf{Description}& \textbf{Link}\\
    \hline
Algorithms for the People&
Elective&
The course explores the ways in which technology affects marginalized communities addressing issues such as geo-fencing, policing, human rights, and financial technologies.  The course is offered among electives for students concentrating in CS.  Algorithms for the People include assigned readings on course themes and a project that evaluates existing solutions or develop and analyze new algorithms that affect marginalized populations.&
Brown University. Created by Seny Kamara, \href{http://algosforthepeople.org/}{2950v-19. “Algorithms for the People.” Algorithms for the People, 26 June 2020}\\
\hline
Philosophy and Society and Ethics and Information Technology&
Required Elective&
Students are required to select among several SEP courses, including  Ethics; Philosophy and Society, and Ethics and Information Technology which are taught in the department of Philosophy.  The Ethics and Philosophy and Society courses cover traditional theories and analysis of applied ethics, while the Ethics and Information Technology course addresses “contemporary ethical issues concerning the use, misuse and development of information technologies”&
University of Colorado at Boulder. \href{https://www.colorado.edu/cs/academics/undergraduate-programs/bachelor-science/bachelor-science-degree-requirements#Logic}{Bachelor of Science Degree Requirements at UC Boulder}.
\\
Social and Ethical Issues in Information Technology&
Required&
incorporates ethical reasoning and the ACM Code of Ethics to examine a wide range of issues including privacy, security and encryption, intellectual property, censorship and computer crime. &
University of Maryland Baltimore County. \href{https://www.csee.umbc.edu/cmsc-304-syllabus/}{CMSC 304 Syllabus}
\\
\hline
Race, Gender, Class, \& Computing &
Elective&
This course explores the diversity, equity, and inclusion (DEI) challenges in computing through an introduction to and discussion of identity as a social construct, its impact on computing departments and organizations, and the resulting impact of technology on various identities.&
Duke. Created by \href{https://courses.cs.duke.edu/fall21/compsci240/}{Nicki Washington}
\\
\hline
Rage Against the Machine + Machine Learning and Society&
Elective&
Two concurrently run courses: Rage Against the Machine (History) and Machine Learning and Society (CSE) . Rage Against the Machine is a class that explores the history of white supremacy in and beyond the United States. Machine Learning and Society talks about the interaction of the ML pipeline with society. Both courses  had a common project (with team members from both courses) on “Ending White Supremacy”&
University at Buffalo. Dale Muller: \href{https://daliamul.wixsite.com/rage}{Rage Against the Machine}, Kenneth Joseph + Atri Rudra \href{http://www-student.cse.buffalo.edu/~atri/ml-and-soc/spr23/}{ML and Society}
\\
\hline
    \end{tabular}
    }
    \caption{Examples of Dedicated courses.}
    \label{tab:dedicated-ex}
\end{table}