\documentclass[sigconf]{acmart}


\usepackage{multirow}
\usepackage{longtable}
\usepackage{xspace}
\usepackage{wrapfig}
\usepackage{graphicx,subcaption,lipsum}


\usepackage{enumitem}

\newenvironment{blockquote}%
  {\list{}{\selectfont\itshape\leftmargin=0.25in\rightmargin=0.25in}\item[]}%
  {\endlist}

\AtBeginDocument{%
  \providecommand\BibTeX{{%
    \normalfont B\kern-0.5em{\scshape i\kern-0.25em b}\kern-0.8em\TeX}}}






\newcommand{\rev}[1]{#1}
\newcommand{\nathalie}[1]{}
\newcommand{\ken}[1]{}
\newcommand{\michel}[1]{}
\newcommand{\nic}[1]{}
\newcommand{\hugo}[1]{}
\newcommand{\fred}[1]{}
\newcommand{\dave}[1]{}



\newcommand{\SystemName}{SystemName\xspace}


\newcommand{\revision}[1]{\textcolor{red}{#1}}



\copyrightyear{2025} 
\acmYear{2025} 
\setcopyright{cc}
 \setcctype{by}
 \acmConference[CHI '25]{CHI Conference on Human Factors in Computing
 Systems}{April 26-May 1, 2025}{Yokohama, Japan}
 \acmBooktitle{CHI Conference on Human Factors in Computing Systems (CHI
 '25), April 26-May 1, 2025, Yokohama,
 Japan}
 \acmDOI{10.1145/3706598.3714259}
 \acmISBN{979-8-4007-1394-1/25/04}

 

\begin{document}





\title[AI-Instruments]{~\textit{AI-Instruments}: Embodying Prompts as Instruments to Abstract \& Reflect Graphical Interface Commands as General-Purpose Tools}

\author{Nathalie Riche}
\affiliation{
  \institution{Microsoft Research}
  \city{Redmond}
  \state{WA}
  \country{USA} 
}
\email{nath@microsoft.com}

\author{Anna Offenwanger}
\affiliation{
  \institution{Microsoft Research }
  \city{Redmond}
  \state{WA}
  \country{USA} 
}
\affiliation{
  \institution{Université Paris-Saclay, CNRS, Inria}
  \city{Orsay}
    \country{France} 
}
\email{anna.offenwanger@gmail.com}

\author{Frederic Gmeiner}
\affiliation{
  \institution{Microsoft Research}
    \city{Redmond}
  \state{WA}
  \country{USA} 
  }
  \affiliation{
  \institution{Carnegie Mellon University}
      \city{Pittsburgh}
  \state{PA}
  \country{USA} 
}
\email{gmeiner@cmu.edu}

\author{David Brown}
\affiliation{
  \institution{Microsoft Research}
  \city{Redmond}
  \state{WA}
  \country{USA} 
}
\email{dabrown@microsoft.com}

\author{Hugo Romat}
\affiliation{
  \institution{Microsoft}
  \city{Redmond}
  \state{WA}
  \country{USA} 
}
\email{romathugo@microsoft.com}

\author{Michel Pahud}
\affiliation{
  \institution{Microsoft Research}
  \city{Redmond}
  \state{WA}
  \country{USA} 
}
\email{mpahud@microsoft.com}



\author{Nicolai Marquardt}
\affiliation{
  \institution{Microsoft Research}
  \city{Redmond}
  \state{WA}
  \country{USA} 
}
\email{nicmarquardt@microsoft.com}

\author{Kori Inkpen}
\affiliation{
  \institution{Microsoft Research}
  \city{Redmond}
  \state{WA}
  \country{USA} 
}
\email{kori@microsoft.com}

\author{Ken Hinckley}
\affiliation{
  \institution{Microsoft Research}
  \city{Redmond}
  \state{WA}
  \country{USA} 
}
\email{kenneth.p.hinckley@gmail.com}




\renewcommand{\shortauthors}{Riche et al.}

\begin{abstract}

Chat-based prompts respond with verbose linear-sequential texts, making it difficult to explore and refine ambiguous intents, back up and reinterpret, or shift directions in creative AI-assisted design work. 
~\textit{AI-Instruments} instead embody “prompts” as interface objects via three key principles: (1)~\textit{Reification} of user-intent as reusable direct-manipulation instruments; (2)~\textit{Reflection} of multiple interpretations of ambiguous user-intents (\textit{Reflection-in-intent}) as well as the range of AI-model responses (\textit{Reflection-in-response}) to inform design "moves" towards a desired result; and (3)~\textit{Grounding} to instantiate an instrument from an example, result, or extrapolation directly from another instrument. Further, AI-Instruments leverage LLM’s to suggest, vary, and refine new instruments, enabling a system that goes beyond hard-coded functionality by generating its own instrumental controls from content. 
We demonstrate four technology probes\rev{, applied to image generation,} and qualitative insights from twelve participants, showing how AI-Instruments address challenges of intent formulation, steering via direct manipulation, and non-linear iterative workflows to reflect and resolve ambiguous intents. 





\end{abstract}

\begin{CCSXML}
<ccs2012>
<concept>
<concept_id>10003120.10003121.10003124</concept_id>
<concept_desc>Human-centered computing~Interaction paradigms</concept_desc>
<concept_significance>500</concept_significance>
</concept>
</ccs2012>
\end{CCSXML}

\ccsdesc[500]{Human-centered computing~Interaction paradigms}







\keywords{instrumental interaction, generative AI interfaces}

\begin{teaserfigure}
  \includegraphics[width=\textwidth]{Figures/Figure-1-camera-ready.pdf}
  \caption{AI-Instruments embody prompts as interface objects, informed by three principles: reification of user intent, reflection, and grounding (left). Visual overview of four technology probes of AI-Instruments -- generative containers, transformative lenses, prompt fragments, and fillable brushes (right).}
  \Description{Visual Abstract of the paper. Visual illustration of a person using AI-Instruments on the left, then a box including the three principles of AI-Instruments in the middle, and sketched illustrations of AI-Instruments on the right (using GUI elements and wireframes). The figure explains that AI-Instruments embody prompts as interface objects, informed by three principles (left): reification of user intent, reflection, and grounding. (right) Visual overview of four technology probes of AI-Instruments -- generative containers, transformative lenses, prompt fragments, and fillable brushes.}
  \label{fig:teaser}
\end{teaserfigure}



\maketitle



\section{Introduction}

Deep Reinforcement Learning (DRL) has emerged as a transformative paradigm for solving complex sequential decision-making problems. By enabling autonomous agents to interact with an environment, receive feedback in the form of rewards, and iteratively refine their policies, DRL has demonstrated remarkable success across a diverse range of domains including games (\eg Atari~\citep{mnih2013playing,kaiser2020model}, Go~\citep{silver2018general,silver2017mastering}, and StarCraft II~\citep{vinyals2019grandmaster,vinyals2017starcraft}), robotics~\citep{kalashnikov2018scalable}, communication networks~\citep{feriani2021single}, and finance~\citep{liu2024dynamic}. These successes underscore DRL's capability to surpass traditional rule-based systems, particularly in high-dimensional and dynamically evolving environments.

Despite these advances, a fundamental challenge remains: DRL agents typically rely on deep neural networks, which operate as black-box models, obscuring the rationale behind their decision-making processes. This opacity poses significant barriers to adoption in safety-critical and high-stakes applications, where interpretability is crucial for trust, compliance, and debugging. The lack of transparency in DRL can lead to unreliable decision-making, rendering it unsuitable for domains where explainability is a prerequisite, such as healthcare, autonomous driving, and financial risk assessment.

To address these concerns, the field of Explainable Deep Reinforcement Learning (XRL) has emerged, aiming to develop techniques that enhance the interpretability of DRL policies. XRL seeks to provide insights into an agent’s decision-making process, enabling researchers, practitioners, and end-users to understand, validate, and refine learned policies. By facilitating greater transparency, XRL contributes to the development of safer, more robust, and ethically aligned AI systems.

Furthermore, the increasing integration of Reinforcement Learning (RL) with Large Language Models (LLMs) has placed RL at the forefront of natural language processing (NLP) advancements. Methods such as Reinforcement Learning from Human Feedback (RLHF)~\citep{bai2022training,ouyang2022training} have become essential for aligning LLM outputs with human preferences and ethical guidelines. By treating language generation as a sequential decision-making process, RL-based fine-tuning enables LLMs to optimize for attributes such as factual accuracy, coherence, and user satisfaction, surpassing conventional supervised learning techniques. However, the application of RL in LLM alignment further amplifies the explainability challenge, as the complex interactions between RL updates and neural representations remain poorly understood.

This survey provides a systematic review of explainability methods in DRL, with a particular focus on their integration with LLMs and human-in-the-loop systems. We first introduce fundamental RL concepts and highlight key advances in DRL. We then categorize and analyze existing explanation techniques, encompassing feature-level, state-level, dataset-level, and model-level approaches. Additionally, we discuss methods for evaluating XRL techniques, considering both qualitative and quantitative assessment criteria. Finally, we explore real-world applications of XRL, including policy refinement, adversarial attack mitigation, and emerging challenges in ensuring interpretability in modern AI systems. Through this survey, we aim to provide a comprehensive perspective on the current state of XRL and outline future research directions to advance the development of interpretable and trustworthy DRL models.
\section{Related Work}

\subsection{CAD Generation}
The compilation of large CAD datasets in recent years~\citep{koch_abc_2019,willis_fusion_2021,jones_automate_2021,willis_joinable_2022} has inspired a wealth of research on synthesizing CAD models. These efforts fall into two broad categories; those which generate CAD geometry directly~\citep{willis_engineering_2021,guo_complexgen_2022,jayaraman_solidgen_2023,nash_polygen_2020,xu2024brepgen,liu2024point2cad}, and those which generate a \emph{procedure} that generates CAD geometry~\citep{wu_deepcad_2021,ellis_learning_2017, ellis_learning_2018, ganin_computer-aided_2021,ren_extrudenet_2022,li_secad-net_2023,xu_skexgen_2022,lambourne_reconstructing_2022,para_sketchgen_2021, seff_vitruvion_2022,willis_fusion_2021,ma2024draw,li2024sfmcad,khan2024cad}. A fundamental challenge with these tools is the ability to control the generation. While many methods can be conditioned on an input allowing for reverse engineering applications~\citep{lambourne_reconstructing_2022,guo_complexgen_2022}, the few methods that directly focus on generation give limited control over their output \citep{jayaraman_solidgen_2023, wu2021deepcad,xu2024brepgen,seff_vitruvion_2022}. The highest degree of control is afforded by those that take sketches as input, such as Free2CAD~\citep{li_free2cad_2022} but these are effectively reverse reverse engineering an existing geometric design rather than enabling high level guidance. The goal of AIDL is to enable control without direct geometric supervision, and to incorporate semantic understanding beyond that of existing CAD programs. We have thus chosen to design our system around \textit{general purpose} language models rather than CAD specific models, and focus on DSL design rather than the design or training of a generative model. Importantly, all prior works use CAD DSLs that have limitations when it comes to LLM needs, as we discuss in Section~\ref{sec:analysis_llm}.


%There have been several attempts in recent years to generate CAD models. Some attempt to directly generate symbolic geometry\cite{willis_engineering_2021,guo_complexgen_2022,jayaraman_solidgen_2023,nash_polygen_2020,xu2024brepgen,liu2024point2cad}, while others use procedural representations~\cite{wu_deepcad_2021,ellis_learning_2017, ellis_learning_2018, ganin_computer-aided_2021,ren_extrudenet_2022,li_secad-net_2023,xu_skexgen_2022,lambourne_reconstructing_2022,para_sketchgen_2021, seff_vitruvion_2022,willis_fusion_2021,ma2024draw,li2024sfmcad,khan2024cad}. Some procedural approaches employ symbolic techniques either alone or in concert with learned heuristics \cite{nandi_functional_2018,xu_inferring_2021,du_inversecsg_2018}. A fundamental challenge with these tools is the ability to control the generation. While many methods can be conditioned on an input allowing for reverse engineering applications~\cite{lambourne_reconstructing_2022,guo_complexgen_2022}, the few methods that directly focus on generation give limited control over their output \cite{jayaraman_solidgen_2023, wu2021deepcad,xu2024brepgen,seff_vitruvion_2022}. The highest degree of control is afforded by those that take sketches as input, such as Free2CAD~\cite{li_free2cad_2022} but these are effectively reverse reverse engineering an existing geometric design rather than enabling high level guidance. 

%While many generative and reconstructive models treat the procedural form of CAD models as linear output and rely on large-data statistics to take care of syntactic and semantic concerns, some works employ symbolic techniques directly, either alone or in concert with learned heuristics. ReIncarnate~\cite{nandi_functional_2018}, Zone Graphs~\cite{xu_inferring_2021}, and InverseCSG~\cite{du_inversecsg_2018} treat reconstruction as a synthesis problem given inferred initial geometry, but use different synthesis techniques; oracle guided heuristics, neurally guided search, and constraint guided synthesis. Program synthesis can also be used to improve the robustness of existing CAD programs.  The goal of AIDL is to enable control without direct geometric supervision, which is why we propose to use an LLM to create CAD models through code, and therefore focus on DSL design rather than the design or training of a generative model. Importantly, all prior work uses CAD DSLs that are have limitations when it comes to LLM needs as will be discussed and surveyed in details in Section~\ref{sec:analysis_llm}. With AIDL, we want to incorporate semantic understanding beyond that of existing CAD programs, and therefore design our system around \textit{general purpose} language models rather than CAD specific models.

%\cite{mathur_constraint_2021} synthesizes symbolic constraints on input parameters that exclude configurations which break the model. Some of these constraints are discoverable through static program analysis, but their work highlights the necessity of access to a CAD kernel to understand the semantics of a CAD program, necessitating \emph{dynamic} analysis as well. \cite{mathur_interactive_2020} improves robustness by replacing specific references with programs synthesized with a small program size heuristic, using the hypothesis that smaller programs will generalize better, thus capturing user intent.



\subsection{Code Generation with LLMs}

Software engineering has been one of the marquee applications of LLMs, so a detailed enumeration of works in the field is beyond the scope of this paper. We instead refer the reader to a survey \citet{zhang2024unifying}, and reserve this section to position AIDL within the space. The majority of research on using LLMs for coding focus on how to make LLMs work more effectively with existing programming languages. A popular approach is to specifically train or fine-tune a model on code repositories and coding specific tasks \citep{li_starcoder_2023,lozhkov_starcoder_2024,grattafiori2023code}, or more recently to use LLMs to generate higher complexity training examples \citep{xu_wizardlm_2023,luo_wizardcoder_2023}. Other approaches tackle prompt complexity through system design, exploring prompt engineering and multi-agent strategies for pre-planning or coordinating a divide-and-conquer strategy \citep{dong_self-collaboration_2023,bairi_codeplan_2023,silver_generalized_2023}. AIDL approaches LLM code generation from an entirely different perspective, by asking which \emph{language features} will best enable an LLM to work with a programming system. Most similar is BOSQUE, a proposed general purpose programming language \citep{marron_towards_2023}. In particular, BOSQUE's embrace of pre and post conditions mirrors AIDL's use of constraints and strong validation, but does not go so far as to employ a solver to enforce constraints.

% Most work on coding LLMs focuses on how to get an LLM to work with existing code and coding systems
% This is done by training models specifically for coding
% fine-tuning existing models for coding tasks
% or creating prompting / querying strategies to improve the performance of LLMs with code
% AIDL asks a different question; what can we do to a coding language to make it work better with an LLM? Most similar to this is the proposed BOSQUE language...

% Code Gen w/ LLMs
% - train models specifically on / for code
% - fine-tune existing models for code
% - prompting / generation strategies for code
% - design languages for use with LLM

%Code generation has been one of the headline applications of the recent LLM explosion. There are several coding-specific LLMs that have been specifically trained or fine-tuned on code repositories and coding specific tasks \cite{li_starcoder_2023,lozhkov_starcoder_2024,grattafiori2023code}. Generating solutions to complex prompts is difficult. Some works explore prompt engineering and multi-agent strategies for pre-planning or coordinating a divide-and-conquer strategy for complexity \cite{dong_self-collaboration_2023,bairi_codeplan_2023,silver_generalized_2023}, while more recently models have been fine-tuned on synthetic, complex examples generated by LLMs \cite{xu_wizardlm_2023,luo_wizardcoder_2023}. LLMs can perform worse generating code in DSL than general purpose languages since they will have fewer or no examples of these in their training data. For context-free languages, grammar prompting \cite{wang_grammar_2023} can constrain output to valid DSL expressions. This approach is not feasible for more complex languages like AIDL, however there are prompting strategies that can increase the likelihood of valid program structures \cite{jain_generating_2023}. Another recent work proposes an LLM specific general purpose language, BOSQUE \cite{marron_towards_2023}, that shares a similar the philosophy of AIDL to design the language around an LLM's strengths and weaknesses. In particular, BOSQUE's embrace of pre and post conditions mirrors AIDL's use of constraints and strong validation, but does not go so far as to employ a solver to enforce constraints.

%\paragraph{Code Generation}
%There is extensive body of work focusing on \em{program synthesis}, the task of generating programs that satisfy user-provided specifications. A number of approaches have been developed to solve the synthesis problem, including enumerative, constraint-based and stochastic search algorithms. However, traditional synthesis algorithms require \em{formal} specifications while our goal is to generate programs from natural language specifications.

\subsection{CAD DSLs}
\label{sec:background}
% We review existing CAD DSLs and give a high-level explanation of how they work.
While there are many CAD DSLs, they can be grouped intro three broad categories: %While these categories are not black and white, they emphasize the main features of their respective languages.
%More specifically, industry standard CAD systems tackle real-world problems with different DSLs for different stages of the design process.
%Whereas the modeling stage often uses a CSG or query-based DSL, the assembly stage uses a constraint-based DSL.

% \felix{Maybe put a figure with the same geometry, designed in all three different languages?} \maaz{This would be nice if we can afford the space but I don't think its necessary.}

\paragraph{Constructive Solid Geometry (CSG)}
% The first step of each CAD design is to create geometric primitives.
In CSG, users can specify 2D and 3D parametric primitives, such as rectangles or spheres, directly in global coordinates.
Using boolean operations, such as union or intersection, users then combine these primitives in a hierarchical tree structure to achieve complex designs.
%Examples of languages for this category are OpenSCAD, and Szalinski \cite{nandi2020synthesizing}.
While some CSG languages, such as OpenSCAD, allow the use of variables or expressions for primitive parameters, they do not support specifying relationships or dependencies between different parts of the geometry.
This absence of dependencies simplifies the abstraction, making CSG widely used in inverse design and reconstruction tasks \citep{du_inversecsg_2018, nandi2020synthesizing, yu2022capri, michel2021dag}. 
However, this limitation also makes modeling more challenging, which is why CSG is not commonly used in most commercial CAD tools.

% CSG operations reference the entire geometry directly and unambiguously, which means that primitives can be safely edited even after the fact.
% Geometric relationships between geometric elements are not part of the language.% itself and are often handled by external solvers in a design tool, e.g. Illustrator's alignment feature for SVG objects. \adriana{I'm not sure I agree that SVG is a programming language and is CSG-like.}
% Due to the binary nature of boolean operations, designs in CSG languages result in a hierarchical tree structure, which is analogous to an abstract syntax tree (AST).

%  % \adriana{would be it useful to talk about how this simplicity allows for applications like reverse engineering cite chandra's work, inverseCSG, capri-net. but is not usually what is most used for modeling because of the needs to specify global coordinates with all primitives.}\felix{It's worth talking about this point.}\maaz{I think briefly qualifying the strengths and weaknesses of CSG DSLs can be nice for less familiar readers but I would not spend more than a few lines.}

\paragraph{Query-based CAD}
Most commercial CAD tools use query-based languages, such as FeatureScript \citep{featurescript}, which employ a sequence of operators to create and modify models (e.g., extrude, fillet, chamfer). 
These operators reference intermediate geometry---e.g., a chamfer operator takes a reference to an edge. 
This referencing creates implicit dependencies, simplifying modeling and enabling easy editing as operations propagate when intermediate geometry is updated. 
However, a challenge arises when edits lead to topological changes, making reference resolution ambiguous. 
For example, if an edge gets split or disappears, where should the chamfer be applied? To address this, these languages do not reference geometry explicitly. 
Instead, geometric references are specified \emph{implicitly} via a language construct called \emph{queries}. 
These queries are resolved during runtime by a solver~\citep{cadquery, featurescript}, which typically uses heuristics to resolve ambiguities. 
This makes automating design challenging, and generative tools that use CAD operators restrict themselves to sequences where references are not needed, such as sketch and extrude~\citep{wu2021deepcad, willis_fusion_2021, lambourne_reconstructing_2022}. 
While recent work allows for the unambiguous direct specification of references~\citep{cascaval2023lineage}, mastering this language is complex and demands significant expertise.

 

\paragraph{Constraint-based CAD}
As the name implies, constraint-based CAD DSLs natively enable users to create geometric constraints between geometric primitives. This frees designers from specifying parameters consistently, allowing for freeform design while ensuring that relationships between parts are preserved. This approach is used in content creation languages like Shape-Assembly \citep{jones2020shapeassembly}, GeoCode \citep{pearl2022geocode}, and SketchGen \citep{para2021sketchgen}.
 In typical commercial CAD tools, constraint-based abstractions are used in sketches---2D drawings that get extruded to form 3D geometry---and during assembly modeling, but not during solid modeling which uses queries. 
These languages do not provide operations to modify primitives or to create intermediate geometry and therefore they reference geometry directly.
Designs specified in these languages are non-hierarchical, all constraints are being solved simultaneously.
 




\section{Example Walkthrough}

Let us take the example of Emma, who is seeking to illustrate a social media post to express the serenity she feels when she spends time outdoors (Figure~\ref{fig:walkthrough1}). She starts from leveraging generative AI to generate a bird. The art style is not quite satisfying but she is not sure what the model is capable of. She selects a generative container from a panel of AI-instruments available to her (\rev{Figure~\ref{fig:walkthrough1}.}1 and 2) and explores different art styles. She finds a simple drawing style she likes, and creates a second generative container to explore other types of birds in the same style (\rev{Figure~\ref{fig:walkthrough1}.}3). She settles on a heron, and moves on composing a more interesting illustration. Since she has an idea of the general composition she wants, she opts for a transformative lens, a second AI-instrument enabling her to position her central character, the bird, in the frame (\rev{Figure~\ref{fig:walkthrough1}.}4). She creates multiple lenses to try multiple backdrops and settles on the forest one (\rev{Figure~\ref{fig:walkthrough1}.}5). As lenses can be layered, she creates a color style one, and layers it on top of the forest backdrop, resulting in an illustration she finds suitable for her post (\rev{Figure~\ref{fig:walkthrough1}.}6).

Two days later, Emma seeks to illustrate her school presentation on medieval castles (Figure~\ref{fig:walkthrough2}). She starts from a drawing generated by AI. She taps-and-holds to expand the fragments the model used to generate the image (\rev{Figure~\ref{fig:walkthrough2}.}1). By tapping on different fragments, she gets to try different variations, such as redrawing the castle as watercolor style (\rev{Figure~\ref{fig:walkthrough2}.}2). As she wants to add color to the illustration, Emma retrieves the palette where she saved multiple fragments related to colors she thought worked great in the past (\rev{Figure~\ref{fig:walkthrough2}.}3), and drags one onto the image. She does like the colors but notices a large white space in the back. She selects a fillable brush, \rev{an AI-instrument that lets} her directly scrub over the specific portions of the image that she wants to revise or refine (\rev{Figure~\ref{fig:walkthrough2}.}4). After she types the outcome she wants and brushes the region, the system generates a mask and applies changes locally (\rev{Figure~\ref{fig:walkthrough2}.}5). Emma is now satisfied with her illustration.





\section{Instrumental Interaction with AI}

Beaudoin-Lafon defines instruments as: \textit{"a mediator or two-way transducer between the user and domain objects."} we expand this definition to AI-instruments: \textit{"an AI-powered mediator or two-way transducer between the user and domain objects."} We describe below the three principles of our proposed model revision: reification of user intent, reflection and grounding. Note that these principles are tightly interconnected and, while differing in certain aspects from the original model also share a lot of similarities. We discuss differences in more depth in Discussion.



\subsection{Reification of User Intent}

Most pre-AI interfaces offer a finite set of functionalities, established at their design by software architects and developer. User experience designers craft a set of graphical interface components and interactions for each functionality to enable users to invoke a finite set of commands through this GUI. Today, LLMs can interpret requests from users in natural language and turn them into the execution of a specific command, or a sequence of commands, unbounding functionalities from a limited set of GUI components. With this major shift in interface design, we propose the reification of \textbf{user intent}, rather than \textbf{commands}.


Reification turns both input and output of generative AI into graphical elements that can be directly manipulated and thus reused by users. In contrast to chat-based interfaces consisting of sequences of [input+output] in \rev{which users can require} to rephrase the input to iterate, reifying input and output enables users to articulate phrases of interaction~\cite{buxton1995chunking} and afford direct manipulation techniques such as lasso selections to specify scopes of intent (Figure~\ref{fig:reification} (1-3)).
In section~\ref{sec:Ai-instruments}, we demonstrate how this instrumental model can leverage the full range of direct manipulation techniques the community developed such as magic lenses~\cite{bier2023toolglass} and attribute objects~\cite{xia2016object}, turning them into AI-instruments encapsulating user intent.




A key capability of generative AI models is their inherent ability to deal with the \textbf{degree of abstraction} of user intent. It offers unparalleled flexibility as users can express high-level or low-level intent. Examples in the literature leverage the high degree of abstraction for content generation. For example, Talebrush~\cite{chung_talebrush_2022} enables users to control the narrative arc of a story (where tension is in the story), which has many implications on the writing itself from adding or sequencing events differently in the story to subtly rewording the language.  Expressing high-level intents is a powerful ability, enabling people to shape content 
in ways that potentially lead to serendipitous discovery of alternative (potentially better) results. However, users face multiple challenges when results are unsatisfactory, understanding how to resolve ambiguity of their intent (C2) or thinking more crisply of the desired outcome (C3). These challenges often require users to lower the degree of abstraction of their intent. On the contrary, expressing intents with a low degree of abstraction lowers the chance to make serendipitous discoveries, as well as get into a class of unwanted model results, making it frustrating for users to steer content generation towards more major changes (C4) or conduct exploratory workflows (C5). These challenges often require users to increase the degree of abstraction of their intent. Figuring out how to navigate degrees of abstraction is a challenge in itself. Users may struggle turning an idea into a set of concrete changes or, conversely, articulate the overarching goal motivating specific changes. Users can leverage AI-instruments themselves to navigate the degree of abstraction of an intent, for example, by using a Generative Container to provide more concrete (resp. abstract) Fragments given one of high-degree (resp. low-degree) of abstraction (Figure~\ref{fig:reification} (4-5)).


\begin{figure}[t]
    \centering
\includegraphics[width=.48\textwidth]{Figures/reification3.png}
\caption{In the chat-based interaction model, interactions consists of a linear sequence of input+output pairs and steering is done by modifying the input (1). Reification enables articulating interactions into phrases for example by reusing the output of the prior input (2). It also affords direct manipulation techniques such as for lasso selection (in red) to specify the scope of the input (3). Reification of user intent enables users to reflect on their intent and navigate dimensions such as its degree of abstraction, using other instruments to make it more concrete (4) or abstract (5) for example.}
\Description{Screenshots of wireframes and examples illustrating the 'reification of user intent'. In the classic chat-based interaction model, interactions consists of a linear sequence of input+output pairs and steering is done by modifying the input (1). Reification enables articulating interactions into phrases for example by reusing the output of the prior input (2). It also affords direct manipulation techniques such as for lasso selection (in red) to specify the scope of the input (3). Reification of user intent also enables users to reflect on their intent and navigate dimensions such as its degree of abstraction, using other instruments to make it more concrete (4) or abstract (5) for example.}
    \label{fig:reification}
\end{figure}


\subsection{Reflection}


Seminal research demonstrated that it is critical to explore alternative designs early and throughout the whole process~\cite{tohidi2006getting, DesignGalleries1997}. It is particularly critical when working with AI because of its "black box" nature~\cite{bathaee2017artificial,hoffman2018explaining}, i.e. the inherent difficulty for users to understand how these models work, and the non-deterministic nature of their outputs.  To capture this aspect, we borrow the term \textbf{reflection} from the design literature and introduce it as a principle for AI-instruments. 

We define reflection as the ability to help users reflect on their possibly ambiguous intent (reflection-in-intent) as well as the ambiguous interpretation made by AI (reflection-in-response), and thus offer the ability to users to steer the content generation towards a satisfying result. 


\begin{figure*}[t]
    \centering
\includegraphics[width=.8\textwidth]{Figures/reflection2.png}
\caption{Reflection-in-intent enables users to gain awareness of the possible  formulations of their intent while reflection-in-response enables users to assess the space of possibilities of the outputs generated by the model given an input. These aspects may help users address the challenges of intent disambiguation, resolution and steering.}
\Description{Screenshots of wireframes and examples illustrating the 'reflection-in-intent' and 'reflection-in-response'. Reflection-in-intent enables users to gain awareness of the possible  formulations of their intent while reflection-in-response enables users to assess the space of possibilities of the outputs generated by the model given an input. These aspects may help users address the challenges of intent disambiguation, resolution and steering.}
    \label{fig:reflection}
\end{figure*}


\textbf{Reflection-in-intent} is the ability of AI-instruments to surface multiple facets of their intent to users. For example,  fragmenting intent into pieces reveals a particular chunking~\cite{buxton1995chunking}. Working with fragments (Figure~\ref{fig:reflection}) may help users refine their intent (1), pivot on a specific aspect (2) or iterate by adding novel aspects (3).



\textbf{Reflection-in-response} is the ability of AI-instruments to offer multiple results of the content generation, while also helping people explore the space of possibilities (Figure~\ref{fig:reflection}), addressing (C3).Reflection-in-response can vary on the type and range of alternatives provided by employing diverse strategies: using model parameters such as its temperature, generating variations of the input, or asking the model to use different context of interpretation. 




\begin{figure*}[t]
    \centering
\includegraphics[width=.8\textwidth]{Figures/grounding4.png}
\caption{Grounding an instrument such as a generative container with an example enables to refer to features to preserve or alter in simple worlds by leveraging AI segmentation (1). Grounding an instrument such as a fillable brush in a specific aspect of an example, for example by selecting a region and extracting its style (2), enables users to use and apply it to other inputs without the need to articulating it in words. The principle of grounding also applies to instruments themselves such as deriving fragments from an example one (3).}
\Description{Wireframes and screenshots used to explain the concept of grounding. Showing a number of images and different operations performed to modify the content. Grounding an instrument such as a generative container with an example enables to refer to features to preserve or alter in simple worlds by leveraging AI segmentation (1). Grounding an instrument such as a fillable brush in a specific aspect of an example, for example by selecting a region and extracting its style (2), enables users to use and apply it to other inputs without the need to articulating it in words. The principle of grounding also applies to instruments themselves such as deriving fragments from an example one (3).}
    \label{fig:grounding}
\end{figure*}



\subsection{Grounding}

The principle of grounding refers to the ability for users to ground instruments from examples of desired outcomes or other instruments. It may be difficult to find the right vocabulary to describe particular aspects of content, especially for images. Instruments leverage AI segmentation to (1) enable users to refer to elements of an example in generic terms,  and (2) extract specific aspects of the content (e.g. style) by selection, storing the result for later (Figure~\ref{fig:grounding}). This builds on the notion of \textit{Variations}, \textit{Parameter Spectrums}, and \textit{Side Views} \cite{TechnologyProbesCHI2003, TerrySideViews2002}, but in a way that leverages the principles of interactive instruments \cite{beaudouin2000instrumental, beaudouin2000reification} as well as the open-ended possibilities of generative AI via our novel AI-instruments, rather than as views or controls with fixed, hand-designed and hard-coded options. AI-instruments can also be grounded in other instruments, enabling exploration of the space of related instruments (Figure~\ref{fig:grounding} (3)).



\section{Examples of AI-Instruments}
\label{sec:Ai-instruments}
To assess the viability of our AI instrumental model, study its differences with existing GUIs and tease out its value compare to existing chat-based AI interaction, we built a technology probe~\cite{TechnologyProbesCHI2003} with four different instruments, grounded in the literature: Fragments, Generative Containers, Transformative Lenses and Fillable Brushes. 
We describe below how this set of instruments surface the principles of our AI-instrumental model, as well as offer complementary interaction capabilities and affordances.




\subsection{Fragments}
 Fragments build on the concept of \textit{Attribute Cards} introduced in Object Oriented Drawing~\cite{xia2016object}, as well as Side View's notions of \textit{Variations} and \textit{Parameter Spectrums} \cite{TerryCreativeNeedsUIDesign2002, TerrySideViews2002}, by using a large language model to extract multiple conceptual dimensions that may be plausibly implied by a prompt. 
 
 Fragments reify an initial prompt used to generate text or image into a set of attribute cards, of the format \texttt{\textbf{[type, value]}} (where \texttt{\textbf{type}} is the category of the extracted dimension, and \texttt{\textbf{value}} is the extracted value within that dimension---such as  \texttt{[tone, enchanting]}, \texttt{[content, castle]} or \texttt{[style, illustration]}). 
 Revealing these conceptual dimensions enables an initial reflection-in-intent, revealing the latent structure of the prompt as seen by the AI model. \rev{Commercial software such as Adobe Firefly~\cite{firefly} offers a similar capability as tags, enabling users to select them from a side panel for subsequent image generation. Applying the principle of reification to tags and turning them into cards affords three core novel interactions} as illustrated in Figure~\ref{fig:fragments}.
 
 First, users can reveal fragments via a long press on the content. Fragments are fully reified as interactive instruments and are dynamically generated---hence open-ended and nondeterministic---in contrast to the fixed, hand-crafted, and hard-coded controls supported by prior work (e.g. ~\cite{xia2016object, TerryCreativeNeedsUIDesign2002, Suggestive3dDrawing2007, WritLarge2017}). 
 Second, via drag and drop, users may remove fragments (by dragging them away), or add new fragments onto existing content in the work space. Adding or removing fragments triggers regeneration of the content. 
 Third, to further support reflection, fragments offer suggestions on demand. By tapping on \includegraphics[width=0.015\textwidth]{Figures/ellispsis.png}, users can generate new variations from any fragment; these suggestions appear in a column below the specific fragment. Users can also invoke additional suggestions for more types of fragments, which are then appended to the row of fragments. 
 
These three core mechanisms support a workflow where, as users work with multiple images in their workspace, they can explore the effect of different fragments via drag-and-drop to ground one image generation into an aspect of another.
 

Fragments use the affordance of attribute cards to break down and reify a complex intent into manageable pieces, each having distinct type and value, that enable people to work with these as more-or-less independent and composable, "pieces of intent." This also encourages a workflow where users can surface useful fragmentary concepts surface that become reusable and specialized instruments in their own right. Such fragments are then available for reapplication to other pieces of content, or even reuse in a different context.

While in principle we could have pursued a design that generated many fragments as automatic suggestions associated with each piece of content, such an approach would introduce clutter and risk overwhelming the user with the "decision paralysis" of too many choices. Our design therefore surfaces only a few fragments at a time, and only in a post-hoc manner upon explicit invocation by the user. Further, we present these newly-invoked fragments in an organized fashion, with two orthogonal dimensions of exploration on demand, by keeping dimension type in horizontal rows of cards, and value variations in vertical columns beneath these.



\begin{figure*}[t]
    \includegraphics[width=0.88\textwidth]{Figures/fragment-probe4.png}
    \caption{Users can expand Fragments with variations of parameter values (1) in vertical columns, or request more suggestions for dimension types (2) at the end of the row. Users can further reuse and transfer Fragments to other content via drag-and-drop(3).}
    \Description{Several wireframe boxes with images and text explain the concept of Fragments. Users can expand Fragments with variations of parameter values (1) in vertical columns, or request more suggestions for dimension types (2) at the end of the row. Users can further reuse and transfer Fragments to other content via drag-and-drop(3). The textboxes with prompt fragments are shown next to an image, and then a person expands these fragments to show variations. When selecting any of these alternative values of a fragment, the content is modified.}
    \label{fig:fragments}
\end{figure*}



\subsection{Transformative Lenses}




\begin{figure*}[tb]
  \centering
  \includegraphics[width=0.95\textwidth]{Figures/lenses-probe3.png}
  
  \caption{Transformative Lenses are placed over initial content, enabling users to "complete" illustrations from pieces of content (1). When users add elements to their composition, lenses regenerate to integrate it (2).}
  \Description{Eight square boxes that illustrate wireframes of the Transformative Lenses. They show in a sequence of how other content gets added to the lens and new images are generated based on the composition of the source images. Transformative Lenses are placed over initial content, enabling users to "complete" illustrations from pieces of content (1). When users add elements to their composition, lenses regenerate to integrate it (2).}
  \label{fig:lenses}
\end{figure*}


Transformative Lenses re-envision the Toolglass and Magic Lens interaction technique ~\cite{bier2023toolglass} as a layered instrument that can be coupled with a generative prompt. 


Layering a Transformative Lens on top of content uses such a prompt to generate a new image that synthesizes the lens and the content. Likewise, a specific piece of image content can be used on top of a lens to recombine the two. Such layerings can be positioned and manipulated to chain multiple effects together. 
As illustrated in Figure~\ref{fig:lenses}, users can leverage lenses to take a piece of content (e.g. a sketch of a suspension bridge), and then re-compose this content within a wider backdrop scene (a city skyline), or even apply a new specific style to the results with a single interaction (e.g. a heavy, black-lined graphic novel style). 

More generally, depending on how the user layers Transformative Lenses and image content, lenses can support image completion from a small piece of content, synthesis and composition of multiple pieces of content into a new image, or regeneration of the underlying image. Note also that blank lenses (which have no image content, but do contain a prompt) can be used. For example, a blank-lens backdrop generated afforded outpainting-like operation ---but here steered by the lens's prompt---to "complete" a scene from an existing piece of image content. 



Users can freely drag, reposition, and resize both content images and lenses, layering them over each other to chain transformations, reflect on the results, and experiment with different combinations. This property also may encourage users to break down their intent into multiple lenses, which can then be applied to multiple pieces of content (grounding). Note that image recomposition and dynamic regeneration occurs after a 2-second idle time to avoid triggering constant image regenerations during dragging or resizing operations. \rev{As users may wish to adjust content under a lens post-generation, the lens is temporarily faded out in the background when the mouse pointer enters it. } 



Complementary to the Fragments described in the previous section, Transformative Lenses afford the design consideration of breaking down the output into pieces (whereas fragments focus on the prompt intent). People can control the composition of images by just moving and layering elements in relation to the lens, limiting the need for precise selection, and encouraging rapid iteration \& experimentation with compositions. However, unlike an undo operation, removing (or otherwise reverting) the layering of Transformative Lens and image-content elements triggers re-generation, and will always lead to a slightly different rendering. 




\subsection{Generative Containers}



Designers use moodboards \cite{BuxtonSketchingUserExperiences2007}, storyboards \cite{Storeoboard2016}, and other techniques for presenting small-multiples in  galleries \cite{DesignGalleries1997, TerryCreativeNeedsUIDesign2002, TEAMSTORM2007} to illustrate and explore a space of possible creative directions. \rev{Structured generation of those alternatives \cite{suh_luminate_2024} allows rapid exploration of design spaces, and techniques to highlight similarities and differences \cite{gero_sensemaking_2024} facilitate the selection, refinement, and comparison of multiple responses.}




As shown in Figure~\ref{fig:containers}, \textit{Generative Containers} provide an AI-instrument that encapsulates these notions using a prompt---shown in the container's header---that is closely associated with a 2x2 small-multiple grid of generated image results. Users can then enter or edit the prompt, or drag and drop  example content---or even another instrument, such as a Fragment---onto the Container to ground it and generate a new small-multiple set of results. 

We designed Generative Containers to enable reflection-in-response, allowing users to quickly get a visual sense of the range of responses a single prompt might produce. And by using Generative Containers to generate different variations of fragments, for example to obtain more concrete image editing suggestions from a high-level intent (Figure~\ref{fig:meta-instrument} left), generative containers also enable reflection-in-intent.

In our current implementation, the Generative Containers probe supports generation of four different variations (in a fixed 2x2 grid). Further, each container is presently limited to a single grounding example as input. However, users can create multiple containers and reuse results by dragging and dropping from one to another. 
In this way Containers afford adding details and varying the prompt to generate a range of example images, encouraging multiple cycles of iteration. Recombining and chaining these together effectively results in a longer, refined prompt that integrates the series of changes from prior interactions. 
\rev{Furthermore, one could expand the Generative Container instrument with other representations beyond our 2x2 grid, such as the dimension plots or stacked vertical dimension grids \cite{suh_luminate_2024}.}



\begin{figure}[htb]
  \begin{subfigure}{.47\textwidth}
  \centering
    \includegraphics[width=\textwidth]{Figures/container-probe4.png}
    \caption{Generative containers enable users to explore possibilities on concrete or abstract dimensions (1). Containers also afford complex exploration paths by reusing the output of one container as the input of another one (2), resulting in refining intent (3).}
    \Description{Generative containers enable users to explore possibilities on concrete or abstract dimensions (1). Containers also afford complex exploration paths by reusing the output of one container as the input of another one (2), resulting in refining intent (3).}
    
    \label{fig:containers}
  \end{subfigure}
  \hfill
  \begin{subfigure}{.47\textwidth}
  \centering
    \vspace{10pt}
    \includegraphics[width=\textwidth]{Figures/brushes-probe3.png}
    \caption{Brushes can extract aspects of content difficult for users to articulate in words, such as drawing style, making it reusable and editable (1). Combined with the selection afforded by brushes, this enables to apply aspects such as style to portions of images (2).}
    \Description{Brushes can extract aspects of content difficult for users to articulate in words, such as drawing style, making it reusable and editable (1). Combined with the selection afforded by brushes, this enables to apply aspects such as style to portions of images (2).}
    \label{fig:brushes}
  \end{subfigure}%
  
  \caption{Generative Containers and Fillable Brushes support different types of content creation tasks. Containers promote the exploration of multiple ideas in parallel, while Brushes offer precise direct manipulation for steering generation. Providing users with both of these AI-instruments enables them to conduct many different activities involved in content creation, enabling  interweaving of both divergent and convergent thinking activities.}
  \Description{Screenshots illustrating generative containers and fillable brushes. Sketched drawings of a beach and palm trees are added to a box that is representing a generative container, which in turn creates variations of those sketches. Generative Containers and Fillable Brushes support different types of content creation tasks. Containers promote the exploration of multiple ideas in parallel, while Brushes offer precise direct manipulation for steering generation. Providing users with both of these AI-instruments enables them to conduct many different activities involved in content creation, enabling  interweaving of both divergent and convergent thinking activities.}
\end{figure}



\subsection{Fillable Brushes}



Fillable Brushes, as illustrated in Figure~\ref{fig:brushes}, 
offer an AI-instrument with the semantics of an "intelligent paint brush" for style transfer scoped to a particular spot on an existing image. 

While previous work has explored brushes that can encapsulate and integrate deterministic modes and commands ~\cite{romat2022style}, our Fillable Brushes instrument applies encapsulated AI-prompts onto content in an intelligent manner as the user scrubs over it with their pen, finger, or other pointing device. And in contrast to the post-hoc notion of Fragments described above, Fillable Brushes apply a brush onto content, structured as an AI-Instrument "command" with a prefix (as opposed to postfix) syntax ~\cite{LexicalPragmaticInput1983}. This offers a familiar interaction model from the way that a highlighting tool turns selected text yellow in a document editor, for example. 



\rev{Encapsulating a prompt or image into a brush to define its function is a powerful interaction techniques to control scope of selection, as demonstrated by Runway motion brushes~\cite{runwayML}. Applying additional principles of our model, enables users to} also "fill" (ground) an empty brush by using existing content as an example, as if the instrument were a color picker that picks up key semantic attributes of the content rather than just its "color." The prompt encapsulated by the Fillable Brush is then automatically populated with descriptive words via generative AI, which the user can further edit if desired. This can be particularly helpful when users want to style something "like this" even when they may lack the vocabulary to describe its visual style. Our Fillable Brushes technology probe supports both content and/or style extraction. \rev{Turning a brush into a persistent object on screen, enables combining brushes together by drag and drop.} Brushes can also be applied multiple times to the same content to emphasize a particular prompt in the result. 

While Fillable Brushes enable users to specify the scope of intent with a high granularity, this does not necessarily require high precision: our implementation leverages the AI-powered Segment Anything Model (SAM)~\cite{kirillov_segment_2023}, which enables users to make approximate selections (i.e. rather than a precise and tedious lasso selection) to indicate an image element. The source content plus the approximate selection (as a set of reference points) is then converted into a precise object selection by the segmentation model. 




\subsection{Generated Instruments and Meta-Instruments}

Beyond the concept of instruments, the instrumental interaction model~\cite{beaudouin2000instrumental} also refers to the concept of meta-instruments, in which \textit{"instruments operate on instruments"}. 
As hinted at in earlier sections, using instruments on other instruments can be particularly useful to derive or compose instruments from the "task detritus" \cite{KirshIntelligentUseOfSpace1995} already produced in the user's workflow and experimentation with other instruments. Using generative containers on Fragments, for example, can help users navigate the degree of abstraction, turning a vague idea into a set of concrete modifications (Figure~\ref{fig:meta-instrument} left). 

However, such generation loops (instruments that generate content, generating instruments that generate other instruments generating content...) could potentially lead to an unwieldy number of elements in the interface. To organize but also generate collections of instruments, we devised a type of meta-instrument we call \textbf{Palettes}.

Akin to menus and containers available in GUIs today, Palettes enable storage and/or generation of different sets of instruments and content if desired. These afford abstraction and generalization of instrumental controls from collected pieces of content that can then serve as examples or generative seeds (Figure~\ref{fig:meta-instrument} right). Palettes of diverse instruments can balance the different affordances and properties of each instrument to provide rich content creation support. They can also help people get past the "cold-start" problem in complex creative design work, by beginning with examples, other pieces of existing content, or past work-artifacts to help overcome so-called "writer's block" or "blank canvas" effects of starting from nothing.



\begin{figure*}
    \centering
    \includegraphics[width=\linewidth]{Figures/meta-instruments.png}
    \caption{Instruments can be used as Meta-Instrument: operate on each other to create related instruments, for example for making a fragment more concrete (left). Specific Meta-Instruments such as palettes (right) can help user organize sets of instruments for easier retrieval and reuse, or, even help generating collection of instruments for a certain task.}
    \Description{Wireframe and screenshots (multiple squares that represent the composition of multiple AI instrument in a meta instrument) -- Instruments can be used as Meta-Instrument: operate on each other to create related instruments, for example for making a fragment more concrete (left). Specific Meta-Instruments such as palettes (right) can help user organize sets of instruments for easier retrieval and reuse, or, even help generating collection of instruments for a certain task.}
    \label{fig:meta-instrument}
\end{figure*}








\section{Implementation}


\paragraph{Overview:}
Our system and all AI-Instruments technology probes were implemented on a web-based platform. We use Javascript and HTML with the fabric.js~\cite{juriy_zaytsev_fabricjs_2024} library for the front-end, and a Node.js~\cite{openjs_foundation_nodejs_2024} server for the back-end managing content and files as well as coordinating communication  with the generative AI models. For the user interface design, we chose to use a sketched user interface look and feel, to encourage our study participants to focus on the concepts rather than the surface details of their specific instantiation in the UI \cite{BuxtonSketchingUserExperiences2007}.  

\paragraph{Leveraging Generative AI models:}
We use the OpenAI GPT-4o~\cite{openai_gpt-4o_2024} model for text transformations and image analysis, and a local Stable Diffusion~\cite{stabilityai_compvisstable-diffusion_nodate} server with a custom processing pipeline for image generation. The AI-Instruments use GPT-4o for analysis of provided input (e.g., for turning provided visual content into a text prompt, analyzing the contents of part of the workspace canvas). For image generation, we use multiple stacked ControlNet~\cite{zhang_controlnet_2023} models with Stable Diffusion to steer the generation of visual content. To preserve aspects of the source input, we use a combination of \textit{Depth}, \textit{Canny Edge}, and \textit{Scribble} ControlNet models, while for preserving art/rendering styles (e.g., the sketch-based output) we use the \textit{Reference} ControlNet model. Depending on the type of image generation, we vary the weight of each ControlNet model (e.g., increasing weight to emphasize content preservation, or decrease weight of another ControlNet to reduce affect of reference style transfer). We use image masks to selectively control which areas are changed or kept, apply inpainting/outpainting scripts, and adjust other parameters such as CFG scale, denoising strength, and control mode. 

\paragraph{Building AI-Instruments:}
We designed a pipeline that can orchestrate the access and requests to the different LLM and diffusion server instances to generate results. Key functionality is wrapped in modules, such as for encapsulating prompts to communicate with one or more models (by using model chaining) to perform a specific task. 
Each AI-instrument then uses a number of these modules for modifying the input or generating new content bases on the user's performed action with the instrument:


\begin{itemize}
    
\item \textit{\textbf{Fragments}} instruments include modules for (1) prompt decomposition which takes a text or visual input and makes a GPT-4o request to generate fragments (returned as collections of \texttt{[type, value]} pairs), (2) fragment extension which takes a prompt and the existing fragments and requests additional fragment dimensions, (3) fragment variation which takes the fragment and parent prompt/content (if applicable) and generates variations of that fragment, and (4) prompt composition which takes a prompt, a modification to the fragments, and returns a modified prompt. The result from the prompt composition is then used to create an updated image with the Stable Diffusion + ControlNet pipeline. 


\item \textit{\textbf{Transformative Lenses}} use a module for composition of the prompt (merging prompts from all source images covered by the lens), before then applying inpainting/outpainting, masks, and ControlNet models to generate the resulting image. 


\item \textit{\textbf{Containers}} use a variation module, taking a prompt and a dimension, and requesting four variations along the provided dimension. Within the prompt we request visually diverse results. The resulting set of prompts is then sent to SD+ControlNet to generate the final set of four images in the container. 


\item \textbf{\textit{Fillable Brushes}} are implemented to either emphasize style or content variations, depending on the intent of the user, which we support by varying the weight of the ControlNet models (e.g., higher reference ControlNet weight for changing the visual style, or increasing weight of Canny-edge/Depth ControlNet to preserve existing content).

When the brush is applied, we perform a segmentation of the source content by feeding the stroke path as control points into Segment Anything~\cite{kirillov_segment_2023}, which results in a segmentation mask of the dominant object(s) selected with the brush stroke. We then use GPT-4o to craft a combined prompt given the source image(s), the segmented content, and the original prompt. Finally, we send this generated prompt together with the source image and segmentation mask to the Stable Diffusion server, using the ControlNet inpainting method.



\end{itemize}
 


\section{User Study}
To better understand the possible benefits and limitations of intent tagging-based interactions, we conducted a lab user study aimed at providing insights into these research questions:
\begin{itemize}[font=\bfseries,
  align=left]
    \item[RQ1] \textit{What are the key differences between chat-based and intent tag-based interactions with GenAI?}
    \item[RQ2] \textit{How do people work with intent tags?}
    \item[RQ3] \textit{What are users’ perceived benefits and challenges for GenAI-driven slide creation with intent tags?}
   
 \end{itemize}


\rev{For RQ1, we decided to compare intent tag-based interactions with chat-based and design gallery-based approaches (text prompting and choosing from a set of options) since these represent the currently most common forms of interacting with GenAI for content creation in commercial systems, such as in OpenAI's ChatGPT/DALL-E \cite{betker_improving_2023}, Adobe's Firefly \cite{adobe_adobe_2024} or Microsoft's PowerPoint Copilot \cite{microsoft_powerpoint_2024b} and Designer \cite{microsoft_powerpoint_2024a} features.}  


\subsection{Participants}
We recruited 12 participants\textit{ (8 self-identified as females, 3 males, and 1 non-binary, age M=31.6 years (SD = 7.65 ))} with professional slide presentation creation experience via email lists at a large software company. The participant pool consisted of individuals with diverse job titles, and the majority of selected participants used PowerPoint at least multiple times per month in their jobs (see Table \ref{tab:participants} in the Appendix). All participants had previous experience using the \textit{Designer} and \textit{Copilot} features in PowerPoint, ensuring existing familiarity with generative AI functionalities for slide deck creation. Participants signed an IRB-approved consent form and were compensated with a \$50 gift card after study completion. 

\subsection{Procedure and Tasks} \label{sec:study-procedure}
The lab study was structured into four phases: 

\textbf{1) On-Boarding (20 min):} At the beginning of the session, after a general study introduction, participants watched a video tutorial demonstrating IntentTagger's core functionalities with a step-by-step example. Following the video, participants were asked to complete two five-minute guided hands-on structured tasks (editing an existing slide and making adjustments to an entire deck) to familiarize themselves with the tools' interface and operation.

\textbf{2) Comparative Tasks ( 2 x 10 min): }
After the onboarding phase, participants completed two comparative tasks, each lasting \rev{10} minutes, to evaluate their ability to create presentations using IntentTagger and a baseline system. 
\rev{In both tasks, participants were asked to create a 6-slide presentation aimed at educating teenagers about the inventions and scientific discoveries of a historical figure (\textit{“The Discoveries of Marie Curie”} or \textit{“The Inventions of Nikola Tesla”}).
Participants alternated between using IntentTagger and Microsoft PowerPoint \cite{microsoft_powerpoint_2024} across the two tasks in randomized order. 
For PowerPoint, participants were restricted to using only the integrated Copilot feature \cite{microsoft_powerpoint_2024b} (chatbot interface with optional document upload) and the Designer feature \cite{microsoft_powerpoint_2024a} (design gallery for slide layouts) for slide generation and modification. 
As a starting point, participants were provided with a Word document containing the relevant Wikipedia article in both tasks.} 
Participants were required to think aloud during the tasks, and after completing each task, they filled out a survey with attitudinal 6-point Likert scale questions. 
To mitigate order effects, the order of systems and presentation topics was randomized across participants. 

\textbf{3) \rev{Semi-structured} Task (10 min):}
In the third phase, participants were tasked with creating a 7-slide presentation from scratch using IntentTagger. 
They chose a topic related to a hobby they enjoy, aiming to convince others of its value and explain how to get started. 
Participants had ten minutes to complete the task, using only the prototype system without directly editing text or images on the canvas. 
They were required to think aloud as they worked, and before starting, they briefly described their topic and intentions for the presentation. 
\rev{We created this task to encourage participants to engage more freely with IntentTagger’s features, focusing on content creation and presentation design related to a topic they are knowledgeable about and emotionally connected to.
Since IntentTagger's deck generation time increases per slide, we deliberately limited the number of slides to six and seven per task in phases 2 and 3 to keep the processing time for each cycle within 15 seconds.}

\textbf{4) Exit Interview (20 min): }
In the final phase, participants participated in a semi-structured interview to provide feedback on their experience with IntentTagger, focusing on its overall utility, comparison with tools like PowerPoint’s Copilot, and the effectiveness of intent tagging interactions for slide creation. They shared insights on the tool’s strengths, areas for improvement, and potential integration into their professional workflows.



\subsection{Collected Data, Measures, and Analysis}

Across the study, we collected the following data:
\begin{itemize}
\item  Video, screen, and audio recordings and machine-generated transcripts of the task think-aloud sessions 
\item Audio recordings and machine-generated transcripts of the post-task interviews 
\item Post-task survey data
\item Participant-created presentations and related IntentTagger project files
\end{itemize}

To compare chat-based and intent tag-based interactions (Q1), we analyzed the post-task surveys from phase 2 that probed on participants' perceived ease of use, efficiency, and control over the slide generation process on a 6-point Likert scale. We applied the Wilcoxon signed-rank test to assess statistical significance and calculated 95\% confidence intervals for mean differences via bootstrapping with 10,000 replications using R \cite{rcoreteam_language_2024}. This approach has been suggested for similar data and studies \cite{zhu_assessing_2018, masson_statslator_2023}. 


To answer how people work with intent tags (Q2), we conducted a video interaction analysis \cite{baumer_comparing_2011} of the video recordings collected in the semi-structured task \textit{(study phase 3)}. We manually coded participants' interactions with IntentTagger's features in Atlas.ti \cite{atlas.ti_atlasti_2024}, such as their interactions with tags or the moments they triggered the outline and slide generation.  

Finally, to investigate users’ perceived benefits and challenges of intent tags (Q3), we conducted a reflexive thematic analysis \cite{braun_reflecting_2019} of the interview transcripts. We followed an iterative inductive coding process (using Marvin \cite{marvin_marvin_2024}) and generated themes through affinity diagramming using Miro \cite{miro_miro_2024}.

\section{Discussion and Future Work}

We first discuss how our model revisits and extends the classic instrumental interaction model, then discuss the application of AI-instruments to different forms of content, outlining future work.

\subsection{Revisiting and Extending the Instrumental Interaction Model}

The instrumental interaction model~\cite{beaudouin2000instrumental} is based on three core principles~\cite{beaudouin2000reification}:
\begin{itemize}
    \item \textit{Reification of commands} refers to the principle of turning systems functionalities into interactive graphical objects in the interface,
    \item \textit{Polymorphism} refers to the principle of applying commands to different types of objects enabling the interface designers to keep the number of interface objects relatively small, 
    \item \textit{Reuse} refers to the ability for users to reapply one command to different objects or apply different commands to one object, with the goal of limiting repetitive user input and/or navigation.
\end{itemize}

In this paper, we revisited and proposed to extend this model in the following ways:

(1) We extended the principle of reification from encompassing a limited set of commands defined for an application to include any intent user may express in natural language. We also unpack two key considerations of reification that one should consider when designing AI-Instruments: the scope of the instrument, and degree of abstraction. We propose to leverage the affordances of existing direct manipulation techniques to convey to users how to specify scope (e.g. select a portion with a brush vs resize the lens). To support users navigating different levels of abstraction of their intent, we propose to use AI-instruments themselves.


(2) We re-framed the concept of polymorphism in instrumental interaction, recasting it as \textit{reflection}. This shifts instrumental interaction from a classic "direct manipulation" technique for graphical user interfaces to a modern AI-augmented technique. Reflection leverages the general concept-translation capabilities of LLM's to support an expansive notion of "polymorphism" without requiring the interface design and system architect to hand-code the parameters, controls, and nuances of how these are interpreted across a wide range of content types. 

Further, by considering reflection from both the user (intent) and system (response) perspectives, we provide users with mechanism to explore both the design space of their intent (i.e. different formulations and disambiguation of intent), as well as the design space of the model response (i.e. different interpretations of user intent by the model). While such notions, in one sense, have been latent in interactive instruments all along, with generative AI many possible forms and interpretations of polymorphism---potentially even for niche or specialized workflows, formats, and types of content, if they are sufficiently represented in the training data for the model---can be made available for user reflection in AI-assisted content creation. 



(3) The third principle of reuse is closely related to our principle of grounding. Grounding extends the notion of \textit{reusing commands} to the capability of \textit{extracting and reusing intent}---whether in terms of one aspect of user intent, a collection of multiple user intents, or other properties of content. This principle of grounding also encapsulates the ability to generate instruments from other instruments, characterized as \textit{meta-instruments} in the nomenclature of instrumental interaction \cite{beaudouin2000instrumental, beaudouin2000reification}. 


(4) A further new challenge raised by AI-instruments is the need to balance the possibility of over-generating instruments, with the power to encapsulate many capabilities---at a high level of abstraction---within a single instrument. In contrast to classic hand-crafted instrumental interaction, AI-generated instruments could potentially lead to an unwieldy number of objects in the workspace, if generation were left unchecked. However, we counterbalance this with strategies to compose instruments and organize them into collections (meta-instruments). Further, we can leverage the generative nature of AI to iteratively refine both content and (meta-)instruments---altering, summarizing, or abstracting instruments and content with each step---as another strategy to harness AI to express aggregated concepts at a high semantic level.



The insights we gained by building a set of technology probes and gathering initial perceptions of 12 content creators, suggest that the principles described in our model can be used inform the design of novel interaction techniques as well as assess existing ones. \rev{As we built each probe, it became clear that design decisions at lower-level, for example pertaining to the specific choice of interactions (e.g. click vs double-click) or their timing (e.g. idle time threshold triggering image generation), can lead to different experiences, especially when multiple probes are used in conjunction.  While our model suggests overarching principles for AI-instruments, specific interaction is bound to differ as sets of these techniques are integrated into specific applications and adapted to different modalities and contexts~\cite{appert2005context, mackay2002interaction}.}

\rev{Additional considerations for integrating AI-instruments in applications include the expectations of users with regard to direct manipulation and instrumental interfaces, as well as when working with AI. For example, a fundamental principle of direct manipulation is ease of reversibility of user actions (e.g. if a fragment is removed from an image triggering a new generation, then added back; the image should revert to its prior state). In contrast, AI models are non-deterministic by nature (e.g. same input, different output). While it is technically feasible to couple AI generation with history and versioning mechanisms to ensure the reversibility of operations, users' attitude on working with AI over the longer term, as well as specific use cases may impact this design decision. }


\subsection{Beyond Images, Applying AI-Instruments to Other Forms of Content}

While extrapolations from participants must be taken with caution, four participants in our study related the use of AI-instrument to content they worked with every day. Interestingly two participants (P2 and P9) commented they would not see the use of instruments for tasks such as writing code. P9 summarized it as \textit{"it [fragments] is a bit harder to use than prompting for tasks like maybe writing code. I would prefer fragmenting for images or plots"}.  However, two different participants (P11 and P12) envisioned using AI-instruments for data analysis and writing (P11) and in the case of P12 leveraging grounding for operating at the artifact level: \textit{"it would be so cool to auto-pick a style (words, design, etc) without me first having to decipher it, and then have it automatically apply to other content (writing, slides, etc)"}.


We experimented with a few of our technology probes (Fragments, Containers and Palettes) to work with textual content and found that our principles generally held. However, further exploration with different types of modality is likely to reveal additional design considerations for the instruments we proposed. Notably, a key issue to address for textual content as opposed to images is the effort required to consume a number of potential outputs (reflection-in-response). Integrating support for helping user skim and get the gist of similarities and differences between textual outputs in Generative Containers (by bolding portions of text or providing summaries or excerpts) would certainly be necessary when using instruments to work with textual documents.  

Another aspect to address is the use of instruments for artifacts composed of multiple pieces of content (e.g. a slide composed of a title and image). Again, while we believe core principles hold for devising instruments to work with content at the artifact level, additional research is needed to delve into how to integrate different aspects of an artifact. For example, one could envision displaying Fragments from different scope of selection, enabling Fragments to operate at the entire slide level or on a subset such as title.



In the future, we plan to pursue these two research directions (designing AI-instruments for heterogeneous content and artifacts composed of multiple pieces of content), further assessing the generalizability of our interaction model and broadening the set of design considerations for AI-instruments.















% \section{Discussion and Future Work}

% %Onscad is insample data cause these LLMS have seen openscad

% The decision space of language design is enormous, so we had to make some decisions about what to explore in the language design of AIDL. In particular, we did not build a new constraint system from scratch and instead developed ours based on an open-source constraint solver. This limited the types of primitives we allow, e.g. ellipses are not currently supported. \jz{Additionally, rectangles in AIDL are constrained to be axis-aligned by default because we found that in most use cases, a rectangle being rotated by the solver was unintuitive, and we included a parameter in the language allows rectangles to be marked as rotatable. While this feature was included in the prompts to the LLM, it was never used by the model. We hope to explore prompt-engineering techniques to rectify this issue in the future. Similarly, we hope to reduce the frequency of solver errors by providing better prompts for explaining the available constraints.} \adriana{Add two other limitations to this paragraph: that we typically noticed that things are axis alignment, say why we use this as default and in the future could try to get the gpt to not use default more often. Mention that we still have Solver failures that could be addressed by better engineering in future. }

% In testing our front-end, we observed that repeated instances of feedback tends to reduce the complexity of models as the LLM would frequently address the errors by removing the offending entity. This leads to unnecessarily removed details. More extensive prompt-engineering could be employed in future work to encourage the LLM to more frequently modify, rather than remove, to fix these errors. \adriana{no idea what this paragraph is trying to say}


% \adriana{This seems  like a future work paragraph so maybe start by saying that in the future you could do other front end or fine tune a model with aidl, we just tested the few shot.  } \jz{In the future, we hope to improve our front-end generation pipeline by finetuning a pretrained LLM on example AIDL programs.} In addition, multi-modal vision-langauge model development has exploded in recent months. Visual modalities are an obvious fit for CAD modeling -- in fact, most procedural CAD models are produced in visual editors -- but we decided not to explore visual inputs yet based on reports ([PH] cite OPENAIs own GPT4V paper) that current vision-language models suffter from the same spatial reasoning issues as purely textual models do (identifying relative positions like above, left of, etc.). This also informed our decision to omit spline curves which are difficult to describe in natural language. This deficit is being addressed by the development of new spatial reasoning datasets ([PH] cite visual math reasoning paper), so allowing visual user input as well as visual feedback in future work with the next generation of models seems promising.

% The decision space of language design is enormous, so it was impossible to explore it all here. We had to make some decisions about what to explore, guided by experience, conjecture, technical limitations, and anecdotal experience. Since we primarily explore the interaction between language design and language models in order to overcome the shortcomings in the latter, we did not wish to focus effort on building new constraint systems. This led us to use an open-source constraint solver to build our solver off of. This limited the types of primitives we allow; in particular, most commercial geometric solvers also support ellipses.

% In testing our generation frontend, we observed that repeated instances of feedback tended to reduce the complexity of models as the LLM would frequently address the errors by removing the offending entity. This is a fine strategy for over-constrained systems, but can unnecessarily remove detail when done in response to a syntax or validation error. More extensive prompt-engineering could be employed to encourage the LLM to more frequently modify, rather than remove, to fix these errors


% In recent months, multi-modal vision-language model development has exploded. Visual modalities are an obvious fit for CAD modeling -- in fact, most procedural CAD models are produced in visual editors -- but we decided not to explore visual input yet based on reports (cite OpenAIs own GPT4V paper) that current vision-language models suffer from the same spatial reasoning issues as purely textual models do (identifying relative positions like above, left of, etc.). This also informed our decision to omit spline curves; they are not easily described in natural language. This deficit is being addressed by the development of new spatial reasoning datasets (cite visual math reasoning paper), so allowing visual user input as well as visual feedback in future work with the next generation of models seems promising.



\section{Conclusion}

AIDL is an experiment in a new way of building graphics systems for language models; what if, instead of tuning a model for a graphics system, we build a graphics system tailored for language models? By taking this approach, we are able to draw on the rich literature of programming languages, crafting a language that supports language-based dependency reasoning through semantically meaningful references, separation of concerns with a modular, hierarchical structure, and that compliments the shortcomings of LLMs with a solver assistance. Taking this neurosymbolic, procedural approach allows our system to tap into the general knowledge of LLMs as well as being more applicable to CAD by promoting precision, accuracy, and editability. Framing AI CAD generation as a language design problem is a complementary approach to model training and prompt engineering, and we are excited to see how advance in these fields will synergize with AIDL and its successors, especially as the capabilities of multi-modal vision-language models improve. AI-driven, procedural design coming to CAD, and AIDL provides a template for that future.

% Using procedural generation instead of direct geometric generation enables greater editability, accuracy, and precision
% Using a general language model allows for generalizability beyond existing CAD datasets and control via common language.
% Approaches code gen in LLMs through language design rather than training the model or constructing complexing querying algorithms. This could be a complimentary approach
% Embedding as a DSL in a popular language allows us to leverage the LLMs syntactic knowledge while exploiting our domain knowledge in the language design
% LLM-CAD languages should hierarchical, semantic, support constraints and dependencies




%In this paper, we proposed AIDL, a language designed specifically for LLM-driven CAD design. The AIDL language simultaneously supports 1) references to constructed geometry (\dgone{}), 2) geometric constraints between components (\dgtwo{}), 3) naturally named operators (\dgthree{}), and 4) first-class hierarchical design (\dgfour{}), while none of the existing languages supports all the above. These novel designs in AIDL allow users to tap into LLMs' knowledge about objects and their compositionalities and generate complex geometry in a hierarchical and constrained fashion. Specifically, the solver for AIDL supports iterative editing by the LLM by providing intermediate feedback, and remedies the LLM's weakness of providing explicit positions for geometries.

%\adriana{This seems  like a future work paragraph so maybe start by saying that in the future you could do other front end or fine tune a model with aidl, we just tested the few shot.  }
%\paragraph{Future work} In recent months, multi-modal vision-language model development has exploded. Visual modalities are an obvious fit for CAD modeling -- in fact, most procedural CAD models are produced in visual editors -- but we decided not to explore visual input yet based on reports (cite OpenAIs own GPT4V paper) that current vision-language models suffer from the same spatial reasoning issues as purely textual models do (identifying relative positions like above, left of, etc.). This also informed our decision to omit spline curves; they are not easily described in natural language. This deficit is being addressed by the development of new spatial reasoning datasets (cite visual math reasoning paper), so allowing visual user input as well as visual feedback in future work with the next generation of models seems promising. 



\begin{acks}

We thank the participants of our AI-Instruments user study for taking the time to provide feedback on our techniques, and the reviewers of this submission for their constructive suggestions to improve this research.

\end{acks}




\bibliographystyle{ACM-Reference-Format}
\bibliography{References}

\onecolumn
\appendix


\end{document}
