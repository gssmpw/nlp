\section{Conclusion}
We operationalized the theory of instrumental interaction for generative AI, with an in-depth unpacking of the principles of reification of user intent, reflection, and grounding. We argue that leveraging this re-appropriated and refined theory can drive the creation of a \textit{new generation of expressive AI-Instruments} that afford better expression of intent, make it easier to discover what is possible, and provide powerful degrees of freedom for steering the generation towards the best possible results. Those new tools and instruments can truly leverage the polymorphic and non-deterministic behavior of generative AI models, unleashing new and empowering forms of expressive HCI+AI experiences. 

Beyond our focus on AI-Instruments, theories play an important role in the advancement of our wider research field~\cite{rogers_hci_2012, halverson_activity_2002}. Rogers argues that there is a need for theories as lenses bringing critical design characteristics into focus, and which can function as a generative source: providing "\textit{design dimensions and constructs to inform the design and selection of interactive representations}"~\cite{rogers_new_2004}. We hope that our work on operationalizing the theory of instrumental interaction for AI can inspire other new -- and re-appropriated -- theories to advance HCI+AI. 








