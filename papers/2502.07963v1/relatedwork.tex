\section{Related Work}
\label{section:related_work}

% - publication bias & spin generally in medical literature (outside of NLP)
\paragraph{Spin in medical literature} Classification and prevalence of spin in medical literature has been studied extensively. 
In this literature, spin is commonly defined in one or more of the following ways: (1) Distorting the interpretation of results to present them more favorably, resulting in misleading conclusions; (2) Discrepancies between results and their favorable interpretation; (3) Attributing causality when the study design does not support it; and (4) Inappropriate extrapolation of results \citep{chiu2017spin}.

A systematic review revealed that spin is more prevalent in trials with nonsignificant primary outcomes and those with a higher risk of bias, such as nonrandomized studies \citep{chiu2017spin}. For instance, $\sim$60\% of RCTs with nonsignificant primary outcomes that were evaluated featured spin in their abstracts. 
Beyond RCTs, spin has also been identified as a common issue in systematic reviews and meta-analyses \citep{yavchitz2016new, nowlin2022spin, qureshi2024development}. Research has demonstrated the presence of spin across various medical disciplines, including oncology \citep{boutron2010reporting, wayant2019evaluation}, psychiatry and psychology \citep{jellison2020evaluation}, dental care \citep{su2023assessment}, wound care \citep{lockyer2013spin}, cardiovascular medicine \citep{khan2019level}, rheumatology \citep{mathieu2012misleading}, obesity research \citep{austin2019evaluation}, and emergency medicine \citep{reynolds2020evaluation}.

% - NLP for detecting spin 
\paragraph{NLP for detecting spin} One of the earliest efforts in automated spin detection was by \citet{koroleva2017contribution}, who proposed a pipeline that used rule-based and deep learning methods to identify key entities and comparative sentences (e.g., ``Treatment A was better than treatment B in terms of efficacy.'') to analyze RCT abstracts from PubMed.
This laid the foundation for subsequent research, including the creation of a corpus of biomedical articles annotated for spin \citep{koroleva2018annotating} and the development of DeSpin, a prototype tool designed to assist authors and reviewers in detecting spin in RCT abstracts \citep{koroleva2020despin}. 
DeSpin adopted a combination of rule-based methods and machine learning models---BERT \citep{devlin2018bert}, BioBERT \citep{lee2020biobert}, and SciBERT \citep{beltagy2019scibert}---to extract trial design information and identify spin. 
Our work extends this line of inquiry, evaluating modern LLMs for spin-related tasks.

% - NLP for summarizing/simplifying medical texts
\paragraph{NLP for simplifying medical texts} Several studies have explored the use of NLP technologies to simplify medical texts, which are often laden with jargon, to make them more accessible to lay readers \citep{ondov2022survey}. Recent advancements in deep learning and LLMs have enabled the simplification of medical abstracts \citep{devaraj2021paragraph, shaib-etal-2023-summarizing} and patient notes, including radiology reports \citep{bala2020patient, jeblick2024chatgpt}. An interactive tool has also been developed to provide simplified summaries of medical articles, assisting lay readers in understanding complex medical texts \citep{august2023paper}. Additionally, efforts have been made to improve evaluation methods for medical text simplification by creating high-quality corpora \citep{devaraj2022evaluating, joseph2023multilingual}. However, existing work does not address spin in medical literature.