\section{Related work}
\label{related2}
As we mentioned earlier, there is a trade-off problem with ANFIS related to its interpretability and accuracy. The ability of fuzzy models to describe their systems' patterns is called interpretability. Some studies concurred that the model structure, the number of input variables, the number of fuzzy rules, the number of linguistic words, and the form of the fuzzy sets are all aspects of interpretability. Because they affect the system's complexity and time consumption. The definition of accuracy relates to the fuzzy model's ability to describe the system being modeled accurately \cite{rini2013balanced}. Numerous attempts have addressed ANFIS issues by enhancing either its interpretability or accuracy, or both using optimization techniques. We have explored the literature and discovered that researchers have pursued two main avenues to resolve these problems, which we will elaborate on in the subsequent sub-sections.
\subsection{Issues with ANFIS Training and Overfitting}
The training of ANFIS architecture constitutes an optimization process to find the best values for its antecedent and consequent parameters. Commonly used derivative-based learning algorithms for this purpose include the Levenberg Marquardt (LM) \cite{ranganathan2004levenberg}, backpropagation (BP)\cite{rumelhart1995backpropagation}, Kalman filter (KF) \cite{welch1995introduction}, and gradient descent (GD) algorithms \cite{baldi1995gradient}. These, however, carry the potential risk of local minimum problems due to the chain rule, and calculating the gradient at each step can be challenging. Additionally, the effectiveness of these algorithms heavily depends on the initial values, and the convergence of the parameters can be relatively slow. Recently, several ANFIS studies have substituted these learning algorithms with evolution optimization or metaheuristic optimization algorithms, including the Genetic Algorithm (GA) \cite{whitley1994genetic}, Differential Evolution (DE) \cite{price2013differential}, Artificial Bee Colony (ABC) \cite{karaboga2010artificial}, and Particle Swarm Optimization (PSO) \cite{chen2013hybrid}.

In \cite{kurniawan2018premise}, the researchers proposed a combination of particle swarm optimization and genetic algorithm. Although PSO is known for its robustness and fast solving of non-linear problems, its quickness could lead to local optimum solution space convergence. To tackle this problem, the researchers merged the GA algorithm with PSO to develop ANFIS-PSOGA for optimal premise parameters.

In \cite{ebrahimi2021accuracy}, a more specific and interpretable fuzzy model (ANFIS-BAT) was proposed for predicting dust concentration in both cold and warm months across semi-arid regions of Iran. The researchers employed the bat algorithm for fine-tuning the premise and consequent parameters of the ANFIS network to minimize the cost function in the learning process. Another training algorithm, BWOA-ANFIS \cite{tightiz2020intelligent}, was proposed to replace the gradient descent in traditional ANFIS. The authors of this study applied the association rules learning technique (ARLT) and then tuned the premise and consequent parameters by utilizing the Black Widow Optimization Algorithm (BWOA).
In the study of  \cite{rajeshwari2022dermatology}, the authors introduced an ANFIS-FA methodology. This system utilized ANFIS  combined with subtractive clustering (SC). This model's unique aspect was using a firefly optimization technique (FA) to improve the optimization of all SC parameters. These parameters, which included the range of influence, squash ratio, accept ratio, and reject ratio, were explicitly optimized to enhance the system's ability to classify and diagnose skin cancer at its early stages.

In \cite{salleh2017optimization}, both the premise and consequent parameters were optimized using the artificial bee colony (ABC) optimization algorithm to enhance the precision of ANFIS in classifying Malaysian SMEs. The ABC algorithm was employed to update these parameters in forward and backward passes instead of the traditional hybrid learning algorithms in conventional ANFIS. Although this technique demonstrated high accuracy, the ABC requires a more efficient exploration strategy.

\subsection{Issues with ANFIS Interpretability and Complexity}\label{related2}

To enhance the interpretability of ANFIS, researchers have attempted to optimize the rule base using various techniques, such as reducing the number of features using feature selection techniques or eliminating redundant rules using different pruning methods.

The process of reducing the number of input variables leads to a lesser number of system parameters and, thus, a reduced number of generated rules. Consequently, this results in a more interpretable model and a less complex ANFIS structure. Several techniques are available for the feature selection, including traditional techniques or filter methods that depend on distance measurement and redundancy features \cite{li2022feature}\cite{saberi2022dual}. Other strategies known as wrapper methods utilize evolutionary algorithms to evaluate the best-selected features, such as binary particle swarm optimization (BPSO), ant colony optimization (ACO), and GA. A classifier is used as an objective function to calculate the minimum error of each subset.
In \cite{rahchamani2021hybrid}, a feature selection based on a genetic algorithm was proposed as the main contribution to simplifying ANFIS complexity to reach high performance by reducing the feature vector in the concrete production industry.
In the study of \cite{birgani2019optimization}, the accuracy of ANFIS was significantly improved while maintaining a less complicated architecture through feature selection. They utilized Principal Component Analysis (PCA) to classify brain tumor MRI images. The efficiency of their technique was demonstrated through the generation of fewer fuzzy rules and an improvement in system accuracy. This was achieved by integrating image segmentation with thresholding techniques and increasing the number of iterations.
Another similar approach is dividing the input variables based on information granules such as fuzzy sets, then using the Apriori algorithm to create an initial rule base that efficiently derives high-quality, interpretable rules from vast datasets. These rules were concurrently selected and tuned to improve the model's performance, as seen in \cite{fazzolari2014multi}, and a similar idea is in \cite{antonelli2016multi}.

Optimizing the rule base based on rule growing and pruning techniques has been implemented to minimize the rule base while maximizing accuracy. The rule base is a crucial part of any fuzzy inference system (FIS), and the quality of its results hinges on the efficacy of these rules. Not all generated rules are essential or contribute significantly to improving accuracy. Many of them are inefficient and can be pruned to reduce the complexity of the FIS system \cite{rini2013balanced}\cite{hussain2015analysis}.

Several techniques have been proposed for rule-based optimization. One such technique is \textit{clustering}, as in \cite{leonori2020generalized}, where various clustering techniques ((one partitional and two other hierarchical strategies) have been proposed to synthesize ANFIS. They addressed the issue of membership function overlap by considering the input space. After the clustering process, a Min-Max classifier is used to refine the membership function definitions. This process generates a limited number of rules while ensuring coverage of the entire input space.
In \cite{suraj2016jaya}, a different clustering technique was adopted to expedite training time, prune irrelevant rules, and enhance the accuracy of classifying motor imagery tasks for controlling light-emitting diodes. Their methodology involved splitting the dataset into two clusters using the k-means clustering algorithm and triggering rules based on each cluster. The Jaya algorithm was combined with ANFIS to determine each group's optimal number of features, subsequently informing the rules' triggering.

Another clustering technique was used in \cite{pramod2021k} to improve the interpretability of grid partitioning ANFIS. This technique proposed K-Means clustering-based Extreme Learning ANFIS (KMELANFIS) for regression purposes. The input space was clustered, and the clustering centers were used to initialize the membership function parameters. The membership functions were reduced using a similarity index method between adjacent ones. Finally, an extreme learning machine (ELM) was used to compute the consequent parameters. However, this technique is limited when applied to a low-dimensional dataset. 

Another approach to rules pruning is to use \textit{thresholding techniques}, as seen in \cite{rini2013balanced} and  \cite{wang2012assessment}, where the elimination of non-essential rules was achieved by employing a threshold set by an expert. All rules with firing strengths below this threshold were discarded. However, these methodologies, reliant on human experts for threshold determination, faced challenges. Especially in cases where expert opinions conflicted based on data types or specific application requirements, choosing the correct threshold value proved problematic. Moreover, choosing to prune rules based on a pre-selected threshold could risk removing some significant rules, negatively impacting the system's accuracy. One possible solution is to use an adaptive threshold, as in \cite{guendouzi2021new}, where the authors used a new threshold-based fitness function that is adaptive.
As outlined in \cite{owoseni2020improved}, an enhanced ANFIS technique leverages a probability trajectory and a k-nearest neighbor-based clustering ensemble to refine its rules. This approach empirically establishes a threshold to select the optimal number of clusters, which can confine this method to a specific dataset type.

A solution to the rules explosion problem, Patch Learning (PL), was presented in \cite{huang2022fuzzy}. Despite its effectiveness, this technique may increase the number of patches, leading to heightened system complexity.

In \cite{held2006extracting}, a unique solution was proposed. The authors designed a three-layer ANFIS architecture for healthy infant sleep classification, utilizing a simple rule-elimination process to balance interpretability and accuracy. They modified the third layer of ANFIS to correspond with the five classes in the fifth layer of conventional ANFIS. Each node in the third layer performed a weighted sum operation of the incoming rules' firing strengths, later altered by the sigmoid function. Consequently, any rule with a normalized average contribution lower than an empirically chosen 7\% threshold was pruned due to its insignificant contribution to the classification.
Moreover, they further streamlined the rules by merging those sharing the same output class and only differing in the fuzzy concept linked with one pattern. A similar rule combination and feature selection methodology were employed in \cite{feng2020accuracy}, where the researchers employed the CFBLS model, which uses a single TSK fuzzy system, streamlining rule interpretation. The input space in CFBLS is uniformly partitioned, typically into 2, 4, or 6 parts, for better data coverage. They adopted a random selection method for features and rules to counter the "rule explosion" arising from numerous features, using a rule-combination matrix and a "don't care" matrix. Parameters were optimized through a ten-fold cross-validation coupled with a grid search. After conducting experiments 30 times per dataset, their approach aimed to harmonize accuracy and complexity in fuzzy learning systems.

In \cite{do2021prediction}, rule optimization was achieved by minimizing the node count, which correlates with the number of generated rules. The authors used a "rule-drop" technique that randomly activated and deactivated nodes in the fuzzification layer during each training step. The choice of nodes was based on probability values that served as a hyperparameter to retain a neuron within the network.
Some techniques, like the one in \cite{huhn2009furia}, bypass the pruning phase and directly learn the first set of rules from the entire training set. This direct learning from data allows for the rule's antecedents to be learned, and if the rule has no antecedent, a default rule is generated. As a result, there is a set of rules for each class.

Some other techniques restrict the number of generated rules, which is close to the thresholding approach, such as in  \cite{tomasiello2022fractional} where the authors proposed a rule reduction technique for regression purposes by using the least-squares method with fractional Tikhonov regularization, and the number of rules restricted to the number of fuzzy terms for each variable resulting in a simplified rule base that efficiently manages high-dimensional inputs while maintaining accuracy.
Another type of thresholding attempts in \cite{alcala2011fuzzy}, where the authors used a search tree for rules generation to list all possible fuzzy item sets of a class, with attributes having an order, and utilizing support thresholds to limit rule expansions; then the candidate Rule Prescreening for subgroup discovery has been conducted to select the most interesting rules by weighting patterns to ensure diverse rule coverage; and finally the genetic post-processing for rule selection and parameter tuning. 

Apart from these techniques, there are several other proposed solutions to improve the interpretability of ANFIS, such as similarity analysis \cite{rajab2019handling}, frequent pattern mining \cite{marimuthu2019oafpm}, and equalization of fuzzy rules with the membership functions\cite{tomasiello2022fractional} that also managed to achieve a degree of an interpretable framework.

In the existing literature, several notable voids have been identified in the context of optimizing ANFIS models:
\begin{itemize}
    \item Feature Selection and Rule Generation Trade-off: Prior research has focused on feature selection techniques to improve ANFIS performance. However, a critical concern is the trade-off between reducing features and maintaining effective rule generation. When reducing features, the number of rules generated might decrease, potentially leading to the exclusion of crucial rules and impacting accuracy. This study takes a distinct approach by using a complete set of features to generate a comprehensive rule set, aiming to balance interpretability and accuracy.

    \item Limitations of Clustering Techniques in ANFIS Pruning: While clustering techniques have shown promise in ANFIS rule pruning, there are still open questions regarding their limitations, especially hierarchical clustering. Issues like sensitivity to data point ordering, scalability concerns, and imbalanced cluster generation need careful consideration. This research introduces grid partitioning as an alternative to hierarchical clustering by uniformly partitioning the relevant domain to address these drawbacks and enhance the ANFIS rule pruning quality.

     \item Subjectivity and Challenges in Thresholding Techniques: Existing literature often relies on subjective expert opinions to select threshold values for ANFIS rule pruning, raising concerns about applicability across diverse data types and domains. This gap highlights the need for objective thresholding techniques that determine optimal pruning thresholds based on data characteristics. The study introduces Binary Particle Swarm Optimization (BPSO) to address this subjectivity concern, enhancing the precision and adaptability of thresholding techniques in ANFIS optimization.

    \item Firing Strengths as an Automatic Rule Pruning Metric: The literature indicates that utilizing firing strengths for rule pruning in ANFIS has been relatively underexplored. Firing strengths within ANFIS offer insights into the significance of individual rules. There is an opportunity to develop techniques that leverage firing strengths as an automatic rule-pruning metric, potentially improving the interpretability and accuracy of ANFIS models.
\end{itemize}