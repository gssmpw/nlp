%%%%%%%%%%%%%%%%%%%%%%%%%%%%%%%%%%%%%%%%%%%%%%%%%%%%%%%%%%%%%%%%%%%%%%%%%%%%
%% Author template for Operations Reseacrh (opre) for articles with no e-companion (EC)
%% Mirko Janc, Ph.D., INFORMS, mirko.janc@informs.org
%% ver. 0.95, December 2010
%%%%%%%%%%%%%%%%%%%%%%%%%%%%%%%%%%%%%%%%%%%%%%%%%%%%%%%%%%%%%%%%%%%%%%%%%%%%
% \documentclass[fleqn,mksc,blindrev]{informs4}
% \documentclass[fleqn,mksc,nonblindrev]{informs4}
\documentclass[fleqn,nonblindrev]{informs4}
%%\documentclass[opre,nonblindrev]{informs3_modified} % current default for manuscript submission

\OneAndAHalfSpacedXI % current default line spacing
%\OneAndAHalfSpacedXII
%\DoubleSpacedXII
%%\DoubleSpacedXI

% If hyperref is used, dvi-to-ps driver of choice must be declared as
%   an additional option to the \documentclass. For example
%\documentclass[dvips,opre]{informs3}      % if dvips is used
%\documentclass[dvipsone,opre]{informs3}   % if dvipsone is used, etc.

%%% OPRE uses endnotes. If you do not use them, put a percent sign before
%%% the \theendnotes command. This template does show how to use them.
\usepackage{endnotes}
\let\footnote=\endnote
% \let\enotesize=\normalsize
\let\enotesize=\small   % Yuyan's change
\def\notesname{Endnotes}%
\def\makeenmark{$^{\theenmark}$}
\def\enoteformat{\rightskip0pt\leftskip0pt\parindent=1.75em
	\leavevmode\llap{\theenmark.\enskip}}

% Private macros here (check that there is no clash with the style)
% figure packages
\usepackage{graphicx}
\usepackage{eqndefns-left}
\usepackage{multirow}
\usepackage{hhline}

% caption package
\usepackage[small, margin=1cm]{caption}

% appendix package
\usepackage{appendix}
% color packages
\usepackage{color}
\definecolor{strcolor}{rgb}{0.6, 0.2, 0.6}
\definecolor{commentcolor}{rgb}{0.3125, 0.5, 0.3125}
\definecolor{keycol}{rgb}{0, 0, 1}


% revision
\newcommand{\rev}[1]{{\color{red} #1}}
\newcommand{\minminc}[1]{{\color{blue} #1}}
\newcommand{\yuyan}[1]{{\color{purple} #1}}

% math packages
%\usepackage{amssymb}
%\usepackage{amsmath}
\usepackage{bbm}

% Code package
\usepackage{listings}
\lstset{
	emph={ROVar, ROUn, ROVarDR, ROExpr, RONormInf, RONorm1, RONorm2,ROConstraint,ROExpect, ROSq, ROConstraintSet,ROIntVar,ROBinVar, ROInfinity,ROModel,ROVarDRArray, ROVarArray, ROMinimize,ROUnArray, ROAbs, ROPos, ROSum, int},emphstyle={\color{strcolor}\bfseries},
	keywordstyle={\color{blue}\bfseries},
	commentstyle={\color{commentcolor}},
	stringstyle={\color{strcolor}\bfseries},
	language=C++,                % choose the language of the code
	basicstyle={\ttfamily\footnotesize}, % the size of the fonts that are used for the code
	numbers=left,                   % where to put the line-numbers
	numberstyle=\footnotesize,      % the size of the fonts that are used for the line-numbers
	stepnumber=1,                   % the step between two line-numbers. If it's 1 each line will be numbered
	numbersep=5pt,                  % how far the line-numbers are from the code
	backgroundcolor=\color{white},  % choose the background color. You must add \usepackage{color}
	showspaces=false,               % show spaces adding particular underscores
	showstringspaces=false,         % underline spaces within strings
	showtabs=false,                 % show tabs within strings adding particular underscores
	frame=single,	                	% adds a frame around the code
	tabsize=2,	                		% sets default tabsize to 2 spaces
	captionpos=b,                   % sets the caption-position to bottom
	breaklines=true,                % sets automatic line breaking
	breakatwhitespace=false,        % sets if automatic breaks should only happen at whitespace
	escapeinside={\%*}{*)},         % if you want to add a comment within your code
	keywords=[1]{for, break, if, else, function}
}
\renewcommand{\lstlistingname}{Code Segment}

% hyperlinks packages
%\usepackage{hyperref}
\usepackage{url}

% numbering
%\numberwithin{equation}{section}
%\numberwithin{table}{section}
%\numberwithin{figure}{section}

% Equation environments
\newcommand {\bea}{\begin{eqnarray}}
	\newcommand {\eea}{\end{eqnarray}}
% \newcommand {\E}[1]{\mathrm{E}\left( #1 \right)}
\newcommand {\Ep}[2]{{\mathrm{E}_{\mathbb{P}_{#1}} \left( #2 \right)}}
\newcommand {\supEp}[1]{\displaystyle \sup_{\mathbb{P} \in \mathbb{F}} \Ep{}{#1}}
\newcommand {\supEpf}[2]{\displaystyle \sup_{\mathbb{P} \in \mathbb{F}_{#1}} \Ep{}{#2}}
\newcommand {\pos}[1]{\paren{#1}^+}
\renewcommand {\neg}[1]{\paren{#1}^-}
\newcommand {\pibound}[1]{\ensuremath{\pi^{#1}\paren{r^0, \mb{r}}}}
\newcommand {\etabound}[1]{\ensuremath{\eta^{#1}\paren{r^0, \mb{r}}}}
\newcommand \conv {\mathrm{conv}}
\newcommand {\p}{{\rm P}}
% mb
\newcommand{\mb}[1]{\mbox{\boldmath \ensuremath{#1}}}
\newcommand{\mbt}[1]{\mb{\tilde{#1}}}
\newcommand{\mbb}[1]{\mb{\bar{#1}}}
\newcommand{\mbbs}[1]{\mbb{\scriptstyle{#1}}}
\newcommand{\mbh}[1]{\mb{\hat{#1}}}
\newcommand{\mbth}[1]{\mbt{\hat{#1}}}
\newcommand{\mbc}[1]{\mb{\check{#1}}}
\newcommand{\mbtc}[1]{\mbt{\check{#1}}}
\newcommand{\mbs}[1]{\mb{\scriptstyle{#1}}}
\newcommand{\mbst}[1]{\mbs{\tilde{#1}}}
\newcommand{\mbsh}[1]{\mbs{\hat{#1}}}
% mc
%\newcommand{\mc}[1]{\mbox{\ensuremath{\mathcal{#1}}}}
\newcommand{\mch}[1]{\hat{\mc{#1}}}
\newcommand{\mcs}[1]{\mc{\scriptstyle{#1}}}
\newcommand{\mcss}[1]{\mc{\scriptscriptstyle{#1}}}
\newcommand{\mcsh}[1]{\hat{\mcs{#1}}}
\newcommand{\mbss}[1]{{\mbox{\boldmath \tiny{$#1$}}}}
\newcommand{\eucnorm}[1]{\left\| #1 \right\|_2}
\newcommand{\dpv}{\displaystyle \vspace{3pt}}
\newcommand{\diag}[1]{\textbf{diag}\paren{#1}}
\newcommand{\yldrk}{\mb{y}^k\paren{\mbt{z}}}
\newcommand{\abs}[1]{\left| #1 \right|}
\DeclareMathOperator{\CVaR}{CVaR}
\DeclareMathOperator{\VaR}{VaR}
%\DeclareMathOperator{\argmin}{\arg\min}
% combinations
\renewcommand{\mbc}[1]{\mb{\mc{#1}}}
%misc
\newcommand{\ceil}[1]{\left\lceil #1  \right\rceil}

% \newtheorem{theorem}{Theorem}
% \newtheorem{lemma}[theorem]{Lemma}
\newtheorem{conj}{Conjecture}
\newtheorem{coro}{Corollary}
%\newtheorem{claim}{Claim}
\newtheorem{Defi}{Definition}
\newtheorem{algorithm}{Algorithm}
%\newtheorem{assumption}{Assumption}

\newcommand{\eg}{\textit{e.g.}}
\newcommand{\ie}{\textit{i.e.}}
\renewcommand{\Re}{\mathbb{R}}

%\renewcommand{\bigtimes}{\mathop{\rm \text{\Large{$\times$}}}}

\def\blot{\quad \mbox{$\vcenter{ \vbox{ \hrule height.4pt
				\hbox{\vrule width.4pt height.9ex \kern.9ex \vrule width.4pt}
				\hrule height.4pt}}$}}

% Natbib setup for author-year style
\usepackage{natbib}
\bibpunct[, ]{(}{)}{,}{a}{}{,}%
\def\bibfont{\fontsize{8}{9.5}\selectfont}%
\def\bibsep{0pt}%
\def\bibhang{16pt}%
\def\newblock{\ }%
\def\BIBand{and}%

% Pan's version from online:
% % Natbib setup for author-number style
% \usepackage{natbib}
%  \bibpunct[, ]{(}{)}{,}{a}{}{,}%
%  \def\bibfont{\small}%
%  \def\bibsep{\smallskipamount}%
%  \def\bibhang{24pt}%
%  \def\newblock{\ }%
%  \def\BIBand{and}%

%Yuyan's additions
\def\todo#1{\textcolor{red}{TODO: #1}}
%%%%% NEW MATH DEFINITIONS %%%%%

\usepackage{amsmath,amsfonts,bm}
\usepackage{derivative}
% Mark sections of captions for referring to divisions of figures
\newcommand{\figleft}{{\em (Left)}}
\newcommand{\figcenter}{{\em (Center)}}
\newcommand{\figright}{{\em (Right)}}
\newcommand{\figtop}{{\em (Top)}}
\newcommand{\figbottom}{{\em (Bottom)}}
\newcommand{\captiona}{{\em (a)}}
\newcommand{\captionb}{{\em (b)}}
\newcommand{\captionc}{{\em (c)}}
\newcommand{\captiond}{{\em (d)}}

% Highlight a newly defined term
\newcommand{\newterm}[1]{{\bf #1}}

% Derivative d 
\newcommand{\deriv}{{\mathrm{d}}}

% Figure reference, lower-case.
\def\figref#1{figure~\ref{#1}}
% Figure reference, capital. For start of sentence
\def\Figref#1{Figure~\ref{#1}}
\def\twofigref#1#2{figures \ref{#1} and \ref{#2}}
\def\quadfigref#1#2#3#4{figures \ref{#1}, \ref{#2}, \ref{#3} and \ref{#4}}
% Section reference, lower-case.
\def\secref#1{section~\ref{#1}}
% Section reference, capital.
\def\Secref#1{Section~\ref{#1}}
% Reference to two sections.
\def\twosecrefs#1#2{sections \ref{#1} and \ref{#2}}
% Reference to three sections.
\def\secrefs#1#2#3{sections \ref{#1}, \ref{#2} and \ref{#3}}
% Reference to an equation, lower-case.
\def\eqref#1{equation~\ref{#1}}
% Reference to an equation, upper case
\def\Eqref#1{Equation~\ref{#1}}
% A raw reference to an equation---avoid using if possible
\def\plaineqref#1{\ref{#1}}
% Reference to a chapter, lower-case.
\def\chapref#1{chapter~\ref{#1}}
% Reference to an equation, upper case.
\def\Chapref#1{Chapter~\ref{#1}}
% Reference to a range of chapters
\def\rangechapref#1#2{chapters\ref{#1}--\ref{#2}}
% Reference to an algorithm, lower-case.
\def\algref#1{algorithm~\ref{#1}}
% Reference to an algorithm, upper case.
\def\Algref#1{Algorithm~\ref{#1}}
\def\twoalgref#1#2{algorithms \ref{#1} and \ref{#2}}
\def\Twoalgref#1#2{Algorithms \ref{#1} and \ref{#2}}
% Reference to a part, lower case
\def\partref#1{part~\ref{#1}}
% Reference to a part, upper case
\def\Partref#1{Part~\ref{#1}}
\def\twopartref#1#2{parts \ref{#1} and \ref{#2}}

\def\ceil#1{\lceil #1 \rceil}
\def\floor#1{\lfloor #1 \rfloor}
\def\1{\bm{1}}
\newcommand{\train}{\mathcal{D}}
\newcommand{\valid}{\mathcal{D_{\mathrm{valid}}}}
\newcommand{\test}{\mathcal{D_{\mathrm{test}}}}

\def\eps{{\epsilon}}


% Random variables
\def\reta{{\textnormal{$\eta$}}}
\def\ra{{\textnormal{a}}}
\def\rb{{\textnormal{b}}}
\def\rc{{\textnormal{c}}}
\def\rd{{\textnormal{d}}}
\def\re{{\textnormal{e}}}
\def\rf{{\textnormal{f}}}
\def\rg{{\textnormal{g}}}
\def\rh{{\textnormal{h}}}
\def\ri{{\textnormal{i}}}
\def\rj{{\textnormal{j}}}
\def\rk{{\textnormal{k}}}
\def\rl{{\textnormal{l}}}
% rm is already a command, just don't name any random variables m
\def\rn{{\textnormal{n}}}
\def\ro{{\textnormal{o}}}
\def\rp{{\textnormal{p}}}
\def\rq{{\textnormal{q}}}
\def\rr{{\textnormal{r}}}
\def\rs{{\textnormal{s}}}
\def\rt{{\textnormal{t}}}
\def\ru{{\textnormal{u}}}
\def\rv{{\textnormal{v}}}
\def\rw{{\textnormal{w}}}
\def\rx{{\textnormal{x}}}
\def\ry{{\textnormal{y}}}
\def\rz{{\textnormal{z}}}

% Random vectors
\def\rvepsilon{{\mathbf{\epsilon}}}
\def\rvphi{{\mathbf{\phi}}}
\def\rvtheta{{\mathbf{\theta}}}
\def\rva{{\mathbf{a}}}
\def\rvb{{\mathbf{b}}}
\def\rvc{{\mathbf{c}}}
\def\rvd{{\mathbf{d}}}
\def\rve{{\mathbf{e}}}
\def\rvf{{\mathbf{f}}}
\def\rvg{{\mathbf{g}}}
\def\rvh{{\mathbf{h}}}
\def\rvu{{\mathbf{i}}}
\def\rvj{{\mathbf{j}}}
\def\rvk{{\mathbf{k}}}
\def\rvl{{\mathbf{l}}}
\def\rvm{{\mathbf{m}}}
\def\rvn{{\mathbf{n}}}
\def\rvo{{\mathbf{o}}}
\def\rvp{{\mathbf{p}}}
\def\rvq{{\mathbf{q}}}
\def\rvr{{\mathbf{r}}}
\def\rvs{{\mathbf{s}}}
\def\rvt{{\mathbf{t}}}
\def\rvu{{\mathbf{u}}}
\def\rvv{{\mathbf{v}}}
\def\rvw{{\mathbf{w}}}
\def\rvx{{\mathbf{x}}}
\def\rvy{{\mathbf{y}}}
\def\rvz{{\mathbf{z}}}

% Elements of random vectors
\def\erva{{\textnormal{a}}}
\def\ervb{{\textnormal{b}}}
\def\ervc{{\textnormal{c}}}
\def\ervd{{\textnormal{d}}}
\def\erve{{\textnormal{e}}}
\def\ervf{{\textnormal{f}}}
\def\ervg{{\textnormal{g}}}
\def\ervh{{\textnormal{h}}}
\def\ervi{{\textnormal{i}}}
\def\ervj{{\textnormal{j}}}
\def\ervk{{\textnormal{k}}}
\def\ervl{{\textnormal{l}}}
\def\ervm{{\textnormal{m}}}
\def\ervn{{\textnormal{n}}}
\def\ervo{{\textnormal{o}}}
\def\ervp{{\textnormal{p}}}
\def\ervq{{\textnormal{q}}}
\def\ervr{{\textnormal{r}}}
\def\ervs{{\textnormal{s}}}
\def\ervt{{\textnormal{t}}}
\def\ervu{{\textnormal{u}}}
\def\ervv{{\textnormal{v}}}
\def\ervw{{\textnormal{w}}}
\def\ervx{{\textnormal{x}}}
\def\ervy{{\textnormal{y}}}
\def\ervz{{\textnormal{z}}}

% Random matrices
\def\rmA{{\mathbf{A}}}
\def\rmB{{\mathbf{B}}}
\def\rmC{{\mathbf{C}}}
\def\rmD{{\mathbf{D}}}
\def\rmE{{\mathbf{E}}}
\def\rmF{{\mathbf{F}}}
\def\rmG{{\mathbf{G}}}
\def\rmH{{\mathbf{H}}}
\def\rmI{{\mathbf{I}}}
\def\rmJ{{\mathbf{J}}}
\def\rmK{{\mathbf{K}}}
\def\rmL{{\mathbf{L}}}
\def\rmM{{\mathbf{M}}}
\def\rmN{{\mathbf{N}}}
\def\rmO{{\mathbf{O}}}
\def\rmP{{\mathbf{P}}}
\def\rmQ{{\mathbf{Q}}}
\def\rmR{{\mathbf{R}}}
\def\rmS{{\mathbf{S}}}
\def\rmT{{\mathbf{T}}}
\def\rmU{{\mathbf{U}}}
\def\rmV{{\mathbf{V}}}
\def\rmW{{\mathbf{W}}}
\def\rmX{{\mathbf{X}}}
\def\rmY{{\mathbf{Y}}}
\def\rmZ{{\mathbf{Z}}}

% Elements of random matrices
\def\ermA{{\textnormal{A}}}
\def\ermB{{\textnormal{B}}}
\def\ermC{{\textnormal{C}}}
\def\ermD{{\textnormal{D}}}
\def\ermE{{\textnormal{E}}}
\def\ermF{{\textnormal{F}}}
\def\ermG{{\textnormal{G}}}
\def\ermH{{\textnormal{H}}}
\def\ermI{{\textnormal{I}}}
\def\ermJ{{\textnormal{J}}}
\def\ermK{{\textnormal{K}}}
\def\ermL{{\textnormal{L}}}
\def\ermM{{\textnormal{M}}}
\def\ermN{{\textnormal{N}}}
\def\ermO{{\textnormal{O}}}
\def\ermP{{\textnormal{P}}}
\def\ermQ{{\textnormal{Q}}}
\def\ermR{{\textnormal{R}}}
\def\ermS{{\textnormal{S}}}
\def\ermT{{\textnormal{T}}}
\def\ermU{{\textnormal{U}}}
\def\ermV{{\textnormal{V}}}
\def\ermW{{\textnormal{W}}}
\def\ermX{{\textnormal{X}}}
\def\ermY{{\textnormal{Y}}}
\def\ermZ{{\textnormal{Z}}}

% Vectors
\def\vzero{{\bm{0}}}
\def\vone{{\bm{1}}}
\def\vmu{{\bm{\mu}}}
\def\vtheta{{\bm{\theta}}}
\def\vphi{{\bm{\phi}}}
\def\va{{\bm{a}}}
\def\vb{{\bm{b}}}
\def\vc{{\bm{c}}}
\def\vd{{\bm{d}}}
\def\ve{{\bm{e}}}
\def\vf{{\bm{f}}}
\def\vg{{\bm{g}}}
\def\vh{{\bm{h}}}
\def\vi{{\bm{i}}}
\def\vj{{\bm{j}}}
\def\vk{{\bm{k}}}
\def\vl{{\bm{l}}}
\def\vm{{\bm{m}}}
\def\vn{{\bm{n}}}
\def\vo{{\bm{o}}}
\def\vp{{\bm{p}}}
\def\vq{{\bm{q}}}
\def\vr{{\bm{r}}}
\def\vs{{\bm{s}}}
\def\vt{{\bm{t}}}
\def\vu{{\bm{u}}}
\def\vv{{\bm{v}}}
\def\vw{{\bm{w}}}
\def\vx{{\bm{x}}}
\def\vy{{\bm{y}}}
\def\vz{{\bm{z}}}

% Elements of vectors
\def\evalpha{{\alpha}}
\def\evbeta{{\beta}}
\def\evepsilon{{\epsilon}}
\def\evlambda{{\lambda}}
\def\evomega{{\omega}}
\def\evmu{{\mu}}
\def\evpsi{{\psi}}
\def\evsigma{{\sigma}}
\def\evtheta{{\theta}}
\def\eva{{a}}
\def\evb{{b}}
\def\evc{{c}}
\def\evd{{d}}
\def\eve{{e}}
\def\evf{{f}}
\def\evg{{g}}
\def\evh{{h}}
\def\evi{{i}}
\def\evj{{j}}
\def\evk{{k}}
\def\evl{{l}}
\def\evm{{m}}
\def\evn{{n}}
\def\evo{{o}}
\def\evp{{p}}
\def\evq{{q}}
\def\evr{{r}}
\def\evs{{s}}
\def\evt{{t}}
\def\evu{{u}}
\def\evv{{v}}
\def\evw{{w}}
\def\evx{{x}}
\def\evy{{y}}
\def\evz{{z}}

% Matrix
\def\mA{{\bm{A}}}
\def\mB{{\bm{B}}}
\def\mC{{\bm{C}}}
\def\mD{{\bm{D}}}
\def\mE{{\bm{E}}}
\def\mF{{\bm{F}}}
\def\mG{{\bm{G}}}
\def\mH{{\bm{H}}}
\def\mI{{\bm{I}}}
\def\mJ{{\bm{J}}}
\def\mK{{\bm{K}}}
\def\mL{{\bm{L}}}
\def\mM{{\bm{M}}}
\def\mN{{\bm{N}}}
\def\mO{{\bm{O}}}
\def\mP{{\bm{P}}}
\def\mQ{{\bm{Q}}}
\def\mR{{\bm{R}}}
\def\mS{{\bm{S}}}
\def\mT{{\bm{T}}}
\def\mU{{\bm{U}}}
\def\mV{{\bm{V}}}
\def\mW{{\bm{W}}}
\def\mX{{\bm{X}}}
\def\mY{{\bm{Y}}}
\def\mZ{{\bm{Z}}}
\def\mBeta{{\bm{\beta}}}
\def\mPhi{{\bm{\Phi}}}
\def\mLambda{{\bm{\Lambda}}}
\def\mSigma{{\bm{\Sigma}}}

% Tensor
\DeclareMathAlphabet{\mathsfit}{\encodingdefault}{\sfdefault}{m}{sl}
\SetMathAlphabet{\mathsfit}{bold}{\encodingdefault}{\sfdefault}{bx}{n}
\newcommand{\tens}[1]{\bm{\mathsfit{#1}}}
\def\tA{{\tens{A}}}
\def\tB{{\tens{B}}}
\def\tC{{\tens{C}}}
\def\tD{{\tens{D}}}
\def\tE{{\tens{E}}}
\def\tF{{\tens{F}}}
\def\tG{{\tens{G}}}
\def\tH{{\tens{H}}}
\def\tI{{\tens{I}}}
\def\tJ{{\tens{J}}}
\def\tK{{\tens{K}}}
\def\tL{{\tens{L}}}
\def\tM{{\tens{M}}}
\def\tN{{\tens{N}}}
\def\tO{{\tens{O}}}
\def\tP{{\tens{P}}}
\def\tQ{{\tens{Q}}}
\def\tR{{\tens{R}}}
\def\tS{{\tens{S}}}
\def\tT{{\tens{T}}}
\def\tU{{\tens{U}}}
\def\tV{{\tens{V}}}
\def\tW{{\tens{W}}}
\def\tX{{\tens{X}}}
\def\tY{{\tens{Y}}}
\def\tZ{{\tens{Z}}}


% Graph
\def\gA{{\mathcal{A}}}
\def\gB{{\mathcal{B}}}
\def\gC{{\mathcal{C}}}
\def\gD{{\mathcal{D}}}
\def\gE{{\mathcal{E}}}
\def\gF{{\mathcal{F}}}
\def\gG{{\mathcal{G}}}
\def\gH{{\mathcal{H}}}
\def\gI{{\mathcal{I}}}
\def\gJ{{\mathcal{J}}}
\def\gK{{\mathcal{K}}}
\def\gL{{\mathcal{L}}}
\def\gM{{\mathcal{M}}}
\def\gN{{\mathcal{N}}}
\def\gO{{\mathcal{O}}}
\def\gP{{\mathcal{P}}}
\def\gQ{{\mathcal{Q}}}
\def\gR{{\mathcal{R}}}
\def\gS{{\mathcal{S}}}
\def\gT{{\mathcal{T}}}
\def\gU{{\mathcal{U}}}
\def\gV{{\mathcal{V}}}
\def\gW{{\mathcal{W}}}
\def\gX{{\mathcal{X}}}
\def\gY{{\mathcal{Y}}}
\def\gZ{{\mathcal{Z}}}

% Sets
\def\sA{{\mathbb{A}}}
\def\sB{{\mathbb{B}}}
\def\sC{{\mathbb{C}}}
\def\sD{{\mathbb{D}}}
% Don't use a set called E, because this would be the same as our symbol
% for expectation.
\def\sF{{\mathbb{F}}}
\def\sG{{\mathbb{G}}}
\def\sH{{\mathbb{H}}}
\def\sI{{\mathbb{I}}}
\def\sJ{{\mathbb{J}}}
\def\sK{{\mathbb{K}}}
\def\sL{{\mathbb{L}}}
\def\sM{{\mathbb{M}}}
\def\sN{{\mathbb{N}}}
\def\sO{{\mathbb{O}}}
\def\sP{{\mathbb{P}}}
\def\sQ{{\mathbb{Q}}}
\def\sR{{\mathbb{R}}}
\def\sS{{\mathbb{S}}}
\def\sT{{\mathbb{T}}}
\def\sU{{\mathbb{U}}}
\def\sV{{\mathbb{V}}}
\def\sW{{\mathbb{W}}}
\def\sX{{\mathbb{X}}}
\def\sY{{\mathbb{Y}}}
\def\sZ{{\mathbb{Z}}}

% Entries of a matrix
\def\emLambda{{\Lambda}}
\def\emA{{A}}
\def\emB{{B}}
\def\emC{{C}}
\def\emD{{D}}
\def\emE{{E}}
\def\emF{{F}}
\def\emG{{G}}
\def\emH{{H}}
\def\emI{{I}}
\def\emJ{{J}}
\def\emK{{K}}
\def\emL{{L}}
\def\emM{{M}}
\def\emN{{N}}
\def\emO{{O}}
\def\emP{{P}}
\def\emQ{{Q}}
\def\emR{{R}}
\def\emS{{S}}
\def\emT{{T}}
\def\emU{{U}}
\def\emV{{V}}
\def\emW{{W}}
\def\emX{{X}}
\def\emY{{Y}}
\def\emZ{{Z}}
\def\emSigma{{\Sigma}}

% entries of a tensor
% Same font as tensor, without \bm wrapper
\newcommand{\etens}[1]{\mathsfit{#1}}
\def\etLambda{{\etens{\Lambda}}}
\def\etA{{\etens{A}}}
\def\etB{{\etens{B}}}
\def\etC{{\etens{C}}}
\def\etD{{\etens{D}}}
\def\etE{{\etens{E}}}
\def\etF{{\etens{F}}}
\def\etG{{\etens{G}}}
\def\etH{{\etens{H}}}
\def\etI{{\etens{I}}}
\def\etJ{{\etens{J}}}
\def\etK{{\etens{K}}}
\def\etL{{\etens{L}}}
\def\etM{{\etens{M}}}
\def\etN{{\etens{N}}}
\def\etO{{\etens{O}}}
\def\etP{{\etens{P}}}
\def\etQ{{\etens{Q}}}
\def\etR{{\etens{R}}}
\def\etS{{\etens{S}}}
\def\etT{{\etens{T}}}
\def\etU{{\etens{U}}}
\def\etV{{\etens{V}}}
\def\etW{{\etens{W}}}
\def\etX{{\etens{X}}}
\def\etY{{\etens{Y}}}
\def\etZ{{\etens{Z}}}

% The true underlying data generating distribution
\newcommand{\pdata}{p_{\rm{data}}}
\newcommand{\ptarget}{p_{\rm{target}}}
\newcommand{\pprior}{p_{\rm{prior}}}
\newcommand{\pbase}{p_{\rm{base}}}
\newcommand{\pref}{p_{\rm{ref}}}

% The empirical distribution defined by the training set
\newcommand{\ptrain}{\hat{p}_{\rm{data}}}
\newcommand{\Ptrain}{\hat{P}_{\rm{data}}}
% The model distribution
\newcommand{\pmodel}{p_{\rm{model}}}
\newcommand{\Pmodel}{P_{\rm{model}}}
\newcommand{\ptildemodel}{\tilde{p}_{\rm{model}}}
% Stochastic autoencoder distributions
\newcommand{\pencode}{p_{\rm{encoder}}}
\newcommand{\pdecode}{p_{\rm{decoder}}}
\newcommand{\precons}{p_{\rm{reconstruct}}}

\newcommand{\laplace}{\mathrm{Laplace}} % Laplace distribution

\newcommand{\E}{\mathbb{E}}
\newcommand{\Ls}{\mathcal{L}}
\newcommand{\R}{\mathbb{R}}
\newcommand{\emp}{\tilde{p}}
\newcommand{\lr}{\alpha}
\newcommand{\reg}{\lambda}
\newcommand{\rect}{\mathrm{rectifier}}
\newcommand{\softmax}{\mathrm{softmax}}
\newcommand{\sigmoid}{\sigma}
\newcommand{\softplus}{\zeta}
\newcommand{\KL}{D_{\mathrm{KL}}}
\newcommand{\Var}{\mathrm{Var}}
\newcommand{\standarderror}{\mathrm{SE}}
\newcommand{\Cov}{\mathrm{Cov}}
% Wolfram Mathworld says $L^2$ is for function spaces and $\ell^2$ is for vectors
% But then they seem to use $L^2$ for vectors throughout the site, and so does
% wikipedia.
\newcommand{\normlzero}{L^0}
\newcommand{\normlone}{L^1}
\newcommand{\normltwo}{L^2}
\newcommand{\normlp}{L^p}
\newcommand{\normmax}{L^\infty}

\newcommand{\parents}{Pa} % See usage in notation.tex. Chosen to match Daphne's book.

\DeclareMathOperator*{\argmax}{arg\,max}
\DeclareMathOperator*{\argmin}{arg\,min}

\DeclareMathOperator{\sign}{sign}
\DeclareMathOperator{\Tr}{Tr}
\let\ab\allowbreak

\newcommand{\prob}[1]{P\left(#1\right)}
\usepackage{amsmath}
\usepackage{algorithm,algorithmic}
 \usepackage{subcaption}
\renewcommand{\algorithmiccomment}[1]{\bgroup\hfill//~#1\egroup}
\renewcommand{\algorithmicrequire}{\textbf{Input:}}
\renewcommand{\algorithmicensure}{\textbf{Output:}}
\usepackage{multirow, booktabs}

\usepackage{listings}
\lstset{
basicstyle=\small\ttfamily,
keywordstyle=\small\ttfamily,
columns=flexible,
breaklines=true
}

%% Setup of theorem styles. Outcomment only one.
%% Preferred default is the first option.
\TheoremsNumberedThrough     % Preferred (Theorem 1, Lemma 1, Theorem 2)
%\TheoremsNumberedByChapter  % (Theorem 1.1, Lema 1.1, Theorem 1.2)
\ECRepeatTheorems

%% Setup of the equation numbering system. Outcomment only one.
%% Preferred default is the first option.
\EquationsNumberedThrough    % Default: (1), (2), ...
% \EquationsNumberedBySection % (1.1), (1.2), ...

% In the reviewing and copyediting stage enter the manuscript number.
%\MANUSCRIPTNO{} % When the article is logged in and DOI assigned to it,
%   this manuscript number is no longer necessary

%\newdimen\setvrulersecondcolumnheightdimen%
%\newbox\setvrulersecondcolumnheightdimenbox%
%%%%%%%%%%%%%%%%%
\gdef\AQ#1{}
\gdef\CQ#1{}
%\setvruler [][1][1][1][1][5pt][5pt][0pt][\textheight]
\begin{document}
	%%%%%%%%%%%%%%%%
	
%	\AIA
% \setcounter{page}{1} %
% \VOL{00}%
% \NO{0}%
% \MONTH{Xxxxx}%
% \YEAR{2017}%
% \FIRSTPAGE{1}%
% \LASTPAGE{16}%
% \FIRSTPAGEAIA{1}%
% \LASTPAGEAIA{16}%
\def\COPYRIGHTHOLDER{INFORMS}%
\def\COPYRIGHTYEAR{2017}%
\def\DOI{\fontsize{7.5}{9.5}\selectfont\sf\bfseries\noindent https://doi.org/10.1287/opre.2017.1714\CQ{Word count = 9740}}
%\def\RECEIVED{November 1, 2016}
%\def\REVISED{June 22, 2017; October 6, 2017}
%\def\ACCEPTED{November 15, 2017}
% \PUBONLINEAIA{}

\RUNAUTHOR{Wang et~al.} %

\RUNTITLE{The Blessing of Reasoning: LLM-Based Contrastive Explanations in Black-Box Recommender Systems}
\TITLE{The Blessing of Reasoning: LLM-Based Contrastive Explanations in Black-Box Recommender Systems}


% A: The Blessing of Explainability: Leveraging LLM Reasoning in Black-Box Recommender Systems
% B: Combining the power of LLMs and DNNs: A Framework for Reasoning-Based Recommendations
% The Blessing of Reasoning: The Value of LLM-Generated Explanations in Black-Box Recommender Systems 
% The Blessing of Explainability: The Value of LLMs' Explanations in Black-Box Recommender Systems 
% The Blessing of Explainability: Leveraging LLM Reasoning in Black-Box Recommender Systems
% The Blessing of Explainability through Large Language Models: The Value of Explanations in Black-Box Recommender Systems
% \todo{Title focus on improve recsys preformance, also improves learning efficiency (i.e. needs fewer examples), maybe something like "LLMs reasoning capability helps design better recsys"}
% The Blessing of Reasoning: The Value of LLM-Generated Explanations in Black-Box Recommender Systems


% \TITLE{Embracing Explainability through LLMs: improves recsys performance}

	
	% Block of authors and their affiliations starts here:
	% NOTE: Authors with same affiliation, if the order of authors allows,
	%   should be entered in ONE field, separated by a comma.
	%   \EMAIL field can be repeated if more than one author

\ARTICLEAUTHORS{
%		\AUTHOR{Jianzhe Zhen,\textsuperscript{a,*} Dick den
%		Hertog,\textsuperscript{a} Melvyn Sim\textsuperscript{b}} 
%\AFF{$^{a}$Department of Econometrics and Operations Research,
%Tilburg University; $^{b}$NUS Business School, National University of
%Singapore}

\AUTHOR{Yuyan Wang\textsuperscript{1}, Pan Li\textsuperscript{2}, Minmin Chen\textsuperscript{3}}
\AFF{\textsuperscript{1}Stanford Graduate School of Business, \textsuperscript{2}Sheller College of Business, Georgia Institute of Technology, \textsuperscript{3}Google, Inc.}


% \AUTHOR{Yuyan Wang}
% \AFF{Stanford Graduate School of Business}

% \AUTHOR{Pan Li}
% \AFF{Sheller College of Business, Georgia Institute of Technology}

% \AUTHOR{Minmin Chen}
% \AFF{Google, Inc.}

%\AUEXTRA{$^{*}$Corresponding author}

%\AFFmail{{\bf Contact:} j.zhen@tilburguniversity.edu,
%d.denhertog@tilburguniversity.edu,\\			melvynsim@nus.edu.sg}%
}
	 % end of the block
	
%\ARTICLEINFO{\textbf{Received:} November 1, 2016\\ \textbf{Revised:} June 22, 2017; October 6, 2017\\ \textbf{Accepted:} November 15, 2017\\ \textbf{Published Online in Articles in Advance:}}

\ABSTRACT{
% Modern recommender systems predict consumer preferences based on consumption history  using machine learning (ML) models. Being black-box in nature, these models rely mostly on data correlations to make predictions, resulting in limited explainability. At the same time, research in explainable AI indicates that enforcing explainability often hurts predictive performance due to reduced model flexibility. In this work, we show that it is possible to improve both explainability and predictive performance by combining large language models (LLMs) with deep neural networks (DNNs).  

% We propose \emph{LR-Recsys}, an LLM-Reasoning-Powered Recommender System, that augments state-of-the-art DNN-based recommender systems with LLMs' reasoning capability. LR-Recsys introduces a contrastive explanation generator that leverages LLMs to produce two types of human-readable explanations: positive explanations for why a consumer might like a product and negative explanations for why they might not. These explanations are embedded via a pre-trained AutoEncoder and combined with consumer and product features as inputs to the DNN. Beyond offering explanability, LLM reasoning powered recommendations also improve learning efficiency and predictive accuracy. 

% To understand why, we provide insights using high-dimensional statistical learning theory. Theoretically, we show that LLMs is equipped with better knowledge of the important variables driving consumer decision-making, and that incorporating such knowledge can improve the learning efficiency of high-dimensional ML models. 

Modern recommender systems use machine learning (ML) models to predict consumer preferences based on consumption history. Although these ``black-box'' models achieve impressive predictive performance, they often suffer from a lack of transparency and explainability. While explainable AI research suggests a tradeoff between the two, we demonstrate that combining large language models (LLMs) with deep neural networks (DNNs) can improve both. We propose LR-Recsys, which augments state-of-the-art DNN-based recommender systems with LLMs' reasoning capabilities. LR-Recsys introduces a contrastive-explanation generator that leverages LLMs to produce human-readable positive explanations (why a consumer might like a product) and negative explanations (why they might not). These explanations are embedded via a fine-tuned AutoEncoder and combined with consumer and product features as inputs to the DNN to produce the final predictions. Beyond offering explainability, LR-Recsys also improves learning efficiency and predictive accuracy. To understand why, we provide insights using high-dimensional multi-environment learning theory. Statistically, we show that LLMs are equipped with better knowledge of the important variables driving consumer decision-making, and that incorporating such knowledge can improve the learning efficiency of ML models. 

Extensive experiments on three real-world recommendation datasets demonstrate that the proposed LR-Recsys framework consistently outperforms state-of-the-art black-box and explainable recommender systems, achieving a 3–14\% improvement in predictive performance. This performance gain could translate into millions of dollars in annual revenue if deployed on leading content recommendation platforms today. Our additional analysis confirms that these gains mainly come from LLMs' strong reasoning capabilities, rather than their external domain knowledge or summarization skills. 

LR-RecSys presents an effective approach to combine LLMs with traditional DNNs, two of the most widely used ML models today. Specifically, we show that LLMs can improve both the explainability and predictive performance of traditional DNNs through their reasoning capability. Beyond improving recommender systems, our findings emphasize the value of combining contrastive explanations for understanding consumer preferences and guiding managerial strategies for online platforms. These explanations provide actionable insights for consumers, sellers, and platforms, helping to build trust, optimize product offerings, and inform targeting strategies.




% To understand why LLM-generated explanations can simultaneously improve predictive performance and explainability rather than creating a trade-off, we provide insights using high-dimensional statistical learning theory. Theoretically, we show that LLMs likely have better knowledge of the important variables driving consumer decision-making, and that incorporating such knowledge can improve the learning efficiency of high-dimensional ML models.

%By incorporating these contrastive explanations, LR-Recsys moves beyond correlational predictions, enabling

% , supporting the theoretical insight that LLMs effectively identify the important variables driving consumer decisions, thereby improving the model's learning efficiency


% Modern recommender systems use machine learning (ML) models to predict whether or how much a consumer will enjoy a product based on her consumption history. However, being black-box in nature, these systems often lack explainability and rely heavily on correlation rather than causation. At the same time, existing explainable AI research suggests that enforcing explainability in ML models hurts predictive performance, as it limits the model's flexibility. To address this challenge, we propose LE-Recsys, an LLM-Explanation-Powered Recommender System, that simultaneously improves both explainability and predictive performance. Our framework introduces a contrastive-explanation generator that leverages generative language models to produce two types of human-readable explanations: positive explanations for why a consumer might like a product and negative explanations for why she might not. These explanations are transformed into embeddings using a pre-trained AutoEncoder and integrated with standard consumer and product features as input to a deep neural network (DNN)-based recommendation model. By incorporating contrastive explanations, the recommendation model can reason about consumer preferences based on their consumption history, rather than blindly relying on correlational patterns to predict choices, thereby improving both learning efficiency and predictive accuracy.

% While our framework is compatible with any generative natural language processing (NLP) model, we find large language models (LLMs) particularly effective due to their advanced reasoning capabilities. To understand why LLM-generated explanations can simultaneously improve predictive performance and explainability rather than creating a trade-off, we provide insights using high-dimensional statistical learning theory. Theoretically, we show that LLMs likely have better knowledge of the important variables driving consumer decision-making, and that incorporating such knowledge can improve the learning efficiency of high-dimensional ML models.

% Extensive experiments on three real-world recommendation datasets demonstrate that our approach consistently outperforms state-of-the-art black-box and explainable recommender systems, improving the predictive performance by 3-14\%. Furthermore, we confirm that these observed improvements mainly come from LLMs' strong reasoning capabilities rather than their external domain knowledge, supporting the theoretical insight that LLMs effectively identify the important variables driving consumer decisions, thereby improving the model's learning efficiency. Beyond improving recommender systems, our findings emphasize the value of contrastive explanations for understanding consumer preferences and guiding managerial strategies for online platforms. These explanations provide actionable insights for consumers, sellers, and platforms, helping to build trust, optimize product offerings, and inform targeting strategies without raising privacy concerns.

% These models, trained on vast datasets, offer rich insights into critical consumer preferences and product attributes that lead to a purchase decision, bridging the gap in current black-box systems.

%%%\todo{maybe remove the last sentence}
}

%\FUNDING{The research of the first author is supported by NWO Grant 613.001.208. The third author acknowledges the funding support from the Singapore Ministry of Education Social Science Research Thematic Grant MOE2016-SSRTG-059.}

\SUBJECTCLASS{\AQ{Please confirm subject classifications.}Recommender systems.}

\AREAOFREVIEW{Marketing.}

\KEYWORDS{Recommender Systems; Large Language Models; Deep Learning; LLM reasoning; LLM-generated explanations.}%{\CQ{Kindly provide the keywords.}}

	%%%%%%%%%%%%%%%%%%%%%%%%%%%%%%%%%%%%%%%%%%%%%%%%%%%%%%%%%%%%%%%%%%%%%%
	
	% Samples of sectioning (and labeling) in OPRE
	% NOTE: (1) \section and \subsection do NOT end with a period
	%       (2) \subsubsection and lower need end punctuation
	%       (3) capitalization is as shown (title style).
	%
	%\section{Introduction.}\label{intro} %%1.
	%\subsection{Duality and the Classical EOQ Problem.}\label{class-EOQ} %% 1.1.
	%\subsection{Outline.}\label{outline1} %% 1.2.
	%\subsubsection{Cyclic Schedules for the General Deterministic SMDP.}
	%  \label{cyclic-schedules} %% 1.2.1
	%\section{Problem Description.}\label{problemdescription} %% 2.
	% Text of your paper here
	
\maketitle

\section{Introduction}
\label{sec:intro}
\section{Introduction}
Surveys provide an essential tool for probing public opinions on societal issues, especially as opinions vary over time and across subpopulations.
However, surveys are also costly, time-consuming, and require careful calibration to mitigate non-response and sampling biases \cite{choi2004catalog, bethlehem2010selection}. 
Recent work suggests that large language models (LLMs) can assist public opinion studies by predicting survey responses across different subpopulations, explored in both social science ~\cite{argyle2023out,bail2024can,ashokkumar2024predicting,manning2024automated} and NLP~\cite{santurkar2023whose,chu2023language,moon-etal-2024-virtual,hamalainen2023evaluating,chiang2023can}.
Such capabilities could substantially enhance the survey development process, not as a replacement for human participants but as a 
tool for researchers to conduct pilot testing, identify subpopulations to over-sample, and test analysis pipelines prior to conducting the full survey  \cite{rothschild2024survey}.

\begin{figure}
    \centering
    \includegraphics[width=1.0\linewidth]{figures/teaser.pdf}
    \caption{Illustration of our method and \OURDATA. We collect survey data from two survey families—ATP from Pew Research~\cite{atp} (forming \OURDATA-Train) and GSS from NORC~\cite{davern2024gss} (forming \OURDATA-Eval). 
    LLMs are fine-tuned on \OURDATA-Train and evaluated on both OpinionQA~\cite{santurkar2023whose} and \OURDATA-Eval to assess generalization of distributional opinion prediction across unseen survey topics, survey families, and subpopulations.
    }
    \label{fig:teaser}
\end{figure}

Prior work in steering language models, \textit{i.e.} conditioning models to reflect the opinions of a specific subpopulation, has primarily investigated different prompt engineering techniques~\cite{santurkar2023whose, moon-etal-2024-virtual, park2024generative}. However, prompting alone has shown limited success in generating completions that accurately reflect the distributions of survey responses collected from human subjects. Off-the-shelf LLMs~\cite{achiam2023gpt, dubey2024llama, jiang2023mistral} have shown to mirror the opinions of certain US subpopulations such as the wealthy and educated \cite{santurkar2023whose,gallegos2024bias,deshpande2023toxicity,kim2023ai}, while generating stereotypical or biased predictions of underrepresented groups ~\cite{cheng2023compost,cheng2023marked,wang2024large}. Furthermore, these models often fail to capture the variation of human opinions within a subpopulation \cite{kapania2024simulacrum, park2024diminished}.
While fine-tuning presents opportunities to address these limitations ~\cite{chu2023language, he2024community}, existing methods fail to train models that accurately predict opinion distributions across diverse survey question topics and subpopulations.

\vspace{-5pt}
\paragraph{The present work.}
Here, we propose directly fine-tuning LLMs on large-scale, high-quality survey data,
consisting of questions about diverse topics and responses from each subpopulation, defined by demographic, socioeconomic, and ideological traits.
By casting pairs of (subpopulation, survey question) as input prompts, we train the LLM to align its response distribution against that of human subjects in a supervised manner.
We posit that survey data is particularly well-suited for fine-tuning LLMs since: (1) We can train the model with clear \textbf{subpopulation-response pairs} that explicitly link group identities and expressed opinions,
which is rare in LLMs' pre-training corpora,
(2) Large-scale opinion polls are carefully designed and calibrated (\textit{e.g.} using post-stratification) to estimate \textbf{representative} human responses, in contrast with LLMs' pre-training data where certain populations are over- or underrepresented, 
(3) Our training objective explicitly aligns model predictions with response \textbf{distributions} from each subpopulation, enabling LLMs to capture variance within human subpopulations.

Training on public opinion survey data has remained under-explored due to the limited availability of structured survey datasets. 
To this end, we curate and release \textbf{\OURDATA} (\textbf{Sub}population-level \textbf{P}ublic \textbf{O}pinion \textbf{P}rediction), a dataset of 70K subpopulation-response distribution pairs ($6.5\times$ larger compared to previous datasets).
We show that fine-tuning LLMs on \OURDATA significantly improves the distributional match between LLM generated and human responses, and improvements are consistent across subpopulations of varying sizes.
Additionally, the improvement generalizes to \textit{unseen} subpopulations, survey waves (topics), and survey families, \textit{i.e.} surveys administered by different institutions.
Such broad generalization is particularly critical for real-world public opinions research, where practitioners are most in need of synthetic data for survey questions or subpopulations (or both) that they have not tested before.

Our contributions are summarized as follows:
\vspace{-3mm}
\begin{itemize}[leftmargin=3.3mm]
\setlength\itemsep{2pt}

\item We show that training LLMs on response distributions from survey data significantly improves their ability to predict the opinions of subpopulations, reducing the Wasserstein distance between LLM and human distributions by 32-46\% compared to top-performing baselines. (\Cref{section_experiments_prediction_of_opinion_distributions})
\vspace{-1mm}
\item We show that the performance of the fine-tuned LLMs strongly generalizes to out-of-distribution data, including unseen subpopulations, new survey waves, and different survey families. 
(\Cref{section_experiments_prediction_of_opinion_distributions} and \Cref{section_experiments_per_group})
\vspace{-1mm}
\item We release \OURDATA, a curated and pre-processed dataset of public opinion survey results that is $6.5\times$ larger than existing datasets, enabling fine-tuning at scale.
\end{itemize}

\section{Related Work}
\label{sec:related_work}
\section{Related Work}
\label{sec:related_work}
\paragraph{Challenges in Long Context Understanding}
LLMs struggle with long contexts despite supporting up to 2M tokens~\cite{dubey2024llama3,reid2024gemini}. 
The ``lost-in-the-middle'' effect~\cite{liu2024lost} and degraded performance on long-range tasks~\cite{li2023loogle} highlight these issues. To address this, ProLong~\cite{prolong} finetunes base models on a large, carefully curated long-context corpus. While this approach improves performance on long-range tasks, it comes at a significant cost, requiring training with an additional 40B tokens and long-input sequences.


%Recent studies have highlighted significant challenges in LLMs' processing of extended contexts. While models like Llama-3~\cite{dubey2024llama3} and Gemini~\cite{reid2024gemini} support context windows up to 128K or even 2M tokens, they struggle with effective utilization of this capacity. 
%The ``lost-in-the-middle'' phenomenon~\cite{liu2024lost} shows that models often fail to leverage information from the middle of long contexts, while \citet{li2023loogle} demonstrated that performance degrades significantly on tasks requiring long-range dependencies. 
%To address this issue, ProLong~\cite{prolong} finetunes base models on a large, carefully curated long-context corpus. While this approach improves performance on long-range tasks, it comes at a significant cost, requiring training with an additional 40B tokens and long-input sequences.

% ProLong~\cite{prolong} provides a solution through extensive continued pretraining (40B tokens) on long-context data, but this approach requires significant computational resources and may not work well for tasks require more than raw long-context abilities.
% These studies suggest that merely increasing the context window size is insufficient; enhancing true long-context understanding remains a significant challenge.

\paragraph{Inference-time Scaling for Long-Context}
The Self-Taught Reasoner (STaR) framework \citep{zelikman2022star} iteratively generates rationales to refine reasoning, with models evaluating answers and finetuning on correct reasoning paths. \citet{wang2024multi} introduced Model-induced Process Supervision (MiPS), automating verifier training by generating multiple completions and assessing accuracy, boosting PaLM 2's performance on math and coding tasks. \citet{li2024large} proposed an inference scaling pipeline for long-context tasks using Bayes Risk-based sampling and fine-tuning, though their evaluation is limited to shorter contexts (10K tokens) compared to ours (128K tokens).

%The Self-Taught Reasoner (STaR) framework, proposed by \citet{zelikman2022star}, presents a method where language models iteratively generate step-by-step rationales to improve reasoning capabilities. This approach involves the model generating rationales for questions, evaluating the correctness of the answers, and fine-tuning based on successful reasoning paths. %Building upon this, \citet{zelikman2024quiet} introduced \textsc{Quiet-STaR}, which enables models to generate internal rationales at each token to enhance predictions. These methods aim to improve inference-time reasoning without extensive human supervision. 
%Furthermore, \citet{wang2024multi} introduced Model-induced Process Supervision (MiPS), an automated data curation method that eliminates the need for human annotation in training verifiers. MiPS involves the model generating multiple completions of an intermediate solution step and calculating the accuracy based on the proportion of correct completions. Their approach significantly improved the performance of PaLM 2 on math and coding tasks. 
%Building on these ideas, \citet{li2024large} proposed an inference scaling pipeline for long-context tasks where LLM outputs are sampled and weighted using Bayes Risk, followed by fine-tuning on preferred outputs. Although their approach shares similarities with ours, their evaluation focuses on much shorter context lengths (around 10K tokens) compared to ours (up to 128K tokens).

% On this line of research, \citet{li2024large} proposed inference scaling pipeline to sample outputs from LLMs and weight them with Bayes Risk. They then finetune the model on preferred outputs. While sharing similarity with our approach, the context length of the problems considered in the paper is significantly shorter (around 10K) than ours (up to 128K).

\paragraph{Agentic Workflow for Long-Context} 
Agentic workflows~\cite{yao2022react} enable LLMs to autonomously manage tasks by generating internal plans and refining outputs iteratively. 
The LongRAG framework~\cite{zhao-etal-2024-dual} enables an LLM and an RAG module to collaborate on long-context tasks by breaking down the input into smaller segments, processing them individually, and integrating the results to form a coherent output.
Chain-of-Agents (CoA)~\cite{zhang2024chain} tackles long-context tasks through decomposition and multi-agent collaboration. In CoA, the input text is divided into segments, each handled by a worker agent that processes its assigned portion and communicates its findings to the next agent in the sequence.
Unlike these, our approach employs a single LLM that orchestrates its own reasoning and retrieval without relying on multiple components. By dynamically structuring its process and iteratively refining long-context information, our model reduces complexity while maintaining efficiency.



% \paragraph{Agentic Workflow for Long-Context}
% The concept of agentic workflows~\cite{yao2022react} in LLMs involves structuring models to autonomously manage tasks by generating and following internal plans or chains of thought. This approach allows models to handle complex tasks by decomposing them into manageable steps and iteratively refining their outputs. For instance, the LongRAG framework~\cite{zhao-etal-2024-dual} enables an LLM and an RAG module to collaborate on long-context tasks by breaking down the input into smaller segments, processing them individually, and integrating the results to form a coherent output. This method enhances the model's ability to manage and reason over extended contexts by leveraging internal planning and iterative refinement. Chain-of-Agents (CoA)~\cite{zhang2024chain} addresses the challenges of processing long-context tasks by leveraging multi-agent collaboration among LLMs. In CoA, the input text is divided into segments, each handled by a worker agent that processes its assigned portion and communicates its findings to the next agent in the sequence.
% Different from these approaches, we focus on an agentic system with a single primary LLM that autonomously orchestrates its reasoning and retrieval processes without relying on multiple interacting components. 
% Instead of distributing tasks across separate entities, our model dynamically structures its own reasoning process, iteratively retrieving, attending to, and refining long-context information within a unified framework. This enables efficient handling of extended contexts while reducing the complexity introduced by multi-agent coordination.




\section{Machine Learning Framework: LR-Recsys}
\label{sec:framework}
% \todo{change LE-Recsys to LR-Recsys}

\subsection{Problem Definition}  
\label{sec:framework_probdefn}

In industrial recommendation platforms, the catalog typically contains millions or billions of items \citep{covington2016deep}. The goal of a recommender system is to select a few relevant items from the catalog in real-time based on the consumer and their current context, then present the results in a ranked list. This is inherently a combinatorial problem as the complexity grows exponentially with the size of the catalog; therefore, solving it directly in real-time is infeasible for large-scale platforms. To address this, a common approach in the industry is to use a greedy algorithm, where each item is assigned an individual ranking score, and the final ranking is determined by the ordering of these scores \citep{liu2009learning}.\endnote{Such greedy solutions have error lower bound guarantees compared to an oracle combinatorial optimization solution \citep{ailon2007efficient, balcan2008robust}, and conveniently reduce the complexity from combinatorial ($\mathcal{O}(n!)$) to log-linear ($\mathcal{O}(n\log{}n)$), making them feasible for real-time, large-scale systems.}

We use $i$ to index consumers and $j$ to index products, while $\vz$ denotes the context (such as time of day, day of the week, or location). The individual ranking score is typically the predicted value of an ideal outcome $y$ (e.g., a click, like, or purchase) that the platform aims to maximize.\endnote{$y$ can also be a multi-dimensional vector when multiple objectives are considered. In these cases, a combination of objectives is used to generate the ranking score. See \citet{wang2024recommending} and \citet{rafieian2024multiobjective} for examples.} When consumer $i$ visits the platform in context $\vz$, the recommender system uses a predictive ML model to estimate the real-time value of $\hat{y}$ for every product $j$ in the catalog:
\begin{equation}
  \label{eqn:ml_recsys}
  \hat{y}(i,j,\vz) = f(\vx_i, \vx_j, \vx_{ij}, \{j_{i_1},...,j_{i_n}\}, \vz),
\end{equation}
where $\vx_{i}$ represents consumer-level features, $\vx_{j}$ represents product-level features, and $\vx_{ij}$ captures the interaction history between consumer $i$ and product $j$. Figure \ref{fig:illustration_baseline} illustrates a typical industrial recommender system. We adopt a state-of-the-art sequential recommender system, which leverages advanced sequence modeling techniques like Transformers \citep{vaswani2017attention} to encode a consumer's sequential consumption history ${j_{i_1},...,j_{i_n}}$ as input features. The function $f(\cdot)$ can be any ML model selected by the developer. Products in the catalog are ranked in descending order of $\hat{y}(i,j,\vz)$ and presented to the consumer accordingly.

% , where the predicted outcome can be either continuous (e.g., a rating) or binary (e.g., click, like, or purchase)
% https://lucid.app/lucidchart/82615591-42af-41ab-aece-f38722fb6512/edit?beaconFlowId=B68358B117CEDB7D&invitationId=inv_a8d3e58d-e4cb-47d9-a270-6fbb9761f10e&page=0_0#
% https://lucid.app/lucidchart/fcc974c4-0c5c-4be3-a763-290c65c3a701/edit?beaconFlowId=C0FFA896091483C2&invitationId=inv_bc226156-f41d-4d94-9e32-a3bd48a4564d&page=0_0
\begin{figure}[hbtp!]
% \vspace{-0.3cm}
    \begin{subfigure}[b]{0.43\textwidth}
        \centering
        \includegraphics[width=\textwidth]{figures/illustration_baseline.png}
        \caption{A typical recommendation framework.}
        \label{fig:illustration_baseline}
    \end{subfigure}
    \hfill
    % \hspace{4mm}
     \centering
    \begin{subfigure}[b]{0.53\textwidth}
        \centering
        \includegraphics[width=\textwidth]{figures/illustration_framework.png}
        \caption{LR-Recsys (detailed architecture in Fig.\ref{fig:llm_recsys_diagram}).}
        \label{fig:illustration_framework}
    \end{subfigure}
    \caption{Comparison between our proposed framework, LR-Recsys, and a typical recommender system.}
  \label{fig:comparison}
 \vspace{-0.4cm}
\end{figure}

\subsection{Motivation: Leveraging LLMs for Explanation Generation}
\label{sec:framework_LLM_reasoning}
The predictive ML model in Fig.\ref{fig:illustration_baseline} is black-box in nature. In other words, it is hard to extract the reasons behind a consumer's choices—such as why a consumer likes or dislikes a product—or provide \emph{explanations} for its own predictions, like why it assigns a high score to product A and a low score to product B. % As a result, the black-box recommender systems in use today may fail to fully leverage the information in the training data to effectively learn and predict the outcomes.  

Inspired by the recent advances in the reasoning capabilities of large language models (LLMs) \citep{brown2020language, lingo2024enhancing, wei2022chain}, we test whether LLMs are able to generate explanations for why a consumer may or may not like a product. In a toy example, we asked OpenAI's latest GPT-4 model \citep{achiam2023gpt} to give reasons on why a consumer might or might not purchase an orange juice product, given her previous purchase history. We included (potentially irrelevant) details for the given product, such as packaging and logo, to understand how LLMs reason through relevant and less relevant information for purchase decisions. 

As illustrated in Fig.\ref{fig:example_explanations}, GPT-4 successfully generated plausible explanations for both positive and negative outcomes. For example, the positive explanation, ``because the consumer regularly buys fruits including oranges, and the product is an orange juice'', highlights the connection between orange juice and her existing preference for oranges (Fig.\ref{fig:pos_explanation}). The negative explanation, ``because the consumer may prefer whole fruits over processed juices'', points out the difference between oranges and orange juice and the consumer's preference for whole fruits (Fig.\ref{fig:neg_explanation}). % Other less relevant product details, such as being in a paper box or having a sun in the logo, do not appear in the explanations.\endnote{It is possible that the hypothetical consumer's purchase decision for the orange juice in this toy example could be influenced by its packaging or logo. However, a more plausible explanation is that the consumer's preference is based on the orange juice itself. The strength of LLMs lies in their ability to identify the most plausible reasons to explain a consumer's choices.}

These explanations provided by LLMs are natural and intuitive to consumers, yet they are not explicitly captured by current recommender systems. This motivates us to directly incorporate such LLM-generated explanations in recommender systems to facilitate their learning, by augmenting the input with these LLM-generated reasons behind the consumers' choices. 

In addition, both positive and negative explanations provide valuable information that would otherwise be difficult for a traditional recommender system to capture. The positive explanation identifies connections between a user's underlying preference and the given product, while the negative explanation pinpoints the differences. These links would typically require thousands of examples for a traditional recommender system to learn, whereas LLMs can accurately identify them with even zero-shot prompting, thanks to their reasoning capabilities built through vast pre-training and post-training.
  

% For example, the positive explanation highlights the connection between oranges and orange juice (both being orange-based products), while the negative explanation points out the distinction (orange juice being a processed product) and suggests the consumer may prefer whole fruits. 

These insights motivate us to design a recommender system with a built-in reasoning component, powered by LLMs, that generates both positive and negative explanations for every $(consumer, product)$ pair to facilitate the model learning. We now provide a high-level overview of our proposed framework, followed by a detailed description of each component in the framework.


\begin{figure}[hbtp!]
% \vspace{-0.4cm}
    \begin{subfigure}[b]{0.49\textwidth}
        \centering
        \includegraphics[width=\textwidth]{figures/pos_explanation.png}
        \caption{Example of a positive explanation.}
        \label{fig:pos_explanation}
    \end{subfigure}
    \hfill
    % \hspace{4mm}
     \centering
    \begin{subfigure}[b]{0.48\textwidth}
        \centering
        \includegraphics[width=\textwidth]{figures/neg_explanation.png}
        \caption{Example of a negative explanation.}
        \label{fig:neg_explanation}
    \end{subfigure}
    \caption{A toy example for positive and negative explanations by GPT-4.}
  \label{fig:example_explanations}
% \vspace{-0.9cm}
\end{figure}

\subsection{Overview of the Proposed Framework}
\label{sec:framework_overview}
We propose \textbf{LR-Recsys}, an \textbf{L}LM-\textbf{R}easoning-Powered \textbf{Rec}ommender \textbf{Sys}tem, that explicitly incorporates a reasoning component into the design of a deep neural network (DNN)-based recommendation framework. Specifically, we first task LLMs with generating explanations for why a favorable outcome (e.g., a like or a purchase) may or may not occur for each sample in the training data. We then encode these explanations as embeddings using a fine-tuned text auto-encoder, and incorporate these embeddings into a DNN component for predicting consumer preference. Figure \ref{fig:illustration_framework} illustrates the high-level concept of LR-Recsys and highlights its differences from traditional recommendation models as in Fig.\ref{fig:illustration_baseline}. % It is important to note that we rely on the reasoning capabilities of LLMs here, \emph{not} their world knowledge. While our experiments show that augmenting product information with LLM-generated profiles can further boost performance, the main performance gains of our framework come from leveraging the reasoning capabilities of LLMs.

Figure \ref{fig:llm_recsys_diagram} presents the complete ML architecture of our proposed LR-Recsys framework, building on the concept illustrated in Fig.\ref{fig:illustration_framework}. Specifically, we design an \emph{contrastive-explanation generator} that takes the consumer's sequential consumption history and a candidate product as input, and outputs embeddings that represent the positive and negative explanations for why the consumer may or may not like the candidate product, given her past consumption history. These embeddings are then concatenated with other consumer, product, and contextual features to form the input layer. The input layer is passed through additional Transformer or MLP layers to create a deep neural network (DNN) recommendation component, which ultimately predicts the outcome for the $(consumer, product)$ pair. During training, the DNN recommendation component dynamically learns how positive and negative explanations contribute to the model prediction by adjusting the neural network weights of the corresponding explanation embeddings through back-propagation. We delve into each of these components in Sections \ref{sec:framework_explanation_generator} to \ref{sec:framework_llm_profile} below.
\begin{figure}[hbtp!]
    \centering
    \includegraphics[width=1.0\linewidth]{figures/llm_recsys_diagram_contrastive.png}
    \caption{(Color online) Detailed architecture of LR-Recsys.}\label{fig:llm_recsys_diagram}
    % \vspace{-0.2in}
\end{figure}


% [old] https://lucid.app/lucidchart/e619058b-1583-4f5d-a821-803e280a6a79/edit?invitationId=inv_40fc6fb6-e822-4534-a4e5-5f3a40382ae5&page=0_0#
% [new] https://lucid.app/lucidchart/69b07707-052b-4463-a543-6db060dc5676/edit?beaconFlowId=1D7CCDA0264F69F1&invitationId=inv_a52296c4-9ee5-44a3-ab23-281b47e3b747&page=0_0 

% \minminc{I would move section 3.5 before 3.4, make section 3.5 an overview of the architecture, with the high-level design idea to put all the information (user, product, history, and llm generated explaination into a consistent form, which is the embeddings), make it more brief, maybe removing some of the equations, and then explain section 3.4 and 3.6. put more energy into section 3.4 since that's the meat of the work.}
% To follow up with Minmin 

\subsection{Contrastive-explanation generator}
\label{sec:framework_explanation_generator}
The goal of the contrastive-explanation generator is to generate positive and negative explanation embeddings for every training sample, which will be used as input to the DNN recommendation component. As discussed in Section \ref{sec:framework_LLM_reasoning}, we leverage LLMs' reasoning capabilities to provide such explanations. 

The architecture of the contrastive-explanation generator is highlighted in the ``Contrastive-Explanation Generator'' box in Fig.\ref{fig:llm_recsys_diagram}. It contains two stages: a pre-trained LLM which produces positive and negative explanations in text in the first stage, and a fine-tuned AutoEncoder that encodes these explanations into embeddings in the second stage. We describe these two stages in detail below.

\subsubsection{Pre-trained LLM for generating textual explanations.}
\label{sec:framework_explanation_generator_pretrainllm}

In the first stage, the pre-trained LLM can be any LLM of choice, allowing us to leverage its general reasoning capabilities. Specifically, given consumer $i$'s consumption history, represented as a sequence of $n$ products $j_{i1}, j_{i2}, ..., j_{in}$, we prompt the LLM to generate explanations for why the consumer may or may not enjoy a candidate product. We refer to these as \emph{positive explanations} and \emph{negative explanations}.

The prompts for positive and negative explanations follow this template: 
\begin{quote}
\emph{
``Provide a reason for why this consumer purchased (or did not purchase) this product, based on the provided profile of the past products the consumer purchased, and the profile of the current product. Answer with exactly one sentence in the following format: `The consumer purchased (or did not purchase) this product because the consumer ... and the product ...'.'' }
\end{quote} 
The prompts can be adapted to fit specific domains; for example, ``purchased this product'' can be changed to ``watched this movie''. The prompt is then concatenated with the consumer's consumption history, which is presented as a sequence of product profiles, and the candidate product profile. These product profiles can range from simple product names available in the observed training data (e.g. ``orange'') to more detailed profiles generated by LLMs for further augmentation, as later described in Section \ref{sec:framework_llm_profile}. 

Detailed examples of the prompts used in our experiments are provided in Section \ref{sec:results}. % The textual explanations generated by the pre-trained LLMs are then fed into a fine-tuned AutoEncoder that converts them into embeddings, which we describe below.

 
% [purchased this product / watched this movie / stayed at this hotel], based on the provided profile of the past [products / movies / hotels] the consumer [purchased / watched / stayed at], and the profile of the current [product / movie / hotel]. Answer with exactly one sentence in the following format: `The consumer [purchased this product / watched this movie / stayed at this hotel] because the consumer ... and the hotel ....' ." 


\subsubsection{Fine-tuned AutoEncoder for converting textual explanations into embeddings.}
\label{sec:framework_explanation_generator_autoencoder}

In the second stage, the generated explanations from the first stage are used to train a fine-tuned AutoEncoder, which encodes both the positive and negative explanations into embeddings. An AutoEncoder, introduced by \citet{hinton2006reducing}, is a type of neural network that learns compressed, efficient representations of data. Like Principal Component Analysis (PCA), it reduces dimensionality while retaining essential features, but it can capture non-linear relationships with the neural network architecture, making it more powerful for complex data. It consists of two main components: an encoder, which compresses the input into a low-dimensional latent space, and a decoder, which reconstructs the input from this compressed representation. Figure \ref{fig:autoencoder} illustrates the concept of an AutoEncoder. The network is trained through back-propagation to minimize reconstruction error between the input data and reconstructed data, ensuring that the learned latent representations capture the essential features. The output of the encoder is used as the low-dimensional representation, or the embedding.

\begin{figure}[hbtp!]
    \centering
    \includegraphics[width=0.5\linewidth]{figures/autoencoder.png}
    \caption{(Color online) AutoEncoder.}\label{fig:autoencoder}
   %  \vspace{-0.2in}
\end{figure}

In the context of text data, AutoEncoders map sentences to lower-dimensional embeddings, where semantically similar sentences are represented by embeddings close to each other in the latent space, while dissimilar sentences are further apart. Given an explanation as the input text sequence \( \vz = \{\vz_1, \dots, \vz_T\} \), where \( T \) is the sequence length, and a reconstructed sequence \( \hat{\vz} = \{\hat{\vz}_1, \dots, \hat{\vz}_T\} \) which is the output of the decoder in Fig.\ref{fig:autoencoder}, our AutoEncoder is trained to minimize the reconstruction loss, which is defined as the cross-entropy (CE) loss between the input sequence $\vz$ and the reconstructed sequence $\hat{\vz}$:
\begin{equation}
\label{eq:reconstruction_loss_pop}
\mathcal{L}_{\text{CE}}(\vz, \hat{\vz}) = - \sum_{t=1}^{T} \log p(\hat{\vz}_t = \vz_t | \vz),
\end{equation}
where $p(\hat{\vz}_t = \vz_t | \vz)$ is the predicted probability of the correct word $\vz_t$ at position $t$. In our case, the fine-tuned AutoEncoder is trained using the generated explanations from the first stage (Section \ref{sec:framework_explanation_generator_pretrainllm}) to encode them into embeddings, with the goal of ensuring that similar explanations are represented by similar vectors in the embedding space. For $N$ training examples, we have $N$ positive explanations $\{\bm{zp}^1, \dots, \bm{zp}^N\}$, and $N$ negative explanations $\{\bm{zn}^1, \dots, \bm{zn}^N\}$. Therefore, the reconstruction loss is computed as 
\begin{equation}
\label{eq:reconstruction_loss_sample}
\mathcal{L}_{\text{recon}}  = \sum_{k=1}^N \left(\mathcal{L}_{\text{CE}}(\bm{zp}^k, \hat{\bm{zp}}^k) + \mathcal{L}_{\text{CE}}(\bm{zn}^k, \hat{\bm{zn}}^k)\right) 
\end{equation}


The detailed model architecture and training of the AutoEncoder are provided in the Appendix \ref{appen:autoencoder}. The ``bottleneck'' layer of the AutoEncoder, highlighted in blue in Fig.\ref{fig:autoencoder} and denoted as $AE(\cdot)$, serves as the low-dimensional embedding for the textual explanations. We used a dimension of 8 as the embedding size. 

Note that rather than using embeddings directly from the LLM, we adopt a separate fine-tuned AutoEncoder to transform the textual explanations into embeddings. This is because we want the embeddings to only focus on differentiating between explanations \emph{within} the subspace of all relevant explanations for the current application, rather than ``wasting'' their capacity on distinguishing explanations from unrelated text in the full universe of texts. Therefore, a fine-tuned AutoEncoder trained exclusively on explanations is a more efficient solution for this task. Another option would be to fine-tune the LLM directly on the generated explanations. However, this approach is significantly more resource-intensive than training a smaller, separate AutoEncoder. For instance, fine-tuning the LLaMA-3-8B model using Low-Rank Adaptation (LoRA) \citep{hu2021lora}, a popular Parameter-Efficient Fine-Tuning (PEFT) method, involves 1,703,936 trainable parameters \citep{ye2024lola}. In contrast, our fine-tuned AutoEncoder has only 52,224 trainable parameters.
% 8*50*64 + 64*8 + 8*64 + 64*8*50 
% This approach allows for better differentiation between the explanations using the learned embeddings without requiring distinctions between explanation texts and unrelated texts. In other words
Using our fine-tuned AutoEncoder, the positive and negative explanations generated by the LLMs are encoded into two distinct embeddings—one for each explanation. These embeddings are then concatenated with consumer, product, and contextual features to form the input layer of the DNN-based recommendation component, which we describe in the following section. 


\subsection{DNN recommendation component}
\label{sec:framework_dnn}  

As illustrated in the ``DNN Recommendation Component'' in Fig.\ref{fig:llm_recsys_diagram}, the DNN recommendation component concatenates the following input to form the input layer of a deep neural network: positive and negative explanation embeddings, consumer embedding, product embedding, sequential history embedding, and contextual features. The input layer then goes through a neural network to generate the final output, which is the predicted likelihood of outcome (e.g., CTR). We introduce each part in detail below. 

\subsubsection{Consumer and product embeddings.} 
\label{sec:framework_dnn_cons_embed}  
We represent each consumer and product as low-dimensional embeddings, which are designed to capture similarities among consumers and products. These embeddings are constructed under the assumption that similar consumers like similar products \citep{breese1998empirical}. Traditionally, these embeddings in recommender systems are derived using matrix factorization \citep{dhillon2021modeling, wang2024recommending}. However, with the success of deep learning techniques, embeddings are now frequently learned via \emph{embedding table lookup}, which we briefly describe below.

Two separate embedding tables are initialized: one for consumers (denoted as $\mathbf{E}_{c}$) and one for products (denoted as $\mathbf{E}_{p}$). Each row in the consumer embedding table corresponds to a unique consumer ID, and each row in the product embedding table corresponds to a unique product ID. For each interaction between a consumer $i$ and a product $j$, their corresponding embeddings are retrieved from the embedding tables:
\begin{equation}
\label{eq:embed}
\mathbf{\ve}^{(c)}_i = \text{Lookup}(\mathbf{E}_{c}, i), \quad \mathbf{\ve}^{(p)}_j = \text{Lookup}(\mathbf{E}_{p}, j),
\end{equation}
where $\text{Lookup}(\mathbf{E}, i)$ refers to taking the $i$-th row of the matrix $\mathbf{E}$, $\mathbf{\ve}^{(c)}_i$ represents the embedding for consumer $i$, and $\mathbf{\ve}^{(p)}_j$ represents the embedding for product $j$. These embeddings are used as the input to the neural network to represent consumer $i$ and product $j$. During training, these embeddings are randomly initialized and are updated through back-propagation to minimize the loss:
\begin{equation}
\label{eq:embedding_update}
\mathbf{E}_{c} \leftarrow \mathbf{E}_{c} - \eta \frac{\partial \mathcal{L}}{\partial \mathbf{E}_{c}}, \quad \mathbf{E}_{p} \leftarrow \mathbf{E}_{p} - \eta \frac{\partial \mathcal{L}}{\partial \mathbf{E}_{p}},
\end{equation}
where $\eta$ is the learning rate, and the loss $\mathcal{L}$ measures the deviation between the predicted outcome (e.g. predicted probability of click) and the actual outcome (e.g. actual clicks) as will be detailed in Section \ref{sec:framework_dnn_loss} below. This process is repeated iteratively during training to improve the embeddings, capturing latent consumer preferences and product characteristics.

\subsubsection{Sequential history embedding.}
\label{sec:framework_dnn_seq_embed} 
The consumer's consumption history is represented as a sequence of $n$ products $\{j_{i1}, j_{i2}, ..., j_{in}\}$. Such sequences cannot be directly consumed by a recommender system which only takes numerical values as inputs. Inspired by recent developments in advanced sequence modeling in the deep learning literature \citep{vaswani2017attention}, we propose to combine two popular sequential modeling architectures to encode the sequence into a \emph{sequence embedding}, a vector of numerical values that can be jointly learned alongside the other parameters of the neural network. 

The sequence embedding should ideally encode two types of information. The first is the \textbf{local and temporal} dependency between consecutive consumed products in the sequence. For example, the consumer may purchase a Nintendo Switch first and, subsequently, several digital Nintendo games in a row. The second type of information is the \textbf{global} relationships within a sequence, allowing each consumption to attend to all other consumptions \emph{regardless} of their distance in the sequence. For example, the consumer may purchase products from a niche category only once in a while (e.g., luxury watches), and we would like the resulting sequence embedding to capture such long-distance relationships. 

% Figure \ref{fig:local_global} illustrates the concepts of local and global attention.
%\begin{figure}[hbtp!]
%    % \vspace{-0.1in}
%    \centering
%    \includegraphics[width=0.6\linewidth]%{figures/local_global.png}
%    \caption{(Color online) Illustration of temporal and %global dependencies in a sequence. for clarity, only %global relationships for $j_4$ are %shown.}\label{fig:local_global}
%     \vspace{-0.2in}
%\end{figure}
We propose to leverage two techniques in the recent sequence modeling research to capture both types of dependencies in the consumer's consumption history. Specifically, the self-attention model used in Transformers \citep{vaswani2017attention} is designed to capture global dependencies; The Gated Recurrent Unit (GRU) \citep{cho2014learning} as a type of recurrent neural networks (RNNs) is known to capture local and temporal dependencies well. %Given the importance of both global and local temporal dependencies in understanding a consumer's sequential consumption history, it is natural to combine self-attention and GRU to leverage the strength of both techniques.
We combine both modules, i.e., a self-attention layer followed by a GRU layer as described below, to form the sequence embedding.\endnote{A similar approach is used by \citet{li2020purs}, but their focus is on incorporating item-level heterogeneity in sequences, rather than capturing both types of dependencies.} 

\subsubsection*{Self-attention layer.}
The self-attention mechanism enables each item in a sequence to attend to all other items, effectively capturing global dependencies regardless of their spatial position within the sequence. This is achieved by computing attention weights that quantify the influence of each element.

For consumer $i$'s consumption history $(j_{i1}, j_{i2}, ..., j_{in})$, we transform this sequence into a series of embeddings. Specifically, we retrieve each product's embedding from the embedding table described in Section \ref{sec:framework_dnn_cons_embed}. The consumption history is represented as $\mathbf{S} = [\vs_1, \vs_2, \dots, \vs_n]$, where each $\vs_l$ is the embedding of the $l$-th product in the sequence. $\vs_l$ is obtained through the lookup operation: $\vs_{l} = \text{Lookup}(\mathbf{E}_{p}, j_{il}), \forall l = 1,\dots,n$. For simplicity, we omit the consumer index $i$ in this notation.

For each input \(\mathbf{s}_l\), we compute \textit{query}, \textit{key}, and \textit{value} using learned weight matrices \(\mathbf{W}^Q\), \(\mathbf{W}^K\), and \(\mathbf{W}^V\):
\begin{equation}
\label{eq:qkv}
\mathbf{Q} = \mathbf{S} \mathbf{W}^Q, \quad \mathbf{K} = \mathbf{S} \mathbf{W}^K, \quad \mathbf{V} = \mathbf{S} \mathbf{W}^V,
\end{equation}
where $\mathbf{Q} = [\mathbf{q}_1, \mathbf{q}_2, \dots, \mathbf{q}_n],  \mathbf{K} = [\mathbf{k}_1, \mathbf{k}_2, \dots, \mathbf{k}_n], \mathbf{V} = [\mathbf{v}_1, \mathbf{v}_2, \dots, \mathbf{v}_n]$. The for each query vector \(\mathbf{q}_l\), compute the \emph{attention score} with each key vector \(\mathbf{k}_j\):
\begin{equation}
\label{eq:score_lj}
\text{score}_{lj} = \frac{\mathbf{q}_l \cdot \mathbf{k}_j^T}{\sqrt{d_k}}
\end{equation}
where $d_k$ is the dimensionality of the key vectors. The value of $\text{score}_{lj}$ can be viewed as the degree of ``attention'' that the $l$-th item in the sequence should give to the $j$-th item. A softmax function is applied to normalize the attention scores into attention weights so that they sum up to 1:
\begin{equation}
\label{eq:alpha_lj}
\alpha_{lj} = \frac{\exp(\text{score}_{lj})}{\sum_{j=1}^{n} \exp(\text{score}_{lj})}.
\end{equation}
The output for each query vector is the weighted sum of the value vectors:
\begin{equation}
\label{eq:z_l}
\mathbf{z}_l = \sum_{j=1}^{n} \alpha_{lj} \mathbf{v}_j
\end{equation}
Thus, the output of the self-attention layer is $\mathbf{Z} = [\mathbf{z}_1, \mathbf{z}_2, \dots, \mathbf{z}_n]$, which is a matrix that represents the consumer's sequential consumption history, accounting for global dependencies among the consumptions.

\subsubsection*{GRU layer.} The Gated Recurrent Unit (GRU) is a type of recurrent neural network (RNN) designed to model sequences by processing them one step at a time, making them naturally suited for capturing local temporal dependencies and sequential order. GRUs are highly effective in tasks that require maintaining memory over time. Temporal dependencies are managed through its two gating mechanisms: the update gate $\mathbf{u}_t$, which controls how much of the past information is retained at time $t$, and the reset gate $\mathbf{r}_t$, which determines how much of the previous state is forgotten at time $t$. These mechanisms are detailed below. 

The output $\mathbf{Z} = [\mathbf{z}_1, \mathbf{z}_2, \dots, \mathbf{z}_n]$ from the self-attention layer is passed through the GRU, which processes the sequence step-by-step to compute the hidden states at each time step $t$. Let the first output of the GRU layer be the first vector from the self-attention layer: $\mathbf{h}_1 = \mathbf{z}_1$. Then starting from $t = 2$, the output of the GRU layer $\rvh_t$ depends on $\rvh_{t-1}$ and $\rvz_t$ through a reset gate $\mathbf{r}_t$ and an update gate $\mathbf{u}_t$:
\begin{equation}
\label{eq:reset_gate}
\mathbf{r}_t = \sigma(\mathbf{W}_r \mathbf{z}_t + \mathbf{U}_r \mathbf{h}_{t-1} + \mathbf{b}_r), 
\end{equation}
\begin{equation}
\label{eq:update_gate}
\mathbf{u}_t = \sigma(\mathbf{W}_u \mathbf{z}_t + \mathbf{U}_u \mathbf{h}_{t-1} + \mathbf{b}_u),
\end{equation}
where $\mathbf{W}_r$, $\mathbf{U}_r$, $\mathbf{W}_u$, and $\mathbf{U}_u$ are learnable weight matrices and $\mathbf{b}_r$ and $\mathbf{b}_u$ are learnable vectors. $\mathbf{r}_t$ controls how much of the previous hidden state \(\mathbf{h}_{t-1}\) to forget; $\mathbf{u}_t$ determines how much of the previous hidden state should be carried forward. The candidate hidden state $\tilde{\mathbf{h}}_t$ is then calculated as 
\begin{equation}
\label{eq:cand_hid_state}
\tilde{\mathbf{h}}_t = \tanh(\mathbf{W}_h \mathbf{z}_t + \mathbf{U}_h (\mathbf{r}_t \odot \mathbf{h}_{t-1}) + \mathbf{b}_h),
\end{equation}
which incorporates the reset gate to adjust the influence of the past state. Here $\odot$ denotes the element-wise product, and $\mathbf{W}_h$, $\mathbf{U}_h$ and $\mathbf{b}_h$ represent learnable weight matrices and vector respectively. Finally, the new hidden state \(\mathbf{h}_t\) is updated by interpolating between the previous hidden state \(\mathbf{h}_{t-1}\) and the candidate hidden state $\tilde{\mathbf{h}}_t$ using the update gate $\mathbf{u}_t$: 
\begin{equation}
\label{eq:new_hid_state}
\mathbf{h}_t = \mathbf{u}_t \odot \mathbf{h}_{t-1} + (1 - \mathbf{u}_t) \odot \tilde{\mathbf{h}}_t.
\end{equation}
After processing the entire sequence in \(\mathbf{Z}\) one by one, 
the output of the GRU layer is the sequence of hidden states $\mathbf{H} = [\mathbf{h}_1, \mathbf{h}_2, \dots, \mathbf{h}_n]$. We use the embedding of the end hidden state, $\mathbf{h}_n$ as the final embedding for the consumer's sequential consumption history, which encapsulates \emph{both} global dependencies from the self-attention layer and temporal dependencies captured by the GRU. 

\subsubsection{Model architecture and loss function.}
\label{sec:framework_dnn_loss}

As shown in the ``DNN Recommendation Component'' in Fig.\ref{fig:llm_recsys_diagram}, the input layer consists of the following six components: the explanation embeddings from the contrastive-explanation generator ($AE(\bm{zp}^k) and AE(\bm{zn}^k)$ in Section \ref{sec:framework_explanation_generator}), denoted as $E_{\text{pos}}$ and $E_{\text{neg}}$; the consumer and product embeddings retrieved through embedding table lookup ($\mathbf{\ve}^{(c)}_i$ and $\mathbf{\ve}^{(p)}_i$ in Section \ref{sec:framework_dnn_cons_embed}), denoted as $E_c$ and $E_p$; the sequence embedding obtained from the self-attention and GRU layers ($h_n^k$ in Section \ref{sec:framework_dnn_seq_embed}), denoted as $E_{\text{seq}}$; and any contextual features relevant to the consumer's decision-making process, such as location, time of day and day of the week etc., denoted as $E_{\text{context}}$. 

To account for the potential interactions among these inputs, we added another self-attention layer to the concatenated input $X_{\text{input}} = [E_{\text{pos}}, E_{\text{neg}}, E_c, E_p, E_{\text{seq}}, E_{\text{context}}]$ to allow each element attend to each other. This is similar to what's described in Section \ref{sec:framework_dnn_seq_embed}, with the only difference being that each element is now an input type (e.g. positive explanation embedding) instead of a product. Let $I = [\text{pos}, \text{neg}, c, p, \text{seq}, \text{context}]$ be the index set corresponding to each element in the input $X_{\text{input}}$. Then similar to Eq.(\ref{eq:qkv})-(\ref{eq:z_l}), the output corresponding to each element is a weighted combination of all elements in $I$:
$$
X_{\text{self-attn}} = [\sum_{i \in I} \alpha_{\text{pos}, i} \vv_i, \sum_{i \in I} \alpha_{\text{neg}, i} \vv_i, \sum_{i \in I} \alpha_{c, i} \vv_i, \sum_{i \in I} \alpha_{p, i} \vv_i, \sum_{i \in I} \alpha_{seq, i} \vv_i, \sum_{i \in I} \alpha_{\text{context}, i} \vv_i],
$$
where, for example, $\alpha_{\text{pos}, c}$ is the attention weight from the positive explanation embedding to the consumer embedding, $\vv_i$ is the learned value vector for each element $i$. See Appendix \ref{appen:self_attn_2} for the mathematical details for obtaining the attention weights and value vectors. 

With the self-attention layer, the input layer $X_{\text{input}}$ is transformed into $X_{\text{self-attn}}$ with the same dimension. The transformed input $X_{\text{self-attn}}$ is then passed through several layers of a multi-layer perceptron (MLP) followed by ReLU activation functions, which is one of the most commonly used architectures for deep neural networks. The final output layer consists of a single neuron with a sigmoid activation function $\sigma(z) = \frac{1}{1 + e^{-z}}$ that outputs a probability score $\hat{y}$ between 0 and 1. In our settings, the probability represented the model's prediction of a positive outcome (e.g., click, like, or purchase) for the $(consumer, product)$ pair.

 %To account for the heterogenous impact of these embedding components during the recommendation process, we add an attention layer on top of these embeddings, and we use the attention values, which we denote as $\{\alpha_{positive},\alpha_{negative},\alpha_{consumer},\alpha_{product},\alpha_{sequence},\alpha_{context}\}$ to construct the weighted concatenated embeddings of the input layer as $W_{concatenate}=[\alpha_{positive}*W_{positive};\alpha_{negative}*W_{negative};\alpha_{consumer}*W_{consumer};\alpha_{product}*W_{product};\alpha_{sequence}*W_{sequence};\alpha_{context}*W_{context}]$. The attention values are determined following the self-attention mechanism \cite{shaw2018self} that we have previously described:
% \begin{equation}
% \alpha_{i} = \frac{exp(e_{i})}{\sum_{t}exp(e_{t})}
% \end{equation}
% where $e_{i}$ represents the average Euclidean distance between each embedding component and other embedding components. The concatenated embeddings of the input layer $W_{concatenate}$ are

The model is trained using binary cross-entropy loss, defined as 
\begin{equation}
\label{eq:cross_entropy_loss}
\mathcal{L}(y, \hat{y}) = -[y \log(\hat{y}) + (1 - y) \log(1 - \hat{y})],
\end{equation}
where \(y\) is the true label (e.g., click, like, or purchase) and $\hat{y}$ is the DNN model's prediction. For a batch of $N$ samples, the average loss is \(\mathcal{L}_{\text{batch}} = \frac{1}{N} \sum_{k=1}^{N} \mathcal{L}(y_k, \hat{y}_k)\). During training, gradients of the loss are computed and used to update the model's trainable parameters via back-propagation. In the DNN component, the trainable parameters include the weight matrices from the self-attention and GRU layers, the consumer and product embedding tables, the self-attention layer, and the MLP layers. 

\subsection{LLM for profile augmentation}
\label{sec:framework_llm_profile}
The framework described in Fig.\ref{fig:llm_recsys_diagram} relies on the \emph{reasoning} capabilities of LLMs. One of the most powerful techniques to improve the reasoning capabilities of LLMs is through Chain-of-Thought (CoT) prompting \citep{wei2022chain}, where the LLMs are encouraged to break down the process of a complex task and ``think step-by-step''. For example, when solving a multi-step arithmetic problem, the LLM is prompted to explain each calculation step before arriving at the final answer, thereby mirroring the natural problem-solving approach used by humans and enhancing the model's ability to handle complex tasks. Motivated by this, we design a CoT-like technique for the explanation generation process: Instead of directly asking the pre-trained LLM for explanations, we first ask it to generate a richer profile of the product based on the name of the product, and then use these LLM-generated profiles to augment the prompt used to generate the final explanations. This is aligned with the ``think step-by-step'' strategy used in CoT: the first step involves obtaining a better understanding of the products through profile augmentation, and the second step leverages these detailed profiles to generate meaningful explanations.

In particular, for every product in the consumer's consumption history, we first ask the LLMs to create a product profile based on the name of the product. An example prompt is the following:
\begin{quote}
\emph{``Create a succinct profile for a product based on its name. This profile should be tailored for use in recommendation systems and must identify the types of consumers who would enjoy the product''.}
\end{quote} 
The word ``product'' can be changed to fit specific product categories, such as ``restaurant'' or ``hotel'' depending on the dataset. These LLM-generated profiles are then used to augment the sequential consumption history of the consumer. Specifically, the consumption history can be represented as a sequence of these augmented profiles to form the input to the contrastive-explanation generator described in Section \ref{sec:framework_explanation_generator_pretrainllm}.

The LLM-generated profile is illustrated as the dotted arrow (``CoT profile augmentation (optional)'') in Fig.\ref{fig:llm_recsys_diagram}. As an example, here is the LLM-generated profile given the prompt above and the product name ``JW Marriott Hotel Hong Kong``:

\begin{quote}
\emph{`` Revitalize body, mind, and spirit when you stay at the 5-star JW Marriott Hotel Hong Kong. Located above Pacific Place, enjoy the views over Victoria Harbour, the mountains, or the glittering downtown Hong Kong skyline.''.}
\end{quote} 

We see that LLMs are indeed capable of augmenting product profiles by generating detailed and contextually rich descriptions, drawing from their vast world knowledge. Given minimal product information, such as a name or category, LLMs can infer and provide additional attributes, such as product features, typical uses, or consumer sentiment, based on similar items from their training data. This world knowledge allows LLMs to enrich product profiles with insights that would not be captured from standard datasets alone.

\noindent \paragraph{\textbf{Remark 1.}} Note that these LLM-generated profiles in Section \ref{sec:framework_llm_profile} are an \emph{optional} component to our framework, as the LLMs can provide meaningful reasons even when minimal product information (e.g. product name) is given. In our experiments in Section \ref{sec:results}, we conduct ablation studies on this optional chain-of-thought component to demonstrate the extra value it added. As a preview of the results in Section \ref{res_understanding_reasoning}, when the augmented profile information is provided to the pre-trained LLM, the contrastive-explanation generator is able to provide slightly better explanations that help the whole system even more. This is aligned with the insights from Chain-of-Thought (CoT) prompting \citep{wei2022chain} where asking the LLMs to ``think step-by-step'' can further improve their reasoning capabilities. However, the primary performance improvement of LR-Recsys still comes from the inherent reasoning capabilities of the LLMs, as LR-Recsys significantly outperforms baseline models even without the LLM-generated profile information.


\subsection{End-to-End Training Process for LR-Recsys}
\label{sec:framework_training}  

Putting everything together, the complete LR-Recsys framework is illustrated in Fig.\ref{fig:llm_recsys_diagram} and detailed as Algorithm \ref{algo:le_recsys} below. The contrastive-explanation generator serves as a pre-training component that is trained before the DNN recommendation component. The output is then fed into the DNN recommendation component as its input. The trainable parameters of the DNN recommendation component include the self-attention and GRU layers, the consumer and product embeddings, and the MLP layers. The specific trainable parameters for each component of the framework and their respective roles are enumerated in Table \ref{tab:trainable} below.

\begin{table}[hbtp!]
  \centering
  \footnotesize
  \setlength\extrarowheight{4pt}
  \setlength{\tabcolsep}{10pt} % Adjust column separation for better spacing
  \begin{tabular}{llp{7cm}}
    \toprule
    \textbf{Component} & \textbf{Trainable Parameters} & \textbf{Role} \\
    \midrule
    Explanation Generator & Fine-tuned AutoEncoder & Encodes explanations into embeddings. \\
    \midrule
    \multirow{5}{*}{\shortstack[l]{DNN Recommendation \\ Component}} 
      & Self-attention and GRU layer & Encodes sequential consumption history. \\ 
      & Consumer embedding table & Encodes consumer preferences. \\ 
      & Product embedding table & Encodes product characteristics. \\ 
      & Self-attention layer &  Captures attention weights among different inputs. \\ 
      & Multi-layer perceptron layers (MLPs) & Captures nonlinearity and feature interactions. \\
    \bottomrule
  \end{tabular}
  \caption{Trainable parameters within each component of the LR-Recsys framework.}
  \label{tab:trainable}
% \vspace{-0.2in}
\end{table}

For the DNN recommendation component, we utilize the input of $N$ observations, $\mathcal{D} = \{(\vx_k, y_k)\}_{k=1}^N$, where $\vx_k$ represents all input features, and $y_k$ represents the outcome (e.g. click, purchase), which serves as the label for training. Detailed information about the model architecture and hyperparameters such as batch size, learning rate, and training epoch is provided in the Appendix \ref{appen:model_architecture_dnn}. The output is a prediction of the outcome (e.g. like, purchase) for each $(consumer, product)$ pair, given real-time contextual features. During a recommendation session, products are ranked in descending order based on the predicted outcomes. 

\begin{algorithm}
\caption{LLM-Explanation-Powered Recommender System (LR-Recsys)}\label{algo:le_recsys}
\label{algo1}
\begin{algorithmic}[1]
   \REQUIRE dataset $\mathcal{D} = \{(\vx_k, y_k)\}_{k=1}^N$, pre-trained LLM, learning rate $\eta$, number of epochs $E$, batch size $B$
   %%%%%%%%%%%%%%%%%%%%%%%%%%%%%%%%%%%%%%%%%
   %%%%%%%%%%%%%Pre training%%%%%%%%%%%%%%%%
   %%%%%%%%%%%%%%%%%%%%%%%%%%%%%%%%%%%%%%%%%
   \hspace*{-1.5\algorithmicindent} \textbf{\emph{Explanation Generator}:}
   \FOR{$(x_k, y_k) \in \mathcal{D}$}                     
        \STATE \textbf{CoT profile augmentation (optional)}: Use LLMs to enrich product profiles for the sequential consumption history $j_1,...,j_n$ and the candidate product;
        \STATE \textbf{Preparing prompts}: Concatenating the positive and negative explanation prompts with sequential consumption history and candidate product; 
        \STATE \textbf{Explanation generation}: Generate $\bm{zp}^k$ (positive explanation) and $\bm{zn}^k$ (negative explanation) using the pre-trained LLM;
        \STATE \textbf{Reconstruction Loss}: Compute  $\mathcal{L}_{\text{CE}}(\bm{zp}^k, \hat{\bm{zp}}^k)$ and $\mathcal{L}_{\text{CE}}(\bm{zn}^k, \hat{\bm{zn}}^k) $ as in Eq.(\ref{eq:reconstruction_loss_pop}).
   \ENDFOR  

   \STATE \textbf{Fine-tuned AutoEncoder Training}: Train the AutoEncoder $AE(\cdot)$ by minimizing the reconstruction loss: $\mathcal{L}_{\text{recon}} = \sum_{k=1}^N \left(\mathcal{L}_{\text{CE}}(\bm{zp}^k, \hat{\bm{zp}}^k) + \mathcal{L}_{\text{CE}}(\bm{zn}^k, \hat{\bm{zn}}^k) \right) $. 

   %%%%%%%%%%%%%%%%%%%%%%%%%%%%%%%%%%%%%%%%%%%%%%%%
   %%%%%%%%%%%%%DNN reccommendation%%%%%%%%%%%%%%%%
   %%%%%%%%%%%%%%%%%%%%%%%%%%%%%%%%%%%%%%%%%%%%%%%%
   \hspace*{-1.5\algorithmicindent} \textbf{\emph{DNN Recommendation Component}:}
   \STATE \textbf{Initialize:} Trainable parameters: embedding table $\mathbf{E} = (\mathbf{E}_{c}$, $\mathbf{E}_{p})$, self-attention layer $\theta_{\text{attn}} = (\mathbf{W}^Q$, $\mathbf{W}^K$, $\mathbf{W}^V$, $\mathbf{W}_r)$, GRU layer $\theta_{\text{GRU}} = (\mathbf{U}_r$, $\mathbf{b}_r$, $\mathbf{W}_u$, $\mathbf{U}_u$, $\mathbf{b}_u$, $\mathbf{W}_h$, $\mathbf{U}_h$, $\mathbf{b}_h)$ and MLP layers $\theta_{\text{MLP}}$
   
   \FOR{epoch = 1 to $E$} 
        \STATE Shuffle dataset $\mathcal{D}$   
        \FOR{each batch $b = 1$ to $\left\lfloor \frac{N}{B} \right\rfloor$} 
            \STATE Select batch $\mathcal{B}_b = \{(\vx_k, y_k)\}_{k=(b-1)B+1}^{bB}$;   \COMMENT{batch training}
            \FOR{each $(\vx_k, y_k) \in \mathcal{B}_b$}
                \STATE Get positive and negative explanation embeddings $AE(\bm{zp}^k)$ and $AE(\bm{zn}^k)$;
                \STATE Get consumer and candidate product embedding $\mathbf{\ve}^{(c)}_i$ and $\mathbf{\ve}^{(p)}_i$ as in Eq.(\ref{eq:embed});
                \STATE Get sequential consumption history embedding $h_n^k$ with the self-attention and GRU layers; 
                \STATE Concatenate the above and any contextual features $c$ as the input layer of the DNN recommendation component: $\vx_k = (AE(\bm{zp}^k), AE(\bm{zn}^k), \mathbf{\ve}^{(c)}_i, \mathbf{\ve}^{(p)}_i, h_n^k, c)$; 
                \STATE Compute prediction: $\hat{y}_k = f_\theta(\vx_k)$;
            \ENDFOR
            \STATE Compute batch loss: $\mathcal{L}_b = \frac{1}{B} \sum_{(\vx_k, y_k) \in \mathcal{B}_b} \mathcal{L}(y_k, \hat{y}_k)$
            \STATE Update all trainable parameters: $\theta \leftarrow \theta - \eta \nabla_\theta \mathcal{L}_b$, where $\theta = (\mathbf{E}, \theta_{\text{attn}}, \theta_{\text{GRU}}, \theta_{\text{self-attn}}, \theta_{\text{MLP}})$
        \ENDFOR
    \ENDFOR    
   \RETURN $\theta$   \COMMENT{all parameters of the DNN recommendation component}
\end{algorithmic}
\end{algorithm}

\noindent \paragraph{\textbf{Remark 2.}} Our LR-Recsys framework is compatible with any pre-trained LLM, and its performance gains mainly come from leveraging the \emph{reasoning abilities} of the LLMs, rather than their dataset-specific knowledge or summarization skills. To validate this, we conducted several experiments. First, we tested older LLMs, such as GPT-2, which were trained on data \emph{predating} the datasets in our experiments, ensuring that the LLMs have \emph{no} prior knowledge of the specific data. Our framework still provides gains in these cases. Second, we evaluated LLMs with varying reasoning capabilities and found that models with stronger reasoning consistently produced better results. Finally, we explored an alternative approach that used LLMs to generate summaries of consumption history, but this method yielded inferior performance compared to when the LLMs were asked to provide reasons. The details of these experiments are outlined in Section \ref{sec:results}. Overall, these findings confirm that the main advantage of our LR-Recsys framework is its ability to leverage the reasoning capabilities of LLMs rather than their dataset knowledge or summarization skills.


% Next in Section \ref{sec:theory}, we provide theoretical asymptotic analysis based on statistical learning theories to shed lights on the theoretical advantage of LR-Recsys. Specifically, challenging the common belief that explainability hurts model performance, we demonstrate how integrating explanations into the training process of a ML framework can actually improve its performance.

% statistical insights to shed light on why LLMs' reasoning capability can help the recommender system. To understand why LLMs' explanations can simultaneously improve predictive performance and explainability, rather than creating a trade-off, in the next section, we provide theoretical insights using high-dimensional statistical learning theory. 


\begin{comment}
==== 

Details in Section \todo{Experiments}

While our experiments show that augmenting product information with LLM-generated profiles can further boost performance, the main performance gains of our framework come from leveraging the reasoning capabilities of LLMs.

is agnostic to the choice of LLMs. To demonstrate that only the reasoning capability of the LLM is used here, not their knowledge about the dataset itself, we conduct experiments (1) on older versions of LLMs (e.g. GPT-2 trained on data up to 2019) whose training data are before the generate of the experiment dataset. (2) We also show that LLMs with better reasoning capabilities generate better results, which confirms that it is the reasoning capability that we leveraged to generate the win. (3) Experiments ablating the LLM-generated profiles also work well.

In our experiments \todo{add experiment to confirm}, we find that more powerful LLMs generate better results (because of their better reasoning capabilities). 


The framework described in Fig.\ref{fig:llm_recsys_diagram} only relies on the \emph{reasoning} capabilities of LLMs. Another strength of LLMs is their \emph{world knowledge} thanks to their vast training data. For example, given minimal product information, such as the name of the product, LLMs can infer and provide additional attributes, such as product features, typical uses, or consumer sentiment, based on similar items from their training data. This world knowledge allows LLMs to enrich product profiles with insights that would not be captured from standard datasets alone. 

Motivated by this, we also explored an \emph{optional} component which leverages LLMs to augment the profiles of the products to serve as the input to our framework. In particular, for every product in the consumer's consumption history, we ask the LLMs to create a product profile based on the name of the product. An example prompt is the following:
\begin{quote}
``Create a succinct profile for a product based on its name. This profile should be tailored for use in recommendation systems and must identify the types of consumers who would enjoy the product''.
\end{quote} 
The word ``product'' can be changed to fit specific product categories such as ``restaurant'' or ``hotel'' depending on the dataset. In Appendix \todo{LLM for profile augmentation}, we present the examples of the LLM's output. These LLM-generated profiles can be used to augment the sequential consumption history of the consumer. Specifically, the sequential consumption history can be represented as a sequence of these augmented profiles to form the input to the Explanation Generator as described in Section \ref{sec:framework_explanation_generator_pretrainllm}. 

Note that these LLM-generated profiles are an \emph{optional} component to our framework, as the LLMs can provide meaningful reasons even when minimal product information (e.g. product name) is given. In our experiments, we conduct ablation studies on this optional component to demonstrate the extra value it added. In particular, when the augmented profile information is provided to the pre-trained LLM, the explanation generator is able to provide better explanations that help the whole system even more. This is an interesting observation as both the explanation generator and the profile augmentation is done by the same pre-trained LLM. By explicitly asking for a more detailed product profile, the LLM is able to provide more helpful explanations. This is similar to the famous chain-of-thought (CoT) prompting technique \citep{wei2022chain} where asking the LLM to ``think step-by-step'' can greatly improve its reasoning capabilities. 

While our experiments show that augmenting product information with LLM-generated profiles can further boost performance, the main performance gains of our framework come from leveraging the reasoning capabilities of LLMs.


% Our DNN recommendation component can be viewed as an augmentation of a state-of-the-art DNN-based recommender system. 




====================

Specifically, given consumer $i$'s consumption history, represented as a sequence of $n$ products $j_{i1}, j_{i2}, ..., j_{in}$, we prompt the LLM to provide explanations for why the consumer may or may not enjoy a candidate product. We refer to these as \emph{positive explanations} and \emph{negative explanations}.

The prompt follows this format: ``Provide a reason for why this consumer [purchased this product / watched this movie / stayed at this hotel], based on the provided profile of the past [products / movies / hotels] the consumer [purchased / watched / stayed at], and the profile of the current [product / movie / hotel]. Answer with exactly one sentence in the following format: `The consumer [purchased this product / watched this movie / stayed at this hotel] because the consumer ... and the hotel ....' "


Given specific datasets, can be adjusted to tailor to the specific domain.  

(include both pos and neg, because both provide perspectives on why and why not, and let the DNN model figure out which is helpful for learning)


(pos / neg prompt: depending on the specific dataset, the prompt used is "define why the consumer consumes a certain product) 

(also introduce fine-tuned autoEncoder here, detail in appendix )

=================

% Google diagram: https://docs.google.com/document/d/1b5uQ2OpeMLUgPv4OptpaxCyRQrqD_pTzifAWs4pMyWI/edit

\end{comment}

\section{Statistical Insights}
\label{sec:theory}

% Why would incorporating LLM-generated explanations improve the performance of an ML recommender system? In this section, we provide mathematical insights into why and how explanations can help the ML models learn, challenging the common belief that explainability typically comes at a cost of model performance.

In this section, we provide mathematical insights into why and how explanations can help the ML models learn. In a typical machine learning setting, the model is trying to learn from a particular \emph{environment} which is characterized by the training data. The goal of the ML model is to predict the outcome $Y$ based on the input $X$, with $X$ typically being high-dimensional vectors characterizing the consumer, the product, and the context. By asking the LLMs to generate why and why not the consumer likes (or purchases) the product, we are asking the LLMs to extract the \emph{probable} decision-making factors that drive the observed outcome (e.g., like or purchase). Incorporating this knowledge in the model training process will intuitively increase the model's learning efficiency, and improve the model's performance given the same amount of training data. In the remainder of this section, we leverage high-dimensional statistical learning theory to mathematically characterize the gains provided by LLM-generated explanations.



% This kind of prediction requires only a strong correlation between the input and the outcome variable. In other words, the ML model tends to rely on correlation rather than causation to make predictions. As a result, they may not know which variables (i.e. which dimensions of $X$) are truly important ones that capture true causal factors in predicting $Y$, but instead rely on the whole high-dimensional set of all available variables in $X$ to make the prediction. This intuitively decreases the signal-to-noise ratio and hurts the models learning efficiency during training \citep{fan2014challenges}. 

% By asking the LLMs to generate why and why not the consumer likes (or purchases) the product, we are asking the LLMs to extract the critical decision making factors that drive the observed outcome (e.g. like or purchase). If these extracted decision making factors are truly the causal factors that explain the relationship between the input $X$ and the outcome $Y$, then incorporating these factors in the model will likely increase the model's learning efficiency, and improving the model's performance given the same amount of training data. In the rest of this section, we mathematically characterize these gains provided by LLM generated explanations, by leveraging high-dimensional statistical learning theory. 

We consider a general high-dimensional learning scenario where the observations $\{\vx_k, y_k\}_{k=1}^N$ follow:
\begin{equation}
\label{eqn:model_full}
y_k = f(\vx_k) + \epsilon_k,
\end{equation}
where $\vx_k = (x_{k1}, ..., x_{kp})$ is a $p$-dimensional vector representing the input feature (independent variable) for observation $k$, $y_k$ is the outcome for observation $k$, $\{\epsilon_k\}_{k=1}^n$ are independent and identically distributed errors, and $f$ is the ground-truth function capturing relationship between $\vx_k$ and $y_k$. $f$ can be any function of choice, ranging from linear regression models to neural networks. The number of variables $p$ can be very large, as in typical high-dimensional learning settings. Let $S^* \subset \{1,...,p\}$ be the index subset of important variables that predicts the outcome, the size of which, $s^* = |S^*|$, is usually significantly smaller than $p$.\endnote{In other words, $|S^*|$ does not include \emph{spurious variables} \citep{fan2014challenges}, which appear to be statistically significant but do not contribute to the true relationship between the dependent variable and independent variables. They are often correlated with both the dependent variable and other predictors due to random chance, measurement errors, or unobserved confounding factors.} For example, there may be numerous attributes that describe a product, but only a few attributes (such as size and color) matter for the consumer's final decision-making. Therefore, the true regression function is 
\begin{equation}
\label{eqn:model_s}
y_k = f^*(\vx_{k,S^*}) + \epsilon_k,
\end{equation}
where $\vx_{k,S^*}$ selects the elements of $\vx_k$ that corresponds to the indices in $S^*$, and $f^*(\cdot)$ captures the relationship between the true important variables $\vx_{k,S^*}$ and the outcome $y_k$. 

\begin{comment}
In next next, we will demonstrate two main points: First, LLMs have better knowledge of $S^*$ than simply relying on the training data alone to infer $S^*$; Second, knowledge of $S^*$ can significantly boost the learning efficiency, thereby improving prediction accuracy when using the same amount of training data. For ease of presentation we first focus on the case where the function family is linear, i.e. the model is 
\begin{equation}
\label{eqn:linear_model_full}
y_k =  \vx_k^T \vbeta^* + \epsilon_k,
\end{equation}
and the true regression function is 
\begin{equation}
\label{eqn:linear_model_s}
y_k =  \vx_{k,S^*}^T \vbeta_{S^*}^* + \epsilon_k,
\end{equation}
where $\vbeta_{S^*}^*$ selects the elements of $\vbeta^*$ that corresponds to the indices in $S^*$. Later in Appendix \todo{theoretical results generalized to ML}, we generalize the theoretical results to general ML models. 
\end{comment}

When we ask an LLM to explain why $X$ leads to $Y$, we are essentially directly asking for $S^*$. To see this, let’s revisit the example in Fig.\ref{fig:example_explanations} from Section \ref{sec:framework_LLM_reasoning}. Here, the feature $X$ describes the sequential consumption history of the consumer (i.e., apple, orange, pear, watermelon) plus the candidate product (i.e., orange juice), while $Y$ represents the outcome (whether the consumer purchased the candidate product). $X$ is potentially high-dimensional as it includes not only the details of each product in the consumption history such as color, texture, taste, and shape of each product, but also categorical indicators such as \texttt{hasBoughtApple}, \texttt{hasBoughtOrange}, along with attributes of the candidate product (e.g., ingredients, packaging, shape). Therefore, the number of features $p$ is large. When the LLM is prompted to explain why a consumer with a particular consumption history might prefer the candidate product, it identifies relevant features, such as \texttt{hasBoughtOrange} for the positive case (Fig.\ref{fig:pos_explanation}) and \texttt{hasOnlyBoughtWholeFruits} for the negative case (Fig.\ref{fig:neg_explanation}) and designates them as the important feature set $S^*$. 

Next, we are going to make two claims using statistical learning theory. First, LLMs are \emph{more likely} to give accurate information about $S^*$ than simply relying on the training data alone to infer $S^*$; Second, having access to a more accurate estimate of $S^*$ can help the ML model learn better within its own environment. We present the theory with high-dimensional linear models first, and generalize to nonlinear ML models later. 


\subsection{LLMs have better knowledge of $S^*$ than the training data itself}   

How are LLMs obtaining the knowledge of $S^*$, the set of important variables? We can use the statistical theory on \emph{multi-environment learning} \citep{peters2016causal} to explain. In multi-environment learning, the goal is to predict the outcome variable $y$ as a function of $\vx$ using data observed from multiple environments, where each \emph{environment} represents a distinct setting or context within which observations are made. For LLMs, their training data comes from a variety of sources, including diverse domains, contexts, and linguistic structures \citep{dubey2024llama, achiam2023gpt}. Therefore, training the LLMs is analogous to learning from multiple environments. Specifically, an LLM has seen many more environments of slightly different distributions of X and Y, but the mapping from X to Y is the same, i.e., the data generating process across these multiple environments is the \emph{same}. For example, in the toy example in Section \ref{sec:framework_LLM_reasoning}, the LLMs have probably seen different orange juice purchase behaviors across numerous shopping scenarios (e.g., grocery store, online, farmer's market) from consumers with different purchase histories. However, the important decision-making variables for whether the consumer will purchase orange juice or not (such as \texttt{isOrangeProduct}, \texttt{isNotWholeFruit}) should remain the same across the different environments. 

Written formally, let $\mathcal{E}$ be the set of environments. Each environment $e \in \mathcal{E}$ observes at least $n$ samples $(\vx_1^{(e)}, y_1^{(e)}), ..., (\vx_n^{(e)}, y_n^{(e)}) \sim \mu^{(e)}$, which are from the model 
\begin{equation}
\label{eqn:multi_env}
% y_k^{(e)} =  {\vx_{k,S^*}^{(e)}}^T \vbeta_{S^*}^* + \epsilon_k^{(e)},
y_k^{(e)} = {\vbeta_{S^*}^*}^T {\vx_{k,S^*}^{(e)}} + \epsilon_k^{(e)},
\end{equation}
where $y_k^{(e)}$ is the outcome (dependent variable) for observation $k$ in environment $e$, and $\vx_k^{(e)} \in \mathbb{R}^p$ represents the full set of variables (including the non-important ones). In a typical multi-environment learning setting \citep{fan2023environment, peters2016causal}, the unknown set of important variables $S^* = \{ j: \beta^*_j \neq 0 \}$ and the model parameters $\vbeta^*$ are the same (or \emph{invariant}) across different environments, but the distribution of the training data, $\mu^{(e)}$, may \emph{vary} across environments. 

Intuitively, if there are more observations seen from multiple environments, then the model may have a better knowledge about $S^*$, which is the same across different environments. For example, when the LLMs see enough consumers purchasing orange juice from many environments, it may be able to recover the important decision-making variables (i.e., $S^*$) for orange juice purchase in a new environment even without seeing any data from this new environment. Indeed, \citet{fan2023environment} recently showed that multi-environment least squares could recover $S^*$ with high probability provided that the number of observations $n$ and the number of environments $|\mathcal{E}|$ is large enough, even in high-dimensional settings where $p > n$. In high-dimensional statistics, this is referred to as \emph{variable selection consistency}. 

We simplify the setup in \citet{fan2023environment} and summarize the variable selection consistency properties for multi-environment learning below. In particular, \citet{fan2023environment} proposes the following environment invariant linear least squares (EILLS) estimator $\hat{\vbeta}_L$ which minimizes the following objective: 
\begin{equation}
\label{eqn:eills_obj}
\hat{\vbeta}_L = \argmin_{\vbeta} \hat{R}(\vbeta) + \gamma \hat{J}(\vbeta) + \lambda ||\vbeta||_0,
\end{equation}
where 
\begin{equation}
\label{eqn:eills_individual_loss}
\begin{aligned}
\hat{R}(\vbeta) &= \sum_{e \in \mathcal{E}} \sum_{k=1}^n (y_k^{(e)} - \vbeta^T \vx_k^{(e)}  )^2 ,\\
\hat{J}(\vbeta) &= \sum_{j=1}^p \mathbf{1} \{\beta_j \neq 0\} \sum_{e \in \mathcal{E}} (\sum_{k=1}^n x_{k,j}^{(e)} (y_k^{(e)} - \vbeta^T \vx_k^{(e)}) )^2. 
\end{aligned}
\end{equation}
Here $\hat{R}(\vbeta)$ is the usual total mean squared loss across all environments, $\hat{J}(\vbeta)$ is the  \emph{invariance regularizer} that encourages the model to focus only on important variables that are useful across all environments, and $||\vbeta||_0$ is the $l_0$-penalty that encourages as few nonzero elements in $\vbeta$ as possible. 

We simplify the convergence results in Theorem 4.5 of \citet{fan2023environment} and obtain the following variable selection consistency results as Lemma \ref{lem:var_selection_ellis} below.
\begin{lem}
\label{lem:var_selection_ellis}
Under conditions detailed in Appendix \ref{appen:lemma_1_cond}, the multi-environment estimator $\hat{\vbeta}_L$ has variable selection consistency, i.e. 
$$ \mathbf{P}(\text{supp}(\hat{\vbeta}_L) = S^*) \rightarrow 1   $$  
as long as $n, p, s^* \rightarrow \infty$ and $n \gg  s^* \beta_{\text{min}}^{-2}\log{p} $, where $s^* = |S^*|$ and $ \beta_{\text{min}} = \min_{j \in S^*} |\beta^*_j| $. 
% \vspace{-0.1in}
\end{lem}

Lemma \ref{lem:var_selection_ellis} states that as long as there are enough observations ($n$) from enough environments ($|\mathcal{E}|$), then the model would be able to recover the true set of important variables with probability approaching 1, even if the potential number of variables $p$ and the number of truly important variables $s^*$ also grows with $n$. Given the fact that LLMs are trained on vast sources of texts, they probably learn from a huge number of observations and numerous environments, making $\mathbf{P}(\text{supp}(\hat{\vbeta}_L) = S^*)$ very close to 1. Such variable selection consistency is only achievable with a large number of observations from multiple environments, which explains why LLMs are much more capable of providing convincing explanations. 

% state-of-the-art methodologies involve adding a penalty term in the loss function that encourages the models to focus on a smaller set of variables that are truly important. [lasso recover S* rate] 

On the contrary, when there is only data from one environment with a limited number of observations (i.e., $n$ much smaller than in the LLM's case), then $\mathbf{P}(\text{supp}(\hat{\vbeta}_L) = S^*)$ is not necessarily close to 1 due to limited $n$. In fact, \citet{fan2023environment} proved that having more than one environment is actually \emph{necessary} for identifying $\vbeta^*$ and $S^*$. We detail their result and our related insights in Appendix \ref{appen:more_than_one_env_needed}. Therefore, LLMs, as discussed above, can obtain a more robust estimate of $S^*$ due to the vast data and numerous environments they have seen.

% LLMs learns invariant across multiple environments 
 
\noindent \paragraph{\textbf{Remark 3.}}
Note that when we ask an LLM for a reason why $\vx$ leads to $y$, we are primarily asking for $S^*$—the important features—rather than the function $f^*$ itself (e.g., the coefficients $\vbeta^*$ if $f^*(\vx) = \vx^T {\vbeta^*} $). For instance, when prompted to explain why a consumer with a specific fruit consumption history might prefer orange juice, the LLM identifies relevant features (like \texttt{hasBoughtOrange} and \texttt{isOrangeProduct}) and assigns them to $S^*$. However, the LLM does \emph{not} quantify the impact of these features on the outcome; for example, it does not predict how much \texttt{hasBoughtOrange} and \texttt{isOrangeProduct} changes the purchase likelihood. Thus, in our framework, LLM explanations highlight important features but do not estimate the prediction function. In the experiments in Section \ref{sec:results}, we also explored using LLMs to directly predict outcomes (i.e., using LLMs to estimate the prediction function $f^*$). This approach led to significantly poorer performance compared to our framework, indicating that while LLMs may excel at identifying the important variables $S^*$, traditional ML models are still needed to estimate the prediction function $f^*$ effectively. This is also aligned with the findings of \citet{ye2024lola}, who observed that LLMs perform poorly as direct prediction functions, and \citet{jeong2024llm}, who found that LLMs are effective at feature selection.

\subsection{Better knowledge of $S^*$ leads to better model performance} 

As discussed above, the number of true important variables $|s^*|$ is usually much smaller than the total number of variables $p$. The performance of the models is measured by the estimation errors for the model parameters $\vbeta^*$, or $||\hat{\vbeta} - \vbeta^* ||_2$ where $||\cdot||_2$ represents the $L_2$ norm. Intuitively, knowing $S^*$ will help the model learn $\vbeta^*$ more efficiently as if it is assisted by an oracle. Next, we mathematically quantify why a better knowledge of $S^*$ leads to better learning efficiency and predictive performance. 

In a high-dimensional learning setting when we don't know $S^*$, the most popular method to help the model focus on important variables is Lasso (Least Absolute Shrinkage and Selection Operator) \citep{tibshirani1996regression}, which adds an $L_1$ penalty to the objective function and encourage the model to shrink the estimates for the non-important variables to zero. The Lasso estimator $\hat{\vbeta}$ is obtained by:
\begin{equation}
\label{eqn:lasso}
\hat{\vbeta}_{\text{Lasso}} = \argmin_{\vbeta} \left\{ \frac{1}{2n} \sum_{k=1}^n (y_k - \vx_k^T \vbeta)^2 + \lambda \sum_{j=1}^p |\beta_j| \right\},
\end{equation}
where $\lambda \geq 0$ is the regularization parameter that controls the amount of shrinkage. Under typical assumptions of high-dimension learning, the convergence rate of Lasso is given by \citet{bickel2009simultaneous}: 
\begin{equation}
\label{eqn:lasso_conv}
||\hat{\vbeta}_{\text{Lasso}} - \vbeta^* ||_2 = O_P\left(\sqrt{\frac{s \log p}{n}}\right),
\end{equation}
where $n$ is the sample size and $O_p$ denotes the order in probability. \endnote{A sequence of random variables \( X_n \) is said to be \( O_P(a_n) \) if for any \( \epsilon > 0 \), there exists a constant \( M > 0 \) such that $\Pr\left( \left| \frac{X_n}{a_n} \right| > M \right) \leq \epsilon \text{ for all sufficiently large } n.$} 

When we do know $S^*$, then a standard ordinary least squares (OLS) would suffice to recover the true model parameters, by estimating the elements of $\vbeta^*$ which are known to be nonzero (i.e., in $S^*$):
\begin{equation}
\label{eqn:ols}
\hat{\vbeta}_{\text{Orc}} = \argmin_{\vbeta \in \{ \vbeta: \beta_j = 0, \; \forall j \in S^* \}} \frac{1}{2n} \sum_{k=1}^n (y_k - \vx_{k}^T \vbeta)^2 .
\end{equation}

The corresponding convergence rate is \citep{tibshirani1996regression}:
\begin{equation}
\label{eqn:oracle_conv}
||\hat{\vbeta}_{\text{Orc}} - \vbeta^* ||_2 = O_P\left(\sqrt{\frac{s}{n}}\right).
\end{equation}

% See Appendix \todo{Convergence results of Lasso and OLS, $https://junwei-lu.github.io/papers/BST235_Notes.pdf$ page 56, full rank $r=s$} for a detailed discussion on how these convergence rates are derived.

Comparing Eq.(\ref{eqn:lasso_conv}) and Eq.(\ref{eqn:oracle_conv}), we see that having knowledge of $S^*$ will increase the convergence rate by a factor of $\sqrt{\log p}$. This is a significant factor, especially in high-dimensional learning settings where $p > n$ and $p \rightarrow \infty$. In a recommender system learning setting, arguably, the number of variables $p$ indeed approaches infinity as the description of the products can be arbitrarily high-dimensional. Therefore, knowing $S^*$ for a recommender system will significantly boost learning efficiency using the same amount of data, ultimately leading to better model performance.

As an illustration, Fig.\ref{fig:convergence_rate_comparison} compares the convergence rates with a fixed number of observations and true important variables ($n=1000$, $s^*=20$) and a varying number of predictors ($p$). As expected, having knowledge of $S^*$ significantly reduces the estimation error, with the reduction becoming larger for larger $p$.

\begin{figure}[hbtp!]
    \centering
    \includegraphics[width=0.5\linewidth]{figures/convergence_rate_comparison.png}
    \caption{(Color online) Convergence rate comparison for the Lasso (unknown $S^*$) and Oracle estimator (known $S^*$).}\label{fig:convergence_rate_comparison}
    % \vspace{-0.2in}
\end{figure}

% Fig.\todo{figure illustrating convergence rate, with fixed n and growing p} illustrates the performance gains with varying values of $n$ and $p$. We see that \todo{with fixed n, knowing $S^*$ would improve the estimation error by X fold.}




% First, by asking the LLMs for explanations, we get a good estimate of the true set of important variables $S^*$ by exploiting LLMs' reasoning capabilities. Then by incorporating the knowledge of $S^*$ from the LLM's explanations as the input $\vx$ to a recommender system, the system is able to learn much more efficiently compared with systems without the knowledge of $S^*$, saving training examples in the order of $\log(p)$. 

So far, we have established the statistical insights about why incorporating LLM-generated explanations, as in our LR-Recsys framework, would help the predictive performance of the model. Note that LR-Recsys does not exactly mirror the high-dimensional learning setup as discussed above, as the theory above is about linear models but LR-Recsys is a nonlinear neural network model. We show that in the nonlinear and ML model cases, the discussions above still hold, but the convergence rate comparisons are slightly more complex. See Appendix \ref{appen:multi_env_nonlinear} for detailed results. 

The theories above suggest that, to achieve optimal convergence, one should discard any features not included in $S^*$. However, in our framework, we propose to keep some low-dimensional non-interpretable features, such as consumer and product embeddings, that are specific to the current environment and find them to be helpful in the experiments. The theoretical assumption is that $S^*$ remains invariant across all environments, but in practice, individual environments may exhibit unique characteristics (e.g., the packaging color of orange juice might matter in some regions but not others). Therefore, our framework allows for the flexibility that individual environments may have their own specificities. 

% In the next section, we validate these statistical insights and training efficiency gains of our LR-Recsys framework on multiple real-world recommender system datasets.


\begin{comment}

=== 

II. How is better knowledge of S* helping ML model learning 
In a regular setting, the ML model is learning from X to Y where X is high dimension. However, the true important variables are only a subset of X, denoted as S*. In the section above we talked about how LLMs have a better knowledge about S*. Now we look at how a better knowledge of S* helps a regular ML model's learning. 

"as if assisted by an oracle" 

In high-dimensional learning, the most popular way to help the model to focus on important variables is Lasso, which adds L1 penalty to the objective function and encourage the model to shrink the estimates for the non-important variables to zero (write math equation of Lasso loss function here). Lasso has a convergence rate: O(sqrt(slogp/n).

When we know which variables are important, i.e. when we know S*, then the convergence should be O(sqrt(s/n)), i.e. does not need to pay the cost of not knowing the true support 
% (Reference: https://junwei-lu.github.io/papers/BST235_Notes.pdf page 56) 

[In our proposed framework, we replace the input x with S*, and keeps the uninterpretable features consumer and item embeddings, so convergence rate should be between the above two, which explains why it learns more efficiently ]
Then how is S* helping ML model who is learning on S
to only focus on the important variables in X 


[end] in the nonlinear case and ML model case, the above discussion also holds. See Appendix \todo{nonlinear case} for more discussions. 

When there are enough environments seen, they true S* can be recovered with high probability:
In the linear case..., In the nonlinear case, see https://arxiv.org/pdf/1706.08576 and https://arxiv.org/pdf/1706.08058 for sequential data.

%%%%%%%%%%%%%%%%%%%%%%%%%%%%%%%%%%%%%%%%%%%%%%%%%%%%%%%%%%%%%%%%%%%%%%%%%%%%%%%%%%%%%%%%%%%%%%%%%%%%%%%%%%%%%%%%%%%%%%%%%%%%%%%%
%%%%%%%%%%%%%%%%%%%%%%%%%%%%%%%%%%%%%%%%%%%%     Appendix     %%%%%%%%%%%%%%%%%%%%%%%%%%%%%%%%%%%%%%%%%%%%%%%%%%%%%%%%%%%%%%%%%%
%%%%%%%%%%%%%%%%%%%%%%%%%%%%%%%%%%%%%%%%%%%%%%%%%%%%%%%%%%%%%%%%%%%%%%%%%%%%%%%%%%%%%%%%%%%%%%%%%%%%%%%%%%%%%%%%%%%%%%%%%%%%%%%%
For general ML, 
I. [https://arxiv.org/pdf/2405.04715 prop 10] variable selection results (when n is large enough, $\hat{S}$ include all true important variable, and does not include all spurious variable. 

II. Page 20 in https://arxiv.org/pdf/1708.06633 $n^(-2beta / (beta + p))$

[In the end, Remark 3: mention that this is different from forcing explainability in model training, because we don't usually know a priori what are the important factors, as a result forcing the models to follow those rules will only decrease (rather than increase) their learning efficiency.]
%%%%%%%%%%%%%%%%%%%%%%%%%%%%%%%%%%%%%%%%%%%%%%%%%%%%%%%%%%%%%%%%%%%%%%%%%%%%%%%%%%%%%%%%%%%%%%%%%%%%%%%%%%%%%%%%%%%%%%%%%%%%%%%%
%%%%%%%%%%%%%%%%%%%%%%%%%%%%%%%%%%%%%%%%%%%%%%%%%%%%%%%%%%%%%%%%%%%%%%%%%%%%%%%%%%%%%%%%%%%%%%%%%%%%%%%%%%%%%%%%%%%%%%%%%%%%%%%%
%%%%%%%%%%%%%%%%%%%%%%%%%%%%%%%%%%%%%%%%%%%%%%%%%%%%%%%%%%%%%%%%%%%%%%%%%%%%%%%%%%%%%%%%%%%%%%%%%%%%%%%%%%%%%%%%%%%%%%%%%%%%%%%%

===

% learning efficiency
% [consumer and product embeddings can be viewed as capturing additional information]

% i=1 are independent and identically distributed p-dimensional covariate vectors,
% {i}n
% i=1 are independent and identically distributed errors and βÅ is a p-dimensional regression coefficient vector.

%  This kind of prediction requires only a strong correlation with the outcome variable.

% The current ML model is one environment, trying to learn the relationship between X and Y. However, ML models tend to rely on correlation rather than causation to make predictions (they can be viewed as gigantic correlation extractors). As a result, they may not know which variables are the truly important ones that capture the underlying data generation process to make the prediction, which decreases the signal-to-noise ratio, and may confuse the ML model during learning.



% Specifically, we leverage high-dimensional statistical learning theory to characterize the gains in statistical learning efficiency provided by incorporating LLM generated explanations in model training. 

% Why LLM's explanation would help? contrary to existing understanding that incorporating explainability hurts model performance. In this section we shed lights on when and how explainability can help a model by adopting statistical analysis on the learning efficiencies of ML models with or without explainability. Specifically, we adopt the multi-environment learning theory in statistics to explain.




I. Why LLMs have better knowledge about S* [focus more discussion on this]
Multi-environment learning theory, when number of observations n and number of environments is large enough, then we have variable selection consistency (Thm 4.5 in https://arxiv.org/pdf/2303.03092). 



\end{comment}





% \section{Actionable Insights}
% Customer aquisition: Which customer segment does my product attract;

% Improving product characteristics: How to improve my product so that it attracts more consumers 
 
% Can we improve the methodology so that our framework can generate actionable insights (and potentially for different segments of the consumers)? E.g. ``restaurant A should improve cleanliness if want to attract more [segment] customers''

\section{Experiments}
\label{sec:results}
%% !!!!!! \todo{New results on summarization: Last row in Table 5 (shall this be ablation or understanding?)}
%% !!!!!! \todo{Change "user" to "consumer"?)}

\subsection{Data}
We conduct a series of experiments on industrial datasets to compare the performance of LR-Recsys against state-of-the-art recommender systems in the literature. We start with introducing the three datasets used. 

\begin{itemize}
\item \textbf{Amazon Movie}: This dataset captures consumer purchasing behavior in the Movies \& TV category on Amazon \citep{ni2019justifying}\footnote{\url{https://nijianmo.github.io/amazon/index.html}}. Collected in 2023, it contains 17,328,314 records from 6,503,429 users and 747,910 unique movies. Each record includes the user ID, movie ID, movie title, user rating (on a 1-5 scale), purchasing timestamp, user’s past purchasing history (as a sequence of movie IDs), and three aspect terms summarizing key movie attributes (e.g., ``thriller'', ``exciting'', ``director'').

\item \textbf{Yelp Restaurant}: This dataset documents users' restaurant check-ins on the Yelp platform\footnote{\url{https://www.yelp.com/dataset}}. It spans 11 metropolitan areas in the United States and comprises 6,990,280 check-in records from 1,987,897 users across 150,346 restaurants. Each record includes the user ID, restaurant ID, check-in timestamp, user rating (on a 1-5 scale), user’s historic visits (as a sequence of restaurant IDs), and three aspect terms summarizing key restaurant features (e.g., ``atmosphere'', ``service'',  ``expensive'').

\item \textbf{TripAdvisor Hotel}: This dataset captures users’ hotel stays on the TripAdvisor platform \citep{li2023personalized}. Collected in 2019, it contains 343,277 hotel stay records from 9,765 users and 6,280 unique hotels. Each record includes the user ID, hotel ID, check-in timestamp, user rating, the list of hotels previously visited by the user, and three aspect terms describing key hotel attributes (e.g., ``beach'', ``price'', ``service''). 
\end{itemize}


% \begin{itemize}
% \item \textbf{Amazon Movie}, which records the consumer purchasing actions in the Movies \& TV department of Amazon \citep{ni2019justifying}\footnote{https://nijianmo.github.io/amazon/index.html}. This dataset was collected in the year of 2023, including 17,328,314 records of 650,3429 users and 747,910 unique movies in total. Each individual record in this dataset contains the information of user ID, movie ID, movie title, user rating (on the scale of 1-5), user purchasing timestamp, user past purchasing history in this department (as a sequence of movie IDs), and three aspect terms that summarize the most important properties of the movie (such as ''thriller'', ''exciting'', and ''director''). 
% \item \textbf{Yelp Restaurant}, which records the users' restaurant check-in behaviors on the Yelp platform \footnote{https://www.yelp.com/dataset}. This dataset spans across 11 metropolitan areas in the United States, and includes 6,990,280 check-in records of 150,346 restaurants from 1,987,897 users. For each individual record, we collect the information of user ID, restaurant ID, check-in timestamp, user rating (on the scale of 1-5), user historic visits (as a sequence of restaurant IDs), and three aspect terms that summarize the most important features of the restaurant (such as ''atmosphere'', ''service'', and ''expensive'').
% \item \textbf{TripAdvisor Hotel}, which records the users' hotel stays on the TripAdvisor platform \citep{li2023personalized}. This dataset was collected in the year of 2019, and it includes 343,277 hotel staying records from 9,765 unique users of 6.280 unique hotels. Each individual record collects the information of user ID, hotel ID, hotel check-in timestamp, user rating, the list of hotels that the user has stayed before, as well as three aspect terms that describe the most unique part of the hotel (such as ''beach'', ''concierge'', and ''breakfast'').
% \end{itemize}

For all three datasets, the recommender system aims to predict the outcome for a given candidate product using the inputs of user ID, candidate item ID, and the user's past purchasing history (specifically, the last five items purchased).\endnote{For the aspect-based recommendation baselines introduced below, the aspect terms are also included as inputs to the model.} The outcome of interest for all three datasets is the product rating (on a 1-5 scale). 

To evaluate the efficacy and robustness of our framework, we examine two prediction settings: (1) a regression task, where the goal is to predict the exact rating value, and (2) a classification task, where the objective is to categorize ratings as either high (4 or 5) or low (1, 2, or 3). In line with established practices in the recommender system literature, we use root mean squared error (RMSE) and mean absolute error (MAE) as evaluation metrics for the regression task. For the classification task, we use the area under the ROC curve (AUC) \citep{hanley1982meaning}. These three datasets represent three distinct business applications with significantly different statistical distributions, making them excellent testbeds for evaluating the generalizability and flexibility of LR-Recsys.

\subsection{Baseline Models}
We compare LR-Recsys against state-of-the-art black-box recommender systems, LLM-based recommender systems, and a wide range of explainable recommender systems. Specifically, we identify the following three groups of 14 state-of-the-art baselines from recent marketing and computer science literature:

% \begin{itemize}
\begin{enumerate}
    \item \textbf{Aspect-Based Recommender Systems}: These models utilize ground-truth aspect terms as additional information to facilitate preference reasoning and generate recommendations.
    \begin{itemize}
        \item \textbf{A3NCF} \citep{cheng20183ncf} constructs a topic model to extract user preferences and item characteristics from reviews, and capture user attention on specific item aspects via an attention network.
        \item \textbf{SULM} \citep{bauman2017aspect} predicts the sentiment of a user about an item’s aspects, identifies the most valuable aspects of their potential experience, and recommends items based on these aspects.
        \item \textbf{AARM} \citep{guan2019attentive} models interactions between similar aspects to enrich aspect connections between users and products, using an attention network to focus on aspect-level importance.
        \item \textbf{MMALFM} \citep{cheng2019mmalfm} applies a multi-modal aspect-aware topic model to estimate aspect importance and predict overall ratings as a weighted linear combination of aspect ratings.
        \item \textbf{ANR} \citep{chin2018anr} learns aspect-based user and item representations through an attention mechanism and models multi-faceted recommendations using a neural co-attention framework.
        \item \textbf{MTER} \citep{le2021explainable} generates aspect-level comparisons between target and reference items, producing recommendations based on these comparative explanations.
    \end{itemize}

    \item \textbf{Sequential Recommender Systems}: These models use sequences of past user behaviors to predict the next likely purchase, leveraging various neural network architectures.
    \begin{itemize}
        \item \textbf{SASRec} \citep{kang2018self} utilizes self-attention to capture long-term semantics in user actions and identify relevant items in a user’s history.
        \item \textbf{DIN} \citep{zhou2018deep} adopts a local activation unit to adaptively learn user interest representations from historical behaviors and predict preferences for candidate items.
        \item \textbf{BERT4Rec} \citep{sun2019bert4rec} adopts bidirectional self-attention and the Cloze objective to model user behavior sequences and avoid information leakage, enhancing recommendation efficiency.
        \item \textbf{UniSRec} \citep{hou2022towards} uses contrastive pre-training to learn universal sequence representations of user preferences, improving recommendation accuracy.
    \end{itemize}

    \item \textbf{Interpretable Recommender Systems}: These models focus on generating high-quality recommendations accompanied by intuitive explanations.
    \begin{itemize}
        \item \textbf{AMCF} \citep{pan2021explainable} maps uninterpretable general features to interpretable aspect features, optimizing for both recommendation accuracy and explanation clarity through dual-loss minimization.
        \item \textbf{PETER} \citep{li2021personalized} predicts words in target explanations using IDs, endowing them with linguistic meaning to generate personalized recommendations.
        \item \textbf{UCEPic} \citep{li2023ucepic} combines aspect planning and lexical constraints to produce personalized explanations through insertion-based generation, improving recommendation performance.
        \item \textbf{PARSRec} \citep{gholami2022parsrec} leverages common and individual behavior patterns via an attention mechanism to tailor recommendations and generate explanations based on these patterns.
    \end{itemize}
% \end{itemize}
\end{enumerate}

\begin{comment}

\begin{itemize}
\item \textbf{Aspect-Based Recommender Systems}, which utilize the ground-truth aspect terms as additional information to facilitate the preference reasoning and to provide recommendations accordingly. These models include: (1) \textbf{A3NCF \citep{cheng20183ncf}}, which constructs a topic model to extract user preferences and item characteristics from user reviews. This topic model guides the representation learning of users and items, and also captures a user’s special attention on each aspect of the targeted item with an attention network. These user/item representations and aspect attention values will then be used for generating recommendations; (2) \textbf{SULM \citep{bauman2017aspect}}, which first predicts the sentiment that the user may have about the item based on what she/he might express about the aspects of the item, and then identifies the most valuable aspects of the user’s potential experience with that item. It further recommends items based on those most important aspects accordingly; (3) \textbf{AARM \citep{guan2019attentive}}, which models the interactions between synonymous and similar aspects to enrich the aspect connections between user and product. It also contains a neural attention network to capture a user’s attention toward aspects when examining different products, and produces recommendations accordingly; (4) \textbf{MMALFM \citep{cheng2019mmalfm}}, which applies a multi-modal aspect-aware topic model to model users’ preferences and items’ features from different aspects, and also estimate the aspect importance of a user toward an item. The overall rating is then predicted via a linear combination of the aspect ratings, which are weighted by the importance of the corresponding aspect; (5) \textbf{ANR \citep{chin2018anr}}, which performs aspect-based representation learning for both users and items via an attention-based component. It also models the multi-faceted process for recommendations by estimating the aspect-level user and item importance based on the neural co-attention mechanism; and (6) \textbf{MTER \citep{le2021explainable}}, which formulates comparative explanations involving aspect-level comparisons between the target item and the reference items, and produces recommendations accordingly.
\item \textbf{Sequential Recommender System}, which utilizes the sequence of user past behaviors to predict the next product that the consumer is likely to purchase using various types of neural network architectures. These models include: (1) \textbf{SASRec \citep{kang2018self}}, which proposes a self-attention-based sequential model to capture long-term semantics in recommendations, and to identify which items are ''relevant'' from a user’s action history; (2) \textbf{DIN \citep{zhou2018deep}}, which designs a local activation unit to adaptively learn the user interest representation from historical behaviors, and then infers the user preference on the candidate item accordingly; (3) \textbf{BERT4Rec \citep{sun2019bert4rec}}, which employs the deep bidirectional self-attention mechanism to model user behavior sequences, and also adopts the Cloze objective to predict the random masked items in the sequence to avoid the information leakage, resulting in more efficient recommendation performance; (4) \textbf{UniSRec \citep{hou2022towards}}, which utilizes the contrastive pre-training technique to learn universal sequence representations that represent the user preferences, and then provide subsequent recommendations accordingly.
\item \textbf{Interpretable Recommender System}, which attempts to develop models that generate not only high-quality recommendations but also intuitive explanations for those recommendations. These models include: (1) \textbf{AMCF \citep{pan2021explainable}}, which presents a novel feature mapping approach that maps the uninterpretable general features onto the interpretable aspect features, achieving both satisfactory accuracy and explainability in the recommendations by simultaneous minimization of rating prediction loss and interpretation loss; (2) \textbf{PETER \citep{li2021personalized}}, which designs a simple and effective learning objective that utilizes the IDs to predict the words in the target explanation, endowing the IDs with linguistic meanings and producing personalized recommendations; (3) \textbf{UCEPic \citep{li2023ucepic}}, which generates high-quality personalized explanations for recommendation results by unifying aspect planning and lexical constraints in an insertion-based generation manner, and these explanations can be subsequently used for improving the recommendation performance; and (4) \textbf{PARSRec \citep{gholami2022parsrec}}, which relies on common behavior patterns as well as individual behaviors to tailor the recommendation strategy for each person through the attention mechanism, and produce the explanations based on these behavioral patterns.
\end{itemize}
\end{comment}

We split each dataset into training and test sets using an 80-20 ratio at the user-temporal level. To ensure a fair comparison, we adopted Grid Search \citep{bergstra2011algorithms} and allocated equal computational resources—in terms of training time and memory usage—to optimize the hyperparameters for both our proposed approach and all baseline models. Detailed hyperparameter settings are provided in Appendix \ref{appen:hyper_param}. We independently ran LR-Recsys and each baseline model ten times and reported the average performance metrics along with their standard deviations.


\subsection{Main Results}
\label{sec:main_results}
Table \ref{main_results} presents the main results. LR-Recsys consistently outperforms all three groups of 14 baseline models across two recommendation tasks and three datasets. Specifically, LR-Recsys achieves an improvement of 5-20\% in RMSE, 15-30\% in MAE, and 2.9-3.7\% in AUC compared to the best-performing baselines. These results demonstrate the efficacy of LR-Recsys and the value of LLM-based contrastive explanations in improving the recommendation performance. 

% Yelp: 
% RMSE 11.3
% MAE 18.6
% AUC 3

% Amazon: 
% RMSE 20
% MAE 33 
% AUC 3.7

% It is also important to note, however, that these performance improvements that we have achieved in the experiments, ranging from 3\% to 15\% in terms of the RMSE, MAE, and AUC metrics, are indeed economically significant according to the discussions in the literature \citep{gunawardana2022evaluating}. In fact, according to a series of recent online experiments conducted at major e-commerce, including Google \citep{zhang2023empowering}, Amazon \citep{chen2024shopping}, Alibaba \citep{zhou2018deep}, and LinkedIn \citep{wang2024limaml}, performance improvements on the accuracy-based metrics (such as RMSE, MAE, and AUC that we use in our paper) in the offline experiments will typically lead to significant business performance increases in the online experiments as well. For example, \citet{li2024variety} proposed a novel recommender system design that manages to achieve a significant improvement of 3\% on average in terms of the AUC and Hit Rate@10 metrics on the three offline public datasets (Yelp, MovieLens, and Alibaba). When deployed at the production system of a major video streaming platform, the authors observe a similar level of 3\% improvement over Click-Through Rate and Video View metrics, which led to an additional \$30 million in revenue for the company. Additionally, according to the industrial practices at Netflix \citep{gomez2015netflix}, even a tiny 0.1\% improvement on the business performance metric would lead to significant economic and business values for the company, while in our experiments, our proposed model consistently achieves performance improvements around or above 3\% in terms of the AUC metric, which is clearly beneficial to the business performance \citep{gunawardana2022evaluating}.

It is important to emphasize that the performance improvements achieved in our experiments are economically significant \citep{gunawardana2022evaluating}. Recent online experiments conducted by major e-commerce platforms, including Google \citep{zhang2023empowering}, Amazon \citep{chen2024shopping}, Alibaba \citep{zhou2018deep}, and LinkedIn \citep{wang2024limaml}, consistently show that improvements in accuracy-based metrics (such as RMSE, MAE, and AUC) during offline testing often translate into substantial business performance gains when deployed in production systems. For example, \citet{li2024variety} proposed a novel recommender system that achieved a 3\% improvement in AUC and Hit Rate@10 on public datasets (Yelp, MovieLens, and Alibaba). When deployed on a major video streaming platform, this improvement translated to a 3\% increase in Click-Through Rate and Video Views, resulting in an additional \$30 million in annual revenue. Netflix highlight that even a 0.1\% improvement in business performance metrics can deliver significant economic value \citep{gomez2015netflix}. Notably, our proposed model consistently achieves performance gains of approximately 3\% or more, validating its efficacy to drive meaningful business impact \citep{gunawardana2022evaluating}.


% Finally, to give the audience a better understanding of how our proposed model works, we present the following case example from our Yelp dataset, where we aim to suggest a premium Thai restaurant for a particular consumer. However, after weighing the positive and negative reasons generated by the contrastive-explanation generator in our LR-Recsys, it seems that an alternative Japanese restaurant would be a better fit for the consumer. Therefore, our proposed recommendation model only predicts a 1.14 rating for this recommended Thai restaurant, which matches the ground-truth rating of 1 and significantly outperforms the prediction of 1.89 by the best-performing baseline model PETER \citep{li2021personalized} in our experiments.

To illustrate how our proposed model operates, we present a case study from the Yelp dataset below. In this scenario, the task is to recommend a premium Thai restaurant to a specific consumer. However, after incorporating the positive and negative reasoning generated by the contrastive-explanation generator in our LR-Recsys, the model determines that an alternative Japanese restaurant would be a better fit for the consumer. Consequently, our recommendation system predicts a rating of 1.14 for the Thai restaurant, closely aligning with the ground-truth rating of 1. This prediction significantly outperforms the baseline model PETER \citep{li2021personalized}, which predicts a rating of 1.89.
\newline

\fbox{%
% \vspace{-0.1in}
    \parbox{\textwidth}{%
\textbf{Consumer Past Visiting History}: O-Ku Sushi, Zen Japanese, MGM Grand Hotel, Sen of Japan, Sushi Bong

\textbf{Restaurant Profile}: ''Siam Thai Kitchen is a Thai restaurant that offers a unique dining experience in the city. The restaurant is known for its authentic Thai cuisine and its warm and inviting atmosphere. The menu features a variety of traditional Thai dishes, as well as some modern twists on classic Thai flavors. The restaurant is perfect for couples, families, and groups of friends who are looking for a delicious and authentic Thai dining experience.''

\textbf{Generated Positive Explanation}: ''The consumer is looking for a unique and flavorful dining experience and the restaurant offers a variety of Asian cuisine.''

\textbf{Generated Negative Explanation}: ''The consumer is looking for a traditional Japanese experience and wants to escape the busy city life, while the restaurant is not a traditional Japanese experience and is located in a city''

\textbf{Positive Explanation Attention Value}: 0.23

\textbf{Negative Explanation Attention Value}: 0.87

\textbf{Predicted Consumer Rating}: 1.14

\textbf{Ground-Truth Consumer Rating}: 1

\textbf{PETER-Predicted Consumer Rating}: 1.89
    }%
}
\newline
% In the remainder of this section, we will conduct a series of additional analyses to understand where the performance improvements that we achieved in the experiments come from, as well as conduct additional ablation studies and robustness checks to demonstrate the flexibility and generalizability of LR-Recsys.

% In the remainder of this section, we conduct a series of additional analyses to explore the sources of the performance gains observed in our experiments. Additionally, we perform ablation studies and robustness checks to further demonstrate the flexibility and generalizability of LR-Recsys.

\begin{table}
\centering
\footnotesize
   \setlength\extrarowheight{2pt}
\resizebox{0.95\textwidth}{!}{
\begin{tabular}{|c|ccc|ccc|ccc|} \hline
 & \multicolumn{3}{c|}{TripAdvisor} & \multicolumn{3}{c|}{Yelp} & \multicolumn{3}{c|}{Amazon Movie} \\ \hline
 & RMSE $\downarrow$ & MAE $\downarrow$ & AUC $\uparrow$ & RMSE & MAE & AUC $\uparrow$ & RMSE $\downarrow$ & MAE $\downarrow$ & AUC $\uparrow$ \\ \hline
\textbf{LR-Recsys (Ours)} & \textbf{0.1889} & \textbf{0.1444} & \textbf{0.7289} & \textbf{0.2149} & \textbf{0.1685} & \textbf{0.7229} & \textbf{0.1673} & \textbf{0.1180} & \textbf{0.7500} \\
 & (0.0010) & (0.0008) & (0.0018) & (0.0010) & (0.0009) & (0.0017) & (0.0010) & (0.0009) & (0.0018) \\
\textbf{\% Improved} & +5.36\%*** & +15.11\%*** & +2.88\%*** & +11.31\%*** & +18.64\%*** & +3.01\%*** & +20.30\%*** & +33.33\%*** & +3.65\%*** \\ \hline
A3NCF & 0.2103 & 0.1811 & 0.6879 & 0.2607 & 0.2181 & 0.6785 & 0.2241 & 0.1903 & 0.6971 \\
 & (0.0019) & (0.0013) & (0.0027) & (0.0023) & (0.0016) & (0.0029) & (0.0029) & (0.0018) & (0.0032) \\
SULM & 0.2191 & 0.1872 & 0.6736 & 0.2823 & 0.2258 & 0.6614 & 0.2477 & 0.1980 & 0.6855 \\
 & (0.0021) & (0.0013) & (0.0027) & (0.0019) & (0.0015) & (0.0029) & (0.0027) & (0.0019) & (0.0027)\\
AARM & 0.2083 & 0.1803 & 0.6901 & 0.2582 & 0.2162 & 0.6801 & 0.2162 & 0.1845 & 0.7032 \\
 & (0.0019) & (0.0014) & (0.0030) & (0.0021) & (0.0015) & (0.0029) & (0.0027) & (0.0018) & (0.0029) \\
MMALFM & 0.2117 & 0.1820 & 0.6894 & 0.2591 & 0.2167 & 0.6801 & 0.2301 & 0.1931 & 0.6931 \\
 & (0.0019) & (0.0014) & (0.0029) & (0.0020) & (0.0016) & (0.0030) & (0.0028) & (0.0020) & (0.0036) \\
ANR & 0.2083 & 0.1804 & 0.6905 & 0.2575 & 0.2145 & 0.6817 & 0.2275 & 0.1915 & 0.6960 \\
 & (0.0017) & (0.0014) & (0.0027) & (0.0021) & (0.0017) & (0.0031) & (0.0026) & (0.0018) & (0.0026)\\ 
MTER & 0.2099 & 0.1825 & 0.6889 & 0.2614 & 0.2169 & 0.6809 & 0.2283 & 0.1906 & 0.6967 \\
 & (0.0019) & (0.0014) & (0.0029) & (0.0021) & (0.0016) & (0.0031) & (0.0026) & (0.0017) & (0.0026) \\ \hline
SASRec & 0.2089 & 0.1731 & 0.7005 & 0.2491 & 0.2135 & 0.6897 & 0.2176 & 0.1869 & 0.7025 \\
 & (0.0007) & (0.0006) & (0.0015) & (0.0011) & (0.0009) & (0.0016) & (0.0013) & (0.0008) & (0.0013) \\
DIN & 0.2022 & 0.1709 & 0.7076 & 0.2479 & 0.2116 & 0.6917 & 0.2155 & 0.1853 & 0.7046 \\
 & (0.0009) & (0.0007) & (0.0017) & (0.0009) & (0.0008) & (0.0015) & (0.0009) & (0.0007) & (0.0013) \\
BERT4Rec & 0.2003 & \underline{0.1701} & \underline{0.7085} & 0.2460 & 0.2101 & 0.6928 & 0.2126 & 0.1832 & 0.7088 \\
 & (0.0009) & (0.0006) & (0.0017) & (0.0009) & (0.0008) & (0.0015) & (0.0009) & (0.0008) & (0.0015) \\
UniSRec & 0.2026 & 0.1720 & 0.7066 & 0.2448 & 0.2093 & 0.6956 & 0.2103 & 0.1810 & 0.7133 \\
 & (0.0015) & (0.0010) & (0.0023) & (0.0013) & (0.0011) & (0.0020) & (0.0011) & (0.0009) & (0.0017) \\ \hline
AMCF & 0.2088 & 0.1755 & 0.6989 & 0.2501 & 0.2123 & 0.6928 & 0.2376 & 0.1863 & 0.7035 \\
 & (0.0019) & (0.0013) & (0.0027) & (0.0016) & (0.0013) & (0.0023) & (0.0013) & (0.0010) & (0.0019) \\
PETER & \underline{0.1996} & 0.1715 & 0.7078 & \underline{0.2423} & \underline{0.2071} & 0.7003 & \underline{0.2099} & \underline{0.1770} & \underline{0.7226} \\
 & (0.0019) & (0.0013) & (0.0027) & (0.0015) & (0.0013) & (0.0022) & (0.0013) & (0.0010) & (0.0019) \\ 
UCEPic & 0.2035 & 0.1723 & 0.7066 & 0.2477 & 0.2099 & \underline{0.7018} & 0.2228 & 0.1801 & 0.7080 \\
 & (0.0015) & (0.0011) & (0.0023) & (0.0015) & (0.0012) & (0.0023) & (0.0011) & (0.0009) & (0.0017) \\ 
PARSRec & 0.2008 & 0.1703 & 0.7080 & 0.2471 & 0.2106 & 0.6923 & 0.2133 & 0.1837 & 0.7069 \\
 & (0.0009) & (0.0007) & (0.0017) & (0.0009) & (0.0008) & (0.0015) & (0.0009) & (0.0007) & (0.0013) \\ \hline 
\end{tabular}
}
\caption{Recommendation performance on three datasets. ``\%Improved'' represents the performance gains of LR-Recsys (ours) compared to the best-performing baseline (underlined). Metrics with $\downarrow$ indicate that lower values are better (e.g., RMSE, MAE), while metrics with $\uparrow$ indicate that higher values are better (e.g., AUC). ***p$<$0.01;**p$<$0.05.}
\label{main_results}
\end{table}



\subsection{Understanding the Improvements} 
In this section, we present additional analyses to decompose and better understand the significant performance gains observed with LR-Recsys in the previous section.

\subsubsection{Improved learning efficiency.} 

Based on the theoretical insights discussed in Section \ref{sec:theory}, incorporating explanations into the recommendation process is expected to significantly improve learning efficiency. This implies that our proposed LR-Recsys should require \emph{less} training data to achieve recommendation performance comparable to the baselines. To demonstrate this, we randomly sample subsets of the three datasets, keeping 12\%, 25\%, and 50\% of the original training data, and train LR-Recsys on these subsets while keeping the same test set for evaluation. The results, presented in Table \ref{efficiency}, show that our model achieves performance equivalent to the best-performing baseline, PETER \citep{li2021personalized}, using as little as 25\% of the training data. These findings validate the improved learning efficiency of LR-Recsys, matching the theoretical insights in Section \ref{sec:theory}.

\begin{table}
\centering
\footnotesize
   \setlength\extrarowheight{3pt}
\resizebox{0.95\textwidth}{!}{
\begin{tabular}{|c|ccc|ccc|ccc|} \hline
 & \multicolumn{3}{c|}{TripAdvisor} & \multicolumn{3}{c|}{Yelp} & \multicolumn{3}{c|}{Amazon Movie} \\ \hline
 & RMSE $\downarrow$ & MAE $\downarrow$  & AUC $\uparrow$ & RMSE $\downarrow$ & MAE $\downarrow$ & AUC $\uparrow$ & RMSE $\downarrow$ & MAE $\downarrow$ & AUC $\uparrow$ \\ \hline
 LR-Recsys & 0.1889 & 0.1444 & 0.7289 & 0.2149 & 0.1685 & 0.7229 & 0.1673 & 0.1180 & 0.7500 \\
 100\% Training Data & (0.0010) & (0.0008) & (0.0018) & (0.0010) & (0.0009) & (0.0017) & (0.0010) & (0.0009) & (0.0018) \\ \hline
LR-Recsys & 0.1938 & 0.1503 & 0.7144 & 0.2188 & 0.1774 & 0.7133 & 0.1791 & 0.1308 & 0.7276 \\
 50\% Training Data & (0.0017) & (0.0013) & (0.0029) & (0.0018) & (0.0017) & (0.0031) & (0.0021) & (0.0017) & (0.0033) \\ \hline
LR-Recsys & 0.2017 & 0.1679 & 0.7020 & 0.2356 & 0.1958 & 0.7004 & 0.1997 & 0.1703 & 0.7173 \\
25\% Training Data  & (0.0026) & (0.0021) & (0.0041) & (0.0027) & (0.0027) & (0.0047) & (0.0039) & (0.0028) & (0.0049) \\ \hline
LR-Recsys & 0.2098 & 0.1796 & 0.6912 & 0.2557 & 0.2140 & 0.6822 & 0.2175 & 0.1866 & 0.7015 \\
12\% Training Data & (0.0036) & (0.0030) & (0.0054) & (0.0039) & (0.0038) & (0.0068) & (0.0066) & (0.0044) & (0.0063) \\ \hline
PETER & 0.1996 & 0.1715 & 0.7078 & 0.2423 & 0.2071 & 0.7003 & 0.2099 & 0.1770 & 0.7226 \\
100\% Training Data & (0.0019) & (0.0013) & (0.0027) & (0.0015) & (0.0013) & (0.0022) & (0.0013) & (0.0010) & (0.0019) \\ \hline
\end{tabular}
}
\caption{Recommendation performance across three datasets using varying percentages of training data for LR-Recsys.}
\label{efficiency}
\end{table}

\subsubsection{The gain is from LLM's reasoning capability.}  
\label{res_understanding_reasoning}
The theoretical insights in Section \ref{sec:theory} highlight that the advantage of LR-Recsys lies in leveraging LLMs' strong \emph{reasoning} capabilities to identify the important variables. Therefore, LLMs with better reasoning capabilities are expected to lead to better recommendation performance. To validate this, we conduct additional experiments within the LR-Recsys framework, using different LLMs with varying reasoning capabilities. As shown in Table \ref{llm_models}, the performance of LR-Recsys with Llama 3.1 is significantly better than LR-Recsys with Llama 3, Mixtral-8$\times$7b, Vicuna-7b-v1.5, Qwen2-7B, or GPT-2. This aligns with the reasoning capability leaderboard at \url{https://huggingface.co/spaces/allenai/ZebraLogic}, where Llama 3.1 demonstrates the highest reasoning capabilities among the tested models. Furthermore, Llama 3 and Mixtral-8$\times$7b also outperform Vicuna-7b-v1.5, Qwen2-7B, and GPT-2 in the reasoning leaderboard, which is also aligned with the results observed in LR-Recsys. These results confirm that better \emph{reasoning} capabilities in LLMs directly translate to improved performance within the LR-Recsys framework. 

Moreover, LR-Recsys significantly outperforms an alternative approach that uses LLMs directly for recommendations—without generating explicit explanations (``LLM Direct Recommendation (with Llama 3.1)'' row in Table \ref{llm_models}). This suggests that only using LLMs for recommendation without tapping into their reasoning abilities is insufficient. Additionally, we find that including LLM-generated product profile information plays only a minor role in the overall effectiveness of the model, as LR-Recsys continues to significantly outperform baseline models even when these augmented profiles are removed (``LR-Recsys w/o Profile Augmentation'' row in Table \ref{llm_models}).

Furthermore, we confirm that the observed performance improvements are not due to information leakage or pre-existing dataset knowledge. For example, the Amazon Movie dataset was collected in 2023, while GPT-2 was pre-trained on data available only up to 2019. Despite this, when GPT-2 is used within the LR-Recsys framework, our approach still outperforms other baselines.



Finally, we also tested utilizing LLMs' summarization capabilities instead of their reasoning abilities within LR-Recsys. Specifically, we replace the positive and negative explanation prompts in the contrastive-explanation generator with the following prompt: 
\begin{quote}
\emph{
``Given the profiles of the watching history of this consumer \{movie\_profile\_seq\}, can you provide a summary of the consumer preference of candidate movies?'' 
}
\end{quote} 
In other words, we leverage the LLM's summarization skills to condense the user's consumption history, and then use this summarized information as input to the DNN instead of the positive and negative explanations. As shown in the last row of Table \ref{llm_models} (``Consumption History Summarization''), the performance of using LLM for summarization is significantly worse than that of LR-Recsys using LLM for explanations. This suggests that the gain from LR-Recsys specifically comes from the \emph{reasons} (positive and negative explanations) provided by LLMs, rather than their ability to summarize consumption history.

Collectively, these findings support the conclusion that the performance gains observed with LR-Recsys are primarily driven by the LLMs' reasoning capabilities, \emph{not} their external dataset knowledge or summarization skills.







% To validate this point, we conduct additional experiments in this section, where we implement multiple versions of LLMs with different levels of reasoning capabilities to construct our proposed model, as well as using LLMs directly to produce recommendations (i.e. without explicitly asking for explanations). As we can observe from Table \ref{llm_models}, recommendation performance obtained from using Llama 3.1 is significantly better than that of Llama 3 and Mixtral-8$\times$7b, and even more so than that of Vicuna-7b-v1.5, Qwen2-7B, and GPT-2. This is consistent with the reasoning capability leaderboard shown at \url{https://huggingface.co/spaces/allenai/ZebraLogic}, where Llama 3.1 is shown to possess the best reasoning capability compared to Llama 3 and Mixtral-8$\times$7b. Furthermore, these three LLMs perform significantly better than Vicuna-7b-v1.5, Qwen2-7B, and GPT-2 in terms of reasoning capability. Therefore, our results confirm that better \emph{reasoning} capabilities in LLMs directly translate to improved performance in our LR-Recsys framework. Furthermore, our proposed model also significantly outperforms the alternative model that uses LLM directly for recommendations, since it does not utilize any benefits coming from the LLM's reasoning capability. We can also verify that the incorporation of the LLM-generated product profile information only plays a small part in terms of the effectiveness of our proposed model, since LR-Recsys still significantly outperforms the baseline models even when the LLM-augmented profile is removed (last row in Table \ref{llm_models}). Lastly, we would like to point out that the performance improvements are not a result of the information/knowledge leakage, since one of our offline datasets, Amazon Movie, was collected in 2023, while GPT-2 uses pre-training data prior to 2019. Even when we use the GPT-2 as the backbone model, we can still achieve performance improvements over prior models in the literature.

\begin{table}
\centering
\footnotesize
   \setlength\extrarowheight{3pt}
\resizebox{1.00\textwidth}{!}{
\begin{tabular}{|c|ccc|ccc|ccc|} \hline
 & \multicolumn{3}{c|}{TripAdvisor} & \multicolumn{3}{c|}{Yelp} & \multicolumn{3}{c|}{Amazon Movie} \\ \hline
 & RMSE $\downarrow$ & MAE $\downarrow$ & AUC $\uparrow$ & RMSE $\downarrow$ & MAE $\downarrow$ & AUC $\uparrow$ & RMSE $\downarrow$ & MAE $\downarrow$ & AUC $\uparrow$ \\ \hline
LR-Recsys with \textbf{Llama 3.1} (Ours) & 0.1889 & 0.1444 & 0.7289 & 0.2149 & 0.1685 & 0.7229 & 0.1673 & 0.1180 & 0.7500 \\
 & (0.0010) & (0.0008) & (0.0018) & (0.0010) & (0.0009) & (0.0017) & (0.0010) & (0.0009) & (0.0018) \\ \hline
LR-Recsys with \textbf{Llama 3}  & 0.1934 & 0.1491 & 0.7260 & 0.2166 & 0.1697 & 0.7210 & 0.1695 & 0.1203 & 0.7472 \\
 & (0.0010) & (0.0008) & (0.0018) & (0.0010) & (0.0009) & (0.0017) & (0.0010) & (0.0009) & (0.0018) \\ \hline
LR-Recsys with \textbf{Mixtral-8} $\times$7b & 0.1910 & 0.1462 & 0.7271 & 0.2163 & 0.1693 & 0.7218 & 0.1691 & 0.1199 & 0.7480 \\
 & (0.0010) & (0.0008) & (0.0018) & (0.0010) & (0.0009) & (0.0017) & (0.0010) & (0.0009) & (0.0018) \\ \hline
LR-Recsys with \textbf{Vicuna-7b-v1.5} & 0.1949 & 0.1502 & 0.7243 & 0.2175 & 0.1703 & 0.7196 & 0.1703 & 0.1210 & 0.7455 \\
 & (0.0010) & (0.0008) & (0.0018) & (0.0010) & (0.0009) & (0.0017) & (0.0010) & (0.0009) & (0.0018) \\ \hline
LR-Recsys with \textbf{Qwen2-7B}  & 0.1966 & 0.1520 & 0.7202 & 0.2193 & 0.1724 & 0.7170 & 0.1727 & 0.1235 & 0.7419 \\
 & (0.0010) & (0.0008) & (0.0018) & (0.0010) & (0.0009) & (0.0017) & (0.0010) & (0.0009) & (0.0018) \\ \hline
LR-Recsys with \textbf{GPT-2}  & 0.1940 & 0.1582 & 0.7169 & 0.2211 & 0.1801 & 0.7144 & 0.1799 & 0.1304 & 0.7288 \\
 & (0.0014) & (0.0010) & (0.0023) & (0.0015) & (0.0013) & (0.0023) & (0.0015) & (0.0013) & (0.0024) \\ \hline
LLM Direct Recommendation (with Llama 3.1)& 0.2233 & 0.1838 & 0.6735 & 0.2976 & 0.2402 & 0.6528 & 0.2680 & 0.2055 & 0.6736 \\
 & (0.0033) & (0.0020) & (0.0036) & (0.0044) & (0.0029) & (0.0046) & (0.0046) & (0.0031) & (0.0055) \\ \hline
LR-Recsys w/o Profile Augmentation & 0.1973 & 0.1520 & 0.7211 & 0.2193 & 0.1719 & 0.7173 & 0.1744 & 0.1239 & 0.7430 \\
 & (0.0013) & (0.0011) & (0.0021) & (0.0012) & (0.0012) & (0.0021) & (0.0013) & (0.0012) & (0.0021) \\ \hline
Consumption History Summarization & 0.1971 & 0.1700 & 0.7055 & 0.2409 & 0.2051 & 0.6990 & 0.2077 & 0.1751 & 0.7236 \\
 & (0.0011) & (0.0008) & (0.0018) & (0.0011) & (0.0009) & (0.0017) & (0.0011) & (0.0010) & (0.0018) \\ \hline
\end{tabular}
}
\caption{Recommendation performance across three datasets using LLMs with varying levels of reasoning capability.}
\label{llm_models}
\end{table}

\subsubsection{``Harder'' examples benefit more from LR-Recsys.} 
% Harder Examples Benefit More.

By leveraging LLMs' reasoning capabilities to identify important variables, LR-Recsys should intuitively provide greater benefits for ``harder'' examples where consumers' decisions are less obvious. To test this hypothesis, we compute the prediction uncertainty for each observation in our datasets, measured as the variance—or “disagreement”—across the predictions made by our model and all baseline models from Table \ref{main_results}. Intuitively, higher prediction uncertainty indicates a more challenging, or ``harder'', prediction task.

In Fig.\ref{fig:uncertainty}, we created plots for each dataset, where the x-axis represents the normalized uncertainty level (scaled between 0 and 1 using min-max normalization \citep{patro2015normalization}), and the y-axis represents the performance improvement of our LR-Recsys over the best-performing baseline (measured by RMSE). As shown by the regression lines in Figures \ref{fig:uncertainty:amazon}, \ref{fig:uncertainty:tripadvisor}, and \ref{fig:uncertainty:yelp}, there is a statistically significant \emph{positive correlation} between uncertainty and performance improvements. Specifically, the performance gains from incorporating explanations are consistently larger for high-uncertainty examples across all three datasets, validating the insight that our LR-Recsys is more beneficial for examples that are ``harder'' or more uncertain.

This observation aligns with the theoretical insights in Section \ref{sec:theory}. For ``harder'' examples, the model is likely uncertain about which input variables to rely on for making predictions, leading to higher prediction uncertainty. In such cases, the knowledge provided by LLMs about the important variables becomes particularly valuable, allowing the model to focus on the most relevant features. Consequently, the performance gains of our LR-Recsys are larger for these more challenging cases.

\begin{figure}[hbtp!]
% \vspace{-0.2in}
    \begin{subfigure}[b]{0.45\textwidth}
        \centering
        \includegraphics[width=\textwidth]{figures/uncertainty_line.jpg}
        \caption{Amazon Dataset.}
        \label{fig:uncertainty:amazon}
    \end{subfigure}
    \hspace{0.5mm}
     \centering
    \begin{subfigure}[b]{0.45\textwidth}
        \centering
        \includegraphics[width=\textwidth]{figures/hotel_uncertainty_line.jpg}
        \caption{TripAdvisor Dataset.}
        \label{fig:uncertainty:tripadvisor}
    \end{subfigure}
    \begin{subfigure}[b]{0.45\textwidth}
        \centering
        \includegraphics[width=\textwidth]{figures/restaurant_uncertainty_line.jpg}
        \caption{Yelp Dataset.}
        \label{fig:uncertainty:yelp}
    \end{subfigure}
   \caption{Performance improvement of LR-Recsys against (normalized) pediction uncertainty.}
  \label{fig:uncertainty}
% \vspace{-0.2in}
\end{figure}


% \begin{figure}[hbtp!]
% \centering
% \includegraphics[width=0.5\linewidth]{figures/uncertainty_line.jpg}
% \caption{Uncertainty Analysis for the Amazon Dataset}\label{fig:uncertainty:amazon}
% \vspace{-0.2in}
% \end{figure}

% \begin{figure}[hbtp!]
% \centering
% \includegraphics[width=0.5\linewidth]{figures/hotel_uncertainty_line.jpg}
% \caption{Uncertainty Analysis for the TripAdvisor Dataset}\label{fig:uncertainty:tripadvisor}
% \vspace{-0.2in}
% \end{figure}

% \begin{figure}[hbtp!]
% \centering
% \includegraphics[width=0.5\linewidth]{figures/restaurant_uncertainty_line.jpg}
% \caption{Uncertainty Analysis for the Yelp Dataset}\label{fig:uncertainty:yelp}
% \vspace{-0.2in}
% \end{figure}

\subsection{Understanding the Role of Contrastive Explanations} 
\label{sec:role_pos_neg}
% We conducted additional analysis to understand how both the positive explanations and the negative explanations contribute to the significant performance improvements from LR-Recsys. To begin with, we implement four variants of our proposed model and test their performance respectively: (1) \textbf{Aspect Terms Only}, where we use the LLM to generate only a few aspect terms that represent the most important properties of the candidate item that the consumer might consider, rather than ask for explicit explanations by leveraging LLM's reasoning capability; (2) \textbf{Positive Explanations Only}, where we use the LLM to generate only positive explanations (without negative explanations); (3) \textbf{General Explanations Only}, where we let the LLM to infer whether the use may or may not like the product, and generate the associated explanations accordingly; and (4) \textbf{Purchashing History Summarization}, where we use the LLM to summarize the purchasing history of the user and use the summarized information for producing recommendations, rather than generating the explanations for recommendations. The results are summarized in Table \ref{generation}. As we can observe from Table \ref{generation}, all of these variants do not perform as well as our proposed LR-Recsys, demonstrating the significant role of both the positive and the negative explanations.

\subsubsection{The need for both positive and negative explanations.} 
\label{sec:dual_explanation}

We conducted analyses to understand how explanations, and in particular both positive and negative explanations, contribute to the significant performance improvements of LR-Recsys. We compared three variants of the contrastive-explanation generator in LR-Recsys:
(1) \textbf{Aspect Terms Only}: The LLM only generates a few aspect terms representing the most important properties of the candidate item that the consumer might consider, without leveraging explicit reasoning-based explanations;
(2) \textbf{Positive Explanations Only}: The LLM generates only positive explanations;
(3) \textbf{General Explanations Only}: The LLM infers whether the user may or may not like the product and generates corresponding general explanations, without distinguishing between positive and negative reasoning;
% (4) \textbf{Purchasing History Summarization}: The LLM summarizes the user's purchasing history and uses this summarized information instead of the generated explanations as input to the DNN.

The detailed prompts used for each variant are listed in Appendix \ref{appen:variants_prompt}. The results, summarized in Table \ref{generation}, show that none of these variants match the performance of our proposed LR-Recsys with the contrastive-explanation generator. This confirms the value of incorporating both positive and negative explanations to improve predictive performance.


\begin{table}
\centering
\footnotesize
   \setlength\extrarowheight{3pt}
\resizebox{0.95\textwidth}{!}{
\begin{tabular}{|c|ccc|ccc|ccc|} \hline
 & \multicolumn{3}{c|}{TripAdvisor} & \multicolumn{3}{c|}{Yelp} & \multicolumn{3}{c|}{Amazon Movie} \\ \hline
 & RMSE $\downarrow$ & MAE $\downarrow$ & AUC $\uparrow$ & RMSE $\downarrow$ & MAE $\downarrow$ & AUC $\uparrow$ & RMSE $\downarrow$ & MAE $\downarrow$ & AUC $\uparrow$ \\ \hline
LR-Recsys (Ours) & \textbf{0.1889} & \textbf{0.1444} & \textbf{0.7289} & \textbf{0.2149} & \textbf{0.1685} & \textbf{0.7229} & \textbf{0.1673} & \textbf{0.1180} & \textbf{0.7500} \\
 & (0.0010) & (0.0008) & (0.0018) & (0.0010) & (0.0009) & (0.0017) & (0.0010) & (0.0009) & (0.0018) \\ \hline
Aspect Terms Only & 0.1975 & 0.1709 & 0.7071 & 0.2413 & 0.2053 & 0.6991 & 0.2083 & 0.1757 & 0.7243 \\
 & (0.0010) & (0.0008) & (0.0018) & (0.0010) & (0.0009) & (0.0017) & (0.0011) & (0.0010) & (0.0018) \\ \hline
Positive Explanations Only & 0.1928 & 0.1480 & 0.7258 & 0.2179 & 0.1708 & 0.7168 & 0.1703 & 0.1209 & 0.7456 \\
 & (0.0010) & (0.0008) & (0.0018) & (0.0010) & (0.0009) & (0.0018) & (0.0010) & (0.0009) & (0.0018) \\ \hline
General Explanations Only & 0.1961 & 0.1633 & 0.7006 & 0.2499 & 0.2279 & 0.6710 & 0.2136 & 0.1797 & 0.7076 \\
 & (0.0014) & (0.0010) & (0.0023) & (0.0015) & (0.0016) & (0.0023) & (0.0015) & (0.0013) & (0.0027) \\ \hline
% Purchasing History Summarization & 0.1971 & 0.1700 & 0.7055 & 0.2409 & 0.2051 & 0.6990 & 0.2077 & 0.1751 & 0.7236 \\
%  & (0.0011) & (0.0008) & (0.0018) & (0.0011) & (0.0009) & (0.0017) & (0.0011) & (0.0010) & (0.0018) \\ \hline
\end{tabular}
}
\caption{Recommendation performance across three datasets using LLMs for alternative generation tasks.}
\label{generation}
\end{table}


\subsubsection{Attention weights on positive and negative explanations.} 
\label{sec:attention_analysis}

% For the classification task, the goal of the recommender system is to predict whether a consumer will give a high rating to a particular product. In Sections \ref{sec:main_results} and \ref{sec:dual_explanation}, we demonstrated that both positive and negative explanations contribute to the improved prediction accuracy of this task. 

We further analyze the contribution of each type of explanation to different predictions. One hypothesis is that high ratings depend more on positive explanations (i.e., reasons the consumer likes the product), while low ratings depend more on negative explanations (i.e., reasons the consumer does not like the product).

To test this, we conduct an \emph{attention value analysis} to quantify the contributions of positive and negative explanations to the classification task. The attention value for the positive explanation ${\bar{\alpha}}_{pos}$ is computed as the average of all the relevant pairwise attention values in the attention layer of the DNN component. Following the notation in Section \ref{sec:framework_dnn_loss}, we define:
\begin{equation}
\label{eq:alpha_pos_neg}
\begin{aligned}
{\bar{\alpha}}_{pos} &= \frac{1}{|I|}{\sum_{i\in I}\alpha_{i, pos}}, \;\;\;\;\;
{\bar{\alpha}}_{neg} &= \frac{1}{|I|}{\sum_{i\in I}\alpha_{i, neg}},
\end{aligned}
\end{equation}
where $I = [\text{pos}, \text{neg}, c, p, \text{seq}, \text{context}]$ represents the index set corresponding to each element in the input $X_{\text{input}}$. In other words, ${\bar{\alpha}}_{pos}$ captures the average ``attention'' that the model puts on the positive explanation embedding when generating the final prediction, and ${\bar{\alpha}}_{neg}$ captures the average ``attention'' put on the negative explanation embedding. Therefore, ${\bar{\alpha}}_{pos}$ and ${\bar{\alpha}}_{neg}$ estimates the relative importance of the positive and negative explanations in producing the final recommendation results. 

% We visualize the distribution of attention values ${\bar{\alpha}}_{pos}$ and ${\bar{\alpha}}_{neg}$ across three datasets in Fig.\ref{fig:attention}. The results show that when a product receives a high rating, LR-Recsys assigns \emph{more} attention to positive explanations than negative ones, as indicated by the distribution of attention weights for positive explanations (blue bars) being skewed further to the \emph{right} compared to negative explanations (orange bars) (Fig.\ref{fig:positive:yelp}, \ref{fig:positive:amazon}, and \ref{fig:positive:tripadvisor}). Conversely, for products receiving a low rating, LR-Recsys assigns more attention to negative explanations, with the distribution of attention weights for positive explanations shifting further to the \emph{left} compared to negative explanations (Fig. \ref{fig:negative:yelp}, \ref{fig:negative:amazon}, and \ref{fig:negative:tripadvisor}). In Appendix \ref{appen:attn_pie_charts}, we also plot the distribution of the attenion values on the other input components (i.e. consumer, item and context embeddings) and find that these explanations together account for a significant (30-40\%) of total model attention values.

% 
We visualize the distribution of attention values on positive explanations (${\bar{\alpha}}_{pos}$) and negative explanations (${\bar{\alpha}}_{neg}$) for the Yelp Dataset in Fig.\ref{fig:attention}, while the results for all three datasets are provided in Appendix \ref{appen:attn_pos_neg}. The results across all three datasets consistently show that when a product receives a high rating, LR-Recsys assigns \emph{more} attention to positive explanations than negative ones, as indicated by the distribution of attention weights for positive explanations (blue bars) being skewed further to the \emph{right} compared to negative explanations (orange bars) (Fig.\ref{fig:positive:yelp}). Conversely, for products receiving a low rating, LR-Recsys assigns more attention to negative explanations, with the distribution of attention weights for positive explanations shifting further to the \emph{left} compared to negative explanations (Fig. \ref{fig:negative:yelp},). In Appendix \ref{appen:attn_pie_charts}, we also plot the distribution of attention values on other input components (i.e., consumer, item, and context embeddings) and find that these explanations together account for a significant (30-40\%) of total attention values. 

The insights from these plots are significant. In addition to confirming the importance of both types of explanations in generating predictions, the separation of attention weight distributions between positive and negative explanations highlights how LR-Recsys effectively leverages contrastive explanations for the prediction task. Across all three datasets, we observe that positive predictions rely more on positive explanations, while negative predictions depend more on negative explanations. This behavior is intuitive: When a product is predicted to receive a high rating, the framework selectively focuses on positive explanations (reasons the consumer may like the product); conversely, when a product is predicted to receive a low rating, the framework is focused on negative explanations (reasons the consumer may \emph{not} like the product). By providing both positive and negative explanations through the contrastive-explanation generator, LR-Recsys is able to intelligently decide which type of explanation to rely on, in order to generate the most accurate predictions. This explains why both types of explanations are critical for the prediction, and why LR-Recsys outperforms other LLM-based recommendation system variants as demonstrated above.

% when producing positive recommendations, the recommender system will pay more attention to positive explanations, as evidenced by their higher attention values compared to the negative explanations; similarly, when producing negative recommendations, the recommender system will pay more attention to negative explanations. Therefore, both explanations are indeed crucial components for determining the final decisions of the recommendation model.

\begin{figure}[hbtp!]
% \vspace{-0.2in}
    \begin{subfigure}[b]{0.45\textwidth}
        \centering
        \includegraphics[width=\textwidth]{figures/restaurant_positive_record.jpg}
        \caption{Positive examples (Yelp Dataset).}
        \label{fig:positive:yelp}
    \end{subfigure}
    \hspace{0.5mm}
     \centering
    \begin{subfigure}[b]{0.45\textwidth}
        \centering
        \includegraphics[width=\textwidth]{figures/restaurant_negative_record.jpg}
        \caption{Negative examples (Yelp Dataset).}
        \label{fig:negative:yelp}
    \end{subfigure}
    
    % \begin{subfigure}[b]{0.45\textwidth}
    %     \centering
    %     \includegraphics[width=\textwidth]{figures/movie_positive_record.jpg}
    %     \caption{Positive examples (Amazon Movie Dataset).}
    %     \label{fig:positive:amazon}
    % \end{subfigure}
    % \hspace{0.5mm}
    %  \centering
    % \begin{subfigure}[b]{0.45\textwidth}
    %     \centering
    %     \includegraphics[width=\textwidth]{figures/movie_negative_record.jpg}
    %     \caption{Negative examples (Amazon Movie Dataset).}
    %     \label{fig:negative:amazon}
    % \end{subfigure}
    
    % \begin{subfigure}[b]{0.45\textwidth}
    %     \centering
    %     \includegraphics[width=\textwidth]{figures/hotel_positive_record.jpg}
    %     \caption{Positive examples (TripAdvisor Dataset).}
    %     \label{fig:positive:tripadvisor}
    % \end{subfigure}
    % \hspace{0.5mm}
    %  \centering
    % \begin{subfigure}[b]{0.45\textwidth}
    %     \centering
    %     \includegraphics[width=\textwidth]{figures/hotel_negative_record.jpg}
    %     \caption{Negative examples (TripAdvisor Dataset).}
    %     \label{fig:negative:tripadvisor}
    % \end{subfigure}
    
   \caption{(Color online) Distribution of attention values on positive and negative explanations for positive and negative examples.}
  \label{fig:attention}
% \vspace{-0.2in}
\end{figure}

% \begin{figure}[hbtp!]
% \centering
% \includegraphics[width=0.5\linewidth]{figures/movie_positive_record.jpg}
% \caption{Attention Value Analysis Positive Explanations in the Amazon Dataset}\label{fig:positive:amazon}
% \vspace{-0.2in}
% \end{figure}

% \begin{figure}[hbtp!]
% \centering
% \includegraphics[width=0.5\linewidth]{figures/movie_negative_record.jpg}
% \caption{Attention Value Analysis Negative Explanations in the Amazon Dataset}\label{fig:negative:amazon}
% \vspace{-0.2in}
% \end{figure}

% \begin{figure}[hbtp!]
% \centering
% \includegraphics[width=0.5\linewidth]{figures/restaurant_positive_record.jpg}
% \caption{Attention Value Analysis Positive Explanations in the Yelp Dataset}\label{fig:positive:yelp}
% \vspace{-0.2in}
% \end{figure}

% \begin{figure}[hbtp!]
% \centering
% \includegraphics[width=0.5\linewidth]{figures/restaurant_negative_record.jpg}
% \caption{Attention Value Analysis Negative Explanations in the Yelp Dataset}\label{fig:negative:yelp}
% \vspace{-0.2in}
% \end{figure}

% \begin{figure}[hbtp!]
% \centering
% \includegraphics[width=0.5\linewidth]{figures/hotel_positive_record.jpg}
% \caption{Attention Value Analysis Positive Explanations in the TripAdvisor Dataset}\label{fig:positive:tripadvisor}
% \vspace{-0.2in}
% \end{figure}

% \begin{figure}[hbtp!]
% \centering
% \includegraphics[width=0.5\linewidth]{figures/hotel_negative_record.jpg}
% \caption{Attention Value Analysis Negative Explanations in the TripAdvisor Dataset}\label{fig:negative:tripadvisor}
% \vspace{-0.2in}
% \end{figure}


\subsubsection{Actionable business insights from aggregated contrastive explanations.} 
\label{sec:actionable_insights}

Another benefit of LR-Recsys is its ability to provide actionable insights for business owners or content creators. For instance, we aggregate all positive and negative explanations associated with a specific restaurant, Siam Thai Kitchen, from the Yelp dataset. A summary of the top keywords (based on word frequencies) reveals that the restaurant excels at offering an authentic Thai dining experience. However, the negative explanations highlight opportunities for improvement, such as expanding the menu with healthier, more upscale options to attract a broader audience. These actionable insights, generated through contrastive explanations in LR-Recsys, are not possible with traditional black-box recommender systems.
\newline
% Another benefit of LR-Recsys is its potential to provide actionable insights for business owners or content creators. As an example, we collect all interaction records associated with a specific restaurant, Siam Thai Kitchen, in the Yelp dataset, and aggregate all of the positive and negative explanations generated in our LR-Recsys framework.

% A summary of the top keywords, based on word frequencies in the generated explanations, is listed below. We see that the restaurant is doing well in terms of offering a unique and authentic Thai dining experience. However, the negative explanations also suggest opportunities for improvement, such as improving its menu to include healthier and more upscale options to attract more consumers. Such actionable insights are not possible with traditional black-box recommender systems, but are made possible by the contrastive explanations generated through our LR-Recsys framework.



\fbox{%
    \parbox{\textwidth}{%
\textbf{Restaurant Name}: Siam Thai Kitchen

\textbf{Restaurant Profile}: ''Siam Thai Kitchen is a Thai restaurant that offers a unique dining experience in the city. The restaurant is known for its authentic Thai cuisine and its warm and inviting atmosphere. The menu features a variety of traditional Thai dishes, as well as some modern twists on classic Thai flavors. The restaurant is perfect for couples, families, and groups of friends who are looking for a delicious and authentic Thai dining experience.''

\textbf{Top Keywords in Positive Explanations}: Unique, Thai, Traditional, Atmosphere, Dining, Experience, Service, Authentic

\textbf{Top Keywords in Negative Explanations}: Healthy, Luxurious, Fine Dining, Upscale, Quick, Snack, Modern 
    }%
}
\newline


\subsection{Ablation Studies and Robustness Checks}
We conduct a series of additional ablation studies and robustness checks to further demonstrate the effectiveness and generalizability of LR-Recsys. We provide the detailed results in Appendix \ref{appen:ablation_robustness}.



\begin{comment}
\subsection{Descriptive analysis of the generated explanations} \hfill\\
\todo{use LLM to summarize}

Profile Prompt: ''Create a succinct profile for a hotel based on its name. This profile should be tailored for use in recommendation systems and must identify the types of consumers who would enjoy the hotel.''

Input: ''Hilton Newark Airport''

Output: ''Stay at the Hilton Newark Airport and experience a comfortable and convenient stay. Ideal for business and leisure travelers, this hotel is located just minutes away from the airport and offers easy access to the city's top attractions. Enjoy the hotel's amenities, including a fitness center, restaurant, and bar.''

Explanation Prompt: ''Given the profiles of restaurants visited by this consumer in the past, can you provide a reason for why this consumer watched the following recommended movie with the following profile? Answer with one sentence with the following format: The consumer visited this restaurant because''

Positive Explanation Output: The consumer visited this restaurant because the consumer is looking for a unique and flavorful dining experience and the restaurant offers a variety of Asian cuisine.

Negative Explanation Output: The consumer did not visit this restaurant because the consumer is looking for a traditional Japanese experience and wants to escape the busy city life and the restaurant is not a traditional Japanese experience and is located in a city.

\todo{Pan: any description analysis / summary statistics you can show here to give insights on (1) what are common positive explanations, and (2) what are common negative explanations? }
\end{comment}

\section{Discussion}
\label{sec:discussion}
\section{Discussion}
\label{sec:discussion}

    \begin{table*}[t!]
    \centering
    \caption{Summary of some related works with ours in computational notebook topic.}
    \begin{tabular}{|m{2cm}|>{\raggedright}m{1.6cm}|>{\centering\arraybackslash}m{0.8cm}|>{\centering\arraybackslash}m{1cm}|>{\centering\arraybackslash}m{1.5cm}|>{\centering\arraybackslash}m{1.3cm}|>{\centering\arraybackslash}m{2.0cm}|>{\raggedright\arraybackslash}m{4.5cm}|}
        \toprule
        
        \textbf{} & \textbf{Purpose} & \textbf{Study Dataset} & \textbf{Module Error} & \textbf{NameError / Cell Order} & \textbf{Input File Error} & \textbf{Non-Executable Found} & \textbf{Dataset Characteristics}\\ 
        \midrule
        
        RELANCER \cite{Zhu2021}                          & Executability        & 4,043             & \cmark    & \xmark    & \xmark    & 47\%      & Collected from Meta Kaggle \cite{kaggle}\\ 
        \hline
        SnifferDog\cite{Wang2021}                        & Executability        & 2,646             & \cmark    & \xmark    & \xmark    & 72.6\%    & Random sample from \cite{Pimentel2019} only those with installable dependency\\ 
        \hline
        Osiris\cite{Wang2020}                            & Reproducibility      & 5,393             & \xmark    & \cmark    & \xmark    & 82.6\%    & Random sample from \cite{Pimentel2019}\\ 
        \hline
        Pimentel et al.\cite{Pimentel2019, Pimentel2021} & Empirical            & $>$1.4M    & N/A       & N/A       & N/A       & 76\%      & Includes high-number of low-starred, rarely reused notebooks from GitHub\\ 
        \hline
        \textbf{This work}                               & Executability        & 42.5K              & \cmark    & \cmark    & \cmark    & \percentPathological & Highly-starred notebooks from GitHub\\ 
        \bottomrule
    \end{tabular}
    \label{related_work_table}
    % \vspace{-3ex}
\end{table*}

    In this section, we distill key findings in a {\em FAQ} format, demonstrating the value of accurate executability categorization, the utility of a notebook corpus once marked unusable, and practices that could prevent non-executability or restore executability in future notebooks. 
    
%\noindent{\em Why do notebook executability measurements in this work differ from prior investigations?} We identify two key factors that distinguish our findings on notebook executability from prior empirical investigations. First, our notebook corpus is primarily sourced from popular repositories—actively shared, reused, and maintained—whereas prior work focused more broadly, including notebooks that are less actively managed or of lower quality. While previous studies provide insights into a general range of public notebooks, our study offers new perspectives on those designed for sharing, adaptation, and reuse. Second, our dataset was collected in 2024, while the most recent prior investigation was conducted in 2021. Over the past three years, several tools aimed at improving notebook quality have emerged \cite{}\textcolor{red}{CITE}, likely contributing to the increase in executability.

\noindent{\em Why are \percentPathological notebooks pathologically non-executable and \percentPartiallyRestored of restorable notebooks still only partially executable?} 
    Prior work has identified that notebooks have alarmingly low numbers of test cases \cite{Pimentel2019}. Developers releasing their notebooks for public use have very limited testing tools to verify reproducibility and executability. For example, notebooks with undefined variables ``NameError" and ``AttributeError" often go unnoticed due to unintended dependencies on session states saved from prior cell execution, leading to those errors being masked on the developer's machine, similar to Case Study 2 in Section \ref{case_study_2}. In a new environment, such states are unavailable, resulting in these errors. 
    %\tien{This may be right, but notebooks originally have execution orders that they can use to execute, even though they may not be consistent. Pimentel \cite{Pimentel2019} and Osiris \cite{Wang2020} looked into this method.} 
    Our findings on pathological errors in notebooks demonstrate the need for static and dynamic analysis tools to catch such errors early in notebook development. 
    %\waris{btw VSCode Jupyter notebook extension addresses this issue.}


\noindent{\em How can LLMs further restore notebook executability with a high success rate?} 
    In the first case study (Section \ref{case_study_1}), Llama-3 successfully generated synthetic data for a `.csv' file. However, in another case \cite{tirthajyoti}, it failed to produce a valid `PNG' file due to limitations in multi-modal generation. This is because most LLMs have been trained on similar textual data formats. Therefore, Llama-3 also faces limitations when generating images, audio, video, or zip files. Multi-modal models such as GPT-4o can generate richer and more diverse inputs. %Our results with language-specific LLMs indicate that these models have the potential to generate synthetic data that promises high executability. 
    This capability can benefit developers who want to share their notebooks without disclosing the input data files. They can leverage LLMs to generate synthetic data (or scripts for synthetic data generation similar to database benchmarks \cite{tpcds}), enhancing executability in remote environments. Additionally, LLM performance tends to degrade when provided with large contexts, which aligns with recent research on LLMs \cite{Leng2024}. Moreover, notebook executability can be enhanced by generating feedback-driven fixes (e.g., multi-shot) for non-executable cells. This includes generating new input files based on error feedback.  
    %For example, if a cell error is caused by a missing file, an LLM could generate the necessary data or adjust paths to match the environment. 
    % Such a framework would greatly benefit notebook developers, enabling quick execution of public notebooks and addressing issues like missing files with minimal manual intervention.


%Moreover, we can further improve the execution of notebook cells by generating feedback-driven fixes for non-executable cells. For instance, we can achieve this by allowing code edits in notebook cells or generating newer input based on the error related to generating input files.  If we get an error in a cell, we can ask the LLM to generate newer file data if it's related to the generated file in the same context or if it requires a code edit (e.g., renaming the file path from a Windows machine path to a Linux machine path) to make it executable. Such a framework will be highly beneficial for notebook developers. It allows them to quickly run public notebooks and automatically address issues such as missing files, which would otherwise require manual effort.
%  Also, if an LLM generates six columns of a CSV correctly but the name or the data in the last column is not correct, we can ask the LLM to generate the last column data correctly. 



\noindent{\em What use is there for partially executable notebooks?}
    Prolonging notebook execution leads to more cells being executed successfully, which translates into more code surfaces that can be executed and understood. This inherently improves code reuse and enhances the chances of notebooks being fully restored. Furthermore, dynamic analysis techniques are limited to executable code only. By improving notebook executability, we increase the application surface of dynamic analysis tools such as dynamic taint analysis \cite{clause2007dytan}, runtime tracing \cite{meliou2011tracing}, automated debugging \cite{parnin2011automated}, symbolic execution \cite{baldoni2018survey}, and Osiris \cite{Wang2020} that can play vital roles in recovering notebook reproducibility. %Additionally, code corpora have proven extremely valuable in training new emerging LLMs. A big challenge is verifying the correctness of training data. More executable cells offer pathways to verified/tested code snippets that can be included in model training. 
    Partial executability metrics can also provide fine-grained measures of the effectiveness of reproducibility and execution-restoration tools. Additionally, it can serve as valuable feedback when applying code repair techniques in notebooks, such as automated cell reordering \cite{Wang2020}. 
    
    
\noindent {\em How do the findings of this study improve notebooks?} 
    We explored several design choices in building our measurement framework. We identify gaps in the notebook ecosystem for code review, debugging, test generation, and build tools. For existing public notebooks, which are growing exponentially \cite{Rule2018}, this measurement framework can be extended to improve fine-grained executability, feeding into dynamic analysis tools to enhance reproducibility. 
    
    Our findings, showing that only \totalRequirement (\percentRequirementInTotal) repositories have REQUIREMENTS files, indicate that there is no standardized build and continuous integration mechanism for notebooks. While this may not apply to exploratory, standalone, one-off notebooks, the most popular repositories provide a range of notebooks that must be properly orchestrated to ensure executability. We believe Python-based build tools, such as {\small{\texttt{disutils}}} \cite{distutil} and {\small{\texttt{setuptools}}} \cite{setuptools}, can be extended to support notebook extensions at the browser level to enforce correct build and dependency management for notebooks.


 \noindent {\em Threats to validity.}
    Outdated, textual, and empty notebooks can cause variations in the results of this study. To mitigate that, we exclude notebooks written in Python 2 since support for Python 2 was discontinued in 2020. We also filter empty or instructional notebooks from our dataset. Similarly, focusing purely on GitHub repositories can bias the results towards a specific class of notebooks. Upon investigation, we identified numerous HuggingFace and Kaggle notebooks that are also hosted on GitHub. While there is always room to increase the scale of the analysis, our emphasis on popular, actively reusable notebooks deprioritizes the need to expand the scale. Our static and dynamic analyses rely on the Python AST parser and Python interpreter to capture errors. These tools can potentially miss or misclassify errors, which can impact our categorization. 
        %We do not capture semantic errors, nor do we measure reproducibility. 
    Since most of our runtime errors are captured through dynamic error checking, an earlier fatal runtime error (that we cannot resolve) may hinder our ability to capture ``FileNotFound" or ``ModuleNotFound" errors. Lastly,  we use a timeout of 5 minutes to execute each notebook. 
        %Any notebook taking more than 5 minutes is considered unanalyzable; thus, we exclude it from our dataset. 
    This approach may exclude notebooks like the ones requiring expensive machine learning training. However, in our notebook corpus, only \percentTimeout of notebooks encountered timeout errors.



\section{Funding and Competing Interests Declarations}
\label{sec:declarations}
% \subsection{Funding and Competing Interests. } 
The author(s) were employed by Google (Google Brain, now Google DeepMind) at the time this project was initiated. Google recognizes a potential need to disclose certain confidential information, and to protect such information from unauthorized use and disclosure. Google had the right to remove its intellectual property or trade secrets subject to the following stipulations: 
\begin{enumerate}
    \item Removing Google confidential information from the paper, including data, code, and statistics. 
    \item Compliance with Google’s obligations as it relates to applicable laws, including the Data Protection Law, security laws, confidentiality requirements, and contractual commitments.
\end{enumerate}



% Acknowledgments here
% \ACKNOWLEDGMENT{The authors are grateful to the associate editor and two anonymous referees for valuable comments on an earlier version of the paper. The research of the first author is supported by NWO Grant 613.001.208. The third author acknowledges the funding support from the Singapore Ministry of Education Social Science Research Thematic Grant MOE2016-SSRTG-059. Disclaimer: Any opinions, findings, and conclusions or recommendations expressed in this material are those of the author(s) and do not reflect the views of the Singapore Ministry of Education or the Singapore Government.}	

\theendnotes

\bibliographystyle{informs2014} % outcomment this and next line in Case 1
\bibliography{reference} % if more than one, comma separated
	

\newpage
\begin{APPENDIX}{The Blessing of Reasoning: LLM-Generated Explanations in Black-Box Recommender Systems}
 \subsection{Lloyd-Max Algorithm}
\label{subsec:Lloyd-Max}
For a given quantization bitwidth $B$ and an operand $\bm{X}$, the Lloyd-Max algorithm finds $2^B$ quantization levels $\{\hat{x}_i\}_{i=1}^{2^B}$ such that quantizing $\bm{X}$ by rounding each scalar in $\bm{X}$ to the nearest quantization level minimizes the quantization MSE. 

The algorithm starts with an initial guess of quantization levels and then iteratively computes quantization thresholds $\{\tau_i\}_{i=1}^{2^B-1}$ and updates quantization levels $\{\hat{x}_i\}_{i=1}^{2^B}$. Specifically, at iteration $n$, thresholds are set to the midpoints of the previous iteration's levels:
\begin{align*}
    \tau_i^{(n)}=\frac{\hat{x}_i^{(n-1)}+\hat{x}_{i+1}^{(n-1)}}2 \text{ for } i=1\ldots 2^B-1
\end{align*}
Subsequently, the quantization levels are re-computed as conditional means of the data regions defined by the new thresholds:
\begin{align*}
    \hat{x}_i^{(n)}=\mathbb{E}\left[ \bm{X} \big| \bm{X}\in [\tau_{i-1}^{(n)},\tau_i^{(n)}] \right] \text{ for } i=1\ldots 2^B
\end{align*}
where to satisfy boundary conditions we have $\tau_0=-\infty$ and $\tau_{2^B}=\infty$. The algorithm iterates the above steps until convergence.

Figure \ref{fig:lm_quant} compares the quantization levels of a $7$-bit floating point (E3M3) quantizer (left) to a $7$-bit Lloyd-Max quantizer (right) when quantizing a layer of weights from the GPT3-126M model at a per-tensor granularity. As shown, the Lloyd-Max quantizer achieves substantially lower quantization MSE. Further, Table \ref{tab:FP7_vs_LM7} shows the superior perplexity achieved by Lloyd-Max quantizers for bitwidths of $7$, $6$ and $5$. The difference between the quantizers is clear at 5 bits, where per-tensor FP quantization incurs a drastic and unacceptable increase in perplexity, while Lloyd-Max quantization incurs a much smaller increase. Nevertheless, we note that even the optimal Lloyd-Max quantizer incurs a notable ($\sim 1.5$) increase in perplexity due to the coarse granularity of quantization. 

\begin{figure}[h]
  \centering
  \includegraphics[width=0.7\linewidth]{sections/figures/LM7_FP7.pdf}
  \caption{\small Quantization levels and the corresponding quantization MSE of Floating Point (left) vs Lloyd-Max (right) Quantizers for a layer of weights in the GPT3-126M model.}
  \label{fig:lm_quant}
\end{figure}

\begin{table}[h]\scriptsize
\begin{center}
\caption{\label{tab:FP7_vs_LM7} \small Comparing perplexity (lower is better) achieved by floating point quantizers and Lloyd-Max quantizers on a GPT3-126M model for the Wikitext-103 dataset.}
\begin{tabular}{c|cc|c}
\hline
 \multirow{2}{*}{\textbf{Bitwidth}} & \multicolumn{2}{|c|}{\textbf{Floating-Point Quantizer}} & \textbf{Lloyd-Max Quantizer} \\
 & Best Format & Wikitext-103 Perplexity & Wikitext-103 Perplexity \\
\hline
7 & E3M3 & 18.32 & 18.27 \\
6 & E3M2 & 19.07 & 18.51 \\
5 & E4M0 & 43.89 & 19.71 \\
\hline
\end{tabular}
\end{center}
\end{table}

\subsection{Proof of Local Optimality of LO-BCQ}
\label{subsec:lobcq_opt_proof}
For a given block $\bm{b}_j$, the quantization MSE during LO-BCQ can be empirically evaluated as $\frac{1}{L_b}\lVert \bm{b}_j- \bm{\hat{b}}_j\rVert^2_2$ where $\bm{\hat{b}}_j$ is computed from equation (\ref{eq:clustered_quantization_definition}) as $C_{f(\bm{b}_j)}(\bm{b}_j)$. Further, for a given block cluster $\mathcal{B}_i$, we compute the quantization MSE as $\frac{1}{|\mathcal{B}_{i}|}\sum_{\bm{b} \in \mathcal{B}_{i}} \frac{1}{L_b}\lVert \bm{b}- C_i^{(n)}(\bm{b})\rVert^2_2$. Therefore, at the end of iteration $n$, we evaluate the overall quantization MSE $J^{(n)}$ for a given operand $\bm{X}$ composed of $N_c$ block clusters as:
\begin{align*}
    \label{eq:mse_iter_n}
    J^{(n)} = \frac{1}{N_c} \sum_{i=1}^{N_c} \frac{1}{|\mathcal{B}_{i}^{(n)}|}\sum_{\bm{v} \in \mathcal{B}_{i}^{(n)}} \frac{1}{L_b}\lVert \bm{b}- B_i^{(n)}(\bm{b})\rVert^2_2
\end{align*}

At the end of iteration $n$, the codebooks are updated from $\mathcal{C}^{(n-1)}$ to $\mathcal{C}^{(n)}$. However, the mapping of a given vector $\bm{b}_j$ to quantizers $\mathcal{C}^{(n)}$ remains as  $f^{(n)}(\bm{b}_j)$. At the next iteration, during the vector clustering step, $f^{(n+1)}(\bm{b}_j)$ finds new mapping of $\bm{b}_j$ to updated codebooks $\mathcal{C}^{(n)}$ such that the quantization MSE over the candidate codebooks is minimized. Therefore, we obtain the following result for $\bm{b}_j$:
\begin{align*}
\frac{1}{L_b}\lVert \bm{b}_j - C_{f^{(n+1)}(\bm{b}_j)}^{(n)}(\bm{b}_j)\rVert^2_2 \le \frac{1}{L_b}\lVert \bm{b}_j - C_{f^{(n)}(\bm{b}_j)}^{(n)}(\bm{b}_j)\rVert^2_2
\end{align*}

That is, quantizing $\bm{b}_j$ at the end of the block clustering step of iteration $n+1$ results in lower quantization MSE compared to quantizing at the end of iteration $n$. Since this is true for all $\bm{b} \in \bm{X}$, we assert the following:
\begin{equation}
\begin{split}
\label{eq:mse_ineq_1}
    \tilde{J}^{(n+1)} &= \frac{1}{N_c} \sum_{i=1}^{N_c} \frac{1}{|\mathcal{B}_{i}^{(n+1)}|}\sum_{\bm{b} \in \mathcal{B}_{i}^{(n+1)}} \frac{1}{L_b}\lVert \bm{b} - C_i^{(n)}(b)\rVert^2_2 \le J^{(n)}
\end{split}
\end{equation}
where $\tilde{J}^{(n+1)}$ is the the quantization MSE after the vector clustering step at iteration $n+1$.

Next, during the codebook update step (\ref{eq:quantizers_update}) at iteration $n+1$, the per-cluster codebooks $\mathcal{C}^{(n)}$ are updated to $\mathcal{C}^{(n+1)}$ by invoking the Lloyd-Max algorithm \citep{Lloyd}. We know that for any given value distribution, the Lloyd-Max algorithm minimizes the quantization MSE. Therefore, for a given vector cluster $\mathcal{B}_i$ we obtain the following result:

\begin{equation}
    \frac{1}{|\mathcal{B}_{i}^{(n+1)}|}\sum_{\bm{b} \in \mathcal{B}_{i}^{(n+1)}} \frac{1}{L_b}\lVert \bm{b}- C_i^{(n+1)}(\bm{b})\rVert^2_2 \le \frac{1}{|\mathcal{B}_{i}^{(n+1)}|}\sum_{\bm{b} \in \mathcal{B}_{i}^{(n+1)}} \frac{1}{L_b}\lVert \bm{b}- C_i^{(n)}(\bm{b})\rVert^2_2
\end{equation}

The above equation states that quantizing the given block cluster $\mathcal{B}_i$ after updating the associated codebook from $C_i^{(n)}$ to $C_i^{(n+1)}$ results in lower quantization MSE. Since this is true for all the block clusters, we derive the following result: 
\begin{equation}
\begin{split}
\label{eq:mse_ineq_2}
     J^{(n+1)} &= \frac{1}{N_c} \sum_{i=1}^{N_c} \frac{1}{|\mathcal{B}_{i}^{(n+1)}|}\sum_{\bm{b} \in \mathcal{B}_{i}^{(n+1)}} \frac{1}{L_b}\lVert \bm{b}- C_i^{(n+1)}(\bm{b})\rVert^2_2  \le \tilde{J}^{(n+1)}   
\end{split}
\end{equation}

Following (\ref{eq:mse_ineq_1}) and (\ref{eq:mse_ineq_2}), we find that the quantization MSE is non-increasing for each iteration, that is, $J^{(1)} \ge J^{(2)} \ge J^{(3)} \ge \ldots \ge J^{(M)}$ where $M$ is the maximum number of iterations. 
%Therefore, we can say that if the algorithm converges, then it must be that it has converged to a local minimum. 
\hfill $\blacksquare$


\begin{figure}
    \begin{center}
    \includegraphics[width=0.5\textwidth]{sections//figures/mse_vs_iter.pdf}
    \end{center}
    \caption{\small NMSE vs iterations during LO-BCQ compared to other block quantization proposals}
    \label{fig:nmse_vs_iter}
\end{figure}

Figure \ref{fig:nmse_vs_iter} shows the empirical convergence of LO-BCQ across several block lengths and number of codebooks. Also, the MSE achieved by LO-BCQ is compared to baselines such as MXFP and VSQ. As shown, LO-BCQ converges to a lower MSE than the baselines. Further, we achieve better convergence for larger number of codebooks ($N_c$) and for a smaller block length ($L_b$), both of which increase the bitwidth of BCQ (see Eq \ref{eq:bitwidth_bcq}).


\subsection{Additional Accuracy Results}
%Table \ref{tab:lobcq_config} lists the various LOBCQ configurations and their corresponding bitwidths.
\begin{table}
\setlength{\tabcolsep}{4.75pt}
\begin{center}
\caption{\label{tab:lobcq_config} Various LO-BCQ configurations and their bitwidths.}
\begin{tabular}{|c||c|c|c|c||c|c||c|} 
\hline
 & \multicolumn{4}{|c||}{$L_b=8$} & \multicolumn{2}{|c||}{$L_b=4$} & $L_b=2$ \\
 \hline
 \backslashbox{$L_A$\kern-1em}{\kern-1em$N_c$} & 2 & 4 & 8 & 16 & 2 & 4 & 2 \\
 \hline
 64 & 4.25 & 4.375 & 4.5 & 4.625 & 4.375 & 4.625 & 4.625\\
 \hline
 32 & 4.375 & 4.5 & 4.625& 4.75 & 4.5 & 4.75 & 4.75 \\
 \hline
 16 & 4.625 & 4.75& 4.875 & 5 & 4.75 & 5 & 5 \\
 \hline
\end{tabular}
\end{center}
\end{table}

%\subsection{Perplexity achieved by various LO-BCQ configurations on Wikitext-103 dataset}

\begin{table} \centering
\begin{tabular}{|c||c|c|c|c||c|c||c|} 
\hline
 $L_b \rightarrow$& \multicolumn{4}{c||}{8} & \multicolumn{2}{c||}{4} & 2\\
 \hline
 \backslashbox{$L_A$\kern-1em}{\kern-1em$N_c$} & 2 & 4 & 8 & 16 & 2 & 4 & 2  \\
 %$N_c \rightarrow$ & 2 & 4 & 8 & 16 & 2 & 4 & 2 \\
 \hline
 \hline
 \multicolumn{8}{c}{GPT3-1.3B (FP32 PPL = 9.98)} \\ 
 \hline
 \hline
 64 & 10.40 & 10.23 & 10.17 & 10.15 &  10.28 & 10.18 & 10.19 \\
 \hline
 32 & 10.25 & 10.20 & 10.15 & 10.12 &  10.23 & 10.17 & 10.17 \\
 \hline
 16 & 10.22 & 10.16 & 10.10 & 10.09 &  10.21 & 10.14 & 10.16 \\
 \hline
  \hline
 \multicolumn{8}{c}{GPT3-8B (FP32 PPL = 7.38)} \\ 
 \hline
 \hline
 64 & 7.61 & 7.52 & 7.48 &  7.47 &  7.55 &  7.49 & 7.50 \\
 \hline
 32 & 7.52 & 7.50 & 7.46 &  7.45 &  7.52 &  7.48 & 7.48  \\
 \hline
 16 & 7.51 & 7.48 & 7.44 &  7.44 &  7.51 &  7.49 & 7.47  \\
 \hline
\end{tabular}
\caption{\label{tab:ppl_gpt3_abalation} Wikitext-103 perplexity across GPT3-1.3B and 8B models.}
\end{table}

\begin{table} \centering
\begin{tabular}{|c||c|c|c|c||} 
\hline
 $L_b \rightarrow$& \multicolumn{4}{c||}{8}\\
 \hline
 \backslashbox{$L_A$\kern-1em}{\kern-1em$N_c$} & 2 & 4 & 8 & 16 \\
 %$N_c \rightarrow$ & 2 & 4 & 8 & 16 & 2 & 4 & 2 \\
 \hline
 \hline
 \multicolumn{5}{|c|}{Llama2-7B (FP32 PPL = 5.06)} \\ 
 \hline
 \hline
 64 & 5.31 & 5.26 & 5.19 & 5.18  \\
 \hline
 32 & 5.23 & 5.25 & 5.18 & 5.15  \\
 \hline
 16 & 5.23 & 5.19 & 5.16 & 5.14  \\
 \hline
 \multicolumn{5}{|c|}{Nemotron4-15B (FP32 PPL = 5.87)} \\ 
 \hline
 \hline
 64  & 6.3 & 6.20 & 6.13 & 6.08  \\
 \hline
 32  & 6.24 & 6.12 & 6.07 & 6.03  \\
 \hline
 16  & 6.12 & 6.14 & 6.04 & 6.02  \\
 \hline
 \multicolumn{5}{|c|}{Nemotron4-340B (FP32 PPL = 3.48)} \\ 
 \hline
 \hline
 64 & 3.67 & 3.62 & 3.60 & 3.59 \\
 \hline
 32 & 3.63 & 3.61 & 3.59 & 3.56 \\
 \hline
 16 & 3.61 & 3.58 & 3.57 & 3.55 \\
 \hline
\end{tabular}
\caption{\label{tab:ppl_llama7B_nemo15B} Wikitext-103 perplexity compared to FP32 baseline in Llama2-7B and Nemotron4-15B, 340B models}
\end{table}

%\subsection{Perplexity achieved by various LO-BCQ configurations on MMLU dataset}


\begin{table} \centering
\begin{tabular}{|c||c|c|c|c||c|c|c|c|} 
\hline
 $L_b \rightarrow$& \multicolumn{4}{c||}{8} & \multicolumn{4}{c||}{8}\\
 \hline
 \backslashbox{$L_A$\kern-1em}{\kern-1em$N_c$} & 2 & 4 & 8 & 16 & 2 & 4 & 8 & 16  \\
 %$N_c \rightarrow$ & 2 & 4 & 8 & 16 & 2 & 4 & 2 \\
 \hline
 \hline
 \multicolumn{5}{|c|}{Llama2-7B (FP32 Accuracy = 45.8\%)} & \multicolumn{4}{|c|}{Llama2-70B (FP32 Accuracy = 69.12\%)} \\ 
 \hline
 \hline
 64 & 43.9 & 43.4 & 43.9 & 44.9 & 68.07 & 68.27 & 68.17 & 68.75 \\
 \hline
 32 & 44.5 & 43.8 & 44.9 & 44.5 & 68.37 & 68.51 & 68.35 & 68.27  \\
 \hline
 16 & 43.9 & 42.7 & 44.9 & 45 & 68.12 & 68.77 & 68.31 & 68.59  \\
 \hline
 \hline
 \multicolumn{5}{|c|}{GPT3-22B (FP32 Accuracy = 38.75\%)} & \multicolumn{4}{|c|}{Nemotron4-15B (FP32 Accuracy = 64.3\%)} \\ 
 \hline
 \hline
 64 & 36.71 & 38.85 & 38.13 & 38.92 & 63.17 & 62.36 & 63.72 & 64.09 \\
 \hline
 32 & 37.95 & 38.69 & 39.45 & 38.34 & 64.05 & 62.30 & 63.8 & 64.33  \\
 \hline
 16 & 38.88 & 38.80 & 38.31 & 38.92 & 63.22 & 63.51 & 63.93 & 64.43  \\
 \hline
\end{tabular}
\caption{\label{tab:mmlu_abalation} Accuracy on MMLU dataset across GPT3-22B, Llama2-7B, 70B and Nemotron4-15B models.}
\end{table}


%\subsection{Perplexity achieved by various LO-BCQ configurations on LM evaluation harness}

\begin{table} \centering
\begin{tabular}{|c||c|c|c|c||c|c|c|c|} 
\hline
 $L_b \rightarrow$& \multicolumn{4}{c||}{8} & \multicolumn{4}{c||}{8}\\
 \hline
 \backslashbox{$L_A$\kern-1em}{\kern-1em$N_c$} & 2 & 4 & 8 & 16 & 2 & 4 & 8 & 16  \\
 %$N_c \rightarrow$ & 2 & 4 & 8 & 16 & 2 & 4 & 2 \\
 \hline
 \hline
 \multicolumn{5}{|c|}{Race (FP32 Accuracy = 37.51\%)} & \multicolumn{4}{|c|}{Boolq (FP32 Accuracy = 64.62\%)} \\ 
 \hline
 \hline
 64 & 36.94 & 37.13 & 36.27 & 37.13 & 63.73 & 62.26 & 63.49 & 63.36 \\
 \hline
 32 & 37.03 & 36.36 & 36.08 & 37.03 & 62.54 & 63.51 & 63.49 & 63.55  \\
 \hline
 16 & 37.03 & 37.03 & 36.46 & 37.03 & 61.1 & 63.79 & 63.58 & 63.33  \\
 \hline
 \hline
 \multicolumn{5}{|c|}{Winogrande (FP32 Accuracy = 58.01\%)} & \multicolumn{4}{|c|}{Piqa (FP32 Accuracy = 74.21\%)} \\ 
 \hline
 \hline
 64 & 58.17 & 57.22 & 57.85 & 58.33 & 73.01 & 73.07 & 73.07 & 72.80 \\
 \hline
 32 & 59.12 & 58.09 & 57.85 & 58.41 & 73.01 & 73.94 & 72.74 & 73.18  \\
 \hline
 16 & 57.93 & 58.88 & 57.93 & 58.56 & 73.94 & 72.80 & 73.01 & 73.94  \\
 \hline
\end{tabular}
\caption{\label{tab:mmlu_abalation} Accuracy on LM evaluation harness tasks on GPT3-1.3B model.}
\end{table}

\begin{table} \centering
\begin{tabular}{|c||c|c|c|c||c|c|c|c|} 
\hline
 $L_b \rightarrow$& \multicolumn{4}{c||}{8} & \multicolumn{4}{c||}{8}\\
 \hline
 \backslashbox{$L_A$\kern-1em}{\kern-1em$N_c$} & 2 & 4 & 8 & 16 & 2 & 4 & 8 & 16  \\
 %$N_c \rightarrow$ & 2 & 4 & 8 & 16 & 2 & 4 & 2 \\
 \hline
 \hline
 \multicolumn{5}{|c|}{Race (FP32 Accuracy = 41.34\%)} & \multicolumn{4}{|c|}{Boolq (FP32 Accuracy = 68.32\%)} \\ 
 \hline
 \hline
 64 & 40.48 & 40.10 & 39.43 & 39.90 & 69.20 & 68.41 & 69.45 & 68.56 \\
 \hline
 32 & 39.52 & 39.52 & 40.77 & 39.62 & 68.32 & 67.43 & 68.17 & 69.30  \\
 \hline
 16 & 39.81 & 39.71 & 39.90 & 40.38 & 68.10 & 66.33 & 69.51 & 69.42  \\
 \hline
 \hline
 \multicolumn{5}{|c|}{Winogrande (FP32 Accuracy = 67.88\%)} & \multicolumn{4}{|c|}{Piqa (FP32 Accuracy = 78.78\%)} \\ 
 \hline
 \hline
 64 & 66.85 & 66.61 & 67.72 & 67.88 & 77.31 & 77.42 & 77.75 & 77.64 \\
 \hline
 32 & 67.25 & 67.72 & 67.72 & 67.00 & 77.31 & 77.04 & 77.80 & 77.37  \\
 \hline
 16 & 68.11 & 68.90 & 67.88 & 67.48 & 77.37 & 78.13 & 78.13 & 77.69  \\
 \hline
\end{tabular}
\caption{\label{tab:mmlu_abalation} Accuracy on LM evaluation harness tasks on GPT3-8B model.}
\end{table}

\begin{table} \centering
\begin{tabular}{|c||c|c|c|c||c|c|c|c|} 
\hline
 $L_b \rightarrow$& \multicolumn{4}{c||}{8} & \multicolumn{4}{c||}{8}\\
 \hline
 \backslashbox{$L_A$\kern-1em}{\kern-1em$N_c$} & 2 & 4 & 8 & 16 & 2 & 4 & 8 & 16  \\
 %$N_c \rightarrow$ & 2 & 4 & 8 & 16 & 2 & 4 & 2 \\
 \hline
 \hline
 \multicolumn{5}{|c|}{Race (FP32 Accuracy = 40.67\%)} & \multicolumn{4}{|c|}{Boolq (FP32 Accuracy = 76.54\%)} \\ 
 \hline
 \hline
 64 & 40.48 & 40.10 & 39.43 & 39.90 & 75.41 & 75.11 & 77.09 & 75.66 \\
 \hline
 32 & 39.52 & 39.52 & 40.77 & 39.62 & 76.02 & 76.02 & 75.96 & 75.35  \\
 \hline
 16 & 39.81 & 39.71 & 39.90 & 40.38 & 75.05 & 73.82 & 75.72 & 76.09  \\
 \hline
 \hline
 \multicolumn{5}{|c|}{Winogrande (FP32 Accuracy = 70.64\%)} & \multicolumn{4}{|c|}{Piqa (FP32 Accuracy = 79.16\%)} \\ 
 \hline
 \hline
 64 & 69.14 & 70.17 & 70.17 & 70.56 & 78.24 & 79.00 & 78.62 & 78.73 \\
 \hline
 32 & 70.96 & 69.69 & 71.27 & 69.30 & 78.56 & 79.49 & 79.16 & 78.89  \\
 \hline
 16 & 71.03 & 69.53 & 69.69 & 70.40 & 78.13 & 79.16 & 79.00 & 79.00  \\
 \hline
\end{tabular}
\caption{\label{tab:mmlu_abalation} Accuracy on LM evaluation harness tasks on GPT3-22B model.}
\end{table}

\begin{table} \centering
\begin{tabular}{|c||c|c|c|c||c|c|c|c|} 
\hline
 $L_b \rightarrow$& \multicolumn{4}{c||}{8} & \multicolumn{4}{c||}{8}\\
 \hline
 \backslashbox{$L_A$\kern-1em}{\kern-1em$N_c$} & 2 & 4 & 8 & 16 & 2 & 4 & 8 & 16  \\
 %$N_c \rightarrow$ & 2 & 4 & 8 & 16 & 2 & 4 & 2 \\
 \hline
 \hline
 \multicolumn{5}{|c|}{Race (FP32 Accuracy = 44.4\%)} & \multicolumn{4}{|c|}{Boolq (FP32 Accuracy = 79.29\%)} \\ 
 \hline
 \hline
 64 & 42.49 & 42.51 & 42.58 & 43.45 & 77.58 & 77.37 & 77.43 & 78.1 \\
 \hline
 32 & 43.35 & 42.49 & 43.64 & 43.73 & 77.86 & 75.32 & 77.28 & 77.86  \\
 \hline
 16 & 44.21 & 44.21 & 43.64 & 42.97 & 78.65 & 77 & 76.94 & 77.98  \\
 \hline
 \hline
 \multicolumn{5}{|c|}{Winogrande (FP32 Accuracy = 69.38\%)} & \multicolumn{4}{|c|}{Piqa (FP32 Accuracy = 78.07\%)} \\ 
 \hline
 \hline
 64 & 68.9 & 68.43 & 69.77 & 68.19 & 77.09 & 76.82 & 77.09 & 77.86 \\
 \hline
 32 & 69.38 & 68.51 & 68.82 & 68.90 & 78.07 & 76.71 & 78.07 & 77.86  \\
 \hline
 16 & 69.53 & 67.09 & 69.38 & 68.90 & 77.37 & 77.8 & 77.91 & 77.69  \\
 \hline
\end{tabular}
\caption{\label{tab:mmlu_abalation} Accuracy on LM evaluation harness tasks on Llama2-7B model.}
\end{table}

\begin{table} \centering
\begin{tabular}{|c||c|c|c|c||c|c|c|c|} 
\hline
 $L_b \rightarrow$& \multicolumn{4}{c||}{8} & \multicolumn{4}{c||}{8}\\
 \hline
 \backslashbox{$L_A$\kern-1em}{\kern-1em$N_c$} & 2 & 4 & 8 & 16 & 2 & 4 & 8 & 16  \\
 %$N_c \rightarrow$ & 2 & 4 & 8 & 16 & 2 & 4 & 2 \\
 \hline
 \hline
 \multicolumn{5}{|c|}{Race (FP32 Accuracy = 48.8\%)} & \multicolumn{4}{|c|}{Boolq (FP32 Accuracy = 85.23\%)} \\ 
 \hline
 \hline
 64 & 49.00 & 49.00 & 49.28 & 48.71 & 82.82 & 84.28 & 84.03 & 84.25 \\
 \hline
 32 & 49.57 & 48.52 & 48.33 & 49.28 & 83.85 & 84.46 & 84.31 & 84.93  \\
 \hline
 16 & 49.85 & 49.09 & 49.28 & 48.99 & 85.11 & 84.46 & 84.61 & 83.94  \\
 \hline
 \hline
 \multicolumn{5}{|c|}{Winogrande (FP32 Accuracy = 79.95\%)} & \multicolumn{4}{|c|}{Piqa (FP32 Accuracy = 81.56\%)} \\ 
 \hline
 \hline
 64 & 78.77 & 78.45 & 78.37 & 79.16 & 81.45 & 80.69 & 81.45 & 81.5 \\
 \hline
 32 & 78.45 & 79.01 & 78.69 & 80.66 & 81.56 & 80.58 & 81.18 & 81.34  \\
 \hline
 16 & 79.95 & 79.56 & 79.79 & 79.72 & 81.28 & 81.66 & 81.28 & 80.96  \\
 \hline
\end{tabular}
\caption{\label{tab:mmlu_abalation} Accuracy on LM evaluation harness tasks on Llama2-70B model.}
\end{table}

%\section{MSE Studies}
%\textcolor{red}{TODO}


\subsection{Number Formats and Quantization Method}
\label{subsec:numFormats_quantMethod}
\subsubsection{Integer Format}
An $n$-bit signed integer (INT) is typically represented with a 2s-complement format \citep{yao2022zeroquant,xiao2023smoothquant,dai2021vsq}, where the most significant bit denotes the sign.

\subsubsection{Floating Point Format}
An $n$-bit signed floating point (FP) number $x$ comprises of a 1-bit sign ($x_{\mathrm{sign}}$), $B_m$-bit mantissa ($x_{\mathrm{mant}}$) and $B_e$-bit exponent ($x_{\mathrm{exp}}$) such that $B_m+B_e=n-1$. The associated constant exponent bias ($E_{\mathrm{bias}}$) is computed as $(2^{{B_e}-1}-1)$. We denote this format as $E_{B_e}M_{B_m}$.  

\subsubsection{Quantization Scheme}
\label{subsec:quant_method}
A quantization scheme dictates how a given unquantized tensor is converted to its quantized representation. We consider FP formats for the purpose of illustration. Given an unquantized tensor $\bm{X}$ and an FP format $E_{B_e}M_{B_m}$, we first, we compute the quantization scale factor $s_X$ that maps the maximum absolute value of $\bm{X}$ to the maximum quantization level of the $E_{B_e}M_{B_m}$ format as follows:
\begin{align}
\label{eq:sf}
    s_X = \frac{\mathrm{max}(|\bm{X}|)}{\mathrm{max}(E_{B_e}M_{B_m})}
\end{align}
In the above equation, $|\cdot|$ denotes the absolute value function.

Next, we scale $\bm{X}$ by $s_X$ and quantize it to $\hat{\bm{X}}$ by rounding it to the nearest quantization level of $E_{B_e}M_{B_m}$ as:

\begin{align}
\label{eq:tensor_quant}
    \hat{\bm{X}} = \text{round-to-nearest}\left(\frac{\bm{X}}{s_X}, E_{B_e}M_{B_m}\right)
\end{align}

We perform dynamic max-scaled quantization \citep{wu2020integer}, where the scale factor $s$ for activations is dynamically computed during runtime.

\subsection{Vector Scaled Quantization}
\begin{wrapfigure}{r}{0.35\linewidth}
  \centering
  \includegraphics[width=\linewidth]{sections/figures/vsquant.jpg}
  \caption{\small Vectorwise decomposition for per-vector scaled quantization (VSQ \citep{dai2021vsq}).}
  \label{fig:vsquant}
\end{wrapfigure}
During VSQ \citep{dai2021vsq}, the operand tensors are decomposed into 1D vectors in a hardware friendly manner as shown in Figure \ref{fig:vsquant}. Since the decomposed tensors are used as operands in matrix multiplications during inference, it is beneficial to perform this decomposition along the reduction dimension of the multiplication. The vectorwise quantization is performed similar to tensorwise quantization described in Equations \ref{eq:sf} and \ref{eq:tensor_quant}, where a scale factor $s_v$ is required for each vector $\bm{v}$ that maps the maximum absolute value of that vector to the maximum quantization level. While smaller vector lengths can lead to larger accuracy gains, the associated memory and computational overheads due to the per-vector scale factors increases. To alleviate these overheads, VSQ \citep{dai2021vsq} proposed a second level quantization of the per-vector scale factors to unsigned integers, while MX \citep{rouhani2023shared} quantizes them to integer powers of 2 (denoted as $2^{INT}$).

\subsubsection{MX Format}
The MX format proposed in \citep{rouhani2023microscaling} introduces the concept of sub-block shifting. For every two scalar elements of $b$-bits each, there is a shared exponent bit. The value of this exponent bit is determined through an empirical analysis that targets minimizing quantization MSE. We note that the FP format $E_{1}M_{b}$ is strictly better than MX from an accuracy perspective since it allocates a dedicated exponent bit to each scalar as opposed to sharing it across two scalars. Therefore, we conservatively bound the accuracy of a $b+2$-bit signed MX format with that of a $E_{1}M_{b}$ format in our comparisons. For instance, we use E1M2 format as a proxy for MX4.

\begin{figure}
    \centering
    \includegraphics[width=1\linewidth]{sections//figures/BlockFormats.pdf}
    \caption{\small Comparing LO-BCQ to MX format.}
    \label{fig:block_formats}
\end{figure}

Figure \ref{fig:block_formats} compares our $4$-bit LO-BCQ block format to MX \citep{rouhani2023microscaling}. As shown, both LO-BCQ and MX decompose a given operand tensor into block arrays and each block array into blocks. Similar to MX, we find that per-block quantization ($L_b < L_A$) leads to better accuracy due to increased flexibility. While MX achieves this through per-block $1$-bit micro-scales, we associate a dedicated codebook to each block through a per-block codebook selector. Further, MX quantizes the per-block array scale-factor to E8M0 format without per-tensor scaling. In contrast during LO-BCQ, we find that per-tensor scaling combined with quantization of per-block array scale-factor to E4M3 format results in superior inference accuracy across models. 

\end{APPENDIX}

		
\end{document} 