% This must be in the first 5 lines to tell arXiv to use pdfLaTeX, which is strongly recommended.
\pdfoutput=1
% In particular, the hyperref package requires pdfLaTeX in order to break URLs across lines.

\documentclass[11pt]{article}

% Recommended, but optional, packages for figures and better typesetting:
\usepackage{microtype}
\usepackage{graphicx}
\usepackage{subfigure}
\usepackage{booktabs} % for professional tables
\usepackage{graphicx}
% hyperref makes hyperlinks in the resulting PDF.
% If your build breaks (sometimes temporarily if a hyperlink spans a page)
% please comment out the following usepackage line and replace
% \usepackage{icml2025} with \usepackage[nohyperref]{icml2025} above.
\usepackage{hyperref}


% Attempt to make hyperref and algorithmic work together better:
\newcommand{\theHalgorithm}{\arabic{algorithm}}

% Change "review" to "final" to generate the final (sometimes called camera-ready) version.
% Change to "preprint" to generate a non-anonymous version with page numbers.
\usepackage[preprint]{acl}

% For theorems and such
\usepackage{amsmath}
\usepackage{amssymb}
\usepackage{mathtools}
\usepackage{amsthm}



\usepackage{times}
\usepackage{latexsym}

% For proper rendering and hyphenation of words containing Latin characters (including in bib files)
\usepackage[T1]{fontenc}
% For Vietnamese characters
% \usepackage[T5]{fontenc}
% See https://www.latex-project.org/help/documentation/encguide.pdf for other character sets

% This assumes your files are encoded as UTF8
\usepackage[utf8]{inputenc}

% This is not strictly necessary, and may be commented out,
% but it will improve the layout of the manuscript,
% and will typically save some space.
\usepackage{microtype}

% This is also not strictly necessary, and may be commented out.
% However, it will improve the aesthetics of text in
% the typewriter font.
\usepackage{inconsolata}

%Including images in your LaTeX document requires adding
%additional package(s)
\usepackage{graphicx} % Required for including images
\usepackage{luatex85}
\pdfimageresolution=300
\usepackage{threeparttable}
\usepackage{amsmath}
\usepackage{amsfonts}
\usepackage{bm}
\usepackage{subcaption}

\usepackage{inconsolata}
\usepackage{algorithm}
\usepackage{algorithmic}
%\usepackage{tcolorbox} % Load tcolorbox with options
\usepackage{xcolor}
\usepackage{tcolorbox}
% \usepackage{colortbl} % Required for cell coloring
\tcbuselibrary{raster} % Load the raster library for tcolorbox
\definecolor{darkblue}{rgb}{0.04, 0.1, 0.35}
\definecolor{lightblue}{rgb}{0.2, 0.6, 0.9}
\definecolor{lightred}{rgb}{0.9, 0.4, 0.4}
\definecolor{lightyellow}{rgb}{0.8, 0.8, 0.35}
\definecolor{lightpurple}{rgb}{0.6, 0.5, 0.7}
\definecolor{lightorange}{rgb}{1.0, 0.5, 0.0}
\definecolor{lightgrey}{rgb}{0.5, 0.5, 0.5}
\usepackage{booktabs} % For professional looking tables
\makeatletter
\newcommand*\bigcdot{\mathpalette\bigcdot@{.5}}
\newcommand*\bigcdot@[2]{\mathbin{\vcenter{\hbox{\scalebox{#2}{$\m@th#1\bullet$}}}}}
\makeatother
% if you use cleveref..
\usepackage[capitalize,noabbrev]{cleveref}

%%%%%%%%%%%%%%%%%%%%%%%%%%%%%%%%
% THEOREMS
%%%%%%%%%%%%%%%%%%%%%%%%%%%%%%%%
\theoremstyle{plain}
\newtheorem{theorem}{Theorem}[section]
\newtheorem{proposition}[theorem]{Proposition}
\newtheorem{lemma}[theorem]{Lemma}
\newtheorem{corollary}[theorem]{Corollary}
\theoremstyle{definition}
\newtheorem{definition}[theorem]{Definition}
\newtheorem{assumption}[theorem]{Assumption}
\theoremstyle{remark}
\newtheorem{remark}[theorem]{Remark}


\newcommand{\bE}{\mathbb{E}}
\newcommand{\bP}{\mathbb{P}}
\newcommand{\cF}{\mathcal{F}}
\newcommand{\cT}{\mathcal{T}}
\newcommand{\act}{$\mathcal{A}\mathcal{C}\mathcal{T}$}
\newcommand{\eps}{\epsilon}
\newcommand{\ub}{^{(b)}}
\newcommand{\tr}{_{tr.}}
\newcommand{\up}{^{\prime}}
\newcommand{\te}{\theta}
\newcommand{\id}{\mathbbm{1}}
\newcommand{\cB}{\mathcal{B}}
\newcommand{\hp}{\mathcal{H}\mathcal{P}}
\newcommand{\lp}{\mathcal{L}\mathcal{P}}
\newcommand{\bT}{\mathbb{T}}
\newcommand{\bZ}{\mathbb{Z}}
\newcommand{\cX}{\mathcal{X}}
\newcommand{\cY}{\mathcal{Y}}
\newcommand{\cA}{\mathcal{A}}
\newcommand{\cS}{\mathcal{S}}
\newcommand{\ust}{^{\star}}
\newcommand{\cL}{\mathcal{L}}
\newcommand{\bN}{\mathbb{N}}
\newcommand{\bR}{\mathbb{R}}
\newcommand{\cC}{\mathcal{C}}
\newcommand{\cU}{\mathcal{U}}
\newcommand{\cG}{\mathcal{G}}
\newcommand{\cD}{\mathcal{D}}
\newcommand{\og}{\omega}
\newcommand{\cJ}{\mathcal{J}}
\newcommand{\ep}{\epsilon}
\newcommand{\cE}{\mathcal{E}}
\newcommand{\cM}{\mathcal{M}}
\newcommand{\cH}{\mathcal{H}}
\newcommand{\uc}{^{uc}}
\newcommand{\ut}{^{(\theta)}}
\newcommand{\tht}{_{\theta(t)}}
\newcommand{\cI}{\mathcal{I}}
% Todonotes is useful during development; simply uncomment the next line
%    and comment out the line below the next line to turn off comments
%\usepackage[disable,textsize=tiny]{todonotes}
%\usepackage[textsize=tiny]{todonotes}

% Standard package includes
\usepackage{times}
\usepackage{latexsym}

% For proper rendering and hyphenation of words containing Latin characters (including in bib files)
%\usepackage[T1]{fontenc}
% For Vietnamese characters
% \usepackage[T5]{fontenc}
% See https://www.latex-project.org/help/documentation/encguide.pdf for other character sets

% This assumes your files are encoded as UTF8
%\usepackage[utf8]{inputenc}

% This is not strictly necessary, and may be commented out,
% but it will improve the layout of the manuscript,
% and will typically save some space.
\usepackage{microtype}

% This is also not strictly necessary, and may be commented out.
% However, it will improve the aesthetics of text in
% the typewriter font.
\usepackage{inconsolata}

%Including images in your LaTeX document requires adding
%additional package(s)
\usepackage{graphicx}
\usepackage{tcolorbox}
% If the title and author information does not fit in the area allocated, uncomment the following
%
%\setlength\titlebox{<dim>}
%
% and set <dim> to something 5cm or larger.

\title{ \textsc{FLAG-Trader}: Fusion LLM-Agent with Gradient-based \\Reinforcement Learning for
Financial Trading}

% Author information can be set in various styles:
% For several authors from the same institution:
% \author{Author 1 \and ... \and Author n \\
%         Address line \\ ... \\ Address line}
% if the names do not fit well on one line use
%         Author 1 \\ {\bf Author 2} \\ ... \\ {\bf Author n} \\
% For authors from different institutions:
% \author{Author 1 \\ Address line \\  ... \\ Address line
%         \And  ... \And
%         Author n \\ Address line \\ ... \\ Address line}
% To start a separate ``row'' of authors use \AND, as in
% \author{Author 1 \\ Address line \\  ... \\ Address line
%         \AND
%         Author 2 \\ Address line \\ ... \\ Address line \And
%         Author 3 \\ Address line \\ ... \\ Address line}

\author{%
Guojun Xiong\textsuperscript{1},
Zhiyang Deng\textsuperscript{2},
Keyi Wang\textsuperscript{3},
Yupeng Cao\textsuperscript{2},
Haohang Li\textsuperscript{2},
Yangyang Yu\textsuperscript{2},
\\
\textbf{Xueqing Peng\textsuperscript{7}},
\textbf{Mingquan Lin\textsuperscript{4}},
\textbf{Kaleb E Smith\textsuperscript{5}},
\textbf{Xiao-Yang Liu Yanglet\textsuperscript{3,6}},
\\
\textbf{Jimin Huang\textsuperscript{7}},
\textbf{Sophia Ananiadou\textsuperscript{8}},
\textbf{Qianqian Xie\textsuperscript{7,*}}
\\
\\
\textsuperscript{1}Harvard University,
\textsuperscript{2}Stevens Institute of Technology,
\textsuperscript{3}Columbia University,
\\
\textsuperscript{4}University of Minnesota,
\textsuperscript{5}NVIDIA,
\textsuperscript{6}Rensselaer Polytechnic Institute,
\\
\textsuperscript{7}TheFinAI,
\textsuperscript{8}University of Manchester
\\
\small{
\textbf{\textsuperscript{*}Correspondence:} \href{xqq.sincere@gmail.com}{xqq.sincere@gmail.com} 
}
}


%\author{
%  \textbf{First Author\textsuperscript{1}},
%  \textbf{Second Author\textsuperscript{1,2}},
%  \textbf{Third T. Author\textsuperscript{1}},
%  \textbf{Fourth Author\textsuperscript{1}},
%\\
%  \textbf{Fifth Author\textsuperscript{1,2}},
%  \textbf{Sixth Author\textsuperscript{1}},
%  \textbf{Seventh Author\textsuperscript{1}},
%  \textbf{Eighth Author \textsuperscript{1,2,3,4}},
%\\
%  \textbf{Ninth Author\textsuperscript{1}},
%  \textbf{Tenth Author\textsuperscript{1}},
%  \textbf{Eleventh E. Author\textsuperscript{1,2,3,4,5}},
%  \textbf{Twelfth Author\textsuperscript{1}},
%\\
%  \textbf{Thirteenth Author\textsuperscript{3}},
%  \textbf{Fourteenth F. Author\textsuperscript{2,4}},
%  \textbf{Fifteenth Author\textsuperscript{1}},
%  \textbf{Sixteenth Author\textsuperscript{1}},
%\\
%  \textbf{Seventeenth S. Author\textsuperscript{4,5}},
%  \textbf{Eighteenth Author\textsuperscript{3,4}},
%  \textbf{Nineteenth N. Author\textsuperscript{2,5}},
%  \textbf{Twentieth Author\textsuperscript{1}}
%\\
%\\
%  \textsuperscript{1}Affiliation 1,
%  \textsuperscript{2}Affiliation 2,
%  \textsuperscript{3}Affiliation 3,
%  \textsuperscript{4}Affiliation 4,
%  \textsuperscript{5}Affiliation 5
%\\
%  \small{
%    \textbf{Correspondence:} \href{mailto:email@domain}{email@domain}
%  }
%}

\begin{document}
\maketitle
\begin{abstract}
Large language models (LLMs) fine-tuned on multimodal financial data have demonstrated impressive reasoning capabilities in various financial tasks. However, they often struggle with multi-step, goal-oriented scenarios in interactive financial markets, such as trading, where complex agentic approaches are required to improve decision-making. To address this, we propose \textsc{FLAG-Trader}, a unified architecture integrating linguistic processing (via LLMs) with gradient-driven reinforcement learning (RL) policy optimization, in which a partially fine-tuned LLM acts as the policy network, leveraging pre-trained knowledge while adapting to the financial domain through parameter-efficient fine-tuning.  Through policy gradient optimization driven by trading rewards, our framework not only enhances LLM performance in trading but also improves results on other financial-domain tasks. We present extensive empirical evidence to validate these enhancements.
\end{abstract}

\section{Introduction}%

Decision-making is at the heart of artificial intelligence systems, enabling agents to navigate complex environments, achieve goals, and adapt to changing conditions. Traditional decision-making frameworks often rely on associations or statistical correlations between variables, which can lead to suboptimal outcomes when the underlying causal relationships are ignored \citep{pearl2009causal}. 
The rise of causal inference as a field has provided powerful frameworks and tools to address these challenges, such as structural causal models and potential outcomes frameworks \citep{rubin1978bayesian,pearl2000causality}. 
Unlike traditional methods, \textit{causal decision-making} focuses on identifying and leveraging cause-effect relationships, allowing agents to reason about the consequences of their actions, predict counterfactual scenarios, and optimize decisions in a principled way \citep{spirtes2000causation}. In recent years, numerous decision-making methods based on causal reasoning have been developed, finding applications in diverse fields such as recommender systems \citep{zhou2017large}, clinical trials \citep{durand2018contextual}, finance \citep{bai2024review}, and ride-sharing platforms \citep{wan2021pattern}. Despite these advancements, a fundamental question persists: 

\begin{center}
    \textit{When and why do we need causal modeling in decision-making?}
\end{center} 

% Numerous decision-making methods based on causal reasoning have been developed recently with wide applications 
% %Decision makings based on causal reasoning have been widely applied 
% in a variety of fields, including 
% recommender systems \citep{zhou2017large}, clinical trials \citep{durand2018contextual}, 
% finance \citep{bai2024review}, 
% ride-sharing platforms \citep{wan2021pattern}, and so on. 


 

% At the intersection of these fields, causal decision-making seeks to answer critical questions: How can agents make decisions when causal knowledge is incomplete? How do we integrate learning and reasoning about causality into real-world decision-making systems? What role do interventions, counterfactuals, and observational data play in guiding decisions? 

% Our review is structured as follows: 
 

This question is closely tied to the concept of counterfactual thinking—reasoning about what might have happened under alternative decisions or actions. Counterfactual analysis is crucial in domains where the outcomes of unchosen decisions are challenging, if not impossible, to observe. For instance, a business leader selecting one marketing strategy over another may never fully know the outcome of the unselected option \citep{rubin1974estimating, pearl2009causal}. Similarly, in econometrics, epidemiology, psychology, and social sciences, \textit{the inability to observe counterfactuals directly often necessitates causal approaches} \citep{morgan2015counterfactuals, imbens2015causal}. 
Conversely, non-causal analysis may suffice in scenarios where alternative outcomes are readily determinable. For example, a personal investor's actions may have negligible impact on stock market dynamics, enabling potential outcomes of alternate investment decisions to be inferred from existing stock price time series \citep{angrist2008mostly}. However, even in cases where counterfactual outcomes are theoretically calculable—such as in environments with known models like AlphaGo—exhaustively computing all possible outcomes is computationally infeasible \citep{silver2017mastering, silver2018general}. 
In such scenarios, causal modeling remains advantageous by offering \textit{structured ways to infer outcomes efficiently and make robust decisions}. 


%This perspective not only enhances the interpretability of decisions but also provides a principled framework for addressing uncertainty, guiding actions, and improving performance across a broad range of applications.

% Data-driven decision-making exists before the causal revolution. \textit{So when and why do we need causal modelling in decision-making?} 
% This is closely related to the presence of counterfactuals in many applications. 
% The counterfactual thinking involves considering what would have happened in an alternate scenario where a different decision or action was taken. 
% In many fields, including econometrics, epidemiology, psychology, and social sciences, accessing outcomes from unchosen decisions is often challenging if not impossible. 
% For example, a business leader who selects one marketing strategy over another may never know the outcome of the unselected option. 
% Conversely, non-causal analysis may be adequate in situations where potential outcomes of alternate actions are more readily determinable: for example, the investment of a personal investor may have minimal impact on the market, therefore her counterfactual investment decision's outcomes can still be calculated with the data of stock price time series. 
% However, it is important to note that even when counterfactuals are theoretically calculable, as in environments with known models like AlphaGo, computing all possible outcomes may not be feasible. 
% In such scenarios, a causal perspective  remains beneficial. 


 

% 1. significance of decision making
% 2. role of causal in decision making
% 3. refer to the https://jair.org/index.php/jair/article/view/13428/26917

% Decision makings based on causal reasoning have been widely applied in a variety of fields, including recommender systems \citep{zhou2017large}, clinical trials \citep{durand2018contextual}, 
% business economics scenarios \citep{shen2015portfolio}, 
% ride-sharing platforms \citep{wan2021pattern}, and so on. 
% However, most existing works primarily assume either sophisticated prior knowledge or strong causal models to conduct follow-up decision-making. To make effective and trustworthy decisions, it is critical to have a thorough understanding of the causal connections between actions, environments, and outcomes.

\begin{figure}[!t]
    \centering
    \includegraphics[width = .75\linewidth]{Figure/3Steps_V2.png}
    \caption{Workflow of the \acrlong{CDM}. $f_1$, $f_2$, and $f_3$ represent the impact sizes of the directed edges. Variables enclosed in solid circles are observed, while those in dashed circles are actionable.}\label{fig:cdm}
\end{figure}


Most existing works primarily assume either sophisticated prior knowledge or strong causal models to conduct follow-up decision-making. To make effective and trustworthy decisions, it is critical to have a thorough understanding of the causal relationships among actions, environments, and outcomes. This review synthesizes the current state of research in \acrfull{CDM}, providing an overview of foundational concepts, recent advancements, and practical applications. Specifically, this work discusses the connections of \textbf{three primary components of decision-making} through a causal lens: 1) discovering causal relationships through \textit{\acrfull{CSL}}, 2) understanding the impacts of these relationships through \textit{\acrfull{CEL}}, and 3) applying the knowledge gained from the first two aspects to decision making via \textit{\acrfull{CPL}}. 

Let $\boldsymbol{S}$ denote the state of the environment, which includes all relevant feature information about the environment the decision-makers interact with, $A$ the action taken, $\pi$ the action policy that determines which action to take, and $R$ the reward observed after taking action $A$. As illustrated in Figure \ref{fig:cdm}, \acrshort{CDM} typically begins with \acrshort{CSL}, which aims to uncover the unknown causal relationships among various variables of interest. Once the causal structure is established, \acrshort{CEL} is used to assess the impact of a specific action on the outcome rewards. To further explore more complex action policies and refine decision-making strategies, \acrshort{CPL} is employed to evaluate a given policy or identify an optimal policy. In practice, it is also common to move directly from \acrshort{CSL} to \acrshort{CPL} without conducting \acrshort{CEL}. Furthermore, \acrshort{CPL} has the potential to improve both \acrshort{CEL} and \acrshort{CSL} by facilitating the development of more effective experimental designs \citep{zhu2019causal,simchi2023multi} or adaptively refining causal structures \citep{sauter2024core}. %However, these are beyond the scope of this paper.

\begin{figure}[!t]
    \centering
    \includegraphics[width = .9\linewidth]{Figure/Table_of_Six_Scenarios_S.png}
    \caption{Common data dependence structures (paradigms) in \acrshort{CDM}. Detailed notations and explanations can be found in Section \ref{sec:paradigms}.}
    \label{Fig:paradigms}
\end{figure}
Building on this framework, decision-making problems discussed in the literature can be further categorized into \textbf{six paradigms}, as summarized in Figure \ref{Fig:paradigms}. These paradigms summarize the common assumptions about data dependencies frequently employed in practice. Paradigms 1-3 describe the data structures in offline learning settings, where data is collected according to an unknown and fixed behavior policy. In contrast, paradigms 4-6 capture the online learning settings, where policies dynamically adapt to newly collected data, enabling continuous policy improvement. These paradigms also reflect different assumptions about state dependencies. The simplest cases, paradigms 1 and 4, assume that all observations are independent, implying no long-term effects of actions on future observations. To account for sequental dependencies, the \acrfull{MDP} framework, summarized in paradigms 2 and 5, assumes Markovian state transition. Specifically, it assumes that given the current state-action pair $(S_t, A_t)$, the next state $S_{t+1}$ and reward $R_t$ are independent of all prior states $\{S_j\}_{j < t}$ and actions $\{A_j\}_{j < t}$. When such independence assumptions do not hold, paradigms 3 and 6 account for scenarios where all historical observations may impact state transitions and rewards. This includes but not limited to researches on \acrfull{POMDP} \citep{hausknecht2015deep, littman2009tutorial}, panel data analysis \citep{hsiao2007panel,hsiao2022analysis}, \acrfull{DTR} with finite stages \citep{chakraborty2014dynamic, chakraborty2013statistical}. 

Each \acrshort{CDM} task has been studied under different paradigms, with \acrshort{CSL} extensively explored within paradigm 1. \acrshort{CEL} and offline \acrshort{CPL} encompass paradigms 1-3, while online \acrshort{CPL} spans paradigms 4-6. By organizing the discussion around these three tasks and six paradigms, this review aims to provide a cohesive framework for understanding the field of \acrlong{CDM} across diverse tasks and data structures.

%Recognizing the importance of long-term effects in decision-making

%Further discussions on these paradigms and their connections to various causal decision-making problems are provided in Section \ref{sec:paradigms}.


\textbf{Contribution.} In this paper, we conduct a comprehensive survey of \acrshort{CDM}. 
Our contributions are as follows. 
\begin{itemize}
    \item We for the first time organize the related causal decision-making areas into three tasks and six paradigms, connecting previously disconnected areas (including economics, statistics, machine learning, and reinforcement learning) using a consistent language. For each paradigm and task, we provide a few taxonomies to establish a unified view of the recent literature.
    \item We provide a comprehensive overview of \acrshort{CDM}, covering all three major tasks and six classic problem structures, addressing gaps in existing reviews that either focus narrowly on specific tasks or paradigms or overlook the connection between decision-making and causality (detailed in Section \ref{sec::related_work}).
    %\item We outline three key challenges that emerge when utilizing CDM in practice. Moreover, we delve into a comprehensive discussion on the recent advancements and progress made in addressing these challenges. We also suggest six future directions for these problems.
    \item We provide real-world examples to illustrate the critical role of causality in decision-making and to reveal how \acrshort{CSL}, \acrshort{CEL} and \acrshort{CPL} are inherently interconnected in daily applications, often without explicit recognition.
    \item We are actively maintaining and expanding a GitHub repository and online book, providing detailed explanations of key methods reviewed in this paper, along with a code package and demos to support their implementation, with URL: \url{https://causaldm.github.io/Causal-Decision-Making}.
\end{itemize}
% Our review is structured as follows: 


%%%%%%%%%%%%%%%%%%%%%%%%%%%%%%%%%%
%  causal helps over "Correlational analysis"
%Correlational analysis, though widely used in various fields, has inherent limitations, particularly when it comes to decision-making. While it identifies relationships between variables, it fails to establish causality, often leading to misinterpretations and misguided decisions. For example, the positive correlation between ice cream sales and drowning incidents is a classic example of how correlational data can be misleading, as both are influenced by a third factor, temperature, rather than causing each other. Such spurious correlations, due to oversight of confounding variables, underscore the necessity of causal modeling in decision making. Causal models excel where correlational analysis falls short, offering predictive power and a deeper understanding of underlying mechanisms. They enable us to predict the outcomes of interventions, even under untested conditions, and provide insights into the processes leading to these outcomes, thereby informing more effective strategies. Moreover, causal models are good at generalizing findings across different contexts, a capability often limited in purely correlational studies. 

%  causal helps in causal RL 
%From another complementary angle, although causal concepts have traditionally not been explicitly incorporated in fields like online bandits \citep{lattimore2020bandit} and \acrfull{RL} \citep{sutton2018reinforcement}, much of the literature in these areas implicitly relies on basic assumptions outlined in Section \ref{sec:prelim_assump} to utilize observed data in place of potential outcomes in their analyses, and there is also a growing recognition of the significance of the causal perspective \citep{lattimore2016causal, zeng2023survey} in these areas. 
% \textbf{Read causal RL survey and summarize. } However, by integrating causal concepts and leverging existing methodologies, we open up possibilities for developing more robust models to remove spurious correlation and selection bias \citep{xu2023instrumental, forney2017counterfactual}, designing more sample-efficient \citep{sontakke2021causal, seitzer2021causal} and robust \citep{dimakopoulou2019balanced, ye2023doubly} algorithms, and improving the generalizability \citep{zhang2017transfer, eghbal2021learning}, explanability \citep{foerster2018counterfactual, herlau2022reinforcement}, and fairness \citep{zhang2018fairness,huang2022achieving,balakrishnan2022scales} of these methods. %, and safety \cite{hart2020counterfactual}

%


%\subsection{Paper Structure}
The remainder of this paper is organized as follows: Section \ref{sec::related_work} provides an overview of related survey papers. Section \ref{sec:preliminary} introduces the foundational concepts, assumptions, and notations that form the foundation for the subsequent discussions. In Section \ref{sec:3task6paradigm}, we offer a detailed introduction to the three key tasks and six learning paradigms in \acrshort{CDM}. Sections \ref{Sec:CSL} through \ref{sec:Online CPL} form the core of the paper, with each section dedicated to a specific topic within \acrshort{CDM}: \acrshort{CSL}, \acrshort{CEL}, Offline \acrshort{CPL}, and Online \acrshort{CPL}, respectively. Section \ref{sec:assump_violated} then explores extensions needed when standard causal assumptions are violated. To illustrate the practical application of the \acrshort{CDM} framework, Section \ref{sec:real_data} presents two real-world case studies. Finally, Section \ref{sec:conclusion} concludes the paper with a summary of our contributions and a discussion of additional research directions that are actively being explored.


%%%
\section{Related Work}
\label{sec:relatedwork}

\subsection{Current AI Tools for Social Service}
\label{subsec:relatedtools}
% the title I feel is quite broad

Harnessing technology for social good has always been a grand challenge in social service \cite{berzin_practice_2015}. As early as the 90s, artificial neural networks and predictive models have been employed as tools for risk assessments, decision-making, and workload management in sectors like child protective services and mental health treatment \cite{fluke_artificial_1989, patterson_application_1999}. The recent rise of generative AI is poised to further advance social service practice, facilitating the automation of administrative tasks, streamlining of paperwork and documentation, optimisation of resource allocation, data analysis, and enhancing client support and interventions \cite{fernando_integration_2023, perron_generative_2023}.

Today, AI solutions are increasingly being deployed in both policy and practice \cite{goldkind_social_2021, hodgson_problematising_2022}. In clinical social work, AI has been used for risk assessments, crisis management, public health initiatives, and education and training for practitioners \cite{asakura_call_2020, gillingham2019can, jacobi_functions_2023, liedgren_use_2016, molala_social_2023, rice_piloting_2018, tambe_artificial_2018}. AI has also been employed for mental health support and therapeutic interventions, with conversational agents serving as on-demand virtual counsellors to provide clinical care and support \cite{lisetti_i_2013, reamer_artificial_2023}.
% commercial solutions include Woebot, which simulates therapeutic conversation, and Wysa, an “emotionally intelligent” AI coach, powered by evidenced-based clinical techniques \cite{reamer_artificial_2023}. 
% Non-clinical AI agents like Replika and companion robots can also provide social support and reduce loneliness amongst individuals \cite{ahmed_humanrobot_2024, chaturvedi_social_2023, pani_can_2024, ta_user_2020}.

Present research largely focuses on \textit{\textbf{AI-based decision support tools}} in social service \cite{james_algorithmic_2023, kawakami2022improving}, especially predictive risk models (PRMs) used to predict social service risks and outcomes \cite{gillingham2019can, van2017predicting}, like the Allegheny Family Screening Tool (AFST), which assesses child abuse risk using data from US public systems \cite{chouldechova_case_2018, vaithianathan2017developing}. Elsewhere, researchers have also piloted PRMs to predict social service needs for the homeless using Medicaid data\cite{erickson_automatic_2018, pourat_easy_2023}, and AI-powered algorithms to promote health interventions for at-risk populations, such as HIV testing among Californian homeless \cite{rice_piloting_2018, yadav_maximizing_2017}.

\subsection{Generative AI and Human-AI Collaboration}
\label{subsec:relatedworkhaicollaboration}
Beyond decision-making algorithms and PRMs, advancements in generative AI, such as large language models (LLMs), open new possibilities for human-AI (HAI) collaboration in social services. 
LLMs have been called "revolutionary" \cite{fui2023generative} and a "seismic shift" \cite{cooper2023examining}, offering "content support" \cite{memmert2023towards} by generating realistic and coherent responses to user inputs \cite{cascella2023evaluating}. Their vastly improved capabilities and ubiquity \cite{cooper2023examining} makes them poised to revolutionise work patterns \cite{fui2023generative}. Generative AI is already used in fields like design, writing, music, \cite{han2024teams, suh2021ai, verheijden2023collaborative, dhillon2024shaping, gero2023social} healthcare, and clinical settings \cite{zhang2023generative, yu2023leveraging, biswas2024intelligent}, with promising results. However, the social service sector has been slower in adopting AI \cite{diez2023artificial, kawakami2023training}.

% Yet, the social service sector is one that could perhaps stand to gain the most from AI technologies. As Goldkind \cite{goldkind_social_2021} writes, social service, as a "values-centred profession with a robust code of ethics" (p. 372), is uniquely placed to inform the development of thoughtful algorithmic policy and practice. 
Social service, however, stands to benefit immensely from generative AI. SSPs work in time-poor environments \cite{tiah_can_2024}, often overwhelmed with tedious administrative work \cite{meilvang_working_2023} and large amounts of paperwork and data processing \cite{singer_ai_2023, tiah_can_2024}. 
% As such, workers often work in time-poor environments and are burdened with information overload and administrative tasks \cite{tiah_can_2024, meilvang_working_2023}. 
Generative AI is well-placed to streamline and automate tasks like formatting case notes, formulating treatment plans and writing progress reports, which can free up valuable time for more meaningful work like client engagement and enhance service quality \cite{fernando_integration_2023, perron_generative_2023, tiah_can_2024, thesocialworkaimentor_ai_nodate}. 

Given the immense potential, there has been emerging research interest in HAI collaboration and teamwork in the Human-Computer Interaction and Computer Supported Cooperative Work space \cite{wang_human-human_2020}. HAI collaboration and interaction has been postulated by researchers to contribute to new forms of HAI symbiosis and augmented intelligence, where algorithmic and human agents work in tandem with one another to perform tasks better than they could accomplish alone by augmenting each other's strengths and capabilities  \cite{dave_augmented_2023, jarrahi_artificial_2018}.

However, compared to the focus on AI decision-making and PRM tools, there is scant research on generative AI and HAI collaboration in the social service sector \cite{wykman_artificial_2023}. This study therefore seeks to fill this critical gap by exploring how SSPs use and interact with a novel generative AI tool, helping to expand our understanding of the new opportunities that HAI collaboration can bring to the social service sector.

\subsection{Challenges in AI Use in Social Service}
\label{subsec:relatedworkaiuse}

% Despite the immense potential of AI systems to augment social work practice, there are multiple challenges with integrating such systems into real-life practice. 
Despite its evident benefits, multiple challenges plague the integration of AI and its vast potential into real-life social service practice.
% Numerous studies have investigated the use of PRMs to help practitioners decide on a course of action for their clients. 
When employing algorithmic decision-making systems, practitioners often experience tension in weighing AI suggestions against their own judgement \cite{kawakami2022improving, saxena2021framework}, being uncertain of how far they should rely on the machine. 
% Despite often being instructed to use the tool as part of evaluating a client, 
Workers are often reluctant to fully embrace AI assessments due to its inability to adequately account for the full context of a case \cite{kawakami2022improving, gambrill2001need}, and lack of clarity and transparency on AI systems and limitations \cite{kawakami2022improving}. Brown et al. \cite{brown2019toward} conducted workshops using hypothetical algorithmic tools 
% to understand service providers' comfort levels with using such tools in their work,
and found similar issues with mistrust and perceived unreliability. Furthermore, introducing AI tools can  create new problems of its own, causing confusion and distrust amongst workers \cite{kawakami2022improving}. Such factors are critical barriers to the acceptance and effective use of AI in the sector.

\citeauthor{meilvang_working_2023} (2023) cites the concept of \textit{boundary work}, which explores the delineation between "monotonous" administrative labour and "professional", "knowledge based" work drawing on core competencies of SSPs. While computers have long been used for bureaucratic tasks like client registration, the introduction of decision support systems like PRMs stirred debate over AI "threatening professional discretion and, as such, the profession itself" \cite{meilvang_working_2023}. Such latent concerns arguably drive the resistance to technology adoption described above. Generative AI is only set to further push this boundary, 
% these concerns are only set to grow in tandem with the vast capabilities of generative and other modern AI systems. Compared to the relatively primitive AI systems in past years, perceived as statistical algorithms \cite{brown2019toward} turning preset inputs like client age and behavioural symptoms \cite{vaithianathan2017developing} into simple numerical outputs indicating various risk scores, modern AI systems are vastly more capable: LLMs 
with its ability to formulate detailed reports and assessments that encroach upon the "core" work of SSPs.
% accept unrestricted and unstructured inputs and return a range of verbose and detailed evaluations according to the user's instructions. 
Introducing these systems exacerbate previously-raised issues such as understanding the limitations and possibilities of AI systems \cite{kawakami2022improving} and risk of overreliance on AI \cite{van2023chatgpt}, and requires a re-examination of where users fall on the algorithmic aversion-bias scale \cite{brown2019toward} and how they detect and react to algorithmic failings \cite{de2020case}. We address these critical issues through an empirical, on-the-ground study that to our knowledge is the first of its kind since the new wave of generative AI.

% W 

% Yet, to date, we have limited knowledge on the real-world impacts and implications of human-AI collaboration, and few studies have investigated practitioners’ experiences working with and using such AI systems in practice, especially within the social work context \cite{kawakami2022improving}. A small number of studies have explored practitioner perspectives on the use of AI in social work, including Kawakami et al. \cite{kawakami2022improving}, who interviewed social workers on their experiences using the AFST; Stapleton et al. \cite{stapleton_imagining_2022}, who conducted design workshops with caseworkers on the use of PRMs in child welfare; and Wassal et al. \cite{wassal_reimagining_2024}, who interviewed UK social work professionals on the use of AI. A common thread from all these studies was a general disregard for the context and users, with many practitioners criticising the failure of past AI tools arising from the lack of participation and involvement of social workers and actual users of such systems in the design and development of algorithmic systems \cite{wassal_reimagining_2024}. Similarly, in a scoping review done on decision-support algorithms in social work, Jacobi \& Christensen \cite{jacobi_functions_2023} reported that the majority of studies reveal limited bottom-up involvement and interaction between social workers, researchers and developers, and that algorithms were rarely developed with consideration of the perspective of social workers.
% so the \cite{yang_unremarkable_2019} and \cite{holten_moller_shifting_2020} are not real-world impacts? real-world means to hear practitioner's voice? I feel this is quite important but i didnt get this point in intro!

% why mentioning 'which have largely focused on existing ADS tools (e.g., AFST)'? i can see our strength is more localized, but without basic knowledge of social work i didnt get what's the 'departure' here orz
% the paragraph is great! do we need to also add one in line 20 21?

\subsection{Designing AI for Social Service through Participatory Design}
\label{subsec:relatedworkpd}
% i think it's important! but maybe not a whole subsection? but i feel the strong connection with practitioners is indeed one of our novelties and need to highlight it, also in intro maybe
% Participatory design (PD) has long been used extensively in HCI \cite{muller1993participatory}, to both design effective solutions for a specific community and gain a deep understanding of that community. Of particular interest here is the rich body of literature on PD in the field of healthcare \cite{donetto2015experience}, which in this regard shares many similarities and concerns with social work. PD has created effective health improvement apps \cite{ryu2017impact}, 

% PD offers researchers the chance to gather detailed user requirements \cite{ryu2017impact}...

Participatory design (PD) is a staple of HCI research \cite{muller1993participatory}, facilitating the design of effective solutions for a specific community while gaining a deep understanding of its stakeholders. The focus in PD of valuing the opinions and perspectives of users as experts \cite{schuler_participatory_1993} 
% In recent years, the tech and social work sectors have awakened to the importance of involving real users in designing and implementing digital technologies, developing human-centred design processes to iteratively design products or technologies through user feedback 
has gained importance in recent years \cite{storer2023reimagining}. Responding to criticisms and failures of past AI tools that have been implemented without adequate involvement and input from actual users, HCI scholars have adopted PD approaches to design predictive tools to better support human decision-making \cite{lehtiniemi_contextual_2023}.
% ; accordingly, in social service, a line of research has begun studying and designing for human-AI collaboration with real-world users (e.g. \cite{holten_moller_shifting_2020, kawakami2022improving, yang_unremarkable_2019}).
Section \ref{subsec:relatedworkaiuse} shows a clear need to better understand SSP perspectives when designing and implementing AI tools in the social sector. 
Yet, PD research in this area has been limited. \citeauthor{yang2019unremarkable} (2019), through field evaluation with clinicians, investigated reasons behind the failure of previous AI-powered decision support tools, allowing them to design a new-and-improved AI decision-support tool that was better aligned with healthcare workers’ workflows. Similarly, \citeauthor{holten_moller_shifting_2020} (2020) ran PD workshops with caseworkers, data scientists and developers in public service systems to identify the expectations and needs that different stakeholders had in using ADS tools.

% Indeed, it is as Wise \cite{wise_intelligent_1998} noted so many years ago on the rise of intelligent agents: “it is perhaps when technologies are new, when their (and our) movements, habits and attitudes seem most awkward and therefore still at the forefront of our thoughts that they are easiest to analyse” (p. 411). 
Building upon this existing body of work, we thus conduct a study to co-design an AI tool \textit{for} and \textit{with} SSPs through participatory workshops and focus group discussions. In the process, we revisit many of the issues mentioned in Section \ref{subsec:relatedworkaiuse}, but in the context of novel generative AI systems, which are fundamentally different from most historical examples of automation technologies \cite{noy2023experimental}. This valuable empirical inquiry occurs at an opportune time when varied expectations about this nascent technology abound \cite{lehtiniemi_contextual_2023}, allowing us to understand how SSPs incorporate AI into their practice, and what AI can (or cannot) do for them. In doing so, we aim to uncover new theoretical and practical insights on what AI can bring to the social service sector, and formulate design implications for developing AI technologies that SSPs find truly meaningful and useful.
% , and drive future technological innovations to transform the social service sector not just within [our country], but also on a global scale.

 % with an on-the-ground study using a real prototype system that reflects the state of AI in current society. With the presumption that AI will continue to be used in social work given the great benefits it brings, we address the pressing need to investigate these issues to ensure that any potential AI systems are designed and implemented in a responsible and effective manner.

% Building upon these works, this study therefore seeks to adopt a participatory design methodology to investigate social workers’ perspectives and attitudes on AI and human-AI collaboration in their social work practice, thus contributing to the nascent body of practitioner-centred HCI research on the use of AI in social work. Yet, in a departure from prior work, which have largely focused on existing ADS tools (e.g., AFST) and were situated in a Western context, our paper also aims to expand the scope by piloting a novel generative AI tool that was designed and developed by the researchers in partnership with a social service agency based in Singapore, with aims of generating more insights on wider use cases of AI beyond what has been previously studied.

% i may think 'While the current lacunae of research on applications of AI in social work may appear to be a limitation, it simultaneously presents an exciting opportunity for further research and exploration \cite{dey_unleashing_2023},' this point is already convincing enough, not sure if we need to quote here
% I like this end! it's a good transition to our study design, do we need to mention the localization in intro as well? like we target at singapore

% Given the increasing prominence and acceptance of AI in modern society, 

% These increased capabilities vastly exacerbate the issues already present with a simpler tool like the AFST: the boundaries and limitations of an LLM system are significantly more difficult to understand and its possible use cases are exponentially greater in scope. 

% Put this in discussion section instead?
% Kawakami et al's work "highlights the importance of studying how collaborative decision-making... impacts how people rely upon and make sense of AI models," They conclude by recommending designing tools that "support workers in understanding the boundaries of [an AI system's] capabilities", and implementing design procedures that "support open cultures for critical discussion around AI decision making". The authors outline critical challenges of implementing AI systems, elucidating factors that may hinder their effectiveness and even negatively affect operations within the organisation.


% Is this needed?:
% talk about the strengths of PD in eliciting user viewpoints and knowledge, in particular when it is a field that is novel or where a certain system has not been used or developed or tested before
%%%
\section{Problem Statement}\label{Sec:Problem Formulation}

We define the financial decision-making process as a finite horizon partially observable Markov decision process (MDP) with time index $\{0,\cdots,T\}$, represented by the tuple:
$\mathcal{M} = (\mathcal{S}, \mathcal{A}, \mathcal{T}, R, \gamma),$
where each component is described in detail below.

\textbf{State.} The state space $\cS=\cX\times\cY$ consists of two components: market observations and trading account balance, i.e.,
$s_t = (m_t, b_t) \in \mathcal{S}.$ Specifically,
     $ m_t = (P_t, N_t) \in \cX$ represents the \textit{market observation process}, includingstock price $P_t$ at time $t$, and financial news sentiment or macroeconomic indicators $N_t$;
   $ b_t = (C_t, H_t) \in \mathcal{Y}$ represents the \textit{trading account balance}, including available cash $C_t$ at time $t$, and number of stock shares $H_t$.

\textbf{Action.} The agent chooses from a discrete set of trading actions
$\mathcal{A} = \{\texttt{Sell}: -1, \texttt{Hold}: 0, \texttt{Buy}: 1\},$
where $a_t=-1$ denotes selling all holdings (liquidate the portfolio),
$a_t=0$ denotes holding (no trading action), and
$a_t=1$ represents buying with all available cash (convert all cash into stocks).


 \textbf{State Transition.} The state transition dynamics are governed by a stochastic process
$s_{t+1} \sim \cT(\cdot | s_t, a_t).$
% Specifically, the stock price $P_t$ follows a stochastic process:
%     $
%     P_{t+1} = P_t + f_m(m_t, a_t) + \xi_t$ with $\quad \xi_t \sim \mathcal{N}(0, \sigma^2).$
The trading account evolves according to the following equations:
    \begin{itemize}
        \item If \texttt{Sell}:
        $        C_{t+1} = C_t + H_t P_{t+1}, \quad H_{t+1} = 0.
        $
        \item If \texttt{Hold}:
        $ C_{t+1} = C_t, \quad H_{t+1} = H_t.$
        \item If \texttt{Buy}:
       $        C_{t+1} = 0, \quad H_{t+1} = H_t + \frac{C_t}{P_{t+1}}.$ 
    \end{itemize}

\textbf{Reward.} 
The agent receives a reward  based on the daily trading profit \& loss (PnLs):
\begin{align*}
    R(s_t, a_t) = SR_t-SR_{t-1},
\end{align*}
where $SR_t$ denotes the Sharpe ratio at day $t$, computed by using the historical PnL from time 0 to time $t$. Moreover, PnL at time $t$ is calculated as
\begin{align*}
pnl_t:=(C_t-C_{t-1})+(H_tP_t-H_{t-1}P_{t-1}).   
\end{align*}
Then, the Sharpe ratio $SR_t$ at time $t$ can be calculated as:
\begin{align}
    \label{eq:sharpe}
    SR_t:=\frac{\mathbb{E}[pnl_1,\cdots,pnl_t]-r_f}{\sigma[pnl_1,\cdots,pnl_t]},
\end{align}
where $\mathbb{E}[pnl_1,\cdots,pnl_t]$ is the sample average of daily PnL up to time $t$, $r_f$ is the risk-free rate, and $\sigma[pnl_1,\cdots,pnl_t]$ is the sample standard deviation of daily PnL up to time $t$.

 
The goal is to find an admissible policy $\pi$ to maximize the expected value of cumulative discounted reward, i.e., 
\begin{align}\label{eq:global_obj}
   \max_{\pi} V^{\pi}(s) = \mathop{\mathbb{E}}\limits_{\substack{s_0=s,a_t\sim \pi(\cdot|s_t)\\ s_{t+1}\sim \cT(\cdot|s_t,a_t)}} \left[ \sum_{t=0}^T \gamma^t R_t \right],
\end{align}
where $R_t$ is a shortened version $R(s_t, a_t)$ and  $ \gamma \in (0,1]$ is the discount factor controlling the importance of future rewards.



Our goal is to train an LLM agent parameterized by $\theta$ to find the optimized policy $\pi_\theta$ for \eqref{eq:global_obj}, i.e.,
\begin{align}
a_t\sim\pi_\theta(\cdot|s_t)=\text{LLM}(\texttt{lang}(s_t);\theta),
\end{align}
where $\texttt{lang}(s_t)$ are the prompts generated by converting state $s_t$ into structured text. The proposed pipeline is illustrated in Figure \ref{fig:FinRL_LLM}. 




% Formally, we model a financial decision-making process as infinite horizon POMDP with time index $\mathbb{T}=\{0,1,2,\cdots, \infty\}$ and discount factor $\gamma\in(0,1]$. This POMDP contains: (1) a state space $\mathcal{S}:=\mathcal{X}\times\mathcal{Y}$ where $\mathcal{X}$ is the observable component and $\mathcal{Y}$ is unobservable component of the financial market; (2) the action space of the agent is $\mathcal{A}$, which is modeled as $\{\textit{``Buy",~``Sell",~``Hold"}\}$; (3) the reward function $R(o,b,a):\mathcal{X}\times\mathcal{Y}\times\mathcal{A}\to\mathbb{R}$ uses daily profit \& loss (PnL) as the output; (4) the observation process $\{o_t\}_{t\in\mathbb{T}}\subseteq\mathcal{X}$ is a multi-dimensional process (5) the reflection process $\{b_t\}_{t\in\mathbb{T}}\subseteq\mathcal{Y}$ represents the agent's self-reflection, which is updated from $b_t$ to $b_{t+1}$ on daily basis \cite{griffiths2023bayes}; (6) the action $a_t\sim\pi_\theta(\cdot|\text{prompt})$ represents the way to make investment decision driven by the language conditioned policy $\pi_\theta$ parameterized by $\theta$. By denoting daily profit \& loss (PnLs) by $r_t(s_t,a_t)=R(s_t:=(o_t, b_t), a_t)$, the optimization objective is to maximize the cumulative reward over a sequence of trading decisions, defined as
% \begin{align}
%    J(\theta) = \mathbb{E}_{(s_t, a_t)\sim \pi_\theta} \left[ \sum_{t=0}^\infty \gamma^t r_t(s_t,a_t) \right].
% \end{align}
% To improve the policy, the gradient of the policy is optimized using the policy gradient loss:
% \begin{align}
%     L(\pi_\theta) = -\mathbb{E}_t \left[ \log \pi_\theta(a_t | s_t) A_t \right],
% \end{align}
% where $A_t$ represents the advantage function evaluating the relative value of choosing a specific action.




% The objective of this research is to fine-tune large language models (LLMs) for financial decision-making in sequential trading environments. The model is designed to generate interpretable, full-sentence outputs such as \textit{"buy AAPL 100 shares,"} which are subsequently mapped into a simplified action space of \textbf{``buy,'' ``sell,'' or ``hold.''} Each decision is based on the current state, represented as $s_t = \{m_t, p_t, h_t\} $, where $m_t$ denotes market indicators, $p_t$ represents historical price information, and $h_t$ reflects the portfolio status.






%%%
\section{\textsc{FLAG-Trader}}
To tackle the challenge of directly fine-tuning an LLM for both alignment and decision-making, we introduce \textsc{FLAG-Trader}, a fused LLM-agent and RL framework for financial stock trading. In \textsc{FLAG-Trader}, a partially fine-tuned LLM serves as the policy network, leveraging its pre-trained knowledge while adapting to the financial domain through parameter-efficient fine-tuning, as shown in Figure \ref{fig:AC-network}. The model processes financial information using a textual state representation, allowing it to interpret and respond to market conditions effectively. Instead of fine-tuning the entire network, only a subset of the LLM’s parameters is trained, striking a balance between adaptation and knowledge retention. In the following, we will present the prompt input design and the detailed architecture of \textsc{FLAG-Trader}.







\subsection{Prompt Input Design}

The first stage of the pipeline involves designing a robust and informative prompt, denoted as \texttt{lang}($s_t$), which is constructed based on the current state 
$s_t$ to guide the LLM in making effective trading decisions. The prompt is carefully structured to encapsulate essential elements that provide context and ensure coherent, actionable outputs. It consists of four key components: a \emph{task description}, which defines the financial trading objective, outlining the problem domain and expected actions; a \emph{legible action space}, specifying the available trading decisions (\texttt{Sell,'' Hold,'' ``Buy''}); a \emph{current state representation}, incorporating market indicators, historical price data, and portfolio status to contextualize the decision-making process; and an \emph{output action}, which generates an executable trading decision. This structured prompt ensures that the LLM receives comprehensive input, enabling it to produce well-informed and actionable trading strategies, as illustrated in Figure \ref{fig:prompt}.




\begin{figure}[h]
    \centering
    \includegraphics[width=0.99\linewidth]{figures/FinRL_Prompt.pdf}
    \caption{The format of input prompt. It contains the task description, the legible action set, the current state description, and the output action format.}
    \label{fig:prompt}
    \vspace{-0.2cm}
\end{figure}

\subsection{\textsc{FLAG-Trader} Architecture}

To incorporate parameter-efficient fine-tuning into the policy gradient framework, we partition the intrinsic parameters of the LLM into two distinct components: the frozen parameters inherited from pretraining, denoted as 
$\theta_{\texttt{forzen}}$, and the trainable parameters, denoted as 
$\theta_{\texttt{train}}$. This separation allows the model to retain general language understanding while adapting to financial decision-making with minimal computational overhead.
Building upon this LLM structure, we introduce a policy network and a value network, both of which leverage the trainable top layers of the LLM for domain adaptation while sharing the frozen layers for knowledge retention. The overall architecture is illustrated in Figure \ref{fig:AC-network}.




\subsubsection{Policy Network Design}
The policy network is responsible for generating an optimal action distribution over the trading decision space $\cA$, conditioned on the observed market state. It consists of three main components:



\emph{State Encoding.}
To effectively process financial data using the LLM, the numerical market state 
$s$ is first converted into structured text using a predefined template\footnote{To simplify notation, we use \texttt{lang}($s_t$) to represent both the state encoding and the prompt, acknowledging this slight abuse of notation for convenience.}
\begin{align}
\texttt{lang}(s) = \text{"Price: \$}p\text{, Vol: }v\text{, RSI: }r\text{,..."}.
\label{eq:template}
\end{align}
This transformation enables the model to leverage the LLM’s textual reasoning capabilities, allowing it to understand and infer trading decisions in a structured, language-based manner.

\emph{LLM Processing.} The tokenized text representation of the state is then passed through the LLM backbone, which consists of:
1) \textbf{Frozen layers} (preserve general knowledge):
Token embeddings $E = \text{Embed}(\texttt{lang}(s))$ pass through LLM frozen layers, i.e.,
\begin{equation}
h^{(1)} = \text{LLM}_{1:N}(E;\theta_{\texttt{frozen}}).
\end{equation}
These layers preserve general knowledge acquired from pretraining, ensuring that the model maintains a strong foundational understanding of language and reasoning.
2) \textbf{Trainable layers} (domain-specific adaptation): The output from the frozen layers is then passed through the trainable layers, which are fine-tuned specifically for financial decision-making, i.e.,
\begin{equation}
h^{(2)} = \text{LLM}_{N+1:N+M}(h^{(1)};\theta_{\text{train}}).
\label{eq:trainable_layer}
\end{equation}
This structure enables efficient adaptation to the financial domain without modifying the entire LLM, significantly reducing training cost while maintaining performance.

\emph{Policy Head.} 
Finally, the processed representation is fed into the policy head, which outputs a probability distribution over the available trading actions according to
\begin{equation}
\texttt{logits} = \textsc{Policy\_Net}(h^{(2)},\theta_P)\in\mathbb{R}^{|\mathcal{A}|},
\label{eq:policy_dist}
\end{equation}
where $\theta_P$ is the parameter of \textsc{Policy\_Net},
with action masking for invalid trades: 
\begin{equation} 
\pi(a|s)\! =\!\! \begin{cases}
0 & a \notin \mathcal{A},\\
\frac{\exp(\texttt{logits}(a))}{\sum_{a'\in \mathcal{A}}\exp(\texttt{logits}(a^\prime))} & \text{otherwise}.
\end{cases}
\label{eq:masking}
\end{equation}
This ensures that actions outside the valid set 
$\cA$ (e.g., selling when no stocks are held) have zero probability, preventing invalid execution.

\subsubsection{Value Network Design}
The value network serves as the critic in the RL framework, estimating the expected return of a given state to guide the policy network's optimization. To efficiently leverage the shared LLM representation, the value network shares the same backbone as the policy network, processing the textual state representation through the frozen and trainable layers  \eqref{eq:template}–\eqref{eq:trainable_layer}. This design ensures efficient parameter utilization while maintaining a structured and informative state encoding. 
After passing through the LLM processing layers, the output 
$h^{(2)}$  is fed into a separate value prediction head, which maps the extracted features to a scalar value estimation:
\begin{equation}
V(s) = \textsc{Value\_Net}(h^{(2)}, \theta_V)\in\mathbb{R}^{1},
\label{eq:value_pred}
\end{equation}
where $\theta_V$ is the parameter of \textsc{Value\_Net}.


\subsection{Online Policy Gradient Learning}
The policy and value networks in \textsc{FLAG-Trader} are trained using an online policy gradient approach, ensuring that the model continuously refines its decision-making ability. The learning process follows an iterative cycle of state observation, action generation, reward evaluation, and policy optimization. The parameters of the model are updated using stochastic gradient descent (SGD), leveraging the computed policy and value losses to drive optimization.

At each training step, we define two key loss functions, i.e.,
\emph{policy loss} 
$\mathcal{L}_P$: measures how well the policy network aligns with the expected advantage-weighted log probability of actions; 
\emph{value loss} $\mathcal{L}_V$: ensures that the value network accurately estimates the expected return.

\begin{remark}
The definitions of \emph{policy loss} and \emph{value loss} may vary across different actor-critic (AC) algorithms. Here, we present a general formulation for clarity and ease of expression. Notably, our framework is designed to be flexible and adaptable, making it compatible with a wide range of AC algorithms.
\end{remark}



Based on these loss functions, the model updates the respective network parameters using backpropagation as follows. 

\textbf{Update Policy Head.} 
The policy network parameters 
 $\theta_P$ are updated via SGD to minimize the \emph{policy loss} $\mathcal{L}_P$ 
\begin{align}\label{eq:P}
    \theta_P\leftarrow\theta_P-\eta \nabla_{\theta_P}\mathcal{L}_P,
\end{align}
where $\eta$ is the learning rate for updating policy head $\theta_P$.

\textbf{Update Value Head.} The value network parameters 
$\theta_V$ are optimized via SGD to minimize the temporal difference (TD) error over policy loss $\mathcal{L}_V$
\begin{align}\label{eq:V}
    \theta_V\leftarrow\theta_V-\eta \nabla_{\theta_V}\mathcal{L}_V.
\end{align}

\textbf{Update Trainable LLM Layers.}
The trainable LLM parameters 
$\theta_{\texttt{train}}$ are updated via SGD jointly based on both the policy and value losses, i.e., $\mathcal{L}_P$ and $\mathcal{L}_V$, allowing the shared LLM representation to align with optimal decision-making:  
\begin{align}\label{eq:train}
    \theta_{\texttt{train}}\leftarrow\theta_{\texttt{train}}-\beta \nabla_{\theta_{\texttt{train}}}(\mathcal{L}_P+\mathcal{L}_V),
\end{align}
where $\beta$ is the learning rate for LLM parameter $\theta_{\texttt{train}}$. 

The updates in \eqref{eq:P}–\eqref{eq:train} are performed iteratively until the stopping criteria are met, as outlined in Algorithm \ref{alg:1}. This iterative learning process effectively balances exploration and exploitation, enhancing policy performance while maintaining stability. To mitigate overfitting and policy divergence, we employ Proximal Policy Optimization (PPO), which constrains updates by limiting the divergence from previous policies, ensuring more controlled and reliable learning. The detailed procedure of how to compute \emph{policy loss} $\cL_P$ and \emph{value loss} $\cL_P$ can be found in Appendix \ref{sec:Appendix_A}.
 






\begin{algorithm}[ht]
\caption{\textsc{FLAG-Trader}}
\begin{algorithmic}[1]
\STATE \textbf{Require:} Pre-trained LLM with parameter  $\theta:=(\theta_{\texttt{frozen}}, \theta_{\texttt{train}})$, environment dynamics $\cT$, reward function $\mathcal{R};$
\STATE Initialize policy network $\theta_P$ and value network $\theta_V$ with shared LLM trainable layers $\theta_{\texttt{train}}$;
\STATE Initialize experience replay buffer $B \leftarrow \emptyset$

\FOR{iteration $t=1,2,\ldots$,}
    \STATE Fetch the current state $s_t$ from the environment and construct an input prompt \texttt{lang}($s_t$);
    \STATE Pass prompt \texttt{lang}($s_t$) through LLM;
    \STATE \textsc{Policy\_Net} outputs $a_t$ from action space $\{\texttt{``buy,'' ``sell,'' ``hold''}\}$ based on \eqref{eq:masking};
    \STATE Execute action $a_t$ in the environment and observe reward $r(s_t, a_t)$ and transition to new state $s_{t+1}$;
    \STATE Store experience tuple $(s_t, a_t, r_t, s_{t+1})$ in replay buffer $B$;
    
    \IF{ $t \mod \tau=0$}
        \STATE Update policy head $\theta_P$ according to \eqref{eq:P};
        \STATE Update value head $\theta_V$ according to \eqref{eq:V};
        \STATE Update the trainable LLM layers $\theta_{\texttt{train}}$ according to \eqref{eq:train}.
        
    \ENDIF
\ENDFOR

\STATE \textbf{Return:} Fine-tuned \textsc{Policy\_Net}($\theta_P$).
\end{algorithmic}
\label{alg:1}
\end{algorithm}



%%%
\section{Experiments}
\label{sec: experiments}

\subsection{Experimental Setup}
\label{sec: experimental_setup}
\begin{figure}[t]
\centering \includegraphics[width=\linewidth]{figure_2.png} \caption{The handheld platform configuration, including the radar, IMU, and onboard computer. The experiments are conducted in a room equipped with a motion capture system to obtain accurate ground truth.}
\label{fig2}
\end{figure}

We conduct experiments using three datasets, comprising a total of 15 sequences. One is our self-collected dataset, captured with a handheld platform as shown in Fig.~\ref{fig2}, while the other two are public radar datasets: ICINS2021~\cite{9470842}, and ColoRadar~\cite{kramer2022coloradar}. The sensors on our platform include a 4D FMCW radar, specifically the Texas Instruments AWR1843BOOST, and an Xsens MTI-670-DK IMU. No additional hardware triggers are used between the sensors, and the sensor data is recorded using an Intel NUC i7 onboard computer. The experiments are conducted in an indoor area equipped with a motion capture system to obtain precise ground truth. The extrinsic calibration between the IMU and the radar is performed manually. To highlight the significance of temporal calibration in RIO, we design the dataset with two levels of difficulty. Sequences 1 to 3 feature standard motion patterns, while Sequences 4 to 7 introduce more rotational motion to induce larger errors due to the time offset, providing a clearer demonstration of its impact.

\begin{figure*}[t]
\centering
\includegraphics[width=\linewidth]{figure_3.png}
\caption{Comparison of estimated trajectories with the ground truth. The \textcolor{black}{black} trajectory is the ground truth, the \textcolor{blue}{blue} one is the EKF-RIO, which does not account for temporal calibration, and the \textcolor{red}{red} one is the proposed RIO with online temporal calibration. Results are presented for Sequence 4, ICINS 1, and ColoRadar 1, representing one sequence from each of the three datasets.}
\label{trajectory}
\end{figure*}

In~\cite{9470842}, the ICINS2021 dataset is collected using a Texas Instruments IWR6843AOP radar sensor, an Analog Devices ADIS16448 IMU sensor, and a camera. A microcontroller board is used for active hardware triggering to accurately capture the timing of the radar measurements. Data is collected using both handheld and drone platforms. The handheld sequences, ``carried\_1'' and ``carried\_2'', are referred to as ``ICINS 1'' and ``ICINS 2'', while the drone sequences, ``flight\_1'' and ``flight\_2'', are referred to as ``ICINS 3'' and ``ICINS 4'', respectively. The ground truth is provided through visual-inertial SLAM, which performs multiple loop closures, offering a pseudo-ground truth. In~\cite{kramer2022coloradar}, the ColoRadar dataset is collected using a Texas Instruments AWR1843BOOST radar sensor, a Microstrain 3DM-GX5-25 IMU sensor, and a LiDAR mounted on a handheld platform. No specific synchronization setup is used between the sensors. The sequences, ``arpg\_lab\_run0'' and ``arpg\_lab\_run1'', are referred to as ``ColoRadar 1'' and ``ColoRadar 2'', while the sequences ``ec\_hallways\_run0'' and ``ec\_hallways\_run1'' are referred to as ``ColoRadar 3'' and ``ColoRadar 4'', respectively. The ground truth is generated via LiDAR-inertial SLAM, which includes loop closures, offering a pseudo-ground truth.
\subsection{Evaluation}
\label{sec: evaluation}

\begin{table}[t]
\centering
\caption{Quantitative Results of Fixed Offset and Online Estimation}
\label{fixed_offset}
\resizebox{\linewidth}{!}{
\begin{tblr}{
  cells = {c},
  cell{1}{1} = {r=2}{},
  cell{1}{2} = {r=2}{},
  cell{1}{3} = {r=2}{},
  cell{1}{4} = {c=2}{},
  cell{1}{6} = {c=2}{},
  cell{3}{1} = {r=6}{},
  cell{3}{2} = {r=5}{},
  cell{3}{5} = {fg=red},
  cell{4}{4} = {fg=red},
  cell{5}{4} = {fg=blue},
  cell{5}{5} = {fg=blue},
  cell{5}{6} = {fg=blue},
  cell{5}{7} = {fg=red},
  cell{6}{6} = {fg=red},
  cell{6}{7} = {fg=blue},
  cell{9}{1} = {r=6}{},
  cell{9}{2} = {r=5}{},
  cell{11}{4} = {fg=red},
  cell{11}{5} = {fg=blue},
  cell{11}{6} = {fg=red},
  cell{11}{7} = {fg=red},
  cell{12}{4} = {fg=blue},
  cell{12}{5} = {fg=red},
  cell{12}{6} = {fg=blue},
  cell{12}{7} = {fg=blue},
  hline{1,3,9,15} = {-}{},
  hline{2} = {4-7}{},
}
\textbf{Sequence} & \textbf{Method} &  \textbf{Time Offset (s)}            & \textbf{APE RMSE} &                & \textbf{RPE RMSE} &                   \\
                  &                 &                                      & Trans. (m)        & Rot. (\degree) & Trans. (m)        & Rot. (\degree)    \\
                  \hline
Sequence 1        & Fixed Offset    & 0.0             & 0.985             & 1.872          & 0.264             & 1.230          \\
                  &                 & -0.05           & 0.647             & 7.561          & 0.166             & 1.549          \\
                  &                 & -0.10           & 0.661             & 2.438          & 0.138             & 0.948          \\
                  &                 & -0.15           & 0.826             & 5.151          & \textbf{0.131}    & 1.196          \\
                  &                 & -0.20           & 0.974             & 2.698          & 0.156             & 1.274          \\
                  & Online Est.     & \textbf{-0.114} & \textbf{0.646}    & \textbf{0.935} & 0.132    & \textbf{0.774} \\
Sequence 4        & Fixed Offset    & 0.0             & 1.737             & 25.885         & 0.118             & 4.074          \\
                  &                 & -0.05           & 1.028             & 15.460         & 0.091             & 2.313          \\
                  &                 & -0.10           & 0.635             & 4.655          & 0.061             & 0.994          \\
                  &                 & -0.15           & 0.649             & 4.275          & 0.068             & 1.083          \\
                  &                 & -0.20           & 0.716             & 12.461         & 0.092             & 2.526          \\
                  & Online Est.     & \textbf{-0.115} & \textbf{0.610}    & \textbf{3.099} & \textbf{0.057}    & \textbf{0.944} 
\end{tblr}
}
\vspace{0.3em}
{\raggedright
\noindent\par {\footnotesize \textsuperscript{*}The initial time offset of `Online Est.' is set to 0.0 and the converged values are shown above.}
\noindent\par {\footnotesize \textsuperscript{**}For each sequence, the lowest error values among the fixed offsets are highlighted in \textcolor{red}{red}, and the second-lowest in \textcolor{blue}{blue}.}
\par}

\end{table}
For the performance comparison, the open-source EKF-RIO \cite{9235254}, which uses the same measurement model but does not account for temporal calibration, is employed. All parameters are kept identical to ensure a fair comparison. In the proposed method, the time offset \( t_d \) is initialized to 0.0 seconds for all sequences, reflecting a typical scenario where the initial time offset is unknown. The experimental results are evaluated using the open-source tool EVO \cite{grupp2017evo}. Figure~\ref{trajectory} illustrates the estimated trajectories compared to the ground truth for visual comparison, with one representative result from each dataset. Due to the stochastic nature of the RANSAC algorithm used in radar ego-velocity estimation, the averaged results from 100 trials across all datasets are presented. We compare the root mean square error (RMSE) of both absolute pose error (APE) and relative pose error (RPE), with the RPE calculated at 10-meter intervals.

APE evaluates the overall trajectory by calculating the difference between the ground truth and the estimated poses for all frames, making it particularly useful for assessing the global accuracy of the estimated trajectory. However, APE can be sensitive to significant rotational errors that occur early or in specific sections, potentially overshadowing smaller errors later in the trajectory. In contrast, RPE focuses on local accuracy by aligning poses at regular intervals and calculating the error, allowing discrepancies over shorter segments to be highlighted. When the temporal calibration between sensors is not accounted for, errors can accumulate over time, making RPE evaluation essential. Both metrics offer valuable insights, providing a comprehensive evaluation of the trajectory.

\subsubsection{Self-Collected Dataset}
The purpose of the self-collected dataset is to identify the actual time offset between the IMU and the radar and evaluate its impact on the accuracy of RIO. Since the handheld platform does not utilize a hardware trigger to synchronize the sensors, the exact time offset is unknown and must be estimated. To address this uncertainty, we evaluate the performance of fixed time offsets over a range of values to determine the interval that provides the best accuracy and estimate the likely time offset range.

As shown in Table \ref{fixed_offset}, error values are analyzed with fixed offsets set at 0.05-second intervals for both Sequence 1 and Sequence 4, which feature different motion patterns. The results show that the time offset falls within the -0.10 to -0.15 second range, where the highest accuracy in terms of APE and RPE is observed for both sequences. The proposed method, which utilizes online temporal calibration, estimates the time offset as -0.114 seconds for Sequence 1 and -0.115 seconds for Sequence 4, closely matching the range found through fixed offset testing. In both cases, the proposed method achieves improved performance in terms of both APE and RPE, demonstrates its effectiveness in accurately estimating the time offset.

\begin{table}[t]
\centering
\caption{Quantitative Results of Comparison study on Self-collected dataset}
\label{table_self}
\resizebox{\linewidth}{!}{
\begin{tblr}{
  cells = {c},
  cell{1}{1} = {r=2}{},
  cell{1}{2} = {r=2}{},
  cell{1}{3} = {c=2}{},
  cell{1}{5} = {c=2}{},
  cell{3}{1} = {r=2}{},
  cell{5}{1} = {r=2}{},
  cell{7}{1} = {r=2}{},
  cell{9}{1} = {r=2}{},
  cell{11}{1} = {r=2}{},
  cell{13}{1} = {r=2}{},
  cell{15}{1} = {r=2}{},
  cell{17}{1} = {r=2}{},
  hline{1,3,5,7,9,11,13,15,17,19} = {-}{},
  hline{2} = {3-6}{},
}
{\textbf{Sequence }\\\textbf{(Trajectory Length)}} & {\textbf{Method } \textbf{($\hat{t}_d$)}} & \textbf{APE RMSE } &                & \textbf{RPE RMSE } &                \\
                                                   &                                         & Trans. (m)         & Rot. (\degree)        & Trans. (m)         & Rot. (\degree)        \\
                                                   \hline
{Sequence 1\\(177 m)}                              & {EKF-RIO (N/A)}                        & 0.985              & 1.872           & 0.264              & 1.230          \\
                                                   & {Ours (-0.114 s)}                      & \textbf{0.646}     & \textbf{0.935}  & \textbf{0.132}     & \textbf{0.774} \\
{Sequence 2\\(197 m)}                              & {EKF-RIO}                              & 2.269              & 2.161           & 0.136              & 1.414          \\
                                                   & {Ours (-0.114 s)}                      & \textbf{0.587}     & \textbf{1.650}  & \textbf{0.064}     & \textbf{0.774} \\
{Sequence 3\\(144 m)}                              & {EKF-RIO}                              & 1.368              & 2.331           & 0.167              & 1.347          \\
                                                   & {Ours (-0.113 s)}                      & \textbf{0.414}     & \textbf{1.140}  & \textbf{0.088}     & \textbf{0.613} \\
{Sequence 4\\(197 m)}                              & {EKF-RIO}                              & 1.737              & 25.885          & 0.118              & 4.074          \\
                                                   & {Ours (-0.115 s)}                      & \textbf{0.610}     & \textbf{3.099}  & \textbf{0.057}     & \textbf{0.944} \\
{Sequence 5\\(190 m)}                              & {EKF-RIO}                              & 2.375              & 7.702           & 0.122              & 1.600          \\
                                                   & {Ours (-0.115 s)}                      & \textbf{1.150}     & \textbf{1.304}  & \textbf{0.069}     & \textbf{0.814} \\
{Sequence 6\\(179 m)}                              & {EKF-RIO}                              & 1.267              & 17.907          & 0.117              & 2.828          \\
                                                   & {Ours (-0.111 s)}                      & \textbf{0.661}     & \textbf{2.551}  & \textbf{0.051}     & \textbf{0.809} \\
{Sequence 7\\(223 m)}                              & {EKF-RIO}                              & 2.757              & 10.092          & 0.116              & 1.863          \\
                                                   & {Ours (-0.112 s)}                      & \textbf{1.596}     & \textbf{6.039}  & \textbf{0.057}     & \textbf{1.365} \\
{Average}                                          & {EKF-RIO}                              & 1.822              & 9.707            & 0.148             & 2.051          \\
                                                   & {Ours (-0.113 s)}                      & \textbf{0.809}     & \textbf{2.388}   & \textbf{0.074}    & \textbf{0.870}   
\end{tblr}
}
\end{table}

Since the radar delay is generally larger than IMU delay, the time offset \( t_d \), representing the difference between these delays, typically takes a negative value. To evaluate the robustness of the estimation, different initial values of \( t_d \) ranging from 0.0 to -0.3 seconds are tested. Figure \ref{sq5} illustrates the estimated time offset for each initial setting, along with the 3-sigma boundaries. As \( t_d \) is estimated from radar ego-velocity, it cannot be determined while the platform is stationary. Once the platform starts moving, the filter begins estimating \( t_d \) and quickly converges to a stable value. The filter converges to a stable time offset of -0.114 ± 0.001 seconds in Sequence 1 and -0.115 ± 0.001 seconds in Sequence 4.

Table \ref{table_self} presents the performance comparison between the proposed method with online temporal calibration and EKF-RIO across seven sequences. The proposed method outperforms EKF-RIO, significantly reducing both APE and RPE across all sequences. Specifically, it reduces APE translation error by an average of 56\%, APE rotation error by 75\%, RPE translation error by 50\%, and RPE rotation error by 58\% compared with EKF-RIO. Despite using the same measurement model, the performance improvement is achieved solely by applying propagation and updates based on a common time stream through the proposed online temporal calibration.

On average, the time offset \( t_d \) is estimated to be -0.113 ± 0.002 seconds, confirming consistent temporal calibration throughout the experiments. Compared with LiDAR-inertial and visual-inertial systems, radar-inertial systems exhibit a significantly larger time offset, as shown in Table~\ref{time_offset_comparison}. Given the radar sensor rate (10 Hz), such a large time offset is significant enough to cause a misalignment spanning more than one data frame. These findings highlight the necessity of temporal calibration in RIO, which is crucial for accurate sensor fusion and reliable pose estimation in real-world applications.

\begin{figure}[t]
\centering
\includegraphics[width=\linewidth]{figure_4.png}
\caption{Time offset estimation with 3-sigma boundaries for different initial values in Sequence 1 and 4.}
\label{sq5}
\end{figure}

\begin{table}[t]
\centering
\caption{Comparison of Time Offset in Multi-Sensor Fusion Systems}
\label{time_offset_comparison}
\begin{tabular}{|c|c|c|} 
\hline
\textbf{Systems} & \textbf{Sensor} & \textbf{Time Offset} \\ 
\hline
LiDAR-Inertial~\cite{10113826} & Velodyne VLP-32 & 0.006 s\\ 
\hline
Visual-Inertial~\cite{li2014online} & PointGrey Bumblebee2 & 0.047 s\\ 
\hline
Radar-Inertial & TI AWR1843BOOST & \textbf{0.113 s} \\
\hline
\end{tabular}
\end{table}

\subsubsection{Open Datasets}
Table \ref{opendataset} presents the results from the two open datasets. The ICINS dataset incorporates a hardware trigger for the radar, which we use to validate the accuracy of the time offset estimation for the proposed method. In this setup, a microcontroller sends radar trigger signals, prompting the radar to begin scanning. The radar data is timestamped based on the actual trigger signal, providing a pseudo-ground truth for time offset estimation. Theoretically, if the sensors are time-synchronized through triggers, the time offset \( t_d \) is expected to be close to 0.0 seconds. The proposed method estimates the time offset to be an average of 0.016 ± 0.003 seconds. Despite this slight discrepancy, the proposed method demonstrates comparable or improved performance on average in both APE and RPE compared with EKF-RIO. Although the ICINS dataset includes hardware-triggered signals for the radar, there is no such trigger signal for the IMU in the dataset, which may introduce a delay in IMU measurements. As defined in Eq.~\eqref{time_offset}, we attribute the estimated positive time offset to this IMU delay, explaining the difference from the expected value.

The ColoRadar dataset, widely used for performance comparison in the RIO field, is utilized to assess if the proposed method generalizes well across different datasets. As shown in Table \ref{opendataset}, the proposed method also demonstrates performance improvements over EKF-RIO in terms of both APE and RPE on average. However, the extent of improvement is smaller compared with the self-collected dataset, which can be explained by differences in trajectory characteristics. The radar ego-velocity model utilizes not only the accelerometer but also the gyroscope measurements. As illustrated in Fig.~\ref{trajectory}, the ColoRadar dataset involves movement over a larger area with less rotation, leading to a smaller impact of the time offset on performance. Nonetheless, the proposed method achieves 33\% reduction in RPE translation error, demonstrating its effectiveness even in this less challenging trajectory. On average, the time offset \( t_d \) is estimated to be -0.111 ± 0.003 seconds, similar to the time offset found in the self-collected dataset. This consistency is likely due to the use of the same radar sensor model in both datasets, further validating the reliability of the proposed method across different environments.

\begin{table}[t]
\centering
\caption{Quantitative Results of Comparison study on Open datasets}
\label{opendataset}
\resizebox{\linewidth}{!}{
\begin{tblr}{
  cells = {c},
  cell{1}{1} = {r=2}{},
  cell{1}{2} = {r=2}{},
  cell{1}{3} = {c=2}{},
  cell{1}{5} = {c=2}{},
  cell{3}{1} = {r=2}{},
  cell{5}{1} = {r=2}{},
  cell{7}{1} = {r=2}{},
  cell{9}{1} = {r=2}{},
  cell{11}{1} = {r=2}{},
  cell{13}{1} = {r=2}{},
  cell{15}{1} = {r=2}{},
  cell{17}{1} = {r=2}{},
  cell{19}{1} = {r=2}{},
  cell{21}{1} = {r=2}{},
  hline{1,3,5,7,9,11,13,15,17,19,21,23} = {-}{},
  hline{2-3} = {3-6}{},
}
{\textbf{Sequence }\\\textbf{(Trajectory Length)}}       & \textbf{Method ($\hat{t}_d$)} & \textbf{APE RMSE}        &                                           & \textbf{RPE RMSE}       &                         \\
                        &                               & Trans. (m)               & Rot. (\degree)                                   & Trans. (m)              & Rot. (\degree)                 \\
                        \hline
{ICINS 1\\(295 m)}      & EKF-RIO (N/A)                 & 1.959                    & 10.694                                    & \textbf{0.093}          & \textbf{0.896}          \\
                        & Ours (0.016 s)                & \textbf{1.922}           & \textbf{10.135}                           & 0.098                   & 0.918          \\
{ICINS 2\\(468 m)}      & EKF-RIO                       & 3.830                    & 23.151                                    & \textbf{0.114}          & 1.289                   \\
                        & Ours (0.013 s)                & \textbf{3.198}           & \textbf{19.235}                           & 0.121                   & \textbf{1.076}          \\
{ICINS 3\\(150 m)}      & EKF-RIO                       & \textbf{1.502}           & \textbf{9.905}                            & 0.130                   & \textbf{1.512}           \\
                        & Ours (0.015 s)                & 1.530                    & 10.189                                    & \textbf{0.126}          & 1.553          \\
{ICINS 4\\(50 m)}       & EKF-RIO                       & \textbf{0.213}           & \textbf{2.091}                            & \textbf{0.076}          & \textbf{0.923}           \\
                        & Ours (0.019 s)                & 0.216                    & 2.098                                     & 0.081                   & \textbf{0.923}          \\
Average                 & EKF-RIO                       & 1.876                    & 11.460                                    & \textbf{0.103}          & 1.155                   \\
                        & Ours (0.016 s)                & \textbf{1.716}           & \textbf{10.414}                           & 0.106                   & \textbf{1.117}          \\
                        \hline
{ColoRadar 1\\(178 m) } & EKF-RIO (N/A)                 & 6.556                    & \textbf{\textbf{1.354}}                   & 0.182                   & \textbf{1.071} \\
                        & Ours (-0.110 s)               & \textbf{\textbf{6.173}}  & 1.382                                     & \textbf{\textbf{0.155}} & 1.188                   \\
{ColoRadar 2\\(197 m) } & EKF-RIO                       & \textbf{\textbf{4.747}}  & 1.238                                     & 0.372                   & 1.375                   \\
                        & Ours (-0.114 s)               & 4.826                    & \textbf{\textbf{0.960}}                   & \textbf{\textbf{0.292}} & \textbf{\textbf{1.180}} \\
{ColoRadar 3\\(197 m) } & EKF-RIO                       & \textbf{\textbf{8.307}}  & 1.969                                     & 0.259                   & 1.015                   \\
                        & Ours (-0.108 s)               & 8.550                    & \textbf{\textbf{1.852}}                   & \textbf{\textbf{0.221}} & \textbf{\textbf{0.879}} \\
{ColoRadar 4\\(144 m) } & EKF-RIO                       & 12.111                   & 2.815                                     & 0.488                   & 1.263                   \\
                        & Ours (-0.112 s)               & \textbf{11.946}          & \textbf{2.756}                            & \textbf{0.200}          & \textbf{1.116} \\
Average                 & EKF-RIO                       & 7.930                    & 1.844                                     & 0.325                   & 1.181                   \\
                        & Ours(-0.111 s)                & \textbf{7.874}           & \textbf{1.737}                            & \textbf{0.217}          & \textbf{1.091}          
\end{tblr}
}
\end{table}


% The \LaTeX{} and Bib\TeX{} style files provided roughly follow the American Psychological Association format.
% If your own bib file is named \texttt{custom.bib}, then placing the following before any appendices in your \LaTeX{} file will generate the references section for you:
% \begin{quote}
% \begin{verbatim}
% \bibliography{custom}
% \end{verbatim}
% \end{quote}

% You can obtain the complete ACL Anthology as a Bib\TeX{} file from \url{https://aclweb.org/anthology/anthology.bib.gz}.
% To include both the Anthology and your own .bib file, use the following instead of the above.
% \begin{quote}
% \begin{verbatim}
% \bibliography{anthology,custom}
% \end{verbatim}
% \end{quote}

% Please see Section~\ref{sec:bibtex} for information on preparing Bib\TeX{} files.

% \subsection{Equations}

% An example equation is shown below:
% \begin{equation}
%   \label{eq:example}
%   A = \pi r^2
% \end{equation}

% Labels for equation numbers, sections, subsections, figures and tables
% are all defined with the \verb|\label{label}| command and cross references
% to them are made with the \verb|\ref{label}| command.

% This an example cross-reference to Equation~\ref{eq:example}.

% Use \verb|\appendix| before any appendix section to switch the section numbering over to letters. See Appendix~\ref{sec:appendix} for an example.

% \section{Bib\TeX{} Files}
% \label{sec:bibtex}

% Unicode cannot be used in Bib\TeX{} entries, and some ways of typing special characters can disrupt Bib\TeX's alphabetization. The recommended way of typing special characters is shown in Table~\ref{tab:accents}.

% Please ensure that Bib\TeX{} records contain DOIs or URLs when possible, and for all the ACL materials that you reference.
% Use the \verb|doi| field for DOIs and the \verb|url| field for URLs.
% If a Bib\TeX{} entry has a URL or DOI field, the paper title in the references section will appear as a hyperlink to the paper, using the hyperref \LaTeX{} package.


\section{Conclusion}
In this paper, we introduced \textsc{FLAG-Trader}, a novel framework that integrates LLMs with RL for financial trading. In particular, \textsc{FLAG-Trader} leverages LLMs as policy networks, allowing for natural language-driven decision-making while benefiting from reward-driven optimization through RL fine-tuning. Our framework enables small-scale LLMs to surpass larger proprietary models by efficiently adapting to market conditions via a structured reinforcement learning approach. Through extensive experiments across multiple stock trading scenarios, we demonstrated that \textsc{FLAG-Trader} consistently outperforms baseline methods, including LLM-agentic frameworks and conventional RL-based trading agents. These results highlight the potential of integrating LLMs with RL to achieve adaptability in financial decision-making.

\newpage
\section*{Limitations and Potential Risk}

Despite its promising results, \textsc{FLAG-Trader} has several limitations. First, while our approach significantly enhances the decision-making ability of LLMs, it remains computationally expensive, particularly when fine-tuning on large-scale market datasets. Reducing computational overhead while maintaining performance is an important direction for future research. Second, financial markets exhibit high volatility and non-stationarity, posing challenges for long-term generalization. Future work should explore techniques such as continual learning or meta-learning to enhance model adaptability in evolving market conditions. Third, while \textsc{FLAG-Trader} effectively integrates textual and numerical data, its reliance on structured prompts could introduce biases in decision-making. Improving prompt design or exploring retrieval-augmented methods may further enhance robustness. Lastly, real-world trading requires stringent risk management, and \textsc{FLAG-Trader} currently optimizes for financial returns without explicitly incorporating risk-sensitive constraints. Extending the framework to integrate risk-aware objectives and dynamic portfolio optimization could provide more robust and practical financial trading solutions.

% \section*{Acknowledgments}

% This document has been adapted
% by Steven Bethard, Ryan Cotterell and Rui Yan
% from the instructions for earlier ACL and NAACL proceedings, including those for
% ACL 2019 by Douwe Kiela and Ivan Vuli\'{c},
% % NAACL 2019 by Stephanie Lukin and Alla Roskovskaya,
% % ACL 2018 by Shay Cohen, Kevin Gimpel, and Wei Lu,
% % NAACL 2018 by Margaret Mitchell and Stephanie Lukin,
% % Bib\TeX{} suggestions for (NA)ACL 2017/2018 from Jason Eisner,
% % ACL 2017 by Dan Gildea and Min-Yen Kan,
% % NAACL 2017 by Margaret Mitchell,
% % ACL 2012 by Maggie Li and Michael White,
% % ACL 2010 by Jing-Shin Chang and Philipp Koehn,
% % ACL 2008 by Johanna D. Moore, Simone Teufel, James Allan, and Sadaoki Furui,
% % ACL 2005 by Hwee Tou Ng and Kemal Oflazer,
% % ACL 2002 by Eugene Charniak and Dekang Lin,
% % and earlier ACL and EACL formats written by several people, including
% % John Chen, Henry S. Thompson and Donald Walker.
% % Additional elements were taken from the formatting instructions of the \emph{International Joint Conference on Artificial Intelligence} and the \emph{Conference on Computer Vision and Pattern Recognition}.

% Bibliography entries for the entire Anthology, followed by custom entries
%\bibliography{anthology,custom}
% Custom bibliography entries only
\bibliography{main}

\newpage
\appendix
\onecolumn
\section{Additional Algorithmic  Details: \textsc{FLAG-Trader} with PPO}\label{sec:Appendix_A}



In this section, we outline a detailed procedure for training the \textsc{FLAG-Trader} architecture via PPO, where the \textsc{Policy\_Net} (actor) and the \textsc{Value\_Net} (critic) share a subset of trainable parameters from a LLM, with
$\theta = \big( \theta_{\texttt{train}}, \theta_P, \theta_V\big)$.
We define $\theta_{policy} = \big( \theta_{\texttt{train}}, \theta_P)$ and $\theta_{value} = \big( \theta_{\texttt{train}}, \theta_V)$ for simplicity.

\textbf{Advantage Estimation.}
We use the Generalized Advantage Estimation (GAE) to compute the advantage function \( A_t \):
\begin{align}
    A_t \;=\; \sum_{k=0}^{T-1} (\gamma \lambda)^k \bigl[r_{t+k} + \gamma V_{\theta_{value}}(s_{t+k+1}) - V_{\theta_{value}}(s_{t+k})\bigr],
\end{align}
where \( \gamma \) is the discount factor, and \( \lambda \) is the GAE parameter.

\textbf{Probability Ratio.}
Let \(\theta_{policy, \mathrm{old}}\) denote the parameters before the current update. The PPO probability ratio is
\begin{align}
    r_t(\theta_{policy}) \;=\; \frac{\pi_{\theta_{policy}}(a_t \mid s_t)}{\pi_{\theta_{policy,\mathrm{old}}}(a_t \mid s_t)}.
\end{align}


\textbf{PPO Clipped Objective.}
PPO clips this ratio to prevent overly large updates. The surrogate objective is
\begin{align}
    \mathcal{L}_P(\theta_{policy}) \;=\; \mathbb{E}_t \Bigl[
  \min\bigl(r_t(\theta_{policy})\,A_t,\; \text{clip}\bigl(r_t(\theta_{policy}),\,1-\varepsilon,\,1+\varepsilon\bigr)\,A_t\bigr)
\Bigr],
\end{align}
where \(\varepsilon\) is a hyperparameter.

\textbf{Value Function Loss.}
The critic (value network) is updated by minimizing the difference between the predicted value \(V_{\theta_{value}}(s_t)\) and the target return \(R_t\). A common choice is:
\begin{align}
    \mathcal{L}_{V}(\theta_{value}) \;=\; \mathbb{E}_t\Bigl[(V_{\theta_{value}}(s_t) - R_t)^2\Bigr].
\end{align}


\textbf{Combined Loss.}
We often add an entropy term to encourage exploration, yielding the overall objective:
\begin{align}
    \mathcal{L}_{\text{total}}(\theta)
\;=\;
-\,\mathcal{L}_{P}(\theta_{policy})
\;+\;
c_1\,\mathcal{L}_{V}(\theta_{value})
\;-\;
c_2\,\mathcal{H}\bigl(\pi_{\theta_{policy}}\bigr),
\end{align}
where \(c_1\) and \(c_2\) are weighting coefficients, and \(\mathcal{H}(\pi_{\theta_{policy}})\) represents the policy entropy.

\textbf{Parameter Updates.}
At each iteration, we apply gradient descent on the total loss:
\begin{align}
\theta_P       &\leftarrow \theta_P       \;-\; \eta \;\nabla_{\theta_P}\,\mathcal{L}_P, \\
\theta_V       &\leftarrow \theta_V       \;-\; \eta \;\nabla_{\theta_V}\,\mathcal{L}_V, \\
\theta_{\texttt{train}} &\leftarrow \theta_{\texttt{train}} - \beta \;\nabla_{\theta_{\text{train}}}\,\mathcal{L}_{\text{total}},
\end{align}
where \(\eta\) and \(\beta\) are learning rates for the policy head, value head, and trainable LLM layers respectively. The algorithm is summerized in Algorithm \ref{alg:flagtrader-ppo}.

\begin{algorithm}[H]
\caption{FLAG-TRADER with PPO}
\label{alg:flagtrader-ppo}
\begin{algorithmic}[1]
\STATE \textbf{Input:} Pre-trained LLM parameters $(\theta_{\texttt{frozen}}, \theta_{\texttt{train}})$; actor parameters $\theta_P$; critic parameters $\theta_V$; environment $\mathcal{E}$; discount factor $\gamma$; GAE parameter $\lambda$; PPO clip $\varepsilon$; learning rates $\eta, \beta$;
\STATE Initialize $\theta_{\text{train}}, \theta_P, \theta_V$; let $\theta_{\mathrm{old}} \leftarrow \theta$
\STATE Initialize replay buffer $B \leftarrow \emptyset$
\FOR{iteration = 1 to \text{max\_iters}}
  \STATE // \textit{Collect Rollouts}
  \FOR{t = 1 to T}
    \STATE Fetch the current state $s_t$ from the environment and construct an input prompt \texttt{lang}($s_t$);
    \STATE Pass prompt \texttt{lang}($s_t$) through LLM;
    \STATE \textsc{Policy\_Net} outputs $a_t$ from action space $\{\texttt{``buy,'' ``sell,'' ``hold''}\}$ based on \eqref{eq:masking};
    \STATE Execute action $a_t$ in the environment and observe reward $r(s_t, a_t)$ and transition to new state $s_{t+1}$;
    \STATE Store experience tuple $(s_t, a_t, r_t, s_{t+1})$ in replay buffer $B$;
    
  \ENDFOR

  \STATE // \textit{Compute Advantage and Targets}
  \FOR{\textbf{each} transition in $B$}
    \STATE Compute $V_{\theta_{value}}(s_t)$ and advantage $A_t$ (e.g., via GAE)
  \ENDFOR

  \STATE // \textit{Perform PPO Updates}
  \FOR{update\_epoch = 1 to K}
    \STATE Sample mini-batch $\mathcal{M}$ from $B$
    \STATE Compute probability ratio $r_t(\theta_{policy}) \;=\; \frac{\pi_{\theta_{policy}}(a_t \mid s_t)}{\pi_{\theta_{policy,\mathrm{old}}}(a_t \mid s_t)}$;
    \STATE Compute PPO loss $\mathcal{L}_P(\theta_{policy}) \;=\; \mathbb{E}_t \Bigl[
  \min\bigl(r_t(\theta_{policy})\,A_t,\; \text{clip}\bigl(r_t(\theta_{policy}),\,1-\varepsilon,\,1+\varepsilon\bigr)\,A_t\bigr)
\Bigr]$;
    \STATE Compute Value loss $\mathcal{L}_{V}(\theta_{value}) \;=\; \mathbb{E}_t\Bigl[(V_{\theta_{value}}(s_t) - R_t)^2\Bigr]$;
    \STATE Compute total loss $\mathcal{L}_{\text{total}}(\theta)
\;=\;
-\,\mathcal{L}_{P}(\theta_{policy})
\;+\;
c_1\,\mathcal{L}_{V}(\theta_{value})
\;-\;
c_2\,\mathcal{H}\bigl(\pi_{\theta_{policy}}\bigr)$;
    \STATE Perform gradient descent on each parameter group:
    \begin{align*}
\theta_P       &\leftarrow \theta_P       \;-\; \eta \;\nabla_{\theta_P}\,\mathcal{L}_P, \\
\theta_V       &\leftarrow \theta_V       \;-\; \eta \;\nabla_{\theta_V}\,\mathcal{L}_V, \\
\theta_{\texttt{train}} &\leftarrow \theta_{\texttt{train}} - \beta \;\nabla_{\theta_{\text{train}}}\,\mathcal{L}_{\text{total}};
\end{align*}
  \ENDFOR

  \STATE // \textit{Update old policy parameters}
  \STATE Update $\theta = \big( \theta_{\texttt{train}}, \theta_P, \theta_V\big)$ by $\theta_{\mathrm{old}} \leftarrow \theta$;
\ENDFOR
\STATE \textbf{Return:} Fine-tuned \textsc{Policy\_Net}($\theta_P$).
\end{algorithmic}
\end{algorithm}




% \begin{algorithm}
% \caption{FinRL-LLM Pipeline with PPO}
% \label{alg:FinRL-LLM-PPO}
% \begin{algorithmic}[1]
% \STATE  \textbf{Input:} Pre-trained LLM policy $\pi_{\theta}$, environment dynamics, reward function $r(s_t, a_t)$, learning parameters (replay buffer size, batch size, PPO clipping parameter $\epsilon$, discount factor $\gamma$, GAE parameter $\lambda$, learning rate $\eta$)
% \STATE  \textbf{Output:} Fine-tuned LLM policy $\pi_{\theta}$

% \STATE  Initialize LLM policy $\pi_{\theta}$ with pre-trained weights
% \STATE  Initialize experience replay buffer $B \leftarrow \emptyset$
% \FOR{each training iteration}
%     \STATE  Fetch the current state $s_t$ from the environment
%     \STATE  Convert $s_t$ into structured text $\text{lang}(s_t)$
%     \STATE  Encode $\text{lang}(s_t)$ into token embeddings $E_t \leftarrow \text{Embed}(\text{lang}(s_t))$
%     \STATE  Pass through LLM:
%     \STATE  $h_t^{(1)} \leftarrow \text{LLM}_{\text{frozen}}(E_t)$
%     \STATE  $h_t^{(2)} \leftarrow \text{LLM}_{\text{trainable}}(h_t^{(1)})$
%     \STATE  Compute action probability distribution:
%     \STATE  $\text{logits}_t \leftarrow \text{POLICY\_NET}(h_t^{(2)})$
%     \STATE  $\pi_{\theta}(a_t | s_t) \leftarrow \text{Softmax}(\text{logits}_t)$
%     \STATE  Sample action $a_t \sim \pi_{\theta}(a_t | s_t)$
%     \STATE  Execute $a_t$ in the environment and observe reward $r_t$ and transition to $s_{t+1}$
%     \STATE  Compute state value $V(s_t) \leftarrow \text{VALUE\_NET}(h_t^{(2)})$
%     \STATE  Store experience $(s_t, a_t, r_t, s_{t+1}, V(s_t), \pi_{\theta}(a_t | s_t))$ in replay buffer $B$

%     \IF{replay buffer $B$ is full}
%         \STATE  Sample a mini-batch of experiences $\{(s_i, a_i, r_i, s_{i+1}, V(s_i), \pi_{\theta_{\text{old}}}(a_i | s_i))\}$ from $B$
%         \STATE  Compute advantage estimation using GAE:
%         \STATE  $A_i = \sum_{l=0}^{T-t} (\gamma \lambda)^l (r_i + \gamma V(s_{i+1}) - V(s_i))$
%         \STATE  Compute log-likelihood ratio:
%         \STATE  $r_i(\theta) = \frac{\pi_{\theta}(a_i | s_i)}{\pi_{\theta_{\text{old}}}(a_i | s_i)}$
%         \STATE  Compute PPO clipped objective:
%         \STATE  $L_{\text{PPO}}(\theta) = \mathbb{E}_t [\min( r_i(\theta) A_i, \text{clip}(r_i(\theta), 1 - \epsilon, 1 + \epsilon) A_i )]$
%         \STATE  Compute value loss:
%         \STATE  $L_V = (V(s_i) - V_{\text{target}}(s_i))^2, \quad V_{\text{target}}(s_i) = r_i + \gamma V(s_{i+1})$
%         \STATE  Compute entropy bonus:
%         \STATE  $S(\pi_{\theta}) = - \sum_a \pi_{\theta}(a | s) \log \pi_{\theta}(a | s)$
%         \STATE  Compute total loss:
%         \STATE  $L_{\text{total}} = -L_{\text{PPO}} + c_1 L_V - c_2 S(\pi_{\theta})$
%         \STATE  Perform gradient descent on $\theta$ to minimize $L_{\text{total}}$
%         \STATE  Update $\pi_{\theta}$, VALUE\_NET, and trainable LLM layers
%         \STATE  Clear replay buffer $B$
%     \ENDIF
% \ENDFOR

% \STATE  \textbf{Return:} Fine-tuned LLM policy $\pi_{\theta}$
% \end{algorithmic}
% \end{algorithm}

% The online PPO module is responsible for training the LLM to improve its policy over time. The learning process involves the following steps:
% \begin{enumerate}
%     \item \textbf{Log-Likelihood Computation}: The LLM generates an action \(a_t\) conditioned on the state \(s_t\) and the prompt. The log-likelihood \(\log \pi_\theta(a_t | s_t, \text{prompt})\) is computed to measure the policy's performance.
%     \item \textbf{Environment Interaction}: The action \(a_t\) is executed in the environment, which transitions to a new state \(s_{t+1}\) and returns a reward \(r(s_t, a_t)\).
%     \item \textbf{Experience Replay}: The observed experience tuple \((s_t, a_t, r_t, s_{t+1})\) is stored in the replay buffer for batch updates.
%     \item \textbf{PPO Objective}: The policy is updated using the PPO objective:
%     \[
%     L_{\text{PPO}}(\theta) = \mathbb{E}_t \left[ \min \left( r_t(\theta) A_t, \text{clip}(r_t(\theta), 1-\epsilon, 1+\epsilon) A_t \right) \right],
%     \]
%     where \(r_t(\theta)\) is the probability ratio, \(A_t\) is the advantage function, and \(\epsilon\) is a hyperparameter controlling the update clipping.
% \end{enumerate}

\section{Additional Experimental Details}


% \subsection{Details on Evaluation Metrics}
% Below is a brief overview of these metrics:\par
% \noindent\textbf{Cumulative Return (CR) \%} measures the total value change of an investment over time by summing daily logarithmic returns, shown in \eqref{eq:cum_return}: 
% \begin{align}
%    \label{eq:cum_return}
%    \textbf{CR} &= \sum_{t=1}^{n} r_i = \sum_{t=1}^{n} \left[ \ln\left(\frac{p_{t+1}}{p_t}\right) \cdot \text{action}_t \right],
% \end{align}
% where $r_i$ is the logarithmic return from day $t$ to $t+1$, $p_t$ and $p_{t+1}$ are the closing prices on days $t$ and $t+1$, respectively, and $\text{action}_t$ is the model's trading decision for day $t$.
% Notice that higher values indicate better strategy effectiveness.

% \noindent\textbf{Sharpe Ratio (SR)} assesses risk-adjusted returns by dividing the average excess return ($R_p$) over the risk-free rate ($R_f$) by its volatility ($\sigma_p$), detailed in \eqref{eq:sharpe}: 
% \begin{equation}
%     \textbf{SR} = \frac{R_p - R_f}{\sigma_p}.
%     \label{eq:sharpe}
% \end{equation}  
% Notice that higher ratios signify better performance.
  
% \noindent\textbf{Annualized Volatility (AV) \% and Daily Volatility (DV) \%} quantify return fluctuations; AV is derived by scaling DV (\textit{standard deviation of daily logarithmic returns}) by the square root of the annual trading days (252), as in \eqref{eq:annuaVol}. This metric highlights potential return deviations across the year.
% \begin{align}
%    \label{eq:annuaVol}
%     \textbf{AV} &= \textbf{DV} \times \sqrt{252}. 
% \end{align} 

% \noindent\textbf{Max Drawdown (MDD) \%} calculates the largest portfolio value drop from peak to trough, as given in \eqref{eq:maxdrawdown}. Lower values indicate lesser risk and higher strategy robustness. 
%     \begin{align}
%     \label{eq:maxdrawdown}
%     \textbf{MDD} = \text{max}(\frac{P_{\text{peak}} - P_{\text{trough}}}{P_{\text{peak}}}).
%     \end{align}



\subsection*{Hyperparameters for Finetuening \textsc{FLAG-Trader} with PPO in Algorithm \ref{alg:flagtrader-ppo}}
\begin{table}[!ht]
\centering
\caption{\textsc{FLAG-Trader} with PPO Finetuning Hyperparameters and Settings.}
\label{tab:parameters:lora}
\scalebox{0.8}{
\begin{tabular}{lll}
\toprule
\textbf{Parameter} & \textbf{Default Value} & \textbf{Description} \\
\toprule
\texttt{total\_timesteps} & 13860 & Total number of timesteps \\
\texttt{learning\_rate} & \(5 \times 10^{-4}\) & Learning rate of optimizer \\
\texttt{num\_envs} & 1 & Number of parallel environments \\
\texttt{num\_steps} & 40 & Steps per policy rollout \\
\texttt{anneal\_lr} & True & Enable learning rate annealing \\
\texttt{gamma} & 0.95 & Discount factor \(\gamma\) \\
\texttt{gae\_lambda} & 0.98 & Lambda for Generalized Advantage Estimation \\
\texttt{update\_epochs} & 1 & Number of update epochs per cycle \\
\texttt{norm\_adv} & True & Advantages whitening \\
\texttt{clip\_coef} & 0.2 & Surrogate clipping coefficient \\
\texttt{clip\_vloss} & True & Clipped loss for value function \\
\texttt{ent\_coef} & 0.05 & Coefficient of entropy term \\
\texttt{vf\_coef} & 0.5 & Coefficient of value function \\
\texttt{kl\_coef} & 0.05 & KL divergence with reference model \\
\texttt{max\_grad\_norm} & 0.5 & Maximum gradient clipping norm \\
\texttt{target\_kl} & None & Target KL divergence threshold \\
\texttt{dropout} & 0.0 & Dropout rate \\
\texttt{llm} & "SmolLM2-135M-Instruct" & Model to fine-tune \\
\texttt{train\_dtype} & "float16" & Training data type \\
\texttt{gradient\_accumulation\_steps} & 8 & Number of gradient accumulation steps \\
\texttt{minibatch\_size} & 32 & Mini-batch size for fine-tuning \\
\texttt{max\_episode\_steps} & 65 & Maximum number of steps per episode \\
\bottomrule
\end{tabular}
}
\end{table}





\end{document}
