% This must be in the first 5 lines to tell arXiv to use pdfLaTeX, which is strongly recommended.
\pdfoutput=1
% In particular, the hyperref package requires pdfLaTeX in order to break URLs across lines.

\documentclass[11pt]{article}

% Recommended, but optional, packages for figures and better typesetting:
\usepackage{microtype}
\usepackage{graphicx}
\usepackage{subfigure}
\usepackage{booktabs} % for professional tables
\usepackage{graphicx}
% hyperref makes hyperlinks in the resulting PDF.
% If your build breaks (sometimes temporarily if a hyperlink spans a page)
% please comment out the following usepackage line and replace
% \usepackage{icml2025} with \usepackage[nohyperref]{icml2025} above.
\usepackage{hyperref}


% Attempt to make hyperref and algorithmic work together better:
\newcommand{\theHalgorithm}{\arabic{algorithm}}

% Change "review" to "final" to generate the final (sometimes called camera-ready) version.
% Change to "preprint" to generate a non-anonymous version with page numbers.
\usepackage[preprint]{acl}

% For theorems and such
\usepackage{amsmath}
\usepackage{amssymb}
\usepackage{mathtools}
\usepackage{amsthm}



\usepackage{times}
\usepackage{latexsym}

% For proper rendering and hyphenation of words containing Latin characters (including in bib files)
\usepackage[T1]{fontenc}
% For Vietnamese characters
% \usepackage[T5]{fontenc}
% See https://www.latex-project.org/help/documentation/encguide.pdf for other character sets

% This assumes your files are encoded as UTF8
\usepackage[utf8]{inputenc}

% This is not strictly necessary, and may be commented out,
% but it will improve the layout of the manuscript,
% and will typically save some space.
\usepackage{microtype}

% This is also not strictly necessary, and may be commented out.
% However, it will improve the aesthetics of text in
% the typewriter font.
\usepackage{inconsolata}

%Including images in your LaTeX document requires adding
%additional package(s)
\usepackage{graphicx} % Required for including images
\usepackage{luatex85}
\pdfimageresolution=300
\usepackage{threeparttable}
\usepackage{amsmath}
\usepackage{amsfonts}
\usepackage{bm}
\usepackage{subcaption}

\usepackage{inconsolata}
\usepackage{algorithm}
\usepackage{algorithmic}
%\usepackage{tcolorbox} % Load tcolorbox with options
\usepackage{xcolor}
\usepackage{tcolorbox}
% \usepackage{colortbl} % Required for cell coloring
\tcbuselibrary{raster} % Load the raster library for tcolorbox
\definecolor{darkblue}{rgb}{0.04, 0.1, 0.35}
\definecolor{lightblue}{rgb}{0.2, 0.6, 0.9}
\definecolor{lightred}{rgb}{0.9, 0.4, 0.4}
\definecolor{lightyellow}{rgb}{0.8, 0.8, 0.35}
\definecolor{lightpurple}{rgb}{0.6, 0.5, 0.7}
\definecolor{lightorange}{rgb}{1.0, 0.5, 0.0}
\definecolor{lightgrey}{rgb}{0.5, 0.5, 0.5}
\usepackage{booktabs} % For professional looking tables
\makeatletter
\newcommand*\bigcdot{\mathpalette\bigcdot@{.5}}
\newcommand*\bigcdot@[2]{\mathbin{\vcenter{\hbox{\scalebox{#2}{$\m@th#1\bullet$}}}}}
\makeatother
% if you use cleveref..
\usepackage[capitalize,noabbrev]{cleveref}

%%%%%%%%%%%%%%%%%%%%%%%%%%%%%%%%
% THEOREMS
%%%%%%%%%%%%%%%%%%%%%%%%%%%%%%%%
\theoremstyle{plain}
\newtheorem{theorem}{Theorem}[section]
\newtheorem{proposition}[theorem]{Proposition}
\newtheorem{lemma}[theorem]{Lemma}
\newtheorem{corollary}[theorem]{Corollary}
\theoremstyle{definition}
\newtheorem{definition}[theorem]{Definition}
\newtheorem{assumption}[theorem]{Assumption}
\theoremstyle{remark}
\newtheorem{remark}[theorem]{Remark}


\newcommand{\bE}{\mathbb{E}}
\newcommand{\bP}{\mathbb{P}}
\newcommand{\cF}{\mathcal{F}}
\newcommand{\cT}{\mathcal{T}}
\newcommand{\act}{$\mathcal{A}\mathcal{C}\mathcal{T}$}
\newcommand{\eps}{\epsilon}
\newcommand{\ub}{^{(b)}}
\newcommand{\tr}{_{tr.}}
\newcommand{\up}{^{\prime}}
\newcommand{\te}{\theta}
\newcommand{\id}{\mathbbm{1}}
\newcommand{\cB}{\mathcal{B}}
\newcommand{\hp}{\mathcal{H}\mathcal{P}}
\newcommand{\lp}{\mathcal{L}\mathcal{P}}
\newcommand{\bT}{\mathbb{T}}
\newcommand{\bZ}{\mathbb{Z}}
\newcommand{\cX}{\mathcal{X}}
\newcommand{\cY}{\mathcal{Y}}
\newcommand{\cA}{\mathcal{A}}
\newcommand{\cS}{\mathcal{S}}
\newcommand{\ust}{^{\star}}
\newcommand{\cL}{\mathcal{L}}
\newcommand{\bN}{\mathbb{N}}
\newcommand{\bR}{\mathbb{R}}
\newcommand{\cC}{\mathcal{C}}
\newcommand{\cU}{\mathcal{U}}
\newcommand{\cG}{\mathcal{G}}
\newcommand{\cD}{\mathcal{D}}
\newcommand{\og}{\omega}
\newcommand{\cJ}{\mathcal{J}}
\newcommand{\ep}{\epsilon}
\newcommand{\cE}{\mathcal{E}}
\newcommand{\cM}{\mathcal{M}}
\newcommand{\cH}{\mathcal{H}}
\newcommand{\uc}{^{uc}}
\newcommand{\ut}{^{(\theta)}}
\newcommand{\tht}{_{\theta(t)}}
\newcommand{\cI}{\mathcal{I}}
% Todonotes is useful during development; simply uncomment the next line
%    and comment out the line below the next line to turn off comments
%\usepackage[disable,textsize=tiny]{todonotes}
%\usepackage[textsize=tiny]{todonotes}

% Standard package includes
\usepackage{times}
\usepackage{latexsym}

% For proper rendering and hyphenation of words containing Latin characters (including in bib files)
%\usepackage[T1]{fontenc}
% For Vietnamese characters
% \usepackage[T5]{fontenc}
% See https://www.latex-project.org/help/documentation/encguide.pdf for other character sets

% This assumes your files are encoded as UTF8
%\usepackage[utf8]{inputenc}

% This is not strictly necessary, and may be commented out,
% but it will improve the layout of the manuscript,
% and will typically save some space.
\usepackage{microtype}

% This is also not strictly necessary, and may be commented out.
% However, it will improve the aesthetics of text in
% the typewriter font.
\usepackage{inconsolata}

%Including images in your LaTeX document requires adding
%additional package(s)
\usepackage{graphicx}
\usepackage{tcolorbox}
% If the title and author information does not fit in the area allocated, uncomment the following
%
%\setlength\titlebox{<dim>}
%
% and set <dim> to something 5cm or larger.

\title{ \textsc{FLAG-Trader}: Fusion LLM-Agent with Gradient-based \\Reinforcement Learning for
Financial Trading}

% Author information can be set in various styles:
% For several authors from the same institution:
% \author{Author 1 \and ... \and Author n \\
%         Address line \\ ... \\ Address line}
% if the names do not fit well on one line use
%         Author 1 \\ {\bf Author 2} \\ ... \\ {\bf Author n} \\
% For authors from different institutions:
% \author{Author 1 \\ Address line \\  ... \\ Address line
%         \And  ... \And
%         Author n \\ Address line \\ ... \\ Address line}
% To start a separate ``row'' of authors use \AND, as in
% \author{Author 1 \\ Address line \\  ... \\ Address line
%         \AND
%         Author 2 \\ Address line \\ ... \\ Address line \And
%         Author 3 \\ Address line \\ ... \\ Address line}

\author{%
Guojun Xiong\textsuperscript{1},
Zhiyang Deng\textsuperscript{2},
Keyi Wang\textsuperscript{3},
Yupeng Cao\textsuperscript{2},
Haohang Li\textsuperscript{2},
Yangyang Yu\textsuperscript{2},
\\
\textbf{Xueqing Peng\textsuperscript{7}},
\textbf{Mingquan Lin\textsuperscript{4}},
\textbf{Kaleb E Smith\textsuperscript{5}},
\textbf{Xiao-Yang Liu Yanglet\textsuperscript{3,6}},
\\
\textbf{Jimin Huang\textsuperscript{7}},
\textbf{Sophia Ananiadou\textsuperscript{8}},
\textbf{Qianqian Xie\textsuperscript{7,*}}
\\
\\
\textsuperscript{1}Harvard University,
\textsuperscript{2}Stevens Institute of Technology,
\textsuperscript{3}Columbia University,
\\
\textsuperscript{4}University of Minnesota,
\textsuperscript{5}NVIDIA,
\textsuperscript{6}Rensselaer Polytechnic Institute,
\\
\textsuperscript{7}TheFinAI,
\textsuperscript{8}University of Manchester
\\
\small{
\textbf{\textsuperscript{*}Correspondence:} \href{xqq.sincere@gmail.com}{xqq.sincere@gmail.com} 
}
}


%\author{
%  \textbf{First Author\textsuperscript{1}},
%  \textbf{Second Author\textsuperscript{1,2}},
%  \textbf{Third T. Author\textsuperscript{1}},
%  \textbf{Fourth Author\textsuperscript{1}},
%\\
%  \textbf{Fifth Author\textsuperscript{1,2}},
%  \textbf{Sixth Author\textsuperscript{1}},
%  \textbf{Seventh Author\textsuperscript{1}},
%  \textbf{Eighth Author \textsuperscript{1,2,3,4}},
%\\
%  \textbf{Ninth Author\textsuperscript{1}},
%  \textbf{Tenth Author\textsuperscript{1}},
%  \textbf{Eleventh E. Author\textsuperscript{1,2,3,4,5}},
%  \textbf{Twelfth Author\textsuperscript{1}},
%\\
%  \textbf{Thirteenth Author\textsuperscript{3}},
%  \textbf{Fourteenth F. Author\textsuperscript{2,4}},
%  \textbf{Fifteenth Author\textsuperscript{1}},
%  \textbf{Sixteenth Author\textsuperscript{1}},
%\\
%  \textbf{Seventeenth S. Author\textsuperscript{4,5}},
%  \textbf{Eighteenth Author\textsuperscript{3,4}},
%  \textbf{Nineteenth N. Author\textsuperscript{2,5}},
%  \textbf{Twentieth Author\textsuperscript{1}}
%\\
%\\
%  \textsuperscript{1}Affiliation 1,
%  \textsuperscript{2}Affiliation 2,
%  \textsuperscript{3}Affiliation 3,
%  \textsuperscript{4}Affiliation 4,
%  \textsuperscript{5}Affiliation 5
%\\
%  \small{
%    \textbf{Correspondence:} \href{mailto:email@domain}{email@domain}
%  }
%}

\begin{document}
\maketitle
\begin{abstract}
Large language models (LLMs) fine-tuned on multimodal financial data have demonstrated impressive reasoning capabilities in various financial tasks. However, they often struggle with multi-step, goal-oriented scenarios in interactive financial markets, such as trading, where complex agentic approaches are required to improve decision-making. To address this, we propose \textsc{FLAG-Trader}, a unified architecture integrating linguistic processing (via LLMs) with gradient-driven reinforcement learning (RL) policy optimization, in which a partially fine-tuned LLM acts as the policy network, leveraging pre-trained knowledge while adapting to the financial domain through parameter-efficient fine-tuning.  Through policy gradient optimization driven by trading rewards, our framework not only enhances LLM performance in trading but also improves results on other financial-domain tasks. We present extensive empirical evidence to validate these enhancements.
\end{abstract}

%具身智能体在复杂场景下 manipulation 的 performance robustness 和泛化能力始终是一个广受关注的研究方向。其中,visuomotor imitation learning 是具身智能体 Policy 的主流范式之一,它允许 agent 从高维视觉观察和机器人本体感知中 effectively 学习 manipulation skills。
%然而,增加场景的复杂度和 visual distraction,会导致在简单场景下表现良好的决策模型性能下降。实际上,不仅是 simple imitation learning policy,先进的多模态 foundation models such as GPT-4o 或 vision language action models (VLA),也不能很好地关注一张语义丰富的图片中的特定的局部问题。对于 robot control or 多模态大模型,其往往侧重于 action prediction, observation mapping or 多模态 alignment,而缺少直观的视觉感知增强。模型需要隐性地或遵循 high-level text instruction 从相关的视觉区域中获得面向任务语义的定位知识。
%To tackle this challenge problem, we introduce Imit Diff, a diffusion transformer imitation learning framework with dual resolution enhancement guided by fine-grained semantics information。具体来说,our work 有三个关键组成部分。
%1) Semanstic Injection. Imit Diff 通过 vision language models (VLM) 和 vision foundation models 的 pretrain knowledge 将面向任务的语义信息和高层文本指导转化为显式的 pixel-level 视觉定位标签,注入到 environment observation中。
%2) Dual Res Fusion。 我们构建了双分辨率图像观测流,使用双分辨率视觉编码器分别提取全局和细粒度视觉特征。多尺度视觉信息随后在 attention block 中进行融合,在保证计算 effiency 的前提下,为全局视觉观测引入多尺度细粒度信息,提升场景理解能力。
%3) Consistency policy on diffusion transformer。Diffusion based imitation policies 通常受到 denoise times 的困扰。我们建立了基于 consistency policy 的 DiT action head。Policy 的决策层可以通过 single step denoise 实现系统高频响应。额外地,受益于较快的 inference time,我们引入 temperal ensemble 改善预测动作的平滑性。
%我们设计了四个在 manipulation 精细度上具有挑战性的现实世界任务来评估 Imit Diff,并通过增加场景复杂度和 visual distraction 来测试模型的场景理解能力。额外地,我们设计了 visual distraction 和 category generalization 的 zero shot 实验来验证模型是否受益于 dual res enhancement framework and fine-grained semantics injection。实验结果表明,Imit Diff outperforms 现有的 strong baselines。
%In summary, the contributions of our work are three-fold:
%1) We propose Imit Diff, a DiT architecture imitation learning framework with dual res enhancement guied by fine-grained semantics information.
%2) 我们构建了 open-set vision foundation models pipeline 来获得显式视觉遮罩。该方法能够有效处理机器人控制场景的运动模糊、遮挡、物体丢失情况。并将其作为 fine-grained 语义信息引导 policy decision。
%3) 我们在DiT上实现了consistency policy,显著减少了模型推理时间。通过异步控制框架,实现了 open-set vision foundation models 工作流下的实时控制。
%The code will be publicly available soon。

\section{Introduction}


\label{Intro}
The performance robustness and generalization capabilities of embodied agents in complex manipulation scenarios have long been a focus of significant research interest \citep{ju2025robo, yuan2024learning}. Visuomotor imitation learning is one of the mainstream paradigms of robot manipulation policy \citep{chi2023diffusion, shridhar2023perceiver, ze2023gnfactor, florence2022implicit, hansen2022pre}. This approach enables agents to derive state estimation and decision-making capabilities from expert demonstrations that incorporate high-dimensional visual observations and robot proprioception \citep{ze20243d}.

However, as scene complexity and visual distractions increase, the performance of decision models that excel in simpler environments tends to degrade \citep{zheng2024instruction, liurobustness}. Not only do simple imitation learning policies face challenges, but even advanced multimodal foundation models, such as GPT-4o \citep{hurst2024gpt} or vision language action models (VLA) \citep{liu2024rdt, brohan2022rt, brohan2023rt, o2023open, kim2024openvla, wen2024diffusion}, struggle to accurately focus on specific details within semantically complex images. In fact, in robot control and embodied multimodal foundation models, the focus is often on action prediction, observation mapping, or multimodal alignment. Therefore, intuitive visual perception enhancement is typically lacking. Models can only acquire task-oriented semantic localization knowledge from relevant visual regions either implicitly or when guided by high-level text instructions \citep{reuss2023multimodal}.

To tackle this challenge problem, we introduce \textbf{Imit Diff}, a diffusion transformer imitation learning framework with dual resolution enhancement guided by fine-grained semantics information. Specifically, our work has three key components:

\begin{enumerate}

\item \textbf{Semanstic injection.} Imit Diff transforms task-oriented semantic information and high-level textual guidance into explicit pixel-level visual localization labels through the pretrain knowledge of vision language models (VLM) and vision foundation models, and injects them into the policy observation.

\item \textbf{Dual resolution (dual res) fusion.} We develop a dual res image observation stream and employed a dual res vision encoder to extract global and fine-grained visual features. The extracted multi-scale visual information is subsequently fused within an attention block, integrating fine-grained details into the global visual feature. This approach enhances scene understanding while maintaining computational efficiency.

\item \textbf{Consistency policy on diffusion transformer (DiT).} Diffusion-based imitation policies often suffer from inefficiencies due to the required denoising steps. To address this, we design a DiT \citep{peebles2023scalable} action head incorporating a consistency policy \citep{song2023consistency}, enabling the decision layer to achieve high-frequency system responses through single-step denoising. Furthermore, leveraging faster inference times, we introduce temperal ensemble to enhance the smoothness of predicted actions.

\end{enumerate}

We design four real-world tasks with challenging manipulation precision to evaluate Imit Diff and test the model's scene understanding capabilities by introducing increased scene complexity and visual distractions. Additionally, we conducted zero-shot experiments on visual distraction and category generalization to assess the benefits of the dual res enhancement framework and fine-grained semantic injection. Experimental results demonstrate that Imit Diff significantly outperforms existing strong baselines. 

In summary, the contributions of our work are three-fold:

\begin{enumerate}

\item We propose Imit Diff, a DiT architecture imitation learning framework with dual res enhancement guied by fine-grained semantics information.

\item We developed an open-set vision foundation model pipeline to generate explicit visual masks. This approach effectively addresses challenges such as motion blur, occlusion, and object loss in robot control scenarios, leveraging the generated masks as fine-grained semantic information to guide policy decisions.

\item We implemented a consistency policy on DiT, which significantly reduced the model inference time. Through the asynchronous control framework, we achieved real-time control under the workflow of open-set vision foundation models.

\end{enumerate}

The code will be made publicly available soon.
%%%
\section{Related Work}
\label{sec:related work}
% In this section, we review the existing literature on point cloud denoising and unsupervised image denoising.
%-------------------------------------------------------------------------
\subsection{Point cloud denoising}

    \subsubsection{Traditional methods}
Traditional approaches to point cloud denoising include statistical methods \cite{computingpointset2003,definingpointset2004,wlop2009HH}, filtering techniques\cite{pointsetsurfaces2001,Robustmoving2005, zaman2017density}, and optimization-based methods \cite{l1sparse2010,clop2014PR,digne2017bilateral,multi-projection2018duan,hu2020featuregraph} . These techniques often rely on handcrafted features and heuristics to distinguish signal from noise. For example, statistical methods may use distribution models to identify and remove outliers. Filtering methods, such as mean or median filters, operate under the assumption that noise is statistically different from the signal. Optimization-based methods formulate denoising as an energy minimization problem, where regularization terms constrain the solution to ensure certain smoothness cirterion or adherence to prior knowledge.

%-------------------------------------------------------------------------
    \subsubsection{Supervised learning based methods}
In recent years, several deep learning-based methods \cite{rakotosaona2020PCN,hermosilla2019TotalDenoising,luo2020DMR,luo_score-based_2021} have been proposed for point cloud denoising. NPD \cite{NPD2019} is the first learning-based point cloud denoising method that directly processes noisy data without requiring noise characteristics or neighboring point definitions. This approach combines local and global information by projecting noisy points onto estimated reference planes, effectively removing noise and enhancing robustness against variations in noise intensity and curvature. PointCleanNet\cite{rakotosaona2020PCN} first removes outlier points and then combines them with residual connectivity to predict the inverse displacement \cite{Guerrero2017PCPNetLL}, and iteratively shifts noisy points to remove noise. Pistilli \etal proposed GPDNet \cite{gpdnet2020}, which is a graph convolutional network to improve denoising robustness at high noise levels. Luo \etal also proposed  DMRDenoise \cite{luo2020DMR}, which filter
points by first downsampling the noisy inputs and reconstructing the local subsurface to perform point upsampling. However, this resampling method is difficult to maintain a good local shape. ScoreDenoise \cite{luo_score-based_2021} is proposed to tackle the aforementioned issues by iteratively updating the point position in implicit gradient fields learned by neural networks. For inference, they follows an iterative procedure with a decaying step size, which stabilizes point movement and prevents over-correction, allowing points to converge gradually toward the underlying geometry. The authors of \cite{de_Silva_Edirimuni_2023_CVPR} proposed IterativePFN - an iterative method that use a novel loss function that utilizes an adaptive ground truth target at each iteration to capture the relationship between intermediate filtering results during training. Zheng \etal proposed a end-to-end network for joint normal filtering-based point cloud denoising \cite{10173632}. They introduce an auxiliary normal filtering task to enhance the network's ability to remove noise while preserving geometric features more accurately.

Supervised methods can achieve impressive results, but are limited by the availability and quality of the training data, as they typically require paired noisy and clean point clouds to train the neural network. The absence of clean data in real-world scenario pose a significant challenge on applicability of these algorithms.

%-------------------------------------------------------------------------
    \subsubsection{Unsupervised learning methods}
Unsupervised learning-based methods for point cloud denoising do not require ground-truth clean data. Instead, these methods leverage the inherent structure or distribution of the point cloud to guide the denoising process. Unsupervised methods show promise in scenarios where clean data is absent or hard to obtain.

Hermosilla \etal first introduced Total Denoising (TotalDn) \cite{hermosilla2019TotalDenoising} as an unsupervised learning approach for point cloud denoising, relying solely on noisy data without requiring clean ground truth. TotalDn approximates the underlying surfaces by regressing points from the distribution of unstructured total noise, utilizing a spatial prior term to refine the estimation of geometry. 

In DMRDenoise \cite{luo2020DMR}, an unsupervised version is proposed which utilizes a loss function that identify local neighborhoods using a probabilistic Gaussian mask on the k-nearest neighbors, which selectively retains points likely to represent the underlying surface. By leveraging an Earth Mover's Distance (EMD) assignment, it achieves a one-to-one correspondence between input and predicted points, aligning them to reduce noise within local neighborhoods.
This approach enhances robustness to noise and adapts well to varied surface geometries. However, the probabilistic masking and EMD calculation add computational complexity, which can slow down inference in dense or noisy point clouds. 

ScoreDenoise \cite{luo_score-based_2021} also introduced an unsupervised version that employs ensemble score function and an adaptive neighborhood-covering loss for model training.  
Score-u is probably the most relevant work to our method. However, the defined score in \cite{luo_score-based_2021} is only an displacement-alike version of the score function and there is no explicit formula relating the estimated score to the final denoising result. Also, the iterative process is computationally expensive, and can suffer from fluctuation, leading to perturbed and unstable solution.

Most recently, Noise4Denoise \cite{noise4Wang2024} method is proposed that use an additional doubly-noisy point cloud to learn noise displacement by comparing the two noise levels. This approach effectively leverages synthetic noise for training, allowing the model to estimate residuals without relying on clean data.
However, in practical applications, noise parameters are often unknown and variable, making it challenging to replicate the exact conditions assumed during training. This reliance on predefined noise characteristics can limit the model's applicability to real-world scenarios where noise distributions may differ significantly from synthetic settings. 
%-------------------------------------------------------------------------
\subsection{Unsupervised image denoising}
Recently unsupervised image denoising has made significant progress. Non-Bayesian methods include PURE \cite{luisier2010image}, SURE \cite{SURE2018} \textit{etc.}, which are based on various unbiased risk estimator under certain noise distribution. Other methods explore self-similarity in natural images \cite{xu2015patch, doi:10.1137/23M1614456} or exploits the statistical properties of noise to achieve denoising effect \cite{gravel2004method}.  

Noise2Noise \cite{2018Noise2NoiseLI} is a pioneering method that does not require clean data, but it requires multiple noisy versions of the same image for training. To address this limitation, methods such as Noise2Void \cite{2018Noise2VoidL}, Noise2Self \cite{2019Noise2SelfBD}, \textit{etc.}, have been developed, which use only a single noisy image. These methods are particularly important for practical applications where obtaining clean images or multiple noisy realizations of the same image is difficult or impossible. Neighbor2Neighbor \cite{huang2021neighbor2neighbor} proposed a two-step method with a a random neighbor sub-sampler that generates training  pairs and a denosing network. Kim \etal proposed Noise2Score\cite{kim_noise2score_2021}, a novel Bayesian framework for self-supervised image denoising without clean data. The core of Noise2Score is the usage of Tweedie's formula, which provides an explicit representation of the denoised image through a score function. Combined with score function estimation, Noise2Score can be applied to image denoising with any exponential family noise. Kim \etal also proposed the Noise Distribution Adaptive Self-Supervised Image Denoising method \cite{kim_noise_2022}, which further extends the application of Noise2Score by combining the Tweedie distribution with score matching. This enables adaptive handling of various noise distributions and dynamically adjusts the denoising process by estimating noise parameters. On the other hand, Xie \etal \cite{scoreXie2024} broadened the denoising scope of Noise2Score by allowing it to handle complex noise models, such as multiplicative and structurally correlated noise, demonstrating broad applicability to diverse noise models.

These development of unsupervised image denoising method motivate us to explore similar ideas in 3D point cloud denoising.




%%%
\section{Problem Statement}\label{Sec:Problem Formulation}

We define the financial decision-making process as a finite horizon partially observable Markov decision process (MDP) with time index $\{0,\cdots,T\}$, represented by the tuple:
$\mathcal{M} = (\mathcal{S}, \mathcal{A}, \mathcal{T}, R, \gamma),$
where each component is described in detail below.

\textbf{State.} The state space $\cS=\cX\times\cY$ consists of two components: market observations and trading account balance, i.e.,
$s_t = (m_t, b_t) \in \mathcal{S}.$ Specifically,
     $ m_t = (P_t, N_t) \in \cX$ represents the \textit{market observation process}, includingstock price $P_t$ at time $t$, and financial news sentiment or macroeconomic indicators $N_t$;
   $ b_t = (C_t, H_t) \in \mathcal{Y}$ represents the \textit{trading account balance}, including available cash $C_t$ at time $t$, and number of stock shares $H_t$.

\textbf{Action.} The agent chooses from a discrete set of trading actions
$\mathcal{A} = \{\texttt{Sell}: -1, \texttt{Hold}: 0, \texttt{Buy}: 1\},$
where $a_t=-1$ denotes selling all holdings (liquidate the portfolio),
$a_t=0$ denotes holding (no trading action), and
$a_t=1$ represents buying with all available cash (convert all cash into stocks).


 \textbf{State Transition.} The state transition dynamics are governed by a stochastic process
$s_{t+1} \sim \cT(\cdot | s_t, a_t).$
% Specifically, the stock price $P_t$ follows a stochastic process:
%     $
%     P_{t+1} = P_t + f_m(m_t, a_t) + \xi_t$ with $\quad \xi_t \sim \mathcal{N}(0, \sigma^2).$
The trading account evolves according to the following equations:
    \begin{itemize}
        \item If \texttt{Sell}:
        $        C_{t+1} = C_t + H_t P_{t+1}, \quad H_{t+1} = 0.
        $
        \item If \texttt{Hold}:
        $ C_{t+1} = C_t, \quad H_{t+1} = H_t.$
        \item If \texttt{Buy}:
       $        C_{t+1} = 0, \quad H_{t+1} = H_t + \frac{C_t}{P_{t+1}}.$ 
    \end{itemize}

\textbf{Reward.} 
The agent receives a reward  based on the daily trading profit \& loss (PnLs):
\begin{align*}
    R(s_t, a_t) = SR_t-SR_{t-1},
\end{align*}
where $SR_t$ denotes the Sharpe ratio at day $t$, computed by using the historical PnL from time 0 to time $t$. Moreover, PnL at time $t$ is calculated as
\begin{align*}
pnl_t:=(C_t-C_{t-1})+(H_tP_t-H_{t-1}P_{t-1}).   
\end{align*}
Then, the Sharpe ratio $SR_t$ at time $t$ can be calculated as:
\begin{align}
    \label{eq:sharpe}
    SR_t:=\frac{\mathbb{E}[pnl_1,\cdots,pnl_t]-r_f}{\sigma[pnl_1,\cdots,pnl_t]},
\end{align}
where $\mathbb{E}[pnl_1,\cdots,pnl_t]$ is the sample average of daily PnL up to time $t$, $r_f$ is the risk-free rate, and $\sigma[pnl_1,\cdots,pnl_t]$ is the sample standard deviation of daily PnL up to time $t$.

 
The goal is to find an admissible policy $\pi$ to maximize the expected value of cumulative discounted reward, i.e., 
\begin{align}\label{eq:global_obj}
   \max_{\pi} V^{\pi}(s) = \mathop{\mathbb{E}}\limits_{\substack{s_0=s,a_t\sim \pi(\cdot|s_t)\\ s_{t+1}\sim \cT(\cdot|s_t,a_t)}} \left[ \sum_{t=0}^T \gamma^t R_t \right],
\end{align}
where $R_t$ is a shortened version $R(s_t, a_t)$ and  $ \gamma \in (0,1]$ is the discount factor controlling the importance of future rewards.



Our goal is to train an LLM agent parameterized by $\theta$ to find the optimized policy $\pi_\theta$ for \eqref{eq:global_obj}, i.e.,
\begin{align}
a_t\sim\pi_\theta(\cdot|s_t)=\text{LLM}(\texttt{lang}(s_t);\theta),
\end{align}
where $\texttt{lang}(s_t)$ are the prompts generated by converting state $s_t$ into structured text. The proposed pipeline is illustrated in Figure \ref{fig:FinRL_LLM}. 




% Formally, we model a financial decision-making process as infinite horizon POMDP with time index $\mathbb{T}=\{0,1,2,\cdots, \infty\}$ and discount factor $\gamma\in(0,1]$. This POMDP contains: (1) a state space $\mathcal{S}:=\mathcal{X}\times\mathcal{Y}$ where $\mathcal{X}$ is the observable component and $\mathcal{Y}$ is unobservable component of the financial market; (2) the action space of the agent is $\mathcal{A}$, which is modeled as $\{\textit{``Buy",~``Sell",~``Hold"}\}$; (3) the reward function $R(o,b,a):\mathcal{X}\times\mathcal{Y}\times\mathcal{A}\to\mathbb{R}$ uses daily profit \& loss (PnL) as the output; (4) the observation process $\{o_t\}_{t\in\mathbb{T}}\subseteq\mathcal{X}$ is a multi-dimensional process (5) the reflection process $\{b_t\}_{t\in\mathbb{T}}\subseteq\mathcal{Y}$ represents the agent's self-reflection, which is updated from $b_t$ to $b_{t+1}$ on daily basis \cite{griffiths2023bayes}; (6) the action $a_t\sim\pi_\theta(\cdot|\text{prompt})$ represents the way to make investment decision driven by the language conditioned policy $\pi_\theta$ parameterized by $\theta$. By denoting daily profit \& loss (PnLs) by $r_t(s_t,a_t)=R(s_t:=(o_t, b_t), a_t)$, the optimization objective is to maximize the cumulative reward over a sequence of trading decisions, defined as
% \begin{align}
%    J(\theta) = \mathbb{E}_{(s_t, a_t)\sim \pi_\theta} \left[ \sum_{t=0}^\infty \gamma^t r_t(s_t,a_t) \right].
% \end{align}
% To improve the policy, the gradient of the policy is optimized using the policy gradient loss:
% \begin{align}
%     L(\pi_\theta) = -\mathbb{E}_t \left[ \log \pi_\theta(a_t | s_t) A_t \right],
% \end{align}
% where $A_t$ represents the advantage function evaluating the relative value of choosing a specific action.




% The objective of this research is to fine-tune large language models (LLMs) for financial decision-making in sequential trading environments. The model is designed to generate interpretable, full-sentence outputs such as \textit{"buy AAPL 100 shares,"} which are subsequently mapped into a simplified action space of \textbf{``buy,'' ``sell,'' or ``hold.''} Each decision is based on the current state, represented as $s_t = \{m_t, p_t, h_t\} $, where $m_t$ denotes market indicators, $p_t$ represents historical price information, and $h_t$ reflects the portfolio status.






%%%
\section{Method}
\label{section:method}
In this section, we first formalize the multi-agent task and MCTS-based data synthesis (§~\ref{subsection:multi-agent-data-synthesis}), then introduce the data influence-oriented data selection
%\cx{not sure whether we should introduce agent-aware here to name our method, we should focus on influence-driven MCTS?} hybrid \cx{also don't think we should say hybrid, focus on our novelty, then it is naturally embedded in existing MCTS methods} data selection using influence scores 
(§~\ref{subsection:hybrid data selection}), and finally present the iterative data synthesis process (§~\ref{subsection:iterative data synthesis}).

\begin{figure*}
    \centering
    % \vspace{-0.5cm}
    \includegraphics[width=0.9\linewidth]{figure/Main_structure.pdf}
    \caption{Overview of our method. (a) illustrates the traversal of a cyclic agent network in topological order. We introduce virtual agents to distinguish the same agent in the traversal. (b) showcases the application of MCTS to generate synthetic multi-agent training data, where the color of each agent represents the magnitude of the node's Q-value. (c) depicts the computation process of influence scores for a non-differentiable metric, highlighting that data points with high Q-values may correspond to low influence scores.} 
    \vspace{-0.3cm}
    \label{fig:framework}
\end{figure*}

\subsection{Multi-Agent Data Synthesis} \label{subsection:multi-agent-data-synthesis}
% How to generalize to multi-agent network
In this work, we model the topology structure for multi-agent collaboration as a directed graph. Concretely, we denote a feasible topology as $\mathcal{G}=(\mathcal{V}, \mathcal{E})$, as demonstrated in Figure~\ref{fig:framework} (a). We allow the presence of cycles in the graph, indicating that multiple rounds of information exchange are permitted among agents. We assume that our agent network can be linearly traversed in topological order $A_1\oplus A_2 \oplus \cdots \oplus A_M$~\cite{Bondy1976, book:gross:2005, qian2024}, where $A_m \in \mathcal{V}$. Different $A_m$ may represent the same agent being visited at different time steps. For clarity and convenience, we use different symbols to distinguish them.

% Why chose MCTS
% \cx{we should have a more mathematical and structured description of our system. Use more equations to present the process.}

In this way, we could utilize MCTS to synthesize training data for MAS. We mainly follow the configuration in Optima~\cite{DBLP:journals/corr/abs-2410-08115} and construct the tree as follows: As shown in Figure~\ref{fig:framework} (b), the synthesis tree begins with a specific task instruction $p$. 

\textbf{Selection}: We select a node $n$ to
expand from the candidate node set, where a node $n=(s,a)$ refers to an agent $A_m$ in state $s$ that takes action $a$. We use the edit distance to filter out nodes that are similar to expanded nodes to obtain the candidate node set.
\begin{equation}   
    N_{\text{cand}} = \{n_j| n_i \in N_{\text{expanded}}, n_j\in N_{\text{all}}, S_{i,j} \geq 0.25 \},
\end{equation}
where $S_{i,j}=\frac{\text{edit\_distance}(n_i, n_j)}{\max (|n_i|,|n_j|)}$ and $\text{edit\_distance}(n_i, n_j)$ represents the edit distance between the action strings of two nodes. $N_\text{all}$ and $N_{\text{expanded}}$ denotes the whole node set and expanded node set. Then we select a node for the candidate set $N_{\text{cand}}$ based on softmax distribution of Q-values.
\begin{equation}
n\sim \text{Softmax}(\{Q(n)\}_{n\in N_{\text{cand}}}),
\end{equation}
where $Q(n) = Q(s, a)$ and the softmax distribution balances exploration and exploitation during the search process.


\textbf{Expansion} For each selected node $n$, we denote the new state as $s'=\text{Trans}(s, a)$, where $\text{Trans}(\cdot)$ is the transit function determined by the environment. Then we sample $d$ actions from LLM paramtered agents $A_{m+1}$: 
\begin{equation}
    \{a_1', \cdots, a_d' \} \sim A_{m+1}(s').
\end{equation}

\textbf{Simulation} For each generated action $a_i'$, we simulate the agent interaction $\tau_i$ until the termination state.
\begin{equation}
    \tau_i = \text{Simulation}(A_{m+2}, \cdots, A_{M}, s', a_i').
\end{equation}
Meanwhile, we construct all $(s, a)$ pairs in the trajectory as new nodes and add them to $N_{\text{all}}$.

\textbf{Backpropagation} Once a trajectory $\tau$ is completed, we can obtain the trajectory reward $R(\tau)$ detailed in Appendix~\ref{appendix:method_details}. We update the Q-value of all nodes in the trajectory with the average Q-value of their children.
\begin{equation}
    Q(n) = Q(s,a) = \sum_{n'\in \text{Child}(n)} \frac{1}{|\text{Child}(n)|} Q(n'),
\end{equation}
where $\text{Child}(n)$ denotes the children set of node $n$. Additionally, due to the complex interactions among multiple agents, the Q-value estimates obtained from $d$ rollouts may be inaccurate. Allocating more inference budget in the data synthesis phase may improve the quality of the generated data and enhance the system’s performance.

We repeat the above process $k$ times and finish the generation process. Then we can construct paired action preferences for agent $A_i$ at state $s$ by selecting the action $a_i^h$ with the highest Q-value and the action $a_i^l$ with the lowest Q-value to form the preference data:
\begin{equation}
    z  = \left(s, a_i^{h}, a_i^{l}\right).
\end{equation}
To update the parameter of agent $A_i$, we utilize the Direct Preference Optimization (DPO) loss to directly encourage the model to prioritize responses that align with preferences $a_i^{h}$ over less preferred ones $a_i^{l}$.
\begin{equation}
\small
    \mathcal{L}_{DPO} = \mathbb{E}_{z} \bigg[ 
    -\log \sigma \bigg( \beta \bigg[
        \log \frac{\pi_\theta(a_i^{h} \mid s)}{\pi_{\text{ref}}(a_i^{h} \mid s)} 
        - \log \frac{\pi_\theta(a_i^{l} \mid s)}{\pi_{\text{ref}}(a_i^{l} \mid s)}
    \bigg] \bigg) \bigg],
\end{equation}
where $\sigma(\cdot)$ denotes the sigmoid function, and $\pi_{\text{ref}}$ represents the reference model, typically the SFT model.



% In this way, we could utilize MCTS to synthesize training data for MAS. As shown in Figure~\ref{fig:framework} (a), the synthesis tree begins with a specific task instruction  $p$, where all nodes at the $i$-th layer correspond to the $i$-th agent $A_i$. DITSinct upstream trajectories $s$ represent varying inputs to agent $A_i$ under different conditions. The divergent branches of a node illustrate the potential actions an agent can take given the particular input. The Q-value for a specific action $Q_i(s, a)$ by agent $A_i$ is estimated through multiple rollouts $\{o_1, \cdots, o_m \}$, which randomly simulate transitions to a terminal state based on the current agents' policy. 
% \begin{equation}
%     Q_i(s,a) = \frac{1}{m} \sum_j R(s,a,o_j),
% \end{equation}
% where $R(s,a,o_j)$ denotes the trajectory-level reward with outputs $o_j$. The Q-value serves as a measure of each agent's contribution to the final answer in the MAS to some extent. \cx{no experimental detail in methodology section} In this work, we follow the MCTS configuration in Optima~\cite{DBLP:journals/corr/abs-2410-08115}, with detail available in Appendix~\ref{}. When the tree search is completed, we can construct paired action preferences for each agent $A_i$ at state $s$ by selecting the action $a_i^{h}$ with the highest Q-value and the action $a_i^{l}$ with the lowest Q-value to form the preference data.
% \begin{equation}
%     z = (x,y^w,y^l) = \left(s, a_i^{h}, a_i^{l}\right),
% \end{equation}
% Since the feasible action space is infinite, MCTS may fail to generate the optimal action within a limited number of $d$ expansions. Additionally, due to the complex interactions among multiple agents, the Q-value estimates obtained from $m$ rollouts may be inaccurate. These factors can result in the collected data containing low-quality samples and significant noise.

% \cx{up to here I am thinking that we may not need Sec 3, we can describe MCTS and DPO as part of our method, then IF as the selection part. This makes the method presentation more streamlined}
% \cx{e.g., we can describe the MCTS process formally, its goal, the roll out, the search space, the priority used in the search, and the DPO upon it. Then how we use Influ to define the search part. (as selection)}
% \cx{after that we can have a 4.3 to recap the full process, including iterative synthetis, training process, and inference}

% As shown in Figure~\ref{fig:scaling}, in this work we first empirically validate that increasing the number of inference tokens during data synthesis for MAS to expand more candidate actions and rollouts more times can improve the quality of the generated data and subsequently enhance the system's performance.\cx{this reads like syntehtic time scaling is property of baseline MCTS, and readers may think it is obvious that more syntheized data is going to lead to better performance. Put synthetize time scaling as an observation, influe-MCTS as the main contribution which leads to the scaling (or at least better scaling) is the right choice.}

\subsection{Data Influence-Oriented Data Selection}
\label{subsection:hybrid data selection}

% The synthesis time scaling approach mentioned in section~\ref{section:multi-agent-data-synthesis} is straightforward but not optimal. This is due to the vast action space, which results in low scaling efficiency. Moreover, selecting the action with the highest Q-value may not yield much improvement for the current model. For example, \cite{} suggests that actions with higher advantages are more valuable for training the current model compared to actions with merely high Q-values. Hence, in this section, we propose a novel scaling dimension for data synthesis in MAS, aimed at finding the most valuable data for model training.

While improving the accuracy of Q-value estimation can enhance data quality to some extent, it is both highly inefficient and suboptimal. During the training phase, the primary goal of synthetic data is to maximize its contribution to model performance improvement, rather than ensuring the data is correct. As shown in Figure~\ref{fig:framework} (c), although the data pair $z_1$ has a higher $Q(s,a_i^h)$, the data pair $z_2$ contributes more significantly to system performance improvement, as reflected by its higher influence score. This discrepancy may arise because the action in $z_2$ provides a greater advantage compared to the less preferred action.

% Our approach is inspired by data influence functions~\cite{}. The function $\mathcal{I}$ was developed to measure the difference in reference loss when a data point is assigned a higher weight in the training dataset.
% % \begin{equation}
% %     \mathcal{M}^* = \arg\min \sum_{i=0}^{n} \frac{1}{n} L(x_i, \mathcal{M})
% % \end{equation}
% \begin{equation}
%     \mathcal{M}_{\epsilon, z_i}^* = \arg\min \sum_{i=0}^{n} \frac{1}{n} L(z_i, \mathcal{M}) + \epsilon L(z_i, \mathcal{M}),
% \end{equation}
% \begin{equation}
%     \mathcal{I}_{\mathcal{M}^*}(z_i, \mathcal{D}_r) \stackrel{\text{def}}{=} \mathcal{F}(\mathcal{D}_r | \mathcal{M}_{\epsilon, z_i}^* ) -\mathcal{F}(\mathcal{D}_r |  \mathcal{M}^* ),
%     \label{equation:definition_data_influence}
% \end{equation}
% where $\mathcal{M}$ represents the agent parameters, $L$ represents the training loss, $\mathcal{F}(\mathcal{D}_r|\mathcal{M})$ indicates the test metric $\mathcal{F}$ achieved by agent parameter $\mathcal{M}$ on the reference dataset $\mathcal{D}_r$.

Hence, in this paper, we introduce the influence score $\mathcal{I}$ to quantify the impact of data on the current agent's performance. 
The influence score $\mathcal{I}$ was developed to measure the difference in loss when a data point is assigned a higher weight in the training dataset.  Suppose the agent $A$ is parameterized by $\theta$. We denote the optimal parameters learned by minimizing the training loss $\mathcal{L}_{\text{tr}}$ on the dataset $\mathcal{D}_{\text{tr}}$, with a data point $z_i$ assigned an additional weight of $\epsilon$, as:
\begin{equation}
    \theta_{\epsilon, z_i}^* = \underset{\theta}{\arg\min} \sum_{z_j\in \mathcal{D}_{\text{tr}}} \frac{1}{|\mathcal{D}_{\text{tr}}|} \mathcal{L}_{\text{tr}}(z_j, \theta) + \epsilon \mathcal{L}_{\text{tr}}(z_i, \theta).
\end{equation}
Under standard assumptions, such as the twice-differentiability and strong convexity of the loss function $\mathcal{L}_{\text{tr}}$, the influence function can be derived via the chain rule of the derivatives~\cite{DBLP:conf/icml/KohL17}:
\begin{equation}
\begin{split}
    \mathcal{I}_{\mathcal{L}_{\text{tr}}}(z_i, \mathcal{D}_{\text{tr}}) &\stackrel{\text{def}}{=} \frac{d \mathcal{L}_{\text{tr}}(z_i, \theta_{\epsilon, z_i}^*)}{d\epsilon} \bigg|_{\epsilon=0} \\
    &\approx -\nabla_\theta \mathcal{L}_{\text{tr}} \big|_{\theta=\theta^*}^T H(\mathcal{D}_{\text{tr}}; \theta_{\epsilon, z_i}^*)^{-1} \nabla_\theta \mathcal{L}_{\text{tr}} \big|_{\theta=\theta^*},
    \label{equation:influence score}
\end{split}
\end{equation}
where $H(\mathcal{D}; \theta) := \nabla_\theta^2\left(\frac{1}{|\mathcal{D}|} \sum_{z\in \mathcal{D}}L_{\text{tr}}(z; \theta)\right)$ and $\nabla_\theta L_{\text{tr}} = \nabla_\theta L_{\text{tr}}(z;\theta)$.

However, the DPO loss does not effectively align with downstream task performance. Our experiments reveal a weak correlation (less than 0.2) between the DPO loss and performance metrics $\mathcal{F}$ such as F1-score or Accuracy on the validation set. This observation is consistent with findings reported in~\citet{DBLP:journals/corr/abs-2406-02900, DBLP:journals/corr/abs-2410-11677}. This indicates that we must redefine the influence score using the changes of non-differentiable performance metrics on the validation set.
\begin{equation}
\mathcal{I}_{\mathcal{F}_{\text{val}}}(z_i, \mathcal{D}_{\text{val}}) := \frac{\mathcal{F}_{\text{val}}(z_i, \theta_{\epsilon,z_i}^*) - \mathcal{F}_{\text{val}}(z_i, \theta^*)}{\epsilon}.
\end{equation}

Due to non-differentiable metric $\mathcal{F}_{\text{val}}$, the influence function cannot be directly derived using gradients or the chain rule. Instead, we use a finite difference method combined with parameter perturbation to approximate the rate of change. The perturbed optimal parameter $\theta_{\epsilon,z_i}^*$ can be rewritten as:
\begin{equation}
    \theta_{\epsilon,z_i}^* = \theta^* + \epsilon\Delta \theta + o(\epsilon),
\end{equation}
where $\Delta\theta$ represents the direction of parameter change. Following~\citet{yu2024mates}, the direct is typically driven by the gradient of the training loss.
\begin{equation}
    \Delta \theta \propto -\nabla_\theta \mathcal{L}_{\text{tr}}(z_i, \theta^*).
\end{equation}
Since the parameter update is dominated by the training loss gradient, we adopt a one-step gradient descent update:
\begin{equation}
    \theta_{\epsilon,z_i}^* \approx \theta^* - \eta\epsilon\nabla_\theta \mathcal{L}_{\text{tr}}(z_i,\theta^*),
\end{equation}
where $\eta$ is the learning rate, and $\epsilon$ is a very small perturbation strength. Combining the finite difference and parameter update, the influence function is approximated as:
\begin{equation}
\begin{split}
    &\mathcal{I}_{\mathcal{F}_{\text{val}}}(z_i, \mathcal{D}_{\text{val}}, \theta^*) \approx \\ &\frac{1}{\epsilon} \left[ \mathcal{F}_{\text{val}}(z_i,  \theta^* - \eta\epsilon\nabla_\theta L_{\text{tr}}(z_i,\theta^*)) \right. 
     \left. - \mathcal{F}_{\text{val}}(z_i,\theta^*) \right].
\end{split}
\end{equation}

Compared to performing simulations to estimate Q-value more accurately, conducting inference on a validation dataset to estimate data influence can better guide the selection of higher-quality data points.

Specifically, Our selection strategy combines Q-values and influence scores to effectively identify the highest-quality pair data: 
\begin{equation}
\begin{split}
    H(z_i) = \mathcal{I}_{\mathcal{F}_{\text{val}}}(z_i, \mathcal{D}_{\text{val}}, \theta) + \gamma \cdot Q(s, a_i^h) , 
    \label{equation:hybrid score}
\end{split}
\end{equation}
where $\theta$ denotes the current parameters of agent $A_m$. Finally, after filtering out low-quality data as described in~\citet{DBLP:journals/corr/abs-2410-08115}, synthetic data are ranked based on the scores, and the Top $\alpha$ are selected to construct the training dataset $\mathcal{D}_{\text{tr}}$.

{
\setlength{\textfloatsep}{0em}
\begin{algorithm}[t]
\caption{DITS-iSFT-DPO}
\label{alg:DITS-isft-dpo}
\begin{algorithmic}[1]
\REQUIRE Initial model $\theta_\text{init}$, problem Set $\mathcal{D}$, validation Set $\mathcal{D}_{\text{val}}$, and max iterations $T$
\ENSURE parameter $\theta_T$
\STATE $\theta_0 \gets \theta_\text{init}$
\FOR{$t = 1$ to $T$}
    \STATE $D_t^{SFT}$ $\gets$ SFTDataCollect($\theta_{t-1}$) \COMMENT{\small{Following~\citet{DBLP:journals/corr/abs-2410-08115}}}
    \STATE $\theta_t$ $\gets$ SFT($D_t^{SFT}$, $\theta_\text{init}$) \COMMENT{\small{Following~\citet{DBLP:journals/corr/abs-2410-08115}}}
    \STATE $\mathcal{D}_t^\text{DPO} \gets \emptyset$
    \FORALL{$p_i \in \mathcal{D}$}
        \STATE $\mathcal{D}_i^\text{DPO} \gets \text{MCTSSynthesis}(\theta_t, p_i)$ 
        \STATE $\mathcal{I}_{\mathcal{F}_{\text{val}}}\gets \text{DataInfluenceCollect}(\mathcal{D}_{\text{val}})$ 
        \STATE $\mathcal{D}_t^\text{DPO} \gets \mathcal{D}_t^\text{DPO} \cup \mathcal{D}_i^\text{DPO}$
    \ENDFOR
    \STATE $\mathcal{D}_t^{DPO} \gets \text{InfluSelection}(\mathcal{D}_t^{DPO}, \mathcal{I}_{\mathcal{F}_{\text{val}}}$)
    \STATE $\theta_{t}$ $\gets$ DPO($\mathcal{D}_t^{DPO}, \theta_t)$
\ENDFOR
\OUTPUT $\theta_T$
\end{algorithmic}
\end{algorithm}

}

\subsection{Iterative Data Synthesis}
\label{subsection:iterative data synthesis}
In addition to utilizing the current model for data synthesis, we propose an iterative refinement approach to generate higher-quality data. By continuously training and enhancing the model, its capabilities improve, enabling the generation of more valuable synthetic data in subsequent iterations. At iteration $t$, we generate the training dataset $\mathcal{D}_{\text{tr}}^t$ based on the parameters $\theta_{t-1}$ and train a new model from the initial model using $\mathcal{D}_{\text{tr}}^t$. The corresponding pseudocode can be found in Algorithm~\ref{alg:DITS-isft-dpo}.



% \cx{Sec 3 and 4 need major revision. Then Introduction.}









%%%


\section{Experiments}
\textbf{Setup.} We evaluate the performance of PINNMamba on three standard PDE benchmarks: convection, wave, and reaction equations, all of which are identified as being affected by failure modes~\cite{krishnapriyan2021characterizing,zhao2024pinnsformer}. The details of those PDEs can be found in Appendix~\ref{apx:setup}.
    We compare PINNMamba with four baseline models, vanilla PINN~\cite{raissi2019physics}, QRes~\cite{bu2021quadratic}, PINNsFormer~\cite{zhao2024pinnsformer}, and KAN~\cite{liu2024kan} .
For fair comparison, we sample 101$\times$101 collection points with uniformly grid sampling, following previous work~\cite{zhao2024pinnsformer,wu2024ropinn}. We also evaluate on PINNacle Benchmark~\cite{hao2023pinnacle} and Navier–Stokes equation~\cite{raissi2019physics}.

\begin{table*}
\vspace{-3mm}
  \caption{Results for solving convection, reaction, and wave equations.}
  \label{sample-table}
  
  \centering
    \small
  \begin{tabular}{l|c|ccc|ccc|ccc}

    \toprule 
  & & \multicolumn{3}{c}{Convection }&\multicolumn{3}{c}{Reaction}&\multicolumn{3}{c}{Wave}\\
    \cmidrule(lr){3-5}\cmidrule(lr){6-8}\cmidrule(lr){9-11}
   Model & \#Params &Loss & rMAE & rRMSE & Loss & rMAE & rRMSE& Loss & rMAE & rRMSE
 \\   \midrule
    PINN&527361& 0.0239 & 0.8514 & 0.8989& 0.1991 & 0.9803 & 0.9785& 0.0320 & 0.4101 & 0.4141\\
    QRes & 396545& 0.0798 & 0.9035 & 0.9245& 0.1991 & 0.9826 & 0.9830& 0.0987 & 0.5349 & 0.5265\\
    PINNsFormer &453561 & 0.0068 & 0.4527 & 0.5217& 3e-6& 0.0146 & 0.0296 & 0.0216 & 0.3559 & 0.3632\\
     KAN&891& 0.0250 & 0.6049 & 0.6587& 7e-6 & 0.0166 & 0.0343& 0.0067 & 0.1433 & 0.1458\\
   \rowcolor{mygray}   PINNMamba  & 285763&0.0001 & \textbf{0.0188} & \textbf{0.0201}&1e-6&\textbf{0.0094}&\textbf{0.0217}& 0.0002 & \textbf{0.0197} & \textbf{0.0199} \\

    \bottomrule
  \end{tabular}
  \normalsize
  \label{tab:diff}
  \vspace{-4mm}
\end{table*}

\begin{figure*}[t!]
    \centering
    \includegraphics[width=\textwidth]{_fig/wave}
    \vspace{-8mm}
    \caption{The ground truth solution, prediction (top), and absolute error (bottom) on wave equations.}
    \label{fig:wave}
    \vspace{-5mm}
  %  \vspace{-1mm}
\end{figure*}

\textbf{Training Details.} We train PINNMamba and all the baseline models 1000 epochs with L-BFGS optimizer~\cite{liu1989limited}.
We set the sub-sequence length to 7 for PINNMamba, and keep the original pseudo-sequence setup for PINNsFormers. The weights of loss terms $[\lambda_\mathcal F,\lambda_\mathcal I,\lambda_\mathcal B]$ are set to $[1,1,10]$ for all three equations, as we find that strengthening the boundary conditions can lead to better convergence. $\lambda_\text{alig}$ is set to 1000 for convection and reaction equations, and auto-adapted by $\lambda_\mathcal F$ for wave equation.
%Loss weights are also actively adapted by neural tangent kernel~\cite{wang2022and} for wave equations for test the orthogonality of PINNMamba with other methods.
All experiments are implemented in PyTorch 2.1.1 and trained on an NVIDIA H100 GPU.  More training details are in Appendix~\ref{apx:hyperparam}. Our code and weights are available at \url{https://github.com/miniHuiHui/PINNMamba}.

\textbf{Metrics.} To evaluate the performance of the models, we take relative Mean Absolute Error (rMAE, a.k.a  $\ell_1$ relative error) and relative Root Mean Square Error (rRMSE, a.k.a $\ell_2$ relative error) following common practive~\cite{zhao2024pinnsformer,wu2024ropinn}. The metrics are formulated as:
\begin{align}
\text { rMAE }(\hat u)&=\frac{\sum_{n=1}^N\left|\hat{u}\left(x_n, t_n\right)-u\left(x_n, t_n\right)\right|}{\sum_{n=1}^{N}\left|u\left(x_n, t_n\right)\right|}, \\
\text { rRMSE }(\hat u)&=\sqrt{\frac{\sum_{n=1}^N\left|\hat{u}\left(x_n, t_n\right)-u\left(x_n, t_n\right)\right|^2}{\sum_{n=1}^N\left|u\left(x_n, t_n\right)\right|^2}},
\end{align}
where N is the number of test points, $u(x,t)$ is the ground truth solution, and $\hat u(x,t)$ is the model's prediction.

\vspace{-2mm}

\subsection{Main Results}
\vspace{-1mm}
We present the rMAE and rRMSE for approximating convection, reaction and wave equation's solution in Table~\ref{tab:diff}. Our model consistently outperforms other model architectures, achieving new state-of-the-art.
Notably, as shown in Fig.~\ref{fig:conv}, for the convection equation, PINNMamba allows sufficient propagation of information about the initial conditions, whereas on all the other models there is a varying degree of distortion in the time coordinates.
    As shown in Fig.~\ref{fig:reac}, PINNMamba can further optimize at the boundary, resulting in a lower error than KAN and PINNsFormer for reaction equations. For problems as intrinsically difficult to optimize as the wave, as in Fig.~\ref{fig:wave}, PINNMamba effectively combats simplicity bias and aligns the scales of multi-order differentiation, and thus achieves significantly higher accuracy. This illustrates that PINNMamba can be effective against PINN's failure modes. It's also worth noting that, PINNMamba has the lowest number of parameters (except KAN), while achieving consistently the best performance.

\begin{table}
\vspace{-3mm}
  \caption{Integrating PINNMamba with advanced training strategies and loss auto-balancing strategy. The rMAE is reported here.}
  
  \centering
    \small
  \begin{tabular}{lccc}

    \toprule 
    Method & Convection & Reaction & Wave\\
   \midrule
   PINNMamba & 0.0188 & 0.0094 & 0.0197\\
   +gPINN & 0.0172& 0.0123 & 0.0264 \\
   +vPINN & 0.0236 & 0.0092& 0.0169\\
   +RoPINN & 0.0102& 0.0099& 0.0121\\
    \midrule
    +NTK &0.0179& 0.0079& 0.0147\\
    +NTK+RoPINN &0.0127& 0.0072& 0.0106\\
   

    \bottomrule
  \end{tabular}
  \normalsize
  \label{tab:para}
  \vspace{-6mm}
\end{table}

\begin{figure*}[t!]
    \centering
    \includegraphics[width=\textwidth]{_fig/reac}
    \vspace{-8mm}
    \caption{The ground truth solution, prediction (top), and absolute error (bottom) on reaction equations.}
    \label{fig:reac}
    \vspace{-5mm}
  %  \vspace{-1mm}
\end{figure*}


\subsection{Combination with Other Methods}
\vspace{-1mm}
Since PINNMamba mainly focuses on model architecture, it can be integrated with other methods effortlessly. 
    We explore the feasibility and their performance in combination with advanced training paradigm, as well as loss balancing.

\textbf{Training Paradigm.} We show the rMAE of PINNMamba when integrated with advanced strategies in Table~\ref{tab:para}. We observe that gPINN~\cite{yu2022gradient} and vPINN~\cite{kharazmi2019variational} erratically deliver some performance gains on some tasks. 
    This is due to the fact that the regularization provided by gPINN and vPINN in the form of a loss function through the gradient and variational residuals has little effect on PINNMamba, since SSM itself is sufficiently regularized. RoPINN~\cite{wu2024ropinn} reduces the PINNMamba's error on convection and wave equations by about 40\%, since it complements the spatial continuity dependency.

\textbf{Neural Tangent Kernel.} Dynamic tuning of losses via Neural Tangent Kernel(NTK)~\cite{wang2022and} has been shown to have the effect of smoothing out the loss landscape. 
PINNMamba also works well with the NTK-adopted loss function. As shown in Table~\ref{tab:para}, NTK can reduce PINNMamba error by 5-25\%. 
The combination of RoPINN and NTK can further improve the overall performance of PINNMamba, which demonstrates the excellent suitability of PINNMamba with other PINN optimization methods.

\begin{figure}[t!]
    \centering
    \includegraphics[width=\linewidth]{_fig/loss_error}
    \vspace{-4mm}
    \caption{Loss and $\ell_1$-Error Curve w.r.t Training Iteration.}
    \label{fig:losserror}
    \vspace{-4mm}
  %  \vspace{-1mm}
\end{figure}
\vspace{-2mm}
\subsection{Loss-Error Consistency Analysis}
\vspace{-1mm}

Our other interest is the role of PINNMamba for the elimination of simplicity bias. Models affected by simplicity bias that fall into over-smoothing solutions will show inconsistent decreasing trends in loss and error during training. 
    As shown in Fig.~\ref{fig:losserror}, in the training process for solving convection equations, the rMAE of PINN doesn't descend as $\mathcal L_\mathcal F$ and $\mathcal L_\mathcal I$. 
        This suggests that PINN is trapped in an over-smoothing solution, which is in agreement with our observation in Fig.~\ref{fig:conv}. 
As a comparison, we find that PINNMamba's losses descent processes show a high degree of consistency with its error descent process. 
    This indicates that PINNMamba does not tend to fall into a local optimum of oversimplified patterns.
        Instead, it tends to exhibit patterns that are consistent with the original PDEs.

\vspace{-2mm}
\subsection{Ablation Study}
\vspace{-1mm}
\begin{table*}
  [t]
  \centering
  \resizebox{\textwidth}{!}{%
  \begin{tabular}{cccccccccccc}
    \toprule \multicolumn{2}{c}{Components}                                                             & \multicolumn{5}{c}{Re-executability Rate (\%)} & \multicolumn{5}{c}{Readability (\#)} \\
    \cmidrule(lr){1-2} \cmidrule(lr){3-7} \cmidrule(lr){8-12}        \hspace{8pt}\labelemoji\hspace{8pt}                                                                & \hspace{8pt}\toolemoji\hspace{8pt}                                      & O0                                 & O1             & O2             & O3             & AVG            & O0             & O1             & O2             & O3             & AVG            \\
    \hline
    \rowcolor[rgb]{0.93,0.93,0.93}\multicolumn{12}{c}{\textbf{Initialize with LLM4Decompile-End-6.7B~\citep{llm4decompile}}}   \\
    \xmark                                                                                              & \xmark                                    & 69.51                              & 46.95          & 50.61          & 46.34          & 53.35          & 3.98 & 3.41 & 3.44 & 3.38 & 3.55 \\
    \cmark                                                                                              & \xmark                                    & 75.61                              & 50.61          & 50.00          & 50.00          & 56.55          & 4.01 & 3.44 & 3.39 & \textbf{3.49} & 3.58 \\
    \xmark                                                                                              & \cmark                                    & 83.54                     & \textbf{56.10}          & 51.22          & 50.61 & 60.37 & 4.05 & 3.51 & 3.51 & 3.42 & 3.62 \\
    \cmark                                                                                              & \cmark                                    & \textbf{85.37}                            & \textbf{56.10}                     & \textbf{51.83} & \textbf{52.43}          & \textbf{61.43} & \textbf{4.13} & \textbf{3.60} & \textbf{3.54} & \textbf{3.49} & \textbf{3.69} \\

    \rowcolor[rgb]{0.93,0.93,0.93}\multicolumn{12}{c}{\textbf{Initialize with Deepseek-Coder-6.7B-base~\citep{deepseekcoder}}} \\
    \xmark                                                                                              & \xmark                                    & 59.15                              & 35.98          & 39.02          & 37.80          & 42.99          & 3.71 & 3.05 & 3.16 & 3.05 & 3.24 \\
    \cmark                                                                                              & \xmark                                    & 66.46                              & 41.46          & 38.41          & 36.59          & 45.73          & 3.76 & 3.17 & \textbf{3.21} & 3.08 & 3.31 \\
    \xmark                                                                                              & \cmark                                    & 70.73                              & 39.63          & 39.02          & 40.24          & 47.41          & 3.90 & 3.17 & 3.08 & 3.11 & 3.31 \\
    \cmark                                                                                              & \cmark                                    & \textbf{79.88}                     & \textbf{45.73} & \textbf{43.90} & \textbf{42.68} & \textbf{53.05} & \textbf{3.96} & \textbf{3.21} & 3.18 & \textbf{3.19} & \textbf{3.38} \\
    \bottomrule
  \end{tabular}%
  }
  \caption{The ablation study of different methods across four optimization levels
  (O0, O1, O2, O3), as well as their average scores (AVG). The results in bold represent the optimal performance. The ~\labelemoji~ and ~\toolemoji~ means Relabedling and Function Call. \textbf{Bold} denotes the best performance.}
  \label{tab:ablation}
\end{table*}

To verify the validity of the various components of the PINNMamba, as shown in Table~\ref{tab:ablation}, we evaluate the performance of models subtracting these components from PINNMamba.

\textbf{Sub-Sequence.} We remove the sub-sequence alignment, which leads to a decrease in model performance, indicating the significance of the agreement formed through alignment in eliminating simplicity bias.
After replacing the sub-sequence with a long sequence of the entire domain, the model shows failure modes, in line with the sequence granularity analysis in Section~\ref{sec:subseq}.

\textbf{Time-Varying SSM.} We replace the selective SSM~\cite{gu2023mamba} with a linear time-invariant structure SSM~\cite{gu2022efficiently}, and there is some decrease in model performance, illustrating the role of predictive diversity in eliminating simplicity bias. 
And when we remove SSM completely and switch to MLP instead, the model has severe failure modes. 
        This demonstrates that SSM's adaptation for \textit{Continuous-Discrete Mismatch} allows the initial condition information to propagate sufficiently in time coordinates.

In addition, we also conducted a sensitivity analysis of the choice of sub-sequence length, activation. See Appendix~\ref{apx:sense}.

\vspace{-3mm}
\subsection{Experiments on Complex Problems}
\vspace{-1mm}
To further demonstrate the generalization of our method, we tested our model on partial PINNacle Benchmark~\cite{hao2023pinnacle} and Navier-Stokes equations. As shown in Fig.~\ref{fig:ns}, PINNMamba achieves the lowest error on the N-S equation. Just like PINNsFormer, PINNMamba also gets out-of-memory on some problems in PINNacle, which we identify as a major limitation of sequence-based methods. We discuss the details of PINNacle experiments in Appendix~\ref{apx:comp}.

\begin{figure}[t!]
    \centering
    \includegraphics[width=\linewidth]{_fig/NS}
    \vspace{-6mm}
    \caption{Absolute Error of pressure prediction of N-S equation}
    \label{fig:ns}
    \vspace{-3mm}
  %  \vspace{-1mm}
\end{figure}


% The \LaTeX{} and Bib\TeX{} style files provided roughly follow the American Psychological Association format.
% If your own bib file is named \texttt{custom.bib}, then placing the following before any appendices in your \LaTeX{} file will generate the references section for you:
% \begin{quote}
% \begin{verbatim}
% \bibliography{custom}
% \end{verbatim}
% \end{quote}

% You can obtain the complete ACL Anthology as a Bib\TeX{} file from \url{https://aclweb.org/anthology/anthology.bib.gz}.
% To include both the Anthology and your own .bib file, use the following instead of the above.
% \begin{quote}
% \begin{verbatim}
% \bibliography{anthology,custom}
% \end{verbatim}
% \end{quote}

% Please see Section~\ref{sec:bibtex} for information on preparing Bib\TeX{} files.

% \subsection{Equations}

% An example equation is shown below:
% \begin{equation}
%   \label{eq:example}
%   A = \pi r^2
% \end{equation}

% Labels for equation numbers, sections, subsections, figures and tables
% are all defined with the \verb|\label{label}| command and cross references
% to them are made with the \verb|\ref{label}| command.

% This an example cross-reference to Equation~\ref{eq:example}.

% Use \verb|\appendix| before any appendix section to switch the section numbering over to letters. See Appendix~\ref{sec:appendix} for an example.

% \section{Bib\TeX{} Files}
% \label{sec:bibtex}

% Unicode cannot be used in Bib\TeX{} entries, and some ways of typing special characters can disrupt Bib\TeX's alphabetization. The recommended way of typing special characters is shown in Table~\ref{tab:accents}.

% Please ensure that Bib\TeX{} records contain DOIs or URLs when possible, and for all the ACL materials that you reference.
% Use the \verb|doi| field for DOIs and the \verb|url| field for URLs.
% If a Bib\TeX{} entry has a URL or DOI field, the paper title in the references section will appear as a hyperlink to the paper, using the hyperref \LaTeX{} package.


\section{Conclusion}
In this paper, we introduced \textsc{FLAG-Trader}, a novel framework that integrates LLMs with RL for financial trading. In particular, \textsc{FLAG-Trader} leverages LLMs as policy networks, allowing for natural language-driven decision-making while benefiting from reward-driven optimization through RL fine-tuning. Our framework enables small-scale LLMs to surpass larger proprietary models by efficiently adapting to market conditions via a structured reinforcement learning approach. Through extensive experiments across multiple stock trading scenarios, we demonstrated that \textsc{FLAG-Trader} consistently outperforms baseline methods, including LLM-agentic frameworks and conventional RL-based trading agents. These results highlight the potential of integrating LLMs with RL to achieve adaptability in financial decision-making.

\newpage
\section*{Limitations and Potential Risk}

Despite its promising results, \textsc{FLAG-Trader} has several limitations. First, while our approach significantly enhances the decision-making ability of LLMs, it remains computationally expensive, particularly when fine-tuning on large-scale market datasets. Reducing computational overhead while maintaining performance is an important direction for future research. Second, financial markets exhibit high volatility and non-stationarity, posing challenges for long-term generalization. Future work should explore techniques such as continual learning or meta-learning to enhance model adaptability in evolving market conditions. Third, while \textsc{FLAG-Trader} effectively integrates textual and numerical data, its reliance on structured prompts could introduce biases in decision-making. Improving prompt design or exploring retrieval-augmented methods may further enhance robustness. Lastly, real-world trading requires stringent risk management, and \textsc{FLAG-Trader} currently optimizes for financial returns without explicitly incorporating risk-sensitive constraints. Extending the framework to integrate risk-aware objectives and dynamic portfolio optimization could provide more robust and practical financial trading solutions.

% \section*{Acknowledgments}

% This document has been adapted
% by Steven Bethard, Ryan Cotterell and Rui Yan
% from the instructions for earlier ACL and NAACL proceedings, including those for
% ACL 2019 by Douwe Kiela and Ivan Vuli\'{c},
% % NAACL 2019 by Stephanie Lukin and Alla Roskovskaya,
% % ACL 2018 by Shay Cohen, Kevin Gimpel, and Wei Lu,
% % NAACL 2018 by Margaret Mitchell and Stephanie Lukin,
% % Bib\TeX{} suggestions for (NA)ACL 2017/2018 from Jason Eisner,
% % ACL 2017 by Dan Gildea and Min-Yen Kan,
% % NAACL 2017 by Margaret Mitchell,
% % ACL 2012 by Maggie Li and Michael White,
% % ACL 2010 by Jing-Shin Chang and Philipp Koehn,
% % ACL 2008 by Johanna D. Moore, Simone Teufel, James Allan, and Sadaoki Furui,
% % ACL 2005 by Hwee Tou Ng and Kemal Oflazer,
% % ACL 2002 by Eugene Charniak and Dekang Lin,
% % and earlier ACL and EACL formats written by several people, including
% % John Chen, Henry S. Thompson and Donald Walker.
% % Additional elements were taken from the formatting instructions of the \emph{International Joint Conference on Artificial Intelligence} and the \emph{Conference on Computer Vision and Pattern Recognition}.

% Bibliography entries for the entire Anthology, followed by custom entries
%\bibliography{anthology,custom}
% Custom bibliography entries only
\bibliography{main}

\newpage
\appendix
\onecolumn
% resource extraction + smelting setup
% complex belts
% rail network (map and train loading)
% biters and defense
% power options 
% fluid/chemical processing
% circuit networks
% main bus design, mall design, modular base design
% assembler and recipes
% inventory GUI
% blueprint GUI
% construction robots


% https://factoriocheatsheet.com
% https://calculatorio.com/resources/
% https://forums.factorio.com/viewtopic.php?t=97410
\newpage
\appendix
\section{Visual Introduction to Factorio}
\label{ref:sec_appendix}

This appendix provides a high-level, illustrated overview of key \textit{Factorio} systems, ensuring that newcomers can grasp the fundamental mechanics of extracting resources, setting up production lines, defending against threats, and automating workflows. Each subsection introduces core concepts, from the simplest mining operations to advanced infrastructures like rail networks, circuit logic, and robot-assisted construction.

\subsection{Resource Extraction and Smelting}
% iron_mining_and_smelting
% 
The foundation of any Factorio factory is consistent raw material throughput. Players begin by placing mining drills on ore patches—such as iron or copper—where the drills extract resources at a steady rate. Ores are usually transported via conveyor belts to nearby smelters, which convert them into plates. A typical early-game setup involves an arrangement of furnaces linked by belts on both the input (ore) and output (finished plates) sides. This workflow underpins the factory’s growth: higher demand for plates necessitates expanding both mining operations and smelting capacity.

\begin{figure}[ht]
    \centering
    \includegraphics[width=\columnwidth]{zimages/resources/iron_mining_and_smelting} % Replace with your image file name
    \caption{An example of early-game resource extraction and smelting in Factorio. Box A shows mining drills extracting iron ore, Box B highlights stone furnaces which take ore and fuel and create plates, and Box C highlights belt routing and inserter mechanics. \cite{ironMiningSmelting}}
    \label{fig:iron_smelting}
\end{figure}
We first refer to Figure~\ref{fig:iron_smelting}. The automation process begins with the mining drills in \textbf{Box A}, which extract raw iron ore (blue material) from resource nodes and place it onto conveyor belts for transport downstream. These drills eliminate the need for manual mining, significantly increasing throughput and setting the foundation for automated workflows. In \textbf{Box B}, the raw iron ore is delivered to stone furnaces, where inserters (mechanical arms) feed the ore into the furnaces and remove the resulting iron plates—a critical intermediate resource—onto separate belts. This dual-belt system, fed by both raw iron ore and coal, ensures a continuous and automated smelting process. Finally, in \textbf{Box C}, the belts are routed efficiently using underground segments to avoid intersection conflicts, enabling seamless transport of resources. Yellow inserters deliver iron ore into the furnaces, while red inserters extract the smelted iron plates, which are then routed onward for further processing. This layered system of extraction, smelting, and material routing illustrates the early-game challenges of compact, efficient factory design in Factorio.


\subsection{Automation with Assemblers and Managing Complexity}
Automation through assemblers is a cornerstone of \textit{Factorio} gameplay, enabling exponential growth in productivity by trading energy and space for vastly higher throughput. The production of \textit{science packs} is central to the objective of unlocking advanced technologies. The earliest science packs are automation (red) and logistic (green) science. The dependencies for crafting these are shown in Figure~\ref{fig:red_green_dependencies}. Taking the example of logistic science, Figure~\ref{fig:green_science_recipe} illustrates that it takes 6s to assemble if the intermediate goods of transport belts and inserters are available. If only the raw materials of iron and copper plates are present then it will take 8.7s since the intermediate goods themselves need to be created. Hence, by automating intermediate goods, the factory can parallelize workflows, ensuring higher efficiency and faster output. 

\begin{figure}[ht]
    \centering
    \includegraphics[width=0.6\columnwidth]{zimages/recipes/red_green_dependencies}
    \caption{Dependency graph for red and green science packs. Inputs include both raw materials and intermediates, reflecting the growing complexity of production chains.}
    \label{fig:red_green_dependencies}
\end{figure}


There are thus many points to consider when designing assembly lines for these finished goods. The throughput of inputs and outputs should be well-matched given the ratios of materials needed in recipes. The demand for a common base resource has to be managed well across different use cases. The factory has to be actively refactored as increased scale means greater space and energy requirements. An example of a compact design which produces both red and green science is shown in Figure~\ref{fig:red_green_assembly}. While it looks efficient, issues may arise when the scale of production needs to increase, since the routing of intermediate goods would be significantly complicated. Efficient layouts must balance immediate needs with scalability, ensuring that adding new production lines or expanding capacity can be achieved without overhauling the entire factory.

\begin{figure}[ht]
    \centering
    \includegraphics[width=0.5\columnwidth]{zimages/recipes/green_science_recipe2}
    \caption{Recipe for logistic (green) science packs. Automating intermediate goods significantly reduces total crafting time from 8.7 seconds (raw) to 6 seconds.}
    \label{fig:green_science_recipe}
\end{figure}



\subsection{Science Packs and the Tech Tree}
% tech tree screenshots, also refer to assembler/recipe screenshot
The science packs produced in Figure~\ref{fig:red_green_assembly} are central to progression in \textit{Factorio}, serving as the primary currency for unlocking new technologies. Science packs are consumed by specialized \textit{laboratory} units, which convert them into research progress. Each research project requires a specific combination and quantity of science packs, introducing dependencies on a wide array of intermediate goods. This makes science packs a natural bottleneck for factory growth, as they represent the culmination of multiple production chains working in harmony.

This reliance on science packs is why science per minute (SPM) emerges as a critical metric for measuring factory productivity. A high SPM indicates that the factory has sufficient capacity not only to produce the required intermediates efficiently but also to scale them as the tech tree demands become more complex. For example, early-game science packs (red and green) require relatively simple intermediates such as gears, transport belts, and inserters, as shown in Figure~\ref{fig:green_science_tech}. However, as the factory evolves, higher-tier science packs (such as blue or utility science) introduce more advanced recipes involving fluids, electronics, and complex multi-stage production.

The tech tree in Figure~\ref{fig:green_science_tech} highlights this progression, showcasing how early-game technologies provide foundational tools like transport belts and inserters, which are then leveraged to unlock more advanced machinery such as trains and assemblers. This cascading dependency structure requires careful planning to ensure that production systems remain adaptable to increasing demands. The iterative process of unlocking technologies feeds back into the factory itself, enabling further automation and resource optimization. The science system is thus a core gameplay mechanic that ties together automation, logistics, and long-term planning, creating a continuous cycle of technological advancement and production refinement.

\begin{figure}[ht]
    \centering
    \includegraphics[width=0.7\columnwidth]{zimages/tech_tree/green_science_tech}
    \caption{Section of the tech tree which shows which technologies have logistic (green) science as a dependency. These include better transport belts, engines, trains, electric cables, circuit networks and more. Each recipe demands dozens if not hundreds of logistic science packs and so SPM becomes the bottleneck for further growth.}
    \label{fig:green_science_tech}
\end{figure}



\subsection{Power Generation Options}
% nuclear + solar panels
% https://www.reddit.com/r/factorio/comments/vmumny/my_473_mw_nuclear_power_plant_complete_with/
% power monitor
% https://forums.factorio.com/viewtopic.php?t=89602
% nuclear only: https://wiki.factorio.com/File:Nuclear_setup.png
Factorio offers diverse power solutions that evolve with the factory’s scale, closely mirroring the progression of energy systems in real-world industrial engineering. Early operations rely on steam engines fueled by coal, providing a reliable but resource-intensive solution. Coal mining introduces logistical challenges, requiring consistent supply chains and raising concerns about pollution, which in the game aggravates hostile aliens and causes them to attack the agent's base (see the next subsection). As research progresses, solar panels and accumulators become viable for renewable energy. While solar panels offer clean, sustainable power, they come with limitations tied to diurnal cycles, requiring accumulators to store excess energy for nighttime use. This trade-off between sustainability and infrastructure demands mirrors the challenges of integrating renewables into modern power grids, where storage and energy distribution systems are key bottlenecks.

Nuclear power, a late-game solution, exemplifies high-density energy production but comes with its own complexities. Players must process uranium, manage heat generation, and design safe reactor layouts to avoid catastrophic failures, echoing real-world concerns around nuclear fuel cycles, reactor safety, and waste management. The spatial footprint of energy systems also becomes a critical factor: steam and nuclear setups require compact layouts with high resource input, while sprawling solar farms demand significant land allocation. A reference for nuclear power is shown in Figure~\ref{fig:nuclear_power}

\begin{figure}[ht]
    \centering
    \includegraphics[width=\columnwidth]{zimages/power/nuclear_power.pdf}
    \caption{Nuclear power generation is actually quite realistic in \textit{Factorio}. Uranium ore is mined, the vast majority of which (99.3\%) is inert U-238. The more valuable U-235 is needed for energy-intensive applications. There is an enrichment process by which U-238 can be refined to make more U-235 provided some initial quantity of U-235. Then this is utilized in nuclear reactors which can produce steam to power turbines. \cite{nuclear}}
    \label{fig:nuclear_power}
\end{figure}

Each energy choice in Factorio presents distinct trade-offs—coal introduces pollution but offers consistency, solar minimizes pollution but requires storage solutions and space, and nuclear delivers immense power but requires advanced materials and precise management. These dynamics force players to weigh efficiency, scalability, and sustainability, capturing the essence of systems engineering in real-world energy infrastructure. By gradually introducing more advanced technologies and requiring players to adapt their power networks, Factorio illustrates the iterative process of scaling energy systems to meet growing demands while addressing environmental and logistical constraints.

\subsection{Biters and Defense}
% biters getting shot
% https://wiki.factorio.com/index.php?curid=46602
% biter base
% https://www.factorio.com/blog/post/fff-358
As factories grow and produce pollution, the indigenous alien lifeforms—commonly called Biters—become increasingly hostile, posing a persistent threat to factory operations. Pollution emitted by the factory spreads across the map, and once it reaches a Biter colony, such as the one depicted in Figure~\ref{fig:biter_colony}, it triggers aggressive behavior. Biters begin spawning in waves to attack the factory, targeting structures and resources critical to production. This introduces a dynamic tension between industrial expansion and the need to secure valuable infrastructure, reflecting real-world trade-offs in industrial development where growth often necessitates heightened security measures.

\begin{figure}[ht]
    \centering
    \includegraphics[width=\columnwidth]{zimages/defense/biters.pdf}
    \caption{Biters are alien residents of the planet where the agent has crash landed. They are docile initially but become aggravated by air pollution from the factory's hydrocarbon-powered operations. Thinking about biters is thus a core trade-off of expanding systems in Factorio.}
    \label{fig:biter_colony}
\end{figure}

\begin{figure}[ht]
    \centering
    \includegraphics[width=\columnwidth]{zimages/defense/biter_defense.pdf}
    \caption{Defense against Biters is essentially a resource sink in \textit{Factorio}. Settings and mods can be used to dramatically change the difficulty associated with defending bases from Biters.}
    \label{fig:biter_defense}
\end{figure}

\begin{figure*}[htb]
    \centering
    \includegraphics[width=0.8\textwidth]{zimages/recipes/red_green_assembly}
    \caption{A red and green science production setup. All belts are running from left to right. Iron and copper plates enter on the bottom-most belt (Box A). There are assemblers throughout the line which have certain recipes selected. For example Box B has the assembler responsible for assembling green circuits. The inserters to the right of Box B automatically pull copper wire from that assembler and the yellow inserter above Box B pulls iron plates from the belt. The red inserter above Box B places finished green circuits onto the belt one tile above the belt with iron and copper plates. Similar assembly happens for gears, belts, and inserters. Ultimately, red and green science packs are produced from the intermediate goods and are ready for further use (Box C).}
    \label{fig:red_green_assembly}
\end{figure*}

Early defenses rely on a combination of walls and gun turrets, as seen in Figure~\ref{fig:biter_defense}, where turrets gun down an approaching wave of Biters at a fortified perimeter. Gun turrets provide reliable protection during the early stages but depend on a steady supply of ammunition, which itself requires dedicated production lines. As the factory evolves, more advanced defensive structures like flamethrower turrets, laser turrets, and artillery become available. Flamethrowers are particularly effective for handling large swarms due to their area-of-effect damage, while laser turrets require no ammunition but demand significant power, introducing another layer of logistical complexity. Artillery, a late-game option, allows players to strike Biter nests at long range, proactively reducing the threat level.



Strategically fortifying perimeters and clearing nearby Biter nests becomes essential as pollution spreads farther and factory operations grow in scale. Defensive layouts must balance resource efficiency with resilience, ensuring that critical areas are well-protected without overextending the factory’s capacity to supply power, ammunition, or repairs. Additionally, players must consider choke points, turret placement, and overlapping fields of fire to maximize defensive effectiveness.

\subsection{Complex Belts and Main Factory Layouts}
% main bus 
% 
% modular design
% 
Conveyor belts are the arteries of a \textit{Factorio} base, transporting materials between production stages with speed and efficiency. While straightforward in the early game, managing belts becomes increasingly complex as factories grow. Scaling introduces challenges such as belt congestion, balancing input and output ratios, and ensuring that each production branch receives the right materials without overloading the system. Designing efficient layouts to manage this complexity is critical for avoiding “spaghetti”—a term used by the community to describe tangled, chaotic belt arrangements that hinder scalability and troubleshooting.

One popular solution to these challenges is the \textbf{main bus} design, as shown in Figure~\ref{fig:main_bus}. A main bus consists of a centralized set of parallel belts carrying essential resources like iron plates, copper plates, gears, and circuits. Branches extend from the main bus to feed production lines, ensuring that critical resources are readily available across the factory. This design prioritizes simplicity and organization, making it easier to scale production by adding new branches or extending the bus itself. However, maintaining a main bus requires careful planning to prevent bottlenecks and to allocate space for future resource additions. Players must also ensure that belts remain balanced to avoid starving downstream branches of materials.

\begin{figure}[ht]
    \centering
    \includegraphics[width=\columnwidth]{zimages/layouts/main_bus.pdf}
    \caption{The main bus design is a common choice for mid-game scaling. Branches for individual component assembly fork off the main bus using belt splitters and underground belts. \cite{mainBus}}
    \label{fig:main_bus}
\end{figure}

An alternative to the main bus approach is the \textbf{city block} design, illustrated in Figure~\ref{fig:city_block}. In this modular approach, the factory is divided into distinct “blocks,” each dedicated to a specific function, such as smelting, circuit production, or science pack assembly. These blocks are connected by train networks, allowing resources to be transported efficiently between distant sections of the factory. The city block layout offers excellent scalability, as additional blocks can be added without disrupting existing workflows. It also improves manageability, as each block operates semi-independently, reducing the risk of widespread factory failures due to localized issues.

Both layouts demonstrate distinct trade-offs. The main bus design excels in compactness and simplicity, making it ideal for medium-sized factories, but it can become unwieldy as the number of resources grows in the late game. City block layouts, while more complex to set up initially, provide unmatched flexibility and extensibility, especially when managing large-scale operations with diverse production needs. Figures~\ref{fig:main_bus} and~\ref{fig:city_block} highlight the strengths of these designs, showcasing how thoughtful belt organization and transportation planning are essential for managing complexity and ensuring smooth factory operation as production demands increase.

\begin{figure}[ht]
    \centering
    \includegraphics[width=\columnwidth]{zimages/layouts/city_blocks2.pdf}
    \caption{The city block design is ideal for late-game mega-base building. Modular base sections are linked using rail networks for loading and unloading of items. \cite{cityBlock}}
    \label{fig:city_block}
\end{figure}




\subsection{Rail Networks}
% plenty of stuff in tutorial
% GUi
% https://steamcommunity.com/sharedfiles/filedetails/?id=2737259470
% intersections and shit
% https://www.reddit.com/r/factorio/comments/8bappn/chunk_aligned_rhd_rail_blueprints_mostly_for_141/
% loading/unloading
% https://www.reddit.com/r/factorio/comments/j0ftu0/consuming_a_full_blue_belt_with_3_stack_inserters/

At mid to late stages of \textit{Factorio}, trains become a critical component of resource logistics, allowing raw materials and finished goods to be transported across vast distances. Tracks are laid out on a tile-based map, with stations configured for specific tasks such as ore pickups and deliveries to smelting or assembly sites. Trains enable players to overcome the limitations of conveyor belts, which can become cumbersome and inefficient for long-range transport, providing a scalable solution that supports factory growth.

\begin{figure}[ht]
    \centering
    \includegraphics[width=\columnwidth]{zimages/rail/train_loading2.pdf}
    \caption{Trains can move large quantities of resources long distances much faster than belts while reusing the same underlying infrastructure, making them crucial for any scalable build. \cite{trainUnloading}}
    \label{fig:_train_loading}
\end{figure}

Designing a robust railway system requires careful planning and mastery of key mechanics. Figure~\ref{fig:_train_loading} shows a typical train loading setup, where numerous inserters work in parallel to load ore into cargo carriages quickly. Efficient loading and unloading are essential to minimize train idle times and ensure smooth throughput. Multiple stations can be linked along a track network, with each station named and assigned schedules dictating when trains should arrive and depart, further streamlining the flow of resources between locations.


\begin{figure}[ht]
    \centering
    \includegraphics[width=\columnwidth]{zimages/rail/_train_intersections.pdf}
    \caption{The variety and customization associated with building rail networks is vast in Factorio. \cite{trainPatterns}}
    \label{fig:_train_intersections}
\end{figure}

Track layouts, particularly intersections and junctions, are another crucial element. Figure~\ref{fig:_train_intersections} highlights several blueprint designs for rail intersections, showcasing patterns optimized for traffic flow and collision avoidance. Proper signaling is necessary to manage multiple trains on shared tracks, with block signals and chain signals controlling which sections of track are reserved for individual trains. Complex networks can handle dozens of trains simultaneously, but poor design or inadequate signaling can lead to congestion or catastrophic collisions, disrupting the factory’s supply chains.

\begin{figure}[ht]
    \centering
    \includegraphics[width=\columnwidth]{zimages/rail/train_gui2.pdf}
    \caption{Factorio gives players the ability to observe and orchestrate train networks with high customization \cite{trainGuide}}
    \label{fig:_train_monitor}
\end{figure}

The train management system extends beyond physical tracks, as shown in Figure~\ref{fig:_train_monitor}, which displays the GUI for monitoring train activity. This interface allows players to track the status of all trains in the network, observe their current locations, and adjust schedules or routes as needed. The train monitor is an invaluable tool for diagnosing delays, optimizing routes, and ensuring that all resource flows remain balanced.

A well-designed railway system is not just a means of transport but a backbone for factory expansion, allowing new outposts and production sites to be integrated seamlessly into the larger network. By balancing efficient loading, modular track designs, and robust train management, players can scale their factories to unprecedented levels while maintaining resource flow and minimizing logistical bottlenecks.


%\subsection{Fluid and Chemical Processing}
%Oil refineries and chemical plants expand production beyond solid intermediates. Crude oil can be refined into multiple products (e.g., petroleum gas, heavy oil, light oil), each feeding into specialized recipes like plastics, sulfur, and lubricants. Pipe networks introduce new routing challenges—especially when storage tanks and by-product management complicate typical belt-based workflows. Fuel selection, cracking processes, and circuit-regulated valves are just a few examples of intricate fluid logistics.

%\subsection{Circuit Networks}
%Circuit networks allow players to program conditional behaviors for factory operations. Wires connect machines and storage, enabling control signals that can, for instance, stop a belt if storage is full or dispatch a train when a station demands resources. While optional, circuit logic can optimize resource usage, reduce power drain, or even orchestrate precise production cycles. Mastering circuits is akin to learning a lightweight scripting language for industrial automation.

\begin{figure}[ht]
    \centering
    \includegraphics[width=\columnwidth]{zimages/robots/constructionbots.pdf}
    \caption{Construction robots automate the placement of arbitrarily complex player-made blueprints. Here the blueprint has been partially constructed by robots and needs to be completed and connected to a source of power. \cite{constructionBots}}
    \label{fig:constructionBots}
\end{figure}


\subsection{Inventory, Blueprints, and Construction Robots}
Players interact with a comprehensive inventory GUI that tracks personal items, crafting queues, and equipment. This interface underpins many of the high-level systems within Factorio, ensuring that even the most complex production chains remain manageable. When testing these systems—particularly in large-scale or late-game scenarios—developers and players alike must verify that inventory updates, crafting queues, and personal equipment management work seamlessly without bottlenecking progress. Such testing is crucial because any inefficiency or bug in inventory handling can cascade throughout a massive base, undermining the player’s ability to grow their automation network.

A prime example of Factorio’s advanced systems is the \textbf{blueprint} feature, which allows users to save layouts ranging from simple assembler setups to sprawling smelter arrays. As shown in Figure \ref{fig:constructionBots}, pasting a blueprint summons construction robots to automatically assemble buildings and belts, provided that the necessary items are available and that the structures remain within the logistic network’s coverage (the robot hub range is visible in the center of the screenshot). High-level system testing involves confirming that these blueprint placements work at scale: robots must reliably build, repair, and upgrade components in the correct order and handle resource shortages gracefully. If the blueprint system or robot AI malfunctions, it can cause partial constructions or idle bots, quickly eroding the advantages of automation and frustrating the player.

\begin{figure}[ht]
    \centering
    \includegraphics[width=\columnwidth]{zimages/robots/destructionbots.pdf}
    \caption{Robots can also be used to efficiently clear out a factory and reclaim the resources. Here the section of the factory has been marked for clearance and robots will swarm it when the player finalizes the selection. \cite{destructionBots}}
    \label{fig:destructionBots}
\end{figure}

Moreover, these same construction robots facilitate large-scale deconstruction, an equally vital aspect of advanced base management. Figure \ref{fig:destructionBots} illustrates the user highlighting a section of the factory for removal—once marked, robots swarm to dismantle it, returning valuable materials to the appropriate storage points. Rigorous testing here ensures that no mismatches occur in item retrieval, that robots can safely access all structures slated for removal, and that the logistic system manages reclaimed items without jams. Essentially, blueprints and bots automate both production and the \textit{creation of production}, making the entire game experience highly recursive and reliant on flawlessly functioning high-level systems. Verifying these features in complex, large-scale conditions is critical for preserving Factorio’s hallmark sense of continual, smoothly scaling automation.




\end{document}
