\section{Experiments}
\begin{table*}[thbp]
\renewcommand{\arraystretch}{1}
\vspace{-0.2cm}
\setlength{\abovecaptionskip}{0.1cm}
\centering
\caption{Performance of stock trading with different LLMs as backbone model across seven stocks.}
\label{tab:stock-trading-performance1}
\begin{threeparttable}
\scalebox{0.70}{
\begin{tabular}{@{}lcccc|cccc|cccc@{}}
\toprule
\textbf{Model} & \multicolumn{4}{c}{\textbf{MSFT}} & \multicolumn{4}{c}{\textbf{JNJ}} & \multicolumn{4}{c}{\textbf{UVV}}\\
\cmidrule(lr){2-5}\cmidrule(lr){6-9}\cmidrule(lr){10-13}
& \textbf{CR\(\uparrow\)} & \textbf{SR\(\uparrow\)} & \textbf{AV\(\downarrow\)} & \textbf{MDD\(\downarrow\)}
& \textbf{CR\(\uparrow\)} & \textbf{SR\(\uparrow\)} & \textbf{AV\(\downarrow\)} & \textbf{MDD\(\downarrow\)}
& \textbf{CR\(\uparrow\)} & \textbf{SR\(\uparrow\)} & \textbf{AV\(\downarrow\)} & \textbf{MDD\(\downarrow\)}\\
\midrule
\textbf{Buy \& Hold}
& 15.340 & 1.039 & 24.980 & 9.428 
& 13.895 & 1.343 & 17.500 & 9.847 
& 36.583 & 2.112 & 29.299 & 15.406\\
\midrule
\multicolumn{13}{c}{\textit{\textbf{Financial Domain Models}}}\\
\textbf{Palmyra-Fin-70B}  
& 14.697 & 0.897 & 27.518 & 9.428 
& 5.748 & 0.450 & 19.317 & 9.367 
& 37.875 & 2.039 & 31.200 & 15.967\\
\midrule
\multicolumn{13}{c}{\textit{\textbf{Proprietary Models}}}\\
\textbf{GPT-o1-preview} 
& 17.184 & 0.962 & 30.000 & 9.428 
& 13.561 & 1.086 & 20.864 & 9.847 
& 41.508 & 2.147 & 32.479 & 9.633\\
\textbf{GPT-4} 
& 16.654 & 0.932 & 30.022 & 9.428 
& 13.712 & 1.103 & 20.894 & 9.860 
& 31.791 & 1.640 & 32.567 & 10.434\\
\textbf{GPT-4o} 
& 12.461 & 0.924 & 22.653 & 6.647
& 9.099 & 0.875 & 17.471 & 7.169 
& 8.043 & 0.496 & 27.241 & 14.889\\
\midrule
\multicolumn{13}{c}{\textit{\textbf{Open-Source Models}}}\\
\textbf{Qwen2.5-72B-Instruct}  
& 7.421 & 0.588 & 21.238 & 6.973 
& 14.353 & 1.140 & 20.995 & 9.812 
& 37.178 & 1.822 & 34.223 & 13.365\\
\textbf{Llama-3.1-70B-Instruct}  
& 17.396 & 1.335 & 21.892 & 7.045 
& 13.868 & 1.121 & 20.779 & 9.825 
& 35.981 & 1.728 & 34.986 & 15.406\\
\textbf{DeepSeek-67B-Chat} 
& 13.941 & 0.834 & 28.081 & 7.850 
& 14.426 & 1.185 & 20.450 & 9.825 
& 29.940 & 1.481 & 33.964 & 15.407\\
% \midrule
\textbf{Yi-1.5-34B-Chat}  
& 22.093 & 1.253 & 29.613 & 9.428 
& 14.004 & 1.180 & 19.938 & 9.847 
& 20.889 & 1.020 & 34.417 & 14.936\\
\textbf{Qwen2.5-32B-Instruct}  
& -0.557 & -0.041 & 22.893 & 8.946 
& 2.905 & 0.292 & 16.725 & 7.169 
& -1.623 & -0.097 & 27.973 & 17.986\\
\textbf{DeepSeek-V2-Lite (15.7B)}  
& 11.904 & 0.694 & 28.796 & 16.094 
& -7.482 & -0.670 & 18.773 & 17.806 
& 33.560 & 1.703 & 33.099 & 12.984\\
% \midrule
\textbf{Yi-1.5-9B-Chat}  
& 19.333 & 1.094 & 29.690 & 9.428 
& 18.606 & 1.611 & 19.409 & 10.986 
& 49.415 & 2.410 & 34.446 & 11.430\\
\textbf{Llama-3.1-8B-Instruct}  
& 22.703 & 1.322 & 28.855 & 7.385 
& 13.988 & 1.486 & 20.460 & 9.969 
& 41.108 & 1.981 & 34.866 & 16.429\\
\textbf{Qwen-2.5-Instruct-7B}  
& -10.305 & -0.724 & 23.937 & 23.371 
& 21.852 & 0.980 & 37.425 & 9.573 
& 11.752 & 0.853 & 22.988 & 15.451\\
\midrule
\multicolumn{13}{c}{\textit{\textbf{FLAG-TRADER }}}\\
\textbf{SmolLM2-135M-Instruct}  
& 20.106 & 1.373 & 24.932 & 9.428 
& 33.724 & 3.344 & 17.174 & 9.320
& 46.799 & 1.463 & 67.758 & 35.039\\
\bottomrule
\end{tabular}
}
\begin{tablenotes}
    \footnotesize
    \item[1] \small{The Buy \& Hold strategy is a passive investment approach commonly used as a baseline strategy, where an investor \\ purchases stocks and holds onto them for an extended period regardless of market fluctuations.}
    \item[2] \small{An upward arrow (\(\uparrow\)) next to a metric indicates that higher values signify better performance, while a downward arrow (\(\downarrow\))} \\
indicates that lower values are preferable.
    \item[3] \small{The numbers highlighted in red indicate the best-performing outcomes for the corresponding metrics.}
\end{tablenotes}
\end{threeparttable}
\vspace{-0.0cm}
\end{table*}


This section describes the overall experimental design and environmental setup for comparing the performance of different trading agents under consistent conditions.

\subsection{Experiment Setup}
For our single-asset trading tasks, we adopt two baselines: the buy-and-hold strategy and the LLM-based trading agent from \texttt{INVESTORBENCH} \cite{li2024investorbench}, which integrates 13 proprietary or open-source large language models. Our proposed model, \textsc{FLAG-Trader} (built on a 135M-parameter LLM), is then evaluated against these baselines for a comprehensive performance comparison.

We focus on five stocks and one crypto: Microsoft Corporation (MSFT), Johnson \& Johnson (JNJ), UVV Corporation (UVV), Honeywell International Inc. (HON), Tesla, Inc. (TSLA) and Bitcoin (BTC). As summarized in Table 1, each agent’s performance is measured across these assets. All language models use a temperature of 0.6 during inference to balance consistency and creativity in their responses.

We report four metrics-Composite Return (CR), Sharpe Ratio (SR), Annualized Volatility (AV), and Maximum Drawdown (MDD) and select final results from the test trajectory corresponding to the median of these metrics. If the median values arise from different epochs, we prioritize the run producing the median SR. Due to varying data availability, warm-up and test periods may differ. For the trading tasks of five stocks, the warm-up period is July 1, 2020, to September 30, 2020, and the test period is October 1, 2020, to May 6, 2021. On the other hand, the warm-up period of BTC trading is from 2023-02-11 to 2023-04-04 and the test period is from 2023-04-05 to 2023-11-05.

We deploy LLMs using the vllm framework, with configurations depending on model size. Small-scale models (under 10B parameters) run on two RTX A6000 GPUs (48GB each), mid-scale models (10B–65B parameters) require four RTX A6000 GPUs, and large-scale models (over 65B parameters) use eight A100 GPUs (80GB each). These setups provide sufficient resources for both inference and training, enabling a fair comparison of trading performance across different assets.
\textsc{FLAG-Trader} is trained by using PPO algorithm, which is detailed in Algorithm \ref{alg:flagtrader-ppo} in Appendix \ref{sec:Appendix_A}.  



\subsection{Evaluation Metrics}
We use four widely recognized financial metrics \cite{hull2007risk} to evaluate and compare the investment performance of various LLM backbones across different tasks: Cumulative Return (CR), Sharpe Ratio (SR), Annualized Volatility (AV), and Maximum Drawdown (MDD). As CR and SR focus more on long-term gains and risk-adjusted returns, they are typically considered more important than AV and MDD for assessing asset trading performance. Accordingly, we treat CR and SR as our primary metrics for the final evaluation. 

\noindent\textbf{Cumulative Return (CR) \%} measures the total value change of an investment over time by summing logarithmic return calculated from daily PnL: 
\begin{align}
   \label{eq:cum_return}
   \textbf{CR} &= \sum_{t=1}^{T} \log{(1+\frac{pnl_t}{C_{t-1}+H_{t-1}P_{t-1}})},
\end{align}
where $pnl_t$ is the PnL at time $t$, and $C_{t-1}+H_{t-1}P_{t-1}$ is the account balance at time $t-1$. Notice that higher values indicate better strategy effectiveness.

\noindent\textbf{Sharpe Ratio (SR)} assesses risk-adjusted returns by dividing the average excess return ($R_p$) over the risk-free rate ($r_f$) by its volatility ($\sigma_p$):
\begin{equation}
    \textbf{SR} = \frac{R_p - r_f}{\sigma_p}.
\end{equation}  
Notice that higher ratios signify better performance.
  
\noindent\textbf{Annualized Volatility (AV) \% and Daily Volatility (DV) \%} quantify return fluctuations; AV is derived by scaling DV (\textit{standard deviation of daily logarithmic returns}) by the square root of the annual trading days (252): 
\begin{align}
   \label{eq:annuaVol}
    \textbf{AV} &= \textbf{DV} \times \sqrt{252}. 
\end{align} 
This metric highlights potential return deviations across the year.

\noindent\textbf{Max Drawdown (MDD) \%} calculates the largest  drop from peak to trough of the value of balance account:
    \begin{align}
    \label{eq:maxdrawdown}
    \textbf{MDD} = \text{max}(\frac{V_{\text{peak}} - V_{\text{trough}}}{V_{\text{peak}}}).
    \end{align}
Notice that  lower values indicate lesser risk and higher strategy robustness. 
\begin{table*}[thbp]
\renewcommand{\arraystretch}{1}
\vspace{-0.2cm}
\setlength{\abovecaptionskip}{0.1cm}
\centering
\caption{Performance of stock trading with different LLMs as backbone model across seven stocks.}
\label{tab:stock-trading-performance2}
\begin{threeparttable}
\scalebox{0.68}{\begin{tabular}{@{}lcccc|cccc|cccc@{}}
\toprule
\textbf{Model} & \multicolumn{4}{c}{\textbf{HON}} &\multicolumn{4}{c}{\textbf{TSLA}} & \multicolumn{4}{c}{\textbf{BTC}} \\
\cmidrule(lr){2-5}\cmidrule(lr){6-9}\cmidrule(lr){10-13}
& \textbf{CR\(\uparrow\)} & \textbf{SR\(\uparrow\)} & \textbf{AV\(\downarrow\)} & \textbf{MDD\(\downarrow\)}
& \textbf{CR\(\uparrow\)} & \textbf{SR\(\uparrow\)} & \textbf{AV\(\downarrow\)} & \textbf{MDD\(\downarrow\)}
& \textbf{CR\(\uparrow\)} & \textbf{SR\(\uparrow\)} & \textbf{AV\(\downarrow\)} & \textbf{MDD\(\downarrow\)} \\
\midrule
\textbf{Buy \& Hold}

& 33.256 & 2.347 & 23.967 & 9.195 
& 39.244 & 0.869 & 75.854 & 37.975
& 21.821 & 0.683 & 37.426 & 20.796 \\
\midrule
\multicolumn{13}{c}{\textit{\textbf{Financial Domain Models}}} \\
\textbf{Palmyra-Fin-70B}  

& 20.016 & 1.464 & 22.974 & 6.824 
& -6.661 & -0.222 & 50.379 & 25.820

& -20.812 & -1.212 & 20.036 & 27.782\\
\midrule
\multicolumn{13}{c}{\textit{\textbf{Proprietary Models}}} \\
\textbf{GPT-o1-preview} 

& 13.162 & 0.776 & 28.511 & 11.558 
& 34.499 & 0.796 & 72.822 & 35.490

& 34.060 & 1.114 & 35.846 & 17.075\\
\textbf{GPT-4} 

& 34.342 & 2.005 & 28.779 & 9.195
& 45.246 & 1.190 & 63.896 & 25.031

& 22.396 & 0.828 & 31.699 & 17.206\\
\textbf{GPT-4o} 
 
& 38.540 & 2.418 & 26.782 & 8.979 
& 45.946 & 1.348 & 57.281 & 21.631

& 14.330 & 0.532 & 31.304 & 17.278\\
\midrule
\multicolumn{13}{c}{\textit{\textbf{Open-Source Models}}} \\
\textbf{Qwen2.5-72B-Instruct}  

& 34.309 & 2.000 & 28.779 & 9.292 
& 39.112 & 1.075 & 61.136 & 26.985

& 0.549 & 0.325 & 1.979 & 0.897\\
\textbf{Llama-3.1-70B-Instruct}  

& 43.944 & 2.646 & 27.903 & 8.993 
& 37.545 & 0.891 & 70.815 & 29.813

& 20.440 & 0.758 & 31.604 & 17.813\\
\textbf{DeepSeek-67B-Chat} 

& 32.536 & 1.909 & 28.628 & 10.782 
& 35.647 & 0.885 & 67.660 & 33.359

& 28.307 & 0.891 & 37.219 & 17.944\\
% \midrule
\textbf{Yi-1.5-34B-Chat}  

& 30.743 & 1.823 & 28.335 & 9.195 
& 35.364 & 0.808 & 73.561 & 35.490

& 13.620 & 0.434 & 36.778 & 22.790 \\
\textbf{Qwen2.5-32B-Instruct}  

& 26.332 & 1.980 & 22.348 & 5.261
& 21.336 & 0.729 & 49.157 & 20.704

& 11.566 & 0.869 & 15.608 & 7.984\\
\textbf{DeepSeek-V2-Lite (15.7B)}  

& 16.686 & 0.974 & 28.771 & 16.806
& 31.458 & 0.744 & 68.524 & 35.404

& 4.804 & 0.153 & 36.846 & 20.562\\
% \midrule
\textbf{Yi-1.5-9B-Chat}  

& 29.028 & 1.700 & 28.682 & 12.588 
& 31.350 & 0.703 & 74.895 & 37.975

 & 7.953 & 0.253 & 36.799 & 26.545 \\
\textbf{Llama-3.1-8B-Instruct}  

& 39.079 & 2.320 & 28.299 & 10.341
& 35.622 & 0.832 & 71.936 & 36.383

& 20.521 & 0.646 & 37.240 & 21.104\\
\textbf{Qwen-2.5-Instruct-7B}  

& 4.291 & 0.285 & 24.933 &14.156 
& 41.203 & 0.925 & 74.862 & 37.975

& 19.477 & 0.612 & 37.289 & 20.796\\
\midrule
\multicolumn{13}{c}{\textit{\textbf{FLAG-TRADER }}} \\
\textbf{SmolLM2-135M-Instruct}  
 
& 34.342 & 2.429 & 23.913 & 10.872
& 50.394 & 1.362 & 64.004 & 37.975

& 45.511 & 1.734 & 30.903 & 24.440\\
\bottomrule
\end{tabular}
}
\begin{tablenotes}
    \footnotesize
    \item[1] \small{The Buy \& Hold strategy is a passive investment approach commonly used as a baseline strategy, where an investor \\ purchases stocks and holds onto them for an extended period regardless of market fluctuations.}
    \item[2] \small{An upward arrow (\(\uparrow\)) next to a metric indicates that higher values signify better performance, while a downward arrow (\(\downarrow\))} \\
indicates that lower values are preferable.
    \item[3] \small{The numbers highlighted in red indicate the best-performing outcomes for the corresponding metrics.}
\end{tablenotes}
\end{threeparttable}
\vspace{-0.0cm}
\end{table*}
\subsection{Experimental Results}
\begin{itemize}
    \item \noindent\textbf{FLAG-Trader achieves superior stock trading performance.} Unlike the baseline agent, which relies on an LLM-agentic framework, \textsc{FLAG-Trader} undergoes an RL-based post-training phase. As shown in Table~\ref{tab:stock-trading-performance1} and Table \ref{tab:stock-trading-performance2}, \textsc{FLAG-Trader} consistently outperforms the baseline across multiple metrics, demonstrating its stronger adaptability and optimization in diverse market environments.

    \item \textbf{Convergence to a (stable) optimal policy.} When using an LLM as the policy network in deep RL, the policy is jointly parameterized by the model’s intrinsic parameters and the prompt. Although the initial prompts strongly influence policy generation during the first few training epochs, this effect diminishes over time. Consequently, the system converges to a relatively stable policy that becomes less sensitive to the initial prompts, indicating that RL training allows the LLM-based agent to refine its strategy and achieve a robust trading policy.

    \item \noindent\textbf{\textsc{FLAG-Trader} enables small-scale models to surpass large-scale counterparts.} While increasing model size generally enhances financial decision-making and robustness— as seen with large proprietary models (e.g., GPT-o1-preview) in the baseline framework-\textsc{FLAG-Trader} leverages an RL-based training pipeline to enable a 135M-parameter open-source model to outperform significantly larger models in financial trading tasks. This demonstrates that a well-designed training strategy can bridge or even surpass the performance gap typically associated with model scale. 


\end{itemize}

 



% \begin{table*}[thbp]
% \renewcommand{\arraystretch}{1}
% \vspace{-0.2cm}
% \setlength{\abovecaptionskip}{0.1cm}
% \centering
% \begin{threeparttable}
% \scalebox{0.50}{
% \begin{tabular}{@{}lcccc|cccc@{}}
% \toprule
% \textbf{Model} & \multicolumn{4}{c}{\textbf{TSLA}} & \multicolumn{4}{c}{\textbf{Average}} \\
% \cmidrule(lr){2-5}\cmidrule(lr){6-9}
% & \textbf{CR\(\uparrow\)} & \textbf{SR\(\uparrow\)} & \textbf{AV\(\downarrow\)} & \textbf{MDD\(\downarrow\)}
% & \textbf{CR\(\uparrow\)} & \textbf{SR\(\uparrow\)} & \textbf{AV\(\downarrow\)} & \textbf{MDD\(\downarrow\)} \\
% \midrule
% \textbf{Buy \& Hold}
% & 39.244 & 0.869 & 75.854 & 37.975
% & 34.099 & 0.732 & 74.012 & 34.953 \\
% \midrule
% \multicolumn{9}{c}{\textit{\textbf{Financial Domain Models}}} \\
% \textbf{Palmyra-Fin-70B}  
% & -6.661 & -0.222 & 50.379 & 25.820

% & -0.453 & 0.031 & 63.660 & 36.564 \\
% \midrule
% \multicolumn{9}{c}{\textit{\textbf{Proprietary Models}}} \\
% \textbf{GPT-o1-preview} 
% & 34.499 & 0.796 & 72.822 & 35.490

% & 25.057 & 0.592 & 69.446 & 34.639 \\
% \textbf{GPT-4} 
% & 45.246 & 1.190 & 63.896 & 25.031

% & 43.696 & 0.972 & 68.654 & 27.339 \\
% \textbf{GPT-4o} 
% & 45.946 & 1.348 & 57.281 & 21.631

% & 39.031 & 1.041 & 56.642 & 21.225 \\
% \midrule
% \multicolumn{9}{c}{\textit{\textbf{Open-Source Models}}} \\
% \textbf{Qwen2.5-72B-Instruct}  
% & 39.112 & 1.075 & 61.136 & 26.985

% & 46.153 & 1.276 & 55.686 & 19.523 \\
% \textbf{Llama-3.1-70B-Instruct}  
% & 37.545 & 0.891 & 70.815 & 29.813

% & 38.946 & 0.864 & 70.907 & 30.738 \\
% \textbf{DeepSeek-67B-Chat} 
% & 35.647 & 0.885 & 67.660 & 33.359

% & 26.941 & 0.717 & 64.975 & 30.030\\
% % \midrule
% \textbf{Yi-1.5-34B-Chat}  
% & 35.364 & 0.808 & 73.561 & 35.490

% & 37.966 & 0.831 & 72.918 & 34.321 \\
% \textbf{Qwen2.5-32B-Instruct}  
% & 21.336 & 0.729 & 49.157 & 20.704

% & 20.884 & 0.823 & 47.341 & 22.541 \\
% \textbf{DeepSeek-V2-Lite (15.7B)}  
% & 31.458 & 0.744 & 68.524 & 35.404

% & 28.745 & 0.813 & 61.192 & 40.439 \\
% % \midrule
% \textbf{Yi-1.5-9B-Chat}  
% & 31.350 & 0.703 & 74.895 & 37.975

% & 22.913 & 0.478 & 72.448 & 36.946 \\
% \textbf{Llama-3.1-8B-Instruct}  
% & 35.622 & 0.832 & 71.936 & 36.383

% & 25.463 & 0.567 & 72.262 & 37.219 \\
% \textbf{Qwen-2.5-Instruct-7B}  
% & 41.203 & 0.925 & 74.862 & 37.975

% & 29.515 & 0.722 & 72.074 & 34.353 \\
% \midrule
% \multicolumn{9}{c}{\textit{\textbf{Fin RL Models}}} \\
% \textbf{SmolLM2-135M-Instruct}  
% & 50.394 & 1.362 & 64.004 & 37.975

% & 35.696 & 2.017 & 36.278 & 19.022 \\
% \bottomrule
% \end{tabular}
% }
% \end{threeparttable}
% \vspace{-0.2cm}
% \end{table*}


