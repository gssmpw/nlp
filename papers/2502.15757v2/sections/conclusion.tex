\section{Conclusion}
We proposed two new deep learning models MLPLOB: A simplified yet effective MLP-based architecture and TLOB, a Transformer-based approach, for the task of stock price trend prediction on Limit Order Book (LOB) data. Both models demonstrated superior performance compared to existing state-of-the-art approaches, with TLOB showing particular promise in handling high-frequency market data. NASDAQ stocks (Tesla, Intel) proved significantly more challenging to predict than Finnish stocks (FI-2010). Our research also showed that prediction accuracy decreases as the forecasting horizon increases, highlighting the inherent challenges of long-term prediction in financial markets.
 
\textbf{Limitations}: When considering practical implementation, we found that defining trend thresholds based on average spread (transaction costs) significantly impacts model evaluation and potential profitability. This finding underscores the critical gap between academic performance metrics and practical trading applicability.

\textbf{Future works} Looking ahead, several avenues for future research emerge. The investigation of scaling laws for financial deep learning models remains an open question, as does the development of more robust approaches to handling increased market efficiency and complexity. Additionally, the exploration of alternative trend definition methodologies that better align with practical trading constraints could prove fruitful.

\textbf{Risks}: Firstly, it is important to acknowledge that the proposed methodologies are not sufficiently mature for practical deployment in live trading environments. Furthermore, the application of deep learning models to stock price prediction and subsequent utilization in trading carries significant inherent risks. A primary concern is the limited explainability of such models. 
Furthermore, automated AI models, increasingly integrated into financial markets, present significant risks to financial stability due to their potential to amplify systemic vulnerabilities. These models, often operating with limited transparency, can trigger rapid and widespread market reactions, exacerbating volatility and potentially leading to cascading failures across the financial system.