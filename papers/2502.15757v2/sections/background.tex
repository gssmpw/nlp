\section{Background}
In the contemporary, highly competitive financial landscape, the predominant mechanism for recording and managing market transactions is the electronic Limit Order Book (LOB). Within a limit order book market, traders can submit orders to buy or sell a specified quantity of an asset at a predetermined price. Three primary order types are prevalent in such markets: (1) \textbf{Market orders}, which are executed immediately at the best available price with a predetermined quantity; (2) \textbf{Limit orders}, allows traders to decide the maximum (in the case of a buy) or the minimum (in the case of a sell) price at which they want to complete the transaction. A quantity is always associated  with the specified price; and (3) \textbf{Cancel orders} (alternatively referred to as deletions), which serve to remove an active limit order.

The LOB is a data structure that maintains and matches active limit orders and market orders in accordance with a predefined set of rules. This structure is transparently accessible to all market participants and is subject to continuous updates with each event, including order placement, modification, cancellation, and execution. The most widely adopted mechanism for order matching is the Continuous Double Auction (CDA) \cite{bouchaud2018trades}. Under the CDA framework, orders are executed whenever the best bid (the highest price a buyer is willing to offer) and the best ask (the lowest price a seller is willing to accept) overlap. This mechanism facilitates continuous and competitive trading among market participants. The price of a security is commonly defined as the mid-price, calculated as the average of the best ask and best bid prices, with the difference between these prices representing the bid-ask spread.

Given that limit orders are organized into distinct depth levels, each comprising bid price, bid size, ask price, and ask size, based on their respective prices, the temporal evolution of a LOB constitutes a complex, multidimensional temporal problem. Research on LOB data can be broadly categorized into four primary types: empirical analyses of LOB dynamics \cite{cont2001empirical, bouchaud2002statistical}, price and volatility forecasting \cite{zhang2019deeplob, sirignano2019deep}, stochastic modeling of LOB dynamics \cite{cont2011statistical, gould2013limit}, and LOB market simulation \cite{byrd2020abides, coletta2021towards, li2020generating}.
