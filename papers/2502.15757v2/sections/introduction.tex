\section{Introduction}
Over the past few decades, the global financial landscape has undergone a profound transformation, transitioning from manual trading operations to sophisticated electronic platforms. This evolution has been so significant that by 2020, electronic trading accounted for over 99\% of equity shares traded in the United States, a stark contrast to just 15\% in 2000 \cite{kissell2020algorithmic}. At the heart of this revolution lies the electronic Limit Order Book (LOB), a dynamic data structure that has become the cornerstone of modern financial markets.
In today's competitive financial world, the majority of the markets utilize electronic LOBs to record trades. 
%Unlike traditional quote-driven marketplaces, where traders were limited to prices set by market makers, modern electronic trading platforms enable market participants to directly view and interact with all resting limit orders. 
The continuous inflow of limit orders, organized by price levels, creates a dynamic structure that evolves over time, reflecting the real-time balance of supply and demand. However, this multidimensional structure, which spans price levels and volumes, presents complex challenges for understanding market behavior, forecasting stock price trends, and simulating realistic market conditions.
The non-stationary nature of LOB data, characterized by its stochastic behavior, makes modeling stock price movements challenging. Traditional statistical methods fail to capture these complexities, especially when attempting to predict short-term price trends. However, recent advancements in deep learning have opened new avenues for tackling these challenges, offering the ability to model the non-linear relationships and temporal dependencies inherent in LOB data.

Stock Price Trend Prediction (SPTP)\footnote{In the literature it is also referred to as mid-price movement prediction.} remains one of the most challenging and economically significant problems in financial markets, attracting significant attention from academic researchers and industry practitioners. One prominent application of SPTP, particularly utilizing Limit Order Book (LOB) data, lies within high-frequency trading, where algorithms attempt to capitalize on short-term price movements. Predicting future market movements is a highly challenging task due to the complexity, non-stationarity, and volatility of financial markets. However, with the growing availability of Limit Order Book (LOB) data and advancements in deep learning, new opportunities have emerged to improve the accuracy of these predictions. This paper explores the application of deep learning models to SPTP using Limit Order Book (LOB) data, which provides the most granular and complete information on stock trades.
Financial markets do not exist in a vacuum; they are continuously shaped by the actions and expectations of countless participants who, according to the Efficient Market Hypothesis (EMH), collectively incorporate all available information into asset prices. When models discover a predictive pattern and traders act on it, the anomaly is quickly competed away, causing a paradox: successful signals sow the seeds of their own demise. Over time, greater liquidity, advanced trading technologies, and the proliferation of algorithmic strategies intensify this effect, i.e., any exploitable signal becomes visible in execution data and erodes more rapidly. Consequently, the apparent decline in forecast accuracy in our findings aligns with EMH principles: as soon as new patterns are detected and exploited, the relentless engine of arbitrage drives markets back toward efficiency. This interplay underlines why forecasting often becomes more difficult the farther we move from idealized, less liquid markets (like FI-2010) toward active, high-efficiency markets (like NASDAQ), thereby illustrating a core tension between the pursuit of alpha and the self-correcting nature of competitive markets.
%The field of stock price prediction has evolved significantly with the advent of deep learning. 
Traditional approaches relied on technical analysis and statistical methods, but recent years have seen a shift toward more sophisticated deep learning methods.
A lot of different types of deep learning architectures have been utilized to tackle the SPTP tasks. Tsantekidis et al. utilized Long Short-Term Memory (LSTM) layers \cite{tsantekidis2017forecasting} and Convolutional Neural Networks (CNNs) \cite{tsantekidis2017using, tsantekidis2020using}. 
Zhang et al. \cite{zhang2019deeplob} introduced the DeepLOB model, which leverages LOB data to predict mid-price movements using a combination of convolutional and LSTM layers.
Recent work \cite{prata2024lob} has highlighted the limitations of existing models, particularly their lack of robustness and generalizability when applied to diverse market conditions and more efficient stocks.
In this paper, we address these limitations by proposing TLOB, a transformer-based approach, that outperforms all the existing models on both benchmark and real-world datasets, paving the way for more reliable SPTP applications. We introduce also an MLP-based model to show that a simple architecture, based on fully connected layers and GeLU activation function, can outperform all the SoTA models.
We list our contributions:
\begin{enumerate}
    \item \textbf{Novel Architecture Proposals}: We introduce two new deep learning models:
    \begin{itemize}
        \item \textbf{MLPLOB}: A simple yet effective MLP-based model inspired by recent advances in the deep learning literature.
        \item \textbf{TLOB}: A transformer-based approach that leverages dual attention mechanisms for both temporal and spatial relationships in LOB data.
    \end{itemize}
    \item \textbf{Comprehensive Evaluation}: We conduct extensive experiments on both the benchmark FI-2010 dataset and a real-world NASDAQ dataset composed of Tesla and Intel stocks, with several baselines, providing insights into model performance across different market conditions and time horizons. We also perform an ablation study investigating the design choices of TLOB. 
    \item \textbf{New Labeling Methods}: We introduce a new labeling method that improves on previous ones, removing the horizon bias.
    \item \textbf{Historical Comparison}: We examine whether stock price prediction has become more difficult over time by comparing model performance on historical data from different periods. 
    \item \textbf{Alternative Threshold Definition}: We propose and evaluate a novel approach to defining trend classification thresholds based on average spread, directly incorporating the primary transaction cost into the prediction framework. 

\end{enumerate}


    