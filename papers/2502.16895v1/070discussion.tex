\chirev{This section discusses consideration, opportunities, and future research directions for LLM-assisted analogical education based on our study results and designed system.}
% We conducted in-class tests and classroom studies to evaluate the effectiveness of LLM-generated analogies on students' understanding and classroom practice. 
% Our findings suggest that, without teacher intervention, \textbf{1)} LLM-generated analogies generally help in problem-solving, especially for biological concepts; \textbf{2)} however, errors and missing information in these analogies can negatively affect understanding, and \textbf{3)} students may overestimate their comprehension due to overconfidence. 
% With teacher intervention, we found that \textbf{4)} teachers' demand for analogies relates to the characteristics of the subject, and that \textbf{5)} effective analogies should align with teaching focus and preferences. \textbf{6)} Teachers are willing to refine analogies during lesson preparation to avoid negative effects, and \textbf{7)} LLM-generated analogies can inspire them to develop new analogies and explore new teaching methods, and \textbf{8)} boost students' classroom engagement and homework performance.
% Based on these insights, we discuss the implications for the future development and evaluation of LLM-assisted education and LLM-generated analogies and our study's limitations.

\chifinal{
\subsection{Subject Differences in LLM-Generated Analogy Effectiveness}
% Study I showed that while LLMs generally produced correct analogies in biology, their physics analogies were often incorrect, as none of the analogies for the five physics concepts in Sec.~\ref{sec:study1_data_preparation} were classified as correct and satisfying, displaying various factual and consistency errors. In Study II, these issues were mitigated by prompting LLMs with our summarized strategies (Sec.~\ref{sec:study22_data_preparation}) to draw on other learned physical concepts and focus on a single aspect of physical concepts when generating analogies. 
% Although physics teachers were willing to refine and use these analogies in the classroom, which helped students stay focused and improve their homework performance, they still commented on many of the analogies as ``interesting but unprofessional.'' 

Our study shows that LLMs generally produce correct and satisfying analogies for biological concepts but generate incorrect or correct yet unprofessional ones for physics.

Several factors appear to contribute to these shortcomings. 
First, physical concepts are highly abstract (e.g., ``mass-energy conversion''), with complex and formula-driven features, making it difficult to find real-life analogies or other concepts that perfectly align with them. 
In contrast, biological concepts are more concrete and observable, often tied to specific structures and functions (e.g., ``mitochondria as the powerhouse of the cell'').
As a result, LLMs would produce forced analogies or oversimplifications for physics while generating satisfying ones for biology.
Second, restricting the physics analogy to a single aspect, as in the strategies used in Study II (Sec.~\ref{sec:study22_data_preparation}), can yield correct and engaging analogies.
However, for highly abstract concepts, these analogies may still be superficial and offer limited support, as noted by physics teachers in Sec.~\ref{sec:study22_results_analysis}.
Third, teaching materials in physics contain more formulas and fewer analogies compared to biology. As a result, LLMs learn fewer physics analogies and generate less effective analogies.

These findings suggest that using LLMs for educational analogy generation is tied to subject characteristics, and we can infer that it may be particularly challenging for subjects lacking clear real-world counterparts (e.g., mathematics). 
In contrast, they might work better for subjects with more directly observable phenomena  (e.g., high school chemistry, biology), which should be confirmed by future studies. 
Nonetheless, analogies help students engage with abstract subjects like physics and math by inspiring interest and sustaining attention. 
More studies are needed to verify the effectiveness and needs of LLM-generated analogies across a broader range of subjects, in conjunction with the review and refinement of teachers before their use.
}

\chirev{
\subsection{Generating High-Quality Analogies}
%The current analogies generated automatically have limitations in scientific accuracy and educational effectiveness. 
The automatically generated analogies have limitations in scientific accuracy and educational effectiveness. 
In our practical system, LLMs summarized principles from human feedback and incorporated them into the next round of analogy generation.
Several teachers in the system evaluation found this approach effective for improving analogy quality.
Building on this, we can incorporate Reinforcement Learning with Human Feedback (RLHF). 
Using teachers' feedback and preferences, we can create a reward model that continually refines the analogies.
%In addition to human-in-the-loop generation, future work should explore automatic methods for generating higher-quality analogies to reduce the burden on teachers. 
To further reduce teachers' workload, future work should explore automatic methods for generating higher-quality analogies.
\chifinal{
First, we could explore multi-agent collaboration methods to further mitigate hallucinations~\cite{SHI2025125723}, including factuality errors and consistency errors, as outlined in Study I.
Besides}, instead of waiting for more advanced general-purpose LLMs to be released~\cite{openai_o1_preview_2024}, we can fine-tune existing models with teacher-adjusted analogies.
Additionally, our system can be transformed into a labeling tool to collect high-quality educational analogy datasets, consisting of revised analogies or new analogies proposed by teachers.
Another bottleneck for generating high-quality analogies is the LLMs' limited domain-specific knowledge. 
In Study II, the physics teacher attributed the current analogies' interesting yet unprofessional nature to the LLMs' limited understanding of abstract physics concepts.
% As a result, LLMs oversimplify the concepts or reinforce incorrect analogies. 
%Future work should focus on fine-tuning LLMs in specific domains and subjects to improve LLMs' understanding of concepts and thus enhance analogy quality.
To address this, we should fine-tune LLMs for specific subjects, improving their understanding and enhancing analogy quality.
Finally, \chifinal{to improve control over the complexity and ethical considerations and make analogies suited for the intended educational level and scenario,} the future analogy generation pipeline should consider factors like students' educational background, cultural context, and prior knowledge.
}


\chirev{
\subsection{LLMs \chifinal{for} Teaching by Analogy}
% \textbf{Practical Benefits.} 
Our Study II and system evaluation show that LLM-generated analogies are valuable to teachers in three progressive aspects: short-term lesson preparation, teaching strategy development, and professional growth. 
For short-term lesson preparation, teachers are able and willing to select and modify LLM-generated analogies or inspire new ones to suit specific concepts and lessons (Sec.~\ref{sec:system_findings}), or use generated content to help set quizzes (Sec.~\ref{sec:study22_results_analysis}). 
For teaching strategy development, continuous use of LLM-generated analogies leads to positive feedback from the classroom and students' homework and iteratively encourages teaching by analogy (Sec.~\ref{sec:study22_results_analysis}). 
Regarding professional growth (Sec.~\ref{sec:system_findings}), providing feedback on LLM-generated analogies helps teachers actively reflect on teaching points and build their knowledge base, while LLM-generated principles based on their feedback also serve as valuable reminders for teachers, supporting their ongoing professional development and enhancing teaching expertise.
Given these benefits, future work should explore the varying needs and develop practical systems to benefit teachers with different experience levels and subjects.
In addition, long-term evaluation of such practical systems and teachers is needed to fully understand the actual benefits.

% \textbf{Potential over-reliance.} 
Besides the benefits, over-reliance on LLM-generated analogies warrants attention. 
In Study II, teachers emphasized during pre-class interviews and demonstrated in classroom teaching that they could avoid over-relying on such content.
However, in the system evaluation, most teachers did not revise analogies within the system but recorded changes elsewhere, following their own lesson preparation habits. 
This suggests that monitoring teachers' interactions with analogies only through system logs might be insufficient, potentially allowing unnoticed over-reliance to develop.
To address this, the system could use pop-up reminders to alert users against over-reliance and encourage them to edit or provide comments when no manual interaction is detected for an extended period.
% Teachers could be encouraged to provide comments or make changes within the system, but this should not be mandatory, as we need to respect teachers' lesson preparation habits.
% In particular, neither Study II nor the system evaluation included novice teachers (e.g., Teaching Assistants), who may lack well-established lesson preparation habits and are more likely to rely on LLM-generated analogies without proper guidance.
Nevertheless, supervision from schools and higher authorities is essential. Additionally, regular updates from system developers to educators~\cite{tan_more_2024,kasneci2023chatgpt} are crucial for maintaining an accurate understanding of model capabilities and ensuring effective use.
% In Study II pre-class interviews, teachers noted that their teaching experience enabled them to use AI-generated contents as references or supplements, maintaining the ability to select, modify, and innovate without over-reliance on AI. 
% They suggested that inexperienced teachers should first build lesson preparation habits through their own thinking or guidance from experienced teachers, rather than relying on AI-generated analogies, which might hinder critical thinking.
% Hence, the use of LLMs to generate analogies and other teaching resources should be closely monitored by schools and senior teachers. 
}

\chirev{
\subsection{\chifinal{Integrating Analogies into LLM-Assisted Education Platforms}}
% \subsection{LLMs \chifinal{for Self-Learning} by Analogy}
\label{sec: discussion_student}

\chifinal{
Integrating analogy generation into LLM-assisted education platforms might benefit teachers and students.

For teachers, the interplay between analogy generation and LLM-assisted teaching material preparation is mutually reinforcing.
First, analogies help develop teaching materials by providing relatable explanations.
For example, LLM-assisted platforms already help novice teachers generate lesson plans\chirev{~\cite{lessonplanner2024uist}}. 
Integrating analogy generation into these platforms can support analogy-based explanations at different teaching stages.
Additionally, participants in our system evaluation demonstrated the potential of using analogies in quiz settings, highlighting its role in automated problem generation.
Second, existing LLM-assisted teaching preparation platforms enhance context-aware analogy generation and verification. 
These platforms usually consider students’ knowledge levels and course context~\cite{GeneratingContextualized}, which could be integrated into analogy generation pipelines to help generate analogies suited for practical scenarios. 
Moreover, existing problem generation platforms could be enhanced to generate quizzes that verify students' understanding of analogy, reinforcing the effectiveness of teaching by analogy.
}

\chifinal{However, the} results of Study I indicate that LLM-generated analogies without human intervention are unreliable for students, making it premature to directly integrate analogy generation into self-learning systems. 
In contrast, teacher-adjusted analogies in Study II ensured correctness and reliably impacted students' classroom feedback and homework performance. 
Therefore, future self-learning systems should only consider pre-set teacher-reviewed analogies for key knowledge points to aid student understanding. 
However, even correct biological analogies from Study I led to negative effects, with students over-relying on them with incorrect learning strategies or becoming subjectively overconfident.
This suggests that the self-learning system should flexibly structure the learning process and, after presenting analogies, guide students back to the textbook with more detailed follow-up questions.
Besides, timely pre-set and teacher-reviewed exercises with feedback and explanation that reveal the limitations of analogies help students reflect on their learning and monitor their learning approaches.
Overall, incorporating analogies into self-learning systems requires careful attention from teachers and system developers to mitigate potential negative effects.
}

% \subsection{Towards LLM-assisted Education Tools Creation with Generated Analogy}
% First, our prompt instructions and principles, refined through multiple iterations in both studies, should apply to future LLM-assisted educational tools that use analogies. 
% Results from Study I indicate that while LLMs can generate satisfactory analogies for biological concepts, analogies for physical concepts may not be effective. 
% Moreover, even with biological concepts, students may over-rely on analogies or overestimate their understanding. 
% Therefore, in LLM-supported self-learning tools without human guidance, for analogies of abstract concepts represented by physics, it require more effort to deconstruct key knowledge points effectively. 
% For analogies of concrete concepts represented by biology, timely tests are needed to assess student understanding, prevent negative effects, and avoid overconfidence.
% Study II shows that teachers can effectively use LLM-generated analogies tailored to their needs to enhance classroom teaching. 
% Thus, tools designed to support teachers should understand and integrate teachers' needs as we did in our pre-class interviews, and generate diverse analogies to inspire lesson planning.
% With the rapid advancement of LLMs, more powerful models will be released~\cite{openai_o1_preview_2024} and able to generate analogies with fewer errors. 
% However, our method, particularly the in-classroom study, offers a complete example and inspiration for future exploration with new LLMs. 
% The principles and requirements summarized in our study also guide future LLMs in generating educational analogies.
% % However, generating analogies that match each abstract feature of scientific concepts is challenging, as some may lack real-life counterparts, even with improved model capabilities.
% % Therefore, it also necessitates careful consideration of specific aspects of analogy generation and requires the involvement and guidance of teachers, as illustrated in our Study II.

% \subsection{Evaluating LLM-assisted Education Beyond Generating Analogies}
% \chirev{Our study focuses on analogies, but the study design can also be adapted to evaluate similar features, such as LLM-generated example-based explanations for physics concepts in Study I.
% Students may complete in-class tests with example-based explanations, followed by teachers using them in the classroom.
% Besides, various features might be integrated into LLM-assisted educational systems that support teachers or students in the future, such as LLM-generated analogies, examples, quizzes~\cite{readingquizmaker}, and lesson plans for novice teachers~\cite{lessonplanner2024uist}.
% It would be important to assess the interplay of analogies and other LLM-assisted features within systems and study their long-term educational impact~\cite{Lyu2024evaluating,chen2024stugptviz}.}
% % For teachers, we aim to understand how they use AI to create analogies linked to other lesson materials, either through in-depth interviews~\cite{tan_more_2024} or by analyzing conversational logs between teachers and AI.

% \chifinal{
% \subsection{Integration of Analogies with LLM-based Educational Tools}

% }


\subsection{Evaluating Broader Analogies in Education}
Analogies serve multiple purposes beyond students' understanding \chirev{and teachers' teaching}, which adds challenges to their evaluation.
In mathematical problems and similar domains, specific procedures involving numeracy and variables often require a different type of analogy, known as procedural analogy~\cite{richland_analogy_2004}. 
Such procedural analogies were also mentioned during our interview with the physics teacher in Study II. 
Due to their rarity and complexity, these analogies, even those crafted by humans, have not been thoroughly evaluated.
LLMs can lower the barrier to creating such analogies due to advanced reasoning abilities and broad subject knowledge, such as linking the formula for a ``spring oscillator'' with that of a ``pendulum.'' 
Our study design can be extended to such analogies by involving calculation questions involving formulas in controlled in-class tests.
Additionally, analogies are utilized for socialization, helping to educate children on becoming better students and enacting behavioral changes~\cite{richland_analogy_2004}.
The evaluation of such analogies involves contexts beyond the classroom, which brings challenges to study design and needs to be explored in the future.
% Analogies also play a significant role for teachers. 
% Using analogies can exemplify teaching models and effective cases can enhance new teachers’ instructional strategies~\cite{oliva_teaching_2007}. 
% In Study II, we also found that LLM-generated analogies prompted teachers to rethink their teaching methods, including the frequency, type, and timing of analogy use. 
% However, a more comprehensive and long-term evaluation is necessary to fully evaluate such impact.

\subsection{Limitation}
Although we have gained lots of evidence and knowledge, our work is limited by student and teacher participation and analogy representation.

% \subsubsection{Limitations of LLM-Generated Analogies}
% \ysy{
% While LLM-generated analogies show promise in facilitating the understanding of scientific concepts, several limitations warrant careful consideration. 
% These challenges mainly arise from the nature of LLMs, which rely on probabilistic pattern recognition instead of true comprehension of scientific principles.
% First, analogies may oversimplify complex phenomena, causing confusion or reinforcing incorrect ideas. For example, simplified analogies in topics like quantum mechanics or cellular processes may omit critical details.
% Additionally, analogies that are not tailored to the student’s prior knowledge, educational background, or cultural context risk alienating or confusing students rather than aiding comprehension.
% Moreover, Biases in training data may lead to stereotypes or inappropriate associations, reducing educational value.
% Furthermore, models trained primarily in English and on mainstream subjects may struggle with less common languages or interdisciplinary topics.
% }

% \ysy{
% Despite these limitations, LLM-generated analogies offer an innovative starting point for enhancing science education. To harness their potential while mitigating challenges, several strategies should be explored. For example, integrating human-in-the-loop systems can ensure outputs are scientifically accurate and pedagogically aligned. Similarly, fine-tuned models trained on expert-reviewed educational content could improve reliability and depth.
% }

\subsubsection{Limitations in Student and Teacher Participation.}
% We thank the high school from China for supporting our study in a real teaching environment and two teachers deeply involved in a one-week study. 
% However, due to practical limitations, it was conducted in only one school and with two teachers. 
% To enhance representativeness, we plan to expand to multiple districts, schools, and teachers.
\chirev{Due to practical limitations, the teachers and students in our first two studies were from one high school, and the sample size was limited.
To explore real-world practicality, we expanded the participant pool by inviting more teachers from different schools with varying teaching experiences to evaluate the practical system.
Additionally, our study is limited to physics and biology due to practical constraints, excluding other subjects like chemistry.
Besides, our study with high school students may not be generalizable to younger students who might not possess developed analogical reasoning abilities or have limited world knowledge~\cite{vendetti_analogical_2015}.
\chifinal{Moreover, our study's demographic generalizability is limited, as all participants are Chinese, while prior research~\cite{richland_cognitive_2007} suggests that U.S. teachers provide cognitive support for analogies less frequently than teachers from Hong Kong or Japan in math instruction.}
In future work, we intend to expand the sample size, range, and diversity of subjects and explore diverse education levels for comprehensive evaluation.}
% Furthermore, study results could vary among students with different academic capabilities. 
% For example, prior research shows a link between patients' numerical skills and the effectiveness of analogies in doctor-patient communication, suggesting similar factors could affect educational contexts~\cite{galesic_using_2013}.
% Additionally, our study is limited to physics and biology due to practical constraints, excluding other subjects like chemistry.
% In future work, We intend to expand the sample size and range of subjects in evaluation.
% We also plan to integrate video analysis~\cite{richland_analogy_2004} for recording classroom practice and AI-supported assessments~\cite{ngoon_classinsight_2024} to achieve comprehensive evaluation.

% \subsubsection{Limitations in Teacher Participation.}
% Our study's limitations arise from the small number of teacher participants, which is also restricted by the high workload and grading focus in high schools. 
% This may not fully capture diverse teaching perspectives.
% For example, teachers in Asia often use more visual cues when explaining analogies compared to teachers in America~\cite{richland_cognitive_2007}.
% % Besides, there may be inconsistency between teachers' self-reported practices and their actual classroom behaviors about analogies~\cite{treagust_science_1992}, potentially skewing our understanding of the real need for analogies in classroom education. 
% % For example, Treagust et al.~\cite{treagust_science_1992} highlighted that although observations from classroom practice revealed infrequent and non-elaborate use of analogies, the majority of interviewed teachers believed they employed them regularly.
% Enhancing data accuracy could involve integrating video analysis~\cite{richland_analogy_2004} and AI-supported assessments~\cite{ngoon_classinsight_2024} rather than relying solely on self-reports and interviews.


\subsubsection{Limitations in Representation of Analogy.}
The practical use of analogies in teaching extends beyond the free-form analogies we generated, incorporating visual aids and dynamic technologies to enhance understanding and interaction~\cite{richland_cognitive_2007,richland_analogy_2015}.
Our interviews in Study II \chirev{and system evaluation} also revealed that teachers \chirev{have the desire to} use images, videos, and physical aids to convey analogies. 
Moving forward, we plan to enrich LLM-generated analogies with rich text, structured representations, and generated visuals to benefit teachers \chirev{in practical systems.}


%Similarly, the results of our study are also limited by the teacher participants.
%First, despite our best efforts to recruit teachers, the demanding workload and focus on grades typical of high school environments limited participation to only two teachers. 
%This limited number might have affected the representativeness of teachers' perspectives in our study.
%However, teaching methods vary between different regions and individual educators. 
%As shown in Study II, LLMs were able to generate the needed analogies for two teachers with different needs, demonstrating the potential to help more groups and individuals.
% \subsection{Evaluating the Broader Role of Analogies and LLMs in Education}

%To enhance the accuracy of our requirement derivation, it would be beneficial to incorporate video analysis of classroom sessions~\cite{richland_analogy_2004} and employ automatic assessments using intelligent AI-supported systems~\cite{ngoon_classinsight_2024}, rather than solely depending on interview findings and self-reported data from teachers.
%In Study II, we may also provide such materials to teachers to reduce their preparation costs.
%However, integrating multimodal representations poses more difficulties for LLMs.
%Our current study design can be used in future research to evaluate the benefits and challenges of expanding the role of large models in supporting teachers and students.

