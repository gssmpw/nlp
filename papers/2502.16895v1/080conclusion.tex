\chirev{
In this work, we first conducted in-class tests and classroom experiments guided by pre-class interviews to evaluate the effectiveness of LLM-generated analogies in two educational scenarios. 
Our in-class tests suggest that LLM-generated analogies could be beneficial for students' understanding, especially on biology concepts, but are unsuitable for self-learning systems without teacher intervention due to students' over-reliance and overconfidence. 
Classroom experiments reveal that teachers effectively create or refine analogies to meet their needs and are encouraged to teach with analogy by positive student feedback in class and homework.
Building on these findings, we developed a practical system for teachers preparing analogies.
Teachers in the system evaluation recognize its real-world effectiveness in lesson preparation about concept explanation and quiz design, teaching methods, and professional growth, despite potential over-reliance issues.
We hope future tool designers could consider these factors to ensure that LLM-generated analogies have a successful impact on teaching and learning.
}
% In this two-stage study, we conducted in-class tests and classroom experiments guided by pre-class interviews, to evaluate the effectiveness and potential benefits of LLM-generated analogies in various educational scenarios. Our findings from the in-class test demonstrate that LLM-generated analogies can enhance comprehension and test accuracy, particularly in biological contexts. However, there is a risk of students relying on incomplete or misleading information or overestimating their understanding without sufficient teacher guidance. Our classroom experiment reveals that teachers are willing to and play a vital role in selecting, refining, and implementing these analogies to align with their teaching goals and the specific needs of their students. Such analogies help maintain student focus, improve classroom and homework performance, and inspire new teaching strategies. Our work suggests integrating our findings into LLM-assisted education tools, conducting long-term evaluations to study the interplay between analogies and other educational features, and exploring the broader impact of LLM-generated analogies on both students and teachers.

