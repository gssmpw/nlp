
\documentclass[sigconf]{acmart}
% \documentclass[manuscript, screen, review, table, dvipsnames]{acmart}
% \documentclass[manuscript,review,screen,anonymous]{acmart}
%%
%% \BibTeX command to typeset BibTeX logo in the docs
\AtBeginDocument{%
  \providecommand\BibTeX{{%
    Bib\TeX}}}



\copyrightyear{2025}
\acmYear{2025}
\setcopyright{acmlicensed}\acmConference[CHI '25]{CHI Conference on Human Factors in Computing Systems}{April 26-May 1, 2025}{Yokohama, Japan}
\acmBooktitle{CHI Conference on Human Factors in Computing Systems (CHI '25), April 26-May 1, 2025, Yokohama, Japan}
\acmDOI{10.1145/3706598.3714313}
\acmISBN{979-8-4007-1394-1/25/04}

%%
%% Submission ID.
%% Use this when submitting an article to a sponsored event. You'll
%% receive a unique submission ID from the organizers
%% of the event, and this ID should be used as the parameter to this command.
%%\acmSubmissionID{123-A56-BU3}
% \acmSubmissionID{9929}
%%
%% For managing citations, it is recommended to use bibliography
%% files in BibTeX format.
%%
%% You can then either use BibTeX with the ACM-Reference-Format style,
%% or BibLaTeX with the acmnumeric or acmauthoryear sytles, that include
%% support for advanced citation of software artefact from the
%% biblatex-software package, also separately available on CTAN.
%%
%% Look at the sample-*-biblatex.tex files for templates showcasing
%% the biblatex styles.
%%

\usepackage{booktabs}
\usepackage{arydshln}
\usepackage{verbatim}
\usepackage{csquotes}
\usepackage{xspace}
\usepackage{mdwlist}
\usepackage{subfigure}
\usepackage{makecell}
\usepackage{array}
\usepackage{enumitem}
\usepackage{colortbl}
\usepackage{dsfont}
\usepackage{tabularx}
\usepackage{bbm}
\usepackage{tikz}
\usepackage{pgfplots}
\usepackage{soul} % 导入 soul 包
\usepackage{listings}
\usepackage[tableposition=top]{caption}
\usepackage{color,xcolor}
\usepackage{multirow} % For multi-row spanned cells
\usepackage{array}
\usepackage{longtable} % For tables that may span multiple pages
%%
%% The majority of ACM publications use numbered citations and
%% references.  The command \citestyle{authoryear} switches to the
%% "author year" style.
%%
%% If you are preparing content for an event
%% sponsored by ACM SIGGRAPH, you must use the "author year" style of
%% citations and references.
%% Uncommenting
%% the next command will enable that style.
%%\citestyle{acmauthoryear}
\setlist[itemize]{left=0pt}
\setlength\tabcolsep{3.5pt}
\newcommand{\chirev}[1]{{\color{black}{#1}}} %revision for chi25 round 1 revision
\newcommand{\ysy}[1]{{\color{black}{#1}}} %revision for chi25 round 1 revision
\newcommand{\chifinal}[1]{{\color{black}{#1}}} %revision for chi25 round 2 (final) revision
\newcommand{\todo}{$\blacksquare$ TODO\xspace}
\newcommand{\eg}{{e.g.}}
\newcommand{\ie}{{i.e.}}
\newcommand{\vs}{{v.s.}}

%%
%% end of the preamble, start of the body of the document source.
\begin{document}

% \tcbset{
%   mybox/.style={
%     colback=gray!20,       % Background color
%     colframe=black,        % Border color
%     fonttitle=\bfseries,   % Title font
%     boxrule=1pt,           % Border thickness
%     arc=0mm,               % Corner rounding
%     left=0mm, right=0mm,   % Horizontal padding
%     top=1mm, bottom=0mm    % Vertical padding
%   }
% }
%%
%% The "title" command has an optional parameter,
%% allowing the author to define a "short title" to be used in page headers.
\title[Unlocking Scientific Concepts with LLM-generated Analogies]{Unlocking Scientific Concepts: How Effective Are LLM-Generated Analogies for Student Understanding and Classroom Practice?}


%%
%% The "author" command and its associated commands are used to define
%% the authors and their affiliations.
%% Of note is the shared affiliation of the first two authors, and the
%% "authornote" and "authornotemark" commands
%% used to denote shared contribution to the research.
\author{Zekai Shao}
\authornotemark[1]
\email{zkshao23@m.fudan.edu.cn}
\affiliation{%
  \institution{Fudan University}
  \state{Shanghai}
  \country{China}
}

\author{Siyu Yuan}
\authornote{Zekai Shao and Siyu Yuan contributed equally to this research.}
\email{syyuan21@m.fudan.edu.cn}
\affiliation{%
  \institution{Fudan University}
  \state{Shanghai}
  \country{China}
}


\author{Lin Gao}
\email{lingao23@m.fudan.edu.cn}
\affiliation{%
  \institution{Fudan University}
  \state{Shanghai}
  \country{China}
}

\author{Yixuan He}
\email{yixuanhe24@m.fudan.edu.cn}
\affiliation{%
  \institution{Fudan University}
  \state{Shanghai}
  \country{China}
}


\author{Deqing Yang}
\authornotemark[2]
\email{yangdeqing@fudan.edu.cn}
\affiliation{%
  \institution{Fudan University}
  \state{Shanghai}
  \country{China}
}


\author{Siming Chen}
\email{simingchen@fudan.edu.cn}
\authornote{Siming Chen and Deqing Yang are the corresponding authors.}
\affiliation{%
  \institution{Fudan University}
  \state{Shanghai}
  \country{China}
}

%%
%% By default, the full list of authors will be used in the page
%% headers. Often, this list is too long, and will overlap
%% other information printed in the page headers. This command allows
%% the author to define a more concise list
%% of authors' names for this purpose.
\renewcommand{\shortauthors}{Zekai Shao, et al.}

%%
%% The abstract is a short summary of the work to be presented in the
%% article.
\begin{abstract}
%\section{Abstract}

Hypotheses are central to information acquisition, decision-making, and discovery. However, many real-world hypotheses are abstract, high-level statements that are difficult to validate directly. 
This challenge is further intensified by the rise of hypothesis generation from Large Language Models (LLMs), which are prone to hallucination and produce hypotheses in volumes that make manual validation impractical. Here we propose \mname, an agentic framework for rigorous automated validation of free-form hypotheses. 
Guided by Karl Popper's principle of falsification, \mname validates a hypothesis using LLM agents that design and execute falsification experiments targeting its measurable implications. A novel sequential testing framework ensures strict Type-I error control while actively gathering evidence from diverse observations, whether drawn from existing data or newly conducted procedures.
We demonstrate \mname on six domains including biology, economics, and sociology. \mname delivers robust error control, high power, and scalability. Furthermore, compared to human scientists, \mname achieved comparable performance in validating complex biological hypotheses while reducing time by 10 folds, providing a scalable, rigorous solution for hypothesis validation. \mname is freely available at \url{https://github.com/snap-stanford/POPPER}.




\end{abstract}

% \begin{CCSXML}
% <ccs2012>
%    <concept>
%        <concept_id>10003120.10003121.10011748</concept_id>
%        <concept_desc>Human-centered computing~Empirical studies in HCI</concept_desc>
%        <concept_significance>300</concept_significance>
%        </concept>
%    <concept>
%        <concept_id>10003120</concept_id>
%        <concept_desc>Human-centered computing</concept_desc>
%        <concept_significance>500</concept_significance>
%        </concept>
%    <concept>
%        <concept_id>10003120.10003121.10003122.10011750</concept_id>
%        <concept_desc>Human-centered computing~Field studies</concept_desc>
%        <concept_significance>100</concept_significance>
%        </concept>
%  </ccs2012>
% \end{CCSXML}

% \ccsdesc[500]{Human-centered computing}
% \ccsdesc[300]{Human-centered computing~Empirical studies in HCI}
% \ccsdesc[100]{Human-centered computing~Field studies}

\begin{CCSXML}
<ccs2012>
   <concept>
       <concept_id>10003120.10003121.10011748</concept_id>
       <concept_desc>Human-centered computing~Empirical studies in HCI</concept_desc>
       <concept_significance>300</concept_significance>
       </concept>
    <concept>
       <concept_id>10003120.10003121.10003129.10011757</concept_id>
       <concept_desc>Human-centered computing~User interface toolkits</concept_desc>
       <concept_significance>100</concept_significance>
       </concept>
   <concept>
       <concept_id>10003120.10003121.10003122.10011750</concept_id>
       <concept_desc>Human-centered computing~Field studies</concept_desc>
       <concept_significance>100</concept_significance>
       </concept>
 </ccs2012>
\end{CCSXML}

\ccsdesc[500]{Human-centered computing}
\ccsdesc[300]{Human-centered computing~Empirical studies in HCI}
\ccsdesc[100]{Human-centered computing~User interface toolkits}
\ccsdesc[100]{Human-centered computing~Field studies}


\keywords{Analogy Generation, Large Language Models, Scientific Concept Understanding, Classroom Study}
  
\sloppy
\maketitle

\section{Introduction}
\label{sec:intro}
\section{Introduction}
% Large Language Models (LLMs) serve as the foundation for a wide range of tasks. 
% Recently researchers have developed methods to equip large language models (LLMs) with external tools, 

Tool-augmented Large Language Models (LLMs) can use external tools such as calculators~\citep{schick2023toolformer}, 
Python interpretors~\citep{pal}, 
APIs~\citep{tang2023toolalpaca}, or 
AI models~\citep{patil2023gorilla} to complement the parametric knowledge of vanilla LLMs and enable them to solve more complex tasks~\citep{schick2023toolformer,patil2023gorilla}. They are often trained on query-response pairs, which embed the ability to use tools {\em directly} into parameters.\looseness-1
% For example, WebGPT~\citep{webgpt} extends GPT-3~\citep{gpt3} to use search engines, especially useful for events that occurred {\em after} GPT-3 was trained.
% and retrieve up-to-date information to improve GPT-3's performance in question answering, 


Despite the growing adoption of tool-augmented LLMs, the ability to selectively unlearn tools has not been investigated. In real-world applications, tool unlearning is essential for addressing critical concerns such as security, privacy, and model reliability. 
For example, consider a tool-augmented LLM deployed in a healthcare system and trained to use APIs for handling patient data. If one of the APIs is later flagged as insecure due to a vulnerability that could expose sensitive information and violate regulations like HIPAA, tool unlearning is necessary to ensure that the LLM can no longer invoke the insecure API. Similarly, when tools undergo major updates, such as the Python transformers package moving from version 3 to version 4, tool unlearning becomes essential to prevent the LLM from generating outdated or erroneous code.
% For example, if a tool-augmented LLM retains knowledge of making insecure HTTP requests, it will cause significant security risks and can become vulnerable to attacks.\footnote{\url{https://datatracker.ietf.org/doc/html/rfc7807}}
The goal of this work is to address this gap by investigating tool unlearning and providing a solution for this overlooked yet essential task.

% Additional scenarios are discussed in \S~\ref{sec:app}. In the aforementioned case, it is necessary for a tool-augmented model to forget its acquired knowledge of using certain tools--an area that has not yet been explored by existing research.
% Consider the following practical scenarios: 1) \emph{Insecure Tools}, where non-trustworthy tools need to be deleted, 2) \emph{Restricted Tools}, where tools may become unavailable due to copyright issues; 3) \emph{Broken Tools/Dependencies}, where tools may become broken, deprecated, or fall out of maintenance; 4) \emph{Unnecessary Tools}, where the requirement for certain tools may no longer be needed; and 5) \emph{Limited Model Capacity}, where the tool-augmented LLM meets capacity limitations. 


% \paragraph{A new task}
We introduce and formalize the new task of \textbf{Tool Unlearning}, which aims to remove the ability of using specific tools from a tool-augmented LLM while preserving its ability to use other tools and perform general tasks of LLMs such as coherent text generation. 
% This is essential for complying with tool deletion requests that often target a small subset of tool. 
Ideally, an effective tool unlearning model should behave as if it had never learned the tools marked for unlearning. 
% When tool deletion requests are received, a successful tool unlearning algorithm should effectively remove knowledge of the targeted tools, as if the model had never encountered them. At the same time, the model’s knowledge of remaining tools and its ability to perform other tasks should be preserved to the greatest extent possible. This is crucial because deletion requests typically focus on a specific subset of tools, which is usually much smaller than the entire tool set.
% \paragraph{Difference to sample unlearning}
Tool unlearning fundamentally differs from traditional sample-level unlearning as it focuses on removing ``skills'' or the ability to use specific tools, rather than removing individual data samples from a model. In addition, success in tool unlearning should be measured by the model’s ability to forget or retain tool-related skills, which differs from traditional metrics such as measuring likelihood of extracting training data in sample-level unlearning.
% While sample-level unlearning focuses on reducing the likelihood of extracting training data, tool unlearning aims to forget the capability to solve tasks that rely on the tools tagged for unlearning, which can be seen as knowledge-level unlearning. (2) Evaluation: Sample unlearning typically uses perplexity or extraction probability as evaluation metrics. In contrast, tool unlearning prioritizes the success rate of using specific tools, ensuring that the model can no longer effectively use the tools targeted for unlearning. (3) Data: Sample unlearning typically requires access to the exact training data, which may not be available in tool unlearning, especially when dealing with closed-source LLM training data. 
These differences are discussed in detail in~\S\ref{sec:diff}.


% \paragraph{Challenge of tool unlearning}
% as opposed to individual data samples, which makes it fun
% and existing unlearning methods are not fundamentally designed for tool removal; 
% 
% similar to sample-level unlearning, in tool unlearning, 
Removing skills requires  
modifying the parameters of LLMs, a process that is computationally expensive and can lead to unforeseen behaviors~\citep{ripple_effect,gu2024model}. In addition, existing membership inference attack (MIA) techniques, a common evaluation method in machine unlearning to determine whether specific data samples were part of training data, are inadequate for evaluating tool unlearning, as they focus on sample-level data rather than tool-based knowledge. 
% and practically difficult due to potential unforeseeable side effects on other tasks when updating LLM's parameters~\citep{ripple_effect,gu2024model}. 
% Additionally, there is no prior Membership Inference Attack (MIA) models, a desired evaluation of unlearning, designed to detect if a tool is present in training set.  


To address these challenges, we propose \method, the first tool unlearning algorithm for tool-augmented LLMs, which satisfies three key properties for effective tool unlearning: 
{\em tool knowledge removal}, which focuses on removing any knowledge gained on tools marked for unlearning; 
{\em tool knowledge retention}, which focuses on preserving the knowledge gained on other remaining tools; and 
{\em general capability retention}, which maintains LLM's general capability on a range of general tasks such as text and code generation using ideas from task arithmetic~\citep{ilharco2023editing,barbulescu2024textual}.
%
In addition, we develop LiRA-Tool, an adaptation of the Likelihood Ratio Attack (LiRA)~\citep{lira} to tool unlearning, to assess whether tool-related knowledge has been successfully unlearned. Our contributions are: 

% When receiving deletion requests, a successful tool unlearning algorithm should remove the knowledge of the tools marked for unlearning, as if the model has never seen such tools before. Meanwhile, the model's knowledge on the remaining tools as well as other tasks should be preserved to the maximum extent. This is important since the deletion request are targeted at specific subset of tools, usually much smaller than the entire tool set, and practically difficult due to potential unforeseeable side effects.

\vspace{-10pt}
\begin{itemize}
\itemsep-1pt
    \item introducing and conceptualizing tool unlearning for tool-augmented LLMs,
    \item \method, which implements three key properties for effective tool unlearning;
    \item LiRA-Tool, which is the first membership inference attack (MIA) for tool unlearning.
\end{itemize}


Extensive experiments on multiple datasets and tool-augmented LLMs show that \method outperforms existing general and LLM-specific unlearning algorithms by $+$ in accuracy on forget tools and retain tools.  In addition, it can save 74.8\% of training time compared to retraining, handle sequential unlearning requests, and retain 95+\% performance in low resource settings.\looseness-1

\section{Related Work}
\label{sec:related}
\xhdr{Domain-specific Tokenizers} 
Tokenizers tailored for specific domains have been employed to process various types of data, including language~\cite{bpe,sentencepiece,wordpiece,Wang2024challenging,Minixhofer2024zeroshot}, images~\cite{ibot,vqgan,Yu2024difftok,Zha2024textok}, videos~\cite{Choudhury2024dontlook}, graphs~\cite{Perozzi2024graphtalk,vqgraph}, and molecular and material sciences~\cite{Fu2024moltok,Tahmid2024birna,Qiao2024mxdna}. While these tokenizers perform well within their respective domains, they are not directly applicable to medical codes, which contains specialized medical semantics. Medical codes reside in relation contexts and are accompanied by textual descriptions. Directly using the tokenizers for languages risks flattening the relationships among codes and failing to preserve the biomedical information. This will lead to fragmented tokenization of medical codes, resulting in loss of contextual information during encoding.
Meanwhile, visual tokenizer typically focus on local pixel-level relationships, which are insufficient for capturing the complex semantics inherent in medical codes. Graph tokenizers are designed to encode structured information from graphs into a discrete token, then enabling LLMs to process relational and topological knowledge effectively. However, graph tokenizers may suffer from information loss when applied to graphs in other domains, making them less flexible and efficient for large, dynamic, and cross-domain graphs. In contrast, our \model tokenizer explicitly incorporates the relevant medical semantics by integrating textual descriptions with graph-based relational contexts.


\xhdr{Vector-Quantized Tokenizers}
Tokenization strategies often vary according to the problem domain and data modality where recent work has highlighted the benefits of discrete tokenization~\cite{du2024role}. This process involves partitioning the input according to a finite set of tokens, often held in a \textit{codebook} (this concept is independent of medical coding despite the similar name), and the quantization process involves learning a mapping from input data to the optimal set of tokens according to a pre-defined objective such as reconstruction loss~\cite{van2017neural}. 

Recent work has highlighted the ability of vector quantized (VQ-based) tokenization to effectively compress semantic information\cite{gu2024rethinking}. This approach is particularly successful for tokenizing inputs with an inherent semantic structure such as graphs~\cite{yang2023vqgraph, wang2024learning}, speech~\cite{zeghidour2021soundstream, baevski2019vq}, and time~\cite{yu2021vector} as well as complex tasks like recommendation retrieval \cite{wang2024learnable, rajput2023recommender, sun2024learning} and image synthesis \cite{zhang2023regularized, yu2021vector}.

Another significant advantage to VQ-based tokenization is the natural integration of multiple modalities. By learning a shared latent space across modalities, each modality can jointly modeled using a common token vocabulary \cite{agarwal2025cosmos, yu2023language}. 
% I think there are probably better citations to use for the line above
TokenFlow leverages a dual-codebook design that allows for correlations across modalities through a dual encoder~\cite{qu2024tokenflow}.


\xhdr{Structured EHR, transformer-based, and foundation models} 
%
Structured EHR models leverage patient records to learn representations for clinical prediction and operational healthcare tasks. These models differ from medical LLMs~\cite{singhal2025toward,tu2024towards,singhal2023large}, which are typically trained on free-text clinical notes~\cite{jiang2023health} and biomedical literature rather than structured EHR data.  
%
BEHRT~\cite{li2020behrt} applies deep bidirectional learning to predict future medical events, encoding disease codes, age, and visit sequences using self-attention. TransformEHR~\cite{transform_ehr} adopts an encoder-decoder transformer with visit-level masking to pretrain on EHRs, enabling multi-task prediction. GT-BEHRT~\cite{gtbehrt} models intra-visit dependencies as a graph, using a graph transformer to learn visit representations before processing patient-level sequences with a transformer encoder.  
%
Other models enhance EHR representations with external knowledge. GraphCare~\cite{graphcare} integrates large language models and biomedical knowledge graphs to construct patient-specific graphs processed via a Bi-attention Augmented Graph Neural Network. Mult-EHR~\cite{mult_ehr} introduces multi-task heterogeneous graph learning with causal denoising to address data heterogeneity and confounding effects. ETHOS~\cite{ethos} tokenizes patient health timelines for transformer-based pretraining, achieving zero-shot performance.  
%
While these models focus on learning patient representations, \model serves a different role as a medical code tokenizer. It can be integrated into any structured EHR, transformer-based, or other foundation model, improving how medical codes are tokenized before being processed. Unlike these models, which rely on predefined tokenization schemes, \model optimizes the tokenization process itself.


\section{Method}
\label{sec:method}





\subsection{Problem Formulation}
\label{sub:problem_formulation}


\textbf{Sample-Level Classification.}  
Consider an input EEG sample $\bm{x} \in \mathbb{R}^{T \times C}$\, where $T$ denotes the number of timestamps and $C$ represents the number of channels. Our objective is to learn an encoder that generates a representation $\bm{h}$\, which can be used to predict the corresponding label $\bm{y} \in \mathbb{R}$\ for the input sample. Specifically, the label $\bm{y}$\ corresponds to either Alzheimer's Disease or Healthy controls.

\textbf{Subject-Level Classification.}
In addition to the corresponding label $\bm{y} \in \mathbb{R}$\, each input EEG sample also has a subject ID $\bm{s} \in \mathbb{R}$\ that indicates which subject the sample belongs to. The ultimate goal of EEG-based AD detection is to determine whether a subject has Alzheimer's Disease. For subject-level classification, we use a majority voting scheme, where the subject is assigned the label corresponding to the majority label of all samples from that subject.



\subsection{Datasets Selection}
\label{sub:datasets_selection}

\textbf{AD Datasets.} We review EEG-based AD detection papers published between 2018 and 2024 to identify potentially available datasets. We find 6 publicly available datasets containing AD subjects: \textbf{AD-Auditory}\cite{lahijanian2024auditory}, \textbf{ADFSU}\cite{vicchietti2023computational}, \textbf{ADFTD}\cite{miltiadous2023dataset,miltiadous2023dice}, \textbf{ADSZ}\cite{alves2022eeg,pineda2020quantile}, \textbf{APAVA}\cite{escudero2006analysis,smith2017accounting}, and \textbf{BrainLat}\cite{prado2023brainlat}. Additionally, we use 3 private datasets: \textbf{Cognision-ERP}\cite{cecchi2015clinical}, \textbf{Cognision-rsEEG}, and \textbf{CNBPM}\cite{ieracitano2019time,amezquita2019novel}, bringing the total number of AD datasets to 9, and the total number of subjects to 813.  We perform preliminary experiments on each dataset individually to assess their quality. For smaller datasets or those showing large performance variability across subjects, we use them for pre-training to alleviate potential data quality issues such as mislabeled subjects, interference from artifacts, collection devices, and collection methods. Five high-quality AD datasets, \textbf{ADFTD}, \textbf{CNBPM}, \textbf{Cognision-rsEEG}, \textbf{Cognision-ERP}, and \textbf{BrainLat}, are used for downstream tasks to evaluate the model performance.

\textbf{Non-AD Datasets.} To enhance the learning of general EEG and AD-specific features, we use datasets of healthy subjects and other neurological diseases for self-supervised pretraining. We aim to increase the diversity of brain conditions, including healthy and diseased states, and increase the number of subjects used for training to reduce the interference of subject-specific patterns. Note that all the non-AD datasets have one commonality: the label is assigned to the subject, which adapts to the subject-level feature extraction. Datasets such as sleep stage detection and mental state classification are unsuitable here. We select publicly available datasets from sources like OpenNEURO\footnote{\url{https://openneuro.org/}}, Temple University Hospital\footnote{\url{https://isip.piconepress.com/projects/tuh_eeg/}}, and Brainclinics\footnote{\url{https://www.brainclinics.com/resources}}. We choose datasets collected in a resting-state condition or involving resting-state tasks with either eyes open or closed to ensure consistency with most downstream AD datasets. In total, we select 7 proper large datasets, each with hundreds or even thousands of subjects. They are \textbf{Depression}~\cite{cavanagh2019multiple,cavanagh2021eeg}, \textbf{PEARL-Neuro}~\cite{dzianok2024pearl}, \textbf{REEG-BACA}~\cite{getzmann2024resting}, \textbf{REEG-PD}~\cite{singh2023evoked}, \textbf{REEG-SRM}~\cite{hatlestad2022bids}, \textbf{TDBrain}~\cite{van2022two}, and \textbf{TUEP}~\cite{veloso2017big}.




\subsection{Data Preprocessing}
\label{sub:data_preprocessing}

Two key challenges in training a large foundation model for time-series-like data are varying channel/variate numbers and heterogeneous sampling frequencies~\cite{liu2024timer,woo2024unified,yang2024biot}. However, we can easily align channels based on their names in EEG and align sampling frequency by resampling. More details and reasons for preprocessing steps are provided in Appendix~\ref{sec:datasets_preprocessing}.

\textbf{Artifacts Removal.} Some datasets have already undergone preprocessing steps during data collection, such as artifact removal and filtering. We perform a secondary preprocessing to align all datasets uniformly for training. All the fine-tuning datasets are guaranteed to be artifacts-free.

\textbf{Channel Alignment.} We align all datasets to a standard set of 19 channels, which include Fp1, Fp2, F7, F3, Fz, F4, F8, T3/T7, C3, Cz, C4, T4/T8, T5/P7, P3, Pz, P4, T6/P8, O1, and O2, based on the international 10-20 system\footnote{\url{https://en.wikipedia.org/wiki/10-20_system_(EEG)}}. For datasets with fewer than 19 channels, we interpolate the missing channels using the MNE EEG processing package\footnote{\url{https://mne.tools/stable/index.html}}. For datasets with more than 19 channels, we select the 19 channels based on the channel name and discard the others. In cases where datasets use different channel montages, such as the Biosemi headcaps with 32, 64, 128 channels\footnote{\url{https://www.biosemi.com/headcap.htm}}, we select the 19 closest channels by calculating the Euclidean distance between their 3D coordinates. The channel alignment allows us to pre-train the models on different datasets with any backbone encoder and perform unified fine-tuning on all AD datasets in one run.
 
\textbf{Frequency Alignment.} In addition to channel alignment, we resample all datasets to a uniform sampling frequency of 128Hz, which is commonly used and preserves the key frequency bands ($\delta$, $\theta$, $\alpha$, $\beta$, $\gamma$), while also reducing noise.

\textbf{Sample Segmentation.} For deep learning training, we segment the EEG trials within each subject into 1-second samples, which results in 128 timestamps per sample, as the sampling frequency is aligned to 128Hz. 

\textbf{Frequency Filtering.} We then apply frequency filtering to each sample, ranging from 0.5Hz to 45Hz, to remove frequency bands that do not correspond to brain activities.

\textbf{Standard Normalization.} After frequency filtering, we perform standard normalization on each sample, applied individually to each channel, to ensure that the data is centered and scaled consistently across all samples and channels.




\begin{figure*}
    \centering
    \includegraphics[width=1.0\linewidth]{figures/methods/lead_details.pdf}
    \caption{\textbf{Details of LEAD method.} \textbf{a)} The pipeline for our method includes data preprocessing, sample indices shuffling, self-supervised pre-training, multi-dataset fine-tuning, sample-level classification, and subject-level AD detection. \textbf{b)} The flowchart of the self-supervised pre-training. A batch of samples is applied with two augmentations $a$ and $b$ to generate two augmented views, $\bm{x}_i^a$ and $\bm{x}_i^b$. The number in each sample is the subject ID. The representation $\bm{z}_i^a$ and $\bm{z}_i^b$ after encoder $f(\cdot)$ and projection head $g(\cdot)$ are used for contrastive learning. Two augmented views from the same sample are positive pairs for sample-level contrast. For subject-level contrast, samples with the same subject IDs are positive pairs. \textbf{c)} The backbone encoder $f(\cdot)$ includes two branches. The temporal branch takes cross-channel patches to embed as tokens. The spatial branch takes the whole series of channels to embed as tokens. The two branches are computed in parallel.
    }
    \label{fig:lead_details}
    \vspace{-5mm}
\end{figure*}



\subsection{Self-Supervised Pretraining}
\label{sub:self_supervised_pretraining}

The region b) in figure~\ref{fig:lead_details} shows the flowchart of self-supervised contrastive pre-training.



\textbf{Representation Learning.} For an input EEG sample $\bm{x}_i$, where $i$ denotes the index of the sample $\bm{x}_i$, we apply data augmentation methods $a$ and $b$ to generate two augmented views, $\bm{x}_i^a$ and $\bm{x}_i^b$. Given a backbone encoder $f(\cdot)$ and a projection head $g(\cdot)$, we compute their representations $\bm{h}_i^a = f(\bm{x}_i^a)$ and $\bm{h}_i^b = f(\bm{x}_i^b)$ after the encoder $f(\cdot)$, and further obtain denser representations $\bm{z}_i^a = g(\bm{h}_i^a)$ and $\bm{z}_i^b = g(\bm{h}_i^b)$ through the projection head $g(\cdot)$. The projection head is designed to benefit contrastive learning, as described in~\cite{chen2020simple}, and it will be discarded during downstream tasks, with only the encoder being used for the downstream task.




\textbf{Sample-Level Contrasting.} Sample-level contrasting is the most widely used framework in contrastive learning, as seen in SimCLR~\cite{chen2020simple} and MOCO~\cite{he2020momentum}. The goal is to perform sample/instance discrimination and learn a representation that can distinguish one sample from others~\cite{wu2018unsupervised}. A pre-trained model using this approach can capture general patterns in EEG data, benefiting downstream tasks by improving performance and reducing the need for labeled data. In this work, we adopt the SimCLR architecture, which treats different augmented views of the same sample as positive pairs and views from different samples as negative pairs. For an input sample $\bm{x}_i \in \mathcal{B}$ in a batch, our sample-level InfoNCE contrastive loss is defined as follows:

\begin{equation}
\label{eq:sample_loss}
\mathcal{L}_{Sam} = 
\mathbb{E}_{\bm{x}_i}
    \left[
    -\textrm{log}
        \frac
            {\textrm{exp}( \textrm{sim}( \bm{z}_i^a, \bm{z}_i^b ) / \tau)}
            {
            \sum_{j}
                \left(
                \textrm{exp}( \textrm{sim}( \bm{z}_i^a, \bm{z}_j^b ) / \tau)
                \right)
            }
    \right]
\end{equation}

where $j$ denotes the index of other samples in the batch $\mathcal{B}$, and $\textrm{sim} (\bm{u}, \bm{v}) = \frac{\bm{u}^T \bm{v}}{\|\bm{u}\| \|\bm{v}\|}$ denotes the cosine similarity between vectors $\bm{u}$ and $\bm{v}$. The parameter $\tau$ is a temperature parameter that adjusts the similarity scale. 




\textbf{Subject-Level Contrasting.} In EEG-based Alzheimer’s disease (AD) detection, each subject is typically associated with a stable medical state. Specifically, once a subject has AD or preclinical signs of AD, all EEG samples from that subject should exhibit features related to AD, meaning they share the same label during deep learning training. This prior knowledge allows us to perform subject-level contrasting, a concept first defined in~\cite{wang2024contrast} and successfully applied in EEG and ECG-based disease detection~\cite{kiyasseh2021clocs, wang2024contrast, abbaspourazad2023large}. In subject-level contrasting, we treat samples from the same subject as positive pairs and samples from different subjects as negative pairs. With an increasing number of subjects used in pre-training, we aim for the model to learn diverse feature types and reduce interference from unrelated subject-specific features during downstream classification. Appendix~\ref{sub:contrastive_learning_modules} and~\ref{sub:subject_contrast_effectiveness} provide more details on the effectiveness and analysis of subject-level contrasting. For an input sample $\bm{x}_i \in \mathcal{B}$ in a batch, our subject-level InfoNCE contrastive loss is defined as follows:

\begin{equation}
\label{eq:subject_loss}
\mathcal{L}_{Sub} = 
\mathbb{E}_{\bm{x}_i}
    \left[
    \mathbb{E}_{\bm{x}_k}
    \left[
    -\textrm{log}
        \frac
            {\textrm{exp}( \textrm{sim}( \bm{z}_i^a, \bm{z}_k^b ) / \tau)}
            {
            \sum_{j}
                \left(
                \textrm{exp}( \textrm{sim}( \bm{z}_i^a, \bm{z}_j^b ) / \tau)
                \right)
            }
    \right]
    \right]
\end{equation}

where $\bm{x}_k$ denotes samples from the same subject as $\bm{x}_i$ in the batch, with the same subject ID $\bm{s}_k = \bm{s}_i$. The function $\textrm{sim} (\bm{u}, \bm{v}) = \frac{\bm{u}^T \bm{v}}{\|\bm{u}\| \|\bm{v}\|}$ represents the cosine similarity, and $\tau$ is a temperature parameter that adjusts the scale. Note that not all neurological diseases can utilize subject-level contrasting. For instance, seizures are a condition where the EEG patterns during a seizure phase differ significantly from those in the regular phase for the same subject.



\textbf{Overall Loss Function.}
The overall loss function is the weighted sum of the sample-level and subject-level contrastive losses is defined as follows: 

\begin{equation}
\label{eq:joint_loss}
\mathcal{L} = 
    {\lambda_1} \mathcal{L}_{Sam} + 
    {\lambda_2} \mathcal{L}_{Sub}  
\end{equation}
where ${\lambda_1} + {\lambda_2} = 1$ are hyper-coefficients that control the relative importance and adjust the scales of each level's loss.


\textbf{Indices Shuffling.} In real-world scenarios, the likelihood of samples with the same subject ID appearing in the same training batch decreases as the number of subjects increases. This can hinder subject-level contrastive learning. To address this issue, we develop an indices shuffling algorithm that shuffles the order of samples in each epoch. The goal is to ensure that samples with the same subject ID are present in the batch while introducing randomness in the sample order every epoch. More algorithm description and Pseudo code details are presented in Appendix~\ref{sec:indices_shuffling_algorithm}.






\subsection{Backbone Encoder Architecture}
\label{sub:backbone_encoder}
We use a simplified version of ADformer~\cite{wang2024adformer} as the backbone encoder $ f(\cdot) $, adopting single-granularity learning only. This architecture is designed for EEG-based AD detection and efficiently captures temporal features along the time dimension and spatial features among channels, as both are critical for EEG feature representation learning. For simplicity, we omit the subscript $ i $ for the input sample $ \bm{x} $ in this subsection, as it is not necessary for the illustration. The temporal and spatial branches are computed in parallel before the projection head or classifier. Both branches use the standard encoder-only transformer, including self-attention, layer normalization, and feed-forward networks. The region c) in figure~\ref{fig:lead_details} illustrates the architecture of the backbone encoder.

\textbf{Temporal Branch.}
Given an input EEG sample $\bm{x} \in \mathbb{R}^{T \times C}$ and patch length $L$, where $T$ and $C$ denote the number of timestamps and channels, respectively. We first segment the input sample into $N$ cross-channel non-overlapping patches to obtain $\bm{x}^{\text{t}} \in \mathbb{R}^{N \times (L \cdot C)}$. Zero padding is applied to ensure that the number of timestamps $T$ is divisible by $L$, resulting in $N = \left\lceil \frac{T}{L} \right\rceil$. The patches $\bm{x}^{\text{t}}$ are then mapped into $D$-dimensional patch embeddings using a linear projection $\bm{W}$, and a fixed positional embedding $\bm{W}_{\text{pos}}$~\cite{vaswani2017attention} is added to produce the final patch embeddings: $\bm{e}^{\text{t}} = \bm{x}^{\text{t}} \bm{W} + \bm{W}_{\text{pos}}$, where $\bm{e}^{\text{t}} \in \mathbb{R}^{N \times D}$, $\bm{W} \in \mathbb{R}^{(L \cdot C) \times D}$, and $\bm{W}_{\text{pos}} \in \mathbb{R}^{N \times D}$. The final patch embeddings $\bm{e}^{\text{t}}$ are used as input tokens for the standard encoder-only transformer. After $M$ encoding layers, we obtain the temporal branch's final representations $\bm{h}^{\text{t}}$.



\textbf{Spatial Branch.}
Given an input EEG sample $\bm{x} \in \mathbb{R}^{T \times C}$, we first transpose the sample and add a fixed channel-wise positional embedding $\bm{W}_{\text{pos}}$ to obtain $\bm{x}^{\text{c}} = \operatorname{Transpose}(\bm{x}) + \bm{W}_{\text{pos}}$, where $\bm{x}^{\text{c}}, \bm{W}_{\text{pos}} \in \mathbb{R}^{C \times T}$. Unlike the temporal branch, where positional embeddings are added after embedding, we add channel-wise positional embeddings on the raw input EEG data since the subsequent up-dimension process destroys the information of raw channel order. For a target channel number $F$ and embedding dimension $D$, we first perform an up-dimensional transformation using a 1-D convolution $\bm{W}_1$ to increase the channel number. Then, we map the entire series of each channel into a latent embedding using a linear projection $\bm{W}_2$ to get the final channel embeddings: $\bm{e}^{\text{c}} = (\bm{W}_1 \bm{x}^{\text{c}}) \bm{W}_2$, where $\bm{e}^{\text{c}} \in \mathbb{R}^{F \times D}$, $\bm{W}_1 \in \mathbb{R}^{F \times C}$, and $\bm{W}_2 \in \mathbb{R}^{T \times D}$. The final channel embeddings $\bm{e}^{\text{c}}$ are used as input tokens for the standard encoder-only transformer. After $M$ encoding layers, we obtain the spatial branch's final representation $\bm{h}^{\text{c}}$. 




\textbf{Projection Head and Classifier.}
For an input EEG sample $\bm{x}$, we obtain the temporal branch's representation $\bm{h}^t$ and the spatial branch's representation $\bm{h}^c$. We concatenate the \textbf{last token} from both representations to form the final representation $\bm{h} = \left[\bm{h}^t[-1] \,||\, \bm{h}^c[-1]\right]$ of the backbone encoder $f(\cdot)$, where $\left[\cdot \,||\, \cdot\right]$ denotes concatenation and $\bm{h} \in \mathbb{R}^{2D}$. For contrastive pre-training, $\bm{h}$ is further projected into a denser representation $\bm{z} \in \mathbb{R}^{D}$ using a projection head $g(\cdot)$, where $g(\cdot)$ consists of a two-layer fully connected network. For downstream classification tasks, $\bm{h}$ is used directly to classify the output label $\bm{y}$ via a linear classifier $c(\cdot)$.






\begin{table*}[t]
    \centering
    \caption{\textbf{Processed Dataset Statistics.} For datasets where the channels are not 19 standard channels in the international 10-20 system, we process them into two versions: one with channel alignment to the 19 channels and one without channel alignment. We present the \#-AD and \#-HC subjects in the supervised or unified fine-tuning task, \eg, 36 AD + 29 HC = 65 subjects.}
    \vspace{2mm}
    \label{tab:processed_data}
    \resizebox{\textwidth}{!}{%
    \begin{tabular}{@{}ll|cccccccc@{}}
    \toprule
    \multicolumn{2}{l|}{\textbf{Datasets}} & \textbf{Confidentiality} & \textbf{Type(subtype)} & \textbf{\#-Subjects} & \textbf{\#-Rate} & \textbf{\#-Channels} & \textbf{\#-Duration} & \textbf{\#-Samples} & \textbf{Tasks}  \\ 
    \midrule
    \multicolumn{2}{l|}{\textbf{AD-Auditory}} & Public &  AD(ERP)  & 35 &  128Hz  & 19 & 1 second &  37,425  &  Self-supervised pre-training \\
    \multicolumn{2}{l|}{\textbf{ADFSU}} & Public &  AD(Resting)  & 92 &  128Hz  & 19 & 1 second &  2,760  &  Self-supervised pre-training \\
    \multicolumn{2}{l|}{\textbf{ADSZ}} & Public &  AD(Resting)  & 48 &  128Hz  & 19 & 1 second &  768   &  Self-supervised pre-training \\
    \multicolumn{2}{l|}{\textbf{APAVA}} & Public &  AD(Resting)  & 23 &  128Hz  & 16 or 19 & 1 second &  5,967  &  Self-supervised pre-training \\
    \multicolumn{2}{l|}{\textbf{ADFTD}} & Public &  AD(Resting)  & 36+29=65 &  128Hz  & 19 & 1 second &  53,215  &  Supervised or Unified Fine-tuning \\
    \multicolumn{2}{l|}{\textbf{BrainLat}} & Public &  AD(Resting)  & 35+32=67 &  128Hz  & 19 & 1 second &  29,788  &  Supervised or Unified Fine-tuning \\
    \multicolumn{2}{l|}{\textbf{CNBPM}} & Private &  AD(Resting)  & 63+63=126 &  128Hz  & 19 & 1 second &  46,336  &  Supervised or Unified Fine-tuning \\
    \multicolumn{2}{l|}{\textbf{Cognision-rsEEG}} & Private &  AD(Resting)  & 97+83=180 &  128Hz  & 7 or 19 & 1 second &  32,400  &  Supervised or Unified Fine-tuning \\
    \multicolumn{2}{l|}{\textbf{Cognision-ERP}} & Private &  AD(ERP)  & 90+87=177 &  128Hz  & 7 or 19 & 1 second &  61,300  &  Supervised or Unified Fine-tuning \\
 

    \midrule

    \multicolumn{2}{l|}{\textbf{Depression}} & Public &  Non-AD(Resting)  & 122 &  128Hz  & 66 or 19 & 1 second &  24,014  &  Self-supervised pre-training \\
    \multicolumn{2}{l|}{\textbf{PEARL-Neuro}} & Public &  Non-AD(Resting)  & 79 &  128Hz  & 127 or 19 & 1 second &  51,670  &  Self-supervised pre-training \\
    \multicolumn{2}{l|}{\textbf{REEG-BACA}} & Public &  Non-AD(Resting)  & 608 &  128Hz  & 65 or 19 & 1 second &  611,269  &  Self-supervised pre-training \\
    \multicolumn{2}{l|}{\textbf{REEG-PD}} & Public &  Non-AD(Resting)  & 149 &  128Hz  & 60 or 19 & 1 second &  23,839  &  Self-supervised pre-training \\
    \multicolumn{2}{l|}{\textbf{REEG-SRM}} & Public &  Non-AD(Resting)  & 109 &  128Hz  & 64 or 19 & 1 second &  32,760  &  Self-supervised pre-training \\
    \multicolumn{2}{l|}{\textbf{TDBrain}} & Public &  Non-AD(Resting)  & 911 &  128Hz  & 33 or 19 & 1 second &  231,689  &  Self-supervised pre-training \\
    \multicolumn{2}{l|}{\textbf{TUEP}} & Public &  Non-AD(Resting)  & 179 &  128Hz  & 19 & 1 second &  143,200  &  Self-supervised pre-training \\
    
    \bottomrule
    \end{tabular}
    } % end of resizebox
\vspace{-5mm}
\end{table*}



\subsection{Important Setups}
\label{sub:important_setups}

\textbf{Subject-Independent.}  
Two main setups are commonly used for evaluation in the EEG-based AD detection domain: subject-dependent~\cite{nour2024novel,kumar2023eegalzheimer} and subject-independent~\cite{watanabe2024deep,chen2024multi}. In the subject-dependent setup, all samples are mixed together and split into training, validation, and test sets, allowing samples from the same subject to appear in all three sets. In contrast, the subject-independent setup splits the training, validation, and test sets based on subjects, ensuring that samples from the same subject are exclusively assigned to one set~\cite{wang2024medformer}. Unlike many existing works that use the subject-dependent setup, we use the subject-independent setup. The subject-dependent setup is unsuitable for real-world scenarios and leads to significant data leakage~\cite{wang2024evaluate}.


\textbf{Unified Fine-tuning.} 
The channel alignment in our data preprocessing step enables us to pre-train the model on various datasets and then fine-tune it on all downstream datasets simultaneously. We refer to this as "unified fine-tuning," where the model is fine-tuned across all downstream AD datasets in one run. The best model is then selected based on the weighted performance across the downstream datasets, ensuring that the model performs optimally on all tasks.


\textbf{Majority Voting.}  
For subject-level EEG-based AD detection, we apply a majority voting scheme to determine the final classification label for each subject. Specifically, for all the samples from one subject (with the same subject ID $\bm{s}$), we find the majority label of these samples and assign this label to this subject. For example, if a subject has 100 samples and more than 50 are classified as AD, the subject will be labeled "AD." The voting mechanism alleviates the interference of outlier samples in a subject.


\section{Study I}
\label{sec:study1}
In this section, we detail the participants, data preparation process, stimuli, procedure, and results analysis of Study I.

% We conducted our first study (Study I) to evaluate the effectiveness of LLM-generated analogies in helping students grasp scientific concepts.
% We iteratively refined the prompting strategy to produce high-quality analogies of the scientific concepts that the students were expected to learn.
% The students were then divided into two groups: one with textbook explanations, and another with both LLM-generated analogies and textbook explanations.
% They then completed an exam about scientific concepts to be learned. 
% We evaluated the effectiveness of the analogies by comparing test accuracy and subjective ratings across the groups.

\subsection{Participants} 
\label{sec:study1_participants}
Two classes of freshmen, totaling 49 \chifinal{Chinese} students from a Chinese high school with which we have a scientific research collaboration, participated in Study I. 
Their ages ranged from 15 to 17, with 26 males and 23 females. 
They had recently started high school physics and biology courses. 
Their entrance exam scores and classroom performance suggested a normal cognitive level, and we did not pre-select students based on their abilities.





\subsection{Data Preparation}
\label{sec:study1_data_preparation}

\begin{figure*}[tb]
    \centering
    \includegraphics[width=0.9\linewidth]{figure/revpipeline.pdf}
    \caption{The pipeline for analogy generation to explain scientific concepts for data preparation in Study I.}
    \Description{This figure presents our pipeline for generating and refining analogies to explain scientific concepts, as used in Study I for data preparation. The process begins with collecting scientific concepts and generating initial analogies using LLMs. Errors are then annotated, and the analogies are refined with revised prompts. Next, the most effective analogy is selected automatically, and patterns are annotated for the final selection.}
    \label{fig:pipeline}
\end{figure*}

\definecolor{titlecolor}{rgb}{0.9, 0.5, 0.1}
\definecolor{anscolor}{rgb}{0.2, 0.5, 0.8}
\definecolor{labelcolor}{HTML}{48a07e}
\begin{table*}[h]
	\centering
	
 % \vspace{-0.2cm}
	
	\begin{center}
		\begin{tikzpicture}[
				chatbox_inner/.style={rectangle, rounded corners, opacity=0, text opacity=1, font=\sffamily\scriptsize, text width=5in, text height=9pt, inner xsep=6pt, inner ysep=6pt},
				chatbox_prompt_inner/.style={chatbox_inner, align=flush left, xshift=0pt, text height=11pt},
				chatbox_user_inner/.style={chatbox_inner, align=flush left, xshift=0pt},
				chatbox_gpt_inner/.style={chatbox_inner, align=flush left, xshift=0pt},
				chatbox/.style={chatbox_inner, draw=black!25, fill=gray!7, opacity=1, text opacity=0},
				chatbox_prompt/.style={chatbox, align=flush left, fill=gray!1.5, draw=black!30, text height=10pt},
				chatbox_user/.style={chatbox, align=flush left},
				chatbox_gpt/.style={chatbox, align=flush left},
				chatbox2/.style={chatbox_gpt, fill=green!25},
				chatbox3/.style={chatbox_gpt, fill=red!20, draw=black!20},
				chatbox4/.style={chatbox_gpt, fill=yellow!30},
				labelbox/.style={rectangle, rounded corners, draw=black!50, font=\sffamily\scriptsize\bfseries, fill=gray!5, inner sep=3pt},
			]
											
			\node[chatbox_user] (q1) {
				\textbf{System prompt}
				\newline
				\newline
				You are a helpful and precise assistant for segmenting and labeling sentences. We would like to request your help on curating a dataset for entity-level hallucination detection.
				\newline \newline
                We will give you a machine generated biography and a list of checked facts about the biography. Each fact consists of a sentence and a label (True/False). Please do the following process. First, breaking down the biography into words. Second, by referring to the provided list of facts, merging some broken down words in the previous step to form meaningful entities. For example, ``strategic thinking'' should be one entity instead of two. Third, according to the labels in the list of facts, labeling each entity as True or False. Specifically, for facts that share a similar sentence structure (\eg, \textit{``He was born on Mach 9, 1941.''} (\texttt{True}) and \textit{``He was born in Ramos Mejia.''} (\texttt{False})), please first assign labels to entities that differ across atomic facts. For example, first labeling ``Mach 9, 1941'' (\texttt{True}) and ``Ramos Mejia'' (\texttt{False}) in the above case. For those entities that are the same across atomic facts (\eg, ``was born'') or are neutral (\eg, ``he,'' ``in,'' and ``on''), please label them as \texttt{True}. For the cases that there is no atomic fact that shares the same sentence structure, please identify the most informative entities in the sentence and label them with the same label as the atomic fact while treating the rest of the entities as \texttt{True}. In the end, output the entities and labels in the following format:
                \begin{itemize}[nosep]
                    \item Entity 1 (Label 1)
                    \item Entity 2 (Label 2)
                    \item ...
                    \item Entity N (Label N)
                \end{itemize}
                % \newline \newline
                Here are two examples:
                \newline\newline
                \textbf{[Example 1]}
                \newline
                [The start of the biography]
                \newline
                \textcolor{titlecolor}{Marianne McAndrew is an American actress and singer, born on November 21, 1942, in Cleveland, Ohio. She began her acting career in the late 1960s, appearing in various television shows and films.}
                \newline
                [The end of the biography]
                \newline \newline
                [The start of the list of checked facts]
                \newline
                \textcolor{anscolor}{[Marianne McAndrew is an American. (False); Marianne McAndrew is an actress. (True); Marianne McAndrew is a singer. (False); Marianne McAndrew was born on November 21, 1942. (False); Marianne McAndrew was born in Cleveland, Ohio. (False); She began her acting career in the late 1960s. (True); She has appeared in various television shows. (True); She has appeared in various films. (True)]}
                \newline
                [The end of the list of checked facts]
                \newline \newline
                [The start of the ideal output]
                \newline
                \textcolor{labelcolor}{[Marianne McAndrew (True); is (True); an (True); American (False); actress (True); and (True); singer (False); , (True); born (True); on (True); November 21, 1942 (False); , (True); in (True); Cleveland, Ohio (False); . (True); She (True); began (True); her (True); acting career (True); in (True); the late 1960s (True); , (True); appearing (True); in (True); various (True); television shows (True); and (True); films (True); . (True)]}
                \newline
                [The end of the ideal output]
				\newline \newline
                \textbf{[Example 2]}
                \newline
                [The start of the biography]
                \newline
                \textcolor{titlecolor}{Doug Sheehan is an American actor who was born on April 27, 1949, in Santa Monica, California. He is best known for his roles in soap operas, including his portrayal of Joe Kelly on ``General Hospital'' and Ben Gibson on ``Knots Landing.''}
                \newline
                [The end of the biography]
                \newline \newline
                [The start of the list of checked facts]
                \newline
                \textcolor{anscolor}{[Doug Sheehan is an American. (True); Doug Sheehan is an actor. (True); Doug Sheehan was born on April 27, 1949. (True); Doug Sheehan was born in Santa Monica, California. (False); He is best known for his roles in soap operas. (True); He portrayed Joe Kelly. (True); Joe Kelly was in General Hospital. (True); General Hospital is a soap opera. (True); He portrayed Ben Gibson. (True); Ben Gibson was in Knots Landing. (True); Knots Landing is a soap opera. (True)]}
                \newline
                [The end of the list of checked facts]
                \newline \newline
                [The start of the ideal output]
                \newline
                \textcolor{labelcolor}{[Doug Sheehan (True); is (True); an (True); American (True); actor (True); who (True); was born (True); on (True); April 27, 1949 (True); in (True); Santa Monica, California (False); . (True); He (True); is (True); best known (True); for (True); his roles in soap operas (True); , (True); including (True); in (True); his portrayal (True); of (True); Joe Kelly (True); on (True); ``General Hospital'' (True); and (True); Ben Gibson (True); on (True); ``Knots Landing.'' (True)]}
                \newline
                [The end of the ideal output]
				\newline \newline
				\textbf{User prompt}
				\newline
				\newline
				[The start of the biography]
				\newline
				\textcolor{magenta}{\texttt{\{BIOGRAPHY\}}}
				\newline
				[The ebd of the biography]
				\newline \newline
				[The start of the list of checked facts]
				\newline
				\textcolor{magenta}{\texttt{\{LIST OF CHECKED FACTS\}}}
				\newline
				[The end of the list of checked facts]
			};
			\node[chatbox_user_inner] (q1_text) at (q1) {
				\textbf{System prompt}
				\newline
				\newline
				You are a helpful and precise assistant for segmenting and labeling sentences. We would like to request your help on curating a dataset for entity-level hallucination detection.
				\newline \newline
                We will give you a machine generated biography and a list of checked facts about the biography. Each fact consists of a sentence and a label (True/False). Please do the following process. First, breaking down the biography into words. Second, by referring to the provided list of facts, merging some broken down words in the previous step to form meaningful entities. For example, ``strategic thinking'' should be one entity instead of two. Third, according to the labels in the list of facts, labeling each entity as True or False. Specifically, for facts that share a similar sentence structure (\eg, \textit{``He was born on Mach 9, 1941.''} (\texttt{True}) and \textit{``He was born in Ramos Mejia.''} (\texttt{False})), please first assign labels to entities that differ across atomic facts. For example, first labeling ``Mach 9, 1941'' (\texttt{True}) and ``Ramos Mejia'' (\texttt{False}) in the above case. For those entities that are the same across atomic facts (\eg, ``was born'') or are neutral (\eg, ``he,'' ``in,'' and ``on''), please label them as \texttt{True}. For the cases that there is no atomic fact that shares the same sentence structure, please identify the most informative entities in the sentence and label them with the same label as the atomic fact while treating the rest of the entities as \texttt{True}. In the end, output the entities and labels in the following format:
                \begin{itemize}[nosep]
                    \item Entity 1 (Label 1)
                    \item Entity 2 (Label 2)
                    \item ...
                    \item Entity N (Label N)
                \end{itemize}
                % \newline \newline
                Here are two examples:
                \newline\newline
                \textbf{[Example 1]}
                \newline
                [The start of the biography]
                \newline
                \textcolor{titlecolor}{Marianne McAndrew is an American actress and singer, born on November 21, 1942, in Cleveland, Ohio. She began her acting career in the late 1960s, appearing in various television shows and films.}
                \newline
                [The end of the biography]
                \newline \newline
                [The start of the list of checked facts]
                \newline
                \textcolor{anscolor}{[Marianne McAndrew is an American. (False); Marianne McAndrew is an actress. (True); Marianne McAndrew is a singer. (False); Marianne McAndrew was born on November 21, 1942. (False); Marianne McAndrew was born in Cleveland, Ohio. (False); She began her acting career in the late 1960s. (True); She has appeared in various television shows. (True); She has appeared in various films. (True)]}
                \newline
                [The end of the list of checked facts]
                \newline \newline
                [The start of the ideal output]
                \newline
                \textcolor{labelcolor}{[Marianne McAndrew (True); is (True); an (True); American (False); actress (True); and (True); singer (False); , (True); born (True); on (True); November 21, 1942 (False); , (True); in (True); Cleveland, Ohio (False); . (True); She (True); began (True); her (True); acting career (True); in (True); the late 1960s (True); , (True); appearing (True); in (True); various (True); television shows (True); and (True); films (True); . (True)]}
                \newline
                [The end of the ideal output]
				\newline \newline
                \textbf{[Example 2]}
                \newline
                [The start of the biography]
                \newline
                \textcolor{titlecolor}{Doug Sheehan is an American actor who was born on April 27, 1949, in Santa Monica, California. He is best known for his roles in soap operas, including his portrayal of Joe Kelly on ``General Hospital'' and Ben Gibson on ``Knots Landing.''}
                \newline
                [The end of the biography]
                \newline \newline
                [The start of the list of checked facts]
                \newline
                \textcolor{anscolor}{[Doug Sheehan is an American. (True); Doug Sheehan is an actor. (True); Doug Sheehan was born on April 27, 1949. (True); Doug Sheehan was born in Santa Monica, California. (False); He is best known for his roles in soap operas. (True); He portrayed Joe Kelly. (True); Joe Kelly was in General Hospital. (True); General Hospital is a soap opera. (True); He portrayed Ben Gibson. (True); Ben Gibson was in Knots Landing. (True); Knots Landing is a soap opera. (True)]}
                \newline
                [The end of the list of checked facts]
                \newline \newline
                [The start of the ideal output]
                \newline
                \textcolor{labelcolor}{[Doug Sheehan (True); is (True); an (True); American (True); actor (True); who (True); was born (True); on (True); April 27, 1949 (True); in (True); Santa Monica, California (False); . (True); He (True); is (True); best known (True); for (True); his roles in soap operas (True); , (True); including (True); in (True); his portrayal (True); of (True); Joe Kelly (True); on (True); ``General Hospital'' (True); and (True); Ben Gibson (True); on (True); ``Knots Landing.'' (True)]}
                \newline
                [The end of the ideal output]
				\newline \newline
				\textbf{User prompt}
				\newline
				\newline
				[The start of the biography]
				\newline
				\textcolor{magenta}{\texttt{\{BIOGRAPHY\}}}
				\newline
				[The ebd of the biography]
				\newline \newline
				[The start of the list of checked facts]
				\newline
				\textcolor{magenta}{\texttt{\{LIST OF CHECKED FACTS\}}}
				\newline
				[The end of the list of checked facts]
			};
		\end{tikzpicture}
        \caption{GPT-4o prompt for labeling hallucinated entities.}\label{tb:gpt-4-prompt}
	\end{center}
\vspace{-0cm}
\end{table*}


\begin{table*}[t]
\centering
\small
\caption{Errors and accuracy of Plain Generation (\textbf{Plain}), Revised Generation (\textbf{Revised}), and Automatic selection (\textbf{Selection$_\texttt{Auto}$}). (\textbf{Data \#}) shows the number of data. \chifinal{($\downarrow$) indicates lower values are better, while ($\uparrow$) indicates higher values are better.}}
\label{tab:error_rates}
\begin{tabular}{lccccccc}
\toprule
\multirow{3}{*}{\textbf{Process}}  & \multirow{3}{*}{\textbf{Data \#}} & \multicolumn{2}{c}{\textbf{Factuality}}     & \multicolumn{2}{c}{\textbf{Consistency}}        & \multirow{3}{*}{\textbf{Accuracy} ($\uparrow$)} \\
\cmidrule(lr){3-4} 
\cmidrule(lr){5-6} 
                      &  & \textbf{Analogy Object}& \textbf{Inappropriate} & \textbf{Object} & \textbf{Logical} & \\ 
                      & & \textbf{Paradox}  ($\downarrow$)& \textbf{Analogy} ($\downarrow$)& \textbf{Confusion} ($\downarrow$)& \textbf{Contradiction} ($\downarrow$)& \\
\midrule
Plain         &     30  & \textbf{0.27}                & 0.23                  & 0.23                & 0.03                 & 0.53    \\
Revised      &      30     & 0.33           & 0.23                  & \textbf{0.17}                & \textbf{0.00}                 & 0.53    \\
Selection$_\texttt{Auto}$   &      10     & 0.30           & \textbf{0.20}                  & 0.20              & \textbf{0.00}                 & \textbf{0.60}    \\ 
% Selection$_\texttt{Final}$     &     4    & \textbf{0.25}           & 0.25                 & 0.25              & \textbf{0.00}                 & \textbf{0.75}    \\ 
\bottomrule
\end{tabular}
\end{table*}
\chirev{As shown in Fig.~\ref{fig:pipeline}, we began by manually selecting ten scientific concepts from physics and biology in Chinese high school textbooks. 
Next, we used the advanced LLM, GPT-4o~\cite{openai_gpt-4_2023} (temperature = 0.7) to generate three analogies for each concept with three principles summarized from education research~\cite{hesse1959defining,gentner1983structure,gentner2017analogy} incorporated in the prompt. 
%Tab.~\ref{tab:instruction_prompt} (I) provides the prompt template used for this task.
Three authors independently identified and annotated errors in generated analogies. 
After repeated discussion, the annotators classified the errors into four types: two related to factuality and two to consistency, as follows.
\begin{itemize}
    % \item \textbf{Concept Paradox}: The analogy inaccurately represents the scientific concept, conflicting with established physical phenomena and commonsense knowledge.
    \item \textbf{Analogy Object Paradox}: The objects of the analogy do not align with physical laws or commonsense knowledge.
    \item \textbf{Inappropriate Analogy}: The analogy fails to accurately mirror the concept, leading to misconceptions.
    \item \textbf{Object Confusion}: The same analogy objects are assigned different roles or functions across various contexts.
    \item \textbf{Logical Contradiction}: The syntax within a sentence or paragraph contradicts itself.
\end{itemize}

%\begin{figure}[!]
\vspace{-0.1cm}
    \centering
    \includegraphics[width=\linewidth]{figures/select.pdf}
    \vspace{-0.6cm}
    \caption{A 3D visualization of the active and sequential training process. The figure shows the selected auxiliary datasets at turn 0, turn 200, and turn 400, for two specific target tasks. The images are presented alongside their ground truth.
    }
    \label{selection}
    \vspace{-0.5cm}
\end{figure}
The inter-rater reliability among annotators reached Fleiss' Kappa of 0.83 for Analogy Object Paradox, 0.94 for Object Confusion, and 1 for the remaining error codes.
Error annotations in subsequent steps achieved similar reliability.
As shown in the first row of Tab.~\ref{tab:error_rates}, out of the 30 generated analogies, 16 were correct. 
The remaining analogies frequently exhibited the first three error types with one analogy containing logical contradiction.
From these errors, we derived four new principles and added them to the prompt template (Tab.~\ref{tab:instruction_prompt} I) to help GPT-4o avoid these errors.
% We then refined the prompt by incorporating four new principles to help GPT-4 avoid these errors. 
% The revised prompt template is shown in Tab.~\ref{tab:instruction_prompt}.
However, even with these improvements, GPT-4o still made errors. 
To address this, we followed prior AI research~\cite{pan2023automatically,yuan-etal-2023-distilling,liu2024large} and allowed GPT-4o to automatically select the best of the three candidate analogies generated for each concept.
\chifinal{The prompt for analogy selection is shown in Tab.~\ref{tab:instruction_prompt} II.}
As shown in the third row of Tab.~\ref{tab:error_rates}, enabling the model to self-correct improved the accuracy of the analogies.}
%: 


%\paragraph{Step 1: Scientific Concept Collection}
%We first manually selected ten scientific concepts from physics and biology, as outlined in a Chinese high school textbook, which the students had not yet studied. We also collected five exam questions per concept from the books to assess the students' understanding of these concepts in our further experiments.

%\paragraph{Step 2: Plain Analogy Generation}
%To align with the educational process for students, we requested the most advanced LLM, \ie, GPT-4o~\cite{openai_gpt-4_2023}, to generate free-form analogies for each scientific concept. 
%We adhered to the prompt settings described in \cite{bhavya_analogy_2022} for the generation process. 
%Tab.~\ref{tab:instruction_prompt} (I) shows the prompt template used for generating. And examples of these analogies are presented in Fig.~\ref{fig:example}.
%\ysy{We did not include demonstrations in the instruction prompt because demonstrations could induce bias in GPT-4~\cite{pmlr-v139-zhao21c}, thereby degrading the generalization of analogy generation.}
%Specifically, to facilitate more effective analogy generation by GPT-4, we incorporated principles from analogy cognition theory~\cite{hesse1959defining,gentner1983structure,gentner2017analogy} into the instruction prompt. 
%Additionally, we provided GPT-4 with textbook content relevant to the scientific concepts to tailor the analogies to students' learning progress.
%We did not include examples in the instruction prompt because there is no high-quality analogy dataset in the educational domain from which students can learn. 
%Examples could induce bias in GPT-4~\cite{pmlr-v139-zhao21c}, thereby degrading the generalization of analogy generation.

%\paragraph{Step 3: Error Annotation for Generated Analogy}
%\sout{Based on the prompt in step 2, we set the temperature to 0.7 for GPT-4 and generated three analogies for each concept.} 
%Three authors served as annotators to evaluate the quality of the generated analogies by annotating the errors. Initially, the annotators followed two common categories of error classification typically used in LLM outputs~\cite{jang-etal-2022-becel,gekhman-etal-2023-trueteacher,chuang2024dola}: 1) Factual Accuracy: the appropriateness of the analogies for the given concept; 2) Consistency: the coherence of objects within the analogies. 
%The three annotators independently applied this classification to their annotations. After annotating 10 analogies, a discussion about error codes and annotations was conducted. Following three discussions, the annotators categorized the error codes into five types: the first three are factual errors, while the last two are inconsistency errors.


%After repeated discussions, the inter-rater reliability among annotators reached Fleiss' Kappa of 0.83 for Analogy Object Paradox, 0.94 for Object Confusion, and 1 for the remaining error codes.
%As shown in the first row of Tab.~\ref{tab:error_rates}.
%Despite the powerful capabilities of GPT-4, it fails to generate appropriate and satisfactory analogies for scientific concepts with our initial principles. 
%Specifically, concerning the factual aspect, GPT-4 attempts to produce analogies that adhere to physical logic. 
%However, the objects within these analogies often contradict physical laws or commonsense knowledge, resulting in errors related to the analogy objects.

%Moreover, in the generated analogies, the same objects are inconsistently assigned different roles or functions in various contexts, indicating that GPT-4 suffers from object confusion.
%For example, when generating analogies for the photoelectric effect, GPT-4 uses a trampoline. 
%In this analogy, the trampoline represents the metal surface, and children represent electrons. 
%The model initially makes an analogy between the force with which a worker presses the trampoline and the frequency of light, \eg, \textit{``First, if someone presses down on a diving board with a very small force (low-frequency light), the diving board will not have enough elasticity to bounce you back up.''}
%However, in later descriptions, this force is equated to light intensity, \eg, \textit{``The greater the force applied (the higher the intensity of light), the more people will be bounced off the diving board in a given period.''}.



%\paragraph{Step 4: Analogy Generation with Revised Prompt and Second Round of Error Annotation}
%In Step 3, we discovered that GPT-4 cannot generate appropriate analogies directly, as it produces several errors. 
%Therefore, we refined the prompt by incorporating new principles to guide GPT-4 in avoiding these errors, enhancing the quality of the generated analogies. 
%Tab.~\ref{tab:instruction_prompt} (II) details the revised prompt template.
%The three annotators applied the error codes used in Step 3 to the analogies generated in this round.
%After two rounds of discussions, the inter-rater reliability among annotators reached Fleiss' Kappa of 0.84 for Analogy Object Paradox, 0.94 for Inappropriate Analogy, 0.92 for Object Confusion, and 1 for the remaining error codes.
%Based on Tab.~\ref{tab:error_rates}, we can find that the revised prompt enhances the coherence of objects within analogies, thereby minimizing confusion and logical contradictions during generation. 
%However, to accurately identify analogous relationships, the model may force objects to conform to patterns that contradict physical laws or common sense, thereby intensifying the analogy object paradox.

%\paragraph{Step 5: Automatic Analogy Selection from LLMs}

%From step 4, it is evident that the model is still prone to errors despite providing clear and instructive guidelines. 
%Previous research in the AI community has demonstrated that models can select the optimal result from multiple outputs through a process known as self-correction~\cite{pan2023automatically,yuan-etal-2023-distilling,liu2024large}. 
%Therefore, we allowed GPT-4 to choose the best result according to the guidelines from three analogies generated for each concept. 

%Despite clear guidelines, the model in Step 4 still makes errors. 
%To address this, we followed the previous research in the AI community~\cite{pan2023automatically,yuan-etal-2023-distilling,liu2024large} and allowed GPT-4 to choose the best result according to the guidelines from three candidate analogies generated for each concept. 

%The chosen prompts are displayed in Tab.~\ref{tab:instruction_prompt} (III). 
%Results in Tab.~\ref{tab:error_rates} show that enabling the model to self-correct enhances the accuracy of the analogies. 
%However, the model struggles with issues such as object confusion. 
%We hypothesize that this difficulty arises because the objects in the analogies are described in disparate text sections, requiring the model to use contextual cues for differentiation. 
%This necessitates a robust capability for understanding long contexts~\cite{li2024can}. 
%Additionally, as GPT-4 is a decoder-only model, its unidirectional attention mechanism may lead to inconsistency~\cite{}\ysy{\todo Add citations}.

\begin{figure*}[t]
    \centering
    \includegraphics[width=0.9\linewidth]{figure/revexample.pdf}
    \caption{\chirev{The examples of final analogies generated from GPT-4o after iterative generation and annotation in Study I.}}
    \Description{This figure shows examples of analogies generated by GPT-4o after multiple iterations. In the ``Correct'' Analogy category, a satisfying analogy for ``immune response" compares the body to a city protected by a security system. An imaginative analogy for ``wave-particle duality" likens it to an amusement park ride, though less accurate. A non-analogy for the ``Doppler effect" fails to form a true analogy. In the Incorrect Analogy category, the analogy for ``nuclear fission'' confuses objects by comparing neutrons to a knife and seeds, leading to inappropriate and paradoxical explanations.}
    \label{fig:example}
\end{figure*}

%\paragraph{Step 6: Pattern Annotation and Final Analogy Selection}

% From Step 5, we obtained ten concepts with ten analogies generated and then selected by the LLM. 
% We re-evaluated the selected concepts and their corresponding analogies to ensure they were suitable for students to learn through analogies. 
Finally, we further categorized ten analogies selected by the LLM into four distinct groups, as illustrated in Fig.~\ref{fig:example}:
\begin{itemize}
    \item \textbf{Correct and Satisfying Analogy}: Analogies in this category are error-free. 
    The objects in these analogies are realistic, align with common sense, and adhere to physical laws, effectively and vividly illustrating scientific concepts.
    \item \textbf{Correct Analogy with Imagination}: Analogies in this category require envisioning non-existent objects or processes to explain a concept. While logically sound, they demand creative thinking and imagination from students.
    \item \textbf{Correct Non-Analogy}: This is more akin to an example-based explanation than a true analogy and is not generally recognized as an analogy in cognitive science.
    \item \textbf{Incorrect Analogy}: This category includes analogies exhibiting previously identified error types. 
    These analogies are inappropriate for students to refer to, as they do not accurately convey the intended concept.
\end{itemize}


The analogies for the five biological concepts fell under the \textbf{Correct and Satisfying Analogy} category. 
In contrast, the five physical concepts were distributed as follows: three under \textbf{Incorrect Analogy}, one under \textbf{Correct Analogy with Imagination}, and one under \textbf{Correct Non-Analogy}.

We limited the number of concepts and analogies to four to avoid overwhelming students with too much new knowledge in further tests. 
To ensure a balance across subjects and categories, we specifically selected two biological concepts categorized as \textbf{Correct and Satisfying Analogy} and two physical concepts, one each under  \textbf{Correct Analogy with Imagination} and \textbf{Incorrect Analogy}, as shown in the right side of Fig.~\ref{fig:example}.
% , with the final selection and error rates detailed in Tab.~\ref{tab:error_rates}. 
We excluded the one \textbf{Correct Non-Analogy} from further consideration, as it is not typically classified as an analogy.


\subsection{Stimuli}
\label{sec:study1_stimuli}
Since students do not have access to electronic devices and are more familiar with and serious about traditional classroom tests, we conducted offline tests in class.

Based on the data preparation, we printed a test paper and two reference materials. 
The test paper comprises 20 multiple-choice questions, with 5 questions assigned to each of the following 4 concepts: Nuclear Fission and Fusion, Wave-Particle Duality, Blood Sugar Regulation, and Immune Response. 
In addition to selecting answers, students are required to complete a 5-point Likert scale rating for self-confidence to measure their subjective satisfaction.
The first reference material provides textbook explanations for the four concepts, while the second adds LLM-generated analogies before the explanations.
% The first reference material consists of textbook explanations for the four concepts. 
% The second reference material includes LLM-generated analogies followed by textbook explanations for each concept.
The test paper and the reference materials present the concepts in the same order. 
They are highlighted in bold, making it easier for students to find and connect the information with the questions.

\subsection{Procedure}
Then, we conducted an in-class test for students in two classes lasting 30 minutes. 
%The session began with a 5-minute introduction where we introduced ourselves and explained that the test was part of a study on intelligent education assisted by AI. 
We first gave a 5-minute introduction for the background of our test.
%After the introduction, we divided the students into two groups based on their seating, distributed two sets of reference materials to each group, and instructed them to indicate their group number on the test paper (either 1 or 2). 
After the introduction, we randomly divided the students into two groups and distributed two sets of reference materials to each group. 
% We instructed them to indicate their group number on the test paper (either 1 or 2). 
We clarified the meaning of self-confidence rating. 
Under our supervision, each student then independently completed the test using the materials provided in 25 minutes.
%Before the test, we allocated an estimated 20 minutes for completion. Observing the students' progress, we extended the time by five minutes, ensuring they had a sufficient 25 minutes to finish.

After the test, we interviewed four students from each group, totaling eight participants.
Each 2-minute interview earned participants a \$2 gift card.
We asked them about any difficulties during the test and, for those with analogies in their materials, how these helped them answer questions alongside textbook concepts.

\subsection{Results Analysis}
\label{sec:study1_results_analysis}
Our selection criteria excluded test papers with incomplete answers. 
After examining the 49 test papers, we excluded 5 that had more than 5 unanswered questions. 
The remaining 44 fully completed papers, 22 from each group, were considered valid data.
Our analysis followed a top-down approach, starting from the overall test (20 questions) and proceeding to finer levels: subject (10 questions each), concept (5 questions each), and individual questions. For further comparison, students' responses were averaged across questions at the first three levels.

% We computed descriptive statistics to gain overall insights and performed statistical tests to determine significance at each level. 
% We identify differences based on both the descriptive results and p-values.
% Given the discrete nature of accuracy and subjective ratings, we employed the exact Wilcoxon-Mann-Whitney test (using the R package \texttt{coin}), solving for ties in the data as well. 
% The test's null hypothesis assumes that the distribution of responses is the same in both groups; thus, a small p-value suggests a statistical difference in student understanding related to the use of analogies. 
% For individual question accuracy, where student responses followed a binomial distribution, we employed Fisher's exact test (using the R package \texttt{stats}). 
% This test assumes that the proportion of correct responses is consistent across groups, with a small p-value indicating a potential association between analogies and student accuracy. 
% We also calculated Kendall's tau correlation coefficient for all pairs of participant response types within each concept and group, as it is suitable for ordered categorical variables and effectively handles ties. 
% A significance level of $p$<0.05 was used for all tests.

\chirev{
We computed descriptive statistics to gain overall insights and performed statistical tests to determine significance at each level. 
The experiment results include the students' answer accuracy and confidence ratings, both can be regarded as ordinal categorical variables.
Thus, we mostly employed the exact Wilcoxon-Mann-Whitney test (using the R package \texttt{coin}) to evaluate the significance of difference between the two groups. 
For one exception, we employed Fisher's exact test (using the R package \texttt{stats}) on students' answer accuracy at the individual question level, where the accuracy is binary (either 0 or 1). 
In any of the tests, a small p-value indicates a potential association between the use of LLM generated analogies and the students’ outcomes, and a significance level is defined as $p$<0.05 in all tests.
We also calculated Kendall's tau correlation coefficient to assess the relationship between students’ objective answer accuracy and subjective confidence ratings within each concept and group.
% Considering the potential for Type II errors due to our small sample size, we base our analysis on both descriptive statistics and p-values.
}

We summarize our findings as follows. The two groups are referred to as Group T (Textbook explanation only) and Group L (Textbook explanation with LLM-generated analogy), while the interviewed students are denoted as T1-T4 and L1-L4.
% We analyzed 49 test papers and found that 5 had more than 5 unanswered questions. 
% The remaining 44 fully answered papers, with 22 from each group, were considered valid data for further analysis. 
% We first computed descriptive statistics for all groups and observed distinct trends in the responses across different concepts and questions. 
% Then, we employed the exact Wilcoxon-Mann-Whitney test to evaluate the impact of analogies on students' understanding of individual concepts by averaging the responses for the five related questions. 
% % We compared the accuracy rates across concepts using the Friedman test followed by a Nemenyi post-hoc analysis. 
% % We regarded the number of correct answers of each participant and the Likert scale results as ordered categorical variables.
% We calculated Kendall's tau correlation coefficient for all pairs of the participant response types in each concept and group.
% For statistical tests on responses to individual questions, we then conducted further analyses: Fisher's exact test on the association of group and accuracy to determine the impact of analogies on whether students answered individual questions correctly, and exact Wilcoxon-Mann-Whitney tests on the association between group and self-confidence and perceived helpfulness to assess how analogies influenced students' subjective satisfaction.
% % we performed Fisher's exact test to determine the impact of analogies on whether students answered individual questions correctly.
% % We also conducted exact Wilcoxon-Mann-Whitney tests on self-confidence and perceived helpfulness to assess how analogies influenced students' subjective satisfaction with specific questions. 
% % The exact Wilcoxon-Mann-Whitney test is also used to evaluate the impact of analogies on students' understanding of individual concepts by averaging the responses for the five related questions.
% We considered statistical significance at a significance level of $p$ < 0.05 for all the cases.
% We summarize the following findings based on the statistical results of student tests and interviews.
% The two groups are denoted as Group T (Textbook explanation only) and Group L (Textbook explanation with LLM-generated analogy), and students interviewed are denoted as T1-T4 and L1-L4 in the following discussion.


\begin{figure}[t]
    \centering
    \includegraphics[width=1\linewidth]{figure/study1_acc_conf_overall.pdf}
    \caption{Boxplots showing the distribution of answer accuracy and confidence ratings for the two groups and comparing the overall test results and the two subjects. ``M'' represents Mean, ``p'' represents the significance of the association between accuracy and group, determined by the exact Wilcoxon-Mann-Whitney test, and * represents significance ($p$<0.05).}
    \label{fig:study1_acc_conf_overall}
    \Description{This figure includes boxplots showing the distribution of answer accuracy and confidence ratings. The plot has facets of accuracy and confidence rating. In each facet, X-labels are all questions, biological questions, and physical questions. The two boxplots of two groups are juxtaposed inside each label. In the accuracy facet on the left, all boxplots share the same 0 to 1 y-axis. Physics questions have overall higher accuracy with no significant difference between groups. Biology questions have overall lower accuracy with significant differences between groups. All questions are thus in the midst with no significant difference between groups. In the confidence rating facet on the right, all boxplots share the same 0 to 5 y-axis. Physics questions have a mildly higher confidence rating at each quantile while biology questions are mildly lower, but the boxplots from different labels largely overlap. All differences in confidence ratings between groups are significant.}
\end{figure}



\textbf{LLM-generated analogies generally aid problem-solving and have a greater impact on biological concepts than physical concepts.}
As shown in the left of Fig.~\ref{fig:study1_acc_conf_overall}, the overall accuracy for physics questions is higher, while a significant association exists between accuracy and group for biology questions ($p$ = 0.042).
Examining individual questions (Fig.~\ref{fig:study1_acc_question}), there are three questions within the two biological concepts where Group L's accuracy largely exceeds Group T's by more than 0.25.
Besides, a strict Fisher’s exact test shows marginally significant associations between accuracy and group for two questions (Q1 and Q3 of “Immune Response”) ($p$ = 0.067).
However, no such clear difference is seen for the physics questions.
% As shown in Fig.~\ref{fig:study1_acc_question}, Although for all individual questions, there is no significant association between accuracy and group, there are three questions for the two biological concepts where the accuracy of Group L exceeds that of Group T by more than 0.25. 
% Associations between analogy and group in two questions (Q1 and Q3 of ``Immune Response'') are marginally significant ($p$ = 0.067). 
% However, there are no questions with a clear difference for the physical concepts.
In the interviews, all four students from Group L (L1-4) coincidentally explained the role of analogies based on subjects.
They noted that explanations for physical concepts are relatively concise, allowing them to understand directly without analogies. 
In contrast, the lengthy explanations for biological concepts made analogies helpful to ``\textit{get an overview and quickly identify key terms}'', as indicated by L4.
% \chirev{Besides, L1-L4 all stated that biological analogies are of higher quality than physical analogies.}
% Similarly, T1 described the biggest difficulty in answering questions as ``\textit{the concept descriptions of biology are so long that I have no desire to read them.}''
% For these two questions, Q1 focused on the core concept, which was clearly illustrated in the provided analogy. 
% Q3 involved knowledge about autoimmune diseases, which wasn't covered in our provided explanation, leaving Group T confused; T2 even asked, ``\textit{What is the autoimmune disease?}'' during the interview. 
% Group L, however, could identify the correct answer by choosing the option inconsistent with the analogy without considering the other unfamiliar things.

\begin{figure}[t]
    \centering
    \includegraphics[width=1\linewidth]{figure/study1_acc_question.pdf}
    \caption{Heatmap of accuracy differences between Group L and Group T for individual questions, with blue indicating higher accuracy for Group L and red for Group T. Each cell contains a bar chart of the respective accuracies and a p-value representing the significance of the association between accuracy and group, determined by Fisher's exact test. For two of the questions, since all students in both groups answered them correctly, Fisher's exact test is not applicable, and $p$ = N/A.}
    \label{fig:study1_acc_question}
    \Description{This figure includes a Heatmap of accuracy differences between Group L and Group T for individual questions. The cells of heatmap are vertically divided by concepts, and then horizontally divided by the question ID in each concept. For ``Immune Response'' and ``Sugar regulation concept'', student accuracy shows much more significant differences in part of the questions (Q1, Q3, Q4 of ``Immune Response'' and Q3 of ``Sugar regulation concept''). For most questions, Group L had higher accuracy, except for Q3 ``Immune Response'' in which Group L showed clearly lower accuracy. However, for Wave-Particle Duality and Nuclear Fission and Fusion concept, the comparison between student accuracy from each group is ambiguous, as all the superiority or inferiority in accuracy is negligible considering the p-value. }
\vspace*{-10pt}
\end{figure}

\textbf{LLM-generated analogies may negatively affect students' understanding without teacher intervention due to \chirev{errors and missing information in analogies and students' incorrect learning strategies with over-reliance}.}
\chirev{Although some students in Group L identified the analogy of ``Nuclear Fission and Fusion'' as an \textbf{Incorrect Analogy} and noted specific LLMs' hallucination during the interview, Group L's accuracy on all five questions was no higher than Group T's (Fig.~\ref{fig:study1_acc_question}).}
% Therefore, we believe that incorrect analogies may harm students' understanding.
Furthermore, we also found that, although the \chirev{\textbf{Correct and Satisfying Analogies}} slightly improved overall answer accuracy, they could also harm students' understanding in some cases due to missing information.
For example, Group L's accuracy for Q4 of ``Immune Response'' was 0.27 lower than Group T's.
Their incorrect answer choices suggest that some students believed that ``plasma cells can recognize antigens.''
However, the textbook explains that ``antigen-presenting cells such as B cells recognize antigens'' and ``plasma cells release antibodies to eliminate antigens'', while the LLM-generated analogy only includes the latter and omits the former.
\chirev{This may be linked to over-reliance issue, as L1 and L3 described their learning strategies during the interview as ``\textit{reading the analogy first, answering the questions, and not revisiting the textbook if the answers seemed clear from the analogy.}''}
% Based on quantitative results and interviews, we argue that some students may rely on the vague information from the analogy without carefully checking the textbook explanations, leading to incorrect answers.
% This highlights the potential risk that, without teacher intervention, students may mistakenly believe they fully understand a concept after grasping only the incomplete or imprecise information from the analogy.

%Group L's accuracy rate in the five questions of ``Nuclear Fission and Fusion'' was all lower than or equal to that of Group T. 
%While the other three analogies slightly improved overall answer accuracy, they also might harm students' understanding.
%The accuracy of Group L for Q4 of ``Immune Response'' was lower than that of Group T (Mean: 0.50 vs. 0.77).
%According to the answers, many students in Group L believe that ``plasma cells can recognize antigens'' is correct.
%However, the textbook explanations mention that ``antigen-presenting cells such as B cells recognize antigens'' and ``plasma cells release antibodies to eliminate antigens'', while the LLM-generated analogy only includes the latter and omits the former.
%During the interview, L1 and L3 mentioned they ``\textit{read the analogy first and then answered the questions}''.



\textbf{Students subjectively appreciate the correct LLM-generated analogies often with overconfidence.}
We \chirev{were surprised to find} the strongest association between group and self-confidence was for ``Wave-Particle Duality'' ($p = 0.025$) among the four concepts. 
This suggests that students were receptive to the \chirev{\textbf{Correct Analogy with Imagination}}. 
However, there was no significant association between group and answer accuracy for this concept. 
We also observed overconfidence among \chirev{\textbf{Correct and Satisfying Analogies} for} biological concepts: a significant association between group and confidence ratings for Q2 of “Blood Sugar Regulation” ($p$ = 0.034), but none with accuracy ($p$ = 1).
As shown on the right of Fig.~\ref{fig:study1_acc_conf_overall}, there are significant associations between group and confidence ratings for both subjects.
Besides, We found a negligible correlation between accuracy and the confidence rating, as the absolute value of Kendall's tau correlation coefficient between them in each group and concept was $\leq0.2$.
% This indicates that Group L's understanding may not align with their subjective satisfaction, suggesting that LLM-generated analogies could lead students to overestimate their grasp of the concept.

% \uline{\textbf{Summary to results of RQ1:}}
% Overall, we found that LLMs are relatively effective at generating helpful analogies for biological concepts, aiding students in understanding complex and detailed concept explanations. 
% However, LLMs often produce incorrect analogies for abstract physical concepts due to the difficulty of finding real-life counterparts, and such analogies are less effective and necessary for understanding physical concepts. 
% Additionally, students might \chirev{overly} rely on incomplete or imprecise information from analogies \chirev{with incorrect learning strategies}, or overestimate their understanding simply because they grasp the analogy.

\chirev{Overall, our empirical evidences suggest that LLM-generated analogies are currently unsuitable for unsupervised self-learning systems. We will discuss LLMs in supporting students' learning by analogy in Sec.~\ref{sec: discussion_student}.}




%\clearpage
\section{Study II}
\label{sec:study2}
In this section, we introduce the pre-class interview (Sec.~\ref{sec:study21}) to gather requirements for the classroom experiments (Sec.~\ref{sec:study22}) that evaluate the actual use of LLM-generated analogies in classroom teaching.



\subsection{Pre-class Interview}
\label{sec:study21}
In this subsection, we outline the participants for our pre-class interview (Sec.~\ref{sec:study21_participants}), the procedure and stimulus (Sec.~\ref{sec:study21_procedure_and_stimuls}), and the findings and derived requirements (Sec.~\ref{sec:study21_findings_and_derived_requirments}) for further classroom experiments. 
\subsubsection{Participants}
\label{sec:study21_participants}
We recruited two \chifinal{Chinese} teachers and two \chifinal{Chinese} students from the same high school as Study I to participate in pre-class interviews. 
The two teachers (T1 and T2; 1 female) are a physics teacher with 6 years of teaching experience and a biology teacher with 3 years of teaching experience. 
Both have a bachelor's degree, are interested in AI-assisted education, and teach first-year high school courses in the semester during our study.
The two senior students (S1 and S2; 1 female) are in their third year of high school, have learned the concepts used in Study I, and have above-average grades.


\subsubsection{Procedure}
\label{sec:study21_procedure_and_stimuls}
We conducted one-on-one semi-structured online interviews with the participants via Tencent Meeting.
The student interviews lasted 40 minutes, with a \$10 gift card for each student, while the teacher interviews lasted 60 minutes, with a \$20 gift card for each teacher.
% Below is a brief description of the interview processes for the teachers and students, respectively.
% Please refer to the supplementary materials for details of the procedure and stimulus.

For teachers, the interview included four steps to understand the requirements of LLM-generated analogies in teaching.
% For teachers, the interview included four steps to understand the potential role and requirements of analogies generated by LLMs in teaching.

\textbf{Step 1: Analogy Orientation.} 
We first presented them with common analogies in the classroom from the literature (\eg, ``light waves and water waves'', ``heart and hydraulic pump'')~\cite{oliva_teaching_2007} to orient them to analogies and ensure the terminology used during the interview.
We asked the teachers to recall the analogies they had used and encouraged them to think aloud about any experiences with analogies throughout the interview.

\textbf{Step 2: Analogy Usage Exploration.}
After that, we conducted interviews using a questionnaire primarily based on the one proposed by~\cite{oliva_teaching_2007} but modified to incorporate insights from the latest research over the past decade.
We first investigated how teachers prepare analogies, such as whether they prepare in advance or improvise and adjust in the class. 
Then we asked teachers to share the characteristics of good analogies in their opinion~\cite{gray_teaching_2021} and whether they agree with the principles we summarized from the literature in Study I. 
We also investigated whether teachers involved students in building analogies during teaching, at which step of introducing knowledge points they used analogies, and whether they followed the six-step theoretical model about analogy~\cite{richland_analogy_2015}.
Other questions covered include whether visual aids were used and whether immediate feedback was provided on students' understanding of analogies.

\textbf{Step 3: AI Usage Exploration.}
% The interview questions then explored the use of AI tools like ChatGPT and AI-assisted education. 
We then asked teachers about their previous experiences with AI tools like ChatGPT, their familiarity with AI-assisted teaching or self-learning. 
We also inquired about their concerns on AI performance and its application in educational settings~\cite{chen2024stugptviz,tan_more_2024}, and whether they believe AI could partially replace teachers to achieve educational goals such as mastering basic concepts, problem-solving, and developing higher-order independent thinking skills.

\textbf{Step 4: Expectations Sharing on LLM-generated Analogies.}
We then showed each teacher the analogies from Study I in their respective teaching subjects.
We asked them to evaluate each analogy's strengths, weaknesses, classroom applicability, and potential for teacher modification
Building on this, we asked teachers to share their expectations for effective AI-generated analogies.

The interviews with the students were focused on their classroom experiences rather than exploring AI usage since their role in the classroom mainly involved receiving information rather than designing analogies, and they spent most of their time without any smart devices or AI.
Initially, students were asked to recall analogies used in class. 
We then presented the ten concepts from Study I, asking students to reflect on their learning experiences. 
Next, we presented the ten analogies from Study I and asked for their feedback on their effectiveness in enhancing their understanding.




\subsubsection{Findings and Derived Requirements}
\label{sec:study21_findings_and_derived_requirments}
%\begin{small}
\begin{longtable}[t]
{@{}p{0.2\linewidth}p{0.39\linewidth}p{0.39\linewidth}@{}}
\caption{A Summary of Interviewing Physics and Biology Teachers.} \label{tab:findings_in_study2_pre_class_interview}\\
\toprule
\textbf{Topic} & \textbf{Physics Teacher (T1)} & \textbf{Biology Teacher (T2)} \\ 
\midrule
\endfirsthead

\multicolumn{3}{c}%
{{\bfseries \tablename\ \thetable{} -- continued from previous page}} \\
\toprule
\textbf{Topic} & \textbf{Physics Teacher (T1)} & \textbf{Biology Teacher (T2)} \\
\midrule
\endhead

\bottomrule
\endfoot

\bottomrule
\endlastfoot

\rowcolor[gray]{0.95}\multicolumn{3}{c}{\textbf{Analogy Usage Exploration}} \\
Analogies Frequency & Sometimes. & Frequent. \\
\midrule
Analogies Feature & Mostly between learned concepts. & Mostly between biology and daily life. \\
\midrule
Source of Analogies & Mostly Prepared analogies between concepts. \newline A few improvised analogies with everyday lives. & Mostly prepared analogies. \newline Nearly no improvised analogies. \\
\midrule
Good Analogy Criterion & Easy to understand and free of scientific errors. & Easy to understand and related to everyday life. \\
\midrule
Agreement with Initial Principles in Study I & Partial agreement: Analogies between similar physical concepts. & Total Agreement. \\
\midrule
Analogy Explanation & Verbal explanation + imagery + teaching aids & Verbal explanation + imagery + teaching aids \\
\midrule
Analogies Usage Scenario & Often used to introduce concepts. \newline Sometimes throughout teaching to help students remember key points & Often used when detailing knowledge points. \\
\midrule
Agreement with the Six-step Model of Practice~\cite{richland_analogy_2015} & Acknowledges most, except for pointing out differences when introducing concepts. & Totally agreement. \\
\midrule
Student Participation in Constructing Analogies & Rare. Sometimes, students offer their ideas, which might be used in the next class. & Rare. Sometimes, students prepare analogies for student-led discussions. \\
\midrule
Students Understanding Examination & Question students with "Have you seen something similar before?", or observe students' expressions & Students complete a few exercises during class, or question students about concept differentiation. \\
\midrule
\rowcolor[gray]{0.95}\multicolumn{3}{c}{\textbf{AI Usage Exploration}} \\
Awareness and Experience with AI & Has used ChatGPT for writing papers, lesson plans, and creating images; knows about Sora. & Has used ChatGPT for tenders and personal use. \\
\midrule
Pros and Cons of AI & Pros: helps write unexpected things. \newline Cons: Needs specific questions; AI usually doesn't follow the instructions. & Pros: Provides broad ideas.  \newline No clear cons due to limited experience. \\
\midrule
Can AI Replace Teachers? & Teachers know students' learning situations, AI does not; AI-generated content needs adjustment. & AI cannot replace but complement teachers. \\
\midrule
\rowcolor[gray]{0.95}\multicolumn{3}{c}{\textbf{Expectations Sharing on LLM-generated Analogies}} \\
Positive Comments on Analogies in Study I & 1. The analogies are all vivid and some of them are interesting & 1. Some analogies are similar to those used in class \newline 2. Identify analogies to try in class for concepts not usually taught with analogies. \\
\midrule
Negative Comments on Analogies in Study I & 1. Analogies don't clarify abstract concepts. \newline 2. Analogies can complicate simple concepts. \newline 3. For concepts that are tested simply, memorization is enough. \newline 4. Pictures could make some concepts clear without analogies. & 1. Analogies shouldn't reflect all but the main concepts; the rest relies on memory. \newline 2. Pictures and animations can visualize familiar organisms without analogies. \newline 3. Although rare, related concepts sometimes are used as analogies. \\
\midrule
Overall Expectations & Vivid analogies between physical concepts. & Analogies from daily life for teaching focus; \newline Interesting analogies to stimulate learning interest.\\
\end{longtable}
\end{small}
\section{Analysis} \label{sec:analysis}
In this section, we provide a comprehensive analysis of Satori. First, we demonstrate that Satori effectively leverages self-reflection to seek better solutions and enhance its overall reasoning performance. Next, we observe that Satori exhibits test-scaling behavior through RL training, where it progressively acquires more tokens to improve its reasoning capabilities. Finally, we conduct ablation studies on various components of Satori's training framework. Additional results are provided in Appendix~\ref{app:results}.



\paragraph{COAT Reasoning v.s. CoT Reasoning.}
\begin{table}[h]
  \begin{center}
  \scriptsize
  \captionsetup{font=small}
  \caption{\textbf{COAT Training v.s. CoT Training.} Qwen-2.5-Math-7B trained with COAT reasoning format (Satori-Qwen-7B) outperforms the same base model but trained with classical CoT reasoning format (Qwen-7B-CoT)}
  \setlength{\tabcolsep}{1.3pt}
  \begin{tabular}{cccccccccc}
    \toprule
    \textbf{Model} & \textbf{GSM8K} & \textbf{MATH500}  &  \textbf{Olym.} & \textbf{AMC2023} & \textbf{AIME2024} \\
    \midrule
    Qwen-2.5-Math-7B-Instruct & 95.2 & 83.6 &41.6& 62.5 &16.7 \\
    Qwen-7B-CoT (SFT+RL) & 93.1 & 84.4  &	42.7 &	60.0 & 10.0 \\
    \midrule
    \textbf{Satori-Qwen-7B}  & 93.2 & 85.6  & 46.6  & 67.5  & 20.0 \\
    \bottomrule
  \end{tabular}
  \label{table:ablation-coat}
  \end{center}
\vspace{-1em}
\end{table}
We begin by conducting an ablation study to demonstrate the benefits of COAT reasoning compared to the classical CoT reasoning. Specifically, starting from the synthesis of demonstration trajectories in the format tuning stage, we ablate the ``reflect'' and  ``explore'' actions, retaining only the ``continue'' actions. Next, we maintain all other training settings, including the same amount of SFT and RL data and consistent hyper-parameters. This results in a typical CoT LLM (Qwen-7B-CoT) without self-reflection or self-exploration capabilities. As shown in Table~\ref{table:ablation-coat}, the performance of Qwen-7B-CoT is suboptimal compared to Satori-Qwen-7B and fails to surpass Qwen-2.5-Math-7B-Instruct, suggesting the advantages of COAT reasoning over CoT reasoning.



\paragraph{Satori Exhibits Self-correction Capability.}
% Please add the following required packages to your document preamble:
% \usepackage{multirow}
\begin{table}[h]
\scriptsize
\captionsetup{font=small}
\caption{\textbf{Satori's Self-correction Capability.} T$\rightarrow$F: negative self-correction; F$\rightarrow$T: positive self-correction.}
\setlength{\tabcolsep}{5pt}
\begin{tabular}{lcccccc}
\toprule
\multirow{3}{*}{\textbf{Model}} & \multicolumn{4}{c}{\textbf{In-Domain}}                                                                            & \multicolumn{2}{c}{\textbf{Out-of-Domain}}              \\ \cmidrule[0.2pt]{2-7} 
                                & \multicolumn{2}{c}{\textbf{MATH500}}                    & \multicolumn{2}{c}{\textbf{OlympiadBench}}              & \multicolumn{2}{c}{\textbf{MMLUProSTEM}}         \\
                                & \textbf{T$\rightarrow$F} & \textbf{F$\rightarrow$T} & \textbf{T$\rightarrow$F} & \textbf{F$\rightarrow$T} & \textbf{T$\rightarrow$F} & \textbf{F$\rightarrow$T} \\ \midrule[0.5pt]
Satori-Qwen-7B-FT                  & 79.4\%                    & 20.6\%                    & 65.6\%                    & 34.4\%                    & 59.2\%                    & 40.8\%                    \\
\textbf{Satori-Qwen-7B}                     & 39.0\%                       & 61.0\%                       & 42.1\%                    & 57.9\%                    & 46.5\%                    & 53.5\%                    \\ \bottomrule
\end{tabular}
\label{table:finegrain-reflect}
\end{table}
We observe that Satori frequently engages in self-reflection during the reasoning process (see demos in Section~\ref{sec:demo}), which occurs in two scenarios: (1) it triggers self-reflection at intermediate reasoning steps, and (2) after completing a problem, it initiates a second attempt through self-reflection. We focus on quantitatively evaluating Satori's self-correction capability in the second scenario. Specifically, we extract responses where the final answer before self-reflection differs from the answer after self-reflection. We then quantify the percentage of responses in which Satori's self-correction is positive (i.e., the solution is corrected from incorrect to correct) or negative (i.e., the solution changes from correct to incorrect). The evaluation results on in-domain datasets (MATH500 and Olympiad) and out-of-domain datasets (MMLUPro) are presented in Table~\ref{table:finegrain-reflect}. First, compared to Satori-Qwen-FT which lacks the RL training stage, Satori-Qwen demonstrates a significantly stronger self-correction capability. Second, we observe that this self-correction capability extends to out-of-domain tasks (MMLUProSTEM). These results suggest that RL plays a crucial role in enhancing the model's true reasoning capabilities.


\paragraph{RL Enables Satori with Test-time Scaling Behavior.}
\begin{figure}[h]
    \centering
    \includegraphics[width=0.5\textwidth]{Figures/rm_shaping_tot_len.pdf}
    \vspace{-2em}
\caption{\textbf{Policy Training Acc. \& Response length v.s. RL Train-time Compute.} Through RL training, Satori learns to improve its reasoning performance through longer thinking.}
\label{fig:test_time_scaling}
\end{figure}
\begin{figure}[h]
    \centering
    \includegraphics[width=0.45\textwidth]{Figures/length_across_levels.pdf}
    \vspace{-1.5em}
\caption{\textbf{Above: Test-time Response Length v.s. Problem Difficulty Level; Below: Test-time Accuracy v.s. Problem Difficulty Level.} Compared to FT model (Satori-Qwen-FT), Satori-Qwen uses more test-time compute to tackle more challenging problems.}
\label{fig:difficulty_level}
\vspace{-1em}
\end{figure}

Next, we aim to explain how reinforcement learning (RL) incentivizes Satori's autoregressive search capability. First, as shown in Figure~\ref{fig:test_time_scaling}, we observe that Satori consistently improves policy accuracy and increases the average length of generated tokens with more RL training-time compute. This suggests that Satori learns to allocate more time to reasoning, thereby solving problems more accurately. One interesting observation is that the response length first decreases from 0 to 200 steps and then increases. Upon a closer investigation of the model response, we observe that in the early stage, our model has not yet learned self-reflection capabilities. During this stage, RL optimization may prioritize the model to find a shot-cut solution without redundant reflection, leading to a temporary reduction in response length. However, in later stage, the model becomes increasingly good at using reflection to self-correct and find a better solution, leading to a longer response length.
 
Additionally, in Figure~\ref{fig:difficulty_level}, we evaluate Satori's test accuracy and response length on MATH datasets across different difficulty levels. Interestingly, through RL training, Satori naturally allocates more test-time compute to tackle more challenging problems, which leads to consistent performance improvements compared to the format-tuned (FT) model.



\paragraph{Large-scale FT v.s. Large-scale RL.}
\begin{table}[h]
  \begin{center}
  \scriptsize
  \captionsetup{font=small}
  \caption{\textbf{Large-scale FT V.S. Large-scale RL} Satori-Qwen (10K FT data + 300K RL data) outperforms same base model Qwen-2.5-Math-7B trained with 300K FT data (w/o RL) across all math and out-of-domain benchmarks.}
  \setlength{\tabcolsep}{1.15pt}
  \vspace{-0.5em}
\begin{tabular}{lccccc}
\toprule
\textbf{(In-domain)}   & \textbf{GSM8K}   & \textbf{MATH500} & \textbf{Olym.} & \textbf{AMC2023} & \textbf{AIME2024} \\ \midrule
Qwen-2.5-Math-7B-Instruct & 95.2 & 83.6                     & 41.6                  & 62.5             & 16.7                 \\
Satori-Qwen-7B-FT (300K)     & 92.3 & 78.2                       & 40.9           & 65.0               & 16.7              \\
\textbf{Satori-Qwen-7B}         & 93.2        & 85.6                     & 46.6           & 67.5             & 20.0                \\ \midrule
\textbf{(Out-of-domain)}  & \textbf{BGQA}    & \textbf{CRUX}  & \textbf{STGQA} & \textbf{TableBench}   & \textbf{STEM}     \\ \midrule
Qwen-2.5-Math-7B-Instruct & 51.3             & 28.0             & 85.3           & 36.3             & 45.2              \\
Satori-Qwen-7B-FT (300K)     & 50.5             & 29.5           & 74.0             & 35.0               & 47.8              \\
\textbf{Satori-Qwen-7B}               & 61.8             & 42.5           & 86.3           & 43.4             & 56.7              \\ \bottomrule
\end{tabular}
  \label{table:ablation-ft-rl}
  \end{center}
\end{table}
We investigate whether scaling up format tuning (FT) can achieve performance gains comparable to RL training. We conduct an ablation study using Qwen-2.5-Math-7B, trained with an equivalent amount of FT data (300K). As shown in Table~\ref{table:ablation-ft-rl}, on the math domain benchmarks, the model trained with large-scale FT (300K) fails to match the performance of the model trained with small-scale FT (10K) and large-scale RL (300K). Additionally, the large-scale FT model performs significantly worse on out-of-domain tasks, demonstrates RL’s advantage in generalization.


\paragraph{Distillation Enables Weak-to-Strong Generalization.} 
\begin{figure}[!t]
    \centering
     \includegraphics[width=0.4\textwidth]
     {Figures/distillation.pdf}
     \vspace{-1.5em}
\caption{\textbf{Format Tuning v.s. Distillation.} Distilling from a Stronger model (Satori-Qwen-7B) to weaker base models (Llama-8B and Granite-8B) are more effective than directly applying format tuning on weaker base models.}
\label{fig:distill}
\vspace{-1em}
\end{figure}
Finally, we investigate whether distilling a stronger reasoning model can enhance the reasoning performance of weaker base models. Specifically, we use Satori-Qwen-7B to generate 240K synthetic data to train weaker base models, Llama-3.1-8B and Granite-3.1-8B. For comparison, we also synthesize 240K FT data (following Section~\ref{subsec:format}) to train the same models. We evaluate the average test accuracy of these models across all math benchmark datasets, with the results presented in Figure~\ref{fig:distill}. The results show that the distilled models outperform the format-tuned models. 

This suggests a new, efficient approach to improve the reasoning capabilities of weaker base models: (1) train a strong reasoning model through small-scale
FT and large-scale RL (our Satori-Qwen-7B) and (2) distill the strong reasoning capabilities of the model into weaker base models. Since RL only requires answer labels as supervision, this approach introduces minimal costs for data synthesis, i.e., the costs induced by a multi-agent data synthesis framework or even more expensive human annotation.



The findings of the teacher interview were summarized in Tab.~\ref{tab:findings_in_study2_pre_class_interview}.
Based on these findings and the interview with senior students, we conclude the requirements for data preparation and classroom study design as follows.

\textbf{Providing analogies to teachers during lesson preparation.}
At the beginning of the interview, both teachers clearly stated that they often use analogies in class, with most being prepared in advance. 
The physics teacher (T1) frequently referred to analogies found in teaching aids. 
% \chirev{and rarely improvised simple everyday analogies in class.}
The biology teacher (T2) listed key knowledge points during lesson preparation and then considered suitable analogies, drawing on personal experience and input from other veteran teachers. 
% The physics teacher occasionally improvised analogies in class, using simple everyday examples to maintain student focus. 
Both teachers and students claimed that students rarely participate in the construction of analogies except in student-led discussion sessions.

% \uline{Based on this, we decided that the teacher would provide the key points of the lesson beforehand, and we would supplement the lesson preparation by providing analogies generated by LLMs.}


\textbf{Generating analogies based on subjects' characteristics and analogy needs.}
Two teachers demonstrated apparent differences in their need for and use of analogies. 
Based on their explanations, we attribute these differences to their subjects' characteristics rather than personal preferences. 
T1 frequently used analogies between concepts like ``electric field and magnetic field,'' noting the abstract nature of physics and the difficulty of finding everyday analogies. 
% As a result, T1 slightly deviated from the standard six-step teaching model, preferring analogies between concepts for effective instruction.
In contrast, T2 primarily employed interesting everyday life analogies, such as likening ``chromosome crossing over'' to ``swapping legs between classmates''. 
% Despite their unconventional nature, these analogies were effective.
% T2 noted that many biological structures and functions are unfamiliar to students, unlike the more common physical phenomena, making it difficult to use conceptual analogies to help them understand.
However, when presented with the analogy for ``blood sugar regulation'' generated in Study I, T2 suggested it could be analogized with ``thyroid hormone regulation'', as functions are related and thus easy for students to grasp. 

% \uline{
% Therefore, in Study II, we mainly generate analogies based on other physical concepts for physical concepts and analogies based on everyday life for biological concepts based on the teachers' needs.
% }

\textbf{Generating analogies for teaching key points and helping students focus.}
Both teachers stated that the primary goal of using analogies was to help students understand key concepts. 
Additionally, they emphasized that some analogies helped students maintain engagement. 
T1 mentioned, \textit{``When I notice students getting sleepy, I occasionally improvise an interesting analogy related to the concept to wake them up.''} 
T2 used images of people and mummies to explain the dry and fresh weight of cells, which are vivid and engaging without distracting students. 

% \uline{Based on this, we plan to generate analogies that effectively illustrate teaching concepts and are engaging while ensuring they accurately reflect some characteristics of the concepts rather than just capturing attention.}


\chirev{\textbf{Generating necessary analogies determined by teachers.}}
Both teachers acknowledged our generated vivid analogies in Study I. 
However, they criticized many of them as being overly complicated and unnecessary. 
For ``nuclear fission and fusion'' and ``auxin,'' T1 and T2 pointed out that students could quickly understand them through pictures and animations. 
Additionally, T1 mentioned that the ``molecular kinetic theory'' is relatively simple and not a key focus of exams, thus only requiring memorization.
T1 also stated that concepts in atomic physics, such as the ``photoelectric effect,'' are too isolated from other physical concepts to be conveyed through analogy. 
Additionally, students interviewed could not recall many concepts taught using analogies and viewed many analogies in Study I as redundant.

\chirev{\textbf{Generating non-complex analogies for certain aspects of the concept.}}
Both teachers emphasized the importance of analogizing only parts of a concept to keep it \chirev{correct} and easy to understand. 
\chirev{
T1 took the incorrect analogy of ``nuclear fission and fusion'' (Fig.~\ref{fig:example}) to illustrate LLMs' difficulty in generating correct physical analogies, noting that forcing analogies for all features leads to factual and semantic errors. 
He explained that physical concepts often involve multiple features, some of which, like ``chain reactions'', can be analogized (e.g., ``dominoes''), while others, such as ``mass-energy conversion'', are too abstract to find counterparts due to their basis in mathematical models.
}
\chirev{For biological analogies,} T2 recommended focusing on negative feedback in ``thyroid hormone regulation'' with an analogy like ``adjusting the temperature with an air conditioner remote control,'' while students should memorize other details. 
S2 recalled an analogy about specific details, in which the teacher compared a ``channel protein'' with a ``fire escape.''
Therefore, for complex concepts with multiple knowledge points, selecting only a specific aspect for the analogy is sufficient.

% \uline{Therefore, in Study II, we would ask teachers to provide the knowledge points they believe require analogies, as determining the necessity of analogies automatically is beyond the scope of this study. 
% For the knowledge points provided, we would generate various analogies focusing on the overall characteristics and specific details, respectively, and ensure that they are concise and easy to understand.}

\textbf{Not necessary to generate perfect analogies.}
Teachers were lenient towards the generated analogies from Study I and managed to extract effective parts from them. 
T2 appreciated the analogy comparing ``nerve impulses'' to the ``efficient operation of stations in an express delivery system,'' though some parts were redundant. 
Additionally, T1 shared his experience using ChatGPT for lesson plans, finding it repetitive and sometimes vague but useful for providing new ideas.


% \uline{Therefore, we do not aim to generate perfect analogies but rather strive to provide analogies that meet the above requirements, allowing teachers to refine and adapt them as needed.}

\textbf{Evaluating LLM-generated analogies in class by teachers.}
Teachers had various approaches to evaluating the effectiveness of analogies in class. 
T1 asked questions like ``Have you seen something similar before?'' or observed the students' expressions, while T2 had the students answer concepts-related questions during class. 
\chirev{Additionally, two teachers expressed cautious optimism about using LLM-generated analogies with their interventions. 
T1 noted that frequently used analogies for physics were concept-based, while AI-generated ones felt more relatable to everyday life, which makes him uncertain about their actual effects. 
Besides, both T1 and T2 anticipated better classroom feedback but were unsure of the effects on students' performance on homework and exams. 
This led to a consensus on conducting a comparative experiment.}
% They also understand student mastery through homework performance.


% \uline{To maintain consistency with their teaching styles, in Study II, we entrusted the method and analysis of evaluating the effectiveness of LLM-generated analogies to the teachers and did not conduct additional tests or evaluations.}

% \uline{\textbf{Summary to results of RQ2:}} 
% Physics teachers usually need analogies based on other concepts to express similar abstract features. 
% In contrast, biology teachers usually require analogies drawn from daily life to describe the structure and function of organisms. 
% Besides, LLM-generated analogies should reflect teachers' specific preferences and needs for which features and parts of the concepts to emphasize, often aligning with their key teaching focus.
% Additionally, teachers seek engaging analogies to capture students' attention.
% However, LLM-generated analogies do not need to be perfect, as teachers can refine them.

\enlargethispage{5pt}

\subsection{Classroom Experiments}
In this subsection, we describe the participants (Sec.~\ref{sec:study22_participants}), data preparation process (Sec.~\ref{sec:study22_data_preparation}), procedure (Sec.~\ref{sec:study21_findings_and_derived_requirments}), and results analysis (Sec.~\ref{sec:study22_results_analysis}) for our classroom experiments. 
\label{sec:study22}
\subsubsection{Participants}
\label{sec:study22_participants}
In this one-week field study, participants included two teachers (T1 and T2) from the pre-class interviews and two first-year high school classes (C1 and C2) they were teaching. 
Each class had 25 students, 12 of whom were girls, and the distribution of their entrance exam scores was very similar.

\subsubsection{Data Preparation}
\label{sec:study22_data_preparation}
Teachers informed us about concepts that might require analogies in the following week of teaching.
The concepts taught by the physics teacher (T1) include average velocity and instantaneous velocity, acceleration, and infinitesimal method.
The concepts taught by the biology teacher (T2) include the various functions of proteins, the adaptation of function and structure, dehydration condensation, the formation of tertiary and quaternary structures, and protein denaturation.
Based on pre-class interviews, we identify four effective strategies to generate analogies from LLMs for classroom practice.


\begin{itemize}
    \item \textbf{Strategy 1: Analogy for Physical Concept.} For physical concepts, analogies often draw on learned physical concepts. 
    % For example, comparing the structure of atoms to the solar system can aid in understanding their complex arrangement.
    \item \textbf{Strategy 2: Analogy for Biological Concept.} For biological concepts, analogies often involve everyday objects. For example, one might use the analogy of fire escape to help understand channel protein.
    \item \textbf{Strategy 3: Vivid Analogy Generation.} Analogies should be vivid and engaging to capture students' attention. 
    % For example, illustrating dry and fresh weight concepts in cells using images of people and mummies can be engaging.
    \item \textbf{Strategy 4: Fine-grained Analogy Generation.} Sometimes, it is sufficient to generate analogies for just one aspect of a concept to provide a detailed explanation of that particular aspect. 
    % For example, creating specific analogies for the tertiary structure of proteins is beneficial.
\end{itemize}

% \begin{tcolorbox}[title=Four Strategies for Generating Analogies for Teachers, mybox]
% \textbf{Strategy 1: Analogy for Physical Concept}

% For physical concepts, analogies typically draw on previously learned physical concepts. For example, comparing the structure of atoms to the solar system can aid in understanding their complex arrangement.

% \textbf{Strategy 2: Analogy for Biological Concept}

% For biological concepts, analogies often involve everyday objects. For example, one might use the analogy of fire escape to help understand channel protein.
% % For example, to help an engineer understand the eye's cross-section, one might use the analogy of a camera's structure.

% \textbf{Strategy 3: Vivid Analogy Generation}

% Analogies should be vivid and engaging to capture students' attention. For example, illustrating dry and fresh weight concepts in cells using images of people and mummies can be both vivid and engaging.

% \textbf{Strategy 4: Fine-grained Analogy Generation}

% Sometimes, it is sufficient to generate analogies for just one aspect of a concept to provide a detailed explanation of that particular aspect. For example, creating specific analogies for the tertiary structure of proteins is beneficial.
% \end{tcolorbox}

Based on the strategies outlined, we can modify the prompt in Tab.~\ref{tab:instruction_prompt} to suit the specific aspect of a concept and the requirements of teachers.
Specifically, we incorporated Strategies 1, 2, and 3 into the \textit{Principles}.
Strategy 4 was added into the \textit{Input Resource} to prevent the model from forgetting.

Following discussions with T1, we generated analogies for average and instantaneous velocity by implementing either strategy 1 or 3. 
Additionally, we generated detailed analogies for the infinitesimal method and acceleration using strategy 4. 
In biology, we produced analogies for proteins by applying either strategy 2 or strategy 3. 
Using strategy 4, we developed detailed analogies for the immune effects of proteins and the formation of tertiary and quaternary structures. 
However, two generated analogies, ``driving speed'' and ``reading speed'', were marked as non-analogies and excluded. 
Besides, the analogies generated with Strategy 1 for physical concepts were not related to other concepts, but we included them as they are vivid analogies.
We generated four physical analogies to T1 and nine biological ones to T2.



\subsubsection{Procedure}
\label{sec:study22_procedure}
In this one-week teaching, T1 and T2 used LLM-generated analogies for C1 and kept the original teaching mode for C2, with each class having 3 lessons for each subject, totaling 12 lessons.
One author attended one C1 lesson taught by T1 and one by T2, observing how teachers used analogies and student reactions without disrupting teaching.
For the remaining lessons within the week, teachers completed our provided record forms after each lesson.
The record forms asked for details on which analogies they chose while preparing for C1, any modifications made to these analogies, and reasons for not selecting others. 
Additionally, the forms inquired about how teachers assessed student feedback during or after class, any differences in feedback between C1 and C2, and whether the feedback met their expectations.
After one week, we conducted one-on-one interviews with T1 and T2, each lasting 20 minutes, to clarify any unclear details in the records, and discuss their experiences with LLM-generated analogies, students' performance, and future expectations.
Both teachers received a \$60 gift card each for their dedicated participation over the week.

\subsubsection{Results Analysis}
\label{sec:study22_results_analysis}

We report the following qualitative findings based on the record forms and interviews.


\textbf{Teachers selected and modified LLM-generated analogies to avoid redundancy, confusion, or misleading students and make them closer to students' daily lives.}
T1 selected two of four analogies and modified one, while T2 chose four of nine analogies and modified two.
In the interview, T2 noted that while the analogies for all four functions of proteins had merits, only two were selected to avoid verbosity in the class. 
Besides, to avoid concept confusion, T2 chose the analogy of ``transport function'' as a ``conveyor belt'' and discarded the analogy of ``catalysis function'' as a "high-speed elevator," due to the transport function of the elevator.
% For modification, \chirev{as shown in the Fig.~\ref{fig:behavior},} 
% T1 modified several analogies in physics that did not clearly distinguish between velocity and speed and explained the difference to clarify the key focus and avoid misleading.
\chirev{
We observed two types of modifications made by teachers to analogies. 
One type involved modifying details, such as T2 changing the security guard's action from ``eliminating'' to ``capturing'' to align with the real-world context \chirev{(Fig.~\ref{fig:behavior}B)}.
Another type involved changing analogy objects, like T1 replacing ``jigsaw puzzle'' with ``pixels on a display screen'' to illustrate the infinitesimal method \chirev{(Fig.~\ref{fig:behavior}C)}.
T1 explained that display screens are more familiar to students than jigsaw puzzles.}

\begin{figure*}[t]
    \centering
    \includegraphics[width=1\linewidth]{figure/revbehavior.pdf}
    \caption{\chirev{Teachers' behaviors on LLM-generated analogies. They may either directly select (A) and use them in class, modify their details (B) or analogy objects (C) to varying extents, or even create entirely new analogies inspired by them (D).}}
    \Description{This figure shows examples of three types of actions that teachers can take with LLM-generated analogies. For analogies of higher quality, such as analogizing ``Average Velocity and Instantaneous Velocity'' to ``videos and snapshots'', teachers directly select and use them in class. Teachers may also modify the details of the analogy, such as changing the security guard’s ``eliminating'' the intruder in the analogy of ``Immune Function of Protein'' to ``capturing and sending him to the police station''; or modify the analogy object, such as changing the ``puzzle and puzzle pieces'' in the analogy of ``Infinitesimal Method'' to the ``entire display and a single pixel''. Teachers may even be inspired by the generated analogies to get new analogies. For example, the generated analogy of ``Dehydration Condensation Reaction'' mentioned the building structure, which inspired the teacher to get a new analogy of ``breaking down the wall between classrooms''.}
    \label{fig:behavior}
\end{figure*}

\looseness-1  \textbf{LLM-generated analogies inspire teachers with new analogies and new teaching methods.}
In the interview, both teachers recognized the novelty of some LLM-generated analogies for concepts, and they had not considered using analogies for those concepts before. 
For example, T1 used ``video and snapshot'' to analogize ``average velocity and instantaneous velocity'' \chirev{(Fig.~\ref{fig:behavior}A)}, while T2 used ``wool folding and weaving'' to analogize the ``tertiary and quaternary structures of protein.''
In addition, T2 developed new analogies inspired by LLM-generated analogies. 
While teaching the ``dehydration condensation reaction,'' T2 explained with a new analogy \chirev{as ``breaking down the wall between classrooms'' (Fig.~\ref{fig:behavior}D).}
In the interview, T2 said, ``\textit{I am not satisfied with the generated one, as comparing the dehydration condensation reaction to mixing building materials doesn't capture the essence. However, \chirev{the buldings environment} inspired me to create a new analogy.}''
Besides, T1 said that participating in this study had changed his teaching style.
T1 used analogies based on everyday life after explaining the concept, which was inconsistent with his pre-class interview response. 
% In the interview, he explained, ``\textit{I didn't use analogies much before but recently I've been trying different ways to incorporate and teach with analogies.}''

\looseness-1  \textbf{LLM-generated analogies boost students' classroom and homework performance \chirev{and encourage teaching with analogy}.}
Both teachers believe that C1 outperforms C2 in both classroom participation and homework.
T1 praised analogies for helping students focus on the class: ``\textit{I can see from the students' eyes that C1 is genuinely paying closer attention, with more students nodding sincerely, rather than just pretending.}''
T1 also reported that C1 outscored C2 by nearly 20\% on a 10-question homework.
He attributed this to C2's confusion between average and instantaneous velocity, causing errors on the two hardest questions."
In the interview, T1 said, ``\textit{I plan to use the analogies in C1 when reviewing the assignments in C2 to explain the concepts again.}''
% In the interview, T1 said, ``\textit{I think there is a clear difference in concept understanding between the two classes. I plan to use the analogies in C1 when reviewing the assignments in C2 to explain the concepts again.}''
As for biological concepts, T2 said, ``\textit{When I explained protein structure using a video, C2 students understood initially but got confused about the tertiary structure, whereas the wool stacking analogy helped C1 students understand the video.}'' 
T2 showed us a fill-in-the-blank question from homework that asked students to summarize protein function. 
Most C1 students summarized correctly, while many C2 students simply copied words from the textbook. 
However, T2 noted that there was no clear difference between the two classes in understanding straightforward concepts like ``Protein denaturation''.
T1 also noted that the physics analogies are still not between concepts and may offer limited help with highly abstract concepts\chirev{, while he added ``\textit{But I'll try more teaching with analogy since the difference between the two classes is clear.''}}

\chirev{Overall, promising feedback from teachers and classroom practice led us to consider designing a practical system to support the preparation of teaching analogies.}
% \chirev{
% Nevertheless, the comparative experiment results motivated the teachers to use analogies more often in future teaching.}

% \uline{\textbf{Summary to results of RQ3.}}
% Teachers are willing to select and modify LLM-generated analogies to ensure they are vivid and engaging, promoting understanding without adding redundancy, confusion, or misleading students. 
% The use of LLM-generated analogies helps maintain student focus, enhances classroom performance and homework outcomes, and inspires teachers to create new analogies and reflect on their teaching methods.
% \chirev{Such effectiveness suggests us to further develop an LLM-assisted system to help teachers construct analogies for classroom teaching. }

\enlargethispage{5pt}





%noting that while AI often provides repetitive content and fails to address specific needs directly, it can still offer ideas that he might not have considered, which helps him with lesson preparation.
%We will also try to generate relatively less used analogy types for teachers' reference.
%However, we also attempt to generate 
%- When analogies are made to biological concepts, they are usually made using everyday objects. For example, an engineer can learn the eye cross-section by taking the analogy of the camera structure.
%- When analogies are made with physical concepts, previously learned physical concepts are usually used. For example, utilizing the analogy of the solar system can facilitate comprehension of the complex structure of atoms.
%The physics teacher (T1) tended to use analogies between physical concepts, such as ``electric field and magnetic field'' or ``Coulomb's law and the law of universal gravitation.'' 
%T1 claimed that physical concepts are often abstract and it can be challenging to find relevant analogies from everyday life. 
%As a result, T1 did not fully align with the six-step teaching model, believing that using analogies highly similar to the objects of concepts themselves was sufficient for physics teaching.
% , drawing on discussions from Steps 1 and 2 and their previous AI experiences in Step 3, along with suggestions for enhancing current analogies.
%We expect participants to explain their needs for analogies used in class from both the teacher and student perspectives.
% Study I has shown that LLM-generated analogies have a \todo effect on students’ understanding of concepts without human intervention.
% We continue to design Study II to first identify the requirements of analogies used in classroom instruction through preclass interviews with teachers and senior students and refine prompting strategies to fit the requirements.
%Then we conduct a field study to evaluate the effectiveness of LLM-generated analogies during classroom teaching.
%In contrast, the biology teacher (T2) almost exclusively used examples from daily life. 
%T2 reported comparing the crossing over of chromosomes to swapping legs between classmates. 
%While some analogies might seem unconventional, they proved effective in teaching.
%- The analogies created should be vivid and engaging to enhance student attention in class. For example, using images of people and mummies to explain the concepts of dry weight and fresh weight of cells can be both vivid and engaging.
%Therefore, when a concept is relatively complex and contains multiple knowledge points, it is sufficient to select only the most important overall feature or a specific aspect for the analogy, rather than trying to cover the entire content.
%Similarly, the two students could not recall many concepts being taught using analogies during interviews. 
%After reviewing the analogies we generated in Study I, they became even more convinced that these analogies were unnecessary for teaching those concepts.
%T2 believed that the teaching focus for ``thyroid hormone regulation'' should be on negative feedback, which can be explained using the analogy of ``adjusting the temperature with an air conditioner remote control,'' while other specific details should be memorized by students. 
%For example, T2 found the analogy comparing ``nerve impulses'' to the ``efficient operation of stations in an express delivery system'' to be unique; 
%while some parts were redundant, one specific paragraph could be adapted for classroom use. 
%Consequently, our discussions with them did not delve into AI usage exploration (Step 3 in teacher interviews).
%Following a similar procedure to Steps 1, 2, and 4 of the teacher interviews, the student sessions focused on their classroom experiences. 
%Initially, students were asked to recall instances in which analogies were used in class, helping to guide analogy usage. 
%Subsequently, we presented all ten concepts from Study I, prompting students to recall how they learned these concepts during class teaching, specifically if analogies were used. 
%Following this, we showcased the ten analogies generated in Study I and asked students to assess the strengths and weaknesses of these from a learned perspective.
%Finally, they shared their opinion on whether viewing the analogies contributed to a better understanding of the concepts.

\chirev{
\section{System}
\label{sec:system}
% In this section, we designed an LLM-assisted system for teachers and conducted a system evaluation to understand the practical effectiveness of LLM in helping analogical teaching.
In this section, we transformed key study findings into an LLM-assisted system for teachers and conducted a system evaluation, highlighting its contribution to teaching by analogy in education.

\subsection{System Design}
% We first confirmed the necessity and functionality of the system by interviewing T1 and T2 and identified design requirements based on interview results and study findings. 
% We then designed a system for teachers to create analogies.
This subsection details the interview for deriving design requirements and system workflow as below.

\subsubsection{Interview}
Given that Study II showed teachers prepared analogies during lesson planning, we first determined the system's necessity and functionality through 20-minute one-on-one online interviews with T1 and T2 via Tencent Meeting. 
The interview mainly consists of two questions. 1) Necessity: Is providing an LLM-assisted analogy generation system necessary for teachers to prepare lessons? 2) Functionality: What functions do they expect?
% Should the system focus on generating analogies for specific concepts, or expand its functionality to create complete lesson plans that integrate analogies?

Both teachers affirmed the first question, expressing a desire to operate the system themselves to gain hands-on experience and long-term support from LLMs.
Regarding the second question, both teachers expressed a preference for the analogy generation mode in Study II. 
They suggested that after specifying the required concepts, the system should generate accurate analogies tailored to their needs, allowing for refinement and management. 
They also noted that in Study II, analogies function as plug-and-play modules to replace or enhance original explanations of scientific concepts, so the generation process need not account for other lesson plan content at this stage.
% For the second question, both teachers noted that in Study II, they only replaced or enhanced their original explanations of scientific concepts with LLM-generated analogies, without altering their lesson plans. 
% Analogies were treated as plug-and-play modules with minimal interaction with the overall plans. 
% Additionally, both emphasized that lesson plans reflect their teaching expertise and core competitiveness, and they currently see no need for LLM-generated lesson plans with analogies.


\subsubsection{Design Requirements}
After confirming the necessity and functionality, we identified three design requirements as below and confirmed them with T1 and T2.

\textbf{R1: It should incorporate principles and strategies identified in previous studies to help generate accurate analogies tailored to teacher needs.}
The general principles identified in Study I (Tab.~\ref{tab:instruction_prompt}) should be integrated into the prompt by default to enhance the accuracy.
The system should allow teachers to select useful prompting strategies identified in Study II and incorporate them into generation following the practice of data preparation of Study II (Sec.~\ref{sec:study22_data_preparation}) to better tailor analogies to their needs.

\textbf{R2: It should enable teachers to input their expectations or automatically generate personalized principles for creating analogies.}
As identified in Study II interviews (Sec.~\ref{sec:study21_findings_and_derived_requirments}), the workflow should allow teachers to input concepts they believe require analogies to be generated.
Besides, to ensure personalized needs, it should also accept user-inputted new principles tailored to their needs and even automatically generate tailored principles from user comments on generated analogies for the prompt (Tab.~\ref{tab:instruction_prompt}).

\textbf{R3: It should allow users to make manual changes and feedback to manage their analogies.}
In Study II, teachers showed a strong willingness and ability to refine and manage generated analogies (Sec.~\ref{sec:study22_results_analysis}), highlighting the need for a system that allows direct editing.
As prompt evolution in Study I showed limited improvement (Tab.~\ref{tab:error_rates}), and teachers found conversational refinement challenging in pre-class interviews in Study II (Tab.~\ref{tab:findings_in_study2_pre_class_interview}), conversational modifications by LLMs are unnecessary.
The system should also enable teachers to manage analogies by classifying them into four categories (useful, inspiring, refinable, and useless) and storing all except the useless ones.


% Instead, the system might summarize personalized principles from teacher feedback to better tailor analogy generation to their specific needs.


% \textbf{It should focus on assisting with analogy preparation rather than generating entire lesson plans.} 
% Given that Study II showed teachers prepared analogies during lesson planning, we asked T1 and T2 about the link between lesson plans and analogies and the need for automatically generating lesson plans with analogies.
% Both teachers noted that in Study II, they only replaced or enhanced their original explanations of scientific concepts with LLM-generated analogies, without altering their lesson plans. 
% Analogies were treated as plug-and-play modules with minimal interaction with the overall plans. 
% Additionally, both emphasized that lesson plans reflect their teaching expertise and core competitiveness, and they currently see no need for LLM-generated lesson plans with analogies.
% \szk{add the corresponding discussion about LLM4lessonplanning.}



\subsubsection{System Workflow}
\label{sec:workflow}

\begin{figure*}[t]
    \centering
    \includegraphics[width=1\textwidth]{figure/revsystem.pdf}
    \caption{\chirev{Our system interface (top) and workflow (bottom). In each round, teachers use Configuration Panel (A) to select strategies (A1) and manage principles (A2) for generation. After entering scientific concepts (B1), they provide feedback (B2), edit (B3), and comment (B4) to each generated analogy in Generation Panel (B). Clicking the ``Save'' button makes the system generate new principles by LLM in Configuration Panel and store approved analogies in Library Panel (C), where teachers may export saved analogies.}}
    \label{fig:system}
    \Description{This figure shows our system interface which allows teachers to interact with three main panels. In Configuration Panel, teachers select strategies and manage principles for analogy generation. They then enter scientific concepts, provide feedback, make manual editing, and add comments to the generated analogies in Generation Panel. Clicking the ``Save'' button enables the system to generate new principles in Configuration Panel and store approved analogies in Library Panel, where all saved analogies can be exported at any time.
    }
\vspace*{-10pt}
\end{figure*}

Based on the design requirements, we built an LLM-assisted system (Fig.~\ref{fig:system}) for teachers to create and refine analogies for teaching. 
Teachers begin by registering an account and follow a workflow as below.

\textbf{Configure strategies and principles for analogy generation.} 
After selecting a teaching subject during registration, the system provides prompting strategies in Configuration Panel (Fig.~\ref{fig:system}A) based on the chosen subject. 
Teachers can click on these strategies for generating analogies (\textbf{R1}).
The principle list starts blank, allowing teachers to manually add or select principles as needed (\textbf{R2}). 
When the user hovers over a principle, they could edit or delete it freely.

\textbf{Generate analogies and provide feedback}. Teachers then input a scientific concept in Generation Panel (Fig.~\ref{fig:system}B) and click the ``Generate'' button (\textbf{R2}). 
The system will provide three cards with generated analogies. 
Each card has four feedback options in the top-right corner: useful, inspiring, refinable, and useless. 
Teachers need to select one feedback option for each analogy card and may edit analogies as needed and provide text comments (\textbf{R3}).

\textbf{Save analogies, optionally generate new principles, and restart.} When saving, modified analogies (excluding those marked as useless) are stored in Library Panel (Fig.~\ref{fig:system}C), along with the corresponding concept (\textbf{R3}). 
If teachers enable the ``Comment-based summarization'' feature in Configuration Panel (Fig.~\ref{fig:system}A), the system will summarize new principles from comments on the three analogies by LLMs and add them to the principles list (\textbf{R2}). 
Teachers can edit or delete any generated principles or disable the automatic summarization feature at any time.
After saving, the system clears the concepts and analogies in Generation Panel (Fig.~\ref{fig:system}B), allowing users to restart. 
Teachers can reconfigure the strategy list, and add, delete, select, or modify the principle list for the new round.

Besides, teachers click right buttons of text for strategies (Fig.~\ref{fig:system}A1), principles (Fig.~\ref{fig:system}A2), and approved analogies (Fig.~\ref{fig:system}C) to view or collapse text, reducing clutter. 
All approved analogies stored in Library Panel can be exported to PDF for teachers' usage.
% Users' interaction logs will be recorded, including modifications and feedback to analogies and modifications and selections of principles.


We implemented the system as a Vue-based web application with a Python Flask backend. More implementation details, such as prompts for automatically generated principles, are provided in the supplementary material.


\subsection{System Evaluation}
This subsection details the participants, procedure, and findings of the system evaluation.
\subsubsection{Participants}
We invited 6 high school physics and biology teachers from 5 schools, including T1 and T2, along with 4 new teachers from different schools to increase diversity. 
The physics teachers had 7 years (T1), 3 years (T3), and 4 years (T5) of experience, while the biology teachers had 4 years (T2), 2 years (T4), and 25 years (T6) of experience. All participants received a \$20 gift card as compensation.

\subsubsection{Procedure}
Our study consists of a tutorial, free exploration, one-week usage, and an interview.

\textbf{Tutorial.} We conducted one-on-one online interviews with each teacher other than T1 and T2 via Tencent Meeting, lasting 20 minutes. 
We briefly revisited Steps 1 and 3 from Study II (Sec.~\ref{sec:study21}) to understand their experience with analogical teaching and AI, and to ensure they were familiar with the study background.

\textbf{Free exploration.} Following the tutorial, we continued the interview with each teacher to demonstrated the basic functionality of the system according to the workflow, after which the teachers were encouraged to explore the system freely. 
During this exploration phase, we prompted them to think aloud and we answered any questions they had to make sure they understand how to use the system. 
This process lasted about 20 minutes. 

\textbf{One-week usage.} Before using the system, we informed the teachers that it would collect data on the analogies and principles they generated, as well as their feedback, manual edits, comments, and other interaction data for research purposes. 
All of them provided informed consent. 
We then asked the teachers to use the system to generate analogies to support their lesson preparation for one week.
The frequency of system use and the content of lesson preparation were determined by the teachers themselves, although we encouraged them to use the system frequently.


\textbf{Interview.}
After one-week usage, we conducted one-on-one semi-structured online interviews with each teacher via Tencent Meeting, lasting nearly 40 minutes each. 
During the interviews, we asked each teacher the following questions: course coverage (Q1), system usability (Q2), satisfaction with the generated analogies (Q3), satisfaction with the generated principles (Q4), satisfaction with the edited analogies (Q5), satisfaction with the edited principles (Q6), and the significance of the system for their future lesson preparation and teaching (Q7).

% \subsubsection{Data Analysis}

% \textbf{Quantitative analysis of user logs.}

% \textbf{Qualitative analysis of interviews.}

\enlargethispage{10pt}


\subsubsection{Findings}
\label{sec:system_findings}
Based on user interaction data and interview results of Q1-Q7, we summarize the following findings.

\textbf{The system's usability and the usefulness of generated analogies were well-received by teachers (Q1-Q3).} 
Teachers reported that they used the system to generate analogies for lessons spanning from one month to half a semester, aiding both reflection on past lessons and preparation for upcoming ones. 
All teachers agreed that the system was easy to use, even if they had nearly no experience using AI (T3, T5, T6).
They generated 15 to 42 analogies (15, 15, 21, 24, 24, 42), with 40\% to 80\% marked as non-useless (``useful'', ``refinable'', ``inspiring'') across users (40\%, 58.3\%, 61.9\%, 64.3\%, 66.7\%, 80\%). 
They all agreed that although many of the generated analogies had various issues, overall, they helped expand their lesson planning ideas and inspired greater use of analogies in teaching.
For example, the system analogizes ``phase difference'' to ``some students doing radio gymnastics faster or slower than classmates.'' T3 said, ``\textit{I hadn’t thought of that, but it’s great, and since my class is right after gymnastics session, I’ll definitely use it.}''
Besides, physics teachers noted that they understood the system’s inability to provide suitable analogies for some complex concepts, as such analogies might not exist anyway.
% Instead of getting stuck, they pragmatically explored analogies for other concepts and got lots of help.
% For example, T3 noted that while the continuous generation of useless analogies could be frustrating, it was often due to the lack of suitable analogies for certain concepts, so he would try other concepts instead of getting stuck.


\textbf{Teachers tried to improve analogy quality by incorporating generated principles or directly regenerating analogies (Q4).} 
Five teachers (excluding T4) continuously enabled the automatic principle generation feature and actively input their comments, producing 10 to 30 principles (10, 13, 17, 18, 30) and incorporating at least half into analogy generation.
They all praised the quality of the principles, with T1 saying, ``\textit{The system infers general principles from my vague intents on comments on specific analogies.}''
For usability of this feature, T3 noted, ``\textit{It doesn’t matter if too many principles are generated during the usage. I just delete the redundant ones—it’s more convenient than summarizing myself.}'' 
For its effectiveness in analogy generation, T1 and T5 felt the principles improved analogy quality, while T3 and T6 were unsure but still incorporated them because ``\textit{adding more correct terms can't hurt.}''
On the contrary, T4 preferred regenerating analogies directly without commenting on analogies and generating principles, saying, ``\textit{If the analogies aren't good, I just regenerate them a few more times as I know the randomness of AI.}''
Only T4 manually added principles at the beginning of usage, while all teachers found it inconvenient and difficult as noted by T3 and T1.

\textbf{The generated analogies and principles benefit teachers more than just concept explanations (Q7).} 
1) The principles help shape teaching expertise. 
T1 said, ``\textit{I can learn teaching techniques from the generated principles, and after using the system for a while, I could even write a teaching paper. It’s like having a discussion about teaching with another experienced teacher. While the principles may not immediately impact teaching, the long-term accumulation is valuable.}'' 
2) The analogies supplement the teacher's knowledge base. 
T6 noted that the generated analogies broaden a teacher's perspective, and saving more analogies gradually builds their teaching knowledge system, which help teachers adapt to different teaching situations. 
3) The analogies also inspire quiz and test creation.
T2 said, ``\textit{Even if some analogies aren't ideal, I save them because having students identify errors helps their learning.}'' T3 stated, ``\textit{Some of the generated `analogies' is example-based explanation, but I save it because these real-life examples can be used to set questions.}''


\textbf{Teachers typically organize analogies externally rather than refining them within the system and there is potential over-reliance (Q5-Q6).}
Despite marking large proportion of generated analogies as ``refinable'' (Mean = 21.4\%) and ``inspiring'' (Mean = 13.8\%), only T1 and T4 edited 1 analogy in the system, respectively. 
T6 explained, ``\textit{Manually modifying so much text is too burdensome for older teachers.}'' 
However, all teachers reported improving analogies to fit their needs through external modifications.
T5 noted that his lesson preparation habit is to record keywords in electronic notes, so the analogies in the system only serve as explanation and inspiration, which he then reorganizes in his notes.
Similarly, T1 and T6 preferred recording analogies in paper notebooks. 
This separation between analogy generation and actual lesson preparation may make it difficult to supervise teachers’ behaviors in teaching and potentially lead to teachers' over-reliance on generated analogies.
This issue may be more pronounced for users like T4, who prefer to regenerate analogies directly without providing any comment, compared to teachers who actively engage by entering comments in the system.

}

\section{Discussion}
\label{sec:discussion}
\chirev{This section discusses consideration, opportunities, and future research directions for LLM-assisted analogical education based on our study results and designed system.}
% We conducted in-class tests and classroom studies to evaluate the effectiveness of LLM-generated analogies on students' understanding and classroom practice. 
% Our findings suggest that, without teacher intervention, \textbf{1)} LLM-generated analogies generally help in problem-solving, especially for biological concepts; \textbf{2)} however, errors and missing information in these analogies can negatively affect understanding, and \textbf{3)} students may overestimate their comprehension due to overconfidence. 
% With teacher intervention, we found that \textbf{4)} teachers' demand for analogies relates to the characteristics of the subject, and that \textbf{5)} effective analogies should align with teaching focus and preferences. \textbf{6)} Teachers are willing to refine analogies during lesson preparation to avoid negative effects, and \textbf{7)} LLM-generated analogies can inspire them to develop new analogies and explore new teaching methods, and \textbf{8)} boost students' classroom engagement and homework performance.
% Based on these insights, we discuss the implications for the future development and evaluation of LLM-assisted education and LLM-generated analogies and our study's limitations.

\chifinal{
\subsection{Subject Differences in LLM-Generated Analogy Effectiveness}
% Study I showed that while LLMs generally produced correct analogies in biology, their physics analogies were often incorrect, as none of the analogies for the five physics concepts in Sec.~\ref{sec:study1_data_preparation} were classified as correct and satisfying, displaying various factual and consistency errors. In Study II, these issues were mitigated by prompting LLMs with our summarized strategies (Sec.~\ref{sec:study22_data_preparation}) to draw on other learned physical concepts and focus on a single aspect of physical concepts when generating analogies. 
% Although physics teachers were willing to refine and use these analogies in the classroom, which helped students stay focused and improve their homework performance, they still commented on many of the analogies as ``interesting but unprofessional.'' 

Our study shows that LLMs generally produce correct and satisfying analogies for biological concepts but generate incorrect or correct yet unprofessional ones for physics.

Several factors appear to contribute to these shortcomings. 
First, physical concepts are highly abstract (e.g., ``mass-energy conversion''), with complex and formula-driven features, making it difficult to find real-life analogies or other concepts that perfectly align with them. 
In contrast, biological concepts are more concrete and observable, often tied to specific structures and functions (e.g., ``mitochondria as the powerhouse of the cell'').
As a result, LLMs would produce forced analogies or oversimplifications for physics while generating satisfying ones for biology.
Second, restricting the physics analogy to a single aspect, as in the strategies used in Study II (Sec.~\ref{sec:study22_data_preparation}), can yield correct and engaging analogies.
However, for highly abstract concepts, these analogies may still be superficial and offer limited support, as noted by physics teachers in Sec.~\ref{sec:study22_results_analysis}.
Third, teaching materials in physics contain more formulas and fewer analogies compared to biology. As a result, LLMs learn fewer physics analogies and generate less effective analogies.

These findings suggest that using LLMs for educational analogy generation is tied to subject characteristics, and we can infer that it may be particularly challenging for subjects lacking clear real-world counterparts (e.g., mathematics). 
In contrast, they might work better for subjects with more directly observable phenomena  (e.g., high school chemistry, biology), which should be confirmed by future studies. 
Nonetheless, analogies help students engage with abstract subjects like physics and math by inspiring interest and sustaining attention. 
More studies are needed to verify the effectiveness and needs of LLM-generated analogies across a broader range of subjects, in conjunction with the review and refinement of teachers before their use.
}

\chirev{
\subsection{Generating High-Quality Analogies}
%The current analogies generated automatically have limitations in scientific accuracy and educational effectiveness. 
The automatically generated analogies have limitations in scientific accuracy and educational effectiveness. 
In our practical system, LLMs summarized principles from human feedback and incorporated them into the next round of analogy generation.
Several teachers in the system evaluation found this approach effective for improving analogy quality.
Building on this, we can incorporate Reinforcement Learning with Human Feedback (RLHF). 
Using teachers' feedback and preferences, we can create a reward model that continually refines the analogies.
%In addition to human-in-the-loop generation, future work should explore automatic methods for generating higher-quality analogies to reduce the burden on teachers. 
To further reduce teachers' workload, future work should explore automatic methods for generating higher-quality analogies.
\chifinal{
First, we could explore multi-agent collaboration methods to further mitigate hallucinations~\cite{SHI2025125723}, including factuality errors and consistency errors, as outlined in Study I.
Besides}, instead of waiting for more advanced general-purpose LLMs to be released~\cite{openai_o1_preview_2024}, we can fine-tune existing models with teacher-adjusted analogies.
Additionally, our system can be transformed into a labeling tool to collect high-quality educational analogy datasets, consisting of revised analogies or new analogies proposed by teachers.
Another bottleneck for generating high-quality analogies is the LLMs' limited domain-specific knowledge. 
In Study II, the physics teacher attributed the current analogies' interesting yet unprofessional nature to the LLMs' limited understanding of abstract physics concepts.
% As a result, LLMs oversimplify the concepts or reinforce incorrect analogies. 
%Future work should focus on fine-tuning LLMs in specific domains and subjects to improve LLMs' understanding of concepts and thus enhance analogy quality.
To address this, we should fine-tune LLMs for specific subjects, improving their understanding and enhancing analogy quality.
Finally, \chifinal{to improve control over the complexity and ethical considerations and make analogies suited for the intended educational level and scenario,} the future analogy generation pipeline should consider factors like students' educational background, cultural context, and prior knowledge.
}


\chirev{
\subsection{LLMs \chifinal{for} Teaching by Analogy}
% \textbf{Practical Benefits.} 
Our Study II and system evaluation show that LLM-generated analogies are valuable to teachers in three progressive aspects: short-term lesson preparation, teaching strategy development, and professional growth. 
For short-term lesson preparation, teachers are able and willing to select and modify LLM-generated analogies or inspire new ones to suit specific concepts and lessons (Sec.~\ref{sec:system_findings}), or use generated content to help set quizzes (Sec.~\ref{sec:study22_results_analysis}). 
For teaching strategy development, continuous use of LLM-generated analogies leads to positive feedback from the classroom and students' homework and iteratively encourages teaching by analogy (Sec.~\ref{sec:study22_results_analysis}). 
Regarding professional growth (Sec.~\ref{sec:system_findings}), providing feedback on LLM-generated analogies helps teachers actively reflect on teaching points and build their knowledge base, while LLM-generated principles based on their feedback also serve as valuable reminders for teachers, supporting their ongoing professional development and enhancing teaching expertise.
Given these benefits, future work should explore the varying needs and develop practical systems to benefit teachers with different experience levels and subjects.
In addition, long-term evaluation of such practical systems and teachers is needed to fully understand the actual benefits.

% \textbf{Potential over-reliance.} 
Besides the benefits, over-reliance on LLM-generated analogies warrants attention. 
In Study II, teachers emphasized during pre-class interviews and demonstrated in classroom teaching that they could avoid over-relying on such content.
However, in the system evaluation, most teachers did not revise analogies within the system but recorded changes elsewhere, following their own lesson preparation habits. 
This suggests that monitoring teachers' interactions with analogies only through system logs might be insufficient, potentially allowing unnoticed over-reliance to develop.
To address this, the system could use pop-up reminders to alert users against over-reliance and encourage them to edit or provide comments when no manual interaction is detected for an extended period.
% Teachers could be encouraged to provide comments or make changes within the system, but this should not be mandatory, as we need to respect teachers' lesson preparation habits.
% In particular, neither Study II nor the system evaluation included novice teachers (e.g., Teaching Assistants), who may lack well-established lesson preparation habits and are more likely to rely on LLM-generated analogies without proper guidance.
Nevertheless, supervision from schools and higher authorities is essential. Additionally, regular updates from system developers to educators~\cite{tan_more_2024,kasneci2023chatgpt} are crucial for maintaining an accurate understanding of model capabilities and ensuring effective use.
% In Study II pre-class interviews, teachers noted that their teaching experience enabled them to use AI-generated contents as references or supplements, maintaining the ability to select, modify, and innovate without over-reliance on AI. 
% They suggested that inexperienced teachers should first build lesson preparation habits through their own thinking or guidance from experienced teachers, rather than relying on AI-generated analogies, which might hinder critical thinking.
% Hence, the use of LLMs to generate analogies and other teaching resources should be closely monitored by schools and senior teachers. 
}

\chirev{
\subsection{\chifinal{Integrating Analogies into LLM-Assisted Education Platforms}}
% \subsection{LLMs \chifinal{for Self-Learning} by Analogy}
\label{sec: discussion_student}

\chifinal{
Integrating analogy generation into LLM-assisted education platforms might benefit teachers and students.

For teachers, the interplay between analogy generation and LLM-assisted teaching material preparation is mutually reinforcing.
First, analogies help develop teaching materials by providing relatable explanations.
For example, LLM-assisted platforms already help novice teachers generate lesson plans\chirev{~\cite{lessonplanner2024uist}}. 
Integrating analogy generation into these platforms can support analogy-based explanations at different teaching stages.
Additionally, participants in our system evaluation demonstrated the potential of using analogies in quiz settings, highlighting its role in automated problem generation.
Second, existing LLM-assisted teaching preparation platforms enhance context-aware analogy generation and verification. 
These platforms usually consider students’ knowledge levels and course context~\cite{GeneratingContextualized}, which could be integrated into analogy generation pipelines to help generate analogies suited for practical scenarios. 
Moreover, existing problem generation platforms could be enhanced to generate quizzes that verify students' understanding of analogy, reinforcing the effectiveness of teaching by analogy.
}

\chifinal{However, the} results of Study I indicate that LLM-generated analogies without human intervention are unreliable for students, making it premature to directly integrate analogy generation into self-learning systems. 
In contrast, teacher-adjusted analogies in Study II ensured correctness and reliably impacted students' classroom feedback and homework performance. 
Therefore, future self-learning systems should only consider pre-set teacher-reviewed analogies for key knowledge points to aid student understanding. 
However, even correct biological analogies from Study I led to negative effects, with students over-relying on them with incorrect learning strategies or becoming subjectively overconfident.
This suggests that the self-learning system should flexibly structure the learning process and, after presenting analogies, guide students back to the textbook with more detailed follow-up questions.
Besides, timely pre-set and teacher-reviewed exercises with feedback and explanation that reveal the limitations of analogies help students reflect on their learning and monitor their learning approaches.
Overall, incorporating analogies into self-learning systems requires careful attention from teachers and system developers to mitigate potential negative effects.
}

% \subsection{Towards LLM-assisted Education Tools Creation with Generated Analogy}
% First, our prompt instructions and principles, refined through multiple iterations in both studies, should apply to future LLM-assisted educational tools that use analogies. 
% Results from Study I indicate that while LLMs can generate satisfactory analogies for biological concepts, analogies for physical concepts may not be effective. 
% Moreover, even with biological concepts, students may over-rely on analogies or overestimate their understanding. 
% Therefore, in LLM-supported self-learning tools without human guidance, for analogies of abstract concepts represented by physics, it require more effort to deconstruct key knowledge points effectively. 
% For analogies of concrete concepts represented by biology, timely tests are needed to assess student understanding, prevent negative effects, and avoid overconfidence.
% Study II shows that teachers can effectively use LLM-generated analogies tailored to their needs to enhance classroom teaching. 
% Thus, tools designed to support teachers should understand and integrate teachers' needs as we did in our pre-class interviews, and generate diverse analogies to inspire lesson planning.
% With the rapid advancement of LLMs, more powerful models will be released~\cite{openai_o1_preview_2024} and able to generate analogies with fewer errors. 
% However, our method, particularly the in-classroom study, offers a complete example and inspiration for future exploration with new LLMs. 
% The principles and requirements summarized in our study also guide future LLMs in generating educational analogies.
% % However, generating analogies that match each abstract feature of scientific concepts is challenging, as some may lack real-life counterparts, even with improved model capabilities.
% % Therefore, it also necessitates careful consideration of specific aspects of analogy generation and requires the involvement and guidance of teachers, as illustrated in our Study II.

% \subsection{Evaluating LLM-assisted Education Beyond Generating Analogies}
% \chirev{Our study focuses on analogies, but the study design can also be adapted to evaluate similar features, such as LLM-generated example-based explanations for physics concepts in Study I.
% Students may complete in-class tests with example-based explanations, followed by teachers using them in the classroom.
% Besides, various features might be integrated into LLM-assisted educational systems that support teachers or students in the future, such as LLM-generated analogies, examples, quizzes~\cite{readingquizmaker}, and lesson plans for novice teachers~\cite{lessonplanner2024uist}.
% It would be important to assess the interplay of analogies and other LLM-assisted features within systems and study their long-term educational impact~\cite{Lyu2024evaluating,chen2024stugptviz}.}
% % For teachers, we aim to understand how they use AI to create analogies linked to other lesson materials, either through in-depth interviews~\cite{tan_more_2024} or by analyzing conversational logs between teachers and AI.

% \chifinal{
% \subsection{Integration of Analogies with LLM-based Educational Tools}

% }


\subsection{Evaluating Broader Analogies in Education}
Analogies serve multiple purposes beyond students' understanding \chirev{and teachers' teaching}, which adds challenges to their evaluation.
In mathematical problems and similar domains, specific procedures involving numeracy and variables often require a different type of analogy, known as procedural analogy~\cite{richland_analogy_2004}. 
Such procedural analogies were also mentioned during our interview with the physics teacher in Study II. 
Due to their rarity and complexity, these analogies, even those crafted by humans, have not been thoroughly evaluated.
LLMs can lower the barrier to creating such analogies due to advanced reasoning abilities and broad subject knowledge, such as linking the formula for a ``spring oscillator'' with that of a ``pendulum.'' 
Our study design can be extended to such analogies by involving calculation questions involving formulas in controlled in-class tests.
Additionally, analogies are utilized for socialization, helping to educate children on becoming better students and enacting behavioral changes~\cite{richland_analogy_2004}.
The evaluation of such analogies involves contexts beyond the classroom, which brings challenges to study design and needs to be explored in the future.
% Analogies also play a significant role for teachers. 
% Using analogies can exemplify teaching models and effective cases can enhance new teachers’ instructional strategies~\cite{oliva_teaching_2007}. 
% In Study II, we also found that LLM-generated analogies prompted teachers to rethink their teaching methods, including the frequency, type, and timing of analogy use. 
% However, a more comprehensive and long-term evaluation is necessary to fully evaluate such impact.

\subsection{Limitation}
Although we have gained lots of evidence and knowledge, our work is limited by student and teacher participation and analogy representation.

% \subsubsection{Limitations of LLM-Generated Analogies}
% \ysy{
% While LLM-generated analogies show promise in facilitating the understanding of scientific concepts, several limitations warrant careful consideration. 
% These challenges mainly arise from the nature of LLMs, which rely on probabilistic pattern recognition instead of true comprehension of scientific principles.
% First, analogies may oversimplify complex phenomena, causing confusion or reinforcing incorrect ideas. For example, simplified analogies in topics like quantum mechanics or cellular processes may omit critical details.
% Additionally, analogies that are not tailored to the student’s prior knowledge, educational background, or cultural context risk alienating or confusing students rather than aiding comprehension.
% Moreover, Biases in training data may lead to stereotypes or inappropriate associations, reducing educational value.
% Furthermore, models trained primarily in English and on mainstream subjects may struggle with less common languages or interdisciplinary topics.
% }

% \ysy{
% Despite these limitations, LLM-generated analogies offer an innovative starting point for enhancing science education. To harness their potential while mitigating challenges, several strategies should be explored. For example, integrating human-in-the-loop systems can ensure outputs are scientifically accurate and pedagogically aligned. Similarly, fine-tuned models trained on expert-reviewed educational content could improve reliability and depth.
% }

\subsubsection{Limitations in Student and Teacher Participation.}
% We thank the high school from China for supporting our study in a real teaching environment and two teachers deeply involved in a one-week study. 
% However, due to practical limitations, it was conducted in only one school and with two teachers. 
% To enhance representativeness, we plan to expand to multiple districts, schools, and teachers.
\chirev{Due to practical limitations, the teachers and students in our first two studies were from one high school, and the sample size was limited.
To explore real-world practicality, we expanded the participant pool by inviting more teachers from different schools with varying teaching experiences to evaluate the practical system.
Additionally, our study is limited to physics and biology due to practical constraints, excluding other subjects like chemistry.
Besides, our study with high school students may not be generalizable to younger students who might not possess developed analogical reasoning abilities or have limited world knowledge~\cite{vendetti_analogical_2015}.
\chifinal{Moreover, our study's demographic generalizability is limited, as all participants are Chinese, while prior research~\cite{richland_cognitive_2007} suggests that U.S. teachers provide cognitive support for analogies less frequently than teachers from Hong Kong or Japan in math instruction.}
In future work, we intend to expand the sample size, range, and diversity of subjects and explore diverse education levels for comprehensive evaluation.}
% Furthermore, study results could vary among students with different academic capabilities. 
% For example, prior research shows a link between patients' numerical skills and the effectiveness of analogies in doctor-patient communication, suggesting similar factors could affect educational contexts~\cite{galesic_using_2013}.
% Additionally, our study is limited to physics and biology due to practical constraints, excluding other subjects like chemistry.
% In future work, We intend to expand the sample size and range of subjects in evaluation.
% We also plan to integrate video analysis~\cite{richland_analogy_2004} for recording classroom practice and AI-supported assessments~\cite{ngoon_classinsight_2024} to achieve comprehensive evaluation.

% \subsubsection{Limitations in Teacher Participation.}
% Our study's limitations arise from the small number of teacher participants, which is also restricted by the high workload and grading focus in high schools. 
% This may not fully capture diverse teaching perspectives.
% For example, teachers in Asia often use more visual cues when explaining analogies compared to teachers in America~\cite{richland_cognitive_2007}.
% % Besides, there may be inconsistency between teachers' self-reported practices and their actual classroom behaviors about analogies~\cite{treagust_science_1992}, potentially skewing our understanding of the real need for analogies in classroom education. 
% % For example, Treagust et al.~\cite{treagust_science_1992} highlighted that although observations from classroom practice revealed infrequent and non-elaborate use of analogies, the majority of interviewed teachers believed they employed them regularly.
% Enhancing data accuracy could involve integrating video analysis~\cite{richland_analogy_2004} and AI-supported assessments~\cite{ngoon_classinsight_2024} rather than relying solely on self-reports and interviews.


\subsubsection{Limitations in Representation of Analogy.}
The practical use of analogies in teaching extends beyond the free-form analogies we generated, incorporating visual aids and dynamic technologies to enhance understanding and interaction~\cite{richland_cognitive_2007,richland_analogy_2015}.
Our interviews in Study II \chirev{and system evaluation} also revealed that teachers \chirev{have the desire to} use images, videos, and physical aids to convey analogies. 
Moving forward, we plan to enrich LLM-generated analogies with rich text, structured representations, and generated visuals to benefit teachers \chirev{in practical systems.}


%Similarly, the results of our study are also limited by the teacher participants.
%First, despite our best efforts to recruit teachers, the demanding workload and focus on grades typical of high school environments limited participation to only two teachers. 
%This limited number might have affected the representativeness of teachers' perspectives in our study.
%However, teaching methods vary between different regions and individual educators. 
%As shown in Study II, LLMs were able to generate the needed analogies for two teachers with different needs, demonstrating the potential to help more groups and individuals.
% \subsection{Evaluating the Broader Role of Analogies and LLMs in Education}

%To enhance the accuracy of our requirement derivation, it would be beneficial to incorporate video analysis of classroom sessions~\cite{richland_analogy_2004} and employ automatic assessments using intelligent AI-supported systems~\cite{ngoon_classinsight_2024}, rather than solely depending on interview findings and self-reported data from teachers.
%In Study II, we may also provide such materials to teachers to reduce their preparation costs.
%However, integrating multimodal representations poses more difficulties for LLMs.
%Our current study design can be used in future research to evaluate the benefits and challenges of expanding the role of large models in supporting teachers and students.



\section{Conclusion}
\label{sec:conclusion}
\chirev{
In this work, we first conducted in-class tests and classroom experiments guided by pre-class interviews to evaluate the effectiveness of LLM-generated analogies in two educational scenarios. 
Our in-class tests suggest that LLM-generated analogies could be beneficial for students' understanding, especially on biology concepts, but are unsuitable for self-learning systems without teacher intervention due to students' over-reliance and overconfidence. 
Classroom experiments reveal that teachers effectively create or refine analogies to meet their needs and are encouraged to teach with analogy by positive student feedback in class and homework.
Building on these findings, we developed a practical system for teachers preparing analogies.
Teachers in the system evaluation recognize its real-world effectiveness in lesson preparation about concept explanation and quiz design, teaching methods, and professional growth, despite potential over-reliance issues.
We hope future tool designers could consider these factors to ensure that LLM-generated analogies have a successful impact on teaching and learning.
}
% In this two-stage study, we conducted in-class tests and classroom experiments guided by pre-class interviews, to evaluate the effectiveness and potential benefits of LLM-generated analogies in various educational scenarios. Our findings from the in-class test demonstrate that LLM-generated analogies can enhance comprehension and test accuracy, particularly in biological contexts. However, there is a risk of students relying on incomplete or misleading information or overestimating their understanding without sufficient teacher guidance. Our classroom experiment reveals that teachers are willing to and play a vital role in selecting, refining, and implementing these analogies to align with their teaching goals and the specific needs of their students. Such analogies help maintain student focus, improve classroom and homework performance, and inspire new teaching strategies. Our work suggests integrating our findings into LLM-assisted education tools, conducting long-term evaluations to study the interplay between analogies and other educational features, and exploring the broader impact of LLM-generated analogies on both students and teachers.




\begin{acks}
The authors want to thank the reviewers for their suggestions. 
The authors also thank SHANGHAI CUNZHI HIGH SCHOOL for its generous support and all participating teachers and students, especially Siwei Ye and Zihan Luo for their deep involvement, and Yuhui Wang for strong support.
This work is supported by Natural Science Foundation of China (NSFC No.62472099, No.62202105, and No.92270121).
\end{acks}


%%
%% The next two lines define the bibliography style to be used, and
%% the bibliography file.
\balance
\bibliographystyle{ACM-Reference-Format}
\bibliography{sample-base}




\end{document}
\endinput
%%
%% End of file `sample-sigconf.tex'.
