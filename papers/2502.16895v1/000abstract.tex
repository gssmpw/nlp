Teaching scientific concepts is essential but challenging, and analogies help students connect new concepts to familiar ideas.
Advancements in large language models (LLMs) enable generating analogies, yet their effectiveness in education remains underexplored.
In this paper, we first conducted a two-stage study involving high school students and teachers to assess the effectiveness of LLM-generated analogies \chirev{in biology and physics through a controlled in-class test} and \chirev{a classroom field study}.
\chirev{Test results suggested that LLM-generated analogies could enhance student understanding particularly in biology, but require teachers' guidance to prevent over-reliance and overconfidence. 
Classroom experiments suggested that teachers could refine LLM-generated analogies to their satisfaction and inspire new analogies from generated ones, encouraged by positive classroom feedback and homework performance boosts.
Based on findings, we developed and evaluated a practical system to help teachers generate and refine teaching analogies.
% A system evaluation demonstrated its practical benefits in various aspects while highlighting potential over-reliance issues.
We discussed future directions for developing and evaluating LLM-supported teaching and learning by analogy.
}
% The first stage involved a structured generation of physical and biological analogies and a controlled in-class test to assess their effectiveness on students' understanding through accuracy and ratings. 
% The second stage combined interviews with teachers and students and a field study to assess their effectiveness in classroom settings.
% \chirev{Based on study findings, we developed a practical system to help teachers generate and refine analogies for real-world teaching. 
% We discussed future directions for developing and evaluating LLM-generated analogies and practical systems.
% }
% Our findings reveal the benefits and challenges of LLM-generated analogies in enhancing student understanding and classroom practice, offering valuable insights for future research and practice.

%Our findings suggest that while LLM-generated analogies can support teaching and lesson preparation, their quality varies and may impede student understanding if not properly curated and implemented. 
%These results underscore the main challenges LLMs encounter in generating analogies for educational purposes and provide valuable insights for future research in this field.


%Recently, with advancements in large language models (LLMs), these models have been used to create analogies. 
%However, the effectiveness of these analogies in educational settings has not been thoroughly evaluated.
%In the first stage, we examined the quality of these analogies through a structured generation process and assessed their influence on students' understanding of concepts through controlled experiments. 
%The second stage, which involved teacher participation, is more reflective of typical educational scenarios. 
%This stage included assessing the demand for and actual classroom use of these analogies through interviews with teachers and students, followed by a practical field study.
% Analogies are commonly used in education to simplify complex scientific concepts by relating them to familiar ideas. 
% Recently, large language models (LLMs) have been employed to create such analogies, while their practical effectiveness remains under-evaluated. 
% We conducted a comprehensive two-stage study with high school students and teachers to assess the usefulness of LLM-generated analogies in aiding student understanding. 
% In the first stage, we examined the quality of these analogies through a structured generation process and assessed their influence on students' understanding of concepts through controlled experiments. 
% The second stage involved evaluating the demand and actual classroom use of these analogies through interviews with teachers and students, followed by a practical field study. 
% Our findings indicate that while LLM-generated analogies can aid teaching and lesson preparation, their quality is inconsistent and can potentially hinder student understanding if not carefully curated and applied.