\begin{table*}[t]
\footnotesize
  \centering
  \caption{\chifinal{Prompt template for GPT-4 to generate analogies (I) and select the analogy from three candidates for the given concept (II).
  {\color[rgb]{0,0.39,0}{\textit{Green texts}}} are new principle after revising.}
  }
    \begin{tabularx}{0.9\linewidth}{X}
    \toprule
    
    \rowcolor[gray]{0.95}\multicolumn{1}{c}{\textbf{I: Analogy Generation with Revised Prompt}} \\
    \midrule
    \makecell[l]{\color{gray}{/* \textit{Task Description} */}\\
    Your task is to use an analogy to explain the scientific concept to students. \\
    Here are some principles for generating appropriate analogies: \\
    \color{gray}{/* \textit{Principles} */}\\
    1. The similarity between the objects in the analogy and those in the scientific concept should be minimal.\\
    2. The relationships in the analogy and the scientific concept should be highly similar.\\
    3. The analogy should use objects that students are very familiar with from everyday experiences.\\
    \color[rgb]{0,0.39,0}{\textit{4. The analogy should accurately identify similar relationships with the scientific concept and avoid forcing non-existent similarities.}}\\
    \color[rgb]{0,0.39,0}{\textit{5. The objects in the analogy and the scientific concept should align with scientific laws and commonsense knowledge.}}\\
    \color[rgb]{0,0.39,0}{\textit{6. An object in the analogy cannot have different roles or functions in different contexts.}}\\
    \color[rgb]{0,0.39,0}{\textit{7. The logic within a sentence or paragraph should not be self-contradictory.}}\\
    \color{gray}{/* \textit{Input Resource} */}\\
    To generate the appropriate analogy according to students' learning progress, we provide you with textbook content related to this scientific concept\\ for your reference.\\
    The textbook content: \texttt{\{input\_text\}}\\
    The scientific concept: \texttt{\{concept\}}\\
    Analogy:
     }\\
     \midrule
     \rowcolor[gray]{0.95}\multicolumn{1}{c}{\textbf{II: Analogy Selection}} \\
    \midrule
    \makecell[l]{\color{gray}{/* \textit{Task Description} */}\\
    There are three candidate analogies which are used to explain the scientific concept based on textbook content. Your task is to select the best analogy\\ from these three candidates. \\
    Here are some principles for generating appropriate analogies: \\
    \color{gray}{/* \textit{Principles} */}\\
    \color{gray}{Same as principles in I (Omit)}\\
    \color{gray}{/* \textit{Input Resource} */}\\
    The textbook content: \texttt{\{input\_text\}}\\
    The scientific concept: \texttt{\{concept\}}\\
    The generated analogies: \\
    Candidate 1: \texttt{\{analogy\_1\}}\\
    Candidate 2: \texttt{\{analogy\_2\}}\\
    Candidate 3: \texttt{\{analogy\_3\}}\\
    You need to give reasons first and then give the answer with the format: \texttt{Final Answer: Candidate X} \\
    Answer:
     }\\
    \bottomrule
    \end{tabularx}
  \label{tab:instruction_prompt}
\end{table*}