\begin{small}
\begin{longtable}[t]
{@{}p{0.2\linewidth}p{0.39\linewidth}p{0.39\linewidth}@{}}
\caption{A Summary of Interviewing Physics and Biology Teachers.} \label{tab:findings_in_study2_pre_class_interview}\\
\toprule
\textbf{Topic} & \textbf{Physics Teacher (T1)} & \textbf{Biology Teacher (T2)} \\ 
\midrule
\endfirsthead

\multicolumn{3}{c}%
{{\bfseries \tablename\ \thetable{} -- continued from previous page}} \\
\toprule
\textbf{Topic} & \textbf{Physics Teacher (T1)} & \textbf{Biology Teacher (T2)} \\
\midrule
\endhead

\bottomrule
\endfoot

\bottomrule
\endlastfoot

\rowcolor[gray]{0.95}\multicolumn{3}{c}{\textbf{Analogy Usage Exploration}} \\
Analogies Frequency & Sometimes. & Frequent. \\
\midrule
Analogies Feature & Mostly between learned concepts. & Mostly between biology and daily life. \\
\midrule
Source of Analogies & Mostly Prepared analogies between concepts. \newline A few improvised analogies with everyday lives. & Mostly prepared analogies. \newline Nearly no improvised analogies. \\
\midrule
Good Analogy Criterion & Easy to understand and free of scientific errors. & Easy to understand and related to everyday life. \\
\midrule
Agreement with Initial Principles in Study I & Partial agreement: Analogies between similar physical concepts. & Total Agreement. \\
\midrule
Analogy Explanation & Verbal explanation + imagery + teaching aids & Verbal explanation + imagery + teaching aids \\
\midrule
Analogies Usage Scenario & Often used to introduce concepts. \newline Sometimes throughout teaching to help students remember key points & Often used when detailing knowledge points. \\
\midrule
Agreement with the Six-step Model of Practice~\cite{richland_analogy_2015} & Acknowledges most, except for pointing out differences when introducing concepts. & Totally agreement. \\
\midrule
Student Participation in Constructing Analogies & Rare. Sometimes, students offer their ideas, which might be used in the next class. & Rare. Sometimes, students prepare analogies for student-led discussions. \\
\midrule
Students Understanding Examination & Question students with "Have you seen something similar before?", or observe students' expressions & Students complete a few exercises during class, or question students about concept differentiation. \\
\midrule
\rowcolor[gray]{0.95}\multicolumn{3}{c}{\textbf{AI Usage Exploration}} \\
Awareness and Experience with AI & Has used ChatGPT for writing papers, lesson plans, and creating images; knows about Sora. & Has used ChatGPT for tenders and personal use. \\
\midrule
Pros and Cons of AI & Pros: helps write unexpected things. \newline Cons: Needs specific questions; AI usually doesn't follow the instructions. & Pros: Provides broad ideas.  \newline No clear cons due to limited experience. \\
\midrule
Can AI Replace Teachers? & Teachers know students' learning situations, AI does not; AI-generated content needs adjustment. & AI cannot replace but complement teachers. \\
\midrule
\rowcolor[gray]{0.95}\multicolumn{3}{c}{\textbf{Expectations Sharing on LLM-generated Analogies}} \\
Positive Comments on Analogies in Study I & 1. The analogies are all vivid and some of them are interesting & 1. Some analogies are similar to those used in class \newline 2. Identify analogies to try in class for concepts not usually taught with analogies. \\
\midrule
Negative Comments on Analogies in Study I & 1. Analogies don't clarify abstract concepts. \newline 2. Analogies can complicate simple concepts. \newline 3. For concepts that are tested simply, memorization is enough. \newline 4. Pictures could make some concepts clear without analogies. & 1. Analogies shouldn't reflect all but the main concepts; the rest relies on memory. \newline 2. Pictures and animations can visualize familiar organisms without analogies. \newline 3. Although rare, related concepts sometimes are used as analogies. \\
\midrule
Overall Expectations & Vivid analogies between physical concepts. & Analogies from daily life for teaching focus; \newline Interesting analogies to stimulate learning interest.\\
\end{longtable}
\end{small}