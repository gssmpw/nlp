
%\captionsetup[lstlisting]{position=bottom}
\lstset{
    backgroundcolor=\color[RGB]{245,245,244},
    breaklines=true,
    breakindent=0pt,
    basicstyle=\ttfamily\small,
    %emph={Structure,Analogy,Free,Form,Word,Generation,Concept},
    emphstyle={\bfseries\color{NavyBlue}}
}\begin{lstlisting}[caption={The revised prompt templates for GPT-4 to generate analogies.},label=lst:analogy_generation_prompt_revise]
Your task is to use an analogy to explain the scientific concept to students. 
Here are some principles for generating appropriate analogies:

1. The objects in the analogy and the scientific concept should align with scientific laws and commonsense knowledge.
2. The similarity between the objects in the analogy and those in the scientific concept should be minimal.
3. The relationships in the analogy and the scientific concept should be highly similar.
4. The analogy should accurately identify similar relationships with the scientific concept and avoid forcing non-existent similarities.
5. The analogy should use objects that students are very familiar with from everyday experiences.
6. An object in the analogy cannot have different roles or functions in different contexts.
7. The logic within a sentence or paragraph should not be self-contradictory.

To generate the appropriate analogy according to students' learning progress, we provide you with textbook content related to this scientific concept for your reference.
The textbook content: {input_text}
The scientific concept: {concept}
Analogy (Answer in Chinese):
\end{lstlisting}