In this section, we introduce the overview, our study design, and the techniques for analogy generation with LLM.

\begin{figure*}[h]
    \centering
    \includegraphics[width=0.9\linewidth]{figure/revoverview.pdf}
    \caption{\chirev{Our Study I address RQ1 through an in-class test. Study II address RQ2 through a pre-class interview and RQ3 through a one-week classroom study. Studies' findings lead us to build an LLM-supported practical system for analogical education.}}
    \Description{The figure presents our two-stage study design addressing three research questions (RQ1, RQ2, and RQ3), which then direct us to design and evaluate an LLM-supported practical system for education by analogy. Study I investigates the effectiveness of large language model-generated analogies for student understanding, assessed through an in-class test and student interviews. Study II explores the types of analogies teachers require and their effectiveness in classroom practice through pre-class interviews and a week-long field study with classroom teaching. }
    \label{fig:overview}
\end{figure*}

\chirev{
\subsection{Overview}
Our work aims to first understand LLM-generated educational analogies' effectiveness through empirical studies and then design practical LLM-assisted educational systems leveraging findings from studies.
For empirical studies, we explore two study settings: one where students solve problems using only LLM-generated analogies and necessary materials without additional guidance (Sec.~\ref{sec:study1}), and another where teachers integrate LLM-generated analogies flexibly into classroom instruction (Sec.~\ref{sec:study2}), considering the following two reasons.
First, these two settings align with those used to evaluate human-made analogies in traditional education research: student-only testing~\cite{thagard_analogy_1992,gick_analogical_1980} and teacher-led classroom practice~\cite{vendetti_analogical_2015,oliva_teaching_2007}.
Second, evaluating in two settings respectively inform the design of systems that incorporate LLM in generating analogies to (1) support self-learning for students~\cite{gao2024fine} and (2) boost teaching for teachers~\cite{lessonplanner2024uist}, within the context of LLM-assisted education research.

Based on study findings, we identify a more feasible direction between supporting student self-learning and teacher instruction by analogy for designing a practical system (Sec.~\ref{sec:system}). 
Drawing from study findings and system evaluation, we offer implications for future LLM-assisted educational systems with analogy (Sec.~\ref{sec:discussion}).

}

\subsection{Study Design}


% In this paper, we conduct a two-stage study to understand the effectiveness and potential role of LLM-generated analogies in education.
% To understand the effectiveness and potential role of LLM-generated analogies in education, our design takes into account the features of LLM-generated content and experimental designs from previous studies that have effectively assessed the effectiveness of manually designed analogies.
% From the perspective of LLM-generated content, LLM can produce different types of analogies, including familiar and unexpected ones, and these can range from high-quality to incorrect or even harmful. 
% Thus, we need to enhance the quality of LLM-generated analogies as much as possible while avoiding the introduction of bias and evaluating the impact of analogies on the understanding of students, which will be detailed in Sec.~\ref{sec:Techniques for Analogy Generation with LLM}.

%In terms of experimental design for evaluating analogies in education, there are typically two approaches.
% From the perspective of study scenarios, 
% Building on previous studies that assessed manually designed analogies, our study considers two settings: with or without teacher intervention.
% Traditionally, two primary approaches are employed in the realm of study design for assessing analogies in problem-solving.

% Traditionally, two approaches evaluate analogies in education: one where students solve problems using only analogies and necessary materials without additional guidance, and another where teachers introduce and explain analogies flexibly during classroom teaching~\cite{brown_analogical_1989}. 
% Although the teacher-led approach is more representative of traditional educational scenarios~\cite{gick_analogical_1980, gray_teaching_2021}, we also consider the first approach due to the rise of LLM-assisted self-learning tools promoting autonomous learning without human intervention~\cite{gao2024fine,chen2024stugptviz}. 
% Controlled experiments in this context allow for swift evaluation of analogy effectiveness through response accuracy, offering immediate feedback compared to classroom-based experiments. 

\chirev{
For both studies, we need to ground the effectiveness of LLM-generated analogies on improving students' concept understanding by comparing the accuracy of problem-solving across different student groups.
Specifically, we may need to compare students' problem-solving accuracy in classroom teaching with and without LLM-generated analogies. 
This comparison is driven by the unique features of LLM-generated analogies compared to traditional ones, as well as teachers’ potential unfamiliarity with them and unclear expectations.
Unlike classic analogies refined and validated over generations, they may be harder for teachers to adapt to support student understanding.
We will confirm this design in the interview with teachers (Sec.~\ref{sec:study21_findings_and_derived_requirments}).
} 
Besides accuracy, it's important to evaluate students' and teachers' subjective satisfaction~\cite{Lyu2024evaluating}, reflected through subjective ratings, classroom feedback, and interviews.

As shown in Fig.~\ref{fig:overview}, we conducted a two-stage study in a Chinese high school on LLM-generated analogies for physics and biology concepts to explore this topic. 

\chifinal{
\subsubsection{Study I} 
We conducted an in-class test to evaluate the effectiveness of LLM-generated analogies in understanding concepts without human intervention (\textbf{RQ1}).}
\begin{itemize}
    \item \textbf{Participants.} 49 Chinese high school freshmen from 2 classes.
    \item \textbf{Procedure.} Students were divided into two groups: one received LLM-generated analogies, while the other did not. They then completed an in-class test with 20 multiple-choice questions for 4 concepts they didn't learn. Afterward, 8 students participated in interviews.
    \item \textbf{Measure.} Effectiveness was assessed through quantitative results of students' answer accuracy and confidence ratings, and qualitative insights from student interviews.
\end{itemize}

\chifinal{
\subsubsection{Study II} Study II consists of two sub-studies.}
We conducted a pre-class interview to identify classroom needs for LLM-generated analogies (\textbf{RQ2}).
\begin{itemize}
    \item \textbf{Participants.} 2 Chinese teachers and 2 Chinese senior students.
    \item \textbf{Procedure.} The interview followed a semi-structured format, allowing participants to discuss their experiences, expectations, and concerns on using LLM-generated analogies in the classroom.
    \item \textbf{Measure.} We identified qualitative findings on their perceptions of analogy use during the interview.
\end{itemize}

We then conducted a controlled field study to evaluate the effectiveness of LLM-generated analogies in classroom practice (\textbf{RQ3}).
\begin{itemize}
    \item \textbf{Participants.} The 2 teachers from the pre-class interview and 50 Chinese students from 2 classes.
    \item \textbf{Procedure.} The teachers taught both classes over one week, delivering a total of 12 lessons. In one class, they incorporated LLM-generated analogies, while in the other, they followed regular instruction without analogies as a control.
    \item \textbf{Measure.} We derived qualitative findings from teachers’ selection and modification of analogies and student feedback.
\end{itemize}

% In \textbf{Study I}, we conducted a controlled experiment with 49 students who completed an in-class test including 4 concepts with 20 questions to evaluate the effectiveness of LLM-generated analogies in understanding concepts without human intervention (\textbf{RQ1}). 
% We assessed the effectiveness of analogies by analyzing students' response accuracy and their subjective satisfaction through confidence rating and also interviewed 8 participating students.

% In \textbf{Study II}, we first interviewed 2 teachers and 2 senior-year students to identify the classroom needs for LLM-generated analogies (\textbf{RQ2}). 
% We then conducted a controlled field study with 2 classes, totaling 12 lessons. 
% One class used analogies in teaching, and the other maintained regular instruction. 
% We documented the teachers' selection and modification of analogies and student feedback to evaluate the effectiveness of analogies in classroom practice (\textbf{RQ3}).


\subsection{Techniques for Analogy Generation with LLM}
\label{sec:Techniques for Analogy Generation with LLM}
% high-level abstract for generation process in two studies
% distill the commonalities

%Traditional research in the AI community has focused on training small but specialized LMs for analogy generation.

%With the development of LLMs, their capabilities for in-context learning and following instructions have demonstrated significant potential in generating satisfying content based solely on textual instructions without parameter updates.
With the development of LLMs, their capabilities have demonstrated significant potential in generating satisfying content.
Therefore, recent studies have utilized LLMs to create analogies using manually-designed instructions. 
We align with this approach by crafting prompts for LLMs grounded in established analogy theories and our interviews.
The prompt consists of three parts:
%To construct prompts, we follow the generally used structure to instruct LLMs
\begin{itemize}
    \item \textbf{Task Description} demonstrates the task that LLMs need to achieve.
    \item \textbf{Principles} highlight the requirements, rules, and constraints that LLMs must follow to complete the task.
    \item \textbf{Input Resource} lists the input materials needed to complete the task.
\end{itemize}

%Specifically, we employ general task description from previous work for analogy generation, \eg, ``Your task is to use an analogy to explain this concept.'' Besides,
%Since the AI community lacks standardized principles for the use of LLMs in educational analogy generation, we initially summarize principles from the education literature review~\cite{hesse1959defining,gentner1983structure,gentner2017analogy}.
%Through manual annotation in \textbf{Study I} and interviews in \textbf{Study II}, we further refine these principles, thereby improving the ability of LLMs to generate analogies that explain concepts on demand and provide insights for future research in educational analogy generation.
%For input resources, we include scientific concepts and textbook content in all analogy generation processes for both \textbf{Study I} and \textbf{Study II}, as textbook content aligns with the learning target of the students being tested and also serves as a reference for LLMs to reduce hallucinations.
\chirev{
Given the lack of standardized guidelines for using LLMs in educational analogy generation, we summarize principles from educational literature~\cite{hesse1959defining,gentner1983structure,gentner2017analogy} and refine them through manual annotation in \textbf{Study I} and interviews in \textbf{Study II}. 
Additionally, inspired by prior work~\cite{yuan-etal-2023-distilling}, Study I uses an over-generation and filtering strategy to select the best analogies, while Study II leaves all candidates for teachers to refine.
The techniques used in the further developed system is determined by the results of two studies.
}



%Additionally, previous studies have shown that LLMs can enhance their performance through an over-generation strategy followed by filtering~\cite{yuan-etal-2023-distilling}. 
%Inspired by this, we adopt a similar approach, prompting the LLMs to generate multiple analogy candidates and asking the LLMs to select the most suitable one to enhance the quality of the analogies in \textbf{Study I}.
%In \textbf{Study II}, we do not perform filtering and leave all candidates for teachers to select and refine.
