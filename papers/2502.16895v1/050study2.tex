In this section, we introduce the pre-class interview (Sec.~\ref{sec:study21}) to gather requirements for the classroom experiments (Sec.~\ref{sec:study22}) that evaluate the actual use of LLM-generated analogies in classroom teaching.



\subsection{Pre-class Interview}
\label{sec:study21}
In this subsection, we outline the participants for our pre-class interview (Sec.~\ref{sec:study21_participants}), the procedure and stimulus (Sec.~\ref{sec:study21_procedure_and_stimuls}), and the findings and derived requirements (Sec.~\ref{sec:study21_findings_and_derived_requirments}) for further classroom experiments. 
\subsubsection{Participants}
\label{sec:study21_participants}
We recruited two \chifinal{Chinese} teachers and two \chifinal{Chinese} students from the same high school as Study I to participate in pre-class interviews. 
The two teachers (T1 and T2; 1 female) are a physics teacher with 6 years of teaching experience and a biology teacher with 3 years of teaching experience. 
Both have a bachelor's degree, are interested in AI-assisted education, and teach first-year high school courses in the semester during our study.
The two senior students (S1 and S2; 1 female) are in their third year of high school, have learned the concepts used in Study I, and have above-average grades.


\subsubsection{Procedure}
\label{sec:study21_procedure_and_stimuls}
We conducted one-on-one semi-structured online interviews with the participants via Tencent Meeting.
The student interviews lasted 40 minutes, with a \$10 gift card for each student, while the teacher interviews lasted 60 minutes, with a \$20 gift card for each teacher.
% Below is a brief description of the interview processes for the teachers and students, respectively.
% Please refer to the supplementary materials for details of the procedure and stimulus.

For teachers, the interview included four steps to understand the requirements of LLM-generated analogies in teaching.
% For teachers, the interview included four steps to understand the potential role and requirements of analogies generated by LLMs in teaching.

\textbf{Step 1: Analogy Orientation.} 
We first presented them with common analogies in the classroom from the literature (\eg, ``light waves and water waves'', ``heart and hydraulic pump'')~\cite{oliva_teaching_2007} to orient them to analogies and ensure the terminology used during the interview.
We asked the teachers to recall the analogies they had used and encouraged them to think aloud about any experiences with analogies throughout the interview.

\textbf{Step 2: Analogy Usage Exploration.}
After that, we conducted interviews using a questionnaire primarily based on the one proposed by~\cite{oliva_teaching_2007} but modified to incorporate insights from the latest research over the past decade.
We first investigated how teachers prepare analogies, such as whether they prepare in advance or improvise and adjust in the class. 
Then we asked teachers to share the characteristics of good analogies in their opinion~\cite{gray_teaching_2021} and whether they agree with the principles we summarized from the literature in Study I. 
We also investigated whether teachers involved students in building analogies during teaching, at which step of introducing knowledge points they used analogies, and whether they followed the six-step theoretical model about analogy~\cite{richland_analogy_2015}.
Other questions covered include whether visual aids were used and whether immediate feedback was provided on students' understanding of analogies.

\textbf{Step 3: AI Usage Exploration.}
% The interview questions then explored the use of AI tools like ChatGPT and AI-assisted education. 
We then asked teachers about their previous experiences with AI tools like ChatGPT, their familiarity with AI-assisted teaching or self-learning. 
We also inquired about their concerns on AI performance and its application in educational settings~\cite{chen2024stugptviz,tan_more_2024}, and whether they believe AI could partially replace teachers to achieve educational goals such as mastering basic concepts, problem-solving, and developing higher-order independent thinking skills.

\textbf{Step 4: Expectations Sharing on LLM-generated Analogies.}
We then showed each teacher the analogies from Study I in their respective teaching subjects.
We asked them to evaluate each analogy's strengths, weaknesses, classroom applicability, and potential for teacher modification
Building on this, we asked teachers to share their expectations for effective AI-generated analogies.

The interviews with the students were focused on their classroom experiences rather than exploring AI usage since their role in the classroom mainly involved receiving information rather than designing analogies, and they spent most of their time without any smart devices or AI.
Initially, students were asked to recall analogies used in class. 
We then presented the ten concepts from Study I, asking students to reflect on their learning experiences. 
Next, we presented the ten analogies from Study I and asked for their feedback on their effectiveness in enhancing their understanding.




\subsubsection{Findings and Derived Requirements}
\label{sec:study21_findings_and_derived_requirments}
%\begin{small}
\begin{longtable}[t]
{@{}p{0.2\linewidth}p{0.39\linewidth}p{0.39\linewidth}@{}}
\caption{A Summary of Interviewing Physics and Biology Teachers.} \label{tab:findings_in_study2_pre_class_interview}\\
\toprule
\textbf{Topic} & \textbf{Physics Teacher (T1)} & \textbf{Biology Teacher (T2)} \\ 
\midrule
\endfirsthead

\multicolumn{3}{c}%
{{\bfseries \tablename\ \thetable{} -- continued from previous page}} \\
\toprule
\textbf{Topic} & \textbf{Physics Teacher (T1)} & \textbf{Biology Teacher (T2)} \\
\midrule
\endhead

\bottomrule
\endfoot

\bottomrule
\endlastfoot

\rowcolor[gray]{0.95}\multicolumn{3}{c}{\textbf{Analogy Usage Exploration}} \\
Analogies Frequency & Sometimes. & Frequent. \\
\midrule
Analogies Feature & Mostly between learned concepts. & Mostly between biology and daily life. \\
\midrule
Source of Analogies & Mostly Prepared analogies between concepts. \newline A few improvised analogies with everyday lives. & Mostly prepared analogies. \newline Nearly no improvised analogies. \\
\midrule
Good Analogy Criterion & Easy to understand and free of scientific errors. & Easy to understand and related to everyday life. \\
\midrule
Agreement with Initial Principles in Study I & Partial agreement: Analogies between similar physical concepts. & Total Agreement. \\
\midrule
Analogy Explanation & Verbal explanation + imagery + teaching aids & Verbal explanation + imagery + teaching aids \\
\midrule
Analogies Usage Scenario & Often used to introduce concepts. \newline Sometimes throughout teaching to help students remember key points & Often used when detailing knowledge points. \\
\midrule
Agreement with the Six-step Model of Practice~\cite{richland_analogy_2015} & Acknowledges most, except for pointing out differences when introducing concepts. & Totally agreement. \\
\midrule
Student Participation in Constructing Analogies & Rare. Sometimes, students offer their ideas, which might be used in the next class. & Rare. Sometimes, students prepare analogies for student-led discussions. \\
\midrule
Students Understanding Examination & Question students with "Have you seen something similar before?", or observe students' expressions & Students complete a few exercises during class, or question students about concept differentiation. \\
\midrule
\rowcolor[gray]{0.95}\multicolumn{3}{c}{\textbf{AI Usage Exploration}} \\
Awareness and Experience with AI & Has used ChatGPT for writing papers, lesson plans, and creating images; knows about Sora. & Has used ChatGPT for tenders and personal use. \\
\midrule
Pros and Cons of AI & Pros: helps write unexpected things. \newline Cons: Needs specific questions; AI usually doesn't follow the instructions. & Pros: Provides broad ideas.  \newline No clear cons due to limited experience. \\
\midrule
Can AI Replace Teachers? & Teachers know students' learning situations, AI does not; AI-generated content needs adjustment. & AI cannot replace but complement teachers. \\
\midrule
\rowcolor[gray]{0.95}\multicolumn{3}{c}{\textbf{Expectations Sharing on LLM-generated Analogies}} \\
Positive Comments on Analogies in Study I & 1. The analogies are all vivid and some of them are interesting & 1. Some analogies are similar to those used in class \newline 2. Identify analogies to try in class for concepts not usually taught with analogies. \\
\midrule
Negative Comments on Analogies in Study I & 1. Analogies don't clarify abstract concepts. \newline 2. Analogies can complicate simple concepts. \newline 3. For concepts that are tested simply, memorization is enough. \newline 4. Pictures could make some concepts clear without analogies. & 1. Analogies shouldn't reflect all but the main concepts; the rest relies on memory. \newline 2. Pictures and animations can visualize familiar organisms without analogies. \newline 3. Although rare, related concepts sometimes are used as analogies. \\
\midrule
Overall Expectations & Vivid analogies between physical concepts. & Analogies from daily life for teaching focus; \newline Interesting analogies to stimulate learning interest.\\
\end{longtable}
\end{small}
\section{Analysis} \label{sec:analysis}
In this section, we provide a comprehensive analysis of Satori. First, we demonstrate that Satori effectively leverages self-reflection to seek better solutions and enhance its overall reasoning performance. Next, we observe that Satori exhibits test-scaling behavior through RL training, where it progressively acquires more tokens to improve its reasoning capabilities. Finally, we conduct ablation studies on various components of Satori's training framework. Additional results are provided in Appendix~\ref{app:results}.



\paragraph{COAT Reasoning v.s. CoT Reasoning.}
\begin{table}[h]
  \begin{center}
  \scriptsize
  \captionsetup{font=small}
  \caption{\textbf{COAT Training v.s. CoT Training.} Qwen-2.5-Math-7B trained with COAT reasoning format (Satori-Qwen-7B) outperforms the same base model but trained with classical CoT reasoning format (Qwen-7B-CoT)}
  \setlength{\tabcolsep}{1.3pt}
  \begin{tabular}{cccccccccc}
    \toprule
    \textbf{Model} & \textbf{GSM8K} & \textbf{MATH500}  &  \textbf{Olym.} & \textbf{AMC2023} & \textbf{AIME2024} \\
    \midrule
    Qwen-2.5-Math-7B-Instruct & 95.2 & 83.6 &41.6& 62.5 &16.7 \\
    Qwen-7B-CoT (SFT+RL) & 93.1 & 84.4  &	42.7 &	60.0 & 10.0 \\
    \midrule
    \textbf{Satori-Qwen-7B}  & 93.2 & 85.6  & 46.6  & 67.5  & 20.0 \\
    \bottomrule
  \end{tabular}
  \label{table:ablation-coat}
  \end{center}
\vspace{-1em}
\end{table}
We begin by conducting an ablation study to demonstrate the benefits of COAT reasoning compared to the classical CoT reasoning. Specifically, starting from the synthesis of demonstration trajectories in the format tuning stage, we ablate the ``reflect'' and  ``explore'' actions, retaining only the ``continue'' actions. Next, we maintain all other training settings, including the same amount of SFT and RL data and consistent hyper-parameters. This results in a typical CoT LLM (Qwen-7B-CoT) without self-reflection or self-exploration capabilities. As shown in Table~\ref{table:ablation-coat}, the performance of Qwen-7B-CoT is suboptimal compared to Satori-Qwen-7B and fails to surpass Qwen-2.5-Math-7B-Instruct, suggesting the advantages of COAT reasoning over CoT reasoning.



\paragraph{Satori Exhibits Self-correction Capability.}
% Please add the following required packages to your document preamble:
% \usepackage{multirow}
\begin{table}[h]
\scriptsize
\captionsetup{font=small}
\caption{\textbf{Satori's Self-correction Capability.} T$\rightarrow$F: negative self-correction; F$\rightarrow$T: positive self-correction.}
\setlength{\tabcolsep}{5pt}
\begin{tabular}{lcccccc}
\toprule
\multirow{3}{*}{\textbf{Model}} & \multicolumn{4}{c}{\textbf{In-Domain}}                                                                            & \multicolumn{2}{c}{\textbf{Out-of-Domain}}              \\ \cmidrule[0.2pt]{2-7} 
                                & \multicolumn{2}{c}{\textbf{MATH500}}                    & \multicolumn{2}{c}{\textbf{OlympiadBench}}              & \multicolumn{2}{c}{\textbf{MMLUProSTEM}}         \\
                                & \textbf{T$\rightarrow$F} & \textbf{F$\rightarrow$T} & \textbf{T$\rightarrow$F} & \textbf{F$\rightarrow$T} & \textbf{T$\rightarrow$F} & \textbf{F$\rightarrow$T} \\ \midrule[0.5pt]
Satori-Qwen-7B-FT                  & 79.4\%                    & 20.6\%                    & 65.6\%                    & 34.4\%                    & 59.2\%                    & 40.8\%                    \\
\textbf{Satori-Qwen-7B}                     & 39.0\%                       & 61.0\%                       & 42.1\%                    & 57.9\%                    & 46.5\%                    & 53.5\%                    \\ \bottomrule
\end{tabular}
\label{table:finegrain-reflect}
\end{table}
We observe that Satori frequently engages in self-reflection during the reasoning process (see demos in Section~\ref{sec:demo}), which occurs in two scenarios: (1) it triggers self-reflection at intermediate reasoning steps, and (2) after completing a problem, it initiates a second attempt through self-reflection. We focus on quantitatively evaluating Satori's self-correction capability in the second scenario. Specifically, we extract responses where the final answer before self-reflection differs from the answer after self-reflection. We then quantify the percentage of responses in which Satori's self-correction is positive (i.e., the solution is corrected from incorrect to correct) or negative (i.e., the solution changes from correct to incorrect). The evaluation results on in-domain datasets (MATH500 and Olympiad) and out-of-domain datasets (MMLUPro) are presented in Table~\ref{table:finegrain-reflect}. First, compared to Satori-Qwen-FT which lacks the RL training stage, Satori-Qwen demonstrates a significantly stronger self-correction capability. Second, we observe that this self-correction capability extends to out-of-domain tasks (MMLUProSTEM). These results suggest that RL plays a crucial role in enhancing the model's true reasoning capabilities.


\paragraph{RL Enables Satori with Test-time Scaling Behavior.}
\begin{figure}[h]
    \centering
    \includegraphics[width=0.5\textwidth]{Figures/rm_shaping_tot_len.pdf}
    \vspace{-2em}
\caption{\textbf{Policy Training Acc. \& Response length v.s. RL Train-time Compute.} Through RL training, Satori learns to improve its reasoning performance through longer thinking.}
\label{fig:test_time_scaling}
\end{figure}
\begin{figure}[h]
    \centering
    \includegraphics[width=0.45\textwidth]{Figures/length_across_levels.pdf}
    \vspace{-1.5em}
\caption{\textbf{Above: Test-time Response Length v.s. Problem Difficulty Level; Below: Test-time Accuracy v.s. Problem Difficulty Level.} Compared to FT model (Satori-Qwen-FT), Satori-Qwen uses more test-time compute to tackle more challenging problems.}
\label{fig:difficulty_level}
\vspace{-1em}
\end{figure}

Next, we aim to explain how reinforcement learning (RL) incentivizes Satori's autoregressive search capability. First, as shown in Figure~\ref{fig:test_time_scaling}, we observe that Satori consistently improves policy accuracy and increases the average length of generated tokens with more RL training-time compute. This suggests that Satori learns to allocate more time to reasoning, thereby solving problems more accurately. One interesting observation is that the response length first decreases from 0 to 200 steps and then increases. Upon a closer investigation of the model response, we observe that in the early stage, our model has not yet learned self-reflection capabilities. During this stage, RL optimization may prioritize the model to find a shot-cut solution without redundant reflection, leading to a temporary reduction in response length. However, in later stage, the model becomes increasingly good at using reflection to self-correct and find a better solution, leading to a longer response length.
 
Additionally, in Figure~\ref{fig:difficulty_level}, we evaluate Satori's test accuracy and response length on MATH datasets across different difficulty levels. Interestingly, through RL training, Satori naturally allocates more test-time compute to tackle more challenging problems, which leads to consistent performance improvements compared to the format-tuned (FT) model.



\paragraph{Large-scale FT v.s. Large-scale RL.}
\begin{table}[h]
  \begin{center}
  \scriptsize
  \captionsetup{font=small}
  \caption{\textbf{Large-scale FT V.S. Large-scale RL} Satori-Qwen (10K FT data + 300K RL data) outperforms same base model Qwen-2.5-Math-7B trained with 300K FT data (w/o RL) across all math and out-of-domain benchmarks.}
  \setlength{\tabcolsep}{1.15pt}
  \vspace{-0.5em}
\begin{tabular}{lccccc}
\toprule
\textbf{(In-domain)}   & \textbf{GSM8K}   & \textbf{MATH500} & \textbf{Olym.} & \textbf{AMC2023} & \textbf{AIME2024} \\ \midrule
Qwen-2.5-Math-7B-Instruct & 95.2 & 83.6                     & 41.6                  & 62.5             & 16.7                 \\
Satori-Qwen-7B-FT (300K)     & 92.3 & 78.2                       & 40.9           & 65.0               & 16.7              \\
\textbf{Satori-Qwen-7B}         & 93.2        & 85.6                     & 46.6           & 67.5             & 20.0                \\ \midrule
\textbf{(Out-of-domain)}  & \textbf{BGQA}    & \textbf{CRUX}  & \textbf{STGQA} & \textbf{TableBench}   & \textbf{STEM}     \\ \midrule
Qwen-2.5-Math-7B-Instruct & 51.3             & 28.0             & 85.3           & 36.3             & 45.2              \\
Satori-Qwen-7B-FT (300K)     & 50.5             & 29.5           & 74.0             & 35.0               & 47.8              \\
\textbf{Satori-Qwen-7B}               & 61.8             & 42.5           & 86.3           & 43.4             & 56.7              \\ \bottomrule
\end{tabular}
  \label{table:ablation-ft-rl}
  \end{center}
\end{table}
We investigate whether scaling up format tuning (FT) can achieve performance gains comparable to RL training. We conduct an ablation study using Qwen-2.5-Math-7B, trained with an equivalent amount of FT data (300K). As shown in Table~\ref{table:ablation-ft-rl}, on the math domain benchmarks, the model trained with large-scale FT (300K) fails to match the performance of the model trained with small-scale FT (10K) and large-scale RL (300K). Additionally, the large-scale FT model performs significantly worse on out-of-domain tasks, demonstrates RL’s advantage in generalization.


\paragraph{Distillation Enables Weak-to-Strong Generalization.} 
\begin{figure}[!t]
    \centering
     \includegraphics[width=0.4\textwidth]
     {Figures/distillation.pdf}
     \vspace{-1.5em}
\caption{\textbf{Format Tuning v.s. Distillation.} Distilling from a Stronger model (Satori-Qwen-7B) to weaker base models (Llama-8B and Granite-8B) are more effective than directly applying format tuning on weaker base models.}
\label{fig:distill}
\vspace{-1em}
\end{figure}
Finally, we investigate whether distilling a stronger reasoning model can enhance the reasoning performance of weaker base models. Specifically, we use Satori-Qwen-7B to generate 240K synthetic data to train weaker base models, Llama-3.1-8B and Granite-3.1-8B. For comparison, we also synthesize 240K FT data (following Section~\ref{subsec:format}) to train the same models. We evaluate the average test accuracy of these models across all math benchmark datasets, with the results presented in Figure~\ref{fig:distill}. The results show that the distilled models outperform the format-tuned models. 

This suggests a new, efficient approach to improve the reasoning capabilities of weaker base models: (1) train a strong reasoning model through small-scale
FT and large-scale RL (our Satori-Qwen-7B) and (2) distill the strong reasoning capabilities of the model into weaker base models. Since RL only requires answer labels as supervision, this approach introduces minimal costs for data synthesis, i.e., the costs induced by a multi-agent data synthesis framework or even more expensive human annotation.



The findings of the teacher interview were summarized in Tab.~\ref{tab:findings_in_study2_pre_class_interview}.
Based on these findings and the interview with senior students, we conclude the requirements for data preparation and classroom study design as follows.

\textbf{Providing analogies to teachers during lesson preparation.}
At the beginning of the interview, both teachers clearly stated that they often use analogies in class, with most being prepared in advance. 
The physics teacher (T1) frequently referred to analogies found in teaching aids. 
% \chirev{and rarely improvised simple everyday analogies in class.}
The biology teacher (T2) listed key knowledge points during lesson preparation and then considered suitable analogies, drawing on personal experience and input from other veteran teachers. 
% The physics teacher occasionally improvised analogies in class, using simple everyday examples to maintain student focus. 
Both teachers and students claimed that students rarely participate in the construction of analogies except in student-led discussion sessions.

% \uline{Based on this, we decided that the teacher would provide the key points of the lesson beforehand, and we would supplement the lesson preparation by providing analogies generated by LLMs.}


\textbf{Generating analogies based on subjects' characteristics and analogy needs.}
Two teachers demonstrated apparent differences in their need for and use of analogies. 
Based on their explanations, we attribute these differences to their subjects' characteristics rather than personal preferences. 
T1 frequently used analogies between concepts like ``electric field and magnetic field,'' noting the abstract nature of physics and the difficulty of finding everyday analogies. 
% As a result, T1 slightly deviated from the standard six-step teaching model, preferring analogies between concepts for effective instruction.
In contrast, T2 primarily employed interesting everyday life analogies, such as likening ``chromosome crossing over'' to ``swapping legs between classmates''. 
% Despite their unconventional nature, these analogies were effective.
% T2 noted that many biological structures and functions are unfamiliar to students, unlike the more common physical phenomena, making it difficult to use conceptual analogies to help them understand.
However, when presented with the analogy for ``blood sugar regulation'' generated in Study I, T2 suggested it could be analogized with ``thyroid hormone regulation'', as functions are related and thus easy for students to grasp. 

% \uline{
% Therefore, in Study II, we mainly generate analogies based on other physical concepts for physical concepts and analogies based on everyday life for biological concepts based on the teachers' needs.
% }

\textbf{Generating analogies for teaching key points and helping students focus.}
Both teachers stated that the primary goal of using analogies was to help students understand key concepts. 
Additionally, they emphasized that some analogies helped students maintain engagement. 
T1 mentioned, \textit{``When I notice students getting sleepy, I occasionally improvise an interesting analogy related to the concept to wake them up.''} 
T2 used images of people and mummies to explain the dry and fresh weight of cells, which are vivid and engaging without distracting students. 

% \uline{Based on this, we plan to generate analogies that effectively illustrate teaching concepts and are engaging while ensuring they accurately reflect some characteristics of the concepts rather than just capturing attention.}


\chirev{\textbf{Generating necessary analogies determined by teachers.}}
Both teachers acknowledged our generated vivid analogies in Study I. 
However, they criticized many of them as being overly complicated and unnecessary. 
For ``nuclear fission and fusion'' and ``auxin,'' T1 and T2 pointed out that students could quickly understand them through pictures and animations. 
Additionally, T1 mentioned that the ``molecular kinetic theory'' is relatively simple and not a key focus of exams, thus only requiring memorization.
T1 also stated that concepts in atomic physics, such as the ``photoelectric effect,'' are too isolated from other physical concepts to be conveyed through analogy. 
Additionally, students interviewed could not recall many concepts taught using analogies and viewed many analogies in Study I as redundant.

\chirev{\textbf{Generating non-complex analogies for certain aspects of the concept.}}
Both teachers emphasized the importance of analogizing only parts of a concept to keep it \chirev{correct} and easy to understand. 
\chirev{
T1 took the incorrect analogy of ``nuclear fission and fusion'' (Fig.~\ref{fig:example}) to illustrate LLMs' difficulty in generating correct physical analogies, noting that forcing analogies for all features leads to factual and semantic errors. 
He explained that physical concepts often involve multiple features, some of which, like ``chain reactions'', can be analogized (e.g., ``dominoes''), while others, such as ``mass-energy conversion'', are too abstract to find counterparts due to their basis in mathematical models.
}
\chirev{For biological analogies,} T2 recommended focusing on negative feedback in ``thyroid hormone regulation'' with an analogy like ``adjusting the temperature with an air conditioner remote control,'' while students should memorize other details. 
S2 recalled an analogy about specific details, in which the teacher compared a ``channel protein'' with a ``fire escape.''
Therefore, for complex concepts with multiple knowledge points, selecting only a specific aspect for the analogy is sufficient.

% \uline{Therefore, in Study II, we would ask teachers to provide the knowledge points they believe require analogies, as determining the necessity of analogies automatically is beyond the scope of this study. 
% For the knowledge points provided, we would generate various analogies focusing on the overall characteristics and specific details, respectively, and ensure that they are concise and easy to understand.}

\textbf{Not necessary to generate perfect analogies.}
Teachers were lenient towards the generated analogies from Study I and managed to extract effective parts from them. 
T2 appreciated the analogy comparing ``nerve impulses'' to the ``efficient operation of stations in an express delivery system,'' though some parts were redundant. 
Additionally, T1 shared his experience using ChatGPT for lesson plans, finding it repetitive and sometimes vague but useful for providing new ideas.


% \uline{Therefore, we do not aim to generate perfect analogies but rather strive to provide analogies that meet the above requirements, allowing teachers to refine and adapt them as needed.}

\textbf{Evaluating LLM-generated analogies in class by teachers.}
Teachers had various approaches to evaluating the effectiveness of analogies in class. 
T1 asked questions like ``Have you seen something similar before?'' or observed the students' expressions, while T2 had the students answer concepts-related questions during class. 
\chirev{Additionally, two teachers expressed cautious optimism about using LLM-generated analogies with their interventions. 
T1 noted that frequently used analogies for physics were concept-based, while AI-generated ones felt more relatable to everyday life, which makes him uncertain about their actual effects. 
Besides, both T1 and T2 anticipated better classroom feedback but were unsure of the effects on students' performance on homework and exams. 
This led to a consensus on conducting a comparative experiment.}
% They also understand student mastery through homework performance.


% \uline{To maintain consistency with their teaching styles, in Study II, we entrusted the method and analysis of evaluating the effectiveness of LLM-generated analogies to the teachers and did not conduct additional tests or evaluations.}

% \uline{\textbf{Summary to results of RQ2:}} 
% Physics teachers usually need analogies based on other concepts to express similar abstract features. 
% In contrast, biology teachers usually require analogies drawn from daily life to describe the structure and function of organisms. 
% Besides, LLM-generated analogies should reflect teachers' specific preferences and needs for which features and parts of the concepts to emphasize, often aligning with their key teaching focus.
% Additionally, teachers seek engaging analogies to capture students' attention.
% However, LLM-generated analogies do not need to be perfect, as teachers can refine them.

\enlargethispage{5pt}

\subsection{Classroom Experiments}
In this subsection, we describe the participants (Sec.~\ref{sec:study22_participants}), data preparation process (Sec.~\ref{sec:study22_data_preparation}), procedure (Sec.~\ref{sec:study21_findings_and_derived_requirments}), and results analysis (Sec.~\ref{sec:study22_results_analysis}) for our classroom experiments. 
\label{sec:study22}
\subsubsection{Participants}
\label{sec:study22_participants}
In this one-week field study, participants included two teachers (T1 and T2) from the pre-class interviews and two first-year high school classes (C1 and C2) they were teaching. 
Each class had 25 students, 12 of whom were girls, and the distribution of their entrance exam scores was very similar.

\subsubsection{Data Preparation}
\label{sec:study22_data_preparation}
Teachers informed us about concepts that might require analogies in the following week of teaching.
The concepts taught by the physics teacher (T1) include average velocity and instantaneous velocity, acceleration, and infinitesimal method.
The concepts taught by the biology teacher (T2) include the various functions of proteins, the adaptation of function and structure, dehydration condensation, the formation of tertiary and quaternary structures, and protein denaturation.
Based on pre-class interviews, we identify four effective strategies to generate analogies from LLMs for classroom practice.


\begin{itemize}
    \item \textbf{Strategy 1: Analogy for Physical Concept.} For physical concepts, analogies often draw on learned physical concepts. 
    % For example, comparing the structure of atoms to the solar system can aid in understanding their complex arrangement.
    \item \textbf{Strategy 2: Analogy for Biological Concept.} For biological concepts, analogies often involve everyday objects. For example, one might use the analogy of fire escape to help understand channel protein.
    \item \textbf{Strategy 3: Vivid Analogy Generation.} Analogies should be vivid and engaging to capture students' attention. 
    % For example, illustrating dry and fresh weight concepts in cells using images of people and mummies can be engaging.
    \item \textbf{Strategy 4: Fine-grained Analogy Generation.} Sometimes, it is sufficient to generate analogies for just one aspect of a concept to provide a detailed explanation of that particular aspect. 
    % For example, creating specific analogies for the tertiary structure of proteins is beneficial.
\end{itemize}

% \begin{tcolorbox}[title=Four Strategies for Generating Analogies for Teachers, mybox]
% \textbf{Strategy 1: Analogy for Physical Concept}

% For physical concepts, analogies typically draw on previously learned physical concepts. For example, comparing the structure of atoms to the solar system can aid in understanding their complex arrangement.

% \textbf{Strategy 2: Analogy for Biological Concept}

% For biological concepts, analogies often involve everyday objects. For example, one might use the analogy of fire escape to help understand channel protein.
% % For example, to help an engineer understand the eye's cross-section, one might use the analogy of a camera's structure.

% \textbf{Strategy 3: Vivid Analogy Generation}

% Analogies should be vivid and engaging to capture students' attention. For example, illustrating dry and fresh weight concepts in cells using images of people and mummies can be both vivid and engaging.

% \textbf{Strategy 4: Fine-grained Analogy Generation}

% Sometimes, it is sufficient to generate analogies for just one aspect of a concept to provide a detailed explanation of that particular aspect. For example, creating specific analogies for the tertiary structure of proteins is beneficial.
% \end{tcolorbox}

Based on the strategies outlined, we can modify the prompt in Tab.~\ref{tab:instruction_prompt} to suit the specific aspect of a concept and the requirements of teachers.
Specifically, we incorporated Strategies 1, 2, and 3 into the \textit{Principles}.
Strategy 4 was added into the \textit{Input Resource} to prevent the model from forgetting.

Following discussions with T1, we generated analogies for average and instantaneous velocity by implementing either strategy 1 or 3. 
Additionally, we generated detailed analogies for the infinitesimal method and acceleration using strategy 4. 
In biology, we produced analogies for proteins by applying either strategy 2 or strategy 3. 
Using strategy 4, we developed detailed analogies for the immune effects of proteins and the formation of tertiary and quaternary structures. 
However, two generated analogies, ``driving speed'' and ``reading speed'', were marked as non-analogies and excluded. 
Besides, the analogies generated with Strategy 1 for physical concepts were not related to other concepts, but we included them as they are vivid analogies.
We generated four physical analogies to T1 and nine biological ones to T2.



\subsubsection{Procedure}
\label{sec:study22_procedure}
In this one-week teaching, T1 and T2 used LLM-generated analogies for C1 and kept the original teaching mode for C2, with each class having 3 lessons for each subject, totaling 12 lessons.
One author attended one C1 lesson taught by T1 and one by T2, observing how teachers used analogies and student reactions without disrupting teaching.
For the remaining lessons within the week, teachers completed our provided record forms after each lesson.
The record forms asked for details on which analogies they chose while preparing for C1, any modifications made to these analogies, and reasons for not selecting others. 
Additionally, the forms inquired about how teachers assessed student feedback during or after class, any differences in feedback between C1 and C2, and whether the feedback met their expectations.
After one week, we conducted one-on-one interviews with T1 and T2, each lasting 20 minutes, to clarify any unclear details in the records, and discuss their experiences with LLM-generated analogies, students' performance, and future expectations.
Both teachers received a \$60 gift card each for their dedicated participation over the week.

\subsubsection{Results Analysis}
\label{sec:study22_results_analysis}

We report the following qualitative findings based on the record forms and interviews.


\textbf{Teachers selected and modified LLM-generated analogies to avoid redundancy, confusion, or misleading students and make them closer to students' daily lives.}
T1 selected two of four analogies and modified one, while T2 chose four of nine analogies and modified two.
In the interview, T2 noted that while the analogies for all four functions of proteins had merits, only two were selected to avoid verbosity in the class. 
Besides, to avoid concept confusion, T2 chose the analogy of ``transport function'' as a ``conveyor belt'' and discarded the analogy of ``catalysis function'' as a "high-speed elevator," due to the transport function of the elevator.
% For modification, \chirev{as shown in the Fig.~\ref{fig:behavior},} 
% T1 modified several analogies in physics that did not clearly distinguish between velocity and speed and explained the difference to clarify the key focus and avoid misleading.
\chirev{
We observed two types of modifications made by teachers to analogies. 
One type involved modifying details, such as T2 changing the security guard's action from ``eliminating'' to ``capturing'' to align with the real-world context \chirev{(Fig.~\ref{fig:behavior}B)}.
Another type involved changing analogy objects, like T1 replacing ``jigsaw puzzle'' with ``pixels on a display screen'' to illustrate the infinitesimal method \chirev{(Fig.~\ref{fig:behavior}C)}.
T1 explained that display screens are more familiar to students than jigsaw puzzles.}

\begin{figure*}[t]
    \centering
    \includegraphics[width=1\linewidth]{figure/revbehavior.pdf}
    \caption{\chirev{Teachers' behaviors on LLM-generated analogies. They may either directly select (A) and use them in class, modify their details (B) or analogy objects (C) to varying extents, or even create entirely new analogies inspired by them (D).}}
    \Description{This figure shows examples of three types of actions that teachers can take with LLM-generated analogies. For analogies of higher quality, such as analogizing ``Average Velocity and Instantaneous Velocity'' to ``videos and snapshots'', teachers directly select and use them in class. Teachers may also modify the details of the analogy, such as changing the security guard’s ``eliminating'' the intruder in the analogy of ``Immune Function of Protein'' to ``capturing and sending him to the police station''; or modify the analogy object, such as changing the ``puzzle and puzzle pieces'' in the analogy of ``Infinitesimal Method'' to the ``entire display and a single pixel''. Teachers may even be inspired by the generated analogies to get new analogies. For example, the generated analogy of ``Dehydration Condensation Reaction'' mentioned the building structure, which inspired the teacher to get a new analogy of ``breaking down the wall between classrooms''.}
    \label{fig:behavior}
\end{figure*}

\looseness-1  \textbf{LLM-generated analogies inspire teachers with new analogies and new teaching methods.}
In the interview, both teachers recognized the novelty of some LLM-generated analogies for concepts, and they had not considered using analogies for those concepts before. 
For example, T1 used ``video and snapshot'' to analogize ``average velocity and instantaneous velocity'' \chirev{(Fig.~\ref{fig:behavior}A)}, while T2 used ``wool folding and weaving'' to analogize the ``tertiary and quaternary structures of protein.''
In addition, T2 developed new analogies inspired by LLM-generated analogies. 
While teaching the ``dehydration condensation reaction,'' T2 explained with a new analogy \chirev{as ``breaking down the wall between classrooms'' (Fig.~\ref{fig:behavior}D).}
In the interview, T2 said, ``\textit{I am not satisfied with the generated one, as comparing the dehydration condensation reaction to mixing building materials doesn't capture the essence. However, \chirev{the buldings environment} inspired me to create a new analogy.}''
Besides, T1 said that participating in this study had changed his teaching style.
T1 used analogies based on everyday life after explaining the concept, which was inconsistent with his pre-class interview response. 
% In the interview, he explained, ``\textit{I didn't use analogies much before but recently I've been trying different ways to incorporate and teach with analogies.}''

\looseness-1  \textbf{LLM-generated analogies boost students' classroom and homework performance \chirev{and encourage teaching with analogy}.}
Both teachers believe that C1 outperforms C2 in both classroom participation and homework.
T1 praised analogies for helping students focus on the class: ``\textit{I can see from the students' eyes that C1 is genuinely paying closer attention, with more students nodding sincerely, rather than just pretending.}''
T1 also reported that C1 outscored C2 by nearly 20\% on a 10-question homework.
He attributed this to C2's confusion between average and instantaneous velocity, causing errors on the two hardest questions."
In the interview, T1 said, ``\textit{I plan to use the analogies in C1 when reviewing the assignments in C2 to explain the concepts again.}''
% In the interview, T1 said, ``\textit{I think there is a clear difference in concept understanding between the two classes. I plan to use the analogies in C1 when reviewing the assignments in C2 to explain the concepts again.}''
As for biological concepts, T2 said, ``\textit{When I explained protein structure using a video, C2 students understood initially but got confused about the tertiary structure, whereas the wool stacking analogy helped C1 students understand the video.}'' 
T2 showed us a fill-in-the-blank question from homework that asked students to summarize protein function. 
Most C1 students summarized correctly, while many C2 students simply copied words from the textbook. 
However, T2 noted that there was no clear difference between the two classes in understanding straightforward concepts like ``Protein denaturation''.
T1 also noted that the physics analogies are still not between concepts and may offer limited help with highly abstract concepts\chirev{, while he added ``\textit{But I'll try more teaching with analogy since the difference between the two classes is clear.''}}

\chirev{Overall, promising feedback from teachers and classroom practice led us to consider designing a practical system to support the preparation of teaching analogies.}
% \chirev{
% Nevertheless, the comparative experiment results motivated the teachers to use analogies more often in future teaching.}

% \uline{\textbf{Summary to results of RQ3.}}
% Teachers are willing to select and modify LLM-generated analogies to ensure they are vivid and engaging, promoting understanding without adding redundancy, confusion, or misleading students. 
% The use of LLM-generated analogies helps maintain student focus, enhances classroom performance and homework outcomes, and inspires teachers to create new analogies and reflect on their teaching methods.
% \chirev{Such effectiveness suggests us to further develop an LLM-assisted system to help teachers construct analogies for classroom teaching. }

\enlargethispage{5pt}





%noting that while AI often provides repetitive content and fails to address specific needs directly, it can still offer ideas that he might not have considered, which helps him with lesson preparation.
%We will also try to generate relatively less used analogy types for teachers' reference.
%However, we also attempt to generate 
%- When analogies are made to biological concepts, they are usually made using everyday objects. For example, an engineer can learn the eye cross-section by taking the analogy of the camera structure.
%- When analogies are made with physical concepts, previously learned physical concepts are usually used. For example, utilizing the analogy of the solar system can facilitate comprehension of the complex structure of atoms.
%The physics teacher (T1) tended to use analogies between physical concepts, such as ``electric field and magnetic field'' or ``Coulomb's law and the law of universal gravitation.'' 
%T1 claimed that physical concepts are often abstract and it can be challenging to find relevant analogies from everyday life. 
%As a result, T1 did not fully align with the six-step teaching model, believing that using analogies highly similar to the objects of concepts themselves was sufficient for physics teaching.
% , drawing on discussions from Steps 1 and 2 and their previous AI experiences in Step 3, along with suggestions for enhancing current analogies.
%We expect participants to explain their needs for analogies used in class from both the teacher and student perspectives.
% Study I has shown that LLM-generated analogies have a \todo effect on students’ understanding of concepts without human intervention.
% We continue to design Study II to first identify the requirements of analogies used in classroom instruction through preclass interviews with teachers and senior students and refine prompting strategies to fit the requirements.
%Then we conduct a field study to evaluate the effectiveness of LLM-generated analogies during classroom teaching.
%In contrast, the biology teacher (T2) almost exclusively used examples from daily life. 
%T2 reported comparing the crossing over of chromosomes to swapping legs between classmates. 
%While some analogies might seem unconventional, they proved effective in teaching.
%- The analogies created should be vivid and engaging to enhance student attention in class. For example, using images of people and mummies to explain the concepts of dry weight and fresh weight of cells can be both vivid and engaging.
%Therefore, when a concept is relatively complex and contains multiple knowledge points, it is sufficient to select only the most important overall feature or a specific aspect for the analogy, rather than trying to cover the entire content.
%Similarly, the two students could not recall many concepts being taught using analogies during interviews. 
%After reviewing the analogies we generated in Study I, they became even more convinced that these analogies were unnecessary for teaching those concepts.
%T2 believed that the teaching focus for ``thyroid hormone regulation'' should be on negative feedback, which can be explained using the analogy of ``adjusting the temperature with an air conditioner remote control,'' while other specific details should be memorized by students. 
%For example, T2 found the analogy comparing ``nerve impulses'' to the ``efficient operation of stations in an express delivery system'' to be unique; 
%while some parts were redundant, one specific paragraph could be adapted for classroom use. 
%Consequently, our discussions with them did not delve into AI usage exploration (Step 3 in teacher interviews).
%Following a similar procedure to Steps 1, 2, and 4 of the teacher interviews, the student sessions focused on their classroom experiences. 
%Initially, students were asked to recall instances in which analogies were used in class, helping to guide analogy usage. 
%Subsequently, we presented all ten concepts from Study I, prompting students to recall how they learned these concepts during class teaching, specifically if analogies were used. 
%Following this, we showcased the ten analogies generated in Study I and asked students to assess the strengths and weaknesses of these from a learned perspective.
%Finally, they shared their opinion on whether viewing the analogies contributed to a better understanding of the concepts.