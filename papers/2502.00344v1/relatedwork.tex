\section{Related work}
With the advancement and widespread adoption of neural network models, some studies have explored their application to the vocalization of non-human animals. Many of such studies employed neural networks to effectively embed animal sounds into lower dimensional spaces, facilitating species detection from their vocal signals~\cite{14, 15, 16}. Some have utilized Transformer-based models, such as HuBERT~\cite{17} and wav2vec 2.0~\cite{18}, enabling more expressive encoding and classification~\cite{19, 20, 21, 22}. 
These previous efforts have predominantly concentrated on analyzing the phonological features of individual syllables. In contrast, this present study emphasizes the sequential relationship among syllables by applying language models to tokenized birdsongs. While transcribing audio sequence into text reduces the overall information contents, it enables analyses focused on sequential patterns, providing novel insights into the structural aspect of communication through the birdsong, a dimension previously overlooked in non-human vocal communication.