\section{Analysis}\label{sec:theory}

In this section we give convergence guarantee for Bilevel ZOFO. Suppose $({ \mathbf{{ \bm{\theta}}}},{ \mathbf{p}})\in\mathbb{R}^{d+d'}$ and ${ \mathbf{s}}\in\mathbb{R}^{d'}$. The following assumptions are made throughout this section.

\begin{assumption}
\label{assumption1} We make the following assumptions:
    \begin{itemize}
        \item $G({ \mathbf{{ \bm{\theta}}}},{ \mathbf{{ \mathbf{p}}}},\cdot)$ can be potentially nonconvex and $G(\cdot,\cdot,{ \mathbf{s}}) $ is $\tau-$ strongly concave; $F({ \mathbf{{ \bm{\theta}}}},{ \mathbf{p}})$ is twice continuously differentiable in ${ \mathbf{{ \bm{\theta}}}}, { \mathbf{p}}$.
        \item $G$ is $\ell$-Lipschitz smooth in $\mathbb{R}^{d+2d'},$ i.e. $\forall ({ \mathbf{{ \bm{\theta}}}}_1,{ \mathbf{p}}_1,{ \mathbf{s}}_1),({ \mathbf{{ \bm{\theta}}}}_2,{ \mathbf{p}}_2,{ \mathbf{s}}_2)\in\mathbb{R}^{d+2d'}$, 
        \begin{multline*}
            \|\nabla G({ \mathbf{{ \bm{\theta}}}}_1,{ \mathbf{p}}_1,{ \mathbf{s}}_1)-\nabla G({ \mathbf{{ \bm{\theta}}}}_2,{ \mathbf{p}}_2,{ \mathbf{s}}_2)\|\leq \\ \ell\|({ \mathbf{{ \bm{\theta}}}}_1,{ \mathbf{p}}_1,{ \mathbf{s}}_1)-({ \mathbf{{ \bm{\theta}}}}_2,{ \mathbf{p}}_2,{ \mathbf{s}}_2)\|.
        \end{multline*}
        We define $\kappa:=\ell/\tau$ as the problem condition number.
        \item $\forall ({ \mathbf{{ \bm{\theta}}}},{ \mathbf{p}},{ \mathbf{s}})\in \mathbb{R}^{d+2d'}$, sample estimates satisfy 
        \begin{equation*}
        \begin{split}
            &\mathbb{E}[G({ \mathbf{{ \bm{\theta}}}},{ \mathbf{p}},{ \mathbf{s}};\xi)]=G({ \mathbf{{ \bm{\theta}}}},{ \mathbf{p}},{ \mathbf{s}}),\\
            &\mathbb{E}[\nabla G({ \mathbf{{ \bm{\theta}}}},{ \mathbf{p}},{ \mathbf{s}};\xi)]=\nabla G({ \mathbf{{ \bm{\theta}}}},{ \mathbf{p}},{ \mathbf{s}}),\\
            &\mathbb{E}\|\nabla G({ \mathbf{{ \bm{\theta}}}},{ \mathbf{p}},{ \mathbf{s}};\xi)-\nabla G({ \mathbf{{ \bm{\theta}}}},{ \mathbf{p}},{ \mathbf{s}})\|^2\leq \frac{\sigma^2}{B}
        \end{split}
        \end{equation*}
        for sample $\xi$ with size $|\xi|=B$ and constant $\sigma>0$.
        \item $\max_{ \mathbf{s}} G({ \mathbf{{ \bm{\theta}}}},{ \mathbf{p}},{ \mathbf{s}})$ is lower bounded.
    \end{itemize}
\end{assumption}

We first discuss the relationship between the optimality condition \eqref{bi_penalized} and \eqref{bi}. We start with defining the $\epsilon$-stationary points of \eqref{bi_penalized} and \eqref{bi} for general bilevel and minimax problems. In the following definitions, the expectation is taken over the randomness in the algorithm that $({ \mathbf{x}},{ \mathbf{y}})$ is generated.

\begin{definition}
Given a bilevel optimization problem $$f^*=\min_{ \mathbf{x}} f({ \mathbf{x}},{ \mathbf{y}}^*({ \mathbf{x}})), { \mathbf{y}}^*({ \mathbf{x}})\in \arg\min_{ \mathbf{z}} g({ \mathbf{x}},{ \mathbf{z}})$$ and any $\epsilon>0$, a point $({ \mathbf{x}}_\epsilon,{ \mathbf{y}}_\epsilon)$ is called an $\epsilon$-stationary point if 
$$\mathbb{E}[\|\nabla f({ \mathbf{x}}_\epsilon,{ \mathbf{y}}^*({ \mathbf{x}}_\epsilon))\|]\leq O(\epsilon), f({ \mathbf{x}}_\epsilon,{ \mathbf{y}}_\epsilon)-\min_{ \mathbf{z}} f({ \mathbf{x}}_\epsilon,{ \mathbf{z}})\leq \epsilon.$$
\end{definition}

\begin{definition}
\label{definition1}
Given a minimax problem $$f^*=\min_{ \mathbf{x}}\max_{ \mathbf{y}} f({ \mathbf{x}},{ \mathbf{y}})$$ and any $\epsilon>0$, a point $({ \mathbf{x}}_\epsilon,{ \mathbf{y}}_\epsilon)$ is called an $\epsilon$-stationary point if 
$$\mathbb{E}[\|\nabla_{ \mathbf{x}} f({ \mathbf{x}}_\epsilon,{ \mathbf{y}}_\epsilon)\|^2]\leq\epsilon^2,\ \mathbb{E}[\|\nabla_{ \mathbf{y}} f({ \mathbf{x}}_\epsilon,{ \mathbf{y}}_\epsilon)\|^2]\leq\epsilon^2.$$
\end{definition}



\begin{lemma}
\label{lemma1}
    If assumption \ref{assumption1} holds and $\lambda=1/\epsilon$, assume that $\nabla^2 F({ \mathbf{{ \bm{\theta}}}},\cdot)$ is Lipschitz continuous and $({ \mathbf{{ \bm{\theta}}}},{ \mathbf{p}},{ \mathbf{s}})$ is an $\epsilon$-stationary point of \eqref{bi_penalized}, then $({ \mathbf{{ \bm{\theta}}}}, { \mathbf{s}})$ is an $\epsilon$-stationary point of \eqref{bi}.
\end{lemma}

% Therefore, we only need to focus on finding an $\epsilon$-stationary point of \eqref{bi_penalized}. For the simplicity of notations, let use define ${ \mathbf{x}}:=({ \mathbf{{ \bm{\theta}}}},{ \mathbf{p}})$, ${ \mathbf{y}}:={ \mathbf{s}}$, $f({ \mathbf{x}},{ \mathbf{y}}):=G({ \mathbf{{ \bm{\theta}}}},{ \mathbf{p}},{ \mathbf{s}})$, ${ \mathbf{y}}^*({ \mathbf{x}}):=\arg\max_{ \mathbf{y}} f({ \mathbf{x}},{ \mathbf{y}})$ and $g({ \mathbf{x}}):=f({ \mathbf{x}},{ \mathbf{y}}^*({ \mathbf{x}})).$

The following is the low effective rank assumption from \cite{MalladiGNDL0A23Mezo}. This assumption avoids dimension $d$ in the total complexity. Following \cite{MalladiGNDL0A23Mezo}, we assume here that ${ \mathbf{z}}^k$ in \eqref{ZO_grad} is sampled from shpere in $\mathbb{R}^{d}$ with radius $\sqrt{d}$ for ease of illustration.

\begin{assumption}
\label{assumption2}
    For any $({ \mathbf{{ \bm{\theta}}}},{ \mathbf{p}},{ \mathbf{s}})\in\mathbb{R}^{d+2d'}$, there exists a matrix $H({ \mathbf{{ \bm{\theta}}}},{ \mathbf{p}},{ \mathbf{s}})$ such that $\nabla^2 G({ \mathbf{{ \bm{\theta}}}},{ \mathbf{p}},{ \mathbf{s}})\preceq H({ \mathbf{{ \bm{\theta}}}},{ \mathbf{p}},{ \mathbf{s}})\preceq \ell\cdot I_d$ and $tr(H({ \mathbf{{ \bm{\theta}}}},{ \mathbf{p}},{ \mathbf{s}}))\leq r \cdot \|H({ \mathbf{{ \bm{\theta}}}},{ \mathbf{p}},{ \mathbf{s}})\|.$
\end{assumption}

\begin{theorem}
\label{theorem1}
    If Assumptions \ref{assumption1} and \ref{assumption2} hold, by setting 
    \begin{equation*}
    \begin{split}
    &\eta=\frac{1}{2\ell}, \zeta=\frac{1}{2\ell r}, \lambda=\frac{1}{\epsilon}, B=O(\sigma^2\epsilon^{-2}),\\
        &\alpha=O(\epsilon \kappa^{-1}(d+d')^{-1.5}),T=O\left(\kappa\log(\kappa\epsilon^{-1})\right),\\
        &K=O(\kappa r \epsilon^{-2})
    \end{split}
    \end{equation*}
    there exists an iteration in Algorithm \ref{bi_minimax_ZOFO} that returns an $\epsilon$-stationary point $({ \mathbf{{ \bm{\theta}}}},{ \mathbf{p}},{ \mathbf{s}})$ for (\ref{equation5}) and it satisfies
    \begin{equation*}
        \begin{split}
            &\mathbb{E}[\|\nabla F({ \mathbf{{ \bm{\theta}}}},{ \mathbf{p}}^*({ \mathbf{{ \bm{\theta}}}});\mathcal{D}_f)\|]\leq O(\epsilon), \\
            &F({ \mathbf{{ \bm{\theta}}}},{ \mathbf{s}};\mathcal{D}_{ \mathbf{p}})-\min_{ \mathbf{p}} F({ \mathbf{{ \bm{\theta}}}},{ \mathbf{p}};\mathcal{D}_{ \mathbf{p}})\leq \epsilon.
        \end{split}
    \end{equation*}
\end{theorem}

\begin{remark}
    The total number of ZO gradient calculations is $$TKB_1+KB_2=O(\sigma^2\kappa^2 r\epsilon^{-4}\log(\kappa\epsilon^{-1})).$$
    This result matches the complexity in previous ZO minimax algorithm in \cite{wang2023zeroth} but solves our bilevel optimization problem (\ref{bi}) and does not depend on the dimensionality $d$ thanks to the efficient rank assumption \ref{assumption2}, providing efficiency guarantee for our algorithm.
\end{remark}