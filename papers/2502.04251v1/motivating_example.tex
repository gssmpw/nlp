% !TeX root = 0_main

\section{Quality Model for Reproduction Steps}
\label{sec:quality_model}

In this paper, we adopt the quality model proposed by Chaparro \etal \cite{Chaparro2019}, with the following quality categories for the steps to reproduce the bug (S2Rs) in a bug report: 
\begin{itemize}
	\item Correct step (\textbf{CS}): the step corresponds to a specific interaction and GUI component on the application.
	\item Ambiguous Step (\textbf{AS}): the step corresponds to multiple interactions on GUI components on the application.
	\item Vocabulary Mismatch (\textbf{VM}): the step does not correspond to any interactions or GUI components on the application due to misaligned terminology.
	\item Missing Steps~(\textbf{MS}): interactions that are required to replicate the bug, but not reported in the bug report.
\end{itemize}

We illustrate the definitions with an example in Figure \ref{fig:bug-report}. 
The bug report presented in the figure comprises six S2Rs, each annotated with the above categories. \noindent\circled{1} The first S2R is \textit{"Change \rev{the} phone setting"}, which does not represent any interactions in the app. 
Therefore, this S2R is annotated as \textbf{VM}.
\noindent\circled{2} The second S2R contains only one individual S2R, \textit{"Open \rev{Mileage Tracker}"}, representing only one app interaction. Therefore, this S2R is annotated as \textbf{CS}. \noindent\circled{3} The third S2R, \textit{"Navigate to the `Service Intervals' screen"}, does not immediately follow after the second step. There is a required intermediate step, \textit{"Open the app menu"}, which must be performed by tapping the "three dots" button in the bottom left menu bar of Screen 1.
Therefore, this missing step is included in the quality report and annotated as \textbf{MS}. 
With this missing step added, the third reported S2R requires a single interaction that can be reliably mapped to the GUI, \ie\ performing a \textit{click} operation on the "Service Interval" button on Screen~1. Therefore, it is categorized as \textbf{CS}. 
\noindent\circled{4} \textit{"Tap on `Add Service Interval'} requires only one interaction in the GUI, \ie\ performing a \textit{click} operation on "Add Service Interval" component on Screen~2, and hence, is annotated as \textbf{CS}. 
\noindent\circled{5} The fifth S2R, \textit{"I entered the information for my next
oil change"}, requires multiple operations. At first, a user has to enter the \textit{"Oil change title"} by performing a \textit{type} operation on the "title" text field at the top of Screen 3. The individual S2R for this interaction is \textit{"Enter Oil change title"}. Secondly, s/he has to enter a value in the "Odometer" text field on Screen 3 by performing a \textit{type} operation which implies an individual S2R: \textit{"Enter distance on odometer field"}. Finally, s/he has to perform a \textit{click} operation on the "Add Service Interval" button on Screen 3. This interaction represents the individual S2R: \textit{"Tap on Add service interval button"}. As three interactions are required to complete the fifth step in the bug report, it is labeled as \textbf{AS}.
\noindent\circled{6} To execute the sixth S2R, \textit{"I added a second service my yearly State Inspection"}, there is another step missing, \textit{"Open the app menu"}, and it is labeled as \textbf{MS}. 
Moreover, the sixth step requires the same three individual S2Rs as the fifth step and annotated as \textbf{AS}. In the next section, we explain how this quality model can be automatically applied to bug reports.


