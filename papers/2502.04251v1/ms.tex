
\documentclass[10pt,conference]{IEEEtran} 

\IEEEoverridecommandlockouts


%%%%%%%%%%%%%%%%%%%%%%%%%%%%%%%%%%%%%%%%%%%%%%%%%%%%%%%
%%%%%%%%%%%%%%%    theorems %%%%%%%%%%%%%%%%%%%%%%%%%%%
%%%%%%%%%%%%%%%%%%%%%%%%%%%%%%%%%%%%%%%%%%%%%%%%%%%%%%%
% \usepackage{mdframed}
\usepackage{kantlipsum}

%%%%%%%%%%%%%%%%%%%%%%%%%%%%%%%%%%%%%%%%%%%%%%%%%%%%%%%
%%%%%%%%%%%%%%%    theorems %%%%%%%%%%%%%%%%%%%%%%%%%%%
%%%%%%%%%%%%%%%%%%%%%%%%%%%%%%%%%%%%%%%%%%%%%%%%%%%%%%%
\theoremstyle{plain}
\newtheorem{theorem}{Theorem}[section]
\newtheorem{proposition}[theorem]{Proposition}
\newtheorem{lemma}[theorem]{Lemma}
\newtheorem{example}[theorem]{Example}
\newtheorem{corollary}[theorem]{Corollary}
\theoremstyle{definition}
\newtheorem{definition}[theorem]{Definition}
\newtheorem{assumption}[theorem]{Assumption}
\theoremstyle{remark}
\newtheorem{remark}[theorem]{Remark}


% \titleformat{\subsection}[runin]% runin puts it in the same paragraph
%        {\normalfont\bfseries}% formatting commands to apply to the whole heading
%        {\thesubsection}% the label and number
%        {0.5em}% space between label/number and subsection title
%        {}% formatting commands applied just to subsection title
%        [.]% punctuation or other commands following subsection title


%%%%%%%%%%%%%%%%%%%%%%%%%%%%%%%%%%%%%%%%%%%%%%%%%%%%%%%
%%%%%%%%%%%%%%%  mathematical notations%%%%%%%%%%%%%%%%
% \usepackage[english]{babel}
% \usepackage[utf8]{inputenc}
% \usepackage[T1]{fontenc}

%% Figures, tables and lists
\usepackage[dvipsnames]{xcolor}
\usepackage{paralist}
\usepackage{graphicx}
\usepackage{subcaption}
\usepackage{longtable} 
\usepackage{multirow}
\usepackage{listings}
\usepackage{makecell}
\usepackage{array}
\usepackage{float}
\usepackage{dsfont}
\usepackage{rotating}
\usepackage{booktabs}
\usepackage{enumerate}
\usepackage{tikz}
\usepackage{pgf}
\usepackage{enumitem}
\usepackage{lipsum} % for generating filler text
\usepackage{titlesec}

%% Math
% \usepackage{amssymb, amsthm,bbm}
\usepackage{mathtools}
\usepackage{mathrsfs}
%% References and author info 
\mathtoolsset{showonlyrefs}
\usepackage{natbib}
\usepackage{authblk}
\usepackage{todonotes}
\usepackage{xr-hyper}


%%%%%%%%%%%%%%%%%%%%%%%%%%%%%%%%%%%%%%%%%%%%%%%%%%%%%%%
\newcommand{\R}{\mathbb R}
\newcommand{\EE}{\mathbb{E}}

\DeclareMathOperator{\Tr}{Tr}
\DeclareMathOperator*{\argmin}{argmin}
\DeclareMathOperator*{\argmax}{argmax}

\newcommand{\bs}[1]{\ensuremath{\boldsymbol{#1}}}
\newcommand{\mc}{\mathcal}
\newcommand{\opt}{^\star}


\newcommand{\diff}{\textnormal{d}}


\def \iid {\stackrel{\textnormal{i.i.d.}}{\sim}}
\def \iidtext {\textnormal{i.i.d.}}





%%%%%%%%%%%%%%%%%%%%%%%%%%%%%%%%%%%%%%%%%%%%%%%%%%%%%%%
%%%%%%%%%%%%%%%%%%%%% colors     %%%%%%%%%%%%%%%%%%%%%%
%%%%%%%%%%%%%%%%%%%%%%%%%%%%%%%%%%%%%%%%%%%%%%%%%%%%%%%
\definecolor{myblue}{rgb}{.8, .8, 1}
\definecolor{mathblue}{rgb}{0.2472, 0.24, 0.6} % mathematica's Color[1, 1--3]
\definecolor{mathred}{rgb}{0.6, 0.24, 0.442893}
\definecolor{mathyellow}{rgb}{0.6, 0.547014, 0.24}


% May add more in future.








\begin{document}

\title{Combining Language and App UI Analysis for the Automated Assessment of Bug Reproduction Steps
\thanks{The first two authors contributed equally to this work.}
}

\author{
	
	\IEEEauthorblockN{Junayed Mahmud\IEEEauthorrefmark{1}, Antu Saha\IEEEauthorrefmark{2}, Oscar Chaparro\IEEEauthorrefmark{2}, Kevin Moran\IEEEauthorrefmark{1}, Andrian Marcus\IEEEauthorrefmark{3}}
    \IEEEauthorblockA{\IEEEauthorrefmark{1}\textit{University of Central Florida (USA)}, 
		    \IEEEauthorrefmark{2}\textit{William \& Mary (USA)},
		    \IEEEauthorrefmark{3}\textit{George Mason University (USA)}
		    \\ \href{mailto:}{junayed.mahmud@ucf.edu}, \href{mailto:}{asaha02@wm.edu}, \href{mailto:}{oscarch@wm.edu},
		    \href{mailto:}{kpmoran@ucf.edu},
		    \href{mailto:}{amarcus7@gmu.edu}
    }
}

\maketitle

%page numbering
\thispagestyle{plain}
\pagestyle{plain}

\begin{abstract}
Bug reports are essential for developers to confirm software problems, investigate their causes, and validate fixes. Unfortunately, reports often miss important information or are written unclearly, which can cause delays, increased issue resolution effort, or even the inability to solve issues. One of the most common components of reports that are problematic is the steps to reproduce the bug(s) (S2Rs), which are essential to replicate the described program failures and reason about fixes. Given the proclivity for deficiencies in reported S2Rs, prior work has proposed techniques that assist reporters in writing or assessing the quality of S2Rs. However, automated understanding of S2Rs is challenging, and requires linking nuanced natural language phrases with specific, semantically related program information. Prior techniques often struggle to form such language $\leftrightarrow$ program connections -- due to issues in language variability and limitations of information gleaned from program analyses. 

To more effectively tackle the problem of S2R quality annotation, we propose a new technique called \tool, which leverages the language understanding capabilities of LLMs to identify and extract the S2Rs from bug reports and \rev{map} them to GUI interactions in a program state model derived via dynamic analysis. We compared \tool to a related state-of-the-art approach and we found that \tool annotates S2Rs 25.2\% better (in terms of F1 score) than the baseline. 
Additionally, \tool suggests more accurate missing S2Rs than the baseline (by 71.4\% in terms of F1 score). 
\looseness=-1
\end{abstract}




\section{Introduction}
\label{sec:intro}
% Image editing methods in diffusion models depend on user-defined control directions - users can unlock their creativity using these methods by specifying the desired manipulation through prompts~\cite{gandikota2023concept}, reference images~\cite{ruiz2022dreambooth, kumari2022customdiffusion, gal2022image, chen2024trainingfreeregionalpromptingdiffusion}, or attribute vectors~\cite{parmar2023zero,hertz2022prompt}. In this work, we ask a fundamentally different question: \emph{Can we automatically discover the underlying visual structure of a concept within diffusion model's knowledge?} %Rather than requiring user-specified controls, we aim to decompose the model's internal knowledge into meaningful directions.

% This question touches on a fundamental limitation in how we interact with diffusion models. Current control methods ~\cite{zhang2023addingconditionalcontroltexttoimage, gandikota2023concept, ye2023ipadaptertextcompatibleimage,ye2023ipadaptertextcompatibleimage, hertz2024stylealignedimagegeneration, li2023photomaker, shi2024instantbooth, chen2024trainingfreeregionalpromptingdiffusion} require users to specify their desired manipulations in advance, limiting interactive creativity. This contrasts with natural human artistic workflows, where creators dynamically explore creative ideas while jointly refining them toward meaningful artistic outcomes~\cite{hoffmann2016modeling}. This synergy between specification and exploration is not new to generative models. Early GAN architectures naturally developed disentangled latent spaces that enabled continuous\cite{harkonen2020ganspace,radford2015unsupervised, wu2021stylespace, shen2020interfacegan}, compositional control over generated images. Users could explore these spaces to discover interesting variations that would be difficult to describe in words~\cite{wu2021stylespace}, then combine them to achieve their creative goals~\cite{grabe2022towards}. 


% While diffusion models have largely superseded GANs in conditional image synthesis~\cite{dhariwal2021diffusion},  their underlying structure remains less understood. Diffusion models achieve remarkable diversity through high-dimensional latents, unlike GANs' compact latent spaces.  With a single prompt, diffusion models can generate radically different variations through different random initializations of input noise. We ask - Is it possible to discover interpretable structure within this vast space of variations?

Text-to-image diffusion models are capable of generating remarkable visual variations from a single prompt through different random initializations. However, this vast creative potential remains largely opaque to users---while we can generate diverse images, we lack understanding of the underlying structure of these variations. This presents a fundamental challenge: how can we discover and expose the latent visual capabilities encoded within these models?

\let\thefootnote\relax \footnote{$^{*}$Correspondence to \texttt{gandikota.ro@northeastern.edu}}

The challenge touches on a key limitation in how we interact with diffusion models today. Current control methods require users to explicitly specify their desired edits in advance through prompts~\cite{gandikota2023concept}, reference images~\cite{zhang2023addingconditionalcontroltexttoimage, chen2024trainingfreeregionalpromptingdiffusion, ruiz2022dreambooth,kumari2022customdiffusion, Ryu_lora, hu2021lora}, or attribute vectors~\cite{ye2023ipadaptertextcompatibleimage, hertz2024stylealignedimagegeneration, li2023photomaker, shi2024instantbooth,parmar2023zero,hertz2022prompt}. That contrasts sharply with natural human creative workflows, where artists dynamically explore creative ideas and jointly refine them toward meaningful artistic outcomes~\cite{hoffmann2016modeling}. The need for pre-specified controls creates a barrier between users and the full creative potential of these models.

Interestingly, earlier generative models like GANs~\cite{gans,karras2019style,brock2018large} naturally developed more interpretable internal structures. Their compact latent spaces often exhibited emergent disentanglement~\cite{harkonen2020ganspace,radford2015unsupervised, wu2021stylespace, shen2020interfacegan}, enabling continuous and compositional control over generated images. Users could explore these spaces to discover interesting variations that would be difficult to describe in words~\cite{wu2021stylespace}, then combine them to achieve their creative goals~\cite{grabe2022towards}.

Diffusion models have largely superseded GANs in conditional image synthesis~\cite{dhariwal2021diffusion}, achieving greater diversity through much higher-dimensional latents. And yet an understanding of the underlying structure of these larger latent spaces has remained elusive. In this work, we ask a fundamental question: \emph{Can we automatically discover the visual structure within a diffusion model's knowledge of a concept?} Rather than requiring user-specified controls, we aim to decompose the model's internal representations into expressive directions that users can explore and combine.

To address these needs, we present \textbf{SliderSpace}, a framework that brings systematic explorability to diffusion models. Given just a text prompt, SliderSpace discovers a canonical set of meaningful, diverse, and controllable directions within the model's knowledge of that concept. Each direction is implemented as a low-rank adapter~\cite{hu2021lora} that can be scaled and composed with others, allowing users to explore and smoothly combine different aspects of variation, as shown in Figure~\ref{fig:intro}.

We ground SliderSpace discovery in three key requirements for meaningful decomposition of a diffusion model's visual manifold: 
\begin{enumerate}
    \item \textbf{Unsupervised Discovery:} The decomposition process should emerge from the intrinsic structure of the model's learned representation, rather than being guided by predefined attributes. This ensures we capture the true topology of the model's knowledge space rather than projecting our assumptions onto it.
    
    \item \textbf{Semantic Orthogonality:} Each discovered control must represent a distinct semantic direction. This is enforced in a semantic feature space, like CLIP, where every slider has an orthogonal effect in embeddings. This prevents discovering multiple controls that create similar semantic effects, making the system more efficient and easier.
    
    \item \textbf{Distribution Consistency:} Directions must induce consistent transformations across both random seeds and prompt variations. 
\end{enumerate}

These requirements naturally lead to our proposed framework, which we formalize in Section~\ref{sec:method}. As we show in our experiments, SliderSpace is architecture-agnostic, working with both conventional U-Net based models like Stable Diffusion~\cite{rombach2022high, rombach2022sd20, podell2023sdxl, turbo, dmd} and recent transformer-based architectures like Flux~\cite{flux}.

We demonstrate the expressiveness of SliderSpace through three applications: First, we show how SliderSpace can decompose high-level concepts into diverse and expressive components, revealing the natural axes of variation in the model's understanding. Second, we explore artistic style variation, where SliderSpace discovers directions that match or exceed the diversity of manually curated artist lists while being judged more useful by human evaluators. Finally, we show how SliderSpace can help reverse the mode collapse commonly observed in distilled diffusion models, restoring diversity while maintaining generation speed.

Beyond providing practical creative control, SliderSpace opens new avenues for understanding and utilizing the latent capabilities of diffusion models. By mapping these models' visual potential into intuitive, composable directions, we take a step toward making their creative possibilities more accessible and interpretable to users.

% Image editing methods in diffusion models unlock the creativity of users. In this work we ask an alternate question: \emph{Can we organize and expose what of the diffusion model is already capable of?}.
% Existing methods for controlling image generation typically require users to manually specify edit directions for desired changes. This process is time-consuming, requires technical expertise, and limits the spontaneity of the creative process. For instance, if a user wants to adjust the smile of a generated person, they must explicitly request this edit, often through imprecise prompt engineering or model fine-tuning. This approach of predefined controls or manual specifications restricts users from fully exploring the latent capabilities of the model. There may be interesting stylistic variations or attributes that the model can generate, but users have no easy way to discover or utilize these.

% Natural visual disentanglement was an emergent property in the latent space of Generative Adversarial Models (GANs) \cite{harkonen2020ganspace,radford2015unsupervised, wu2021stylespace, shen2020interfacegan}. In particular, it has been observed that StyleGAN~\cite{karras2019style} stylespace neurons offer detailed control over many meaningful aspects of images that would be difficult to describe in words~\cite{wu2021stylespace}. However, diffusion models do not share such a compact latent space~\cite{park2023unsupervised}; and efforts to uncover such a space in the semantic embeddings of the text conditioning have met with limited success \nik{Nick - is there a specific citation you were thinking about?}.

% In this work we introduce \textbf{SliderSpace}, which takes a step towards uncovering an analogous low dimensional representation of diffusion models' visual breadth; in essence treating the diffusion model as many generators sharing parameters, where a particular generator is defined by a specific prompt. For a given prompt we sample many random seeds (and optionally prompt expansions using an LLM), generate the corresponding images, and apply an off the shelf feature extractor (in this work CLIP, but our method can be applied to any differentiable feature extractor). We use PCA to analyze these features, and for each of the leading $k$ principal components we train a LoRA \cite{} which causes the diffusion model to produces images which increase the feature magnitude along that component when passed back through the same feature extractor. This leads to a 'Slider' for each principal component, because each LoRA can be scaled and applied to the original diffusion model, continuously varying those visual features in the generated results (as measured, in our case, by CLIP).

% There are many other works that enhance the controllability of diffusion models. One common approach is enabling users to add spatial constraints to a generation either manually, or via a reference image \cite{zhang2023addingconditionalcontroltexttoimage, chen2024trainingfreeregionalpromptingdiffusion}, a second is leveraging more abstract embeddings (e.g. identity, style) extracted from a reference image \cite{ye2023ipadaptertextcompatibleimage, hertz2024stylealignedimagegeneration, li2023photomaker, shi2024instantbooth}, a third is finetuning a foundation model to better generate a concept important to the user \cite{ruiz2022dreambooth, kumari2022customdiffusion, Ryu_lora, hu2021lora}, and a fourth (most relevant to this work) is finding low-rank adaptors of the model based on a prompt or small training set which can be scaled to provide continous control over one aspect of generated image (e.g. night vs day, basic vs luxury, etc.) \cite{gandikota2023concept}. SliderSpace is complementary to all of these methods and offers something distinct. All of the other methods we are aware require the user (and / or model designer) to know in advance what type of control they want. In contrast SliderSpace assists users in discovering and controlling hidden capabilities present in the diffusion model's distribution of possible generations.

%We propose that truly intuitive creative control in a text-to-image model should meet three key criteria: \emph{discoverability}, \emph{intuitiveness}, and \emph{specificity}. The model should reveal controllable attributes that may not be immediately obvious, offer controls that are easy to understand and manipulate, and ensure each control affects a distinct attribute of the generated image.

% We demonstrate the utility and power of SliderSpace using three applications built on top of SDXL-DMD \cite{dmd}, because its fast generation speed lends itself well to the continuous control offered by SliderSpace.

% First, we study concept decomposition (Section \ref{sec:concept_exp}), where we learn sliders for a specific concept (e.g. 'monster', 'waterfall', 'car'). Through quantitative metrics of diversity and text alignment we demonstrate that the learned sliders dramatically boost the diversity of generations when randomly applied without harming text alignment; we also ask humans to qualitatively judge these results in a user study where they find the SliderSpace results to be more 'Diverse', 'Useful', and 'Creative' than our baselines.

% Second, we attempt to compare the automatic discoveries of SliderSpace to a large scale manual study of artistic styles (Section \ref{sec:art_exp}), open-sourced by ParrotZone \cite{parrotzone}. In this study SDXL was prompted with over 4300 artist names,  and based on visual inspection the cases of successful stylistic mimicry recorded. Quantitatively SliderSpace more closely matches the distribution of artistic variation discovered by ParrotZone than other baselines, and in our user studies was judged to be significantly more 'Diverse' and 'Useful' than the baselines. To our surprise humans even judged SliderSpace results to be slightly more 'Diverse' than the results generated by the manually discovered artist names of \cite{parrotzone}.

% Third, we attempt to use SliderSpace to reverse the mode collapse commonly observed in distilled few-step diffusion models relative to the original teacher model (Section \ref{sec:diverse_exp}). We quantitatively demonstrate that applying SliderSpace to SDXL-DMD leads to more closely matching the distribution of images by the original teacher, SDXL.

%Through extensive experiments on various state-of-the-art text-to-image models, we demonstrate that SliderSpace significantly enhances user control and creative expression in AI-assisted image generation tasks. Our method enables a range of applications, including concept decomposition and control, diversity improvement in generated images, customization dissection and edits, and the exploration of artistic styles inherent in the model.

% SliderSpace goes beyond providing a practical tool for enhanced creative control. By mapping the visual potential of diffusion models it can open new avenues for generative creativity and deepens our understanding of each model's hidden potential.

% !TeX root = 0_main

\section{Quality Model for Reproduction Steps}
\label{sec:quality_model}

In this paper, we adopt the quality model proposed by Chaparro \etal \cite{Chaparro2019}, with the following quality categories for the steps to reproduce the bug (S2Rs) in a bug report: 
\begin{itemize}
	\item Correct step (\textbf{CS}): the step corresponds to a specific interaction and GUI component on the application.
	\item Ambiguous Step (\textbf{AS}): the step corresponds to multiple interactions on GUI components on the application.
	\item Vocabulary Mismatch (\textbf{VM}): the step does not correspond to any interactions or GUI components on the application due to misaligned terminology.
	\item Missing Steps~(\textbf{MS}): interactions that are required to replicate the bug, but not reported in the bug report.
\end{itemize}

We illustrate the definitions with an example in Figure \ref{fig:bug-report}. 
The bug report presented in the figure comprises six S2Rs, each annotated with the above categories. \noindent\circled{1} The first S2R is \textit{"Change \rev{the} phone setting"}, which does not represent any interactions in the app. 
Therefore, this S2R is annotated as \textbf{VM}.
\noindent\circled{2} The second S2R contains only one individual S2R, \textit{"Open \rev{Mileage Tracker}"}, representing only one app interaction. Therefore, this S2R is annotated as \textbf{CS}. \noindent\circled{3} The third S2R, \textit{"Navigate to the `Service Intervals' screen"}, does not immediately follow after the second step. There is a required intermediate step, \textit{"Open the app menu"}, which must be performed by tapping the "three dots" button in the bottom left menu bar of Screen 1.
Therefore, this missing step is included in the quality report and annotated as \textbf{MS}. 
With this missing step added, the third reported S2R requires a single interaction that can be reliably mapped to the GUI, \ie\ performing a \textit{click} operation on the "Service Interval" button on Screen~1. Therefore, it is categorized as \textbf{CS}. 
\noindent\circled{4} \textit{"Tap on `Add Service Interval'} requires only one interaction in the GUI, \ie\ performing a \textit{click} operation on "Add Service Interval" component on Screen~2, and hence, is annotated as \textbf{CS}. 
\noindent\circled{5} The fifth S2R, \textit{"I entered the information for my next
oil change"}, requires multiple operations. At first, a user has to enter the \textit{"Oil change title"} by performing a \textit{type} operation on the "title" text field at the top of Screen 3. The individual S2R for this interaction is \textit{"Enter Oil change title"}. Secondly, s/he has to enter a value in the "Odometer" text field on Screen 3 by performing a \textit{type} operation which implies an individual S2R: \textit{"Enter distance on odometer field"}. Finally, s/he has to perform a \textit{click} operation on the "Add Service Interval" button on Screen 3. This interaction represents the individual S2R: \textit{"Tap on Add service interval button"}. As three interactions are required to complete the fifth step in the bug report, it is labeled as \textbf{AS}.
\noindent\circled{6} To execute the sixth S2R, \textit{"I added a second service my yearly State Inspection"}, there is another step missing, \textit{"Open the app menu"}, and it is labeled as \textbf{MS}. 
Moreover, the sixth step requires the same three individual S2Rs as the fifth step and annotated as \textbf{AS}. In the next section, we explain how this quality model can be automatically applied to bug reports.




% !TeX root = 0_main

\section{\tool: Automated S2R Quality Assessment}
\label{sec:approach}
\begin{figure*}[t]
		\vspace{-2em}
		\centering
		\includegraphics[width=1\linewidth]{figures/approach.pdf}
		\caption{{The \tool Approach}}
		\label{fig:approach}
\end{figure*}

This section presents \tool, an automated approach that leverages an LLM and a graph-based app execution model to assess the quality of the steps to reproduce (S2Rs) in textual bug reports.  
\tool identifies, extracts, and processes the S2Rs from a bug report to detect which ones are correct, ambiguous, missing, or phrased using language that does not correspond to a target app, according to the quality model described in \Cref{sec:quality_model}.
 \tool generates a quality report with annotations that provide feedback to the reporter about problematic S2Rs and includes generated missing S2Rs.
\tool has four main components, as illustrated in \Cref{fig:approach}:
\begin{enumerate}
	\item \textbf{S2R sentence identification}: \tool identifies the sentences that describe any S2Rs (\Cref{sec:identification-phase}).
	\item \textbf{Individual S2R extraction}: \tool extracts phrases describing individual S2Rs from S2R sentences (\Cref{sec:indiv-s2rs-approach}).
	\item \textbf{App execution model generation}: \tool builds a graph-based model using automated and manual app execution (\Cref{sec:execution-model}).  
	\item \textbf{S2R quality assessment}: \tool maps individual S2Rs to GUI-level interactions captured in the app execution model, providing feedback about high- and low-quality S2Rs as well as missing steps in a quality report (\Cref{sec:quality-assessment-annotations}).
\end{enumerate}

We leverage the language processing capabilities of LLMs \rev{(\ie\ GPT-4)} across the three phases, integrating these with GUI-level dynamic app analysis to assess S2R quality. \rev{The selection of GPT-4 as the LLM was based on its demonstrated effectiveness in language and bug understanding tasks, including bug reproduction~\cite{Feng2024} and analysis~\cite{Bo2024}.}
In the remainder of this section, we detail \tool's components or phases.



\subsection{S2R Sentence Identification Phase}
\label{sec:identification-phase}

\tool automatically identifies sentences that describe any steps to reproduce (S2R) in the bug report (see the blue sentences in Fig. \ref{fig:bug-report}). 
This is necessary as the bug report typically includes other content, notably the observed (OB) and expected app behaviors (EB). 
We formulate this task as a text classification task, using LLMs. 
\tool decomposes the bug report into a list of sentences and asks the LLM to identify which of these sentences describe any S2Rs. 
Sentence parsing is done using the Stanford CoreNLP toolkit~\cite{manning2014stanford} and heuristics. 
We experimented with three types of prompts, each one providing a different context to facilitate the task for the LLM (\eg the definition of S2Rs and guidelines on how to distinguish them from other content like the OB and EB). 
\Cref{sec:prompt_development} describes the process we followed to develop and evaluate these prompts.
\looseness=-1



\subsection{Individual S2R Extraction Phase}
\label{sec:indiv-s2rs-approach}

After identifying S2R sentences, \tool asks the LLM to extract the individual S2Rs from these sentences in a particular format (described below). 
Individual S2Rs are phrases that describe a single, atomic interaction with the app. 
Individual S2R extraction is needed because S2R sentences may describe multiple interactions with the app together with content such as the OB (\eg ``I opened the app and clicked on the Start button'' or ``The app crashes if the user checks the Angle Box''). 
In addition, different S2R sentences may describe the same interaction (\eg "... the user checks the Angle Box" and "give the Exercise a name and check the Angle Box"). 
\tool resolves this redundancy by asking the LLM to provide only one S2R among all extracted individual S2Rs that describe the same interaction.
\looseness=-1

The output of this phase is a list of individual S2Rs extracted in the order they appear in the sentences from left to right and top to bottom. 
\tool asks the LLM to represent the individual S2Rs in the following format: 
\texttt{\small[action][object][preposition][object2]}. 
The \texttt{\small[action]} is a verb associated with the app interaction (tap, long tap, enter, \etc). 
The \texttt{\small[object]} is the GUI component upon which the action is performed. 
The \texttt{\small[object2]} is additional information related to the object connected by a \texttt{\small[preposition]}. 
For example, the S2R ``Click any button on this page" is formatted as \texttt{\small[Click] [any button] [on] [this page]}.

We designed and evaluated three prompt types to extract individual S2Rs via GPT-4. 
Each prompt implements a different approach, providing different contexts about the task (\eg examples that illustrate how to accomplish the task). 
\Cref{sec:prompt_development} details the prompts and the process we followed to design and evaluate them.
\looseness=-1
 
\subsection{App Execution Model Generation Phase}
\label{sec:execution-model}

\tool's quality assessment relies on mapping individual S2Rs to interactions that can be executed on the app to replicate the reported bug. 
This requires collecting and representing possible user GUI interactions, for which we adapt graph-based representations and dynamic app execution strategies from prior work~\cite{song2022toward,saha2024toward}.

\tool creates an app execution model represented as a directed graph, $G = (V, E)$, where $V$ represents the set of unique GUI screens for an app, and $E$ represents the set of unique interactions that users can perform on the GUI components of the screens. 
A GUI screen (\ie node) is represented as a hierarchy of the GUI components and layouts. 
Two GUI screens with different GUI component hierarchies are considered distinct graph nodes. 
Each interaction (\ie edge) in $E$ is represented by a unique tuple in the form of ($v_x, v_y, e, c)$, where $c$ is a GUI component of screen $v_x$ and  $e$ in an action (tap, type, \etc) performed on $c$, resulting in a transition to another screen $v_y$. 
Each edge contains additional interaction metadata such as the interacted GUI component type, ID, text (\ie label), and description. 

To build the execution model for an app, \tool parses GUI interaction traces collected from automated app exploration and manual app usage.  \tool executes an adapted version of the \CrashScope tool~\cite{Moran2016, Moran2017}, which implements multiple
automated exploration strategies to interact with the UI components of app screens, trying to exercise as many app screens and GUI components as possible.
In the process, \CrashScope collects app screenshots and XML-based GUI hierarchies and metadata for the exercised app UI screens and components. 
As \CrashScope may \rev{fail} to interact with certain GUI screens and components that app users would normally interact with, \tool can also make use of interaction data collected from manual app usage and testing. 
In this paper, for the development set, we used the set of traces collected by Saha \etal~\cite{saha2024toward} which consists of 10-12 manually recorded feature interaction traces for each of the 5 test applications. For the prompt development dataset, two authors collected the same number of traces for each of the 31 apps. These recordings include all the app GUI interactions starting from launching the application to the last step related to carrying out an application feature (more details of this process, used for prompt development and evaluation, are found in \Cref{sec:dev_dataset}). 
In practical applications of \tool, manual executions can be collected in several ways.
For example, developers can enable user monitoring features in the app and perform record-and-replay during in-house or crowd-sourced app testing~\cite{du2022semcluster}. 
\tool parses the interaction traces generated by \CrashScope and the traces collected during app usage/testing to build the graph, according to the graph format we previously described in this section (details found in \Cref{sec:dev_dataset}). 
 
\subsection{S2R Quality Assessment Phase}
\label{sec:quality-assessment-annotations}

The app execution model captures possible interaction sequences that a user could perform when using or testing an app as paths in the graph. 
To assess the quality of the S2Rs, \tool attempts to map each individual S2R to interactions (\ie edges) along these paths. 
To do so, \tool implements an LLM-guided depth-first-search (DFS) graph traversal 
to establish the correspondence between an individual S2R and interactions on a given screen. 

Any S2Rs that cannot be mapped to a graph interaction are labeled as having a Vocabulary Mismatch (\textbf{VM}).  
S2Rs that map to multiple interactions performed on a single screen (\ie a node) are labeled as Ambiguous Steps (\textbf{AS}). 
Those that map to single interactions within a sequence are labeled as Correct Steps (\textbf{CS}). 
Finally, for the mapped S2Rs that correspond to non-consecutive interactions spanning different screens in a path, additional interactions are required to connect them to form a complete path. 
These additional interactions are used to generate individual S2Rs that are labeled as Missing Steps~(\textbf{MS}) and used to fill in the "gaps" between the existing S2Rs.

\subsubsection{Mapping Individual S2Rs to Interactions on a Screen}
\label{sec:qualtiy_phase:mapping_single_screen}
Mapping an individual S2R (S2R, hereon)
to interactions on a given screen is supported by GPT-4. 
For a graph node (\ie a screen), \tool asks GPT-4 to identify which of the outgoing edges (\ie interactions) from that node correspond to the S2R. 
Both the S2R and graph interactions are represented textually: the S2R is extracted from the bug report, while each interaction is represented as a tuple of textual information (\eg the event description and the label of the interacted GUI component).  
We designed and evaluated a set of prompts using different prompting strategies to accomplish this mapping in a 2-step manner: a first prompt asks GPT-4 to return a yes/no answer on whether an individual S2R maps to the interactions of a given screen and if the answer is yes, a second prompt asks GPT-4 to return the list of corresponding interactions. The methodology used to develop and evaluate the prompts is detailed in \Cref{sec:prompt_development_methodology}. 

\subsubsection{Graph Traversal and S2R Mapping to Interaction Paths}
\label{sec:graph-traversal}

To map all the S2Rs from a bug report to app interaction sequences, \tool implements an algorithm that traverses the graph in a depth-first-search (DFS) manner, aiming to \rev{map} the S2Rs to interactions along the DFS paths. 
When S2Rs map to non-consecutive interactions within a path, \tool connects these interactions by selecting the shortest path between the nodes where these interactions occur. 
Since multiple paths may map to the S2Rs, \tool selects the path with the most mapped S2Rs or the shortest path, if multiple paths have the same number of mapped S2Rs.

The DFS traversal of the graph is guided by the LLM-based mapping approach from \Cref{sec:qualtiy_phase:mapping_single_screen}, as only edges that map to S2Rs are traversed, avoiding the need to explore the entire graph. 
While none of the S2Rs can map to any interaction in the graph (in which case \tool would traverse the graph entirely), this scenario is expected to be rare, as we assume reporters would describe at least one S2R using the app’s vocabulary and the graph is as complete as possible, covering a broad range of screens and interactions.

\textbf{\textit{Algorithm Details.}} 
\tool's DFS-based graph traversal algorithm is recursive. 
It receives an S2R $s$ and graph node~$n$ as input, where $s$ is the first item in the S2R list $L$ (the bug report S2Rs). 
The algorithm returns either: the best DFS path $p$ (starting from $n$) that maps to a subset of S2Rs in $L$ (possibly including $s$), or no path if no S2Rs can be mapped. 

The traversal begins with the first S2R from the bug report and the starting node of the graph, which contains "\textit{open app}" interactions that navigate to the screens users usually see upon launching the application. 

The algorithm has two main logic branches:
\begin{enumerate}
	\item If S2R $s$ does not map to any of the outgoing interactions~$I$ from $n$, the algorithm recurses, attempting to map  $s$ on each node connected to $n$ by $I$.  If this traversal results in no DFS paths mapped to $s$ or following S2Rs in $L$, $s$ is labeled as having a Vocabulary Mismatch (\textbf{VM}), and the algorithm recurses with the next S2R in $L$ at the current node $n$. 
	This means that the S2R $s$ cannot be mapped to any node in the (sub)graph starting from $n$, then the algorithm attempts to map the next S2R.
	\item Conversely, if $s$ maps to interactions in $I$, the algorithm checks whether there are one or more mapped interactions. If there is a single interaction, $s$ is labeled as a Correct Step~(\textbf{CS}); if there are multiple, it is labeled as an Ambiguous Step (\textbf{AS}). 
	The algorithm then recurses with the next S2R in $L$ on each node connected to $n$ by only the mapped interactions from $I$. 
	Essentially, if the algorithm succeeds at mapping $s$ to interactions from $n$, then it proceeds with attempting to map the next S2R to the resulting nodes after navigating to the mapped interactions.
\end{enumerate}

It is possible that $s$ maps to interactions in $I$ (second branch above), but there are "gaps" between the previous mapped S2R and $s$: if their mapped interactions are not consecutive in the DFS path. 
If this is the case, the algorithm connects them by determining the shortest path between the involved nodes. 
The interactions used to connect the nodes are then labeled as Missing Steps (\textbf{MS}). 
Note that this shortest path may include interactions outside the DFS path, as we are not limiting the shortest path search to the DFS path alone. A shorter path may exist that bypasses parts of the DFS path.

After traversing a node with a given S2R (in either branch above), it is possible that when calling the algorithm recursively on a set of interactions (\ie when navigating down DFS paths), it returns multiple DFS paths mapped to the S2Rs. 
If this is the case, the algorithm selects the DFS path to return based on the following criteria: prioritizing the path with the most mapped S2Rs in $L$, or, if paths have the same number, choosing the shortest path.
\looseness=-1

The traversal continues until all S2Rs in $L$ have been exhausted or until none of the S2Rs are mapped to any DFS paths. 
If all S2Rs have been mapped, but there are still nodes along a DFS path, the algorithm does not proceed to check additional nodes down the current DFS path.
To prevent re-processing nodes and their interactions, the algorithm marks each (node, S2R) pair as visited before it processes the node and S2R.
\looseness=-1

\subsubsection{Quality Report Generation}

The returned DFS path contains interactions mapped to all or a subset of the S2Rs from the bug report. Each S2R is labeled as either a Correct Step (CS), Ambiguous Step (AS), or Vocabulary Mismatch Step (VM).  
In addition, interactions identified to fill in the "gaps" between S2Rs are labeled as Missing Steps (MS). 
For evaluation purposes, we also mark the corresponding S2Rs with missing steps as MS, so that we can perform a fine-grained analysis of results (more details found in Section \ref{sec:empirical_evaluation}).


% !TeX root = 0_main

\section{\tool's Prompt Development and Evaluation}
\label{sec:prompt_development}

This section describes how we developed and evaluated the LLM prompts for three distinct tasks: (i) S2R sentence identification, (ii) individual S2R extraction, and (iii) individual S2R mapping to app interactions. 
We adopted a rigorous, comprehensive, and data-driven approach in which we designed an initial prompt that was iteratively evaluated and refined into new prompts. 
Prompt development and evaluation followed a quantitative and qualitative methodology based on a set of Android app bug reports. 
Overall, we designed and evaluated 12 prompt templates across all three tasks.
\rev{To generate GPT-4 responses with the prompts for all tasks, we used a temperature of 0 to minimize randomness/non-determinism in the responses.}

\subsection{Development Dataset Construction}
\label{sec:dev_dataset}

We constructed a dataset of 54 bug reports and corresponding ground truth data, with manually identified S2R sentences, individual S2Rs, and interactions mapped to each S2R.

\subsubsection{Bug Report Collection}

We selected the 54 bug reports from the dataset released by Saha \etal~\cite{saha2024toward}, which contains reproducible mobile app bug reports from the AndroR2 dataset~\cite{wendland2021,Johnson2022}. 
These reports describe bugs for 31 Android apps of various domains (\eg web browsing, WiFi network diagnosis, and finance tracking). 
The reported bugs span different bug types, namely crashes (15 reports), output problems~(19), UI cosmetic issues (13), and navigation problems (7). 

\subsubsection{S2R Sentence Labeling}
\label{sec:identification_data_dev}

Two authors annotated the 1,031 sentences present in the bug reports as either S2R or non-S2R, following the S2R criteria and methodology defined by Chaparro \etal~\cite{Chaparro2017}. 
One author annotated each sentence, while the second author validated the annotations, recording disagreements and their rationale. 
The authors agreed on the annotations for 1,002 sentences (97.2\%, 0.91 Cohen's kappa~\cite{Cohen}), which represents near-perfect agreement. 
Disagreements were resolved via discussion. 
The most common reasons for disagreements were content misinterpretations and mistakes (\eg\ a sentence describing the observed behavior, not S2Rs). 
In total, the 54 bug reports contain 189 S2R sentences (3.5 per report on average), while the remaining 842 sentences describe non-S2R content.
\looseness=-1

\subsubsection{Individual S2Rs Extraction}
\label{sec:extraction_data_dev}
Two authors manually inspected the 189 S2R sentences to extract individual S2Rs (phrases describing a single interaction with the app). 
One author read and extracted the individual S2Rs in the format defined in \Cref{sec:indiv-s2rs-approach}. 
The extracted S2Rs were validated by a second author. 
They discussed disagreements to reach a consensus where needed. 
From the 189 S2R sentences, we extracted 246 individual S2Rs with an agreement rate of 97.6\%. 

\subsubsection{S2Rs to GUI Interaction Mapping}
\label{sec:qa_data_dev}

To create ground truth mappings between individual S2Rs and GUI app interactions, we first built the execution models (\ie graphs) for the 31 apps corresponding to the bug reports. 
To do so, we executed the \CrashScope tool~\cite{Moran2016} using the corresponding APKs (from the original dataset~\cite{saha2024toward,Johnson2022}) and a Pixel 2 Android emulator. 
We also used the manual interaction traces collected as part of Saha \etal's dataset~\cite{saha2024toward}. Both the \CrashScope and manual interaction traces consist of GUI-event execution traces and (video) screen captures showing the executed interactions. 
We used Song \etal's toolkit~\cite{song2022burt} to parse the traces and build the execution graphs.

Two authors manually inspected the execution data, graphs, and reproduction screen captures to map each S2R to graph nodes and interactions. 
One author first inspected this data to identify the GUI screen and target GUI component for each S2R. 
Then, the author identified the graph node corresponding to such screen, and within it, the interaction corresponding to the S2R. 
In the process, missing steps and the path that represented a minimal bug reproduction scenario were identified. 
A second author followed the same procedure to verify the interactions/nodes mapped to the S2Rs and the reproduction paths identified by the first author. 
Both authors discussed any disagreements, involving a third author where necessary.  
\looseness=-1

We applied the above methodology on a sample of 10 bug reports, in such a way that they spanned different bugs types, 
apps of different domains (9 apps), and S2R types (taps, types, \etc). 
The two authors created the ground truth for 46 individual S2Rs among 49 individual S2Rs for the 10 bug reports, agreeing on 43 S2Rs (agreement rate of 93.5\%). The excluded three individual S2Rs did not have corresponding app interactions in the execution model because they are performed outside the app (\eg\ \textit{"install the app"}), and hence, are not included in the graph.
Common reasons for disagreements were unclear individual S2Rs and misinterpretation of graph nodes/interactions. 
During the data creation process, we realized that it would take the two authors a prohibitive amount of effort to create the data for the remaining 44 bug reports. 
Therefore, we decided to focus on the S2R mapping prompt development using only the 10 bug reports and redirect our effort to curating the test data used for \tool's evaluation (see \Cref{sec:empirical_evaluation}). 

\subsection{Prompt Development Methodology}
\label{sec:prompt_development_methodology}

For each of three tasks where \tool uses GPT-4, our overall data-driven methodology used three prompting strategies, commonly used in software engineering research~\cite{hou2023large}: 
\begin{itemize}
	\item Zero Shot (ZS) prompting: starting from a base prompt template that includes the task description, input, and response format, we iteratively executed, evaluated, and refined the template until the performance plateaued. This involved computing performance metrics (precision, recall, and F1 score) against the ground truth, qualitatively analyzing false positives (FP) and negatives (FN), and adjusting the prompt to address those cases. For example, as S2R sentence identification is a classification task, two authors investigated the FP and FN of the GPT-4 responses to derive the classification criteria (\Cref{fig:prompt-structure}a) to better guide GPT-4 in the S2R sentence classification task.
    \rev{This process resulted in four versions of each type of prompt template. To determine if performance plateaued, we monitored the F1 score. For example, from version 3 to version 4 the F1 score decreased by 0.001 for the S2R identification task. Based on this minimal change, we selected version 3 as the optimal prompt for this phase.}
	\item Few Shot (FS) prompting: starting from the obtained ZS template, we created a base FS template containing positive and negative examples selected from the remaining bugs of Saha \etal's dataset~\cite{saha2024toward} and the expected output. The example bug reports are representative of each task and selected based on certain criteria, \eg\ various bug types (crash, output, \etc), and bug reports with different wordings and structures. We iteratively executed, evaluated, and refined the template until the performance no longer improved, in the same way we did it in ZS prompting.
	\item Chain of Thought (CoT) prompting: starting from the obtained FS template, we created a base CoT template containing explanations for \rev{the outcome of the positive and negative examples. The explanation for the outcome was designed by two authors after discussion and reaching a consensus.} We iteratively executed, evaluated, and refined the template until the performance plateaued, in the same way we did it in ZS and FS prompting.
\end{itemize}


\begin{figure}
	\centering
	\includegraphics[width=\linewidth]{figures/prompt.pdf}
	\caption{Structure of the Developed Prompts}
	\label{fig:prompt-structure}
\end{figure}

This methodology resulted in three prompt templates (one from each prompting strategy) for S2R identification and three templates for individual S2R extraction. 
For S2R mapping, since we defined mapping as a 2-step task, we designed two prompts for each strategy, resulting in six prompts. 
The 2-step task consisted of first asking GPT-4 to return a yes/no answer on whether an individual S2R maps to the interactions of a given screen and if the answer is yes, asking GPT-4 to return the list of interactions that the S2R maps to. 
Our tests revealed that this approach led to less noisy answers from GPT-4, compared to executing only the second step. 
In total, we developed 12 prompt templates. To help visualize our prompt templates, \Cref{fig:prompt-structure} illustrates the various components associated with the prompts for each task---our detailed templates are found in our replication package~\cite{package,doi}.

\subsection{Prompt Evaluation Results}
\label{sec:development_results}

We evaluated the prompt templates for S2R identification and extraction in terms of precision, recall, and F1 score, by executing these two phases in isolation. 
The F1 score was used to rank the templates. 
The S2R mapping prompt templates were evaluated by executing \tool's S2R quality assessment phase and evaluating the resulting S2R-interaction mappings. 
Since S2R mapping is a 2-step task, we evaluated each of the prompts based on the \# and \% of \textit{hits}, defined as follows.
For the first prompt, it is the number (and proportion) of correct predictions for the presence or absence of an S2R-interaction mapping in a given screen (out of the total number of predictions). 
For the second prompt, it is the number (and proportion) of correctly identified interactions for each individual S2R (out of the total number of individual S2Rs).


\begin{table}[t!]
	\centering
	\caption{Prompt Template Performance for S2R Identification}
	\label{tab:identification_results_dev_set}
	\resizebox{\columnwidth}{!}{%
		\begin{tabular}{c|c|c|c|c|c|c}
			\hline
			\textbf{Template} & \textbf{Precision} & \textbf{Recall} & \textbf{F1} & \textbf{\#TP} & \textbf{\#FP} & \textbf{\#FN} \\ \hline
			ZS & 0.929              & 0.968           & 0.948       & 183           & 14            & 6             \\ \hline
			FS   & 0.897              & 0.963           & 0.929       & 182           & 21            & 7             \\ \hline
			CoT  & 0.915              & 0.963           & 0.938       & 182           & 17            & 7             \\ \hline
		\end{tabular}%
	}
\end{table}

\begin{table}[t!]
	\centering
	\caption{Prompt Template Performance for S2R Extraction}
	\label{tab:indiv-s2r-study-results}
	\resizebox{\columnwidth}{!}{%
		\begin{tabular}{c|c|c|c|c|c|c}
			\hline
			\textbf{Template} & \textbf{Precision} & \textbf{Recall} & \textbf{F1} & \textbf{\#TP} & \textbf{\#FP} & \textbf{\#FN} \\ \hline
			ZS              & 0.918              & 0.951           & 0.934       & 234            & 21             & 12             \\ \hline
			FS              & 0.897              & 0.951           & 0.923       & 234            & 27             & 12             \\ \hline
			CoT             & 0.810              & 0.951           & 0.875       & 234            & 55             & 12             \\ \hline
		\end{tabular}%
	}
	
\end{table}

\begin{table}[t!]
	\centering
	\caption{Prompt Performance for S2R-Interaction Mapping}
	\label{tab:indiv-s2r-matching-results}
	\resizebox{\columnwidth}{!}{%
		\begin{tabular}{c|ccc|cc}
			\hline
			\multirow{2}{*}{\textbf{Template}} & \multicolumn{3}{c|}{\textbf{1st-step template}}                                                             & \multicolumn{2}{c}{\textbf{2nd-step template}}          \\ \cline{2-6} 
			& \multicolumn{1}{c|}{\textbf{\# Predictions}} & \multicolumn{1}{c|}{\textbf{\# Hits}} & \textbf{Hit Rate} & \multicolumn{1}{c|}{\textbf{\# Hits}} & \textbf{Hit Rate} \\ \hline
			{ZS}                      & \multicolumn{1}{c|}{939}                    & \multicolumn{1}{c|}{887}              & 94.5\%            & \multicolumn{1}{c|}{30}               & 76.9\%            \\ \hline
			{FS}                      & \multicolumn{1}{c|}{970}                    & \multicolumn{1}{c|}{912}              & 94.0\%            & \multicolumn{1}{c|}{26}               & 66.7\%            \\ \hline
			{CoT}                     & \multicolumn{1}{c|}{1214}                   & \multicolumn{1}{c|}{1152}             & 94.9\%            & \multicolumn{1}{c|}{18}               & 46.2\%            \\ \hline
		\end{tabular}%
	}

\end{table}

\Cref{tab:identification_results_dev_set,tab:indiv-s2r-study-results,tab:indiv-s2r-matching-results} show the performance of the designed prompt templates for the three tasks: S2R identification, individual S2R extraction, and S2R mapping. 
Among the three templates for the S2R identification task, \textit{ZS} achieved the best performance across the three metrics having the lowest \# of FP (14) and FN~(6). Likewise, for the individual S2R extraction task, the \textit{ZS} template achieved the highest precision (0.918) with the lowest \# of FP (21), sharing the same \# of FN (12) with the other two prompts. Regarding the S2R mapping task, \tool with all three templates for the 1st-step prompt achieved a similar hit rate (94.0\% to 94.9\%) and with \textit{ZS} template for the 2nd-step prompt achieved the best hit rate of 76.9\%. 

Interestingly, although prior research has shown the superiority of \textit{CoT} prompts over \textit{ZS} and \textit{FS} prompts~\cite{hou2023large,Feng2024}, this is not the case for our tasks. Via qualitative analysis of GPT-4 responses, we observed that GPT-4 with \textit{FS} and \textit{CoT} prompts tends to include more unintended text in the responses compared to \textit{ZS} prompt which results in more false positives, \eg\ \textit{CoT} template for S2R extraction generated 55 FPs while \textit{ZS} template generated 21 FPs only. We conjecture that the long and complicated input (\eg\ bug reports can be long, and interaction information can be complicated) made the task difficult for GPT-4. Moreover, having three or four examples with reasoning made the prompts even longer.

As for all three tasks, \textit{ZS} templates outperformed the other two, we utilized the \textit{ZS} templates for implementing \tool.


% !TeX root = 0_main

\section{\tool's Evaluation Design}
\label{sec:empirical_evaluation}
\tool's evaluation has two main goals: (i) to evaluate \tool's ability to provide correct quality annotations for real bug reports, 
and (ii) to examine how well \tool can infer missing S2R information in bug reports. 
We apply \tool to a test dataset (see \Cref{sec:test_dataset}) comprising 21 bug reports, in order to provide a comparison with prior work. We aim to answer the following research questions (RQs):
\begin{itemize}
	\item \textbf{RQ$_{1}$:} How effective is \tool in generating correct S2R quality annotations?
	\item \textbf{RQ$_{2}$:} How accurately can \tool infer missing S2Rs?
\end{itemize}

\subsection{Evaluation Dataset}
\label{sec:test_dataset}

We used the bug reports (\ie\ \textit{test set}) used by Chaparro \etal~\cite{Chaparro2019}, which allow us to provide a direct comparison with their approach, \EulerC. 
This dataset contains 24 bug reports \rev{of various kinds ( crashes, UI problems, and navigation problems)} from six Android applications \rev{of different domains (web browsing, WiFi network diagnosis, finance tracking, \etc). The diverse evaluation set, separate from the development set, enabled us to assess the generalizability of the developed prompts across different bug reports.}  
We discarded three bug reports, as follows: (1) two bug reports~\cite{aard81, aard104} from the Aard Dictionary App~\cite{aardapp}, because the app version 1.4.1 is unable to load its dictionary database, and (2) one bug from Time Tracker app~\cite{atimetracker1}, because we could not generate the execution model for this app as the bug report requires a rotation action which \tool does not support.
Hence, our test set contains 21 bug reports from the original \EulerC dataset. 

Since this dataset does not contain any ground truth information for evaluating \tool, we constructed the ground truth manually.  We used the same methodology discussed in \Cref{sec:identification_data_dev,sec:extraction_data_dev} to do so for identifying S2R sentences and extracting individual S2Rs.

To construct the quality assessment ground truth, the first two authors mapped the extracted individual S2Rs to \rev{GUI} interactions manually following the methodology discussed in \Cref{sec:qa_data_dev}. App execution models for the bug reports were built by parsing execution traces collected via \CrashScope's app exploration and manual app \rev{usage}.  
One author identified the reproduction interactions on the generated data and mapped such interactions with the extracted individual S2Rs from the bug report. 
They collected the mapped interactions for each individual S2R, as well as the interactions that are required to reproduce the bug, but not reported in the bug report, \ie ground truth for missing steps. 
Each individual S2R was mapped with one or more interactions in the execution model path, as needed. Using the mapped interactions and the quality assessment model (discussed in \Cref{sec:quality_model}), they assigned quality labels to each individual S2R. 
A second author performed the same steps and validated the interactions in the reproduction scenario as well as the quality annotations.  Disagreements were resolved via discussion.

In summary, we identified 73 S2R sentences out of the 275 sentences present in the 21 bug reports with a near-perfect agreement between the two authors (98.2\% agreement rate and 0.88 Cohen's kappa \cite{Cohen}). 
From the 73 S2R sentences, we extracted 82 individual S2Rs with an agreement rate of 93.9\% between the two authors.
We discarded four individual S2Rs as they represent rotation operation and the current version of \tool does not support this operation. 
We assigned the remaining 78 individual S2Rs quality annotations (\ie\ 70 S2Rs as CS, 7 S2Rs as AS, 1 S2R as VM, and 38 S2Rs as MS). 
We identified 158 missing interactions, \ie\ missing steps for the 38 MS positions (\ie S2Rs with filled-in missing interactions). 
For constructing the annotations ground truth, the two authors agreed on 90\% of the cases. Cohen's kappa for individual S2R extraction and mapping is inapplicable since the labeling is not based on a discrete set of labels. 

\subsection{Baseline Approach}

We considered \EulerC~\cite{Chaparro2019} as the baseline approach, which also aims to assess the quality of S2Rs in a bug report. 
It identifies the S2R sentences from a bug report using deep learning techniques (\eg\ CNN~\cite{o2015introduction}, Bi-LSTM~\cite{zhou2016attention}). 
It identifies individual S2Rs via analysis of discourse patterns and assigns quality annotations by employing keyword-based mapping to app UI information.  
\rev{\EulerC and \tool generate similar quality reports, therefore we can directly compare the \tool reports to the original \EulerC reports  provided by \EulerC's replication package \cite{Chaparro2019}, 
to answer the RQs.}
\looseness=-1

\subsection{Evaluation Methodology} 

We executed \tool with the 21 bug reports on the test set,  producing the quality report for each bug report, including the quality annotations and missing steps. To answer \textbf{RQ$_{1}$}, we compared the \tool assigned quality annotations with the ground truth quality annotations. To answer \textbf{RQ$_{2}$}, we evaluated the generated missing steps by \tool against the ground truth missing steps. For both RQs, we computed precision, recall, and F1 score. We applied the same process for \EulerC and qualitatively analyzed the false positives (FP) and negatives (FN) to understand the limitations of both approaches.

\section{Results}
\label{sec:results}
Following \nksr, we evaluate our method using metrics including the standard Chamfer-$L_1$ Distance~(CD-$L_1 \times 10^{-2}$, $\downarrow$) and F-score~($\uparrow$) with a threshold~($\delta{=}0.010$). 
We also report additional metrics proposed in \nksr~including Chamfer-$L_1$ Distance by Completeness (Comp.~$\times 10^{-2}$, $\downarrow$) and Accuracy (Acc.~$\times 10^{-2}$, $\downarrow$) in the \texttt{Supplementary Material}. 
We evaluate our method on multiple datasets, under two settings including in-domain evaluation for accuracy estimation -- training set and test set are from same dataset, and cross-domain evaluation for generalization ability estimation where training set and test set are from different datasets. 
Additionally, for cross-domain evaluation we use the following datasets prepared by the leading voxel-based baseline, \nksr, and one additional dataset from RangeUDF~\cite{wang2022rangeudf}:

\begin{itemize}
    \item \synthetic{}  is a synthetic dataset created from ShapeNet objects~\cite{chang2015shapenet}. Each scene contains 2-3 objects. 
    Following prior works~\cite{wang2022rangeudf,chibane2020ndf}, we re-scale the synthetic rooms to roughly match real-world scale.
    There are 3750 scenes as training set and \ws{995 scenes} as the test set. 
    \item \scannet{} is a real-world indoor scene dataset. We use the setting from previous work~\cite{wang2022rangeudf, tang2021SACon, peng2020convoccnet, boulch2022poco} where we train on 1201 rooms and test on 312 rooms. 
    \item \carla is a large-scale outdoor driving scene prepared by NKSR~\cite{huang2023neural} using the CARLA simulator~\cite{dosovitskiy2017carla}. 
    \ws{Following NSKR~\cite{huang2023neural}, we test on two subsets including the 'Original' subset (10 random drives simulated on 3 towns) and the 'Novel' subset (3 drives from an additional town only for testing).}
    To avoid exploding GPU memory during training, we follow NKSR~\cite{huang2023neural} to divide a large scene into patches. The resultant training set has {3757} patches. 
    \item \scenenn{}  is a real-world indoor dataset prepared by RangeUDF~\cite{wang2022rangeudf} which we used for cross-domain evaluation. We only use its test set which consists of 20 scenes.
\end{itemize}



\begin{table*}
\centering
\resizebox{\linewidth}{!}{
\setlength{\tabcolsep}{3pt}
\begin{tabular}{LccccccccccccC}
\toprule
Methods & & \multicolumn{3}{c}{\ws{{\bf \synthetic}}}  &  \multicolumn{3}{c}{{\bf \scannet}} & \multicolumn{3}{c}{\ws{{\bf \carla(Original)}}} & \multicolumn{3}{c}{\ws{{\bf \carla(Novel)}}} \\ 
 \cmidrule(lr){3-5} \cmidrule(lr){6-8} \cmidrule(lr){9-11} \cmidrule(lr){12-14} 
&Primitive& CD ($10^{-2}$) $\downarrow$ & F-Score  $\uparrow$ & Latency (s) $\downarrow$  & CD ($10^{-2}$) $\downarrow$ & F-Score  $\uparrow$ & Latency (s) $\downarrow$  & CD (cm) $\downarrow$ & F-Score  $\uparrow$ & Latency (s) $\downarrow$ & CD (cm) $\downarrow$ & F-Score  $\uparrow$ & Latency (s) $\downarrow$ \\        
\midrule
SA-CONet~\cite{tang2021SACon} & Voxels & {0.496} & {93.60} & - & - & - & - & - & - & - & - & - & -\\
ConvOcc~\cite{peng2020convoccnet} & Voxels & {0.420} & {96.40} & - & - & - & - & - & - & - & - & - & -\\
NDF~\cite{chibane2020ndf} & Voxels & {0.408} & {95.20} & - & 0.385  & 96.40  & -  & - & - & - & - & - & -\\
RangeUDF~\cite{wang2022rangeudf} & Voxels & {0.348} & {97.80} & {-} & 0.286 & 98.80 & - & - & - & - & - & - & -\\
\ws{TSDF-Fusion~\cite{zeng20163dmatch}} & -  & - & - & - & - & - & - & 8.1 & 80.2 & - & 7.6 & 80.7 & - \\
\ws{POCO~\cite{boulch2022poco}} & - & - & - & - & - & - & - & 7.0 & 90.1 & - & 12.0 & 92.4 & - \\
\ws{SPSR~\cite{kazhdan2013screened}} & - & - & - & - & - & - & - & 13.3 & 86.5 & - & 11.3 & 88.3 & - \\
\nksr & Voxels &  \underline{0.346} &  \underline{97.41} & \underline{0.40} & \underline{0.246} & \underline{99.51} & \underline{1.54} &  \underline{3.9} &  \underline{93.9} &  \underline{2.0} &  \underline{2.9} &  \underline{96.0} &  \underline{1.8} \\
\nksr (more data) & Voxels & - & - & - & - & - & - & {3.6} & {94.0} & {2.0} & {3.0} & {96.0} & {1.8}\\
Ours~(Minkowski)~\cite{choy20194d} \scriptsize{(w/ KNN)} & Voxels & - & \todo{} & \todo{} & 0.254 & 99.41 & 0.46 & 3.4 & 97.2 &1.9 & 2.7 & 98.1 & 2.0 \\
Ours~(Minkowski)~\cite{choy20194d} & Voxels & - & \todo{} & \todo{} & 0.301 & 98.48 & 0.31 & 3.8 & 96.2 & 1.5 & 3.0 & 97.4 & 1.5\\
\rowcolor{1st} Ours \scriptsize{(w/ KNN)} & Points &{0.321} & {98.34} & {0.13} & {0.243} & {99.61} & {0.48} &{3.2} & {97.5} & {3.2} &{2.6} & {98.3} & {3.4}\\
\rowcolor{1st}Ours & Points & {0.360} & {96.32} & 0.14 & 0.257 & 99.33 & 0.49 & {3.3} & {97.4} & 1.7 & {2.7} & {98.2} & 1.7 \\

\bottomrule
\end{tabular}
}
\caption{\textbf{In-domain evaluation} -- We show that our method achieves the best accuracy (CD and F-score) with significantly improved time efficiency~(inference latency).
Note we retrain \nksr (numbers are underlined) for fairer comparison, \ws{as the training data for \nksr is different from ours -- i.e., they reported some models trained on a ``mix'' of datasets, which is impossible to reproduce.
}
}
\label{tab:indomain}
\end{table*}


\paragraph{Evaluation pipeline}
To evaluate our method, we first extract the mesh with Dual Marching Cubes~\cite{schaefer2004dual} on the predicted SDF, and then compute the CD and F-score between 100k points sampled on the mesh, and 100k points sampled from the ground-truth dense point cloud.
We use the same approach as \nksr to prepare the input point clouds for training and evaluation from the ground-truth dense point clouds through downsampling.
Specifically, for indoor datasets (i.e., \synthetic, 
\scannet and \scenenn), we uniformly sample 10K points sampled from the ground truth dense point cloud. 
For outdoor driving scenes~(i.e., \carla), we follow the evaluation pipeline from \nksr.
We sample sparse input point clouds with a sparse 32-beam LiDAR with a ray distance noise of 0-5 cm and pose noise of $0-3^\circ$, and obtain the ground truth from a noise-free dense 256-beam LiDAR.

\begin{figure*}
\centering
\includegraphics[width=\linewidth]{visualizations/test_set_results.pdf}
\caption{
{\textbf{Qualitative results on \carla and \synthetic}} -- our method achieves high quality surface reconstructions which preserve more details than \nksr~which loses information due to quantization for large and non-uniformly sampled datasets like Carla.
}
\label{fig:qual_results_carla_syn}
\end{figure*}
 
\begin{figure*}
\centering
\vspace{-1em}
\includegraphics[width=.95\linewidth]{visualizations/scannet_results_0.pdf}
\caption{
Qualitative results on \scannet: We compare our method with prior SOTA~\cite{huang2023neural} and Ours~(Minkowski)~\cite{choy20194d} that is more comparable as it only differs from ours in the backbone. Our method achieves reconstruction of similar quality to the SOTA. It also \textit{significantly} outperforms Ours~(Minkowski), highlighting the importance of point-based methods. 
% \TODO{callouts too small? almost no zoom? why?}
}
\vspace{-1em}
\label{fig:scannet_results}
\end{figure*}
  

\paragraph{Implementation details}
We base our feature backbone on PointTransformerV3~\cite{wu2024point} with 4-levels.
The PointNet-style network is a 2-layered residual connection MLP, with hidden dimension of $32$ and output feature dimension of $32$.    
The grid size used in neighborhood function is $0.01$ meters.
Following \nksr, we use the similar coefficients for loss terms -- i.e., $\lambda_{\text{SDF}}$ is $300$ and $\lambda_{\text{mask}}$ is $150$.
However, we empirically set $\lambda_{\text{Eikonal}}$ to $10$~(\nksr does not need this regularizer thanks to its specialized surface solver).
We train our model with a batch size of $4$ on either a single \texttt{NVIDIA RTX A6000 ADA} or an \texttt{NVIDIA L40S}, and a learning rate of $10^{-3}$.
We adopt the Adam optimizer with default parameters.
We set the maximum number of epochs to 200 and employ a cosine learning rate decay starting from epoch 120.


\begin{table*}
\centering
\resizebox{\linewidth}{!}{
\setlength{\tabcolsep}{2pt}
\begin{tabular}{LccccccccccC}
\toprule
Methods & & \multicolumn{3}{c}{{\bf \synthetic $\rightarrow$ \scannet}}  &  \multicolumn{3}{c}{{{\bf \scannet $\rightarrow$ \synthetic}}} & \multicolumn{3}{c}{{{\bf \scannet $\rightarrow$ \scenenn}}} \\ 
 \cmidrule(lr){3-5} \cmidrule(lr){6-8} \cmidrule(lr){9-11}
&Primitive& CD ($10^{-2}$) $\downarrow$ & F-Score  $\uparrow$ & {Latency (s) $\downarrow$ } & CD ($10^{-2}$) $\downarrow$ & F-Score  $\uparrow$ & {Latency (s) $\downarrow$ } & CD ($10^{-2}$) $\downarrow$ & F-Score  $\uparrow$ & {Latency (s) }$\downarrow$ \\       
\midrule
SA-CONet~\cite{tang2021SACon} & Voxels & 0.845 & 77.80 & - & - & - & - & - & - & - \\
ConvOcc~\cite{peng2020convoccnet} & Voxels & 0.776 & 83.30  & - & - & - & - & - & - & - \\
NDF~\cite{chibane2020ndf} & Voxels & 0.452 & 96.00 & - & {0.568} & {88.10} & - & 0.425 & 94.80 & - \\
RangeUDF~\cite{wang2022rangeudf} & Voxels & {0.303} & {98.60} & {-} & 0.481& 91.50 & - & 0.324 & 97.80 & - \\
\nksr & Voxels & {0.329} & {97.37} & {2.02} & {0.351} & {97.41} & {0.46} & {0.268} & {99.18} & {1.95} \\
\rowcolor{1st} Ours (w/ KNN) & Points & {0.284} & {98.65} & {0.54} & {0.327} &{98.37} & {0.13} & {0.277} & {99.00} & {0.50} \\
\bottomrule
\end{tabular}
}
\caption{\textbf{Cross-domain evaluation} -- we achieve the best generalization ability in two cases with much better time efficiency. In the other case where we generalize from \scannet to \scenenn, we achieve accuracy on par with the SOTA baseline~\cite{huang2023neural} with less than a half of their latency.  
}
\vspace{-1.4em}
\label{tab:across_domain}
\end{table*}


\paragraph{Reconstruction latency}
For both our models and NKSR, we record the reconstruction latency for all indoor scenes on a single \texttt{NVIDIA RTX 3090}, and for large outdoor scenes on a single \texttt{NVIDIA L40s} given that more GPU memory is required.
We omit data loading time, and only record the average forward pass time. 

\subsection{In-domain evaluation}
We compare against \nksr~(the current state-of-the-art), RangeUDF~\cite{wang2022rangeudf},  SPSR~\cite{kazhdan2013screened}, NDF~\cite{chibane2020ndf}, ConvOcc~\cite{peng2020convoccnet} and SA-CONet~\cite{tang2021SACon}.     
We further include a baseline that replaces our backbone with MinkowskiNet~\cite{choy20194d} (i.e., Ours~(Minkowski)) to show the degraded performance due to the information loss caused by voxelization.

\paragraph{Quantitative results -- \Cref{tab:indomain}}
Across indoor and outdoor datasets, our method outperforms baselines in terms of accuracy and time efficiency. Especially in outdoor datasets, our method achieves the best surface reconstruction with the smallest latency -- nearly \textit{half} of the second best's latency.
In indoor datasets, which have relatively uniform sampling patterns, we achieve accuracy on par with the previous state-of-the-art, but with significantly improved time efficiency.
Note that we achieve this advantage even with KNN because, in smaller indoor point clouds, the highly engineered KNN implementation has similar time efficiency to that of our neighborhood function.
We further detail our analysis on this matter in the \texttt{Supplementary Material}. 
We also note that our approximate neighborhood function is still effective, as it outperforms the directly comparable baseline MinkowskiNet~\cite{choy20194d}, which shares the same structure except for the backbone and neighborhood function.


\paragraph{Qualitative results -- \Cref{fig:qual_results_carla_syn,fig:scannet_results}}
We show that our method tends to reconstruct surfaces of the best quality among the compared methods.
Especially, on the non-uniform large scale \carla, our method tends to preserve more details than the previous state-of-the-art~\cite{huang2023neural}, which voxelizes the point cloud.   

\subsection{Cross-domain evaluation -- \Cref{tab:across_domain}}
We further test the generalization ability of our method with a cross-domain evaluation.
We evaluate models trained with dataset A on other a different dataset B; we denote this as~A $\rightarrow$ B. 
As shown in \Cref{tab:across_domain}, there are three cases in total.
In two cases (i.e., \synthetic $\rightarrow$ \scannet and \scannet $\rightarrow$ \synthetic), our method achieves the best accuracy with the best time efficiency. 
In another case (\scannet $\rightarrow$ \scenenn), we achieve accuracy on par with SOTA~\cite{huang2023neural} with a much better time efficiency, i.e., less than a half of the latency required by the SOTA~\cite{huang2023neural}.

\subsection{Ablation studies}
Our ablations are executed on \scannet, as it is a real-world dataset, and is equipped with precise ground truth surface meshes.

\begin{table}
\centering
\resizebox{.9\columnwidth}{!}{
\begin{tabular}{LccccccC}
\toprule
{\bf Neighbor Num.} & {CD (10\textsuperscript{-2})} $\downarrow$ & {F-score} $\uparrow$ & Latency (s) $\downarrow$ \\ \midrule
 2 & 0.246 & 99.56 & 109 \\
 4 & 0.244 & 99.59 & 127 \\
 \rowcolor{1st} 
8 & {0.243} & 99.61 & 151 \\
16 & 0.256 & 99.28 & 187 \\
\bottomrule
\end{tabular}
}
\caption{{\bf The impact of neighborhood size} -- larger neighborhoods lead to increased computational cost, and we find that 8 neighbors gives the best balance of cost and quality.}
\label{tab:numpts_neighbor}
\vspace{-1em}
\end{table}

\paragraph{Impact of neighborhood size -- \Cref{tab:numpts_neighbor}}
We analyze the impact of neighborhood size on performance. Larger neighborhood size leads to increased computation overhead. 
We show that the 8-nearest neighboring points gives the best trade-off between accuracy and time efficiency.
Considering a large number (e.g., 16) of neighboring points degrades performance as the the aggregation module has limited capacity to predict the precise SDF from a large local point cloud.

\begin{table}
\centering
\resizebox{.95\columnwidth}{!}{
\begin{tabular}{@{}lcccccc@{}}
\toprule
\makecell{\bf Num. of hidden\\\bf layers in $\aggregation$} & CD (10\textsuperscript{-2}) $\downarrow$ & F-score $\uparrow$ & Latency (s) $\downarrow$ \\ \midrule
 2 & 0.257 & 99.33 & 152 \\
 4 & 0.256 & 99.32 & 166 \\
\bottomrule
\end{tabular}
}
\caption{{\bf Impact of capacity of $\aggregation$} -- we find that increasing the number of layers in $\aggregation$ beyond 2 decreases time efficiency without substantially improving the reconstruction quality.}
\label{tab:agg_capacity}
\vspace{-1em}
\end{table}

\paragraph{Impact of capacity of $\aggregation$ -- \Cref{tab:agg_capacity}} 
We report how the capacity of the aggregation module $\aggregation$ (i.e., different number of hidden layers) impacts the performance.
We observe that aggregation modules of higher capacity give better performance but degraded time efficiency. However, as shown in~\Cref{tab:agg_capacity}, a very large capacity (4 layers) for $\aggregation$ does not help.
We show that we we use 2 layers to have a good trade-off between accuracy and time efficiency. 
We supplement~\Cref{tab:agg_capacity} with an analysis across even more levels in the \texttt{Supplementary Material}.

\begin{table}
\centering
\resizebox{.9\columnwidth}{!}{
\begin{tabular}{@{}lcccc@{}}
\toprule
\textbf{Num. of scales} &KNN & Minkowski & Z-order & Hilbert  \\ \midrule
0 & 1.00 & 0.17 & 0.44  & \cellcolor{1st}0.46  \\
1 & 1.00 & 0.29 & 0.48  & \cellcolor{1st}0.50  \\
2 & 1.00 & 0.38 & 0.49  & \cellcolor{1st}0.52  \\
3 & 1.00 & 0.44 & 0.49  & \cellcolor{1st}0.53  \\ %
\bottomrule
\end{tabular}
}
\caption{\textbf{Recall rate of our Hilbert-curve based $\neighbor$} -- we find that the Hilbert curve consistently outperforms both the Z-order curve~\cite{morton1966computer} and the one-ring neighborhood from Minkowski relative to the exact k-nearest neighbors.
}
\vspace{-1em}
\label{tab:locality_neighbor}
\end{table}

\paragraph{Analysis of neighbors retrieved by~$\neighbor$ -- \Cref{tab:locality_neighbor}}
\at{We now investigate the quality of the point neighborhoods retrieved by various possible implementations for $\neighbor$.
In particular, we are interested to experimentally study whether our serialization indeed preserves locality.
To quantify this, we treat the neighborhood retrieved with KNN as the ground-truth.}
We report the recall rate of a local neighborhood by comparing it with this ground truth~(we ignore the precision rate because we remove false positives with a distance threshold).
We also report the recall rate of the one-ring neighborhood retrieved in Minkowski~\cite{choy20194d}.
We show that the recall rate of our Hilbert $\neighbor$ is the best across variants, and across all scales.

\begin{table}[t]
\centering
\resizebox{\columnwidth}{!}{
\begin{tabular}{L rr rR}
\toprule
Methods & \multicolumn{2}{c}{Uniform} & \multicolumn{2}{c}{Non-Uniform}   \\ 
\cmidrule(r){1-1}
\cmidrule(lr){2-3}
\cmidrule(l){4-5}
\nksr & 0.246 & 480s & 0.273 & 668s  \\
Ours~(Minkowski)~\cite{choy20194d}  & 0.301 & 97s & 0.349 & 94s \\
Ours~(Minkowski)~\cite{choy20194d} {(w/ KNN)} & 0.254 & 145s & 0.294 & 155s \\
\rowcolor{1st} Ours~(w/ serialization) & {0.257} & {152s} & {0.296} & {145s} \\
\rowcolor{1st} Ours~(w/ KNN) & \textbf{0.243} & \textbf{151s} & \textbf{0.273} & \textbf{142s}  \\
\bottomrule
\end{tabular}
}
\caption{
\textbf{The impact of sampling} -- we evaluate uniform vs non-uniform sampling on ScanNet. We find that our method achieves the best accuracy (in terms of CD ($10^{-2}$)) and good time efficiency compared to \nksr~for both sampling types.
}
\vspace{-1em}
\label{tab:nonuniform_scannet}
\end{table}

\paragraph{The impact of sampling pattern --~\Cref{tab:nonuniform_scannet}} 
We report the impact of sampling pattern on performance by evaluating models on ScanNet point clouds that are uniformly or non-uniformly sampled. 
{To non-uniformly sample the ScanNet point clouds, we first partitioned the scene into eight blocks and randomly sampled a different number of points from each block. The number of samples followed an arithmetic sequence with a common difference of 200. Finally, we padded the last block to ensure that the total number of points remained 10K.}
 
We show that our method achieves better robustness to non-uniform sampling than the baselines, highlighting the importance of avoiding quantization of the point cloud for high quality surface reconstruction. 



\section{Related Work}

% Reaction Diffusion
\paragraph{Wave-based Computing}
While prior work on wave-based computing in trainable task-oriented neural networks remains scarce, there is a rich history of using wave-like or other spatiotemporal field dynamics generally for computation.  
Early work studied the ability for waves to perform simple logical operations and thereby compute in a distributed manner \citep{pwc, wave_compute}, while other work has studied the ability for physical water waves to act as literal instantiations of classic `reservoir computers' \citep{maksymov2023analoguephysicalreservoircomputing}. Classically, the domain of `Neural Field Theory' has studied the role of spatiotemporal field dynamics in neural computation from a rigorous mathematical standpoint, although to-date these models have not been adapted to deep-neural network task-oriented performance. We refer readers to \cite{nft} for a thorough review of such models. 

More recently, \cite{hughes2019wave} have noted the analogy between the wave equation and recurrent neural networks, as we have done here, and used this to suggest that wave-based RNNs with learnable wave speeds may perform a type of analog computation. The authors use this to perform acoustic signal classification in a simplified setting, similar to our study in spirit, but differing in how waves are used and their computational purpose. Most related to the present study, \cite{BALKENHOL20244288} use an architecture similar to ours, with a Laplacian recurrent operator, damping, and gating, to show that when provided with an audio signal at a specific spatial location of the network, neurons at more distant locations can perfectly reconstruct the signal. The authors also show that this network is able to reproduce electrical recordings from macaque monkeys in response to simple grating stimuli, hypothesizing that their detection of high frequency waves is highly related to the transfer of information over large cortical distances.  

\looseness=-1
In terms of task-oriented wave-based models, recent work by \cite{felix} extensively studies the computational abilities of oscillatory neural networks, and specifically notes the emergence of traveling waves in these models in response to visual stimuli. Similarly, work by \cite{nwm, wrnn} studies wave-based RNNs for sequence processing and prediction. Our work fundamentally differs from these in the precise study of how these waves may be utilized for the spatial integration of visual information, as is hypothesized to happen in the visual cortex. Furthermore, our work uniquely demonstrates that a timeseries based readout is crucial for performing this type of integration, inspired by Kac's question, opening the door for future novel applications of these models. 

\vspace{-4mm}
\paragraph{Recurrence vs. Depth}
Another relevant line of research concerns the ability to trade off depth for recurrence in CNNs. 
Early work in this area was performed by \citet{liao2020bridginggapsresiduallearning}, with a more extensive recent study performed by \citet{schwarzschild2022the}. The authors demonstrate how iterating a single convolutional layer in a deep CNN yields similar performance to equivalently deep fully untied CNNs. Our work differs from these in that we demonstrate the advantage of a timeseries readout mechanism, inspired by Kac's question, whereas prior work can be seen as using the 'last' hidden state mechanism, that we see underperforms in this work. Interestingly, our findings thus suggest a potential novel method to improve the performance of these recurrent alternatives to deep networks through the use of our readout, a direction we intend to study in future work. Other more machine learning focused work has studied the impact of various weight-sharing schemes in deep convolutional networks \citep{eigen2014understandingdeeparchitecturesusing, jastrzębski2018residualconnectionsencourageiterative, boulch2017sharesnetreducingresidualnetwork}, however these share the same distinction with the present study in terms of their readout mechanism, while our proposed timeseries readouts appear to be uniquely linked to the wave dynamics that emerge in our models. 


\subsubsection{Binding By Synchrony}
Finally, we believe our work shares an interesting connection with the ``binding by synchrony'' concept \citep{Singer:2007} from early neuroscience research. Specifically, while our model's `binding' of parts into wholes does not rely on precise zero-lag synchrony—where oscillators within an object are perfectly in phase, as in the original framework; our method does rely on traveling waves of activity within objects that can be interpreted as a type of phase-lag synchrony. The ``binding operation'' then involves a transformation of the time signal using a suitable linear projection (our proposed timeseries readout). We believe this connection is valuable precisely since it enables a connection with the extensive historical literature on this concept, while simultaneously forming novel predictions on how such phenomena might manifest in natural neural systems. 
On the machine learning side of this concept, our work shares a strong connection with a class of object-centric learning methods which leverage a notion of synchrony of neural activations to define `bound' visual units for computational purposes. This includes models such as complex autoencoders \citep{lowe_complex-valued_2022, lowe_rotating_2024, stanic_contrastive_2024, gopalakrishnan_recurrent_2024} and recent Artificial Kuramoto Oscillatory Neurons (AKOrN) \cite{miyato_artificial_2024}. 
Unlike our method, the waves in the AKOrN model are not used directly as a representation themselves, but instead are neglected through the use of the `last hidden state' readout method. Perhaps most related to our work, \cite{liboni_image_2023} use a complex-valued recurrent neural network designed to generate traveling waves for image segmentation, with binding information encoded in the temporal phase sequence of these waves. This method can indeed be seen as using traveling waves to integrate information spatially, but contains no trainable components, offering a more theoretical exposition to the problem, as opposed to the task-oriented empirical study presented here. 



% !TeX root = 0_main

\section{Threats to Validity and Limitations}
\label{sec:threats}
\textbf{Construct Validity.}
The main threats to construct validity stem from manually verifying the matching of the interactions extracted from the S2R sentences to the information on the execution model and constructing a ground truth dataset. 
To mitigate this threat, two authors independently carried out the manual verification tasks and ground truth creation, following well-defined and replicable methodologies.
More so, we computed and reported agreement levels, which are very high in all cases. 
\looseness=-1

\textbf{Internal Validity.}
Selecting the optimal prompt can be challenging for any use of GPT-4, let alone for multiple distinct tasks, and this process of finding the best prompt impacts the performance of our approach.
We selected the best prompt by evaluating 14 prompt templates, using three prompting strategies (\ie zero-shot, few-shot, and chain-of-thought) on a rich development set of bug reports from multiple applications.
\looseness=-1 

\textbf{External Validity.}
Our results are compared with the state-of-the-art, \EulerC, where we used 21 bug reports from their original dataset across six applications. We could not increase the dataset size for comparison due to difficulties in running the \EulerC tool. However, our approach is built by analyzing a dataset with bug reports from nine different applications consisting of four types of bugs. Therefore, \tool can be generalized to diverse types of bug reports.
\rev{Moreover, AstroBR currently supports the most frequently used GUI interactions in Android applications (tap, long tap, \etc). While the lack of support for certain types of interactions (\eg rotation) is a limitation, this is not due to the inherent design of the approach, and the support of these features can be added through additional engineering effort in future work.}

\textbf{Limitations.}
\tool's performance depends on the completeness of the app execution model.
The automated execution information collected with \CrashScope may result in an incomplete execution model.
To overcome this issue, we collected information from manual app executions. 

\section{Conclusion}\label{Sec:con}
This work introduces a benchmarking for model-free varying-diameter log-grasping with a forestry crane, including the structure of the environment, design of reward functions, and a modified proximal policy optimization (mPPO) algorithm. Under the assumption that the log pose is given, extensive simulations are presented to show the effectiveness of the reward shape and the exploration capability of the mPPO over other algorithms. The overall success rate of the grasping task of varying-diameter wood logs, varying log poses, and randomized initial configurations of the forestry crane exceeds $96\%$. 

\textbf{Limitation.} Although our method shows promising results, we recognize many aspects that require further attention, particularly regarding the sim-to-real gap. For instance, while the simulation offers many benefits, real-world uncertainties such as sensor noise, actuation delays, and unexpected disturbances will require more robust handling. The computational efficiency, especially the training time, can be further optimized by leveraging GPU acceleration. Additionally, incorporating transfer learning techniques may help improve the generalization to physical systems. In future work, we will focus on deploying the learned model in real-world demonstrations and aim to refine the agent’s ability to adapt to dynamic, unpredictable conditions. 


%Additionally, the training process can integrate object estimation and imitation learning. 
%Closing the sim-to-real gap is not a trivial problem since we consider a large-scale hydraulically actuated robot. This is also our main focus for the future work. 




%\section*{Acknowledgment}
%\addcontentsline{toc}{section}{Acknowledgment}
%\lipsum[1]


\section*{Acknowledgements}

This material is based upon work supported by: the MIT Climate and Sustainability Consortium Scholars Program, MIT J-WAFS seed grant \#2040131, National Science Foundation award \#2330423, and Caltech Resnick Sustainability Institute Impact Grant ``Continuous, accurate and cost-effective counting of migrating salmon for conservation and fishery management in the Pacific Northwest.'' Thanks to Erik Young and Suzanne Stathatos for input and discussions, and Bill Hanot for initial conversations on echogram generation.

\balance

\bibliographystyle{IEEEtran}
\bibliography{references}

\end{document}
