
\documentclass[10pt,conference]{IEEEtran} 

\IEEEoverridecommandlockouts


\def\method{\text MixMin~}
\def\methodnospace{\text MixMin}
\def\genmethod{$\mathbb{R}$\text Min~}
\def\genmethodnospace{ $\mathbb{R}$\text Min}



\begin{document}

\title{Combining Language and App UI Analysis for the Automated Assessment of Bug Reproduction Steps
\thanks{The first two authors contributed equally to this work.}
}

\author{
	
	\IEEEauthorblockN{Junayed Mahmud\IEEEauthorrefmark{1}, Antu Saha\IEEEauthorrefmark{2}, Oscar Chaparro\IEEEauthorrefmark{2}, Kevin Moran\IEEEauthorrefmark{1}, Andrian Marcus\IEEEauthorrefmark{3}}
    \IEEEauthorblockA{\IEEEauthorrefmark{1}\textit{University of Central Florida (USA)}, 
		    \IEEEauthorrefmark{2}\textit{William \& Mary (USA)},
		    \IEEEauthorrefmark{3}\textit{George Mason University (USA)}
		    \\ \href{mailto:}{junayed.mahmud@ucf.edu}, \href{mailto:}{asaha02@wm.edu}, \href{mailto:}{oscarch@wm.edu},
		    \href{mailto:}{kpmoran@ucf.edu},
		    \href{mailto:}{amarcus7@gmu.edu}
    }
}

\maketitle

%page numbering
\thispagestyle{plain}
\pagestyle{plain}

\begin{abstract}
Bug reports are essential for developers to confirm software problems, investigate their causes, and validate fixes. Unfortunately, reports often miss important information or are written unclearly, which can cause delays, increased issue resolution effort, or even the inability to solve issues. One of the most common components of reports that are problematic is the steps to reproduce the bug(s) (S2Rs), which are essential to replicate the described program failures and reason about fixes. Given the proclivity for deficiencies in reported S2Rs, prior work has proposed techniques that assist reporters in writing or assessing the quality of S2Rs. However, automated understanding of S2Rs is challenging, and requires linking nuanced natural language phrases with specific, semantically related program information. Prior techniques often struggle to form such language $\leftrightarrow$ program connections -- due to issues in language variability and limitations of information gleaned from program analyses. 

To more effectively tackle the problem of S2R quality annotation, we propose a new technique called \tool, which leverages the language understanding capabilities of LLMs to identify and extract the S2Rs from bug reports and \rev{map} them to GUI interactions in a program state model derived via dynamic analysis. We compared \tool to a related state-of-the-art approach and we found that \tool annotates S2Rs 25.2\% better (in terms of F1 score) than the baseline. 
Additionally, \tool suggests more accurate missing S2Rs than the baseline (by 71.4\% in terms of F1 score). 
\looseness=-1
\end{abstract}




\section{Introduction}
Backdoor attacks pose a concealed yet profound security risk to machine learning (ML) models, for which the adversaries can inject a stealth backdoor into the model during training, enabling them to illicitly control the model's output upon encountering predefined inputs. These attacks can even occur without the knowledge of developers or end-users, thereby undermining the trust in ML systems. As ML becomes more deeply embedded in critical sectors like finance, healthcare, and autonomous driving \citep{he2016deep, liu2020computing, tournier2019mrtrix3, adjabi2020past}, the potential damage from backdoor attacks grows, underscoring the emergency for developing robust defense mechanisms against backdoor attacks.

To address the threat of backdoor attacks, researchers have developed a variety of strategies \cite{liu2018fine,wu2021adversarial,wang2019neural,zeng2022adversarial,zhu2023neural,Zhu_2023_ICCV, wei2024shared,wei2024d3}, aimed at purifying backdoors within victim models. These methods are designed to integrate with current deployment workflows seamlessly and have demonstrated significant success in mitigating the effects of backdoor triggers \cite{wubackdoorbench, wu2023defenses, wu2024backdoorbench,dunnett2024countering}.  However, most state-of-the-art (SOTA) backdoor purification methods operate under the assumption that a small clean dataset, often referred to as \textbf{auxiliary dataset}, is available for purification. Such an assumption poses practical challenges, especially in scenarios where data is scarce. To tackle this challenge, efforts have been made to reduce the size of the required auxiliary dataset~\cite{chai2022oneshot,li2023reconstructive, Zhu_2023_ICCV} and even explore dataset-free purification techniques~\cite{zheng2022data,hong2023revisiting,lin2024fusing}. Although these approaches offer some improvements, recent evaluations \cite{dunnett2024countering, wu2024backdoorbench} continue to highlight the importance of sufficient auxiliary data for achieving robust defenses against backdoor attacks.

While significant progress has been made in reducing the size of auxiliary datasets, an equally critical yet underexplored question remains: \emph{how does the nature of the auxiliary dataset affect purification effectiveness?} In  real-world  applications, auxiliary datasets can vary widely, encompassing in-distribution data, synthetic data, or external data from different sources. Understanding how each type of auxiliary dataset influences the purification effectiveness is vital for selecting or constructing the most suitable auxiliary dataset and the corresponding technique. For instance, when multiple datasets are available, understanding how different datasets contribute to purification can guide defenders in selecting or crafting the most appropriate dataset. Conversely, when only limited auxiliary data is accessible, knowing which purification technique works best under those constraints is critical. Therefore, there is an urgent need for a thorough investigation into the impact of auxiliary datasets on purification effectiveness to guide defenders in  enhancing the security of ML systems. 

In this paper, we systematically investigate the critical role of auxiliary datasets in backdoor purification, aiming to bridge the gap between idealized and practical purification scenarios.  Specifically, we first construct a diverse set of auxiliary datasets to emulate real-world conditions, as summarized in Table~\ref{overall}. These datasets include in-distribution data, synthetic data, and external data from other sources. Through an evaluation of SOTA backdoor purification methods across these datasets, we uncover several critical insights: \textbf{1)} In-distribution datasets, particularly those carefully filtered from the original training data of the victim model, effectively preserve the model’s utility for its intended tasks but may fall short in eliminating backdoors. \textbf{2)} Incorporating OOD datasets can help the model forget backdoors but also bring the risk of forgetting critical learned knowledge, significantly degrading its overall performance. Building on these findings, we propose Guided Input Calibration (GIC), a novel technique that enhances backdoor purification by adaptively transforming auxiliary data to better align with the victim model’s learned representations. By leveraging the victim model itself to guide this transformation, GIC optimizes the purification process, striking a balance between preserving model utility and mitigating backdoor threats. Extensive experiments demonstrate that GIC significantly improves the effectiveness of backdoor purification across diverse auxiliary datasets, providing a practical and robust defense solution.

Our main contributions are threefold:
\textbf{1) Impact analysis of auxiliary datasets:} We take the \textbf{first step}  in systematically investigating how different types of auxiliary datasets influence backdoor purification effectiveness. Our findings provide novel insights and serve as a foundation for future research on optimizing dataset selection and construction for enhanced backdoor defense.
%
\textbf{2) Compilation and evaluation of diverse auxiliary datasets:}  We have compiled and rigorously evaluated a diverse set of auxiliary datasets using SOTA purification methods, making our datasets and code publicly available to facilitate and support future research on practical backdoor defense strategies.
%
\textbf{3) Introduction of GIC:} We introduce GIC, the \textbf{first} dedicated solution designed to align auxiliary datasets with the model’s learned representations, significantly enhancing backdoor mitigation across various dataset types. Our approach sets a new benchmark for practical and effective backdoor defense.




\begin{figure*}[t]
    \centering
    \includegraphics[width=0.97\linewidth]{figs/motivating_example_fig}
    \caption{Motivating example}
    \label{fig:motivating-example}
\end{figure*}


\section{Methodology}
\label{sec:approach}

In this section, we present the \textbf{N}eurosymbolic \textbf{P}rogram \textbf{C}omprehension (\framework) framework, which leverages SHAP values (refer to \secref{sec:background}) to interpret and guide model predictions. We first describe our approach to identifying patterns in SHAP values for input features. Next, we explain how these patterns are transformed into symbolic rules to improve model performance, particularly in scenarios with low prediction confidence.

%%%%%%%%%%%%%%%%%%% NPC PIPELINE  
\begin{figure}[ht]
		\centering
  \vspace{-1.5em}
  \includegraphics[width=0.45\textwidth]{images/pipeline.pdf}
		\caption{Description of \framework framework as a sequence of steps.}
    \label{fig:npc_pipeline}
    \vspace{-1em}
\end{figure}
%%%%%%%%%%%%%%%%%%

%%%%%%% RESULTS TABLE 

\begin{table*}[t]
\centering
\fontsize{11pt}{11pt}\selectfont
\begin{tabular}{lllllllllllll}
\toprule
\multicolumn{1}{c}{\textbf{task}} & \multicolumn{2}{c}{\textbf{Mir}} & \multicolumn{2}{c}{\textbf{Lai}} & \multicolumn{2}{c}{\textbf{Ziegen.}} & \multicolumn{2}{c}{\textbf{Cao}} & \multicolumn{2}{c}{\textbf{Alva-Man.}} & \multicolumn{1}{c}{\textbf{avg.}} & \textbf{\begin{tabular}[c]{@{}l@{}}avg.\\ rank\end{tabular}} \\
\multicolumn{1}{c}{\textbf{metrics}} & \multicolumn{1}{c}{\textbf{cor.}} & \multicolumn{1}{c}{\textbf{p-v.}} & \multicolumn{1}{c}{\textbf{cor.}} & \multicolumn{1}{c}{\textbf{p-v.}} & \multicolumn{1}{c}{\textbf{cor.}} & \multicolumn{1}{c}{\textbf{p-v.}} & \multicolumn{1}{c}{\textbf{cor.}} & \multicolumn{1}{c}{\textbf{p-v.}} & \multicolumn{1}{c}{\textbf{cor.}} & \multicolumn{1}{c}{\textbf{p-v.}} &  &  \\ \midrule
\textbf{S-Bleu} & 0.50 & 0.0 & 0.47 & 0.0 & 0.59 & 0.0 & 0.58 & 0.0 & 0.68 & 0.0 & 0.57 & 5.8 \\
\textbf{R-Bleu} & -- & -- & 0.27 & 0.0 & 0.30 & 0.0 & -- & -- & -- & -- & - &  \\
\textbf{S-Meteor} & 0.49 & 0.0 & 0.48 & 0.0 & 0.61 & 0.0 & 0.57 & 0.0 & 0.64 & 0.0 & 0.56 & 6.1 \\
\textbf{R-Meteor} & -- & -- & 0.34 & 0.0 & 0.26 & 0.0 & -- & -- & -- & -- & - &  \\
\textbf{S-Bertscore} & \textbf{0.53} & 0.0 & {\ul 0.80} & 0.0 & \textbf{0.70} & 0.0 & {\ul 0.66} & 0.0 & {\ul0.78} & 0.0 & \textbf{0.69} & \textbf{1.7} \\
\textbf{R-Bertscore} & -- & -- & 0.51 & 0.0 & 0.38 & 0.0 & -- & -- & -- & -- & - &  \\
\textbf{S-Bleurt} & {\ul 0.52} & 0.0 & {\ul 0.80} & 0.0 & 0.60 & 0.0 & \textbf{0.70} & 0.0 & \textbf{0.80} & 0.0 & {\ul 0.68} & {\ul 2.3} \\
\textbf{R-Bleurt} & -- & -- & 0.59 & 0.0 & -0.05 & 0.13 & -- & -- & -- & -- & - &  \\
\textbf{S-Cosine} & 0.51 & 0.0 & 0.69 & 0.0 & {\ul 0.62} & 0.0 & 0.61 & 0.0 & 0.65 & 0.0 & 0.62 & 4.4 \\
\textbf{R-Cosine} & -- & -- & 0.40 & 0.0 & 0.29 & 0.0 & -- & -- & -- & -- & - & \\ \midrule
\textbf{QuestEval} & 0.23 & 0.0 & 0.25 & 0.0 & 0.49 & 0.0 & 0.47 & 0.0 & 0.62 & 0.0 & 0.41 & 9.0 \\
\textbf{LLaMa3} & 0.36 & 0.0 & \textbf{0.84} & 0.0 & {\ul{0.62}} & 0.0 & 0.61 & 0.0 &  0.76 & 0.0 & 0.64 & 3.6 \\
\textbf{our (3b)} & 0.49 & 0.0 & 0.73 & 0.0 & 0.54 & 0.0 & 0.53 & 0.0 & 0.7 & 0.0 & 0.60 & 5.8 \\
\textbf{our (8b)} & 0.48 & 0.0 & 0.73 & 0.0 & 0.52 & 0.0 & 0.53 & 0.0 & 0.7 & 0.0 & 0.59 & 6.3 \\  \bottomrule
\end{tabular}
\caption{Pearson correlation on human evaluation on system output. `R-': reference-based. `S-': source-based.}
\label{tab:sys}
\end{table*}



\begin{table}%[]
\centering
\fontsize{11pt}{11pt}\selectfont
\begin{tabular}{llllll}
\toprule
\multicolumn{1}{c}{\textbf{task}} & \multicolumn{1}{c}{\textbf{Lai}} & \multicolumn{1}{c}{\textbf{Zei.}} & \multicolumn{1}{c}{\textbf{Scia.}} & \textbf{} & \textbf{} \\ 
\multicolumn{1}{c}{\textbf{metrics}} & \multicolumn{1}{c}{\textbf{cor.}} & \multicolumn{1}{c}{\textbf{cor.}} & \multicolumn{1}{c}{\textbf{cor.}} & \textbf{avg.} & \textbf{\begin{tabular}[c]{@{}l@{}}avg.\\ rank\end{tabular}} \\ \midrule
\textbf{S-Bleu} & 0.40 & 0.40 & 0.19* & 0.33 & 7.67 \\
\textbf{S-Meteor} & 0.41 & 0.42 & 0.16* & 0.33 & 7.33 \\
\textbf{S-BertS.} & {\ul0.58} & 0.47 & 0.31 & 0.45 & 3.67 \\
\textbf{S-Bleurt} & 0.45 & {\ul 0.54} & {\ul 0.37} & 0.45 & {\ul 3.33} \\
\textbf{S-Cosine} & 0.56 & 0.52 & 0.3 & {\ul 0.46} & {\ul 3.33} \\ \midrule
\textbf{QuestE.} & 0.27 & 0.35 & 0.06* & 0.23 & 9.00 \\
\textbf{LlaMA3} & \textbf{0.6} & \textbf{0.67} & \textbf{0.51} & \textbf{0.59} & \textbf{1.0} \\
\textbf{Our (3b)} & 0.51 & 0.49 & 0.23* & 0.39 & 4.83 \\
\textbf{Our (8b)} & 0.52 & 0.49 & 0.22* & 0.43 & 4.83 \\ \bottomrule
\end{tabular}
\caption{Pearson correlation on human ratings on reference output. *not significant; we cannot reject the null hypothesis of zero correlation}
\label{tab:ref}
\end{table}


\begin{table*}%[]
\centering
\fontsize{11pt}{11pt}\selectfont
\begin{tabular}{lllllllll}
\toprule
\textbf{task} & \multicolumn{1}{c}{\textbf{ALL}} & \multicolumn{1}{c}{\textbf{sentiment}} & \multicolumn{1}{c}{\textbf{detoxify}} & \multicolumn{1}{c}{\textbf{catchy}} & \multicolumn{1}{c}{\textbf{polite}} & \multicolumn{1}{c}{\textbf{persuasive}} & \multicolumn{1}{c}{\textbf{formal}} & \textbf{\begin{tabular}[c]{@{}l@{}}avg. \\ rank\end{tabular}} \\
\textbf{metrics} & \multicolumn{1}{c}{\textbf{cor.}} & \multicolumn{1}{c}{\textbf{cor.}} & \multicolumn{1}{c}{\textbf{cor.}} & \multicolumn{1}{c}{\textbf{cor.}} & \multicolumn{1}{c}{\textbf{cor.}} & \multicolumn{1}{c}{\textbf{cor.}} & \multicolumn{1}{c}{\textbf{cor.}} &  \\ \midrule
\textbf{S-Bleu} & -0.17 & -0.82 & -0.45 & -0.12* & -0.1* & -0.05 & -0.21 & 8.42 \\
\textbf{R-Bleu} & - & -0.5 & -0.45 &  &  &  &  &  \\
\textbf{S-Meteor} & -0.07* & -0.55 & -0.4 & -0.01* & 0.1* & -0.16 & -0.04* & 7.67 \\
\textbf{R-Meteor} & - & -0.17* & -0.39 & - & - & - & - & - \\
\textbf{S-BertScore} & 0.11 & -0.38 & -0.07* & -0.17* & 0.28 & 0.12 & 0.25 & 6.0 \\
\textbf{R-BertScore} & - & -0.02* & -0.21* & - & - & - & - & - \\
\textbf{S-Bleurt} & 0.29 & 0.05* & 0.45 & 0.06* & 0.29 & 0.23 & 0.46 & 4.2 \\
\textbf{R-Bleurt} & - &  0.21 & 0.38 & - & - & - & - & - \\
\textbf{S-Cosine} & 0.01* & -0.5 & -0.13* & -0.19* & 0.05* & -0.05* & 0.15* & 7.42 \\
\textbf{R-Cosine} & - & -0.11* & -0.16* & - & - & - & - & - \\ \midrule
\textbf{QuestEval} & 0.21 & {\ul{0.29}} & 0.23 & 0.37 & 0.19* & 0.35 & 0.14* & 4.67 \\
\textbf{LlaMA3} & \textbf{0.82} & \textbf{0.80} & \textbf{0.72} & \textbf{0.84} & \textbf{0.84} & \textbf{0.90} & \textbf{0.88} & \textbf{1.00} \\
\textbf{Our (3b)} & 0.47 & -0.11* & 0.37 & 0.61 & 0.53 & 0.54 & 0.66 & 3.5 \\
\textbf{Our (8b)} & {\ul{0.57}} & 0.09* & {\ul 0.49} & {\ul 0.72} & {\ul 0.64} & {\ul 0.62} & {\ul 0.67} & {\ul 2.17} \\ \bottomrule
\end{tabular}
\caption{Pearson correlation on human ratings on our constructed test set. 'R-': reference-based. 'S-': source-based. *not significant; we cannot reject the null hypothesis of zero correlation}
\label{tab:con}
\end{table*}

\section{Results}
We benchmark the different metrics on the different datasets using correlation to human judgement. For content preservation, we show results split on data with system output, reference output and our constructed test set: we show that the data source for evaluation leads to different conclusions on the metrics. In addition, we examine whether the metrics can rank style transfer systems similar to humans. On style strength, we likewise show correlations between human judgment and zero-shot evaluation approaches. When applicable, we summarize results by reporting the average correlation. And the average ranking of the metric per dataset (by ranking which metric obtains the highest correlation to human judgement per dataset). 

\subsection{Content preservation}
\paragraph{How do data sources affect the conclusion on best metric?}
The conclusions about the metrics' performance change radically depending on whether we use system output data, reference output, or our constructed test set. Ideally, a good metric correlates highly with humans on any data source. Ideally, for meta-evaluation, a metric should correlate consistently across all data sources, but the following shows that the correlations indicate different things, and the conclusion on the best metric should be drawn carefully.

Looking at the metrics correlations with humans on the data source with system output (Table~\ref{tab:sys}), we see a relatively high correlation for many of the metrics on many tasks. The overall best metrics are S-BertScore and S-BLEURT (avg+avg rank). We see no notable difference in our method of using the 3B or 8B model as the backbone.

Examining the average correlations based on data with reference output (Table~\ref{tab:ref}), now the zero-shoot prompting with LlaMA3 70B is the best-performing approach ($0.59$ avg). Tied for second place are source-based cosine embedding ($0.46$ avg), BLEURT ($0.45$ avg) and BertScore ($0.45$ avg). Our method follows on a 5. place: here, the 8b version (($0.43$ avg)) shows a bit stronger results than 3b ($0.39$ avg). The fact that the conclusions change, whether looking at reference or system output, confirms the observations made by \citet{scialom-etal-2021-questeval} on simplicity transfer.   

Now consider the results on our test set (Table~\ref{tab:con}): Several metrics show low or no correlation; we even see a significantly negative correlation for some metrics on ALL (BLEU) and for specific subparts of our test set for BLEU, Meteor, BertScore, Cosine. On the other end, LlaMA3 70B is again performing best, showing strong results ($0.82$ in ALL). The runner-up is now our 8B method, with a gap to the 3B version ($0.57$ vs $0.47$ in ALL). Note our method still shows zero correlation for the sentiment task. After, ranks BLEURT ($0.29$), QuestEval ($0.21$), BertScore ($0.11$), Cosine ($0.01$).  

On our test set, we find that some metrics that correlate relatively well on the other datasets, now exhibit low correlation. Hence, with our test set, we can now support the logical reasoning with data evidence: Evaluation of content preservation for style transfer needs to take the style shift into account. This conclusion could not be drawn using the existing data sources: We hypothesise that for the data with system-based output, successful output happens to be very similar to the source sentence and vice versa, and reference-based output might not contain server mistakes as they are gold references. Thus, none of the existing data sources tests the limits of the metrics.  


\paragraph{How do reference-based metrics compare to source-based ones?} Reference-based metrics show a lower correlation than the source-based counterpart for all metrics on both datasets with ratings on references (Table~\ref{tab:sys}). As discussed previously, reference-based metrics for style transfer have the drawback that many different good solutions on a rewrite might exist and not only one similar to a reference.


\paragraph{How well can the metrics rank the performance of style transfer methods?}
We compare the metrics' ability to judge the best style transfer methods w.r.t. the human annotations: Several of the data sources contain samples from different style transfer systems. In order to use metrics to assess the quality of the style transfer system, metrics should correctly find the best-performing system. Hence, we evaluate whether the metrics for content preservation provide the same system ranking as human evaluators. We take the mean of the score for every output on each system and the mean of the human annotations; we compare the systems using the Kendall's Tau correlation. 

We find only the evaluation using the dataset Mir, Lai, and Ziegen to result in significant correlations, probably because of sparsity in a number of system tests (App.~\ref{app:dataset}). Our method (8b) is the only metric providing a perfect ranking of the style transfer system on the Lai data, and Llama3 70B the only one on the Ziegen data. Results in App.~\ref{app:results}. 


\subsection{Style strength results}
%Evaluating style strengths is a challenging task. 
Llama3 70B shows better overall results than our method. However, our method scores higher than Llama3 70B on 2 out of 6 datasets, but it also exhibits zero correlation on one task (Table~\ref{tab:styleresults}).%More work i s needed on evaluating style strengths. 
 
\begin{table}%[]
\fontsize{11pt}{11pt}\selectfont
\begin{tabular}{lccc}
\toprule
\multicolumn{1}{c}{\textbf{}} & \textbf{LlaMA3} & \textbf{Our (3b)} & \textbf{Our (8b)} \\ \midrule
\textbf{Mir} & 0.46 & 0.54 & \textbf{0.57} \\
\textbf{Lai} & \textbf{0.57} & 0.18 & 0.19 \\
\textbf{Ziegen.} & 0.25 & 0.27 & \textbf{0.32} \\
\textbf{Alva-M.} & \textbf{0.59} & 0.03* & 0.02* \\
\textbf{Scialom} & \textbf{0.62} & 0.45 & 0.44 \\
\textbf{\begin{tabular}[c]{@{}l@{}}Our Test\end{tabular}} & \textbf{0.63} & 0.46 & 0.48 \\ \bottomrule
\end{tabular}
\caption{Style strength: Pearson correlation to human ratings. *not significant; we cannot reject the null hypothesis of zero corelation}
\label{tab:styleresults}
\end{table}

\subsection{Ablation}
We conduct several runs of the methods using LLMs with variations in instructions/prompts (App.~\ref{app:method}). We observe that the lower the correlation on a task, the higher the variation between the different runs. For our method, we only observe low variance between the runs.
None of the variations leads to different conclusions of the meta-evaluation. Results in App.~\ref{app:results}.
%%%%%%%%

\subsection{Pattern Identification}
\label{sec:npc_pattern_identification}

Drawing inspiration from probing classifier techniques widely used in NLP \cite{hewitt_designing_2019} and SE \cite{lopez_ast-probe_2022, troshin_probing_2022}, our framework leverages supervised machine learning techniques to identify patterns in the SHAP values computed for specific predictions in classification tasks. Probing techniques work by examining the latent representations of a model to determine the extent to which specific types of information are encoded. Specifically, a supervised model (\eg classifier) is trained to predict properties of interest from the neural network's hidden representations \cite{belinkov_probing_2021}. In the context of our framework, we propose training classifiers to predict target classes from SHAP value distributions enabling the formulation of symbolic rules, as illustrated in \figref{fig:npc_pipeline}.

First, given a set of inputs $\mathbb{X}$ that the \lcm predicts as belonging to a specific class $y \in Y$ (\eg Secure/Insecure), we compute SHAP values ($\phi$) for each input $x \in \mathbb{X}$. The SHAP values are calculated relative to the expected predicted class: $y = \mathbb{E}[f(\mathbb{X})]$. Inspired by syntax decomposition \cite{syntax_capabilities, palacio_towards_2024, docode}, we apply an alignment function $\delta(w_i): w_i \to \mu \in \mathbb{M}$ to tag tokens $w_i \in x$ with meaningful AST types $\mathbb{M}$, defined by the programming language grammar. This process produces a SHAP tensor for each target class: ${(i, w_i, \phi_i, \mu_i)}$, where $i$ is the position, $w_i$ is the token, $\phi_i$ is the SHAP value, and $\mu_i$ is the associated AST type. The entire process is depicted in region \circled{1} of \figref{fig:npc_pipeline}.

After computing the SHAP tensors for each target class in $Y$, we merge them and group the $\phi$ values by the AST tag associated with their corresponding tokens. We define position ranges as $[a, b], \quad 0 \leq a \leq b \leq \max{|x|: x \in \mathbb{X}}$. For each range, we train a supervised model (\eg logistic regression, decision tree, random forest) to identify curves that best capture the relationship between $\phi$ values and feature positions. Curves with an accuracy exceeding $60\%$ and a well-defined decision boundary for the target class (\ie intersection with the x-axis) provide evidence of patterns in specific AST type positions where SHAP values influence the model's decisions. The computed curves allow us to identify regions and position ranges where a feature’s $\phi$ value (\ie SHAP value corresponding to a specific AST node) consistently influences the overall prediction of the expected outputs either positively or negatively.

\subsection{Symbolic Rules}
From the identified patterns in SHAP value distributions, we derive symbolic rules encapsulating feature structures that align with expected model predictions. These rules consist of two parts: (i) configurations positively correlated with the predicted label, forming symbolic rules for correctly predicted patterns, and; (ii) complementary rules for configurations linked to lower prediction reliability, enabling targeted model adjustments in uncertain cases. We derive these rules by grouping SHAP-influential features within each type $\mu \in \mathbb{M}$ and formulating conditions based on both feature presence and SHAP value contributions. For instance, if a feature linked to an AST node consistently shows high SHAP values for insecure code at the input's start, it may represent a necessary condition for an \textbf{\textit{insecure}} prediction in the rule. As illustrated in region \circled{3} of \figref{fig:npc_pipeline}, the derived symbolic rules can be applied during the post-training stage of an ML pipeline, for instance, in supervised fine-tuning and knowledge distillation to facilitate knowledge transfer between models.

% !TeX root = 0_main

\section{\tool's Prompt Development and Evaluation}
\label{sec:prompt_development}

This section describes how we developed and evaluated the LLM prompts for three distinct tasks: (i) S2R sentence identification, (ii) individual S2R extraction, and (iii) individual S2R mapping to app interactions. 
We adopted a rigorous, comprehensive, and data-driven approach in which we designed an initial prompt that was iteratively evaluated and refined into new prompts. 
Prompt development and evaluation followed a quantitative and qualitative methodology based on a set of Android app bug reports. 
Overall, we designed and evaluated 12 prompt templates across all three tasks.
\rev{To generate GPT-4 responses with the prompts for all tasks, we used a temperature of 0 to minimize randomness/non-determinism in the responses.}

\subsection{Development Dataset Construction}
\label{sec:dev_dataset}

We constructed a dataset of 54 bug reports and corresponding ground truth data, with manually identified S2R sentences, individual S2Rs, and interactions mapped to each S2R.

\subsubsection{Bug Report Collection}

We selected the 54 bug reports from the dataset released by Saha \etal~\cite{saha2024toward}, which contains reproducible mobile app bug reports from the AndroR2 dataset~\cite{wendland2021,Johnson2022}. 
These reports describe bugs for 31 Android apps of various domains (\eg web browsing, WiFi network diagnosis, and finance tracking). 
The reported bugs span different bug types, namely crashes (15 reports), output problems~(19), UI cosmetic issues (13), and navigation problems (7). 

\subsubsection{S2R Sentence Labeling}
\label{sec:identification_data_dev}

Two authors annotated the 1,031 sentences present in the bug reports as either S2R or non-S2R, following the S2R criteria and methodology defined by Chaparro \etal~\cite{Chaparro2017}. 
One author annotated each sentence, while the second author validated the annotations, recording disagreements and their rationale. 
The authors agreed on the annotations for 1,002 sentences (97.2\%, 0.91 Cohen's kappa~\cite{Cohen}), which represents near-perfect agreement. 
Disagreements were resolved via discussion. 
The most common reasons for disagreements were content misinterpretations and mistakes (\eg\ a sentence describing the observed behavior, not S2Rs). 
In total, the 54 bug reports contain 189 S2R sentences (3.5 per report on average), while the remaining 842 sentences describe non-S2R content.
\looseness=-1

\subsubsection{Individual S2Rs Extraction}
\label{sec:extraction_data_dev}
Two authors manually inspected the 189 S2R sentences to extract individual S2Rs (phrases describing a single interaction with the app). 
One author read and extracted the individual S2Rs in the format defined in \Cref{sec:indiv-s2rs-approach}. 
The extracted S2Rs were validated by a second author. 
They discussed disagreements to reach a consensus where needed. 
From the 189 S2R sentences, we extracted 246 individual S2Rs with an agreement rate of 97.6\%. 

\subsubsection{S2Rs to GUI Interaction Mapping}
\label{sec:qa_data_dev}

To create ground truth mappings between individual S2Rs and GUI app interactions, we first built the execution models (\ie graphs) for the 31 apps corresponding to the bug reports. 
To do so, we executed the \CrashScope tool~\cite{Moran2016} using the corresponding APKs (from the original dataset~\cite{saha2024toward,Johnson2022}) and a Pixel 2 Android emulator. 
We also used the manual interaction traces collected as part of Saha \etal's dataset~\cite{saha2024toward}. Both the \CrashScope and manual interaction traces consist of GUI-event execution traces and (video) screen captures showing the executed interactions. 
We used Song \etal's toolkit~\cite{song2022burt} to parse the traces and build the execution graphs.

Two authors manually inspected the execution data, graphs, and reproduction screen captures to map each S2R to graph nodes and interactions. 
One author first inspected this data to identify the GUI screen and target GUI component for each S2R. 
Then, the author identified the graph node corresponding to such screen, and within it, the interaction corresponding to the S2R. 
In the process, missing steps and the path that represented a minimal bug reproduction scenario were identified. 
A second author followed the same procedure to verify the interactions/nodes mapped to the S2Rs and the reproduction paths identified by the first author. 
Both authors discussed any disagreements, involving a third author where necessary.  
\looseness=-1

We applied the above methodology on a sample of 10 bug reports, in such a way that they spanned different bugs types, 
apps of different domains (9 apps), and S2R types (taps, types, \etc). 
The two authors created the ground truth for 46 individual S2Rs among 49 individual S2Rs for the 10 bug reports, agreeing on 43 S2Rs (agreement rate of 93.5\%). The excluded three individual S2Rs did not have corresponding app interactions in the execution model because they are performed outside the app (\eg\ \textit{"install the app"}), and hence, are not included in the graph.
Common reasons for disagreements were unclear individual S2Rs and misinterpretation of graph nodes/interactions. 
During the data creation process, we realized that it would take the two authors a prohibitive amount of effort to create the data for the remaining 44 bug reports. 
Therefore, we decided to focus on the S2R mapping prompt development using only the 10 bug reports and redirect our effort to curating the test data used for \tool's evaluation (see \Cref{sec:empirical_evaluation}). 

\subsection{Prompt Development Methodology}
\label{sec:prompt_development_methodology}

For each of three tasks where \tool uses GPT-4, our overall data-driven methodology used three prompting strategies, commonly used in software engineering research~\cite{hou2023large}: 
\begin{itemize}
	\item Zero Shot (ZS) prompting: starting from a base prompt template that includes the task description, input, and response format, we iteratively executed, evaluated, and refined the template until the performance plateaued. This involved computing performance metrics (precision, recall, and F1 score) against the ground truth, qualitatively analyzing false positives (FP) and negatives (FN), and adjusting the prompt to address those cases. For example, as S2R sentence identification is a classification task, two authors investigated the FP and FN of the GPT-4 responses to derive the classification criteria (\Cref{fig:prompt-structure}a) to better guide GPT-4 in the S2R sentence classification task.
    \rev{This process resulted in four versions of each type of prompt template. To determine if performance plateaued, we monitored the F1 score. For example, from version 3 to version 4 the F1 score decreased by 0.001 for the S2R identification task. Based on this minimal change, we selected version 3 as the optimal prompt for this phase.}
	\item Few Shot (FS) prompting: starting from the obtained ZS template, we created a base FS template containing positive and negative examples selected from the remaining bugs of Saha \etal's dataset~\cite{saha2024toward} and the expected output. The example bug reports are representative of each task and selected based on certain criteria, \eg\ various bug types (crash, output, \etc), and bug reports with different wordings and structures. We iteratively executed, evaluated, and refined the template until the performance no longer improved, in the same way we did it in ZS prompting.
	\item Chain of Thought (CoT) prompting: starting from the obtained FS template, we created a base CoT template containing explanations for \rev{the outcome of the positive and negative examples. The explanation for the outcome was designed by two authors after discussion and reaching a consensus.} We iteratively executed, evaluated, and refined the template until the performance plateaued, in the same way we did it in ZS and FS prompting.
\end{itemize}


\begin{figure}
	\centering
	\includegraphics[width=\linewidth]{figures/prompt.pdf}
	\caption{Structure of the Developed Prompts}
	\label{fig:prompt-structure}
\end{figure}

This methodology resulted in three prompt templates (one from each prompting strategy) for S2R identification and three templates for individual S2R extraction. 
For S2R mapping, since we defined mapping as a 2-step task, we designed two prompts for each strategy, resulting in six prompts. 
The 2-step task consisted of first asking GPT-4 to return a yes/no answer on whether an individual S2R maps to the interactions of a given screen and if the answer is yes, asking GPT-4 to return the list of interactions that the S2R maps to. 
Our tests revealed that this approach led to less noisy answers from GPT-4, compared to executing only the second step. 
In total, we developed 12 prompt templates. To help visualize our prompt templates, \Cref{fig:prompt-structure} illustrates the various components associated with the prompts for each task---our detailed templates are found in our replication package~\cite{package,doi}.

\subsection{Prompt Evaluation Results}
\label{sec:development_results}

We evaluated the prompt templates for S2R identification and extraction in terms of precision, recall, and F1 score, by executing these two phases in isolation. 
The F1 score was used to rank the templates. 
The S2R mapping prompt templates were evaluated by executing \tool's S2R quality assessment phase and evaluating the resulting S2R-interaction mappings. 
Since S2R mapping is a 2-step task, we evaluated each of the prompts based on the \# and \% of \textit{hits}, defined as follows.
For the first prompt, it is the number (and proportion) of correct predictions for the presence or absence of an S2R-interaction mapping in a given screen (out of the total number of predictions). 
For the second prompt, it is the number (and proportion) of correctly identified interactions for each individual S2R (out of the total number of individual S2Rs).


\begin{table}[t!]
	\centering
	\caption{Prompt Template Performance for S2R Identification}
	\label{tab:identification_results_dev_set}
	\resizebox{\columnwidth}{!}{%
		\begin{tabular}{c|c|c|c|c|c|c}
			\hline
			\textbf{Template} & \textbf{Precision} & \textbf{Recall} & \textbf{F1} & \textbf{\#TP} & \textbf{\#FP} & \textbf{\#FN} \\ \hline
			ZS & 0.929              & 0.968           & 0.948       & 183           & 14            & 6             \\ \hline
			FS   & 0.897              & 0.963           & 0.929       & 182           & 21            & 7             \\ \hline
			CoT  & 0.915              & 0.963           & 0.938       & 182           & 17            & 7             \\ \hline
		\end{tabular}%
	}
\end{table}

\begin{table}[t!]
	\centering
	\caption{Prompt Template Performance for S2R Extraction}
	\label{tab:indiv-s2r-study-results}
	\resizebox{\columnwidth}{!}{%
		\begin{tabular}{c|c|c|c|c|c|c}
			\hline
			\textbf{Template} & \textbf{Precision} & \textbf{Recall} & \textbf{F1} & \textbf{\#TP} & \textbf{\#FP} & \textbf{\#FN} \\ \hline
			ZS              & 0.918              & 0.951           & 0.934       & 234            & 21             & 12             \\ \hline
			FS              & 0.897              & 0.951           & 0.923       & 234            & 27             & 12             \\ \hline
			CoT             & 0.810              & 0.951           & 0.875       & 234            & 55             & 12             \\ \hline
		\end{tabular}%
	}
	
\end{table}

\begin{table}[t!]
	\centering
	\caption{Prompt Performance for S2R-Interaction Mapping}
	\label{tab:indiv-s2r-matching-results}
	\resizebox{\columnwidth}{!}{%
		\begin{tabular}{c|ccc|cc}
			\hline
			\multirow{2}{*}{\textbf{Template}} & \multicolumn{3}{c|}{\textbf{1st-step template}}                                                             & \multicolumn{2}{c}{\textbf{2nd-step template}}          \\ \cline{2-6} 
			& \multicolumn{1}{c|}{\textbf{\# Predictions}} & \multicolumn{1}{c|}{\textbf{\# Hits}} & \textbf{Hit Rate} & \multicolumn{1}{c|}{\textbf{\# Hits}} & \textbf{Hit Rate} \\ \hline
			{ZS}                      & \multicolumn{1}{c|}{939}                    & \multicolumn{1}{c|}{887}              & 94.5\%            & \multicolumn{1}{c|}{30}               & 76.9\%            \\ \hline
			{FS}                      & \multicolumn{1}{c|}{970}                    & \multicolumn{1}{c|}{912}              & 94.0\%            & \multicolumn{1}{c|}{26}               & 66.7\%            \\ \hline
			{CoT}                     & \multicolumn{1}{c|}{1214}                   & \multicolumn{1}{c|}{1152}             & 94.9\%            & \multicolumn{1}{c|}{18}               & 46.2\%            \\ \hline
		\end{tabular}%
	}

\end{table}

\Cref{tab:identification_results_dev_set,tab:indiv-s2r-study-results,tab:indiv-s2r-matching-results} show the performance of the designed prompt templates for the three tasks: S2R identification, individual S2R extraction, and S2R mapping. 
Among the three templates for the S2R identification task, \textit{ZS} achieved the best performance across the three metrics having the lowest \# of FP (14) and FN~(6). Likewise, for the individual S2R extraction task, the \textit{ZS} template achieved the highest precision (0.918) with the lowest \# of FP (21), sharing the same \# of FN (12) with the other two prompts. Regarding the S2R mapping task, \tool with all three templates for the 1st-step prompt achieved a similar hit rate (94.0\% to 94.9\%) and with \textit{ZS} template for the 2nd-step prompt achieved the best hit rate of 76.9\%. 

Interestingly, although prior research has shown the superiority of \textit{CoT} prompts over \textit{ZS} and \textit{FS} prompts~\cite{hou2023large,Feng2024}, this is not the case for our tasks. Via qualitative analysis of GPT-4 responses, we observed that GPT-4 with \textit{FS} and \textit{CoT} prompts tends to include more unintended text in the responses compared to \textit{ZS} prompt which results in more false positives, \eg\ \textit{CoT} template for S2R extraction generated 55 FPs while \textit{ZS} template generated 21 FPs only. We conjecture that the long and complicated input (\eg\ bug reports can be long, and interaction information can be complicated) made the task difficult for GPT-4. Moreover, having three or four examples with reasoning made the prompts even longer.

As for all three tasks, \textit{ZS} templates outperformed the other two, we utilized the \textit{ZS} templates for implementing \tool.


% !TeX root = 0_main

\section{\tool's Evaluation Design}
\label{sec:empirical_evaluation}
\tool's evaluation has two main goals: (i) to evaluate \tool's ability to provide correct quality annotations for real bug reports, 
and (ii) to examine how well \tool can infer missing S2R information in bug reports. 
We apply \tool to a test dataset (see \Cref{sec:test_dataset}) comprising 21 bug reports, in order to provide a comparison with prior work. We aim to answer the following research questions (RQs):
\begin{itemize}
	\item \textbf{RQ$_{1}$:} How effective is \tool in generating correct S2R quality annotations?
	\item \textbf{RQ$_{2}$:} How accurately can \tool infer missing S2Rs?
\end{itemize}

\subsection{Evaluation Dataset}
\label{sec:test_dataset}

We used the bug reports (\ie\ \textit{test set}) used by Chaparro \etal~\cite{Chaparro2019}, which allow us to provide a direct comparison with their approach, \EulerC. 
This dataset contains 24 bug reports \rev{of various kinds ( crashes, UI problems, and navigation problems)} from six Android applications \rev{of different domains (web browsing, WiFi network diagnosis, finance tracking, \etc). The diverse evaluation set, separate from the development set, enabled us to assess the generalizability of the developed prompts across different bug reports.}  
We discarded three bug reports, as follows: (1) two bug reports~\cite{aard81, aard104} from the Aard Dictionary App~\cite{aardapp}, because the app version 1.4.1 is unable to load its dictionary database, and (2) one bug from Time Tracker app~\cite{atimetracker1}, because we could not generate the execution model for this app as the bug report requires a rotation action which \tool does not support.
Hence, our test set contains 21 bug reports from the original \EulerC dataset. 

Since this dataset does not contain any ground truth information for evaluating \tool, we constructed the ground truth manually.  We used the same methodology discussed in \Cref{sec:identification_data_dev,sec:extraction_data_dev} to do so for identifying S2R sentences and extracting individual S2Rs.

To construct the quality assessment ground truth, the first two authors mapped the extracted individual S2Rs to \rev{GUI} interactions manually following the methodology discussed in \Cref{sec:qa_data_dev}. App execution models for the bug reports were built by parsing execution traces collected via \CrashScope's app exploration and manual app \rev{usage}.  
One author identified the reproduction interactions on the generated data and mapped such interactions with the extracted individual S2Rs from the bug report. 
They collected the mapped interactions for each individual S2R, as well as the interactions that are required to reproduce the bug, but not reported in the bug report, \ie ground truth for missing steps. 
Each individual S2R was mapped with one or more interactions in the execution model path, as needed. Using the mapped interactions and the quality assessment model (discussed in \Cref{sec:quality_model}), they assigned quality labels to each individual S2R. 
A second author performed the same steps and validated the interactions in the reproduction scenario as well as the quality annotations.  Disagreements were resolved via discussion.

In summary, we identified 73 S2R sentences out of the 275 sentences present in the 21 bug reports with a near-perfect agreement between the two authors (98.2\% agreement rate and 0.88 Cohen's kappa \cite{Cohen}). 
From the 73 S2R sentences, we extracted 82 individual S2Rs with an agreement rate of 93.9\% between the two authors.
We discarded four individual S2Rs as they represent rotation operation and the current version of \tool does not support this operation. 
We assigned the remaining 78 individual S2Rs quality annotations (\ie\ 70 S2Rs as CS, 7 S2Rs as AS, 1 S2R as VM, and 38 S2Rs as MS). 
We identified 158 missing interactions, \ie\ missing steps for the 38 MS positions (\ie S2Rs with filled-in missing interactions). 
For constructing the annotations ground truth, the two authors agreed on 90\% of the cases. Cohen's kappa for individual S2R extraction and mapping is inapplicable since the labeling is not based on a discrete set of labels. 

\subsection{Baseline Approach}

We considered \EulerC~\cite{Chaparro2019} as the baseline approach, which also aims to assess the quality of S2Rs in a bug report. 
It identifies the S2R sentences from a bug report using deep learning techniques (\eg\ CNN~\cite{o2015introduction}, Bi-LSTM~\cite{zhou2016attention}). 
It identifies individual S2Rs via analysis of discourse patterns and assigns quality annotations by employing keyword-based mapping to app UI information.  
\rev{\EulerC and \tool generate similar quality reports, therefore we can directly compare the \tool reports to the original \EulerC reports  provided by \EulerC's replication package \cite{Chaparro2019}, 
to answer the RQs.}
\looseness=-1

\subsection{Evaluation Methodology} 

We executed \tool with the 21 bug reports on the test set,  producing the quality report for each bug report, including the quality annotations and missing steps. To answer \textbf{RQ$_{1}$}, we compared the \tool assigned quality annotations with the ground truth quality annotations. To answer \textbf{RQ$_{2}$}, we evaluated the generated missing steps by \tool against the ground truth missing steps. For both RQs, we computed precision, recall, and F1 score. We applied the same process for \EulerC and qualitatively analyzed the false positives (FP) and negatives (FN) to understand the limitations of both approaches.

\section{Experimental Results}
In this section, we present the main results in~\secref{sec:main}, followed by ablation studies on key design choices in~\secref{sec:ablation}.

\begin{table*}[t]
\renewcommand\arraystretch{1.05}
\centering
\setlength{\tabcolsep}{2.5mm}{}
\begin{tabular}{l|l|c|cc|cc}
type & model     & \#params      & FID$\downarrow$ & IS$\uparrow$ & Precision$\uparrow$ & Recall$\uparrow$ \\
\shline
GAN& BigGAN~\cite{biggan} & 112M & 6.95  & 224.5       & 0.89 & 0.38     \\
GAN& GigaGAN~\cite{gigagan}  & 569M      & 3.45  & 225.5       & 0.84 & 0.61\\  
GAN& StyleGan-XL~\cite{stylegan-xl} & 166M & 2.30  & 265.1       & 0.78 & 0.53  \\
\hline
Diffusion& ADM~\cite{adm}    & 554M      & 10.94 & 101.0        & 0.69 & 0.63\\
Diffusion& LDM-4-G~\cite{ldm}   & 400M  & 3.60  & 247.7       & -  & -     \\
Diffusion & Simple-Diffusion~\cite{diff1} & 2B & 2.44 & 256.3 & - & - \\
Diffusion& DiT-XL/2~\cite{dit} & 675M     & 2.27  & 278.2       & 0.83 & 0.57     \\
Diffusion&L-DiT-3B~\cite{dit-github}  & 3.0B    & 2.10  & 304.4       & 0.82 & 0.60    \\
Diffusion&DiMR-G/2R~\cite{liu2024alleviating} &1.1B& 1.63& 292.5& 0.79 &0.63 \\
Diffusion & MDTv2-XL/2~\cite{gao2023mdtv2} & 676M & 1.58 & 314.7 & 0.79 & 0.65\\
Diffusion & CausalFusion-H$^\dag$~\cite{deng2024causal} & 1B & 1.57 & - & - & - \\
\hline
Flow-Matching & SiT-XL/2~\cite{sit} & 675M & 2.06 & 277.5 & 0.83 & 0.59 \\
Flow-Matching&REPA~\cite{yu2024representation} &675M& 1.80 & 284.0 &0.81 &0.61\\    
Flow-Matching&REPA$^\dag$~\cite{yu2024representation}& 675M& 1.42&  305.7& 0.80& 0.65 \\
\hline
Mask.& MaskGIT~\cite{maskgit}  & 227M   & 6.18  & 182.1        & 0.80 & 0.51 \\
Mask. & TiTok-S-128~\cite{yu2024image} & 287M & 1.97 & 281.8 & - & - \\
Mask. & MAGVIT-v2~\cite{yu2024language} & 307M & 1.78 & 319.4 & - & - \\ 
Mask. & MaskBit~\cite{weber2024maskbit} & 305M & 1.52 & 328.6 & - & - \\
\hline
AR& VQVAE-2~\cite{vqvae2} & 13.5B    & 31.11           & $\sim$45     & 0.36           & 0.57          \\
AR& VQGAN~\cite{vqgan}& 227M  & 18.65 & 80.4         & 0.78 & 0.26   \\
AR& VQGAN~\cite{vqgan}   & 1.4B     & 15.78 & 74.3   & -  & -     \\
AR&RQTran.~\cite{rq}     & 3.8B    & 7.55  & 134.0  & -  & -    \\
AR& ViTVQ~\cite{vit-vqgan} & 1.7B  & 4.17  & 175.1  & -  & -    \\
AR & DART-AR~\cite{gu2025dart} & 812M & 3.98 & 256.8 & - & - \\
AR & MonoFormer~\cite{zhao2024monoformer} & 1.1B & 2.57 & 272.6 & 0.84 & 0.56\\
AR & Open-MAGVIT2-XL~\cite{luo2024open} & 1.5B & 2.33 & 271.8 & 0.84 & 0.54\\
AR&LlamaGen-3B~\cite{llamagen}  &3.1B& 2.18& 263.3 &0.81& 0.58\\
AR & FlowAR-H~\cite{flowar} & 1.9B & 1.65 & 296.5 & 0.83 & 0.60\\
AR & RAR-XXL~\cite{yu2024randomized} & 1.5B & 1.48 & 326.0 & 0.80 & 0.63 \\
\hline
MAR & MAR-B~\cite{mar} & 208M & 2.31 &281.7 &0.82 &0.57 \\
MAR & MAR-L~\cite{mar} &479M& 1.78 &296.0& 0.81& 0.60 \\
MAR & MAR-H~\cite{mar} & 943M&1.55& 303.7& 0.81 &0.62 \\
\hline
VAR&VAR-$d16$~\cite{var}   & 310M  & 3.30& 274.4& 0.84& 0.51    \\
VAR&VAR-$d20$~\cite{var}   &600M & 2.57& 302.6& 0.83& 0.56     \\
VAR&VAR-$d30$~\cite{var}   & 2.0B      & 1.97  & 323.1 & 0.82 & 0.59      \\
\hline
\modelname& \modelname-B    &172M   &1.72&280.4&0.82&0.59 \\
\modelname& \modelname-L   & 608M   & 1.28& 292.5&0.82&0.62\\
\modelname& \modelname-H    & 1.1B    & 1.24 &301.6&0.83&0.64\\
\end{tabular}
\caption{
\textbf{Generation Results on ImageNet-256.}
Metrics include Fréchet Inception Distance (FID), Inception Score (IS), Precision, and Recall. $^\dag$ denotes the use of guidance interval sampling~\cite{guidance}. The proposed \modelname-H achieves a state-of-the-art 1.24 FID on the ImageNet-256 benchmark without relying on vision foundation models (\eg, DINOv2~\cite{dinov2}) or guidance interval sampling~\cite{guidance}, as used in REPA~\cite{yu2024representation}.
}\label{tab:256}
\end{table*}

\subsection{Main Results}
\label{sec:main}
We conduct experiments on ImageNet~\cite{deng2009imagenet} at 256$\times$256 and 512$\times$512 resolutions. Following prior works~\cite{dit,mar}, we evaluate model performance using FID~\cite{fid}, Inception Score (IS)~\cite{is}, Precision, and Recall. \modelname is trained with the same hyper-parameters as~\cite{mar,dit} (\eg, 800 training epochs), with model sizes ranging from 172M to 1.1B parameters. See Appendix~\secref{sec:sup_hyper} for hyper-parameter details.





\begin{table}[t]
    \centering
    \begin{tabular}{c|c|c|c}
      model    &  \#params & FID$\downarrow$ & IS$\uparrow$ \\
      \shline
      VQGAN~\cite{vqgan}&227M &26.52& 66.8\\
      BigGAN~\cite{biggan}& 158M&8.43 &177.9\\
      MaskGiT~\cite{maskgit}& 227M&7.32& 156.0\\
      DiT-XL/2~\cite{dit} &675M &3.04& 240.8 \\
     DiMR-XL/3R~\cite{liu2024alleviating}& 525M&2.89 &289.8 \\
     VAR-d36~\cite{var}  & 2.3B& 2.63 & 303.2\\
     REPA$^\ddagger$~\cite{yu2024representation}&675M &2.08& 274.6 \\
     \hline
     \modelname-L & 608M&1.70& 281.5 \\
    \end{tabular}
    \caption{
    \textbf{Generation Results on ImageNet-512.} $^\ddagger$ denotes the use of DINOv2~\cite{dinov2}.
    }
    \label{tab:512}
\end{table}

\noindent\textbf{ImageNet-256.}
In~\tabref{tab:256}, we compare \modelname with previous state-of-the-art generative models.
Out best variant, \modelname-H, achieves a new state-of-the-art-performance of 1.24 FID, outperforming the GAN-based StyleGAN-XL~\cite{stylegan-xl} by 1.06 FID, masked-prediction-based MaskBit~\cite{maskgit} by 0.28 FID, AR-based RAR~\cite{yu2024randomized} by 0.24 FID, VAR~\cite{var} by 0.73 FID, MAR~\cite{mar} by 0.31 FID, and flow-matching-based REPA~\cite{yu2024representation} by 0.18 FID.
Notably, \modelname does not rely on vision foundation models~\cite{dinov2} or guidance interval sampling~\cite{guidance}, both of which were used in REPA~\cite{yu2024representation}, the previous best-performing model.
Additionally, our lightweight \modelname-B (172M), surpasses DiT-XL (675M)~\cite{dit} by 0.55 FID while achieving an inference speed of 9.8 images per second—20$\times$ faster than DiT-XL (0.5 images per second). Detailed speed comparison can be found in Appendix \ref{sec:speed}.



\noindent\textbf{ImageNet-512.}
In~\tabref{tab:512}, we report the performance of \modelname on ImageNet-512.
Similarly, \modelname-L sets a new state-of-the-art FID of 1.70, outperforming the diffusion based DiT-XL/2~\cite{dit} and DiMR-XL/3R~\cite{liu2024alleviating} by a large margin of 1.34 and 1.19 FID, respectively.
Additionally, \modelname-L also surpasses the previous best autoregressive model VAR-d36~\cite{var} and flow-matching-based REPA~\cite{yu2024representation} by 0.93 and 0.38 FID, respectively.




\noindent\textbf{Qualitative Results.}
\figref{fig:qualitative} presents samples generated by \modelname (trained on ImageNet) at 512$\times$512 and 256$\times$256 resolutions. These results highlight \modelname's ability to produce high-fidelity images with exceptional visual quality.

\begin{figure*}
    \centering
    \vspace{-6pt}
    \includegraphics[width=1\linewidth]{figures/qualitative.pdf}
    \caption{\textbf{Generated Samples.} \modelname generates high-quality images at resolutions of 512$\times$512 (1st row) and 256$\times$256 (2nd and 3rd row).
    }
    \label{fig:qualitative}
\end{figure*}

\subsection{Ablation Studies}
\label{sec:ablation}
In this section, we conduct ablation studies using \modelname-B, trained for 400 epochs to efficiently iterate on model design.

\noindent\textbf{Prediction Entity X.}
The proposed \modelname extends next-token prediction to next-X prediction. In~\tabref{tab:X}, we evaluate different designs for the prediction entity X, including an individual patch token, a cell (a group of surrounding tokens), a subsample (a non-local grouping), a scale (coarse-to-fine resolution), and an entire image.

Among these variants, cell-based \modelname achieves the best performance, with an FID of 2.48, outperforming the token-based \modelname by 1.03 FID and surpassing the second best design (scale-based \modelname) by 0.42 FID. Furthermore, even when using standard prediction entities such as tokens, subsamples, images, or scales, \modelname consistently outperforms existing methods while requiring significantly fewer parameters. These results highlight the efficiency and effectiveness of \modelname across diverse prediction entities.






\begin{table}[]
    \centering
    \scalebox{0.92}{
    \begin{tabular}{c|c|c|c|c}
        model & \makecell[c]{prediction\\entity} & \#params & FID$\downarrow$ & IS$\uparrow$\\
        \shline
        LlamaGen-L~\cite{llamagen} & \multirow{2}{*}{token} & 343M & 3.80 &248.3\\
        \modelname-B& & 172M&3.51&251.4\\
        \hline
        PAR-L~\cite{par} & \multirow{2}{*}{subsample}& 343M & 3.76 & 218.9\\
        \modelname-B&  &172M& 3.58&231.5\\
        \hline
        DiT-L/2~\cite{dit}& \multirow{2}{*}{image}& 458M&5.02&167.2 \\
         \modelname-B& & 172M&3.13&253.4 \\
        \hline
        VAR-$d16$~\cite{var} & \multirow{2}{*}{scale} & 310M&3.30 &274.4\\
        \modelname-B& &172M&2.90&262.8\\
        \hline
        \baseline{\modelname-B}& \baseline{cell} & \baseline{172M}&\baseline{2.48}&\baseline{269.2} \\
    \end{tabular}
    }
    \caption{\textbf{Ablation on Prediction Entity X.} Using cells as the prediction entity outperforms alternatives such as tokens or entire images. Additionally, under the same prediction entity, \modelname surpasses previous methods, demonstrating its effectiveness across different prediction granularities. }%
    \label{tab:X}
\end{table}

\noindent\textbf{Cell Size.}
A prediction entity cell is formed by grouping spatially adjacent $k\times k$ tokens, where a larger cell size incorporates more tokens and thus captures a broader context within a single prediction step.
For a $256\times256$ input image, the encoded continuous latent representation has a spatial resolution of $16\times16$. Given this, the image can be partitioned into an $m\times m$ grid, where each cell consists of $k\times k$ neighboring tokens. As shown in~\tabref{tab:cell}, we evaluate different cell sizes with $k \in \{1,2,4,8,16\}$, where $k=1$ represents a single token and $k=16$ corresponds to the entire image as a single entity. We observe that performance improves as $k$ increases, peaking at an FID of 2.48 when using cell size $8\times8$ (\ie, $k=8$). Beyond this, performance declines, reaching an FID of 3.13 when the entire image is treated as a single entity.
These results suggest that using cells rather than the entire image as the prediction unit allows the model to condition on previously generated context, improving confidence in predictions while maintaining both rich semantics and local details.





\begin{table}[t]
    \centering
    \scalebox{0.98}{
    \begin{tabular}{c|c|c|c}
    cell size ($k\times k$ tokens) & $m\times m$ grid & FID$\downarrow$ & IS$\uparrow$ \\
       \shline
       $1\times1$ & $16\times16$ &3.51&251.4 \\
       $2\times2$ & $8\times8$ & 3.04& 253.5\\
       $4\times4$ & $4\times4$ & 2.61&258.2 \\
       \baseline{$8\times8$} & \baseline{$2\times2$} & \baseline{2.48} & \baseline{269.2}\\
       $16\times16$ & $1\times1$ & 3.13&253.4  \\
    \end{tabular}
    }
    \caption{\textbf{Ablation on the cell size.}
    In this study, a $16\times16$ continuous latent representation is partitioned into an $m\times m$ grid, where each cell consits of $k\times k$ neighboring tokens.
    A cell size of $8\times8$ achieves the best performance, striking an optimal balance between local structure and global context.
    }
    \label{tab:cell}
\end{table}



\begin{table}[t]
    \centering
    \scalebox{0.95}{
    \begin{tabular}{c|c|c|c}
      previous cell & noise time step &  FID$\downarrow$ & IS$\uparrow$ \\
       \shline
       clean & $t_i=0, \forall i<n$& 3.45& 243.5\\
       increasing noise & $t_1<t_2<\cdots<t_{n-1}$& 2.95&258.8 \\
       decreasing noise & $t_1>t_2>\cdots>t_{n-1}$&2.78 &262.1 \\
      \baseline{random noise}  & \baseline{no constraint} &\baseline{2.48} & \baseline{269.2}\\
    \end{tabular}
    }
    \caption{
    \textbf{Ablation on Noisy Context Learning.}
    This study examines the impact of noise time steps ($t_1, \cdots, t_{n-1} \subset [0, 1]$) in previous entities ($t=0$ represents pure Gaussian noise).
    Conditioning on all clean entities (the ``clean'' variant) results in suboptimal performance.
    Imposing an order on noise time steps, either ``increasing noise'' or ``decreasing noise'', also leads to inferior results. The best performance is achieved with the "random noise" setting, where no constraints are imposed on noise time steps.
    }
    \label{tab:ncl}
\end{table}


\noindent\textbf{Noisy Context Learning.}
During training, \modelname employs Noisy Context Learning (NCL), predicting $X_n$ by conditioning on all previous noisy entities, unlike Teacher Forcing.
The noise intensity of previous entities is contorlled by noise time steps $\{t_1, \dots, t_{n-1}\} \subset [0, 1]$, where $t=0$ corresponds to pure Gaussian noise.
We analyze the impact of NCL in~\tabref{tab:ncl}.
When conditioning on all clean entities (\ie, the ``clean'' variant, where $t_i=0, \forall i<n$), which is equivalent to vanilla AR (\ie, Teacher Forcing), the suboptimal performance is obtained.
We also evaluate two constrained noise schedules: the ``increasing noise'' variant, where noise time steps increase over AR steps ($t_1<t_2< \cdots < t_{n-1}$), and the `` decreasing noise'' variant, where noise time steps decrease ($t_1>t_2> \cdots > t_{n-1}$).
While both settings improve over the ``clean'' variant, they remain inferior to our final ``random noise'' setting, where no constraints are imposed on noise time steps, leading to the best performance.




        


% !TEX root = ../main.tex

\section{Related work}
\label{sec:related_work}

\subsection{Visual unsupervised anomaly localization}

% In recent years the creation of the MVTec AD benchmark~\cite{mvtec} has given impetus to the development of new methods for visual unsupervised anomaly detection and localization. We review several main approaches which have representatives among top-5 methods on the localization track of the MVTec AD leaderboard
% The MVTec AD benchmark~\cite{mvtec}, developed in recent years, has been instrumental in propelling research towards new methods in visual unsupervised anomaly detection and localization.
In this section, we review several key approaches, each represented among the top five methods on the localization track of the MVTec AD benchmark~\cite{mvtec}, developed to stir progress in visual unsupervised anomaly detection and localization. 
% \footnote{\url{https://paperswithcode.com/sota/anomaly-detection-on-mvtec-ad}}.
% \paragraph{Synthetic anomalies} In unsupervised setting, real anomalies are either not present or not labeled in the training images. Some methods~\cite{memseg,mood_top1}, however, propose synthetic procedures that corrupt random regions in the images and train a segmentation model to predict the corrupted regions' masks.

\paragraph{Synthetic anomalies.} In unsupervised settings, real anomalies are typically absent or unlabeled in training images. To simulate anomalies, researchers synthetically corrupt random regions by replacing them with noise, random patterns from a special set~\cite{memseg}, or parts of other training images~\cite{mood_top1}. A segmentation model is trained to predict binary masks of corrupted regions, providing well-calibrated anomaly scores for individual pixels. While straightforward to train, these models may overfit to synthetic anomalies and struggle with real ones.
% . Unlabeled real anomalies in training images cannot be included in the binary masks, leading the model to predict zero scores for these regions and resulting in false negatives.

% One limitation of this approach is that the models may overfit to synthetic anomalies and generalize poorly to real anomalies. Another limitation is that training images may contain real anomalies which are unlabeled and cannot be included in the training binary masks. Thus, segmentation model is trained to predict zero scores for these regions which leads to false negatives.

% \paragraph{Reconstruction-based} Reconstruction-based methods build a generative model that takes an image $x$ as input and generates its normal (anomaly-free) version $\hat{x}$. Then anomaly scores are obtained as pixel-wise reconstruction errors between $x$ and $\hat{x}$. SotA methods from this family, e.g. DRAEM~\cite{draem}, DiffusionAD~\cite{diffusionad}, POUTA~\cite{pouta}, present a combination of reconstruction-based and synthetic-based approaches. First, they train a generative model to reconstruct synthetically corrupted image regions. Then, they train a segmentation model that takes a corrupted image and its reconstructed version as input and predicts the mask of the corrupted regions.

\paragraph{Reconstruction-based.} 
% In reconstruction-based methods, anomaly scores are obtained as reconstruction errors between the input image $x$ and generated normal (anomaly-free) counterpart $\hat{x}$.
% Reconstruction-based methods build a generative model that takes an image $x$ as input and generates its normal (anomaly-free) version $\hat{x}$. Then anomaly scores are obtained as reconstruction errors between $x$ and $\hat{x}$.
Trained solely on normal images, reconstruction-based approaches~\cite{autoencoder, vae, fanogan}, poorly reconstruct anomalous regions, allowing pixel-wise or feature-wise discrepancies to serve as anomaly scores. Later generative approaches~\cite{draem, diffusionad, pouta} integrate synthetic anomalies. The limitation stemming from anomaly-free train set assumption still persists -- if anomalous images are present, the model may learn to reconstruct anomalies as well as normal regions, undermining the ability to detect anomalies through differences between $x$ and $\hat{x}$.
% Early approaches, such as Autoencoders~\cite{autoencoder} and Variational Autoencoders~\cite{vae}, are trained solely on normal images. During inference, these models poorly reconstruct anomalous regions, allowing pixel-wise squared errors ${(x - \hat{x})^2}$ to serve as anomaly scores. Methods like f-AnoGAN~\cite{fanogan} enhance this by training W-GAN~\cite{wgan} $g$ to generate normal images and an encoder $f$ to map images to the GAN's latent space, ensuring ${\hat{x} = g(f(x)) \approx x}$. Anomalies are detected using a weighted average of reconstruction errors in pixel space and discrepancies in feature maps from GAN discriminator.

% State-of-the-art methods such as DRAEM~\cite{draem}, DiffusionAD~\cite{diffusionad}, and POUTA~\cite{pouta} integrate synthetic anomalies into the reconstruction process. They first train a generative model (autoencoder / diffusion model) to reconstruct synthetically corrupted regions. Then, they train a segmentation model that takes both the corrupted image and its reconstruction as input to predict masks of the corrupted regions.

% A major limitation of reconstruction-based methods is the assumption that the training set contains only normal images. If anomalous images are present, the generative model may learn to reconstruct anomalies as well as normal regions, undermining the ability to detect anomalies through differences between $x$ and $\hat{x}$.

% The earliest methods from this family are based on Autoencoder~\cite{autoencoder} or Variational Autoencoder~\cite{vae}, which are trained on anomaly-free images. At the inference stage, when it takes an image $x$ with anomalies it is intended to badly reconstruct the anomalous regions in $\hat{x}$, so that pixel-wise squared errors $(x - \hat{x})^2$ can be used as anomaly scores.

% Another method, f-AnoGAN~\cite{fanogan} at the first step trains W-GAN~\cite{wgan}, consisting of generator $g$ and discriminator $d$, to generate anomaly-free images $x \sim g(z)$ from latent variables $z \sim \mathcal{N}(0, I)$. Then, at the second step, it trains encoder $f$ to map anomaly-free images $x$ to the GAN's latent space, s.t. $\hat{x} = g(f(x)) \approx x$. At the inference stage, when $x$ is anomalous image, generator is assumed to generate its anomaly-free version $\hat{x}$, as it is trained only on normal images. Anomaly score are then obtained as a weighted average of reconstruction errors $(x - \hat{x})^2$ in pixel space and squared differences $(\varphi_d(x) - \varphi_d(x'))^2$ between feature maps $\varphi_d(x)$ and $\varphi_d(x')$ taken intermediate layers of GAN discriminator $d$.

% The SotA reconstruction-based methods, e.g. DRAEM~\cite{draem}, DiffusionAD~\cite{diffusionad}, POUTA~\cite{pouta}, present a combination with the approach based on synthetic anomalies. First, they train a generative model, e.g. autoencoder~\cite{draem,pouta} or diffusion model~\cite{diffusionad}, to reconstruct synthetically corrupted image regions. Then, they train a segmentation model that takes a corrupted image and its reconstructed version as input and predicts the mask of the corrupted regions.

% The main limitation of reconstruction-based methods is that they assume that training set does not contain anomalous images. Otherwise, generative model may learn to reconstruct anomalous regions as well as normal ones, which does not allow to detect anomalies by comparison of $x$ and $\hat{x}$.

\paragraph{Density-based.} Density-based methods for anomaly detection model the distribution of the training image patterns. As modeling of the joint distribution of raw pixel values is infeasible, these methods usually model the marginal or conditional distribution of pixel-wise deep feature vectors.

Some methods~\cite{ttr, pni} perform a non-parametric density estimation using memory banks. More scalable flow-based methods~\cite{fastflow,cflow,msflow}, leverage normalizing flows to assign low likelihoods to anomalies. From this family, we selected MSFlow as a representative baseline, because it is simpler than PNI, and yields similar top-5 results on the MVTec AD. 


\subsection{Medical unsupervised anomaly localization}
While there's no standard benchmark for pathology localization on CT images, MOOD~\cite{mood} offers a relevant benchmark with synthetic target anomalies. Unfortunately, at the time of preparing this work, the benchmark is closed for submissions, preventing us from evaluating our method on it. We include the top-performing method from MOOD~\cite{mood_top1} in our comparison, that relies on synthetic anomalies.

Other recognized methods for anomaly localization in medical images are reconstruction-based: variants of AE / VAE~\cite{autoencoder, dylov}, f-AnoGAN~\cite{fanogan}, and diffusion-based~\cite{latent_diffusion}. These approaches highly rely on the fact that the the training set consists of normal images only. However, it is challenging and costly to collect a large dataset of CT images of normal patients. While these methods work acceptable in the domain of 2D medical images and MRI, the capabilities of the methods have not been fully explored in a more complex CT data domain. We have adapted these methods to 3D.




% !TeX root = 0_main

\section{Threats to Validity and Limitations}
\label{sec:threats}
\textbf{Construct Validity.}
The main threats to construct validity stem from manually verifying the matching of the interactions extracted from the S2R sentences to the information on the execution model and constructing a ground truth dataset. 
To mitigate this threat, two authors independently carried out the manual verification tasks and ground truth creation, following well-defined and replicable methodologies.
More so, we computed and reported agreement levels, which are very high in all cases. 
\looseness=-1

\textbf{Internal Validity.}
Selecting the optimal prompt can be challenging for any use of GPT-4, let alone for multiple distinct tasks, and this process of finding the best prompt impacts the performance of our approach.
We selected the best prompt by evaluating 14 prompt templates, using three prompting strategies (\ie zero-shot, few-shot, and chain-of-thought) on a rich development set of bug reports from multiple applications.
\looseness=-1 

\textbf{External Validity.}
Our results are compared with the state-of-the-art, \EulerC, where we used 21 bug reports from their original dataset across six applications. We could not increase the dataset size for comparison due to difficulties in running the \EulerC tool. However, our approach is built by analyzing a dataset with bug reports from nine different applications consisting of four types of bugs. Therefore, \tool can be generalized to diverse types of bug reports.
\rev{Moreover, AstroBR currently supports the most frequently used GUI interactions in Android applications (tap, long tap, \etc). While the lack of support for certain types of interactions (\eg rotation) is a limitation, this is not due to the inherent design of the approach, and the support of these features can be added through additional engineering effort in future work.}

\textbf{Limitations.}
\tool's performance depends on the completeness of the app execution model.
The automated execution information collected with \CrashScope may result in an incomplete execution model.
To overcome this issue, we collected information from manual app executions. 

\section{Conclusions}\label{sec-conclusions}
We present Interaction-aware Conformal Prediction (ICP) to explicitly address the mutual influence between robot and humans in crowd navigation problems. We achieve interaction awareness by proposing an iterative process of robot motion planning based on human motion uncertainty and conformal prediction of the human motion dependent on the robot motion plan. Our crowd navigation simulation experiments show ICP strikes a good balance of performance among navigation efficiency, social awareness, and uncertainty quantification compared to previous works. ICP generalizes well to navigation tasks across different crowd densities, and its fast runtime and manageable memory usage indicates potential for real-world applications.

In future work, we will address infeasible robot planning solutions with adaptive failure probability and conduct real-world crowd navigation experiments to evaluate the effectiveness of ICP. As ICP is a task-agnostic algorithm, we would like to explore its applications in manipulation settings, such as collaborative manufacturing.

\begin{credits}
\subsubsection{\ackname} This work was supported by the National Science Foundation under Grant No. 2143435 and by the National Science Foundation under Grant CCF 2236484.
\end{credits}

\section*{Acknowledgments}
We thank an anonymous Prolific user whose answer to \autoref{xhw_study::question::perception} inspired our paper's title.

Work on this paper was funded by the Deutsche Forschungsgemeinschaft (DFG, German Research Foundation) under Germany's Excellence Strategy---\href{https://casa.rub.de}{EXC 2092 CASA}---390781972, through the DFG grant 389792660 as part of \href{https://perspicuous-computing.science}{TRR~248}, and by the Volkswagen Foundation grants AZ~9B830, AZ~98509, and AZ~98514 \href{https://explainable-intelligent.systems}{\enquote{Explainable Intelligent Systems}} (EIS). 

The Volkswagen Foundation and the DFG had no role in preparation, review, or approval of the manuscript; or the decision to submit the manuscript for publication. 
The authors declare no other financial interests.

\balance

\bibliographystyle{IEEEtran}
\bibliography{references}

\end{document}
