% !TeX root = 0_main

\section{Threats to Validity and Limitations}
\label{sec:threats}
\textbf{Construct Validity.}
The main threats to construct validity stem from manually verifying the matching of the interactions extracted from the S2R sentences to the information on the execution model and constructing a ground truth dataset. 
To mitigate this threat, two authors independently carried out the manual verification tasks and ground truth creation, following well-defined and replicable methodologies.
More so, we computed and reported agreement levels, which are very high in all cases. 
\looseness=-1

\textbf{Internal Validity.}
Selecting the optimal prompt can be challenging for any use of GPT-4, let alone for multiple distinct tasks, and this process of finding the best prompt impacts the performance of our approach.
We selected the best prompt by evaluating 14 prompt templates, using three prompting strategies (\ie zero-shot, few-shot, and chain-of-thought) on a rich development set of bug reports from multiple applications.
\looseness=-1 

\textbf{External Validity.}
Our results are compared with the state-of-the-art, \EulerC, where we used 21 bug reports from their original dataset across six applications. We could not increase the dataset size for comparison due to difficulties in running the \EulerC tool. However, our approach is built by analyzing a dataset with bug reports from nine different applications consisting of four types of bugs. Therefore, \tool can be generalized to diverse types of bug reports.
\rev{Moreover, AstroBR currently supports the most frequently used GUI interactions in Android applications (tap, long tap, \etc). While the lack of support for certain types of interactions (\eg rotation) is a limitation, this is not due to the inherent design of the approach, and the support of these features can be added through additional engineering effort in future work.}

\textbf{Limitations.}
\tool's performance depends on the completeness of the app execution model.
The automated execution information collected with \CrashScope may result in an incomplete execution model.
To overcome this issue, we collected information from manual app executions. 