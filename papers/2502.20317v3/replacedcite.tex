\section{Related Work}
\label{sec-relatedwork}
\textbf{Retrieval-augmented Generation (RAG):}
RAG enhances generative tasks by retrieving relevant information from external knowledge sources____ and has been widely used to improve question-answering ____. With LLMs, RAG has been used for mitigating hallucinations____, enhancing interpretability____, and enabling dynamic knowledge updates____. This work essentially leverages the idea of RAG to retrieve supporting entities from TG-KBs to contextualize answer generation. Depending on concrete types of knowledge being retrieved, existing retrievers can be categorized into structural and textual retrieval, which are reviewed next.


\noindent \textbf{Textual and Structural Retrieval:} Early textual retrieval models, such as TF-IDF and BM25____, rely on lexical similarity and keyword matching____. Modern approaches address this limitation by learning dense representations____. Beyond textual retrieval, structural retrieval leverages graph-based techniques to extract structured knowledge. Methods such as graph traversal____, community detection____, and graph machine learning models, including graph neural networks____, play a crucial role in structural retrieval. Our approach integrates the strengths of both textual and structural retrieval by infusing the mixture-of-expert philosophy into retrieval design.

Due to page limitation, a comprehensive version of the related work is attached in Appendix~\ref{app-comprehensive}.