\section{Limitations}
In this paper, we integrate a mixture of expert philosophy into retrieval design and propose a Mixture of structural-and-textual Retrieval (MoR) to adaptively retrieve textual and structural knowledge. The limitations of MoR can be categorized into two main aspects in the following:

\textbf{Lack of Domain-Specific Knowledge:} Our proposed MoR, similar to other baselines, does not exhibit significantly higher performance on PRIME than AMAZON and MAG. The reason is the lack of biomedical knowledge required to comprehend biomedical questions, extract key information, navigate relevant entities and relations, and rerank retrieved candidates. This suggests that current state-of-the-art retrieval models, even paired with LLMs' intelligence, still struggle to handle domain-specific knowledge effectively. Such limitations may extend to other specialized domains, such as finance and law. Future research could integrate domain-specific knowledge into retrieval.


\textbf{Reranking at Every Traversal Layer:} Our current MoR adaptively routes retrieved candidates into the Top-K positions at the final layer via reranking, effectively implementing a conventional Mixture of Experts (MoE) routing mechanism. Despite the state-of-the-art performance we have achieved in Table~\ref{tab-merged}, this routing mechanism could also be applied to intermediate layers, where after each retrieval step, candidates are reranked, and only Top-K proceeds to the next round of traversal and retrieval. This enables every layer of mixed traversal to emulate the router design of the MoE.


\textbf{Multi-Trajectory Reranking:} While our current Structural Reranker is designed to compute ranking scores by leveraging the full spectrum of trajectory information from multiple traversed paths ending at each candidate (as illustrated in Figure~\ref{fig-framework}), our implementation currently utilizes only the most informative trajectory (i.e., the one with the longest traversed path) due to implementation complexity. Future work should explore adaptive methods to fully integrate the complete set of traversed paths into the candidate ranking process and compare the effectiveness of leveraging traversed paths at different levels.


