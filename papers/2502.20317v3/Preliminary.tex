\begin{figure*}[t!]
    \centering
    \includegraphics[width=1\linewidth]{figures/framework.pdf}
    \vspace{-4ex}
    \caption{Our MoR framework consists of a planning module to generate a planning graph, a reasoning module to conduct mixed traversal, and an organizing module to rerank the retrieved candidates.}
    \label{fig-framework}
    \vspace{-3ex}
\end{figure*}

\section{Preliminary}\label{sec-preliminary}

\textbf{Notations:} A Text-rich Graph Knowledge Base (TG-KB) $\mathcal{B}$ generally consists of a set of connected nodes $\mathcal{V}$ in the graph with each node $v \in \mathcal{V}$ associated with its corresponding document $\mathcal{D}_v \in \mathcal{D}$ and category $\mathcal{E}_v \in \mathcal{E}$. When retrieving nodes with supporting documents from $\mathcal{B}$ for answering a given query $Q \in \mathcal{Q}$, we typically follow certain rationale encapsulating the underlying logic of that query~\cite{xu-etal-2024-adaption, xue2024enhancing}, which can be characterized by a text-attributed planning graph $G$. In many existing works~\cite{Jin2024GraphCA, wu2024stark}, this planning graph can be usually decomposed into multiple reasoning paths $G = \{\mathcal{P}_i\}_{i=1}^{|G|}$ where the $i^{\text{th}}$ reasoning path $\mathcal{P}_i = (p_{i1} \rightarrow p_{i2} \rightarrow, ..., \rightarrow p_{iL_i})$ is a distinctive reasoning chain of length $L_i$ encoding a unique logic and the $j^{\text{th}}$ node $p_{ij}$ corresponds to an entity in $\mathcal{B}$ with its own category $\mathcal{E}_{p_{ij}}$ and textual restriction $\mathcal{T}_{p_{ij}}$ extracted from the query. For example, in Figure~\ref{fig-motivation}(a), the query \textit{Publications by Point...} has a planning graph with two paths, i.e., $\mathcal{P}_1=(\text{Institution}\rightarrow\text{Author}\rightarrow\text{Paper})$ and $\mathcal{P}_2=(\text{Field-of-Study}\rightarrow\text{Paper})$, where the category and textual restriction of the first node on $\mathcal{P}_1$ are $\mathcal{E}_{p_{11}}=\text{Institution}$ and $\mathcal{T}_{p_{11}}=<\textit{Point Park Univerisity}>$, respectively. Comprehensive notations are summarized in Table~\ref{tab-symbols} in Appendix~\ref{sec-notation}. 

\textbf{Problem Setup:} With the above notations, the investigated problem here is to retrieve entities $\mathcal{C} \subseteq \mathcal{V}$ supporting answering a given query $Q$.




\textbf{Textual Retrieval} retrieves candidates based on the textual signals of both the query and documents. One common strategy is to retrieve candidates ${\widetilde{\mathcal{C}}}$ from the whole documents $\mathcal{D}$ that have Top-K textual similarity to query $Q$ measured by lexical or semantic similarity~\cite{vijaymeena2016survey}. The textual retrieval used in MoR retrieves documents for a given query by matching them with textual descriptions in the query, e.g., matching \textit{stellar populations in tidal tails} shown in Figure~\ref{fig-motivation}.


\textbf{Structural Retrieval} retrieves candidates by applying prescribed rules to structured databases such as knowledge graphs and SQL~\cite{guo2023retrieval}. Common strategies include graph-based traversal (e.g., BFS, DFS) and rule fetching~\cite{Jiang2023StructGPTAG}. Specifically, MoR conducts structural retrieval by traversing neighbors of certain categories from the generated planning graph. For example, in Figure~\ref{fig-motivation}(a), only "Paper" typed neighbors of the Author can be traversed by our structural retrieval.
