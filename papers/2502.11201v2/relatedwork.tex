\section{Related Work}
\label{sec:related_work}

Our work intersects with three key research areas: NoSQL databases, text-to-SQL systems, and natural language interfaces for data systems. In this section, we review relevant studies from each field.

\subsection{NoSQL Databases}
NoSQL databases have become a crucial component of modern data management systems, primarily due to their flexibility, scalability, and high performance. These attributes are essential for handling the large volumes of unstructured and semi-structured data commonly encountered in contemporary applications. NoSQL databases allow for efficient horizontal scaling and offer robust options for distributed data storage, making them an ideal choice for large-scale applications that require high availability and fault tolerance.

In the fields of databases and data mining, current research focuses on several key areas within NoSQL databases. These include enhancing scalability and reliability in distributed environments \cite{10.1145/1978915.1978919}, improving query capabilities for efficient large-scale data processing \cite{MAHAJAN2019120}, and ensuring data security and privacy in the context of increasing cloud deployments \cite{10.1109/TrustCom.2011.70}. Additionally, many scholars are exploring the differences between NoSQL and SQL, aiming to uncover the development potential and future directions of NoSQL technologies.

Despite the maturation of NoSQL-related technologies, challenges remain in terms of accessibility and usability, particularly for users unfamiliar with specific query languages. This paper introduces the Text-to-NoSQL task, which aims to enable users to interact with databases using intuitive NLQs. This technology not only broadens the user base but also enhances the efficiency and productivity of data management operations by bridging the gap between technical and user language capabilities.

\vspace{-5pt}
\subsection{Text-to-SQL}
Text-to-SQL aims to automatically convert NLQs into SQL queries, effectively bridging the gap between non-expert users and traditional relational database systems. This task has been extensively studied within both the database and natural language processing communities. Early research efforts relied on predefined rules or query enumeration to tackle the Text-to-SQL task \cite{10.14778/3407790.3407858,DBS-078,10.1145/3318464.3389776}, or treated it as a sequence-to-sequence problem, leveraging encoder-decoder architectures to predict the final SQL queries \cite{qi-etal-2022-rasat,popescu-etal-2022-addressing,10.5555/3304222.3304323}. With the rapid advancement of deep learning, numerous techniques such as attention mechanisms \cite{10191914}, graph representations \cite{hui-etal-2022-s2sql,10.1609/aaai.v37i11.26536,qi-etal-2022-rasat,wang-etal-2020-rat,xu-etal-2018-sql,zheng-etal-2022-hie}, and syntax parsing \cite{guo-etal-2019-towards,10.1609/aaai.v37i11.26535,scholak-etal-2021-picard,10.1145/3534678.3539305} have been employed to aid the Text-to-SQL task, resulting in significant improvements. To evaluate the performance of Text-to-SQL models in real-world scenarios, several large-scale benchmark datasets have been constructed and released, including WikiSQL \cite{zhong2017seq2sqlgeneratingstructuredqueries}, Spider \cite{yu-etal-2018-spider}, KaggleDBQA \cite{lee-etal-2021-kaggledbqa}, and BIRD \cite{NEURIPS2023_83fc8fab}. 

With the advent of large language models (LLMs), the Text-to-SQL task has entered a new era of development \cite{10.14778/3641204.3641221,NEURIPS2023_72223cc6,dong2023c3zeroshottexttosqlchatgpt,10.1145/3654930,nan-etal-2023-enhancing,ren2024purplemakinglargelanguage,10.1145/3589292}. By pre-training on massive text corpora, LLMs have developed strong natural language understanding capabilities, enabling them to accomplish various downstream tasks with input prompts. The paradigm of leveraging LLMs for Text-to-SQL tasks has emerged as the new mainstream approach. For instance, Ren \texttt{et al.} \cite{ren2024purplemakinglargelanguage} designed a more intricate pipeline involving Schema Pruning, Skeleton Prediction, and Demonstration Selection to generate SQL queries from LLM prompts. They further refined the SQL output by adjusting it to fit specific database schemas and SQL dialects and addressed hallucination issues.

Although significant progress has been made in Text-to-SQL conversion technology, research in the field of Text-to-NoSQL is still in its infancy. Moreover, due to substantial differences in database structures, query languages, and application scenarios, existing Text-to-SQL research cannot be directly applied to Text-to-NoSQL. Specifically, NoSQL databases typically feature more complex schemas, and query syntax varies significantly across different systems, posing substantial challenges for SQL-to-NoSQL conversion. As a result, achieving Text-to-NoSQL through a cascaded approach of Text-to-SQL and SQL-to-NoSQL is not only technically demanding but also resource-intensive. This conclusion has been experimentally validated in the technical report cited in the footnote of Section~\ref{sec:intro}.

\subsection{Natural Language Interfaces for Data Systems}
The application of Natural Language Interfaces to Data Systems (NLIDS) in the industry has been evolving for a considerable period. Numerous applications have demonstrated the practicality of natural language interfaces in data interaction. For instance, Salesforce Einstein integrates natural language processing, enabling users to query Customer Relationship Management (CRM) data using simple language. Amazon QuickSight allows users to pose questions about their datasets without complex query syntax. Google Cloud's Natural Language API assists businesses in classifying and understanding text data. IBM Watson Assistant provides a conversational interface, allowing users to interact with backend systems via natural language. Microsoft Power BI features a Q\&A function that translates NLQs into visual data insights. These applications are designed to enhance data accessibility for non-technical users.

In the academic realm, research on NLIDS is similarly well-explored. Studies such as WikiSQL \cite{zhong2017seq2sql} and Spider \cite{yu-etal-2018-spider} have provided large-scale datasets to evaluate models that convert NLQs into SQL queries, assisting the examination of this conversion process in real-world systems. Ni \texttt{et al.} \cite{10.1007/s10796-022-10295-0} and Staniek \texttt{et al} \cite{staniek-etal-2024-text}. have explored the conversion of NLQs into query languages for graph databases and the Overpass geographic database, respectively, further expanding the research boundaries of NLIDS. Furthermore, Song \texttt{et al.}'s work \cite{song2024speech,song2022voicequerysystem} extends NLIDS model inputs from text to audio, highlighting further innovation in the field. A survey on NLIDS reviewed the current state of this technology and the open challenges it faces \cite{zcan2020StateOT}.

However, these studies have not directly addressed research on NLIDS for NoSQL databases. Therefore, in this work, we introduce a large-scale dataset, TEND, aimed at evaluating the ability of models to translate NLQs into query languages specific to NoSQL databases. This initiative is intended to foster further research on NLIDS for NoSQL databases within the database and data mining communities. In addition, we have developed SMART, a novel system, and have conducted extensive experiments to demonstrate the feasibility of this new task.