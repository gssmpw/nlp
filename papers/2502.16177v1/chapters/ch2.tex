\section{Datasets}
A keystroke dataset is essential for effectively evaluating keystroke dynamics-based authentication algorithms. The dataset's quality, size and type (fixed or free-text) play a critical role in assessing algorithm performance. After some consideration we chose three datasets \textbf{Aalto 136M}\cite{ref:aalto-136m}, \textbf{Buffalo} \cite{ref:buffalo}, and \textbf{Nanglae-Bhattarakosol} \cite{ref:2dataset}.\\

\subsection{Aalto}

Aaalto 136M dataset includes data from 168.960 participants. It is fixed-text since each participant completed one session which consisted of typing 15 English sentences drawn from a corpus. The 15 sentences were randomly taken from a set of 1.525 sentences consisting of at least 3 words and a maximum of 70 characters. They were asked to type as fast and accurate as possible. Subjects were allowed to make typing errors, correct them or add new characters when typing, and as a result, they could type more than 70 characters. The dataset is categorized as a controlled fixed-text dataset because it involves transcribing - subjects did not type contents of their own but were shown what to type. There are around 810 keystrokes per user which translates to a total of 136.857.600 keystrokes in total.

\subsection{Buffalo}

Buffalo contains both free-text and fixed-text keystrokes from 157 participants. The data was collected in three sessions spread out over 4 months. On average there was a 28 days gap between 2 sessions to consider the temporal variations. The users had two tasks:

\begin{itemize}
    \item  The first task was copying the 2005 commencement speech of Steve Jobs at Stanford University. The speech was divided into 3 parts with each part of equal length, the subjects had to type one part in each session (fixed text);
    \item The second task consisted of multiple subtasks including answering survey questions and describing a picture, sending an email with attachment and free internet browsing (free text).
\end{itemize}

Each session lasted for about 50 minutes with 30 minutes for the first task and 20 minutes for the second task. In one section of the dataset the same keyboard was used across sessions, while in the other section subjects used different keyboards. On average there are around 17000 keystrokes for each subject. Some participant's data had to be discarded due to mistakes, so the dataset contains 148 users instead of 157.\\

\subsection{Nanglae-Bhattarakosol}

Nanglae-Bhattarakosol \cite{ref:nanglae} contains fixed-text data coming from 108 users. They entered static information (name, last name, email address, phone number) 10 times per entry using an iPhone's touchscreen.\\

\subsection{Considerations on datasets}
We chose these three different datasets because they have different characteristics: Aalto has by far the most users and the most data, however it was collected in a single session; meanwhile Buffalo has less data but it's more interesting since it was collected over three sessions. Nanglae-Bhattarakosol was chosen because of its small size.