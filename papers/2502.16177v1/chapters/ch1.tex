\section{Introduction to Keystroke Recognition}
Authentication methods follow three categories: something you \textit{know}, like passwords, something you \textit{have}, like tokens and something you \textit{are}, like biometrics \cite{ref:glasgow intro}. Biometrics can be divided into two categories, physiological biometrics and behavioral biometrics. The former refers to characteristics of the human body that do not change (face, iris, fingerprint) while the latter involves using a person's action to distinguish them.\\

Keystroke dynamics falls into the behavioral biometrics category. It works by measuring and assessing people’s typing rhythm on digital devices like computer keyboards or mobile phones\cite{ref:survey intro}.\\

\subsection{Advantages}
Keystroke Recognition has numerous advantages, like:
\begin{itemize}
    \item High acceptability and collectability. It does not require specialized hardware unlike many other biometric systems such as fingerprint recognition and iris recognition. The only physical component needed is a standard computer keyboard. This means that they can even be deployed remotely.
    \item Uniqueness. Modern computer keyboard software can measure events with high precision, meaning that it is nearly impossible to replicate another user's typing rythm.
    \item Universality. Almost everyone uses a keyboard for various activities.
    \item Low cost. Because there is no need for specialized hardware, the costs are low.
\end{itemize}

\subsection{Disadvantages}
However there are some disadvantages:
\begin{itemize}
    \item Low accuracy. This kind of biometric system is generally worse than other systems like iris recognition or face recognition.
    \item Low permanence. Since this is a behavioral biometric, it has lower permanency than physiological biometrics since typing patterns can change with time.
\end{itemize}

\subsection{Timing Features and Data Types}
The timing features of keystrokes are latency and hold time \cite{ref:air force intro}. Latency is the time between consecutive key presses/releases while hold time is the time between key press and key release. This means that between two consecutive keystrokes there are 6 timing features that can be analyzed: hold time of two keys, time between key presses, time between key releases, time between first key press and second key release, time between first key release and second key press.\\

\begin{figure}[h]
    \centering
    \includegraphics[width=0.75\linewidth]{images/image.png}
    \caption{Relevant keystroke events}
\end{figure}

These typing features are extracted from keystroke data, which can be divided into two types:
\begin{itemize}
    \item free-text, when the user can write freely. It is more complex to implement and it suffers from inconsistencies in user input. Those characteristics make it more secure and suitable for continuous authentication.
    \item fixed-text, when the user must type a predefined string. This means that comparison is easier to implement and more precise. However it has limited usability.
\end{itemize}