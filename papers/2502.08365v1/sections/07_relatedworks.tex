\section{Related Works}
\label{sec:relatedworks}

Below, we summarize the most relevant work investigating multi-agent exploration and task-agnostic exploration in single-agent scenarios.
\vspace{-9pt}
\paragraph*{Multi-Agent Exploration.}~Recently, fostering exploration in order to boost performances in (deep) MARL has gained much attention recently. A large set of works proposed to address it via reward-shaping based on many heuristics:~\citet{wang2019influence} adds a term maximizing the mutual-information between per-agent interactions;~\citet{zhang2021made} proposes to optimize the deviation from (jointly) explored regions while~\citet{zhang2023self} proposes to optimize directly the entropy over per-agent observations; more recently,~\citet{xu2024population} proposed an heuristic reward-shaping enforcing diversity between different agents, and notices that~\citet{wang2019influence} fails to address the task it introduced. Up to our knowledge, this work is the first in covering both the theoretical properties of multi-agent (task-agnostic) exploration and the optimization of single-trial objectives. Finally, we notice that a that a similar notion of Convex Markov Games was introduced in a concurrent and preliminary work~\citep{gemp2025convexmarkovgamesframework}, together with some results on existence of equilibria.
\vspace{-9pt}
\paragraph*{Task-Agnostic Exploration and Policy Optimization.}~Entropy maximization in MDPs was first introduced in~\citet{hazan2019provably} and then investigated extensively in a blossoming of different works, addressing the entropy over (trajectories of) states or even observations~\citep[to name a few][]{jin2020sample, golowich2022planning, pmlr-v202-tiapkin23a, zamboni2024limits, savas2022entropy}. Finally, the use of trust-region schemes~\cite{schulman2017trustregionpolicyoptimization} is ubiquitous in RL. We considered an importance-sampling policy gradient estimator inspired by the work of~\citet{metelli2020pois}. It is yet possible to use other forms of IS estimators, as non-parametric k-NN estimators proposed in~\citet{muttirestelli2020}.
