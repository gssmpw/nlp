\section{Sparsification of Simplicial Complexes}
\label{sec:sparse}

We assume from now on that we consider the \( k\)-spasification of the weighted simplicial complex \( \mc K \) and will sometimes omit the index \( k \); to simplify the notations, we assume that \( W_{k-1} = I \), though every statement below holds in the general case. 

The task of spectral sparsification we are concerned with thus can be generally described as follows: 
\begin{problem}
For a given weighted simplicial complex \( \mc K \) and the sensitivity level \( \eps > 0 \), find a simplicial complex \( \mc L \) such that \( \V i = \mc V_i(\mc L)\) for all \( i = 0,\dots,k\), \( \mc V_{k+1}(\mc L) \subset \V {k+1} \) with \( m_{k+1}(\mc L) \ll m_{k+1}(\mc K) \), and the corresponding up-Laplacians are spectrally \( \eps \)-close, \( \Lu k (\mc L ) \underset{\eps}{\approx} \Lu k (\mc K) \). 
\end{problem}

The seminal result by Spielman and Srivastava, \cite{spielman2008graph}, (with generalization into the simplicial case in \cite{osting2017spectral}) suggests that the sparsifier \( \mc L \) can be randomly sampled from \( \V {k+1} \) with a probability measure defined by the generalized effective resistance \( \b r \) of simplices in \( \V {k+1}\). Specifically, 

\begin{theorem}[Simplicial Sparsification, \cite{spielman2008graph, osting2017spectral}] \label{thm:sparsify}
      For any \( \eps > \frac{1}{\sqrt{m_k}} > 0 \), a \(k\)-sparse complex \( \mc L \) can be sampled as follows:
      \begin{enumerate}[leftmargin=*, label=(\arabic*)]
            \item compute the probability measure \( \b p  \) on \( \V {k+1} \) proportional to the generalized effective resistance (GER) vector \( \b r \),  \( \b p \sim W_{k+1}^2 \b r \), where \( \b r = \diag \left( B_{k+1}^\top ( \Lu k )^\dagger B_{k+1} \right) \);
            \item sample \(q(m_k)\) simplices \( \sigma_i \) from \( \V {k+1}\) according to the probability measure \( \b p  \) with  \( q ( m_k ) \ge 9 C^2 m_{k} \log ( m_{k} / \eps^2  )\), for some absolute constant  \( C>0 \);
            \item form a sparse simplicial complex \( \mc L \) with all the sampled \((k+1)\)-simplexes (and all their faces) with the weight \( \frac{ w_{k+1} (\sigma_i) }{ q(m_k) \b p(\sigma_i)  } \); weights of simplices, repeated during sampling, are accumulated.
      \end{enumerate}
      Then, with probability at least \(1/2\), the up-Laplacian of the sparsifier \( \mc L \) is spectrally \(\eps\)-close to the original one,  \( \Lu k (\mc L ) \underset{\eps}{\approx} \Lu k (\mc K) \).
\end{theorem}

The bottleneck in the sparsification by sampling described in Theorem~\ref{thm:sparsify} is the computation of the GER vector 
\(
\b r =  \diag \left( B_{k+1}^\top ( \Lu k )^\dagger B_{k+1} \right),
\)
which requires the pseudo-inverse \( ( \Lu k )^\dagger \), a highly computationally demanding operation with complexity \( \mathcal{O}( m_k^3 + 2 k m_k m_{k+1} ) \). This poses the central problem of the current work:

\begin{problem}
      Find a computationally efficient and arbitrarily close approximation of the generalized effective resistance vector \( \b r \) for a given weighted simplicial complex \( \mc K \) and simplices of fixed \( k \)-th order.
\end{problem}

In the rest of the paper, we formulate and discuss a novel method for approximating the GER vector using efficiently computable spectral densities of the up-Laplacian operators \( \Lu k \). Note that the sparsification process described in Theorem~\ref{thm:sparsify} allows for approximate sparsifying measures and exhibits low sensitivity to them, as we discuss below.

\begin{remark}
      Since \( \b p \) is a probability measure over \( \V {k+1} \), it is natural to measure its perturbations in terms of \( \frac{1}{m_{k+1}} \), i.e., 
      \(
      \left| \b p (\sigma_i) - \b p^{(\delta)}(\sigma_i) \right|< \frac{\delta}{m_{k+1}}
      \)
      (the size of the perturbation is meaningful only in relation to the actual support of the measure). As shown in Figure~\ref{fig:perturb_measure}, random perturbations of \( \b p \) can, on average, slow down the convergence of the approximately sampled complex \( \mc{L}^{(\delta)} \) to the original simplicial complex \( \mc{K} \) in terms of the number of sampled simplices. However, even for moderately high values of \( \delta \), such as \( \delta = 0.5 \), the convergence rate remains largely unaffected.
\end{remark}

\begin{figure}[t]
      \centering
      \vspace{-10pt}
      \includegraphics[width = 1.0\columnwidth]{figures/perturb.pdf}
      \caption{ Convergence of the sampled simplicial complex \( \mc L \) to the original complex \( \mc K \) at \(0\)-order in terms of the spectrum of \( L_0 \) Laplacian operator. Left pane: convergence rate vs the number of sampled edges \( q \) for various perturbed measures \( \b p^{(\delta)}\). Right pane: convergence rate for chosen values of \( \delta \) in relation to the unperturbed sparsifier. \( m_0 = 100 \), \( m_1 = \frac{m_0(m_0-1)}{3}\). All curves are averaged over \( 25 \) random perturbations for VR-complex (see Section~\ref{sec:benchmark}). \label{fig:perturb_measure}}
\end{figure}






