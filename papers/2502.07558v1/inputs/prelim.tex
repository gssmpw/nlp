



\section{Preliminaries}
\label{sec:SC}

\subsection{ Notation }
We use \( \sigma(A )\) to denote the spectrum of a symmetric operator \( A \). \( \sigma_+(A )\) denotes the strictly positive part of the spectrum.
We say that two symmetric operators \( A \) and \( B \) are semi-positive ordered \( A \succeq B \) iff \( A - B  \succeq 0 \), meaning \( A - B \) is a symmetric semi-positive definite operator.
Two operators \( A \) and \( B \) are spectrally \( \eps \)-close, \( A \underset{\eps}{\approx} B \) iff 
      \begin{equation*}
            ( 1 - \eps ) B \preceq A \preceq ( 1 + \eps ) B.     
      \end{equation*}
We use \(\odot \) to denote element-wise matrix multiplication.
Finally, for a finite set \( \mc S \),  \( | \mc S | \) corresponds to its cardinality.


\subsection{Simplicial complexes}

Classical pairwise graph models consist of two sets, \( G =\{ \mc V_0, \mc V_1 \} \), where \( \mc V_0 \) is a set of nodes and \( \mc V_1 \) is a set of edges between the nodes. Instead, one may consider the structured generalization, which includes higher-order interactions between the nodes, known as \emph{simplicial complexes}. Let us assume that \( \mc V_0 = \{ v_1, v_2, \dots v_{m_0}\}\); then each subset \( \sigma \) of \( \mc V_0 \), \( \sigma = [ v_{i_0}, v_{i_1}, \dots v_{i_k} ]\), is called a \emph{simplex} of order \( k \) (\(k\)-simplex) with its maximal proper subsets of order \( ( k - 1 ) \) known as \emph{faces} of \( \sigma \). 
\begin{definition}[Simplicial complex, \cite{Lim15}]
      A collection of simplices \( \mc K \) on the node set \( \mc V_0 \) is a \emph{simplicial complex}, if each simplex \( \sigma \) enters \( \mc K \) with all its faces. Additionally, we say that \( \mc K = \bigcup_{k=0}^{\dim K} \V k \) where \( \V k \) is a set of simplices of order \( k \) and \( \dim K \) is the maximal order of simplices in \( \mc K \). We provide a small example of a simplicial complex in Figure~\ref{fig:orientation}.
\end{definition}

Let us denote the number of \( k \)-simplices in \( \mc K\) by \( m_k\), \( m_k = | \V k | \). The sparsity of $\mc K$ at the level of \( k\)-simplices is defined by the relation between \( m_k \) and \( m_{k+1}\); we refer to it as the \( k \)-sparsity of the simplicial complex. In particular, the \( 0 \)-sparsity is given by the relation between the number of nodes \( m_0 \) and the number of edges \( m_1 \) and is what usually defines the sparsity of the graph. Note that the \( k \)-sparsity is not on its own indicative of the \( ( k + 1 ) \)-sparsity as the two depend on the intrinsic topology of the simplicial complex. Consequently, if one aims to find a similar but sparser simplicial complex, it is natural to consider such a problem for a fixed \( k \) rather than attempting to define a unified notion of sparsifier across all simplex orders.

We now formalize the concept of closeness between simplicial complexes, aiming to identify a simplicial complex \(\mc L \) that is \(k\)-sparser than \( \mc K \) while still preserving target properties and structures of the original $\mc K$. 
Following \cite{spielman2008graph, osting2017spectral}, we use the notion of sparsification via the spectral closeness of higher-order Laplacian operators \( L_k \)
that naturally describe the topology of the underlying simplicial complex. For this purpose, the operators \( L_k \) are formally defined below.






\subsection{Laplacian Operators and Topology}

Since each simplex \( \sigma \) enters \( \mc K \) with all its faces, there exists a map matching it to its boundary formed by its faces.

\begin{figure}[t]
      \centering
      \begin{tikzpicture}
            \begin{scope}[shift={(-0.75, 0)}]
            \draw[fill = rwth-blue, opacity = 0.4] (0,0) -- (1.5,0) -- (0.75, -1.5*3/4) -- cycle; 
            \draw[fill = rwth-blue, opacity = 0.6] (0,0) -- (1.5,0) -- (0.75, 1.5*3/4) -- cycle; 

            \Vertex[x=0, y=0, label = 1, style={color = rwth-magenta}, fontcolor = white, size = 0.4 ]{v1}
            \Vertex[x=1.5, y=0, label = 3, style={color = rwth-magenta}, fontcolor = white, size = 0.4 ]{v2}
            \Vertex[x=0.75, y=-1.5*3/4, label = 2, style={color = rwth-magenta}, fontcolor = white, size = 0.4 ]{v3}
            \Vertex[x=0.75, y=1.5*3/4, label = 4, style={color = rwth-magenta}, fontcolor = white, size = 0.4 ]{v4}
            \Vertex[x=2.25, y=1.5*3/4, label = 5, style={color = rwth-magenta}, fontcolor = white, size = 0.4 ]{v5}
            \Edge[Direct](v1)(v2)
            \Edge[Direct](v1)(v3)
            \Edge[Direct](v1)(v4)
            \Edge[Direct](v3)(v2)
            \Edge[Direct](v2)(v4)
            \Edge[Direct](v3)(v5)
            \Edge[Direct](v4)(v5)
            \node at (0.75, 1.5*3/4*1/3 ) { \AxisRotator[rotate=0] };
            \node at (0.75, -1.5*3/4*1/3 ) { \AxisRotator[rotate=-60] };
            \end{scope}

            \begin{scope}[shift = {(2.5, -0.5 )}]
                  \node[anchor=north west,align=left,] at ( 0, 1.5*3/4 ) { \( \textcolor{rwth-magenta}{\mc V_0(\mc K)} : [1], [2], [3], [4], [5] \) };
                  \node[anchor=north west,align=left,] at ( 0, 0.9*3/4 ) { \( 
                  \begin{aligned}
                        \mc V_1(\mc K) : & \; [1, 2], [1, 3], [1, 4], \\[-2pt]
                        & \; [ 2, 3], [3, 4 ], [3, 5], [4, 5 ]  
                  \end{aligned}
                  \) };
                  \node[anchor=north west,align=left,] at ( 0, -.2*3/4 ) { \( 
                  \begin{aligned}
                        \textcolor{rwth-blue}{\mc V_2(\mc K)} : [ 1, 2, 3], [ 1, 3, 4]
                  \end{aligned}
                  \) };
            \end{scope}

            \begin{scope}[shift={(-2.75, -3.25)}]
            \Vertex[x=2.5, y=1.5*3/4, label = 1, style={color = rwth-magenta}, fontcolor = white, size = 0.3 ]{t1}
            \Vertex[x=3.5, y=1.5*3/4, label = 2, style={color = rwth-magenta}, fontcolor = white, size = 0.3 ]{t2}
            \Vertex[x=2.5, y=1.1*3/4, label = 1, style={color = rwth-magenta}, fontcolor = white, size = 0.3 ]{t3}
            \Vertex[x=3.5, y=1.1*3/4, label = 3, style={color = rwth-magenta}, fontcolor = white, size = 0.3 ]{t4}
            \Vertex[x=2.5, y=0.7*3/4, label = 1, style={color = rwth-magenta}, fontcolor = white, size = 0.3 ]{t5}
            \Vertex[x=3.5, y=0.7*3/4, label = 4, style={color = rwth-magenta}, fontcolor = white, size = 0.3 ]{t6}
            \Vertex[x=2.5, y=0.3*3/4, label = 2, style={color = rwth-magenta}, fontcolor = white, size = 0.3 ]{t7}
            \Vertex[x=3.5, y=0.3*3/4, label = 3, style={color = rwth-magenta}, fontcolor = white, size = 0.3 ]{t8}
            \Vertex[x=2.5, y=-0.1*3/4, label = 3, style={color = rwth-magenta}, fontcolor = white, size = 0.3 ]{t9}
            \Vertex[x=3.5, y=-0.1*3/4, label = 4, style={color = rwth-magenta}, fontcolor = white, size = 0.3 ]{t10}
            \Vertex[x=2.5, y=-0.5*3/4, label = 3, style={color = rwth-magenta}, fontcolor = white, size = 0.3 ]{t11}
            \Vertex[x=3.5, y=-0.5*3/4, label = 5, style={color = rwth-magenta}, fontcolor = white, size = 0.3 ]{t12}
            \Vertex[x=2.5, y=-0.9*3/4, label = 4, style={color = rwth-magenta}, fontcolor = white, size = 0.3 ]{t13}
            \Vertex[x=3.5, y=-0.9*3/4, label = 5, style={color = rwth-magenta}, fontcolor = white, size = 0.3 ]{t14}
            \Edge[Direct](t1)(t2)
            \Edge[Direct](t3)(t4)
            \Edge[Direct](t5)(t6)
            \Edge[Direct](t7)(t8)
            \Edge[Direct](t9)(t10)
            \Edge[Direct](t11)(t12)
            \Edge[Direct](t13)(t14)

            \draw[->, line width = 1.0] (2.1, 1.9*3/4)--(2.1, -1.5*3/4);
            \draw[->, line width = 1.0] (2.1, 1.9*3/4)--(3.85, 1.9*3/4);
            \node[ anchor=south ] at ( 3., 1.9*3/4 ) { \small orientation };
            \node[ anchor = south, rotate = 90 ] at (2.1, 0.2*3/4 ) { \small ordering };

            \end{scope}

            \begin{scope}[shift = {(1,-0.5)}]
            \node[] at ( 3.25, -2.0 ) { \( B_2 \textcolor{rwth-blue}{[1, 2, 3 ]} = \overbrace{\textcolor{rwth-red}{(+1)} [1, 2]}^{\substack{\text{1st in}\\\text{order}}} + (-1) [1, 3] + (+1) [2, 3] \) };
            \node[] at ( 3.25, -3.4 ) { \( B_2 \textcolor{rwth-blue}{[1, 3, 4 ]} = \underbrace{\textcolor{rwth-red}{(+1)} [1, 3]}_{\substack{\text{1st in}\\\text{order}}} + (-1) [1, 4] + (+1) [3, 4] \) };  
            \end{scope}
      \end{tikzpicture}
      \caption{ Example of a simplicial complex with ordering and orientation: nodes from \( \V 0 \) in magenta, triangles from \( \V 2 \) in blue. Orientation of edges and triangles is shown by arrows; the action of the \( B_2 \) operator is given for both triangles. Adapted from \cite{savostianov2024cholesky}. \label{fig:orientation}}
\end{figure}


In order to formally define such a map, assume that simplices in \( \V k \) have a fixed ordering and chosen orientations as a matter of bookkeeping (e.g. lexicographically). Then, let us consider linear spaces \( \mc C_k\) of formal sums of simplices in \( \V k \); i.e. \( \mc C_0 \) is the space of node states, \( \mc C_1 \) the space of edge flows, and so on; note that each such space is isomorphic to $\ds R^{m_k}$, \( \mc C_k \cong \ds R^{m_k} \),  and that it can be viewed as the space of simplicial flows on \( \V k\). Then, the action of the boundary map \( B_k\) on \( \sigma = [ v_{i_0}, v_{i_1}, \dots v_{i_k} ] \in \V k \) is defined as an alternating sum:
\begin{equation*}
      B_k : \mc C_k \mapsto \mc C_{k-1}, \quad B_k \sigma = \sum_{j=0}^{k} (-1)^j \sigma_{\bar j }
\end{equation*}
where \( \sigma_{\bar j } \) denotes the face of \( \sigma \) that does not include \( v_{i_j}\). Given a fixed order and orientation, simplices in \( \V k \) and \( \V {k-1}\) form canonical bases in \( \mc C_k\) and \( \mc C_{k-1}\) respectively, thus allowing us to use a matrix representation of the boundary operators. For simplicity, from now on, we will always assume this is the case. With a slight abuse of notation, will use the symbol \( B_k \) to denote the matrix representation of the boundary operator. An example for a simplicial complex and the action of the boundary operator is provided in Figure~\ref{fig:orientation}. 


The alternating sum in the definition of boundary operators \( B_k \) upholds the fundamental lemma of homology (``a boundary of a boundary is zero''), \( B_k B_{ k + 1 } = 0 \), and thus induces the topologically sound decomposition of simplicial flows known as \emph{Hodge decompositon}:
\begin{equation}
      \label{eq:hodge_decomposition}
      \ds R^{m_k} = \lefteqn{\overbrace{\phantom{\im B_k^\top \oplus  \ker \left( B_k^\top B_k + B_{k+1} B_{k+1}^\top \right)}}^{\ker B_{k+1}^\top}} \im B_k^\top \oplus
      \underbrace{\ker \left( B_k^\top B_k + B_{k+1} B_{k+1}^\top \right) \oplus  \im B_{k+1}}_{\ker B_k}.          
\end{equation}


\begin{definition}[Hodge Laplacian operator]
      The operator \( L_k = B_k^\top B_k + B_{k+1} B_{k+1}^\top \) is known as the \emph{Hodge} or \emph{higher-order} Laplacian operator and has the following properties: 
      \begin{enumerate}
            \item the elements of \( \ker L_k \) in the Hodge decomposition correspond to the \( k \)-dimensional holes in \( \mc K \) (connected components for \( k = 0 \), 1-dimensional holes for \( k = 1\), and so on);
            \item the first term \( \Ld k = B_k^\top B_k \) is known as the \emph{down-Laplacian} and describes the relation between \( k\)- and \( (k-1)\)-simplices. For \( k = 1 \), \( \im \Ld k = \im B_k^\top \) contains so-called gradient flows on the edges;
            \item the second term \( \Lu k = B_{k+1} B_{k+1}^\top \) is known as the \emph{up-Laplacian} and describes the relation between \( k\)- and \( (k+1)\)-simplices. For \( k = 1 \), \( \im \Lu k = \im B_{k+1} \) contains so-called curl flows.
      \end{enumerate}
\end{definition}


Higher-order Laplacian operators \( L_k \), as well as their down- and up-terms, are frequently used to describe dynamical processes and random walks on simplicial complexes, \cite{schaub2019random}, to inject simplicial structure into graph neural networks, \cite{ebli2020simplicial}. Moreover, the eigenvectors of Hodge Laplacians can be used for spectral clustering, \cite{grande2024node,ebli2019notion}, or to characterize the stability of topological features, \cite{guglielmi2023quantifying}.

Note that one can generalize boundary and Laplacian operators to the weighted case without significant difficulty. Let \( W_k \) be a diagonal matrix such that its \(i\)-th diagonal entry contains the weight of the \(i\)-th simplex in \( \V k \), \( [W_k]_{ii} = \sqrt{w_k(\sigma_i)}\). Then,  \(W_{k-1}^{-1} B_k W_k \) provides a weighted version of $B_k$ that preserves all the fundamental topological features of $B_k$, and the same is true for the corresponding weighted higher-order Laplacian~\cite{guglielmi2023quantifying}.

By its definition, the up-Laplacian \( \Lu{k} \) describes the relationship between simplicies in \( \V{k} \) and \( \V{k+1} \). At the same time, its spectrum encodes information about the overall topology of the simplicial complex. As a result, one may use the operator \( \Lu{k} \) to measure the closeness between simplicial complexes for the task of \( k \)-sparsification, as we describe below.

