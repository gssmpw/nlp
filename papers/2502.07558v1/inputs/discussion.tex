\section{ Discussion }
\label{sec:discussion}

In this work we have proposed a fast method of approximating generalized effective resistance vector for simplices of an arbitrary order \( k \) inside simplicial complex \( \mc K \) with the algorithmic complexity \( \mc O \left( \delta^{-3} m_k^{ 1 + \frac{4}{k+1}} \right)\), Theorem~\ref{thm:error}, allowing for efficient \( k\)-sparsification of \( \mc K \) through subsampling, Theorem~\ref{thm:sparsify}. The novel approach is based on the connection between GER vector \( \b r \) and the family of local density of states of the corresponding higher-order down-Laplacian operator \( \Ld {k+1}\), Theorem~\ref{thm:GER_DOS}. We avoid the pathologic behaviour of the pre-existing KPM approximation methods for LDoS by suggesting a kernel-ignoring decomposition de facto decomposing symmetrized spectral densities via odd Chebyshev polynomials. Developed approach follows the theoretical estimates, Theorem~\ref{thm:error} and Equation~\eqref{eq:err1}, for the approximation error which allows us to choose the number of moments \( M \) and number of Monte Carlo vectors \( N_z \) controlling the final approximation error which is shown to be sufficient at a moderate level for a close-to-efficient sparsification, Figure~\ref{fig:perturb_measure}. Given the fact that the developed method is only dependent on \( \Lu k \) being a Hodge Laplacian, it can also be directly applied to cell complexes.

Given the sufficient improvement of the asymptotic of the computational complexity of the method for \(k \ge 1 \), the introduction of the sparsified complex in label spreading, spectral clustering or generic simplicial complex GNN tasks may sufficiently decrease priorly prohibitive computational costs, \cite{yang2022simplicial,ebli2019notion}; additionally, the cost of trajectory classification as well as landmark detection algorithms can be directly scaled down by transitioning to a sparser, but spectrally similar model, \cite{grande2024topological}. Separately, one may notice that the existence of the efficient sparsifier \( \mc L \) effectively breaches the gap between dense complexes and ones for which one can obtain a preconditioner through the collapsible subcomplex, \cite{savostianov2024cholesky}, resulting into an efficient solver for linear systems \( \Lu k \b x = \b b \) for arbitrary simplicial complex. Additionally, the developed method admits graph sparsification as a special case; at the same time, the computational complexity for \( k = 0 \) would clearly exceed the complexity of preexisting graph sparsification algorthims. Note that we do not suggest the application of KID-approximation for the case of the classical graph, since it is not the focus of the current work.

Finally, one may note, that the approximation quality given in Theorem~\ref{thm:error} may still be improved in the pathological scenarios such as over-represented motifs in the graph (see \cite{dong2019network}) through similar filtration technique. Whilst we avoid explicitly defining a generalized motif for simplices of the arbitrary order and whether their detrimental effect of LDoS is generic or non-generic for simplicial complexes, it is promising future venue of research as well as other applications of KID-approximated LDoS for simplicial complexes.