\section{Introduction}

Graph models for relational data are ubiquitous, enabling the integration of structural information into various tasks such as link prediction, node importance ranking, label propagation, and machine learning algorithms, with numerous applications across disciplines. At the same time, such models are inherently restricted to pairwise interactions between agents, \cite{benson2016higher,battiston2020networks}. Many natural systems include polyadic interactions, such as chemical reactions, co-authorship networks, and social connections. To address this limitation, higher-order models of relational data have been introduced, including hypergraphs, motifs, cell and simplicial complexes. Such higher-order models have been shown to influence a variety of processes, such as promoting synchronization \cite{gambuzza2021stability}, impacting label spreading and critical node identification, \cite{tudisco2021nonlinear,prokopchik2022nonlinear}, and supporting higher-order random walks and trajectory classification \cite{schaub2019random}.

However, as the size of a system grows, the number of interactions scales accordingly, rendering most existing network-based algorithms computationally prohibitive. The tractability of the system and the memory consumption are inevitably affected in the same way; the transition to higher-order models only accentuates the problem. As a result, it is natural to posit the question of sparsification: for a given high-order model \( \mc K \), can one find a model \( \mc L \) with similar key properties (such as similar topology or comparable rates of information propagation) but with asymptotically significantly fewer order-\( k\) interactions?

This idea of the sparsification is closely related to the Lottery Ticket theorem for the graph neural network (or simplicial complex neural networks in particular, \cite{ebli2020simplicial, yang2022simplicial}): the transition to a sparser, but similar simplicial complex inside the convolutional layer implies pre-training pruning of connections of each message passing filter.

In the current work, we consider the task of efficient sparsification of simplicial complexes at the level of simplices of order \( k \). 
For this, the spectral information of the induced higher-order Laplacian operators \( L_k \) is critical, as it describes the topology of the complex, governs its stability, may be used to define clusters, and affects the performance of many numerical methods for the analysis of simplicial complexes; consequently, we are computing sparsifiers that maintain the spectrum or specific spectral properties of the original operator \( L_k \) (note that the spectral properties of all lower order Laplacians are unaffected by the sparsification). { To be more specific, each \(L_k\) operator is composed of down- and up-Laplacian terms, \(L_k = \Ld k + \Lu k\); since \( \Lu k \) desribes up-adjacency between simplices of orders \( k \) and \( k+1\), one uses its spectrum to control the similarity of the sparsifier.}
The result by Spielman and Srivastava, \cite{spielman2008graph, spielman2011spectral} and its more recent generalization to simplicial complexes \cite{osting2017spectral} state that for any simplicial complex \( \mathcal K \) one can find a spectrally close sparser complex \( \mathcal L \) with \( m_{k+1}(\mathcal L ) = \mathcal O ( m_{k} (\mathcal K) \log m_{k}(\mathcal K) ) \) where \( m_k(\mathcal K)\) denotes the number of simplices of order $k$ in the simplicial complex \(\mathcal K\). The sparsifier \( \mathcal L \) can be sampled from the original complex \( \mathcal K \) according to a sampling probability proportional to the so-called generalized effective resistance (GER) vector \( \mathbf r \).

Although a variety of methods for the efficient computation of the GER vector have been proposed for the classical graph case \( k = 0 \), \cite{cohen2014solving, spielman2014nearly, kelner2013simple}, this remains a major computational bottleneck for higher-order cases. In this work, we show that GER vectors can be directly computed using functional descriptors of the spectral information known as the network's local densities of states (LDoS), introduced in \cite{dong2019network}. 
Due to the Hodge decomposition, \cite{Lim15,schaub2019random}, high-order up-Laplacian operators \( \Lu k \) have high-dimensional kernels which are detrimental for existing quasi-polynomial methods to approximate LDoS.
To resolve this issue, we suggest a novel method based on a kernel-ignoring decomposition. Additionally, we provide error estimates, which allows us to guarantee the method's advantageous computational complexity \( \mathcal O \left( \delta^{-3} {m_{k+1}^4}{m_k^{-3}} \right)\) where \( \delta \) controls the approximation error.
The performance of the developed method is illustrated on the family of Vietoris--Rips simplicial complexes, \cite{hausmann1994vietoris}, for various density levels and orders of simplices.







\paragraph{Main Contributions } 
Our main contributions are as follows: \textbf{(i)} We show that the sparsification measure used for simplicial complexes is directly related to the local densities of states of a higher-order down-Laplacian {\( \Ld {k+1} \)(whose spectrum is inherited from \( \Lu k \), \cite{guglielmi2023quantifying})}.
This measure has previously been defined in terms of the full spectrum of the up-Laplacian; the transition to LDoS enables an efficient approximation.
\textbf{(ii)} We develop a novel kernel-ignoring decomposition (KID) for the efficient approximation of LDoS; the suggested method avoids pathological spectral structures of the up-Laplacian, which prevented a successful application of the preexisting methods.
\textbf{(iii)} Finally, we prove that our method outperforms previous approaches in terms of algorithmic complexity, bounding the number of parameters required to obtain a desired approximation error. 



\subsection*{ Related work }
 
\paragraph{Simplicial complexes.}

Simplicial complexes generalize models for relational data to higher-order interactions, aligning with the intrinsic topology of the data \cite{quillen2006homotopical, munkres2018elements}. The corresponding higher-order Hodge Laplacian operators \( L_k \) define homology groups which, for example, describe \( k \)-dimensional holes in the complex \cite{Lim15}. In the thermodynamic limit, key topological features are preserved, as \( L_k \) operators on simplicial complex induced by point clouds on manifolds converge to their continuous counterparts \cite{chen2021decomposition,chen2021helmholtzian}. The spectral information of Hodge Laplacians has diverse applications: it can be used to determine topological stability of the simplicial complex \cite{guglielmi2023quantifying}, 
% identify turning points in configuration models, REF FUTURE HEHE {\color{red}FT: add something}, 
be exploited for spectral clustering, \cite{ebli2019notion,grande2024node}, and trajectory classification through simplicial random walks, \cite{schaub2019random,grande2024topological}. Moreover, \( L_k \) can act as structural shift operators for signal processing, \cite{barbarossa2020topological}, and, thus, injected into graph neural networks to model higher-order interactions, \cite{ebli2020simplicial,roddenberry2021principled,yang2022simplicial}. 

\paragraph{Sparsification.}

Numerous graph sparsification algorithms have been proposed over the years, each aiming to preserve specific properties of the original system, such as cut costs, \cite{benczur1996approximating,ahn2012graph}, clustering, \cite{satuluri2011local}, or classification scores, \cite{li2022graph}. Due to the prevalence of the graph spectral methods, \cite{spielman2008graph,spielman2011spectral} introduced the concept of spectral sparsification, which preserves the spectrum of the corresponding graph Laplacian operator by sampling edges based on their generalized effective resistance, \cite{tetali1991random}. This idea was later extended to simplicial complexes in \cite{osting2017spectral}. Several methods have been proposed in the classical graph case to efficiently compute (or approximate) the effective resistance, which is inherently computationally prohibitive as it requires access to the entries of the pseudo-inverse of the Laplacian \cite{spielman2014nearly, cohen2014solving}. This computational challenge directly links the graph and simplicial sparsification problems to efficient solvers for linear systems involving \( L_k \) operators. Unlike the graph case, available approaches for higher-order Laplacians\cite{savostianov2024cholesky,cohen2014solving2,kyng2016approximate} are either uniquely applicable for sparse simplicial complexes or their performance suffers in the densest cases. 


\paragraph{Network density of states.}

The notion of spectral densities as functional descriptors of the spectral information for various operators is well-established across various areas of physics, \cite{weisse2006kernel}. It was first introduced in the context of network analysis in \cite{dong2019network} where it was observed the presence of pathological structures in the spectrum of the graph Laplacian associated with over-represented motifs in the graph. These structures create potential complications that hinder the application of the kernel polynomial method \cite{weisse2006kernel}, the algorithm of choice when aiming to approximate the spectral density.




\paragraph*{Outline.} The rest of the paper is organized as follows: Section~\ref{sec:SC} provides a brief introduction to simplicial complexes. In Section~\ref{sec:sparse}, we present our main sparsification result and its connection to the local spectral density of states, used as efficiently computable proxies for spectral information. Section~\ref{sec:KID} outlines the proposed novel approach for efficiently computing the sparsifying probability measure. Finally, numerical experiments are presented in Section~\ref{sec:benchmark}, followed by concluding remarks in Section~\ref{sec:discussion}.

