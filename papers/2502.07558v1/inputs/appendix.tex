

%\section{Simplicial complexes}



\section{Proof of Theorem~\ref{thm:GER_DOS} }

\begin{proof}
      Let \( B_{k+1} W_{k+1} = U S V^\top\) where \( S \) is diagonal and invertible and both \( U \) and \( V \) are orthogonal (so it is a truncated SVD decomposition of \( B_{k+1} W_{k+1} \) matrix with eliminated obsolete kernel). Then:
      \begin{equation}
            ( \Lu k )^\dagger = \left( B_{k+1} W_{k+1}^2 B_{k+1}^\top \right)^\dagger = \left( U S^2 U^\top \right)^\dagger = U S^{-2} U^\top
      \end{equation}
      \begin{equation}
            \begin{aligned}
                  \b r & = \diag \left( W_{k+1} B_{k+1}^\top ( \Lu k )^\dagger B_{k+1} W_{k+1} \right)  =  \\
                  & = \diag \left(  V S U^\top U S^{-2} U^\top U S V^\top \right) = \diag \left(  V V^\top \right)
            \end{aligned}
      \end{equation}
      As a result, \( \b r_i = \| V_{ i \cdot } \|^2 = \sum_j | v_{ij} |^2  \), so the \(i\)-th entry of the resistance is defined by the sum of square of \(i\)-th components of eigenvectors \( \b v_j \) of \( \Ld {k+1} = W_{k+1} B_{k+1}^\top B_{k+1} W_{k+1} \) operator where \( \b v_j \perp \ker \Ld {k+1} \).

      Note that 
      \begin{equation}
            \mu_i ( \lambda \mid \Ld {k+1} ) = \sum_{j=1}^{m_{k+1}} \left| \b e_i^\top \b q_j \right|^2 \bm\delta \left( \lambda - \lambda_j \right)  = \sum_{j=1}^{m_{k+1}} \left| q_{ij} \right|^2 \bm\delta \left( \lambda - \lambda_j \right) 
      \end{equation}
      so 
      \begin{equation}
            \begin{aligned}
                  \b r_i & = \| V_{ i \cdot } \|^2 = \sum_j | v_{ij} |^2 = \int_{ \ds R \backslash \{ 0 \} }  \sum_{j=1}^{m_{k+1}} \left| q_{ij} \right|^2 \bm\delta \left( \lambda - \lambda_j \right)  d\lambda =\\
                  & = \int_{ \ds R \backslash \{ 0 \} } \mu_i ( \lambda \mid \Ld {k+1} ) d\lambda = \int_{\ds R } (1 - \ds 1_0(\lambda)) \mu_i ( \lambda \mid \Ld {k+1} ) d\lambda
            \end{aligned}
      \end{equation}
\end{proof}


\section{Derivation for Equation~\ref{eq:diagonal}}\label{app:eq11}

Let \( \mu_j ( x \mid A )  \) be a general case for LDoS; then for an arbitrary polynomial function \( f(x) \) and \( A = Q \Lambda Q^\top \),
\begin{equation}
   \begin{aligned}
         \left\langle  \mu_j ( x \mid A ), f(x) \right\rangle & = \int_{-\infty}^{+\infty}  \bm\delta \left( x - \lambda_i \right) f(x) dx = \\
         & = \sum_{i=1}^{n} \left| \b e_j^\top \b q_i \right|^2 \int_{-\infty}^{+\infty}  \bm\delta \left( x - \lambda_i \right) f(x) dx = \\
         & = \sum_{i=1}^{n} \left| \b e_j^\top \b q_i \right|^2 f(\lambda_i) =  \sum_{i=1}^{n} \left| q_{ji} \right|^2 f(\lambda_i) 
   \end{aligned}
\end{equation}
Since \( f(A) = Q f(\Lambda) Q^\top \), we get
\begin{equation}
      \left[ f(A) \right]_{jj} = \b e_j^\top Q f(\Lambda) Q^\top \b e_j = \b q_j^\top f(\Lambda) \b q_j = \sum_{i=1}^{n} \left| q_{ji} \right|^2 f(\lambda_i)  
\end{equation}
 
The case of symmetrized \( \tilde \mu_j ( x \mid H ) \) is marginally different; indeed, since \( T_m(x) \) is odd and \( T_m(0) = 0 \):
\begin{equation}
      \begin{aligned}
            \left\langle  \tilde \mu_j ( x \mid H ), T_m(x) \right\rangle & = \sum_{\substack{i=1 \\ \lambda_i \ne 0 }}^{n} \left| q_{ji} \right|^2 \left( T_m(\lambda_i) - T_m(-\lambda_i) \right) = \\
            & = 2 \sum_{\substack{i=1 \\ \lambda_i \ne 0 }}^{n} \left| q_{ji} \right|^2 T_m(\lambda_i) = 2 \sum_{i=1}^{n} \left| q_{ji} \right|^2 T_m(\lambda_i) 
      \end{aligned}
\end{equation}
So 
\begin{equation} 
      d_{mj} = 2 \sum_{i=1}^{n} \left| q_{ji} \right|^2 T_m(\lambda_i) = 2 \left[ T_m(H) \right]_{jj}
\end{equation}


\section{Proof of Theorem~\ref{thm:error}}

Before showing the result directly, we briefly show an auxiliary fact:
\begin{lemma}
  For a given simplicial complex \( \mc K \) and GER vector \( \b r \) it holds that \( \| \b r \|_1 = m_{k} - \sum_{i=-1}^{k-1} (-1)^{k-1-i} ( m_i - \beta_{i+1} ) \) where \( \beta_k = \dim \ker L_k \) denotes \( k\)-th Betti's number and \( m_{-1} = 0 \).
\end{lemma}
\begin{proof}
      Note that by the proof of Theorem~\ref{thm:GER_DOS}, \( \b r_i \ge 0 \), and
      \begin{equation}
            \| \b r \|_1 = \sum_i \b r_i = \sum_{i, j} | v_{ij} |^2
      \end{equation}
      Since each of the right singular vectors \( V_{ \cdot j} \) of \( B_{k+1} W_{k+1}\) has unit length, then \( \sum_{i, j} | v_{ij} |^2 = \text{number of columns of } V = m_k - \dim \ker \Lu k \)  given the truncated SVD. Given Hodge decomposition \eqref{eq:hodge_decomposition}, \( \dim \ker \Lu k = \dim \im B_k^\top + \dim \ker L_k = \sum_{i=-1}^{k-1} (-1)^{k-1-i} ( m_i - \beta_{i+1} )\) due to the spectral inheritance principle for Hodge Laplacians, \cite{guglielmi2023quantifying}, yielding the result. 
\end{proof}

Finally, to show the estimate for the approximation of the sparsifying norm, we consider the estimate 
\begin{equation}
      \ds E \left\| h^{(j)} - \wh h^{(j)}_M \right\|_\infty \le \frac{1}{\Delta h} \left( \frac{6L}{M} + \frac{2 \| K_{\Delta h} \|_\infty}{\pi \sqrt{N_z}} \right)
\end{equation}
which verbatim translates to the estimate on GER vector \( \b r\). Given each \( \b r_i \ge  0\), the measure \( \b p = \frac{1}{\| \b r \|_1} \b r \) with its approximation quality measured in \( 1/m_{k+1}\). As a result, to get \( \| \b p - \wh{ \b p} \|_\infty \le \delta \), it is sufficient to show:
\begin{equation}
      \frac{6L}{M} + \frac{2 \| K_{\Delta h} \|_\infty}{\pi \sqrt{N_z}} \le \frac{\delta}{m_{k+1}} \| \b r \|_1 
\end{equation}
by choosing \( M \) and \( N_z \) such that 
\begin{equation}
      \frac{6L}{M} \le \frac{\delta}{2m_{k+1}} \| \b r \|_1 , \qquad \frac{2 \| K_{\Delta h} \|_\infty}{\pi \sqrt{N_z}} \le \frac{\delta}{2m_{k+1}}\| \b r \|_1
\end{equation}
assuming that for a sufficiently dense (namely, \( \frac{m_k}{2} \ge m_{k-1} + \beta_k \) ) simplicial complex \( \| \b r \|_1 = m_{k} -\sum_{i=-1}^{k-1} (-1)^{k-1-i} ( m_i - \beta_{i+1} ) \ge \frac{1}{2}m_{k}\).


\begin{comment}
\section{ Proof of the bound for numbers of simplices }

\begin{lemma}
 \( m_{k+1} \le m_k^{ 1 + \frac{1}{k+1} }\)
  \end{lemma}

  \begin{proof}
      Oh no \dots
  \end{proof}
\end{comment}




%%
%% If your work has an appendix, this is the place to put it.

