\begin{figure}[t]
      \centering
      \begin{tikzpicture}
            \begin{scope}[shift={(-0.75, 0)}]
            \draw[fill = rwth-blue, opacity = 0.4] (0,0) -- (1.5,0) -- (0.75, -1.5*3/4) -- cycle; 
            \draw[fill = rwth-blue, opacity = 0.6] (0,0) -- (1.5,0) -- (0.75, 1.5*3/4) -- cycle; 

            \Vertex[x=0, y=0, label = 1, style={color = rwth-magenta}, fontcolor = white, size = 0.4 ]{v1}
            \Vertex[x=1.5, y=0, label = 3, style={color = rwth-magenta}, fontcolor = white, size = 0.4 ]{v2}
            \Vertex[x=0.75, y=-1.5*3/4, label = 2, style={color = rwth-magenta}, fontcolor = white, size = 0.4 ]{v3}
            \Vertex[x=0.75, y=1.5*3/4, label = 4, style={color = rwth-magenta}, fontcolor = white, size = 0.4 ]{v4}
            \Vertex[x=2.25, y=1.5*3/4, label = 5, style={color = rwth-magenta}, fontcolor = white, size = 0.4 ]{v5}
            \Edge[Direct](v1)(v2)
            \Edge[Direct](v1)(v3)
            \Edge[Direct](v1)(v4)
            \Edge[Direct](v3)(v2)
            \Edge[Direct](v2)(v4)
            \Edge[Direct](v3)(v5)
            \Edge[Direct](v4)(v5)
            \node at (0.75, 1.5*3/4*1/3 ) { \AxisRotator[rotate=0] };
            \node at (0.75, -1.5*3/4*1/3 ) { \AxisRotator[rotate=-60] };
            \end{scope}

            \begin{scope}[shift = {(2.5, -0.5 )}]
                  \node[anchor=north west,align=left,] at ( 0, 1.5*3/4 ) { \( \textcolor{rwth-magenta}{\mc V_0(\mc K)} : [1], [2], [3], [4], [5] \) };
                  \node[anchor=north west,align=left,] at ( 0, 0.9*3/4 ) { \( 
                  \begin{aligned}
                        \mc V_1(\mc K) : & \; [1, 2], [1, 3], [1, 4], \\[-2pt]
                        & \; [ 2, 3], [3, 4 ], [3, 5], [4, 5 ]  
                  \end{aligned}
                  \) };
                  \node[anchor=north west,align=left,] at ( 0, -.2*3/4 ) { \( 
                  \begin{aligned}
                        \textcolor{rwth-blue}{\mc V_2(\mc K)} : [ 1, 2, 3], [ 1, 3, 4]
                  \end{aligned}
                  \) };
            \end{scope}

            \begin{scope}[shift={(-2.75, -3.25)}]
            \Vertex[x=2.5, y=1.5*3/4, label = 1, style={color = rwth-magenta}, fontcolor = white, size = 0.3 ]{t1}
            \Vertex[x=3.5, y=1.5*3/4, label = 2, style={color = rwth-magenta}, fontcolor = white, size = 0.3 ]{t2}
            \Vertex[x=2.5, y=1.1*3/4, label = 1, style={color = rwth-magenta}, fontcolor = white, size = 0.3 ]{t3}
            \Vertex[x=3.5, y=1.1*3/4, label = 3, style={color = rwth-magenta}, fontcolor = white, size = 0.3 ]{t4}
            \Vertex[x=2.5, y=0.7*3/4, label = 1, style={color = rwth-magenta}, fontcolor = white, size = 0.3 ]{t5}
            \Vertex[x=3.5, y=0.7*3/4, label = 4, style={color = rwth-magenta}, fontcolor = white, size = 0.3 ]{t6}
            \Vertex[x=2.5, y=0.3*3/4, label = 2, style={color = rwth-magenta}, fontcolor = white, size = 0.3 ]{t7}
            \Vertex[x=3.5, y=0.3*3/4, label = 3, style={color = rwth-magenta}, fontcolor = white, size = 0.3 ]{t8}
            \Vertex[x=2.5, y=-0.1*3/4, label = 3, style={color = rwth-magenta}, fontcolor = white, size = 0.3 ]{t9}
            \Vertex[x=3.5, y=-0.1*3/4, label = 4, style={color = rwth-magenta}, fontcolor = white, size = 0.3 ]{t10}
            \Vertex[x=2.5, y=-0.5*3/4, label = 3, style={color = rwth-magenta}, fontcolor = white, size = 0.3 ]{t11}
            \Vertex[x=3.5, y=-0.5*3/4, label = 5, style={color = rwth-magenta}, fontcolor = white, size = 0.3 ]{t12}
            \Vertex[x=2.5, y=-0.9*3/4, label = 4, style={color = rwth-magenta}, fontcolor = white, size = 0.3 ]{t13}
            \Vertex[x=3.5, y=-0.9*3/4, label = 5, style={color = rwth-magenta}, fontcolor = white, size = 0.3 ]{t14}
            \Edge[Direct](t1)(t2)
            \Edge[Direct](t3)(t4)
            \Edge[Direct](t5)(t6)
            \Edge[Direct](t7)(t8)
            \Edge[Direct](t9)(t10)
            \Edge[Direct](t11)(t12)
            \Edge[Direct](t13)(t14)

            \draw[->, line width = 1.0] (2.1, 1.9*3/4)--(2.1, -1.5*3/4);
            \draw[->, line width = 1.0] (2.1, 1.9*3/4)--(3.85, 1.9*3/4);
            \node[ anchor=south ] at ( 3., 1.9*3/4 ) { \small orientation };
            \node[ anchor = south, rotate = 90 ] at (2.1, 0.2*3/4 ) { \small ordering };

            \end{scope}

            \begin{scope}[shift = {(1,-0.5)}]
            \node[] at ( 3.25, -2.0 ) { \( B_2 \textcolor{rwth-blue}{[1, 2, 3 ]} = \overbrace{\textcolor{rwth-red}{(+1)} [1, 2]}^{\substack{\text{1st in}\\\text{order}}} + (-1) [1, 3] + (+1) [2, 3] \) };
            \node[] at ( 3.25, -3.4 ) { \( B_2 \textcolor{rwth-blue}{[1, 3, 4 ]} = \underbrace{\textcolor{rwth-red}{(+1)} [1, 3]}_{\substack{\text{1st in}\\\text{order}}} + (-1) [1, 4] + (+1) [3, 4] \) };  
            \end{scope}
      \end{tikzpicture}
      \caption{ Example of a simplicial complex with ordering and orientation: nodes from \( \V 0 \) in magenta, triangles from \( \V 2 \) in blue. Orientation of edges and triangles is shown by arrows; the action of the \( B_2 \) operator is given for both triangles. Adapted from \cite{savostianov2024cholesky}. \label{fig:orientation}}
\end{figure}
