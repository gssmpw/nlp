\section{Additional background}
\label{appendix:background_appendix}

\subsection{Homology}
\label{appendix:homology}

\textit{Homology groups}, introduced by Henri Poincaré in the early 20th century
in his \textit{Analysis Situs}, are algebraic constructions that describe
"holes" in a topological space.
Here, we provide a brief summary of the notions used to define them.

Let $\calk$ be a simplicial complex. A \textit{$\homologyDim$-chain} $c$ is a formal sum of $\homologyDim$-simplices with \textit{modulo 2} coefficients\footnote{The coefficients can be chosen in another field or ring (which yields different homology groups), but $\bbz/2\bbz$ is the most common choice.} : $c=\sum\alpha_i\sigma_i$ with %$\dim\sigma_i=\homologyDim$ and
$\alpha_i\in\bbz/2\bbz$. Two $\homologyDim$-chains can be summed simplex by simplex, which permits to define the group of $\homologyDim$-chains, noted $\chain{\homologyDim}$.
The \textit{boundary} of a $\homologyDim$-simplex is the sum of its $(\homologyDim-1)$-dimensional faces, i.e. more formally, if $\sigma=(v_0,\ldots,v_k)$, then its boundary is \[\partial_\homologyDim\sigma=\sum\limits_{i=0}^k(v_0,\ldots,\hat{v_i},\ldots,v_k)\]
where the hat means that $v_i$ is omitted. Then the boundary of a $\homologyDim$-chain $c$ is defined as the modulo 2 sum of the boundaries of its simplices, i.e. $\partial_\homologyDim c=\sum\alpha_i\partial_\homologyDim\sigma_i$. This defines the \textit{$\homologyDim$-th boundary map} $\partial_\homologyDim$, which is a morphism $\partial_\homologyDim:\chain{\homologyDim}\rightarrow\chain{\homologyDim-1}$.

A \textit{$\homologyDim$-cycle} is a $\homologyDim$-chain whose boundary is 0. The group of $\homologyDim$-cycles is noted $\cycle{\homologyDim}=\ker\partial_\homologyDim$.
A \textit{$\homologyDim$-boundary} is a $\homologyDim$-chain which is the boundary of some $(\homologyDim+1)$-chain. The group of $\homologyDim$-boundaries is noted $\bound{\homologyDim}=\Ima\partial_{\homologyDim+1}$.
The fundamental lemma of homology states that
$\partial_{\homologyDim-1}\circ\partial_\homologyDim=0$, i.e., that the boundary
of a boundary is always an empty chain. In particular this implies the
inclusions
$\bound{\homologyDim}\subseteq\cycle{\homologyDim}\subseteq\chain{\homologyDim}$.

Two $\homologyDim$-cycles $a,b\in\cycle{\homologyDim}$ are \textit{homologous}
whenever $a=b+\partial c$ for some $c\in\chain{\homologyDim+1}$. The set of the
equivalence classes for this relation forms the
$\homologyDim$-th homology group of $\calk$:
$\homol{\homologyDim}=\cycle{\homologyDim}/\bound{\homologyDim}$. Its rank, i.e.
the maximal number of linearly independent classes (called \textit{generators}), is called the
\textit{$\homologyDim$-th Betti number} of $\calk$:
$\betti{\homologyDim}(\calk)=\rank\homol{\homologyDim}=\log_2\bigl(
|\homol{\homologyDim }|\bigr)$. The Betti numbers can be easily
interpreted geometrically for small values of $\homologyDim$: for a simplicial complex $\calk$ embedded in $\bbr^3$,
$\betti{0}(\calk)$ is the number of connected components, $\betti{1}(\calk)$ is the number of
cycles, and $\betti{2}(\calk)$ is the number of voids.

\subsection{Minmax length triangulations}
\label{appendix:MMLTs}
In this section, we give more details on the computation of minmax length ($\mml$) triangulations, that are all taken from~\cite{edelsbrunner_quadratic_1993}. General position is supposed, i.e., no two edges are equally long.
A \emph{2-edge} is an edge whose right and left lenses are both non-empty.
Conversely, a \emph{1-edge} is an edge such that one of its half-lens is empty but not the other one.
By definition, 1- and 2-edges are not $\rng$ edges.
Note that every triangulation of an
$\rng$ polygon has a 2-edge (its longest edge is always a 2-edge).

An \emph{expandable 2-edge} is a 2-edge $e=pq$ for which there exists $x\in\llens(e)$ and $y\in\rlens(e)$ such that the edges $px$, $xq$, $qy$, $yp$ are either $\rng$ edges, or 1-edges with an empty right lens (see \autoref{fig:expandability}).
Expandability is a property that somehow allows an edge to be the longest edge of a triangulation.
Indeed, let $e$ be an expandable 2-edge inside an
$\rng$-polygon $\Pi$.
Then it is possible to construct by recurrence a triangulation of $\Pi$ such that the only 2-edge -- and thus longest edge -- in that triangulation is $e$ itself.

Besides, the \emph{2-edge lemma} (Sec. 5.3 of~\cite{edelsbrunner_quadratic_1993}) states that there exists a $\mml$ triangulation of $\Pi$ that contains an expandable 2-edge.
Therefore, to find a $\mml$ triangulation of an
$\rng$-polygon, it suffices to find the shortest expandable 2-edge $e$, which
is the longest edge of any $\mml$-triangulation. This is the only information we
use in our application.
However, if a complete $\mml$ triangulation is wanted, $e=pq$ has to be found together with $x$ and $y$ defined as above.
Then, the incomplete polygons defined by $px$, $xq$, $qy$, $yp$ have to be triangulated (Sec. 5.1 of~\cite{edelsbrunner_quadratic_1993}).

\section{Raw data for \autoref{fig:counter-example}}
\label{appendix:counter-example-data}
The point clouds $X\subset\bbr^3$ and $Z\subset\bbr^2$ used in the counter-example depicted in \autoref{fig:counter-example} are given by the following coordinates:
\[X=
\begin{bmatrix}
	1     &  0     & 0     \\
	 0.5  &  0.866 & 0     \\
	-0.48 &  0.667 & 0     \\
	-0.73 & -0.199 & 0.433 \\
	-0.48 & -1.065 & 0     \\
	 0.5  & -0.866 & 0
\end{bmatrix}
\quad
Z=
\begin{bmatrix}
	1     & 0     \\
	0.5   & 0.866 \\
	-0.48 & 0.667 \\
	-0.98 &-0.199 \\
	-0.48 &-1.065 \\
	0.5   &-0.866
\end{bmatrix}.
\]

\section{Additional quantitative analysis}
\label{appendix:quantitative}
\begin{figure*}
	\centering
	\scriptsize{
	\begin{tabular}{|c|r||r|r||r|r|r||r|r|r||r|}
		\hline
		\multirow{2}{*}{Dataset} & \multirow{2}{*}{Indicator} & \multicolumn{2}{c||}{Global methods} & \multicolumn{3}{c||}{Locally topology-aware methods} & \multicolumn{4}{c|}{Globally topology-aware methods} \\
		\cline{3-11}
		& &
		\methodText{PCA}{\cite{pearson1901liii}} &
		\methodText{MDS}{\cite{torgerson1952multidimensional}} &
		\methodText{Isomap}{\cite{tenenbaum_global_2000}} &
		\methodText{t-SNE}{\cite{van2008visualizing}} &
		\methodText{UMAP}{\cite{mcinnes2018umap}} &
		\methodText{TopoMap}{\cite{doraiswamy2020topomap}} &
		\methodText{\tAE}{\cite{moor2020topological}} &
		\methodText{\carriereMethod}{\cite{carriere2021optimizing}} &
		\methodText{\tAE++}{} \\
		\hline

		\multirow{7}{*}{\datasetText{\datathreeblobs}} & $\metwasserzero$ & 2.0e+01 & 1.5e+01 & 2.6e+01 & 6.7e+02 & 2.1e+02 & \textbf{4.2e-04} & 2.0e+01 & 1.2e+01 & \underline{1.0e+01}\\
		& $\metwasser$ & 4.5e-01 & 5.0e-01 & 4.7e-01 & 2.3e+01 & 7.2e-01 & 4.3e+00 & 5.8e-01 & \textbf{1.1e-01} & \underline{4.4e-01}\\
		& $\metdistor$ & \textbf{2.7e-01} & \underline{4.1e-01} & 1.1e+00 & 3.5e+01 & 8.5e+00 & 7.0e+00 & 1.3e+00 & 1.5e+00 & 6.4e-01\\
		& $\LC$ & \textbf{1.00} & \underline{0.99} & 0.98 & 0.85 & 0.89 & 0.86 & 0.95 & 0.93 & 0.97\\
		& $\TA$ & \textbf{0.89} & \underline{0.88} & 0.72 & 0.38 & 0.42 & 0.48 & 0.70 & 0.66 & 0.78\\
		& $\Trust$ & 0.93 & 0.94 & 0.92 & \textbf{0.99} & \underline{0.98} & 0.96 & 0.97 & 0.97 & 0.96\\
		& $\Cont$ & \textbf{0.99} & 0.98 & 0.98 & 0.98 & 0.98 & 0.92 & 0.98 & 0.97 & \underline{0.98}\\
		\hline
		\multirow{7}{*}{\datasetText{\datatwist}} & $\metwasserzero$ & 1.1e-01 & 1.3e-01 & 4.5e-01 & 2.9e+00 & 6.6e-01 & \textbf{7.5e-06} & 8.6e-02 & 2.8e-02 & \underline{5.2e-03}\\
		& $\metwasser$ & 1.1e+00 & 2.2e-01 & 1.9e+01 & 3.3e+00 & 1.5e+01 & 1.8e-01 & 1.6e-01 & \textbf{1.8e-05} & \underline{1.7e-03}\\
		& $\metdistor$ & \underline{2.2e-01} & \textbf{2.1e-01} & 2.4e+00 & 3.9e+00 & 2.8e+00 & 8.2e-01 & 3.8e-01 & 7.3e-01 & 7.8e-01\\
		& $\LC$ & \textbf{0.99} & \underline{0.99} & 0.86 & 0.99 & 0.58 & 0.86 & 0.98 & 0.99 & 0.99\\
		& $\TA$ & \textbf{0.91} & \underline{0.89} & 0.63 & 0.87 & 0.42 & 0.64 & 0.85 & 0.87 & 0.88\\
		& $\Trust$ & 0.98 & 0.99 & \textbf{1.00} & 0.99 & 1.00 & 0.99 & 0.99 & 0.96 & \underline{1.00}\\
		& $\Cont$ & 0.99 & 0.99 & \textbf{1.00} & 0.99 & 1.00 & 0.98 & 0.99 & 0.98 & \underline{1.00}\\
		\hline
		\multirow{7}{*}{\datasetText{\datakfour}} & $\metwasserzero$ & 1.2e-01 & 6.9e-02 & 1.1e-01 & 3.0e+01 & 2.0e+01 & \textbf{2.3e-07} & 1.9e-01 & 7.8e-02 & \underline{5.8e-02}\\
		& $\metwasser$ & 2.7e-01 & 2.8e-01 & 3.8e-01 & 4.7e+01 & 3.9e-01 & 1.4e-01 & 6.5e-02 & \textbf{7.9e-04} & \underline{2.4e-02}\\
		& $\metdistor$ & \underline{2.7e-01} & \textbf{1.8e-01} & 5.1e-01 & 1.6e+01 & 9.1e+00 & 5.7e-01 & 1.0e+00 & 6.8e-01 & 3.4e-01\\
		& $\LC$ & \underline{0.83} & \textbf{0.89} & 0.65 & 0.66 & 0.53 & 0.57 & 0.75 & 0.68 & 0.77\\
		& $\TA$ & \underline{0.61} & \textbf{0.63} & 0.39 & 0.48 & 0.34 & 0.38 & 0.51 & 0.47 & 0.56\\
		& $\Trust$ & 0.97 & 0.98 & 0.95 & \textbf{1.00} & 1.00 & 0.99 & 0.98 & 0.98 & \underline{1.00}\\
		& $\Cont$ & \underline{1.00} & 1.00 & 1.00 & 0.99 & 0.99 & 0.98 & 0.99 & 0.99 & \textbf{1.00}\\
		\hline
		\multirow{7}{*}{\datasetText{\datakfive}} & $\metwasserzero$ & 2.7e-01 & 1.5e-01 & 1.8e-01 & 9.3e+01 & 1.8e+01 & \textbf{7.3e-07} & 2.2e-01 & 1.3e-01 & \underline{9.0e-02}\\
		& $\metwasser$ & 2.8e-01 & 3.5e-01 & 1.3e-01 & 4.7e+01 & 9.7e-01 & 4.8e-01 & 1.6e-01 & \textbf{2.2e-03} & \underline{8.0e-02}\\
		& $\metdistor$ & \underline{3.1e-01} & \textbf{2.4e-01} & 4.7e-01 & 2.1e+01 & 9.8e+00 & 1.2e+00 & 9.2e-01 & 8.6e-01 & 3.4e-01\\
		& $\LC$ & \underline{0.81} & \textbf{0.86} & 0.67 & 0.76 & 0.58 & 0.41 & 0.73 & 0.72 & 0.81\\
		& $\TA$ & \textbf{0.63} & \underline{0.61} & 0.43 & 0.55 & 0.41 & 0.33 & 0.50 & 0.50 & 0.57\\
		& $\Trust$ & 0.93 & 0.97 & 0.96 & \textbf{1.00} & \underline{1.00} & 0.99 & 0.99 & 0.97 & 1.00\\
		& $\Cont$ & 0.99 & 0.99 & \textbf{1.00} & 0.99 & 0.98 & 0.97 & 0.99 & 0.99 & \underline{1.00}\\
		\hline
		\multirow{7}{*}{\datasetText{\datacoil}} & $\metwasserzero$ & 1.2e+02 & 8.1e+01 & 3.0e+00 & 1.5e+02 & 1.5e+02 & \textbf{2.3e-06} & 1.9e+00 & 4.5e+00 & \underline{1.2e+00}\\
		& $\metwasser$ & 1.2e+01 & 1.0e+01 & 6.0e+02 & 1.3e+01 & 1.2e+01 & 2.7e+01 & 7.5e+00 & \underline{9.1e-02} & \textbf{1.6e-02}\\
		& $\metdistor$ & \underline{2.9e+00} & \textbf{2.0e+00} & 2.1e+01 & 4.9e+00 & 6.2e+00 & 7.2e+00 & 8.2e+00 & 6.7e+00 & 1.9e+01\\
		& $\LC$ & \textbf{0.94} & \underline{0.93} & 0.83 & 0.88 & 0.85 & 0.74 & 0.81 & 0.62 & 0.88\\
		& $\TA$ & \textbf{0.81} & 0.74 & 0.47 & \underline{0.77} & 0.65 & 0.44 & 0.56 & 0.38 & 0.73\\
		& $\Trust$ & 0.97 & 0.97 & \underline{1.00} & \textbf{1.00} & 0.99 & 0.92 & 0.96 & 0.91 & 0.99\\
		& $\Cont$ & 0.96 & 0.97 & 0.98 & \textbf{1.00} & \underline{0.99} & 0.96 & 0.97 & 0.94 & 0.99\\
		\hline
		\multirow{7}{*}{\datasetText{\datamocap}} & $\metwasserzero$ & 4.3e+03 & 2.2e+03 & \underline{3.9e+02} & 6.8e+03 & 6.8e+03 & \textbf{2.4e-03} & 1.6e+03 & 7.9e+02 & 1.9e+03\\
		& $\metwasser$ & 2.1e+03 & 1.9e+03 & 1.0e+04 & 1.6e+03 & 1.5e+03 & 2.1e+03 & 1.2e+03 & \textbf{9.4e+00} & \underline{1.2e+01}\\
		& $\metdistor$ & \underline{2.4e+01} & \textbf{1.7e+01} & 1.3e+02 & 9.9e+01 & 1.0e+02 & 8.5e+01 & 2.6e+01 & 5.6e+01 & 2.9e+01\\
		& $\LC$ & \underline{0.94} & \textbf{0.95} & 0.67 & 0.90 & 0.47 & 0.43 & 0.89 & 0.61 & 0.83\\
		& $\TA$ & \underline{0.72} & \textbf{0.73} & 0.53 & 0.64 & 0.34 & 0.35 & 0.68 & 0.42 & 0.59\\
		& $\Trust$ & 0.98 & 0.98 & \underline{1.00} & 1.00 & \textbf{1.00} & 0.97 & 0.98 & 0.93 & 1.00\\
		& $\Cont$ & 0.98 & 0.99 & \textbf{1.00} & 1.00 & 0.99 & 0.98 & 0.99 & 0.98 & \underline{1.00}\\
		\hline
		\multirow{7}{*}{\datasetText{\datasinglecell}} & $\metwasserzero$ & 4.6e+05 & 4.6e+05 & 3.5e+05 & 4.6e+05 & 4.6e+05 & \textbf{1.2e-01} & 4.6e+05 & \underline{1.6e+05} & 3.8e+05\\
		& $\metwasser$ & 3.0e+03 & 3.1e+03 & 1.8e+05 & 1.8e+03 & 1.8e+03 & 2.8e+04 & 1.8e+03 & \textbf{6.4e+02} & \underline{8.6e+02}\\
		& $\metdistor$ & \underline{3.1e+01} & \textbf{2.3e+01} & 4.8e+02 & 9.6e+01 & 1.0e+02 & 1.2e+03 & 9.8e+01 & 3.2e+02 & 1.5e+02\\
		& $\LC$ & \textbf{0.99} & \underline{0.98} & 0.98 & 0.88 & 0.81 & 0.62 & 0.72 & 0.74 & 0.85\\
		& $\TA$ & \textbf{0.94} & \underline{0.86} & 0.84 & 0.68 & 0.54 & 0.40 & 0.55 & 0.52 & 0.61\\
		& $\Trust$ & \underline{0.99} & 0.99 & 0.99 & \textbf{0.99} & 0.99 & 0.96 & 0.97 & 0.96 & 0.97\\
		& $\Cont$ & \underline{0.99} & 0.99 & 0.99 & \textbf{0.99} & 0.99 & 0.95 & 0.94 & 0.96 & 0.96\\
		\hline

	\end{tabular}
	}
	%\vspace{2mm}
	\caption{\PH{}-based metrics and other common indicators used in DR (see \autoref{appendix:quantitative} for a description). The best value (before rounding) for an indicator is written in bold, the second best (before rounding) is underlined.}
	\label{table:quantitative}
\end{figure*}

% \begin{table*}
	\centering
	\scriptsize{
	\begin{tabular}{|c|r||r|r||r|r|r||r|r|r|r||r|}
		\hline
		\multirow{2}{*}{Dataset} & \multirow{2}{*}{Indicator} & \multicolumn{2}{c||}{Global methods} & \multicolumn{3}{c||}{Locally topology-aware methods} & \multicolumn{5}{c|}{Globally topology-aware methods} \\
		\cline{3-12}
		& &
		\methodText{PCA}{\cite{pearson1901liii}} &
		\methodText{MDS}{\cite{torgerson1952multidimensional}} &
		\methodText{Isomap}{\cite{tenenbaum_global_2000}} &
		\methodText{t-SNE}{\cite{van2008visualizing}} &
		\methodText{UMAP}{\cite{mcinnes2018umap}} &
		\methodText{TopoMap}{\cite{doraiswamy2020topomap}} &
		\methodText{\tAE}{\cite{moor2020topological}} &
		\methodText{\tAE1}{} &
		\methodText{\carriereMethod}{\cite{carriere2021optimizing}} &
		\methodText{\tAE++}{} \\
		\hline

		\multirow{7}{*}{\datasetText{\datathreeblobs}} & $\metwasserzero$ & 2.0e+01 & 1.5e+01 & 2.6e+01 & 6.7e+02 & 2.1e+02 & \textbf{4.2e-04} & 2.0e+01 & 1.4e+01 & 1.2e+01 & \underline{1.0e+01}\\
		& $\metwasser$ & 4.5e-01 & 5.0e-01 & 4.7e-01 & 2.3e+01 & 7.2e-01 & 4.3e+00 & 5.8e-01 & \underline{3.1e-01} & \textbf{1.1e-01} & 4.4e-01\\
		& $\metdistor$ & \textbf{2.7e-01} & \underline{4.1e-01} & 1.1e+00 & 3.5e+01 & 8.5e+00 & 7.0e+00 & 1.3e+00 & 1.0e+00 & 1.5e+00 & 6.4e-01\\
		& $\LC$ & \textbf{1.00} & \underline{0.99} & 0.98 & 0.85 & 0.89 & 0.86 & 0.95 & 0.95 & 0.93 & 0.97\\
		& $\TA$ & \textbf{0.89} & \underline{0.88} & 0.72 & 0.38 & 0.42 & 0.48 & 0.70 & 0.64 & 0.66 & 0.78\\
		& $\Trust$ & 0.93 & 0.94 & 0.92 & \textbf{0.99} & \underline{0.98} & 0.96 & 0.97 & 0.97 & 0.97 & 0.96\\
		& $\Cont$ & \textbf{0.99} & 0.98 & 0.98 & 0.98 & 0.98 & 0.92 & 0.98 & 0.97 & 0.97 & \underline{0.98}\\
		\hline
		\multirow{7}{*}{\datasetText{\datatwist}} & $\metwasserzero$ & 1.1e-01 & 1.3e-01 & 4.5e-01 & 2.9e+00 & 6.6e-01 & \textbf{7.5e-06} & 8.6e-02 & 1.4e-01 & 2.8e-02 & \underline{5.2e-03}\\
		& $\metwasser$ & 1.1e+00 & 2.2e-01 & 1.9e+01 & 3.3e+00 & 1.5e+01 & 1.8e-01 & 1.6e-01 & \underline{6.9e-05} & \textbf{1.8e-05} & 1.7e-03\\
		& $\metdistor$ & \underline{2.2e-01} & \textbf{2.1e-01} & 2.4e+00 & 3.9e+00 & 2.8e+00 & 8.2e-01 & 3.8e-01 & 3.1e-01 & 7.3e-01 & 7.8e-01\\
		& $\LC$ & \textbf{0.99} & \underline{0.99} & 0.86 & 0.99 & 0.58 & 0.86 & 0.98 & 0.98 & 0.99 & 0.99\\
		& $\TA$ & \textbf{0.91} & \underline{0.89} & 0.63 & 0.87 & 0.42 & 0.64 & 0.85 & 0.82 & 0.87 & 0.88\\
		& $\Trust$ & 0.98 & 0.99 & \textbf{1.00} & 0.99 & 1.00 & 0.99 & 0.99 & 1.00 & 0.96 & \underline{1.00}\\
		& $\Cont$ & 0.99 & 0.99 & \textbf{1.00} & 0.99 & 1.00 & 0.98 & 0.99 & 1.00 & 0.98 & \underline{1.00}\\
		\hline
		\multirow{7}{*}{\datasetText{\datakfour}} & $\metwasserzero$ & 1.2e-01 & 6.9e-02 & 1.1e-01 & 3.0e+01 & 2.0e+01 & \textbf{2.3e-07} & 1.9e-01 & \underline{5.1e-02} & 7.8e-02 & 5.8e-02\\
		& $\metwasser$ & 2.7e-01 & 2.8e-01 & 3.8e-01 & 4.7e+01 & 3.9e-01 & 1.4e-01 & 6.5e-02 & \underline{5.3e-03} & \textbf{7.9e-04} & 2.4e-02\\
		& $\metdistor$ & \underline{2.7e-01} & \textbf{1.8e-01} & 5.1e-01 & 1.6e+01 & 9.1e+00 & 5.7e-01 & 1.0e+00 & 4.2e-01 & 6.8e-01 & 3.4e-01\\
		& $\LC$ & \underline{0.83} & \textbf{0.89} & 0.65 & 0.66 & 0.53 & 0.57 & 0.75 & 0.78 & 0.68 & 0.77\\
		& $\TA$ & \underline{0.61} & \textbf{0.63} & 0.39 & 0.48 & 0.34 & 0.38 & 0.51 & 0.56 & 0.47 & 0.56\\
		& $\Trust$ & 0.97 & 0.98 & 0.95 & \textbf{1.00} & 1.00 & 0.99 & 0.98 & 1.00 & 0.98 & \underline{1.00}\\
		& $\Cont$ & 1.00 & 1.00 & 1.00 & 0.99 & 0.99 & 0.98 & 0.99 & \underline{1.00} & 0.99 & \textbf{1.00}\\
		\hline
		\multirow{7}{*}{\datasetText{\datakfive}} & $\metwasserzero$ & 2.7e-01 & 1.5e-01 & 1.8e-01 & 9.3e+01 & 1.8e+01 & \textbf{7.3e-07} & 2.2e-01 & 1.4e-01 & 1.3e-01 & \underline{9.0e-02}\\
		& $\metwasser$ & 2.8e-01 & 3.5e-01 & 1.3e-01 & 4.7e+01 & 9.7e-01 & 4.8e-01 & 1.6e-01 & \underline{7.4e-02} & \textbf{2.2e-03} & 8.0e-02\\
		& $\metdistor$ & \underline{3.1e-01} & \textbf{2.4e-01} & 4.7e-01 & 2.1e+01 & 9.8e+00 & 1.2e+00 & 9.2e-01 & 6.4e-01 & 8.6e-01 & 3.4e-01\\
		& $\LC$ & \underline{0.81} & \textbf{0.86} & 0.67 & 0.76 & 0.58 & 0.41 & 0.73 & 0.77 & 0.72 & 0.81\\
		& $\TA$ & \textbf{0.63} & \underline{0.61} & 0.43 & 0.55 & 0.41 & 0.33 & 0.50 & 0.55 & 0.50 & 0.57\\
		& $\Trust$ & 0.93 & 0.97 & 0.96 & \textbf{1.00} & \underline{1.00} & 0.99 & 0.99 & 1.00 & 0.97 & 1.00\\
		& $\Cont$ & 0.99 & 0.99 & \textbf{1.00} & 0.99 & 0.98 & 0.97 & 0.99 & 0.99 & 0.99 & \underline{1.00}\\
		\hline
		\multirow{7}{*}{\datasetText{\datacoil}} & $\metwasserzero$ & 1.2e+02 & 8.1e+01 & 3.0e+00 & 1.5e+02 & 1.5e+02 & \textbf{2.3e-06} & 1.9e+00 & \underline{3.3e-01} & 4.5e+00 & 1.2e+00\\
		& $\metwasser$ & 1.2e+01 & 1.0e+01 & 6.0e+02 & 1.3e+01 & 1.2e+01 & 2.7e+01 & 7.5e+00 & \underline{3.2e-02} & 9.1e-02 & \textbf{1.6e-02}\\
		& $\metdistor$ & \underline{2.9e+00} & \textbf{2.0e+00} & 2.1e+01 & 4.9e+00 & 6.2e+00 & 7.2e+00 & 8.2e+00 & 2.0e+01 & 6.7e+00 & 1.9e+01\\
		& $\LC$ & \textbf{0.94} & \underline{0.93} & 0.83 & 0.88 & 0.85 & 0.74 & 0.81 & 0.87 & 0.62 & 0.88\\
		& $\TA$ & \textbf{0.81} & 0.74 & 0.47 & \underline{0.77} & 0.65 & 0.44 & 0.56 & 0.72 & 0.38 & 0.73\\
		& $\Trust$ & 0.97 & 0.97 & \underline{1.00} & \textbf{1.00} & 0.99 & 0.92 & 0.96 & 0.99 & 0.91 & 0.99\\
		& $\Cont$ & 0.96 & 0.97 & 0.98 & \textbf{1.00} & \underline{0.99} & 0.96 & 0.97 & 0.98 & 0.94 & 0.99\\
		\hline
		\multirow{7}{*}{\datasetText{\datamocap}} & $\metwasserzero$ & 4.3e+03 & 2.2e+03 & \underline{3.9e+02} & 6.8e+03 & 6.8e+03 & \textbf{2.4e-03} & 1.6e+03 & 4.2e+03 & 7.9e+02 & 1.9e+03\\
		& $\metwasser$ & 2.1e+03 & 1.9e+03 & 1.0e+04 & 1.6e+03 & 1.5e+03 & 2.1e+03 & 1.2e+03 & 3.0e+01 & \textbf{9.4e+00} & \underline{1.2e+01}\\
		& $\metdistor$ & \underline{2.4e+01} & \textbf{1.7e+01} & 1.3e+02 & 9.9e+01 & 1.0e+02 & 8.5e+01 & 2.6e+01 & 5.1e+01 & 5.6e+01 & 2.9e+01\\
		& $\LC$ & \underline{0.94} & \textbf{0.95} & 0.67 & 0.90 & 0.47 & 0.43 & 0.89 & 0.48 & 0.61 & 0.83\\
		& $\TA$ & \underline{0.72} & \textbf{0.73} & 0.53 & 0.64 & 0.34 & 0.35 & 0.68 & 0.42 & 0.42 & 0.59\\
		& $\Trust$ & 0.98 & 0.98 & \underline{1.00} & 1.00 & \textbf{1.00} & 0.97 & 0.98 & 0.99 & 0.93 & 1.00\\
		& $\Cont$ & 0.98 & 0.99 & \textbf{1.00} & 1.00 & 0.99 & 0.98 & 0.99 & 0.99 & 0.98 & \underline{1.00}\\
		\hline
		\multirow{7}{*}{\datasetText{\datasinglecell}} & $\metwasserzero$ & 4.6e+05 & 4.6e+05 & 3.5e+05 & 4.6e+05 & 4.6e+05 & \textbf{1.2e-01} & 4.6e+05 & 2.9e+05 & \underline{1.6e+05} & 3.8e+05\\
		& $\metwasser$ & 3.0e+03 & 3.1e+03 & 1.8e+05 & 1.8e+03 & 1.8e+03 & 2.8e+04 & 1.8e+03 & 1.3e+03 & \textbf{6.4e+02} & \underline{8.6e+02}\\
		& $\metdistor$ & \underline{3.1e+01} & \textbf{2.3e+01} & 4.8e+02 & 9.6e+01 & 1.0e+02 & 1.2e+03 & 9.8e+01 & 2.6e+02 & 3.2e+02 & 1.5e+02\\
		& $\LC$ & \textbf{0.99} & \underline{0.98} & 0.98 & 0.88 & 0.81 & 0.62 & 0.72 & 0.88 & 0.74 & 0.85\\
		& $\TA$ & \textbf{0.94} & \underline{0.86} & 0.84 & 0.68 & 0.54 & 0.40 & 0.55 & 0.64 & 0.52 & 0.61\\
		& $\Trust$ & \underline{0.99} & 0.99 & 0.99 & \textbf{0.99} & 0.99 & 0.96 & 0.97 & 0.98 & 0.96 & 0.97\\
		& $\Cont$ & \underline{0.99} & 0.99 & 0.99 & \textbf{0.99} & 0.99 & 0.95 & 0.94 & 0.98 & 0.96 & 0.96\\
		\hline

	\end{tabular}
	}
	\vspace{2mm}
	\caption{\PH{}-based metrics and other common indicators used in DR (see \autoref{appendix:quantitative} for a description). The best value (before rounding) for an indicator is written in bold, the second best (before rounding) is underlined.}
	\label{table:quantitativeWithTopoAE1}
\end{table*}

See \autoref{table:quantitative} for topological metrics and some other commonly used DR indicators computed on the examples presented in the paper.

\subsection{Description of the indicators}
\label{appendix:quantitativeDescription}

\paragraph{Pairwise distances-based indicators}
In addition to the metric distortion $\metdistor$, one can measure the linear correlation $\LC\in[-1,1]$ between the pairwise distances in $\inputPointCloud$ and in $\latentPointCloud$.
Having $\LC=1$ means that these pairwise distances are perfectly correlated in high and low dimension.

\paragraph{Triplets-based indicator}
The \textit{triplet accuracy} $\TA\in[0,1]$ is the proportion of triangles whose three edges have the same relative order (in terms of length) both in high and low dimension~\cite{wang_understanding_2021}.
It measures to some extent the preservation of the global structure of
$\inputPointCloud$, which is preserved when $\TA=1$.

\paragraph{Rank-based indicators}
The rank of a point $\inputPointCloud_i$ relative to another point $\inputPointCloud_j$ is the integer $\rho_{ij}\in\bbn$ such that $\inputPointCloud_j$ is the $\rho_{ij}$-th nearest neighbor of $\inputPointCloud_i$.
There exists indicators based on these ranks~\cite{lee_quality_2009, venna2006local} that measure the preservation of the nearest neighbors.
In particular, the \emph{trustworthiness} ($\Trust\in[0,1]$) is penalized when a
high rank $\rho_{ij}$ in the input becomes low in the representation,
i.e., when faraway points in high-dimension become neighbors in
low-dimension.
On the contrary, the \emph{continuity} ($\Cont\in[0,1]$) is penalized when a low
rank $\rho_{ij}$ in the input becomes high in the representation, i.e.,
 when neighbors in high-dimension are projected to faraway points in
low-dimension.
These indicators are computed for a number $K$ of nearest neighbors
(in our experiments, $K=10$).

\subsection{Observations}
\label{appendix:quantitativeObservations}

TopoMap has the best results for $\metwasserzero$, which is expected since it preserves by design exactly \PH{0}.
Although in theory this quantity should be exactly 0, the approximation
performed by the auction algorithm when estimating
the Wasserstein distance (to compute $\metwasserzero$) makes it slightly positive.
Our approach (\tAE++) has competitive results for this metric as well, which was expected since, as \tAE, it constrains the preservation of \PH{0} in the sense of \autoref{lemma:TopoAE0_bound}.

Global indicators, namely the metric distortion $\metdistor$, the linear correlation $\LC$ between pairwise distances and the triplet accuracy $\TA$, are  best preserved by global methods, i.e., PCA and MDS.
Our approach (\tAE++) presents competitive results for these indicators when
comparing to other locally topology-aware methods, while it clearly
outperforms globally topology-aware methods.

Finally, \tAE++ also presents good results for neighborhood quality
indicators ($\Trust$ and $\Cont$). Indeed, in our
datasets, the correct embedding of the input cycle(s) favors the preservation of
the neighborhood.
More precisely, a self-intersection in the projection of an input cycle penalizes the trustworthiness (since faraway vertices would be projected as neighbors near this intersection), while a broken input cycle penalizes the continuity (since two neighbors in the input generator would be embedded faraway in low-dimension).
