\section{Conclusion}
\label{sec:conclusion}

We have introduced \tAE++, a dimensionality reduction method
aiming at accurately visualizing the cyclic patterns present in high
diemnsional data. For this, we revisited TopoAE~\cite{moor2020topological},
provided a novel theoretical analysis of its original formulation
for \PH{0}, and introduced a new generalization to \PH{1}.
We have shown experimentally that our novel projection
provides an improved balance between the topological accuracy and the
visual preservation of the input $1$-cycles.
As a side benefit of our work, to overcome the computational overhead
due to the \PH{1} computation, we have presented
a novel, fast, geometric algorithm that computes Rips \PH{}
for planar point clouds, which may be of independent interest.

In future work, the possibility to handle automatically bad local minima during
the minimization, e.g., by penalizing
self-intersections of input generators, could be investigated.
Besides, an interesting extension would be to generalize the method to either
higher-dimensional latent spaces (e.g., $\calz=\bbr^3$, still
for visualization purpose), or constraints on higher-dimensional homology
(e.g., constraining \PH{2} to preserve the \emph{cavities}).
However, this might prove challenging in terms of computational cost, since the generalization of our fast planar Rips \PH{1} algorithm does not seem trivial to such a context.
Finally, another interesting extension would be to consider
other filtrations of specific interest for the applications.
