\section{Related work}

The literature related to our work can be classified into two
categories: general purpose DR techniques
(\autoref{sec:relatedWorkGeneralPurpose}) and topology-aware techniques
(\autoref{sec:relatedWorkTopology}).

\subsection{General purpose dimensionality reduction}
\label{sec:relatedWorkGeneralPurpose}

Numerous DR techniques have been proposed and the related literature has been
summarized in several books~\cite{borg97, dimensionReductionBook} and surveys
\cite{surveyDimensionReduction2, surveyDimensionReduction1, NonatoA19}.
Principal Component Analysis (PCA)~\cite{pearson1901liii} is by far the most
popular linear DR technique.
Although it is an indispensable tool for data analysis,
its linear nature does not always allow it to apprehend complex non-linear
phenomena. One of the first non linear DR methods is the multidimensional
scaling (MDS)~\cite{torgerson1952multidimensional}. It aims at preserving as far
as possible the pairwise distances in the high- and low-dimensional point
clouds.
Another approach, particularly related to our work,
consists in optimizing an autoencoder neural network~\cite{hinton_reducing_2006}.
The \textit{encoder} is used to represent the explicit projection map from the
high-dimensional input space to the low-dimensional representation
space, while the \textit{decoder} tries to reconstruct the input data
from its encoded representation.
We will refer to these methods as \emph{global} methods.

Global methods have been used successfully in many applications, but
they do not take into
account the possible distribution of the input points over some implicit,
unknown manifold. This may lead to the unwanted preservation of distances
between points that are close in the ambient space but far apart on this
manifold. \emph{Locally topology-aware} methods have therefore been
introduced to address this issue. For instance,
Isomap~\cite{tenenbaum_global_2000}
preserves geodesic distances on a captured manifold structure of the
input data.
%\remove{Because it suffers from computational
%inefficiencies, Isomap was sped up with the use of landmark points (L-Isomap
%\cite{silva2003global}).}
Other methods also feature neighborhood preservation objectives.
For example, Local Linear Embedding (LLE)~\cite{roweis2000nonlinear} relies
on linear reconstructions of local neighborhoods.
Other methods leverage additional landmarks~\cite{silva2003global} or user-provided
control points~\cite{joia:tvcg:2011}.
%Some local methods additionally support user
%constraints expressed as control points~\cite{joia:tvcg:2011}.

All these methods try to preserve local
Euclidean distances when projecting to a lower dimension.
However, this can sometimes lack relevance in the applications,
especially with high-dimensional datasets for which
the distribution of pairwise Euclidean distances tend to be uniform.
For such cases, local distance preservation fails at characterizing
well relevant local relations.
To alleviate this issue, SNE~\cite{hinton2002stochastic} and later
t\nobreakdash-SNE~\cite{van2008visualizing} use a conditional probability
formulation to represent similarities between points and try to
have similar distributions both in high- and low-dimension thanks to a
Kullback--Leibler divergence minimization.
More recently UMAP has been introduced~\cite{mcinnes2018umap} along a
theoretical foundation on category theory.
It provides results that are similar visually to t-SNE, but in a more
scalable way.
Variants were later introduced to better preserve the global structure in the embedding, such as TriMAP~\cite{amid2022trimap} that constrains the proximity order within triplets of points, or PaCMAP~\cite{wang_understanding_2021} that adds constraints on more distant point pairs.
Although these methods succeed in preserving the local topology, they are not
explicitly aware of the global structure
of the input, which can lead to the loss of noteworthy global,
topological features.

\subsection{Globally topology-aware dimensionality reduction}
\label{sec:relatedWorkTopology}

Topology-based methods have become popular over the last
two decades in data analysis and
visualization~\cite{heine16} and have been applied to various areas:
astrophysics~\cite{sousbie11, shivashankar2016felix},
biological imaging~\cite{beiBrain18, carr04, topoAngler},
quantum chemistry~\cite{chemistry_vis14,harshChemistry, D2CP05893F},
fluid dynamics~\cite{kasten_tvcg11, NauleauVBBT22},
material sciences~\cite{gyulassy_vis07, gyulassy_vis15, SolerPDPCT19},
turbulent combustion~\cite{gyulassy_ev14, laney_vis06}. They leverage tools that
define concise signatures of the data based on its topological properties and
that have been summarized in topological data analysis reference
books ~\cite{edelsbrunner_computational_2010, zomorodian_computational_2010}
and surveys~\cite{chazal_introduction_2021}.

Several DR methods have been proposed
by the visualization community to preserve specific topological signatures
of the input data. For instance, terrain metaphors have been
investigated for the visualization of an input high-dimensional scalar
field, in the form of a three-dimensional terrain, whose elevation yields an
identical contour tree~\cite{Weber:2007} or an identical set of separatrices
\cite{gerber2010, gerber2013}.
This framework has been extended to density
estimators~\cite{OesterlingHJS10,
OesterlingSTHKEW10, OHJSH11, Oesterling0WS13} for the support of
high-dimensional point clouds. However, such metaphors completely discard
the metric information of the input space~\cite{OesterlingHJS10}, possibly
placing next to each other points which are arbitrarily far
in the input space (and reciprocally). Yan et al.~\cite{abs-1806-08460}
introduced a DR approach driven by the Mapper structure~\cite{SinghMC07}, an
approximation of the Reeb graph~\cite{reeb46}, which can capture in practice
large handles in the data, however without guarantees, since the number of handles in the considered manifold is only an upper bound on the number of loops in the Reeb graph~\cite{edelsbrunner_computational_2010}.

To incorporate the metric information from the input data while
preserving at the same time some of their topological characteristics, several
approaches have focused on driving the projection by
the \emph{persistence diagram}
of the Rips filtration of the point cloud (see \autoref{sec:persistentHomology}
for  a technical description).
Carriere et al.~\cite{carriere2021optimizing} presented a generic persistence
optimization framework with an application to dimensionality reduction.
Their approach explicitly minimizes the Wasserstein distance
(\autoref{sec:persistentHomology}) between the $1$-dimensional persistence
diagrams in high and low dimensions. However, this approach solely focuses on
this penalization term. As a result,
although the number and persistence of cycles  may be well-preserved,
the solver tends to produce cycles in low dimensions which involve arbitrary
points (e.g., which were not necessarily located along the cycles in high
dimensions), which challenges visual interpretation, as later
detailed in \autoref{sec:results:analysis}.

To enforce a correspondence between the topological
features at the data point level, additional structures need to be preserved.
For the specific case of $0$-dimensional persistent homology (\PH{0}),
Doraiswamy et al. introduced \emph{TopoMap}~\cite{doraiswamy2020topomap}, an
algorithm which constructively preserves the \emph{persistence pairs}
(\autoref{sec:persistentHomology}) through the preservation of the minimum
spanning tree of the data. An accelerated version, with improved layouts, has
recently been proposed~\cite{guardieiro2024topomap++}.
Alternative approaches have considered the usage of an optimization framework
(typically based on an autoencoder neural network
\cite{hinton_reducing_2006}), with the integration of specific topology-aware
losses~\cite{moor2020topological,barannikov2021representation,
nelson2022topology,trofimov2023learning,schonenberger2020witness}. Among them,
a prominent approach is the \emph{Topological Autoencoders}
(TopoAE)~\cite{moor2020topological}. Its loss aims at preserving
the diameter of the simplices involved in
persistence pairs, when going from high to low dimensions and reciprocally.
However, the above techniques focused in practice on the
preservation of \PH{0} and did not, to our knowledge, report experiments
regarding the preservation of higher dimensional PH.
Specifically, we show in \autoref{sec:analysis} that, while a zero
TopoAE loss indeed implies a preservation of the persistence pairs for \PH{0},
it is not the case for higher dimensional PH. We provide a counter example for
\PH{1}, which is addressed by our novel, generalized loss.
