\begin{figure*}
	\adjustbox{width=1.013\linewidth,center}{
		\scriptsize{
		\begin{tabular}{|p{0cm}r||rr|rrr|rrr||r|}
			\hline
			
			\multicolumn{2}{|c||}{\multirow{2}{*}{Input}}&
			\multicolumn{2}{c|}{Global methods}&
			\multicolumn{3}{c|}{Locally topology-aware methods}&
			\multicolumn{4}{c|}{Globally topology-aware methods}\\
			
			\cline{3-11}
			\multicolumn{2}{|c||}{}
			&\multicolumn{1}{c}{PCA\cite{pearson1901liii}}
			&\multicolumn{1}{c|}{MDS\cite{torgerson1952multidimensional}}
			&\multicolumn{1}{c}{Isomap\cite{tenenbaum_global_2000}}
			&\multicolumn{1}{c}{t-SNE\cite{van2008visualizing}}
			&\multicolumn{1}{c|}{UMAP\cite{mcinnes2018umap}}
			&\multicolumn{1}{c}{TopoMap\cite{doraiswamy2020topomap}}
			&\multicolumn{1}{c}{\tAE\cite{moor2020topological}}
			&\multicolumn{1}{c||}{\carriereMethod\cite{carriere2021optimizing}}
			&\multicolumn{1}{c|}{\tAE++}\\
			
			\hline \raisebox{1.55cm}{\multirow{4}{*}{\rotatebox{90}{\datathreeblobs{} $(n=800)$}}} &
			\raisebox{-.5mm}{\includegraphics[width=.1\linewidth]{figures/data/3blobs_input.png}}&
			\includegraphics[width=.1\linewidth]{results/figures_cropped/3blobs_PCA.png}&
			\includegraphics[width=.1\linewidth]{results/figures_cropped/3blobs_MDS.png}&
			\includegraphics[width=.1\linewidth]{results/figures_cropped/3blobs_Isomap.png}&
			\includegraphics[width=.1\linewidth]{results/figures_cropped/3blobs_tSNE.png}&
			\includegraphics[width=.1\linewidth]{results/figures_cropped/3blobs_UMAP.png}&
			\includegraphics[width=.1\linewidth]{results/figures_cropped/3blobs_TopoMap.png}&
			\includegraphics[width=.1\linewidth]{results/figures_cropped/3blobs_TopoAE.png}&
			\includegraphics[width=.1\linewidth]{results/figures_cropped/3blobs_Carriere.png}&
			\includegraphics[width=.1\linewidth]{results/figures_cropped/3blobs_TopoAE++.png}\\

			& $\metwasser(\inputPointCloud,\latentPointCloud)$ & 4.5e-01 & 5.0e-01 & 4.7e-01 & 2.3e+01 & 7.2e-01 & 4.3e+00 & 5.8e-01 & \textbf{1.1e-01} & \underline{4.4e-01}\\
			& $\metdistor(\inputPointCloud,\latentPointCloud)$ & \textbf{2.7e-01} & \underline{4.1e-01} & 1.1e+00 & 3.5e+01 & 8.5e+00 & 7.0e+00 & 1.3e+00 & 1.5e+00 & 6.4e-01\\
			& $\timing$ & 0.00049 & 2.5 & 0.16 & 0.76 & 2.6 & 0.086 & 81 & 830 & 5.0\\
			
			\hline \raisebox{1.25cm}{\multirow{4}{*}{\rotatebox{90}{\datatwist{} $(n=100)$}}} &
			\raisebox{-.5mm}{\includegraphics[width=.1\linewidth]{figures/data/1twist_input.png}}&
			\includegraphics[width=.1\linewidth]{results/figures_cropped/1twist_PCA.png}&
			\includegraphics[width=.1\linewidth]{results/figures_cropped/1twist_MDS.png}&
			\includegraphics[width=.1\linewidth]{results/figures_cropped/1twist_Isomap.png}&
			\includegraphics[width=.1\linewidth]{results/figures_cropped/1twist_tSNE.png}&
			\includegraphics[width=.1\linewidth]{results/figures_cropped/1twist_UMAP.png}&
			\includegraphics[width=.1\linewidth]{results/figures_cropped/1twist_TopoMap.png}&
			\includegraphics[width=.1\linewidth]{results/figures_cropped/1twist_TopoAE.png}&
			\includegraphics[width=.1\linewidth]{results/figures_cropped/1twist_Carriere.png}&
			\includegraphics[width=.1\linewidth]{results/figures_cropped/1twist_TopoAE++.png}\\

			& $\metwasser(\inputPointCloud,\latentPointCloud)$ & 1.1e+00 & 2.2e-01 & 1.9e+01 & 3.3e+00 & 1.5e+01 & 1.8e-01 & 1.6e-01 & \textbf{1.8e-05} & \underline{1.7e-03}\\
			& $\metdistor(\inputPointCloud,\latentPointCloud)$ & \underline{2.2e-01} & \textbf{2.1e-01} & 2.4e+00 & 3.9e+00 & 2.8e+00 & 8.2e-01 & 3.8e-01 & 7.3e-01 & 7.8e-01\\
			& $\timing$ & 0.00049 & 0.073 & 0.0072 & 0.20 & 2.3 & 0.74 & 2.2 & 6.3 & 2.5\\
			
			\hline \raisebox{1.1cm}{\multirow{4}{*}{\rotatebox{90}{\datakfour{} $(n=300)$}}} &
			\raisebox{-.5mm}{\includegraphics[width=.1\linewidth]{data/K4.png}} &
			\includegraphics[width=.1\linewidth]{results/figures_cropped/K4_PCA.png}&
			\includegraphics[width=.1\linewidth]{results/figures_cropped/K4_MDS.png}&
			\includegraphics[width=.1\linewidth]{results/figures_cropped/K4_Isomap.png}&
			\includegraphics[width=.1\linewidth]{results/figures_cropped/K4_tSNE.png}&
			\includegraphics[width=.1\linewidth]{results/figures_cropped/K4_UMAP.png}&
			\includegraphics[width=.1\linewidth]{results/figures_cropped/K4_TopoMap.png}&
			\includegraphics[width=.1\linewidth]{results/figures_cropped/K4_TopoAE.png}&
			\includegraphics[width=.1\linewidth]{results/figures_cropped/K4_Carriere.png}&
			\includegraphics[width=.1\linewidth]{results/figures_cropped/K4_TopoAE++.png}\\

			& $\metwasser(\inputPointCloud,\latentPointCloud)$ & 2.7e-01 & 2.8e-01 & 3.8e-01 & 4.7e+01 & 3.9e-01 & 1.4e-01 & 6.5e-02 & \textbf{7.9e-04} & \underline{2.4e-02}\\
			& $\metdistor(\inputPointCloud,\latentPointCloud)$ & \underline{2.7e-01} & \textbf{1.8e-01} & 5.1e-01 & 1.6e+01 & 9.1e+00 & 5.7e-01 & 1.0e+00 & 6.8e-01 & 3.4e-01\\
			& $\timing$ & 0.00048 & 0.22 & 0.026 & 0.46 & 2.4 & 1.7 & 11 & 63 & 6.2\\
			\hline
		\end{tabular}
		}
	}
	\caption{Comparison of DR methods on three synthetic 3D point clouds (see
	\autoref{sec:test_data} for a description), along with the metric distortion
	$\metdistor(\inputPointCloud,\latentPointCloud)$ between the input $\inputPointCloud$ and
	its planar embedding $\latentPointCloud$, the Wasserstein distance
	$\metwasser(\inputPointCloud,\latentPointCloud)=
	\calw_2\bigl(\dgmrips{1}(\inputPointCloud),\dgmrips{1}(\latentPointCloud)\bigr)$
	between their respective 1-dimensional persistence diagrams, and the running
	time $t$ in seconds. The best value for an indicator is written in bold, the second best is underlined.
	In the second line, a generator of the most persistent \PH{1} pair in the high-dimensional input
	is projected in the 2D embedding in
	transparency and slightly smoothed for visualization purposes (to
	distinguish it from the point cloud). The generator color map depicts its
	arc-length parameterization. In the third line, the input -- that is sampled
	around the edges of a tetrahedron in 3D -- features 3 significantly persistent \PH{1}
	pairs, and a generator for each is represented both in the input and in the 2D
	embeddings with a specific color. Quantitatively, our approach (TopoAE++) generates
	planar embeddings with competitive Wasserstein distances. Qualitatively, it produces
	planar embeddings yielding less crossings of the projected high-dimensional
	persistent generators (colored curves), thereby producing visualizations that
	depict more faithfully the topological handles present in high dimensions.}
	\label{fig:tableSyntheticData}
\end{figure*}


\section{Introduction}
\IEEEPARstart{D}{imensionality} reduction (DR)~\cite{borg97,dimensionReductionBook}
is a fundamental tool in data science for apprehending high-dimensional data.
Such datasets are typically represented as a set of $n$ points $\inputPointCloud = \{x_0, x_1, 
\dots, x_{n - 1}\}$ living in a high-dimensional Euclidean space 
$\mathbb{R}^{\highDim}$. Then, the goal of DR is to compute an embedding 
$\latentPointCloud = \projection(\inputPointCloud)$ of this input point cloud 
in a lower-dimensional Euclidean space $\mathbb{R}^{\lowDim}$, 
with $\lowDim \ll \highDim$. This process can be motivated in practice by 
the need to either \emph{(i)} simplify the input data (to facilitate further 
processing and analysis) or to \emph{(ii)} generate a visualization of the 
data that can be interpreted by a human observer (e.g., $\lowDim=2$ yields planar views of the input point cloud).


Unless $\inputPointCloud$ is already \emph{flat} (e.g., living in an
$\lowDim$-dimensional hyperplane), its projection to $\mathbb{R}^{\lowDim}$
will necessarily involve a form of \emph{distortion}. Depending on the
application needs, several distortion criteria have been introduced in the
literature.
Then, a vast corpus of DR algorithms~\cite{surveyDimensionReduction2,
surveyDimensionReduction1, NonatoA19} have been proposed, either
\emph{(i)} to minimize a specific distortion criterion,
or \emph{(ii)} to preserve certain geometrical features (thereby
indirectly minimizing the associated distortion criterion).
For instance, the seminal multidimensional scaling algorithm~\cite{dimensionReductionBook} aims at minimizing metric distortion,
which is achieved in practice by trying to preserve pairwise distances.

In many use-cases, it may be desirable to preserve \emph{topological}
characteristics when projecting down to a visual space. For instance, the
preservation of the \emph{clusters} present in high-dimensions enables a quick
visual interpretation of the main trends in the dataset.
\emph{TopoMap}~\cite{doraiswamy2020topomap}, for example, guarantees that, after its
projection, the hierarchical clustering of the dataset is identical 
in high and low dimensions. This provides users with strong confidence 
regarding the structural interpretation of the projected data. 
Technically, the preservation of hierarchical clusters
relates to the preservation of the $0$-dimensional persistent homology (\PH{0}) of the Rips filtration of $\inputPointCloud$ (see \autoref{sec:persistentHomology} for  a 
technical description). 
In this context, several topology-aware DR 
methods~\cite{moor2020topological, doraiswamy2020topomap,
trofimov2023learning} have been proposed, but these mostly focused in 
practice on preserving PH$^0$, i.e., the data point clusters (see
\autoref{sec:relatedWorkTopology}).

In this work, we investigate the preservation of more 
advanced topological features than clusters (captured by PH$^0$). In 
particular, we focus on the preservation of the cyclic structures,
the \emph{topological handles}
captured by PH$^1$, when projecting to a planar visual space (i.e.,
$\lowDim = 2$). The presence of such cycles in the input high-dimensional point 
cloud can imply the existence of periodic phenomena in the data, or the 
possibility of multiple paths between clusters (e.g., multiple sequences
of transition states).
As showcased later in
\autoref{sec:results:analysis},
% (when discussing \autoref{fig:tableRealData}),
the preservation of
such cyclic structures can be of paramount importance for the visual
interpretation of real-life high-dimensional datasets.

However, preserving \PH{1} while projecting to the plane is a 
significantly more challenging problem than preserving \PH{0}.
First, from a computational point of view, the $0$-dimensional case is a very 
specific instance of PH computation, which can be efficiently computed
 with a union-find data structure~\cite{cormen}
 (with a runtime complexity that is virtually linear with the number of
considered edges in the Rips complex).
As of dimension $1$, generic PH computation algorithms need to be considered, with a worst case runtime complexity that is cubic with the number
of considered simplices.
Second, as discussed by several authors~\cite{moor2020topological, doraiswamy2020topomap}, PH$^0$ preservation can be obtained by finding a planar layout of $\inputPointCloud$ which preserves its minimum spanning tree (MST).
Since trees can always be embedded in the plane, the existence of (at least) one 
solution (i.e., a projection preserving \PH{0}) is guaranteed.
Unfortunately, this observation no longer holds for higher dimensional PH.
As discussed later in \autoref{sec:results:analysis},
% (and illustrated
% in \autoref{fig:tableK5}),
it is easy to design examples of
high-dimensional point clouds whose \PH{1} cannot be preserved through planar
projections (e.g., a dense sampling of a non-planar
graph, see \autoref{fig:tableK5}).
% For instance, a non-planar graph such as the  complete $K_5$ graph
% cannot be projected to the plane without edge crossing.
% This implies that an input point cloud densely sampling this graph in 3D (see \autoref{fig:tableK5}) cannot be projected to the plane while preserving
% \PH{1}, extra cycles would be created. %(the graph's edge crossings in the plane would result in the creation of
%extra cycles in the Rips filtration of the point cloud).
The existence of such cases, for which an exact preservation of
\PH{1} cannot be obtained,
illustrates the difficulty of the problem, and the need for efficient optimization methods for preserving \PH{1} as accurately as possible, which is the topic of this paper.


