
Specifically, this paper revisits the \emph{Topological Autoencoders} 
(TopoAE), a prominent technique in topology-aware dimensionality reduction,
which we extend into TopoAE++, for the accurate visualization of the cyclic
patterns present in the data. 
We make the following new contributions:

\begin{enumerate}
\item \emph{A novel theoretical analysis of TopoAE's loss
				($\ltopoae{0}$):}
\begin{itemize}
	%  [leftmargin=1cm]
	\item We show that $\ltopoae{0}$ is an upper bound of the
		Wasserstein distance between the $0$-dimensional persistence diagrams in high
		and low dimensions, with regard to the Rips filtration. In particular, when $\ltopoae{0} = 0$,
		the persistence pairs are identical in high and low
		dimensions for the $0$-dimensional persistent homology (\PH{0}) of
		the Rips filtration.
	
	\item We provide a counter example showing that the above property does not hold for a naive extension of TopoAE to $1$-dimensional persistent homology (\PH{1}), with a loss noted $\ltopoae{1}$.
\end{itemize}


 \item \emph{A generalization of TopoAE's loss for $1$-dimensional
persistent homology (\PH{1}):} We introduce a new loss called
\emph{cascade distortion} (CD), noted $\lcascae{1}$, which addresses the above
counter example. This term favors an isometric embedding of the $2$-chains
filling persistent $1$-cycles. We show that for
\PH{0} this loss generalizes TopoAE's loss (i.e., $\lcascae{0} = \ltopoae{0}$).
Extensive experiments show the practical performance
of this loss for the accurate preservation of
generators through the projection.
 \item \emph{An efficient computation algorithm for TopoAE++:}
     To accelerate runtime, we provide a new,
fast algorithm for the exact computation of \PH{} for Rips filtrations in the
plane. This geometric algorithm relies on the
fast identification of cycle-killing triangles via local minmax triangulations of
the relative neighborhood graph, leveraging duality
for the efficient computation of \PH{1}. We believe this contribution to be of independent interest.
 \item \emph{Implementation:} We provide a C++ implementation of our
algorithms that can be used for reproducibility.
\end{enumerate}

