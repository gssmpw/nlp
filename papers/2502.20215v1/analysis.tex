\section{Topological Autoencoder Analysis}
\label{sec:analysis}

In this section we provide a novel theoretical
analysis of the TopoAE loss function~\cite{moor2020topological} and its
link with the Wasserstein distance between the persistence diagrams in high and
low dimensions. Our aim is to better understand its limitations and the
configurations that may cause them.

\subsection{Relation to the Wasserstein distance}
\label{sec:topoae-wasserstein}

We first show below that the topological regularization term $\ltopoae{0}$ (\autoref{eq:topoae0})
upper bounds the $L_2$-Wasserstein distance $\calw_2$ between both 0-dimensional persistence diagrams.
\begin{lemma}
	\label{lemma:TopoAE0_bound}
	For any point clouds $\inputPointCloud$ and $\latentPointCloud$ of equal 
	size, we have the following inequality:
	\[\calw_2\bigl(\dgmrips{0}(\inputPointCloud),
	\dgmrips{0}(\latentPointCloud)\bigr)^2\leq\ltopoae{0}(
	\inputPointCloud,\latentPointCloud).\]
	Besides, under general position hypothesis (unique pairwise distances), if $\ltopoae{0}(\inputPointCloud,\latentPointCloud)=0$, then $\dgmrips{0}(\inputPointCloud)=\dgmrips{0}(\latentPointCloud)$ and the \PH{0} pairs are the same, i.e. $\emst(\inputPointCloud)=\emst(\latentPointCloud)$.
	\end{lemma}
	\begin{proof}
	Let $\calk$ be the (abstract) 1-dimensional simplicial complex
	$\calk=\emst(\inputPointCloud)\cup\emst(\latentPointCloud)$ (see \autoref{fig:wassersteinBound}). Let
	$\delta_\inputPointCloud:\calk\rightarrow\bbr$ (resp.
	$\delta_\latentPointCloud:\calk\rightarrow\bbr$) be
	the simplex diameter function in $\inputPointCloud$ (resp. in
	$\latentPointCloud$).
	\begin{figure}
		\centering
		\def\svgwidth{\linewidth}
		\input{figures/proofWassersteinBound.pdf_tex}
		\caption{Notations for the proof of \autoref{lemma:TopoAE0_bound}. In this example
		$\emst(\inputPointCloud)=\bigl\{\{v_1,v_2\},\{v_2,v_3\},\{v_3,v_4\}\bigr\}$ (in red) and
		$\emst(\latentPointCloud)=\bigl\{\{v_1,v_2\},$\allowbreak$\{v_2,v_3\},$\allowbreak$\{v_2,v_4\}\bigr\}$ (in blue). Therefore
		$\calk=\bigl\{\{v_1,v_2\},$\allowbreak$\{v_2,v_3\},$\allowbreak$\{v_2,v_4\},$\allowbreak$\{v_3,v_4\}\bigr\}$ (in yellow). Here $\ltopoae{0}(\inputPointCloud, \latentPointCloud)$ evaluates as
		$\bigl(\delta_X(\{v_2,v_4\})-\delta_Z(\{v_2,v_4\})\bigr)^2+
		 \bigl(\delta_X(\{v_3,v_4\})-\delta_Z(\{v_3,v_4\})\bigr)^2
		=2(\sqrt{2}-1)^2$.
		Note that this provides an example where
		$\ltopoae{0}(\inputPointCloud, \latentPointCloud)>0$ but
		$\dgmrips{0}(\inputPointCloud)=\dgmrips{0}(\latentPointCloud)$, i.e. the
		converse of the stated implication (\autoref{lemma:TopoAE0_bound})
is not true.}
		\label{fig:wassersteinBound}
	\end{figure}
	Because the minimum spanning tree contains exactly all the edges that kill a 0-dimensional persistent homology class, and because $\emst(\inputPointCloud)\subset\calk$ and $\emst(\latentPointCloud)\subset\calk$, we have $\dgmrips{0}(\inputPointCloud)=\dgm{0}(\delta_\inputPointCloud)$ and $\dgmrips{0}(\latentPointCloud)=\dgm{0}(\delta_\latentPointCloud)$.
	Hence, the following inequalities:
	\begin{align*}		
		&\calw_2\bigl(\dgmrips{0}(\inputPointCloud),\dgmrips{0}(\latentPointCloud)\bigr) ^ 2 = \calw_2\bigl(\dgm{0}(\delta_\inputPointCloud),
		\dgm{0}(\delta_\latentPointCloud)\bigr)^2 \\
		&\leq\sum\limits_{\substack{\sigma\in\calk \\ \dim(\sigma)\in\{0,1\}}}|\delta_\inputPointCloud(\sigma)-\delta_\latentPointCloud(\sigma)|^2 \text{\begin{tabular}{ll}
				(by Cellular Wasserstein \\
				stability theorem~\cite{skraba_wasserstein_2023})
		\end{tabular}}\\
		&=\sum\limits_{\substack{\sigma\in\calk \\ \dim(\sigma)=1}}|\delta_\inputPointCloud(\sigma)-\delta_\latentPointCloud(\sigma)|^2 \text{ (Rips filtration)}\\
		&\leq\sum\limits_{\substack{\sigma\in\emst(\inputPointCloud) \\ \dim(\sigma)=1}}|\delta_\inputPointCloud(\sigma)-\delta_\latentPointCloud(\sigma)|^2 + \sum\limits_{\substack{\sigma\in\emst(\latentPointCloud) \\ \dim(\sigma)=1}}|\delta_\inputPointCloud(\sigma)-\delta_\latentPointCloud(\sigma)|^2\\
		&=\text{\small{$\|A^\inputPointCloud[\critical{0}(\inputPointCloud)]-A^\latentPointCloud[\critical{0}(\inputPointCloud)]\|_2^2+\|A^\latentPointCloud[\critical{0}(\latentPointCloud)]-A^\inputPointCloud[\critical{0}(\latentPointCloud)]\|_2^2$}}\\
		&=\ltopoae{0}(\inputPointCloud,\latentPointCloud).
	\end{align*}
	In particular, when $\ltopoae{0}(\inputPointCloud,\latentPointCloud)=0$, then $\dgmrips{0}(\inputPointCloud)=\dgmrips{0}(\latentPointCloud)$. In that case, general position hypothesis (unique pairwise distances) on $\inputPointCloud$ and $\latentPointCloud$ -- that guarantees the uniqueness of their respective $\emst$ -- implies $\emst(\inputPointCloud)=\emst(\latentPointCloud)$, hence the same \PH{0} pairs.
\end{proof}

This result is, to our knowledge, the first theoretical
guarantee that justifies
using the original, $0$-dimensional version of the TopoAE
loss function (\autoref{eq:topoae0}) as a regularization term for preserving
\PH{0}. In particular, if this term is zero, we know that the representation
$\latentPointCloud$ has the same \PH{0} -- and therefore the same 0-th
persistence diagram -- than the input $\inputPointCloud$.

\subsection{Counter-example for higher dimensional PH}
Unfortunately, \autoref{lemma:TopoAE0_bound} does not generalize to higher-dimensional \PH{}.
Indeed, as soon as we add the edges that create or destroy \PH{1}
classes (\autoref{sec:topologicalAutoencoders}), a
zero $\ltopoae{1}$ loss (\autoref{eq:topoae}) no longer implies
equality of the persistence diagrams:
\begin{equation}
	\label{eq:counter-ex}
	\ltopoae{1}(\inputPointCloud,\latentPointCloud)=0 \centernot\implies\left\{
	\begin{array}{l}
		\dgmrips{0}(\inputPointCloud)=\dgmrips{0}(\latentPointCloud)\\
		\dgmrips{1}(\inputPointCloud)=\dgmrips{1}(\latentPointCloud)
	\end{array}
	\right..
\end{equation}

\begin{figure}
	\centering
	\def\svgwidth{\linewidth}\input{figures/counter-ex.pdf_tex}
	\caption{An example for the non-implication of
	\autoref{eq:counter-ex}. The
	left point cloud $\latentPointCloud$ is in $\bbr^2$ and is a deformed regular
	hexagon; the right one $\inputPointCloud$ is in $\bbr^3$ and its leftmost point
	has been rotated with respect to the dotted yellow axis with angle $\pi/3$ (see \autoref{appendix:counter-example-data} for raw coordinates).
	Both $\rng$s are pictured in black and the length of their edges are 1.
	The triangle that kills the cycle in $\inputPointCloud$ (resp.
	$\latentPointCloud$) is shown in blue (resp. red) with its longest edge
	highlighted in bold. For $\latentPointCloud$, the remaining cascade edge
	(see \autoref{eq:GCS}) is also shown (its length is 1.623).
	The 0-th persistence diagrams are the same: 
	$\dgmrips{0}(\inputPointCloud)=\dgmrips{0}(\latentPointCloud)=\{(0,1)\text{ with
	multiplicity 5}, (0,+\infty)\}$; but not the 1-th persistence diagrams:
	$\dgmrips{1}(\latentPointCloud)=\{(1,1.819)\}$ and
	$\dgmrips{1}(\inputPointCloud)=\{(1,1.732)\}$. However,
	the critical edges considered in $\ltopoae{1}$ (i.e., $\rng$ edges
	and 1-cycle killing edges, highlighted in yellow) have the
	same length in both $\inputPointCloud$ and $\latentPointCloud$, therefore
	$\ltopoae{1}(\inputPointCloud,\latentPointCloud)=0$. On the contrary, the two
	additional edges considered in the 1-dimensional \emph{cascade distortion}
	(highlighted in purple, see \autoref{eq:cascae})
	are longer in $\latentPointCloud$ than in $\inputPointCloud$,
	hence $\lcascae{1}(\inputPointCloud,\latentPointCloud)>0$.}
	\label{fig:counter-example}
\end{figure}

\autoref{fig:counter-example} provides a detailed counter-example for two point
clouds $\inputPointCloud\subset\bbr^3$ and $\latentPointCloud\subset\bbr^2$
which illustrates this non-implication.
This example shows that preserving the length
of topologically critical edges is not enough to guarantee the preservation
of $\PH{\homologyDim}$ when $\homologyDim\geq1$.
