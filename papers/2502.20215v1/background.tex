\section{Background}
\label{sec:background}

This section presents the technical background to our work. We refer the
reader to textbooks~\cite{edelsbrunner_computational_2010,
zomorodian_computational_2010} for comprehensive introductions to computational
topology.

\subsection{Geometric structures}
\label{sec:geometry}

We introduce below geometric concepts related to persistent homology (\autoref{sec:geometricPH}) that are relevant to our algorithm.

\begin{figure}
	\centering
	\def\svgwidth{\linewidth}
	\input{figures/geometric_with_mml_beside.pdf_tex}
	\caption{Left: geometric graphs for a point cloud $\inputPointCloud$ in the plane. Its $\emst$ is
depicted in red. The edges from the $\rng$ which are not in the $\emst$ are shown in continuous yellow, while those which are in the $\ug$ but not in the $\rng$ are shown in dashed yellow. The
remaining Delaunay edges are shown in gray.
Three lenses are shown:
the purple one is devoid of points of $\inputPointCloud$, therefore the associated yellow edge belongs to the $\rng$;
the blue one -- associated with the dashed yellow edge -- is devoid of points of the link of that edge but contains another point of $\inputPointCloud$ (the blue one), therefore this edge is in the $\ug$ but not in the $\rng$;
the green one contains the green point and the associated gray edge is not in $\rng$ nor in $\ug$.
Right: within the highlighted bottom $\rng$-polygon, a $\mml$ triangulation has
replaced the Delaunay triangulation, with its longest edge highlighted in
black. Note that for the two other polygons, the Delaunay triangulation is
already a $\mml$ triangulation within these polygons.}
	\label{fig:geometric_graphs}
\end{figure}

We consider in this section a point cloud $\inputPointCloud$ containing $n$ points, in a
$d$-dimensional space $\bbr^d$ endowed with the ambient Euclidean metric. The
lens of an edge $(a,b)\in \inputPointCloud^2$ is defined as the points of
$\inputPointCloud$ whose distance to either $a$ or $b$ is smaller than $\dist{a}{b}$.
The relative neighborhood graph $\rng(\inputPointCloud)$
\cite{toussaint_relative_1980} is the graph on $\inputPointCloud$ that features
every edge with an empty lens, i.e., $\rng(\inputPointCloud)=\{e\in
\inputPointCloud^2\mid\lens(e)=\varnothing\}$.
The $\rng$ is known to be a subset of the Delaunay triangulation
\cite{toussaint_relative_1980}. In the plane,
it can be computed in time $\calo(n\log n)$~\cite{supowit_relative_1983,
jaromczyk_note_1987, lingas_linear-time_1994}.
The Urquhart graph $\ug(\inputPointCloud)$ \cite{urquhart_algorithms_1980} is
another subgraph of the Delaunay triangulation that is obtained by removing
the longest edge of each triangle. It can also be thought as the
subgraph of $\del(\inputPointCloud)$ that only keeps edges whose lens does not
contain any points of its link%
\footnote{In an abstract simplicial complex $\calk$ (i.e., a family of sets that
is closed under taking subsets, where sets represent simplices and their
subsets represent their faces), the link of a simplex $\sigma\in\calk$ is the set
$\{\tau\in\calk\mid\sigma\cap\tau=\varnothing\text{ and }\sigma\cup\tau\in\calk\}$.
It provides a notion of neighborhood boundary for $\sigma$.}.
Finally, we write $\emst(\inputPointCloud)$ the Euclidean minimum spanning tree
of $\inputPointCloud$, i.e., the tree spanning $\inputPointCloud$ that minimizes
the sum of the Euclidean distances of its edges. Under general position
hypothesis\footnote{i.e., we suppose that no $d+2$ points lie on a
common $(d-1)$-sphere (for the Delaunay triangulation
\cite{edelsbrunner_computational_2010}), and that pairwise distances are unique
for the uniqueness of the $\emst$, which can be enforced
% in practice
via small
perturbations.},
we have the following inclusions
\cite{toussaint_relative_1980} (\autoref{fig:geometric_graphs}, left, shows a case where the three inclusions are strict):
\begin{equation}
\emst(\inputPointCloud)\subseteq\rng(\inputPointCloud)
\subseteq\ug(\inputPointCloud)\subseteq\del(\inputPointCloud).
\end{equation}

Our algorithm also relies on the minmax length ($\mml$)
triangulations of a planar point cloud $\inputPointCloud\subset\bbr^2$. It is defined as
a triangulation that minimizes the length of its longest edge
(\autoref{fig:geometric_graphs}, right). Such a triangulation can be computed in quadratic time
$\calo(n^2)$ with an algorithm~\cite{edelsbrunner_quadratic_1993}
that breaks down the construction of such a triangulation over
each polygon formed by the $\rng$ and the convex hull of $\inputPointCloud$.

\begin{figure*}
	\begin{tabular}{@{}c@{}@{}c@{}@{}c@{}@{}c@{}@{}c@{}@{}c@{}}
	\includegraphics[width=.166\textwidth]{figures/rips/curve_rips0.00.png}&
	\includegraphics[width=.166\textwidth]{figures/rips/curve_rips0.14.png}&
	\includegraphics[width=.166\textwidth]{figures/rips/curve_rips0.24.png}&
	\includegraphics[width=.166\textwidth]{figures/rips/curve_rips0.34.png}&
	\includegraphics[width=.166\textwidth]{figures/rips/curve_rips0.39.png}&
	\includegraphics[width=.166\textwidth]{figures/rips/curve_rips0.50.png}\\
	$\betti{0}=n,\;\betti{1}=0$&
	$\betti{0}=2,\;\betti{1}=2$&
	$\betti{0}=1,\;\betti{1}=1$&
	$\betti{0}=1,\;\betti{1}=2$&
	$\betti{0}=1,\;\betti{1}=1$&
	$\betti{0}=1,\;\betti{1}=0$\\
	\end{tabular}
	\caption{Rips complexes of the same planar point cloud $\inputPointCloud$,
	for increasing diameter thresholds $\diameterThreshold$
	(only the 2-skeletons, i.e., the vertices, edges and triangles, of the
	simplicial complexes are shown). This threshold increase
	induces a sequence of nested simplicial complexes,
	whose topology varies along the process. From left to right, the number of
	connected components is successively $n$, then $2$, then $1$,
	while the number of handles is successively $0$, $2$, $1$, $2$, $1$ and $0$.}
	\label{fig:rips}
\end{figure*}

We conclude this section with a geometric construction that is often used in the topological data analysis of point clouds:
the Rips complex of $\inputPointCloud$ with diameter threshold $\diameterThreshold\geq0$
is the simplicial complex featuring all the simplices $\sigma\subset
\inputPointCloud$ whose diameter
$\delta(\sigma)=\max\limits_{x,y\in\sigma}\lVert x-y\rVert_2$ is smaller than
$\diameterThreshold$ (i.e., every edge included in $\sigma$ must be shorter than
$\diameterThreshold$) (see \autoref{fig:rips}):
\begin{equation}
	\rips_r(\inputPointCloud)=\{\sigma\subset\inputPointCloud\mid \delta(\sigma)\leq
	\diameterThreshold\}.
\end{equation}
We write $\calk^\homologyDim=\{\sigma\in\calk\mid\dim\sigma=\homologyDim\}$ the set of $\homologyDim$-simplices of a simplicial complex $\calk$, and $\skeleton{\calk}{k}=\{\sigma\in\calk\mid\dim\sigma\leq\homologyDim\}$ its $\homologyDim$-\emph{skeleton}.
Then, $\rips_r^{(\homologyDim)}(\inputPointCloud)$ is the Rips complex of
$\inputPointCloud$ with diameter
% threshold
$\diameterThreshold\geq0$ and maximum
simplex dimension $\homologyDim$.

\subsection{Persistent homology}
\label{sec:persistentHomology}

Persistent homology, introduced independently by several authors
\cite{B94,frosini99,robins99,edelsbrunner02}, focuses on the evolution of the
topological properties of an increasing sequence of $m$ nested
simplicial complexes $\bbk=(\filt{0},\filt{1},\ldots,\filt{m-1})$,
which is called a \textit{filtration}. Persistent homology can be applied on
various filtrations, like the sublevel sets of a scalar field on a mesh. However
a filtration of particular interest for point clouds is the Rips filtration
(\autoref{fig:rips}), which consists in the sequence of Rips complexes with an
increasing threshold $r$. More precisely, the Rips filtration of $X$ can be
written as
$\bigl(\rips_{0}(X),\rips_{r_1}(X),\ldots,\rips_{r_{m-1}}(X),
\rips_{\infty}(X)\bigr) $ with increasing $r_i$'s such that the Rips complex
changes between $r_i$ and $r_{i+1}$ (i.e.
$\rips_{r_i}(X)\subsetneq\rips_{r_{i+1}}(X)$).

\textit{Persistent homology groups} for a filtration $\bbk$ are algebraic constructions that contain \textit{persistent homology classes} and that are defined from homology groups (see \autoref{appendix:homology} for a brief summary or~\cite{edelsbrunner_computational_2010} for a more complete introduction).
Informally, these classes represent topological features (that can be
geometrically interpreted as clusters for \PH{0}, cycles for \PH{1}, voids for
the \PH{2}\ldots) which appear at some point in the filtration and may disappear
later. Formally, for $0\leq i\leq j\leq m$, the $\homologyDim$-th persistent
homology group $\PHG{i}{j}$ is the image of the $\homologyDim$-th homology group
$\HG{i}$ by the inclusion morphism $\morphism{i}{j}:\HG{i}\rightarrow\HG{j}$.
Its classes can be understood as those which already exist in $\HG{i}$ and that
still exist in $\HG{j}$. Therefore, we say that a class $\PHC\in\HG{i}$ is
\textit{born} entering $\filt{i}$ when $\gamma\not\in\PHG{i-1}{i}$, i.e., when
it is not the image of a pre-existing class by the inclusion morphism. We also
say that such a class born at $\filt{i}$ \textit{dies} entering $\filt{j}$ when
$\morphism{i}{j-1}(\gamma)\not\in\PHG{i-1}{j-1}$ but
$\morphism{i}{j}(\gamma)\in\PHG{i-1}{j}$, i.e., when $\gamma$ is merged into a
pre-existing class (possibly the 0 class) when going from $\filt{j-1}$ to
$\filt{j}$. In this case, the birth of $\gamma$ is associated with a
$\homologyDim$-simplex $\sigma_i\in\filt{i}$ and its death with
$(\homologyDim+1)$-simplex $\sigma_j\in\filt{j}$. Such a pair $(\sigma_i,
\sigma_j)$ is called a \textit{persistence pair}.
If the filtration is defined by scalar values $f$ on the simplices
(e.g., the diameter $\delta$ for the Rips filtration), the \textit{persistence}
of such a class is the positive value $f(\sigma_j)-f(\sigma_i)$.

In the specific case of a Rips filtration, we can say more simply that for any $\homologyDim$, a \PH{\homologyDim} pair of positive persistence $(\sigma_i,\sigma_j)$ is created (resp. killed) by the longest edge of $\sigma_i$ (resp. $\sigma_j$).
Indeed, a Rips complex is entirely determined by its 1-skeleton since the diameter of any simplex is the length of the longest edge it features (it is a \textit{flag complex}). In the following we refer to these edges as $\homologyDim$-\textit{critical edges}.

\textit{Persistence diagrams} (\autoref{fig:wasserstein}) are concise
topological representations that summarize the persistent homology groups.
More precisely, the $\homologyDim$-th persistence diagram is a 2D
multiset where each $\homologyDim$-dimensional persistence pair $(\sigma_i,
\sigma_j)$ is represented by a point
$\bigl(f(\sigma_i),f(\sigma_j)\bigr)\in\bbr^2$ above
the diagonal.
We write $\dgm{k}(f)$ for the $\homologyDim$-th persistence diagram of the filtration given by a scalar function $f:\calk\rightarrow\bbr$, or directly $\dgmrips{k}(\inputPointCloud)$ for the $\homologyDim$-th persistence diagram of the Rips filtration of $\inputPointCloud$.
Many points clouds have the same persistence diagrams:
in particular, $\dgmrips{k}(\inputPointCloud)$ is invariant to point permutations in $\inputPointCloud$.

Persistence diagrams (of equal homology dimension) are usually compared with metrics born of optimal transport that come with stability properties \cite{cohen2006vines,skraba_wasserstein_2023}.
Let $\cald, \cald'\subset\bbr^2$ be two persistence diagrams to be
compared. We define the \textit{augmented} diagrams as
$\augmented{\cald}=\cald\cup\Delta\cald'$ and
$\augmented{\cald'}=\cald'\cup\Delta\cald$ where $\Delta$ is the map that
projects to
the diagonal, i.e., $\Delta:(b,d)\mapsto(\frac{b+d}{2},
\frac{b+d}{2})$. This ensures that $|\augmented{\cald}|=|\augmented{\cald'}|$.
The $p$-cost $c_p(x,y)$ between two points $x\in\augmented{\cald}$ and
$y\in\augmented{\cald'}$ is defined as 0 if both $x$ and $y$ are on the diagonal
and as $\bbr^2$'s $p$-norm $\|x-y\|_p$ otherwise.
Then, the $L_q$-Wasserstein distance $\calw_q^p$ is:
\begin{equation}
\calw_q^p(\cald,\cald')=
\min\limits_{\psi\in\Psi}\left(\sum\limits_{x\in\augmented{\cald}}c_p\bigl(x,
\psi(x)\bigr)^q\right)^{1/q},
\end{equation}
where $\Psi=\mathrm{Bij}(\augmented{\cald},\augmented{\cald'})$ (\autoref{fig:wasserstein}).
\begin{figure}
	\centering
	\includegraphics[width=.95\linewidth]{figures/wasserstein.pdf}
	\caption{Two point clouds (red and blue) in the plane represented
	with their $\rng$, along with their respective 1-dimensional
	persistence diagrams (right). The edges in
	$\rng\setminus\emst$ are highlighted in bold and yellow.
	The number of non-diagonal points in each diagram is exactly the number of
	$\rng$-polygons in the associated point cloud (3 for the red one, 2 for the blue
	one), and also the number of $\rng\setminus\emst$ edges. The optimal assignment inducing
	the Wasserstein distance $\calw_2$ between them is shown in black.}
	\label{fig:wasserstein}
\end{figure}
Because for any fixed $q$, all $\calw_q^p$ are bi-Lipschitz equivalent
\cite{skraba_wasserstein_2023}, we focus on the distances $\calw_p=\calw_p^p$
with $p<\infty$. In practice, the Wasserstein distances $\calw_p$ can be
approximated thanks to the \textit{auction} algorithm, an asymmetric and generic
approach introduced by Bertsekas~\cite{bertsekas_new_1981} that can be
specialized to persistence diagrams with dedicated data structures to leverage
their geometric structure~\cite{kerber2017geometry}.

\subsection{Persistence homology computation}
\label{sec:persistenceComputation}

Since the introduction of computational PH
\cite{edelsbrunner02}, numerous algorithms and implementations have been
proposed, based either on the boundary matrix reduction approach or on its
geometric interpretation known as the \texttt{PairCells} algorithm~\cite{zomorodian_computational_2010},
which we detail in the following as our
novel loss (introduced in \autoref{sec:newLoss}) is defined relatively to the
\textit{cascade}, an object computed by this algorithm.
Indeed, \texttt{PairCells} computes not only persistent pairs but also
representatives of each \PH{} class for any simplicial complex $\calk$ with its
simplices endowed with distinct scalar values $f(\sigma)$ that are increasing by
inclusion (which induces the filtration of sublevel sets).
For each $k$-simplex $\sigma\in\calk$, it stores a $k$-chain $\cascade(\sigma)$ that is initially equal to $\sigma$, and whose $(k-1)$-boundary $\partial(\cascade(\sigma))$ is, after the execution of the algorithm, a representative of the \PH{} class killed by $\sigma$.
In the following we will call this boundary a \emph{persistent generator}
of the \PH{} class~\cite{Iuricich22, guillou2023discrete}.
In addition, these cascades will enable us in \autoref{sec:newLoss} to find $2$-chains that fill the $1$-cycles.

\texttt{PairCells} (\autoref{algo:pairCells}) takes one $k$-simplex
$\sigma\in\calk$ at a time, by increasing value $f(\sigma)$, and applies the
\texttt{EliminateBoundaries} procedure (\autoref{algo:eliminateBoundaries}) on
it.
This procedure runs a \textit{homologous propagation}\cite{guillou2023discrete} from $\sigma$: the $(k-1)$-boundary $\partial(\cascade(\sigma))$ is expanded step by step, each time selecting its \emph{youngest} $(k-1)$-simplex $\tau$ (i.e., the one with the highest value $f(\tau)$) and adding modulo~2 to $\cascade(\sigma)$ the cascade associated with the partner of $\tau$, if it exists (see \autoref{fig:eliminateBoundaries}).
The procedure stops either when an unpaired simplex $\tau$ is selected, in which
case $\sigma$ and $\tau$ are paired together since $\sigma$ kills the
$(k-1)$-dimensional homology class introduced by $\tau$; or when the boundary
$\partial\bigl(\cascade(\sigma)\bigr)$ becomes empty, in which case we know
that $\sigma$ creates a $k$-dimensional homology class.
This algorithm can be understood as a geometric interpretation of the persistence algorithms based on boundary matrix reduction.

\begin{figure*}
	\def\svgwidth{\linewidth}
	\input{figures/eliminateBoundaries.pdf_tex}
	\caption{Some steps of two executions of the \texttt{\texttt{EliminateBoundaries}} \autoref{algo:eliminateBoundaries} procedure run on the 2-simplices $\sigma_0$ (top line) and $\sigma_1$ (bottom line), corresponding to the creation of two \PH{1} pairs in the previous planar point cloud (with its smallest cycle removed for clarity).
	Unpaired edges -- which are in $\rng(\inputPointCloud)\setminus\emst(\inputPointCloud)$ --
	are depicted in yellow.
	At each step, $\cascade[\sigma]$ is shown in red, and its boundary
$\partial(\cascade[\sigma])$ is highlighted in bold, with its longest edge,
i.e., $\tau=\text{Youngest}\bigl(\partial(\cascade[\sigma])\bigr)$, dotted.
	When it exists, $\cascade\bigl[\partner[\tau]\bigr]$ is shown is blue.
	A persistent pair $(\tau, \sigma)$ is created when $\partner[\tau]=\varnothing$, i.e., when $\tau$ coincides with a yellow unpaired edge (rightmost images).
	In both executions, $\rips_{\delta(\sigma)}^1$ is shown in light gray.
	}
	\label{fig:eliminateBoundaries}
\end{figure*}

In practice, if we are not interested in the cascades themselves, it can be
faster to only store the boundary of each simplex's cascade
$\partial\bigl(\cascade(\sigma)\bigr)$
%\in\calb_{k-1}(\calk)$
(instead of the
whole cascade itself) and to perform a modulo~2 addition operation on those
boundaries when expanding a cascade
(\autoref{algo:eliminateBoundaries:propagation} of
\autoref{algo:eliminateBoundaries})~\cite{guillou2023discrete}. Another
acceleration is the preliminary pairing of the \textit{apparent pairs}, pairs of
simplices with the same filtration
value~\cite{guillou2023discrete,bauer_ripser_2021}
(some of them appear in \autoref{fig:eliminateBoundaries} as these dotted red edges whose partner's cascade is a single triangle).

\begin{algorithm}
	\DontPrintSemicolon
	\label{algo:pairCells}\caption{\texttt{PairCells}}
	\KwIn{$\calk$}
	\For{$\sigma\in\calk$ by increasing $f(\sigma)$}{
		$\partner[\sigma]\gets\varnothing$\;
		$\cascade[\sigma]\gets\sigma$\;
		\texttt{EliminateBoundaries}($\sigma$)\;
		\If{$\partial(\cascade[\sigma])\neq\varnothing$} {
			$\tau\gets\text{Youngest}\bigl(\partial(\cascade[\sigma])\bigr)$\;
			$\partner[\sigma]\gets\tau$\;
			$\partner[\tau]\gets\sigma$\;
		}
	}
\end{algorithm}

\begin{algorithm}
	\DontPrintSemicolon
	\label{algo:eliminateBoundaries}\caption{\texttt{EliminateBoundaries}}
	\KwIn{$\sigma\in\calk$}
	\While{$\partial(\cascade[\sigma])\neq\varnothing$}{
		$\tau\gets\text{Youngest}\bigl(\partial(\cascade[\sigma])\bigr)$\;
		\eIf{$\partner[\tau]=\varnothing$} {
			return\;
		}{
$\cascade[\sigma]\gets\cascade[\sigma]+\cascade\bigl[\partner[\tau]\bigr]$\;
\label{algo:eliminateBoundaries:propagation}
		}
	}
\end{algorithm}

\subsection{Geometric interpretation of persistent homology}
\label{sec:geometricPH}

Recent boundary matrix reduction-based open-source implementations include
\textit{Gudhi}~\cite{maria2014gudhi}, \textit{PHAT}~\cite{bauer2017phat} or
\textit{Ripser}~\cite{bauer_ripser_2021}. In particular, \textit{Ripser} is
specialized in the PH computation of Rips filtrations (given by point clouds or
distance matrices) yielding much faster and more memory-efficient execution than
previous implementations.
% on these filtrations.
However, it is sometimes possible to leverage the Euclidean embedding
of point clouds for which we want to compute the PH of the Rips filtration to
speed up the computation. In this section, we highlight some known results that
link \PH{0} and \PH{1} of the Rips filtration to some geometric structures
defined in \autoref{sec:geometry}.

The first result is that the set of edges that destroy
a \PH{0} class of the Rips filtration (i.e., that reduce the
number of connected components by 1) is exactly the set of edges of the
Euclidean minimum spanning tree (see
\cite{doraiswamy2020topomap} for example for a proof). It allows to compute
the \PH{0} of the Rips filtration by constructing the minimum spanning tree of
the input point cloud, e.g., with
Kruskal's algorithm based on the union-find
data structure~\cite{cormen}, in time complexity
$\calo\left(n^2\alpha(n^2)\right)$ where $\alpha$ is the inverse of the
Ackermann function. This approach is largely used in practice, e.g. in
\textit{Ripser}~\cite{bauer_ripser_2021}.

On the contrary, the edges in the complementary of $\emst(X)$ are the edges that
create a \PH{1} class~\cite{skraba2017randomly}, possibly with zero
persistence -- in which case this class dies immediately with a triangle of
same diameter.
More recently, Koyama et al.~\cite{koyama2023reduced} showed that the edges of
$\rng(X)\setminus\emst(X)$ are exactly the edges that create a \PH{1}
class of positive persistence.
A corollary
is that the quantity $|\rng(X)\setminus\emst(X)|$ is exactly the number of \PH{1} classes with positive persistence (see
\autoref{fig:wasserstein}).
Koyama et al. also introduced \textit{Euclidean PH1}, a method that specializes
the \PH{1} computation for Rips filtrations of
point clouds in $\bbr^2$ and $\bbr^3$, using the above geometric result.
Compared to other Rips PH algorithm like \textit{Ripser}, they improve both
running time and memory consumption in this specific case thanks to the
construction from the $\rng$ of a reduced Vietoris--Rips complex that features
far fewer triangles than the original one, but that has the same \PH{0} and
\PH{1}. This permits to get a much smaller boundary matrix to be reduced.

\subsection{Topological Autoencoders}
\label{sec:topologicalAutoencoders}

Our method builds on \textit{Topological Autoencoders}
(TopoAE)~\cite{moor2020topological}, a dimension reduction technique that
combines an autoencoder with a topological feature preservation loss.
We therefore briefly summarize this approach below.

Let $\calx$ be the input, high-dimensional space and $\calz$ the latent, low-dimensional space. For visualization purposes, in the following we take $\calz=\bbr^2$. An \textit{autoencoder}~\cite{hinton_reducing_2006} is a composition $\dec\circ\enc$ of an \textit{encoder} $\enc:\calx\rightarrow\calz$ and a \textit{decoder} $\dec:\calz\rightarrow\calx$. We write $Z=\enc{X}\in\calz$ the representation of an input vector $X\in\calx$ and $\Tilde{X}=\dec(\enc{X})$ its reconstruction.
In practice, $\enc$ and $\dec$ are neural networks, e.g., with one or several fully connected layer(s), possibly with one or several additional convolutional layer(s) if $\calx$ represents an image space.
\begin{figure}
	\centering
	\def\svgwidth{.9\linewidth}
	\input{figures/TopologicalAutoencoder.pdf_tex}
	\caption{An autoencoder with reconstruction term $\call_r$ and topological regularization term $\call_t$. Here both the encoder and the decoder are fully-connected networks with 1 hidden layer.}
	\label{fig:autoencoder}
\end{figure}
The
% basic
reconstruction loss
% function
of an autoencoder is simply the MSE between an
input and its reconstruction:
\begin{equation}
	\call_r=\|\dec(\enc X)-X\|^2.
\end{equation}

For constraining autoencoders to preserve topological structures, one can add to
this reconstruction term a topological regularization term $\call_t$ which
compares some topological abstraction of the input and its low-dimensional
representation (\autoref{fig:autoencoder}).
Moore et al.~\cite{moor2020topological} discussed
a regularization term considering directly the lengths of
carefully selected edges:
\begin{equation} \label{eq:topoae}
	\begin{split}
\ltopoae{d}=&\|A^\inputPointCloud[\critical{d}(\inputPointCloud)]
-A^\latentPointCloud[\critical{d}(\inputPointCloud)]\|_2^2 +\\
		&\|A^\latentPointCloud[\critical{d}(\latentPointCloud)]-A^\inputPointCloud[\critical{d}(\latentPointCloud)]\|_2^2
	\end{split}
	,
\end{equation}
where $A^\inputPointCloud[\cdot]$ and $A^\latentPointCloud[\cdot]$ denote
evaluating the distance matrix of $\inputPointCloud$ and $\latentPointCloud$
respectively, and where $\critical{d}(\inputPointCloud)$ (resp.
$\critical{d}(\latentPointCloud)$) denotes all $\homologyDim$-critical edges,
for any $0\leq\homologyDim\leq d$, in $\inputPointCloud$ (resp.
$\latentPointCloud$), i.e., the edges that either create or kill a
\PH{\homologyDim} class of positive persistence (see
\autoref{sec:persistentHomology}).

In practice however, the method is only documented
for \PH{0}, so that
$\critical{0}(\inputPointCloud)=\emst(\inputPointCloud)$ and
$\critical{0}(\latentPointCloud)=\emst(\latentPointCloud)$ (see
\autoref{sec:geometricPH}).
In that case, the regularization term rewrites as:
\begin{equation} \label{eq:topoae0}
	\begin{split}
		\ltopoae{0}=&\|A^\inputPointCloud[\emst(\inputPointCloud)]-A^\latentPointCloud[\emst(\inputPointCloud)]\|_2^2+\\
		&\|A^\latentPointCloud[\emst(\latentPointCloud)]-A^\inputPointCloud[\emst(\latentPointCloud)]\|_2^2
	\end{split}.
\end{equation}
The idea of this loss function is to force the length of topologically
critical edges in the input to be preserved in the representation (first line of 
\autoref{eq:topoae0}), and reciprocally (second line). More precisely, the 
second term tries to push away in $\calz$ the ends of the edges in 
$\emst(\latentPointCloud)\setminus\emst(\inputPointCloud)$ that are, 
intuitively, too short in $\calz$. The hope is that at the end of the 
optimization, the length of both $\emst$'s edges match in high and low 
dimension, inducing the same 0-th persistence diagrams 
(which we show in \autoref{sec:topoae-wasserstein}).

Finally, the weights of the autoencoder are optimized iteratively to minimize the value of the loss function. Note that this involves computing the required critical edges of the current representation $\latentPointCloud$ at each iteration.
