\section{Cascade Distortion}
\label{sec:newLoss}

This section presents a new penalty term called \emph{cascade distortion} (CD) which addresses the above counter-example.
This term favors an isometric embedding of the $2$-chains filling persistent
$1$-cycles
(i.e., that are computed thanks to the \texttt{PairCells}
algorithm), leading to a more accurate geometrical representation of
the persistent $1$-cycles in low dimensions.

\subsection{Formulation}
\label{sec:newLoss:formulation}

Let $\calp^k$ be the set of \PH{k} pairs with a positive persistence, i.e.,
$\calp^k=\{(\tau,\sigma)\in\simplicesofdim{\calk}{k}\times\simplicesofdim{\calk}{k+1}\mid
\partner[\tau]=\sigma\text{ and }\delta(\tau)<\delta(\sigma)\}$.
We now define
the $d$-dimensional \textit{global cascade skeleton} ($\skelcasc^d$) as the set
of edges that appear in the cascade of any \PH{k} pair % of any $\calp^k$
with $0\leq k\leq d$:
\begin{equation} \label{eq:GCS}
\skelcasc^d(\inputPointCloud)=
\bigcup\limits_{k=0}^d\bigcup\limits_{(\tau,\sigma)\in\calp^k}\cascade[
\sigma]^{(1)}.
\end{equation}
In \autoref{fig:eliminateBoundaries},
$\skelcasc^1(\inputPointCloud)$ appears in the bottom-right image as the set of
red edges.
This leads us to propose
the following $d$-dimensional \textit{Cascade Distortion} (CD) loss function,
which looks similar to the original TopoAE
loss (\autoref{eq:topoae}), but which constrains the length of all 
the edges appearing within the global cascade of $\inputPointCloud$, instead of its critical edges only:

\begin{equation} \label{eq:cascae}
\begin{split}
	\lcascae{d}(\inputPointCloud,\latentPointCloud)=&\|A^\inputPointCloud[\skelcasc^d(\inputPointCloud)]-A^\latentPointCloud[\skelcasc^d(\inputPointCloud)]\|_2^2+\\&\|A^\latentPointCloud[\critical{d}(\latentPointCloud)]-A^\inputPointCloud[\critical{d}(\latentPointCloud)]\|_2^2
\end{split}.
\end{equation}
In practice, we focus our study on $\lcascae{1}$ for visualization purposes.
Intuitively, the role on the first term (top line of \autoref{eq:cascae})
is to favor an isometric embedding of the input cycles and the 2-chains that
fill them, so that their shapes are preserved at best.
The second term (bottom line of \autoref{eq:cascae}), like in
\tAE, involves pushing away vertices that are too close in $\latentPointCloud$,
hence penalizing the cycles that exist in $\latentPointCloud$ but not in
$\inputPointCloud$.
This way, the first term favors a faithful projection of the high-dimensional cycles, while the second term removes spurious low-dimensional cycles from the planar projection.
In addition, this loss enables the usage of a dedicated algorithm for the fast
computation of $\critical{1}(\latentPointCloud)$ at each
optimization iteration (\autoref{sec:algorithms}).

In practice, $\skelcasc^1(\inputPointCloud)$ is computed using a modified
version of \texttt{PairCells}, which, similarly to~\cite{Iuricich22},
uses a two-step approach.
First, the persistence pairs $\calp^1$ with a positive persistence are computed
with an efficient, dedicated algorithm, e.g., \textit{Ripser}~\cite{bauer_ripser_2021}.
Then, for each pair $(\tau,\sigma)\in\calp^1$ in ascending order, we run the
\texttt{EliminateBoundary} procedure (\autoref{algo:eliminateBoundaries}) from $\sigma$ until the youngest
edge of the boundary $\partial(\cascade[\sigma])$ is $\tau$, storing the cascade
edges encountered in the meantime. At each step,
$\text{Youngest}\bigl(\partial(\cascade[\sigma])\bigr)$ forms either an
apparent pair (with a triangle that is searched at this moment), or a positive persistence
pair that was handled previously.
Note that it is mandatory to choose the same order as \textit{Ripser} on the
triangles with identical diameter (i.e., the reverse colexicographic order).
This is more efficient than a naive execution of \texttt{PairCells},
since it benefits from the accelerations of \textit{Ripser}.

\subsection{Relation to TopoAE}
\label{sec:relationTopoAECascadeAE}

The above \textit{Cascade Distortion} term (\autoref{eq:cascae}) can be understood as a generalization of the original TopoAE topological regularization term in that when restricting to \PH{0}, both formulations are equivalent.
\begin{lemma}
	For any point cloud $\inputPointCloud$, $\skelcasc^0(\inputPointCloud)=\emst(\inputPointCloud)$.
	Therefore, the 0-dimensional CD and TopoAE regularization
	terms are equal:
	$\ltopoae{0}(\inputPointCloud, \latentPointCloud)=\lcascae{0}(\inputPointCloud, \latentPointCloud)$.
\end{lemma}
\begin{proof}
	Let $\inputPointCloud$ be a point cloud. First note that for any $\sigma$, we have $\sigma\in\cascade[\sigma]$, hence:
	\[
		\underbrace{\emst(\inputPointCloud)
		= \bigcup\limits_{(\tau,\sigma)\in\calp^0}\{\sigma\}}
	    _{\text{see \autoref{sec:geometricPH}}}
		\subset\underbrace{\bigcup\limits_{(\tau,\sigma)\in\calp^0}\cascade[\sigma]
		=\skelcasc^0(\inputPointCloud)}
		_{\text{by definition (\autoref{eq:GCS})}}.
	\]
	Reciprocally, note that for any vertex $\tau$, $\partner[\tau]$ is an edge that kills a \PH{0} class, hence is an $\emst$ edge.
	Then, with \autoref{algo:eliminateBoundaries:propagation}
	(\autoref{algo:eliminateBoundaries}), $\cascade[\sigma]$ writes as the
	sum of $\sigma$ and cascades of edges of the form $\partner[\tau]$.
	Therefore, by recurrence on the edges by increasing length, for any $\sigma\in\emst(X)$, i.e., for any $(\tau,\sigma)\in\calp^0$, we have $\cascade[\sigma]\subset\emst(\inputPointCloud)$.
	Hence $\skelcasc^0(\inputPointCloud)\subset\emst(\inputPointCloud)$.
	In the end, $\skelcasc^0(\inputPointCloud)=\emst(\inputPointCloud)$.
\end{proof}
For higher-dimensional \PH{} however, we can only say that
$\ltopoae{d}(\inputPointCloud,
\latentPointCloud)\leq\lcascae{d}(\inputPointCloud, \latentPointCloud)$, i.e.,
the latter is more restrictive than the former.
For instance, for $d=1$, in the counter-example of \autoref{fig:counter-example} where we had $\dgmrips{1}(\inputPointCloud)\neq\dgmrips{1}(\latentPointCloud)$ but $\ltopoae{1}(\inputPointCloud,\latentPointCloud)=0$, the 1-dimensional cascade distortion is here indeed positive : $\lcascae{1}(\inputPointCloud,\latentPointCloud)>0$.
More generally, $\lcascae{1}$ additionally constrains the length of the edges of the 2-chains that fill
the persistent 1-cycles, hence favoring an isometric embedding of those chains.
Intuitively, this can be understood as trying to preserve the whole
\textit{shape} of the 1-cycles, instead of only their birth and death.

In the following, we call TopoAE++ the dimensionality reduction
technique obtained by replacing in TopoAE the topological regularization term
$\ltopoae{0}$ with $\lcascae{1}$, and by including the acceleration
presented in the next section.
