\section{Fast algorithm for Rips PH in the plane}
\label{sec:algorithms}

Because the computation of the persistence diagram of the low-dimensional representation is executed at each step of the optimization process, it is a critical stage in terms of computation time. Therefore, we propose in this section a new fast algorithm that computes \PH{} for the Rips filtration of a two-dimensional point cloud, which is useful when our embedding space is $\bbr^2$, e.g., for visualization purpose.
This algorithm is purely geometric and involves no matrix reduction step,
contrary to other algorithms (e.g. \cite{maria2014gudhi, bauer2017phat,
bauer_ripser_2021, koyama2023reduced}),
which allows it to be significantly
faster and less memory-intensive (see \autoref{sec:performance:rips2d} for timing results).

\subsection{Geometrical results}
\label{sec:algorithm:geometry}

We rely on the fact that the edges that introduce a cycle are exactly the edges in $\rng(\inputPointCloud)\setminus\emst(\inputPointCloud)$ (as previously exploited in the related work~\cite{koyama2023reduced}, see \autoref{sec:geometricPH}). Consequently, the number of points in the 1-th persistence diagram is the number of polygons formed by $\rng(\inputPointCloud)$, which is also the cardinal of $\rng(\inputPointCloud)\setminus\emst(\inputPointCloud)$. In contrast to previous work~\cite{koyama2023reduced} (which also exploited boundary matrix reduction), we additionally claim that the critical edges that kill these \PH{1} classes
can be found efficiently through geometric -- rather than algebraic -- considerations, namely with a minmax length triangulation of each $\rng$-polygons (see \autoref{fig:MMLT-PD}).

\begin{figure}
	\centering
	\includegraphics[width=\linewidth]{figures/PD_geometry+CH.pdf}
	\caption{$\rng$-polygons (green, blue and red polygons) with their minmax length triangulations (gray edges). The arrows show to which critical edges (birth: dashed; death: continuous) is associated each point of the 1-dimensional persistence diagram. Each birth edge is in the $\rng$ and each death edge is the longest edge of the minmax length triangulations of its $\rng$-polygon. Missing edges from the convex hull are shown in yellow, as well as the remaining polygons that the convex hull forms with the $\rng$.}
	\label{fig:MMLT-PD}
\end{figure}

\begin{lemma}
	\label{lemma:interpolygonal-edges}
	In the plane, an edge that intersects an $\rng$ edge kills no \PH{1}
class of positive persistence.
\end{lemma}
\begin{proof}
	Let $ab$ be an edge that intersects an $\rng$ edge $e=p_1p_2$.
	Suppose that $ab$ is the longest edge of the triangles $abc$ and $abd$, one of which kills a \PH{1} class $\gamma$ of which the quadrilateral $acbd$ is a representative 1-cycle (see \autoref{fig:interpolygonal-edges}).
	Because $\lens(e)=\varnothing$ (since $e\in\rng(\inputPointCloud)$), $|e|<|ab|$.
	In addition, $c$ and $d$ are in $\lens(a,b)$.
	Therefore, the 2-chain $acp_1+cbp_1+bdp_2+dap_2+ap_1p_2+bp_1p_2$, whose boundary is $acbd$, contains only triangles of diameter $<|ab|$.
	Hence, the class $\gamma$ is killed before the value $|ab|$, therefore not by $ab$.
\end{proof}
\begin{figure}
	\centering
	\def\svgwidth{.6\linewidth}\input{figures/interpolygonal-edges.pdf_tex}
	\caption{Notations for proof of \autoref{lemma:interpolygonal-edges}.
	$e=p_1p_2$ (in red) is supposed to be an $\rng$ edge, hence an empty lens. An edge $ab$ (light
	green) that crosses e is shown to not be able to kill any \PH{1} class of positive persistence.}
	\label{fig:interpolygonal-edges}
\end{figure}

Therefore, the edges that kill a \PH{1} class of positive persistence can only
lie inside either an $\rng$-polygon, or a polygon formed by the $\rng$ and the
convex hull of $\inputPointCloud$ (\autoref{fig:MMLT-PD}).
Then, we have the following lemma for each $\rng$-polygon.

\begin{lemma}
	\label{lemma:MML-edges}
	Let $\Pi$ be a $\rng$-polygon. Under general position hypothesis (unique pairwise distances), the longest edge $\edgeMML$, of length noted $\delta_{\mml}$, of any minmax length triangulation of $\Pi$ kills a \PH{1} class $\gamma$ with positive persistence for which $\Pi$ is a representative.
\end{lemma}
\begin{proof}
	Let $\Pi$ be a $\rng$-polygon with point set $P$. Let $\edgeMML$ be the
	longest edge of the minmax length triangulations of this polygon, which is well-defined with the general position hypothesis. We have the
	following:
	\begin{itemize}
		\item for diameter threshold $r=\delta_{\mml}$, $\rips_{r}(P)$ contains a $\mml$ triangulation of $\Pi$, which can be seen as a 2-chain whose boundary is $\Pi$;
		\item for any diameter threshold $r<\delta_{\mml}$, $\rips_{r}(P)$ contains no 2-chain whose boundary is $\Pi$: otherwise, one could construct
		%, starting from the longest edge of that 2-chain,
 		a triangulation of $\Pi$ whose longest edge is of length at most $r$, hence smaller than $\delta_{\mml}$.
	\end{itemize}
	Hence, $\edgeMML$ kills a \PH{1} class $\gamma$ with positive persistence for which $\Pi$ is a representative.
\end{proof}

Therefore, in the plane, as the number of $\rng$-polygons is exactly the number of \PH{1} classes with a positive persistence, the edges that destroy the latter are exactly the edges found in the above lemma, i.e., that are the longest of the $\mml$ triangulations of an $\rng$-polygon (see \autoref{fig:MMLT-PD}).
%Hence, in the following, we write $\ripsKiller$ and  $\delta_{\rips}$ instead of $\edgeMML$ and $\delta_{\mml}$.

Finally, for any $\rng$-polygon, we state a relation between $\deathValue$
and the length $\delta_{\dr}$ of the longest Delaunay edge $\DRKiller$ inside this polygon,
enabling the search of
$\ripsKiller$ among a smaller subset of the polygon's
diagonals (see \autoref{sec:algorithm:description}).

\begin{lemma}
	\label{lemma:bounded_rips_delrips}
	In the plane,
	the values $\deathValue$ and $\delta_{\dr}$ of an $\rng$-polygon verify:
	\[\frac{\sqrt{3}}{2}\delta_{\dr}\leq\deathValue\leq\delta_{\dr}.\]
\end{lemma}
\begin{proof}
	Let $\Pi$ be an $\rng$-polygon and $P$ its point set. Let $\DRKiller\in\del(P)$ be the
	longest Delaunay edge inside $\Pi$, of length $\delta_{\dr}=|\DRKiller|$.
	The right inequality results from the Delaunay triangulation inside $\Pi$ being a 2-chain whose boundary is $\Pi$ and whose maximum diameter is $\delta_{\dr}$: hence the associated Rips \PH{1} class $\gamma$ is killed at value $\deathValue\leq\delta_{\dr}$.
	For the left inequality, because $\DRKiller$ is a Delaunay edge, there exists
	a circumcircle $\circumcircle$ to $\DRKiller$ of radius
	$r(\circumcircle)\geq|\DRKiller|/2$ that contains no other point of $P$ (see
	\autoref{fig:bounded_rips_delrips}). Now any 2-chain with $\Pi$ as boundary has
	to contain the center $c(\circumcircle)$ of that circle. However, the smallest
	triangles with no vertex within $\circumcircle$ and that contain $c(\circumcircle)$ are exactly the equilateral
	triangles circumscribed by $\circumcircle$, and their side length is
	$\sqrt{3}r(\circumcircle)\geq\frac{\sqrt{3}}{2}|\DRKiller|$.
	Therefore, the associated Rips \PH{1} class
	$\gamma$ is killed by a triangle $\tau$ of diameter at least
	$\frac{\sqrt{3}}{2}|\DRKiller|$, hence
	$\frac{\sqrt{3}}{2}\delta_{\dr}\leq\deathValue$.
\end{proof}
\begin{figure}
	\centering
	\def\svgwidth{.8\linewidth}\input{figures/bounded_delrips.pdf_tex}
	\caption{Notations for the proof of \autoref{lemma:bounded_rips_delrips}.
	The $\rng$-polygon $\Pi$ (bold black), is such that
	its Delaunay triangulation (in gray) is not a $\mml$ triangulation. Hence,
	its cycle killing edge $\ripsKiller$ (bold red) is shorter than its
	longest Delaunay edge $\DRKiller$ (blue), i.e., $\deathValue < \delta_{\dr}$.
	The circumcircle $\circumcircle$ of the longest Delaunay edge $\DRKiller$ contains
	no points of $\inputPointCloud$ in its interior.}
	\label{fig:bounded_rips_delrips}
\end{figure}

\subsection{Algorithm description}
\label{sec:algorithm:description}

Leveraging the above results, we introduce a fast algorithm that computes
\PH{} for the Rips filtration of a 2-dimensional point cloud. It is
summarized in \autoref{algo:2dpersistence} and we describe below more precisely
the different stages.

\begin{algorithm}
	\DontPrintSemicolon
	\label{algo:2dpersistence}\caption{2D Rips persistence method overview}
	\KwIn{Point cloud $X\subset\bbr^2$}
	\KwOut{Persistence diagrams $\dgmrips{0}(\inputPointCloud)$ and $\dgmrips{1}(\inputPointCloud)$ with associated simplices}
	Compute the Delaunay triangulation $\del(\inputPointCloud)$\;
	Compute the Urquhart graph $\ug(\inputPointCloud)$\;
	Apply Kruskal's algorithm on $\ug(\inputPointCloud)$ to compute
$\emst(\inputPointCloud)$ and obtain $\dgmrips{0}(X)$\;
	Check the emptiness of the lens of the edges in $\ug(\inputPointCloud)\setminus\emst(\inputPointCloud)$ to compute $\rng(\inputPointCloud)$\;
	For each $\rng$-polygon, find its local minmax length triangulation longest edge (\autoref{algo:findMMLedge})\;
	Apply Kruskal's algorithm on the dual $\rng(\inputPointCloud)$ graph
to obtain $\dgmrips{1}(X)$\;
\end{algorithm}

\begin{algorithm}
	\DontPrintSemicolon
	\label{algo:findMMLedge}\caption{Find the longest edge of a minmax length triangulation of an $\rng$-polygon of $\inputPointCloud$}
	\KwIn{Points $P\subset\inputPointCloud$ of an $\rng$-polygon}
	\KwIn{Length $\delta_{\dr}$ of the longest Delaunay edge}
	\KwOut{Edge that destroy the associated 1-cycle}
	$\text{candidates}\gets\{\}$\;
	\For{$e=v_1v_2\in\binom{P}{2}$ s.t. $|e|\in\left[\frac{\sqrt{3}}{2}\delta_{\dr},\delta_{\dr}\right]$}{
		\tcp{check if $e$ is a 2-edge}
		\If{$\llens(e)\neq\varnothing$ and $\rlens(e)\neq\varnothing$} {
			\tcp{check if $e$ is expandable}
			\For{$x\in\llens(e)$}{
				\If{$\rlens(v_1x)=\varnothing$ and $\rlens(xv_2)=\varnothing$}{
					$\text{LeftExpandable}\gets$ true\;
					break\;
				}
			}
			\For{$y\in\rlens(e)$}{
				\If{$\rlens(v_2y)=\varnothing$ and $\rlens(yv_1)=\varnothing$}{
					$\text{RightExpandable}\gets$ true\;
					break\;
				}
			}
			\If{$\text{\normalfont{LeftExpandable}}$ and $\text{\normalfont{RightExpandable}}$} {
				$\text{candidates}\gets\text{candidates}\cup\{e\}$\;
			}
		}
	}
	\Return{$\min(\text{candidates}, |\cdot|)$}
\end{algorithm}

\paragraph{Delaunay triangulation} We use
CGAL~\cite{fabri_cgal_2009} to compute the Delaunay triangulation,
with identifiers on vertices and
triangles.
This implementation deals with the boundary (simplices on the convex hull) with a
virtual point at infinity.

\paragraph{Relative neighborhood graph} To compute $\rng(\inputPointCloud)$, we first compute
$\ug(\inputPointCloud)$ by keeping Delaunay edges whose link (made of two points) is outside their
lens. We then apply Kruskal's algorithm (which uses a
union-find data structure~\cite{cormen}) on $\ug(\inputPointCloud)$ to compute
$\emst(\inputPointCloud)$,
of which $\dgmrips{0}(X)$ can be deduced
(see \autoref{sec:geometry} and \autoref{sec:geometricPH}). Then, checking the
emptiness of the lens of the edges in $\ug(\inputPointCloud)\setminus\emst(\inputPointCloud)$ using disk queries on
a precomputed k-d tree, we determine whether they belong to $\rng(\inputPointCloud)$, and
construct a data structure that represents $\rng(\inputPointCloud)$ and its polygons (see next
paragraph).

\paragraph{Data structure for $\rng$-polygons} During the computation of $\ug(\inputPointCloud)$ and $\rng(\inputPointCloud)$, we maintain a union-find data structure on Delaunay triangles that permits to deduce efficiently the $\rng$-polygons. More precisely, as soon as we encounter a non-$\rng$ edge (either during $\ug(\inputPointCloud)$ computation or during $\rng(\inputPointCloud)$ refinement), we merge (if not already merged) the two classes associated with its two adjacent triangles. We also keep for each current class the longest deleted Delaunay edge, which needs to be maintained at each merge. Besides, we store $\ug(\inputPointCloud)$ and $\rng(\inputPointCloud)$ edges as quad-edges (i.e., an edge is stored as the identifiers of its two extremities and the identifiers of its two adjacent triangles). This allows an implicit representation of the $\rng$-polygons, each of which is associated with a root of the union-find data structure over triangles. For the next stage, we also recover the vertices associated with each polygon.

\paragraph{Minmax length triangulations} Now all we need is to find the  edges that kill persistent cycles: using \autoref{lemma:MML-edges}, we search the longest edge $\edgeMML$ of a $\mml$ triangulation of each polygon. For that we partially follow the algorithm described in~\cite{edelsbrunner_quadratic_1993} that computes $\mml$ triangulations in quadratic time, even though there is no need here to compute a full triangulation since we only need its longest edge. Therefore for each $\rng$-polygon, we enumerate the polygon diagonals with length between $\frac{\sqrt{3}}{2}\simeq0.866$ and 1 times the length of the longest deleted Delaunay edge within the polygon (see \autoref{lemma:bounded_rips_delrips}). Among them, we search the smallest that is an \emph{expandable 2-edge} (a property introduced in~\cite{edelsbrunner_quadratic_1993} that allows an edge to be the longest of a triangulation of the polygon, see \autoref{appendix:MMLTs} for more details)
using disk queries on a k-d tree on the vertices of the polygon (see \autoref{fig:expandability} and \autoref{algo:findMMLedge}).

\begin{figure}
	\centering
	\def\svgwidth{.75\linewidth}\input{figures/expandability.pdf_tex}
	\caption{Illustration of the expandability -- tested by
	\autoref{algo:findMMLedge} -- of the 2-edge $e=v_1v_2$. Indeed there exists
	$x,y\in\lens(e)$ (red area) on either side of $e$ such that the
	"inner" half-lenses (green areas) of the edges $v_1x$, $xv_2$, $v_2y$,
	$yv_1$ are empty.
	Thus, there exists a planar triangulation of $\Pi$ whose longest edge is $e$.
	Besides, $e$ happens to be the smallest of the expandable 2-edges in
	$\Pi$: $e$ is therefore the longest edge of the $\mml$ triangulations of $\Pi$.}
	\label{fig:expandability}
\end{figure}

\paragraph{Persistent pairs} The last step consists in pairing edges that create cycles with those that kill them (a cycle-creating edge is not necessarily associated with a killer edge of an adjacent polygon, in case of nested cycles).
We solve this relying on duality, an overall strategy also employed in
the seminal persistence algorithm~\cite{edelsbrunner02}.
Specifically, for our setup, we compute \PH{0} of the dual of $\rng(\inputPointCloud)$ with opposite filtration values, which is
the
% wanted
\PH{1} of the input. In other words, we apply Kruskal's algorithm on a
dual graph where dual nodes are $\rng$-polygons $\Pi$ with filtration value
$-\deathValue(\Pi)$, and where dual edges are cycle-creating edges
$e\in\rng(\inputPointCloud)\setminus\emst(\inputPointCloud)$ with filtration value $-|e|$.

\subsection{Parallelization}
\label{sec:algorithm:parallelization}

Several stages can be parallelized for accelerating the running time. The most
critical stage is the computation of minmax length triangulations
(it represents around 50\% of the computation time on average in sequential for uniform point clouds). It can be
naively parallelized on the different $\rng$-polygons. However, these polygons
can be of varying sizes, leading to a poor speed-up in some cases.
Instead, we develop a slightly more efficient parallelization, which consists in storing in a single array every candidate edge (of length
in $\left[\frac{\sqrt{3}}{2}\delta_{\dr},\delta_{\dr}\right]$) of every polygon, and to
parallelize the expandability checks on this array.

Finally, other stages can be easily parallelized with a less noticeable
effect, such as using a parallel sorting implementation for sorting the
$\ug(\inputPointCloud)$ edges by length (before applying Kruskal's
algorithm on them to get $\emst(\inputPointCloud)$).
