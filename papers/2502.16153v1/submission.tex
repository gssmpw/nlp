%%\usepackage{comment}
\PassOptionsToPackage{table,xcdraw}{xcolor}

%% This is file `sample-second-authordraft.tex',
%% generated with the docstrip utility.
%%
%% The original source files were:
%%
%% samples.dtx  (with options: `all,proceedings,bibtex,authordraft')
%% 
%% IMPORTANT NOTICE:
%% 
%% For the copyright see the source file.
%% 
%% Any modified versions of this file must be renamed
%% with new filenames distinct from sample-sigconf-authordraft.tex.
%% 
%% For distribution of the original source see the terms
%% for copying and modification in the file samples.dtx.
%% 
%% This generated file may be distributed as long as the
%% original source files, as listed above, are part of the
%% same distribution. (The sources need not necessarily be
%% in the same archive or directory.)
%%
%%
%% Commands for TeXCount
%TC:macro \cite [option:text,text]
%TC:macro \citep [option:text,text]
%TC:macro \citet [option:text,text]
%TC:envir table 0 1
%TC:envir table* 0 1
%TC:envir tabular [ignore] word
%TC:envir displaymath 0 word
%TC:envir math 0 word
%TC:envir comment 0 0
%%
%%
%% The first command in your LaTeX source must be the \documentclass
%% command.
%%
%% For submission and review of your manuscript please change the
%% command to \documentclass[manuscript, screen, review]{acmart}.
%%
%% When submitting camera ready or to TAPS, please change the command
%% to \documentclass[sigconf]{acmart} or whichever template is required
%% for your publication.
%%
%%
%%\documentclass[review,anonymous]{acmart}
% \documentclass[sigconf,anonymous]{acmart}
%\documentclass[manuscript,review,anonymous]{acmart}
%\documentclass[acmsmall]{acmart}
\PassOptionsToPackage{prologue,dvipsnames}{xcolor}
\documentclass[sigconf,screen]{acmart}

\usepackage[table,xcdraw]{xcolor}
\usepackage{tikz}
\usetikzlibrary{patterns}
\usepackage{multirow}
\usepackage[table,xcdraw]{xcolor}
\usepackage{colortbl}
\usepackage{tabularx}
\usepackage{graphicx}
%%
%% \BibTeX command to typeset BibTeX logo in the docs
\AtBeginDocument{%
  \providecommand\BibTeX{{%
    Bib\TeX}}}

%% Rights management information.  This information is sent to you
%% when you complete the rights form.  These commands have SAMPLE
%% values in them; it is your responsibility as an author to replace
%% the commands and values with those provided to you when you
%% complete the rights form.
\copyrightyear{2025}
\acmYear{2025}
\setcopyright{acmlicensed}\acmConference[CHI '25]{CHI Conference on Human Factors in Computing Systems}{April 26-May 1, 2025}{Yokohama, Japan}
\acmBooktitle{CHI Conference on Human Factors in Computing Systems (CHI '25), April 26-May 1, 2025, Yokohama, Japan}
\acmDOI{10.1145/3706598.3714120}
\acmISBN{979-8-4007-1394-1/2025/04}


%%
%% Submission ID.
%% Use this when submitting an article to a sponsored event. You'll
%% receive a unique submission ID from the organizers
%% of the event, and this ID should be used as the parameter to this command.
%%\acmSubmissionID{123-A56-BU3}

%%
%% For managing citations, it is recommended to use bibliography
%% files in BibTeX format.
%%
%% You can then either use BibTeX with the ACM-Reference-Format style,
%% or BibLaTeX with the acmnumeric or acmauthoryear sytles, that include
%% support for advanced citation of software artefact from the
%% biblatex-software package, also separately available on CTAN.
%%
%% Look at the sample-*-biblatex.tex files for templates showcasing
%% the biblatex styles.
%%

%%
%% The majority of ACM publications use numbered citations and
%% references.  The command \citestyle{authoryear} switches to the
%% ''author year'' style.
%%
%% If you are preparing content for an event
%% sponsored by ACM SIGGRAPH, you must use the ''author year'' style of
%% citations and references.
%% Uncommenting
%% the next command will enable that style.
%%\citestyle{acmauthoryear}


%%
%% end of the preamble, start of the body of the document source.
\begin{document}

%%
%% The ''title'' command has an optional parameter,
%% allowing the author to define a ''short title'' to be used in page headers.
\title{Understanding Screenwriters' Practices, Attitudes, and Future Expectations in Human-AI Co-Creation}

%%
%% The ''author'' command and its associated commands are used to define
%% the authors and their affiliations.
%% Of note is the shared affiliation of the first two authors, and the
%% ''authornote'' and ''authornotemark'' commands
%% used to denote shared contribution to the research.
% \iffalse

\author{Yuying Tang}
%\orcid{1234-5678-9012}
%\authornotemark[1]
%\email{webmaster@marysville-ohio.com}
\affiliation{%
  \institution{The Hong Kong University of Science and Technology }
  \city{Hong Kong SAR}
  \country{China}
}
\email{ytangdh@connect.ust.hk}

\author{Haotian Li}
\authornote{The work was done when Haotian Li was at HKUST. Haotian Li is the corresponding author.}
% \orcid{0000-0001-9547-3449}
\affiliation{%
  \institution{Microsoft Research Asia}
  \city{Beijing}
  \country{China}
}
\email{haotian.li@microsoft.com}

\author{Minghe Lan}
%\orcid{1234-5678-9012}
%\authornotemark[1]
%\email{webmaster@marysville-ohio.com}
\affiliation{%
  \institution{Central Academy of Fine Arts}
  \city{Beijing}
  \country{China}
}
\email{lan_shan99@outlook.com}

\author{Xiaojuan Ma}
%\orcid{1234-5678-9012}
%\authornotemark[1]
\affiliation{%
  \institution{The Hong Kong University of Science and Technology}
  \city{Hong Kong SAR}
  \country{China}
}
\email{mxj@cse.ust.hk}

\author{Huamin Qu}
% \orcid{0000-0002-3344-9694}
\affiliation{%
  \institution{The Hong Kong University of Science and Technology}
  \city{Hong Kong SAR}
  \country{China}
}
\email{huamin@cse.ust.hk}
% \fi


%%
%% By default, the full list of authors will be used in the page
%% headers. Often, this list is too long, and will overlap
%% other information printed in the page headers. This command allows
%% the author to define a more concise list
%% of authors' names for this purpose.
% \renewcommand{\shortauthors}{Tang et al.}
\newcommand{\haotian}[1]{\textcolor{teal}{#1}}
\newcommand{\revision}[1]{\textcolor{red}{#1}}
% \definecolor{red}{HTML}{B53E27}

% \ccsdesc[500]{Human-centered computing~Interaction design process and methods}
\ccsdesc[500]{Human-centered computing~Empirical studies in HCI}
\ccsdesc[500]{General and reference~Art and Design}
\ccsdesc[500]{Computing methodologies~Artificial intelligence}



\keywords{Creativity Support; Screenwriting; Qualitative Methods; Human-AI Co-Creation}

%%
%% The abstract is a short summary of the work to be presented in the
%% article.
\begin{abstract}

%With the rise of AI technologies and their growing influence in the screenwriting field, understanding the opportunities and concerns related to AI’s role in screenwriting is essential for enhancing human-AI co-creation. Through semi-structured interviews with 23 screenwriters, we explored their creative practices, attitudes, and expectations in collaborating with AI for screenwriting. Based on participants' responses, we identified the key stages in which they commonly integrated AI, including story structure \& plot development, screenplay text, goal \& idea generation, and dialogue. \textcolor{black}{We observed that differing attitudes toward AI integration shape screenwriters' practices across workflow stages and the broader industry. These attitudes are influenced by AI's capabilities and limitations, and at times, reflect contradictions depending on the specific use case context. Additionally, w}e categorized their expected assistance using four distinct roles of AI: actor, audience, expert, and executor. Our findings provide insights into AI's impact on screenwriting practices and offer suggestions on how AI can affect the future of screenwriting.
%We also observed that despite divergent attitudes toward AI, all participants expressed a desire for AI to better support the screenwriting process. 

With the rise of AI technologies and their growing influence in the screenwriting field, understanding the opportunities and concerns related to AI's role in screenwriting is essential for enhancing human-AI co-creation. Through semi-structured interviews with 23 screenwriters, we explored their creative practices, attitudes, and expectations in collaborating with AI for screenwriting. Based on participants' responses, we identified the key stages in which they commonly integrated AI, including story structure \& plot development, screenplay text, goal \& idea generation, and dialogue. Then, we examined how different attitudes toward AI integration influence screenwriters' practices across various workflow stages and their broader impact on the industry. Additionally, we categorized their expected assistance using four distinct roles of AI: actor, audience, expert, and executor. Our findings provide insights into AI's impact on screenwriting practices and offer suggestions on how AI can benefit the future of screenwriting.


%With the rise of AI technologies and their growing influence in the screenwriting industry, understanding the opportunities and concerns surrounding AI's role in screenwriting is crucial for advancing human-AI co-creation. Through a qualitative study involving semi-structured interviews with 23 screenwriters, we explored their creative practices, attitudes, and the potential future roles of AI within the screenwriting workflow. Based on their experience, we found that their use of AI primarily focused on story structure \& plot, screenplay text, goal \& idea generation, and dialogue stages. Despite divergent attitudes toward AI, all screenwriters expressed a desire for AI to better support the screenwriting process. We categorized their future expectations for AI into four distinct roles in the human-AI co-creation process in screenwriting: actor, audience, expert, and executor. Our findings provide an in-depth understanding of AI's impact on screenwriting practices and offer suggestions on how AI can affect the future of screenwriting.

%With the rise of AI technologies and their growing influence in screenwriting industries, understanding the opportunities and concerns surrounding AI's integration into screenwriting is crucial for advancing human-AI co-creation. Through a qualitative study involving semi-structured interviews with 23 screenwriters, we explored their existing creative practices, their attitudes, and the potential future roles of AI within the screenwriting workflow. We found that, based on their user experience, despite divergent attitudes toward AI, all screenwriters expressed a desire for AI to better support the screenwriting process. We categorized their future expectations for AI into four distinct roles in the human-AI co-creation process in screenwriting: actor, audience, expert, and executor. Our findings provide an in-depth understanding of AI's impact on the screenwriting practice and offer suggestions on how AI can affect the future of screenwriting.

%With the rise of AI technologies and their growing influence in creative industries, particularly in light of the recent Hollywood screenwriters' strike, understanding AI's integration into screenwriting has become crucial. This paper examines the integration of AI into screenwriting, analyzing existing practices, screenwriters' attitudes, and the potential future roles of AI within the screenwriting workflow. Through a qualitative study involving semi-structured in-depth interviews with screenwriters (N = 23), we identified the anticipated future need for AI tools at each stage of the workflow. Despite varying attitudes toward AI—ranging from positive to negative—all screenwriters expressed a desire for AI to better support the screenwriting process. We categorized their future expectations for AI into four distinct roles: actor, audience, expert, and executor. Our findings provide a comprehensive overview of AI's impact on the screenwriting industry and offer guidance on how AI can shape the future of screenwriting.
\end{abstract}

%%
%% The code below is generated by the tool at http://dl.acm.org/ccs.cfm.
%% Please copy and paste the code instead of the example below.
%%
\begin{comment}
\begin{CCSXML}
<ccs2012>
 <concept>
  <concept_id>00000000.0000000.0000000</concept_id>
  <concept_desc>Do Not Use This Code, Generate the Correct Terms for Your Paper</concept_desc>
  <concept_significance>500</concept_significance>
 </concept>
 <concept>
  <concept_id>00000000.00000000.00000000</concept_id>
  <concept_desc>Do Not Use This Code, Generate the Correct Terms for Your Paper</concept_desc>
  <concept_significance>300</concept_significance>
 </concept>
 <concept>
  <concept_id>00000000.00000000.00000000</concept_id>
  <concept_desc>Do Not Use This Code, Generate the Correct Terms for Your Paper</concept_desc>
  <concept_significance>100</concept_significance>
 </concept>
 <concept>
  <concept_id>00000000.00000000.00000000</concept_id>
  <concept_desc>Do Not Use This Code, Generate the Correct Terms for Your Paper</concept_desc>
  <concept_significance>100</concept_significance>
 </concept>
</ccs2012>
\end{CCSXML}

\ccsdesc[500]{Do Not Use This Code~Generate the Correct Terms for Your Paper}
\ccsdesc[300]{Do Not Use This Code~Generate the Correct Terms for Your Paper}
\ccsdesc{Do Not Use This Code~Generate the Correct Terms for Your Paper}
\ccsdesc[100]{Do Not Use This Code~Generate the Correct Terms for Your Paper}

%%
%% Keywords. The author(s) should pick words that accurately describe
%% the work being presented. Separate the keywords with commas.
\keywords{Do, Not, Us, This, Code, Put, the, Correct, Terms, for,
  Your, Paper}
%% A ''teaser'' image appears between the author and affiliation
%% information and the body of the document, and typically spans the
%% page.

\end{comment}

%%
%% This command processes the author and affiliation and title
%% information and builds the first part of the formatted document.
\maketitle

\section{Introduction}

%Screenwriting practices can be seen as a form of creative writing, while the development of screenplays aligns with film production processes. The role of a screenwriter can be understood in relation to other types of creative writers, as well as to roles typically associated with producers or directors~\cite{batty2015screenwriter}. This suggests that screenwriting is distinct from traditional forms of creative writing, with its focus on crafting scripts designed for visual and auditory media rather than solely text-based content~\cite{batty2014screenwriters, kerrigan2016re}. 

%Screenwriting refers to the creative process of screenplay, which is often considered a form of creative writing. However, since its final output typically combines both visual and auditory elements in film and television, the responsibilities of a screenwriter differ from those of other creative writers~\cite{batty2014screenwriters, kerrigan2016re}. Specifically, the role of a screenwriter can be contextualized by comparing it with other types of creative writers, and by examining how it intersects with the responsibilities typically held by producers or directors~\cite{batty2015screenwriter}.

%Screenwriting refers to the process of crafting screenplays for visual media, encompassing the description of plot structure, dialogue, and scene. It is frequently considered a subset of creative writing, due to its imaginative and narrative-driven qualities~\cite{kerrigan2016re, batty2015screenwriter}. However, screenwriting differs substantially from traditional creative writing as the final product typically integrates both visual and auditory components in film and television, all of which contribute to the storytelling in film, television, or other audiovisual formats\cite{dunnigan2019screenwriting}.

%Screenwriting refers to the process of crafting screenplays for visual media, encompassing the detailed description of plot structure, dialogue, and scenes. It is often considered a subset of creative writing due to its imaginative and narrative-driven qualities~\cite{kerrigan2016re, batty2015screenwriter}. However, screenwriting is not entirely equivalent to traditional creative writing, as its ultimate goal is to integrate both visual and auditory elements, which aim to support storytelling in film, television, or other audiovisual formats~\cite{dunnigan2019screenwriting}.

%In the era of rapid technology advancements, the application of AI in the field of screenwriting has garnered widespread attention\cite{anguiano2023hollywood}, with extensive discussion about the synergy of AI and creativity~\cite{10.1145/3613904.3642731, 10.1145/3613905.3650929}. Previous research has addressed various challenges in screenwriting, including information retrieval~\cite{sanghrajka2017lisa, pavel2015sceneskim}, logical reasoning~\cite{sanghrajka2017lisa, kapadia2015computer}, emotional support~\cite{goyal2010toward}, and collaborative writing~\cite{schank2013scripts, bartindale2016tryfilm}. Solutions have also been proposed for visualization~\cite{10.1145/3172944.3172972, kim2017visualizing, won2014generating}. However, with the development of advanced AI technologies, such as text-to-image generation, and large language models, the capabilities of AI have evolved significantly, advancing from mere processing and understanding to generation. Recent studies suggest that generative AI technologies and tools can now support screenwriting~\cite{luchen2023chatgpt, chow2020ghost, 10.1145/3544548.3581225}. Moreover, a previous empirical study have shown that AI can generate screenplay elements that are comparable to, or even superior to, those produced by humans in certain cases~\cite{ccelik4ai}. Additionally, other studies have explored its use in screenwriting education~\cite{brako2023robots}, and discussed screenplay copyright challenges posed by AI~\cite{kavitha2023copyright}. Overall, while screenwriters have started integrating AI into their tasks, previous studies focused on part of the entire screenwriting workflow and cannot present a panorama of why and how screenwriters leverage AI to improve their productivity and creativity in the workflow.
%Furthermore, their perception and expectation of AI have not been thoroughly investigated, hindering the development of co-creation tools to better meet screenwriters' requirements. 

%Consequently, although earlier empirical studies have provided valuable insights, they have not addressed the questions we focus in this study. 

%Our research aims to understand the current state of AI use in the entire screenwriting workflow, screenwriters' attitudes towards AI, and their expectations for future AI functionalities. The goal is to identify the challenges currently faced in screenwriting, explore how screenwriters are using AI tools to address these challenges, and provide guidelines for designing future AI-assisted screenwriting tools that can solve problems in the workflow in ways that align with screenwriters' expectations. This will further promote human-AI co-creation and extend existing work in this area. Specifically, our study focuses on addressing the following key questions:

%Our study aims to understand the current state of AI practices throughout the screenwriting workflow, screenwriters' attitudes toward AI, and their expectations for future AI capabilities. We aim to identify the challenges currently faced in screenwriting, explore how screenwriters are using AI tools to address these challenges, and offer guidelines for designing future AI-assisted screenwriting tools that align with screenwriters' needs. 

%Our study seeks to understand current AI practices in screenwriting, screenwriters' attitudes toward AI, and their expectations for future AI advancements to provide a overview picture for this field. We aim to identify the challenges in the screenwriting process, explore how AI tools are being used to address these issues, and provide suggestions for designing AI-assisted tools that meet screenwriters' needs. Our approach will further promote human-AI co-creation and build upon existing work in this area. Specifically, our study focuses on addressing the following key questions:

%The AI discussed in this context primarily refers to prediction techniques supported by data-driven deep learning, which rely on data priors and inductive reasoning. 

%Our study aims to understand current AI practices in screenwriting, screenwriters' attitudes toward AI, and their expectations for future AI advancements to offer a comprehensive overview of the field. Specifically, we seek to identify the challenges in the screenwriting process, explore how AI tools are currently used to address these challenges, and provide suggestions for designing AI-assisted tools that meet screenwriters' needs. Our approach will further promote human-AI co-creation and build upon existing work in this area. Therefore, our study focuses on addressing the following key questions:

%Our study aims to understand current AI practices in screenwriting, screenwriters' attitudes toward AI, and their expectations for future AI, providing a comprehensive overview of the field. Specifically, we seek to identify the challenges in the screenwriting workflow, explore how AI tools are currently used to address these challenges, and offer suggestions for designing AI-assisted tools that meet screenwriters' needs. Our goal is not for AI to replace screenwriters, but to improve the quality and experience of their work. This approach will build upon existing research and further promote human-AI co-creation. Therefore, our study focuses on addressing the following key questions:

%Our study aims to understand current AI practices in screenwriting, screenwriters' attitudes toward AI, and their expectations for future AI, providing a comprehensive overview of the field. Specifically, we seek to identify the challenges in the screenwriting workflow, explore how AI tools are currently used to address these challenges, and offer suggestions for designing AI-assisted tools that meet screenwriters' needs. Our goal is not for AI to replace screenwriters, but to enhance the quality and experience of their work. This approach builds upon existing research and further promotes human-AI co-creation. Our study focuses on addressing the following key questions:

% Based on their usage of AI in the workflow, we will examine screenwriters’ attitudes toward AI and their expectations for future functionalities. Specifically, we aim to identify the challenges in the screenwriting workflow, explore how AI tools are currently being used to address these challenges, and offer suggestions for designing AI-assisted tools that meet screenwriters' needs. Our goal is not for AI to replace screenwriters but to enhance the quality and experience of their work, promoting human-AI co-creation.

%To fill the gap, our research aims to understand screenwriters' current practice, attitudes, and future expectations of co-creating with AI across the entire screenwriting workflow. 
%This study focuses on addressing three research questions:

%Screenwriting involves crafting screenplays for visual media, detailing plot structure, dialogue, and scenes. While it is often categorized as a subset of creative writing due to its narrative-driven and imaginative nature~\cite{kerrigan2016re, batty2015screenwriter}, it differs in its focus on integrating visual and auditory elements to support storytelling in film, television, and other audiovisual formats~\cite{dunnigan2019screenwriting}.  

\begin{comment}
    
Screenwriting refers to the process of crafting screenplays for visual media, encompassing the detailed description of plot structure, dialogue, and scenes. It is often considered a subset of creative writing due to its imaginative and narrative-driven qualities~\cite{kerrigan2016re, batty2015screenwriter}. However, screenwriting is not entirely equivalent to traditional creative writing, as its ultimate goal is to integrate both visual and auditory elements, which aim to support storytelling in film, television, or other audiovisual formats~\cite{dunnigan2019screenwriting}.

\textcolor{black}{The rapid advancement of technology has brought significant attention to the application of AI in screenwriting~\cite{anguiano2023hollywood}. Previous research has explored challenges in screenwriting, such as story background information retrieval~\cite{pavel2015sceneskim}, ensuring narrative consistency~\cite{sanghrajka2017lisa, kapadia2015computer}, analyzing emotions through plot units~\cite{goyal2010toward}, collaborative screenwriting~\cite{schank2013scripts, bartindale2016tryfilm}, and 2D and 3D visualizations of the script content~\cite{10.1145/3172944.3172972, kim2017visualizing, won2014generating}. With developments in AI technologies such as text-to-image generation and large language models, AI capabilities have shifted from processing and understanding to content generation. Recent studies highlight the role of generative AI in supporting screenwriting~\cite{luchen2023chatgpt, chow2020ghost, 10.1145/3544548.3581225}, including instances where AI has produced screenplay elements surpassing human efforts~\cite{ccelik4ai}, its applications in screenwriting education~\cite{brako2023robots}, and the copyright challenges associated with its use~\cite{kavitha2023copyright}.}

\textcolor{black}{While these studies demonstrate AI's growing role, they primarily focus on specific workflow stages or features. They do not fully examine how traditional screenwriting practices are being transformed by AI or capture screenwriters' attitudes toward AI and their expectations for future AI tools. Meanwhile, although previous works have explored various ways to integrate AI to improve efficiency in screenwriting, they have also raised concerns among screenwriters, including strikes driven by fears of being replaced by AI~\cite{bhattacharya2023hollywood, cheng2024research}. This underscores the significant impact AI has already had on the screenwriting profession. Therefore, we argue that a deeper understanding of screenwriters’ needs is essential to inform the design of future AI systems that better support human-AI co-creation in screenwriting.}

\textcolor{black}{To address this gap, our research explores screenwriters' current practices, attitudes, and expectations for co-creating with AI, focusing on three research questions:}
\end{comment}

%Screenwriting refers to the process of crafting screenplays for visual media, encompassing detailed descriptions of plot structure, dialogue, and scenes. While it is often categorized as a subset of creative writing due to its imaginative and narrative-driven nature~\cite{kerrigan2016re, batty2015screenwriter}, screenwriting differs significantly in its goal of integrating visual and auditory elements to support storytelling in film, television, or other audiovisual formats~\cite{dunnigan2019screenwriting}. \textcolor{black}{Although both creative writing and screenwriting face challenges such as a lack of inspiration and the absence of valuable feedback~\cite{10.1145/3544548.3580782}, the unique demands of screenwriting require writers to incorporate visual and auditory elements into their thinking~\cite{duncan2020guide}. Screenwriting, as a form, aims to evoke emotional responses from the audience through performance, dialogue, and action, relying on the interplay between characters and scene settings~\cite{10.1145/3656650.3656688}.}
%Although both creative writing and screenwriting face challenges such as a lack of inspiration and the absence of valuable feedback~\cite{10.1145/3544548.3580782}, the unique demands of screenwriting require writers to integrate visual and auditory elements into their thinking to craft high-quality films~\cite{duncan2020guide}.}

%\textcolor{black}{Although both creative writing and screenwriting face challenges such as a lack of inspiration, content inconsistency, and limited emotional resonance, the unique demands of screenwriting require screenwriters to integrate visual and auditory elements into their thinking to craft high-quality films.}

%Although both creative writing and screenwriting face challenges such as a lack of inspiration, content inconsistency, and limited emotional resonance, the unique demands of screenwriting require screenwriters to possess the ability to integrate visual and auditory elements into their thinking to craft high-quality films.}

%They must consider potential approaches to create screenplays that demonstrate high quality in both visual and auditory presentation to effectively address these issues.}

%\textcolor{black}{Screenwriting involves crafting screenplays for visual media, aiming to evoke emotional responses through detailed plot structures, dialogue, and scenes~\cite{howard1993tools}. Although often regarded as a subset of creative writing due to its imaginative and narrative focus~\cite{kerrigan2016re, batty2015screenwriter}, screenwriting is distinct in its integration of visual and auditory elements to support storytelling in film, television, and other audiovisual formats~\cite{dunnigan2019screenwriting}. While both creative writing and screenwriting face challenges, such as a lack of inspiration and guidance~\cite{10.1145/3544548.3580782}, screenwriting uniquely requires the incorporation of visual and auditory considerations into the creative process, adding complexity to these challenges~\cite{duncan2020guide}.}

\textcolor{black}{Screenwriting involves the creation of screenplays that integrate visual and auditory elements to support storytelling in film, television, and other audiovisual formats~\cite{dunnigan2019screenwriting}, with the aim of evoking emotional responses through detailed plot structures, dialogue, and scenes~\cite{howard1993tools}. 
Belonging to the area of creative writing, screenwriting also has an imaginative and narrative focus~\cite{kerrigan2016re, batty2015screenwriter}.
As a result, it faces similar challenges to creative writing, such as a lack of inspiration and guidance~\cite{10.1145/3544548.3580782}. 
Futhermore, its distinctive emphasis on incorporating visual and auditory considerations adds layers of complexity to the creative processes and necessitates tailored approaches to addressing these challenges~\cite{duncan2020guide}.}

The rapid advancement of technology has enabled the application of AI in screenwriting~\cite{anguiano2023hollywood}. Previous research demonstrates that AI can support screenwriters by summarizing and cross-checking content to ensure the consistency of screenplay text~\cite{sanghrajka2017lisa, kapadia2015computer}, as well as analyzing emotions through plot units~\cite{goyal2010toward}. \textcolor{black}{With developments ranging from natural language processing (NLP) to deep learning (DL), AI capabilities have evolved from processing and summarizing content to generating it, offering innovative solutions to screenwriting challenges.} For instance, AI can generate character roles with diverse appearances and dialogue styles~\cite{10.1145/3613904.3642105}, assisting users in sparking creative ideas.

\textcolor{black}{These advancements in AI have shown their ability to enhance screenwriters’ productivity and creativity, leading to economic and industry-wide impacts~\cite{chow2020ghost}. The role of AI in screenwriting was notably highlighted during the Hollywood writers' strike when the Writers Guild of America and production companies reached an agreement defining AI as a supporting tool rather than a replacement for writers~\cite{nyt_wga_strike_ai}. The agreement prohibits AI from receiving credit, ensures fair compensation for human writers, and allows optional AI assistance for tasks such as script drafting. This reflects the growing integration of advanced AI technologies in screenwriting and their transformative impact on the industry, motivating further exploration of AI’s role in screenwriting.}

%\textcolor{black}{The rapid advancement of technology has brought significant attention to the application of AI in screenwriting, enabling screenwriters to tackle these challenges with AI assistance~\cite{anguiano2023hollywood}. Previous research demonstrates that AI can support screenwriters by summarizing and cross-checking content to ensure script consistency~\cite{sanghrajka2017lisa, kapadia2015computer} and analyzing emotions through plot units~\cite{goyal2010toward}. With developments in AI technologies such as text-to-image generation and large language models, AI capabilities have evolved from processing and summarizing to content generation, offering innovative solutions to screenwriting challenges. For instance, generative AI can adopt character roles with diverse appearances and dialogue styles~\cite{10.1145/3613904.3642105}, helping users spark creative ideas, maintain consistency, and provide potential emotional interpretation for characters and plots. Moreover, AI has already demonstrated its potential to enhance screenwriters’ productivity and creativity, with significant economic and industry-wide implications~\cite{chow2020ghost}. The role of AI in screenwriting was highlighted during the Hollywood writers' strike when the Writers Guild of America and production companies reached an agreement that defined AI as a tool rather than a replacement for writers~\cite{nyt_wga_strike_ai}. The agreement prohibits AI from receiving credit, ensures fair compensation for human writers, and permits optional AI assistance for tasks such as script drafting. This reflects the growing integration of advanced AI technologies in screenwriting and their transformative effect on the industry, motivating further exploration into AI’s role in screenwriting.}

%The use of AI in screenwriting was underscored during the Hollywood writers' strike, where the Writers Guild of America and production companies reached an agreement to establish guidelines for AI usage in the industry~\cite{nyt_wga_strike_ai}. These guidelines define AI as a tool, not a replacement for writers, prohibiting AI from being credited, ensuring fair compensation for human writers, and allowing optional AI assistance for tasks like drafting scripts. This reflects the increasing integration of advanced AI technologies in screenwriting and their transformative impact on the industry. Such growing integration motivates us to delve deeper into the use of AI in screenwriting.}

%\textcolor{black}{While prior studies highlight AI’s expanding role and have examined its impact on screenwriting education~\cite{brako2023robots}, there is limited understanding of how AI is currently utilized across different stages of the screenwriting workflow. To address this gap, we propose \textbf{RQ1}, which aims to explore screenwriters’ existing practices and their engagement with AI tools.}
\textcolor{black}{While prior studies highlight AI’s expanding role and its impact on screenwriting education~\cite{brako2023robots} and the broader filmmaking process~\cite{naji2024employing}, there remains limited understanding of how AI is applied across different stages of the screenwriting workflow. To address this gap, we propose \textbf{RQ1}, which seeks to explore screenwriters’ current practices, including their task allocation involving AI tools.}

\textcolor{black}{Additionally, existing research has primarily focused on topics such as attitudes~\cite{10.1145/3656650.3656688}, copyright~\cite{kavitha2023copyright}, and labor considerations~\cite{chow2020ghost}. However, it provides limited insights into the nuanced relationship between screenwriters’ diverse attitudes toward AI and its practical effects across various stages of their workflows. To address this gap, we propose \textbf{RQ2}, which focuses on understanding the reasons behind screenwriters’ differing attitudes and how these attitudes influence their practices regarding AI integration and their perceptions of AI’s impact.}

\textcolor{black}{Finally, recent work by Grigis and Angeli examined the features, strategies, and outcomes of LLM-assisted playwriting~\cite{10.1145/3656650.3656688}, and the Dramatron system, developed by Mirowski et al.~\cite{10.1145/3544548.3581225}, demonstrated the potential of integrating LLMs into screenwriting workflows. However, these studies focus solely on LLMs, neglecting advancements in AI-generated images and videos. A broader understanding of how emerging and rapidly evolving AI technologies may address the dynamic and evolving needs of screenwriters is still lacking. To address this gap, we propose \textbf{RQ3}, which aims to explore screenwriters’ detailed expectations for future AI tools.}

%\textcolor{black}{While prior studies highlight AI’s expanding role and have examined its impact on screenwriting education~\cite{brako2023robots}, there is limited understanding of how AI is currently utilized specifically in the screenwriting workflow. To address this gap, we propose \textbf{RQ1}, which aims to explore screenwriters’ existing practices and their engagement with AI tools. Additionally, existing research has primarily explored topics such as attitudes~\cite{10.1145/3656650.3656688}, copyright~\cite{kavitha2023copyright}, and labor considerations~\cite{chow2020ghost}. However, it provides limited insights into the nuanced relationship between screenwriters’ diverse attitudes toward AI and its practical effects across different stages of their workflows. To address this gap, we propose \textbf{RQ2}, focusing on understanding the reasons behind screenwriters’ differing attitudes and how these attitudes influence their actions regarding AI integration and their perceptions of AI’s impact. Finally, recent work by Grigis and Angeli examined the features, strategies, and outcomes of LLM-assisted screenwriting~\cite{10.1145/3656650.3656688}, while the system Dramatron, developed by Mirowski et al.~\cite{10.1145/3544548.3581225}, demonstrated the potential of integrating LLMs into screenwriting workflows. However, these studies focused exclusively on LLM technology, overlooking developments in other areas such as AI-generated images and videos. There is still a lack of broader implications of how emerging and ever-expanding AI technologies may address the evolving and dynamic needs of screenwriters. To address this gap, we propose \textbf{RQ3}, which aims to explore screenwriters’ detailed expectations for future AI tools.}
%Additionally, existing research has predominantly addressed issues related to attitudes~\cite{10.1145/3656650.3656688}, copyright~\cite{kavitha2023copyright}, and labor considerations~\cite{chow2020ghost}, but it lacks insights into the detailed relationship between screenwriters’ diverse attitudes toward AI and the practical impacts of AI on their various workflow stages.
%Additionally, existing research has primarily focused on copyright~\cite{kavitha2023copyright} and labor considerations~\cite{chow2020ghost}, and lacks insights into the relationship between screenwriters’ varying attitudes toward AI and the practical effects of AI on their workflows.
%Additionally, existing work focused primarily on a specific aspect such as copyright~\cite{kavitha2023copyright}, lacking insights into the relationship between screenwriters’ different attitudes toward AI and the practical effects of AI on their workflows.
%Additionally, the impact of AI on copyright in screenwriting has garnered attention~\cite{kavitha2023copyright}, but existing research lacks insights into the relationship between screenwriters’ attitudes toward AI and the practical effects of AI on their workflows.

%Finally, the recent work by Grigis and Angeli for understanding perceived features, interaction strategies and outcomes for LLM assist screenwriting with limited interviewees and the system Dramatron made by Mirowski等人has demonstrated the potential of integrating LLMs into screenwriting workflows~\cite{10.1145/3544548.3581225}. However, this study focused solely on LLM technology, overlooking other advancements such as AI-generated images or videos. It did not account for the broader scope of emerging AI technologies or consider screenwriters’ evolving needs in this context. To address this gap, we propose \textbf{RQ3}, which aims to explore screenwriters’ detailed expectations for future AI tools.}
%Finally, the recent system Dramatron has demonstrated that combining LLMs with strictly linear workflows can support certain aspects of screenwriting~\cite{10.1145/3544548.3581225}. However, such systems limit creative flexibility and fail to address screenwriters’ expectations for AI functionality and their detailed needs. To bridge this gap, we propose \textbf{RQ3}, aiming to understand screenwriters’ detailed expectations for future AI tools.}

%\textcolor{orange}{To enhance our understanding of AI's current usages, attitudes, and potential roles in screenwriting, we conducted an in-depth interview study with professionals in the industry. In our study, we began with a holistic examination of the existing practices of working with AI (\textbf{RQ1}). This understanding serves as the foundation for further exploration of professionals' attitudes toward AI (\textbf{RQ2}). However, such aspects have been overlooked in previous studies. For instance, Brako et al. focused on the impact of AI on screenwriting education but neglected the specific ways AI is used within screenwriters' actual creative workflows~\cite{brako2023robots}. Additionally, while Kavitha's research addresses the impact of AI on copyright in screenwriting~\cite{kavitha2023copyright}, it lacks insights into the relationship between screenwriters’ attitudes toward AI and the practical effects of AI on their workflows. Building on these gaps, we aimed to further explore the reasons behind screenwriters’ differing attitudes and how these attitudes influence their actions regarding AI integration and their perceptions of AI’s impact. To better understand screenwriters’ needs, we also investigated their expectations for AI in the future (\textbf{RQ3}), with the goal of providing comprehensive insights for the development of AI systems designed to support screenwriting.}

%\textcolor{black}{Based on these, this study addresses the following three research questions:}

\begin{enumerate}
    \item[\textbf{RQ1:}] How is AI involved in existing practices within the screenwriting workflow?
    \item[\textbf{RQ2:}] What are the attitudes of screenwriters towards the integration of AI into their workflow?
    \item[\textbf{RQ3:}] How do screenwriters envision the future roles of AI in the creative process?
\end{enumerate}

% \vspace{0.3cm}

% \noindent \textbf{RQ1:} How is AI involved in existing practices within the screenwriting workflow?

% %What are the existing practices AI involved within the screenwriting workflow?

% %What are the existing practices involving AI within the screenwriting workflow, and how is AI being integrated into these processes?

% %What and how are the existing practices involving AI within the screenwriting workflow?
% %What are the existing practices involving AI in task allocation and tool usage within the screenwriting process?


% \vspace{0.1cm}

% \noindent \textbf{RQ2:} What are the attitudes of screenwriters towards the integration of AI into their workflow?

% \vspace{0.1cm}

% \noindent \textbf{RQ3:} How do screenwriters envision the future roles of AI in the creative process?

% \vspace{0.3cm}

We conducted in-depth interviews with 23 screenwriters to address these research questions. For \textbf{RQ1}, most participants (78\%) reported integrating AI into workflows, primarily in stages such as story structure \& plot, screenplay text, goal \& idea generation, and dialogue. \textcolor{black}{For \textbf{RQ2}, participants highlighted AI's strengths in reducing trial-and-error costs, expanding knowledge boundaries, and inspiring overlooked creative ideas but noted challenges such as high usage barriers, inaccuracy, and limited emotional understanding. Opinions on structured and pastiche generation were mixed. Regarding broader societal impacts, participants expressed optimism about AI’s potential to support divergent thinking and industry communication but raised concerns about authorship and copyright. Views on AI as a competitor in screenwriting were divided.} For \textbf{RQ3}, we identified participants' desired AI features in four roles: ``actor'' for character simulation, ``audience'' for feedback, ``expert'' for professional advice, and ``executor'' for completing tasks. These findings provide insights to guide the development of future AI tools for screenwriting.

%We conducted in-depth interviews with 23 screenwriters to address these research questions. For \textbf{RQ1}, most participants (78\%) reported integrating AI into workflows, primarily in stages such as story structure \& plot, screenplay text, goal \& idea generation, and dialogue. \textcolor{black}{For \textbf{RQ2}, participants highlighted AI's strengths in reducing trial-and-error costs, expanding knowledge boundaries, and inspiring overlooked ideas but noted challenges such as high usage barriers, inaccuracy, and limited emotional understanding. Opinions on structured and pastiche generation were mixed. Regarding broader impacts, participants expressed optimism about AI’s potential to support divergent thinking and industry communication but raised concerns about authorship and copyright. Views on AI as a competitor in screenwriting were divided.} For \textbf{RQ3}, we identified participants' desired AI features in four roles: ``actor'' for character simulation, ``audience'' for feedback, ``expert'' for professional advice, and ``executor'' for completing tasks. These findings provide insights to guide the development of future AI tools for screenwriting.



%We conducted in-depth interviews with 23 screenwriters to address these research questions. For \textbf{RQ1}, we analyzed workflows to identify challenges addressed through AI and summarized tasks allocated to AI at each stage. Most participants (78\%) reported integrating AI into workflows, primarily in stages such as story structure \& plot, screenplay text, goal \& idea generation, and dialogue. \textcolor{black}{For \textbf{RQ2}, participants highlighted AI's strengths in reducing trial-and-error costs, expanding knowledge boundaries, and inspiring overlooked ideas but noted challenges such as high usage barriers, inaccuracy, and limited emotional understanding. Opinions on structured and pastiche generation were mixed. Regarding broader impacts, participants expressed optimism about AI’s potential to support divergent thinking and industry communication but raised concerns about authorship and copyright. Views on AI as a competitor in screenwriting were divided.} For \textbf{RQ3}, we identified participants' desired AI features in four roles: ``actor'' for character simulation, ``audience'' for feedback, ``expert'' for professional advice, and ``executor'' for completing tasks. These findings provide insights to guide the development of future AI tools for screenwriting.

\begin{comment}
We addressed these research questions through in-depth interviews with 23 screenwriters. For \textbf{RQ1}, we analyzed workflows to identify challenges addressed through AI collaboration and summarized AI's allocation across stages. Most participants (78\%) reported integrating AI into workflows, with prominent use in stages such as story structure \& plot, screenplay text, goal \& idea generation, and dialogue. \textcolor{black}{For \textbf{RQ2}, we examined participants' attitudes toward AI. They highlighted strengths, including reducing trial-and-error costs, expanding knowledge boundaries, and inspiring overlooked ideas, while identifying challenges such as high usage barriers, inaccuracy, and limited emotional understanding. Opinions on structured and pastiche generation were mixed. Participants expressed optimism about AI’s potential to support divergent thinking and improve industry communication but raised concerns related to authorship and copyright. Views on AI as a competitor in screenwriting were divided.} For \textbf{RQ3}, participants outlined desired AI features categorized into four roles. The ``actor'' role simulates characters, the ``audience'' role provides feedback, the ``expert'' role delivers professional advice, and the ``executor'' role transforms ideas into outcomes. These findings offer valuable insights for guiding the design of future AI-powered tools for screenwriting.

    

We addressed these research questions through in-depth interviews with 23 screenwriters. 
Specifically, to address \textbf{RQ1}, based on screenwriters' workflow, we first understood their challenges that they hope to resolve through collaborating with AI.
Next, we summarized how AI is allocated to address their needs.
We found that the majority of screenwriters (78\%) have integrated AI into their traditional workflows.
Among the stages where AI is currently applied, story structure \& plot, screenplay text, goal \& idea generation, and dialogue are particularly prominent.

Then, based on the practices, their attitudes toward AI usage were examined and summarized (\textbf{RQ2}), revealing a mix of optimism about its potential to enhance efficiency and creativity, concerns over its limitations in handling complex tasks, and contradictory opinions on its effectiveness in structured writing and content pastiche.
Finally, we asked the participants to consider their desired assistance from AI, such as ideal AI features and their interactions with AI.
We realized that the demand for additional AI support is most evident for story structures and plots, screenplay text, and dialogue stages.
To better characterize how AI can work with screenwriters, we summarize four roles, i.e., actor, audience, expert, and executor, that AI can serve in future co-creation (\textbf{RQ3}).

``Actor'' enhances creativity by simulating characters, ``audience'' provides feedback from diverse perspectives, ``expert'' offers professional advice and workflow guidance, and ``executor'' carries out creative tasks to boost efficiency.
Based on our interview results, we further identified a series of design opportunities for AI-powered tools to support screenwriting.
\end{comment}

%Then, based on the practices, their attitudes toward AI usage was enquired and summarized (\textbf{RQ2}).
%\haotian{Can we add one finding about attitude here?}

%These roles include actor, audience, expert, and executor.
%\haotian{Please briefly introduce the roles.}

% First, we found that the majority of screenwriters (78\%) have integrated AI into their traditional workflows. Among the stages where AI is currently applied, story structure \& plot, screenplay text, goal \& idea generation, and dialogue are particularly prominent. 
% The demand for additional AI support is most evident in the stages of story structure \& plot development, screenplay text, and dialogue stages. Additionally, screenwriters expressed expectations for future AI tools, specifically hoping for AI to take on four roles in the screenwriting workflow: actor, audience, expert, and executor. 


%Through in-depth interviews with 23 screenwriters, we addressed these research questions. First, we found that the majority of screenwriters (78\%) have integrated AI into their traditional workflows. Among the stages where AI is currently applied, goal \& idea generation, character development, story structure \& plot, and dialogue are particularly prominent. The demand for additional AI support is most evident in the stages of story structure \& plot development, screenplay text, and dialogue stages. Additionally, screenwriters expressed expectations for future AI tools, specifically hoping for AI to take on four roles in the screenwriting workflow: actor, audience, expert, and executor. In summary, our study offers the following contributions:

%Through in-depth interviews with 23 screenwriters, we addressed these research questions. First, we found that the majority of screenwriters (78\%) have integrated AI into their traditional workflows. Among the stages where AI is currently utilized, goal \& idea generation, as well as character development, stand out as stages where screenwriters express higher satisfaction. However, the demand for further AI support is pronounced in the story structure \& plot development, screenplay text, and dialogue stages. When it comes to the advantages and disadvantages of AI, we found that screenwriters' attitudes are multifaceted. On the one hand, they acknowledge AI's potential to enhance their workflows and hold optimistic views about its future development. On the other hand, they express concerns about long-term over-reliance on AI and its current limited capabilities. Additionally, based on screenwriters' expectations, we identified four role-oriented needs for future AI tools: actor, audience, expert, and executor. In summary, our study offers the following contributions:

%\textcolor{black}{We addressed these research questions through in-depth interviews with 23 screenwriters. For \textbf{RQ1}, we analyzed screenwriters' workflows to identify challenges addressed through AI collaboration and summarized AI’s allocation across workflow stages. Most screenwriters (78\%) reported integrating AI into traditional workflows, with prominent applications in story structure \& plot, screenplay text, goal \& idea generation, and dialogue.}

%\textcolor{black}{To address \textbf{RQ2}, we examined participants' attitudes toward AI integration. Participants noted AI’s strengths in reducing trial-and-error costs, expanding knowledge boundaries, and inspiring overlooked ideas while highlighting challenges such as high usage barriers, inaccuracy, and limited emotional understanding. Opinions on structured and pastiche generation were mixed. From a broader perspective, participants were optimistic about AI’s potential to support divergent thinking and improve industry communication but expressed concerns about authorship and copyright. Views on AI as a competitor in screenwriting were divided.}

%\textcolor{black}{For \textbf{RQ3}, participants identified desired AI features and interactions, which we categorized into four roles. The ``actor'' role enhances creativity by simulating characters, particularly in the character, and goal \& idea stages. The ``audience'' role offers feedback from diverse perspectives during the goal \& idea, character, and screenplay text stages. The ``expert'' role provides professional advice across all stages. The ``executor'' role transforms ideas into outcomes, especially in the character and story structure \& plot stages. These findings inform design opportunities for future AI-powered tools to support screenwriting.}

%\textcolor{black}{For \textbf{RQ3}, participants identified desired AI features and interactions, which we categorized into four roles: ``actor,'' enhancing creativity by simulating characters, particularly in the character, and goal \& idea stages; ``audience,'' offering feedback from diverse perspectives during the goal \& idea, character, and screenplay text stages; ``expert,'' providing professional advice across all stages; and ``executor,'' transforming ideas into outcomes, especially in the character and story structure \& plot stages. These findings informed design opportunities for future AI-powered tools to support screenwriting.}

%We addressed these research questions through in-depth interviews with 23 screenwriters. Specifically, to address \textbf{RQ1}, we first analyzed screenwriters' workflows to understand the challenges they hope to resolve through collaboration with AI. Next, we summarized how AI is allocated to meet their needs across different workflow stages. Our findings reveal that the majority of screenwriters (78\%) have integrated AI into their traditional workflows. Among the stages where AI is currently applied, story structure \& plot, screenplay text, goal \& idea generation, and dialogue are particularly prominent. Then, based on their practices, participants' attitudes toward AI were examined and summarized (\textbf{RQ2}). \textcolor{black}{Participants highlighted AI’s strengths in reducing trial-and-error costs, expanding knowledge boundaries, and inspiring overlooked ideas but noted challenges, including high usage barriers, inaccuracy, and limited emotional understanding. Opinions on structured and pastiche generation were mixed. From a broader perspective, while participants were optimistic about AI's potential to support divergent thinking and improve industry communication, concerns about authorship and copyright persisted. Views on AI's impact as a competitor in the future of screenwriting remained divided.} Finally, we asked participants to consider their desired assistance from AI, including ideal features and preferred interactions. To better characterize how AI could collaborate with screenwriters, we identified four roles that AI could serve in future co-creation (\textbf{RQ3}): ``actor,'' ``audience,'' ``expert,'' and ``executor.'' ``Actor'' enhances creativity by simulating characters, \textcolor{black}{particularly in the character, and goal \& idea stages}. ``audience'' provides feedback from diverse perspectives \textcolor{black}{during the goal \& idea, character, and screenplay text stages to refine the work}, ``expert'' offers professional advice and guidance, \textcolor{black}{and is seen as beneficial across all stages}, and ``executor'' transforms ideas into outcomes, \textcolor{black}{especially in the character and story structure \& plot stages}. Based on our interview results, we further identified a series of design opportunities for AI-powered tools to support screenwriting.

%\textcolor{black}{Participants acknowledged AI’s strengths in rapid result generation and information management, enhancing productivity and creativity, but criticized its high usage barriers, inaccuracy, lack of originality, and limited emotional support. Opinions on structured and pastiche generation were mixed. From a broader perspective, while participants were optimistic about AI's potential to support divergent thinking and improve industry communication, concerns about its lack of unique life experiences and ethical issues persisted. Views on AI as a competitor's impact on the future of screenwriting remained divided.}
%Participants acknowledged AI's strengths in rapid result generation and information management but criticized its high usage barriers, inaccuracy, lack of originality, and limited emotional support. Opinions on structured and pastiche generation were mixed. From the broader perspective of AI's impact on the screenwriting industry, while participants were optimistic about AI's potential to support divergent thinking and enhance industry communication, concerns about its lack of unique life experiences and ethical issues persisted. Views on AI as a competitor's impact on screenwriting's future also remained divided.} 
%\textcolor{black}{Additionally, we found that the demand for further AI support is most evident in the divergent stages of character, goal \& idea, story structures and plots, dialogue, and screenplay text. The functions with the greatest potential to meet participants' expectations include targeted continuation and expansion based on input, providing guidance and optimization, supporting multimodal output formats, simulating actors' external behaviors, user as an external character engaging with AI, and user as an observer engaging with AI.} 



In summary, our study offers the following contributions:

\begin{itemize}
\item We provide insights into the current AI-involved screenwriting practices, revealing the multilayered and diverse ways in which AI tools are utilized.

\item We summarize the complex attitudes screenwriters hold towards AI technology and identify the specific stages within the workflow where AI will be needed in the future.

\item \textcolor{black}{We identify four potential roles that AI could play in screenwriting, aiming to clarify screenwriters' specific functional expectations for AI tools in future co-creation.}
%We propose a role-oriented design framework for future AI tools, further clarifying screenwriters' specific functional expectations for AI.
\end{itemize}


\section{Related Work}
\textcolor{black}{Our study builds on prior research on AI integration in creative processes, AI tools for creative writing, and AI-supported screenwriting, highlighting gaps specific to screenwriting. These gaps stem from its unique characteristics, including audiovisual storytelling~\cite{senje2017formatting}, structured principles~\cite{mckee1997substance}, and multi-stakeholder collaboration~\cite{taylor2024one, cake2021collaborative, batty2018script}. Previous studies have not fully incorporated evolving AI capabilities to consider these screenwriting characteristics, underscoring the need for further investigation.}

%\textcolor{black}{Our study builds on previous research in AI integration in creative processes, AI tools for creative writing, and AI-supported screenwriting.}

%\textcolor{black}{Our study builds on prior research into AI integration in creative processes, AI tools for creative writing, and AI-supported screenwriting, identifying research gaps specific to screenwriting as a unique form of creative writing within the broader creative industry.}

%\textcolor{black}{Our study builds on prior research into AI integration in creative processes, AI tools for creative writing, and AI-supported screenwriting, identifying specific research gaps in screenwriting. These gaps arise from the unique characteristics of screenwriting, including audiovisual storytelling~\cite{senje2017formatting}, structured principles~\cite{mckee1997substance}, and collaboration with multiple stakeholders~\cite{taylor2024one, cake2021collaborative, batty2018script}, which distinguish it within the broader creative industry.}

%\textcolor{black}{Our study builds on prior research into AI integration in creative processes, AI tools for creative writing, and AI-supported screenwriting, identifying specific research gaps in AI integration for screenwriting. These gaps arise from the unique characteristics of screenwriting, including audiovisual storytelling~\cite{senje2017formatting}, structured principles~\cite{mckee1997substance}, and collaboration with multiple stakeholders~\cite{taylor2024one, cake2021collaborative, batty2018script}, which distinguish it within the broader creative industry and warrant further attention.}

\subsection{\textcolor{black}{AI Integration in Creative Processes}}

\textcolor{black}{Advancements in AI technology have significantly impacted various creative domains~\cite{anantrasirichai2022artificial, 10.1145/3613904.3642726}, reshaping fields such as art~\cite{cetinic2022understanding} and design~\cite{lee2023impact}. In art, AI has contributed to interactive media and visual representation~\cite{anadol2018archive, sun2023ai, tang2023ai}. In design, AI has been integrated into product design workflows, fostering innovation in idea generation~\cite{hong2023generative, chiou2023designing} and advancing user interface design through features such as adaptability~\cite{cheng2023play, zhang2023layoutdiffusion}. Furthermore, the impact of AI-generated image tools on workflows has been explored~\cite{mustafa2023impact, tang2024s}, highlighting AI’s potential to democratize graphic design~\cite{tang2024exploring} and emphasizing the need for interdisciplinary collaboration to enhance AI systems~\cite{meron2022graphic}. As AI-generated video capabilities advance, AI is also reshaping fashion design through dynamic visuals~\cite{liu2019toward, karras2023dreampose} and supporting the creation of short-form social media videos~\cite{10.1145/3613904.3642476}. Additionally, other research highlights AI's role in various creative domains, such as design~\cite{10.1145/3613904.3642812}, and data storytelling~\cite{10.1145/3613904.3642726}.}

%\textcolor{black}{The growing importance of effective human-AI co-creation, as highlighted in prior research within visual-based creative domains, motivates us to investigate whether the diverse AI capabilities and applications in these areas have similarly impacted text-based creative works.}

The growing importance of effective human-AI co-creation, as highlighted in prior research within visual-based creative domains, motivates us to investigate how diverse AI capabilities and applications have impacted text-based creative work.


%\textcolor{black}{The significance of effective human-AI co-creation has become increasingly evident, building on prior research in these visual-based creative domains. This inspires us to explore whether the diverse AI capabilities and applications observed in visual-based domains have also influenced text-based creative works.}


\subsection{\textcolor{black}{AI Tools for Creative Writing}}
\textcolor{black}{Creative writing, as a text-based creative work encompassing novels, short stories, poetry, and essays, emphasizes linguistic artistry, stylistic diversity, and imaginative narrative freedom~\cite{ramet2011creative, smith2020writing, mcvey2008all}. The development of AI technologies has significantly supported this creative field~\cite{10.1145/3613904.3642529}, evolving from natural language processing to advanced deep learning systems. This progression spans basic tools like spell checkers~\cite{peterson1980computer}, crowdsourcing platforms~\cite{kim2017mechanical, kim2014ensemble}, and crowd role-play approaches~\cite{huang2020heteroglossia}, to AI-driven character role-playing systems~\cite{10.1145/3613904.3642105}. Prior studies have shown that AI reduces creative workloads by providing valuable suggestions~\cite{gero2019stylistic, bernstein2010soylent} and refining content for professional use~\cite{hui2018introassist, roemmele2015creative, clark2018creative}, thereby helping writers overcome creative blocks. Specifically, AI-supported creative writing processes are currently categorized into story development and character creation~\cite{radford2019language, yang2019sketching, calderwood2020novelists, clark2018creative, coenen2021wordcraft}.}

%\textcolor{black}{Creative writing, as a text-based creative work encompassing novels, short stories, poetry, and essays, emphasizes linguistic artistry, stylistic diversity, and imaginative freedom~\cite{ramet2011creative, smith2020writing, mcvey2008all}. The development of AI technologies has significantly supported this field~\cite{10.1145/3613904.3642529}, evolving from natural language processing to advanced deep learning systems. This progression spans basic tools like spell checkers~\cite{peterson1980computer}, crowdsourcing platforms~\cite{kim2017mechanical, kim2014ensemble}, and crowd role-play approaches~\cite{huang2020heteroglossia}, to AI-driven character role-playing systems~\cite{10.1145/3613904.3642105}. Prior studies have shown that AI reduces creative workloads by providing valuable suggestions~\cite{gero2019stylistic, bernstein2010soylent} and refining content for professional use~\cite{hui2018introassist, roemmele2015creative, clark2018creative}, thereby helping writers overcome creative blocks. Specifically, AI-supported creative writing processes are currently categorized into story development and character creation~\cite{radford2019language, yang2019sketching, calderwood2020novelists, clark2018creative, coenen2021wordcraft}.}

\textcolor{black}{In story development, platforms such as Dramatica~\cite{dramatica} and AI Dungeon~\cite{aidungeon} support story generation. TaleStream combines AI recommendation algorithms with database-driven trope suggestions for narratives but has yet to integrate evolving generative AI technologies~\cite{chou2023talestream}. Rule-based methods for story extension have also been explored~\cite{lebowitz1984creating, meehan1977tale, riedl2010narrative}. However, we found that these approaches primarily focus on text-based support and do not incorporate visual elements. In character creation, prior works demonstrate that AI can simulate characters' personalities~\cite{cavazza2001characters}, infer characters' relationships~\cite{chaturvedi2017unsupervised}, facilitate AI-character dialogues~\cite{10.1145/3613904.3642105}, and map character trajectories~\cite{chung2022talebrush}. These tools showcase the potential of AI for inspiring character development~\cite{10.1145/3450741.3465253, shakeri2021saga, yuan2022wordcraft}, but they are often limited to isolated stages and lack integration across the creative workflow. Additionally, challenges in human-AI collaboration remain. Grigis and Angeli recently examined the limitations of LLM-assisted writing, particularly in handling taboos and conflicts~\cite{10.1145/3656650.3656688}. Dhillon et al. emphasized the importance of personalized systems to accommodate the diverse experiences of writers~\cite{10.1145/3613904.3642134}, and Lee et al. provided broad recommendations for the development of future AI tools~\cite{10.1145/3613904.3642697}. Meanwhile, Biermann et al. underscored writers’ concerns about the potential impact of AI on creativity~\cite{biermann2022tool}.}

\textcolor{black}{Based on these, we note that existing research largely focuses on text-based AI support, often overlooking the consideration of visual elements in the creative process. 
This limitation reduces the suitability of these approaches for certain types of creative writing, particularly screenwriting, which is inherently characterized by audiovisual and dynamic storytelling~\cite{senje2017formatting}. Furthermore, existing studies often focus on isolated stages of the creative process, overlooking the complete workflow that integrates structured principles~\cite{mckee1997substance} and collaboration with stakeholders~\cite{taylor2024one, cake2021collaborative, batty2018script}, both of which are essential to screenwriting. These gaps underscore the need to explore AI's current applications in screenwriting.}


%\textcolor{black}{Based on these, we note that existing works primarily focus on text-based AI support formats, often overlooking the need to integrate visual elements in the creation processes. 
%\textcolor{black}{Based on these, we note that existing research predominantly emphasizes text-based AI support, often neglecting the importance of integrating visual elements into the creative process. 
%\textcolor{black}{Based on these, we note that existing research largely focuses on text-based AI support, overlooking the role of visual elements in the creative process. This limitation reduces the suitability of these approaches for certain types of creative writing, particularly screenwriting, which is inherently characterized by audiovisual storytelling~\cite{senje2017formatting}. Furthermore, existing studies often focus on isolated stages of the creative process, overlooking the complete workflow that integrates structured principles~\cite{mckee1997substance} and collaboration with stakeholders~\cite{taylor2024one, cake2021collaborative, batty2018script}, both of which are essential to screenwriting. These gaps underscore the need to explore AI's current applications in screenwriting.}

%Additionally, challenges in human-AI collaboration persist. Lee et al. offered recommendations for future AI tools~\cite{10.1145/3613904.3642697}, and Dhillon et al. emphasized the need for personalized systems to accommodate writers’ varying experiences~\cite{10.1145/3613904.3642134}. Meanwhile, Biermann et al. highlighted writers’ concerns about AI’s influence on creativity~\cite{biermann2022tool}.}

%\textcolor{black}{Based on these, we note that existing works primarily focus on text-based output formats for AI support, often overlooking the potential of visual elements. This limitation reduces the suitability of these approaches for certain types of creative writing, particularly screenwriting, which is inherently characterized by audiovisual storytelling~\cite{senje2017formatting}. Additionally, existing studies frequently address isolated stages of the creative process, neglecting the entire workflow combined with structured principles~\cite{mckee1997substance} and the collaboration with stakeholders~\cite{taylor2024one, cake2021collaborative, batty2018script}, both of which are integral to screenwriting. These gaps underscore the need to explore AI's current applications in screenwriting.}
%These gaps underscore the importance of exploring AI applications in screenwriting and investigating how AI can be effectively integrated into screenwriting workflows.}

\subsection{\textcolor{black}{AI-Supported Screenwriting}}

\textcolor{black}{The development of AI technologies has already influenced screenwriting~\cite{batty2015screenwriter, anguiano2023hollywood}. Previous applications of AI tools in screenwriting have primarily focused on three areas: information processing, emotional support, and visualization.}

%\textcolor{black}{For information processing, AI tools assist with retrieving background information~\cite{pavel2015sceneskim}, ensuring narrative consistency~\cite{sanghrajka2017lisa, kapadia2015computer}, and supporting story continuation~\cite{valls2016error, mateas2003experiment}. Traditional tools like Final Draft~\cite{finaldraft} integrate AI for scene organization and formatting, while AI systems like ChatGPT~\cite{chatgpt, luchen2023chatgpt} and DeepStory~\cite{deepstory} offer predictive analytics and co-writing capabilities~\cite{chow2020ghost}, streamlining the screenwriting process. For emotion support, earlier studies have explored emotion analysis in narratives~\cite{stapleton2003interactive}. Goyal et al.'s AESOP system analyzes emotions through plot units, helping screenwriters understand emotional trajectories~\cite{goyal2010toward}. However, it relies on basic emotion tags (e.g., positive, negative, neutral), failing to capture complex emotional dynamics. Su et al.'s work on simulating basic and mixed emotions~\cite{su2007personality} improved character expression but lacked deeper insight into emotions tied to plans and reasoning, limiting its integration into the screenwriting workflow. For visualization, previous research is divided into data visualization and visual representation. In data visualization, studies have summarized storylines~\cite{tapaswi2014storygraphs} and managed character arcs~\cite{kim2017visualizing}, providing a foundation for data-driven AI systems. In visual representation, AI tools use NLP to create 2D and 3D visualizations of screenplay content~\cite{10.1145/3172944.3172972, kim2017visualizing}, including character interactions~\cite{won2014generating} and scenes~\cite{hanser2009scenemaker}. However, these tools lack real-time feedback, and suitable visual representation, and rely on the quality of screenplay content. Beyond these main types of previous research, a recent work closely related to ours is the Dramatron system~\cite{10.1145/3544548.3581225}, which explores the integration of LLMs into screenwriting. However, it focuses only on LLMs, neglecting the impact of AI-generated images and videos.}

%\textcolor{black}{Previous empirical research shows that AI can generate screenplay elements comparable to those of human writers~\cite{ccelik4ai}, using tools like ChatGPT~\cite{chatgpt, luchen2023chatgpt} and DeepStory~\cite{deepstory}, enhancing efficiency, reducing costs~\cite{naji2024employing}, and offering inspiration~\cite{brako2023robots}. However, human creativity remains indispensable~\cite{naji2024employing, song2022analysis}. Furthermore, ethical concerns such as biases~\cite{chow2020ghost} and copyright issues~\cite{kavitha2023copyright} remain significant challenges. However, these studies have neither explored the specific ways AI has been integrated into different stages of screenwriting nor examined how screenwriters' attitudes concretely influence their actual practices.}

\textcolor{black}{For information processing, AI tools assist with retrieving background information~\cite{pavel2015sceneskim}, ensuring narrative consistency~\cite{sanghrajka2017lisa, kapadia2015computer}, and organizing content~\cite{valls2016error, mateas2003experiment}. Tools like Final Draft~\cite{finaldraft} integrate AI for scene management and formatting, streamlining the screenwriting process. For emotional support, earlier studies have explored emotion analysis in narratives~\cite{stapleton2003interactive}. Goyal et al.'s AESOP system analyzes emotions through plot units, helping screenwriters understand emotional trajectories~\cite{goyal2010toward}. However, it relies on basic emotion tags (e.g., positive, negative, neutral), failing to capture complex emotional dynamics. Su et al.'s work on simulating basic and mixed emotions~\cite{su2007personality} improved character expression but lacked deeper insights into emotions tied to plans and reasoning, limiting its integration into screenwriting workflows. For visualization, previous research is divided into data visualization and visual representation. In data visualization, studies have summarized storylines~\cite{tapaswi2014storygraphs} and managed character arcs~\cite{kim2017visualizing}, providing a foundation for data-driven AI systems. In visual representation, AI tools use NLP to create 2D and 3D visualizations of screenplay content~\cite{10.1145/3172944.3172972, kim2017visualizing}, including character interactions~\cite{won2014generating} and scenes~\cite{hanser2009scenemaker}. However, these tools lack real-time feedback, personalized representation styles, and rely heavily on the quality of the screenplay content. Beyond these primary research areas, a recent work closely related to ours is the Dramatron system~\cite{10.1145/3544548.3581225}, which explores the integration of LLMs into screenwriting. However, it focuses solely on LLMs, neglecting the impact of AI-generated images and videos.}

\textcolor{black}{Additionally, previous empirical research has shown that AI can generate screenplay elements comparable to those created by human writers~\cite{ccelik4ai}, leveraging tools like ChatGPT~\cite{chatgpt, luchen2023chatgpt} and DeepStory~\cite{deepstory} to enhance efficiency, reduce costs~\cite{naji2024employing}, and inspire creativity~\cite{brako2023robots}. Despite these advancements, human creativity remains irreplaceable~\cite{naji2024employing, song2022analysis}. Meanwhile, ethical concerns, including biases~\cite{chow2020ghost} and copyright issues~\cite{kavitha2023copyright}, have sparked significant debate. However, these studies have not investigated how AI is integrated into specific stages of screenwriting and how screenwriters' attitudes concretely influence their practices.}

%\textcolor{black}{Additionally, previous empirical research has demonstrated that AI can generate screenplay elements comparable to those created by human writers~\cite{ccelik4ai}, utilizing tools such as ChatGPT~\cite{chatgpt, luchen2023chatgpt} and DeepStory~\cite{deepstory} to enhance efficiency, reduce costs~\cite{naji2024employing}, and provide creative inspiration~\cite{brako2023robots}. Despite these advancements, human creativity remains indispensable~\cite{naji2024employing, song2022analysis}. Meanwhile, ethical concerns, such as biases~\cite{chow2020ghost} and copyright issues~\cite{kavitha2023copyright}, have also sparked considerable discussion. However, these studies have not explored the specific ways in which AI is integrated into different stages of screenwriting, nor have they examined how screenwriters' attitudes concretely influence their practices.}

\textcolor{black}{To address these gaps, our study aims to examine screenwriters' current practices and attitudes toward AI integration at various stages within the workflow. Additionally, we investigate screenwriters’ expectations for future AI tools, considering both the potential of emerging AI technologies and possibilities beyond current advancements.} 
Ultimately, we provide suggestions for designing tailored human-AI co-creation tools that meet screenwriters' needs.

%\textcolor{black}{To address these gaps, our study aims to examine screenwriters' current practices and attitudes toward AI integration at specific stages of their workflows. Then, we investigate screenwriters’ expectations for future AI tools, taking into account both the potential of emerging AI technologies and possibilities beyond current advancements. Ultimately, our findings aim to offer suggestions for designing human-AI co-creation tools for screenwriting that are specifically tailored to the needs of screenwriters.}

%\textcolor{black}{To address these gaps, our study aims to explore screenwriters' current practices and attitudes toward AI integration at specific stages of the workflow. Additionally, it investigates screenwriters’ expectations for future AI tools, considering both the capabilities of evolving AI technologies and possibilities beyond current advancements. Overall, our findings seek to provide suggestions for the future design of human-AI co-creation tools for screenwriting that are tailored to the specific needs of screenwriters.}

%\textcolor{black}{Building on prior research in visual-based creative domains, the significance of effective human-AI co-creation has become increasingly evident. This inspires us to investigate whether the diverse AI capabilities and applications demonstrated in visual-based creative tasks have influenced text-based creative works.}

%\textcolor{black}{Advancements in AI have transformed creative domains such as the arts and design~\cite{anantrasirichai2022artificial, 10.1145/3613904.3642726, cetinic2022understanding, lee2023impact, 10.1145/3613904.3642812}. In art, AI supports interactive media and visual representation~\cite{anadol2018archive, sun2023ai, tang2023ai}. In design, it fosters innovation in product workflows~\cite{hong2023generative, chiou2023designing} and advances user interface adaptability~\cite{cheng2023play, zhang2023layoutdiffusion}. AI-generated image tools further democratize graphic design~\cite{tang2024exploring, mustafa2023impact, tang2024s}, with interdisciplinary collaboration advocated to improve AI systems~\cite{meron2022graphic}. As AI-generated video capabilities expand, AI is transforming fashion design~\cite{liu2019toward, karras2023dreampose} and assisting social media creators~\cite{10.1145/3613904.3642476}.}

%\textcolor{black}{Beyond these visual-based creative domains, AI’s impact also extends to diverse areas such as data-driven work~\cite{10.1145/3613905.3650841, muller2019data, wang2021much, 10.1145/3613904.3642726} and text-based work~\cite{10.1145/3544548.3580782}. Notably, creative writing, as a text-based domain, increasingly emphasizes human-AI co-creation. For instance, Kim et al. investigated AI's role in creative language arts, such as poetry and novels~\cite{10.1145/3613904.3642529}, highlighting the evolving significance of AI tools in enhancing the creative process.}


%\subsection{\textcolor{black}{General-Purpose AI Tools for Creative Writing}}


%\textcolor{black}{Creative writing, encompassing novels, short stories, poetry, and essays, prioritizes linguistic artistry, stylistic diversity, and imaginative freedom~\cite{ramet2011creative, smith2020writing, mcvey2008all}. AI technologies have advanced this field, evolving from natural language processing to deep learning systems. Tools range from basic spell checkers~\cite{peterson1980computer},  crowdsourcing~\cite{kim2017mechanical, kim2014ensemble}, crowd role-play~\cite{huang2020heteroglossia}, to AI-driven character role-playing~\cite{10.1145/3613904.3642105}, reducing workloads and aiding writers in overcoming creative blocks~\cite{gero2019stylistic, hui2018introassist, roemmele2015creative, clark2018creative}. Current applications focus on story development~\cite{radford2019language, yang2019sketching} and character creation~\cite{cavazza2001characters, chung2022talebrush}.}

%\textcolor{black}{For story development, rule-based methods for story extension~\cite{lebowitz1984creating, meehan1977tale, riedl2010narrative} and platforms like Dramatica~\cite{dramatica} and AI Dungeon~\cite{aidungeon} facilitate text-based story creation but lack integration of visual elements. TaleStream integrates database-driven trope suggestions but lacks generative AI capabilities~\cite{chou2023talestream}. For character creation, AI has demonstrated its ability to simulate virtual personalities~\cite{cavazza2001characters}, infer relationships~\cite{chaturvedi2017unsupervised}, assist in dialogues~\cite{10.1145/3613904.3642105}, and map character trajectories~\cite{chung2022talebrush}. While these tools inspire character development~\cite{10.1145/3450741.3465253, shakeri2021saga, yuan2022wordcraft}, they remain limited to isolated stages and lack integration across workflows. Additionally, challenges in human-AI collaboration persist, with calls for personalized systems~\cite{10.1145/3613904.3642134}, improved tool design~\cite{10.1145/3613904.3642697}, and attention to creativity concerns~\cite{biermann2022tool}.}

%\textcolor{black}{Overall, prior works primarily focus on text-based solutions, neglecting the potential of visual-based approaches. This limitation renders them unsuitable for screenwriting, which is fundamentally related to audiovisual elements. Furthermore, existing studies often target specific stages of the creative process, failing to address the structured workflows and involvement of multiple stakeholders characteristic of screenwriting. These gaps underscore the need for targeted research to investigate AI applications and their integration throughout the entire screenwriting workflow.}


%\subsection{\textcolor{black}{Specialized AI Tools and Empirical Studies for Screenwriting}}

%\textcolor{black}{Previous AI tool applications in screenwriting primarily assist in three areas: information processing, emotion support, and visualization. For information processing, AI aids in background retrieval~\cite{pavel2015sceneskim}, narrative consistency~\cite{sanghrajka2017lisa}, and story continuation~\cite{valls2016error}. Emotion support focuses on analyzing and simulating emotions in narratives~\cite{goyal2010toward, su2007personality}, but existing systems rely on basic emotion tags and lack the nuance required for screenwriting. In visualization, AI tools range from data visualization~\cite{tapaswi2014storygraphs} to character and scene representation~\cite{kim2017visualizing}, but they lack real-time feedback and depend on high-quality screenplay input. Overall, while these tools~\cite{batty2015screenwriter, anguiano2023hollywood} support screenwriters, they often fail to address needs such as integrating emotional depth and narrative complexity. Furthermore, they do not incorporate advances in AI technologies that could offer new solutions.}

%\textcolor{black}{Empirical research shows that AI can generate screenplay elements comparable to those of human writers~\cite{ccelik4ai}, using tools like ChatGPT~\cite{chatgpt, luchen2023chatgpt} and DeepStory~\cite{deepstory}, enhancing efficiency, reducing costs~\cite{naji2024employing}, and offering inspiration~\cite{brako2023robots}. However, human creativity remains indispensable~\cite{naji2024employing, song2022analysis}. Recent studies have explored LLM-assisted screenwriting but focus exclusively on LLMs~\cite{10.1145/3656650.3656688, 10.1145/3544548.3581225}, neglecting advancements in AI-generated images and videos, leaving the broader potential of evolving AI technologies underexplored. To address these gaps, our study investigates screenwriters’ expectations for future AI tools to meet evolving needs.}

%\textcolor{black}{Furthermore, ethical concerns such as biases~\cite{chow2020ghost} and copyright issues~\cite{kavitha2023copyright} remain significant challenges. Understanding screenwriters' current practices and attitudes toward AI integration at specific workflow stages could inform strategies to mitigate these issues. In general, our findings aim to offer guidelines for the future design of human-AI co-creation tools tailored to the needs of screenwriters.}

%Our study builds upon previous research in AI integration in Creative Processes, General-Purpose AI Tools for Creative Writing, and Specialized AI Tools and Empirical Studies for Screenwriting.}
%AI-supported creative writing, AI-powered tools in screenwriting, the application of AI in other creative processes, and empirical studies in screenwriting.


%Creative writing, encompassing novels, short stories, poetry, and essays, prioritizes linguistic artistry, stylistic diversity, and authorial freedom~\cite{ramet2011creative, smith2020writing, mcvey2008all}. The evolution of AI in creative writing has progressed from simple spell checkers~\cite{peterson1980computer} to advanced ML and NLP applications for human-AI co-creation in plot development, character creation, and content review~\cite{radford2019language, yang2019sketching, calderwood2020novelists, clark2018creative, coenen2021wordcraft}. Previous research has synthesized findings from 115 articles on AI-assisted writing, offering recommendations for future tools~\cite{10.1145/3613904.3642697}.

%One of AI’s most impactful contributions of previous works has been its ability to reduce the creative workload for writers. AI tools offer valuable suggestions~\cite{gero2019stylistic, bernstein2010soylent}, refine content in professional contexts~\cite{hui2018introassist, swanson2008say, roemmele2015creative, clark2018creative}, and help writers overcome creative bottlenecks. Studies have examined the attitudes of writers toward these generative AI tools, revealing concerns about AI's role in shaping creative ideas~\cite{biermann2022tool}. Platforms like HaLLMark Effect have been proposed to facilitate collaboration between writers and AI models, further streamlining the creative process~\cite{10.1145/3613904.3641895}.

%AI tools for narrative structuring and storytelling including commercial tools like Dramatica \footnote{https://dramatica.com/}, and AI Dungeon \footnote{https://aidungeon.com/} assist in narrative structuring, while ASM's structured story model provides computational understanding for cultural storytelling differences~\cite{finlayson2012learning}. Methods for extending stories based on rules and objectives~\cite{lebowitz1984creating, meehan1977tale, riedl2010narrative} and tools like TaleStream~\cite{chou2023talestream} have also been explored. AI has been utilized in crowdsourced story creation and structuring creative leadership~\cite{huang2020heteroglossia, kim2017mechanical, kim2014ensemble}. AI has also been employed for specific creative tasks such as newspaper headline creation~\cite{gatti2016heady} and poetry generation~\cite{ghazvininejad2017hafez}.

%AI tools for narrative structuring and creative tasks include commercial platforms like Dramatica\footnote{https://dramatica.com/} and AI Dungeon\footnote{https://aidungeon.com/}, while ASM's structured story model provides a computational understanding of cultural differences in storytelling~\cite{finlayson2012learning}. Methods for extending stories based on rules and objectives have been explored~\cite{lebowitz1984creating, meehan1977tale, riedl2010narrative}, along with tools like TaleStream~\cite{chou2023talestream}. Additionally, AI has been utilized in crowdsourced story creation and creative leadership structuring~\cite{huang2020heteroglossia, kim2017mechanical, kim2014ensemble}, as well as for specific creative tasks such as newspaper headline creation~\cite{gatti2016heady} and poetry generation~\cite{ghazvininejad2017hafez}.


%In character development, research has focused on virtual character simulation~\cite{cavazza2001characters}, relationship inference~\cite{chaturvedi2017unsupervised}, and tools like TaleBrush for mapping character destinies~\cite{chung2022talebrush}. Studies indicate that LLM-driven systems inspire character creation through dialogue~\cite{10.1145/3450741.3465253, shakeri2021saga, yuan2022wordcraft, 10.1145/3411763.3450391, 10.1145/3613904.3642105, yeh2024ghostwriter}. Collaborative efforts have also examined tasks like newspaper headline creation~\cite{gatti2016heady} and poetry generation~\cite{ghazvininejad2017hafez}.

%AI in character development has been another key focus. Tools have been developed for simulating virtual characters~\cite{cavazza2001characters}, inferring character relationships~\cite{chaturvedi2017unsupervised}, and mapping character destinies through systems like TaleBrush~\cite{chung2022talebrush}. Research highlights the use of large language models (LLMs) in generating character dialogue and inspiring new characters~\cite{10.1145/3450741.3465253, shakeri2021saga, yuan2022wordcraft}. 

%These AI tools reduce creative workload~\cite{quinn2016cost}, offer suggestions~\cite{gero2019stylistic, bernstein2010soylent, singh2023hide, peng2020exploring}, and assist in professional settings~\cite{hui2018introassist, swanson2008say, roemmele2015creative, clark2018creative}. Writers' attitudes toward generative AI tools have been studied, with concerns about AI's role in transforming creative ideas into text~\cite{biermann2022tool}. Platforms like HaLLMark Effect have been proposed to enhance collaboration between writers and LLMs~\cite{10.1145/3613904.3641895}. Research shows varying interactions based on writers' experience levels, emphasizing the need for personalized, user-centered AI tools~\cite{10.1145/3613904.3642134}. However, existing studies often focus on general writers, and limited control may restrict AI's applicability in specialized domains~\cite{chung2022talebrush}.

%User experience in AI-assisted writing has also garnered attention. Paramveer S. Dhillon et al. indicate that writers' interactions with AI tools vary based on their level of experience, emphasizing the importance of designing personalized, user-centered AI systems~\cite{10.1145/3613904.3642134}. While AI’s potential to assist in creative writing is evident, its current applications often focus on general writing tasks. However, there is limited control for domain-specific needs, such as in screenwriting, where the demands for structuring narrative, dialogue, and character arcs are more complex~\cite{chung2022talebrush}. Our research specifically addresses this gap by providing insights into screenwriting as a distinct domain and offering tailored AI applications to better support screenwriters' workflows.

%Creative writing, encompassing novels, short stories, poetry, and essays, prioritizes linguistic artistry, stylistic diversity, and authorial freedom~\cite{ramet2011creative, smith2020writing, mcvey2008all}. The emergence of AI in creative writing has introduced potential transformations, evolving from early attempts at simple spell checkers~\cite{peterson1980computer} to advanced deep learning applications for human-AI co-creation in structure development, character creation, and content review~\cite{radford2019language, yang2019sketching, calderwood2020novelists, clark2018creative, coenen2021wordcraft}, increasingly influencing the creative process. Specifically, AI tools for narrative structuring in creative writing include commercial platforms like Dramatica\footnote{https://dramatica.com/} and AI Dungeon\footnote{https://aidungeon.com/}, etc. Methods for extending stories based on rules and objectives have been explored~\cite{lebowitz1984creating, meehan1977tale, riedl2010narrative}, along with tools like TaleStream~\cite{chou2023talestream}. Additionally, AI has been utilized in crowdsourced story creation and creative leadership structuring~\cite{huang2020heteroglossia, kim2017mechanical, kim2014ensemble}, as well as for specific creative tasks such as newspaper headline creation~\cite{gatti2016heady} and poetry generation~\cite{ghazvininejad2017hafez}. Moreover, AI assistance in character development has also been another key focus, with tools developed for simulating virtual characters~\cite{cavazza2001characters}, inferring character relationships~\cite{chaturvedi2017unsupervised}, and mapping character destinies through systems like TaleBrush~\cite{chung2022talebrush}. Research also highlights the use of large language models (LLMs) in generating character dialogue and inspiring new characters~\cite{10.1145/3450741.3465253, shakeri2021saga, yuan2022wordcraft}. 

%Furthermore, user experience in AI-supported creative writing has garnered attention. The main contributions of AI in previous works typically lie in its ability to reduce the creative workload for writers by offering valuable suggestions~\cite{gero2019stylistic, bernstein2010soylent}, refining content in professional contexts~\cite{hui2018introassist, swanson2008say, roemmele2015creative, clark2018creative}, and helping writers overcome creative bottlenecks. Platforms like HaLLMark Effect have been proposed to facilitate collaboration between writers and AI models, streamlining the creative process~\cite{10.1145/3613904.3641895}. However, challenges in human-AI collaboration are also arisen in this field. Biermann et al. have examined the attitudes of writers toward these generative AI tools, revealing concerns about AI's role in shaping creative ideas~\cite{biermann2022tool}. Dhillon et al. indicate that writers' interactions with AI tools vary based on their experience level, emphasizing the importance of designing personalized, user-centered AI systems~\cite{10.1145/3613904.3642134}. Lee et al. synthesized findings from 115 articles on AI-assisted writing, offering recommendations for future tools~\cite{10.1145/3613904.3642697}. Through previous research, we found that AI's evident potential to assist in creative writing. However, its current applications often focus on general writing tasks. There is limited support for domain-specific needs, such as in screenwriting, where the demands for structure, plot, dialogue, and character arcs are more complex~\cite{chung2022talebrush}. Therefore, our research specifically addresses this gap by providing insights into screenwriting as a distinct domain and offering suggestions for developing AI functions to better support screenwriters.

%\subsection{AI-supported Creative Writing}

%Creative writing, including novels, short stories, poetry, and essays, emphasizes linguistic artistry, stylistic diversity, and imaginative freedom~\cite{ramet2011creative, smith2020writing, mcvey2008all}. \textcolor{black}{The integration of AI has transformed this field, evolving from basic spell checkers~\cite{peterson1980computer} to advanced deep learning systems. This progression ranges from crowdsourcing~\cite{kim2017mechanical, kim2014ensemble} and crowd role-play~\cite{huang2020heteroglossia} to AI-driven character role-playing~\cite{10.1145/3613904.3642105}, enhancing creative processes. Previous studies demonstrate that AI reduces creative workloads by providing valuable suggestions~\cite{gero2019stylistic, bernstein2010soylent} and refining content for professional use~\cite{hui2018introassist, roemmele2015creative, clark2018creative}, thereby assisting writers in overcoming creative blocks. Currently, AI-supported creative writing processes are commonly divided into story development and character creation~\cite{radford2019language, yang2019sketching, calderwood2020novelists, clark2018creative, coenen2021wordcraft}.}

%Creative writing, including novels, short stories, poetry, and essays, emphasizes linguistic artistry, stylistic diversity, and imaginative freedom~\cite{ramet2011creative, smith2020writing, mcvey2008all}. \textcolor{black}{The development of AI technologies has supported this field, evolving from natural language processing to advanced deep learning systems. This progression ranges from basic spell checkers~\cite{peterson1980computer}, to crowdsourcing~\cite{kim2017mechanical, kim2014ensemble}, and crowd role-play~\cite{huang2020heteroglossia}, evolving into AI-driven character role-playing~\cite{10.1145/3613904.3642105}, all of which enhance creative processes. Previous studies have shown that AI reduces creative workloads by providing valuable suggestions~\cite{gero2019stylistic, bernstein2010soylent} and refining content for professional use~\cite{hui2018introassist, roemmele2015creative, clark2018creative}, thus helping writers overcome creative blocks. Specifically, AI-supported creative writing processes are currently divided into story development and character creation~\cite{radford2019language, yang2019sketching, calderwood2020novelists, clark2018creative, coenen2021wordcraft}.}

%\textcolor{black}{For story development, platforms like Dramatica~\cite{dramatica} and AI Dungeon~\cite{aidungeon} support story generation, while TaleStream combines AI recommendation algorithms with database-driven trope suggestions for narratives but has yet to integrate evolving generative AI technologies~\cite{chou2023talestream}. Rule-based methods for story extension have also been explored~\cite{lebowitz1984creating, meehan1977tale, riedl2010narrative}. We found that these approaches primarily focus on text-based support and do not incorporate visual elements. For character creation, previous works show that AI simulates virtual personalities~\cite{cavazza2001characters}, infers relationships~\cite{chaturvedi2017unsupervised}, facilitates AI-character dialogues~\cite{10.1145/3613904.3642105}, and maps character trajectories~\cite{chung2022talebrush}. These tools demonstrate various potentials of AI for inspiring character development~\cite{10.1145/3450741.3465253, shakeri2021saga, yuan2022wordcraft}, but they remain limited to isolated stages and lack integration across creative workflows. Additionally, challenges in human-AI collaboration persist. Lee et al. offered recommendations for future AI tools~\cite{10.1145/3613904.3642697}, and Dhillon et al. emphasized the need for personalized systems to accommodate writers’ varying experiences~\cite{10.1145/3613904.3642134}. Meanwhile, Biermann et al. highlighted writers’ concerns about AI’s influence on creativity~\cite{biermann2022tool}. }

%\textcolor{black}{Based on these, we observed that previous works primarily focus on text-based supports and challenges, neglecting the needs and potential for supporting audiovisual storytelling. This limitation renders these approaches inadequately suited to the field of screenwriting, which is inherently characterized by audiovisual storytelling. Furthermore, these works lack an understanding of AI integration across the entire creative process, highlighting the necessity of further exploring AI’s current applications and potential requirements within the screenwriting workflow.}


\begin{comment}

\textcolor{black}{Creative writing, including novels, short stories, poetry, and essays, emphasizes linguistic artistry, stylistic diversity, and imaginative freedom~\cite{ramet2011creative, smith2020writing, mcvey2008all}. The integration of AI has transformed this field, evolving from basic spell checkers~\cite{peterson1980computer} to advanced deep learning systems. This progression ranges from crowdsourcing~\cite{kim2017mechanical, kim2014ensemble} and crowd role-play~\cite{huang2020heteroglossia} to AI-driven character role-playing~\cite{10.1145/3613904.3642105}, enhancing creative processes. Previous studies demonstrate that AI reduces creative workloads by providing valuable suggestions~\cite{gero2019stylistic, bernstein2010soylent} and refining content for professional use~\cite{hui2018introassist, roemmele2015creative, clark2018creative}, thereby assisting writers in overcoming creative blocks. Currently, AI-supported creative writing processes are commonly divided into story development and character creation~\cite{radford2019language, yang2019sketching, calderwood2020novelists, clark2018creative, coenen2021wordcraft}.}

\textcolor{black}{For story development, platforms like Dramatica~\cite{dramatica} and AI Dungeon~\cite{aidungeon} support story generation, while TaleStream combines AI recommendation algorithms with database-driven trope suggestions for narratives but has yet to integrate evolving generative AI technologies~\cite{chou2023talestream}. Rule-based methods for story extension have also been explored~\cite{lebowitz1984creating, meehan1977tale, riedl2010narrative}. We found that these approaches primarily focus on text-based support and do not incorporate visual elements. For character creation, AI simulates virtual personalities~\cite{cavazza2001characters}, infers relationships~\cite{chaturvedi2017unsupervised}, facilitates AI-character dialogues~\cite{10.1145/3613904.3642105}, and maps character trajectories with tools like TaleBrush~\cite{chung2022talebrush}. These tools demonstrate various potentials of AI for inspiring character development~\cite{10.1145/3450741.3465253, shakeri2021saga, yuan2022wordcraft}, but they remain limited to isolated stages and lack integration across creative workflows. Additionally, challenges in human-AI collaboration persist. Biermann et al. highlighted writers’ concerns about AI’s influence on creativity~\cite{biermann2022tool}. Dhillon et al. emphasized the need for personalized systems to accommodate writers’ varying experiences~\cite{10.1145/3613904.3642134}. Lee et al. synthesized findings from 115 studies, offering recommendations for future AI tools~\cite{10.1145/3613904.3642697}. }

\textcolor{black}{Based on these findings, we observed that most research focuses on text-based creative writing, neglecting the unique challenges of audiovisual storytelling, such as screenwriting, and lacks an understanding of AI integration across the entire creative process. This leaves critical gaps about how AI can support audiovisual storytelling processes, including various stages such as idea generation, plot development, and dialogue creation, etc. To address these gaps, our study explores screenwriters’ needs across different workflow stages within the context of evolving technologies and proposes design opportunities for future AI tools tailored to screenwriting.}

\textcolor{black}{Creative writing, including novels, short stories, poetry, and essays, emphasizes linguistic artistry, stylistic diversity, and imaginative freedom~\cite{ramet2011creative, smith2020writing, mcvey2008all}. The integration of AI has transformed this field, evolving from basic spell checkers~\cite{peterson1980computer} to advanced deep learning systems. This progression spans crowdsourcing~\cite{kim2017mechanical, kim2014ensemble}, crowd role-play~\cite{huang2020heteroglossia}, and AI-driven character role-playing~\cite{10.1145/3613904.3642105}, enhancing creative processes. Studies show that AI reduces creative workloads by providing valuable suggestions~\cite{gero2019stylistic, bernstein2010soylent} and refining content for professional use~\cite{hui2018introassist, roemmele2015creative, clark2018creative}, thereby assisting writers in overcoming creative blocks. Currently, AI-supported creative writing processes are commonly divided into story development and character creation~\cite{radford2019language, yang2019sketching, calderwood2020novelists, clark2018creative, coenen2021wordcraft}.}

\textcolor{black}{For story structure, platforms like Dramatica~\cite{dramatica} and AI Dungeon~\cite{aidungeon} support narrative structuring, while TaleStream generates trope suggestions~\cite{chou2023talestream}. Rule-based methods for story extension have also been explored~\cite{lebowitz1984creating, meehan1977tale, riedl2010narrative}. For character creation, AI simulates virtual personalities~\cite{cavazza2001characters}, infers relationships~\cite{chaturvedi2017unsupervised}, facilitates AI-character dialogues~\cite{10.1145/3613904.3642105}, and maps character trajectories with tools like TaleBrush~\cite{chung2022talebrush}. These tools demonstrate AI’s potential for inspiring character creation~\cite{10.1145/3450741.3465253, shakeri2021saga, yuan2022wordcraft} but remain limited to isolated stages of development and lack integration across entire creative workflows.}

\textcolor{black}{Challenges in human-AI collaboration persist. Biermann et al. highlighted writers’ concerns about AI’s influence on creativity~\cite{biermann2022tool}, while Dhillon et al. emphasized the need for personalized systems to accommodate writers’ varying experiences~\cite{10.1145/3613904.3642134}. Lee et al. synthesized findings from 115 studies, offering recommendations for future AI tools~\cite{10.1145/3613904.3642697}. However, most research focuses on text-based creative writing, neglecting the unique challenges of audiovisual storytelling, such as screenwriting. This leaves critical questions about how AI can support processes like idea generation, plot development, dialogue creation, and visual integration. To address these gaps, this research explores screenwriters’ needs and proposes design guidelines for future AI tools tailored to screenwriting.}

Creative writing, encompassing novels, short stories, poetry, and essays, emphasizes linguistic artistry, stylistic diversity, and the freedom of imagination~\cite{ramet2011creative, smith2020writing, mcvey2008all}. The integration of AI into this field has brought transformative changes, evolving from early spell checkers~\cite{peterson1980computer} to advanced deep learning systems. This progression, from crowdsourcing~\cite{kim2017mechanical, kim2014ensemble} and crowd role-play~\cite{huang2020heteroglossia} to AI-driven character role-playing~\cite{10.1145/3613904.3642105}, further enhances creative writing. Previous studies demonstrate that AI reduces creative workloads by providing valuable suggestions~\cite{gero2019stylistic, bernstein2010soylent} and refining content for professional use~\cite{hui2018introassist, roemmele2015creative, clark2018creative}, thereby assisting writers in overcoming creative blocks. Currently, AI-supported creative writing processes are commonly divided into story development and character creation~\cite{radford2019language, yang2019sketching, calderwood2020novelists, clark2018creative, coenen2021wordcraft}.

For story structure development, platforms such as Dramatica~\cite{dramatica} and AI Dungeon~\cite{aidungeon} assist with narrative structuring. Previous research has explored rule-based methods for story extension~\cite{lebowitz1984creating, meehan1977tale, riedl2010narrative}, and TaleStream aids story ideation by generating trope suggestions~\cite{chou2023talestream}. However, these studies primarily address text-based storytelling, offering limited approaches to audiovisual narratives. For character creation, AI has been utilized to simulate virtual personalities~\cite{cavazza2001characters}, infer relationships~\cite{chaturvedi2017unsupervised}, facilitate writing through AI-character dialogues~\cite{10.1145/3613904.3642105}, and map character trajectories for story generation using tools like TaleBrush~\cite{chung2022talebrush}. While these advancements demonstrate AI’s role in inspiring character creation~\cite{10.1145/3450741.3465253, shakeri2021saga, yuan2022wordcraft}, they remain limited to assisting individual stages of character and story development and lack integration with the broader creative writing workflow.

Moreover, challenges in human-AI collaboration persist. Biermann et al. examined writers’ concerns about AI's influence on creative ideas~\cite{biermann2022tool}, while Dhillon et al. highlighted the varying interactions with AI tools based on writers’ experience, emphasizing the need for personalized, user-centered systems~\cite{10.1145/3613904.3642134}. Lee et al. synthesized findings from 115 studies on AI-assisted writing, offering recommendations for improving future tools~\cite{10.1145/3613904.3642697}. Despite these advancements, most research focuses on text-based creative writing, paying insufficient attention to the challenges of audiovisual storytelling, such as screenwriting. This gap leaves unanswered questions about how AI can support the creation processes of screenwriting, such as idea generation, plot development, dialogue creation, and the integration of visual elements. Therefore, this research aims to address these limitations in audiovisual storytelling by understanding screenwriters' needs and offering design guidelines for future AI tools tailored to screenwriting.

    

Creative writing, which includes novels, short stories, poetry, and essays, emphasizes linguistic artistry, stylistic diversity, and authorial freedom~\cite{ramet2011creative, smith2020writing, mcvey2008all}. The introduction of AI into creative writing has led to potential transformations, evolving from early spell checkers~\cite{peterson1980computer} to advanced deep learning tools for co-creation in structure development, character creation, and content review~\cite{radford2019language, yang2019sketching, calderwood2020novelists, clark2018creative, coenen2021wordcraft}. Commercial platforms like Dramatica~\cite{dramatica} and AI Dungeon~\cite{aidungeon} provide AI-assisted narrative structuring. Rule-based methods for story extension have also been explored~\cite{lebowitz1984creating, meehan1977tale, riedl2010narrative}, along with tools like TaleStream~\cite{chou2023talestream}. AI is further used in crowdsourced story creation and creative leadership~\cite{huang2020heteroglossia, kim2017mechanical, kim2014ensemble}, as well as specific tasks like headline generation~\cite{gatti2016heady} and poetry writing~\cite{ghazvininejad2017hafez}. AI tools also aid in character development through simulating virtual characters~\cite{cavazza2001characters}, inferring relationships~\cite{chaturvedi2017unsupervised}, and mapping character fates with systems like TaleBrush~\cite{chung2022talebrush}. Large language models (LLMs) have been utilized to generate dialogue and inspire character creation~\cite{10.1145/3450741.3465253, shakeri2021saga, yuan2022wordcraft}.

User experience in AI-assisted creative writing has also been explored, with AI reducing creative workloads by offering valuable suggestions~\cite{gero2019stylistic, bernstein2010soylent}, refining content in professional settings~\cite{hui2018introassist, swanson2008say, roemmele2015creative, clark2018creative}, and helping writers overcome creative blocks. Platforms like HaLLMark Effect have been introduced to streamline collaboration between writers and AI models~\cite{10.1145/3613904.3641895}. However, challenges in human-AI collaboration persist. Biermann et al. revealed writers' concerns about AI's influence on creative ideas~\cite{biermann2022tool}, while Dhillon et al. highlighted how interactions with AI tools vary based on writers' experience, underscoring the need for personalized, user-centered systems~\cite{10.1145/3613904.3642134}. Lee et al. synthesized findings from 115 articles on AI-assisted writing, providing recommendations for future tools~\cite{10.1145/3613904.3642697}. While AI shows promise in creative writing, its current applications are generally limited to basic tasks. There remains a lack of support for the more complex demands of screenwriting, such as structure, plot, dialogue, and character development~\cite{chung2022talebrush}. Our research addresses this gap by offering insights into screenwriting as a distinct domain and proposing AI functionalities tailored to screenwriters' needs.
\end{comment}

%\subsection{AI-powered Tools in Screenwriting}

%\textcolor{black}{The development of AI technologies has impacted screenwriting~\cite{batty2015screenwriter, anguiano2023hollywood}. Previous research highlights that AI can primarily assist screenwriting in three areas: information processing, emotion support, and visualization.}

%\textcolor{black}{For information processing, AI tools assist with retrieving background information~\cite{pavel2015sceneskim}, ensuring narrative consistency~\cite{sanghrajka2017lisa, kapadia2015computer}, and supporting story continuation~\cite{valls2016error, mateas2003experiment}. Traditional tools like Final Draft~\cite{finaldraft} integrate AI for scene organization and formatting, while AI systems like ChatGPT~\cite{chatgpt, luchen2023chatgpt} and DeepStory~\cite{deepstory} offer predictive analytics and co-writing capabilities~\cite{chow2020ghost}, streamlining the screenwriting process. For emotion support, earlier studies have explored emotion analysis in narratives~\cite{stapleton2003interactive}. Goyal et al.'s AESOP system analyzes emotions through plot units, helping screenwriters understand emotional trajectories~\cite{goyal2010toward}. However, it relies on basic emotion tags (e.g., positive, negative, neutral), failing to capture complex emotional dynamics. Su et al.'s work on simulating basic and mixed emotions~\cite{su2007personality} improved character expression but lacked deeper insight into emotions tied to plans and reasoning, limiting its integration into the screenwriting workflow. For visualization, previous research is divided into data visualization and visual representation. In data visualization, studies have summarized storylines~\cite{tapaswi2014storygraphs} and managed character arcs~\cite{kim2017visualizing}, providing a foundation for data-driven AI systems. In visual representation, AI tools use NLP to create 2D and 3D visualizations of screenplay content~\cite{10.1145/3172944.3172972, kim2017visualizing}, including character interactions~\cite{won2014generating} and scenes~\cite{hanser2009scenemaker}. However, these tools lack real-time feedback, and suitable visual representation, and rely on the quality of screenplay content. Beyond these main types of previous research, a recent work closely related to ours is the Dramatron system~\cite{10.1145/3544548.3581225}, which explores the integration of LLMs into screenwriting. However, it focuses only on LLMs, neglecting the impact of AI-generated images and videos.}

%\textcolor{black}{Overall, while these tools assist screenwriters, many have limitations that could be addressed by evolving AI technologies. Our study aims to understand how screenwriters expect AI to be applied and offer suggestions for future system design, thereby filling gaps in the field of screenwriting.}

%\textcolor{black}{For information processing, AI tools help retrieve story background information~\cite{pavel2015sceneskim}, ensure narrative consistency~\cite{sanghrajka2017lisa, kapadia2015computer}, and support story continuation~\cite{valls2016error, mateas2003experiment}. Traditional screenwriting tools, such as Final Draft~\cite{finaldraft}, integrate AI for scene organization and formatting, while AI systems like ChatGPT~\cite{chatgpt, luchen2023chatgpt} and DeepStory~\cite{deepstory} provide predictive analytics and co-writing capabilities~\cite{chow2020ghost}, reducing manual effort and improving screenplay structure.} \textcolor{black}{For emotion support, previous works explore emotion analysis and simulation in narratives~\cite{stapleton2003interactive}. Goyal et al.'s AESOP system uses AI to analyze emotions through plot units and generate emotional states related to characters~\cite{goyal2010toward}, aiding screenwriters in understanding emotional trajectories. However, it relies on basic emotion tags (e.g., positive, negative, neutral), missing more complex emotional dynamics. Su et al.'s work on simulating basic and mixed emotions~\cite{su2007personality} enhanced character expression but lacked a nuanced understanding of emotions tied to plans and reasoning, limiting its integration across the screenwriting workflow.} \textcolor{black}{For visualization, previous work can be divided into data visualization and visual representation. In data visualization, prior research has summarized storylines~\cite{tapaswi2014storygraphs} and managed character arcs~\cite{kim2017visualizing}, providing a foundation for data-driven AI systems. In visual representation, AI tools primarily use NLP to generate 2D and 3D visualizations of screenplay content~\cite{10.1145/3172944.3172972, kim2017visualizing}, including character interactions~\cite{won2014generating} and scenes~\cite{hanser2009scenemaker}. However, these tools lack real-time feedback and a suitable visual representation and depend on the quality of the screenplay content.}

%\textcolor{black}{Furthermore, recent work, the Dramatron system~\cite{10.1145/3544548.3581225}, highlights the potential of integrating LLMs into screenwriting but focuses only on LLMs, without addressing the impact of AI-generated images and videos.} 


\begin{comment}
As generative AI evolves, its growing impact on screenwriting, combining creative writing and film production, has become evident~\cite{batty2015screenwriter, anguiano2023hollywood}. While screenwriting shares imaginative elements with creative writing, it focuses on storytelling through visual and auditory media~\cite{batty2014screenwriters, kerrigan2016re}.

Traditional screenwriting tools like Final Draft~\cite{finaldraft}, now enhanced with AI, assist in organizing scenes and formatting screenplays. AI tools have also advanced in areas such as information retrieval, logical consistency checks, and emotional state mapping~\cite{sanghrajka2017lisa, goyal2010toward}, with applications like AESOP identifying and mapping character emotions~\cite{goyal2010toward}. Additionally, AI is widely used in collaborative screenwriting and story continuation~\cite{schank2013scripts, valls2016error, mateas2003experiment, lehnert1981plot, li2012crowdsourcing}. Visualization is also a key focus of previous AI exploration in screenwriting, with tools like Cardinal visualizing structure, characters, and scenes using NLP~\cite{10.1145/3172944.3172972}, and Story Explorer managing character arcs and event chronology~\cite{kim2017visualizing}. Other tools visualize multi-character interactions and generate scene possibilities~\cite{won2014generating}, automatically summarize character interactions~\cite{tapaswi2014storygraphs}, and create multi-character animations~\cite{kapadia2016canvas}. There are also several additional tools support tangible storyboard creation~\cite{bartindale2012storycrate, bartindale2016tryfilm} and fictional world-building~\cite{poulakos2015towards}.

Recent advancements, including ChatGPT~\cite{chatgpt, luchen2023chatgpt} and DeepStory~\cite{deepstory}, offer predictive analytics and co-writing capabilities~\cite{chow2020ghost}. The system Dramatron, developed by Mirowski et al.~\cite{10.1145/3544548.3581225}, demonstrated the potential of integrating LLMs into screenwriting workflows. However, these studies focused exclusively on LLM technology, overlooking developments in other areas such as AI-generated images and videos. Therefore, our study aims first to understand screenwriters' current workflows and then explore the specific ways in which AI is used within this workflow.
\end{comment}

%Dramatron, for instance, uses large language models to generate coherent screenplays through hierarchical text generation, from log lines to complete scripts~\cite{10.1145/3544548.3581225}. However, these tools often follow a linear process, while screenwriting is typically nonlinear, involving frequent exploration, iteration, and remixing~\cite{10.1145/3544548.3581225}. Therefore, our study aims first to understand screenwriters' current flexible and variable nonlinear workflows and then explore the specific ways in which AI is used within this workflow.



%Generative AI's growing impact on screenwriting, a blend of creative writing and film production, is becoming more evident~\cite{batty2015screenwriter, anguiano2023hollywood}. While screenwriting shares elements with creative writing, it focuses on storytelling through visual and auditory media~\cite{batty2014screenwriters, kerrigan2016re}.

%Traditional tools like Final Draft \footnote{http://www.http://finaldraft.com.} and Adobe Story \footnote{https://story.adobe.com.}, now enhanced with AI, organize scenes and format screenplays. AI also assists in information retrieval, logical consistency checks, and emotional state mapping~\cite{sanghrajka2017lisa, goyal2010toward}. For example, AESOP identifies and maps character emotional states~\cite{goyal2010toward}. AI has also been applied to collaborative screenwriting and continuation~\cite{schank2013scripts, valls2016error, mateas2003experiment, lehnert1981plot, li2012crowdsourcing}.


%Traditional screenwriting tools like Final Draft \footnote{http://www.http://finaldraft.com.} and Adobe Story \footnote{https://story.adobe.com.}, now enhanced with AI, are commonly used for organizing scenes and formatting screenplays. AI tools have also advanced in areas such as information retrieval, logical consistency checks, and emotional state mapping in scripts~\cite{sanghrajka2017lisa, goyal2010toward}. There have also been advances in emotional applications, such as AESOP, which can identify character emotional states and map them to characters in a story~\cite{goyal2010toward}. Furthermore, AI has been widely applied in collaborative screenwriting and continuation~\cite{schank2013scripts, valls2016error, mateas2003experiment, lehnert1981plot, li2012crowdsourcing}. Given that the ultimate goal of screenwriting is to integrate visual and auditory elements, many studies have explored AI's role in visualizing screenplays. For example, Cardinal uses NLP techniques to visualize screenplay structure, characters, story nodes, and simple scenes based on character dialogue and actions~\cite{10.1145/3172944.3172972}. Story Explorer visualizes character story arcs and background information, providing script management that allows users to quickly specify the chronological order of events in a film~\cite{kim2017visualizing}. Other studies have proposed a tool for visualizing multi-character interactions using simple graphics and generating multiple potential scene possibilities from an action database~\cite{won2014generating}. Tapaswi et al. developed a visualization tool that automatically summarizes character interactions in completed films and TV shows~\cite{tapaswi2014storygraphs}. CANVAS allows users to create multi-character animations by automatically completing storyboards~\cite{kapadia2016canvas}. Additionally, there are tools for creating storyboards tangibly~\cite{bartindale2012storycrate, bartindale2016tryfilm} and tools that allow the creation of fictional story worlds, filling them with characters and objects for automatic narrative synthesis~\cite{poulakos2015towards}.

%While these AI-assisted tools address specific screenplay tasks, advancements in technology have significantly enhanced the capabilities of NLP, computer vision, text-to-image generation, and large language models. AI tools, such as ChatGPT\footnote{https://openai.com/chatgpt/}\cite{luchen2023chatgpt}, and DeepStory\footnote{https://www.deepstory.ai/}, are increasingly being integrated into the screenwriting process, providing predictive analytics and even co-writing capabilities~\cite{chow2020ghost}. Recently, the Dramatron system demonstrated that large language models could enhance screenwriting capabilities by generating coherent screenplays and drama scripts through hierarchical text generation. The system sequentially generates complete scripts, including titles, character lists, plot points, location descriptions, and dialogue, starting from a log line~\cite{10.1145/3544548.3581225}. This research emphasizes a linear approach to script content generation but lacks flexibility in the workflow. While previous research has summarized common patterns and experiences of screenwriters~\cite{lawson1960theory, catron2017elements, smiley2005playwriting}, each screenwriter has a unique process in practice. Screenwriting is a nonlinear process with open-ended goals, not strictly following a predefined sequence. Screenwriters explore multiple solutions, frequently shifting from one to another, actively reviewing, remixing, and iterating them until a satisfactory result is achieved~\cite{10.1145/3544548.3581225}. Therefore, our study aims first to understand screenwriters' current flexible and variable nonlinear workflows and then explore the specific ways in which AI is used within these workflows.

%\subsection{The Use of AI in Other Creativity Processes}

%\textcolor{black}{Advancements in AI technology have also impacted various other creative domains~\cite{anantrasirichai2022artificial, 10.1145/3613904.3642726}, reshaping fields such as the arts~\cite{cetinic2022understanding} and design~\cite{lee2023impact, 10.1145/3613904.3642812}. In art, AI contributes to interactive media and visual representation~\cite{anadol2018archive, sun2023ai, tang2023ai}, while Kim et al. explored AI’s influence on creative language arts, such as poetry, novels, and screenplays~\cite{10.1145/3613904.3642529}. In design, AI integrates into product design workflows, fostering innovation~\cite{hong2023generative, chiou2023designing}, and advancing user interface design with innovations like adaptability~\cite{cheng2023play, zhang2023layoutdiffusion}. Meanwhile, with the impact of AI-generated image tools~\cite{mustafa2023impact, tang2024s}, AI’s democratizing potential in graphic design has been explored~\cite{tang2024exploring}, with a call for interdisciplinary collaboration to improve AI systems~\cite{meron2022graphic}. As AI-generated video capabilities develop, AI is also transforming fashion design with dynamic visuals~\cite{liu2019toward, karras2023dreampose} and aiding social media short-form video creators~\cite{10.1145/3613904.3642476}. Additionally, other research highlights AI’s role across various creative domains, assisting knowledge workers~\cite{10.1145/3613905.3650841}, data scientists~\cite{muller2019data, wang2021much}, and data storytelling~\cite{10.1145/3613904.3642726}.}

%\textcolor{black}{Based on these previous works, effective human-AI co-creation is increasingly emphasized across creative fields. Screenwriting, as part of creative industry, should also embrace this trend, driven by AI advancements. However, the complexity of narrative structures and emotional depth in audiovisual storytelling makes existing research from other creative fields insufficient for understanding AI integration in screenwriting. This underscores the need for targeted research to bridge these gaps and advance AI’s practical application in screenwriting.}

%Across these fields, there is an increasing emphasis on effective human-AI co-creation. As a component of the creative industry, screenwriting should also embrace this trend, driven by advancements in AI technologies. However, due to the inherent complexity of narrative structures and the emotional depth involved in audiovisual storytelling, existing findings from other creative fields are insufficient to fully understand the integration of AI in screenwriting. This highlights the need for targeted research to address these gaps and advance the practical application of AI in screenwriting.}

%AI's development in creative fields extends beyond writing~\cite{anantrasirichai2022artificial}, reshaping authorship and creativity in areas such as visual art~\cite{Bisson, Cai, roose2022ai}, interactive media arts~\cite{cetinic2022understanding, anadol2018archive, sun2023ai, tang2023ai}, design~\cite{lee2023impact}, and visualization~\cite{wang2019human}. In design, AI has become integral to workflows~\cite{10.1145/3613904.3642812}, particularly in product design, where it supports visual generation and fosters innovation, despite challenges related to diversity~\cite{hong2023generative, chiou2023designing}. Diffusion models have further enhanced innovation in user interface design~\cite{cheng2023play, zhang2023layoutdiffusion}. AI is also transforming fashion design through technologies like Attribute-GAN and DreamPose, enabling dynamic visuals~\cite{liu2019toward, karras2023dreampose}. In graphic design, Tang et al. have explored AI’s democratizing potential~\cite{tang2024exploring, tang2024s}, while Mustafa analyzed the impact of AI-assisted tools~\cite{mustafa2023impact}. Meron stressed the importance of interdisciplinary collaboration between computer science and graphic design to enhance AI systems~\cite{meron2022graphic}. Additionally, previous research has examined AI's role in data scientists' workflows~\cite{muller2019data, wang2021much} and how large language models assist knowledge workers~\cite{10.1145/3613905.3650841}. Across these fields, there is a shared call for fostering effective human-AI co-creation. AI is viewed as a tool to enhance, not replace, human creativity. However, challenges remain, including concerns about creative control, content homogenization, and the potential loss of diversity in artistic expression. These concerns are commonly reflected across the creative industries, where practitioners seek to balance the innovative potential of AI with maintaining the uniqueness and authenticity of human contributions. 

%As a part of the creative field, screenwriters also face these challenges. However, due to the complexity of narrative structures, character development, emotional depth, and the integration of audiovisual language in the screenwriting process, findings from other creative fields are insufficient to address the issues surrounding AI integration in this context. This highlights the necessity of conducting targeted research to bridge the existing knowledge gap and advance the practical application of AI in screenwriting.

%Given our focus on the screenwriting, it faces the same challenges as general creativity studies. However, due to the involvement of complex narrative structures, character development, emotional depth, and the integration of audiovisual language in the screenwriting process, findings from other creative fields are insufficient to address the issues surrounding AI integration in screenwriting. This highlights the necessity of conducting targeted research in this area to bridge the existing knowledge gap and advance the practical application of AI in screenwriting.

%Given our focus on the screenwriting context, which presents unique challenges compared to general creativity studies, findings from other creative fields are insufficient to address the issues surrounding AI integration in screenwriting. Screenwriting, in particular, involves complex narrative structures, character development, and emotional depth, which involve different approches of AI integration. This highlights the need for targeted research in this area to bridge the existing knowledge gap and advance the practical application of AI in screenwriting. 


%However, given our focus on the screenwriting context, which presents unique challenges compared to general creativity studies, findings from other creative fields are insufficient to address the issues surrounding AI integration in screenwriting. This highlights the need for targeted research in this area to bridge the existing knowledge gap and advance the practical application of AI in screenwriting.


%However, since our focus is on the screenwriting context, which differs from general creativity studies, the results from other creative fields cannot address the specific issues related to the integration of AI in screenwriting that we are concerned with.

%AI's development in creative fields extends beyond writing~\cite{anantrasirichai2022artificial}, reshaping authorship and creativity in areas such as visual art~\cite{Bisson, Cai, roose2022ai}, interactive media arts~\cite{cetinic2022understanding, anadol2018archive, sun2023ai, tang2023ai}, design~\cite{lee2023impact}, and visualization~\cite{wang2019human}. In design, AI has become integral to workflows, particularly in product design, where it supports visual generation and fosters innovation, despite challenges related to diversity~\cite{hong2023generative, chiou2023designing}. Diffusion models have further enhanced innovation in user interface design~\cite{cheng2023play, zhang2023layoutdiffusion}. AI is also transforming fashion design through technologies like Attribute-GAN and DreamPose, enabling dynamic visuals~\cite{liu2019toward, karras2023dreampose}. In graphic design, Tang et al. have explored AI’s democratizing potential~\cite{tang2024exploring, tang2024s}, while Mustafa analyzed the impact of AI-assisted tools~\cite{mustafa2023impact}. Meron stressed the importance of interdisciplinary collaboration between computer science and graphic design to enhance AI systems~\cite{meron2022graphic}. Additionally, previous research has examined AI's role in data scientists' workflows~\cite{muller2019data, wang2021much} and how large language models assist knowledge workers~\cite{10.1145/3613905.3650841}. Based on these explorations of creative processes involving AI, our results both diverge from and align with Zhou et al.'s research on creative design~\cite{10.1145/3613904.3642812}. We have identified distinct future roles for AI, with a focus on understanding how screenwriters utilize AI and their expectations. Our findings contribute to the broader application of AI in creative domains, offering valuable new perspectives.

%AI's development in creative fields goes beyond writing~\cite{anantrasirichai2022artificial}, redefining authorship and creativity in areas like visual art~\cite{Bisson, Cai, roose2022ai}, photography~\cite{PostPhotographicPerspectivesFellowship}, and interactive media arts~\cite{cetinic2022understanding, anadol2018archive, sun2023ai, tang2023ai}. In design, AI has been integral to workflows, particularly in product design, where it aids in visual generation and innovation, despite limitations in diversity~\cite{lee2023impact, hong2023generative, chiou2023designing}. Diffusion models in user interface design have enhanced innovation~\cite{cheng2023play, zhang2023layoutdiffusion}. AI is also transforming fashion design through technologies like Attribute-GAN and DreamPose for dynamic visuals~\cite{liu2019toward, karras2023dreampose}. In graphic design, AI's democratizing potential has been studied through the work of Tang et al.~\cite{tang2024exploring, tang2024s} and Mustafa's analysis of AI-assisted tools~\cite{mustafa2023impact}. Additionally, Meron stressed the need for interdisciplinary collaboration between computer science and graphic design to improve AI systems~\cite{meron2022graphic}. Previous research has also focused on AI's role in data scientists' workflows~\cite{muller2019data, wang2019human, wang2021much} and how large language models assist knowledge workers~\cite{10.1145/3613905.3650841}. Based on these previous explorations in creativity processes with AI, understanding how screenwriters use AI is crucial for grasping AI's broader role in creativity and contributing to research on AI in creative industries.

%AI's development in creative fields goes beyond writing~\cite{anantrasirichai2022artificial}, redefining authorship and creativity in areas like visual art~\cite{Bisson, Cai, roose2022ai}, photography~\cite{PostPhotographicPerspectivesFellowship}, and interactive media arts~\cite{cetinic2022understanding, anadol2018archive, sun2023ai, tang2023ai}. In design, AI has been integral to workflows, particularly in product design, where it aids in visual generation and innovation, despite limitations in diversity~\cite{lee2023impact, hong2023generative, chiou2023designing}. Diffusion models in user interface design have enhanced innovation~\cite{cheng2023play, zhang2023layoutdiffusion}. AI is also transforming fashion design through technologies like Attribute-GAN and DreamPose for dynamic clothing visuals~\cite{liu2019toward, karras2023dreampose}. In graphic design, AI's democratizing potential has been studied through the work of Tang et al.~\cite{tang2024exploring, tang2024s} and Mustafa's analysis of AI-assisted tools~\cite{mustafa2023impact}. Meron stressed the need for interdisciplinary collaboration between computer science and graphic design to improve AI systems~\cite{meron2022graphic}. Previous research has also focused on AI's role in data scientists' workflows~\cite{muller2019data, wang2019human, wang2021much} and how large language models assist knowledge workers~\cite{10.1145/3613905.3650841}. Therefore, exploring how screenwriters use AI is crucial for understanding AI's broader role in creativity and contributing to AI research in creative industries.

%The development of AI in creative fields extends beyond creative writing~\cite{anantrasirichai2022artificial}. AI is redefining authorship and creativity in art, with many artists incorporating AI into their creative processes~\cite{manovich2019defining, mezei2020leonardo, goenaga2020critique}, including visual art~\cite{Bisson, Cai, roose2022ai}, photography~\cite{PostPhotographicPerspectivesFellowship}, and interactive media arts~\cite{cetinic2022understanding, anadol2018archive, sun2023ai, tang2023ai}. In the field of design, AI has long been integrated into workflows. In product design, AI facilitates visual generation, sparking innovative thinking despite some limitations in diversity~\cite{lee2023impact, hong2023generative, chiou2023designing}. Gmeiner et al. have explored the challenges and opportunities in helping engineers and architectural designers adopt AI-based tools for co-creation~\cite{gmeiner2023exploring}. In user interface design, the adoption of diffusion models has made design elements more intuitive, enhancing innovation~\cite{cheng2023play, zhang2023layoutdiffusion}. In fashion design, technologies such as Attribute-GAN and DreamPose have played a crucial role in automatically generating clothing-matching pairs and transforming static images into dynamic clothing showcase videos~\cite{liu2019toward, karras2023dreampose}. In graphic design, Tang et al.'s research examined the use of AI by designers in the field~\cite{tang2024s} and highlighted the democratizing potential of AI tools through a comparative study of professional and non-professional users in the art and design fields~\cite{tang2024exploring}. Mustafa studied the impact of AI-assisted and AI-based design tools on various specific tasks in graphic design~\cite{mustafa2023impact}. Meron emphasized that addressing issues in AI-assisted design systems requires interdisciplinary collaboration between computer science experts and graphic design specialists~\cite{meron2022graphic}. Additionally, in terms of integrating AI into different types of work and workflows, previous researchers have focused on data scientists' workflows, their use of AI, and their expectations~\cite{muller2019data, wang2019human, wang2021much}, as well as empirical studies on how knowledge workers in enterprises use large language models to assist their work~\cite{10.1145/3613905.3650841}. Therefore, in an era where AI technology is ubiquitous, gaining an in-depth understanding of how screenwriters use AI in actual creation, as an underexplored but highly significant research area within the creative domain, will not only deepen our understanding of AI's role in creativity but also provide valuable contributions to research on AI in the creativity field.

%\subsection{Empirical Study in Screenwriting}

%\textcolor{black}{Previous studies have explored the connection between screenwriting and creativity~\cite{bourgeois2014creativity, conor2010everybody, conor2010screenwriting}. Notable projects such as the short film script ``\textit{Sunspring}'' (2016)~\cite{sunspring} and the interactive playwriting project at the Young Vic (2021)~\cite{youngvic} have highlighted new trends in AI-powered screenwriting~\cite{millard2014screenwriting, ogle2019screenwriting, thorne2020hey}. Çelik's research suggests that AI can generate screenplay elements comparable to, and sometimes superior to, those created by human writers~\cite{ccelik4ai, naji2024employing}, though human artistic creativity remains irreplaceable~\cite{song2022analysis}. Brako et al. explored AI as a co-creation tool in screenwriting education~\cite{brako2023robots}, noting that it accelerates creativity by offering new inspiration. Other studies have examined AI's role in filmmaking, including screenwriting with LLMs like ChatGPT, showing that AI can enhance efficiency and reduce costs~\cite{naji2024employing, 10.1145/3656650.3656688}. However, these studies lack an understanding of how AI specifically integrates into the various stages of the screenwriting workflow. They are also limited to LLM assistance and do not consider the potential impact of AI-generated images and videos, which may provide unprecedented support for audiovisual storytelling in screenwriting. Thus, the broader implications of emerging AI technologies in screenwriting remain underexplored.}

%\textcolor{black}{Moreover, Chow's research has shown that biases in AI datasets related to race, gender, or ideology can affect creativity in the film industry~\cite{chow2020ghost}, and there are also challenges related to screenplay copyright~\cite{kavitha2023copyright}. Understanding how screenwriters' attitudes and expectations toward AI integration in specific stages of the screenwriting workflow may provide potential approaches or insights for mitigating or altering the impact of these issues.}

%\textcolor{black}{Overall, our study aims to fill these gaps by analyzing screenwriters' current workflows, challenges, and task allocation with AI, as well as their attitudes toward AI integration and expectations for future AI roles. Our findings aim to inform the design of human-AI co-creation tools tailored to the needs of screenwriters.}


%Previous studies have established a connection between screenwriting and creativity~\cite{bourgeois2014creativity, conor2010everybody, conor2010screenwriting}. In the context of evolving digital and AI technologies across various media, short film scripts such as ``\textit{Sunspring}'' (2016)~\cite{sunspring} and the interactive playwriting project at the Young Vic in London (2021)~\cite{youngvic} have garnered significant attention, showcasing new trends and modes in AI-augmented screenwriting in the digital era~\cite{millard2014screenwriting, ogle2019screenwriting, thorne2020hey}. Çelik's research indicates that AI can generate screenplay elements that are comparable to, and sometimes even superior to, those produced by human writers~\cite{ccelik4ai, naji2024employing}. Additionally, Brako et al. has explored the use of AI as a co-creation tool in screenwriting education~\cite{brako2023robots}, highlighting that AI can accelerate the creative process and provide new sources of inspiration. However, the evaluation of works needs to be adapted to fit the AI co-creation environment. Other studies have explored the entire filmmaking process using various AI tools, including screenwriting with ChatGPT, showing that AI technology can significantly improve filmmaking efficiency and reduce costs. These studies suggest that the combination of AI and human creativity is key to the future sustainability of the film industry, where lower-level tasks may be automated by AI, but human artistic creativity and emotional input remain irreplaceable~\cite{song2022analysis}. However, research by Chow has found that potential biases related to race, gender, or ideology in AI datasets could affect the diversity and novelty of creativity in the film industry~\cite{chow2020ghost}, and these biases also pose challenges to screenplay copyright~\cite{kavitha2023copyright}.

%Overall, previous research has rarely addressed the specific needs for future AI functionalities as informed by feedback from screenwriters after actual use, lacking concrete insights into AI's practical applications and potential expectations in this domain. Therefore, our research aims to analyze screenwriters' practices, including task allocation across different stages of the workflow with AI, combining their current attitudes toward AI integration in screenwriting with their idealized expectations for future AI functionalities. This leads us to propose future design opportunities for human-AI co-creation tools tailored to the needs of screenwriters.

%Previous studies have established a strong connection between screenwriting and creativity~\cite{bourgeois2014creativity, conor2010everybody, conor2010screenwriting}. In the context of evolving digital and AI technologies and various media, short film scripts like ``\textit{Sunspring}'' (2016) \footnote{https://www.garethjmsaunders.co.uk/2019/10/15/sunspring-a-sci-fi-film-script-written-by-ai/} and the interactive playwriting project at the Young Vic in London (2021) \footnote{https://www.theguardian.com/stage/2021/aug/24/rise-of-the-robo-drama-young-vic-creates-new-play-using-artificial-intelligence} have garnered significant attention, demonstrating new trends and modes in the digital era's AI-augmented screenwriting~\cite{millard2014screenwriting, ogle2019screenwriting, thorne2020hey}. Studies have shown that AI can generate screenplay elements that are comparable to, and sometimes even superior to, those produced by human writers~\cite{ccelik4ai}. Additionally, research has explored the use of AI as a co-creation tool in screenwriting education~\cite{brako2023robots}, noting that AI can accelerate the creative process and provide new sources of inspiration, but that the evaluation of works needs to be adjusted to fit the AI co-creation environment. Other studies have used various AI tools to explore the entire filmmaking process, including screenwriting with ChatGPT, demonstrating that AI technology can significantly improve filmmaking efficiency and reduce costs. Research also suggests that the combination of AI and human creativity is key to the future sustainability of the film industry, where lower-level tasks may be automated by AI, but human artistic creativity and emotional input remain irreplaceable~\cite{naji2024employing, song2022analysis}. However, studies have found that due to potential biases related to race, gender, or ideology in data sets, the application of AI in the film industry may impact the diversity and novelty of creativity~\cite{chow2020ghost}, and it also poses challenges to the copyright of screenplays~\cite{kavitha2023copyright}.

%Overall, previous research has rarely addressed the specific needs for future AI functionalities as feedback from screenwriters after actual use. Therefore, our research aims to analyze screenwriters' task allocation across different stages of the workflow with AI, combining their current attitudes toward AI integration into the screenwriting process with their idealized expectations for future AI functionalities, to propose design guidelines for AI co-creation tools suited to screenwriters. Our goal is to enable future AI tools to effectively address the relevant challenges in screenwriting workflows as expected by screenwriters, thereby further promoting the development of human-AI co-creation.
\begin{table*}
%{H}
\centering
\footnotesize
%\tiny
%\scriptsize
%\scriptsize % 将字体大小缩小到更小号
\caption{Demographic Information of Participants. From left to right, each column presents the participant number, age, gender, background, years (Y) of screenwriting experience, training methods received, prior use of AI in screenwriting (yes/no), and their self-reported proficiency in using AI for screenwriting (5 = very proficient, 1 = not proficient at all). The specific training methods are represented by the following abbreviations in the table: institution courses (IC), classic scripts (CS), online videos (OV), instructor books (IB), and other (O).}
\Description{Description for Table 1:
The table displays demographic information of participants. It contains information organized by participant number (No.), age, gender, background (e.g., student, professional, enthusiast), years of screenwriting experience (Y), training methods received, prior use of AI in screenwriting (yes or no), and self-reported proficiency in using AI for screenwriting rated on a scale of 1 to 5 (5 being very proficient and 1 being not proficient at all).

Training methods are abbreviated as follows:
- IC: institution courses
- CS: classic scripts
- OV: online videos
- IB: instructor books
- O: other

The table lists 23 participants (P1–P18 and N1–N5), each row detailing:
1. Age range from 22 to 32.
2. Gender distribution includes male and female participants.
3. Backgrounds vary between students, professionals, and enthusiasts.
4. Years of screenwriting experience range from 0.5 years to 8 years.
5. Participants report training through a combination of methods like IC, CS, OV, IB, or O.
6. Prior use of AI in screenwriting shows some participants have used AI (marked "Yes") and others have not (marked "No").
7. Self-reported AI proficiency scores range from 1 (not proficient) to 5 (very proficient).} 

\label{tab:participant_data}
\begin{tabular}{|p{0.8cm}|p{0.8cm}|p{1.1cm}|p{1.7cm}|p{2.2cm}|p{2cm}|p{1.1cm}|p{2.5cm}|}


\hline
\textbf{No.} & \textbf{Age} & \textbf{Gender} & \textbf{Background}   & \textbf{Experience(Y)} & \textbf{Training}       & \textbf{Use AI} & \textbf{Proficiency of AI} \\ \hline
\textcolor{black}{P1}  & 24  & Female & Student      & 1             & IB, CS         & Yes    & 4                 \\ \hline
\textcolor{black}{P2}  & 26  & Male   & Professional & 7             & IC, CS         & Yes    & 4                 \\ \hline
\textcolor{black}{P3}  & 23  & Female & Enthusiast   & 0.5           & IC, CS, O      & Yes    & 3                 \\ \hline
\textcolor{black}{P4}  & 25  & Male   & Professional & 7             & IC, CS, OV     & Yes    & 4                 \\ \hline
\textcolor{black}{P5}  & 23  & Female & Student      & 5             & IC, IB, CS     & Yes    & 3                 \\ \hline
\textcolor{black}{P6}  & 24  & Female & Student      & 2             & IC, IB, CS     & Yes    & 3                 \\ \hline
\textcolor{black}{P7}  & 25  & Male   & Professional & 8             & IC, IB, CS     & Yes    & 4                 \\ \hline
\textcolor{black}{P8}  & 26  & Male   & Student      & 2             & IC, CS, OV, IB & Yes    & 5                 \\ \hline
\textcolor{black}{P9}  & 27  & Female & Enthusiast   & 1             & IC, CS, IB     & Yes    & 4                 \\ \hline
\textcolor{black}{P10} & 23  & Female & Student      & 5             & IC, CS, IB     & Yes    & 2                 \\ \hline
\textcolor{black}{P11} & 27  & Female & Professional & 6             & IC, IB, CS     & Yes    & 2                 \\ \hline
\textcolor{black}{P12} & 24  & Female & Enthusiast   & 0.5           & IC, CS, IB     & Yes    & 4                 \\ \hline
\textcolor{black}{P13} & 25  & Male   & Enthusiast   & 6             & IC, CS, IB     & Yes    & 4                 \\ \hline
\textcolor{black}{P14} & 27  & Male   & Professional & 2             & IC, CS, OV, IB & Yes    & 4                 \\ \hline
\textcolor{black}{P15} & 32  & Male   & Professional & 8             & IC, CS, OV, IB & Yes    & 5                 \\ \hline
\textcolor{black}{P16} & 23  & Female & Enthusiast   & 4             & IC, CS, IB     & Yes    & 4                 \\ \hline
\textcolor{black}{P17} & 23  & Female & Professional & 4             & IC, CS, OV     & Yes    & 4                 \\ \hline
\textcolor{black}{P18} & 22  & Female & Student      & 3             & IC, CS, IB     & Yes    & 3                 \\ \hline
\textcolor{black}{N1}  & 23  & Male   & Student      & 3             & IC, CS         & No     & 1                 \\ \hline
\textcolor{black}{N2}  & 22  & Female & Enthusiast   & 1             & CS             & No     & 1                 \\ \hline
\textcolor{black}{N3}  & 25  & Male   & Enthusiast   & 2             & IC, IB, CS     & No     & 1                 \\ \hline
\textcolor{black}{N4}  & 29  & Female & Professional & 6             & IC, CS, OV, IB & No     & 1                 \\ \hline
\textcolor{black}{N5}  & 23  & Female & Enthusiast   & 4             & IC, CS, IB     & No     & 1                 \\ \hline
\end{tabular}
    \label{tab:participants}
\end{table*}

\section{Methodology}

%Our study utilized a qualitative approach, conducting semi-structured interviews with 23 participants from screenwriting backgrounds. The objective of these interviews was to gain in-depth understanding into the integration of AI tools in screenwriting, with a particular focus on participants' existing usage, attitudes toward AI, and future expectations.

\textcolor{black}{Our study utilized a qualitative approach, conducting semi-structured interviews with 23 participants with screenwriting backgrounds. The objective of these interviews was to gain understanding into the integration of AI tools in screenwriting, with a particular focus on participants' usage, attitudes toward AI, and future expectations.}


%Our research employed a qualitative approach to address the research questions through semi-structured interviews. We invited 23 participants with backgrounds in screenwriting. The goal of these interviews was to gain deep insights into the application of AI tools in screenwriting scenarios, specifically focusing on users' usage, future expectations, and attitudes within this context.
 
\subsection{Participants}

Participants with screenwriting backgrounds, totaling 23, were recruited for this study through snowball sampling~\cite{goodman1961snowball}. All interviews were independently conducted by the first author. All participant information is based on self-reports: the gender distribution was 9 males and 14 females, with participants aged between 22 and 32 years (average age: 25). Their screenwriting backgrounds primarily consisted of students, enthusiasts, and professionals, with all participants having received professional training in screenwriting methods. On average, they had approximately four years of screenwriting-related experience, which included either studies, work, or a combination of both (refer to Table ~\ref{tab:participant_data}). Their average self-reported proficiency level with AI in screenwriting was three out of five, suggesting a moderate level of familiarity. The majority of participants (78\%) had experience using AI in screenwriting and are referred to as P (e.g., P1). The remaining 22\% had not used AI tools in screenwriting and are referred to as N (e.g., N1). We did not use prior experience with AI as an inclusion criterion during recruitment for the following reasons. Participants without AI experience offered valuable insights into most research questions, including workflows, challenges, attitudes, and expectations toward AI, thereby enriching the findings. We retained responses from all 23 participants to achieve more comprehensive results.
\textcolor{black}{Each interview was conducted individually via an online meeting platform to ensure participant engagement and thorough data collection.}

%Twenty-three participants with screenwriting backgrounds were recruited for this study through snowball sampling~\cite{goodman1961snowball}. The gender distribution was 9 males and 14 females, with participants aged between 22 and 32 years (average age: 25). Their backgrounds included screenwriting students, enthusiasts, and professionals, all of whom had received professional training in screenwriting methods. Training methods were categorized as institution courses (IC), classic scripts (CS), online videos (OV), instructor books (IB), and other (O). Participants had an average of four years of screenwriting experience (see Table ~\ref{tab:participant_data}). Most participants (78\%) had used AI in screenwriting, with an average familiarity level of 3 on a 5-point scale (5 = very proficient, 1 = not proficient). The interviews were conducted individually via an online meeting platform to ensure personalized engagement and effective data collection.

%Twenty-three participants with screenwriting backgrounds were recruited for this study through snowball sampling~\cite{goodman1961snowball}. All interviews were independently conducted by the first author. All participant information is based on self-reports: the gender distribution was 9 males and 14 females, with participants aged between 22 and 32 years (average age: 25). Their screenwriting backgrounds primarily consisted of students, enthusiasts, and professionals, with all participants having received professional training in screenwriting methods. On average, they had approximately four years of screenwriting-related experience, which included either studies, work, or a combination of both (refer to Table ~\ref{tab:participant_data}). \textcolor{black}{Their average self-reported proficiency level with AI in screenwriting was 3 out of 5, suggesting a moderate level of familiarity. The majority of participants (78\%) had experience using AI in screenwriting and are referred to as P (e.g., P1). The remaining 22\% had not used AI tools in screenwriting and are referred to as N (e.g., N1). We did not use prior experience with AI as an inclusion criterion during recruitment for the following reasons. Participants without AI experience offered valuable insights into most research questions, including traditional workflows, challenges, attitudes, and expectations toward AI, thereby enriching the findings. We retained responses from all 23 participants to achieve more comprehensive results.} Each interview was conducted individually via an online meeting platform to ensure full participant engagement and thorough data collection.

%Participants’ prior experience with AI was not used as an inclusion criterion during recruitment, meaning that regardless of whether they had previously used AI in screenwriting, they were eligible to participate in this study. 
%This decision was made because one focus of the study was to understand current screenwriting practices, with the proportions of P and N participants also representing part of the study’s results. Additionally, 


%\textcolor{black}{The majority of participants (78\%) had used AI in the screenwriting process; these participants are referred to as P (e.g., P1, P2...). Their average level of proficiency with AI in screenwriting was 3. The remaining 22\% of participants had not previously used AI tools in screenwriting and are referred to as N (e.g., N1, N2...).} 
%Each interview was conducted individually via an online meeting platform to ensure full engagement and comprehensive data collection.
%The majority of our participants (78\%) had used AI in the screenwriting process, with the average level of proficiency with AI in screenwriting being 3. Each interview was conducted individually via an online meeting platform, enabling full engagement and ensuring thorough data collection.




\subsection{Apparatus and Materials}

%The equipment used in this study included a laptop and an integrated recording device within the online meeting system. To gain a deeper understanding of the screenwriting process, we designed 27 open-ended questions for the entire interview process, divided into the following three sections. The specific interview questions are available in the supplementary materials.

The study utilized a laptop and an online meeting system's integrated recording device. To explore the screenwriting process, we designed 27 open-ended questions, divided into three sections. The full list of questions is provided in the supplementary materials.

\begin{itemize}
\item Basic information and daily screenwriting workflow (\textbf{RQ1}) (e.g., screenplay types and themes, tools, workflow stages, and challenges)

\item Experience with AI tools in the screenwriting process (\textbf{RQ1}, \textbf{RQ2}) (e.g., familiarity and experience with AI, AI integration into workflow)

\item Discussion of potential AI assistance (\textbf{RQ3}) (e.g., ideal AI features, ideal interaction methods, visualization, role-playing)
\end{itemize}

The interviews began by focusing on participants' typical workflows, then explored their specific use and experiences with AI tools, and finally, inquired about their attitudes toward potential AI tools. The potential AI features mentioned in the interviews are based on previous related studies, and we aimed to understand participants' current views and future expectations for these features.

\subsection{Procedure}

%\textcolor{black}{The study followed widely recognized ethical frameworks~\cite{wma_helsinki_declaration}, prioritizing the safety, rights, and dignity of all participants. Participants were fully informed about the study’s background, objectives, and purpose prior to providing informed consent and participating in data collection. Their rights were clearly explained, including the voluntary nature of participation, the option to withdraw at any time without penalty, and the assurance of data confidentiality. Privacy protection measures were explicitly outlined, such as anonymizing all responses and securely storing the collected data. To enable informed decision-making, participants were given one week to review the detailed consent form and carefully consider their participation, ensuring that no undue pressure or coercion was applied.}

The study followed widely recognized ethical frameworks~\cite{wma_helsinki_declaration}, prioritizing the safety, rights, and dignity of all participants. Participants were fully informed about the study’s background, objectives, and purpose prior to providing informed consent and participating in data collection. \textcolor{black}{Their rights were clearly explained, including the voluntary nature of participation, the option to withdraw at any time without penalty, and the assurance of confidentiality.} Privacy protection measures were explicitly outlined, such as anonymizing all responses and securely storing the collected data. To enable informed decision-making, participants were given one week to review the detailed consent form and carefully consider their participation, ensuring that no undue pressure or coercion was applied.

%\textcolor{black}{After obtaining informed consent, anonymized background information was collected from participants} (see Table ~\ref{tab:participant_data}). Subsequently, in-depth, one-on-one interviews were conducted with each participant using a semi-structured format guided by a pre-prepared outline of 27 open-ended questions. Each interview lasted approximately 80 to 100 minutes, allowing participants to express their perspectives in detail. To ensure accurate and comprehensive data collection, interviews were recorded using the recording feature of the online meeting platform. \textcolor{black}{All recordings were securely stored in a restricted-access environment and used solely for analysis. Throughout the research process, strict confidentiality was maintained, adhering to ethical standards and safeguarding participants’ privacy.}

%After obtaining informed consent, anonymized background information was collected from participants (see Table ~\ref{tab:participant_data}). Subsequently, in-depth, one-on-one interviews were conducted with each participant using a semi-structured format guided by a pre-prepared outline of 27 open-ended questions. Each interview lasted approximately 80 to 100 minutes, allowing participants to express their perspectives in detail. \textcolor{red}{To ensure accurate and comprehensive data collection, interviews were recorded using the feature of the online meeting platform.} All recordings were securely stored in a restricted-access environment and used solely for analysis. Throughout the research process, strict confidentiality was maintained, adhering to ethical standards and safeguarding participants’ privacy.

After obtaining informed consent, anonymized background information was collected from participants (see Table ~\ref{tab:participant_data}). Subsequently, in-depth, one-on-one interviews were conducted with each participant using a semi-structured format guided by a pre-prepared outline of 27 open-ended questions. Each interview lasted approximately 80 to 100 minutes, allowing participants to express their perspectives in detail. \textcolor{black}{To ensure accurate and comprehensive data collection, interviews were recorded using the online meeting platform.} All recordings were securely stored in a restricted-access environment and used solely for analysis. Throughout the research process, strict confidentiality was maintained, adhering to ethical standards and safeguarding participants’ privacy.



%\textcolor{black}{The study adhered to widely recognized ethical frameworks~\cite{wma_helsinki_declaration}, prioritizing the safety, rights, and dignity of all participants. Informed consent was obtained from participants before data collection, ensuring they were fully informed about the study’s background, objectives, and purpose. Participants’ rights were clearly explained, including the voluntary nature of participation, the option to withdraw at any time without penalty, and the confidentiality of their data. Privacy protection measures were explicitly outlined, such as anonymizing all responses and securely storing collected data. To facilitate informed decision-making, participants were given one week to review the detailed consent form and carefully consider their participation, ensuring no undue pressure or coercion was applied.}

%\textcolor{black}{After obtaining informed consent, anonymized participant background information was collected} (see Table ~\ref{tab:participant_data}). One-on-one semi-structured interviews, guided by 27 open-ended questions, were conducted, each lasting 80 to 100 minutes. Interviews were recorded via the online meeting platform’s recording feature for accurate data collection. \textcolor{black}{All recordings were securely stored in a restricted-access environment, used solely for analysis, and strict confidentiality was maintained to safeguard participants’ privacy.}

%\textcolor{black}{After obtaining informed consent, anonymized background information was collected from participants} (see Table ~\ref{tab:participant_data}). Subsequently, in-depth, one-on-one interviews were conducted using a semi-structured format guided by a pre-prepared outline of 27 open-ended questions. Each interview lasted approximately 80 to 100 minutes, providing participants with ample opportunity to express their perspectives in detail. To ensure accurate and comprehensive data collection, interviews were recorded using the online meeting platform’s recording feature, capturing participants’ responses in their entirety. \textcolor{black}{All recordings were securely stored in a restricted-access environment and used solely for analysis. Throughout the research process, strict confidentiality was maintained, upholding ethical standards and safeguarding participants’ privacy.}

%\textcolor{black}{The study strictly adhered to widely recognized ethical frameworks~\cite{wma_helsinki_declaration}, ensuring the safety, rights, and dignity of all participants. Informed consent was obtained from all participants prior to data collection. Participants were provided with a comprehensive explanation of the study’s background, objectives, and purpose to ensure they were fully informed. This process clarified their rights, including the voluntary nature of their participation, their right to withdraw from the study at any time without penalty, and the confidentiality of their data. Measures to protect privacy were explicitly outlined, such as the anonymization of all responses and the secure storage of collected data. To allow participants ample time to make an informed decision, they were given one week to review the detailed consent form and carefully consider their participation, ensuring that no undue pressure or coercion was applied.}

%\textcolor{black}{After obtaining informed consent, anonymized background information was collected from participants} (see Table ~\ref{tab:participant_data}). Subsequently, we conducted in-depth, one-on-one interviews using a semi-structured interview format guided by an interview outline consisting of 27 open-ended questions prepared in advance. Each interview lasted approximately 80 to 100 minutes, providing participants the opportunity to express their perspectives comprehensively. To ensure accurate and reliable data collection, we utilized the recording feature of the online meeting platform, which allowed for a complete and detailed capture of participants’ responses. \textcolor{black}{All recordings were securely stored in a restricted-access environment and were used exclusively for analysis purposes. Strict confidentiality was maintained throughout the research process, ensuring compliance with ethical standards and safeguarding participants’ privacy.}


%The study initially sought informed consent from the participants. We outlined the background and objectives of the interviews to ensure that participants were fully informed about the purpose of the research. Following this, we collected background information from the participants (see Table ~\ref{tab:participant_data}). Subsequently, we conducted in-depth, one-on-one interviews with each participant using a semi-structured interview method based on an interview outline consisting of 27 open-ended questions prepared in advance. Each interview lasted approximately 80 to 100 minutes. To ensure the interview content was accurately captured, we utilized the recording feature of the online meeting platform to document participants' responses comprehensively, allowing for a comprehensive analysis of their views.
\begin{figure*}
%[H]
 \centering         
\includegraphics[width=1\textwidth]{figure/overview.jpg} 
 \caption{\textcolor{black}{Overview of This Study’s Findings}. This figure summarizes the three themes aligned with the research questions: existing practices, \textcolor{black}{attitudes toward AI integration}, and future expectations for AI, including \textcolor{black}{nine} sub-themes and key findings.}
 \label{overview}
\Description{Description for Figure 1:
This Figureure summarizes the three themes aligned with the research questions: Existing practices, Attitudes towards AI integration, and Future expectations for AI, including nine sub-themes and all key findings.

Theme 1: Existing Practices
Sub-theme 1: Workflow: Screenwriters' creative process is divided into several steps, including:
  1. Goal and Idea
  2. Synopsis and Outline
  3. Character
  4. Story Structure and Plot
  5. Dialogue
  6. Screenplay Text
Sub-theme 2: Challenges faced by screenwriters during creation include:
  - Lack of Inspiration
  - Insufficient coherence
  - Inadequate content depth
  - Lack of emotional resonance
Sub-theme 3: Task Allocated to AI: Screenwriters' satisfaction with AI taking on certain tasks:
  - Satisfied with AI for Goal and Idea and Character stages.
  - Not satisfied with AI for Story Structure and Plot, Dialogue, and Screenplay Text stages.
Theme 2: Attitudes Towards AI
This section explores both positive and negative attitudes of screenwriters toward AI, divided as follows:
Sub-theme 4: Attitudes Toward Current AI Integration
1. Positive:
    - Rapid generation to reduce trial-and-error costs
- Efficient retrieval and summarization to expand knowledge boundaries
- Visual generation to inspire overlooked ideas
2. Negative
    - Limited knowledge and skills of screenwriters reducing their willingness to use AI
- Inaccuracy and uncontrollability limiting AI models' usage in complex narrative tasks
- Lack of authentic experiences hindering AI models' emotion perception 
3.Contradictory
-Structured text generation capability: Enhanced efficiency for structural tasks, and Unsuitable for complex narrative tasks
- Pastiche text and image generation capability: Rejection in realistic genres, and Adoption in speculative and non-realistic genres
Sub-theme 5: Attitudes Toward the Impact of AI Integration:
1. Positive:
    - AI’s potential to enhance all stages that require divergent thinking
- AI’s potential to improve communication among stakeholders
2.Negative:
- Authorship and copyright
3. Contradictory
- Al as competitor: Enhancing storytelling quality, and Facing job displacement risks
Theme 3: Future Expectations for AI
This section lists screenwriters' expectations for AI development in the future, divided into the following categories:
Sub-theme 6: Actor
1. Simulating characters based on multiple requirements:
  - Modeling internal emotions
  - Simulating external behaviors
  - Representing character environments
  - Supporting multimodal output formats
2. Engaging with screenwriters with different methods. Possible roles that screenwriters might take when interacting with AI:
  - Internal character
  - External character
  - Observer
Sub-theme 7: Audience
  - Evaluating as mass audience
  - Providing feedback as a specific group of audience
Sub-theme 8: Expert
  - Providing professional guidance and optimization suggestions
  - Promoting new workflows and cultivating multi-skilled filmmakers
Sub-theme 9: Executor
  - Managing continuation: Refining details, and Generating complete screenplays
- Visualizing presentation: Visualizing plot structures and character relationships, and Visualizing emotional rhythms}
 \end{figure*}
 
 
\subsection{Data Analysis}

%Our analysis involved reviewing quotes extracted from approximately 1,992 minutes of transcribed audio from 23 participants. We collected 117 pages of interview notes and 558 pages of transcripts. Two authors conducted a qualitative analysis using an inductive approach to open coding, following the six stages of thematic analysis~\cite{braun2006using}: 1) Familiarizing ourselves with the data: We reviewed transcripts and recordings to ensure a full understanding of the content. 2) Generating initial codes: We independently coded sections based on the interview outline, including screenwriting workflow, AI tool feedback, and interaction methods. 3) Searching for themes: We compared initial codes and explored emerging themes. To adhere to the reflective thematic analysis method, we did not establish inter-rater reliability but instead acknowledged the influence of both authors on the analysis. 4) Reviewing themes: After multiple discussions, we refined the themes, aligning them with the data and identifying relevant examples. 5) Defining and naming themes: Themes were finalized, and their definitions clarified. 6) Generating the report: The first author reviewed the final coding and discussed it with the other authors to produce the final report. This process led us to identify three main themes with ten sub-themes, which will be presented in the following sections with detailed findings and examples as Fig. \ref{overview}:


Our analysis involved reviewing quotes extracted from approximately 1,992 minutes of transcribed audio from 23 participants. We collected 117 pages of interview notes and 558 pages of transcripts. Two authors conducted a qualitative analysis using an inductive approach to open coding, following the six stages of thematic analysis~\cite{braun2006using}: 1) Familiarizing ourselves with the data: We reviewed transcripts and recordings to ensure a full understanding of the content. 2) Generating initial codes: We independently coded sections based on the interview outline, including screenwriting workflow, AI tool feedback, and interaction methods. 3) Searching for themes: We compared initial codes and explored emerging themes. To adhere to the reflective thematic analysis method, we did not establish inter-rater reliability but instead acknowledged the influence of both authors on the analysis. 4) Reviewing themes: After multiple discussions, we refined the themes, aligning them with the data and identifying relevant examples. 5) Defining and naming themes: Themes were finalized, and their definitions were clarified. 6) Generating the report: The first author reviewed the final coding and discussed it with the other authors to produce the final report. This process led us to identify three themes with \textcolor{black}{nine} sub-themes, which will be presented in the following sections with detailed findings and examples as Fig. \ref{overview}:

%Our analysis involved examining quotes extracted from approximately 1,992 minutes of transcribed audio recordings from the 23 participants. A total of 117 pages of interview notes and 558 pages of interview transcripts were collected. The two authors conducted a qualitative analysis of the interview recordings and texts based on the interview outlines. Specifically, because our study was exploratory, we chose an inductive approach to open coding and strictly followed the six stages of thematic analysis as outlined~\cite{braun2006using}: 1) Familiarizing ourselves with the data: The two authors read through the interview transcripts and replayed the interview recordings to ensure complete familiarity with the content. 2) Generating initial codes: The two authors independently coded the initial codes based on sections in the interview outline: basic information and daily screenwriting workflow, feedback on the use of AI tools in the screenwriting process, and evaluation of potential interaction methods. 3) Searching for themes: The two authors discussed their independent initial codes and jointly explored related themes. To stay true to the reflective thematic analysis method, we did not aim to establish inter-rater reliability but instead acknowledged the influence of each author on the analysis, followed by further discussions. 4) Reviewing the themes: After four rounds of discussions, the two authors reviewed the themes together three times to check whether the data aligned with the themes, revising the themes as necessary. After another two rounds of discussions and adjustments, the sub-themes under these themes were unified, and all relevant specific examples were identified from the original interviews for use in writing the article. 5) Defining and naming themes: The two authors discussed the themes, merged the content of each theme, and clarified the final definitions and names of each theme. 6) Generating the report: The first author re-read the final coding content of each theme, discussed it with the other author, and formed the final results report. Through these steps, this study conducted a detailed analysis of the following three themes, which include a total of ten sub-themes. The following sections will present the corresponding findings and specific examples in detail:

\begin{itemize}
\item  \textbf{Section~\ref{sec:Practices}:} Existing practices: workflow, challenges, and task allocated to AI (\textbf{RQ1})

\item  \textbf{Section~\ref{sec:Attitudes}:} Attitudes: positive, negative, and contradictory cases (\textbf{RQ2})

\item  \textbf{Section~\ref{sec:Expectations}:} Future expectations: actor, audience, expert, and executor (\textbf{RQ3})

\end{itemize}

 
\section{\textcolor{black}{Findings 1: Existing Practices}}\label{sec:Practices}

%Seventy-eight percent of our 23 participants reported having used AI in the screenwriting process and were able to provide real-life insights into the application of AI by screenwriters. However, 5 participants (22\%) could only share their views on future expectations and attitudes toward AI. Therefore, the content of Sections~\ref{sec:challenges} and \ref{sec:Allocation} includes specific feedback from the 18 participants (78\%) who stated they had used AI in the screenwriting process. The content of the other sections includes the perspectives of all 23 participants. Currently, the tools mentioned by participants are primarily divided into two categories: traditional screenwriting tools and AI tools. AI tools include AI text generation tools (e.g., ChatGPT, WPS AI, Kimi, AI Dungeon), AI image generation tools (e.g., Midjourney, Runway), and AI sound generation tools (e.g., DeepMusic).

Seventy-eight percent of our 23 participants reported having used AI in the screenwriting process, providing feedback on AI's application. Therefore, \textcolor{black}{Sections~\ref{sec:workflow} and~\ref{sec:challenges}} include perspectives from all 23 participants, while the \textcolor{black}{Section \ref{sec:Allocation}} reflects specific feedback from the 18 participants (78\%) who had used AI in screenwriting. The tools mentioned by participants fall into two categories: traditional screenwriting tools and AI tools. The AI tools they used include AI text generation tools (e.g., ChatGPT, WPS AI, Kimi, AI Dungeon), AI image generation tools (e.g., Midjourney, Runway), and AI sound generation tools (e.g., DeepMusic).


\subsection{Workflow}\label{sec:workflow}
\begin{figure*}
%[H]
 \centering         
\includegraphics[width=1\textwidth]{figure/workflow.jpg} 
 \caption{A Common Screenwriting Workflow Summarized from 23 Participants. The six blue blocks represent common stages in the nonlinear workflow, without a specific order. The detailed workflows are provided in the supplementary material.}
 \label{workflow}
\Description{Description for Figure 2:
This diagram illustrates a common traditional screenwriting workflow, summarized from the practices of 23 participants. The process is represented by six blue blocks, each outlining a different stage in the nonlinear workflow. These stages can occur in any order, without a strict sequence. 

Here is a breakdown of each step in the process:

1. Goal & Idea: This stage focuses on capturing creative ideas and concepts through life experiences, observations, reading, and conversations.

2. Synopsis & Outline: The initial ideas and inspiration are expanded into a concise story summary or a more detailed outline.

3. Character: The development of characters involves creating backgrounds and personality traits, including their motivations, goals, relationships, and development arcs.

4. Story Structure & Plot: This step involves arranging the overall framework and key plot points, including the beginning, development, climax, and resolution, ensuring a coherent narrative arc.

5. Dialogue: Writing dialogues that reflect the characters' personalities and backgrounds, while also advancing the story.

6. Screenplay Text: A complete screenplay that includes scene descriptions, action directives, and dialogue, forming the full script for a film or TV series.}
 \end{figure*}
 
%As mentioned in Section 2.4 regarding the complexity of the screenwriting workflow, we first summarized the participants’ workflows. Through interviews with 23 screenwriters, we identified a common screenwriting workflow encompassing the following key stages: goal \& idea, synopsis \& outline, character, story structure \& plot, dialogue, and screenplay text as Fig. \ref{workflow}. Despite individual differences, we identified a consensus on the core stages involved in screenwriting, highlighting shared practices across creative approaches. This commonality lends credibility to the workflow we have summarized. Our subsequent analysis of task allocation will be based on this workflow framework.

As mentioned in Section 2.4 regarding the complexity of the screenwriting workflow, we first summarized the participants' workflows. Through interviews with 23 screenwriters, we identified a common workflow encompassing the following key stages: goal \& idea, synopsis \& outline, character, story structure \& plot, dialogue, and screenplay text as Fig. \ref{workflow}. Despite individual differences, we found consensus on the core stages involved in screenwriting, highlighting shared practices across creative approaches. This commonality supports the workflow we have summarized. Our subsequent analysis will proceed within the context of this workflow.
%Our subsequent analysis will be based on this framework.

%Although some screenwriters exhibited variations in their specific processes, the overall structure remained consistent. For instance, some screenwriters prefer to return to the outline stage after developing characters and plots to make revisions before proceeding to write the screenplay (P1). Others focus on refining character traits and relationships before completing the synopsis (P1). In certain cases, screenwriters might advance directly to the screenplay after establishing the plot (P12), while others incorporate additional steps, such as scene outlines or act breakdowns, before moving forward (P4, P10).

 
\subsection{Challenges}\label{sec:challenges}
Before exploring how screenwriters use AI, we first analyzed the challenges they face in their current workflow, which can be categorized into four main aspects: lack of inspiration, insufficient coherence, inadequate content depth, and lack of emotional resonance. Among the participants, 15 out of 23 reported frequently experiencing a lack of inspiration, primarily due to a limited understanding of the real world and personal experience, which hinders the generation of innovative ideas in character development and plot creation. Additionally, 22 out of 23 participants mentioned issues with coherence in their screenplays. \textcolor{black}{The importance of aligning dialogue with context and character settings was emphasized by 13 participants, and 10 participants highlighted the need for logical coherence in plot structure.} Furthermore, seven out of 23 participants felt their screenplays often lacked depth, particularly in story themes and emotional dialogue. Another eight participants expressed difficulties in creating content that resonates with the audience, especially in dialogue and plot. In subsequent Sections~\ref{sec:Allocation}, we found that screenwriters have already begun exploring the use of AI to address some of these challenges.

%Before exploring how screenwriters use AI, we first analyzed the challenges they face in their current workflow, which can be categorized into four main aspects: lack of inspiration, insufficient coherence, inadequate content depth, and lack of emotional resonance. Among the participants, 15 out of 23 reported frequently experiencing a lack of inspiration during the screenwriting process, primarily due to limited understanding of the real world and personal experience, which hinders the generation of innovative ideas in character development and plot creation. Additionally, 22 out of 23 participants mentioned encountering issues with coherence in their screenplay, with 13 of them emphasizing that dialogue must align with the context and character settings, and 10 participants highlighted the critical importance of ensuring the logical coherence of plot structure to the overall screenplay. Furthermore, 7 out of 23 participants felt that their screenplays often lacked depth, particularly in the development of story themes and the emotional depth of dialogue. Another 8 participants expressed challenges in creating content that resonates with the audience, particularly in dialogue and plot elements. Additionally, some participants noted that they struggle with visualizing scene spaces, making it difficult to seamlessly imagine and translate screenplays into actual production, and they also mentioned a lack of professional guidance. In subsequent Sections~\ref{sec:Allocation} , we found that screenwriters have already begun to explore the use of AI to address some of these challenges.



\subsection{Task Allocated to AI} \label{sec:Allocation}

%\textcolor{black}{To understand the tasks currently allocated to AI by screenwriters, we analyzed the usage across the six workflow stages mentioned above, based on responses from the 18 participants who had previously used AI in screenwriting, as shown in Table~\ref{tab:Task Allocation 23}}. In categorizing these tasks, we distinguished between human-provided information and AI-generated content. Human-provided information refers to the input screenwriters contribute when interacting with AI, while AI-generated content refers to the output that AI produces based on the screenwriter's input, as outlined in Table \ref{tab:Task Allocation}.

%\textcolor{black}{To understand screenwriters' use of AI across different workflow stages, we analyzed the usage and satisfaction across the six workflow stages mentioned above, based on responses from the 18 participants who had previously used AI in screenwriting, as shown in Table~\ref{tab:Task Allocation 23}. The five participants who had not used AI in screenwriting (N) were not included in this section as they did not provide relevant data. However, we found that their reasons for not having tried AI in screenwriting were primarily due to perceiving AI as having a high usage threshold (N3, and N4) or assuming that AI-generated results would be unsatisfactory (N1, N2, and N5). In addition, to further understand the tasks currently allocated to AI by screenwriters at different workflow stages, we categorized specific tasks by distinguishing between human-provided information and AI-generated content.} Human-provided information refers to the input screenwriters contribute when interacting with AI, while AI-generated content refers to the output produced by AI based on the screenwriter's input, as outlined in Table~\ref{tab:Task Allocation}.

To understand screenwriters' use of AI across different workflow stages, we analyzed the usage and satisfaction across the six workflow stages mentioned above, based on responses from the \textcolor{black}{18} participants who had previously used AI in screenwriting, as shown in Table~\ref{tab:Task Allocation 23}. In addition, to further understand the tasks currently allocated to AI by screenwriters at different workflow stages, we categorized specific tasks by distinguishing between human-provided information and AI-generated content. Human-provided information refers to the input screenwriters contribute when interacting with AI, while AI-generated content refers to the output produced by AI based on the screenwriter's input, as outlined in Table~\ref{tab:Task Allocation}. \textcolor{black}{Note that the five participants who had not used AI in screenwriting (N) were not included in this section as they did not provide relevant data. We found that their reasons for not having tried AI in screenwriting were primarily due to perceiving AI as having a steep learning curve (N3, and N4) or assuming that AI-generated results would be unsatisfactory (N1, N2, N4, and N5).}

%To illustrate the tasks allocated to AI, we categorized the information into human-provided and AI-generated. Human-provided information refers to the input that the screenwriter contributes when using AI, while AI-generated content refers to the output that the AI produces based on the screenwriter’s input, as shown in Table \ref{tab:Task Allocation}.



\begin{comment}
\begin{table*}
\caption{Task Allocation with AI in the Screenwriting Workflow Among 23 Screenwriters. Green represents participants who used AI at this stage and were satisfied with AI output. Violet represents participants who used AI at this stage but were not satisfied with AI output. Blank indicates that the participants have not yet used AI tools at this stage.}

\vspace{0.1cm}
\label{tab:Task Allocation 23}
\tiny
\begin{tabularx}{\textwidth}{|>{\centering\arraybackslash}X|>{\centering\arraybackslash}X|>{\centering\arraybackslash}X|>{\centering\arraybackslash}X|>{\centering\arraybackslash}X|>{\centering\arraybackslash}X|>{\centering\arraybackslash}X|}

\hline
\textbf{No. / Workflow Stage} & \textbf{Goal \& Idea} & \textbf{Synopsis \& Outline} & \textbf{Character} & \textbf{Story Structure \& Plot} & \textbf{Dialogue} & \textbf{Screenplay Text} \\ \hline
N1 & & & & & & \\ \hline
N2 & & & & & & \\ \hline
P1 & \cellcolor[HTML]{9fe4a0} & & & \cellcolor[HTML]{feb4b4} & \cellcolor[HTML]{feb4b4} & \\ \hline
N3 & & & & & & \\ \hline
P2 & & \cellcolor[HTML]{feb4b4} & \cellcolor[HTML]{9fe4a0} & \cellcolor[HTML]{feb4b4} & & \\ \hline
P3 & \cellcolor[HTML]{9fe4a0} & & & & \cellcolor[HTML]{feb4b4} & \\ \hline
P4 & \cellcolor[HTML]{9fe4a0} & \cellcolor[HTML]{feb4b4} & \cellcolor[HTML]{9fe4a0} & \cellcolor[HTML]{feb4b4} & \cellcolor[HTML]{feb4b4} & \cellcolor[HTML]{feb4b4} \\ \hline
P5 & & & & & & \cellcolor[HTML]{feb4b4} \\ \hline
P6 & & & & \cellcolor[HTML]{feb4b4} & & \\ \hline
P7 & & & & & & \cellcolor[HTML]{feb4b4} \\ \hline
P8 & & & & \cellcolor[HTML]{9fe4a0} & \cellcolor[HTML]{feb4b4} & \\ \hline
P9 & & & & \cellcolor[HTML]{9fe4a0} & \cellcolor[HTML]{feb4b4} & \\ \hline
P10 & & & & & & \cellcolor[HTML]{feb4b4} \\ \hline
P11 & \cellcolor[HTML]{9fe4a0} & & & & \cellcolor[HTML]{feb4b4} & \\ \hline
P12 & & & & \cellcolor[HTML]{9fe4a0} & & \cellcolor[HTML]{feb4b4} \\ \hline
P13 & & & & \cellcolor[HTML]{feb4b4} & & \\ \hline
N4 & & & & & & \\ \hline
P14 & & & & \cellcolor[HTML]{feb4b4} & & \cellcolor[HTML]{feb4b4} \\ \hline
P15 & & & \cellcolor[HTML]{feb4b4} & \cellcolor[HTML]{9fe4a0} & & \cellcolor[HTML]{9fe4a0} \\ \hline
P16 & \cellcolor[HTML]{9fe4a0} & & & \cellcolor[HTML]{feb4b4} & & \cellcolor[HTML]{9fe4a0} \\ \hline
P17 & \cellcolor[HTML]{9fe4a0} & \cellcolor[HTML]{9fe4a0} & \cellcolor[HTML]{9fe4a0} & \cellcolor[HTML]{feb4b4} & & \\ \hline
P18 & & & & & & \cellcolor[HTML]{feb4b4} \\ \hline
N5 & & & & & & \\ \hline
Participants used AI & 6 & 3 & 4 & 12 & 6 & 9 \\ \hline
\end{tabularx}
% \textit{Note:} Green represents participants who used AI at this stage and were satisfied with AI output. Red represents participants who used AI at this stage but were not satisfied with AI output. Blank indicates that the participants have not yet used AI tools at this stage.
\end{table*}

\end{comment}

\newcolumntype{L}[1]{>{\raggedright\let\newline\\\arraybackslash\hspace{0pt}}m{#1}}
\newcolumntype{C}[1]{>{\centering\let\newline\\\arraybackslash\hspace{0pt}}m{#1}}
\newcolumntype{R}[1]{>{\raggedleft\let\newline\\\arraybackslash\hspace{0pt}}m{#1}}

% Please add the following required packages to your document preamble:
% \usepackage[table,xcdraw]{xcolor}
% Beamer presentation requires \usepackage{colortbl} instead of \usepackage[table,xcdraw]{xcolor}

\begin{table*}
\caption{\textcolor{black}{Task Allocation with AI in the Screenwriting Workflow Among 18 Screenwriters Who Have Previously Used AI in Screenwriting.} Green represents participants who used AI at this stage and were satisfied with AI output, \textcolor{black}{denoted as ``S'' in the table}. Red represents participants who used AI at this stage but were dissatisfied with AI output, \textcolor{black}{denoted as “D” in the table}. Blank indicates that the participants have not yet used AI tools at this stage.}
\Description{This table represents task allocation with AI in the screenwriting workflow among 18 screenwriters who have previously used AI in screenwriting. It provides information about whether participants used AI at different stages of the workflow and their satisfaction with the AI-generated output.
Key information:
  Rows represent workflow stages:
    1. Goal \& Idea
    2. Synopsis \& Outline
    3. Character
    4. Story Structure \& Plot
    5. Dialogue
    6. Screenplay Text
  Columns represent participants (P1 to P18), with a "Total" column summarizing usage counts for each stage.
Color codes and notations:
  Green cells marked as "S" indicate participants who used AI at a specific stage and were satisfied with its output.
  Red cells marked as "D" indicate participants who used AI at a specific stage but were dissatisfied with its output.
  Blank cells indicate that participants have not yet used AI tools at that stage.
Summary of usage:
1. Goal \& Idea: AI was used by 8 participants, all satisfied.
2. Synopsis \& Outline: AI was used by 3 participants, with mixed satisfaction levels.
3. Character: AI was used by 5 participants, with mixed satisfaction levels.
4. Story Structure \& Plot: AI was used by 12 participants, with mixed satisfaction levels.
5. Dialogue: AI was used by 6 participants, all dissatisfied.
6. Screenplay Text: AI was used by 9 participants, with mixed satisfaction levels.
The table highlights the stages where AI tools are used most often (e.g., Story Structure \& Plot) and participant satisfaction.}
\label{tab:Task Allocation 23}


\vspace{0.1cm}


\scriptsize

% Define patterns for red and green cells

\newcommand{\patterncellred}{\cellcolor[HTML]{EF949F}\makebox[0pt][c]{D}D\tikz[overlay] \fill[pattern=north west lines, pattern color=black] (0,0) rectangle (\linewidth,\baselineskip);}
\newcommand{\patterncellgreen}{\cellcolor[HTML]{ADD88D}\makebox[0pt][c]{S}S\tikz[overlay] \fill[pattern=horizontal lines, pattern color=black] (0,0) rectangle (\linewidth,\baselineskip);}


\centering
\begin{tabular}{|c|c|c|c|c|c|c|c|c|c|c|c|c|c|c|c|c|c|c|c|}

\hline
\multicolumn{1}{|c|}{\textbf{Stage/Participant}} & \multicolumn{1}{c|}{\textcolor{black}{\textbf{P1}}} & \textcolor{black}{\textbf{P2}}                       & \textcolor{black}{\textbf{P3}}                       & \textcolor{black}{\textbf{P4}}                       & \textcolor{black}{\textbf{P5}}                       & \textcolor{black}{\textbf{P6}}                       & \textcolor{black}{\textbf{P7}}                       & \textcolor{black}{\textbf{P8}}                       & \textcolor{black}{\textbf{P9}}                       & \textcolor{black}{\textbf{P10}}                      & \textcolor{black}{\textbf{P11}}                      & \textcolor{black}{\textbf{P12}}                      & \textcolor{black}{\textbf{P13}}                      & \textcolor{black}{\textbf{P14}}                      & \textcolor{black}{\textbf{P15}}                      & \textcolor{black}{\textbf{P16}}                      & \textcolor{black}{\textbf{P17}}                      & \textcolor{black}{\textbf{P18}}                      & \textbf{Total} 

\\ \hline
Goal \& Idea            & \cellcolor[HTML]{ADD88D}\makebox[0pt][c]{S} &                          & \cellcolor[HTML]{ADD88D}\makebox[0pt][c]{S} & \cellcolor[HTML]{ADD88D}\makebox[0pt][c]{S}                        &                          &                          & \cellcolor[HTML]{ADD88D}\makebox[0pt][c]{S}                        &                          &                          &                          & \cellcolor[HTML]{ADD88D}\makebox[0pt][c]{S} &                          &                          & \cellcolor[HTML]{ADD88D}\makebox[0pt][c]{S}{\color[HTML]{588E31} } &                          & \cellcolor[HTML]{ADD88D}\makebox[0pt][c]{S}                        & \cellcolor[HTML]{ADD88D}\makebox[0pt][c]{S} &                          & 8     \\ \hline
Synopsis \& Outline     &                          &                          &                          & \cellcolor[HTML]{EF949F}\makebox[0pt][c]{D}                        & \cellcolor[HTML]{EF949F}\makebox[0pt][c]{D} &                          &                                                 &                          &                          &                          &                          &                          &                          &                                                 &                          &                                                 & \cellcolor[HTML]{ADD88D}\makebox[0pt][c]{S} &                          & 3     \\ \hline
Character               &                          &                          &                          & \cellcolor[HTML]{ADD88D}\makebox[0pt][c]{S}{\color[HTML]{ADD88D} } &                          &                          &                                                 &                          & \cellcolor[HTML]{ADD88D}\makebox[0pt][c]{S} &                          &                          &                          &                          &                                                 & \cellcolor[HTML]{ADD88D}\makebox[0pt][c]{S} & \cellcolor[HTML]{EF949F}\makebox[0pt][c]{D}                        & \cellcolor[HTML]{EF949F}\makebox[0pt][c]{D} &                          & 5     \\ \hline
Story Structure \& Plot & \cellcolor[HTML]{EF949F}\makebox[0pt][c]{D} & \cellcolor[HTML]{EF949F}\makebox[0pt][c]{D} &                          & \cellcolor[HTML]{EF949F}\makebox[0pt][c]{D}                        &                          & \cellcolor[HTML]{EF949F}\makebox[0pt][c]{D} &                                                 & \cellcolor[HTML]{ADD88D}\makebox[0pt][c]{S} & \cellcolor[HTML]{EF949F}\makebox[0pt][c]{D} &                          & \cellcolor[HTML]{EF949F}\makebox[0pt][c]{D} & \cellcolor[HTML]{EF949F}\makebox[0pt][c]{D} & \cellcolor[HTML]{EF949F}\makebox[0pt][c]{D} &                                                 & \cellcolor[HTML]{ADD88D}\makebox[0pt][c]{S} & \cellcolor[HTML]{EF949F}\makebox[0pt][c]{D}{\color[HTML]{588E31} } & \cellcolor[HTML]{EF949F}\makebox[0pt][c]{D} &                          & 12    \\ \hline
Dialoge                 & \cellcolor[HTML]{EF949F}\makebox[0pt][c]{D} &                          &                          & \cellcolor[HTML]{EF949F}\makebox[0pt][c]{D}{\color[HTML]{588E31} } &                          &                          &                                                 & \cellcolor[HTML]{EF949F}\makebox[0pt][c]{D} & \cellcolor[HTML]{EF949F}\makebox[0pt][c]{D} &                          & \cellcolor[HTML]{EF949F}\makebox[0pt][c]{D} &                          &                          &                                                 & \cellcolor[HTML]{EF949F}\makebox[0pt][c]{D} &                                                 &                          &                          & 6     \\ \hline
Screenplay Text         &                          &                          &                          & \cellcolor[HTML]{EF949F}\makebox[0pt][c]{D}                        &                          &                          & \cellcolor[HTML]{EF949F}\makebox[0pt][c]{D}{\color[HTML]{92D050} } &                          &                          & \cellcolor[HTML]{EF949F}\makebox[0pt][c]{D} &                          & \cellcolor[HTML]{EF949F}\makebox[0pt][c]{D} & \cellcolor[HTML]{EF949F}\makebox[0pt][c]{D} & \cellcolor[HTML]{EF949F}\makebox[0pt][c]{D}                        & \cellcolor[HTML]{EF949F}\makebox[0pt][c]{D} & \cellcolor[HTML]{ADD88D}\makebox[0pt][c]{S}                        & {\color[HTML]{ADD88D} }  & \cellcolor[HTML]{EF949F}\makebox[0pt][c]{D} & 9     \\ \hline
\end{tabular}
\end{table*}



\begin{comment}
    
% \begin{tabular}{|c|l|l|c|c|l|l|l|l|l|l|l|l|l|l|l|l|l|l|l|l|l|l|l|l|}
% \tiny
\scriptsize
\begin{tabularx}{\textwidth}{|c|>{\centering\arraybackslash}X|>{\centering\arraybackslash}X|>{\centering\arraybackslash}X|>{\centering\arraybackslash}X|>{\centering\arraybackslash}X|>{\centering\arraybackslash}X|>{\centering\arraybackslash}X|>{\centering\arraybackslash}X|>{\centering\arraybackslash}X|>{\centering\arraybackslash}X|>{\centering\arraybackslash}X|>{\centering\arraybackslash}X|>{\centering\arraybackslash}X|>{\centering\arraybackslash}X|>{\centering\arraybackslash}X|>{\centering\arraybackslash}X|>{\centering\arraybackslash}X|>{\centering\arraybackslash}X|>{\centering\arraybackslash}X|>{\centering\arraybackslash}X|>{\centering\arraybackslash}X|>{\centering\arraybackslash}X|>{\centering\arraybackslash}X|c|}
% \begin{tabularx}{\textwidth}{|c|C{0.18cm}|c|c|c|c|c|c|c|c|c|c|c|c|c|c|c|c|c|c|c|c|c|C{0.22cm}|c|}
\hline
\textbf{Stage/No.} & \tiny \textbf{N1}        &\tiny \textbf{N2} &\tiny \textbf{P1}  &\tiny \textbf{N3} &\tiny \textbf{P2}                       &\tiny \textbf{P3}                       &\tiny \textbf{P4}                       & \tiny \textbf{P5}                       & \tiny \textbf{P6}                       &\tiny \textbf{P7}                     & \tiny \textbf{P8}                      &\tiny \textbf{P9}                      &\tiny \textbf{P10}                      &\tiny \textbf{P11}                      & \tiny \textbf{P12}                      & \tiny \textbf{P13}                      & \tiny \textbf{N4} &\tiny \textbf{P14}                      & \tiny \textbf{P15}                      &\tiny \textbf{P16}                      &\tiny \textbf{P17}                      &\tiny \textbf{P18}                      &\tiny \textbf{N5} & \textbf{Total} \\ \hline
Goal \& Idea                   &           &    & \cellcolor[HTML]{9fe4a0} &                         &                          & \cellcolor[HTML]{9fe4a0} & \cellcolor[HTML]{9fe4a0} &                          &                          &                          &                          &                          &                          & \cellcolor[HTML]{9fe4a0} &                          &                          &     &                          &                          & \cellcolor[HTML]{9fe4a0} & \cellcolor[HTML]{9fe4a0} &                          &     & 6     \\ \hline
Synopsis \& Outline            &           &    &                          &                         & \cellcolor[HTML]{feb4b4} &                          & \cellcolor[HTML]{feb4b4} &                          &                          &                          &                          &                          &                          &                          &                          &                          &     &                          &                          &                          & \cellcolor[HTML]{9fe4a0} &                          &     & 3     \\ \hline
Character                       &           &    &                          &                         & \cellcolor[HTML]{9fe4a0} &                          & \cellcolor[HTML]{9fe4a0} &                          &                          &                          &                          &                          &                          &                          &                          &                          &     &                          & \cellcolor[HTML]{9fe4a0} &                          & \cellcolor[HTML]{9fe4a0} &                          &     & 4     \\ \hline
Story Structure \& Plot        & \textbf{} &    & \cellcolor[HTML]{feb4b4} &                         & \cellcolor[HTML]{feb4b4} &                          & \cellcolor[HTML]{feb4b4} &                          & \cellcolor[HTML]{feb4b4} &                          & \cellcolor[HTML]{9fe4a0} & \cellcolor[HTML]{9fe4a0} &                          &                          & \cellcolor[HTML]{9fe4a0} & \cellcolor[HTML]{feb4b4} &     & \cellcolor[HTML]{feb4b4} & \cellcolor[HTML]{9fe4a0} & \cellcolor[HTML]{feb4b4} & \cellcolor[HTML]{feb4b4} &                          &     & 12    \\ \hline
Dialogue                        &           &    & \cellcolor[HTML]{feb4b4} &                         &                          &   & \cellcolor[HTML]{feb4b4} &                          &                          &                          & \cellcolor[HTML]{feb4b4} & \cellcolor[HTML]{feb4b4} &                          & \cellcolor[HTML]{feb4b4} &                          &                          &     &                          &                          &                          &                          &                          &     & 5     \\ \hline
Screenplay Text                 & \textbf{} &    &                          &                         &                          &                          & \cellcolor[HTML]{feb4b4} & \cellcolor[HTML]{feb4b4} &                          & \cellcolor[HTML]{feb4b4} &                          &                          & \cellcolor[HTML]{feb4b4} &                          & \cellcolor[HTML]{feb4b4} &                          &     & \cellcolor[HTML]{feb4b4} & \cellcolor[HTML]{9fe4a0} & \cellcolor[HTML]{9fe4a0} &                          & \cellcolor[HTML]{feb4b4} &     & 9     \\ \hline
\end{tabularx}

\end{comment}

\subsubsection{Goal \& Idea Stage}
In this stage, screenwriters often encounter challenges related to a lack of inspiration (Section~\ref{sec:challenges}). Consequently, they sought AI assistance. According to our interview results, P1, P3, P4, P7, P11, P14, P16, and P17 stated that they used AI at this stage and felt that AI could meet their needs. The human-provided information at this stage is primarily to determine the subject matter, genre, and elements to be included in the screenplay. Screenwriters use AI to obtain inspiration, assist with information retrieval, determine the theme direction, establish world-building, generate concept images, create story outlines, and develop character biographies. The needs of screenwriters at this stage are often vague, but they hope that AI can help them explore different concepts and inspire their next steps.
%In this stage, screenwriters often encounter challenges related to a lack of inspiration (Section~\ref{sec:challenges}). Consequently, they sought AI assistance. According to our interview results, 7 participants stated that they used AI at this stage and felt that AI could meet their current needs (\textcolor{black}{P1, P3, P4, P7, P11, P14, P16, and P17}). The human-provided information at this stage is primarily to determine the subject matter, genre, and elements to be included in the screenplay. Screenwriters use AI to obtain inspiration, assist with information retrieval, determine the theme direction, establish world-building, generate concept images, create story outlines, and develop character biographies. The needs of screenwriters at this stage are often vague, but they hope that AI can help them explore different concepts and inspire their next creative steps.

\subsubsection{Synopsis \& Outline Stage}
Only \textcolor{black}{P17} mentioned successfully using AI at this stage, stating that the human-provided information during use was script elements and character design allowing AI to generate synopsis \& outlines. 
\textcolor{black}{P17} stated, ``\textit{I sometimes feed the client's requirements into the AI to see what feedback it gives. It can help me set the tone by triggering certain keywords, which I then rearrange to get closer to what I want.}'' However, \textcolor{black}{P4 and P5} expressed dissatisfaction with AI's effectiveness at this stage. \textcolor{black}{P4} noted, ``\textit{At the beginning, I tried using keywords to generate an outline, and I also tried using an outline to generate a script, but both attempts failed. It felt like AI couldn’t understand the relationship between the characters. The content produced was strange and disconnected.}'' These responses indicate that AI, due to its lack of understanding of human emotions and complex human relationships, does not have sufficient capability to generate synopses and outlines. This aligns with the limitations in AI's current abilities (Section~\ref{sec:capabilities}).

%screenwriters prefer to handle story synopsis \& outline tasks themselves rather than relying on AI.
%\haotian{I do not think the feedback can have the implication. It can only suggest that AI may not have sufficient ability to generate outline.}

\subsubsection{Character Stage}
\textcolor{black}{P4, P9, and P15} mentioned that their tasks using AI primarily involved providing character-related needs, such as naming characters and developing settings and script elements, with AI generating names and biographies based on their input. As \textcolor{black}{P4} noted: ``\textit{For example, when I suddenly needed a name, I used AI to generate a few names for me, and I thought they were quite interesting and playful.}'' \textcolor{black}{P15} stated: ``\textit{I had a spaceship captain character. I knew roughly his age and that he was a villain. The AI quickly gave me a detailed profile, including his height, weight, tragic childhood, and his current evil goals. I could then make modifications based on this. It was much faster than trying to come up with everything from scratch, especially since gathering material for this kind of character is so complex.}'' Participants indicated that, with current inputs, AI meets basic expectations for character-related tasks like generating names and biographies. However, when it comes to understanding relationships between characters, AI falls short (\textcolor{black}{P16 and P17}).

%\textcolor{black}{P4, P9, and P15} mentioned that their tasks using AI primarily involved providing specific character-related needs, such as naming characters and developing character settings and script elements, with AI generating names and biographies based on their input. As \textcolor{black}{P4} noted: ``\textit{For example, when I suddenly needed a name, I used AI to generate a few names for me, and I thought they were quite interesting and playful.}'' \textcolor{black}{P15} stated: ``\textit{I had a spaceship captain character. I knew roughly his age and that he was a villain. The AI quickly gave me a detailed profile, including his height, weight, tragic childhood, and his current evil goals. I could then make modifications based on this. It was much faster than trying to come up with everything from scratch, especially since gathering material for this kind of character is so complex.}'' Participants indicated that, with current inputs, AI meets basic expectations for character-related tasks like generating names and biographies. However, when it comes to understanding relationships between characters, AI falls short (\textcolor{black}{P16 and P17}).

%\textcolor{black}{P2, P4, P15, and P17} mentioned their tasks using AI primarily involved providing specific character-related needs, such as naming characters and providing character settings and script elements, with AI generating names and biographies based on this input. As \textcolor{black}{P4} mentioned: ``\textit{For example, when I suddenly needed a name, I used AI to generate a few names for me, and I thought they were quite interesting and playful.}'' \textcolor{black}{P15} stated: ``\textit{I had a spaceship captain character. I knew roughly his age and that he was a villain. The AI quickly gave me a detailed profile, including his height, weight, tragic childhood, and his current evil goals. I could then make modifications based on this. It was much faster than trying to come up with everything from scratch, especially since gathering material for this kind of character is so complex.}'' Based on our interviews, participants indicated that, given the current inputs, AI is able to meet basic expectations for character-related tasks such as generating character names and biographies.


\subsubsection{Story Structure \& Plot Stage}
In this stage, screenwriters often meet challenges in maintaining the coherence of story pieces and having an in-depth structure (Section~\ref{sec:challenges}). As a result, they would like to seek AI's assistance. Compared to other stages, more screenwriters preferred to work with AI in this stage. 
According to our interview results, 12 participants reported using AI, but only two of them stated that AI could produce satisfactory results (\textcolor{black}{P8 and P15}). 
This stage had the most AI users and involved various human-provided informations, including providing genre, basic plots, storyboards, character settings, story outlines, scene settings, and reference images. 
AI tasks included expanding plot details, summarizing, organizing, continuing the plot, generating multiple plot possibilities, and generating multiple scene possibilities. Despite the specific and varied needs of screenwriters at this stage, \textcolor{black}{P1, P2, P4, P6, P9, P11, P12, P13, P16, and P17} expressed dissatisfaction with AI's performance. 
\textcolor{black}{P13} noted that while AI-generated structures were technically sound, the content was rigid. 
\textcolor{black}{P16} felt that while AI-generated images helped expand plot details, AI-generated text was often dull and logically incoherent. 
\textcolor{black}{P2, P6, and P17} criticized AI’s limited understanding of film and television resources, with \textcolor{black}{P17} stating: ``\textit{The content generated by the AI is unusable for me.}'' This is a process that participants frequently feel inefficient and time-consuming. These collective insights underscore the challenges screenwriters face when integrating AI into the story structure creation process, pointing to the need for significant improvements in AI's ability to understand and generate coherent, in-depth creative structure (detailed in Section~\ref{sec:capabilities}).


\subsubsection{Dialogue Stage}
At this stage, screenwriters often face challenges related to insufficient content depth and emotional resonance (Section~\ref{sec:challenges}), prompting them to explore the potential of AI assistance. However, P1, P4, P8, P9, P11, and P15 attempted to use AI for dialogue generation, but none were fully satisfied with its assistance. Their inputs include genre, character design, basic dialogue, and detailed dialogue scenarios. The primary concern among participants was that AI struggled with handling more complex scenarios, particularly in maintaining the nuanced interactions required for effective dialogue writing. For example, \textcolor{black}{P8} attempted to use ChatGPT to write a dialogue between two animals for a children’s story but found that AI couldn’t handle highly complex scenarios. \textcolor{black}{P11} mentioned, ``\textit{AI generated a short story with a Wong Kar-Wai style, including dialogue and a literary feel, but it was only useful for brief inspiration. It wasn’t suitable for actual screenwriting work.}'' These insights reveal a common sentiment among participants that, while AI can provide initial creative sparks, its current capabilities are inadequate for developing fully-realized dialogue in screenwriting, primarily due to its limited understanding of human emotions and complex interpersonal relationships. This observation aligns with the disadvantages of AI's current abilities as highlighted by participants (Section~\ref{sec:capabilities}).

%At this stage, screenwriters often face challenges related to insufficient content depth and emotional resonance (Section~\ref{sec:challenges}), prompting them to explore the potential of AI assistance. However, P1, P4, P8, P9, P11, and P15 attempted to use AI for dialogue generation, but none were fully satisfied with its assistance. Their inputs include genre, character design, basic dialogue, and dialogue scenarios. The primary concern among participants was that AI struggled with handling more complex scenarios, particularly in maintaining the nuanced interactions required for effective dialogue writing. For example, \textcolor{black}{P8} attempted to use ChatGPT to write a dialogue between two animals for a children’s story but found that AI couldn’t handle more complex scenarios. \textcolor{black}{P11} mentioned, ``\textit{AI generated a short story with a Wong Kar-Wai style, including dialogue and a literary feel, but it was only useful for brief inspiration. It wasn’t suitable for actual screenwriting work.}'' These insights reveal a common sentiment among participants that, while AI can provide initial creative sparks, its current capabilities are inadequate for developing fully-realized dialogue in screenwriting, primarily due to its limited understanding of human emotions and complex interpersonal relationships. This observation aligns with the disadvantages of AI's current abilities as highlighted by participants (Section~\ref{sec:capabilities}).

%At this stage, screenwriters often face challenges related to insufficient content depth and emotional resonance (Section~\ref{sec:challenges}), prompting them to explore the potential of AI assistance. However, 5 participants attempted to use AI for dialogue generation, none were fully satisfied with its assistance (\textcolor{black}{P1, P4, P8, P9, and P11}). Their inputs include genre, character design, basic dialogue, and dialogue scenarios. The primary concern among participants was that AI struggled with handling more complex scenarios, particularly in maintaining the nuanced interactions required for effective dialogue writing. For example, \textcolor{black}{P8} attempted to use ChatGPT to write a dialogue between two animals for a children’s story but found that AI couldn’t handle more complex scenarios. \textcolor{black}{P11} mentioned, ``\textit{AI generated a short story with a Wong Kar-Wai style, including dialogue and a literary feel, but it was only useful for brief inspiration. It wasn’t suitable for actual screenwriting work.}'' These insights reveal a common sentiment among participants that, while AI can provide initial creative sparks, its current capabilities are inadequate for developing fully-realized dialogue in screenwriting, primarily due to its limited understanding of human emotions and complex interpersonal relationships. This observation aligns with the disadvantages of AI's current abilities as highlighted by participants (Section~\ref{sec:capabilities}).


\subsubsection{Screenplay Text Stage}
In this stage, screenwriters are often concerned about the inadequate content depth of the screenplay text (Section~\ref{sec:challenges}), leading 9 participants to experiment with using AI to assist in the entire screenplay creation process. However, only P16 felt that AI could optimize screenplay content, with \textcolor{black}{P16} noting that AI was effective in transforming plain sentences into narration in a specific style. The remaining participants (\textcolor{black}{P4, P7, P10, P12, P13, P14, P15, and P18}) were generally dissatisfied with AI’s performance. \textcolor{black}{P10} described AI-generated scripts as \textit{``completely useless and misleading.''} \textcolor{black}{P12} pointed out that AI only completed ``\textit{10\% to 20\% of the work, with the rest needing to be done manually.}'' \textcolor{black}{P18} also criticized AI’s output as ``\textit{too crude and not specific enough.}'' Based on the perspectives of our participants, the majority expressed that due to AI's current limitations in contextual understanding and logical coherence (Section~\ref{sec:capabilities}), it is generally unable to generate satisfactory screenplay text in most situations.


\subsubsection{Summary}
\textcolor{black}{In our study, screenwriters primarily used AI tools in four key stages: story structure \& plot development, screenplay text, goal \& idea generation, and dialogue, particularly for repetitive tasks like plot revisions (refer to Table~\ref{tab:Task Allocation 23} and Table~\ref{tab:Task Allocation}). However, they reported less AI use in the synopsis \& outline stages, preferring to rely on personal efforts. Regarding satisfaction, participants acknowledged AI’s ability to inspire ideas and meet basic needs in the goal \& idea generation and character development stages. However, fewer participants were satisfied with AI’s performance in the story structure \& plot development, screenplay text, and dialogue stages, due to its limitations (see Section~\ref{sec:capabilities}). Future research should consider enhancing AI capabilities in these areas.}

%Within the scope of our study, screenwriters primarily utilize AI tools during four key stages: goal \& idea generation, story structure \& plot development, dialogue, and screenplay text. This includes repetitive tasks such as continuous dialogue and plot revisions, where they hope AI can provide better assistance to enhance efficiency. In contrast, participants reported less AI usage within the synopsis \& outline stage and the character stage, as these stages often involve the core thematic exploration and character potential, where screenwriters prefer to rely on personal expression and fundamental perspectives rather than AI assistance.

%Regarding AI usage satisfaction, participants generally acknowledged AI’s ability to inspire ideas, with current tools meeting their basic needs in the goal \& idea generation and character development stages. However, fewer participants expressed satisfaction with AI’s performance in the story structure, plot development, and screenplay text stages. This dissatisfaction may result from AI’s limitations in logical reasoning, contextual coherence, and its inability to fully grasp complex human emotions (Section~\ref{sec:capabilities}). Additionally, none of the participants were satisfied with AI’s performance in the dialogue stage, likely due to its failure to capture emotional nuances and the complexities of character relationships, leading to insufficient depth and resonance. Therefore, future research should prioritize enhancing AI capabilities in these areas and improving tools for lower satisfaction stages.

%Regarding AI usage satisfaction, participants generally acknowledged AI’s ability to inspire ideas, with current tools meeting their basic needs in the goal \& idea generation and character development stages. However, fewer participants were satisfied with AI’s performance in story structure \& plot development, and screenplay text stages. This dissatisfaction may stem from AI’s limitations in logical reasoning, contextual coherence, and its inability to fully grasp complex human emotions, as noted during interviews (detailed in Section~\ref{sec:capabilities}). Furthermore, none of the participants were satisfied with AI’s performance in the dialogue stage, likely due to its failure to capture emotional nuances and the complexities of character relationships, resulting in insufficient depth and resonance. Therefore, future research should prioritize enhancing AI capabilities in these areas and improving tools for stages where satisfaction was lower.

%Overall, within our study's scope, screenwriters mainly use AI tools during four key stages: goal \& idea generation, story structure \& plot development, dialogue, and screenplay text. This includes repetitive tasks like ongoing dialogue and plot revisions, where screenwriters hope AI can boost efficiency. In contrast, participants reported less AI usage during the synopsis/outline, and character development stages, which often involve core thematic exploration and character depth, where screenwriters tend to rely more on personal expression than AI assistance.

%Regarding satisfaction with AI usage, participants generally recognized AI's ability to inspire ideas, and current AI tools appear to adequately meet their basic needs during the goal \& idea generation and character development stage. However, fewer participants expressed satisfaction with AI's performance in the story structure \& plot development, and screenplay text stages. This dissatisfaction may stem from AI's limitations in logical reasoning, contextual coherence, and its misunderstanding of complex human emotions, as noted by participants during the interviews (detailed in Section~\ref{sec:capabilities}). Additionally, none of the participants were satisfied with AI's performance in the dialogue stage, likely due to AI's inability to fully grasp the nuances of emotions and the intricate dynamics of character relationships, resulting in insufficient content depth and emotional resonance. Since these capabilities and elements are crucial to screenwriting, without significant advancements in these areas, future AI tools may struggle to substantially improve the user experience for screenwriters. Therefore, future research might consider focusing more on enhancing AI capabilities in these related fields and improving AI tools for the stages where participants indicated less satisfaction.

\begin{table*}
\caption{Task Allocated to AI in the Screenwriting Workflow}
\Description{Description for Table 3:
This Table provides a detailed breakdown of tasks allocated to AI in the screenwriting workflow, dividing responsibilities between human-provided information and AI-generated content across different workflow stages. The workflow stages are as follows:
1. Goal \& Idea
 Human-provided Information: Includes Genre (e.g., P3, P16) and Elements (e.g., P1, P3, P4, P7, P11, P14, P17).
 AI-generated Content: Provides Inspiration (P7, P14, P16), Assisted retrieval of information (P1, P4), Topic direction (P3), World-building (P17), Concept images (P11), Synopsis (P17), and Character biography (P17).
2. Synopsis \& Outline
 Human-provided Information: Includes Elements (P17) , Character design (P4), and Brief outline (P5).
 AI-generated Content: Generates Synopsis (P17) and Full outline (P4, P5).
3. Character
 Human-provided Information: Involves Naming requirements (P4), Character design (P15), and Elements (P9, P16, P17).
 AI-generated Content: Creates Character names (P4), Character biographies (P9, P15, P17), and Character relationship (P16, P17).
4. Story Structure \& Plot
 Human-provided Information: Includes Genre (P1), Basic plot (P2, P6, P8, P9, P15), Character design (P4, P6, P16, P17), Synopsis (P1, P12, P13, P16, P17), Scene design (P16), Scene reference image (P16), and Element (P16).
 AI-generated Content: Expands plot details (P1, P4, P9, P11, P12, P13), Plot continuation (P9, P16), Multiple plot possibilities (P1, P2, P6, P8, P9, P15, P17), and Multiple scene possibilities (P16).
5. Dialogue
 Human-provided Information: Covers Genre (P11), Character design (P4, P8, P9), Dialogue scenarios (P1, P4), Synopsis (P1), and No details were mentioned by P15.
 AI-generated Content: Includes Basic dialogue (P4, P8, P9), Stylized dialogue (P11), Continuation writing dialogue (P1, P9), and Generating complete dialogue (P1, P15).
6. Screenplay Text
 Human-provided Information: Includes Genre (P7, P10, P16, P18), Element ((P10, P16, P18), Screenplay text (P12, P16), Synopsis (P4, P13, P14), Character design (P18), and No details were mentioned by P15.
 AI-generated Content: Translation (P16), Narration (P16), Screenplay text (P4, P7, P10, P13, P14, P15, P18), and Format (P12).
This table highlights how human input guides AI outputs at each stage and shows specific participants associated with the contributions and outputs.}
\label{tab:Task Allocation}
\footnotesize
\begin{tabular}{|l|l|l|}
\hline
\textbf{Workflow Stage }                          & \textbf{Human-provided Information}                                                                                                                                                                                                                                                & \textbf{AI-generated Content}                                                                                                                                                                                                                                                   \\ \hline
\multirow{8}{*}{Goal \& Idea}            & \multirow{8}{*}{\begin{tabular}[c]{@{}l@{}}1. Genre (P3, P16)\\ 2. Element (P1, P3, P4, P7, P11, P14, P17)\end{tabular}}                                                                                                                                                  & \multirow{8}{*}{\begin{tabular}[c]{@{}l@{}}1. Inspiration (P7, P14, P16)\\ 2. Assisted retrieval of information (P1, P4)\\ 3. Topic direction (P3)\\ 4. World building (P17)\\ 5. Concept image (P11)\\ 6. Synopsis (P17)\\ 7. Character biography (P17)\end{tabular}} \\
                                         &                                                                                                                                                                                                                                                                           &                                                                                                                                                                                                                                                                        \\
                                         &                                                                                                                                                                                                                                                                           &                                                                                                                                                                                                                                                                        \\
                                         &                                                                                                                                                                                                                                                                           &                                                                                                                                                                                                                                                                        \\
                                         &                                                                                                                                                                                                                                                                           &                                                                                                                                                                                                                                                                        \\
                                         &                                                                                                                                                                                                                                                                           &                                                                                                                                                                                                                                                                        \\
                                         &                                                                                                                                                                                                                                                                           &                                                                                                                                                                                                                                                                        \\
                                         &                                                                                                                                                                                                                                                                           &                                                                                                                                                                                                                                                                        \\ \hline
\multirow{4}{*}{Synopsis \& Outline}     & \multirow{4}{*}{\begin{tabular}[c]{@{}l@{}}1. Element (P17)\\ 2. Character design(P4)\\ 3. Brief outline (P5)\end{tabular}}                                                                                                                                               & \multirow{4}{*}{\begin{tabular}[c]{@{}l@{}}1.Synopsis (P17)\\ 2.Full outline(P4, P5)\end{tabular}}                                                                                                                                                                     \\
                                         &                                                                                                                                                                                                                                                                           &                                                                                                                                                                                                                                                                        \\
                                         &                                                                                                                                                                                                                                                                           &                                                                                                                                                                                                                                                                        \\
                                         &                                                                                                                                                                                                                                                                           &                                                                                                                                                                                                                                                                        \\ \hline
\multirow{4}{*}{Character}               & \multirow{4}{*}{\begin{tabular}[c]{@{}l@{}}1. Naming requirements (P4)\\ 2. Character design (P15)\\ 3. Element (P9, P16, P17)\end{tabular}}                                                                                                                              & \multirow{4}{*}{\begin{tabular}[c]{@{}l@{}}1. Character name (P4)\\ 2. Character biography (P9, P15, P17)\\ 3. Character relationship (P16, P17)\end{tabular}}                                                                                                         \\
                                         &                                                                                                                                                                                                                                                                           &                                                                                                                                                                                                                                                                        \\
                                         &                                                                                                                                                                                                                                                                           &                                                                                                                                                                                                                                                                        \\
                                         &                                                                                                                                                                                                                                                                           &                                                                                                                                                                                                                                                                        \\ \hline
\multirow{8}{*}{Story Structure \& Plot} & \multirow{8}{*}{\begin{tabular}[c]{@{}l@{}}1. Genre (P11)\\ 2. Basic plot (P2, P6, P8, P9, P15)\\ 3. Character design (P4, P6, P16, P17)\\ 4. Synopsis (P1, P12, P13, P16, P17)\\ 5. Scene design (P16)\\ 6. Scene reference image (P16)\\ 7. Element (P16)\end{tabular}} & \multirow{8}{*}{\begin{tabular}[c]{@{}l@{}}1. Expand plot details (P1, P4, P9, P11, P12, P13)\\ 2. Plot continuation (P9, P16)\\ 3. Multiple plot possibilities (P1, P2, P6, P8, P9, P15, P17)\\ 4. Multiple scene possibilities (P16)\end{tabular}}                   \\
                                         &                                                                                                                                                                                                                                                                           &                                                                                                                                                                                                                                                                        \\
                                         &                                                                                                                                                                                                                                                                           &                                                                                                                                                                                                                                                                        \\
                                         &                                                                                                                                                                                                                                                                           &                                                                                                                                                                                                                                                                        \\
                                         &                                                                                                                                                                                                                                                                           &                                                                                                                                                                                                                                                                        \\
                                         &                                                                                                                                                                                                                                                                           &                                                                                                                                                                                                                                                                        \\
                                         &                                                                                                                                                                                                                                                                           &                                                                                                                                                                                                                                                                        \\
                                         &                                                                                                                                                                                                                                                                           &                                                                                                                                                                                                                                                                        \\ \hline
\multirow{6}{*}{Dialogue}                & \multirow{6}{*}{\begin{tabular}[c]{@{}l@{}}1. Genre (P11)\\ 2. Character design (P4, P8, P9)\\ 3. Dialogue scenarios (P1, P4)\\ 4. Synopsis (P1)\\ (No details were mentioned by P15)\end{tabular}}                                                                       & \multirow{6}{*}{\begin{tabular}[c]{@{}l@{}}1. Basic dialogue (P4, P8, P9)\\ 2. Stylized dialogue (P11)\\ 3. Continue writing dialogue (P1, P9)\\ 4. Generate complete dialogue (P1, P15)\end{tabular}}                                                                 \\
                                         &                                                                                                                                                                                                                                                                           &                                                                                                                                                                                                                                                                        \\
                                         &                                                                                                                                                                                                                                                                           &                                                                                                                                                                                                                                                                        \\
                                         &                                                                                                                                                                                                                                                                           &                                                                                                                                                                                                                                                                        \\
                                         &                                                                                                                                                                                                                                                                           &                                                                                                                                                                                                                                                                        \\
                                         &                                                                                                                                                                                                                                                                           &                                                                                                                                                                                                                                                                        \\ \hline
\multirow{7}{*}{Screenplay Text}         & \multirow{7}{*}{\begin{tabular}[c]{@{}l@{}}1. Genre (P7, P10, P16, P18)\\ 2. Element (P10, P16, P18)\\ 3. Screenplay text (P12, P16)\\ 4. Synopsis (P4, P13, P14)\\ 5. Character design (P18)\\ (No details were mentioned by P15)\end{tabular}}                          & \multirow{7}{*}{\begin{tabular}[c]{@{}l@{}}1. Translation (P16)\\ 2. Narration (P16)\\ 3. Screenplay text (P4, P7, P10, P13, P14, P15, P18)\\ 4. Format (P12)\end{tabular}}                                                                                            \\
                                         &                                                                                                                                                                                                                                                                           &                                                                                                                                                                                                                                                                        \\
                                         &                                                                                                                                                                                                                                                                           &                                                                                                                                                                                                                                                                        \\
                                         &                                                                                                                                                                                                                                                                           &                                                                                                                                                                                                                                                                        \\
                                         &                                                                                                                                                                                                                                                                           &                                                                                                                                                                                                                                                                        \\
                                         &                                                                                                                                                                                                                                                                           &                                                                                                                                                                                                                                                                        \\
                                         &                                                                                                                                                                                                                                                                           &                                                                                                                                                                                                                                                                        \\ \hline
\end{tabular}
\end{table*}

\section{\textcolor{black}{Findings 2: Attitudes Toward AI Integration}}\label{sec:Attitudes}

%\textcolor{black}{Understanding participants' attitudes toward AI integration in screenwriting highlights its advantages and limitations, categorized into two dimensions: current AI integration, reflecting the perspectives of participants with prior AI experience, and the broader impact, incorporating insights from all participants to provide a comprehensive view of AI's potential.}

%\textcolor{black}{Understanding participants' attitudes toward AI integration in screenwriting reveals its advantages and limitations across two dimensions: current AI integration, based on perspectives of participants with prior AI experience, and broader impact, incorporating insights from all participants for a comprehensive view of AI's potential.}

\textcolor{black}{Understanding participants' attitudes toward AI integration in screenwriting reveals its advantages and limitations across two dimensions: current AI integration, reflecting the perspectives of participants with prior AI experience, and broader impact, incorporating insights from all participants to provide a comprehensive view of AI's potential.}

%\textcolor{black}{Understanding participants' attitudes toward AI integration in screenwriting reveals its advantages and limitations, categorized into two dimensions: current AI integration, based on participants who with prior AI experience perspectives, and the broader impact of AI integration, incorporating insights from all participants to provide a comprehensive perspective on AI's potential.}

%Understanding participants' attitudes toward integrating AI into screenwriting workflows reveals their perceptions of AI's advantages and limitations. This insight helps identify potential strategies for balancing differing attitudes among screenwriters in the future. Participants' attitudes were categorized into two dimensions: attitudes toward current AI capabilities and attitudes toward AI's future development. The section on attitudes toward current AI capabilities reflects the perspectives of participants with prior experience using AI in screenwriting, while the section on attitudes toward AI's future development incorporates the views of all participants, regardless of their prior AI usage, providing a more comprehensive understanding of screenwriters' perspectives on AI's potential.}

\subsection{\textcolor{black}{Attitudes Toward Current AI Integration}}

\textcolor{black}{Participants with prior experience using AI in screenwriting, totaling 18, shared their perspectives on current AI capabilities.} These perspectives were categorized into three aspects: positive, negative, and contradictory. 
Additionally, the workflow stages discussed in this section align with those in Section~\ref{sec:Allocation} and  Tables~\ref{tab:Task Allocation 23} and~\ref{tab:Task Allocation}
% as well as in Table~\ref{tab:Task Allocation 23} and Table~\ref{tab:Task Allocation}.

\subsubsection{\textcolor{black}{Positive}} \label{sec:Current Positive}

\textcolor{black}{Many participants with prior experience using AI shared positive perspectives, which influenced their integration of AI into screenwriting workflows.}

\textcolor{black}{\textbf{Rapid Generation to Reduce Trial-and-Error Costs.}}
The value of AI’s ability to rapidly generate diverse ideas and narrative directions, particularly during the goal \& idea stage, was emphasized by 11 participants. P16 noted that AI’s brainstorming capabilities effectively addressed creative blocks, enabling screenwriters to explore new possibilities and refine initial concepts more efficiently. Similarly, participants (P8, and P15) highlighted AI’s utility in suggesting unconventional and alternative plotlines during the story structure \& plot stage, assisting in experimentation with innovative story forms and fresh perspectives.

\textcolor{black}{\textbf{Efficient Retrieval and Summarization to Expand Knowledge Boundaries.}}
AI's capability to retrieve and summarize diverse, unstructured information, especially during the goal \& idea generation and character development stages, was highlighted by 11 participants. P17 noted that AI tools efficiently reviewed extensive materials and extracted key insights from real-world research, providing valuable background information and allowing screenwriters to focus on refining creative details. P13 emphasized AI's potential for personalized knowledge expansion, stating, ``\textit{AI-generated visuals provide access to spaces difficult to explore in real life, like Mars or the universe... It expands our thinking and helps us beyond our knowledge}.'' P13 also noted that AI delivers more accurate and relevant content than traditional search engines when tailored to the screenplay's context, enhancing its utility.
%P13 also observed that AI, when contextualized to the screenplay, surpasses traditional search engines by delivering more precise and relevant content, enhancing its utility.}
%\textcolor{black}{Ten out of 18 participants recognized AI's ability to retrieve and summarize large datasets, particularly during the goal \& idea generation and character development stages. P17 noted that AI tools streamlined tasks such as reviewing extensive materials and extracting key insights from real-world research, providing valuable background information and enabling screenwriters to focus on refining more creative details. Additionally, P13 highlighted AI’s potential for personalized knowledge expansion, stating, ``\textit{AI-generated visuals provide access to spaces difficult to explore in real life, like Mars or the universe... It expands our thinking and helps us beyond our knowledge}.'' P13 further observed that when AI understands the screenplay context, it outperforms traditional search engines by delivering more precise and contextually relevant content, thereby enhancing its utility.}

\textcolor{black}{\textbf{Visual Generation to Inspire Overlooked Ideas.}}
P11, P14, and P16 also identified AI's ability to create visual imagery as a source of unexpected inspiration during the goal \& idea stage. P11 explained, ``\textit{AI generates concept art from just a few prompts, often revealing overlooked elements that inspire through color and composition},'' and emphasized the importance of focusing on elements beyond the initial prompt, which expands imaginative possibilities. P14 added, ``\textit{AI-generated visuals, such as character expressions or clothing, offer perspectives I would never consider},'' highlighting how visualizations allow screenwriters to preview potential visual effects and uncover overlooked creative opportunities.
%While AI’s capability to generate diverse textual possibilities is well recognized, participants also identified its ability to create visual imagery as a source of unexpected inspiration during the goal \& idea stage (P11, P14, and P16).
%P11 explained, ``\textit{AI generates concept art from just a few prompts, often revealing overlooked elements that inspire through color and composition},'' and emphasized focusing on elements beyond the initial prompt, which expands imaginative possibilities. 


%\paragraph{\textcolor{black}{Enhancing Productivity with AI}}\textcolor{black}{Fourteen of the 18 participants who used AI emphasized its critical role in enhancing productivity by automating time-intensive tasks and generating rapid results across various screenwriting stages (refer to Table~\ref{tab:Task Allocation 23} and~\ref{tab:Task Allocation}). \textbf{Quantity of Idea Generation:} During the goal \& idea stage, P11 and P16 highlighted AI's ability to efficiently generate multiple suggestions. P16 noted, ``\textit{AI can brainstorm many scenarios for me},'' enabling screenwriters to explore diverse narratives and overcome creative blocks. \textbf{Efficient Information Processing:} P2, P4, P15, and P17 emphasized AI’s capabilities in information retrieval and summarization, particularly during character development. Tasks such as generating character names and refining profiles were streamlined. P17 remarked, ``\textit{It helps us review a large amount of material quickly},'' allowing screenwriters to extract insights from research more effectively and focus on complex storytelling decisions.}

%\paragraph{\textcolor{black}{Sparking Creativity through AI}}  \textcolor{black}{Beyond productivity, 15 participants highlighted AI’s role in enhancing creativity as a collaborative tool for inspiration and exploration. \textbf{Narrative Form Exploration:} During the story structure \& plot stage, AI's ability to propose branching storylines and explore ``what-if'' scenarios (P8, P9, P12, and P15) expanded screenwriters’ creative options, enabling experimentation with diverse narrative elements and storytelling possibilities.}

%\paragraph{\textcolor{black}{Sparking Creativity through AI}}  
%\textcolor{black}{Beyond productivity, 15 participants highlighted AI’s role in enhancing creativity as a collaborative tool for inspiration and exploration. \textbf{Novelty of Idea Generation:} During the goal \& idea stage, AI’s ability to generate new ideas helped participants spark creativity, especially when inspiration was lacking. P11 described AI as offering a ``\textit{creative and infinite imaginative space},'' while P16 noted that AI-generated images were even more effective than text in brainstorming, providing intuitive and visually engaging content. These tools supported crafting nuanced characters (P2, P4, P15, and P17). \textbf{Narrative Form Exploration:} During the story structure \& plot stage, AI's ability to propose branching storylines and explore ``what-if'' scenarios (P8, P9, P12, and P15) expanded screenwriters’ creative options, enabling experimentation with diverse narrative elements and storytelling possibilities.}

%Eighteen participants who had previously used AI in screenwriting expressed their attitudes toward certain current AI capabilities. These attitudes were categorized into three aspects: positive, negative, and contradictory.}

%\textcolor{black}{Fourteen of the 18 participants who had used AI highlighted its role in improving productivity by facilitating the rapid generation of results and automating time-intensive tasks. For example, during the goal \& idea stage, P11 and P16 found AI particularly effective in streamlining workflows by \textbf{generating multiple suggestions quickly}. P16 explained, ``\textit{AI can brainstorm many scenarios for me},'' allowing screenwriters to explore diverse narrative possibilities and overcome initial hurdles in idea generation. Participants, including P2, P4, P15, and P17, emphasized that AI's \textbf{information retrieval and summarization capabilities} significantly enhanced creative efficiency by providing rapid access to vast databases. During the character development stage, AI assisted tasks such as generating character names and refining detailed profiles. Specifically, P17 highlighted AI's ability to consolidate and summarize extensive material, noting that it "\textit{helps us review a large amount of material quickly}," which enabled screenwriters to efficiently extract key information from real-world events or research topics. These functionalities allowed screenwriters to dedicate more time and focus to complex creative decisions. Furthermore, during the story structure \& plot stage, participants like P8, P9, P12, and P15 found AI invaluable for structuring narratives and organizing plot elements, reducing the cognitive effort required to create coherent storylines. By automating these tasks, AI tools effectively enhanced productivity across multiple stages of the screenwriting workflow (refer to Table~\ref{tab:Task Allocation 23} and~\ref{tab:Task Allocation}).}

%\textcolor{black}{Fourteen of the 18 participants who had used AI emphasized its pivotal role in enhancing productivity by facilitating the rapid generation of results and automating time-intensive tasks across various stages of the screenwriting workflow (refer to Table~\ref{tab:Task Allocation 23} and~\ref{tab:Task Allocation}). \textbf{Multiple Idea Generation:} During the goal \& idea stage, P11 and P16 highlighted AI’s effectiveness in streamlining workflows by generating multiple suggestions efficiently. P16 noted, ``\textit{AI can brainstorm many scenarios for me},'' enabling screenwriters to explore diverse narrative possibilities and overcome initial creative blocks. \textbf{Efficient Information Processing:} P2, P4, P15, and P17 underscored AI’s capabilities in information retrieval and summarization, which were particularly effective during the character development stage. AI facilitated tasks such as generating character names and refining detailed profiles. Specifically, P17 remarked on AI’s ability to consolidate and summarize large volumes of material, stating, ``\textit{It helps us review a large amount of material quickly},'' allowing screenwriters to extract critical insights from real-world events or research topics more effectively. These capabilities freed screenwriters to focus on higher-order creative decisions and complex storytelling challenges.}


%\paragraph{\textcolor{black}{Sparking Creativity through AI.}} \textcolor{black}{Beyond its productivity benefits, 15 of the 18 participants highlighted AI's significant role in enhancing creativity as a collaborative tool for inspiration and narrative exploration. \textbf{Creative Idea Generation:} During the goal \& idea stage, AI's ability to generate new ideas proved instrumental in sparking participants’ creativity, particularly when initial inspiration was lacking. For instance, P11 emphasized that AI provides a “\textit{creative and infinite imaginative space}.” Similarly, P16 observed that AI-generated images were even more effective than text in facilitating brainstorming. These images not only offered a wide range of ideas but also conveyed them through intuitive and visually engaging content, fostering innovation. Meanwhile, during the character development stage, AI-generated visual suggestions for character traits provided valuable support to screenwriters, enabling them to craft characters with greater depth and nuance (P2, P4, P15, and P17). \textbf{Narrative Form Exploration:} Furthermore, AI was particularly valued during the story structure \& plot stage for its ability to generate alternative branching storylines based on the screenwriter's original ideas and explore "what-if" scenarios (P8, P9, P12, and P15). This functionality provided screenwriters with a broader range of potential options, enabling them to experiment with different combinations of narrative elements and explore more diverse storytelling possibilities.}

%Beyond its productivity benefits, 15 of the 18 participants highlighted AI's role in enhancing creativity as a collaborative tool for idea generation and inspiration. During the goal \& idea stage, participants relied on AI's capability to expand creative possibilities. For instance, P16 described how AI's brainstorming features generated numerous narrative scenarios, helping overcome creative blocks and providing fresh perspectives. This ability to suggest alternative directions and new ideas was instrumental in inspiring participants when initial inspiration was limited. Participants also acknowledged AI's contributions to enhancing creative output during the character development stage. By offering suggestions by AI-generated images for character traits, AI helped screenwriters craft richer and more nuanced characters (P2, P4, P15, and P17). Additionally, its role in the story structure \& plot stage was particularly valued for its capacity to propose alternative plotlines or explore ``what-if'' scenarios (P8, P9, P12, and P15). This enabled screenwriters to experiment with unconventional narrative forms and approaches, significantly improving their creative potential.

\begin{comment}
    
\textcolor{black}{Fourteen out of 18 participants who had used AI reported that it \textbf{facilitates rapid generation of results and expands possibilities across textual and visual forms}. For instance, P11 and P16 found AI particularly helpful during the goal \& idea stage. P16 argued, ``\textit{AI can brainstorm many scenarios for me},'' assisting in moments when initial inspiration was lacking. P2, P4, P15, and P17 noted its utility in the character development stage, such as generating names or refining detailed character profiles. Similarly, P8, P9, P12, and P15 highlighted the advantages of AI during the story structure \& plot stage (refer to Table~\ref{tab:Task Allocation 23} and~\ref{tab:Task Allocation}).}

\textcolor{black}{Moreover, 12 of the 18 participants recognized AI’s ability to \textbf{assist with information retrieval, organization, and summarization, as well as text polishing}. For example, P17 mentioned that AI's access to vast databases helps fill knowledge gaps and supports rapid information processing: ``\textit{It quickly helps us review a large amount of material},'' which was particularly beneficial for the goal \& idea stage.}
\end{comment}

\subsubsection{\textcolor{black}{Negative}} \label{sec:capabilities}

\textcolor{black}{All participants with prior AI experience expressed concerns about its capabilities, which impacted their use of AI in screenwriting workflows.}

% \textcolor{black}{\textbf{High Barriers Reduce Usage Willingness.}}  
%\haotian{\textbf{Limited Knowledge and Skills of AI Technologies.}}  
%\textcolor{red}{\textbf{Limited Knowledge and Skills of Screenwriters to Reduce Their Willingness to Use AI}}
\textcolor{black}{\textbf{Limited Knowledge and Skills of Screenwriters Reducing Their Willingness to Use AI.}}
\textcolor{black}{P1, P4, P8, P17, and P18 identified high barriers to effectively using AI due to their limited training, aligning with challenges noted by N3 and N4 in Section~\ref{sec:Allocation}. P17 observed, ``\textit{We lack a systematic method to train AI. It may have the capability we need, but we fail to articulate our requirements clearly. It is less about training AI and more about training ourselves}.'' These challenges reflect participants’ limited knowledge and skills in leveraging AI technologies effectively.}

% \textcolor{black}{\textbf{Inaccuracy and Uncontrollability Limit Usage Purposes.}} 
%\haotian{\textbf{Inaccuracy and Uncontrollability of AI Models.}}  
%\textcolor{red}{\textbf{Inaccuracy and Uncontrollability of AI Models to Limit Usage in Complex Narrative Tasks.}}
\textcolor{black}{\textbf{Inaccuracy and Uncontrollability Limiting AI Models' Usage in Complex Narrative Tasks.}}
\textcolor{black}{P1, P3, P8, P9, P10, P11, P16, P17, and P18 cited AI’s inaccuracy and lack of control as major challenges, echoing concerns raised by N1, N2, and N5 in Section~\ref{sec:Allocation}. P8, P10, and P18 avoided using AI for critical tasks like complex story development, fearing misinformation. P10 noted, ``\textit{When analyzing Kubrick’s A Clockwork Orange, AI misrepresented his style and falsely attributed films by other directors to him}.'' These concerns highlight mistrust in AI’s reliability for tasks requiring accuracy and factual consistency.}

% \textcolor{black}{\textbf{Limited Understanding Constrains Emotional Depth.}}  
%\haotian{\textbf{Insufficient Emotional Depth of AI-generated Content.}}  
%\textcolor{red}{\textbf{Superficial Understanding of the Real World Reducing Emotional Depth.}}
%\textcolor{black}{\textbf{Lack of Authenticate Experiences Hindering Emotion Perception}}

\textcolor{black}{\textbf{Lack of Authentic Experiences Hindering AI Models' Emotion Perception.}}
\textcolor{black}{AI lacks the ability to replicate authentic experiences, which are deeply tied to personal emotions and life events, as noted by 12 participants.} P2 explained, ``\textit{Creativity stems from my emotions—what makes me deeply pained or joyful.}'' P4, P7, and P8 avoided AI for tasks demanding deep emotional resonance, such as pivotal scenes. P7 observed, ``\textit{AI captures simple emotions, like a father’s love depicted as both smiling happily. But love is far more complex than that.}'' Participants highlighted that human creativity integrates nuanced, dynamic emotional flows across dialogue, behavior, and context, whereas AI relies on simplistic, discrete representations, limiting its application in screenwriting.

\begin{comment}

Twelve participants noted that AI lacks the ability to replicate authentic experiences, which are deeply tied to personal emotions and life events. 
\textcolor{black}{\textbf{High Barriers Reduce Usage Willingness.}}
\textcolor{black}{Five out of 18 participants identified high barriers to using AI effectively (P1, P4, P8, P17, P18). This aligns with reasons cited by N3 and N4 in Section~\ref{sec:Allocation}, who avoided using AI due to similar challenges. P17 explained, ``\textit{We lack a systematic method to train AI. Perhaps it has the ability to provide what we need, but we fail to articulate our requirements clearly. It is not necessarily about training AI but about training ourselves}.'' P18 added, ``\textit{A significant amount of effort is needed to understand its algorithms, which is quite exhausting. Ideally, AI should intuitively understand human intentions regardless of the format used}.'' These perspectives also highlight participants' lack of knowledge and skills in effectively utilizing AI technologies.}

\textcolor{black}{\textbf{Inaccuracy and Uncontrollability Limit Usage Purposes.}}
\textcolor{black}{Nine of 18 participants cited the inaccuracy and uncontrollability of AI-generated content as a limitation to its practical use, echoing concerns raised by N1, N2, and N5 in Section~\ref{sec:Allocation}. Participants (P8, P10, P18) avoided using AI for critical tasks like complex story development due to fears of misinformation. P10 remarked, ``\textit{AI can confidently produce false information. For instance, when analyzing Kubrick’s A Clockwork Orange, it misrepresented Kubrick’s style and falsely attributed other films to him}.'' This highlights mistrust in AI’s reliability for tasks requiring factual accuracy.}

\textbf{Limited Understanding Constrains Emotional Depth.}  
Ten participants highlighted AI's inability to replicate authentic experiences, noting that human creativity is deeply tied to personal emotions and life experiences. P2 stated, ``\textit{Creativity stems from my emotions—what makes me deeply pained or joyful.}'' P4, P7, and P8 avoided using AI for tasks requiring deep emotional resonance, such as pivotal emotional scenes. P7 remarked, ``\textit{AI only captures simple emotional expressions, like a father’s love for his daughter, often depicted as both smiling happily. But love is far more complex than that.}'' Participants emphasized that humans interpret emotions with nuanced, dynamic flows integrating dialogue, behavior, and context, while AI relies on simplistic, generic, and discrete emotional representations, limiting its utility for screenwriting.
\end{comment}

%\textcolor{black}{\textbf{Limited Understanding Constrains Emotional Depth.}}  
%\textcolor{black}{Ten participants highlighted AI's lack of authentic experiences, emphasizing that human creativity is deeply rooted in personal life experiences, which current AI cannot replicate. For example, P2 remarked, ``\textit{Creativity stems from my emotions—what makes me deeply pained or joyful.}'' P4, P7, and P8 avoided using AI for tasks requiring deep emotional resonance or complex character development, such as pivotal emotional scenes. P7 observed, ``\textit{AI only captures simple emotional expressions, like a father’s love for his daughter, often depicted as both smiling happily. But love is far more complex than that.}'' They emphasized that humans interpret emotions with nuance, characterized by a dynamic and coherent emotional flow that integrates sensory perceptions from dialogue, behavior, and other contextual elements. In contrast, AI’s dependence on simplistic, generic, and discrete emotional representations presents a significant limitation for screenwriting.}

%\textcolor{black}{Nine of 18 participants cited the inaccuracy and uncontrollability of AI-generated content as a limitation to its practical use. This skepticism echoes concerns raised by N1, N2, and N5 in Section~\ref{sec:Allocation}, who doubted the usability of AI outputs. Participants (P8, P10, P18) avoided using AI for critical tasks like research or complex story development due to fears of misinformation or errors. P10 noted, ``\textit{AI can confidently produce false information. For instance, when analyzing Kubrick’s A Clockwork Orange, it misrepresented Kubrick’s style and falsely attributed films to him}.'' This underscores mistrust in AI’s reliability for tasks requiring factual accuracy.}


%\paragraph{\textcolor{black}{Avoidance of AI for stages requiring detailed adjustments or high accuracy.}}  

%\textcolor{black}{\textbf{High Barriers to Usage:} Participants who found AI challenging to use (P1, P4, P8, P17, P18) avoided tasks requiring detailed input or nuanced outputs, citing the effort needed to train themselves or the AI as outweighing the benefits. This led them to exclude AI from workflow steps such as dialogue generation or crafting emotional nuance. \textbf{Inaccuracy and Uncontrollability:} Nine of 18 participants identified inaccuracy and uncontrollability as major barriers, echoing skepticism from N1, N2, and N5 in Section~\ref{sec:Allocation}. P8, P10, and P18 avoided AI for critical tasks like research or story development due to concerns over misinformation and errors. P10 noted, ``\textit{AI can confidently produce false information. For instance, when analyzing Kubrick’s A Clockwork Orange, it misrepresented Kubrick’s style and falsely attributed films to him.}.'' This underscores mistrust in AI’s reliability for tasks requiring factual accuracy.}  

%\paragraph{\textcolor{black}{Limited adoption of AI for stages emphasizing complex emotional content.}}  

%\textcolor{black}{\textbf{Lack of Emotional Support:} 
%Ten participants emphasized AI's lack of authentic experiences, noting that human creativity is deeply rooted in personal life experiences, which current AI cannot replicate. For example, P2 stated, ``\textit{Creativity stems from my emotions—what makes me deeply pained or joyful.}'' P4, P7, and P8 avoided using AI for tasks requiring deep emotional resonance or complex character development, such as pivotal emotional scenes. P7 remarked, ``\textit{AI only captures simple emotional expressions, like a father’s love for his daughter, often depicted as both smiling happily. But love is far more complex than that.}'' He emphasized that while humans interpret emotions with nuance, AI relies on generic representations.}
%\textcolor{black}{\textbf{Lack of Emotional Support:} P4, P7, and P8 avoided AI for tasks requiring deep emotional resonance or complex character development, such as pivotal emotional scenes. P7 remarked, ``\textit{AI only captures simple emotional expressions, like a father’s love for his daughter, often depicted as both smiling happily. But love is far more complex than that}.'' He emphasized that while humans interpret emotions with nuance, AI relies on generic representations.}

%\textcolor{black}{\textbf{Lack of Originality:} P13, and P14 criticized AI for producing formulaic and repetitive content, restricting its use to tasks where originality was less critical, such as creating rough frameworks or basic scene outlines. P13 observed, ``\textit{You can see similarities between AI-generated content and previous works}.'' This reflects a cautious approach to integrating AI into workflows. \textbf{Lack of Emotional Support:} P4, P7, and P8 avoided AI for tasks requiring deep emotional resonance or complex character development, such as pivotal emotional scenes. P7 remarked, ``\textit{AI only captures simple emotional expressions, like a father’s love for his daughter, often depicted as both smiling happily. But love is far more complex than that}.'' He emphasized that while humans interpret emotions with nuance, AI relies on generic representations.}

%As a result, participants reserved creative and emotionally complex tasks for humans, limiting AI to simpler tasks like idea generation or formatting.}  

\begin{comment}
    
\textcolor{black}{All participants who had previously used AI expressed concerns about its capabilities. These negative views had several implications for their practices and usage of AI in screenwriting workflows.}

\paragraph{\textcolor{black}{Avoidance of AI for stages requiring detailed adjustments or high accuracy.}}

\textcolor{black}{\textbf{High Barriers to Usage:} Participants who found AI challenging to use (P1, P4, P8, P17, P18) avoided tasks requiring detailed input or nuanced outputs. They felt the time and effort needed to train themselves or the AI outweighed the benefits, leading them to exclude AI from workflow steps such as dialogue generation or crafting emotional nuance. \textbf{Inaccuracy and Uncontrollability:} Nine of 18 participants highlighted the inaccuracy and uncontrollability of AI-generated content as major barriers to practical use, reflecting the skepticism noted by N1, N2, and N5 in Section~\ref{sec:Allocation} about AI's usability. Participants (P8, P10, P18) avoided using AI for critical tasks like research or story development due to fears of misinformation and errors. P10 remarked, ``\textit{AI can confidently produce false information. For instance, when analyzing Kubrick’s A Clockwork Orange, it misrepresented Kubrick’s style and falsely attributed films to him. Without thorough research, such inaccuracies can be misleading}.'' This highlights concerns over AI’s reliability in tasks requiring factual accuracy or creative alignment.}

%\textcolor{black}{\textbf{Inaccuracy and Uncontrollability:} Nine of 18 participants cited the inaccuracy and uncontrollability of AI-generated content as barriers to its practical use. This skepticism echoes concerns raised by N1, N2, and N5 in Section~\ref{sec:Allocation}, who doubted the usability of AI outputs. Participants (P8, P10, P18) avoided using AI for critical tasks like research or complex story development due to fears of misinformation or errors. P10 noted, ``\textit{AI can lie. It confidently produces false information. For example, when I analyzed Kubrick’s A Clockwork Orange, it generated content that misrepresented Kubrick’s style and even attributed films to him that he did not create. Without proper research, you could easily be misled}.'' This highlights concerns over AI’s reliability for tasks requiring factual accuracy or deep creative alignment.}

%\textcolor{black}{\textbf{High Barriers to Usage:} Participants who found AI difficult to use (P1, P4, P8, P17, P18) avoided relying on AI for tasks requiring detailed input or nuanced outputs. They felt that the effort needed to train themselves or the AI was not worth the time investment, leading them to exclude AI from specific steps in their workflows, such as dialogue generation or emotional nuance crafting.} \textcolor{black}{\textbf{Inaccuracy and Uncontrollability:} Nine out of 18 participants emphasized the uncontrollability and inaccuracy of AI-generated content, which they felt reduced its practicality. This aligns with the skepticism expressed by N1, N2, and N5 in Section~\ref{sec:Allocation}, who refrained from using AI due to doubts about the usability of AI-generated outputs. Participants (P8, P10, P18) avoided using AI for critical narrative steps like research or complex story development due to concerns about misinformation or errors. P10 remarked, ``\textit{AI can lie. It confidently produces false information. For example, when I analyzed Kubrick’s A Clockwork Orange, it generated content that misrepresented Kubrick’s style and even attributed films to him that he did not create. Without proper research, you could easily be misled}.'' This suggests that AI was seen as unreliable for tasks demanding factual accuracy or deep creative alignment.}

\paragraph{\textcolor{black}{Limited adoption of AI for stages emphasizing originality or complex emotional content.}}

\textcolor{black}{\textbf{Lack of Originality:} Participants (P13, P14) criticized AI for producing formulaic and repetitive content, limiting its use to tasks where originality was not critical, such as creating rough frameworks or basic scene outlines. P13 noted, ``\textit{You can see similarities between AI-generated content and previous works}.'' This cautious adoption reflects screenwriters' selective integration of AI into their workflows. \textbf{Lack of Emotional Support:} Participants (P4, P7, P8) avoided using AI for tasks requiring deep emotional resonance or complex character development, such as pivotal emotional scenes. P7 noted, ``\textit{Current AI cannot understand complex emotions. For example, it cannot grasp the inner struggles of a single person. It only captures simple emotional expressions—like a father’s love for his daughter, often depicted as both smiling happily. But love is far more complex than that}.'' He emphasized that while humans interpret emotions with nuance, AI relies on generic representations. As a result, participants assigned creative and emotionally complex tasks to humans, limiting AI to simpler tasks like idea generation or formatting.}

\end{comment}

%\textcolor{black}{\textbf{Lack of Originality:} Participants (P13, P14) criticized AI for producing formulaic and repetitive content, which led them to restrict AI use to tasks where originality was not essential, such as creating rough frameworks or basic scene outlines. P13 observed, ``\textit{You can see similarities between AI-generated content and previous works}.'' This selective adoption reflects a cautious approach by screenwriters to integrating AI into their workflows. \textbf{Lack of Emotional Support:} Participants (P4, P7, P8) avoided using AI for tasks requiring deep emotional resonance or complex character development, such as writing pivotal emotional scenes. P7 explained, ``\textit{Current AI cannot understand complex emotions. For instance, AI cannot grasp the inner struggles of a single person. It only understands the simplest emotional expressions—like a father’s love for his daughter. If you ask it to depict that love, it would show a father holding his daughter with both smiling happily. But we all know love is far more complex than that}.'' He emphasized that humans interpret and express emotions in nuanced ways, whereas AI relies on generic representations without understanding deeper, subtler expressions. Consequently, participants used AI for simpler tasks like generating ideas or formatting, reserving creative and emotionally complex tasks for human input.}


%All participants who had previously used AI expressed concerns about its capabilities. First, 5 out of 18 participants identified the \textbf{high barriers to using AI effectively} (P1, P4, P8, P17, P18). This aligns with the reasons cited by N3 and N4 in Section~\ref{sec:Allocation}, who avoided using AI due to similar challenges. P17 explained, ``\textit{We lack a systematic method to train AI. Perhaps it has the ability to provide what we need, but we fail to articulate our requirements clearly. It is not necessarily about training AI but about training ourselves}.'' P18 added, ``\textit{A significant amount of effort is needed to understand its algorithms, which is quite exhausting. Ideally, AI should intuitively understand human intentions regardless of the format used}.''}

%\textcolor{black}{Additionally, 9 out of 18 participants emphasized the \textbf{uncontrollability and inaccuracy of AI-generated content, which they felt reduced its practicality}. This aligns with the skepticism expressed by N1, N2, and N5 in Section~\ref{sec:Allocation}, who refrained from using AI due to doubts about the usability of AI-generated outputs. P8 noted, ``\textit{AI can provide a framework, but filling in the details is extremely difficult}.'' Similarly, P10 remarked, ``\textit{AI can lie. It confidently produces false information. For example, when I analyzed Kubrick’s A Clockwork Orange, it generated content that misrepresented Kubrick’s style and even attributed films to him that he did not create. Without proper research, you could easily be misled}.'' P18 added, ``\textit{I do not want to waste an evening interacting with AI, only to end up with results that fail to meet my aims}.''}

%\textcolor{black}{Moreover, 13 out of 18 participants criticized AI-generated content for its \textbf{lack of originality, describing it as formulaic and repetitive}. P13 observed, ``\textit{You can see similarities between AI-generated content and previous works}.'' P14 added, ``\textit{AI writes in a very formulaic way, like something thoughtlessly created. It lacks practical value and only reflects the most clichéd approaches}.'' This limitation underscores AI's current inability to provide personalized and stylistic expression tailored to the needs of individual screenwriters.}

%\textcolor{black}{Furthermore, 3 out of 18 participants highlighted the \textbf{limitations of current AI in offering emotional support} (\textcolor{black}{P4, P7, P8}). P4 remarked, ``\textit{To stand out among existing film and television works, there must be something that truly resonates with people on a deep emotional level. Current AI lacks the capability to produce such novel or impactful emotional content}.'' P7 explained, ``\textit{Current AI cannot understand complex emotions. For instance, AI cannot grasp the inner struggles of a single person. It only understands the simplest emotional expressions—like a father’s love for his daughter. If you ask it to depict that love, it would show a father holding his daughter with both smiling happily. But we all know love is far more complex than that}.'' He emphasized that humans interpret and express emotions in nuanced ways, whereas AI relies on generic representations without understanding deeper, subtler expressions.}

\begin{comment}
    
\textcolor{black}{All participants who had previously used AI expressed concerns about its capabilities. First, 5 out of 18 participants noted the \textbf{high barriers to using AI effectively} (P1, P4, P8, P17, P18). This aligns with the reasons mentioned by N3 and N4 in Section~\ref{sec:Allocation}, who had not used AI due to similar concerns. P17 explained, ``\textit{We lack a systematic method to train AI. Perhaps it has the ability to provide what we need, but we fail to articulate our requirements clearly. It is not necessarily about training AI but about training ourselves}.'' P18 added, ``\textit{A significant amount of effort is needed to understand its algorithms, which is quite exhausting. Ideally, AI should intuitively understand human intentions regardless of the format used}.''}

\textcolor{black}{Additionally, 9 out of 18 participants highlighted the \textbf{uncontrollability and inaccuracy of AI-generated content, which they felt reduced its efficiency and practicality}. This aligns with the skepticism expressed by N1, N2, and N5 in Section~\ref{sec:Allocation}, who had not used AI due to doubts about the usability of AI-generated content. P8 noted, ``\textit{AI can provide a framework, but filling in the details is extremely difficult}.'' Similarly, P10 stated, ``\textit{AI can lie. It confidently produces false information. For example, when I analyzed Kubrick’s A Clockwork Orange, it generated content that misrepresented Kubrick’s style and even attributed films to him that he did not create. Without proper research, you could easily be misled}.'' P18 also commented, ``\textit{I do not want to waste an entire evening interacting with AI, only to end up with results that fail to meet my aims}.''}

\textcolor{black}{Moreover, 13 out of 18 participants criticized AI-generated content for its \textbf{lack of originality, calling it formulaic and repetitive}. P13 observed, ``\textit{You can see similarities between AI-generated content and previous works}.'' P14 added, ``\textit{AI writes in a very formulaic way like something thoughtlessly created. It lacks practical value and only reflects the most clichéd approaches}.'' This limitation also highlights the current inability of AI to provide personalized and stylistic expression tailored to the needs of different screenwriters.}

\textcolor{black}{Furthermore, 3 out of 18 participants emphasized the \textbf{limitations of current AI in offering emotional support} (\textcolor{black}{P4, P7, P8}). P4 remarked, ``\textit{To stand out among existing film and television works, there must be something that truly resonates with people on a deep emotional level. Current AI lacks the capability to produce such novel or impactful emotional content}.'' P7 stated, ``\textit{Current AI cannot understand complex emotions. For instance, AI cannot grasp the inner struggles of a single person. It only understands the simplest emotional expressions—like a father’s love for his daughter. If you ask it to depict that love, it would show a father holding his daughter with both smiling happily. But we all know love is far more complex than that}.'' He emphasized that humans interpret and express the same emotion in many nuanced ways, whereas AI tends to rely on the most generic forms without understanding deeper, subtler emotional expressions. Additionally, P8 expressed a desire for AI to provide realistic simulations of character interactions, akin to experiencing continuous, authentic emotional engagement with the character. He noted that achieving this level of emotional sophistication in AI would require significant advancements.}

\end{comment}
%\textcolor{black}{All participants who had previously used AI expressed concerns about its capabilities. First, 5 out of 18 participants noted the \textbf{high barriers to using AI effectively} (P1, P4, P8, P17, P18). P17 explained, ``\textit{We lack a systematic method to train AI. Perhaps it has the ability to provide what we need, but we fail to articulate our requirements clearly. It is not necessarily about training AI but about training ourselves}.'' P18 added, ``\textit{A significant amount of effort is needed to understand its algorithms, which is quite exhausting. Ideally, AI should intuitively understand human intentions regardless of the format used}.''}

%\textcolor{black}{Additionally, 9 out of 18 participants highlighted the \textbf{uncontrollability and inaccuracy of AI-generated content, which they felt reduced its efficiency and practicality}. P8 noted, ``\textit{AI can provide a framework, but filling in the details is extremely difficult}.'' Similarly, P10 stated, ``\textit{AI can lie. It confidently produces false information. For example, when I analyzed Kubrick’s A Clockwork Orange, it generated content that misrepresented Kubrick’s style and even attributed films to him that he did not create. Without proper research, you could easily be misled}.'' P18 also commented, ``\textit{I do not want to waste an entire evening interacting with AI, only to end up with results that fail to meet my aims}.''}

%\textcolor{black}{Moreover, 13 out of 18 participants criticized AI-generated content for its \textbf{lack of originality, calling it formulaic and repetitive}. P4 remarked, ``\textit{To stand out among existing film and television works, there must be something that truly resonates with people on a deep emotional level. Current AI lacks the capability to produce such novel or impactful content}.'' P13 observed, ``\textit{You can see similarities between AI-generated content and previous works}.'' P14 added, ``\textit{AI writes in a very formulaic way like something thoughtlessly created. It lacks practical value and only reflects the most clichéd approaches}.'' }


%\subsubsection{\textcolor{black}{Contradictory Views}}

\subsubsection{Contradictory}

\textcolor{black}{``Contradictory'' refers to the varying attitudes expressed by participants toward certain AI capabilities, depending on usage contexts.} From our findings, feedback from 18 participants revealed mixed opinions on structured and pastiche generation, influencing the integration of AI into workflows.
%``Contradictory'' refers to inconsistent evaluations of AI capabilities across screenwriting tasks. Feedback from 18 participants highlighted mixed responses to structured and pastiche generation, impacting AI integration in workflows.}
%\paragraph{\textcolor{black}{Structured Text Generation Capability.}}  
%\textcolor{black}{\textbf{Support for Beginners and Structural Efficiency:} Participants who viewed structured text generation positively used AI for tasks requiring predefined rules, such as screenplay outlines or foundational story structures (P2, P10, P15, P16). AI’s efficiency in generating structured outputs helped screenwriters, particularly beginners, reduce cognitive load and focus on creative aspects (P5). P2 noted that AI could provide a ``\textit{relatively complete screenplay structure},'' allowing users to prioritize nuanced storytelling. \textbf{Avoidance Due to Lack of Depth and Creativity:} Critics of structured text generation, including P1, P12, and P18, avoided AI for dialogue writing or complex narrative development, describing the content as mechanical and overly template-based. These participants reserved such tasks for human effort, valuing creativity over AI's structural precision.}  

%\paragraph{\textcolor{black}{Structured Text Generation Capability.}}  
%\textcolor{black}{\textbf{Enhanced Efficiency for Structural Tasks:} Participants who had a positive view of structured text generation appreciated its utility for stages or tasks requiring predefined rules, such as screenplay outlines or foundational story structures (P2, P10, P15, P16). AI's ability to generate structured outputs efficiently was noted by P2, who described it as providing a ``\textit{relatively complete screenplay structure}.'' P15 further added that for beginners, AI helps reduce cognitive load, enabling screenwriters to focus more on the creative aspects of their work. \textbf{Unsuitable for Complex Narrative Tasks:} Participants who criticized structured text generation (P1, P12, and P18) highlighted its rigidity and lack of creativity, making it unsuitable for dialogue writing or intricate narrative development. They viewed the generated content as overly structured-driven, hindering meaningful plot progression and leaving the output as repetitive structures lacking substantive story development.}  

\textbf{Structured Text Generation Capability.}
Participants argued that AI provides \textit{enhanced efficiency for structural tasks}. P2, P8, P12, P15, and P17 expressed positive views on structured text generation, highlighting its utility for predefined tasks such as screenplay outlines or foundational structures. P12 described that AI can assist us more with format-related tasks. P15 emphasized its value for beginners, as it reduces cognitive load and allows for a greater focus on creative aspects. 
On the other hand, participants also expressed that AI is \textit{unsuitable for complex narrative tasks.} P1, P12, P13, P16, and P18, who were critical of structured text generation, pointed out that it is overly structure-driven, resulting in repetitive frameworks and limited meaningful plot progression. They deemed it unsuitable for dialogue writing or intricate narrative development within the complete screenplay text.
%Participants expressed that AI is \textit{unsuitable for complex narrative tasks}. P1, P12, P13, P16, and P18, who were critical of structured text generation, pointed out that it was overly structure-driven, resulting in repetitive frameworks and limited meaningful plot progression. They deemed it unsuitable for dialogue writing or intricate narrative development within the complete screenplay text.}  

\textbf{Pastiche Text and Image Generation Capability.}
Participants mentioned that AI has been \textit{adopted in speculative and non-realistic genres}. P9, P11, P12, P14, and P17 valued AI's imaginative potential for idea generation in science fiction and surrealism. P14 remarked, ``\textit{AI is good at piecing together different elements and creating connections and narrative forms that differ from human thinking},'' enabling the exploration of unconventional connections and narrative styles well-suited to these genres. On the contrary, participants highlighted their \textit{rejection of AI usage in realistic genres}. P4, P7, P8, P9, P12, P14, P16, and P17 avoided AI-generated pastiche content in logically coherent genres like documentaries. P4 and P16 criticized AI's poor contextual understanding, which caused illogical plot twists and unrelated character behaviors, disrupting the storyline.

%On the contrary, participants highlighted their \textbf{rejection in realistic genres} of AI usage. 

%\paragraph{\textcolor{black}{Pastiche Text and Image Generation Capability.}}  
%\textcolor{black}{\textbf{Rejection in Realistic Genres:} Participants working in genres requiring logical coherence and consistency, such as documentaries (P4, P7, P8, P9, P16, P17), avoided AI-generated pastiche content. P4 and P16 argued AI's lack of contextual understanding, which disrupted workflows by introducing illogical plot twists or unrelated character behaviors, requiring manual corrections (P16). \textbf{Adoption in Speculative and Non-Realistic Genres:} Conversely, participants in science fiction and surrealism embraced AI’s imaginative potential during idea generation (P7, P12, P14). P7 noted that AI-generated images ``\textit{can uncover interesting details human eyes might miss},'' while P14 observed, ``\textit{AI is good at piecing together different elements and creating connections and narrative forms that differ from human thinking, especially in surrealism or postmodern pastiche},'' enabling exploration of unconventional connections and narrative styles beyond traditional approaches.}

%\paragraph{\textcolor{black}{Novelty Content Generation Capability.}}
%\textcolor{black}{\textbf{Novelty of Idea Generation:} During the goal \& idea stage, AI’s ability to generate new ideas helped participants spark creativity, especially when inspiration was lacking. P11 described AI as offering a ``\textit{creative and infinite imaginative space},'' while P16 noted that AI-generated images were even more effective than text in brainstorming, providing intuitive and visually engaging content. These tools supported crafting nuanced characters (P2, P4, P15, and P17). \textbf{Lack of Originality:} However, P13, and P14 criticized AI for producing formulaic and repetitive content, restricting its use to tasks where originality was less critical, such as creating rough frameworks or basic scene outlines. P13 observed, ``\textit{You can see similarities between AI-generated content and previous works}.'' This reflects a cautious approach to integrating AI into workflows.}

%\textcolor{black}{``Contradictory'' refers to instances where participants provided inconsistent evaluations of the same AI capability across different screenwriting tasks. Based on feedback from 18 participants, we identified two current AI capabilities—structured text generation and pastiche text and image generation—that elicited mixed and contradictory responses. These contradictions stem from the architectural design, training methodologies, and probabilistic language modeling of AI systems. The discrepancies in participants' attitudes primarily arise from varying user requirements in different contexts.}

%\textcolor{black}{``Contradictory'' refers to instances where participants offered inconsistent evaluations of the same AI capability across different screenwriting tasks. Based on feedback from 18 participants, we identified two AI capabilities, structured text generation and pastiche text and image generation, that elicited mixed and contradictory responses, directly influenced how participants incorporated or avoided AI in their workflows.}

%These contradictions arise from the architectural design, training methodologies, and probabilistic language modeling of AI systems. Discrepancies in participants' attitudes are primarily driven by varying user requirements across different contexts.}

\begin{comment}

\textcolor{black}{``Contradictory'' refers to inconsistent evaluations of AI capabilities across screenwriting tasks. Feedback from 18 participants highlighted structured and pastiche generation as eliciting mixed responses, affecting AI integration in workflows.}

\paragraph{\textcolor{black}{Structured Text Generation Capability.}}
\textcolor{black}{\textbf{Support for Beginners and Structural Efficiency:} Participants who viewed structured text generation favorably used AI for tasks requiring adherence to predefined rules, such as creating screenplay outlines or foundational story structures (P2, P10, P15, P16). AI’s ability to generate structured outputs efficiently helped screenwriters, particularly beginners, reduce cognitive load and focus on creative aspects (P5). P2 remarked that AI could provide a ``\textit{relatively complete screenplay structure},'' enabling users to prioritize nuanced storytelling. \textbf{Avoidance Due to Lack of Depth and Creativity:} Participants who criticized structured text generation for its rigidity and lack of originality, P1, P12, and P18, avoided using AI for dialogue writing or intricate narrative development. They perceived the content generated as mechanical and overly reliant on templates, which hindered meaningful plot advancement. This led them to reserve these tasks for manual effort, favoring human creativity over AI's structural precision.}

\paragraph{\textcolor{black}{Pastiche Text and Image Generation Capability.}}  
\textcolor{black}{\textbf{Rejection in Realistic Genres:} P4, P7, P8, P9, P16, and P17 working in genres that demand logical coherence and narrative consistency, such as documentaries, refrained from using AI-generated pastiche content. P4 and P16 cited AI's inability to maintain contextual understanding, which disrupted workflows during tasks like story structure development. For example, AI's tendency to introduce illogical plot twists or unrelated character behaviors required participants to interrupt their process and make manual corrections, as described by P16. \textbf{Adoption in Speculative and Non-Realistic Genres:} Conversely, participants creating science fiction and surrealism genres embraced AI's imaginative potential (P7, P12, and P14) in the goal \& idea generation stage. P7 and P14 found AI-generated pastiche content useful for generating unconventional ideas and collage-like imagery that inspired storytelling. P7 commented, AI-generated images ``\textit{can uncover interesting details human eyes might miss},'' P14 added, ``\textit{AI is good at piecing together different elements and creating connections and narrative forms that differ from human thinking, especially in surrealism or postmodern pastiche},'' allowing screenwriters to explore connections and narrative styles beyond traditional human thinking.}  

    
\textbf{\textcolor{black}{Structured Text Generation Capability.}}  
\textcolor{black}{This capability refers to AI's ability to produce coherent, logically organized, and grammatically accurate text that adheres to predefined structural rules or templates. The debate on AI's structured writing capabilities centers on the tension between creativity and structural coherence. While AI excels at producing structured content, its reliance on conventional patterns often results in outputs that lack originality and emotional depth. For instance, AI-generated dialogues are frequently described as overly dependent on traditional writing templates and rigid structures, leading to content that appears mechanical and unnatural, rather than grounded in an understanding of complex logic or emotions. Several participants (\textcolor{black}{P1, P12, and P18}) criticized this limitation. P1 noted, ``\textit{AI tends to overuse conjunctions, and the information density is low},'' resulting in narratives that rigidly adhere to structured writing methods without effectively advancing the plot through substantive and meaningful content. Conversely, some participants (\textcolor{black}{P2, P10, P15, and P16}) emphasized AI's strength in providing a robust structural foundation. Given that screenwriting is inherently structure-driven, participants highlighted that AI’s capabilities in this area help reduce cognitive load. P2 argued, ``\textit{AI can quickly provide users with a relatively complete screenplay structure},'' allowing creators to focus on more nuanced and creative aspects of storytelling. P15 also acknowledged the utility of this feature for beginners, even though he no longer relies on it himself.}

\textbf{\textcolor{black}{Pastiche Text and Image Generation Capability.}}  
\textcolor{black}{This capability refers to AI's ability to produce content by integrating or juxtaposing disparate elements, often resulting in outputs that resemble a patchwork of unrelated or loosely connected components in terms of meaning or context. Opinions on AI's content pastiche capability are often shaped by the requirements of different genres. In realistic genres such as documentaries, where strict logical consistency is essential, participants (\textcolor{black}{P4, P7, P8, P9, P16, and P17}) expressed concerns about AI's limited contextual understanding and coherence. P4 highlighted this limitation in dialogue creation, stating, ``\textit{AI-generated dialogue feels too strange to me, and it does not align with the relationships I’ve established between characters}.'' Similarly, \textcolor{black}{P7 and P9} criticized AI's inability to handle multi-threaded and logically interconnected narratives. P16 remarked, ``\textit{AI might suddenly generate an illogical plot twist, like a good character suddenly becoming evil... this inconsistency forces me to interrupt the generation process and rethink it myself},'' which disrupted her workflow during the story structure \& plot stage. P17 also noted that AI's failure to grasp the logical relationships between stories and characters hindered her work in the same stage. However, in speculative genres like science fiction or surrealism, participants praised AI's imaginative potential. These genres often benefit from creative leaps, where unconventional or disjointed combinations enhance storytelling. Participants (\textcolor{black}{P7, P12, and P14}) emphasized AI's suitability for crafting content in non-realistic genres. P7 commented, ``\textit{AI is currently best at piling up imagery... it may not fully grasp the underlying emotions, but it can uncover interesting details human eyes might miss},'' highlighting how the collage-style images generated by AI have inspired his creative process. P12 highlighted AI's strengths in science fiction, where ``\textit{imaginative styles are crucial}.'' P14 added, ``\textit{AI is good at piecing together different elements and creating connections and narrative forms that differ from human thinking, especially in surrealism or postmodern pastiche}.''}
\end{comment}

\begin{comment}
    
\textcolor{black}{``Contradictory'' refers to instances where participants provided inconsistent evaluations of the same AI capability across different screenwriting tasks. Based on feedback from 18 participants, we identified two current AI capabilities—structured writing and content pastiche—that elicited mixed and contradictory responses. These contradictions stem from the architectural design, training methodologies, and probabilistic language modeling of AI systems. The discrepancies primarily arise from varying user requirements in different contexts, with appreciation or criticism of AI capabilities largely depending on the specific use case.}

\textbf{\textcolor{black}{Contradictions in Structured Writing Capability.}}
\textcolor{black}{Structured writing refers to an LLM's ability to generate text that adheres to logical, grammatical, and rhetorical conventions. This capability is grounded in principles such as sequential language modeling, self-attention mechanisms, pretraining on diverse corpora, context windowing with positional encoding, and fine-tuning for domain-specific tasks. These elements, rooted in the Transformer architecture and probabilistic modeling, enable LLMs to produce text that is both logically coherent and stylistically consistent.} \textcolor{black}{The debate on AI's structured writing capabilities centers on the tension between creativity and structural coherence. While AI excels at producing structured content, its reliance on conventional patterns often results in outputs that lack originality and emotional depth. For instance, AI-generated dialogues are frequently described as mechanical and unnatural due to their dependence on word-to-word associations rather than an understanding of complex logic or emotions. Several participants (\textcolor{black}{P1, P12, and P18}) criticized this limitation, particularly in screenwriting. P1 noted, ``\textit{AI tends to overuse conjunctions, and the information density is low},'' leading to narratives filled with redundant structured content that fail to advance a script’s plot effectively.} \textcolor{black}{Conversely, some participants (\textcolor{black}{P2, P10, P15, and P16}) emphasized AI's strength in providing a robust structural foundation. Given that screenwriting is inherently structure-driven, participants highlighted that AI’s capabilities in this area reduce cognitive load. P2 observed that AI’s ability to handle routine and formulaic tasks ``\textit{can quickly provide users with a relatively complete structure},'' enabling creators to focus on more nuanced and creative aspects of storytelling.}

\textbf{\textcolor{black}{Contradictions in Content Pastiche Capability.}}
\textcolor{black}{AI's ability to create content pastiches in text and images relies on principles such as probabilistic modeling, representation learning, and generative frameworks. Techniques like self-attention, latent space exploration, and multimodal alignment empower AI systems to emulate and synthesize diverse styles across modalities.} \textcolor{black}{Opinions on AI's content pastiche capabilities were divided, often shaped by the requirements of different genres. In realistic genres such as documentaries, where strict logical consistency is essential, participants (\textcolor{black}{P4, P7, P8, P9, P16, and P17}) expressed concerns about AI’s limited contextual understanding and coherence. P4 emphasized how this limitation affects dialogue creation, ``\textit{AI-generated dialogue feels too strange to me, and it does not align with the relationships I’ve established between characters}.'' Additionally, \textcolor{black}{P7 and P9} criticized AI's inability to handle multi-threaded and logically interconnected narratives. P16 remarked, ``\textit{AI might suddenly generate an illogical plot twist like a good character suddenly becoming evil... this inconsistency forces me to interrupt the generation process and rethink it myself},'' which disrupted her workflow during the story structure \& plot stage. Similarly, P17 noted that AI's failure to grasp the logical relationships between stories and characters hindered her workflow during the story structure \& plot stages.} \textcolor{black}{In speculative genres like science fiction or surrealism, however, participants praised AI's imaginative potential. These genres often benefit from creative leaps, where unconventional or disjointed combinations enhance storytelling. Participants (\textcolor{black}{P7, P12, and P14}) highlighted AI's suitability for crafting content in non-realistic genres. P7 stated, ``\textit{AI is currently best at piling up imagery... it may not fully grasp the underlying emotions, but it can uncover interesting details human eyes might miss}.'' P12 noted AI’s strengths in science fiction, where ``\textit{imaginative styles are crucial}.'' P14 added, ``\textit{AI is good at piecing together different elements and creating connections and narrative forms that differ from human thinking, especially in surrealism or postmodern pastiche}.''}

\subsection{\textcolor{black}{Attitudes Toward the Impact of AI Integration}}

\textcolor{black}{When discussing broader attitudes toward the impact of AI integration, all 23 participants shared their perspectives, which were also categorized into three aspects: positive, negative, and contradictory.}


\subsubsection{\textcolor{black}{Positive}} \label{sec:Positive Potential}
\textcolor{black}{Most participants indicated that AI has the potential to act as a collaborator, bringing more positive impacts to the creative process. Nineteen of the 23 participants, regardless of prior AI experience, expressed optimism about \textbf{AI's potential to support divergent thinking workflow stages}. For example, P1, P6, P8, P11, and P13 anticipated its enhanced application in the dialogue and screenplay text stages if AI could develop a deeper understanding of story content and character emotions. P1 specifically desired AI to provide tailored suggestions based on an understanding of plot content to assist with rapid iterative revisions. P8 emphasized AI's potential to generate diverse dialogues by interpreting emotions in specific scenarios, stating, ``\textit{For instance, in the event of a car accident, how would the characters converse? I might envision two or three scenarios based on the setting and characters, but AI could generate a large number of variations instantly. This would significantly improve my efficiency in dialogue creation}.''  Speculative opinions from participants who had not yet used AI also reflected a positive outlook. N1 suggested that if AI could visualize scenes, it could benefit the goal \& idea stage as well as the story structure \& plot stage. N3 proposed that if AI could understand characters' true inner emotions and facilitate interactions with them, it would significantly aid dialogue creation. Based on participants' feedback, AI was perceived as having the potential to positively influence various stages of the creative process, including the goal \& idea stage, story structure \& plot stage, dialogue stage, and screenplay text stage, all of which involve divergent thinking processes.}  

\textcolor{black}{Additionally, 10 out of 23 participants believed AI could enhance efficiency and reduce costs. Eight participants further predicted that AI would reshape screenwriting workflows by \textbf{improving communication across the industry}, reducing the traditionally high communication costs among various stakeholders in the filmmaking process. For instance, P13 noted, ``\textit{When pitching scripts to investors, AI could allow us to present a visual preview of the story instead of asking investors to imagine it, showing them exactly what the story might look like. This is very exciting}.'' Such applications could gain widespread adoption as the industry increasingly recognizes the value of AI in streamlining workflows and enhancing collaboration.}


\subsubsection{\textcolor{black}{Negative}}

\textcolor{black}{Ten out of 23 participants expressed greater trust in human creativity, emphasizing that each \textbf{creator's unique life experiences shape their distinct perspectives and approaches} (\textcolor{black}{P1, P2, P8, P13, P14, and N1}). For instance, P2 highlighted, ``\textit{I feel that creativity stems from my own emotions—what makes me deeply pained or joyful. AI struggles to replicate such strong, personal feelings}.'' Similarly, P8 noted, ``\textit{As screenwriters, it is not just about understanding a character but capturing the fleeting moment of interaction and expressing that personal experience to others}.'' In contrast, AI-generated content lacks the depth of personal experience and expression, which some participants viewed as essential for creative work (\textcolor{black}{P1, P6, P11, and N1}). P6 explained, ``\textit{Screenwriting or filmmaking is a form of self-expression, and AI-generated content diminishes this aspect for creators}.''}

\textcolor{black}{Additionally, three out of 23 participants raised \textbf{ethical concerns} about AI, particularly regarding authorship (\textcolor{black}{N2}) and intellectual property (\textcolor{black}{P8 and P17}). On authorship, N2 remarked, ``\textit{If AI can write a person's story in a brilliant way, then AI is the author, not the user, who becomes merely a transcription tool}.'' Intellectual property was another contentious issue. P8 argued that if AI merely provides a framework or basic ideas that the screenwriter modifies, the screenwriter should retain copyright over the final content. P17 offered a more nuanced perspective, suggesting that iterative refinement of AI-generated content with user input can produce results distinct from the original material. However, P17 acknowledged that some colleagues strongly disagreed, equating such use of AI with plagiarism. These ethical concerns shaped screenwriters' attitudes toward AI, impacting their willingness to adopt AI tools as questions of ownership, originality, and the boundaries of human and machine collaboration remain unresolved.}

\subsubsection{\textcolor{black}{Contradictory}}
\textcolor{black}{Participants' perspectives on AI's social impact on the industry reveal conflicting views about AI as a competitor. On the positive side, ten of the 23 participants believed AI could \textbf{help screenwriters create higher-quality content} by optimizing storytelling approaches and meeting the needs of independent filmmakers in the self-media era. P13 likened AI to an ``\textit{electronic catfish},'' suggesting it could stimulate healthy competition in the industry: ``\textit{AI provides fresh blood to the industry, encouraging humans to create better stories.}'' Participants noted that such a scenario would become more likely as AI advances in generating higher-quality and more original content. Similarly, P18 emphasized the role of AI advancements, stating, ``\textit{Whether it is AI for text or image techniques, they are continuously advancing in the screenwriting industry,}'' describing this as part of societal progress. Conversely, P4, P7, N2, and N5 expressed concerns about AI as a competitor, fearing it could \textbf{weaken screenwriters' creative abilities and even replace them entirely}. P7 cautioned that delegating simpler tasks to AI might diminish the human capacity for complex work over time, warning, ``\textit{If you delegate simple tasks to AI, you may lose the ability to handle complex tasks over time.}'' As AI technology continues to evolve, addressing these concerns will be essential to ensure a balanced integration into creative workflows.}  

\end{comment}

\subsection{\textcolor{black}{Attitudes Toward the Impact of AI Integration}}  

%\textcolor{black}{Participants' attitudes toward AI integration in screenwriting were also categorized into three aspects: positive, negative, and contradictory.}  
\textcolor{black}{When discussing broader attitudes toward the impact of AI integration, all 23 participants shared their perspectives, which were also categorized into three aspects: positive, negative, and contradictory.}

\subsubsection{\textcolor{black}{Positive}} \label{sec:Positive Potential} 

\textcolor{black}{Participants anticipated increased human-AI collaboration across various stages of screenwriting workflows and the broader development of the film industry.}

\textcolor{black}{\textbf{AI’s Potential to Enhance All Stages that Require Divergent Thinking}.}
Most participants regarded AI as a creative collaborator. P8 noted its potential to generate diverse dialogues efficiently by understanding scenarios, particularly for dialogue and screenplay text stages. Even participants without AI experience expressed optimism, suggesting AI could assist in scene visualization (N1) and character interactions (N3) for dialogue creation. \textcolor{black}{AI was also perceived as enhancing goal and idea development, as well as story structure and plot creation, both driven by divergent thinking.}
%\textcolor{black}{Most participants viewed AI as a creative collaborator. P8 highlighted its potential to efficiently generate diverse dialogues by interpreting scenarios, particularly in dialogue and screenplay text generation. Even participants without prior AI experience expressed optimism, suggesting AI could aid in scene visualization (N1) and character interactions for dialogue creation (N3). Additionally, AI was seen as positively impacting goal \& idea development, as well as story structure \& plot creation, both rooted in divergent thinking approaches.}
%\textcolor{black}{Most participants regarded AI as a collaborator that enhances creativity, particularly in dialogue and screenplay text generation. P8 noted that AI has the potential to efficiently generate diverse dialogues by effectively interpreting scenarios. Even participants without prior AI experience expressed optimism, suggesting that AI could assist in scene visualization (N1) or facilitate character interactions to support dialogue creation (N3). Additionally, AI was also perceived as positively influencing other stages, such as goal \& idea development, and story structure \& plot creation, which are primarily rooted in divergent thinking approaches.}

\textcolor{black}{\textbf{AI’s Potential to Improve Communication among Stakeholders}.}
\textcolor{black}{P4, P7, P8, P13, P14, and P15 highlighted AI's potential to improve efficiency and reduce costs by facilitating better stakeholder communication. For instance, P13 suggested that AI-generated visual previews could be used to pitch screenplays or outlines to investors and directors. Such features were seen as tools to streamline workflows and foster collaboration as the industry increasingly adopts AI innovations.}


\begin{comment}
    
\textcolor{black}{Participants viewed AI as fostering collaboration across various stages of screenwriting workflows and the development of the film industry.}

\paragraph{\textcolor{black}{Optimizing Human-AI Collaboration.}}
\textcolor{black}{Most participants viewed AI as a collaborator that enhances the creative process. Nineteen participants highlighted \textbf{AI’s potential to support all divergent thinking}, especially in dialogue and screenplay text generation. P8 noted that AI could efficiently generate diverse dialogues by interpreting scenarios. Participants without prior AI experience also expressed optimism, suggesting AI could assist in scene visualization (N1) or enable character interactions to aid dialogue creation (N3). Overall, AI was seen as positively influencing stages such as goal \& idea development, story structure \& plot creation, dialogue, and screenplay text generation.}

\paragraph{\textcolor{black}{Improving Efficiency and Reducing Costs.}}  
\textcolor{black}{Additionally, 10 participants believed AI could enhance efficiency and reduce costs. Eight noted \textbf{AI's potential to improve industry communication}, with applications like visual previews for pitching scripts to investors (P13). Such features could streamline workflows and promote collaboration as the industry adopts AI innovations.}  

\end{comment}

\subsubsection{\textcolor{black}{Negative}}  \label{sec:ethical concerns}  

\textcolor{black}{Participants expressed concerns about ethical issues surrounding AI, which influenced their attitudes toward its impact. Three participants specifically raised concerns about \textbf{authorship and copyright}. N2 questioned whether AI should be acknowledged as the true author, while P8 and P17 discussed copyright implications. P8 contended that screenwriters should retain copyright when AI serves merely as a framework provider. P17 suggested that iterative refinement with user input could produce distinct outcomes. However, P17 also noted that some colleagues equated such processes with plagiarism. These unresolved issues of ownership, originality, and collaboration contributed to participants' hesitancy to fully adopt AI tools.}

%\textcolor{black}{Participants expressed concerns about AI's limitations in ethical issues, which influenced their attitude toward AI's impact.}

%\paragraph{\textcolor{black}{Concerns About AI's Ethical Implications.}}  
%\textcolor{black}{Three participants raised concerns about \textbf{authorship and intellectual property rights}. N2 questioned whether AI should be recognized as the true author, while P8 and P17 explored the implications for copyright. P8 argued that screenwriters should retain copyright if AI only provides a framework, whereas P17 suggested that iterative refinement with user input could lead to distinct outcomes. However, P17 also noted that some colleagues equated this process with plagiarism. These unresolved issues surrounding ownership, originality, and collaboration contributed to participants' reluctance to fully adopt AI tools.}

%\paragraph{\textcolor{black}{Concerns About AI's Emotional Depth.}}  \textcolor{black}{Ten participants emphasized AI's \textbf{lack of authentic experiences}, highlighting that human creativity is deeply rooted in personal life experiences, which AI cannot replicate. For instance, P2 remarked, ``\textit{Creativity stems from my emotions—what makes me deeply pained or joyful.}'' Similarly, P8 argued that AI lacks the depth of personal expression essential for meaningful creative work. Some participants, such as P6, perceived AI-generated content as diminishing the role of self-expression in the creative process.} 
%\paragraph{\textcolor{black}{AI as a Creative Collaborator}}
%\paragraph{\textcolor{black}{Limited understanding of emotional expression.}}  
%\paragraph{\textcolor{black}{Concerns over AI's Ethical concerns.}}
%\textcolor{black}{Three participants raised \textbf{authorship and intellectual property rights}. N2 questioned whether AI should be recognized as the true author, while P8 and P17 explored copyright implications. P8 argued that screenwriters should retain copyright if AI only provides a framework, whereas P17 suggested that iterative refinement with user input could lead to distinct outcomes. However, P17 also mentioned that some colleagues equated this process with plagiarism. These unresolved issues surrounding ownership, originality, and collaboration contributed to participants' hesitance in adopting AI tools.}


%\textcolor{black}{Most participants viewed AI as a collaborator that enhances the creative process. Nineteen participants highlighted \textbf{AI’s potential to support divergent thinking workflow stages}, especially in dialogue and screenplay text generation. P8 noted that AI could efficiently generate diverse dialogues by interpreting scenarios. Participants without prior AI experience also expressed speculative optimism, suggesting AI could assist in scene visualization (N1) or enable character interactions to aid dialogue creation (N3). Overall, AI was seen as positively influencing stages such as goal \& idea development, story structure \& plot creation, dialogue, and screenplay text generation.}

%\textcolor{black}{Most participants viewed AI as a collaborator that could enhance the creative process. Nineteen participants expressed optimism about \textbf{AI’s potential to support divergent thinking workflow stages}, particularly in dialogue and screenplay text stages. For instance, P8 emphasized AI's capacity to generate diverse dialogues by interpreting scenarios, significantly improving efficiency. Participants without AI experience also offered speculative optimism, suggesting AI could aid scene visualization (N1) or facilitate interactions with characters to assist dialogue creation (N3). AI was seen as potentially influencing goal \& idea stages, story structure \& plot stages, dialogue, and screenplay text stages positively.}  

%\textcolor{black}{Ten participants emphasized \textbf{human creativity's foundation in personal life experiences}, which AI cannot replicate. For example, P2 noted, ``\textit{Creativity stems from my emotions—what makes me deeply pained or joyful.}'' Similarly, P8 argued that AI lacks the depth of personal expression essential for creative work. Some participants viewed AI-generated content as diminishing self-expression (P6).}  

%\textcolor{black}{Three participants highlighted ethical considerations related to \textbf{authorship and intellectual property rights}. N2 questioned whether AI could be considered the true author, while P8 and P17 discussed copyright issues. P8 argued that screenwriters should retain copyright if AI provides a framework, whereas P17 noted iterative refinement with user input could produce distinct results. However, P17 mentioned that some colleagues equated this process with plagiarism. These concerns shaped participants’ hesitance toward AI adoption due to unresolved issues around ownership, originality, and collaboration.}

%\textcolor{black}{Three participants raised \textbf{ethical concerns} regarding authorship and intellectual property. N2 questioned whether AI could be considered the true author, while P8 and P17 discussed copyright issues. P8 argued that screenwriters should retain copyright if AI only provides a framework, whereas P17 noted iterative refinement with user input could produce distinct results. However, some colleagues equated this process with plagiarism. These concerns shaped participants’ hesitance toward AI adoption due to unresolved issues around ownership, originality, and human-machine collaboration.}  


\begin{table*}

\caption{Summary of the Four Roles of AI in Screenwriters' Expectations}
\Description{Description for Table 4:
This Table provides a summary of the four roles of AI as expected by screenwriters. It consists of two columns: the role and its definition. Each role highlights how AI is perceived to assist screenwriters in different capacities:
1. Actor: AI is expected to enhance creative abilities by embodying characters and assisting screenwriters in expanding their knowledge boundaries.
2. Audience: AI aids screenwriters by evaluating the value of their creations from the perspective of diverse audience groups, thereby increasing acceptance and recognition of their work.
3. Expert: AI is seen as an authoritative guide, offering evaluations and suggestions while introducing new workflows based on professional expertise.
4. Executor: AI fulfills creative tasks according to specified demands, focusing on improving work efficiency.}
\label{tab:role}

\begin{tabular}{|p{1.5cm}|p{15cm}|}
     \hline
\textbf{Role}     & \textbf{Definition}                                                                                                                                          \\ \hline
Actor    & Enhancing creative abilities by embodying characters and assisting screenwriters in expanding knowledge boundaries.                           \\ \hline
Audience & Assisting screenwriters in evaluating the value of their creations from the perspective of diverse audience groups, increasing the acceptance and recognition of their work. \\ \hline
Expert   & Offering authoritative evaluations and suggestions, guiding new workflows through professional expertise.                                           \\ \hline
Executor & Fulfilling creative tasks according to specified demands, improving work efficiency.                                                                \\ \hline
\end{tabular}
\end{table*}


\subsubsection{\textcolor{black}{Contradictory}}  \label{sec:Contradictory AI as a competitor}
\textcolor{black}{Participants expressed conflicting views on \textbf{AI as a competitor.}
Those with positive opinions believed that AI has the potential to \textit{\textcolor{black}{enhance storytelling quality.}} \textcolor{black}{P4, P7, P13, P14, and P18 believed AI could stimulate competition and contribute to enhanced content quality, with P13 likening it to an ``\textit{electronic catfish}'' that encourages better storytelling. P18 highlighted that continuous advancements in AI techniques signify societal progress.} However, participants also thought there was a possibility of AI causing screenwriters to \textit{\textcolor{black}{face job displacement risks.}} \textcolor{black}{P1, P4, P6, P7, P13, and P18 highlighted concerns about job displacement due to AI. P7 specifically expressed fears about the potential erosion of professional skills, which could escalate into unemployment risks as AI continues to advance. P7 noted, ``\textit{If you delegate simple tasks to AI, you may lose the ability to handle complex tasks over time.}''}} 


%Participants expressed AI have the potential to \textit{\textcolor{black}{enhance storytelling quality.}} \textcolor{black}{P4, P7, P13, P14, and P18 believed AI could stimulate competition and contribute to enhanced content quality, with P13 likening it to an ``\textit{electronic catfish}'' that encourages better storytelling. P18 highlighted that continuous advancements in AI techniques signify societal progress.} 


%However, participants also thought there would have the possibility of AI to make screenwriters\textit{\textcolor{black}{face job displacement risks.}} \textcolor{black}{P1, P4, P6, P7, P13, and P18 highlighted concerns about job displacement due to AI. P7 specifically expressed fears about the potential erosion of professional skills, which could escalate into unemployment risks as AI continues to advance. P7 noted, ``\textit{If you delegate simple tasks to AI, you may lose the ability to handle complex tasks over time.}''}

%\textcolor{red}{P1, P4, P6, P7, P13, and P18 raised fears of job displacement due to AI. P7 remarked, ``\textit{If you delegate simple tasks to AI, you may lose the ability to handle complex tasks over time}.'' P18 expressed that life is inherently exhausting, requiring constant effort and competition, and stopping this effort might even lead to being surpassed by AI.}

%\textcolor{black}{Two participants mentioned the displacement AI may cause, P7 expressed concerns about the potential erosion of professional skills, which could escalate into an unemployment risk as AI continues to develop. P7 explained, ``\textit{If you delegate simple tasks to AI, you may lose the ability to handle complex tasks over time}.''}

%\textcolor{black}{Participants expressed conflicting views on AI as a competitor. On the positive side, 10 participants believed AI could stimulate competition and \textbf{improve content quality}, likening it to an ``\textit{electronic catfish}'' that encourages better storytelling (P13). P18 highlighted continuous advancements in AI techniques as part of societal progress. Conversely, participants like P7 expressed concerns that reliance on AI might weaken human creativity, warning that delegating simpler tasks could diminish the capacity for complex work. Balancing these perspectives will be crucial as AI integrates into creative workflows.}

%P17 recalled a debate with colleagues, stating, ``\textit{My colleague argued, 'Even if you adapt AI-generated content repeatedly, it remains derivative.' I countered, 'By that logic, if you incorporate a memorable scene from a novel you read years ago into your story without recalling its origin, is that not also plagiarism?'}'' These ethical concerns shaped screenwriters' attitudes toward AI, impacting their willingness to adopt AI tools as questions of ownership, originality, and the boundaries of human and machine collaboration remain unresolved.}

%\textcolor{black}{When discussing attitudes toward the future development and potential of AI, all participants, regardless of their prior AI usage, shared their perspectives. These attitudes were also categorized into three aspects: positive attitudes, negative attitudes, and contradictory views.}

%\textcolor{black}{Nineteen of the 23 participants, regardless of prior AI experience, expressed optimism about AI's potential to \textbf{address various creative needs}. For example, five of the 18 participants who had used AI before anticipated its application in the dialogue and screenplay text stages, while two participants who had not used AI before suggested it could also benefit the goal \& idea stage, story structure \& plot stage, and dialogue stage. P1 expressed a desire for AI to not only generate multiple possibilities quickly but also provide specific suggestions to achieve better results and assist in rapid iterative revisions. P8 further acknowledged AI's potential to generate diverse dialogue patterns for scenarios he provided, ``\textit{For instance, in the event of a car accident, how would the characters converse? I might envision two or three scenarios based on the setting and characters, but AI could generate a large number of variations instantly. This would significantly improve my efficiency in dialogue creation}.'' Based on these participants' feedback, AI's capabilities were perceived as having the potential to positively influence various stages of the creative process, including the goal \& idea stage, story structure \& plot stage, dialogue stage, and screenplay text stage, all of which involve divergent thinking processes.}

%\textcolor{black}{Additionally, six out of 23 participants raised ethical concerns about AI, such as authorship (\textcolor{black}{N2}), intellectual property (\textcolor{black}{P8 and P17}), and the risk of unemployment (\textcolor{black}{P4, P7, N2, and N5}). Regarding authorship, N2 remarked, ``\textit{If AI can write a person's story in a brilliant way, then AI is the author, not the user, who becomes merely a transcription tool}.'' Intellectual property issues were also a point of debate. P8 argued that AI should not own intellectual property if it merely provides a framework or basic ideas, while P17 shared a more nuanced view, stating that iterative refinement of AI-generated content with user input can result in outputs distinct from the original material. However, P17 acknowledged that some colleagues strongly disagreed, considering such use of AI akin to plagiarism. P17 contended, ``\textit{Even if you adapt AI-generated content repeatedly, it remains derivative. By that logic, if you incorporate a memorable scene from a novel into your story years later without recalling its origin, is that not also plagiarism?}'' Furthermore, four participants expressed concerns about AI weakening screenwriters' creative abilities (\textcolor{black}{P4, P7, N2, and N5}) and the fear of being replaced. P7 worried that advancing AI might eventually eliminate screenwriting roles entirely, cautioning, ``\textit{If you delegate simple tasks to AI, you may lose the ability to handle complex tasks over time}.''}



\begin{comment}
Participants’ attitudes toward integrating AI into screenwriting workflows reflected a balance of its perceived advantages and limitations, resulting in both positive and negative perspectives. Additionally, contradictory attitudes were observed in certain cases.

\subsection{Positive Attitudes}

\textcolor{black}{When discussing screenwriters’ perspectives on incorporating AI into their workflows, positive attitudes were generally categorized into two main themes:}

\subsubsection{\textcolor{black}{Satisfied with AI's Current Capabilities}}

\textcolor{black}{Fourteen out of 18 participants who had used AI reported that it facilitates rapid generation of results and expands possibilities across textual and visual forms. For instance, P11 and P16 found AI particularly helpful during the goal \& idea stage, while P2, P4, P15, and P17 noted its utility in the character stage, such as generating names or refining detailed character profiles. Similarly, P8, P9, P12, and P15 highlighted the advantages of AI during the story structure \& plot stage (refer to Table~\ref{tab:Task Allocation 23} and~\ref{tab:Task Allocation}).}

\textcolor{black}{Moreover, 12 of the 18 participants recognized AI’s ability to assist with information retrieval, organization, and summarization, as well as text polishing. For example, P17 mentioned that AI's access to vast databases helps fill knowledge gaps and supports rapid information processing: ``\textit{It quickly helps us review a large amount of material,}'' which was particularly beneficial for the goal \& idea stage.}


\subsubsection{\textcolor{black}{Optimism About AI's Future Potential}}

\textcolor{black}{Nineteen of the 23 participants, regardless of prior AI experience, expressed optimism about AI’s potential to meet various creative needs. For instance, five out of 18 AI users anticipated its application in the dialogue stage and screenplay text stage, while two non-AI users suggested it could also benefit the goal \& idea stage, story structure \& plot stage, and dialogue stage. Based on participants’ feedback, AI’s capabilities were perceived as having the potential to positively influence various stages of the creative process, including the goal \& idea stage, the story structure \& plot stage, the dialogue stage, and the screenplay text stage, all of which involve divergent processes.}

%Based on participants’ feedback, AI’s capabilities were perceived as having the potential to positively impact all stages except the synopsis \& outline stage, which participants viewed as requiring a more convergent, rather than divergent, approach.}

\textcolor{black}{Additionally, 10 out of 23 participants believed AI could enhance efficiency and reduce costs. Eight participants went further, predicting that AI would reshape screenwriting workflows by improving communication across the industry. For example, P13 noted: ``\textit{When pitching scripts to investors, AI could allow us to present a visual preview of the story instead of asking investors to imagine it, showing them exactly what the story might look like. This is very exciting.}''}

\textcolor{black}{Ten of the 18 participants with prior AI experience also demonstrated a deeper understanding of AI potential positive impact on the film industry. They anticipated AI optimizing storytelling approaches, fostering competition to produce higher-quality content, and meeting the needs of independent filmmakers in the self-media era. Specifically, P13 likened AI to an ``\textit{electronic catfish},'' suggesting it could stimulate healthy industry competition: ``\textit{AI provides fresh blood to the industry, encouraging humans to create better stories.}'' Similarly, P18 emphasized that AI advancements signify progress and adaptation: ``\textit{Whether it’s AI for graphics or AI for screenwriting, we are continuously moving forward, replacing some lower-cost labor.}'' This positions AI as both a competitor and a collaborator in the creative process.}

\end{comment}

\begin{comment}
    
Participants' attitudes toward AI integration in screenwriting reflected their views on its current advantages and disadvantages, resulting in both positive and negative perspectives. Furthermore, we identified contradictory attitudes in some cases.

\subsection{Positive Attitudes}

When discussing screenwriters' attitudes toward the introduction of AI technology into their workflow, positive attitudes can be summarized into two categories:

\subsubsection{Acceptance of AI Usage}
Of the 23 participants, 21 (excluding N2 and P5) expressed willingness to incorporate AI into their screenwriting workflow. These participants argued AI could improve efficiency and reduce costs, with 12 stating it saves time and labor, and 7 noting its effectiveness in visualizing ideas to improve communication with other creative personnel. Nine noted AI’s positive impact on information retrieval, and 10 highlighted its role in organizing and refining text to enhance writing quality and flow. Five participants mentioned that chatbot-based character simulations met some screenwriters' needs, and four acknowledged AI-generated images as helpful in content creation. 

%21 out of 23 participants (excluding N2 and P5) indicated their willingness to incorporate AI into their screenwriting workflow. 9 participants acknowledged that AI's information retrieval capabilities have already positively impacted their work. Additionally, 10 participants argued that AI's ability to organize and refine text was widely recognized, with participants noting that AI effectively improves the quality and flow of writing. In terms of role-playing, 5 participants mentioned that simulations conducted through chatbots have already met some screenwriters' needs, and 4 participants agreed that AI-generated image tools have also played a supportive role in content creation. Participants generally believed that AI could enhance efficiency and reduce costs. For example, 12 participants expressed that AI can save time and labor costs in the writing process by quickly generating multiple content options. And 7 participants thought that AI can also effectively improve communication efficiency with other creative personnel by visualizing screenwriters' ideas.

\subsubsection{Optimism About AI Development}

Eight out of 23 participants expressed positive expectations for AI's development. N1, P2, P4, P13, and P17 pointed to AI as an essential tool in their creative process. P13, P14, and P18 argued that AI would foster competition in the screenwriting industry, driving higher-quality stories. Specifically, P13 described AI as an ``\textit{electronic catfish},'' suggesting it would stimulate healthy competition: ``\textit{AI gives our entire industry fresh blood... encouraging humans to create better stories.}'' P18 added that AI advancements signify human progress and adaptation to technological change: ``\textit{Within the entire industry, whether it’s AI for graphics or AI for screenwriting... we are continually moving forward, replacing some of the lower-cost labor.}''

%8 out of 23 participants expressed positive expectations for the development of AI. P11 and P17 believed that since AI lacks emotions, it would not replace humans in the future. N1, P2, P4, P13, and P17 were more confident in future AI technologies and looked forward to AI becoming an essential tool in their creative process. Additionally, P13, P14, and P18 believed that the introduction of AI would foster competition within the screenwriting industry, which would, in turn, drive the creation of higher-quality stories. P13 metaphorically described AI as an ``\textit{electronic catfish},'' suggesting that it would stimulate healthy competition within the industry: ``\textit{AI gives our entire industry fresh blood... encouraging humans to create better stories.}'' P18 further added that AI advancements represent human progress and the continuous adaptation to technological change: ``\textit{Within the entire industry, whether it’s AI for graphics or AI for screenwriting... we are continually moving forward, replacing some of the lower-cost labor.}''



\subsection{Negative Attitudes}

Participants also raised concerns regarding both the usage and challenges AI may encounter. These negative attitudes can also be summarized into two categories:

\subsubsection{Concerns About AI Usage}

Fifteen out of 23 participants expressed concerns about using AI. Seven participants noted that the entry barrier for AI is relatively high and can complicate workflows. Eight participants preferred human creation, emphasizing the reliability of personal experiences and creativity. They stressed the importance of human input in the creative process. P5 stated, ``\textit{I wouldn't use it for creation; I would only use it for improvement,}'' indicating a preference for AI as a supplementary tool rather than the primary driver. P6 expanded on this, saying that his stories and characters require his unique perspective: ``\textit{I try to use my understanding to voice a perspective that shows the world as rich and multidimensional.}'' P14 suggested that while AI could assist with planning, such as generating ideas, it still requires the screenwriter's personal experience to develop them. N2 and P5 stated that, although they do not reject AI in screenwriting, they are not particularly interested in using it themselves. Additionally, some participants expressed concern that AI-generated content might conflict with personal creativity. Specifically, three participants mentioned worries about AI leading to job displacement, fearing it could weaken human creativity or even replace screenwriters.
\end{comment}
%Regarding ethical and moral issues, four participants raised concerns about them. P4 highlighted the restrictive nature of moral standards, stating, ``\textit{Moral standards... could make AI-generated stories overly righteous and lose their appeal. Many compelling stories might be stifled due to moral constraints.}'' These constraints were seen as contributing to homogeneity in AI outputs, reducing diversity and creativity. 

%15 out of 23 participants explicitly expressed concerns about using AI. First, 7 participants pointed out that the entry barrier for using AI is relatively high and can sometimes complicate the workflow. Second, 8 participants preferred human creation, believing that personal experiences and creative abilities are more reliable. They also emphasized the importance of human creativity in the process. P5 stated, ``\textit{I wouldn't use it for creation; I would only use it for improvement.}'' This suggests that P5 prefers to see AI as a supplementary tool rather than as the primary driver of the creative process. P6 further elaborated on this stance, noting that his stories and characters need to be refined and imbued with his unique understanding: ``\textit{I try to use my understanding to voice a perspective that shows the world as rich and multidimensional.}'' P14 suggested that AI could take on the planning aspects of screenwriting, such as generating ideas, but ultimately, it still requires the screenwriter to fill in these ideas with personal experience. N2 and P5 explicitly stated that, while they do not reject AI's involvement in the screenwriting process, they are not particularly interested in using AI themselves. Additionally, some participants mentioned conflicts between AI-generated content and personal creative expression, fearing that the use of AI could weaken or limit human creativity and may eventually replace screenwriters altogether. 

%Regarding ethical and moral issues, 4 participants specifically mentioned the moral constraints and intellectual property issues that may arise from using AI. P1 pointed out that AI could not effectively restrict the content and themes input by users, which might lead to misuse. P4 further emphasized the potential limitations moral constraints could impose on creativity, saying, ``\textit{Moral standards... could make AI-generated stories overly righteous and lose their appeal. Many compelling stories might be stifled due to moral constraints.}'' These moral constraints also led to output homogeneity, lacking diversity and creativity. Moreover, 3 participants argued that the historical trend where technological advancements, such as AI, have led to job displacement.


%\subsubsection{Concerns About AI Capabilities} All participants expressed varying concerns about AI's capabilities. Twenty participants noted that AI lacks subjectivity, leading to content that often lacks innovation and is filled with redundancy and clichés. Meanwhile, AI cannot replicate human life experiences or emotional depth, making it inadequate for personal expression and stylization. It also cannot autonomously generate inspiration or goals, particularly in understanding unstructured information and complex logical relationships, such as contextual reasoning and multi-character interactions. Additionally, 12 participants raised concerns about the accuracy and applicability of AI-generated content, viewing it as uncontrollable and unreliable. This, they argued, reduces the efficiency and practicality of using AI. 

%All participants expressed varying degrees of concern regarding AI's capabilities. First, 20 participants emphasized that AI lacks subjectivity, leading to content generation based on existing datasets that often lack innovation and exhibit high levels of redundancy and clichés. AI cannot possess the life experiences and emotional depth that humans do, making it insufficient in terms of personal expression and stylization. It cannot autonomously generate inspiration or goal, especially when understanding unstructured information and complex logical relationships, such as contextual reasoning clues and multi-character relationships. Additionally, 12 participants mentioned issues with the accuracy and applicability of AI-generated content. They believed that AI-generated content is uncontrollable and not always reliable, requiring significant human intervention and adjustment, thus reducing the efficiency and practicality of using AI. And 4 participants remain uncertain whether AI technology will meet their expectations in the future. 

\begin{comment}
    
\subsection{Contradictory Cases}

\textcolor{black}{``Contradictory cases'' refer to instances where participants provided inconsistent evaluations of the same AI capability across different screenwriting tasks. Based on feedback from all 23 participants, we identified two AI capabilities—structured writing and content pastiche—that elicited mixed and contradictory responses. These capabilities are rooted in AI's architectural design, training methodologies, and probabilistic language modeling. The contradictions primarily arise from differing user requirements in varying contexts, with appreciation or criticism of AI capabilities largely dependent on the specific use case.}

\subsubsection{\textcolor{black}{Contradictions in Structured Writing Capability}}

\textcolor{black}{Structured writing refers to an LLM's ability to generate text that adheres to logical, grammatical, and rhetorical conventions. This capability is grounded in principles such as sequential modeling of language, self-attention mechanisms, pretraining on diverse corpora, context windowing with positional encoding, and fine-tuning for domain-specific tasks. These elements, rooted in the Transformer architecture and probabilistic modeling, enable LLMs to produce text that is both logically coherent and stylistically consistent.}

\textcolor{black}{The debate on AI's structured writing capabilities centers on the tension between creativity and structural coherence. While AI excels at producing structured content, its reliance on conventional patterns often results in outputs that lack originality and emotional depth. For instance, AI-generated dialogues are frequently described as mechanical and unnatural due to their dependence on word-to-word associations rather than an understanding of complex logic or emotions. Several participants (P1, P12, P18, N5) criticized this limitation, particularly in screenwriting. P1 noted, ``\textit{AI tends to overuse conjunctions, and the information density is low,}'' leading to narratives filled with redundant structured content that fail to effectively advance a script's plot.}

\textcolor{black}{Conversely, some participants (P2, P10, P15, P16, N2, N3) emphasized AI's strength in providing a robust structural foundation. Screenwriting, being inherently structure-driven, benefits from AI’s capabilities in this area. P2 observed that AI’s ability to handle routine and formulaic tasks reduces cognitive load, enabling creators to focus on more nuanced and creative aspects of storytelling. Similarly, N3 remarked, ``\textit{The hardest part of writing a script is structure, but that's AI's strong suit.}''}

\subsubsection{\textcolor{black}{Contradictions in Content Pastiche Capability}}

\textcolor{black}{AI's ability to create content pastiches in text and images relies on principles such as probabilistic modeling, representation learning, and generative frameworks. Techniques like self-attention, latent space exploration, and multimodal alignment empower AI systems to emulate and synthesize diverse styles across modalities.}

\textcolor{black}{Opinions on AI's content pastiche capabilities were divided, often shaped by the requirements of different genres. In realistic genres such as documentaries, where strict logical consistency is essential, participants (P8, P16, N2) expressed concerns about AI’s limited contextual understanding and coherence. P16 remarked, ``\textit{AI might suddenly generate an illogical plot twist, like a good character suddenly becoming evil... this inconsistency forces me to interrupt the generation process and rethink it myself.}'' Similarly, N2 commented, ``\textit{AI often finds two related things in the database and puts them together, but it doesn't know why they're connected... this can result in a lack of overall logic in screenwriting.}''}

\textcolor{black}{In speculative genres like science fiction or surrealism, however, participants praised AI’s imaginative potential. These genres often benefit from creative leaps, where unconventional or disjointed combinations can enhance storytelling. Participants P7, P12, and P14 highlighted AI's suitability for crafting content in non-realistic genres. P7 stated, ``\textit{AI is currently best at piling up imagery... it may not fully grasp the underlying emotions, but it can uncover interesting details human eyes might miss.}'' P12 noted AI’s strengths in science fiction, where ``\textit{imaginative}'' styles are crucial. P14 added, ``\textit{AI is good at piecing together different elements and creating connections and narrative forms that differ from human thinking, especially in surrealism or postmodern pastiche.}''}


\subsubsection{\textcolor{black}{Summary}}

\textcolor{black}{Participants, regardless of prior AI experience, expressed contradictory views on these two capabilities, reflecting the nuanced perspectives screenwriters hold. Evaluations of AI's structured writing and content pastiche capabilities largely depend on their context. These inconsistencies highlight the need for greater customization and personalization of AI systems to align with the specific requirements of individual users and use cases.}
\end{comment}

\begin{comment}
    
``Contradictory cases'' refer to instances where participants gave inconsistent evaluations of the same AI feature across different screenwriting tasks. Based on participant feedback, we identified two cases where mixed attitudes emerged, reflecting differing opinions among participants. These complex views on AI’s abilities in structured writing and content pastiche highlight both appreciation and criticism, depending on the specific usage cases.

Regarding AI's structured writing feature, some participants (P1, P12, P18, N5) felt AI's expression was too mechanical and unsuitable for screenwriting. P1 remarked, ``\textit{AI tends to overuse conjunctions, and the information density is low.}'' However, other participants (N2, P1, N3, P2, P10, P15, P16) acknowledged the advantages of AI’s structured writing in screenwriting and other tasks relying on structure. N3 noted, ``\textit{The hardest part of writing a script is structure, but that's AI's strong suit.}'' P2 further explained that AI could handle routine, formulaic output, reducing the mental load on creators for basic tasks.

Similarly, opinions were divided on AI's content pastiche feature. Some participants (P7, P12, P16) found AI helpful in creating non-realistic themed genres. P7 stated, ``\textit{AI is currently best at piling up imagery... it may not fully grasp the underlying emotions, but it can uncover interesting details human eyes might miss.}'' P12 noted AI’s strength in science fiction, where a more ``\textit{imaginative}'' style is needed. P14 argued: ``\textit{AI is good at piecing together different elements and creating connections and narrative forms that differ from human thinking, especially in surrealism or postmodern pastiche.}'' However, some participants (N2, P8, P12, P14, P16) expressed concerns about AI's ability to understand context and generate coherent content. N2 noted, ``\textit{AI often finds two related things in the database and puts them together, but it doesn't know why they're connected... this can result in a lack of overall logic in screenwriting.}'' P16 added, ``\textit{AI might suddenly generate an illogical plot twist, like a good character suddenly becoming evil... this inconsistency forces me to interrupt the generation process and rethink it myself.}''

These contrasting views underscore the nuanced perspectives screenwriters hold toward AI, depending on the specific context. The inconsistencies suggest a need for deeper customization or personalization of AI for individual users in various use cases.
\end{comment}

\begin{comment}
``Contradictory cases''  refer to instances where participants gave inconsistent evaluations of the same AI feature across different screenwriting tasks. Based on participant feedback, we identified two cases where mixed and complex attitudes emerged, reflecting differing opinions among participants. These complex attitudes towards AI’s abilities in structured writing and content pastiche demonstrate both appreciation and criticism, depending on the user's specific cases.

Regarding AI's structured writing feature, some participants (P1, P12, P18, N5) felt that AI's expression tends to be mechanical and unsuitable for screenwriting. P1 remarked that AI often fails to capture the subtle emotions in human dialogue, saying, ``\textit{You can tell it's not written by a human... AI tends to overuse conjunctions, and the information density is low.}'' However, other participants (N2, P1, N3, P2, P10, P15, P16) acknowledged the advantages of AI’s structured writing ability in screenwriting and other writing that heavily relies on structure. 
N3 noted that while AI-written scripts may lack aesthetic appeal, they are structurally sound: ``\textit{The hardest part of writing a script is structure, but that's AI's strong suit.}'' P2 further explained that AI could handle routine, formulaic text output, reducing the mental load on creators for basic writing tasks. 

Similarly, opinions were divided on AI's content pastiche feature. Some participants (P7, P12, P16) felt that AI is particularly helpful for the creation of non-realistic themed genres.
P7 stated, ``\textit{AI is currently best at piling up imagery... while it may not fully grasp the underlying emotions, it can uncover interesting details that human eyes might miss.}'' 
P12 noted that AI excels in specific genres, such as science fiction writing, where a more ``\textit{imaginative}'' style is needed. P14 elaborated on AI’s potential in high-concept creation, arguing that AI can generate creative ideas that may seem unrealistic in realist fiction but are innovative in surrealist or postmodern pastiche styles: ``\textit{AI is good at piecing together different elements, and it can create connections and narrative forms that differ from human thinking, especially in surrealism or postmodern pastiche.}'' 
However, the content pastiche capability of AI led some participants (N2, P8, P12, P14, P16) to express concerns about its ability to understand the context and generate coherent content, which could negatively impact the creative process. 
N2 noted, ``\textit{AI often finds two related things in the database and puts them together, but it doesn't know why they're connected... this can result in a lack of overall logic in screenwriting.}'' 
P16 also mentioned, ``\textit{AI might suddenly generate an illogical plot twist like a good character suddenly becoming evil... this inconsistency forces me to interrupt the generation process and rethink it myself.}''

These contrasting views underscore the nuanced perspectives screenwriters have toward AI, depending on the specific demands of the task at hand. The inconsistencies in participant perspectives suggest the need for a deeper understanding of how AI may require customization or personalization for individual users. 
\end{comment}


\section{\textcolor{black}{Findings 3: Future Expectations}}\label{sec:Expectations}

Participants expressed varied expectations for AI in screenwriting, and we categorized these future expectations into four distinct roles for AI: actor, audience, expert, and executor (refer to Table~ \ref{tab:role}). \textcolor{black}{To further understand how screenwriters' expectations of these roles influence their workflow stages, we also mapped each specific expectation to the stages where participants envisioned these expectations could be applied (refer to Table~ \ref{tab:expectations and stages})}.




% Beamer presentation requires \usepackage{colortbl} instead of \usepackage[table,xcdraw]{xcolor}
\begin{table*}
\caption{\textcolor{black}{Summary of Screenwriters' Expectations for Four Roles and Their Potential Application to Workflow Stages. Blue cells indicate the stages where participants expected these roles to be applicable, denoted as ``E'' in the table. Blank cells represent stages where no application was mentioned. For simplicity, this table uses ``Idea'' to represent ``Goal \& Idea,'' ``Outline'' to represent ``Synopsis \& Outline,'' ``Plot'' to represent ``Story Structure \& Plot,'' and ``Screenplay'' to represent ``Screenplay Text.''}}
\scalebox{0.88}{
\Description{This Table summarizes screenwriters' expectations for four roles of AI (Actor, Audience, Expert, Executor) and their potential application to workflow stages. Blue cells indicate the stages where participants expected these roles to be applicable, denoted as ''E'' in the table. Blank cells represent stages where no application was mentioned:
  "Idea" for Goal \& Idea,
  "Outline" for Synopsis \& Outline,
  "Character" for Character,
  "Plot" for Story Structure \& Plot,
  "Dialogue" for Dialogue,
  "Screenplay" for Screenplay Text.
Role and Workflow Stage Mapping:
1. Actor:
Simulating characters based on multiple requirements:
Modeling internal emotions: Applicable to Idea, Character, Plot and Dialogue stages.
          Simulating external behaviors: Applicable to Idea, Character, Plot, Dialogue and Screenplay stages.
          Representing character environments: Applicable to Idea, Character, Plot, Dialogue and Screenplay stages.
          Supporting multimodal output formats: Applicable to Idea, Character, Plot, Dialogue and Screenplay stages.
Engaging with screenwriters with different methods:
          Internal character: Applicable to Idea, Character and Dialogue stages.
          External character: Applicable to Idea, Character, Plot and Dialogue stages.
          Observer: Applicable to Idea, Character, Plot, Dialogue and Screenplay stages.
2. Audience:
Evaluating as mass audience: Applicable to Idea, Character, Plot and Screenplay stages.
Providing feedback as a specific group of audience: Applicable to Idea stage.
3. Expert:
Providing professional guidance and optimization suggestions: Applicable to all stages.
Promoting new workflows and cultivating filmmakers: Applicable to Idea, Plot and Screenplay stages.
4. Executor:
Managing continuation: 
          Refining details: Applicable to all stages.
          Generating complete screenplays: Applicable to Screenplay stage.
Visualizing Presentation:
          Visualizing plot structures and character relationships: Applicable to Outline, Character, Plot and Screenplay stages.
          Visualizing emotional rhythms: Applicable to Character and Plot stage.
This table provides a comprehensive mapping of AI roles to various stages, showcasing how different stages of screenwriting workflow can benefit from AI involvement.}
\label{tab:expectations and stages}
\footnotesize
\centering


\begin{tabular}{|cll|c|c|c|c|c|c|}
\hline
\multicolumn{3}{|c|}{\textbf{Expectation/Stage}}                                                                                                                                                                  & \textbf{Idea}                                            & \textbf{Outline}                  & \textbf{Character}                                       & \textbf{Plot}                     & \textbf{Dialogue}                 & \textbf{Screenplay}               \\ \hline
\multicolumn{1}{|c|}{}                           & \multicolumn{1}{l|}{}                                                                       & Modeling internal emotions                              & \cellcolor[HTML]{CEDCFF}\makebox[0pt][c]{E}                        &                          & \cellcolor[HTML]{CEDCFF}\makebox[0pt][c]{E}                        & \cellcolor[HTML]{CEDCFF}\makebox[0pt][c]{E} & \cellcolor[HTML]{CEDCFF}\makebox[0pt][c]{E} &                          \\ \cline{3-3}
\multicolumn{1}{|c|}{}                           & \multicolumn{1}{l|}{}                                                                       & Simulating external behaviors                           & \cellcolor[HTML]{CEDCFF}\makebox[0pt][c]{E}                        &                          & \cellcolor[HTML]{CEDCFF}\makebox[0pt][c]{E}                        & \cellcolor[HTML]{CEDCFF}\makebox[0pt][c]{E} & \cellcolor[HTML]{CEDCFF}\makebox[0pt][c]{E} & \cellcolor[HTML]{CEDCFF}\makebox[0pt][c]{E} \\ \cline{3-3}
\multicolumn{1}{|c|}{}                           & \multicolumn{1}{l|}{}                                                                       & Representing character environments                     & \cellcolor[HTML]{CEDCFF}\makebox[0pt][c]{E}                        &                          & \cellcolor[HTML]{CEDCFF}\makebox[0pt][c]{E}                        & \cellcolor[HTML]{CEDCFF}\makebox[0pt][c]{E} & \cellcolor[HTML]{CEDCFF}\makebox[0pt][c]{E} & \cellcolor[HTML]{CEDCFF}\makebox[0pt][c]{E} \\ \cline{3-3}
\multicolumn{1}{|c|}{}                           & \multicolumn{1}{l|}{\multirow{-4}{*}{Simulating characters based on multiple requirements}} & Supporting multimodal output formats                    & \cellcolor[HTML]{CEDCFF}\makebox[0pt][c]{E}                        &                          & \cellcolor[HTML]{CEDCFF}\makebox[0pt][c]{E}                        & \cellcolor[HTML]{CEDCFF}\makebox[0pt][c]{E} & \cellcolor[HTML]{CEDCFF}\makebox[0pt][c]{E} & \cellcolor[HTML]{CEDCFF}\makebox[0pt][c]{E} \\ \cline{2-3}
\multicolumn{1}{|c|}{}                           & \multicolumn{1}{l|}{}                                                                       & Internal character                                      & \cellcolor[HTML]{CEDCFF}\makebox[0pt][c]{E}                        &                          & \cellcolor[HTML]{CEDCFF}\makebox[0pt][c]{E}                        &                          & \cellcolor[HTML]{CEDCFF}\makebox[0pt][c]{E} &                          \\ \cline{3-3}
\multicolumn{1}{|c|}{}                           & \multicolumn{1}{l|}{}                                                                       & External character                                      & \cellcolor[HTML]{CEDCFF}\makebox[0pt][c]{E}                        &                          & \cellcolor[HTML]{CEDCFF}\makebox[0pt][c]{E}                        & \cellcolor[HTML]{CEDCFF}\makebox[0pt][c]{E} & \cellcolor[HTML]{CEDCFF}\makebox[0pt][c]{E} &                          \\ \cline{3-3}
\multicolumn{1}{|c|}{\multirow{-7}{*}{Actor}}    & \multicolumn{1}{l|}{\multirow{-3}{*}{Engaging with screenwriters with different methods}}     & Observer                                                & \cellcolor[HTML]{CEDCFF}\makebox[0pt][c]{E}{\color[HTML]{ADD88D} } &                          & \cellcolor[HTML]{CEDCFF}\makebox[0pt][c]{E}                        & \cellcolor[HTML]{CEDCFF}\makebox[0pt][c]{E} & \cellcolor[HTML]{CEDCFF}\makebox[0pt][c]{E} & \cellcolor[HTML]{CEDCFF}\makebox[0pt][c]{E} \\ \hline
\multicolumn{1}{|c|}{}                           & \multicolumn{2}{l|}{Evaluating as mass audience}                                                                                                      & \cellcolor[HTML]{CEDCFF}\makebox[0pt][c]{E}                        &                          & \cellcolor[HTML]{CEDCFF}\makebox[0pt][c]{E}                        & \cellcolor[HTML]{CEDCFF}\makebox[0pt][c]{E} &                          & \cellcolor[HTML]{CEDCFF}\makebox[0pt][c]{E} \\ \cline{2-3}
\multicolumn{1}{|c|}{\multirow{-2}{*}{Audience}} & \multicolumn{2}{l|}{Providing feedback as a specific group of audience}                                                                               & \cellcolor[HTML]{CEDCFF}\makebox[0pt][c]{E}                        &                          &                                                 &                          &                          &                          \\ \hline
\multicolumn{1}{|c|}{}                           & \multicolumn{2}{l|}{Providing professional guidance and optimization suggestions}                                                                     & \cellcolor[HTML]{CEDCFF}\makebox[0pt][c]{E}                        & \cellcolor[HTML]{CEDCFF}\makebox[0pt][c]{E} & \cellcolor[HTML]{CEDCFF}\makebox[0pt][c]{E}                        & \cellcolor[HTML]{CEDCFF}\makebox[0pt][c]{E} & \cellcolor[HTML]{CEDCFF}\makebox[0pt][c]{E} & \cellcolor[HTML]{CEDCFF}\makebox[0pt][c]{E} \\ \cline{2-3}
\multicolumn{1}{|c|}{\multirow{-2}{*}{Expert}}   & \multicolumn{2}{l|}{Promoting new workflows and cultivating multi-skilled filmmakers}                                                                 & \cellcolor[HTML]{CEDCFF}\makebox[0pt][c]{E}                        &                          &                                                 & \cellcolor[HTML]{CEDCFF}\makebox[0pt][c]{E} &                          & \cellcolor[HTML]{CEDCFF}\makebox[0pt][c]{E} \\ \hline
\multicolumn{1}{|c|}{}                           & \multicolumn{1}{c|}{}                                                                       & Refining details                                        & \cellcolor[HTML]{CEDCFF}\makebox[0pt][c]{E}                        & \cellcolor[HTML]{CEDCFF}\makebox[0pt][c]{E} & \cellcolor[HTML]{CEDCFF}\makebox[0pt][c]{E}                        & \cellcolor[HTML]{CEDCFF}\makebox[0pt][c]{E} & \cellcolor[HTML]{CEDCFF}\makebox[0pt][c]{E} & \cellcolor[HTML]{CEDCFF}\makebox[0pt][c]{E} \\ \cline{3-3}
\multicolumn{1}{|c|}{}                           & \multicolumn{1}{c|}{\multirow{-2}{*}{Managing continuation}}                                & Generating complete screenplays                         &                                                 &                          &                                                 &                          &                          & \cellcolor[HTML]{CEDCFF}\makebox[0pt][c]{E} \\ \cline{2-3}
\multicolumn{1}{|c|}{}                           & \multicolumn{1}{c|}{}                                                                       & Visualizing plot structures and character relationships &                                                 & \cellcolor[HTML]{CEDCFF}\makebox[0pt][c]{E} & \cellcolor[HTML]{CEDCFF}\makebox[0pt][c]{E}                        & \cellcolor[HTML]{CEDCFF}\makebox[0pt][c]{E} &                          & \cellcolor[HTML]{CEDCFF}\makebox[0pt][c]{E} \\ \cline{3-3}
\multicolumn{1}{|c|}{\multirow{-4}{*}{Executor}} & \multicolumn{1}{c|}{\multirow{-2}{*}{Visualizing presentation}}                             & Visualizing emotional rhythms                           &                                                 &                          & \cellcolor[HTML]{CEDCFF}\makebox[0pt][c]{E}{\color[HTML]{ADD88D} } & \cellcolor[HTML]{CEDCFF}\makebox[0pt][c]{E} &                          &                          \\ \hline
\end{tabular}
}
\end{table*}

\begin{comment}
% Please add the following required packages to your document preamble:
% \usepackage{multirow}
% \usepackage[table,xcdraw]{xcolor}
% Beamer presentation requires \usepackage{colortbl} instead of \usepackage[table,xcdraw]{xcolor}
\begin{table}[]
\begin{tabular}{|cll|c|c|c|c|c|c|}
\hline
\multicolumn{3}{|c|}{Expectation/Stage}                                                                                                                                                                  & Idea                                            & Outline                  & Character                                       & Plot                     & Dialogue                 & Screenplay               \\ \hline
\multicolumn{1}{|c|}{}                           & \multicolumn{1}{l|}{}                                                                       & Modeling internal emotions                              & \cellcolor[HTML]{CEDCFF}\makebox[0pt][c]{E}                        &                          & \cellcolor[HTML]{CEDCFF}\makebox[0pt][c]{E}                        & \cellcolor[HTML]{CEDCFF}\makebox[0pt][c]{E} & \cellcolor[HTML]{CEDCFF}\makebox[0pt][c]{E} &                          \\ \cline{3-9} 
\multicolumn{1}{|c|}{}                           & \multicolumn{1}{l|}{}                                                                       & Simulating external behaviors                           & \cellcolor[HTML]{CEDCFF}\makebox[0pt][c]{E}                        &                          & \cellcolor[HTML]{CEDCFF}\makebox[0pt][c]{E}                        & \cellcolor[HTML]{CEDCFF}\makebox[0pt][c]{E} & \cellcolor[HTML]{CEDCFF}\makebox[0pt][c]{E} & \cellcolor[HTML]{CEDCFF}\makebox[0pt][c]{E} \\ \cline{3-9} 
\multicolumn{1}{|c|}{}                           & \multicolumn{1}{l|}{}                                                                       & Representing character environments                     & \cellcolor[HTML]{CEDCFF}\makebox[0pt][c]{E}                        &                          & \cellcolor[HTML]{CEDCFF}\makebox[0pt][c]{E}                        & \cellcolor[HTML]{CEDCFF}\makebox[0pt][c]{E} & \cellcolor[HTML]{CEDCFF}\makebox[0pt][c]{E} & \cellcolor[HTML]{CEDCFF}\makebox[0pt][c]{E} \\ \cline{3-9} 
\multicolumn{1}{|c|}{}                           & \multicolumn{1}{l|}{\multirow{-4}{*}{Simulating characters based on multiple requirements}} & Supporting multimodal output formats                    & \cellcolor[HTML]{CEDCFF}\makebox[0pt][c]{E}                        &                          & \cellcolor[HTML]{CEDCFF}\makebox[0pt][c]{E}                        & \cellcolor[HTML]{CEDCFF}\makebox[0pt][c]{E} & \cellcolor[HTML]{CEDCFF}\makebox[0pt][c]{E} & \cellcolor[HTML]{CEDCFF}\makebox[0pt][c]{E} \\ \cline{2-9} 
\multicolumn{1}{|c|}{}                           & \multicolumn{1}{l|}{}                                                                       & Internal character                                      & \cellcolor[HTML]{CEDCFF}\makebox[0pt][c]{E}                        &                          & \cellcolor[HTML]{CEDCFF}\makebox[0pt][c]{E}                        &                          & \cellcolor[HTML]{CEDCFF}\makebox[0pt][c]{E} &                          \\ \cline{3-9} 
\multicolumn{1}{|c|}{}                           & \multicolumn{1}{l|}{}                                                                       & External character                                      & \cellcolor[HTML]{CEDCFF}\makebox[0pt][c]{E}                        &                          & \cellcolor[HTML]{CEDCFF}\makebox[0pt][c]{E}                        & \cellcolor[HTML]{CEDCFF}\makebox[0pt][c]{E} & \cellcolor[HTML]{CEDCFF}\makebox[0pt][c]{E} &                          \\ \cline{3-9} 
\multicolumn{1}{|c|}{\multirow{-7}{*}{Actor}}    & \multicolumn{1}{l|}{\multirow{-3}{*}{Engaging with screenwriters with different roles}}     & Observer                                                & \cellcolor[HTML]{CEDCFF}\makebox[0pt][c]{E}{\color[HTML]{ADD88D} } &                          & \cellcolor[HTML]{CEDCFF}\makebox[0pt][c]{E}                        & \cellcolor[HTML]{CEDCFF}\makebox[0pt][c]{E} & \cellcolor[HTML]{CEDCFF}\makebox[0pt][c]{E} & \cellcolor[HTML]{CEDCFF}\makebox[0pt][c]{E} \\ \hline
\multicolumn{1}{|c|}{}                           & \multicolumn{2}{l|}{Evaluating as mass audience}                                                                                                      & \cellcolor[HTML]{CEDCFF}\makebox[0pt][c]{E}                        &                          & \cellcolor[HTML]{CEDCFF}\makebox[0pt][c]{E}                        & \cellcolor[HTML]{CEDCFF}\makebox[0pt][c]{E} &                          & \cellcolor[HTML]{CEDCFF}\makebox[0pt][c]{E} \\ \cline{2-9} 
\multicolumn{1}{|c|}{\multirow{-2}{*}{Audience}} & \multicolumn{2}{l|}{Providing feedback as a specific group of audience}                                                                               & \cellcolor[HTML]{CEDCFF}\makebox[0pt][c]{E}                        &                          &                                                 &                          &                          &                          \\ \hline
\multicolumn{1}{|c|}{}                           & \multicolumn{2}{l|}{Providing professional guidance and optimization suggestions}                                                                     & \cellcolor[HTML]{CEDCFF}\makebox[0pt][c]{E}                        & \cellcolor[HTML]{CEDCFF}\makebox[0pt][c]{E} & \cellcolor[HTML]{CEDCFF}\makebox[0pt][c]{E}                        & \cellcolor[HTML]{CEDCFF}\makebox[0pt][c]{E} & \cellcolor[HTML]{CEDCFF}\makebox[0pt][c]{E} & \cellcolor[HTML]{CEDCFF}\makebox[0pt][c]{E} \\ \cline{2-9} 
\multicolumn{1}{|c|}{\multirow{-2}{*}{Expert}}   & \multicolumn{2}{l|}{Promoting new workflows and cultivating multi-skilled filmmakers}                                                                 & \cellcolor[HTML]{CEDCFF}\makebox[0pt][c]{E}                        &                          &                                                 & \cellcolor[HTML]{CEDCFF}\makebox[0pt][c]{E} &                          & \cellcolor[HTML]{CEDCFF}\makebox[0pt][c]{E} \\ \hline
\multicolumn{1}{|c|}{}                           & \multicolumn{1}{c|}{}                                                                       & Refining details                                        & \cellcolor[HTML]{CEDCFF}\makebox[0pt][c]{E}                        & \cellcolor[HTML]{CEDCFF}\makebox[0pt][c]{E} & \cellcolor[HTML]{CEDCFF}\makebox[0pt][c]{E}                        & \cellcolor[HTML]{CEDCFF}\makebox[0pt][c]{E} & \cellcolor[HTML]{CEDCFF}\makebox[0pt][c]{E} & \cellcolor[HTML]{CEDCFF}\makebox[0pt][c]{E} \\ \cline{3-9} 
\multicolumn{1}{|c|}{}                           & \multicolumn{1}{c|}{\multirow{-2}{*}{Managing continuation}}                                & Generating complete screenplays                         &                                                 &                          &                                                 &                          &                          & \cellcolor[HTML]{CEDCFF}\makebox[0pt][c]{E} \\ \cline{2-9} 
\multicolumn{1}{|c|}{}                           & \multicolumn{1}{c|}{}                                                                       & Visualizing plot structures and character relationships &                                                 & \cellcolor[HTML]{CEDCFF}\makebox[0pt][c]{E} & \cellcolor[HTML]{CEDCFF}\makebox[0pt][c]{E}                        & \cellcolor[HTML]{CEDCFF}\makebox[0pt][c]{E} &                          & \cellcolor[HTML]{CEDCFF}\makebox[0pt][c]{E} \\ \cline{3-9} 
\multicolumn{1}{|c|}{\multirow{-4}{*}{Executor}} & \multicolumn{1}{c|}{\multirow{-2}{*}{Visualizing presentation}}                             & Visualizing emotional rhythms                           &                                                 &                          & \cellcolor[HTML]{CEDCFF}\makebox[0pt][c]{E}{\color[HTML]{ADD88D} } & \cellcolor[HTML]{CEDCFF}\makebox[0pt][c]{E} &                          &                          \\ \hline
\end{tabular}
\end{table}
\end{comment}

\subsection{Actor: Enhancing Creative Abilities by Embodying Characters and Assisting Screenwriters in Expanding Knowledge Boundaries}
All 23 participants expressed future expectations for AI-related in the role of an ``actor.'' Traditional screenwriting methods typically rely on the screenwriter's personal experience and relevant materials. However, due to the limitations of their knowledge base, participants sought a reliable source of inspiration to deepen their understanding of character identities and traits. To address these needs, we categorized the screenwriters' expectations of AI as an ``actor'' into two key areas: AI functions and screenwriter engagement. \textcolor{black}{Additionally, as participants perceived the ``actor'' as being related to and aligned with the screenplay's characters, this role was expected to provide more detailed character information, supporting the exploration of the screenplay's goals, along with the characters' motivations and emotions. Participants anticipated that the ``actor'' role would be primarily applied during the goal \& idea, character, and dialogue stages (as shown in Table~\ref{tab:expectations and stages}).}
%\textcolor{black}{Additionally, participants anticipated the ``actor'' role to be applied primarily during the character and goal \& idea stages (as Table~\ref{tab:expectations and stages}) as they perceived the ``actor'' as being related to or aligned with the screenplay's characters, providing more detailed information about characters to support the exploration of the screenplay's goal, as well as the characters' motivations and emotions.}
%Additionally, participants expected the ``actor'' role to be useful primarily during the character, and goal \& idea stages, as shown in Table~\ref{tab:expectations and stages}. This is because they believed the ``actor'' would be related to or aligned with the screenplay's characters and could support emotional depth, aiding in the characters's motivations and emotions.}
%In contrast, they did not expect the ``actor'' to assist in the synopsis \& outline stage, as screenwriters typically rely on foundational world-building, character, and storyline rather than detailed actor emotions and behaviors in this stage. Consistent with the findings in Section 4.3.2, participants reported limited past AI use at this stage and expected less future assistance compared to other stages.}

%Additionally, participants expected the ``actor'' role to be most useful during the character, and goal \& idea stages, as shown in Table~\ref{tab:expectations and stages}. This is because they believed the ``actor'' would be aligned with the screenplay's characters and could support emotional depth, aiding in the characters's motivations and emotions. In contrast, they did not expect the ``actor'' to assist in the synopsis \& outline stage, as screenwriters typically rely on foundational world-building, character, and storyline rather than detailed actor information in this stage. Consistent with the findings in Section 4.3.2, participants reported limited past AI use at this stage and expected less future assistance compared to other stages.}


%\textcolor{black}{Additionally, We found that participants expected the ``actor'' role to be most applicable during the character, and goal \& idea stages, as shown in Table~\ref{tab:expectations and stages}. This is because they anticipated that the ``actor'' would be related or consistent with the characters in the screenplay and could provide support with emotional depth, helping screenwriters better convey the characters' motivations and emotions. In contrast, participants did not expect the ``actor'' role to assist in the synopsis \& outline stage. This is because, at this stage, screenwriters typically do not require detailed information about the actors. Instead, they can complete the synopsis \& outline based on foundational world-building, character, and storyline. Aligning with the results in Section~\ref{sec:Allocation}.2, screenwriters mentioned that they had seldom used AI at this stage previously and expected less assistance in the future compared to other stages.}


%This categorization helps us better understand how AI can support screenwriters by facilitating meaningful human-AI collaboration through tailored functions and engagement that align with user expectations.

\subsubsection{Simulating Characters Based on Multiple Requirements} \label{sec:Simulating}

Many participants argue that screenplay logic is inherently visual, and AI should use both text and visual representations when generating character actions and emotional responses. Excluding N2, the remaining 22 participants expect AI to generate detailed character images, including behaviors, actions, emotional changes, and complex relationships while interacting with screenwriters in real-time. This would help screenwriters intuitively understand characters' internal emotions and external behaviors, making the characters more vivid and multidimensional.


%\subsubsection{Simulating Characters Based on Multiple Requirements} Many participants argue that screenplay logic is inherently visual, and AI should use both text and visual representations when generating character actions and emotional responses. Excluding N2, the remaining 22 participants expect AI to generate character images, including behaviors, actions, emotional changes, and relationships while interacting with screenwriters in real-time. This would help screenwriters intuitively understand characters' internal emotions and external behaviors, making the characters more vivid and multidimensional.
%\textcolor{black}{P2, P3, P5, P6, P7, P11, P13, P14, P17, N1, and N3} expect AI to generate character images, including behaviors, actions, emotional changes, and relationships, and interact with screenwriters in real-time. 

%Participants believe that the logic of a screenplay is inherently visual, so when generating character actions and emotional responses, AI should rely not only on text but also on visual representations. N1, N3, P2, P3, P5, P6, P7, P11, P13, P14, and P17 expect AI to generate character images, including character behaviors, actions, emotional changes, and relationships, and interact with screenwriters in real-time, helping them to understand characters’ internal emotions and external behaviors more intuitively, making the characters in the screenplay more vivid and multidimensional.

%Although current AI image-generation tools can generate character images to some extent, several participants expressed higher expectations for AI in portraying more complex character behaviors, actions, emotional changes, and relationships. 

%\textbf{Modeling Internal Emotions.} P2, P11, P13, and P14 suggest that AI can stimulate creative inspiration by continuously demonstrating certain emotional performances. P17 pointed out that AI’s real-time display of facial expressions and psychological states would help screenwriters grasp the inner world of characters, assisting in the adjustment of the screenplay's rhythm and the internal monologues of the characters.

%P2, P11, P13, and P14 suggest that AI can stimulate creative inspiration by continuously demonstrating certain emotional performances. P17 pointed out that AI’s real-time display of facial expressions and psychological states would help screenwriters grasp the inner world of characters, thereby adjusting the rhythm of the screenplay and the internal monologues of the characters.

\textbf{Modeling Internal Emotions.} \textcolor{black}{P2, P4, P5, P11, P13, P14, P17, and N4 emphasized} that AI has the potential to stimulate creative inspiration by consistently portraying specific emotional performances. In particular, \textcolor{black}{P17} argued that AI's real-time display of facial expressions and psychological states could provide screenwriters with deeper insights into a character's inner world, enabling more precise adjustments to the screenplay's rhythm and enhancing the depth of the characters' internal monologues.

%\textcolor{black}{Ten participants who have previously used AI in screenwriting emphasized} that AI has the potential to stimulate creative inspiration by consistently portraying specific emotional performances. In particular, \textcolor{black}{P17} argued that AI's real-time display of facial expressions and psychological states could provide screenwriters with deeper insights into a character's inner world, enabling more precise adjustments to the screenplay's rhythm and enhancing the depth of the characters' internal monologues.


%\textcolor{black}{Participants who have previously used AI in screenwriting (P2, P11, P13, and P14)} suggested that AI has the potential to stimulate creative inspiration by consistently portraying certain emotional performances. \textcolor{black}{P17} noted that AI's real-time display of facial expressions and psychological states could assist screenwriters in understanding a character's inner world, enabling better adjustments to the screenplay's rhythm and enhancing the characters' internal monologues. 
%\textcolor{black}{Participants who had not previously used AI in screenwriting did not express demand for this functionality, possibly due to limited awareness of AI’s capabilities in providing emotional support. In contrast, participants with prior AI experience observed that current AI tools struggle to model characters' internal emotions effectively. They suggested that improving this capability could significantly aid screenwriters in the creative process.}
\textbf{Simulating External Behaviors.} P2, P4, P5, P6, P9, P10, P11, P12, P13, P14, P15, P16, and N4 expressed needs for the portrayal of character behaviors in the screenplay. \textcolor{black}{P12} suggested that AI could reference existing films of similar types to provide valuable scenes and emotional expression patterns for the screenplay creation. \textcolor{black}{P16} mentioned that by dialoguing with AI-embodied characters, screenwriters could explore how characters react in specific situations and incorporate these reactions into the screenplay. \textcolor{black}{P2 and P13} argue that AI can simulate characters' behaviors in various backgrounds, enabling screenwriters to understand and express the emotional complexity of characters more comprehensively without the limitations of geography and resources.


%\textbf{Simulating External Behaviors.} Participants (P2, P4, P5, P6, P9, P10, P11, P12, P13, P14, P15, P16, and N4) expressed specific needs for the portrayal of character behaviors in the screenplay. \textcolor{black}{P12} suggested that AI could reference existing films of similar types to provide valuable scenes and emotional expression patterns for the screenplay creation. \textcolor{black}{P16} mentioned that by dialoguing with AI-embodied characters, screenwriters could explore how characters react in specific situations and incorporate these reactions into the screenplay. \textcolor{black}{P2 and P13} argue that AI can simulate characters’ behaviors in various backgrounds, enabling screenwriters to understand and express the emotional complexity of characters more comprehensively without the limitations of geography and resources.


%\textbf{Simulating External Behaviors.} Twelve participants with prior experience using AI in screenwriting, along with N4, who has not used AI in the screenwriting process, expressed specific needs for the portrayal of character behaviors in the screenplay. \textcolor{black}{P12} suggested that AI could reference existing films of similar types to provide valuable scenes and emotional expression patterns for the screenplay creation. \textcolor{black}{P16} mentioned that by dialoguing with AI-embodied characters, screenwriters could explore how characters react in specific situations and incorporate these reactions into the screenplay. \textcolor{black}{P2 and P13} argue that AI can simulate characters’ behaviors in various backgrounds, enabling screenwriters to understand and express the emotional complexity of characters more comprehensively without the limitations of geography and resources.

%\textcolor{black}{P2, P6, P9, P11, P12, P13, P14, P16, and N4} expressed specific needs for the portrayal of character behaviors in the screenplay. 

%\textbf{Ensuring Character Data Authenticity.} Participants expect AI to generate characters based on real data from similar types of people in reality, reflecting authentic behaviors and states of similar real-life individuals. \textcolor{black}{P12} stated, ``\textit{I will definitely have it mark the source, and I will verify it myself because I need to ensure that what I write is accurate.}'' \textcolor{black}{N4} added, ``\textit{What matters is the accuracy of its responses... It’s crucial that the line it delivers fits the character’s identity.}''

\textbf{Representing Character Environments.} \textcolor{black}{P2, P3, P5, P6, P8, P10, P11, P12, P13, P15, P16, N1, N3, and N4} mentioned that traditional screenwriting relies on imagination, whereas AI could help screenwriters visually present their ideas by generating scenes and character images, thereby enhancing the efficiency of both creation and communication during teamwork. Although current AI image generation tools can generate scenes to some extent, many participants expressed higher expectations for AI’s ability to generate the structure of the space, local environment, scene layout, and final imagery in which characters engage in dialogue.

%\textcolor{red}{Traditional screenwriting, which relies heavily on imagination, was noted by 11 participants with prior experience using AI in screenwriting, as well as by N1, N3, and N4, who have not used AI in the screenwriting process.} They suggested that AI could assist screenwriters by visually presenting their ideas through the generation of scenes and character images, thereby improving both the efficiency of creation and communication in collaborative settings. While current AI image generation tools can produce scenes to a certain extent, many participants expressed higher expectations for AI’s capability to generate detailed spatial structures, local environments, scene layouts, and final imagery depicting characters engaged in dialogue.



\textbf{Supporting Multimodal Output Formats.} P2, P3, P4, P6, P10, P11, P12, P13, P14, P15, P16, N1, N3, and N5 strongly expected multimodal output, such as video generation, image generation, text generation, and speech generation. Among these participants, 10 favored output combining text with image and video generation. \textcolor{black}{P16} emphasized the advantages of integrating multimodal information, stating, ``\textit{It’s best to provide a variety of information, such as text, action descriptions, images, dynamic videos, etc., which can present more detailed scenes and help screenwriters better understand and create.}'' Meanwhile, \textcolor{black}{P15, N1, and N5} specifically mentioned the importance of combining text and speech output. \textcolor{black}{P15} noted, ``\textit{I think the story is already deep enough through text, but if understanding is difficult, we can supplement with sound. Even without visuals, you can still feel the emotion through sound.}'' Furthermore, \textcolor{black}{P2, P13, and P16} highlighted AI’s potential in offering immersive experiences and enhancing reality perception. \textcolor{black}{P2} stated, ``\textit{I prefer it to be 3D projection right next to me.}''
%Eleven out of 23 participants strongly expected multimodal output, such as video generation, image generation, text generation, and speech generation. Among these participants, a significant number favored output combining text with image and video generation (\textcolor{black}{P3, P4, P6, P10, P11, P13, P14, P16, and N1}). \textcolor{black}{P16} emphasized the advantages of integrating multimodal information, stating, ``\textit{It’s best to provide a variety of information, such as text, action descriptions, images, dynamic videos, etc., which can present more detailed scenes and help screenwriters better understand and create.}'' Meanwhile, \textcolor{black}{P15, N1, and N5} specifically mentioned the importance of combining text and speech output. \textcolor{black}{P15} noted, ``\textit{I think the story is already deep enough through text, but if understanding is difficult, we can supplement with sound. Even without visuals, you can still feel the emotion through sound.}'' Furthermore, \textcolor{black}{P2, P13, P14, and P16} mentioned AI’s potential in offering immersive experiences and enhancing reality perception. \textcolor{black}{P2} stated, ``\textit{I prefer it to be 3D projection right next to me.}''

\subsubsection{Engaging with Screenwriters with Different Methods.} \label{sec:Engaging}
After expressing their functional expectations, all participants emphasized the need to combine AI-generated content with customizable parameters for preliminary setup. They expect to define characters through parameters such as writing style, detailed story outline, character information, backstories, reactions in specific situations, and current scenarios. Following this setup, 18 participants expressed a strong desire to enhance their portrayal of characters by engaging in dialogue with the AI ``actor.'' This feature allows screenwriters to maintain control and guide the AI to obtain the necessary information. Ideal interactions with the AI ``actor'' suggested by participants fall into three categories (Using \textit{Titanic} as an example, where Jack and Rose are the main characters and AI is playing the role of Jack, there are three categories of screenwriter engagement as illustrated in Fig. \ref{actor}.): (1) the screenwriter interacts with AI Jack as a character within the screenplay’s story world (e.g., screenwriter as Rose), (2) the screenwriter engages with AI Jack as a character outside the screenplay’s story world (e.g., screenwriter as a bystander), and (3) the screenwriter observes interactions between AI Rose and AI Jack (e.g., screenwriter as an observer). Meanwhile, some participants (\textcolor{black}{P2, P6, P9, P13, P15, N1, N4, and N5}) hoped to seamlessly switch between these methods to offer more choice and immersive experience, as characters reveal different facets to different people. As \textcolor{black}{N1} stated: “\textit{If you view a character from just one perspective, it's impossible to create a well-rounded character. Since they show different sides to different people.}”


%\subsubsection{Engaging with Screenwriters with Different Roles.} After expressing their functional expectations, all participants emphasized the need to combine AI-generated content with customizable parameters for preliminary setup. They expect to define characters through parameters such as writing style, story outline, character information, backstories, reactions in specific situations, and current scenarios. Following this setup, 15 out of 23 screenwriters expressed a desire to enhance their portrayal of characters by engaging in dialogue with the AI ``actor.'' This feature allows screenwriters to maintain control and guide the AI to obtain the necessary information. Ideal interactions with the AI ``actor'' suggested by participants fall into three categories (Using \textit{Titanic} as an example, where Jack and Rose are the main characters and AI is playing the role of Jack, there are three categories of screenwriter engagement as illustrated in Fig. \ref{actor}.): (1) the screenwriter interacts with AI Jack as a character within the screenplay’s story world (e.g., screenwriter as Rose), (2) the screenwriter engages with AI Jack as a character outside the screenplay’s story world (e.g., screenwriter as a bystander), and (3) the screenwriter observes interactions between AI Rose and AI Jack (e.g., screenwriter as an observer). Meanwhile, some participants (\textcolor{black}{P2, P4, P6, P9, P13, P15, N1, N2, and N5}) hoped to switch between these roles to offer more choice and experience, as characters reveal different facets to different people. As \textcolor{black}{N1} stated: “\textit{If you view a character from just one perspective, it's impossible to create a well-rounded character. Since they show different sides to different people.}”

%After expressing their functional expectations, all participants mentioned the need to combine AI-generated content with customizable parameters for further preliminary setup. Participants expect to define characters through parameters such as writing style, story outline, character information, backstories, character reactions in specific situations, and the current scenario, etc.After completing the preliminary setup, 17 out of 23 screenwriters expressed a desire to enhance their understanding and portrayal of characters by engaging in direct or indirect dialogue with the AI ``actor.'' This feature emphasizes the screenwriter’s control and guidance over the AI to ensure that they can effectively obtain the needed information. The ideal interactions and engagements with the AI ``actor'' suggested by all participants can be summarized into the following aspects:(Using \textit{Titanic} as an example, where Jack and Rose are the main characters and AI is playing the role of Jack, there are three categories of screenwriter engagement as illustrated in Fig. \ref{actor}.) The first category involves the screenwriter interacting with AI Jack as a character within the screenplay (e.g., the screenwriter as Rose). The second category involves the screenwriter engaging with AI Jack as a character outside of the screenplay (e.g., the screenwriter as a bystander). The third category involves the screenwriter observing the interaction between AI Rose and AI Jack as a character outside of the screenplay (e.g., the screenwriter as a screenwriter). However, some participants hoped to switch identities and dialogue perspectives through three categories during the usage, offering more choices and experiences (N1, N2, P2, P4, P6, P9, P13, P15, N5). Since characters reveal different aspects of themselves to different people, multifaceted interaction becomes more critical. As N1 stated: “\textit{If you view a character from just one perspective, it's impossible to create a well-rounded character. They show different sides to different people.}”


\begin{figure*}
%[H]
 \centering         
\includegraphics[width=1\textwidth]{figure/actor.jpg} 
 \caption{The Three Categories of Screenwriter Engagement with AI Actors}
 \Description{Description for Figure 3:
This chart illustrates three different modes of screenwriter engagement with AI during the creative process: Internal Character, External Character, and Observer. Using the characters Jack and Rose from Titanic as examples, the chart explains how screenwriters interact with AI in the story world. Here are the detailed descriptions for each mode:
1. Screenwriter as an Internal Character
- In the first box on the left, the screenwriter is represented as an internal character within the story world, such as playing the role of Rose.
- In this setup, the screenwriter interacts with the AI, which plays another character, like Jack.
- Through communication (indicated by two-way arrows), the screenwriter as Rose engages in dialogue and interaction with the AI as Jack.
- This means the screenwriter directly participates in the story world, experiencing and advancing the narrative as an internal character.
2. Screenwriter as an External Character
- The middle box shows the screenwriter as an external character, such as a bystander who does not directly participate in the core plot.
- The AI still plays the main role of Jack, while the screenwriter, as an external character, communicates and interacts with the AI's character.
- In this mode, there is still interaction between the screenwriter and the AI character, but the screenwriter is not a central character, participating in the story from a secondary perspective.
3. Screenwriter as an Observer
- In the third box on the right, the screenwriter is positioned as an observer, acting in their role as a screenwriter rather than a character.
- In this case, the screenwriter does not directly participate in the story but instead watches the interaction between AI characters. The chart shows the AI playing both Rose and Jack, while the screenwriter observes their dialogue.
- In this mode, the screenwriter does not intervene in the development of the story but gathers inspiration or insights from watching the interactions between the AI-controlled characters.}
 \label{actor}
 \end{figure*}

\textbf{Screenwriter as an Internal Character.} The first category involves interacting with the AI ``actor'' as specific characters within the screenplay’s story world (\textcolor{black}{P2, P4, P5, P7, P9, P10, P11, P15, P16, N1, N2, and N4}). \textcolor{black}{P10} suggested that refining character portrayals through multiple character perspectives within the story world would lead to more precise and nuanced character development. They argued that this form of interaction could directly influence the final screenplay, showing specific behaviors. As \textcolor{black}{P16} noted: ``\textit{I feel that directly playing a character benefits my creativity. By embodying a character, I can better understand how another character would react in a given situation... it helps to think about what each character would do or say in the moment. This immersive experience is very helpful for the creative process.}''

%\textbf{Screenwriter as an Internal Character.} The first category involves interacting with the AI ``actor'' as specific characters within the screenplay’s story world (\textcolor{black}{P2, P4, P5, P7, P10, P11, P15, P16, N1, N2, and N4}). \textcolor{black}{P10} suggested that refining character portrayals through various character perspectives within the story world would lead to more precise character development. They argued that this form of interaction could directly influence the final screenplay, showing specific behaviors. As \textcolor{black}{P16} noted: ``\textit{I feel that directly playing a character benefits my creativity. By embodying a character, I can better understand how another character would react in a given situation... it helps to think about what each character would do or say in the moment. This immersive experience is very helpful for the creative process.}''

%The first category involves interacting with the AI ``actor'' as some specific characters within the screenplay (N2, P10, P16). P9 suggested that continuously refining character portrayals through different character perspectives within the screenplay would result in more precise characters. They hoped that this form of interaction would have more potential to directly influence the final screenplay, yielding immediate results by showcasing only specific behaviors in front of particular characters, without the need to construct additional background stories or delve deeply into the character's biography. As P16 mentioned: ``\textit{I feel that directly playing a character benefits my creativity. By embodying a character, I can better understand how another character would react in a given situation... it helps to think about what each character would do or say in the moment. This immersive experience is very helpful for the creative process.}''


\textbf{Screenwriter as an External Character.} The second category involves interacting with the AI ``actor'' as characters outside the screenplay’s story world, such as the screenwriter, a bystander, or a family member not depicted in the screenplay (\textcolor{black}{P2, P6, P10, P11, P12, P13, P15, N1, N3, and N4}). Since these characters are not directly tied to the screenplay, the interactions do not immediately affect the final text. The goal is to gain a deeper understanding of the character through the perspectives of additional personas imagined by the screenwriters, creating more realistic and well-rounded characters enriched by detailed backgrounds and biographies. \textcolor{black}{P2} noted, ``\textit{It's necessary to use a 'god's eye view' to see the various forms a character takes. It's very effective!}'' \textcolor{black}{P11} compared talking to AI characters to interviewing real people, saying, ``\textit{I might, as the screenwriter, directly ask the character about the plot, reading them dialogue from my script and seeing how they respond.}''

%The second category involves interacting with the AI ``actor'' as some characters outside the screenplay, such as the screenwriter, a bystander, or a member of the character's family not depicted in the screenplay (P2, P9, P11, P13, N4). Since these characters are not directly related to the screenplay content, the outcomes of these interactions with the AI do not immediately translate into the final screenplay. The purpose of this approach is to gain a comprehensive understanding of the character through the perspectives of various additional personas envisioned by the screenwriters outside the screenplay, to create characters that are as realistic and well-rounded as possible, enriched by detailed background stories and character biographies, rather than being merely fabricated identities. P2 mentioned: ``\textit{It's necessary to use a 'god's eye view' to see the various forms a character takes. It's very effective!}'' P9 said: ``\textit{If you refine a character's portrait by using different perspectives, like their mother, a passerby, or their girlfriend, it becomes more accurate.}'' P11 likened talking to AI characters to interviewing real people, saying: ``\textit{I might directly ask the character about the plot, reading them dialogue from my script and seeing how they respond.}''


\textbf{Screenwriter as an Observer.} When screenwriters choose to observe interactions between AI ``actors'' as bystanders, they prefer the AI to generate content autonomously, providing inspiration for the screenplay, especially in terms of characters' dialogue and behaviors (\textcolor{black}{P2, P3, P4, P6, P7, P8, P9, P10, P11, P12, P13, P14, P15, P17, P18, N1, N3, N4, and N5}). \textcolor{black}{P2 and N3} noted that character interaction processes generate stories, which are fundamental to screenwriting. Meanwhile, screenwriters also want AI-generated content to incorporate more specific user settings to avoid random generation (\textcolor{black}{P3, P4, P11, and P12}). \textcolor{black}{P3} stated, ``\textit{For screenwriting, I hope it’s the interaction between those characters, not between them and me, but I want to be able to guide them.}'' \textcolor{black}{P11} added that if AI characters evolved to a level close to human self-awareness, allowing them to act freely in pre-set scenes could lead to more interesting interactions. \textcolor{black}{P4} emphasized the expectation of AI autonomy but noted that completely random behaviors could be inefficient: ``\textit{There are too many possibilities with random generation, making it challenging to use. It would be more convenient if I could control it in stages.}''

%When screenwriters choose to observe interactions between AI ``actors'' as bystanders, they prefer the AI to generate content autonomously and provide inspiration for the screenplay, particularly in terms of characters’ dialogue and behaviors (N3, P2, P3, P4, P10, P11, P12, P13, P15, P17, P18, N5). N3 and P2 noted that interactions between characters generate stories, which are the foundation of screenwriting, making it crucial to observe the AI characters' interaction process. However, screenwriters also hope that the AI-generated content incorporates more specific user settings to avoid completely random generation (P3, P4, P11, P12). P3 stated, ``\textit{For screenwriting, I hope it’s the interaction between those characters, not between them and me, but I want to be able to guide them.}'' P11 added that if AI characters evolve to a level close to human self-awareness, allowing them to freely act in pre-set scenes could result in more interesting interactions. P4 further emphasized the expectation of AI autonomy but also pointed out that completely random character behaviors could lead to inefficiency, proposing a potential solution that combines staged control and autonomy: ``\textit{There are too many possibilities with random generation, making it challenging to use. It would be more convenient if I could control it in stages.}''


\subsection{Audience: Assisting Screenwriters in Evaluating the Value of Their Creations from the Perspective of Diverse Audience Groups, Increasing the Acceptance and Recognition of Their Work}
\textcolor{black}{P2, P4, P5, P6, P13, and N2 envisioned AI acting as an ``audience,'' offering timely, multidimensional feedback to enhance the marketability of screenplays.} As shown in Table~\ref{tab:expectations and stages}, this role could be applied in the goal \& idea, character, story structure \& plot, and screenplay text stages. Feedback during these stages could refine ideas, enhance character development for emotional resonance, and align final screenplays with market expectations.

%\textcolor{red}{Participants (P2, P4, P5, P6, P13, and N2) envisioned AI acting as an ``audience,'' offering timely, multidimensional feedback to enhance the appeal and marketability of screenplays.} As shown in Table~\ref{tab:expectations and stages}, this role could be applied in the goal \& idea, character, story structure \& plot, and screenplay text stages. Feedback during these stages could refine ideas, enhance character development for emotional resonance, and align final screenplays with market expectations.

%\textcolor{black}{Six out of 23 participants expected that AI, as an ``audience,'' would offer timely, multidimensional feedback, helping screenwriters improve the appeal and marketability of their work. Participants indicated that the ``audience'' role has the potential to be applied in the goal \& idea, character, and screenplay text stages, as shown in Table~\ref{tab:expectations and stages}. Providing audience feedback during these stages would help screenwriters refine their ideas, optimize character development for emotional resonance, and receive final feedback on the screenplay to better align with or predict market expectations.}

%During the interviews, 6 out of 23 participants expressed the expectation that AI, as an ``audience,'' would provide timely, multidimensional feedback. Specifically, it would assist screenwriters in enhancing the appeal and marketability of their work by evaluating its value from the perspective of a broad audience, allowing them to continuously refine their creations throughout the creative process.

%During the interviews, 6 out of 23 participants expressed the hope that AI could assist screenwriters by evaluating the value of their creations from the perspective of a board audience. Overall, participants expect AI, in its role as an ``audience,'' to provide timely, multidimensional feedback, allowing screenwriters to continuously refine their work throughout the creative process, ultimately making it more appealing and marketable.

\subsubsection{Evaluating as Mass Audience}
By simulating mass audience feedback, AI can help screenwriters adjust their creative direction to better align their work with audience expectations. \textcolor{black}{N2} said, ``\textit{AI represents the average level presented by the database it represents... You don’t need to show the book to ten different people from different cultural backgrounds; just show it to an AI.}'' This approach allows screenwriters to adjust their content promptly to meet the emotional expectations of the audience. \textcolor{black}{P2} argued that AI, through extensive data analysis, could distill ``\textit{the things that truly resonate with everyone,}'' thereby achieving broader emotional resonance. This capability would make AI an indispensable tool in the creative process, helping screenwriters craft works that touch the audience deeply. \textcolor{black}{P4} further emphasized that AI could provide ``\textit{purely neutral feedback,}'' which is particularly important for screenwriters who face interference from clients or other stakeholders during the creative process, as AI’s objectivity can offer more impartial references for screenplay revisions. \textcolor{black}{P5} stated that during the creative process, screenwriters often struggle to determine how to gain broader audience recognition, and AI’s data-driven feedback can help screenwriters approach the optimal solution. He suggested that AI should be involved in every stage of the creative process, including screenwriting, directing, and post-production, to ensure that the work maximizes its appeal to the audience. \textcolor{black}{P13} remarked, ``\textit{AI helps my work achieve both critical acclaim and commercial success... It helps us figure out what to focus on sooner and what to avoid.}'' Through AI’s feedback mechanism, screenwriters can identify the market potential of their work earlier and make targeted optimizations, thereby increasing its commercial value. 

\subsubsection{Providing Feedback as a Specific Group of Audience}
In addition to focusing on mass audiences, participants emphasized the importance of considering minority and underrepresented groups, such as individuals from specific cultural backgrounds. \textcolor{black}{P6} noted that audience preferences vary, and AI could simulate responses from diverse cultural, regional, and social backgrounds, providing feedback to help screenwriters make adjustments before releasing their work. This approach would help them avoid creative missteps caused by cultural differences or biases against minority groups.

%\subsubsection{Providing Feedback as a Specific Group of Audience} In addition to focusing on mass audiences, participants emphasized the importance of considering minority and underrepresented groups, such as individuals from specific cultural backgrounds. \textcolor{black}{P6} noted that audience preferences vary, and AI could simulate responses from diverse cultural, regional, and social backgrounds, providing valuable feedback to help screenwriters make adjustments before releasing their work. This approach would help them avoid creative missteps caused by cultural differences or biases against minority groups.
%In addition to focusing on the mass audience, participants also emphasized their attention to other minority and underrepresented groups, such as individuals from specific cultures. P6 pointed out that since audience preferences vary, AI could simulate responses from audiences of different cultures, regions, and social backgrounds, providing valuable feedback to help screenwriters make adjustments prior to the release of their work. This would allow them to avoid creative missteps due to cultural differences or biases towards minority groups.


\subsection{Expert: Offering Authoritative Evaluations and Suggestions, Guiding New Workflows Through Professional Expertise}
\textcolor{black}{P1, P4, P5, P6, P7, P8, P10, P11, P13, P15, N1, N2, and N4 expected AI to serve as an ``expert,'' offering reliable guidance and alternative solutions. As shown in Table~\ref{tab:expectations and stages}, they anticipated that the ``expert'' role could support all workflow stages, providing advice to address their limitations despite their professional training.}

%Nine of 23 participants expressed the desire for AI to assume the role of an ``expert,'' with the primary task of providing reliable, high-quality guidance and suggesting alternative solutions. As shown in Table~\ref{tab:expectations and stages}, participants believed the ``expert'' role would be beneficial across all workflow stages. This may stem from their expectation that, despite their professional training, AI could offer valuable advice to help overcome their limitations.}
%Among the 23 participants, 9 expressed the desire for AI to take on the role of an ``expert.'' They argued that AI’s primary task in this role would be to provide reliable, high-quality guidance for existing screenplay content, offering insights for alternative solutions. Despite their professional training, participants expect AI to offer advice to overcome their limitations. Their functional expectations include the following:

%Participants argue that AI’s primary task as an expert is to provide reliable and high-quality guidance for existing screenplay content or creative stages, offering in-depth insights for alternative solutions. Although all participants are professionally trained and possess the basic skills of screenwriting, they also expect AI to provide advice due to the limitations of their abilities. Specifically, their functional expectations for AI mainly include the following aspects:

\subsubsection{Providing Professional Guidance and Optimization Suggestions}
Participants \textcolor{black}{who have previously used AI in screenwriting} expect it to act as a mentor, advisor, or consultant. They anticipate that AI could identify contradictions and logical issues in the screenplay and suggest improvements, such as adding details, adjusting pacing, refining themes, modifying dialogue, revising story direction, optimizing core expressions, and enhancing aesthetic appeal (\textcolor{black}{P6, P7, P8, P10, and P11}). Additionally, participants expect AI to provide direct solutions, such as character design and plot modifications, to improve both efficiency and quality (\textcolor{black}{P5}). AI is also seen as a potential tool to assist less experienced film professionals, helping them avoid common pitfalls and enhancing the overall capabilities of the team (\textcolor{black}{P5, and P7}).
%Participants expect AI to act as a mentor, advisor, or consultant, identifying contradictions and logical issues in the screenplay and suggesting improvements such as adding details, adjusting pacing, refining themes, modifying dialogue, revising story direction, optimizing core expressions, and enhancing aesthetic appeal (\textcolor{black}{P6, P7, P8, P10, and P11}). They also expect AI to provide direct solutions, like character design and plot modifications, to improve efficiency and quality (\textcolor{black}{P5}). AI is also expected to assist less experienced film professionals, helping them avoid potential problems and enhancing the capabilities of the entire team (\textcolor{black}{P5, and P7}). 
%\textcolor{black}{Participants who had not used AI in screenwriting did not explicitly express expectations for such guidance. This could be attributed to their limited familiarity with AI's potential to address professional challenges and the absence of tools in their existing workflows that actively offer such suggestions.}

%identifying contradictions and logical relationships in the screenplay and offering suggestions such as adding details, adjusting pacing, refining themes, modifying dialogue, revising story direction, optimizing core expressions, and enhancing the style and aesthetic appeal of the screenplay (P6, P7, P8, P10, P11, P1). Additionally, AI should directly provide optimal solutions, such as character design and plot modification suggestions, to improve the efficiency and quality of screenwriting (P5). Participants also expect AI to leverage its rich professional knowledge to assist those in the film industry who lack relevant experience, helping them avoid potential problems in their work and thus improving the professional capabilities of the entire team (P5, P7).

\subsubsection{Promoting New Workflows and Cultivating Multi-skilled Filmmakers}
Participants have high expectations for AI-generated image tools to foster new workflows and support multi-skilled filmmakers. \textcolor{black}{N1} stated, ``\textit{Screenwriting primarily involves writing... but imagining scenes can be exhausting... Ideally, AI should not only convert descriptions into images but also add creativity to these visuals, helping screenwriters understand the visual aspects of later production.}'' \textcolor{black}{N4} remarked, ``\textit{I prefer directing and cinematography, and I write only out of necessity... If AI could continuously provide high-quality screenplays, I would gladly use it.}'' \textcolor{black}{P4} highlighted AI’s potential to reduce costs and provide visual references for filmmakers who take on multiple roles, such as directors and cinematographers. \textcolor{black}{P15} noted that AI's visual capabilities could integrate story and visuals early, minimizing unnecessary revisions: ``\textit{If the visuals are established while the screenwriter is creating the world, nothing will be wasted, leading to greater efficiency.}''

%Participants have high expectations for AI-generated image tools to promote new workflows and support multi-skilled filmmakers. N1 noted, ``\textit{Screenwriting primarily involves writing... but when screenwriters imagine scenes, it can be exhausting... Ideally, AI should not only convert descriptions into images but also imbue these visuals with creativity, allowing screenwriters to gain insight into the visual aspects of later production.}'' N3 added, ``\textit{I prefer directing and cinematography, and I write only out of necessity... If AI could provide continuous, high-quality screenplays, I would gladly use it.}'' P4 emphasized AI’s potential to reduce costs and offer visual references for multi-skilled filmmakers, such as directors and cinematographers. P15 pointed out that AI’s visual presentations could integrate story and visuals early in the process, thus minimizing unnecessary revisions: ``\textit{If the visuals are established while the screenwriter is building the world, nothing will be wasted, leading to greater efficiency.}''

%Participants have high expectations for AI-generated image tools in promoting new workflows and cultivating multi-skilled filmmakers. Many participants hope that AI can serve not only as a powerful assistant for screenwriters but also play a greater role in various aspects of film production. For example, N1 mentioned, ``\textit{Screenwriting is mostly about writing... But when screenwriters imagine scenes during the writing process, it can sometimes feel exhausting... Ideally, AI should provide more visual content to the author, not just convert scene descriptions into images... AI should be able to imbue these visual contents with more subjectivity and creativity.}'' N3 said, ``\textit{I actually prefer directing and cinematography, and I only do screenwriting out of necessity. I don’t really enjoy writing. If there were a tool that could liberate me and provide a continuous stream of high-quality screenplays, I would be very willing to use it.}'' P4 discussed AI’s potential assistance from the perspective of multi-skilled filmmakers, believing that although AI’s role may be limited for pure screenwriters, it can significantly reduce costs and provide intuitive visual references for those who also serve as directors, art designers, or cinematographers. P15 pointed out that in the past, art design typically followed the screenwriting process, but with AI’s visual presentation capabilities, screenwriters can incorporate visual elements into their creation early on, ensuring a closer integration of story and visuals while reducing unnecessary revisions: ``\textit{If the visuals are already established when the screenwriter is writing this world... what they write won’t be wasted, making it more efficient.}''

\subsection{Executor: Fulfilling Creative Tasks According to Specified Demands, Improving Work Efficiency} \label{sec:Executor}
\textcolor{black}{Except for P1, P2, P10, P15, and N1, all other participants wanted AI to act as an ``executor,'' performing tasks based on user-defined requirements. As shown in Table~\ref{tab:expectations and stages}, they expected this role to be applied mainly in the character, story structure \& plot, and screenplay stages. With clear objectives but uncertainty about outcomes, participants viewed AI as a tool to reduce the effort of translating goals into actionable results.}

%Twenty out of 23 participants expressed the desire for AI to take on the role of an ``executor,'' carrying out specific tasks based on user requirements. As shown in Table~\ref{tab:expectations and stages}, participants expected the ``executor'' role to be appied primarily in the character and story structure \& plot stages. Because participants have clear objectives but are uncertain about the outcome, they consider using AI to ease the workload of translating objectives into actionable results.}

%During the screenwriting process, 20 out of 23 participants expressed the hope that AI could take on the role of an ``executor,'' fulfilling specific tasks based on user requirements. These participants often have clear creative goals but are unsure about the final result, so they expect AI to reduce the workload of translating ideas into actionable outcomes.

%fulfilling specified content based on user requirements. These users typically have clear creative goals but may be uncertain about the final result, so they hope AI can reduce the workload involved in translating their ideas into actionable outcomes.

\subsubsection{Managing Continuation}
P4, P6, P8, P11, P12, P14, P17, P18, N2, N3, N4, and N5 expressed the desire for AI when acting as an ``executor,'' to better manage tasks related to the continuation and expansion of existing content. While current tools offer some capability in this area, a gap remains between participants' expectations and current standards. 

\textbf{Refining Details.} 
%Participants emphasized the need for AI to ensure contextual coherence and smooth continuation (\textcolor{black}{P12, P13, P16, P17, and N4}), as well as to enhance and refine the details (\textcolor{black}{P4, P8, P11, P13, and P18}). 
Participants emphasized the need for AI to ensure contextual coherence (\textcolor{black}{P4, P6, P12, N3, and N5}), as well as to enhance and polish specific details (\textcolor{black}{P8, P11, P17, P18, N2, and N5}). \textcolor{black}{N3} remarked, ``\textit{Coherence in a screenplay is crucial... it must first understand the details... then process and generate something according to the logic, not its own creation.}'' \textcolor{black}{P18} hoped AI could refine details like dialogue tone and attitude to more accurately reflect both the character's and the screenwriter’s intent.

\textbf{Generating Complete Screenplays.} In addition to managing detailed tasks, \textcolor{black}{P14, and N4} expressed a desire for AI to generate complete screenplays. They argued that even with AI advancements, high-level screenwriters would still play a key role. \textcolor{black}{P14} stated, ``\textit{Even if AI does the entire screenplay, it still needs someone with taste to filter and organize the content... A screenwriter would just have a better tool.}'' \textcolor{black}{N2} added that if AI could generate a cohesive story with a rich plot and character depth, it would be a significant achievement.

%Nine out of 23 participants expressed the desire for AI when acting as an ``executor,'' to better handle tasks related to the continuation and expansion of existing content. While current tools possess some capability in this area, there remains a gap between participants' expectations and current standards. These needs focus on contextual coherence and continuation (P12, P13, N4, P16, and P17), as well as detail refinement (P4, P8, P11, P13, and P18). N4 remarked, ``\textit{Coherence in a screenplay is crucial; it must first understand... Then process and generate something according to others' logic, not its own creation.}'' Meanwhile, P18 hoped that AI could focus on details such as dialogue tone and attitude to more accurately reflect both the character's and the screenwriter’s intent.


%These needs focus on contextual coherence, continuation, and detail refinement. P12, P13, N4, P16, and P17 emphasized AI’s role in maintaining contextual coherence. N4 remarked, ``\textit{Coherence in a screenplay is crucial; it must first understand... Then process and generate something according to others' logic, not its creation.}'' Meanwhile, P4, P8, P11, P13, and P18 stressed the importance of AI in refining details, with P18 hoping AI can focus on dialogue tone and attitude to more accurately reflect the creator’s intent.

%to better fulfill tasks related to the targeted continuation and expansion of existing content. Although some tools currently possess a certain degree of capability in this area, there is still a considerable gap between what participants expect and the current standards. These needs mainly focus on contextual coherence, continuation, and detail processing. P12, P13, N4, P16, and P17 all emphasized AI’s critical role in contextual coherence and continuation tasks. N4 remarked, ``\textit{Coherence in a screenplay is crucial; it must first understand... Then process and generate something according to others' logic, not its creation.}'' Meanwhile, P4, P8, P11, P13, and P18 emphasized the importance of AI in handling details. P18 hopes that AI can focus on processing details in dialogue, including tone and attitude so that AI can more accurately reflect the creator’s intentions and provide feedback more appropriately.

\subsubsection{Visualizing Presentation} \label{sec:Presentation}
While previous studies have addressed some challenges through visualization, participants in this study expressed specific expectations for visualizing the plot, character relationships, and emotions.
%Although previous studies have attempted to address some challenges faced by screenwriters through visualization, the participants in this interview expressed more specific expectations regarding the visualization of plot, character relationships, and emotions.

\textbf{Visualizing Plot Structures and Character Relationships.} \textcolor{black}{P3, P6, P7, P9, P12, P13, P14, P16, P18, N3, and N4} highlighted the potential of AI’s visualization function in enhancing screenwriting efficiency. \textcolor{black}{P3} suggested that AI could organize screenplays using mind maps, helping to clarify complex plot structures. \textcolor{black}{P6} added, ``\textit{When I don’t know how to advance the story, AI can generate new plot points visually and show how they connect with existing content,}'' streamlining the creation process. \textcolor{black}{P9} envisioned AI creating connections between characters within the story, generating multiple plot paths and endings.

%P3, P6, P9, P12, and N4 highlighted AI’s potential, believing that this feature could enhance screenwriting efficiency. P3 noted that AI could organize the screenplay using mind maps, a visual approach that is particularly useful for clarifying complex plot structures. P6 further envisioned, ``\textit{When I don’t know how to advance the story, AI can generate new plot points visually and show how they connect with existing content,}'' making the creation process smoother. P9 envisioned that within the story framework, AI could create connections between different characters, generating various plot development paths and endings.

\textbf{Visualizing Emotional Rhythms.} \textcolor{black}{P5, P7, P12, P13, P14, P17, and N3} hoped AI could abstract complex emotional cues into visual content. \textcolor{black}{P7} proposed using curves to represent emotional changes, with time on the horizontal axis and emotional fluctuations on the vertical, helping screenwriters manage emotional rhythm. He also suggested AI could display the proportional impact of events on a character’s emotions using pie charts, such as 40\% family, 30\% love, and 10\% work, which would be valuable for analysis. \textcolor{black}{N3} added that AI could use visualization to track characters’ emotional states, making the emotional narrative more coherent.

%P7, N4, and P17 expressed the hope that AI could become a tool for abstracting complex emotional cues into easily understandable visual content. P7 pointed out that screenwriters need to abstract the emotional changes of characters into curves, with the horizontal axis representing time and the vertical axis representing emotional fluctuations, to better control the emotional rhythm during the screenwriting process. He also suggested that AI could use pie charts or similar formats to display the proportional impact of different events on a character’s emotions. For example, 40\% family, 30\% love, 10\% work, which would be an extremely valuable analysis tool for screenwriters. N4 also stated that AI could use visualization techniques to help screenwriters clearly see the emotional states of characters at different stages, making the emotional narrative more coherent.

%N2, N3, and P14 expressed a strong desire for AI to directly generate complete screenplays. They noted that even with widespread AI applications, high-level screenwriters would still play an indispensable role in the creative process. P14 stated, ``\textit{Even if AI does the entire screenplay, it still needs someone with taste to filter and organize the content... A screenwriter would just have a better tool.}'' N2 added that if AI could write a cohesive story rich in plot and character depth, it would be a significant achievement.

\subsection{Summary}
Participants identified four key roles for AI in screenwriting: ``actor,'' ``audience,'' ``expert,'' and ``executor.'' \textcolor{black}{As an ``actor,'' AI simulates characters, particularly in the goal \& idea, character, and dialogue stages. In the ``audience'' role, AI provides feedback during the goal \& idea, character, story structure \& plot, and screenplay text stages. As an ``expert,'' AI supports all stages by offering guidance. In the ``executor'' role, AI transforms ideas into outcomes, primarily applied in the character, story structure \& plot, and screenplay text stages. These roles underscore AI's ability to complement human creativity rather than replace it.}

%Participants identified four key roles for AI in screenwriting: ``actor,'' ``audience,'' ``expert,'' and ``executor.'' \textcolor{black}{As an ``actor,'' AI would simulate characters, particularly in the character and goal \& idea stages. In the ``audience'' role, AI would provide feedback during the goal \& idea, character, and screenplay text stages. As an ``expert,'' AI would support all stages by offering guidance. Finally, in the ``executor'' role, AI would help transform ideas into outcomes, especially in the character and story structure \& plot stages. These roles highlight AI's potential to complement, rather than replace, human creativity.}
%Participants envisioned AI assuming four key roles in screenwriting: ``actor'', ``audience'', ``expert'', and ``executor''. \textcolor{black}{In the ``actor'' role, AI is expected to simulate characters, particularly in the character and goal \& idea stages. In the ``audience'' role, AI would provide feedback during the goal \& idea, character, and screenplay text stages to refine the work. In the ``expert'' role, AI is seen as beneficial across all stages, offering guidance and optimizing workflows. In the ``executor'' role, AI is expected to help transform ideas into outcomes, especially in the character and story structure \& plot stages. These roles emphasize AI's potential to complement, not replace, human creativity.}

%Participants envisioned AI assuming four key roles in screenwriting: ``actor'', ``audience'', ``expert'', and ``executor''. \textcolor{black}{As the ``actor'' role, AI is expected to support emotional character development, primarily in the character and goal \& idea stages. As the ``audience'' role, AI would provide data-driven feedback during the goal \& idea, character, and screenplay text stages, helping screenwriters refine their work based on diverse emotional responses. As the ``expert'' role, AI is seen as beneficial across all workflow stages, offering guidance and optimizing workflows. Finally, as the ``executor'' role, AI is anticipated to assist in expanding content and visualizing presentations, mainly during the character and story structure \& plot stages, thereby improving efficiency. These roles highlight AI's potential to complement, rather than replace, human creativity.} 

%Participants envision AI taking on four key roles in the screenwriting workflow: ``actor'', ``audience'', ``expert'', and ``executor''. As an ``actor'', AI should simulate character behaviors, offering immersive, multimodal experiences that enhance emotional and psychological character development, transforming narrative creation into a multidimensional, interactive process. As an ``audience'', AI's role would provide data-driven feedback, allowing screenwriters to test and refine stories based on emotional responses from diverse cultural backgrounds, enabling informed, predictive insights into audience reception. In the role of ``expert'',  AI is expected to offer content guidance and streamline workflows. Lastly, as an ``executor'', AI is anticipated to expand content and visualize presentations, thereby improving efficiency. These roles highlight the potential of AI to enhance, rather than diminish, human creativity.


%In our study, Section 4.1 reveals that the screenwriting workflow involves multiple stages. Section~\ref{sec:Allocation} highlights the diverse AI tasks at each stage and how AI is currently integrated to support stage-specific needs. In Section 5, we further explored participants' attitudes toward the integration of AI into the screenwriting workflow, and Section~\ref{sec:Expectations} outlines the four role-oriented functionalities for future expectations. 
%Based on our findings, the design opportunities we propose for the future focus on emotional intelligence, flexibility, and multifaceted outputs.

%Our findings suggest that the screenwriting workflow involves multiple stages (Section 4.1). Section~\ref{sec:Allocation} highlights the diverse AI requirements at each stage, emphasizing how AI can be effectively integrated to support stage-specific functions. Furthermore, Section~\ref{sec:Expectations} underscores the necessary functionalities and flexibility required to meet different role-oriented expectations. The design opportunities we propose are informed by participants' attitudes and related with the design principles outlined by Weisz et al~\cite{10.1145/3613904.3642466}.

%Therefore, when designing AI tools for these stages, developers should focus on generating multiple narrative branches, providing previews of story options, and offering visualization, as noted in Section 7.4.2. This may also involve immersive, multimodal presentations, as mentioned in Section 7.1.1, incorporating technologies like XR and 3D projection, to enable screenwriters to explore and compare these possibilities more effectively.

%\subsection{AI for Multithreaded and Multimodal Generation}
%Based on the results in Section 5.3, during the goal \& idea generation stages, as well as the story structure \& plot development stages, participants expressed a need for AI to manage a broader range of tasks (Table \ref{tab:Task Allocation}) to explore diverse creative possibilities more efficiently and make informed decisions early in the process. This aligns with the design for generative variability principle proposed by Weisz et al., which emphasizes that generative AI should assist users in managing and producing diverse outputs~\cite{10.1145/3613904.3642466}. Therefore, when designing AI tools for these stages, developers should focus on generating multiple narrative branches, providing previews of story options, and offering visualization, as noted in Section 7.4.2, to enable screenwriters to explore and compare these possibilities effectively.
%During the goal \& idea generation stages, as well as the story structure \& plot development stages, participants expressed a need for AI to manage a broader array of tasks (Table \ref{tab:Task Allocation}), enabling screenwriters to explore diverse creative possibilities more efficiently and make informed decisions early in the process. This aligns with the design for generative variability principle proposed by Weisz et al., which emphasizes that generative AI should assist users in managing and producing diverse outputs~\cite{10.1145/3613904.3642466}. Therefore, when designing AI tools for these stages, developers should focus on generating multiple narrative branches, providing previews of story options, and offering visual interfaces for screenwriters to explore and compare these possibilities.

%During the goal \& idea generation stages, as well as the story structure \& plot development stages, participants expressed a need for AI to manage a broader array of tasks (Table \ref{tab:Task Allocation}). These tasks include generating a wider range of potential outputs and narrative branches and providing quicker previews of these outputs. Such capabilities would enable screenwriters to explore diverse creative possibilities more efficiently and make informed decisions early in the process. Therefore, AI tools designed for these stages should be equipped with multi-path creative generation capabilities to meet the complex demands of screenwriters. This aligns with the Design for Generative Variability principle proposed by Weisz et al., which emphasizes the role of generative AI should assist users in managing and producing diverse outputs~\cite{10.1145/3613904.3642466}.

\section{Future Design Opportunities} \label{sec:Opportunities}

%\textcolor{black}{Section 6 highlights the need for flexible AI systems to simulate diverse ``actor'' and ``audience'' roles, addressing screenwriters' demands for nuanced character portrayals and predictive audience feedback. Current tools leverage big data for screenwriting and feedback analysis~\cite{pavel2015sceneskim, sanghrajka2017lisa, mutlu2020future}, but participants emphasized the importance of AI capable of emotional simulation and managing complex character dynamics. Findings from Section~\ref{sec:Allocation} further stress the need for AI to support tasks such as goal \& idea generation, story structure, and plot development (Table \ref{tab:Task Allocation}). Additionally, Section~\ref{sec:Presentation} and 6.1.1 emphasize immersive representations' role in enhancing creativity, highlighting the necessity of multithreaded and multimodal generation.} \textcolor{black}{To meet these demands, our research identifies two aspects: model development and interaction design, providing guidelines for future AI systems that align with screenwriters' needs for emotional intelligence, role adaptability, and multithreaded output.}

%\textcolor{black}{Section 6 highlights the need for flexible AI systems capable of simulating diverse ``actor'' and ``audience'' roles, enabling nuanced character portrayals and predictive audience feedback. Current tools are limited to basic static and discrete emotional representations~\cite{assunccao2022overview, goyal2010toward}, falling short of participants' expectations for continuous and dynamic emotional simulation. Moreover, while existing tools utilize big data for screenwriting and feedback analysis~\cite{pavel2015sceneskim, sanghrajka2017lisa, mutlu2020future}, participants stressed the importance of AI systems capable of managing complex character dynamics. Findings from Section~\ref{sec:Allocation} underscore the need for AI to support tasks across goal \& idea generation, story structure, and plot development stages (Table \ref{tab:Task Allocation}). Additionally, Section~\ref{sec:Presentation} and 6.1.1 highlight the role of immersive representations in fostering creativity, emphasizing the necessity of multithreaded and multimodal generation. Based on these findings, our research identifies two key aspects: model development and interaction design, providing guidance for developing future AI systems that address screenwriters' needs for emotional intelligence, role adaptability, and multithreaded outputs.}

%\textcolor{black}{Section 6 highlights the need for flexible AI systems capable of simulating diverse ``actor'' and ``audience'' roles, enabling nuanced character portrayals and predictive audience feedback. Current tools are limited to basic static and discrete emotional representations~\cite{assunccao2022overview, goyal2010toward}, falling short of participants' expectations for continuous and dynamic emotional simulation. Moreover, while existing tools utilize big data for screenwriting and feedback analysis~\cite{pavel2015sceneskim, sanghrajka2017lisa, mutlu2020future}, participants emphasized the importance of AI systems capable of managing complex character dynamics.Findings from Section~\ref{sec:Allocation} underscore the need for AI to support multiple tasks across goal \& idea generation, story structure, and plot development stages (Table \ref{tab:Task Allocation}). Additionally, Section~\ref{sec:Presentation} and 6.1.1 highlight the role of immersive representations, emphasizing the necessity of multithreaded and multimodal generation. Based on these findings, our research identifies two aspects: model development and interaction design, providing guidance for developing future AI systems that address screenwriters' needs for emotional intelligence, role adaptability, and multithreaded outputs.}

%\textcolor{black}{Section 6 highlights the need for flexible AI systems capable of simulating diverse ``actor'' and ``audience'' roles, enabling nuanced character portrayals and predictive audience feedback. Current tools are limited to basic, static, and discrete emotional representations~\cite{assunccao2022overview, goyal2010toward}, falling short of participants' expectations for continuous and dynamic emotional simulation. Moreover, while existing tools utilize big data for screenwriting and feedback analysis~\cite{pavel2015sceneskim, sanghrajka2017lisa, mutlu2020future}, participants emphasized complex character dynamics management. Findings from Section~\ref{sec:Allocation} underscore the need for AI to support multiple tasks across stages of goal \& idea generation, as well as story structure \& plot (Table \ref{tab:Task Allocation}). Additionally, Section~\ref{sec:Presentation} and 6.1.1 highlight the role of immersive representations. Based on these findings, our research identifies two key aspects: model development and interaction design, providing guidance for developing future AI systems that address screenwriters' needs for emotional intelligence, role adaptability, and multithreaded outputs.}

\textcolor{black}{This section summarizes participants' needs identified in the findings and proposes design opportunities. Section 6 emphasizes the need for flexible AI systems capable of simulating diverse ``actor'' and ``audience'' roles to enable nuanced character portrayals and predictive audience feedback. Current tools are constrained by basic, static, and discrete emotional representations~\cite{assunccao2022overview, goyal2010toward}, failing to meet participants' expectations for continuous and dynamic emotional simulation. Additionally, while existing tools leverage big data for screenwriting and feedback analysis~\cite{pavel2015sceneskim, sanghrajka2017lisa, mutlu2020future}, participants highlighted the importance of managing complex character dynamics. Findings from Section~\ref{sec:Allocation} further underscore the need for AI to support a range of tasks in different stages, including goal \& idea generation, and story structure \& plot development (Table \ref{tab:Task Allocation}). Moreover, Section~\ref{sec:Presentation} and Section~\ref{sec:Simulating} emphasize the significance of immersive representations. Based on these findings, we identify two key opportunities: model development and interaction design. These provide guidance for developing future AI systems that address screenwriters' needs for emotional intelligence, role adaptability, and multithreaded outputs.}

%Based on these findings, we identify two aspects of opportunities: model development and interaction design, which guide the development of future AI systems to address screenwriters' needs for emotional intelligence, role adaptability, and multithreaded outputs.}

\subsection{\textcolor{black}{Model Development}}  
\textcolor{black}{
We envision that there are two crucial steps, data collection and model training, in the lifecycle of model development can be improved based on findings from our research.}
% Our proposed model development consists of two steps: data collection and model training.}

\subsubsection{\textcolor{black}{Data Collection}}  
%\textcolor{black}{Based on our findings mentioned at the beginning of Section 7, to achieve these screenwriters' needs, building multimodal, fine-grained datasets across diverse scenarios is central to effective AI development. 
%\textcolor{black}{Based on our findings mentioned at the beginning of Section~\ref{sec:Opportunities}, building multimodal, fine-grained datasets across diverse scenarios is essential for addressing screenwriters' needs. We propose three key dimensions for acquiring data: social media feedback, high-quality works, and crowdsourced observations.}
\textcolor{black}{Based on our findings mentioned in Section~\ref{sec:Opportunities}, building multimodal, fine-grained datasets across diverse scenarios is essential for addressing screenwriters' needs. We propose three key dimensions for acquiring data: social media feedback, high-quality works, and crowdsourced observations.}

\textbf{\textcolor{black}{Social Media Feedback.}}  
\textcolor{black}{Social computing methods analyze audience feedback derived from social media activity, fan discussions, and sentiment trends. These analyses provide valuable insights into emotional responses and narrative expectations across various demographics. By categorizing feedback based on genre preferences, emotional resonance, and storytelling trends, AI systems can better align their outputs with specific screenplay genres. For ``actor'' roles, social data offers insights into audience perceptions of character traits, relationships, and dynamics, thereby enhancing the AI's ability to generate resonant and relatable characters.}

\textbf{\textcolor{black}{High-Quality Films and Screenplays.}}  
\textcolor{black}{Analyzing high-quality films and screenplays uncovers patterns in character relationships, backstories, and narrative evolution. These insights could contribute to the creation of robust datasets that train AI systems to produce realistic character portrayals and interconnected storylines. This approach aids screenwriters by enriching the depth and complexity of character development within intricate narratives.}
%\textcolor{black}{Analyzing high-quality films and screenplays uncovers patterns in character relationships, backstories, and narrative evolution. These insights have the potential to contribute to the creation of robust datasets that train AI systems to produce realistic character portrayals and coherent, interconnected storylines. This approach aids screenwriters by enriching the depth and complexity of character development within intricate narratives.}

\textbf{\textcolor{black}{Crowdsourced Observation Networks.}}
\textcolor{black}{Crowdsourcing, widely used in the creative domain~\cite{kim2017mechanical, kim2014ensemble, huang2020heteroglossia}, supports observation networks as a scalable method to collect diverse data for AI training. Engaging individuals from varied geographic and cultural contexts, these networks provide insights for creating realistic, contextually nuanced characters and settings. Standardized guidelines can specify data requirements and recording methods, while mobile apps facilitate multimodal uploads, such as text, images, audio, and video. Incentive mechanisms and validation processes, including peer reviews or AI checks, ensure data reliability and quality.}

%\textcolor{black}{Crowdsourcing, widely used in the creative domain~\cite{kim2017mechanical, kim2014ensemble, huang2020heteroglossia}, supports crowdsourced observation networks as a scalable method to collect diverse data for AI training. Engaging individuals from varied geographic and cultural contexts, these networks provide insights for creating realistic, contextually nuanced characters and settings. Standardized guidelines can specify data requirements and recording methods, while mobile apps facilitate multimodal uploads, such as text, images, audio, and video. Incentive mechanisms and validation processes, including peer reviews or AI checks, ensure data reliability and quality.}
%Crowdsourcing, widely applied in the creative domain~\cite{kim2017mechanical, kim2014ensemble, huang2020heteroglossia}, underpins crowdsourced observation networks as a scalable method for collecting diverse data to train AI systems. By engaging individuals from varied geographic and cultural contexts, these networks gather insights to develop realistic, contextually nuanced characters and settings. Standardized guidelines could define data requirements and recording methods, while mobile apps enable multimodal data uploads, including text, images, audio, and video. Incentive mechanisms and validation processes, such as peer reviews or AI checks, could ensure the reliability and quality of the data.


\subsubsection{\textcolor{black}{Model Training}} \label{sec:Model Training}

%\textcolor{black}{To align with screenwriters' expectations for conflict-driven narratives~\cite{10.1145/3656650.3656688}, datasets can be structured based on the theory of conflict~\cite{lawson1936theory, eisenstein2014film}, categorizing data into four types: Human vs. Nature, Human vs. Society, Human vs. Human, and Human vs. Self.}

\textcolor{black}{Section~\ref{sec:Current Positive} and Section~\ref{sec:Simulating} highlight the need to deepen AI’s understanding of contextual factors and characters' internal emotions, while Section~\ref{sec:Presentation} emphasizes narrative-driven emotional integration, focusing on plot context and character relationships. Accordingly, we provide suggestions for training models that are centered on context and characters. Meanwhile, to align with screenwriters' expectations for conflict-driven narratives~\cite{10.1145/3656650.3656688}, datasets can be structured using the theory of conflict~\cite{lawson1936theory, eisenstein2014film}, categorized into Human vs. Nature, Human vs. Society, Human vs. Human, and Human vs. Self.}

\textbf{\textcolor{black}{Contextual Emotion Modeling.}}
\textcolor{black}{Screenwriters require AI systems capable of dynamically adapting to evolving narrative contexts. Future systems should represent emotions as continuous dynamic trajectories and incorporate diverse conflict types such as Human vs. Nature and Human vs. Society.}
\textcolor{black}{To ensure that AI models can generate contextually appropriate responses, it is possible to leverage reinforcement learning from human feedback (RLHF) techniques to train attention-based language models that can effectively capture the contextual information following humans' approaches~\cite{niu2021review}.}
Additionally, multidimensional emotion modeling~\cite{gafa2023emotivita} has the potential to enable continuous dynamic simulations, facilitating coherent emotional shifts that enhance narrative depth and maintain alignment with story progression.
%\textcolor{black}{Screenwriters require AI systems capable of dynamically adapting to evolving narrative contexts. Future systems should represent emotions as continuous dynamic trajectories and incorporate conflict types such as Human vs. Nature and Human vs. Society. To ensure that AI models can generate contextually appropriate responses, it is possible to leverage reinforcement learning from human feedback (RLHF) techniques to train attention-based language models that can capture the contextual information following humans' approaches~\cite{niu2021review}. Additionally, multidimensional emotion modeling~\cite{gafa2023emotivita} has the potential to make continuous dynamic simulations, facilitating coherent emotional shifts that enhance narrative depth and maintain alignment with story progression.}
%Leveraging reinforcement learning and attention mechanisms to process real-time contextual feedback~\cite{niu2021review}, AI could generate contextually appropriate responses. 


%\textcolor{black}{Screenwriters require AI systems capable of dynamically adapting to evolving narrative contexts. Future systems should represent emotions as continuous dynamic trajectories and incorporate conflict types such as Human vs. Nature and Human vs. Society. By leveraging reinforcement learning and attention mechanisms to process real-time contextual feedback~\cite{niu2021review}, AI could generate contextually appropriate responses. Additionally, multidimensional emotion modeling~\cite{gafa2023emotivita} has the potential to make continuous dynamic simulations, facilitating gradual and coherent emotional shifts that enhance narrative depth and maintain alignment with story progression.}

%\textcolor{black}{Screenwriters require AI systems that dynamically adapt to evolving narrative contexts. Current models, which rely on static and discrete emotional representations~\cite{assunccao2022overview, goyal2010toward}, fall short of this need. Future systems should treat emotions as continuous dynamic lines and incorporate conflict types such as Human vs. Nature and Human vs. Society. By utilizing reinforcement learning and attention mechanisms to process real-time contextual feedback~\cite{niu2021review}, AI could generate contextually appropriate responses. Multidimensional emotion modeling could enable continuous dynamic simulation~\cite{gafa2023emotivita}, producing gradual and coherent emotional shifts that enhance narrative depth and alignment with story progression.}

\textbf{\textcolor{black}{Character Emotion Modeling.}}  
\textcolor{black}{To create emotionally engaging storytelling, AI should simulate evolving character relationships and emotional arcs~\cite{smith2019simulating}. 
To achieve the target, an additional GNN model might be trained on subtle emotional cues enhances modeling for conflicts like Human vs. Human and Human vs. Self.
It can serve as a powerful method to model the character relationships~\cite{wu2020comprehensive}, with nodes as characters and edges capturing emotional or conflict-driven interactions. Incorporating the prediction results of this GNN model has the potential to better capture the dynamic and nuanced changes in characters' emotions. 
}

%To enable emotionally engaging storytelling, AI must effectively simulate evolving character relationships and emotional arcs~\cite{smith2019simulating}. Training on subtle emotional cues can enhance the modeling of conflicts such as Human vs. Human and Human vs. Self. Graph Neural Networks (GNNs) present a promising approach for representing character relationships as dynamic graphs, where nodes represent characters and edges capture their emotional or conflict-driven interactions~\cite{wu2020comprehensive}. By training GNNs on datasets that reflect relationship dynamics, AI systems can predict how interactions evolve over time and influence the narrative. This approach could support emotional consistency and enrich character trajectories, empowering screenwriters to craft compelling and nuanced narratives.

\textbf{\textcolor{black}{Cross-Disciplinary Collaboration.}}  
\textcolor{black}{Advancing AI's emotional intelligence requires collaboration among cognitive scientists, psychologists, screenwriters, and AI developers~\cite{assunccao2022overview, zhao2022emotion}.} 
%A Human-in-the-Loop (HITL) approach, such as the technique of reinforcement learning from human feedback (RLHF), could bridge technical and creative expertise by integrating expert feedback into AI development~\cite{wu2023toward}.
\textcolor{black}{A Human-in-the-Loop (HITL) approach, such as RLHF or model fine-tuning with human-labeled datasets, could bridge technical and creative expertise by integrating expert feedback into AI development~\cite{wu2023toward}.}
This approach has the potential for accurate modeling of emotional nuances and adaptability to diverse narrative contexts. The iterative process enhances AI's reliability and ensures creative alignment with screenwriters' evolving needs.

\subsection{\textcolor{black}{Interaction Design}}
%To address emotional intelligence, role adaptability, and multithreaded output, we propose tailored interaction design guidelines.

%\textcolor{black}{We propose tailored interaction design guidelines to address the needs for emotional intelligence, role adaptability, and multithreaded outputs.}
%We propose different interaction design guidelines tailored to address emotional intelligence, role adaptability, and multithreaded output.}
\textcolor{black}{We propose tailored interaction design guidelines to effectively address the needs for emotional intelligence, role adaptability, and multithreaded dynamic outputs.}

\subsubsection{For Emotional Intelligence Enhancement}
\textcolor{black}{Based on the model training approach focused on context and characters as discussed in Section~\ref{sec:Model Training}, and reflecting the scene and character emphasis in screenwriting, we propose three interaction methods.}

\textcolor{black}{\textbf{Contextual Interactions.}  
The interface should incorporate contextual factors, allowing screenwriters to shape emotional outputs based on evolving environments and stories. Features like real-time feedback, visualizations (e.g., heatmaps or evolving arcs), and contextual manipulation tools can enable adjustments to variables like settings or plot points to refine emotional trajectories. Tools such as “contextual snapshots” can help maintain narrative coherence across scene transitions~\cite{10.1145/2508244.2508251}.}

\textcolor{black}{\textbf{Character-Driven Interactions.}  
The interface should support dynamic emotional simulations, enabling the refinement of characters’ emotional responses and relationships. Tools such as timelines or radial graphs can visualize emotional evolution, while sliders allow for fine-tuning of intensity and pace. Relationship maps can simulate changes, such as conflict or reconciliation, ensuring emotional depth and alignment with character arcs~\cite{weiland2023creating}.}

%\textcolor{black}{\textbf{Context-Character Interwoven Interactions.}  To integrate context and character dynamics, the interface should align characters’ emotions with contextual elements. Real-time scenario testing can explore interactions, such as weather changes affecting emotional states. Moreover, incorporating collaborative modes could encourage contributions from screenwriters and stakeholders, providing shared visualizations to align with unified storytelling objectives. Additionally, as discussed in Section~\ref{sec:Multimodal}, multimodal outputs could further enhance creativity and improve communication efficiency.}

\textcolor{black}{\textbf{Context-Character Interwoven Interactions.}  
To integrate context and character dynamics, the interface should align characters’ emotions with contextual elements. Real-time scenario testing can explore interactions, such as weather changes affecting emotional states. Moreover, incorporating collaborative modes could encourage contributions from screenwriters and stakeholders, providing visualizations to align with unified storytelling objectives. Additionally, as discussed in Section~\ref{sec:Multimodal}, multimodal outputs could further enhance creativity and improve communication efficiency.}

\subsubsection{For Flexibility and Role Adaptation}
\textcolor{black}{Interactive solutions like modular input and multi-agent systems provide structured and dynamic methods for seamlessly adapting ``actor'' and ``audience'' roles in collaborative screenwriting.}

%\textcolor{black}{Interactive solutions like modular input and multi-agent systems provide structured and dynamic methods for adapting ``actor'' and ``audience'' roles in screenwriting.}

\textcolor{black}{\textbf{Modular Input.}}  
\textcolor{black}{For ``actor'' roles, modular components create detailed character maps, detailing traits, backstories, relationships, and emotional states. These interfaces allow screenwriters to dynamically refine profiles, supporting seamless and flexible role transitions~\cite{ye2023mplug}. For ``audience'' roles, modular inputs define segments by demographics, preferences, or emotional metrics, integrating with ``actor'' modules to simulate audience responses to character roles, plot points, or themes.}
%\textcolor{black}{For ``actor'' roles, modular components create character maps, detailing traits, backstories, relationships, and emotional states. These interfaces allow screenwriters to dynamically refine profiles, supporting seamless role transitions~\cite{ye2023mplug}. For ``audience'' roles, modular inputs define segments by demographics, preferences, or emotional metrics, integrating with ``actor'' modules to simulate audience responses to character roles, plot points, or themes.}

\textcolor{black}{\textbf{Multi-Agent Systems.}}  
\textcolor{black}{For ``actor'' roles, characters function as independent agents interacting dynamically in a shared story world~\cite{10.1145/3586183.3606763}. Screenwriters can modify interactions and preview narrative changes to inspire creativity. For ``audience'' roles, agents simulate diverse demographics, providing feedback on engagement and exploring how narrative elements spark discussions or attention, helping screenwriters craft stories that resonate with varied audiences or encourage thematic exploration.}

\subsubsection{For Multithreaded and Multimodal Generation}\label{sec:Multimodal}
\textcolor{black}{Narrative branching and immersive visualization present potential avenues for further exploring AI's capabilities in screenwriting.}
%AI could enhance creativity and efficiency in screenwriting workflows by providing narrative branching and immersive visualization.}

\textcolor{black}{\textbf{Intuitive Interaction with Narrative Branches.}}  
Participants highlighted the need for AI to generate and manage diverse narrative outputs, essential for exploring creative possibilities and making informed decisions. Features supporting multiple narrative branches would enable screenwriters to experiment with plot variations and assess the broader implications of changes in story elements. Real-time previews of these branches could facilitate intuitive adjustments, enhancing both the efficiency and effectiveness of the screenwriting process.


\textcolor{black}{\textbf{Immersive Visualization of Characters and Scenes.}}  
Participants suggested visualization tools are critical for transforming abstract concepts into tangible representations, and further emphasize the potential of immersive technologies, such as extended reality (XR) and 3D projection, to enhance this process. Integrating these technologies would allow AI tools to provide real-time visualizations of characters and scenes, offering insights into dynamics, spatial relationships, and emotional pacing. This capability could significantly enrich narrative development and improve the overall screenwriting experience.

\begin{comment}

\textcolor{black}{Building on our findings, we propose future design opportunities focused on enhancing emotional intelligence, increasing flexibility, and enabling multifaceted outputs.}

\subsection{AI for Emotional Intelligence Enhancement}

%\textcolor{black}{Results from Section~\ref{sec:capabilities}, Section 6.1 (AI as an ``actor''), and Section~\ref{sec:Presentation} (AI as an ``executor'', especially visualizing emotional rhythms) emphasize the need for advanced emotional support in screenwriting. Screenwriters expect AI tools to move beyond basic emotion recognition, to manage dynamic emotional networks, simulate complex character relationships, and track their evolution in intricate narratives. These expectations align with the mental model design principles proposed by Weisz et al.~\cite{10.1145/3613904.3642466}. Furthermore, the findings from Section~\ref{sec:Current Positive} and Section 6.1.1, especially representing character environments, reveal that screenwriters expect AI could have a deeper contextual understanding of emotion. Meanwhile, Section 6.1.1, especially modeling internal emotions, highlights the expectation of a deeper characters' emotions understanding. Moreover, Section~\ref{sec:Presentation} visualizing plot structures and character relationships suggests narrative-driven integration including context and characters. Based on these findings, we propose the following strategies for continuous dynamic emotion-simulation.}

\textcolor{black}{Findings from Section~\ref{sec:capabilities}, Section 6.1 (AI as an ``actor''), and Section~\ref{sec:Presentation} (AI as an ``executor,'' particularly in visualizing emotional rhythms) highlight the need for advanced emotional support in screenwriting. Screenwriters expect AI tools to go beyond basic emotion recognition~\cite{velagaleti2024empathetic, martinez2005emotions} by managing dynamic emotional networks, simulating complex character relationships, and tracking their evolution in intricate narratives, aligning with the mental model design principles proposed by Weisz et al.~\cite{10.1145/3613904.3642466}. Additionally, Section~\ref{sec:Current Positive} and Section 6.1.1 reveal expectations for a deeper understanding of contextual and characters' internal emotions. Meanwhile, Section~\ref{sec:Presentation} emphasizes the necessity of narrative-driven emotional integration, incorporating plot context and character relationships. Based on these findings, we propose specific strategies for continuous dynamic emotion simulation.}

\textcolor{black}{\subsubsection{Training Models}  
Traditional AI models treat emotions as discrete states (e.g., happy, sad, angry)~\cite{assunccao2022overview, goyal2010toward}, oversimplifying their dynamic nature. Continuous emotion models provide a more effective alternative, as demonstrated in prior work using evolving fortune trajectories in character arcs to drive story development~\cite{chung2022talebrush}. However, this approach does not fully support continuous, dynamic emotional storytelling. To address this, we propose the following:}

\textcolor{black}{\textbf{Understanding Contextual Emotions.}  
Screenwriters expect AI-generated content to adapt responsively to narrative contexts, aligning with dynamic environments and story progression. Current systems offer limited contextual integration, with challenges in managing complex influences and generating contextually appropriate responses. Future AI systems should integrate real-time contextual feedback through reinforcement learning and attention mechanisms. These methods could decompose story settings into key influences, allowing AI to respond dynamically to multifaceted contexts. Continuous emotion modeling in multidimensional spaces would enable gradual emotional shifts tied to narrative developments, enhancing coherence and depth.}
%AI-generated content should adapt to narrative contexts, aligning with evolving environments and story progression. Current systems lack robust contextual integration. Future AI systems should explore the usage of reinforcement learning and attention mechanisms for real-time contextual feedback, decomposing story settings into key influences. Continuous emotion modeling in multidimensional spaces would allow emotional shifts tied to narrative changes, enhancing depth and coherence.}

\textcolor{black}{\textbf{Understanding Character Emotions.}  
AI tools should simulate dynamic character relationships and emotional evolution. While current models focus on static emotions, future systems should capture emotions, conflicts, and interactions in real time. Algorithms trained on subtle cues could enhance character trajectories by integrating interpersonal dynamics and internal struggles, ensuring emotional consistency and alignment with character arcs~\cite{smith2019simulating}.}

\textcolor{black}{\textbf{Cross-Disciplinary Collaboration for Model Development.}  
Advancing AI emotional intelligence necessitates collaboration among diverse experts~\cite{assunccao2022overview, zhao2022emotion}, including cognitive scientists, psychologists, screenwriters, and AI developers. Cognitive scientists provide insights into emotional expression, psychologists address interpersonal dynamics, and screenwriters ensure alignment with narrative requirements. These interdisciplinary perspectives inform dataset construction (Section 7.2.1) and facilitate the development of tools adaptable to narrative contexts and evolving emotional landscapes, effectively bridging technical capabilities with creative demands.}

%AI tools should simulate dynamic character relationships and emotional evolution. While current models focus on static emotions, future systems must capture emotions, conflicts, and interactions in real-time. Algorithms trained on subtle cues could enhance character trajectories by integrating interpersonal dynamics and internal struggles, ensuring emotional consistency and alignment with story arcs.}

%\textcolor{black}{\textbf{Cross-Disciplinary Collaboration for Model Development.}  
%Advancing AI emotional intelligence requires collaboration between different experts~\cite{assunccao2022overview}, such as cognitive scientists, psychologists, screenwriters, and AI developers. Cognitive scientists contribute insights into emotional expression, psychologists focus on interpersonal dynamics, and screenwriters ensure alignment with narrative needs. These perspectives guide dataset construction (Section 7.2.1) and support the creation of tools adaptable to narrative contexts and evolving emotional landscapes, bridging technical capabilities with creative demands.}


    
\textcolor{black}{Results from Section~\ref{sec:capabilities}, Section 6.1.1 (AI as an ``actor''), and Section~\ref{sec:Presentation} (AI as an ``executor'') highlight the need for advanced emotional support in screenwriting. Screenwriters expect AI tools to go beyond basic emotion recognition to manage dynamic emotional networks, simulate nuanced character relationships, and track their evolution in complex narratives. These expectations align with the mental model design principles proposed by Weisz et al.~\cite{10.1145/3613904.3642466}. Based on these findings, we propose the following strategies:}

\textcolor{black}{\subsubsection{Training Continuous Dynamic Emotion-Simulation Models}  
Traditional AI models treat emotions as discrete states (e.g., happy, sad, angry)~\cite{assunccao2022overview, goyal2010toward}, oversimplifying their dynamic nature. Continuous emotion models offer a more effective approach. Prior work using character arcs with evolving ``fortune'' trajectories informs story development~\cite{chung2022talebrush}, but does not fully support continuous emotion-driven storytelling. To address this, we propose the following:}  

\textcolor{black}{\textbf{Model Training for Contextual Understanding.}  
AI-generated content should adapt responsively to narrative contexts, aligning with evolving environments and story progression. Current systems lack robust contextual integration. Future AI systems should employ real-time contextual feedback using reinforcement learning and attention mechanisms to decompose story settings into key influences. Continuous emotion modeling in multidimensional spaces would enable gradual emotional shifts tied to narrative changes, enhancing depth and coherence.}

\textcolor{black}{\textbf{Model Training for Character Understanding.}  
AI tools should simulate nuanced character relationships and emotional evolution across storylines. While current models focus on static emotional states, future systems should dynamically capture emotions, conflicts, and interactions. Algorithms trained on subtle interaction cues could enhance character-driven trajectories, integrating interpersonal dynamics and internal struggles. This ensures emotional consistency and alignment with story arcs, supporting screenwriters in crafting complex, multidimensional characters.}

\textcolor{black}{\textbf{Cross-Disciplinary Collaboration for Model Development.}  
Advancing AI emotional intelligence requires input from cognitive scientists, psychologists, screenwriters, and AI developers. Cognitive scientists provide insights into emotional expression, psychologists focus on interpersonal dynamics, and screenwriters ensure alignment with narrative needs. These perspectives guide dataset construction (Section 7.2.1) and support the creation of tools that adapt to narrative contexts and evolving emotional landscapes. This interdisciplinary approach enhances AI's ability to produce immersive, emotionally resonant stories, bridging technical capabilities with creative demands.}
\end{comment}

%\textcolor{black}{\textbf{Cross-Disciplinary Collaboration for Model Development.}  Advancing AI emotional intelligence requires collaboration among cognitive scientists, psychologists, screenwriters, and AI developers. Cognitive scientists provide insights into emotional expression, psychologists contribute expertise on interpersonal dynamics, and screenwriters ensure alignment with storytelling needs. These perspectives inform dataset construction (Section 7.2.1) and support the development of tools capable of adapting to narrative contexts and evolving emotional landscapes. This interdisciplinary approach enhances AI's ability to generate immersive, emotionally resonant stories, bridging technical capabilities with creative demands.}

%\textcolor{black}{Results from Section 6.1, focusing on AI’s role as an ``actor,'' indicate that screenwriting often includes complex emotional narratives, requiring AI systems to advance beyond basic emotion recognition or static relationship modeling. However, previous work falls short of addressing the depth of screenwriters’ demands for emotional intelligence~\cite{goyal2010toward, tapaswi2014storygraphs}. Specifically, our findings (Sections~\ref{sec:capabilities} and 6.1) suggest that screenwriters envision future AI tools capable of managing dynamic emotional networks, simulating nuanced emotional connections between characters, and depicting their evolution within complex storylines. Ideally, these systems would allow characters' behaviors and dialogue to evolve in harmony with underlying emotions, thereby supporting screenwriters in crafting characters with greater depth and cohesion. This expectation aligns with the design principles for mental models discussed by Weisz et al.~\cite{10.1145/3613904.3642466}. We argue that critical steps toward achieving this goal include enhancing AI’s capacity to process cohesive emotional intelligence and improving contextual sensitivity to more accurately reflect emotional complexity. These considerations lead us to propose several promising yet challenging strategies:}

%\textcolor{black}{Results from Sections~\ref{sec:capabilities}, which address current AI limitations in emotional support, Section 6.1.1, focusing on AI’s role as an ``actor,'' and Section~\ref{sec:Presentation}, examining AI’s role as an ``executor,'' underscore the more complex emotional support needs of narratives in screenwriting. These narratives require AI systems to go beyond basic emotion recognition and static relationship modeling. Our findings suggest that screenwriters anticipate future AI tools capable of managing dynamic emotional networks, simulating nuanced character relationships, and tracking their evolution within intricate storylines combined with visualizations. Such tools would enhance the creation of characters with greater depth and cohesion, aligning with the mental model design principles discussed by Weisz et al.~\cite{10.1145/3613904.3642466}. Based on these insights, we propose several promising but challenging strategies:}

\begin{comment}
\textcolor{black}{Results from Section~\ref{sec:capabilities}, Section 6.1.1 (AI as an ``actor''), and Section~\ref{sec:Presentation} (AI as an ``executor'') highlight the need for advanced emotional support in screenwriting. Screenwriters expect AI tools to go beyond basic emotion recognition to manage dynamic emotional networks, simulate nuanced character relationships, and track their evolution in complex narratives. These expectations align with the mental model design principles proposed by Weisz et al.~\cite{10.1145/3613904.3642466}. Based on these findings, we propose two strategies:}

\textcolor{black}{\subsubsection{Training Continuous Dynamic Emotion-Simulation Models}  
Traditional AI models often treat emotions as discrete states (e.g., happy, sad, angry)~\cite{assunccao2022overview, goyal2010toward}, which oversimplifies their dynamic nature. A more effective approach involves continuous emotion models. Previous work combining character arcs with evolving ``fortune'' trajectories has informed story development~\cite{chung2022talebrush}, but it does not fully support continuous emotion-driven storytelling. To address this, we propose the following guidelines:}  

\textcolor{black}{\textbf{Model Training for Contextual Understanding.} 
Screenwriters expect AI-generated content to adapt responsively to narrative contexts, aligning with dynamic environments and story progression. Current systems offer limited contextual integration, with challenges in managing complex influences and generating contextually appropriate responses. Future AI systems should integrate real-time contextual feedback through reinforcement learning and attention mechanisms. These methods could decompose story settings into key influences, allowing AI to respond dynamically to multifaceted contexts. Continuous emotion modeling in multidimensional spaces would enable gradual emotional shifts tied to narrative developments, enhancing coherence and depth.}

\textcolor{black}{\textbf{Model Training for Character Understanding.}  
AI tools must simulate nuanced character relationships and track emotional evolution across storylines. Current systems model static emotional states, but future models should dynamically capture characters’ emotions, conflicts, and interactions. Algorithms trained on subtle interaction cues could enhance character-driven emotional trajectories, incorporating influences such as interpersonal dynamics and internal struggles. These capabilities ensure logical emotional consistency and alignment with story arcs, enabling AI to support screenwriters in crafting multidimensional, emotionally complex characters.}

\textcolor{black}{\textbf{Cross-Disciplinary Collaboration for Model Development.}  
Advancing AI emotional intelligence requires collaboration among cognitive scientists, psychologists, screenwriters, and AI developers. Cognitive scientists provide insights into emotional expression, psychologists contribute expertise on interpersonal dynamics, and screenwriters ensure alignment with storytelling needs. These perspectives inform dataset construction, as detailed in Section 7.2.1, and support the development of tools capable of adapting to narrative contexts and evolving emotional landscapes. This interdisciplinary approach enhances AI's ability to generate immersive, emotionally resonant stories, bridging technical capabilities with creative demands.}
    
\textcolor{black}{Results from Section~\ref{sec:capabilities}, Section 6.1.1 (AI as an ``actor''), and Section~\ref{sec:Presentation} (AI as an ``executor'') underscore the need for advanced emotional support in screenwriting. Screenwriters anticipate AI tools that extend beyond basic emotion recognition to manage dynamic emotional networks, simulate nuanced character relationships, and track their evolution within complex storylines. These expectations align with the mental model design principles proposed by Weisz et al.~\cite{10.1145/3613904.3642466}. Based on these findings, we propose two strategies:}

\textcolor{black}{\subsubsection{Training Continuous Dynamic Emotion-Simulation Models}  
Traditional AI models often simulate emotions as discrete states (e.g., happy, sad, angry)~\cite{assunccao2022overview, goyal2010toward}, a simplification that fails to capture the nuanced and dynamic nature of emotions. A more effective approach would involve utilizing continuous emotion models. For instance, prior research has combined character arcs with evolving ``fortune'' trajectories to drive character and story development~\cite{chung2022talebrush}. However, this method falls short of enabling fully continuous, emotion-driven storytelling. To address these gaps, we propose the following guidelines:}  

\textcolor{black}{\textbf{Model Training for Contextual Understanding.} 
Screenwriters anticipate AI-generated content that adapts responsively to narrative contexts, aligning more deeply with evolving environments and story dynamics. While current AI systems offer limited context integration, managing complex environmental influences and providing rapid, contextually appropriate responses remains a challenge. Future AI systems should incorporate real-time contextual feedback to adapt emotional outputs based on narrative progression. This could be achieved through reinforcement learning and contextual attention mechanisms, which would decompose complex story settings into key influencing factors, such as the background and setting of the narrative world. These mechanisms would enable AI to respond dynamically to multifaceted contextual inputs, ensuring emotional portrayals remain accurate and logically consistent across scenes. By leveraging continuous emotion modeling and affective computing, AI could simulate gradual emotional shifts driven by changes in narrative context. Representing emotions as cohesive trajectories in a multidimensional space would allow for interconnected storylines where environmental factors and plot developments dynamically shape the emotional tone, enhancing narrative coherence and depth.}

\textcolor{black}{\textbf{Model Training for Character Understanding.}  
Screenwriters also envision AI tools capable of simulating nuanced character relationships and tracking their emotional evolution throughout complex storylines. While current AI systems can represent static emotional states, future systems must advance by dynamically modeling characters’ inner emotions, conflicts, and interpersonal dynamics. Machine learning algorithms trained on subtle cues indicative of character interactions and emotional changes could significantly enhance AI's ability to generate character-driven emotional trajectories. These models would need to capture multidimensional influences, such as interactions between characters, individual internal struggles, and their evolution alongside the narrative. Such capabilities would enable AI to maintain logical consistency in character emotions across scenes, ensuring alignment with the overall story arc and character development. By embedding emotional depth and complexity into characters, AI could better assist screenwriters in crafting narratives with well-rounded, multidimensional characters.}

\textcolor{black}{\textbf{Cross-Disciplinary Collaboration for Model Development.}  
Advancing AI emotional intelligence requires interdisciplinary collaboration among cognitive scientists, psychologists, screenwriters, and AI developers to interpret and represent complex emotional expressions accurately. Cognitive scientists offer insights into how emotions are experienced and expressed based on emotion theory, while psychologists, drawing from narrative psychology, contribute expertise on interpersonal dynamics and character authenticity to inform AI model designs that better approximate human emotional behavior. Screenwriters bring domain knowledge on emotional nuances and narrative coherence, ensuring that AI-generated content aligns with storytelling needs. These insights form the theoretical framework for constructing datasets relevant to model training. Potential methods for obtaining these datasets are discussed in detail in Section 7.2.1, highlighting their critical role in enabling effective model training. Furthermore, these experts can evaluate and refine AI tools, ensuring their effectiveness and authenticity in practical applications. This collaborative approach supports the development of AI systems capable of dynamically adapting to narrative contexts and characters’ evolving emotional landscapes. Ultimately, such systems enhance screenwriters' ability to craft immersive, emotionally resonant stories, bridging the gap between technical capabilities and creative demands.}

\end{comment}

%\textcolor{black}{\textbf{Cross-Disciplinary Collaboration for Model Development.}  Advancing AI emotional intelligence requires interdisciplinary collaboration. Cognitive scientists, psychologists, screenwriters, and AI developers should work together to interpret and represent complex emotional expressions. Cognitive scientists contribute insights into how emotions are experienced and expressed based on emotion theory. Psychologists, drawing on narrative psychology, offer knowledge of interpersonal dynamics and character authenticity, which can inform AI model designs that more closely approximate human emotional behavior. Screenwriters provide expertise on emotional nuances and narrative coherence. These combined insights guide the application of reinforcement learning and contextual attention mechanisms to model dynamic emotional trajectories and ensure contextually adaptive responses. Additionally, these professionals play a vital role in evaluating and refining AI tools, ensuring their effectiveness and authenticity. Such collaboration fosters the development of AI systems capable of dynamically adapting to narrative contexts and characters’ evolving emotional landscapes, ultimately supporting screenwriters in crafting immersive and emotionally resonant stories.} 

\begin{comment}
\subsubsection{\textcolor{black}{Designing Interactions}}

\textcolor{black}{Effective interaction design is essential for integrating contextual understanding and character-driven emotional dynamics. Interfaces must enable collaboration between screenwriters and AI, ensuring emotional trajectories align with narrative progression and character evolution while remaining intuitive to use.}

\textcolor{black}{\textbf{Contextual Interactions.}  
The interface should incorporate contextual factors, allowing screenwriters to shape emotional outputs based on evolving environments and stories. Key features could include real-time feedback, with visualizations such as heatmaps or evolving arcs to track environmental influences on emotions. Context manipulation tools should enable users to adjust variables like settings or plot points to observe their effects on emotional trajectories. “Contextual snapshots” can help track and refine scene-to-scene transitions, maintaining narrative coherence~\cite{10.1145/2508244.2508251}.}

\textcolor{black}{\textbf{Character-Driven Interactions.}  
The interface should facilitate dynamic emotional simulations, enabling users to refine characters’ emotional responses and relationships. Visual tools like timelines or radial graphs can illustrate emotional evolution, internal conflicts, and relationships. Interactive controls, such as sliders, allow for adjustments to emotional intensity, subtlety, or pace. Relationship mapping diagrams can help simulate changes like conflict or reconciliation, ensuring emotional depth and alignment with narrative arcs.}

\textcolor{black}{\textbf{Context-Character Interwoven Interactions.}  
To capture the interplay between context and character dynamics, the interface should enable narrative-driven integration. For example, tools could align a character’s emotions with contextual elements, such as amplifying fear in a dark forest. Real-time scenario testing would allow users to dynamically explore character-context interactions, such as evaluating the impact of weather changes on emotional states. Collaborative modes would facilitate contributions from screenwriters and stakeholders, offering shared visualizations to ensure alignment with storytelling objectives. Additionally, multimodal outputs (as discussed in Section~\ref{sec:Multimodal}) could further enhance creativity and communication efficiency.}

new new
\textcolor{black}{Effective interaction design is crucial for integrating contextual understanding and character-driven emotional dynamics. Interactions must enable seamless collaboration between screenwriters and AI, ensuring emotional trajectories reflect narrative progression and character evolution while offering intuitive user control.}

\textcolor{black}{\textbf{Contextual Interactions.}  
The interface should integrate contextual factors, allowing screenwriters to shape emotional outputs in response to evolving environments and stories. Key features include real-time contextual feedback with visualizations, such as heatmaps or evolving arcs, to track environmental influences on emotions. Context manipulation tools can let users adjust variables like settings or plot points to observe their impact on emotional trajectories. “Contextual snapshots” could support scene-to-scene transitions, helping track and adjust how earlier scenes influence future emotional states for narrative coherence.}

\textcolor{black}{\textbf{Character-Driven Interactions.}  
The interface should enable dynamic, character-driven emotional simulations, allowing users to refine emotional responses and relationships. Visualizations like timelines or radial graphs can depict emotional evolution, internal conflicts, and relationships. Interactive tools, such as sliders, can adjust emotional intensity, subtlety, or pace, while relationship mapping diagrams help users simulate how changes like conflict or reconciliation affect emotions. These tools ensure emotional depth and alignment with narrative arcs.}

\textcolor{black}{\textbf{Context-Character Interwoven Interactions.}  
To capture the interplay between context and character dynamics, the interface should support narrative-driven integration. For instance, tools can align a character’s emotions with contextual factors, such as intensifying fear in a dark forest. Real-time scenario testing allows users to explore character-context interactions dynamically, such as analyzing the impact of weather changes on emotional states. Collaborative modes enable input from screenwriters and other stakeholders, providing shared visualizations and ensuring alignment with storytelling goals. Multimodal outputs (as detailed in Section~\ref{sec:Multimodal}) could further enhance creative processes and communication efficiency.}


    new
    
Effective interaction design is essential for integrating contextual understanding and character-driven emotional dynamics into the interface. To achieve this, interactions must enable seamless collaboration between screenwriters and AI systems, ensuring that emotional trajectories reflect narrative progression and character evolution while providing intuitive control for the user.

\textcolor{black}{\textbf{Contextual Interactions.}  
The interface should enable the seamless integration of contextual factors, allowing screenwriters to shape emotional outputs in response to evolving environments and story progression. Key design features include real-time contextual feedback, which visually represents emotional dynamics through cues like heatmaps or evolving arcs, helping users track environmental influences on emotional portrayals. Context manipulation tools would let screenwriters modify variables such as settings, atmosphere, or plot points to simulate alternative scenarios and dynamically observe their impact on emotional trajectories. Additionally, scene-based emotional continuity can be supported through features like “contextual snapshots,” allowing users to track scene-to-scene transitions and adjust how earlier scenes influence future emotional states, ensuring narrative coherence.}

\textcolor{black}{\textbf{Character-Driven Interactions.}  
The interface should support dynamic, character-driven emotional simulations, enabling screenwriters to refine characters' emotional responses and relationships through interactive controls. Character emotion visualizations, such as timelines, radial graphs, or trajectory charts, can depict emotional evolution, internal conflicts, and relationships, allowing comparisons to highlight how interactions influence emotional states. Interactive emotional adjustments would let users fine-tune characters' responses by manipulating emotional trajectories or setting specific emotional targets for pivotal moments, with controls like sliders to adjust intensity, subtlety, or pace. Moreover, relationship mapping tools could visualize and modify interpersonal dynamics through interactive diagrams, simulating how changes like conflict or reconciliation affect individual and collective emotional states.}

\textcolor{black}{\textbf{Context-Character Interwoven Interactions.}  
The interface should seamlessly integrate context and character dynamics to capture their interplay in emotional simulations. This includes narrative-driven integration, enabling screenwriters to align characters' emotional states with evolving contexts. For example, a character’s fear may intensify in a dark forest, while tools highlight how contextual factors, such as weather or setting, influence emotions and allow users to adjust these connections. Real-time scenario testing facilitates dynamic exploration of character-context interactions. For instance, modifying a stormy setting can reveal its impact on a character’s anxiety, with features to replay and analyze scenarios for refinement. Collaborative interaction modes allow input from screenwriters and other production stakeholders, supporting shared visualizations to monitor emotional changes throughout the narrative and ensuring alignment with storytelling objectives. Additionally, immediate multimodal outputs could be a potential method to demonstrate the collective manipulation of context and character elements (as detailed in Section~\ref{sec:Multimodal}), further enhancing both creative processes and communication efficiency.}


\textcolor{black}{\subsubsection{Training Continuous Dynamic Emotion-Simulation Models}
Traditional AI models tend to simulate emotions as discrete states (e.g., happy, sad, angry)~\cite{assunccao2022overview, goyal2010toward}, but this simplification does not capture the dynamic nature of emotions. A more effective approach would involve continuous emotion models. For example, previous research has combined character stories with evolving ``fortune'' arcs to drive character and story development~\cite{chung2022talebrush}. However, this approach does not fully address continuous, emotion-driven storytelling. We propose leveraging continuous emotion modeling and affective computing to simulate gradual emotional shifts, while machine learning algorithms could be trained to recognize subtle cues indicative of character dynamics. This approach would allow emotions to be represented as cohesive trajectories within a multidimensional space. Such emotional continuity could aid in generating interconnected storylines and facilitate mutual reinforcement between the evolution of emotions and plot development.}

\textcolor{black}{\subsubsection{Improving Context-Aware Emotional Responses}
Screenwriters expect AI-generated content to adapt responsively to narrative context, enabling a deeper understanding of characters across scenarios. Although current AI can integrate context to some extent, managing complex environmental influences and providing rapid responses remains challenging. Future systems will need to incorporate real-time contextual feedback that adapts emotional outputs based on narrative progression and interpersonal cues. Potential strategies include integrating reinforcement learning and contextual attention mechanisms, which may involve decomposing complex settings and interpersonal dynamics within the story world into influential factors across multiple dimensions, such as relationships between characters and the story world background, interactions among different characters, and each character's inner emotions and conflicts. This approach would enable AI to respond adaptively to the narrative environment and the emotional development of characters in multi-dimensional ways, producing more contextually accurate emotional portrayals while ensuring that character emotions and reactions remain logically consistent across scenes and align with the overall story arc.}

%Screenwriters expect AI-generated content to adapt responsively to narrative context, enabling a deeper understanding of characters across various scenarios. Although current AI can integrate context to some extent, managing complex environmental influences and providing rapid responses remains challenging. Future systems will need to incorporate real-time contextual feedback that adapts emotional outputs based on narrative progression and interpersonal cues. Potential strategies include integrating reinforcement learning and contextual attention mechanisms, which may involve decomposing complex settings and interpersonal dynamics within the story world into influential factors across multiple dimensions, such as relationships between characters and the story world background, interactions among different characters, and each character's inner emotions and conflicts. This approach would enable AI to respond adaptively to the narrative environment and the emotional development of different characters in multi-dimensional ways, producing more contextually accurate emotional portrayals while ensuring that character emotions and reactions remain logically consistent across scenes and align with the overall story arc.}

\textcolor{black}{\subsubsection{Creating Emotionally-Driven Development Templates}
Drawing on established characterizations from existing high-quality screenwriting, creating development templates tailored to various character archetypes and relational dynamics could streamline the management of complex emotional relationships. Embedding these templates in AI tools could provide screenwriters with access to customized frameworks that support emotionally coherent interactions and archetypal emotional pathways. To avoid rigid template constraints, screenwriters could further customize these frameworks to suit their individual styles and preferences, thereby enhancing efficiency while preserving both personal creative features and emotional depth in character development.}

\textcolor{black}{\subsubsection{Promoting Cross-Disciplinary Collaboration}
Developing AI capabilities in emotional intelligence also requires extensive interdisciplinary collaboration. Insights from cognitive science on emotion theory, combined with narrative psychology’s studies of character development, can inform AI model designs that more closely approximate human emotional behavior. Collaboration among AI technical developers, screenwriters, and psychologists could further refine AI models, enabling more accurate understanding and representation of complex emotional expressions.}

\end{comment}

%According to the results in Section~\ref{sec:Allocation}, in stages such as the synopsis/outline and character development, future AI tools should prioritize improving user experience by offering streamlined interfaces that make simple AI tasks more accessible. In the dialogue and screenplay text stages, AI needs to demonstrate greater emotional and relational understanding to provide meaningful support to screenwriters.

%Results from Section 6.1 on the AI role of the ``actor'' reveal that screenwriters frequently manage complex emotional relationships and interactions. Although previous studies have explored emotion and relationships~\cite{goyal2010toward, tapaswi2014storygraphs}, they do not fully address screenwriters' needs. Overall, according to our findings (Sections~\ref{sec:capabilities} and 6.1), screenwriters expect future AI tools to manage dynamic emotional relationship networks, displaying emotional connections and simulating their evolution across scenarios. Screenwriters also anticipate AI tools that align behavior and dialogue with internal emotions and relationships, helping to create multidimensional characters through personalized role-playing functionalities. This relates to Weisz et al.'s design for mental models principle~\cite{10.1145/3613904.3642466}. 

%Future efforts should focus on developing intuitive AI systems tailored to screenwriters' emotional intelligence needs. We recommend that researchers consider the evolving emotions of characters in screenwriting, along with changes in story structure over time, the influence of different characters, and the impact of various environments on character development. This will require prioritizing interdisciplinary collaboration, integrating cognitive science, screenwriting, and AI to create advanced emotion-simulating models, along with diverse development templates and automated consistency checks to ensure both narrative coherence and emotional depth.

%For future efforts, we recommend that researchers consider the evolving emotions of characters in screenwriting, along with changes in story structure over time, the influence of different characters, and the impact of various environments on character development. This will require prioritizing interdisciplinary collaboration, integrating cognitive science, screenwriting, and AI to create advanced emotion-simulating models, along with diverse development templates and automated consistency checks to ensure both narrative coherence and emotional depth.


%According to the results in Section~\ref{sec:Allocation}, in stages such as the synopsis/outline, and character, future AI tools should prioritize improving user experience by offering streamlined interfaces that make simple AI tasks more accessible. In the dialogue and screenplay text stages, AI needs to demonstrate greater emotional and relational understanding to provide meaningful support to screenwriters. As noted in Section~\ref{sec:capabilities}, AI’s ability to generate nuanced, emotionally rich content requires improvement. Technological advancements should focus on managing complex narratives, character relationships, and human emotional depth. This relates to the design for mental models principle by Weisz et al., which emphasizes that AI systems should be intuitive and emotionally attuned to users' needs, facilitating their integration into creative processes~\cite{10.1145/3613904.3642466}. 

%Future efforts should focus on developing intuitive AI systems tailored to screenwriters' needs. We recommend that researchers consider the evolving emotions of characters in screenwriting, along with changes in story structure over time, the influence of different characters, and the impact of various environments on character development. This will likely require prioritizing interdisciplinary collaboration, integrating cognitive science, screenwriting, and AI to create advanced emotion-simulating models, along with diverse development templates and automated consistency checks to ensure both narrative coherence and emotional depth.

%Overall, future efforts should focus on developing intuitive AI systems tailored to screenwriters' needs. We recommend that researchers prioritize interdisciplinary collaboration, integrating cognitive science, screenwriting, and AI to create advanced emotion-simulating models, along with diverse development templates and automated consistency checks to ensure narrative coherence and emotional depth.

%In stages such as the synopsis, outline, and character development, future AI tools should prioritize improving user experience by providing streamlined interfaces that make simple AI tasks more accessible. In the dialogue and screenplay text stages, AI needs to demonstrate greater emotional and relational understanding to offer meaningful support to screenwriters. This aligns with the design for mental models principle outlined by Weisz et al., which emphasizes that AI systems should be designed to be more intuitive and emotionally attuned to users' needs, enhancing their integration into creative processes~\cite{10.1145/3613904.3642466}. Additionally, the accuracy and quality of AI-generated tasks are critical for maintaining narrative coherence. Enhancing AI functionality and usability in these areas can significantly reduce manual effort for screenwriters and reshape their workflow. We recommend that designers and developers of screenwriting tools focus on developing user-friendly AI tools with emotional intelligence capabilities, providing advanced character development templates and automated consistency checks to ensure both narrative coherence and emotional depth. As noted in Section~\ref{sec:capabilities}, AI's ability to generate nuanced, emotionally rich content requires improvement. Technological advancements should focus on managing complex narratives, character relationships, and human emotional complexity. Researchers should prioritize interdisciplinary collaboration, integrating cognitive science, creative writing, and AI to develop better emotion-simulating models. Future efforts should emphasize creating intuitive, user-friendly AI systems that meet screenwriters' specific needs.

%\subsection{AI for Usability and Emotional Intelligence Enhancement}
%In workflow stages such as the synopsis \& outline, and character development stages, future AI tools should focus on enhancing the user experience by offering streamlined interfaces that make simple AI tasks more accessible. Meanwhile, the dialogue and screenplay text stages require greater emotional and relational understanding from AI, which could provide more substantial assistance to screenwriters. This is consistent with the Design for Mental Models principle outlined by Weisz et al., which suggests that AI systems should be designed to be more comprehensible and emotionally attuned to users' needs, making them easier to integrate into creative processes~\cite{10.1145/3613904.3642466}. The accuracy and quality of these tasks are crucial for maintaining overall narrative coherence. Improving the functionality and usability of AI in these stages could significantly reduce the manual effort required from screenwriters and reshape their workflow. For example, tools could offer advanced character templates or automated consistency checks to ensure that character arcs align with the story's progression and dialogue, thereby reducing the potential for narrative inconsistencies. 

\begin{comment}
    

\subsection{AI for Flexibility and Role Adaptation}
%\textcolor{black}{The findings from Section 6 highlight screenwriters' expectations for AI systems to offer greater flexibility in simulating diverse ``actors'' and ``audiences.'' Kim et al. observed that creators aim to produce works resonating with varied audience groups, but achieving this remains challenging~\cite{10.1145/3613904.3642529}. This necessitates adaptable AI capable of managing complex character dynamics and anticipating audience responses. While current tools utilize big data for screenwriting and feedback analysis~\cite{pavel2015sceneskim, sanghrajka2017lisa, mutlu2020future}, screenwriters seek more nuanced character portrayals and predictive feedback to enhance emotional resonance. Enhancing AI adaptability not only reduces manual effort and reshapes workflows but also aligns with co-creation design principles~\cite{10.1145/3613904.3642466}. This section outlines strategies to achieve these advancements.}

%\textcolor{black}{The findings from Section 6 highlight screenwriters' need for AI systems with greater flexibility in simulating diverse ``actors'' and ``audiences.'' Kim et al. noted that creators strive to produce works that resonate with varied audiences, yet this remains a significant challenge~\cite{10.1145/3613904.3642529}. Addressing this requires adaptable AI capable of managing complex character dynamics and predicting audience responses. While current tools leverage big data for screenwriting and feedback analysis~\cite{pavel2015sceneskim, sanghrajka2017lisa, mutlu2020future}, screenwriters seek more nuanced character portrayals and predictive feedback to enhance emotional resonance. Improving AI adaptability reduces manual effort, reshapes workflows, and aligns with co-creation design principles~\cite{10.1145/3613904.3642466}. This section proposes strategies to advance these capabilities.}

\textcolor{black}{Section 6 highlights the need for AI systems with greater flexibility in simulating diverse ``actors'' and ``audiences.'' Kim et al. identified the challenge of creating works that resonate with varied audiences~\cite{10.1145/3613904.3642529}. Addressing this requires adaptable AI capable of managing complex character dynamics and predicting audience responses. While current tools leverage big data for screenwriting and feedback analysis~\cite{pavel2015sceneskim, sanghrajka2017lisa, mutlu2020future}, screenwriters seek more nuanced portrayals and predictive feedback to enhance emotional resonance. This section proposes strategies to address these needs.}

\subsubsection{\textcolor{black}{Training Models with Diverse Roles Data.}}
\textcolor{black}{Training models for diverse ``actor'' and ``audience'' roles use social computing methods, with high-quality films offering insights for ``actor'' roles. These datasets enable AI to adapt to diverse roles, supporting realistic character portrayals and tailored audience feedback in screenplay development.}

%\textcolor{black}{Training models for diverse ``actor'' and ``audience'' roles use social computing methods, with high-quality films offering additional insights for ``actor'' roles. These datasets enable AI to adapt to diverse roles, supporting realistic character portrayals and tailored audience feedback in screenplay development.}

\textcolor{black}{\textbf{Computing for Social Media Feedback.} Social computing methods analyze social media activity, fan discussions, sentiment trends, and audience-generated content to capture emotional responses and expectations across demographics. These methods categorize audience feedback by genre preferences, emotional resonance, and narrative expectations, tailoring recommendations to screenplay genres. Tracking dynamic audience feedback ensures alignment with societal trends. For ``actor'' roles, social data reveals audience perceptions of traits, relationships, and development, improving AI's ability to create characters that resonate with audiences.}
%Social computing methods analyze social media activity, fan discussions, sentiment trends, and audience-generated content to capture expectations and emotional responses across demographics. These methods categorize ``audience'' feedback by genre preferences, emotional resonance, and narrative expectations, tailoring recommendations to align with screenplay genres. Tracking dynamic audience feedback ensures relevance to societal trends. For ``actor'' roles, social data uncovers audience perceptions of traits, relationships, and development, enhancing AI's ability to create realistic characters that resonate with audiences.}

\textcolor{black}{\textbf{Analyzing High-Quality Films and Screenplays.} High-quality films and screenplays provide insights into multi-character dynamics, relationships, and evolution. By analyzing traits, backstories, and interactions, AI systems can identify patterns and create robust datasets to train for realistic and coherent character portrayals. This supports screenwriters in developing multi-dimensional characters within interconnected narratives.}

\subsubsection{\textcolor{black}{Interactive Solutions for Diverse Roles}}
\textcolor{black}{Interactive solutions like modular input and multi-agent systems provide structured and dynamic methods for adapting ``actor'' and ``audience'' roles in screenwriting.}

\textcolor{black}{\textbf{Modular Input.}}  
\textcolor{black}{For ``actor'' roles, labeled modular components form a character map, including traits, backstories, relationships, and emotional states. Modular interfaces allow screenwriters to refine profiles dynamically, ensuring smooth role transitions~\cite{ye2023mplug}. For ``audience'' roles, modular inputs define audience segments based on demographics, preferences, or emotional metrics. These tools integrate with ``actor'' modules, enabling simulations of audience responses to roles, development, plot points, or themes.}
%Modular input systems structure ``actor'' and ``audience'' roles. }

\textcolor{black}{\textbf{Multi-Agent Systems.}}  
\textcolor{black}{For ``actor'' roles, characters act as independent agents interacting dynamically within a shared story world~\cite{10.1145/3586183.3606763}. Screenwriters can adjust interactions and preview narrative changes, fostering creative inspiration. For ``audience'' roles, agents simulate diverse demographics, providing feedback on how narrative changes affect engagement. Simulations explore whether specific elements spark discussions or increase attention, helping screenwriters balance storytelling to resonate with diverse groups or encourage thematic debates.}
%Multi-agent systems enable dynamic simulations for ``actor'' and ``audience'' roles. 

%\textcolor{black}{The findings from Section 6 highlight screenwriters' expectations for AI systems to offer greater flexibility in simulating diverse ``actors'' and ``audiences.'' Kim et al. observed that creators aim to produce works resonating with varied audience groups, but achieving this remains challenging~\cite{10.1145/3613904.3642529}. This necessitates adaptable AI capable of managing complex character dynamics and anticipating diverse audience responses. While current tools utilize big data for screenwriting and audience feedback analysis~\cite{pavel2015sceneskim, sanghrajka2017lisa, mutlu2020future}, screenwriters seek more nuanced character portrayals and predictive feedback to enhance emotional resonance. Enhancing AI adaptability not only reduces manual effort and reshapes workflows but also aligns with co-creation design principles~\cite{10.1145/3613904.3642466}. This section outlines strategies to achieve these advancements.}

%\textcolor{black}{The findings from Section 6 emphasize screenwriters' expectations for AI systems to provide greater flexibility in simulating diverse ``actors'' and ``audiences.'' Previous research by Kim et al. highlighted that creators have consistently strived to produce works that resonate with diverse audience groups and elicit positive feedback, yet this remains a persistent challenge~\cite{10.1145/3613904.3642529}. This underscores the need for adaptable AI systems capable of managing complex character dynamics and anticipating varied audience responses. While current tools leverage big data for screenwriting and audience feedback analysis~\cite{pavel2015sceneskim, sanghrajka2017lisa, mutlu2020future}, screenwriters demand more nuanced character portrayals and predictive feedback to enhance emotional resonance. Improving AI's adaptability not only reduces manual workload and reshapes workflows but also aligns with co-creation design principles~\cite{10.1145/3613904.3642466}. This section details several strategies to achieve these advancements.} 

%\textcolor{black}{The findings from Section 6 highlight screenwriters' expectations for AI to provide flexibility in simulating diverse ``actors'' and ``audiences.'' Managing character dynamics and anticipating audience responses require adaptable AI systems. While current tools utilize big data for screenwriting and feedback analysis~\cite{pavel2015sceneskim, sanghrajka2017lisa, mutlu2020future}, screenwriters seek more nuanced portrayals and predictive feedback for emotional resonance. Enhancing AI's flexibility can reduce workload, reshape workflows, and align with co-creation design principles~\cite{10.1145/3613904.3642466}. This section outlines strategies for achieving these advancements.}
%\textcolor{black}{The findings from Section 6 emphasize the evolving role of AI in supporting screenwriters by providing flexibility in simulating diverse roles, such as different ``actors'' and ``audiences''. Screenwriters often manage complex character dynamics and anticipate diverse audience responses, requiring AI systems to adapt seamlessly to these varying demands. Current tools leverage big data for screenwriting enhancements~\cite{pavel2015sceneskim, sanghrajka2017lisa} and analyze audience feedback in relation to market trends~\cite{mutlu2020future}. However, screenwriters expect AI to go beyond these capabilities by offering more nuanced character portrayals and predictive feedback for emotional resonance across diverse audience segments. Enhancing AI's flexibility and role adaptation could significantly reduce screenwriters' manual workload while reshaping workflows. These advancements align with the principles of co-creation in design~\cite{10.1145/3613904.3642466}, promoting deeper AI integration into creative processes. This section outlines the necessary model training processes and interactive technical solutions to achieve these goals.}

%\textcolor{black}{Diverse ``actor'' roles and ``audience'' roles can be trained using social computing methods, complemented by analyzing relationship patterns among multi-characters in high-quality films for multi ``actor'' roles. This approach enables a deeper understanding of how diverse ``actor'' roles interact and develop within a shared story world, uncovering the connections between their narrative progression and underlying relationship patterns, serving as a dataset for training AI to quickly understand different ``actor'' roles.}

%\textcolor{black}{\textbf{Social Computing Methods for Diverse ``Actor'' and ``Audience'' Roles.} Social computing methods offer valuable insights for training both diverse ``actor'' roles and ``audience'' roles. By analyzing social media activity, fan discussions, sentiment trends, and audience-generated content, AI systems can capture the expectations and emotional responses of diverse demographics. These methods enable the categorization of ``audience'' feedback by genre preferences, emotional resonance, and narrative expectations, thereby tailoring AI-generated recommendations for various screenplay genres. For ``actor'' roles, social computing data can reveal audience perceptions of character traits, relationships, and development, enhancing AI's ability to align character portrayals with audience expectations.}
\end{comment}

%\textcolor{black}{Interactive technical solutions, such as modular input systems and multi-agent systems, provide structured and dynamic methods for adapting both ``actor'' and ``audience'' roles in screenwriting.}

\begin{comment}
    
\subsubsection{\textcolor{black}{Training Models with Diverse Roles Data.}}
\textcolor{black}{Training models for diverse ``actor'' and ``audience'' roles leverage social computing methods, with high-quality films offering additional insights for ``actor'' roles. These approaches create datasets that enable AI to adapt to diverse roles, supporting realistic character portrayals and tailored audience feedback in screenplay development.}

\textcolor{black}{\textbf{Social Computing Methods for Diverse ``Actor'' and ``Audience'' Roles.} Social computing methods provide valuable insights for training both diverse ``actor'' roles and ``audience'' roles. By analyzing social media activity, fan discussions, sentiment trends, and audience-generated content, AI systems can capture the expectations and emotional responses of diverse demographic groups. These methods enable the categorization of ``audience'' feedback based on genre preferences, emotional resonance, and narrative expectations, tailoring AI-generated recommendations to align with various screenplay genres. Additionally, tracking dynamic audience feedback over time on social media could ensure that ``audience'' data remains relevant to current societal trends. For ``actor'' roles, social computing data can uncover audience perceptions of character traits, relationships, and development, enhancing AI's ability to craft character portrayals that meet audience expectations effectively.}

\textcolor{black}{\textbf{Analyzing High-Quality Films and Screenplays for Diverse ``Actor'' Roles.} High-quality films and screenplays provide a complementary approach for training diverse ``actor'' roles, particularly in understanding relationship patterns among multi-characters. By analyzing character traits, backstories, and interactions within the same story world, AI systems can identify common multi-character dynamics and patterns. This analysis offers insights into how characters evolve and interact, forming a robust dataset to train AI for realistic and coherent character portrayals. These findings enable AI to adapt to varying ``actor'' roles and support screenwriters in crafting multi-dimensional characters within interconnected narratives.}

\subsubsection{\textcolor{black}{Interactive Solutions for Diverse Roles}}
\textcolor{black}{Interactive solutions, such as modular input and multi-agent systems, provide structured and dynamic methods for adapting both ``actor'' and ``audience'' roles in screenwriting.}

\textcolor{black}{\textbf{Modular Input.}}
\textcolor{black}{Modular input systems provide a structured approach for adapting both ``actor'' roles and ``audience'' roles. For ``actor'' roles, modular character inputs can be organized into labeled components, forming a character label map that includes essential details such as personality traits, backstories, relationships, plot points, and emotional states. Screenwriters can use modular editing interfaces to input, adjust, and refine character profiles dynamically, ensuring seamless transitions between different actor roles~\cite{ye2023mplug}.} \textcolor{black}{Similarly, for ``audience'' roles, modular inputs can define distinct audience segments based on demographic data, genre preferences, or emotional resonance metrics. Screenwriters can use these tools to input hypothetical audience profiles or select predefined audience modules. These inputs should integrate with different ``actor'' modular components to simulate and predict how various audience groups might respond to different ``actor'' roles, character development, plot points, or thematic elements.}

\textcolor{black}{\textbf{Multi-Agent Systems.}}
\textcolor{black}{Multi-agent systems enhance flexibility by enabling dynamic interactions and simulations for both ``actor'' and ``audience'' roles. For ``actor'' roles, each character can be represented as an independent agent within a shared story world, capable of interacting dynamically with others~\cite{10.1145/3586183.3606763}. Screenwriters can include or exclude different ``actors'' from specific scenes and preview how different interactions alter narrative outcomes. This approach fosters dynamic multi-character transformations and relationships, offering valuable creative inspiration for screenwriters.} \textcolor{black}{For ``audience'' roles, multi-agent systems can simulate interactions between various audience segments to provide detailed feedback on screenplay elements. For instance, agents could represent different audience demographics, simulating how narrative changes affect emotional engagement across segments. Such simulations could explore whether certain narrative elements spark debates among specific groups, potentially increasing audience attention and interest in the content. This enables screenwriters to balance storytelling elements, ensuring resonance with diverse audience groups or encouraging discussions on particular themes.}
\end{comment}

%\textcolor{black}{\textbf{Analyzing High-Quality Films for Diverse ``Actor'' Roles.}  High-quality films provide a complementary approach for training diverse ``actor'' roles, particularly in understanding relationship patterns among multi-characters. By analyzing character traits, backstories, and interactions within the same story world, AI systems can identify common multi-character dynamics and relationship patterns. This analysis offers insights into how characters evolve and interact, forming a robust dataset to train AI for realistic and coherent character portrayals. These findings enable AI to quickly adapt to varying ``actor'' roles and support screenwriters in crafting multi-dimensional characters within interconnected narratives.}

%\textcolor{black}{\textbf{Training for Diverse ``Actor'' Roles} Training AI for dynamic ``actor'' role adaptation involves collecting and analyzing data on character traits, backstories, relationships, and emotional states. Social computing methods, such as examining character-related discussions on online forums, fan communities, and social media platforms, can offer valuable insights into audience expectations for character development. These datasets can enhance the AI model’s understanding of diverse personality traits, emotional trajectories, and interactions across different scenarios. For example, user-generated content, such as fan theories or alternative character interpretations, can help train models to generate nuanced portrayals that capture the complexity of character dynamics. Additionally, analyzing high-quality films to examine differences in personality traits, personal experiences, and relational dynamics among characters within the same story can uncover common multi-character features and relationship patterns. These findings can inform training datasets for developing multi-actor story frameworks. By leveraging such patterns, AI systems can provide recommendations or generate structured frameworks to assist screenwriters in efficiently coordinating relationships among multiple characters in a story. This approach facilitates consistent character interactions and supports creative storytelling involving diverse or interconnected actors.}

%\textcolor{black}{\textbf{Training for Diverse ``Audience'' Roles.} To address the need for more predictive audience feedback, AI systems require a comprehensive understanding of audience diversity. Social computing methods, such as analyzing social media activity, sentiment analysis within specific communities, surveys, and interviews, can gather data from diverse demographic and cultural groups. By categorizing audience feedback based on genre preferences, emotional resonance, and narrative expectations, AI models can be trained to provide tailored recommendations across various screenplay genres. However, developers must account for potential cultural biases within these datasets and establish safeguards to avoid perpetuating stereotypes, striking a balanced approach to ensure inclusivity and fairness.}

\begin{comment}

\textcolor{black}{Findings from Section 6.1 on the role of AI as an ``actor'' suggest that screenwriters frequently navigate the complex dynamics of various characters within a story world. Specifically, Section 6.1.1 emphasizes the importance of accurately representing character traits, while Section 6.1.2 highlights users' demands for diverse ``actor'' roles, reflecting screenwriters' expectations for both the reliability of information provided by AI and smooth transitions between roles. Additionally, while current AI tools leverage big data to enhance screenwriting~\cite{pavel2015sceneskim, sanghrajka2017lisa}, such as by analyzing audience feedback in relation to market trends~\cite{mutlu2020future}, screenwriters further expect AI to simulate different types of ``audience'' feedback (Section 6.2), which could help them achieve more emotional resonance. Enhancing AI functionality and flexibility in these areas could significantly reduce the manual workload for screenwriters and reshape their workflow. This aligns with the principle of co-creation in design proposed by Weisz et al.~\cite{10.1145/3613904.3642466}, which underscores the importance of deeply integrating AI into creative processes. To achieve these goals, we recommend that designers and developers consider implementing modular information components that allow users to quickly input various character identities and traits or employing multi-agent systems to enhance flexibility and accuracy in character portrayals tailored to different ``actor'' role requirements. Additionally, gaining a deeper understanding of audience needs may enable the provision of more predictive feedback, assisting in meeting diverse ``audience'' role expectations.}

\textcolor{black}{\subsubsection{Applying Modular Components or Multi-Agents for Shifting Various Actor Roles}
In conjunction with our findings in Section 4, character modular information components can be categorized into labeled elements, forming a character label map that encompasses essential information, personality traits, backstory, relationships, specific plot points, emotional states, and more. Modular editing and input would enable AI to quickly understand each character and facilitate transitions and dynamic adjustments across different roles~\cite{ye2023mplug}.}

\textcolor{black}{Moreover, multi-agent systems offer a promising approach for efficiently switching between diverse roles. By configuring numerous distinct characters and their respective designs within a single story world~\cite{10.1145/3586183.3606763}, screenwriters can selectively include or exclude characters as needed in specific scenes, allowing for a preview of varied outcomes and storylines that emerge from different character interactions within the same story world or specific scenarios. This approach enables convenient switching between actor roles. Furthermore, integrating multi-agent systems can foster more nuanced relationships and developments among characters, transforming previously static or linear interactions into dynamic, multi-character transformations, thereby providing screenwriters with additional creative inspiration.}

\textcolor{black}{\subsubsection{Understanding Audience Needs for Shifting Various Audience Roles}  
Screenwriters highlight the importance of leveraging big data and emotion recognition to adapt plots and character development for deeper emotional resonance with diverse audiences. This demand for varied feedback from different audience groups requires designers to develop a nuanced understanding of target audiences across screenplay genres, enabling relevant audience group recommendations and collective feedback. Approaches to achieving this might include analyzing social media comments across different communities or conducting surveys with distinct audience segments, such as interviews or questionnaires. However, developers must remain attentive to potential cultural biases in this context, ensuring that data usage is managed within appropriate and reasonable boundaries.}

\end{comment}

%Findings from Section 6.1 on the AI role as the ``actor'' indicate that screenwriters often navigate complex character dynamics. Specifically, Section 6.1.1 emphasizes the importance of accuracy in characteristics, and Section 6.1.2 highlights user demands for different ``actor'' roles, reflecting screenwriters' expectations for the reliability of the information provided by AI and flexible transitions between different roles. Enhancing AI functionality and flexibility in these areas could significantly reduce manual effort for screenwriters and reshape their workflow. This is related to the design for co-creation principle by Weisz et al.~\cite{10.1145/3613904.3642466}, which emphasizes the deeper integration of AI into creative processes.

%To achieve this, we recommend that designers and developers consider creating modular information components that allow users to quickly input different character identities and traits, thereby enhancing the coherence, flexibility, and accuracy of screenplay content. Specifically, character information can be categorized into various element labels, forming a character label map that includes basic information, personality traits, backstory, relationships, specific plot points, emotional states, etc. Modular editing and input would enable AI to quickly comprehend the character and facilitate transitions and dynamic adjustments between different roles\cite{ye2023mplug}.

%Additionally, while current AI tools leverage big data to optimize screenwriting~\cite{pavel2015sceneskim, sanghrajka2017lisa}, such as analyzing audience feedback based on market trends data~\cite{mutlu2020future}, screenwriters expect AI to simulate diverse ``audience'' feedback (Section 6.2), using big data and emotion recognition to adjust plots and characters for emotional resonance. This demand requires designers to understand the target audience for different types of screenplays, providing relevant audience group recommendations and collective feedback. However, developers should also carefully consider potential cultural biases in AI algorithms in this context.

\begin{comment}
    

\subsection{AI for Multithreaded and Multimodal Generation} \label{sec:Multimodal}

Results in Section~\ref{sec:Allocation} highlight the need for AI to manage a broader range of tasks across goal \& idea generation, as well as story structure \& plot development stages (Table \ref{tab:Task Allocation}). \textcolor{black}{Furthermore, Section~\ref{sec:Presentation} and 6.1.1 underscore the value of immersive representations in enhancing the creative process. Based on these findings, we propose two strategies.}

\textcolor{black}{\textbf{Intuitive Interaction with Narrative Branches.}}  
Participants emphasized the need for AI to assist in generating and managing diverse narrative outputs, \textcolor{black}{which is essential for exploring creative possibilities and enabling informed decision-making.} Future features that support the creation of multiple narrative branches would allow screenwriters to experiment with different plot outcomes and understand the broader implications of changes in specific story elements. Providing real-time previews of these branches would enable intuitive adjustments, improving both the efficiency and effectiveness of the screenwriting workflow.

\textcolor{black}{\textbf{Immersive Visualization of Characters and Scenes.}}  
As noted in Section~\ref{sec:Presentation}, visualization tools are crucial for translating abstract ideas into tangible representations. Additionally, Section 6.1.1 highlights the potential of immersive multimodal technologies, such as extended reality (XR) or 3D projection, to enhance this process further. By integrating these technologies, AI tools can offer screenwriters an immersive experience, visualizing characters and scenes in real time. This capability would provide valuable insights into character dynamics, spatial relationships, and emotional pacing, significantly enriching narrative development and the overall screenwriting process.

\end{comment}

%Based on the results in Section~\ref{sec:Allocation}, during the goal-setting and idea generation stages, as well as the story structure \& plot development stages, participants expressed a need for AI to manage a broader range of tasks (Table \ref{tab:Task Allocation}) to facilitate the exploration of diverse creative possibilities and support early, informed decision-making. These needs align with the generative variability design principle proposed by Weisz et al.~\cite{10.1145/3613904.3642466}.

%\textcolor{black}{\textbf{Diverse Narrative Branches.}}  
%\textcolor{black}{Participants expressed the need for AI to assist in managing and generating diverse narrative outputs.} Developers should prioritize features that enable the creation of multiple narrative branches. This functionality would allow screenwriters to explore and understand different plot outcomes and observe how changes in one aspect of the story might impact the overall narrative. Providing real-time previews of these story options would empower screenwriters to make adjustments quickly and intuitively, thus enhancing both the efficiency and effectiveness of their creative process.

%\textcolor{black}{\textbf{Immersive Visualization of Characters and Scenes.}}  

%As noted in Section~\ref{sec:Presentation}, visualization tools offer a more tangible representation of abstract concepts. Furthermore, as discussed in Section 6.1.1, immersive, multimodal technologies, such as extended reality (XR) or 3D projection, can further enhance this process. By integrating these technologies, AI tools can offer screenwriters an immersive experience, enabling them to craft their narratives in innovative ways. \textcolor{black}{These tools could assist in visualizing characters and scenes as if unfolding in real time, providing a deeper understanding of character dynamics, spatial relationships, and emotional pacing in the future.}

%Additionally, visualization plays a crucial role in helping screenwriters evaluate various creative directions. 

%Based on the results in Section~\ref{sec:Allocation}, during the goal \& idea generation stages, as well as the story structure \& plot development stages, participants expressed a need for AI to manage a broader range of tasks (Table \ref{tab:Task Allocation}) to help them explore diverse creative possibilities more efficiently and make informed decisions early in the process. This aligns with the design for generative variability principle proposed by Weisz et al., which emphasizes that generative AI should assist users in managing and producing diverse outputs~\cite{10.1145/3613904.3642466}. To address this need, when designing AI tools for these stages, developers should focus on features that enable the generation of multiple narrative branches. This would allow screenwriters to visualize different plot outcomes and see how changes in one area of the story might affect the overall narrative. Providing previews of these story options in real-time would give screenwriters the ability to make adjustments quickly and intuitively, enhancing both the efficiency and effectiveness of their creative process.

%Additionally, visualization plays a crucial role in helping screenwriters evaluate different creative directions. As noted in Section~\ref{sec:Presentation}, visualization tools can provide a more tangible representation of abstract ideas. Furthermore, as discussed in Section 6.1.1, immersive, multimodal presentations, such as those involving extended reality (XR) or 3D projection technologies, can further enhance this process. By integrating such technologies, AI tools can offer screenwriters a more immersive experience, allowing them to create their narratives in new ways. These could help screenwriters visualize characters and scenes as if they were unfolding in real-time, providing a deeper understanding of character dynamics, spatial relationships, and emotional pacing.  

%Results from Section 6.1 on the AI role of the ``actor'' reveal that screenwriters frequently manage complex character relationships and emotional interactions. Section 6.1.2 highlights user demands for different roles, and Section 6.1.1 emphasizes the importance of accuracy in character traits, reflecting screenwriters' expectations for flexible transitions between different roles and the reliability of the information provided by AI. This is related to the design for co-creation principle by Weisz et al.~\cite{10.1145/3613904.3642466}, which emphasizes the deeper integration of AI into creative processes.

%Although previous studies have explored emotion and relationship visualization~\cite{goyal2010toward, tapaswi2014storygraphs} and scene generation~\cite{won2014generating}, they do not fully address screenwriters' needs. According to our findings (Sections~\ref{sec:capabilities} and 6.1), screenwriters expect future AI tools to manage dynamic character relationship networks, displaying emotional connections and simulating their evolution across scenarios. Screenwriters also anticipate AI tools that align behavior and dialogue with internal emotions and relationships, helping to create multidimensional characters through personalized role-playing functionalities. Enhancing AI functionality and flexibility in these areas could significantly reduce manual effort for screenwriters and reshape their workflow. To achieve this, we recommend that designers and developers consider creating modular information components that allow users to quickly input different character identities and traits, thereby enhancing the coherence, flexibility, and accuracy of screenplay content. Specifically, character information can be categorized into various element labels, forming a character label map that includes basic information, personality traits, backstory, relationships, specific plot points, emotional states, etc. Modular editing and input would enable AI to quickly comprehend the character and facilitate transitions and dynamic adjustments between different roles.

%Additionally, while current AI tools leverage big data to optimize screenwriting~\cite{pavel2015sceneskim, sanghrajka2017lisa}, such as analyzing audience feedback and market trends~\cite{mutlu2020future}, screenwriters expect AI to simulate diverse ``audience'' feedback (Section 6.2), using big data and emotion recognition to adjust plots and characters for emotional resonance. This demand requires designers to understand the target audience for different types of screenplays, providing relevant audience group recommendations and collective feedback. However, developers should also carefully consider potential cultural biases in AI algorithms in this context.

%Results from Section 6.1 on the AI role of ``actor'' reveal that screenwriters frequently manage complex character relationships and emotional interactions. While studies have explored emotion and relationship visualization~\cite{goyal2010toward, tapaswi2014storygraphs} and scene generation~\cite{won2014generating}, they do not fully meet screenwriters' needs. According to our findings (Sections~\ref{sec:capabilities} and 6.1), screenwriters expect future AI tools to create and manage dynamic character relationship networks, displaying emotional connections and simulating their evolution across different scenarios. Screenwriters anticipate AI tools that, by aligning external behavior and dialogue with a character's internal emotion and relationships, can help create more multidimensional characters through personalized role-playing functionalities. Additionally, while current AI tools use big data to optimize screenwriting~\cite{pavel2015sceneskim, sanghrajka2017lisa}, such as analyzing audience feedback and market trends~\cite{mutlu2020future}, screenwriters want AI to simulate diverse audience feedback (Section 6.2) , using big data and emotion recognition technologies to adjust plot and characters for emotional resonance. This aligns with the design for co-creation principle by Weisz et al.~\cite{10.1145/3613904.3642466}.

%Results from Section 6.1 on the AI role of ``actor'' reveal that screenwriters frequently manage complex character relationships and emotional interactions. Although studies have explored emotion and relationship visualization~\cite{goyal2010toward, tapaswi2014storygraphs} and scene generation~\cite{won2014generating}, they do not fully address screenwriters' needs. According to our findings (Sections~\ref{sec:capabilities} and 6.1), screenwriters expect future AI tools to manage dynamic character relationship networks, displaying emotional connections and simulating their evolution across scenarios. Screenwriters also anticipate AI tools that align behavior and dialogue with internal emotions and relationships, helping to create multidimensional characters through personalized role-playing functionalities. Additionally, while current AI tools leverage big data for optimizing screenwriting~\cite{pavel2015sceneskim, sanghrajka2017lisa}, such as analyzing audience feedback and market trends~\cite{mutlu2020future}, screenwriters want AI to simulate diverse audience feedback (Section 6.2), using big data and emotion recognition to adjust plot and characters for emotional resonance. This aligns with the design for co-creation principle by Weisz et al.~\cite{10.1145/3613904.3642466}.

%We suggest that designers focus on developing AI tools that allow flexible role adaptation, enabling screenwriters to simulate complex emotional interactions and dynamically manage character relationships. Additionally, AI tools should simulate diverse audience reactions and provide feedback, supporting screenplay refinement based on audience insights. Furthermore, designers should carefully consider potential cultural biases in AI algorithms in this context.

%We suggest that designers pay more attention on developing AI tools that allow flexible role adaptation, enabling screenwriters to simulate complex emotional interactions and dynamically manage character relationships. Additionally, AI tools should simulate diverse audience reactions and provide emotional feedback, supporting screenplay refinement based on audience insights. Besides, designers should also consider the potential cultural biases in AI algorithms in this context.

%Results from Section 7.1 on the AI role of ``actor'' reveal that screenwriters often deal with complex character relationships and emotional interactions. Although existing studies have explored emotion and relationship visualization~\cite{goyal2010toward, tapaswi2014storygraphs} and assisted scene generation~\cite{won2014generating}, they still fall short of fully meeting screenwriters' needs. According to our findings in Sections~\ref{sec:capabilities} and 7.1, screenwriters expect future AI tools to support the creation and management of dynamic character relationship networks, not only displaying emotional connections between characters but also simulating the evolution of these relationships in different scenarios. Screenwriters desire AI tools that can provide personalized role-playing functionalities, enabling them to generate behavior and dialogue that align with a character's internal goals, external behaviors, and emotional relationships. This would assist screenwriters in developing characters with greater depth and multidimensionality.

%Furthermore, while existing AI tools already leverage big data analysis to optimize screenwriting~\cite{pavel2015sceneskim, sanghrajka2017lisa}, such as by analyzing audience feedback, market trends, and successful case studies to provide data-driven creative suggestions for better market adaptability~\cite{mutlu2020future}, screenwriters in Section 7.2 expressed a desire for AI to simulate audience feedback across diverse backgrounds, cultures, and age groups in its ``audience'' role. Screenwriters envision AI tools that utilize big data and emotion recognition technologies to simulate the audience's emotional responses to scripts, allowing for adjustments in plot and character to achieve broader emotional resonance. This aligns with the Design for Co-Creation principle emphasized by Weisz et al., which suggests that AI should not only generate content but also empower creators to actively participate in and influence the generation process, resulting in more emotionally resonant and personalized content~\cite{10.1145/3613904.3642466}. Additionally, AI tools should be capable of analyzing and optimizing the emotional tension of scripts, automatically detecting emotional highs and lows, and providing adjustment suggestions to ensure a coherent and engaging emotional experience throughout the story. While existing tools~\cite{chung2022talebrush} can generate stories based on user-drawn emotion curves, current screenwriters are more focused on visualizing emotions in existing scripts and gathering further feedback from audiences to enhance coherence and appeal. Implementing this functionality requires careful attention to potential cultural biases, which may subtly influence script development through AI algorithms.

%\subsection{Integrating AI Throughout the Screenwriting Workflow}
%Our findings suggest that the screenwriting process encompasses multiple stages (Section 5.1). Section~\ref{sec:Allocation} highlights the varying AI needs across these different workflow stages, underscoring the necessity for a comprehensive approach that considers how AI can effectively integrate stage-specific functions. For instance, during the goal \& idea generation stages, as well as the story structure \& plot development stages, participants expressed a need for AI to manage a broader array of tasks. These tasks include generating a wider range of potential outputs and narrative branches and providing quicker previews of these outputs. Such capabilities would enable screenwriters to explore diverse creative possibilities more efficiently and make informed decisions early in the process. Therefore, AI tools designed for these stages should be equipped with multi-path creative generation capabilities that cater to the complex demands of screenwriters.

%Meanwhile, in workflow stages where AI tasks are currently less prevalent, such as the synopsis \& outline stage and the character development stage, future AI tools could focus on enhancing the user experience by offering streamlined interfaces that make simple AI tasks more accessible. Additionally, the dialogue stage and screenplay text stage require more emotional and relationship understanding from AI, which could provide greater assistance to screenwriters. The accuracy and quality of these tasks are critical to overall narrative coherence. Improving the functionality and usability of AI in these stages could significantly reduce the manual effort required from screenwriters and alter their workflow. For example, tools could offer advanced character templates or automated consistency checks to ensure that character arcs align with the story's progression and dialogue, thereby reducing the potential for narrative inconsistencies.

%Previous AI tools have typically focused on specific phases, such as character development~\cite{10.1145/3172944.3172972, kapadia2016canvas} and story plot generation~\cite{kim2017visualizing, poulakos2015towards, bartindale2016tryfilm}, although they have not fully met screenwriters' needs according to our study. Furthermore, these tools often result in a lack of continuity across the entire workflow, creating gaps that hinder the seamless integration of AI into the complete screenwriting process. Our research shows that screenwriters have attempted to incorporate AI at many stages, albeit with mixed results. Therefore, it is essential to consider that future AI tools should support the complete workflow, requiring a holistic approach that ensures smooth transitions from goal \& idea generation to screenplay text writing. This integration should aim to eliminate disruptions caused by the current stage-specific tool limitations. By developing AI systems that can seamlessly guide screenwriters through the entire process, from initial brainstorming and structuring to the final screenplay, creators can benefit from a more cohesive and efficient workflow that leverages AI's strengths at every stage.


%\subsection{Leveraging Flexibility Through AI}

%Our study suggests that there is an expectation from participants for AI flexibility and capabilities, especially in enhancing dynamic character relationships and simulating audience feedback.

%Results from Section 7.1 reveal that screenwriters often deal with complex character relationships and emotional interactions. Although existing studies have explored emotion and relationship visualization~\cite{goyal2010toward, tapaswi2014storygraphs} and assisted scene generation~\cite{won2014generating}, they still fall short of fully meeting screenwriters' needs. According to our findings in Section~\ref{sec:capabilities} and 7.1, screenwriters expect future AI tools to support the establishment and management of dynamic multi-character relationship networks, not only displaying emotional connections between characters but also simulating the evolution of these relationships in different scenarios. This would ensure consistency between characters' internal goals and external behaviors while offering plot suggestions based on emotions and goals to better develop character depth and multidimensionality. Additionally, as demonstrated by the feedback in Section 7.1, screenwriters desire AI tools that can provide personalized role-playing functionalities, allowing them to generate behavior and dialogue that align with a character's defined internal goals, external behaviors, and emotional relationships. Such AI-driven character generation and interaction, based on real-world data, can enhance the authenticity and complexity of characters, helping screenwriters to better understand and develop their characters, ultimately leading to higher-quality screenplay.

%Furthermore, while existing AI tools already leverage big data analysis to optimize screenwriting~\cite{pavel2015sceneskim, sanghrajka2017lisa}, such as by analyzing audience feedback, market trends, and successful case studies to provide data-driven creative suggestions for better market adaptability~\cite{mutlu2020future}, screenwriters in Section 7.2 expressed a desire for AI to simulate audience feedback across diverse backgrounds, cultures, and age groups. Screenwriters envision AI tools that utilize big data and emotion recognition technologies to simulate audience's emotional responses to scripts, allowing for adjustments in plot and character to achieve broader emotional resonance. Additionally, AI tools should be capable of analyzing and optimizing the emotional tension of scripts, automatically detecting emotional highs and lows, and providing adjustment suggestions to ensure a coherent and engaging emotional experience throughout the story. While existing tools~\cite{chung2022talebrush} can generate stories based on user-drawn emotion curves, current screenwriters are more focused on visualizing emotions in existing scripts and gathering further feedback from audiences to enhance coherence and appeal. In implementing this functionality, particular attention should be paid to potential cultural biases, which may subtly influence the direction of script development through AI algorithms.


\section{Discussion}

%Existing research highlights both the opportunities and challenges of AI in creative fields~\cite{10.1145/3613904.3642731, 10.1145/3613904.3642854, 10.1145/3613904.3642105}. Given the increasing integration of AI into these domains, these challenges are likely to persist and, if we do not pay attention to, could lead to more severe long-term consequences, which should be taken seriously by relevant stakeholders. Our qualitative study examined the impact of AI in the screenwriting domain, providing specific insights for screenwriters, researchers, and stakeholders to bridge the gap between AI technology and creative practice. Specifically, we discuss the current negative impacts and propose potential suggestions through the lenses of ethical limitations, agency and identity, and collaborative training.

In this section, we propose aspects that future AI tools in screenwriting should focus on to enhance human-AI co-creation. Specifically, we suggest potential solutions through the lenses of agency, identity, and training. We also acknowledge the limitations of our research and outline potential directions for future studies.

%Existing research highlights both the opportunities and challenges of AI in creative fields. As AI continues to be integrated into these domains~\cite{10.1145/3613904.3642731, 10.1145/3613905.3650929}, these challenges are likely to persist, and failing to address them could result in more severe long-term consequences. Therefore, our qualitative study examines the impact of AI in the screenwriting domain, providing specific insights for screenwriters, researchers, and stakeholders to bridge the gap between AI technology and creative practice. Specifically, we propose potential solutions through the lenses of agency, identity, and training.

%Existing research highlights both the opportunities and challenges of AI in creative fields~\cite{10.1145/3613904.3642731, 10.1145/3613904.3642854, 10.1145/3613904.3642105}. While prior work has proposed design principles for the application of AI~\cite{10.1145/3613904.3642466}, our qualitative study explored the impact of AI in the screenwriting domain, offering specific insights for screenwriters, researchers, and stakeholders to bridge the gap between AI technology and creative practice. Specifically, we provide suggestions for fostering human-AI co-creation in screenwriting through the following three key aspects.

%We conducted a qualitative study to assess AI's impact on screenwriting and provide suggestions for screenwriters, researchers, and stakeholders to bridge the gap between technology and practice. Overall, future efforts should focus on collaborating to develop a framework for the responsible use of AI~\cite{10.1145/3613904.3642466}, ensuring that AI enhances rather than diminishes human creativity.

%We conducted a qualitative study to assess AI's impact on screenwriting and offer suggestions for screenwriters, researchers, and stakeholders to bridge the gap between technology and practice. Overall, future work should collaborate to develop a framework for the responsible use of AI~\cite{10.1145/3613904.3642466}, ensuring that AI enhances rather than diminishes human creativity.

%We conducted a qualitative study to assess AI's impact on screenwriting and provide suggestions for screenwriters, researchers, and all stakeholders to bridge the gap between technology and practice. Overall, stakeholders, including screenwriters, researchers, and industry leaders, should collaborate to create a framework for AI's responsible use~\cite{10.1145/3613904.3642466}, ensuring that AI enhances rather than diminishes human creativity.

%\subsection{Enhancing AI Capabilities and Usability} \label{sec:Democratization}
%An unexpected finding from our study highlights a significant gap between the AI functionalities that screenwriters desire and the capabilities currently offered by AI tools. Specifically, participants expressed the need for AI to assist with the directed continuation and expansion of existing content, as well as data visualization. Although some researchers have made significant progress in these areas and developed related tools~\cite{schank2013scripts, valls2016error, mateas2003experiment, 10.1145/3172944.3172972, tapaswi2014storygraphs, poulakos2015towards}, these innovations have not been widely adopted in practice. Our analysis suggests two possible reasons for this disconnect. First, many of these innovative tools have not been commercialized or open-sourced, making them inaccessible to most screenwriters. Second, as indicated by the results in Section 5 and supplemental material, screenwriters typically prefer using basic document editing software like Word or WPS. Only a few participants mentioned using professional screenwriting software such as Final Draft. Additionally, although some have experimented with AI chatbots like ChatGPT, even these tools are perceived as having a high entry threshold. Thus, the complexity of the tools mentioned in previous studies may also limit their adoption among screenwriters. To bridge this gap, collaboration between industry and academia is crucial for advancing AI-assisted screenwriting. Stakeholders must work towards democratizing access to AI technologies, making them available to a broader range of screenwriters, including those working independently or within smaller production companies. This could involve developing affordable AI software, offering open-source tools, and providing targeted support to help screenwriters integrate these technologies into their workflows.

%Moreover, as highlighted in Section~\ref{sec:capabilities}, the concerns raised by screenwriters about the limitations of current AI in emotional depth and creativity underscore the need for continued advancement in AI technology. To make AI truly valuable in the field of screenwriting, technological development should focus on enhancing AI’s ability to understand and generate nuanced, emotionally rich content. This may include improving AI’s performance in handling complex narrative structures, multi-character relationships, and other aspects of human emotional complexity. Therefore, researchers should prioritize interdisciplinary collaboration, combining insights from cognitive science, creative writing, and artificial intelligence to develop models that better simulate human emotions. Future work by researchers and developers should emphasize the design of intuitive, user-friendly AI systems that align with the specific needs of screenwriters. 

%Our study revealed a gap between the AI functionalities screenwriters desire and current AI tools. Participants expressed the need for AI to assist with content expansion and visualization in Section 7.4. Although some researchers have made progress in these areas and developed related tools~\cite{schank2013scripts, valls2016error, mateas2003experiment, 10.1145/3172944.3172972, tapaswi2014storygraphs, poulakos2015towards}, these innovations have not been widely adopted in practice, likely due to their complexity and lack of accessibility. To address this, collaboration between industry and academia is necessary to democratize access to AI tools. For instance, researchers could consider focusing on comparing the use of AI tools among different types of screenwriters, such as professionals, enthusiasts, and students, or across various screenwriting genres. This approach could help further tailor AI tools to meet diverse user needs, involving creating affordable AI software and offering open-source tools to help screenwriters integrate AI more effectively into their workflows, thereby advancing the democratization of AI in this field. 

%As noted in Section~\ref{sec:capabilities}, AI's ability to generate nuanced, emotionally rich content requires improvement. Technological advancements should focus on managing complex narratives, character relationships, and human emotional complexity. Researchers should prioritize interdisciplinary collaboration, integrating cognitive science, creative writing, and AI to develop better emotion-simulating models. Future efforts should emphasize creating intuitive, user-friendly AI systems that meet screenwriters' specific needs.

%According to the findings in Section 6.1, screenwriters perceive AI as both a potential competitor and a collaborator in the creative process. On one hand, the competitive pressure from AI encourages screenwriters to elevate their creative standards and explore new possibilities. On the other hand, collaboration between AI and screenwriters offers opportunities for narrative innovation. However, our interviews revealed that some participants lack the skills needed to effectively prompt AI tools. Therefore, it is crucial for screenwriters to acquire the knowledge and skills necessary to integrate AI into their creative processes effectively, viewing it as a tool to enhance their creative capabilities. To achieve this, AI experts and other stakeholders should actively organize various training initiatives, such as workshops, courses, and conferences, to facilitate greater exposure to and understanding of AI technologies among screenwriters. These training programs should focus on educating screenwriters about the development of AI technology and how to best apply AI tools in their creative work, guiding them to discover the potential applications of AI across different stages of the creative process. This approach not only enhances screenwriters' technical abilities but also boosts their confidence in using AI technologies, fostering further integration of human-AI collaboration in the creative process. Furthermore, tutorials and workshops can serve as effective approaches to address technical barriers.

%\subsection{Considering Ethical Limitation}

%Section 5.2.1 emphasized that AI's strict moral standards can hinder its application in screenwriting. Previous research highlights the importance of technomoral values such as honesty, empathy, and flexibility for integrating AI into creative practices~\cite{vallor2016technology, flick2022ethics}. However, these studies do not fully address the unique needs of screenwriting. While screenwriters strive to create emotionally resonant stories, they also seek to craft narratives that are distinctive and memorable. Predictable or conventional stories often fail to captivate audiences, making them less worth the effort required for production. Unique stories, often involving themes like crime or minority experiences (as mentioned in Section 6.2.2), are more likely to engage audiences. However, AI's moral constraints frequently limit its capacity to explore such themes.

%Section 5.2.1 emphasized that AI's strict moral standards can hinder its application in screenwriting. Previous research highlights the importance of technomoral values such as honesty, empathy, and flexibility for integrating AI into creative practices~\cite{vallor2016technology, flick2022ethics}. However, these studies do not fully address the unique needs of screenwriting. While screenwriters strive to create emotionally resonant stories, they also seek to craft narratives that are distinctive and memorable. Predictable or conventional stories often fail to captivate audiences, making them less worth the effort required for production. Unique stories, often involving themes like crime or minority experiences (as mentioned in Section 6.2.2), are more likely to engage audiences. However, AI's constraints frequently limit its capacity to explore such themes.

%Future developers must acknowledge the significant impact these restrictions have on screenwriting and consider how to train AI on ethical issues that balance creative freedom with responsible use. Designers should also offer screenwriters clear guidelines and appropriate use cases for AI tools. For example, in non-realist genres, AI’s content pastiche feature (as discussed in Section 5.3) could provide substantial creative support. Conversely, as noted in Section 6.2.2, AI may be less suitable for handling screenplays involving crime. By further exploring AI's ethical limitations and mapping its capabilities, developers can equip screenwriters with a more effective framework for using AI tools, minimizing the negative effects of ethical constraints while enhancing screenwriters' confidence in AI-assisted creativity.

%Section 5.2.1 highlighted that AI's strong moral standards can hinder its use in screenwriting. Previous studies emphasize key technomoral values like honesty, empathy, and flexibility as crucial for integrating AI into creative practices~\cite{vallor2016technology, flick2022ethics}. However, these studies do not fully address the specific needs of screenwriting. While screenwriters seek to create emotionally resonant stories, they also aim to produce unique and engaging narratives that stand out in audiences' mind. Conventional, predictable stories are neither appealing nor worth the extensive production effort. Unique stories are more likely to captivate audiences, often involving crime or minority experiences as mentioned in Section 6.2.2. However, AI's moral standards often limit its ability to explore such themes.

%Future developers need to recognize the fundamental impact of these restrictions on screenwriting. They must consider how to train AI on ethical issues that balance restrictions on creativity with the potential to offer more flexible, tailored tools for screenwriters. This might involve implementing identity verification systems to unlock certain functions in AI tools, providing screenwriters with more creative freedom within a controlled environment\cite{shen2023anything}. While this is a complex challenge, it must be addressed if AI is to significantly advance its utility in screenwriting, particularly in mitigating risks such as biases and the homogenization of content.

%Section 6.2 highlights screenwriters' ethical concerns, emphasizing the need for clear guidelines. Previous research stresses the importance of human creativity amidst technological advancements and advocates for interdisciplinary teams to promote responsible AI use in enhancing creativity~\cite{suchacka2021human}. Key technomoral values like honesty, empathy, and flexibility are crucial for embedding within creative AI practices~\cite{vallor2016technology, flick2022ethics}. However, these studies do not fully address screenwriting-specific needs. 

%\subsection{Preserving Creative Agency and Identity}

%Previous research emphasizes the importance of preserving human creativity amidst technological advancements and advocates for interdisciplinary teams to promote the responsible use of AI in creative processes~\cite{suchacka2021human}. Building on this, our study suggests that future guidelines for screenwriting should specifically consider agency and identity in the context of human-AI collaboration. In Section~\ref{sec:Allocation}, AI integration in task allocation highlights the need for human input at each stage of the workflow. In Section 6.1.2, where AI acts as an ``actor,'' and in Section 6.4, where AI functions as an ``executor,'' screenwriters similarly emphasized the importance of maintaining control and its potential impacts. They aim to ensure that AI serves as an extension of the screenwriter's creative intent rather than acting as an autonomous creator. However, as AI takes on more complex tasks, such as dialogue, structure/plot, and even generating complete screenplays, it could challenge a screenwriter’s control over the narrative. We suggest that future AI screenwriting tool developers consider how AI tools can grant more control to screenwriters over the output. Stakeholders should also think about how to maintain or regain agency by enabling screenwriters to become more adept at directing AI outputs (as discussed in Section~\ref{sec:Facilitating}).

\subsection{Preserving Creative Agency}

Previous research emphasizes the importance of preserving human creativity amidst technological advancements and advocates for interdisciplinary teams to promote the responsible use of AI in creative processes~\cite{suchacka2021human}. Building on this, our study suggests that future guidelines for screenwriting should specifically consider agency in the context of human-AI collaboration. In Section~\ref{sec:Allocation}, AI integration in task allocation highlights the need for human input at each stage of the workflow. In Section~\ref{sec:Engaging} and Section~\ref{sec:Executor}, where AI acts as an ``actor'' and functions as an ``executor,'' respectively, screenwriters emphasized the importance of maintaining control over the creative process and its potential impacts. They aim to ensure that AI serves as an extension of the screenwriter's creative intent rather than acting as an autonomous creator. However, as AI takes on more complex tasks, such as dialogue, structure and plot development, and even generating complete screenplays, it could challenge a screenwriter’s control over the narrative. We suggest that future AI screenwriting tool developers consider how AI tools can grant more control to screenwriters over the output, such as by providing interactive feedback loops and improving transparency in AI decision-making. Stakeholders should also think about how to maintain or regain agency by enabling screenwriters to become more adept at directing AI outputs (as discussed in Section~\ref{sec:Facilitating}).



%\textcolor{black}{The integration of AI into the creative process is reshaping traditional notions of authorship, urging screenwriters to reconsider their contributions. As discussed in Section 6.1.2, screenwriters may increasingly view themselves as curators or observers of AI-generated content rather than sole authors. The traditional image of the screenwriter as the sole architect of a narrative is evolving into a collaborative model, where humans and AI co-create content. This shift raises critical questions about originality, ownership, and the essence of creative contribution, particularly as AI systems assume more autonomous roles in content generation. These challenges necessitate a reevaluation of authorial identity and the adoption of new strategies for asserting screenwriters' creative roles in a human-AI collaborative landscape. Below, we outline actionable strategies for maintaining and redefining authorial identity, framing this evolution not as a diminishment of the screenwriter’s identity but as an opportunity for expansion and innovation.}

\subsection{Redefining Authorial Identity}

\textcolor{black}{The traditional view of screenwriters as sole creators is shifting toward a collaborative human-AI model. As noted in Section 6.1.2, screenwriters are gradually becoming curators or observers of AI-generated content in their expectations, raising concerns about originality, ownership, and creative contributions (Section~\ref{sec:ethical concerns}). To address these concerns, we propose strategies to maintain authorial identity and frame this shift as an opportunity for innovation.}

\textcolor{black}{\subsubsection{Preserving Creative Contribution through Personalization}  
AI tools can enhance creative identity by aligning outputs with individual styles. For instance, tools like Stable Diffusion enable artists to embed their style into AI training~\cite{stablediffusion2022}. Similarly, screenwriters could guide AI outputs by uploading screenplays or outlines to refine tone, pacing, and structure while preserving stylistic specifics. Additionally, prior research highlights AI’s ability to merge trends with creators’ styles in short videos~\cite{10.1145/3613904.3642476}, suggesting its potential to synthesize real-world events or news with screenwriters' preferences, balancing external influences and unique styles in screenwriting.}

\textcolor{black}{\subsubsection{Maintaining Ownership and Accountability through Transparency}  
Transparency is essential for preserving authorial identity~\cite{10.1145/3613904.3641895}. We propose tracking AI’s role at different stages, such as dialogue generation or plot development, by documenting the proportion of AI versus human contributions. This systematic breakdown allows screenwriters to reflect on their processes, recognize AI’s role, and identify areas for improvement. This approach could also safeguard creative ownership, support ethical guidelines, and inform standards for attribution and intellectual property, aiding in copyright decisions based on human and AI contributions.}


\begin{comment}
    
\textcolor{black}{The traditional view of screenwriters as sole narrative creators is evolving toward a collaborative human-AI model. As noted in Section 6.1.2, screenwriters are increasingly seen as curators or observers of AI-generated content, raising concerns about originality, ownership, and creative contribution (Section~\ref{sec:ethical concerns}). We propose strategies for maintaining authorial identity, framing this shift as an opportunity for innovation rather than a reduction of the screenwriter’s role.}

\textcolor{black}{\subsubsection{Preserving Creative Contribution through Personalization}  
Advances in visual arts demonstrate how AI can enhance creative identity. For example, tools like Stable Diffusion allow artists to embed their style into AI training, aligning outputs with their vision~\cite{stablediffusion2022}. Similarly, AI tools for screenwriting could integrate style-training mechanisms, enabling screenwriters to upload scripts or outlines to guide AI outputs. These tools should refine tone, pacing, and structure, while preserving stylistic specifics. Prior research shows AI’s potential to merge trends with creators’ styles in short videos~\cite{10.1145/3613904.3642476}, and this can extend to screenwriting, where AI synthesizes real-world events with artistic preferences, balancing external influences with the screenwriter’s unique style.}

\textcolor{black}{\subsubsection{Maintaining Ownership and Accountability through Transparency}  
Transparency is critical for preserving authorial identity, as discussed by Hoque et al.~\cite{10.1145/3613904.3641895}. We propose tracking AI’s role in various stages of screenwriting, such as dialogue generation or plot development, to systematically attribute contributions. Systems should document the proportion of AI versus human-generated content, providing a breakdown for each segment. This enables screenwriters to reflect on their creative processes, recognize AI’s role, and identify areas for further AI support. Transparent documentation safeguards creative ownership, informs ethical guidelines, and supports standards for attribution and IP, also helping determine copyright ownership based on human and AI contributions.}

\end{comment}

%\textcolor{black}{The traditional view of screenwriters as sole narrative creators is shifting toward a collaborative human-AI paradigm. As noted in Section 6.1.2, screenwriters increasingly act as curators or observers of AI-generated content rather than sole authors. This transition raises concerns about originality, ownership, and creative contribution (Section~\ref{sec:ethical concerns}). To address these challenges, we propose strategies for upholding authorial identity, framing this shift as an opportunity for innovation rather than a diminishment of the screenwriter’s role.}

%\textcolor{black}{\subsubsection{Preserving Creative Contribution through Personalization}
%Advances in visual arts demonstrate how AI can enhance creative identity. For instance, tools like Stable Diffusion allow artists to embed their style into AI training processes, aligning outputs with their vision~\cite{stablediffusion2022}. Similarly, AI tools for screenwriting could integrate style-training mechanisms, enabling screenwriters to upload scripts, dialogue samples, or narrative outlines to guide AI outputs. Beyond replication, these tools should support refining tone, pacing, and structure, while maintaining stylistic specificity in techniques, perspectives, or genres. Prior research shows AI’s ability to merge trends with creators’ styles for short videos~\cite{10.1145/3613904.3642476}. Extending this to screenwriting, AI could help synthesize real-world events with artistic preferences. For example, a screenwriter specializing in suspense could use AI to draw inspiration from unusual news stories for original narratives. This approach balances external influences with personal artistic input, embedding the screenwriter’s unique style and thematic elements into AI-generated content.}

%\textcolor{black}{\subsubsection{Maintaining Ownership and Accountability through Transparency}  
%Transparency is essential for preserving authorial identity, aligning with the work of Hoque et al.~\cite{10.1145/3613904.3641895}. Building on this foundation, we propose further strategies tailored for screenwriters. For instance, by logging AI’s role at various stages of screenwriting (e.g., dialogue generation, plot development) alongside screenwriter interventions, contributions can be systematically tracked and attributed. Systems should document the proportion of AI-generated versus human-generated content, offering a detailed breakdown for each segment. This enables screenwriters to reflect on their creative processes, recognize AI’s contributions, and identify areas where additional AI support may be beneficial. Transparent documentation not only safeguards creative ownership but also informs ethical guidelines and supports standards for attribution and IP. Furthermore, such records can be used to determine copyright ownership based on the proportion of human and AI contributions.}

%\textcolor{black}{\subsubsection{Reconceptualizing Creative Contribution through Personalization}  

%Transparency is crucial for preserving authorial identity. By logging AI’s role in different stages of screenwriting (e.g., dialogue, plot development) alongside screenwriter interventions, contributions can be tracked and attributed. Systems should record the proportion of AI- versus human-generated content, providing a detailed breakdown for each segment. This helps screenwriters reflect on their processes, recognize AI’s contributions, and identify areas for further AI support. Transparent documentation safeguards creative ownership, informs ethical guidelines, and supports attribution and IP standards. Such documentation can also determine copyright ownership based on the proportion of human and AI contributions.}

%\textcolor{black}{The traditional view of the screenwriter as the sole creator of a narrative is evolving into a collaborative human-AI model. As noted in Section 6.1.2, screenwriters may increasingly act as curators or observers of AI-generated content rather than sole authors. This transformation raises essential questions about originality, ownership, and creative contribution, especially as AI systems gain greater autonomy in content creation. We propose the following strategies to maintain and redefine authorial identity, framing this shift as an opportunity for innovation rather than a diminishment of the screenwriter’s role.}

%\textcolor{black}{The traditional perception of screenwriters as sole narrative creators is evolving into a collaborative human-AI paradigm. As discussed in Section 6.1.2, screenwriters are increasingly taking on roles as curators or observers of AI-generated content rather than exclusive authors. This transition raises critical concerns about originality, ownership, and creative contribution (as outlined in Section~\ref{sec:ethical concerns}). To address these challenges, we propose strategies to maintain and redefine authorial identity, positioning this shift as an opportunity for innovation rather than a diminishment of the screenwriter’s role.}

\begin{comment}

\textcolor{black}{The traditional perception of screenwriters as sole narrative creators is evolving into a collaborative human-AI paradigm. As discussed in Section 6.1.2, screenwriters are increasingly taking on roles as curators or observers of AI-generated content rather than sole authors. This transition raises concerns about originality, ownership, and creative contribution (as outlined in Section~\ref{sec:ethical concerns}). To address these challenges, we propose strategies to maintain and redefine authorial identity, positioning this shift as an opportunity for innovation rather than a diminishment of the screenwriter’s role.}

\textcolor{black}{\subsubsection{Reconceptualizing Creative Contribution through Personalization} Advances in the visual arts demonstrate how AI can be harnessed to preserve and enhance creative identity. For example, tools like Stable Diffusion enable artists to incorporate their own artwork or stylistic preferences into the training process, ensuring AI-generated outputs align with their artistic vision~\cite{stablediffusion2022}. These practices highlight AI’s potential to serve as an enabler, rather than a detractor, of individual creativity. When applied to screenwriting, similar AI tools could integrate style-training mechanisms, allowing screenwriters to upload previous scripts, dialogue samples, or narrative outlines, creating a personalized dataset to guide AI outputs. Beyond replication, these tools should offer customizability, enabling users to refine tone, pacing, and structure, and support stylistic specificity in narrative techniques, perspectives, or genres. Prior research has shown how AI can merge popular trends with a creator’s unique style to enhance content for short videos~\cite{10.1145/3613904.3642476}. Extending this concept, AI could assist screenwriters in synthesizing relevant real-world events with their artistic preferences. For instance, a screenwriter specializing in suspense might use personalized AI to extract unusual real-life events from news sources, drawing inspiration for original stories or adaptations. Additionally, screenwriters can enhance their contribution to the screenplay by incorporating their understanding of complex emotions and nuanced details, drawing on personal experiences. This approach balances external influences with personal artistic input, embedding the screenwriter’s unique stylistic and thematic signatures into AI-generated content.}

\textcolor{black}{\subsubsection{Maintaining Ownership and Accountability through Transparency} Transparency is another critical factor in addressing authorial identity. By logging AI’s role at various stages of screenwriting (e.g., dialogue, plot development) alongside the screenwriter’s interventions, both contributions can be accurately tracked and attributed, safeguarding the screenwriter's creative identity within collaborative AI environments. Specifically, systems should log the proportion of AI-generated versus human-generated content in the final script, while also providing a detailed breakdown of contributions in each segment. This approach enables screenwriters to reflect on their creative processes, recognize areas where AI has enhanced their work, and identify opportunities for further AI support. Transparent documentation facilitates accurate attribution of credit, helps maintain creative ownership, and informs the development of ethical guidelines and standards, including protocols for attribution and intellectual property (IP) rights. The proportion of human and AI contributions can be used to determine the copyright ownership of the final screenplay.}

\end{comment}

%\textcolor{black}{\subsubsection{Personalization in AI Tools.} Advances in the visual arts illustrate how AI can be leveraged to preserve and enhance creative identity. Tools like Stable Diffusion empower artists by allowing them to integrate their own artwork or stylistic preferences into the training process, ensuring AI-generated outputs remain true to their artistic vision~\cite{stablediffusion2022}. These practices underline the potential for AI to act as an enabler rather than a diminisher of individual creativity. Translating this approach to screenwriting, AI tools could implement style-training mechanisms that allow screenwriters to input prior scripts, dialogue samples, or narrative outlines, creating a personalized dataset to inform AI outputs. Beyond replication, these tools should offer advanced customizability, enabling users to set precise guidelines for tone, pacing, and narrative structure, thereby embedding the screenwriter’s unique stylistic and thematic signatures into the generated content.}

\begin{comment}

\textcolor{black}{\subsubsection{Preserving Screenwriters' Unique Style}
Developing AI tools with customization features is essential to allow screenwriters to embed their distinct voices and artistic visions into AI-generated content. Such tools could include models designed for quick configuration or personalized retrieval, support for refining tonal consistency, and features tailored to specific styles, such as narrative techniques, perspectives, or genres. Screenwriters should learn to leverage iterative processes, purposefully modifying and refining AI-generated outputs to align with their creative intent, thereby asserting authorship through meaningful adjustments. Furthermore, prior research has suggested that AI can align popular trends with creators' unique identities to support the creation of online short videos~\cite{10.1145/3613904.3642476}. Building on this, we propose that AI can assist screenwriters in efficiently gathering relevant real-world events and integrating them with their creative preferences. For example, a screenwriter specializing in suspense could use AI to extract specific types of unusual real-life events from a vast number of news reports as inspiration for adaptation or original storytelling.}

\textcolor{black}{\subsubsection{Establishing Ethical Standards}
Developing clear ethical guidelines and standards for attribution and intellectual property rights is equally critical. These measures help delineate the boundaries between human and AI contributions, safeguarding the professional integrity of screenwriters and fostering trust within the industry. For example, transparently documenting AI-generated content in the creative process can mitigate disputes over originality and authorship, providing a foundation for fair crediting practices.}

\end{comment}

%The integration of AI into the creative process may significantly influence how screenwriters perceive their roles, potentially leading them to see themselves more as observers of AI-generated content rather than as sole authors (Section 6.1.2). This shift necessitates a reevaluation of authorship, requiring screenwriters to redefine and assert their creative identity in new ways as AI tools become more prevalent. As AI systems generate more content autonomously, screenwriters might struggle with questions of originality, ownership, and the essence of their contribution to the final work. This evolution challenges traditional notions of authorship and creativity, prompting a need for new frameworks that acknowledge the collaborative nature of human-AI collaboration.

%Developers and stakeholders should consider implementing features that allow screenwriters to imprint their unique style and vision onto AI-generated content. Additionally, establishing clear guidelines and ethical standards regarding attribution and intellectual property rights can help preserve screenwriters' identities and professional integrity in an AI-augmented environment. By engaging with these challenges, the field can foster a more harmonious integration of AI technologies that enhance creativity while respecting and preserving the vital role of the screenwriter.

%Moreover, AI integration in the creative process may influence how screenwriters perceive their roles, potentially leading them to see themselves more as observers of AI-generated content rather than as sole authors (Section 6.1.2). This shift may prompt a reevaluation of authorship, requiring screenwriters to redefine and assert their creative identity in new ways as AI tools become more prevalent, which will be crucial for future AI tools in screenwriting.

%Previous research emphasizes the importance of preserving human creativity amid technological advancements and advocates for interdisciplinary teams to promote the responsible use of AI in creative processes~\cite{suchacka2021human}. In Section~\ref{sec:Allocation}, the integration of AI in task allocation highlights the need for human input at each stage of the workflow. In Section 6.1.2, where AI acts as an ``actor,'' and in Section 6.4, where AI functions as an ``executor,'' screenwriters similarly emphasized the importance of maintaining control and its potential impacts. 

%Our study suggests that guidelines should prioritize preserving agency and identity in human-AI collaboration. Specifically, AI's role in complex tasks should allow screenwriters greater control over AI outputs to ensure they maintain creative agency. Additionally, as AI becomes more integrated, there may be a shift in the perception of authorship, with screenwriters potentially becoming observers of AI-generated content rather than sole creators, as noted in Section 6.1.2. This shift prompts a reevaluation of creative identity, which will be a key consideration for future AI tools in screenwriting.

\subsection{Facilitating Collaborative Training} \label{sec:Facilitating}

\textcolor{black}{Sections~\ref{sec:Allocation} and~\ref{sec:capabilities} reveal the steep learning curve of AI tools, highlighting gaps in screenwriters' understanding and training for effective integration. Screenwriters view AI as both a competitor (Section~\ref{sec:Contradictory AI as a competitor}) and a collaborator (Section~\ref{sec:Positive Potential}), emphasizing the need for balanced training. Collaboration among AI developers, educators, and screenwriters is essential for designing effective strategies for these training programs. Inspired by Kicklighter et al.'s generative AI strategies for animation education~\cite{kicklighter2024empowering}, we propose a customized training workshop program for screenwriters.}

\subsubsection{\textcolor{black}{Understanding AI Capabilities and Ethical Considerations}}
\textcolor{black}{Workshops should begin by introducing AI's strengths, limitations, and applications, guiding screenwriters in setting realistic expectations. This is especially crucial for those unfamiliar with using AI in screenwriting. Screenwriters should recognize the value AI brings, such as generating diverse ideas during brainstorming. Ethical considerations should be embedded within the training to develop technical skills, build confidence, and enhance human-AI collaboration, ultimately improving both efficiency and creativity.}

\subsubsection{\textcolor{black}{Integrating AI into Workflow}}
\textcolor{black}{Workshops offer a practical setting for screenwriters to share experiences, explore peer strategies for AI integration, and experiment with AI across different stages of the screenwriting process. Building on Kim et al.'s approach for short video producers~\cite{10.1145/3544548.3581225}, these workshops can facilitate brainstorming sessions, enabling screenwriters to leverage AI's strengths in strategically allocating tasks between AI and humans. This approach could foster technical proficiency, build confidence, and enhance collaboration efficiency.}

\subsubsection{\textcolor{black}{Promoting Awareness of Human Strengths}}
\textcolor{black}{To balance AI's roles as both a competitor and collaborator, training should emphasize unique human strengths, such as emotional depth and personal experience, which remain beyond the capabilities of current AI (Sections~\ref{sec:capabilities} and~\ref{sec:ethical concerns}). Workshops should actively encourage reflection on these unique human attributes, promoting the exploration of innovative strategies to maintain creative control while effectively utilizing AI tools.}

%\textcolor{black}{To balance AI's roles as both a competitor and collaborator, training should emphasize human strengths, such as emotional depth and personal experience, which remain beyond the capabilities of current AI (Sections~\ref{sec:capabilities} and~\ref{sec:ethical concerns}). Workshops should encourage reflection on these unique human attributes, promoting the exploration of strategies to maintain creative control while effectively utilizing AI tools.}

%\textcolor{black}{Our findings indicate that screenwriters view AI as both a competitor (Section~\ref{sec:Contradictory AI as a competitor}) and a collaborator (Section~\ref{sec:Positive Potential}) in the creative process. As noted in Sections~\ref{sec:Allocation} and~\ref{sec:capabilities}, the steep learning curve of AI tools highlights a lack of comprehensive understanding among screenwriters on how to effectively integrate AI into workflows. Targeted training is essential to address these challenges. Collaboration between AI developers, educators, and screenwriting practitioners is necessary to design initiatives for effective AI integration. Inspired by the generative AI strategies for animation education proposed by Kicklighter et al.~\cite{kicklighter2024empowering}, we propose tailored training approaches for screenwriting.}


%\textcolor{black}{Sections~\ref{sec:Allocation} and~\ref{sec:capabilities} highlight the steep learning curve associated with AI tools, underscoring the lack of comprehensive understanding and training among screenwriters on how to effectively integrate AI into their workflows. Moreover, our findings reveal that screenwriters perceive AI as both a competitor (Section~\ref{sec:Contradictory AI as a competitor}) and a collaborator (Section~\ref{sec:Positive Potential}) in the creative process. Consequently, targeted training that addresses both perspectives is crucial for overcoming these challenges. To this end, collaboration between AI developers, educators, and screenwriting professionals is essential to create initiatives that promote effective AI integration. Drawing inspiration from the generative AI strategies for animation education proposed by Kicklighter et al.~\cite{kicklighter2024empowering}, we suggest tailored training approaches for screenwriting that focus on two training objectives.} 

%\textcolor{black}{Sections~\ref{sec:Allocation} and~\ref{sec:capabilities} highlight the steep learning curve of AI tools, reflecting gaps in screenwriters' understanding and training for effective AI integration. Screenwriters perceive AI as both a competitor (Section~\ref{sec:Contradictory AI as a competitor}) and a collaborator (Section~\ref{sec:Positive Potential}), underscoring the need for targeted training to balance these. Collaboration among AI developers, educators, and screenwriters is crucial for designing effective integration strategies. Drawing on Kicklighter et al.'s generative AI strategies for animation education~\cite{kicklighter2024empowering}, we propose tailored training with two key objectives.}

%\textcolor{black}{Sections~\ref{sec:Allocation} and~\ref{sec:capabilities} reveal the steep learning curve of AI tools, highlighting gaps in screenwriters' understanding and training for effective integration. Screenwriters view AI as both a competitor (Section~\ref{sec:Contradictory AI as a competitor}) and a collaborator (Section~\ref{sec:Positive Potential}), emphasizing the need for balanced training. Collaboration among AI developers, educators, and screenwriters is essential to design effective strategies for these trainings. Inspired by Kicklighter et al.'s generative AI strategies for animation education~\cite{kicklighter2024empowering}, we propose tailored trainings with two key objectives.}

%\textcolor{black}{Sections~\ref{sec:Allocation} and~\ref{sec:capabilities} reveal the steep learning curve of AI tools, highlighting gaps in screenwriters' understanding and training for effective integration. Screenwriters view AI as both a competitor (Section~\ref{sec:Contradictory AI as a competitor}) and a collaborator (Section~\ref{sec:Positive Potential}), emphasizing the need for balanced training. Collaboration among AI developers, educators, and screenwriters is essential for designing effective strategies for these training programs. Inspired by Kicklighter et al.'s generative AI strategies for animation education~\cite{kicklighter2024empowering}, we propose tailored training programs with two key objectives.}

%\textcolor{black}{Sections~\ref{sec:Allocation} and~\ref{sec:capabilities} reveal the steep learning curve of AI tools, highlighting the gap in screenwriters' understanding and training on effective AI integration. Our findings show that screenwriters perceive AI as both a competitor (Section~\ref{sec:Contradictory AI as a competitor}) and a collaborator (Section~\ref{sec:Positive Potential}) in the creative process. Therefore, targeted training addressing both perspectives is necessary. Collaboration between AI developers, educators, and screenwriting professionals is essential to design effective AI integration strategies. Inspired by Kicklighter et al.'s generative AI strategies for animation education~\cite{kicklighter2024empowering}, we propose tailored training focusing on two key objectives.}

%Sections~\ref{sec:Allocation} and~\ref{sec:capabilities} highlight the steep learning curve of AI tools, revealing the lack of understanding and training among screenwriters on effective AI integration. Our findings also show that screenwriters view AI as both a competitor (Section~\ref{sec:Contradictory AI as a competitor}) and a collaborator (Section~\ref{sec:Positive Potential}) in the creative process. Therefore, targeted training addressing both perspectives is essential. Collaboration between AI developers, educators, and screenwriting professionals is needed to design effective AI integration initiatives. Inspired by the generative AI strategies for animation education by Kicklighter et al.~\cite{kicklighter2024empowering}, we propose tailored training approaches focusing on two aims.}


%\subsubsection{\textcolor{black}{Training for Integrating AI into Workflows}} \textcolor{black}{This training aims to help screenwriters with no prior AI knowledge understand how to integrate AI into different stages of the screenwriting process.}

%The training aims to help screenwriters without AI knowledge understand how to integrate AI into various stages of the screenwriting workflow.}

\begin{comment}

\textcolor{black}{\textbf{Introduction to AI Capabilities.}}
\textcolor{black}{The training should familiarize screenwriters with AI’s strengths, limitations, and applications, setting realistic expectations. Screenwriters should understand where AI adds value (e.g., idea generation and audience trend analysis). For example, AI can suggest diverse ideas during brainstorming (Sections~\ref{sec:Current Positive} and~\ref{sec:Positive Potential}), facilitating the integration of AI into specific tasks.}
%The training should introduce screenwriters to AI’s strengths, limitations, and practical applications, setting realistic expectations. Screenwriters should learn where AI adds value (e.g., idea generation, audience analysis, story structure evaluation) and where human creativity remains irreplaceable. For instance, AI can offer diverse plotline suggestions during brainstorming (Sections~\ref{sec:Current Positive} and~\ref{sec:Positive Potential}). This understanding will allow for effective integration of AI into specific tasks.}

\textcolor{black}{\textbf{Practical Application in Workflow.}}
\textcolor{black}{Training should guide screenwriters in developing AI-integrated workflows that optimize both AI and human creativity. Workshops or hands-on sessions will help them design customized workflows and improve task allocation. Ethical considerations should be incorporated into these strategies to build technical skills, increase confidence, and enhance human-AI collaboration, improving both efficiency and creativity.}

%The training should guide screenwriters in developing AI-integrated workflows that leverage both AI and human creativity. Workshops or hands-on sessions will allow them to design custom workflows, optimizing task allocation. Ethical considerations should be integrated into this process. These strategies will build technical skills, increase confidence, and enhance human-AI collaboration, improving both efficiency and creativity.}

\subsubsection{\textcolor{black}{Training for Balancing Screenwriters' Views}} \textcolor{black}{This training aims to assist screenwriters with AI experience in balancing the two distinct perspectives on AI’s role in the creative process.}

%The training aims to assist screenwriters with experience in AI utilization in effectively balancing the influence of the two distinct perspectives.}

\textcolor{black}{\textbf{Promoting Awareness of Human Strengths.}}
\textcolor{black}{For screenwriters with AI experience, the training should focus on balancing AI’s role as both a competitor and collaborator. Emphasis should be placed on human strengths, such as emotional depth, cultural intuition, and personal experience, which AI cannot replicate. Workshops should encourage reflection on these aspects and explore ways to maintain creative control while using AI. Understanding AI’s emotional limitations (Sections~\ref{sec:capabilities} and~\ref{sec:ethical concerns}) will help ensure that AI complements, rather than replaces, human creativity.}
%For screenwriters with AI experience, training should focus on balancing AI’s roles as both a competitor and collaborator. The emphasis should be on human strengths, such as emotional depth, cultural intuition, and personal experience, which AI cannot replicate. Workshops should encourage reflection on these aspects and explore ways to maintain creative control while using AI. Understanding AI’s emotional limitations (Sections~\ref{sec:capabilities} and~\ref{sec:ethical concerns}) will help ensure that AI complements, rather than replaces, human creativity.}

\textcolor{black}{\textbf{Ethical and Copyright Considerations.}}
\textcolor{black}{Training should address ethical challenges, particularly those related to copyright, authorship, and accountability (Section 8.2). Screenwriters should learn to navigate these issues, ensuring transparency and compliance with copyright laws. This will help them uphold ethical standards and professional integrity while incorporating AI tools into their creative processes.}
%The training should address ethical challenges, especially concerning copyright, authorship, and accountability (Section 8.2). Screenwriters should learn to navigate these issues, ensuring transparency and compliance with copyright laws. This will help them preserve ethical standards and their professional integrity while incorporating AI tools into their creative process.}

    
\textcolor{black}{\textbf{Understanding AI Capabilities and Ethical Considerations}}  
\textcolor{black}{Training should start with an overview of AI’s functions, strengths, and limitations to set realistic expectations and prevent over-reliance. Screenwriters should focus on tasks where AI excels, such as idea generation, audience trend analysis, and story structure evaluation. For instance, AI can suggest diverse plotlines or alternative perspectives during brainstorming, acting as a creative collaborator (Sections~\ref{sec:Current Positive} and~\ref{sec:Positive Potential}). Repetitive tasks like proofreading or formatting can be assigned to AI, freeing screenwriters to focus on storytelling. Recognizing AI’s limitations helps screenwriters address its role as a competitor. By emphasizing uniquely human strengths, such as emotional depth, cultural intuition, narrative complexity, and personal experience, screenwriters can maintain their creative edge and highlight the value of human creativity (Sections~\ref{sec:capabilities} and~\ref{sec:ethical concerns}). Screenwriters must also consider ethical aspects, particularly around copyright and authorship (Section 8.2). Adhering to regulations ensures accountability, ethical practices, and the preservation of the screenwriter’s creative identity.}

\textcolor{black}{\textbf{Workshops for AI-Integrated Workflow Training}}  
\textcolor{black}{Workshops provide a practical platform for screenwriters to share experiences and peer strategies for AI integration, drawing from Kim et al.'s approach for short video producers~\cite{10.1145/3544548.3581225}. These workshops can train screenwriters to design workflows that strategically allocate tasks between AI and humans, leveraging AI’s strengths discussed earlier. By combining traditional workflows with AI tools, screenwriters can enhance their technical skills, confidence, and collaboration efficiency. Workshops also foster the development of effective AI-integrated workflows while preserving human creativity, and strengthening human-AI partnerships in storytelling.}

\textcolor{black}{Our findings show that screenwriters view AI as both a competitor (Section~\ref{sec:Contradictory AI as a competitor}) and a collaborator (Section~\ref{sec:Positive Potential}) in the creative process. As noted in Section~\ref{sec:Allocation} and Section~\ref{sec:capabilities}, the steep learning curve associated with AI tools indicates that screenwriters lack a comprehensive understanding of how to integrate AI into their workflows effectively. Targeted training is essential to address these challenges. Therefore, collaboration among AI developers, educators, and screenwriting practitioners is necessary to design initiatives that enable effective AI integration. Drawing from the generative AI strategies for animation education proposed by Kicklighter et al.~\cite{kicklighter2024empowering}, we propose tailored training approaches for screenwriting.}

\textcolor{black}{\subsubsection{Understanding AI Capabilities and Ethical Considerations}  
Training should begin with an overview of AI’s functions, strengths, and limitations to help screenwriters set realistic expectations and avoid over-reliance. Screenwriters should focus on tasks where AI excels, such as generating ideas, identifying audience trends, and analyzing story structures. For instance, AI can provide diverse plotlines or alternative perspectives during brainstorming, serving as a creative collaborator (Sections~\ref{sec:Current Positive} and~\ref{sec:Positive Potential}). Repetitive tasks, such as proofreading or formatting, can be delegated to AI, allowing screenwriters to concentrate on storytelling. Understanding AI’s limitations also helps screenwriters address its potential threats as a competitor. By emphasizing human strengths, such as emotional depth, cultural intuition, complex narrative crafting, and personal experience, screenwriters can maintain their creative edge and highlight the value of human creativity (Sections~\ref{sec:capabilities} and~\ref{sec:ethical concerns}).  
Additionally, screenwriters must understand the ethical considerations of AI-generated content, particularly concerning copyright and authorship (Section 8.2). Adhering to relevant regulations promotes accountability, ethical practices, and the preservation of the screenwriter’s creative voice.}

\textcolor{black}{\subsubsection{Workshops for AI-Integrated Workflow Training}  
Workshops provide a practical platform for screenwriters to share experiences and learn peer strategies for AI integration, drawing inspiration from Kim et al.'s approach for short video producers~\cite{10.1145/3544548.3581225}. These workshops can train screenwriters to design AI-integrated workflows that strategically allocate tasks between AI and humans, leveraging AI’s capabilities discussed earlier. By blending traditional workflows with AI tools, screenwriters can enhance technical skills, confidence, and collaboration effectiveness. Workshops also support the development of efficient AI-integrated workflows while maintaining human creativity, fostering stronger human-AI partnerships in storytelling.}
\end{comment}

%\textcolor{black}{Our findings indicate that screenwriters perceive AI as both a competitor (Section~\ref{sec:Contradictory AI as a competitor}) and a collaborator (Section~\ref{sec:Positive Potential}) in the creative process. Additionally, there is an expectation for AI to act as a specialized ``expert'' in skill acquisition, facilitating users' transitions into different work roles beyond traditional screenwriter roles, such as combining screenwriter, director, and producer identities into one user. This shift, discussed in Section 6.3, is driven by independent content creators like YouTubers and filmmakers. Based on these findings, targeted training to navigate AI’s dual role is essential. AI developers, educators, and screenwriting practitioners must collaborate to create initiatives that help screenwriters integrate AI effectively into their workflows. Building on the generative AI strategies for animation education proposed by Kicklighter et al.~\cite{kicklighter2024empowering}, we propose tailored training approaches for screenwriting.}

\begin{comment}
    \textcolor{black}{Our findings show that screenwriters view AI as both a competitor (Section~\ref{sec:Contradictory AI as a competitor}) and a collaborator (Section~\ref{sec:Positive Potential}) in the creative process. As noted in Section~\ref{sec:Allocation} and Section~\ref{sec:capabilities}, the steep learning curve of AI tools indicates that screenwriters lack a comprehensive understanding of how to integrate AI into their workflows effectively. Targeted training is essential to address this challenge. By emphasizing human strengths, such as personal experiences and interpreting complex emotions (Sections~\ref{sec:capabilities} and~\ref{sec:ethical concerns}), screenwriters can assert their creative identity when AI is perceived as a competitor (Section 8.2). Simultaneously, leveraging AI’s capabilities for idea generation and inspiration positions it as a collaborator (Sections~\ref{sec:Current Positive} and~\ref{sec:Positive Potential}). Therefore, collaboration among AI developers, educators, and screenwriting practitioners is needed to design initiatives enabling effective AI integration. Drawing from the generative AI strategies for animation education proposed by Kicklighter et al.~\cite{kicklighter2024empowering}, we propose tailored training approaches for screenwriting.}

\textcolor{black}{\subsubsection{Understanding AI Capabilities and Ethical Considerations}  
Training should start with an overview of AI’s functions, strengths, and limitations to help screenwriters set realistic expectations and avoid over-reliance. Screenwriters should focus on tasks where AI excels, such as generating ideas, identifying audience trends, and analyzing story structures. For instance, AI can provide diverse plotlines or alternative perspectives during brainstorming, serving as a creative partner. Repetitive tasks like proofreading or formatting can be delegated to AI, allowing screenwriters to focus on storytelling. Understanding AI’s limitations also helps screenwriters address its potential threats as a competitor. By emphasizing human strengths like emotional depth, cultural intuition, complex narrative crafting, and personal experience, screenwriters can maintain their creative edge and highlight the value of human creativity.  
Additionally, screenwriters must understand the ethical considerations of AI-generated content, particularly around copyright and authorship (Section 8.2). Adhering to relevant regulations promotes accountability, ethical practices, and the preservation of the screenwriter’s creative voice.}

\textcolor{black}{Our findings indicate that screenwriters perceive AI as both a competitor (Section~\ref{sec:Contradictory AI as a competitor}) and a collaborator (Section~\ref{sec:Positive Potential}) in the creative process. Furthermore, as highlighted in Section~\ref{sec:Allocation} and Section~\ref{sec:capabilities}, the steep learning curve associated with AI tools suggests that screenwriters still lack a comprehensive understanding of how to effectively integrate AI into their workflows. Targeted training is, therefore, essential to navigate AI’s dual role. By emphasizing uniquely human strengths, such as personal experiences and the ability to interpret complex emotions (as discussed in Section~\ref{sec:capabilities} and~\ref{sec:ethical concerns}), screenwriters can assert their creative identity when AI is perceived as a competitor (as discussed in Section 8.2). At the same time, leveraging AI’s strengths in idea generation and inspiration can position it as a valuable collaborator (Section~\ref{sec:Current Positive} and~\ref{sec:Positive Potential}). Thus, AI developers, educators, and screenwriting practitioners must work together to design initiatives that enable screenwriters to effectively integrate AI into their workflows. Building on the generative AI strategies for animation education proposed by Kicklighter et al.~\cite{kicklighter2024empowering}, we propose tailored training approaches specifically designed for screenwriting.}

\textcolor{black}{\subsubsection{Understanding the Capabilities of AI and Ethical Considerations}
Training should begin with an overview of AI’s functions, strengths, and limitations to help screenwriters set realistic expectations and avoid over-reliance on AI tools. To collaborate effectively with AI, screenwriters should focus on tasks where AI excels, such as generating ideas, identifying audience trends, and analyzing story structures. For instance, AI can offer diverse plotlines or alternative perspectives during brainstorming, acting as a creative partner. Delegating repetitive tasks like proofreading or formatting to AI allows screenwriters to focus on more creative aspects of storytelling. Understanding AI's capabilities also helps screenwriters address potential threats posed by AI as a competitor. By emphasizing uniquely human strengths, such as emotional depth, cultural intuition, complex narrative crafting, and personal experience, areas where AI currently falls short, screenwriters can maintain their creative edge and highlight the value of human creativity.
Additionally, screenwriters must thoroughly understand the ethical considerations surrounding AI-generated content, particularly issues related to copyright and authorship (as discussed in Section 8.2). Adhering to regulations recognized by relevant stakeholders is essential for promoting accountability, upholding ethical practices, and preserving the distinctiveness of the screenwriter's creative voice.}


\textcolor{black}{\subsubsection{Leveraging Workshops for AI-Integrated Workflow Training.}
To further support screenwriters in their workflows, we can draw inspiration from Kim et al.'s workshop approach for assisting short video producers~\cite{10.1145/3544548.3581225}. Similarly, workshops provide an effective platform for screenwriters to share experiences and insights on AI usage, fostering peer learning to improve efficiency while maintaining human creativity. These workshops can also train participants to design AI-integrated workflows that strategically allocate tasks between AI and humans, leveraging the capabilities of AI discussed earlier. By combining traditional screenwriting workflows with a deeper understanding of how to incorporate AI tools at various stages, screenwriters can enhance their technical knowledge and skills in utilizing AI tools, boost their confidence in using AI, and foster more effective collaboration between humans and AI.}

\end{comment}

%\textcolor{black}{\subsubsection{Designing Hybrid Workflows in Screenwriting through Workshops}
%\textcolor{black}{\subsubsection{Designing Hybrid Workflows in Screenwriting through Workshops} Workshops can be a potential approach to provide an environment for screenwriters to share experiences and insights on AI usage, promoting peer learning to enhance efficiency while preserving human creativity. These workshops can also train participants to design hybrid workflows that strategically allocate tasks between AI and humans, building on the AI capabilities discussed earlier. By integrating traditional screenwriting processes with a deeper understanding of how to incorporate AI tools at various stages, screenwriters can improve their technical skills, increase confidence in using AI, and foster more effective collaboration between humans and AI.}



%\textcolor{black}{\subsubsection{Implementing Hybrid Workflows}  Workshops can encourage screenwriters to share experiences and insights about AI usage and foster peer learning to maximize efficiency while safeguarding human creativity. They can also train participants to design hybrid workflows that strategically balance tasks between AI and humans, building on the previously discussed AI capabilities. By integrating traditional screenwriting workflows with a deeper understanding of how to incorporate AI tools at various stages, screenwriters can enhance their technical skills, boost confidence in AI use, and foster more effective human-AI collaboration.}


%\textcolor{black}{As noted in Section 5.1, screenwriters perceive AI as both a competitor and a collaborator in the creative process. As a competitor, AI challenges traditional screenwriter roles, potentially reducing job opportunities and undermining creative authority. Conversely, as a collaborator, AI offers opportunities for innovation and knowledge expansion. By leveraging AI's analytical and generative strengths, screenwriters can explore diverse narrative structures, unconventional character development, and novel plotlines that may not be achievable through human effort alone.}  

%\textcolor{black}{Our findings also reveal a growing expectation for AI to serve as a collaborator in screenwriting. As discussed in Section 6.3, the rise of independent content creators, such as YouTubers and independent filmmakers, has prompted screenwriters to view AI as a specialized ``expert,'' assisting in skill acquisition and enabling transitions into versatile roles beyond traditional ones, such as screenwriter or director. Additionally, screenwriters expect AI to act as a specialized ``executor'' (Section 6.4), automating tasks like script formatting, background research, and dialogue refinement, thereby allowing screenwriters to focus on core creative activities.}

%\textcolor{black}{To navigate AI's dual role as a collaborator and competitor, targeted training is essential. Stakeholders, including AI developers, educators, and industry leaders, should collaborate to design comprehensive initiatives that equip screenwriters with the skills to integrate AI into their workflows effectively. Building upon the generative AI integration strategies for animation education proposed by Kicklighter et al.~\cite{kicklighter2024empowering}, we propose tailored training approaches for screenwriters to maximize AI's potential in the screenwriting field.}

%\textcolor{black}{\subsubsection{Understanding the Capabilities of AI and Ethic Considerations}Training should begin with an overview of AI’s functions, strengths, and limitations to help screenwriters set realistic expectations and avoid over-reliance on AI tools. To leverage AI as a collaborator, screenwriters should focus on tasks where AI excels, such as generating ideas, identifying audience trends, or analyzing story structures. For instance, AI can offer diverse plotlines or alternative perspectives during brainstorming, acting as a creative partner. Delegating repetitive tasks like proofreading or formatting to AI further allows screenwriters to concentrate on more creative storytelling tasks. Understanding AI's capabilities also helps screenwriters address the potential threats posed by its role as a competitor. By emphasizing uniquely human strengths, such as emotional depth, cultural intuition, complex narrative crafting, and personal experience—areas where AI currently falls short—screenwriters can maintain their creative edge and reinforce the value of human creativity. Screenwriters are supposed to thoroughly understand the ethical considerations surrounding AI-generated content, particularly issues related to copyright and authorship (as discussed in Section 8.2). Adhering to regulations recognized by stakeholders is vital for promoting accountability, upholding ethical practices, and preserving the distinctiveness of the screenwriter's creative voice.}  



%\textcolor{black}{\subsubsection{Adhering to Ethical Standards}  Screenwriters are supposed to thoroughly understand the ethical considerations surrounding AI-generated content, particularly issues related to copyright and authorship (as discussed in Section 8.2). Adhering to regulations recognized by stakeholders is vital for promoting accountability, upholding ethical practices, and preserving the distinctiveness of the screenwriter's creative voice.} 

%\textcolor{black}{\subsubsection{Ethical Implications}  
%Screenwriters are supposed to have a thorough understanding of the ethical considerations surrounding AI-generated content, particularly issues related to copyright and authorship (as discussed in Section 8.2). This understanding is essential for safeguarding their professional integrity. Additionally, documenting AI usage is critical for ensuring transparency in the creative process. Establishing clearer boundaries between human and AI contributions is vital to promoting accountability, and ethical practices, and preserving the distinctiveness of the screenwriter's creative voice.}

%As noted in Section 5.1, screenwriters view AI as both a competitor and a collaborator in the creative process. On one hand, AI poses a threat by encroaching on screenwriters' roles and potentially shrinking their job market. On the other hand, AI helps screenwriters expand their knowledge, and collaboration between AI and screenwriters creates opportunities for narrative innovation, motivating screenwriters to enhance their creativity and explore new possibilities. With the rise of YouTubers and independent filmmakers, as discussed in Section 6.3, screenwriters expect AI to serve as an ``expert'', helping to fill in gaps in their skill sets and allowing them to become more versatile creators.

%However, our interviews revealed that some participants lack the skills to effectively use AI tools. Therefore, it is crucial for screenwriters to acquire the necessary knowledge and skills to integrate AI into their creative processes. To address this, AI experts and other stakeholders should organize training initiatives, such as workshops, courses, and conferences, providing screenwriters with greater exposure to and understanding of AI technologies. These programs should focus on educating screenwriters about AI development and how to apply AI tools at different stages of the screenwriting workflow. This will not only improve their technical skills but also boost their confidence in using AI, fostering deeper human-AI collaboration in screenwriting.

%Previous research stresses the importance of human creativity amidst technological advancements and advocates for interdisciplinary teams to promote responsible AI use in enhancing creativity~\cite{suchacka2021human}. Our study suggests that guidelines should focus on agency and identity in human-AI collaboration, addressing risks such as biases, intellectual property issues, and the homogenization of content. AI's role in complex tasks like character development can both enhance and reduce a screenwriter's control, making it essential for screenwriters to maintain agency by directing AI outputs. The integration of AI may also shift the perception of authorship, with screenwriters potentially becoming curators of AI-generated content rather than sole creators, prompting a reevaluation of creative identity. Preserving narrative voice and style will be key in navigating these changes.


%The results in Section 6.2, which highlight screenwriters' concerns about ethical issues, underscore the urgent need for clear ethical guidelines. Prior research has emphasized the importance of human creativity in the face of technological advancements, exploring AI's impact on the labor market, creativity, and adaptability. These studies advocate for interdisciplinary research teams to support future AI research in sociology and philosophy, promoting the responsible use of AI to enhance creativity~\cite{suchacka2021human}. Additionally, key technomoral values, such as honesty, humility, empathy, care, civility, and flexibility, have been identified as crucial for embedding within any practice involving creative AI techniques~\cite{vallor2016technology, flick2022ethics}. However, this research is insufficient for addressing the specific needs of screenwriting.

%Building on this foundation, our study suggests that future guidelines for screenwriting should specifically consider agency and identity in the context of human-AI collaboration and address the risks of AI perpetuating biases, infringing on intellectual property rights, or contributing to the homogenization of creative content. A critical aspect to consider is how AI tools might impact screenwriters' sense of agency in the creative process. As AI takes on more complex tasks, such as character development and plot structuring, it can both enhance and diminish a screenwriter’s control over the narrative. Reflecting on how screenwriters can maintain or regain agency by becoming more adept at directing AI outputs is essential. This approach would ensure that AI serves as an extension of the screenwriter's creative will rather than acting as an autonomous creator. Moreover, the integration of AI into the creative process could influence how screenwriters perceive their roles, perhaps leading them to see themselves more as curators or directors of AI-generated content rather than as sole authors. This shift may prompt a reevaluation of authorship, with screenwriters needing to define and assert their creative identity in new ways as AI tools become more prevalent. Additionally, AI could challenge or reinforce a screenwriter's unique narrative voice and style. Understanding how screenwriters can navigate these changes to preserve their creative identity in their work will be crucial.

%Overall, stakeholders, including screenwriters, researchers, industry leaders, etc., should collaboratively develop a framework for the responsible use of AI. This framework should ensure that AI remains a tool to enhance human creativity rather than diminish the role of the screenwriter.

%\haotian{I think it might be interesting to recommend one or two potential studies about ethical guidelines. And discuss what they may not cover or are insufficient for screenwriting. it can make the discussion more fruitful.}

\subsection{Limitation and Future Work}

%Although our study provides valuable insights into the integration of AI in screenwriting, it is not without limitations. One of the primary limitations is that all our participants were from China, which may introduce cultural and regional biases in the results. Additionally, the participants’ professional backgrounds and years of screenwriting experience were not directly correlated, as all information was self-reported by the participants. Furthermore, the proficiency levels in AI usage among the participants varied, potentially leading to differing perspectives. However, we did not specifically compare the viewpoints of participants with different AI proficiency levels, which could be a potential direction for future research. As AI capabilities continue to expand, there is also research value in exploring how advancements in AI technology might influence screenwriters' job satisfaction, professional identity, and authorship recognition in the context of human-AI co-creation. Additionally, the development of more intuitive, user-friendly AI interfaces and the commercialization or open-sourcing of advanced AI tools could make these technologies more accessible to a broader range of screenwriters. By addressing these limitations and expanding the scope of research, the field can move closer to realizing the full potential of AI in screenwriting, ultimately providing screenwriters with more innovative and effective tools.

%Although our study provides valuable insights into the integration of AI in screenwriting, it is not without limitations. One primary limitation is that all our participants were from China, which may introduce cultural and regional biases in the results. Additionally, the participants’ professional backgrounds and years of screenwriting experience were not directly correlated, as all information was self-reported by the participants. Furthermore, the proficiency levels in AI usage among the participants varied, potentially leading to differing perspectives. However, we did not specifically compare the viewpoints of participants with different AI proficiency levels, which could be a direction for future research. As AI capabilities continue to expand, there is also value in exploring how advancements in AI technology might influence screenwriters' job satisfaction, professional identity, and authorship recognition in the context of human-AI co-creation. Additionally, the development of more intuitive AI interfaces and the commercialization or open-sourcing of advanced AI tools could make these technologies more accessible to a broader range of screenwriters. By addressing these limitations and expanding the scope of research, the field can move closer to realizing the full potential of AI in screenwriting, ultimately providing screenwriters with more innovative and effective tools.

Although our study provides valuable insights into the integration of AI in screenwriting, it is not without limitations. One primary limitation is that all participants were from China, which may introduce cultural and regional biases in the results. Additionally, the participants’ professional backgrounds and years of screenwriting experience were not directly correlated, as all information was self-reported by the participants. Furthermore, proficiency levels in AI usage among participants varied, potentially leading to differing perspectives. However, we did not specifically compare the viewpoints of participants with different AI proficiency levels, which could be a direction for future research. As AI capabilities continue to expand, there is value in exploring how advancements in AI technology might influence screenwriters' job satisfaction, professional identity, and authorship recognition in the context of human-AI co-creation. Additionally, the development of more intuitive AI interfaces and the commercialization or open-sourcing of advanced AI tools could make these technologies more accessible to a broader range of screenwriters. By addressing these limitations and expanding the scope of research, the field can move closer to realizing the potential of AI in screenwriting, ultimately providing screenwriters with more innovative and effective tools.

\section{Conclusion}
%In this study, we conducted a qualitative analysis of in-depth interviews with 23 screenwriters, gaining insights into the current practices, attitudes, and future roles of AI in screenwriting. We found that the majority of screenwriters (78\%) have already integrated AI into their traditional workflows, particularly in the stages of character, story structure \& plot, and dialogue, where they expressed both the usage and further need for AI assistance. Additionally, we explored screenwriters' attitudes toward AI integration in their workflows and summarized their future AI needs into four distinct roles: actor, audience, expert, and executor. These findings indicate that while the integration of AI into screenplay writing presents promising opportunities, it also poses significant challenges, highlighting the need for collaborative efforts among screenwriters, researchers, and stakeholders to ensure that AI can be seamlessly integrated into each step of the screenwriting workflow. Future work should prioritize the development of advanced AI for emotional and creative content, democratize AI access, facilitate collaborative training, and establish ethical guidelines. Overall, this study provides valuable insights into human-AI co-creation in screenwriting and offers guidance for the development of future AI-assisted tools that enhance creativity, efficiency, and usability.

%In this study, we conducted a qualitative analysis of in-depth interviews with 23 screenwriters to gain insights into the current practices, attitudes, and future roles of AI in screenwriting. We found that the majority of screenwriters (78\%) have already integrated AI into their traditional workflows, particularly in the stages of story structure \& plot, screenplay text, goal \& idea generation, and dialogue. Except for the goal \& idea stage, participants expressed dissatisfaction with the current AI tools, expressing a desire for further AI assistance in other stages. Additionally, we explored screenwriters' attitudes toward AI integration and summarized their future AI needs into four distinct roles: actor, audience, expert, and executor. These findings indicate that while AI integration into screenwriting presents promising opportunities, it also poses significant challenges, emphasizing the need for collaborative efforts among screenwriters, researchers, and stakeholders to ensure AI can be effectively integrated into the screenwriting workflow. Future work should focus on developing advanced AI for multithreaded and multimodal generation, enhancing usability and emotional intelligence, and improving flexibility and adaptation in AI roles, while preserving creative agency and identity, and facilitating collaborative training. Overall, this study provides valuable insights into human-AI co-creation in screenwriting and offers suggestions for the development of future AI tools.

%In this study, we conducted a qualitative analysis of interviews with 23 screenwriters to explore current AI practices, attitudes, and future roles in screenwriting. The majority (78\%) have integrated AI into traditional workflows, particularly in the stages of story structure \& plot, screenplay text, goal \& idea generation, and dialogue. Based on their responses, further AI support is needed in the stages where they primarily use AI, with only the goal \& idea generation stage currently meeting their needs. Additionally, we identified four key AI roles for future development: actor, audience, expert, and executor. According to these findings, we suggest that future research focus on enhancing emotional intelligence, improving flexibility, and advancing AI for multithreaded and multimodal generation. Furthermore, promoting human-AI co-creation in screenwriting should emphasize preserving creative agency and identity, as well as facilitating collaborative training. Overall, this study offers recommendations for the development of future AI tools in screenwriting.

In this study, we conducted a qualitative analysis of interviews with 23 screenwriters to explore their feedback on current practices, attitudes, and future roles of AI in screenwriting. The majority (78\%) have integrated AI into traditional workflows, particularly in the stages of story structure \& plot, screenplay text, goal \& idea generation, and dialogue. \textcolor{black}{Based on their responses, we gained deeper insights into screenwriters' attitudes toward AI integration, which influence various workflow stages and the broader industry.} We also identified four AI roles for future expectations: actor, audience, expert, and executor. \textcolor{black}{Based on these findings, we recommend that future research focus on enhancing emotional intelligence, improving role flexibility, and advancing multithreaded and multimodal generation. Furthermore, promoting human-AI co-creation should prioritize preserving creative agency and identity while facilitating collaborative training.} Overall, this study provides suggestions for future AI tools in screenwriting.

\begin{acks}
This work was partially supported by the Hong Kong RGC GRF grant 16218724 and the EdUHK-HKUST Joint Centre for Artificial Intelligence (JC\_AI) research scheme (Grant No. FB454). We extend our gratitude to Rui Sheng, Linping Yuan, Runhua Zhang, Liwenhan Xie, Xian Xu, and Yujia He for their valuable feedback and discussions on this work. We also sincerely thank all participants for their time and effort. Lastly, we greatly appreciate the reviewers for their insightful and constructive suggestions.
\end{acks}
%, which have significantly improved the quality of this paper.
%This work was partially supported by the Hong Kong RGC GRF grant 16218724 and the EdUHK-HKUST Joint Centre for Artificial Intelligence (JC\_AI) research scheme (Grant No. FB454). We extend our gratitude to Rui Sheng, Runhua Zhang, Liwenhan Xie, and Yujia He for their valuable feedback and discussions on this work. Finally, we sincerely thank all participants for their time and effort.


%In this study, we conducted a qualitative analysis of interviews with 23 screenwriters to explore current AI practices, attitudes, and future roles in screenwriting. The majority (78\%) have integrated AI into traditional workflows, particularly in the stages of story structure \& plot, screenplay text, goal \& idea generation, and dialogue. Based on their responses, further AI support is needed in the stages where AI is primarily used, with only the goal \& idea generation stage currently meeting their expectations. Additionally, we identified four AI roles for future development: actor, audience, expert, and executor. According to these findings, we suggest that future research focus on enhancing emotional intelligence, improving flexibility, and advancing AI for multithreaded and multimodal generation. Moreover, promoting human-AI co-creation in screenwriting should emphasize preserving creative agency and identity, as well as facilitating collaborative training. Overall, this study offers suggestions for the development of future AI tools in screenwriting.

\bibliographystyle{ACM-Reference-Format}
\bibliography{software}

%%
%% If your work has an appendix, this is the place to put it.
\appendix

\end{document}
}
\end
\endinput
%%
%% End of file `sample-sigconf-authordraft.tex'.
