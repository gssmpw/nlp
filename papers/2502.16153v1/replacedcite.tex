\section{Related Work}
\textcolor{black}{Our study builds on prior research on AI integration in creative processes, AI tools for creative writing, and AI-supported screenwriting, highlighting gaps specific to screenwriting. These gaps stem from its unique characteristics, including audiovisual storytelling____, structured principles____, and multi-stakeholder collaboration____. Previous studies have not fully incorporated evolving AI capabilities to consider these screenwriting characteristics, underscoring the need for further investigation.}

%\textcolor{black}{Our study builds on previous research in AI integration in creative processes, AI tools for creative writing, and AI-supported screenwriting.}

%\textcolor{black}{Our study builds on prior research into AI integration in creative processes, AI tools for creative writing, and AI-supported screenwriting, identifying research gaps specific to screenwriting as a unique form of creative writing within the broader creative industry.}

%\textcolor{black}{Our study builds on prior research into AI integration in creative processes, AI tools for creative writing, and AI-supported screenwriting, identifying specific research gaps in screenwriting. These gaps arise from the unique characteristics of screenwriting, including audiovisual storytelling____, structured principles____, and collaboration with multiple stakeholders____, which distinguish it within the broader creative industry.}

%\textcolor{black}{Our study builds on prior research into AI integration in creative processes, AI tools for creative writing, and AI-supported screenwriting, identifying specific research gaps in AI integration for screenwriting. These gaps arise from the unique characteristics of screenwriting, including audiovisual storytelling____, structured principles____, and collaboration with multiple stakeholders____, which distinguish it within the broader creative industry and warrant further attention.}

\subsection{\textcolor{black}{AI Integration in Creative Processes}}

\textcolor{black}{Advancements in AI technology have significantly impacted various creative domains____, reshaping fields such as art____ and design____. In art, AI has contributed to interactive media and visual representation____. In design, AI has been integrated into product design workflows, fostering innovation in idea generation____ and advancing user interface design through features such as adaptability____. Furthermore, the impact of AI-generated image tools on workflows has been explored____, highlighting AI’s potential to democratize graphic design____ and emphasizing the need for interdisciplinary collaboration to enhance AI systems____. As AI-generated video capabilities advance, AI is also reshaping fashion design through dynamic visuals____ and supporting the creation of short-form social media videos____. Additionally, other research highlights AI's role in various creative domains, such as design____, and data storytelling____.}

%\textcolor{black}{The growing importance of effective human-AI co-creation, as highlighted in prior research within visual-based creative domains, motivates us to investigate whether the diverse AI capabilities and applications in these areas have similarly impacted text-based creative works.}

The growing importance of effective human-AI co-creation, as highlighted in prior research within visual-based creative domains, motivates us to investigate how diverse AI capabilities and applications have impacted text-based creative work.


%\textcolor{black}{The significance of effective human-AI co-creation has become increasingly evident, building on prior research in these visual-based creative domains. This inspires us to explore whether the diverse AI capabilities and applications observed in visual-based domains have also influenced text-based creative works.}


\subsection{\textcolor{black}{AI Tools for Creative Writing}}
\textcolor{black}{Creative writing, as a text-based creative work encompassing novels, short stories, poetry, and essays, emphasizes linguistic artistry, stylistic diversity, and imaginative narrative freedom____. The development of AI technologies has significantly supported this creative field____, evolving from natural language processing to advanced deep learning systems. This progression spans basic tools like spell checkers____, crowdsourcing platforms____, and crowd role-play approaches____, to AI-driven character role-playing systems____. Prior studies have shown that AI reduces creative workloads by providing valuable suggestions____ and refining content for professional use____, thereby helping writers overcome creative blocks. Specifically, AI-supported creative writing processes are currently categorized into story development and character creation____.}

%\textcolor{black}{Creative writing, as a text-based creative work encompassing novels, short stories, poetry, and essays, emphasizes linguistic artistry, stylistic diversity, and imaginative freedom____. The development of AI technologies has significantly supported this field____, evolving from natural language processing to advanced deep learning systems. This progression spans basic tools like spell checkers____, crowdsourcing platforms____, and crowd role-play approaches____, to AI-driven character role-playing systems____. Prior studies have shown that AI reduces creative workloads by providing valuable suggestions____ and refining content for professional use____, thereby helping writers overcome creative blocks. Specifically, AI-supported creative writing processes are currently categorized into story development and character creation____.}

\textcolor{black}{In story development, platforms such as Dramatica____ and AI Dungeon____ support story generation. TaleStream combines AI recommendation algorithms with database-driven trope suggestions for narratives but has yet to integrate evolving generative AI technologies____. Rule-based methods for story extension have also been explored____. However, we found that these approaches primarily focus on text-based support and do not incorporate visual elements. In character creation, prior works demonstrate that AI can simulate characters' personalities____, infer characters' relationships____, facilitate AI-character dialogues____, and map character trajectories____. These tools showcase the potential of AI for inspiring character development____, but they are often limited to isolated stages and lack integration across the creative workflow. Additionally, challenges in human-AI collaboration remain. Grigis and Angeli recently examined the limitations of LLM-assisted writing, particularly in handling taboos and conflicts____. Dhillon et al. emphasized the importance of personalized systems to accommodate the diverse experiences of writers____, and Lee et al. provided broad recommendations for the development of future AI tools____. Meanwhile, Biermann et al. underscored writers’ concerns about the potential impact of AI on creativity____.}

\textcolor{black}{Based on these, we note that existing research largely focuses on text-based AI support, often overlooking the consideration of visual elements in the creative process. 
This limitation reduces the suitability of these approaches for certain types of creative writing, particularly screenwriting, which is inherently characterized by audiovisual and dynamic storytelling____. Furthermore, existing studies often focus on isolated stages of the creative process, overlooking the complete workflow that integrates structured principles____ and collaboration with stakeholders____, both of which are essential to screenwriting. These gaps underscore the need to explore AI's current applications in screenwriting.}


%\textcolor{black}{Based on these, we note that existing works primarily focus on text-based AI support formats, often overlooking the need to integrate visual elements in the creation processes. 
%\textcolor{black}{Based on these, we note that existing research predominantly emphasizes text-based AI support, often neglecting the importance of integrating visual elements into the creative process. 
%\textcolor{black}{Based on these, we note that existing research largely focuses on text-based AI support, overlooking the role of visual elements in the creative process. This limitation reduces the suitability of these approaches for certain types of creative writing, particularly screenwriting, which is inherently characterized by audiovisual storytelling____. Furthermore, existing studies often focus on isolated stages of the creative process, overlooking the complete workflow that integrates structured principles____ and collaboration with stakeholders____, both of which are essential to screenwriting. These gaps underscore the need to explore AI's current applications in screenwriting.}

%Additionally, challenges in human-AI collaboration persist. Lee et al. offered recommendations for future AI tools____, and Dhillon et al. emphasized the need for personalized systems to accommodate writers’ varying experiences____. Meanwhile, Biermann et al. highlighted writers’ concerns about AI’s influence on creativity____.}

%\textcolor{black}{Based on these, we note that existing works primarily focus on text-based output formats for AI support, often overlooking the potential of visual elements. This limitation reduces the suitability of these approaches for certain types of creative writing, particularly screenwriting, which is inherently characterized by audiovisual storytelling____. Additionally, existing studies frequently address isolated stages of the creative process, neglecting the entire workflow combined with structured principles____ and the collaboration with stakeholders____, both of which are integral to screenwriting. These gaps underscore the need to explore AI's current applications in screenwriting.}
%These gaps underscore the importance of exploring AI applications in screenwriting and investigating how AI can be effectively integrated into screenwriting workflows.}

\subsection{\textcolor{black}{AI-Supported Screenwriting}}

\textcolor{black}{The development of AI technologies has already influenced screenwriting____. Previous applications of AI tools in screenwriting have primarily focused on three areas: information processing, emotional support, and visualization.}

%\textcolor{black}{For information processing, AI tools assist with retrieving background information____, ensuring narrative consistency____, and supporting story continuation____. Traditional tools like Final Draft____ integrate AI for scene organization and formatting, while AI systems like ChatGPT____ and DeepStory____ offer predictive analytics and co-writing capabilities____, streamlining the screenwriting process. For emotion support, earlier studies have explored emotion analysis in narratives____. Goyal et al.'s AESOP system analyzes emotions through plot units, helping screenwriters understand emotional trajectories____. However, it relies on basic emotion tags (e.g., positive, negative, neutral), failing to capture complex emotional dynamics. Su et al.'s work on simulating basic and mixed emotions____ improved character expression but lacked deeper insight into emotions tied to plans and reasoning, limiting its integration into the screenwriting workflow. For visualization, previous research is divided into data visualization and visual representation. In data visualization, studies have summarized storylines____ and managed character arcs____, providing a foundation for data-driven AI systems. In visual representation, AI tools use NLP to create 2D and 3D visualizations of screenplay content____, including character interactions____ and scenes____. However, these tools lack real-time feedback, and suitable visual representation, and rely on the quality of screenplay content. Beyond these main types of previous research, a recent work closely related to ours is the Dramatron system____, which explores the integration of LLMs into screenwriting. However, it focuses only on LLMs, neglecting the impact of AI-generated images and videos.}

%\textcolor{black}{Previous empirical research shows that AI can generate screenplay elements comparable to those of human writers____, using tools like ChatGPT____ and DeepStory____, enhancing efficiency, reducing costs____, and offering inspiration____. However, human creativity remains indispensable____. Furthermore, ethical concerns such as biases____ and copyright issues____ remain significant challenges. However, these studies have neither explored the specific ways AI has been integrated into different stages of screenwriting nor examined how screenwriters' attitudes concretely influence their actual practices.}

\textcolor{black}{For information processing, AI tools assist with retrieving background information____, ensuring narrative consistency____, and organizing content____. Tools like Final Draft____ integrate AI for scene management and formatting, streamlining the screenwriting process. For emotional support, earlier studies have explored emotion analysis in narratives____. Goyal et al.'s AESOP system analyzes emotions through plot units, helping screenwriters understand emotional trajectories____. However, it relies on basic emotion tags (e.g., positive, negative, neutral), failing to capture complex emotional dynamics. Su et al.'s work on simulating basic and mixed emotions____ improved character expression but lacked deeper insights into emotions tied to plans and reasoning, limiting its integration into screenwriting workflows. For visualization, previous research is divided into data visualization and visual representation. In data visualization, studies have summarized storylines____ and managed character arcs____, providing a foundation for data-driven AI systems. In visual representation, AI tools use NLP to create 2D and 3D visualizations of screenplay content____, including character interactions____ and scenes____. However, these tools lack real-time feedback, personalized representation styles, and rely heavily on the quality of the screenplay content. Beyond these primary research areas, a recent work closely related to ours is the Dramatron system____, which explores the integration of LLMs into screenwriting. However, it focuses solely on LLMs, neglecting the impact of AI-generated images and videos.}

\textcolor{black}{Additionally, previous empirical research has shown that AI can generate screenplay elements comparable to those created by human writers____, leveraging tools like ChatGPT____ and DeepStory____ to enhance efficiency, reduce costs____, and inspire creativity____. Despite these advancements, human creativity remains irreplaceable____. Meanwhile, ethical concerns, including biases____ and copyright issues____, have sparked significant debate. However, these studies have not investigated how AI is integrated into specific stages of screenwriting and how screenwriters' attitudes concretely influence their practices.}

%\textcolor{black}{Additionally, previous empirical research has demonstrated that AI can generate screenplay elements comparable to those created by human writers____, utilizing tools such as ChatGPT____ and DeepStory____ to enhance efficiency, reduce costs____, and provide creative inspiration____. Despite these advancements, human creativity remains indispensable____. Meanwhile, ethical concerns, such as biases____ and copyright issues____, have also sparked considerable discussion. However, these studies have not explored the specific ways in which AI is integrated into different stages of screenwriting, nor have they examined how screenwriters' attitudes concretely influence their practices.}

\textcolor{black}{To address these gaps, our study aims to examine screenwriters' current practices and attitudes toward AI integration at various stages within the workflow. Additionally, we investigate screenwriters’ expectations for future AI tools, considering both the potential of emerging AI technologies and possibilities beyond current advancements.} 
Ultimately, we provide suggestions for designing tailored human-AI co-creation tools that meet screenwriters' needs.

%\textcolor{black}{To address these gaps, our study aims to examine screenwriters' current practices and attitudes toward AI integration at specific stages of their workflows. Then, we investigate screenwriters’ expectations for future AI tools, taking into account both the potential of emerging AI technologies and possibilities beyond current advancements. Ultimately, our findings aim to offer suggestions for designing human-AI co-creation tools for screenwriting that are specifically tailored to the needs of screenwriters.}

%\textcolor{black}{To address these gaps, our study aims to explore screenwriters' current practices and attitudes toward AI integration at specific stages of the workflow. Additionally, it investigates screenwriters’ expectations for future AI tools, considering both the capabilities of evolving AI technologies and possibilities beyond current advancements. Overall, our findings seek to provide suggestions for the future design of human-AI co-creation tools for screenwriting that are tailored to the specific needs of screenwriters.}

%\textcolor{black}{Building on prior research in visual-based creative domains, the significance of effective human-AI co-creation has become increasingly evident. This inspires us to investigate whether the diverse AI capabilities and applications demonstrated in visual-based creative tasks have influenced text-based creative works.}

%\textcolor{black}{Advancements in AI have transformed creative domains such as the arts and design____. In art, AI supports interactive media and visual representation____. In design, it fosters innovation in product workflows____ and advances user interface adaptability____. AI-generated image tools further democratize graphic design____, with interdisciplinary collaboration advocated to improve AI systems____. As AI-generated video capabilities expand, AI is transforming fashion design____ and assisting social media creators____.}

%\textcolor{black}{Beyond these visual-based creative domains, AI’s impact also extends to diverse areas such as data-driven work____ and text-based work____. Notably, creative writing, as a text-based domain, increasingly emphasizes human-AI co-creation. For instance, Kim et al. investigated AI's role in creative language arts, such as poetry and novels____, highlighting the evolving significance of AI tools in enhancing the creative process.}


%\subsection{\textcolor{black}{General-Purpose AI Tools for Creative Writing}}


%\textcolor{black}{Creative writing, encompassing novels, short stories, poetry, and essays, prioritizes linguistic artistry, stylistic diversity, and imaginative freedom____. AI technologies have advanced this field, evolving from natural language processing to deep learning systems. Tools range from basic spell checkers____,  crowdsourcing____, crowd role-play____, to AI-driven character role-playing____, reducing workloads and aiding writers in overcoming creative blocks____. Current applications focus on story development____ and character creation____.}

%\textcolor{black}{For story development, rule-based methods for story extension____ and platforms like Dramatica____ and AI Dungeon____ facilitate text-based story creation but lack integration of visual elements. TaleStream integrates database-driven trope suggestions but lacks generative AI capabilities____. For character creation, AI has demonstrated its ability to simulate virtual personalities____, infer relationships____, assist in dialogues____, and map character trajectories____. While these tools inspire character development____, they remain limited to isolated stages and lack integration across workflows. Additionally, challenges in human-AI collaboration persist, with calls for personalized systems____, improved tool design____, and attention to creativity concerns____.}

%\textcolor{black}{Overall, prior works primarily focus on text-based solutions, neglecting the potential of visual-based approaches. This limitation renders them unsuitable for screenwriting, which is fundamentally related to audiovisual elements. Furthermore, existing studies often target specific stages of the creative process, failing to address the structured workflows and involvement of multiple stakeholders characteristic of screenwriting. These gaps underscore the need for targeted research to investigate AI applications and their integration throughout the entire screenwriting workflow.}


%\subsection{\textcolor{black}{Specialized AI Tools and Empirical Studies for Screenwriting}}

%\textcolor{black}{Previous AI tool applications in screenwriting primarily assist in three areas: information processing, emotion support, and visualization. For information processing, AI aids in background retrieval____, narrative consistency____, and story continuation____. Emotion support focuses on analyzing and simulating emotions in narratives____, but existing systems rely on basic emotion tags and lack the nuance required for screenwriting. In visualization, AI tools range from data visualization____ to character and scene representation____, but they lack real-time feedback and depend on high-quality screenplay input. Overall, while these tools____ support screenwriters, they often fail to address needs such as integrating emotional depth and narrative complexity. Furthermore, they do not incorporate advances in AI technologies that could offer new solutions.}

%\textcolor{black}{Empirical research shows that AI can generate screenplay elements comparable to those of human writers____, using tools like ChatGPT____ and DeepStory____, enhancing efficiency, reducing costs____, and offering inspiration____. However, human creativity remains indispensable____. Recent studies have explored LLM-assisted screenwriting but focus exclusively on LLMs____, neglecting advancements in AI-generated images and videos, leaving the broader potential of evolving AI technologies underexplored. To address these gaps, our study investigates screenwriters’ expectations for future AI tools to meet evolving needs.}

%\textcolor{black}{Furthermore, ethical concerns such as biases____ and copyright issues____ remain significant challenges. Understanding screenwriters' current practices and attitudes toward AI integration at specific workflow stages could inform strategies to mitigate these issues. In general, our findings aim to offer guidelines for the future design of human-AI co-creation tools tailored to the needs of screenwriters.}

%Our study builds upon previous research in AI integration in Creative Processes, General-Purpose AI Tools for Creative Writing, and Specialized AI Tools and Empirical Studies for Screenwriting.}
%AI-supported creative writing, AI-powered tools in screenwriting, the application of AI in other creative processes, and empirical studies in screenwriting.


%Creative writing, encompassing novels, short stories, poetry, and essays, prioritizes linguistic artistry, stylistic diversity, and authorial freedom____. The evolution of AI in creative writing has progressed from simple spell checkers____ to advanced ML and NLP applications for human-AI co-creation in plot development, character creation, and content review____. Previous research has synthesized findings from 115 articles on AI-assisted writing, offering recommendations for future tools____.

%One of AI’s most impactful contributions of previous works has been its ability to reduce the creative workload for writers. AI tools offer valuable suggestions____, refine content in professional contexts____, and help writers overcome creative bottlenecks. Studies have examined the attitudes of writers toward these generative AI tools, revealing concerns about AI's role in shaping creative ideas____. Platforms like HaLLMark Effect have been proposed to facilitate collaboration between writers and AI models, further streamlining the creative process____.

%AI tools for narrative structuring and storytelling including commercial tools like Dramatica \footnote{https://dramatica.com/}, and AI Dungeon \footnote{https://aidungeon.com/} assist in narrative structuring, while ASM's structured story model provides computational understanding for cultural storytelling differences____. Methods for extending stories based on rules and objectives____ and tools like TaleStream____ have also been explored. AI has been utilized in crowdsourced story creation and structuring creative leadership____. AI has also been employed for specific creative tasks such as newspaper headline creation____ and poetry generation____.

%AI tools for narrative structuring and creative tasks include commercial platforms like Dramatica\footnote{https://dramatica.com/} and AI Dungeon\footnote{https://aidungeon.com/}, while ASM's structured story model provides a computational understanding of cultural differences in storytelling____. Methods for extending stories based on rules and objectives have been explored____, along with tools like TaleStream____. Additionally, AI has been utilized in crowdsourced story creation and creative leadership structuring____, as well as for specific creative tasks such as newspaper headline creation____ and poetry generation____.


%In character development, research has focused on virtual character simulation____, relationship inference____, and tools like TaleBrush for mapping character destinies____. Studies indicate that LLM-driven systems inspire character creation through dialogue____. Collaborative efforts have also examined tasks like newspaper headline creation____ and poetry generation____.

%AI in character development has been another key focus. Tools have been developed for simulating virtual characters____, inferring character relationships____, and mapping character destinies through systems like TaleBrush____. Research highlights the use of large language models (LLMs) in generating character dialogue and inspiring new characters____. 

%These AI tools reduce creative workload____, offer suggestions____, and assist in professional settings____. Writers' attitudes toward generative AI tools have been studied, with concerns about AI's role in transforming creative ideas into text____. Platforms like HaLLMark Effect have been proposed to enhance collaboration between writers and LLMs____. Research shows varying interactions based on writers' experience levels, emphasizing the need for personalized, user-centered AI tools____. However, existing studies often focus on general writers, and limited control may restrict AI's applicability in specialized domains____.

%User experience in AI-assisted writing has also garnered attention. Paramveer S. Dhillon et al. indicate that writers' interactions with AI tools vary based on their level of experience, emphasizing the importance of designing personalized, user-centered AI systems____. While AI’s potential to assist in creative writing is evident, its current applications often focus on general writing tasks. However, there is limited control for domain-specific needs, such as in screenwriting, where the demands for structuring narrative, dialogue, and character arcs are more complex____. Our research specifically addresses this gap by providing insights into screenwriting as a distinct domain and offering tailored AI applications to better support screenwriters' workflows.

%Creative writing, encompassing novels, short stories, poetry, and essays, prioritizes linguistic artistry, stylistic diversity, and authorial freedom____. The emergence of AI in creative writing has introduced potential transformations, evolving from early attempts at simple spell checkers____ to advanced deep learning applications for human-AI co-creation in structure development, character creation, and content review____, increasingly influencing the creative process. Specifically, AI tools for narrative structuring in creative writing include commercial platforms like Dramatica\footnote{https://dramatica.com/} and AI Dungeon\footnote{https://aidungeon.com/}, etc. Methods for extending stories based on rules and objectives have been explored____, along with tools like TaleStream____. Additionally, AI has been utilized in crowdsourced story creation and creative leadership structuring____, as well as for specific creative tasks such as newspaper headline creation____ and poetry generation____. Moreover, AI assistance in character development has also been another key focus, with tools developed for simulating virtual characters____, inferring character relationships____, and mapping character destinies through systems like TaleBrush____. Research also highlights the use of large language models (LLMs) in generating character dialogue and inspiring new characters____. 

%Furthermore, user experience in AI-supported creative writing has garnered attention. The main contributions of AI in previous works typically lie in its ability to reduce the creative workload for writers by offering valuable suggestions____, refining content in professional contexts____, and helping writers overcome creative bottlenecks. Platforms like HaLLMark Effect have been proposed to facilitate collaboration between writers and AI models, streamlining the creative process____. However, challenges in human-AI collaboration are also arisen in this field. Biermann et al. have examined the attitudes of writers toward these generative AI tools, revealing concerns about AI's role in shaping creative ideas____. Dhillon et al. indicate that writers' interactions with AI tools vary based on their experience level, emphasizing the importance of designing personalized, user-centered AI systems____. Lee et al. synthesized findings from 115 articles on AI-assisted writing, offering recommendations for future tools____. Through previous research, we found that AI's evident potential to assist in creative writing. However, its current applications often focus on general writing tasks. There is limited support for domain-specific needs, such as in screenwriting, where the demands for structure, plot, dialogue, and character arcs are more complex____. Therefore, our research specifically addresses this gap by providing insights into screenwriting as a distinct domain and offering suggestions for developing AI functions to better support screenwriters.

%\subsection{AI-supported Creative Writing}

%Creative writing, including novels, short stories, poetry, and essays, emphasizes linguistic artistry, stylistic diversity, and imaginative freedom____. \textcolor{black}{The integration of AI has transformed this field, evolving from basic spell checkers____ to advanced deep learning systems. This progression ranges from crowdsourcing____ and crowd role-play____ to AI-driven character role-playing____, enhancing creative processes. Previous studies demonstrate that AI reduces creative workloads by providing valuable suggestions____ and refining content for professional use____, thereby assisting writers in overcoming creative blocks. Currently, AI-supported creative writing processes are commonly divided into story development and character creation____.}

%Creative writing, including novels, short stories, poetry, and essays, emphasizes linguistic artistry, stylistic diversity, and imaginative freedom____. \textcolor{black}{The development of AI technologies has supported this field, evolving from natural language processing to advanced deep learning systems. This progression ranges from basic spell checkers____, to crowdsourcing____, and crowd role-play____, evolving into AI-driven character role-playing____, all of which enhance creative processes. Previous studies have shown that AI reduces creative workloads by providing valuable suggestions____ and refining content for professional use____, thus helping writers overcome creative blocks. Specifically, AI-supported creative writing processes are currently divided into story development and character creation____.}

%\textcolor{black}{For story development, platforms like Dramatica____ and AI Dungeon____ support story generation, while TaleStream combines AI recommendation algorithms with database-driven trope suggestions for narratives but has yet to integrate evolving generative AI technologies____. Rule-based methods for story extension have also been explored____. We found that these approaches primarily focus on text-based support and do not incorporate visual elements. For character creation, previous works show that AI simulates virtual personalities____, infers relationships____, facilitates AI-character dialogues____, and maps character trajectories____. These tools demonstrate various potentials of AI for inspiring character development____, but they remain limited to isolated stages and lack integration across creative workflows. Additionally, challenges in human-AI collaboration persist. Lee et al. offered recommendations for future AI tools____, and Dhillon et al. emphasized the need for personalized systems to accommodate writers’ varying experiences____. Meanwhile, Biermann et al. highlighted writers’ concerns about AI’s influence on creativity____. }

%\textcolor{black}{Based on these, we observed that previous works primarily focus on text-based supports and challenges, neglecting the needs and potential for supporting audiovisual storytelling. This limitation renders these approaches inadequately suited to the field of screenwriting, which is inherently characterized by audiovisual storytelling. Furthermore, these works lack an understanding of AI integration across the entire creative process, highlighting the necessity of further exploring AI’s current applications and potential requirements within the screenwriting workflow.}


\begin{comment}

\textcolor{black}{Creative writing, including novels, short stories, poetry, and essays, emphasizes linguistic artistry, stylistic diversity, and imaginative freedom____. The integration of AI has transformed this field, evolving from basic spell checkers____ to advanced deep learning systems. This progression ranges from crowdsourcing____ and crowd role-play____ to AI-driven character role-playing____, enhancing creative processes. Previous studies demonstrate that AI reduces creative workloads by providing valuable suggestions____ and refining content for professional use____, thereby assisting writers in overcoming creative blocks. Currently, AI-supported creative writing processes are commonly divided into story development and character creation____.}

\textcolor{black}{For story development, platforms like Dramatica____ and AI Dungeon____ support story generation, while TaleStream combines AI recommendation algorithms with database-driven trope suggestions for narratives but has yet to integrate evolving generative AI technologies____. Rule-based methods for story extension have also been explored____. We found that these approaches primarily focus on text-based support and do not incorporate visual elements. For character creation, AI simulates virtual personalities____, infers relationships____, facilitates AI-character dialogues____, and maps character trajectories with tools like TaleBrush____. These tools demonstrate various potentials of AI for inspiring character development____, but they remain limited to isolated stages and lack integration across creative workflows. Additionally, challenges in human-AI collaboration persist. Biermann et al. highlighted writers’ concerns about AI’s influence on creativity____. Dhillon et al. emphasized the need for personalized systems to accommodate writers’ varying experiences____. Lee et al. synthesized findings from 115 studies, offering recommendations for future AI tools____. }

\textcolor{black}{Based on these findings, we observed that most research focuses on text-based creative writing, neglecting the unique challenges of audiovisual storytelling, such as screenwriting, and lacks an understanding of AI integration across the entire creative process. This leaves critical gaps about how AI can support audiovisual storytelling processes, including various stages such as idea generation, plot development, and dialogue creation, etc. To address these gaps, our study explores screenwriters’ needs across different workflow stages within the context of evolving technologies and proposes design opportunities for future AI tools tailored to screenwriting.}

\textcolor{black}{Creative writing, including novels, short stories, poetry, and essays, emphasizes linguistic artistry, stylistic diversity, and imaginative freedom____. The integration of AI has transformed this field, evolving from basic spell checkers____ to advanced deep learning systems. This progression spans crowdsourcing____, crowd role-play____, and AI-driven character role-playing____, enhancing creative processes. Studies show that AI reduces creative workloads by providing valuable suggestions____ and refining content for professional use____, thereby assisting writers in overcoming creative blocks. Currently, AI-supported creative writing processes are commonly divided into story development and character creation____.}

\textcolor{black}{For story structure, platforms like Dramatica____ and AI Dungeon____ support narrative structuring, while TaleStream generates trope suggestions____. Rule-based methods for story extension have also been explored____. For character creation, AI simulates virtual personalities____, infers relationships____, facilitates AI-character dialogues____, and maps character trajectories with tools like TaleBrush____. These tools demonstrate AI’s potential for inspiring character creation____ but remain limited to isolated stages of development and lack integration across entire creative workflows.}

\textcolor{black}{Challenges in human-AI collaboration persist. Biermann et al. highlighted writers’ concerns about AI’s influence on creativity____, while Dhillon et al. emphasized the need for personalized systems to accommodate writers’ varying experiences____. Lee et al. synthesized findings from 115 studies, offering recommendations for future AI tools____. However, most research focuses on text-based creative writing, neglecting the unique challenges of audiovisual storytelling, such as screenwriting. This leaves critical questions about how AI can support processes like idea generation, plot development, dialogue creation, and visual integration. To address these gaps, this research explores screenwriters’ needs and proposes design guidelines for future AI tools tailored to screenwriting.}

Creative writing, encompassing novels, short stories, poetry, and essays, emphasizes linguistic artistry, stylistic diversity, and the freedom of imagination____. The integration of AI into this field has brought transformative changes, evolving from early spell checkers____ to advanced deep learning systems. This progression, from crowdsourcing____ and crowd role-play____ to AI-driven character role-playing____, further enhances creative writing. Previous studies demonstrate that AI reduces creative workloads by providing valuable suggestions____ and refining content for professional use____, thereby assisting writers in overcoming creative blocks. Currently, AI-supported creative writing processes are commonly divided into story development and character creation____.

For story structure development, platforms such as Dramatica____ and AI Dungeon____ assist with narrative structuring. Previous research has explored rule-based methods for story extension____, and TaleStream aids story ideation by generating trope suggestions____. However, these studies primarily address text-based storytelling, offering limited approaches to audiovisual narratives. For character creation, AI has been utilized to simulate virtual personalities____, infer relationships____, facilitate writing through AI-character dialogues____, and map character trajectories for story generation using tools like TaleBrush____. While these advancements demonstrate AI’s role in inspiring character creation____, they remain limited to assisting individual stages of character and story development and lack integration with the broader creative writing workflow.

Moreover, challenges in human-AI collaboration persist. Biermann et al. examined writers’ concerns about AI's influence on creative ideas____, while Dhillon et al. highlighted the varying interactions with AI tools based on writers’ experience, emphasizing the need for personalized, user-centered systems____. Lee et al. synthesized findings from 115 studies on AI-assisted writing, offering recommendations for improving future tools____. Despite these advancements, most research focuses on text-based creative writing, paying insufficient attention to the challenges of audiovisual storytelling, such as screenwriting. This gap leaves unanswered questions about how AI can support the creation processes of screenwriting, such as idea generation, plot development, dialogue creation, and the integration of visual elements. Therefore, this research aims to address these limitations in audiovisual storytelling by understanding screenwriters' needs and offering design guidelines for future AI tools tailored to screenwriting.

    

Creative writing, which includes novels, short stories, poetry, and essays, emphasizes linguistic artistry, stylistic diversity, and authorial freedom____. The introduction of AI into creative writing has led to potential transformations, evolving from early spell checkers____ to advanced deep learning tools for co-creation in structure development, character creation, and content review____. Commercial platforms like Dramatica____ and AI Dungeon____ provide AI-assisted narrative structuring. Rule-based methods for story extension have also been explored____, along with tools like TaleStream____. AI is further used in crowdsourced story creation and creative leadership____, as well as specific tasks like headline generation____ and poetry writing____. AI tools also aid in character development through simulating virtual characters____, inferring relationships____, and mapping character fates with systems like TaleBrush____. Large language models (LLMs) have been utilized to generate dialogue and inspire character creation____.

User experience in AI-assisted creative writing has also been explored, with AI reducing creative workloads by offering valuable suggestions____, refining content in professional settings____, and helping writers overcome creative blocks. Platforms like HaLLMark Effect have been introduced to streamline collaboration between writers and AI models____. However, challenges in human-AI collaboration persist. Biermann et al. revealed writers' concerns about AI's influence on creative ideas____, while Dhillon et al. highlighted how interactions with AI tools vary based on writers' experience, underscoring the need for personalized, user-centered systems____. Lee et al. synthesized findings from 115 articles on AI-assisted writing, providing recommendations for future tools____. While AI shows promise in creative writing, its current applications are generally limited to basic tasks. There remains a lack of support for the more complex demands of screenwriting, such as structure, plot, dialogue, and character development____. Our research addresses this gap by offering insights into screenwriting as a distinct domain and proposing AI functionalities tailored to screenwriters' needs.
\end{comment}

%\subsection{AI-powered Tools in Screenwriting}

%\textcolor{black}{The development of AI technologies has impacted screenwriting____. Previous research highlights that AI can primarily assist screenwriting in three areas: information processing, emotion support, and visualization.}

%\textcolor{black}{For information processing, AI tools assist with retrieving background information____, ensuring narrative consistency____, and supporting story continuation____. Traditional tools like Final Draft____ integrate AI for scene organization and formatting, while AI systems like ChatGPT____ and DeepStory____ offer predictive analytics and co-writing capabilities____, streamlining the screenwriting process. For emotion support, earlier studies have explored emotion analysis in narratives____. Goyal et al.'s AESOP system analyzes emotions through plot units, helping screenwriters understand emotional trajectories____. However, it relies on basic emotion tags (e.g., positive, negative, neutral), failing to capture complex emotional dynamics. Su et al.'s work on simulating basic and mixed emotions____ improved character expression but lacked deeper insight into emotions tied to plans and reasoning, limiting its integration into the screenwriting workflow. For visualization, previous research is divided into data visualization and visual representation. In data visualization, studies have summarized storylines____ and managed character arcs____, providing a foundation for data-driven AI systems. In visual representation, AI tools use NLP to create 2D and 3D visualizations of screenplay content____, including character interactions____ and scenes____. However, these tools lack real-time feedback, and suitable visual representation, and rely on the quality of screenplay content. Beyond these main types of previous research, a recent work closely related to ours is the Dramatron system____, which explores the integration of LLMs into screenwriting. However, it focuses only on LLMs, neglecting the impact of AI-generated images and videos.}

%\textcolor{black}{Overall, while these tools assist screenwriters, many have limitations that could be addressed by evolving AI technologies. Our study aims to understand how screenwriters expect AI to be applied and offer suggestions for future system design, thereby filling gaps in the field of screenwriting.}

%\textcolor{black}{For information processing, AI tools help retrieve story background information____, ensure narrative consistency____, and support story continuation____. Traditional screenwriting tools, such as Final Draft____, integrate AI for scene organization and formatting, while AI systems like ChatGPT____ and DeepStory____ provide predictive analytics and co-writing capabilities____, reducing manual effort and improving screenplay structure.} \textcolor{black}{For emotion support, previous works explore emotion analysis and simulation in narratives____. Goyal et al.'s AESOP system uses AI to analyze emotions through plot units and generate emotional states related to characters____, aiding screenwriters in understanding emotional trajectories. However, it relies on basic emotion tags (e.g., positive, negative, neutral), missing more complex emotional dynamics. Su et al.'s work on simulating basic and mixed emotions____ enhanced character expression but lacked a nuanced understanding of emotions tied to plans and reasoning, limiting its integration across the screenwriting workflow.} \textcolor{black}{For visualization, previous work can be divided into data visualization and visual representation. In data visualization, prior research has summarized storylines____ and managed character arcs____, providing a foundation for data-driven AI systems. In visual representation, AI tools primarily use NLP to generate 2D and 3D visualizations of screenplay content____, including character interactions____ and scenes____. However, these tools lack real-time feedback and a suitable visual representation and depend on the quality of the screenplay content.}

%\textcolor{black}{Furthermore, recent work, the Dramatron system____, highlights the potential of integrating LLMs into screenwriting but focuses only on LLMs, without addressing the impact of AI-generated images and videos.} 


\begin{comment}
As generative AI evolves, its growing impact on screenwriting, combining creative writing and film production, has become evident____. While screenwriting shares imaginative elements with creative writing, it focuses on storytelling through visual and auditory media____.

Traditional screenwriting tools like Final Draft____, now enhanced with AI, assist in organizing scenes and formatting screenplays. AI tools have also advanced in areas such as information retrieval, logical consistency checks, and emotional state mapping____, with applications like AESOP identifying and mapping character emotions____. Additionally, AI is widely used in collaborative screenwriting and story continuation____. Visualization is also a key focus of previous AI exploration in screenwriting, with tools like Cardinal visualizing structure, characters, and scenes using NLP____, and Story Explorer managing character arcs and event chronology____. Other tools visualize multi-character interactions and generate scene possibilities____, automatically summarize character interactions____, and create multi-character animations____. There are also several additional tools support tangible storyboard creation____ and fictional world-building____.

Recent advancements, including ChatGPT____ and DeepStory____, offer predictive analytics and co-writing capabilities____. The system Dramatron, developed by Mirowski et al.____, demonstrated the potential of integrating LLMs into screenwriting workflows. However, these studies focused exclusively on LLM technology, overlooking developments in other areas such as AI-generated images and videos. Therefore, our study aims first to understand screenwriters' current workflows and then explore the specific ways in which AI is used within this workflow.
\end{comment}

%Dramatron, for instance, uses large language models to generate coherent screenplays through hierarchical text generation, from log lines to complete scripts____. However, these tools often follow a linear process, while screenwriting is typically nonlinear, involving frequent exploration, iteration, and remixing____. Therefore, our study aims first to understand screenwriters' current flexible and variable nonlinear workflows and then explore the specific ways in which AI is used within this workflow.



%Generative AI's growing impact on screenwriting, a blend of creative writing and film production, is becoming more evident____. While screenwriting shares elements with creative writing, it focuses on storytelling through visual and auditory media____.

%Traditional tools like Final Draft \footnote{http://www.http://finaldraft.com.} and Adobe Story \footnote{https://story.adobe.com.}, now enhanced with AI, organize scenes and format screenplays. AI also assists in information retrieval, logical consistency checks, and emotional state mapping____. For example, AESOP identifies and maps character emotional states____. AI has also been applied to collaborative screenwriting and continuation____.


%Traditional screenwriting tools like Final Draft \footnote{http://www.http://finaldraft.com.} and Adobe Story \footnote{https://story.adobe.com.}, now enhanced with AI, are commonly used for organizing scenes and formatting screenplays. AI tools have also advanced in areas such as information retrieval, logical consistency checks, and emotional state mapping in scripts____. There have also been advances in emotional applications, such as AESOP, which can identify character emotional states and map them to characters in a story____. Furthermore, AI has been widely applied in collaborative screenwriting and continuation____. Given that the ultimate goal of screenwriting is to integrate visual and auditory elements, many studies have explored AI's role in visualizing screenplays. For example, Cardinal uses NLP techniques to visualize screenplay structure, characters, story nodes, and simple scenes based on character dialogue and actions____. Story Explorer visualizes character story arcs and background information, providing script management that allows users to quickly specify the chronological order of events in a film____. Other studies have proposed a tool for visualizing multi-character interactions using simple graphics and generating multiple potential scene possibilities from an action database____. Tapaswi et al. developed a visualization tool that automatically summarizes character interactions in completed films and TV shows____. CANVAS allows users to create multi-character animations by automatically completing storyboards____. Additionally, there are tools for creating storyboards tangibly____ and tools that allow the creation of fictional story worlds, filling them with characters and objects for automatic narrative synthesis____.

%While these AI-assisted tools address specific screenplay tasks, advancements in technology have significantly enhanced the capabilities of NLP, computer vision, text-to-image generation, and large language models. AI tools, such as ChatGPT\footnote{https://openai.com/chatgpt/}____, and DeepStory\footnote{https://www.deepstory.ai/}, are increasingly being integrated into the screenwriting process, providing predictive analytics and even co-writing capabilities____. Recently, the Dramatron system demonstrated that large language models could enhance screenwriting capabilities by generating coherent screenplays and drama scripts through hierarchical text generation. The system sequentially generates complete scripts, including titles, character lists, plot points, location descriptions, and dialogue, starting from a log line____. This research emphasizes a linear approach to script content generation but lacks flexibility in the workflow. While previous research has summarized common patterns and experiences of screenwriters____, each screenwriter has a unique process in practice. Screenwriting is a nonlinear process with open-ended goals, not strictly following a predefined sequence. Screenwriters explore multiple solutions, frequently shifting from one to another, actively reviewing, remixing, and iterating them until a satisfactory result is achieved____. Therefore, our study aims first to understand screenwriters' current flexible and variable nonlinear workflows and then explore the specific ways in which AI is used within these workflows.

%\subsection{The Use of AI in Other Creativity Processes}

%\textcolor{black}{Advancements in AI technology have also impacted various other creative domains____, reshaping fields such as the arts____ and design____. In art, AI contributes to interactive media and visual representation____, while Kim et al. explored AI’s influence on creative language arts, such as poetry, novels, and screenplays____. In design, AI integrates into product design workflows, fostering innovation____, and advancing user interface design with innovations like adaptability____. Meanwhile, with the impact of AI-generated image tools____, AI’s democratizing potential in graphic design has been explored____, with a call for interdisciplinary collaboration to improve AI systems____. As AI-generated video capabilities develop, AI is also transforming fashion design with dynamic visuals____ and aiding social media short-form video creators____. Additionally, other research highlights AI’s role across various creative domains, assisting knowledge workers____, data scientists____, and data storytelling____.}

%\textcolor{black}{Based on these previous works, effective human-AI co-creation is increasingly emphasized across creative fields. Screenwriting, as part of creative industry, should also embrace this trend, driven by AI advancements. However, the complexity of narrative structures and emotional depth in audiovisual storytelling makes existing research from other creative fields insufficient for understanding AI integration in screenwriting. This underscores the need for targeted research to bridge these gaps and advance AI’s practical application in screenwriting.}

%Across these fields, there is an increasing emphasis on effective human-AI co-creation. As a component of the creative industry, screenwriting should also embrace this trend, driven by advancements in AI technologies. However, due to the inherent complexity of narrative structures and the emotional depth involved in audiovisual storytelling, existing findings from other creative fields are insufficient to fully understand the integration of AI in screenwriting. This highlights the need for targeted research to address these gaps and advance the practical application of AI in screenwriting.}

%AI's development in creative fields extends beyond writing____, reshaping authorship and creativity in areas such as visual art____, interactive media arts____, design____, and visualization____. In design, AI has become integral to workflows____, particularly in product design, where it supports visual generation and fosters innovation, despite challenges related to diversity____. Diffusion models have further enhanced innovation in user interface design____. AI is also transforming fashion design through technologies like Attribute-GAN and DreamPose, enabling dynamic visuals____. In graphic design, Tang et al. have explored AI’s democratizing potential____, while Mustafa analyzed the impact of AI-assisted tools____. Meron stressed the importance of interdisciplinary collaboration between computer science and graphic design to enhance AI systems____. Additionally, previous research has examined AI's role in data scientists' workflows____ and how large language models assist knowledge workers____. Across these fields, there is a shared call for fostering effective human-AI co-creation. AI is viewed as a tool to enhance, not replace, human creativity. However, challenges remain, including concerns about creative control, content homogenization, and the potential loss of diversity in artistic expression. These concerns are commonly reflected across the creative industries, where practitioners seek to balance the innovative potential of AI with maintaining the uniqueness and authenticity of human contributions. 

%As a part of the creative field, screenwriters also face these challenges. However, due to the complexity of narrative structures, character development, emotional depth, and the integration of audiovisual language in the screenwriting process, findings from other creative fields are insufficient to address the issues surrounding AI integration in this context. This highlights the necessity of conducting targeted research to bridge the existing knowledge gap and advance the practical application of AI in screenwriting.

%Given our focus on the screenwriting, it faces the same challenges as general creativity studies. However, due to the involvement of complex narrative structures, character development, emotional depth, and the integration of audiovisual language in the screenwriting process, findings from other creative fields are insufficient to address the issues surrounding AI integration in screenwriting. This highlights the necessity of conducting targeted research in this area to bridge the existing knowledge gap and advance the practical application of AI in screenwriting.

%Given our focus on the screenwriting context, which presents unique challenges compared to general creativity studies, findings from other creative fields are insufficient to address the issues surrounding AI integration in screenwriting. Screenwriting, in particular, involves complex narrative structures, character development, and emotional depth, which involve different approches of AI integration. This highlights the need for targeted research in this area to bridge the existing knowledge gap and advance the practical application of AI in screenwriting. 


%However, given our focus on the screenwriting context, which presents unique challenges compared to general creativity studies, findings from other creative fields are insufficient to address the issues surrounding AI integration in screenwriting. This highlights the need for targeted research in this area to bridge the existing knowledge gap and advance the practical application of AI in screenwriting.


%However, since our focus is on the screenwriting context, which differs from general creativity studies, the results from other creative fields cannot address the specific issues related to the integration of AI in screenwriting that we are concerned with.

%AI's development in creative fields extends beyond writing____, reshaping authorship and creativity in areas such as visual art____, interactive media arts____, design____, and visualization____. In design, AI has become integral to workflows, particularly in product design, where it supports visual generation and fosters innovation, despite challenges related to diversity____. Diffusion models have further enhanced innovation in user interface design____. AI is also transforming fashion design through technologies like Attribute-GAN and DreamPose, enabling dynamic visuals____. In graphic design, Tang et al. have explored AI’s democratizing potential____, while Mustafa analyzed the impact of AI-assisted tools____. Meron stressed the importance of interdisciplinary collaboration between computer science and graphic design to enhance AI systems____. Additionally, previous research has examined AI's role in data scientists' workflows____ and how large language models assist knowledge workers____. Based on these explorations of creative processes involving AI, our results both diverge from and align with Zhou et al.'s research on creative design____. We have identified distinct future roles for AI, with a focus on understanding how screenwriters utilize AI and their expectations. Our findings contribute to the broader application of AI in creative domains, offering valuable new perspectives.

%AI's development in creative fields goes beyond writing____, redefining authorship and creativity in areas like visual art____, photography____, and interactive media arts____. In design, AI has been integral to workflows, particularly in product design, where it aids in visual generation and innovation, despite limitations in diversity____. Diffusion models in user interface design have enhanced innovation____. AI is also transforming fashion design through technologies like Attribute-GAN and DreamPose for dynamic visuals____. In graphic design, AI's democratizing potential has been studied through the work of Tang et al.____ and Mustafa's analysis of AI-assisted tools____. Additionally, Meron stressed the need for interdisciplinary collaboration between computer science and graphic design to improve AI systems____. Previous research has also focused on AI's role in data scientists' workflows____ and how large language models assist knowledge workers____. Based on these previous explorations in creativity processes with AI, understanding how screenwriters use AI is crucial for grasping AI's broader role in creativity and contributing to research on AI in creative industries.

%AI's development in creative fields goes beyond writing____, redefining authorship and creativity in areas like visual art____, photography____, and interactive media arts____. In design, AI has been integral to workflows, particularly in product design, where it aids in visual generation and innovation, despite limitations in diversity____. Diffusion models in user interface design have enhanced innovation____. AI is also transforming fashion design through technologies like Attribute-GAN and DreamPose for dynamic clothing visuals____. In graphic design, AI's democratizing potential has been studied through the work of Tang et al.____ and Mustafa's analysis of AI-assisted tools____. Meron stressed the need for interdisciplinary collaboration between computer science and graphic design to improve AI systems____. Previous research has also focused on AI's role in data scientists' workflows____ and how large language models assist knowledge workers____. Therefore, exploring how screenwriters use AI is crucial for understanding AI's broader role in creativity and contributing to AI research in creative industries.

%The development of AI in creative fields extends beyond creative writing____. AI is redefining authorship and creativity in art, with many artists incorporating AI into their creative processes____, including visual art____, photography____, and interactive media arts____. In the field of design, AI has long been integrated into workflows. In product design, AI facilitates visual generation, sparking innovative thinking despite some limitations in diversity____. Gmeiner et al. have explored the challenges and opportunities in helping engineers and architectural designers adopt AI-based tools for co-creation____. In user interface design, the adoption of diffusion models has made design elements more intuitive, enhancing innovation____. In fashion design, technologies such as Attribute-GAN and DreamPose have played a crucial role in automatically generating clothing-matching pairs and transforming static images into dynamic clothing showcase videos____. In graphic design, Tang et al.'s research examined the use of AI by designers in the field____ and highlighted the democratizing potential of AI tools through a comparative study of professional and non-professional users in the art and design fields____. Mustafa studied the impact of AI-assisted and AI-based design tools on various specific tasks in graphic design____. Meron emphasized that addressing issues in AI-assisted design systems requires interdisciplinary collaboration between computer science experts and graphic design specialists____. Additionally, in terms of integrating AI into different types of work and workflows, previous researchers have focused on data scientists' workflows, their use of AI, and their expectations____, as well as empirical studies on how knowledge workers in enterprises use large language models to assist their work____. Therefore, in an era where AI technology is ubiquitous, gaining an in-depth understanding of how screenwriters use AI in actual creation, as an underexplored but highly significant research area within the creative domain, will not only deepen our understanding of AI's role in creativity but also provide valuable contributions to research on AI in the creativity field.

%\subsection{Empirical Study in Screenwriting}

%\textcolor{black}{Previous studies have explored the connection between screenwriting and creativity____. Notable projects such as the short film script ``\textit{Sunspring}'' (2016)____ and the interactive playwriting project at the Young Vic (2021)____ have highlighted new trends in AI-powered screenwriting____. Çelik's research suggests that AI can generate screenplay elements comparable to, and sometimes superior to, those created by human writers____, though human artistic creativity remains irreplaceable____. Brako et al. explored AI as a co-creation tool in screenwriting education____, noting that it accelerates creativity by offering new inspiration. Other studies have examined AI's role in filmmaking, including screenwriting with LLMs like ChatGPT, showing that AI can enhance efficiency and reduce costs____. However, these studies lack an understanding of how AI specifically integrates into the various stages of the screenwriting workflow. They are also limited to LLM assistance and do not consider the potential impact of AI-generated images and videos, which may provide unprecedented support for audiovisual storytelling in screenwriting. Thus, the broader implications of emerging AI technologies in screenwriting remain underexplored.}

%\textcolor{black}{Moreover, Chow's research has shown that biases in AI datasets related to race, gender, or ideology can affect creativity in the film industry____, and there are also challenges related to screenplay copyright____. Understanding how screenwriters' attitudes and expectations toward AI integration in specific stages of the screenwriting workflow may provide potential approaches or insights for mitigating or altering the impact of these issues.}

%\textcolor{black}{Overall, our study aims to fill these gaps by analyzing screenwriters' current workflows, challenges, and task allocation with AI, as well as their attitudes toward AI integration and expectations for future AI roles. Our findings aim to inform the design of human-AI co-creation tools tailored to the needs of screenwriters.}


%Previous studies have established a connection between screenwriting and creativity____. In the context of evolving digital and AI technologies across various media, short film scripts such as ``\textit{Sunspring}'' (2016)____ and the interactive playwriting project at the Young Vic in London (2021)____ have garnered significant attention, showcasing new trends and modes in AI-augmented screenwriting in the digital era____. Çelik's research indicates that AI can generate screenplay elements that are comparable to, and sometimes even superior to, those produced by human writers____. Additionally, Brako et al. has explored the use of AI as a co-creation tool in screenwriting education____, highlighting that AI can accelerate the creative process and provide new sources of inspiration. However, the evaluation of works needs to be adapted to fit the AI co-creation environment. Other studies have explored the entire filmmaking process using various AI tools, including screenwriting with ChatGPT, showing that AI technology can significantly improve filmmaking efficiency and reduce costs. These studies suggest that the combination of AI and human creativity is key to the future sustainability of the film industry, where lower-level tasks may be automated by AI, but human artistic creativity and emotional input remain irreplaceable____. However, research by Chow has found that potential biases related to race, gender, or ideology in AI datasets could affect the diversity and novelty of creativity in the film industry____, and these biases also pose challenges to screenplay copyright____.

%Overall, previous research has rarely addressed the specific needs for future AI functionalities as informed by feedback from screenwriters after actual use, lacking concrete insights into AI's practical applications and potential expectations in this domain. Therefore, our research aims to analyze screenwriters' practices, including task allocation across different stages of the workflow with AI, combining their current attitudes toward AI integration in screenwriting with their idealized expectations for future AI functionalities. This leads us to propose future design opportunities for human-AI co-creation tools tailored to the needs of screenwriters.

%Previous studies have established a strong connection between screenwriting and creativity____. In the context of evolving digital and AI technologies and various media, short film scripts like ``\textit{Sunspring}'' (2016) \footnote{https://www.garethjmsaunders.co.uk/2019/10/15/sunspring-a-sci-fi-film-script-written-by-ai/} and the interactive playwriting project at the Young Vic in London (2021) \footnote{https://www.theguardian.com/stage/2021/aug/24/rise-of-the-robo-drama-young-vic-creates-new-play-using-artificial-intelligence} have garnered significant attention, demonstrating new trends and modes in the digital era's AI-augmented screenwriting____. Studies have shown that AI can generate screenplay elements that are comparable to, and sometimes even superior to, those produced by human writers____. Additionally, research has explored the use of AI as a co-creation tool in screenwriting education____, noting that AI can accelerate the creative process and provide new sources of inspiration, but that the evaluation of works needs to be adjusted to fit the AI co-creation environment. Other studies have used various AI tools to explore the entire filmmaking process, including screenwriting with ChatGPT, demonstrating that AI technology can significantly improve filmmaking efficiency and reduce costs. Research also suggests that the combination of AI and human creativity is key to the future sustainability of the film industry, where lower-level tasks may be automated by AI, but human artistic creativity and emotional input remain irreplaceable____. However, studies have found that due to potential biases related to race, gender, or ideology in data sets, the application of AI in the film industry may impact the diversity and novelty of creativity____, and it also poses challenges to the copyright of screenplays____.

%Overall, previous research has rarely addressed the specific needs for future AI functionalities as feedback from screenwriters after actual use. Therefore, our research aims to analyze screenwriters' task allocation across different stages of the workflow with AI, combining their current attitudes toward AI integration into the screenwriting process with their idealized expectations for future AI functionalities, to propose design guidelines for AI co-creation tools suited to screenwriters. Our goal is to enable future AI tools to effectively address the relevant challenges in screenwriting workflows as expected by screenwriters, thereby further promoting the development of human-AI co-creation.
\begin{table*}
%{H}
\centering
\footnotesize
%\tiny
%\scriptsize
%\scriptsize % 将字体大小缩小到更小号
\caption{Demographic Information of Participants. From left to right, each column presents the participant number, age, gender, background, years (Y) of screenwriting experience, training methods received, prior use of AI in screenwriting (yes/no), and their self-reported proficiency in using AI for screenwriting (5 = very proficient, 1 = not proficient at all). The specific training methods are represented by the following abbreviations in the table: institution courses (IC), classic scripts (CS), online videos (OV), instructor books (IB), and other (O).}
\Description{Description for Table 1:
The table displays demographic information of participants. It contains information organized by participant number (No.), age, gender, background (e.g., student, professional, enthusiast), years of screenwriting experience (Y), training methods received, prior use of AI in screenwriting (yes or no), and self-reported proficiency in using AI for screenwriting rated on a scale of 1 to 5 (5 being very proficient and 1 being not proficient at all).

Training methods are abbreviated as follows:
- IC: institution courses
- CS: classic scripts
- OV: online videos
- IB: instructor books
- O: other

The table lists 23 participants (P1–P18 and N1–N5), each row detailing:
1. Age range from 22 to 32.
2. Gender distribution includes male and female participants.
3. Backgrounds vary between students, professionals, and enthusiasts.
4. Years of screenwriting experience range from 0.5 years to 8 years.
5. Participants report training through a combination of methods like IC, CS, OV, IB, or O.
6. Prior use of AI in screenwriting shows some participants have used AI (marked "Yes") and others have not (marked "No").
7. Self-reported AI proficiency scores range from 1 (not proficient) to 5 (very proficient).} 

\label{tab:participant_data}
\begin{tabular}{|p{0.8cm}|p{0.8cm}|p{1.1cm}|p{1.7cm}|p{2.2cm}|p{2cm}|p{1.1cm}|p{2.5cm}|}


\hline
\textbf{No.} & \textbf{Age} & \textbf{Gender} & \textbf{Background}   & \textbf{Experience(Y)} & \textbf{Training}       & \textbf{Use AI} & \textbf{Proficiency of AI} \\ \hline
\textcolor{black}{P1}  & 24  & Female & Student      & 1             & IB, CS         & Yes    & 4                 \\ \hline
\textcolor{black}{P2}  & 26  & Male   & Professional & 7             & IC, CS         & Yes    & 4                 \\ \hline
\textcolor{black}{P3}  & 23  & Female & Enthusiast   & 0.5           & IC, CS, O      & Yes    & 3                 \\ \hline
\textcolor{black}{P4}  & 25  & Male   & Professional & 7             & IC, CS, OV     & Yes    & 4                 \\ \hline
\textcolor{black}{P5}  & 23  & Female & Student      & 5             & IC, IB, CS     & Yes    & 3                 \\ \hline
\textcolor{black}{P6}  & 24  & Female & Student      & 2             & IC, IB, CS     & Yes    & 3                 \\ \hline
\textcolor{black}{P7}  & 25  & Male   & Professional & 8             & IC, IB, CS     & Yes    & 4                 \\ \hline
\textcolor{black}{P8}  & 26  & Male   & Student      & 2             & IC, CS, OV, IB & Yes    & 5                 \\ \hline
\textcolor{black}{P9}  & 27  & Female & Enthusiast   & 1             & IC, CS, IB     & Yes    & 4                 \\ \hline
\textcolor{black}{P10} & 23  & Female & Student      & 5             & IC, CS, IB     & Yes    & 2                 \\ \hline
\textcolor{black}{P11} & 27  & Female & Professional & 6             & IC, IB, CS     & Yes    & 2                 \\ \hline
\textcolor{black}{P12} & 24  & Female & Enthusiast   & 0.5           & IC, CS, IB     & Yes    & 4                 \\ \hline
\textcolor{black}{P13} & 25  & Male   & Enthusiast   & 6             & IC, CS, IB     & Yes    & 4                 \\ \hline
\textcolor{black}{P14} & 27  & Male   & Professional & 2             & IC, CS, OV, IB & Yes    & 4                 \\ \hline
\textcolor{black}{P15} & 32  & Male   & Professional & 8             & IC, CS, OV, IB & Yes    & 5                 \\ \hline
\textcolor{black}{P16} & 23  & Female & Enthusiast   & 4             & IC, CS, IB     & Yes    & 4                 \\ \hline
\textcolor{black}{P17} & 23  & Female & Professional & 4             & IC, CS, OV     & Yes    & 4                 \\ \hline
\textcolor{black}{P18} & 22  & Female & Student      & 3             & IC, CS, IB     & Yes    & 3                 \\ \hline
\textcolor{black}{N1}  & 23  & Male   & Student      & 3             & IC, CS         & No     & 1                 \\ \hline
\textcolor{black}{N2}  & 22  & Female & Enthusiast   & 1             & CS             & No     & 1                 \\ \hline
\textcolor{black}{N3}  & 25  & Male   & Enthusiast   & 2             & IC, IB, CS     & No     & 1                 \\ \hline
\textcolor{black}{N4}  & 29  & Female & Professional & 6             & IC, CS, OV, IB & No     & 1                 \\ \hline
\textcolor{black}{N5}  & 23  & Female & Enthusiast   & 4             & IC, CS, IB     & No     & 1                 \\ \hline
\end{tabular}
    \label{tab:participants}
\end{table*}