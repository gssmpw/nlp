\section{Threats to Validity}


\textbf{Internal Validity:} An LLM may appear to address the security flaws within a given prompt; however, if the prompt was a part of the LLM's training data, the generated response could merely reflect that information, and not the model's generalizability. To counteract this threat, we created the Transformed-Dataset. This dataset includes transformed SO questions devoid of any security mentions, either in questions or responses. Therefore, we expect 
%that none of the three LLMs has encountered these specific questions or any notions of security regarding them, allowing us to assess the model's capability to identify and address security issues in new data.
none of the three LLMs to have encountered these specific questions or related security concepts, allowing us to assess their ability to address security issues in new data. 

\textbf{External Validity:} The results of our study may not generalize to all LLMs or programming questions. %\textcolor{blue}{For instance, the inclusion of other LLMs, such as code-specific models that place a heavier emphasis on code and provide more concise natural language output, is outside the scope of our study and can be explored further in future work.} However, to investigate the security awareness of cutting-edge LLMs, we included three of the most popular models: \gpt, \llama, and \claude. Further, we collected our datasets from a SO dump spanning 9 years (March 2015-March 2024). We selected questions related to two popular programming languages, Python and JavaScript, to ensure  our findings adaptable to a wide range of development scenarios and questions.
\textcolor{blue}{One key limitation of our study is the exclusion of code-specific LLMs, such as DeepSeek-Coder \cite{deepseekcoder} and Qwen-Coder \cite{hui2024qwen2}, which are designed specifically for code-related tasks. These models place a stronger emphasis on code comprehension and may yield different security-related responses compared to the conversational LLMs we evaluated. While our focus was on conversational LLMs widely used in developer interactions, we acknowledge that analyzing the security awareness of code-specific models is an important avenue for future research. In investigating the security awareness of conversational LLMs, however, we included three of the most popular models: \gpt, \llama, and \claude. Further, we collected our datasets from a SO dump spanning 9 years (March 2015–March 2024). We selected questions related to two popular programming languages, Python and JavaScript, to ensure our findings remain adaptable to a wide range of development scenarios and questions.}


\textbf{Construct Validity:} Much of the qualitative analysis done throughout this work has been the result of manual efforts by the authors. To ensure the reliability of our analyses, we %used Cohen's Kappa to measure inter-rater agreement among the authors. 
calculated the inter-rater agreement using Cohen Kappa coefficient, and iteratively discussed and resolved conflicts to reach sufficient agreement ($>$ 0.6). 
We chose to use an static analysis tool for automatically identifying the security vulnerabilities in code snippets which can sometimes produce false positives or false negatives. To mitigate this threat, in line with prior work, we relied on CodeQL, as it offers higher precision at the expense of a lower recall. As a result, by prioritizing precision, we ensure that the vulnerabilities included in our dataset are more accurate and reliable.

% although commonly used, codeql performs best in real-world projects and codebases and not necessarily code snippets from questions. The differences in the code of the development and the deployment environment affects the performance of CodeQL % Hard coded credentials example

% There can be instances of the LLM providing improved code (security-wise) to the vulnerable code they have been prompted with. In these instances the LLM may not provide any mention of the existing vulnerability in the prompt. (we could start running codeql on the code snippets of the LLMs to see if the vulnerability still exists or not)

% Could the vulnerabilities that were also pointed out by SO users, easier to detect?
