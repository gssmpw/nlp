\section{Related Work}

% \subsection{Large Language Models in Software Engineering}
% The impact of LLMs on SE practices
% Large language models have deeply influenced various aspects of software engineering, offering new possibilities for code generation, summarization, and debugging by answering queries related to software development practices. Tools like GitHub Copilot \cite{githubcopilot}, OpenAI's GPT series \cite{openai2023chatgpt}, Anthropic's Claude \cite{claude}, and Meta's LLaMA\cite{llama} generate code snippets based on natural language prompts, accelerating the development process and reducing the time spent on repetitive coding tasks. Developers regularly use LLMs for code generation, debugging, and general information retrieval purposes \cite{saud2024chatting}. 

% General study of LLMs in SE
% Given the significant influence of LLMs on software engineering practices, researchers have extensively studied the impact of LLMs in the field. A study by Das et al. analyzed the interactions of developers with ChatGPT, revealing its utility in tasks such as code generation, comprehension, debugging, ideation, and validation \cite{das2024investigating}. Their findings also indicate that developers frequently leverage ChatGPT during the ideation phase, receiving suggestions, new approaches, and potential solutions to their problems in form of \textit{natural language}. Likewise, developers use LLMs to summarize complex code segments, making it easier to understand and modify existing codebases.

Researchers have studied the adoption of LLMs in security-related software engineering tasks. Multiple studies have explored the possibilities and limitations of LLMs in vulnerability detection \citep{zhou2024large, noever2023can, bakhshandeh2023using, purba2023software, cheshkov2023evaluation, wang2023defecthunter, liu2023harnessing}. Some of these studies have demonstrated the superior performance of LLMs compared to traditional methods such as static analysis tools \citep{zhou2024large, noever2023can, bakhshandeh2023using}, while others have highlighted the shortcomings of LLMs, particularly their tendency to generate false positives \citep{purba2023software, cheshkov2023evaluation, Banerjee23}. Additionally, LLMs have been employed to detect malicious code \citep{plate, eli}, generate test cases \citep{yao2023llm, zhang2023well}, and to fix defective and/or vulnerable code \citep{jiang2023impact, pearce2023examining, xia2022practical, jin2023inferfix}. Beyond the use of LLMs in performing security-related tasks, researchers have also studied the security issues caused due to the use of LLMs. We discuss some of these studies in Sections \ref{concerns_with_llms_in_se} and \ref{concerns_in_llms_vs_SO}.

\subsection{Security Concerns with using LLMs} 
\label{concerns_with_llms_in_se}
% Code generation LLMs
% Following the widespread adoption of LLMs by developers, many researchers have shown interest in studying how this integration impacts the security of the produced software. 
Numerous studies studied the security of the code generated by LLMs. %In one of the earlier works in this area, 
\cite{pearce2022asleep} examined security of the code generated by GitHub's Copilot \citep{githubcopilot}. They found that around 40\% of the code generated by Copilot contained vulnerabilities when analyzed with CodeQL \citep{GitHubCodeQL}. Other studies have also evaluated the security of code from different code generation models, comparing it to human-written code \citep{siddiq2022securityeval, asare2023github}. %In their study, Copilot was prompted to generate code that, if not implemented properly, would introduce vulnerabilities to software. The generated code was then analyzed using CodeQL \cite{GitHubCodeQL}, revealing that approximately 40\% of the Copilot's code contained vulnerabilities. Other studies have examined the security of the code written by different code generation models and even compared it to human-written code \cite{siddiq2022securityeval, asare2023github}. %Similar studies have also taken into account conversational LLMs, evaluating the security of code generated by models such as ChatGPT \cite{khoury2023secure, siddiq2024quality}. Unlike these studies, our work highlights the security of the LLM-generated responses in form of natural language.
% In a different study, Siddiq et al published an evaluation dataset for assessing the security of the LLM generated code snippets \cite{siddiq2022securityeval}. In addition, they use this dataset to find the security vulnerabilities in code generated by Copilot and Incoder \cite{fried2022incoder}. In a more recent publication, researchers compared the security of the Copilot generated code with that of human developers, concluding that Copilot is less likely to generate vulnerable code \cite{asare2023github}.

% Conversational LLMs
In addition to the code generation models such as Insider and Copilot, LLMs have also been scrutinized for the security of their generated code. \cite{khoury2023secure} tasked ChatGPT with generating 21 programs likely to contain vulnerabilities. The results revealed that ChatGPT only provided secure code in five out of 21 programs. %and was able to fix its generated code in seven other cases \cite{khoury2023secure}.
Similarly, \cite{siddiq2024quality} used CodeQL, Bandit, and Pylint to evaluate the security of code generated by ChatGPT, finding a number of security issues related to improper resources and exception management.

LLMs have been proven useful in detecting security vulnerabilities~\cite{lu2024grace, zhou2024large, thapa2022transformer, yao2024survey}. Based on these studies, and supported by the preliminary validation of our motivational study we initially hypothesized that LLMs possess the security knowledge needed to warn users of potential threats. However, our findings show that LLMs do not always recognize when or how to apply this knowledge effectively. Different from previous studies, our work focuses on determining LLMs' abilities in proactive prevention i.e., proactively identifying and flagging vulnerabilities. To the best of our knowledge, we are the first to assess LLMs'  capability in communicating critical information (causes, exploits, and fixes) to enhance developer understanding and  encourage the adoption of secure coding practices. %Our evaluation examines the ability of LLMs for \textbf{proactive prevention} i.e., proactively identifying and flagging vulnerabilities. While we might not yet expect this from LLMs, this will become critical as AI-driven tools become more integrated into development workflows. 
%\textcolor{blue}{add a short sentence how our work is different from rest and what are we contributing}%Unlike these studies, our work does not focus on the code, but rather natural language output of LLMs. By analyzing how LLMs use natural language to communicate, or fail to communicate, vital security information, we aim to understand their role in shaping developers' security awareness.% and practices.

% Prompt engineering & Output optimization
% The concerns regarding the LLM approaches in many software engineering tasks have inspired researchers to seek methods for improving the LLM outputs. Prompt engineering \cite{homoliak2024enhancing, pearce2022asleep, yetistiren2022assessing} and output optimizing \cite{vernikos2023small, wang2023factcheck} are among the techniques, commonly used to improve LLM generated content. One notable study demonstrated the use of scenario-specific, iterative, and general clause methods to rephrase the prompts of GitHub Copilot in order to enhance the security of the generated code \cite{homoliak2024enhancing}. The use of certain prompt engineering techniques in our work is meant to improve the performance of the LLMs with respect to our own particular metrics, as further described in Section \ref{methodology_case_study}.

% Offensive LLM applications
% It's worth noting that the concerns about the use of LLMs in the software space extend beyond the limitations of LLMs in performing certain software engineering tasks. As pointed out by Yao et al., LLMs can and have been actively used for offensive applications\includecomment{, against security and privacy}\cite{yao2024survey}. These offensive applications can range from hardware to user-level attacks. In an example of an OS-level attack, Happe and Cito demonstrated how providing ChatGPT with an SSH connection to a Linux virtual machine in a closed feedback loop resulted in a successful privilege escalation attack \cite{happe2023getting}. The offensive applications of LLMs are, however, beyond the scope of this study.


\subsection{Stack Overflow vs. LLMs: Security Implications} 
\label{concerns_in_llms_vs_SO}
% Intro
Popularity of Stack Overflow has been significantly impacted by the emergence of LLMs\citep{da2024chatgpt}. Despite the varying quality of LLM-generated answers, many developers prefer ChatGPT due to their well-articulated language \citep{kabir2024stack}. While a rich body of literature on SO and the security concerns associated with its usage remains \citep{yang2016security, licorish2021contextual, fischer2017stack, zhang2018code}, researchers are still exploring the consequences of relying on LLM responses instead of user-provided SO responses. 
% This transition has inspired a number of studies to compare user-generated responses on SO to LLM-generated responses. 
% often comparing LLM-generated code to traditional user-generated content on SO. 

% Security-focused comparison of SO & LLMs
\cite{da2024chatgpt} compared the reliability of LLM-generated responses to user-provided answers on SO. Their results indicate that despite a high degree of textual similarity, the LLM responses are not reliable. Furthermore, they report a considerable decline in user activity on SO since the introduction of ChatGPT. %,  suggesting that some developers are relying on LLM responses to answer their questions instead of SO.
% They They also compared the reliab responses to the answers generated by LLaMA-2 and ChatGPT-3.5 \cite{da2024chatgpt}. Their results indicate a high degree of textual similarity between ChatGPT and SO responses. %with the unpopular features of frameworks and libraries to be the most challenging questions for the LLMs. 
In another study, ChatGPT was prompted with 517 SO questions, revealing that 52\% of the responses contained incorrect and fabricated information\citep{kabir2024stack}. %Regardless, 35\% of the involved study participants still preferred the LLM responses to the SO responses, overlooking the incorrect information \cite{kabir2024stack}. 
% Others have shown 74\% of the user-provided responses and 71\% of the ChatGPT-generated responses to the same privacy-related SO questions to be accurate \cite{delile2023evaluating}. and discovered multiple vulnerabilities in ChatGPT's code, responding to SO security questions \cite{hamer2024just}.
\cite{delile2023evaluating} have shown that 74\% of the user-provided responses and 71\% of the ChatGPT-generated responses to privacy-related SO questions are accurate. In another study, \cite{hamer2024just} discovered multiple vulnerabilities in ChatGPT's code when responding to SO security questions. While they focused on whether LLMs generate vulnerable code, we aim to assess if LLMs, when prompted with vulnerable code, inform users about the security implications.
%In contrast to these studies, our work does not attempt to assess the correctness of the LLM-generated responses to SO questions. Instead, we aim to understand whether LLMs, prompted with vulnerable code, would inform users about the security implications of their code.

% In another study, researchers, prompted ChatGPT with 92 privacy related SO questions and manually analyzed both the responses Finding 74\% of the SO responses and 71\% of the ChatGPT responses to be accurate \cite{delile2023evaluating}. Further, Hmaer et al. prompted ChatGPT with security-related SO questions and used CodeQL to detect existing vulnerabilities, finding a number of vulnerabilities both in SO and ChatGPT answers \cite{hamer2024just}.  In contrast to these studies, our work does not attempt to assess the correctness of the LLM-generated responses to SO questions. Instead, we aim to understand whether LLMs, prompted with a question that contains vulnerable code, would inform users about the security implications of their code or if they only answer the question.

% Overall Distinguishing Factors of our Work
Overall, our work is distinct in two ways. First, instead of assessing the security of LLM-generated code, we evaluate the natural language responses. Second, unlike studies that prompt LLMs with specific security-related tasks such as vulnerability detection, we deliberately avoid any mention of security in our prompts. Instead, we present LLMs with only SO questions that contain insecure code. This emulates real-life programming Q\&A and allows us to evaluate the security awareness of LLMs in practical applications. %emulating real-life programming Q\&A. This allows us to evaluate the security-awareness of LLMs and their ability to inform users about the vulnerable code. %in various plausible real-world scenarios.
% We aim to understand whether LLMs proactively contribute to security awareness in software development or if they potentially encourage insecure practices by turning a blind eye to the users' insecure code. 
% Lastly, our use of prompt engineering methods is meant to improve the security awareness of the LLM answers with respect to our particular specific metrics, described in Section \ref{methodology_case_study}, as opposed to common tasks such as vulnerability detection or secure code generation.
