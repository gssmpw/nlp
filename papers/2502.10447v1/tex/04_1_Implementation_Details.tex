\subsection{Implementation Details}
\label{sec:implementation}

\begin{table*}[t]
\resizebox{\textwidth}{!}{
\begin{tabular}{cccccccccccc}
\hline
\multirow{2}{*}{$\alpha$} & \multirow{2}{*}{Method} & \multicolumn{5}{c}{$\beta = 0.2$}                                                                                                         & \multicolumn{5}{c}{$\beta = 0.4$}                                                                                     \\ \cline{3-12} 
                          &                         & Precision                 & Recall                    & F1                        & Accuracy                  & AUC                       & Precision             & Recall                & F1                    & Accuracy              & AUC                   \\ \hline
\multirow{7}{*}{0.2}      & FedAvg                  & 53.94 $\pm$ 3.56          & 54.05 $\pm$ 3.53          & 53.21 $\pm$ 3.90          & 53.41 $\pm$ 3.71          & 69.39 $\pm$ 2.05          & 53.58 $\pm$ 2.34      & 53.80 $\pm$ 2.77      & 52.30 $\pm$ 2.32      & 52.64 $\pm$ 2.14      & 68.20 $\pm$ 2.72      \\
                          & FedProx                 & 55.81 $\pm$ 1.58          & 54.37 $\pm$ 5.19          & 54.32 $\pm$ 2.21          & 54.62 $\pm$ 2.18          & 68.29 $\pm$ 2.93          & 54.39 $\pm$ 2.69      & 54.36 $\pm$ 1.82      & 53.27 $\pm$ 2.99      & 53.52 $\pm$ 2.97      & 68.11 $\pm$ 2.74      \\
                          & FedMed-GAN              & 54.62 $\pm$ 1.02          & 55.18 $\pm$ 2.73          & 53.46 $\pm$ 0.97          & 53.52 $\pm$ 1.07          & 68.80 $\pm$ 2.16          & 53.87 $\pm$ 3.66      & 54.66 $\pm$ 3.57      & 53.35 $\pm$ 3.61      & 53.41 $\pm$ 3.63      & 68.09 $\pm$ 2.53      \\
                          & FedMI                   & 54.77 $\pm$ 2.94          & 55.22 $\pm$ 4.36          & 54.26 $\pm$ 2.82          & 54.40 $\pm$ 2.83          & 69.15 $\pm$ 2.73          & 54.03 $\pm$ 3.88      & 54.46 $\pm$ 3.56      & 53.54 $\pm$ 4.22      & 53.63 $\pm$ 4.14      & 67.61 $\pm$ 3.93      \\
                          & MFCPL                   & 54.54 $\pm$ 1.88          & \textbf{56.06 $\pm$ 4.14} & 53.88 $\pm$ 1.64          & 54.07 $\pm$ 1.93          & 69.39 $\pm$ 2.71          & 53.86 $\pm$ 5.42      & 54.01 $\pm$ 5.44      & 53.22 $\pm$ 5.16      & 53.30 $\pm$ 5.34      & 67.58 $\pm$ 3.67      \\
                          & PmcmFL                  & 53.17 $\pm$ 1.60          & 52.73 $\pm$ 3.50          & 52.58 $\pm$ 1.81          & 52.64 $\pm$ 2.03          & 67.51 $\pm$ 4.07          & 49.03 $\pm$ 1.55      & 48.01 $\pm$ 3.03      & 48.38 $\pm$ 2.00      & 48.46 $\pm$ 2.31      & 63.71 $\pm$ 2.50      \\
                          & ClusMFL                    & \textbf{57.16 $\pm$ 2.32}     & 54.73 $\pm$ 3.93              & \textbf{56.56 $\pm$ 2.36}     & \textbf{56.92 $\pm$ 2.41}     & \textbf{72.81 $\pm$ 3.64}     & \textbf{56.06 $\pm$ 1.31} & \textbf{55.44 $\pm$ 2.19} & \textbf{55.38 $\pm$ 1.07} & \textbf{55.49 $\pm$ 1.10} & \textbf{72.50 $\pm$ 2.02} \\ \hline
\multirow{7}{*}{0.4}      & FedAvg                  & 53.75 $\pm$ 3.86          & 54.26 $\pm$ 3.76          & 52.76 $\pm$ 4.64          & 53.08 $\pm$ 4.34          & 67.91 $\pm$ 3.48          & 52.60 $\pm$ 3.00      & 53.95 $\pm$ 3.66      & 51.84 $\pm$ 3.06      & 51.98 $\pm$ 3.07      & 67.03 $\pm$ 3.22      \\
                          & FedProx                 & 54.36 $\pm$ 3.50          & 54.28 $\pm$ 4.17          & 53.77 $\pm$ 3.48          & 53.96 $\pm$ 3.51          & 68.26 $\pm$ 3.53          & 52.16 $\pm$ 2.20      & 52.93 $\pm$ 3.76      & 51.35 $\pm$ 2.13      & 51.54 $\pm$ 2.07      & 67.43 $\pm$ 3.64      \\
                          & FedMed-GAN              & 52.97 $\pm$ 1.63          & 52.89 $\pm$ 2.41          & 52.43 $\pm$ 1.74          & 52.53 $\pm$ 1.63          & 67.25 $\pm$ 3.45          & 53.56 $\pm$ 3.57      & 54.02 $\pm$ 3.95      & 52.54 $\pm$ 3.97      & 52.86 $\pm$ 3.63      & 67.91 $\pm$ 2.93      \\
                          & FedMI                   & 55.40 $\pm$ 2.85          & 53.82 $\pm$ 2.45          & 54.84 $\pm$ 2.75          & 54.95 $\pm$ 2.69          & 68.59 $\pm$ 2.48          & 54.01 $\pm$ 2.07      & 52.36 $\pm$ 1.98      & 53.62 $\pm$ 2.14      & 53.85 $\pm$ 2.06      & 66.70 $\pm$ 2.63      \\
                          & MFCPL                   & 55.07 $\pm$ 4.34          & 54.66 $\pm$ 3.93          & 54.34 $\pm$ 4.37          & 54.51 $\pm$ 4.57          & 66.60 $\pm$ 2.58          & 52.84 $\pm$ 4.14      & 54.69 $\pm$ 3.92      & 51.70 $\pm$ 3.89      & 51.87 $\pm$ 3.74      & 67.81 $\pm$ 3.48      \\
                          & PmcmFL                  & 51.71 $\pm$ 5.54          & 49.80 $\pm$ 6.21          & 50.77 $\pm$ 5.21          & 50.88 $\pm$ 5.42          & 65.54 $\pm$ 6.22          & 52.45 $\pm$ 3.96      & 50.03 $\pm$ 5.74      & 50.40 $\pm$ 4.72      & 50.77 $\pm$ 4.66      & 63.31 $\pm$ 5.17      \\
                          & ClusMFL                    & \textbf{56.59 $\pm$ 4.53} & \textbf{54.68 $\pm$ 4.49} & \textbf{56.17 $\pm$ 4.08} & \textbf{56.37 $\pm$ 4.10} & \textbf{73.25 $\pm$ 4.11} & \textbf{54.83 $\pm$ 6.13} & \textbf{54.88 $\pm$ 6.09} & \textbf{54.22 $\pm$ 5.89} & \textbf{54.40 $\pm$ 6.03} & \textbf{72.22 $\pm$ 4.47} \\ \hline
\end{tabular}
}
\caption{Performance Comparison Across Different Federated Learning Methods (Mean $\pm$ Standard Deviation \%) under Different Settings.}
\label{main result}
\end{table*}


\paragraph{Datasets.}
For the robust AVSR benchmark, we utilize the LRS3 dataset~\citep{afouras2018lrs3}, which consists of 433 hours of audio-visual speech from 5,000+ speakers. Following the experimental setup of \citet{shi2022robust}, we extract audio noise samples from the MUSAN~\citep{snyder2015musan} dataset, targeting different noise types such as \textit{babble}, \textit{music}, and \textit{natural} noises, along with \textit{speech} noise from LRS3. These noises are randomly augmented into the audio data, corrupting 25\% of the training set with a signal-to-noise ratio (SNR) sampled from $\mathcal{N}(0, 5)$. We measure performance using the word error rate (WER), primarily under noisy conditions with SNRs of \{$-$10, $-$5, 0, 5, 10\}\,dB, specifically N-WER\,\citep{kim2024learning} which highlights the significance of visual cues in noise-corrupted environments.


For multilingual evaluations, the MuAViC dataset \cite{anwar2023muavic} is used, featuring 1,200 hours of audio-visual content from 8,000+ speakers across 9 languages, sourced from LRS3-TED\,\cite{afouras2018lrs3} and mTEDx\,\cite{elizabeth2021multilingual}. We use 8 languages (excluding English) for multilingual AVSR and 6 languages for X-to-English audio-visual speech-to-text translation (AVS2TT) tasks. We assess the models using WER for transcription and the BLEU score\,\cite{papineni2002bleu} for translation.

\vspace*{-8pt}
\paragraph{\ourmodel model description.}

Our \ourmodel framework is developed in two configurations: \textsc{Base} and \textsc{Large}. The \textsc{Base} model consists of 12 Transformer~\citep{vaswani2017attention} encoder layers and 6 decoder layers, while the \textsc{Large} model incorporates 24 encoder layers and 9 decoder layers. Both models’ audio-visual encoders are derived from the AV-HuBERT-\textsc{Base}/-\textsc{Large} models, pretrained on a noise-augmented corpus of LRS3 \citep{afouras2018lrs3} + VoxCeleb2 \citep{chung2018voxceleb2}. Our MoE implementation activates top-2 out of 8 experts in every FFN layer within the decoder~\citep{jiang2024mixtral}, while the hierarchical architecture engages the top-1 expert from each audio and visual group. To facilitate the expert group specialization, load biasing is used with audio or video randomly dropped in 25\% probability.

\begin{table}[t]
    \centering
    \caption{Comparison with other methods (FT). All the compared results except PAD are borrowed from TINN. Training epochs of SODA and MFNet-T are 200 and 300.}
    \label{tab:sota_comparison}
    \centering
    \scriptsize
    \setlength{\tabcolsep}{0.8mm}{
    \scalebox{1.0}{
    \begin{tabular}{l c c c c c c}
        \toprule
        Methods & Params(M) & SODA & MFNet-T \\
        \midrule
        DeepLab V3+ \citep{deeplabv3+} & 62.7  & 68.73 & 49.80  \\
        PSPNet \citep{pspnet} & 68.1 & 68.68 & 45.24 \\
        UPerNet \citep{upernet} & 72.3 & 67.45 & 48.56 \\
        SegFormer \citep{segformer} & 84.7 & 67.86 & 50.68 \\
        ViT-Adapter \citep{vitadapter} & 99.8 & 68.12 & 50.62 \\
        Mask2Former \citep{mask2former} & 216.0 & 67.58 & 51.30 \\
        MaskDINO \citep{maskdino} & 223.0 & 66.32 & 51.03 \\
        \midrule
        EC-CNN \citep{soda} & 54.5 & 65.87 & 47.56 \\
        MCNet \citep{mcnet} & 35.7 & 63.89 & 43.15 \\
        PAD (MAE-B) \citep{pad} & 164.9 & 69.71 & 50.14 \\
        TINN \citep{tinn} & 85.3 & 69.45 & 51.93  \\
        \midrule
        \rowcolor{cyan!15} UNIP-T (MAE-L) & 11.0 & 67.29 & 50.39 \\
        \rowcolor{cyan!15} UNIP-S (MAE-L) & 41.9 & \underline{71.35} & \underline{53.76} \\
        \rowcolor{cyan!15} UNIP-B (MAE-L) & 163.7 & \textbf{72.19} & \textbf{54.35} \\
        \bottomrule
    \end{tabular}}}
    \vspace{-4mm}
\end{table}

As summarized in Table\,\ref{tab:main_result}, the \textsc{Base} model of \ourmodel holds 359M parameters, and the \textsc{Large} configuration contains 1B. Despite its larger model capacity, due to the sparse activation of these parameters, only about half are active during token processing, amounting to 189M for \textsc{Base} and 553M for \textsc{Large} model. This setup ensures computational efficiency which is comparable to the smaller AV-HuBERT counterparts. For more details on the model description and computation cost, refer to Appendix\,\ref{appx:model_details} and \ref{appx:computation_cost}.
