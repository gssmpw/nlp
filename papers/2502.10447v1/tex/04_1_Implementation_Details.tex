\subsection{Implementation Details}
\label{sec:implementation}

\begin{table*}[t]
\centering
\footnotesize

\begin{subtable}{\textwidth}
\centering
\begin{adjustbox}{width=\textwidth, center}
\begin{tabular}{llccccccccccc}
\toprule
\multirow{2}{*}{Model} & \multirow{2}{*}{Modality} & \multicolumn{11}{c}{mAP (\%)} \\
\cmidrule(lr){3-13}
 & & 0.20s & 0.40s & 0.60s & 0.80s & 1.00s & 1.20s & 1.40s & 1.60s & 1.80s & 2.00s & Avg \\
\midrule
\multirow{4}{*}{\centering Transformer} 
 & A & 72.2 & 65.2 & 60.3 & 56.8 & 54.4 & 53.1 & 52.4 & 52.0 & 51.6 & 51.4  & 56.9 $\pm$ 0.05 
 \\
 & V & 52.0 & 51.7 & 51.6 & 51.3 & 51.1 & 50.9 & 50.8 & 50.5 & 50.3 & 50.1  & 51.0 $\pm$ 0.08 
 \\
 & A+V & 73.8 & 66.9 & 62.1 & 58.5 & 56.3 & 55.0 & 54.1 & 53.7 & 53.3 & 53.0 & 58.7 $\pm$ 0.13
 \\
 & A+V\textsuperscript{P} & 73.4 & 66.8 & 61.8 & 58.3 & 56.1 & 54.8 & 54.1 & 53.5 & 53.2 & 52.7  & 58.5 $\pm$ 0.26 
 \\
\midrule
\multirow{4}{*}{\centering GRU} 
 & A & 71.5 & 65.0 & 60.1 & 57.0 & 55.0 & 53.8 & 52.9 & 52.2 & 51.5 & 50.9  & 57.0 $\pm$ 0.30 
 \\
 & V & 53.0 & 52.7 & 52.4 & 52.0 & 51.7 & 51.6 & 51.2 & 51.1 & 50.8 & 50.6 & 51.7 $\pm$ 0.29 
 \\
 & A+V & 73.5 & 68.1 & 63.7 & 60.7 & 59.1 & 58.1 & 57.2 & 56.3 & 55.4 & 54.4 & 60.6 $\pm$ 0.17 
 \\
 & A+V\textsuperscript{P} & 70.8 & 64.9 & 60.1 & 56.9 & 55.0 & 53.8 & 53.0 & 52.4 & 51.8 & 51.4 & 57.0 $\pm$ 0.29 
 \\
\midrule
\multirow{4}{*}{\centering Mamba} 
 & A & 67.5 & 62.2 & 58.4 & 55.7 & 54.0 & 52.9 & 52.0 & 51.1 & 50.2 & 49.6 & 55.4 $\pm$ 0.62 
 \\
 & V & 52.2 & 51.8 & 51.5 & 51.1 & 50.9 & 50.7 & 50.5 & 50.4 & 50.0 & 49.7 & 50.9 $\pm$ 0.21 
 \\
 & A+V & 71.8 & 65.4 & 60.5 & 57.1 & 55.0 & 53.9 & 53.5 & 53.1 & 52.3 & 51.8 & 57.4 $\pm$ 0.26 
 \\
 & A+V\textsuperscript{P} & 68.9 & 63.2 & 59.1 & 56.0 & 54.0 & 52.7 & 51.8 & 51.4 & 50.7 & 50.1 & 55.8 $\pm$ 0.43 
 \\
\bottomrule
\end{tabular}
\end{adjustbox}
\label{tab:main_result_a}
\caption*{(a) Results on EasyCom} %
\end{subtable}


\begin{subtable}{\textwidth}
\centering
\begin{adjustbox}{width=\textwidth, center}
\begin{tabular}{llccccccccccc}
\toprule
\multirow{2}{*}{Model} & \multirow{2}{*}{Modality} & \multicolumn{11}{c}{mAP (\%)} \\
\cmidrule(lr){3-13}
 & & 0.20s & 0.40s & 0.60s & 0.80s & 1.00s & 1.20s & 1.40s & 1.60s & 1.80s & 2.00s & Avg \\
\midrule
\multirow{4}{*}{\centering Transformer} 
 & A & 78.8 & 74.9 & 71.8 & 69.7 & 68.1 & 67.0 & 66.3 & 65.7 & 65.1 & 64.7 & 69.2 $\pm$ 0.03
 \\
 & V & 58.7 & 58.5 & 58.4 & 58.2 & 58.1 & 58.0 & 57.9 & 57.8 & 57.7 & 57.7 & 58.0 $\pm$ 0.27
 \\
 & A+V & 78.1 & 74.3 & 71.5 & 69.4 & 68.0 & 67.0 & 66.3 & 65.7 & 65.3 & 64.9 & 69.0 $\pm$ 0.24
 \\
 & A+V\textsuperscript{P} & 78.4 & 74.5 & 71.5 & 69.4 & 67.9 & 66.7 & 65.9 & 65.4 & 65.0 & 64.5 & 68.9 $\pm$ 0.18
\\
\midrule
\multirow{4}{*}{\centering GRU} 
 & A & 78.6 & 74.8 & 71.8 & 69.6 & 68.1 & 66.9 & 66.2 & 65.6 & 65.2 & 64.8  & 69.2 $\pm$ 0.25
 \\
 & V & 58.6 & 58.3 & 58.1 & 57.9 & 57.8 & 57.8 & 57.7 & 57.6 & 57.5 & 57.5  & 57.9 $\pm$ 0.61
 \\
 & A+V & 76.4 & 73.0 & 70.4 & 68.5 & 67.1 & 66.3 & 65.6 & 65.2 & 64.7 & 64.4  & 68.2 $\pm$ 0.42
 \\
 & A+V\textsuperscript{P} & 76.9 & 73.4 & 70.6 & 68.6 & 67.3 & 66.3 & 65.6 & 65.1 & 64.7 & 64.4  & 68.3 $\pm$ 0.18
 \\
\midrule
\multirow{4}{*}{\centering Mamba} 
 & A & 77.4 & 73.6 & 70.5 & 68.5 & 66.9 & 65.8 & 65.0 & 64.3 & 63.9 & 63.5  & 67.9 $\pm$ 0.37
 \\
 & V & 58.2 & 58.1 & 57.9 & 57.8 & 57.6 & 57.5 & 57.5 & 57.4 & 57.4 & 57.3 & 57.7 $\pm$ 0.28
 \\
 & A+V & 76.0 & 72.5 & 69.8 & 67.9 & 66.6 & 65.6 & 64.8 & 64.2 & 63.8 & 63.5  & 67.5 $\pm$ 0.18
 \\
 & A+V\textsuperscript{P} & 74.1 & 70.8 & 68.1 & 66.2 & 64.8 & 63.9 & 63.2 & 62.7 & 62.3 & 62.0 & 65.8 $\pm$ 0.23
 \\
\bottomrule
\end{tabular}
\end{adjustbox}
\label{tab:main_result_b}
\caption*{(b) Results on Ego4D} 
\end{subtable}

\caption{Mean average precision (mAP) scores on (a)~EasyCom and (b)~Ego4D at time steps 0.20\,s to 2.00\,s for Transformer, GRU, and Mamba architectures under Audio~(A), Visual~(V), and Audio+Visual~(A+V) modalities. Models with \textsuperscript{P} are pretrained on YT-Conversation. Per-timestep values come from a single random seed, while ``Avg'' shows mean $\pm$ SE over five seeds (see \Cref{app:stat_results} for full multi-seed results). Using both A and V yields the best performance overall.}




\label{tab:main_result}
\end{table*}














\paragraph{Datasets.}
For the robust AVSR benchmark, we utilize the LRS3 dataset~\citep{afouras2018lrs3}, which consists of 433 hours of audio-visual speech from 5,000+ speakers. Following the experimental setup of \citet{shi2022robust}, we extract audio noise samples from the MUSAN~\citep{snyder2015musan} dataset, targeting different noise types such as \textit{babble}, \textit{music}, and \textit{natural} noises, along with \textit{speech} noise from LRS3. These noises are randomly augmented into the audio data, corrupting 25\% of the training set with a signal-to-noise ratio (SNR) sampled from $\mathcal{N}(0, 5)$. We measure performance using the word error rate (WER), primarily under noisy conditions with SNRs of \{$-$10, $-$5, 0, 5, 10\}\,dB, specifically N-WER\,\citep{kim2024learning} which highlights the significance of visual cues in noise-corrupted environments.


For multilingual evaluations, the MuAViC dataset \cite{anwar2023muavic} is used, featuring 1,200 hours of audio-visual content from 8,000+ speakers across 9 languages, sourced from LRS3-TED\,\cite{afouras2018lrs3} and mTEDx\,\cite{elizabeth2021multilingual}. We use 8 languages (excluding English) for multilingual AVSR and 6 languages for X-to-English audio-visual speech-to-text translation (AVS2TT) tasks. We assess the models using WER for transcription and the BLEU score\,\cite{papineni2002bleu} for translation.

\vspace*{-8pt}
\paragraph{\ourmodel model description.}

Our \ourmodel framework is developed in two configurations: \textsc{Base} and \textsc{Large}. The \textsc{Base} model consists of 12 Transformer~\citep{vaswani2017attention} encoder layers and 6 decoder layers, while the \textsc{Large} model incorporates 24 encoder layers and 9 decoder layers. Both models’ audio-visual encoders are derived from the AV-HuBERT-\textsc{Base}/-\textsc{Large} models, pretrained on a noise-augmented corpus of LRS3 \citep{afouras2018lrs3} + VoxCeleb2 \citep{chung2018voxceleb2}. Our MoE implementation activates top-2 out of 8 experts in every FFN layer within the decoder~\citep{jiang2024mixtral}, while the hierarchical architecture engages the top-1 expert from each audio and visual group. To facilitate the expert group specialization, load biasing is used with audio or video randomly dropped in 25\% probability.

\begin{table}[!t]
\centering
\caption{Comparison of \methodname~with related backdoor attacks.}
\label{table:sota_comparison}
\begin{adjustbox}{max width=\linewidth}
\begin{threeparttable}
\begin{tabular}{c|c|c|c|c|}
\cline{2-5}
 & \textbf{\begin{tabular}[c]{@{}c@{}}Provides\\ Concealed\\ Backdoor\\ Feature?\end{tabular}} & \textbf{\begin{tabular}[c]{@{}c@{}}Without\\ Modifying\\ Training \\ Process?\end{tabular}} & \textbf{\begin{tabular}[c]{@{}c@{}}Requires Victim\\Model Access\\for Data\\ Poisoining?\end{tabular}} & \textbf{\begin{tabular}[c]{@{}c@{}}Camouflaging\\ Without\\ Auxiliary\\ Data?\end{tabular}} \\ \hline
\multicolumn{1}{|c|}{TrojanNN~\cite{trojannn}} & \redcross & \greentick & $\square$ & \multirow{13}{*}{\begin{tabular}[c]{@{}c@{}}Not\\ Applicable\end{tabular}} \\ \cline{1-4}
\multicolumn{1}{|c|}{SIG~\cite{sig}} & \redcross & \greentick & \textcolor{OliveGreen}{No Access} &  \\ \cline{1-4}
\multicolumn{1}{|c|}{BadNets~\cite{badnets}} & \redcross & \greentick & \textcolor{OliveGreen}{No Access} &  \\ \cline{1-4}
\multicolumn{1}{|c|}{ReFool~\cite{refool}} & \redcross & \greentick & \textcolor{OliveGreen}{No Access} &  \\ \cline{1-4}
\multicolumn{1}{|c|}{Input-Aware~\cite{inputaware}} & \redcross & \redcross & $\square$ &  \\ \cline{1-4}
\multicolumn{1}{|c|}{Blind~\cite{blind}} & \redcross & \redcross$^{\star}$ & \textcolor{OliveGreen}{No Access} &  \\ \cline{1-4}
\multicolumn{1}{|c|}{LIRA~\cite{lira}} & \redcross & \redcross & $\square$ &  \\ \cline{1-4}
\multicolumn{1}{|c|}{SSBA~\cite{ssba}} & \redcross & \greentick & \textcolor{OliveGreen}{No Access} &  \\ \cline{1-4}
\multicolumn{1}{|c|}{WaNet~\cite{wanet}} & \redcross & \greentick & \textcolor{OliveGreen}{No Access} &  \\ \cline{1-4}
\multicolumn{1}{|c|}{LF~\cite{lf}} & \redcross & \greentick & $\square$ &  \\ \cline{1-4}
\multicolumn{1}{|c|}{FTrojan~\cite{ftrojan}} & \redcross & \greentick & \textcolor{OliveGreen}{No Access} &  \\ \cline{1-4}
\multicolumn{1}{|c|}{BppAttack~\cite{bppattack}} & \redcross & \greentick & \textcolor{OliveGreen}{No Access} &  \\ \cline{1-4}
\multicolumn{1}{|c|}{PoisonInk~\cite{poisonink}} & \redcross & \greentick & \textcolor{OliveGreen}{No Access} &  \\ \hline \hline
\multicolumn{1}{|c|}{Di et al.~\cite{DBLP:conf/nips/DiDA0S23}} & \greentick & \greentick & $\square$ & \greentick \\ \hline
\multicolumn{1}{|c|}{Liu et al.~\cite{DBLP:conf/aaai/LiuWHM24}} & \greentick & \greentick & $\blacksquare^{\dagger}$ & \greentick \\ \hline
\multicolumn{1}{|c|}{UBA-Inf~\cite{uba}} & \greentick & \greentick & $\textcolor{gray!60}{\blacksquare}^{\ddagger}$ & \redcross \\ \hline
\multicolumn{1}{|c|}{\textbf{\methodname~[Ours]}} & \greentick & \greentick & \textcolor{OliveGreen}{No Access} & \greentick \\ \hline
\end{tabular}
\begin{tablenotes}
    \item $\square$: Represents white-box model access.
    \item $\blacksquare$: Represents black-box model access.
    \item $\textcolor{gray!60}{\blacksquare}$: Represents substitute model access.
    \item $\star$: Changes the training code to maliciously modify loss value.
    \item $\dagger$: Non-data poisoning attack mode requires Black-Box model access to synthesize samples for a successful attack.
    \item $\ddagger$: Substitute model is trained on auxiliary data.
\end{tablenotes}
\end{threeparttable}
\end{adjustbox}
\end{table}

As summarized in Table\,\ref{tab:main_result}, the \textsc{Base} model of \ourmodel holds 359M parameters, and the \textsc{Large} configuration contains 1B. Despite its larger model capacity, due to the sparse activation of these parameters, only about half are active during token processing, amounting to 189M for \textsc{Base} and 553M for \textsc{Large} model. This setup ensures computational efficiency which is comparable to the smaller AV-HuBERT counterparts. For more details on the model description and computation cost, refer to Appendix\,\ref{appx:model_details} and \ref{appx:computation_cost}.
