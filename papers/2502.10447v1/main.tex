%%%%%%%% ICML 2025 EXAMPLE LATEX SUBMISSION FILE %%%%%%%%%%%%%%%%%

\documentclass{article}

% Recommended, but optional, packages for figures and better typesetting:
\usepackage{microtype}
\usepackage{graphicx}
\usepackage{subfigure}
\usepackage{booktabs} % for professional tables
\usepackage{xcolor}
\usepackage{subcaption}

% hyperref makes hyperlinks in the resulting PDF.
% If your build breaks (sometimes temporarily if a hyperlink spans a page)
% please comment out the following usepackage line and replace
% \usepackage{icml2025} with \usepackage[nohyperref]{icml2025} above.
\usepackage{hyperref}


% Attempt to make hyperref and algorithmic work together better:
\newcommand{\theHalgorithm}{\arabic{algorithm}}

% Use the following line for the initial blind version submitted for review:
\usepackage[accepted]{icml2025}

% If accepted, instead use the following line for the camera-ready submission:
% \usepackage[accepted]{icml2025}

% Optional math commands from https://github.com/goodfeli/dlbook_notation.
%%%%% NEW MATH DEFINITIONS %%%%%

\usepackage{amsmath,amsfonts,bm}
\usepackage{derivative}
% Mark sections of captions for referring to divisions of figures
\newcommand{\figleft}{{\em (Left)}}
\newcommand{\figcenter}{{\em (Center)}}
\newcommand{\figright}{{\em (Right)}}
\newcommand{\figtop}{{\em (Top)}}
\newcommand{\figbottom}{{\em (Bottom)}}
\newcommand{\captiona}{{\em (a)}}
\newcommand{\captionb}{{\em (b)}}
\newcommand{\captionc}{{\em (c)}}
\newcommand{\captiond}{{\em (d)}}

% Highlight a newly defined term
\newcommand{\newterm}[1]{{\bf #1}}

% Derivative d 
\newcommand{\deriv}{{\mathrm{d}}}

% Figure reference, lower-case.
\def\figref#1{figure~\ref{#1}}
% Figure reference, capital. For start of sentence
\def\Figref#1{Figure~\ref{#1}}
\def\twofigref#1#2{figures \ref{#1} and \ref{#2}}
\def\quadfigref#1#2#3#4{figures \ref{#1}, \ref{#2}, \ref{#3} and \ref{#4}}
% Section reference, lower-case.
\def\secref#1{section~\ref{#1}}
% Section reference, capital.
\def\Secref#1{Section~\ref{#1}}
% Reference to two sections.
\def\twosecrefs#1#2{sections \ref{#1} and \ref{#2}}
% Reference to three sections.
\def\secrefs#1#2#3{sections \ref{#1}, \ref{#2} and \ref{#3}}
% Reference to an equation, lower-case.
\def\eqref#1{equation~\ref{#1}}
% Reference to an equation, upper case
\def\Eqref#1{Equation~\ref{#1}}
% A raw reference to an equation---avoid using if possible
\def\plaineqref#1{\ref{#1}}
% Reference to a chapter, lower-case.
\def\chapref#1{chapter~\ref{#1}}
% Reference to an equation, upper case.
\def\Chapref#1{Chapter~\ref{#1}}
% Reference to a range of chapters
\def\rangechapref#1#2{chapters\ref{#1}--\ref{#2}}
% Reference to an algorithm, lower-case.
\def\algref#1{algorithm~\ref{#1}}
% Reference to an algorithm, upper case.
\def\Algref#1{Algorithm~\ref{#1}}
\def\twoalgref#1#2{algorithms \ref{#1} and \ref{#2}}
\def\Twoalgref#1#2{Algorithms \ref{#1} and \ref{#2}}
% Reference to a part, lower case
\def\partref#1{part~\ref{#1}}
% Reference to a part, upper case
\def\Partref#1{Part~\ref{#1}}
\def\twopartref#1#2{parts \ref{#1} and \ref{#2}}

\def\ceil#1{\lceil #1 \rceil}
\def\floor#1{\lfloor #1 \rfloor}
\def\1{\bm{1}}
\newcommand{\train}{\mathcal{D}}
\newcommand{\valid}{\mathcal{D_{\mathrm{valid}}}}
\newcommand{\test}{\mathcal{D_{\mathrm{test}}}}

\def\eps{{\epsilon}}


% Random variables
\def\reta{{\textnormal{$\eta$}}}
\def\ra{{\textnormal{a}}}
\def\rb{{\textnormal{b}}}
\def\rc{{\textnormal{c}}}
\def\rd{{\textnormal{d}}}
\def\re{{\textnormal{e}}}
\def\rf{{\textnormal{f}}}
\def\rg{{\textnormal{g}}}
\def\rh{{\textnormal{h}}}
\def\ri{{\textnormal{i}}}
\def\rj{{\textnormal{j}}}
\def\rk{{\textnormal{k}}}
\def\rl{{\textnormal{l}}}
% rm is already a command, just don't name any random variables m
\def\rn{{\textnormal{n}}}
\def\ro{{\textnormal{o}}}
\def\rp{{\textnormal{p}}}
\def\rq{{\textnormal{q}}}
\def\rr{{\textnormal{r}}}
\def\rs{{\textnormal{s}}}
\def\rt{{\textnormal{t}}}
\def\ru{{\textnormal{u}}}
\def\rv{{\textnormal{v}}}
\def\rw{{\textnormal{w}}}
\def\rx{{\textnormal{x}}}
\def\ry{{\textnormal{y}}}
\def\rz{{\textnormal{z}}}

% Random vectors
\def\rvepsilon{{\mathbf{\epsilon}}}
\def\rvphi{{\mathbf{\phi}}}
\def\rvtheta{{\mathbf{\theta}}}
\def\rva{{\mathbf{a}}}
\def\rvb{{\mathbf{b}}}
\def\rvc{{\mathbf{c}}}
\def\rvd{{\mathbf{d}}}
\def\rve{{\mathbf{e}}}
\def\rvf{{\mathbf{f}}}
\def\rvg{{\mathbf{g}}}
\def\rvh{{\mathbf{h}}}
\def\rvu{{\mathbf{i}}}
\def\rvj{{\mathbf{j}}}
\def\rvk{{\mathbf{k}}}
\def\rvl{{\mathbf{l}}}
\def\rvm{{\mathbf{m}}}
\def\rvn{{\mathbf{n}}}
\def\rvo{{\mathbf{o}}}
\def\rvp{{\mathbf{p}}}
\def\rvq{{\mathbf{q}}}
\def\rvr{{\mathbf{r}}}
\def\rvs{{\mathbf{s}}}
\def\rvt{{\mathbf{t}}}
\def\rvu{{\mathbf{u}}}
\def\rvv{{\mathbf{v}}}
\def\rvw{{\mathbf{w}}}
\def\rvx{{\mathbf{x}}}
\def\rvy{{\mathbf{y}}}
\def\rvz{{\mathbf{z}}}

% Elements of random vectors
\def\erva{{\textnormal{a}}}
\def\ervb{{\textnormal{b}}}
\def\ervc{{\textnormal{c}}}
\def\ervd{{\textnormal{d}}}
\def\erve{{\textnormal{e}}}
\def\ervf{{\textnormal{f}}}
\def\ervg{{\textnormal{g}}}
\def\ervh{{\textnormal{h}}}
\def\ervi{{\textnormal{i}}}
\def\ervj{{\textnormal{j}}}
\def\ervk{{\textnormal{k}}}
\def\ervl{{\textnormal{l}}}
\def\ervm{{\textnormal{m}}}
\def\ervn{{\textnormal{n}}}
\def\ervo{{\textnormal{o}}}
\def\ervp{{\textnormal{p}}}
\def\ervq{{\textnormal{q}}}
\def\ervr{{\textnormal{r}}}
\def\ervs{{\textnormal{s}}}
\def\ervt{{\textnormal{t}}}
\def\ervu{{\textnormal{u}}}
\def\ervv{{\textnormal{v}}}
\def\ervw{{\textnormal{w}}}
\def\ervx{{\textnormal{x}}}
\def\ervy{{\textnormal{y}}}
\def\ervz{{\textnormal{z}}}

% Random matrices
\def\rmA{{\mathbf{A}}}
\def\rmB{{\mathbf{B}}}
\def\rmC{{\mathbf{C}}}
\def\rmD{{\mathbf{D}}}
\def\rmE{{\mathbf{E}}}
\def\rmF{{\mathbf{F}}}
\def\rmG{{\mathbf{G}}}
\def\rmH{{\mathbf{H}}}
\def\rmI{{\mathbf{I}}}
\def\rmJ{{\mathbf{J}}}
\def\rmK{{\mathbf{K}}}
\def\rmL{{\mathbf{L}}}
\def\rmM{{\mathbf{M}}}
\def\rmN{{\mathbf{N}}}
\def\rmO{{\mathbf{O}}}
\def\rmP{{\mathbf{P}}}
\def\rmQ{{\mathbf{Q}}}
\def\rmR{{\mathbf{R}}}
\def\rmS{{\mathbf{S}}}
\def\rmT{{\mathbf{T}}}
\def\rmU{{\mathbf{U}}}
\def\rmV{{\mathbf{V}}}
\def\rmW{{\mathbf{W}}}
\def\rmX{{\mathbf{X}}}
\def\rmY{{\mathbf{Y}}}
\def\rmZ{{\mathbf{Z}}}

% Elements of random matrices
\def\ermA{{\textnormal{A}}}
\def\ermB{{\textnormal{B}}}
\def\ermC{{\textnormal{C}}}
\def\ermD{{\textnormal{D}}}
\def\ermE{{\textnormal{E}}}
\def\ermF{{\textnormal{F}}}
\def\ermG{{\textnormal{G}}}
\def\ermH{{\textnormal{H}}}
\def\ermI{{\textnormal{I}}}
\def\ermJ{{\textnormal{J}}}
\def\ermK{{\textnormal{K}}}
\def\ermL{{\textnormal{L}}}
\def\ermM{{\textnormal{M}}}
\def\ermN{{\textnormal{N}}}
\def\ermO{{\textnormal{O}}}
\def\ermP{{\textnormal{P}}}
\def\ermQ{{\textnormal{Q}}}
\def\ermR{{\textnormal{R}}}
\def\ermS{{\textnormal{S}}}
\def\ermT{{\textnormal{T}}}
\def\ermU{{\textnormal{U}}}
\def\ermV{{\textnormal{V}}}
\def\ermW{{\textnormal{W}}}
\def\ermX{{\textnormal{X}}}
\def\ermY{{\textnormal{Y}}}
\def\ermZ{{\textnormal{Z}}}

% Vectors
\def\vzero{{\bm{0}}}
\def\vone{{\bm{1}}}
\def\vmu{{\bm{\mu}}}
\def\vtheta{{\bm{\theta}}}
\def\vphi{{\bm{\phi}}}
\def\va{{\bm{a}}}
\def\vb{{\bm{b}}}
\def\vc{{\bm{c}}}
\def\vd{{\bm{d}}}
\def\ve{{\bm{e}}}
\def\vf{{\bm{f}}}
\def\vg{{\bm{g}}}
\def\vh{{\bm{h}}}
\def\vi{{\bm{i}}}
\def\vj{{\bm{j}}}
\def\vk{{\bm{k}}}
\def\vl{{\bm{l}}}
\def\vm{{\bm{m}}}
\def\vn{{\bm{n}}}
\def\vo{{\bm{o}}}
\def\vp{{\bm{p}}}
\def\vq{{\bm{q}}}
\def\vr{{\bm{r}}}
\def\vs{{\bm{s}}}
\def\vt{{\bm{t}}}
\def\vu{{\bm{u}}}
\def\vv{{\bm{v}}}
\def\vw{{\bm{w}}}
\def\vx{{\bm{x}}}
\def\vy{{\bm{y}}}
\def\vz{{\bm{z}}}

% Elements of vectors
\def\evalpha{{\alpha}}
\def\evbeta{{\beta}}
\def\evepsilon{{\epsilon}}
\def\evlambda{{\lambda}}
\def\evomega{{\omega}}
\def\evmu{{\mu}}
\def\evpsi{{\psi}}
\def\evsigma{{\sigma}}
\def\evtheta{{\theta}}
\def\eva{{a}}
\def\evb{{b}}
\def\evc{{c}}
\def\evd{{d}}
\def\eve{{e}}
\def\evf{{f}}
\def\evg{{g}}
\def\evh{{h}}
\def\evi{{i}}
\def\evj{{j}}
\def\evk{{k}}
\def\evl{{l}}
\def\evm{{m}}
\def\evn{{n}}
\def\evo{{o}}
\def\evp{{p}}
\def\evq{{q}}
\def\evr{{r}}
\def\evs{{s}}
\def\evt{{t}}
\def\evu{{u}}
\def\evv{{v}}
\def\evw{{w}}
\def\evx{{x}}
\def\evy{{y}}
\def\evz{{z}}

% Matrix
\def\mA{{\bm{A}}}
\def\mB{{\bm{B}}}
\def\mC{{\bm{C}}}
\def\mD{{\bm{D}}}
\def\mE{{\bm{E}}}
\def\mF{{\bm{F}}}
\def\mG{{\bm{G}}}
\def\mH{{\bm{H}}}
\def\mI{{\bm{I}}}
\def\mJ{{\bm{J}}}
\def\mK{{\bm{K}}}
\def\mL{{\bm{L}}}
\def\mM{{\bm{M}}}
\def\mN{{\bm{N}}}
\def\mO{{\bm{O}}}
\def\mP{{\bm{P}}}
\def\mQ{{\bm{Q}}}
\def\mR{{\bm{R}}}
\def\mS{{\bm{S}}}
\def\mT{{\bm{T}}}
\def\mU{{\bm{U}}}
\def\mV{{\bm{V}}}
\def\mW{{\bm{W}}}
\def\mX{{\bm{X}}}
\def\mY{{\bm{Y}}}
\def\mZ{{\bm{Z}}}
\def\mBeta{{\bm{\beta}}}
\def\mPhi{{\bm{\Phi}}}
\def\mLambda{{\bm{\Lambda}}}
\def\mSigma{{\bm{\Sigma}}}

% Tensor
\DeclareMathAlphabet{\mathsfit}{\encodingdefault}{\sfdefault}{m}{sl}
\SetMathAlphabet{\mathsfit}{bold}{\encodingdefault}{\sfdefault}{bx}{n}
\newcommand{\tens}[1]{\bm{\mathsfit{#1}}}
\def\tA{{\tens{A}}}
\def\tB{{\tens{B}}}
\def\tC{{\tens{C}}}
\def\tD{{\tens{D}}}
\def\tE{{\tens{E}}}
\def\tF{{\tens{F}}}
\def\tG{{\tens{G}}}
\def\tH{{\tens{H}}}
\def\tI{{\tens{I}}}
\def\tJ{{\tens{J}}}
\def\tK{{\tens{K}}}
\def\tL{{\tens{L}}}
\def\tM{{\tens{M}}}
\def\tN{{\tens{N}}}
\def\tO{{\tens{O}}}
\def\tP{{\tens{P}}}
\def\tQ{{\tens{Q}}}
\def\tR{{\tens{R}}}
\def\tS{{\tens{S}}}
\def\tT{{\tens{T}}}
\def\tU{{\tens{U}}}
\def\tV{{\tens{V}}}
\def\tW{{\tens{W}}}
\def\tX{{\tens{X}}}
\def\tY{{\tens{Y}}}
\def\tZ{{\tens{Z}}}


% Graph
\def\gA{{\mathcal{A}}}
\def\gB{{\mathcal{B}}}
\def\gC{{\mathcal{C}}}
\def\gD{{\mathcal{D}}}
\def\gE{{\mathcal{E}}}
\def\gF{{\mathcal{F}}}
\def\gG{{\mathcal{G}}}
\def\gH{{\mathcal{H}}}
\def\gI{{\mathcal{I}}}
\def\gJ{{\mathcal{J}}}
\def\gK{{\mathcal{K}}}
\def\gL{{\mathcal{L}}}
\def\gM{{\mathcal{M}}}
\def\gN{{\mathcal{N}}}
\def\gO{{\mathcal{O}}}
\def\gP{{\mathcal{P}}}
\def\gQ{{\mathcal{Q}}}
\def\gR{{\mathcal{R}}}
\def\gS{{\mathcal{S}}}
\def\gT{{\mathcal{T}}}
\def\gU{{\mathcal{U}}}
\def\gV{{\mathcal{V}}}
\def\gW{{\mathcal{W}}}
\def\gX{{\mathcal{X}}}
\def\gY{{\mathcal{Y}}}
\def\gZ{{\mathcal{Z}}}

% Sets
\def\sA{{\mathbb{A}}}
\def\sB{{\mathbb{B}}}
\def\sC{{\mathbb{C}}}
\def\sD{{\mathbb{D}}}
% Don't use a set called E, because this would be the same as our symbol
% for expectation.
\def\sF{{\mathbb{F}}}
\def\sG{{\mathbb{G}}}
\def\sH{{\mathbb{H}}}
\def\sI{{\mathbb{I}}}
\def\sJ{{\mathbb{J}}}
\def\sK{{\mathbb{K}}}
\def\sL{{\mathbb{L}}}
\def\sM{{\mathbb{M}}}
\def\sN{{\mathbb{N}}}
\def\sO{{\mathbb{O}}}
\def\sP{{\mathbb{P}}}
\def\sQ{{\mathbb{Q}}}
\def\sR{{\mathbb{R}}}
\def\sS{{\mathbb{S}}}
\def\sT{{\mathbb{T}}}
\def\sU{{\mathbb{U}}}
\def\sV{{\mathbb{V}}}
\def\sW{{\mathbb{W}}}
\def\sX{{\mathbb{X}}}
\def\sY{{\mathbb{Y}}}
\def\sZ{{\mathbb{Z}}}

% Entries of a matrix
\def\emLambda{{\Lambda}}
\def\emA{{A}}
\def\emB{{B}}
\def\emC{{C}}
\def\emD{{D}}
\def\emE{{E}}
\def\emF{{F}}
\def\emG{{G}}
\def\emH{{H}}
\def\emI{{I}}
\def\emJ{{J}}
\def\emK{{K}}
\def\emL{{L}}
\def\emM{{M}}
\def\emN{{N}}
\def\emO{{O}}
\def\emP{{P}}
\def\emQ{{Q}}
\def\emR{{R}}
\def\emS{{S}}
\def\emT{{T}}
\def\emU{{U}}
\def\emV{{V}}
\def\emW{{W}}
\def\emX{{X}}
\def\emY{{Y}}
\def\emZ{{Z}}
\def\emSigma{{\Sigma}}

% entries of a tensor
% Same font as tensor, without \bm wrapper
\newcommand{\etens}[1]{\mathsfit{#1}}
\def\etLambda{{\etens{\Lambda}}}
\def\etA{{\etens{A}}}
\def\etB{{\etens{B}}}
\def\etC{{\etens{C}}}
\def\etD{{\etens{D}}}
\def\etE{{\etens{E}}}
\def\etF{{\etens{F}}}
\def\etG{{\etens{G}}}
\def\etH{{\etens{H}}}
\def\etI{{\etens{I}}}
\def\etJ{{\etens{J}}}
\def\etK{{\etens{K}}}
\def\etL{{\etens{L}}}
\def\etM{{\etens{M}}}
\def\etN{{\etens{N}}}
\def\etO{{\etens{O}}}
\def\etP{{\etens{P}}}
\def\etQ{{\etens{Q}}}
\def\etR{{\etens{R}}}
\def\etS{{\etens{S}}}
\def\etT{{\etens{T}}}
\def\etU{{\etens{U}}}
\def\etV{{\etens{V}}}
\def\etW{{\etens{W}}}
\def\etX{{\etens{X}}}
\def\etY{{\etens{Y}}}
\def\etZ{{\etens{Z}}}

% The true underlying data generating distribution
\newcommand{\pdata}{p_{\rm{data}}}
\newcommand{\ptarget}{p_{\rm{target}}}
\newcommand{\pprior}{p_{\rm{prior}}}
\newcommand{\pbase}{p_{\rm{base}}}
\newcommand{\pref}{p_{\rm{ref}}}

% The empirical distribution defined by the training set
\newcommand{\ptrain}{\hat{p}_{\rm{data}}}
\newcommand{\Ptrain}{\hat{P}_{\rm{data}}}
% The model distribution
\newcommand{\pmodel}{p_{\rm{model}}}
\newcommand{\Pmodel}{P_{\rm{model}}}
\newcommand{\ptildemodel}{\tilde{p}_{\rm{model}}}
% Stochastic autoencoder distributions
\newcommand{\pencode}{p_{\rm{encoder}}}
\newcommand{\pdecode}{p_{\rm{decoder}}}
\newcommand{\precons}{p_{\rm{reconstruct}}}

\newcommand{\laplace}{\mathrm{Laplace}} % Laplace distribution

\newcommand{\E}{\mathbb{E}}
\newcommand{\Ls}{\mathcal{L}}
\newcommand{\R}{\mathbb{R}}
\newcommand{\emp}{\tilde{p}}
\newcommand{\lr}{\alpha}
\newcommand{\reg}{\lambda}
\newcommand{\rect}{\mathrm{rectifier}}
\newcommand{\softmax}{\mathrm{softmax}}
\newcommand{\sigmoid}{\sigma}
\newcommand{\softplus}{\zeta}
\newcommand{\KL}{D_{\mathrm{KL}}}
\newcommand{\Var}{\mathrm{Var}}
\newcommand{\standarderror}{\mathrm{SE}}
\newcommand{\Cov}{\mathrm{Cov}}
% Wolfram Mathworld says $L^2$ is for function spaces and $\ell^2$ is for vectors
% But then they seem to use $L^2$ for vectors throughout the site, and so does
% wikipedia.
\newcommand{\normlzero}{L^0}
\newcommand{\normlone}{L^1}
\newcommand{\normltwo}{L^2}
\newcommand{\normlp}{L^p}
\newcommand{\normmax}{L^\infty}

\newcommand{\parents}{Pa} % See usage in notation.tex. Chosen to match Daphne's book.

\DeclareMathOperator*{\argmax}{arg\,max}
\DeclareMathOperator*{\argmin}{arg\,min}

\DeclareMathOperator{\sign}{sign}
\DeclareMathOperator{\Tr}{Tr}
\let\ab\allowbreak

% \usepackage{microtype}
\usepackage{geometry}
% \usepackage{subfig}
\usepackage{booktabs} 
\usepackage{bbm}
\usepackage{mathtools}
% \usepackage{amsthm}
\usepackage{nccmath}
\usepackage{setspace}

\usepackage{caption}
\usepackage{subcaption}

\usepackage[linesnumbered,ruled,vlined]{algorithm2e}
% \usepackage{algorithmic}
% \usepackage{algorithm}

\SetKwInput{KwInput}{Input}                % Set the Input
\SetKwInput{KwOutput}{Output}              % set the Output
\newcommand\mycommfont[1]{\footnotesize\ttfamily\textcolor{blue}{#1}}
\SetCommentSty{mycommfont}
\newcommand{\algcapsty}[1]{\small\sffamily\bfseries{#1}}
\SetAlCapSty{algcapsty}

\usepackage[T1]{fontenc}
\usepackage{wrapfig,lipsum,booktabs}

% \usepackage{natbib}
\usepackage{soul}
\usepackage{dsfont}
\usepackage{enumerate}
\usepackage{enumitem}

% \usepackage{kotex}
% \usepackage{hyperref}
% \usepackage[hidelinks]{hyperref}
% \usepackage{amsmath}
% \usepackage{amsthm}
\usepackage{amsfonts}
\usepackage{bbm}
\usepackage{dsfont}
\usepackage[Symbol]{upgreek}
\usepackage{lscape}
\usepackage{caption}
\usepackage{balance}
\usepackage{xspace}
\usepackage{float}
\usepackage{kotex}

\usepackage{wasysym}
%\usepackage[table,xcdraw,dvipsnames]{xcolor}
\usepackage{xcolor}
\usepackage{multirow}
\usepackage{array, boldline, rotating}

\usepackage{amssymb}% http://ctan.org/pkg/amssymb
\usepackage{pifont}% http://ctan.org/pkg/pifont
\newcommand{\cmark}{\ding{51}\xspace}%
\newcommand{\omark}{\textbf{$\mathcal{O}$}\xspace}%
\newcommand{\xmark}{\ding{55}\xspace}%

\newcommand{\ds}[1]{\mathds{#1}}
\newcommand{\mc}[1]{\mathcal{#1}}
\newcommand{\bb}[1]{\mathbbm{#1}}

% %%%%%% Theorem Related Things %%%%%%
% \theoremstyle{plain}
% \newtheorem{thm}{Theorem}
% \newtheorem{cor}{Corollary}
% \newtheorem{lem}{Lemma}
% \newtheorem{prop}{Proposition}

% \theoremstyle{definition}
% \newtheorem{defn}{Definition}
% \newtheorem{assum}{Assumption}



% Citation

% \let\oldeqcite\cite
% \renewcommand*\cite[1]{(\oldcite{#1})}
\let\oldeqref\eqref
\renewcommand*\eqref[1]{(\ref{#1})}

% % Highlight (incl. note)
\newcommand{\smnote}[1]{\textbf{\textcolor{Cyan}{SM: #1}}}
\newcommand{\jhnote}[1]{\textbf{\textcolor{Orange}{JH: #1}}}
\newcommand{\yes}[1]{\textcolor{blue}{[YES]}}
\newcommand{\no}[1]{\textcolor{orange}{[NO]}}
\newcommand{\na}[1]{\textcolor{gray}{[N/A]}}
%\newcommand\bg[1]{\textcolor{blue}{#1}} % JH
\newcommand\jt[1]{\textcolor{brown}{#1}} % JT
\newcommand\jh[1]{\textcolor{black}{#1}} % JH
\newcommand\sm[1]{\textcolor{blue}{#1}} % SM
\newcommand{\eg}{\emph{e.g.,~}}
\newcommand{\ie}{\emph{i.e.,~}}


% \renewcommand\jt[1]{\textcolor{black}{#1}} % JT
% \renewcommand\jh[1]{\textcolor{black}{#1}} % JH

% % Separation (paragraph)
\newcommand{\myparagraph}[1]{\vspace{0.07cm}\noindent\textbf{#1}~}

% % math op.

% % Font
\def\code#1{\texttt{#1}}
\DeclarePairedDelimiter\norm{\lVert}{\rVert}

% % % Definition
% \theoremstyle{definition}
\newcommand\scalemath[2]{\scalebox{#1}{\mbox{\ensuremath{\displaystyle #2}}}}



\newcommand{\thickhline}{\hlineB{4}}
\newcommand{\bfcode}[1]{\code{\textbf{#1}}}


\definecolor{LightCyan}{rgb}{0.88,1,1}
\definecolor{Blue}{rgb}{0, 0.3, 0.6}
\definecolor{Orange}{rgb}{0.8, 0.4, 0}
\definecolor{Green}{rgb}{0.0, 0.8, 0.0 }
\definecolor{Red}{rgb}{0.95, 0.55, 0.6}
\definecolor{Skyblue}{rgb}{0.6, 0.6, 0.95 }



% Supplementary title
\NewDocumentEnvironment{suptitle}{ +b }{
    \twocolumn[{#1}]%
}{}

\NewDocumentCommand{\supptitle}{s}{
\begin{suptitle}
        \centering
        % \rule{\textwidth}{0.07cm}\\[-0.34cm]
        \rule{\textwidth}{0.03cm}\\[0.1cm]
        -Supplementary Material-\\[0.2cm]
        {\Large 
            \textbf{\mytitle }
        }\\%[0.40cm]
        \rule{\textwidth}{0.03cm}\\[0.2cm]
\end{suptitle}}

% For theorems and such
\usepackage{mathtools}
\usepackage{amsthm}

% if you use cleveref..
\usepackage[capitalize,noabbrev]{cleveref}

%%%%%%%%%%%%%%%%%%%%%%%%%%%%%%%%
% THEOREMS
%%%%%%%%%%%%%%%%%%%%%%%%%%%%%%%%
\theoremstyle{plain}
\newtheorem{theorem}{Theorem}[section]
\newtheorem{proposition}[theorem]{Proposition}
\newtheorem{lemma}[theorem]{Lemma}
\newtheorem{corollary}[theorem]{Corollary}
\theoremstyle{definition}
\newtheorem{definition}[theorem]{Definition}
\newtheorem{assumption}[theorem]{Assumption}
\theoremstyle{remark}
\newtheorem{remark}[theorem]{Remark}

% Todonotes is useful during development; simply uncomment the next line
%    and comment out the line below the next line to turn off comments
%\usepackage[disable,textsize=tiny]{todonotes}
\usepackage[textsize=tiny]{todonotes}


% The \icmltitle you define below is probably too long as a header.
% Therefore, a short form for the running title is supplied here:
\icmltitlerunning{MoHAVE: Mixture of Hierarchical Audio-Visual Experts for Robust Speech Recognition}

\begin{document}

\twocolumn[
\icmltitle{{\vspace{-5pt}\includegraphics[width=18pt]{desert_emoji.png}}\,MoHAVE: Mixture of Hierarchical\\Audio-Visual Experts for Robust Speech Recognition}

% It is OKAY to include author information, even for blind
% submissions: the style file will automatically remove it for you
% unless you've provided the [accepted] option to the icml2025
% package.

% List of affiliations: The first argument should be a (short)
% identifier you will use later to specify author affiliations
% Academic affiliations should list Department, University, City, Region, Country
% Industry affiliations should list Company, City, Region, Country

% You can specify symbols, otherwise they are numbered in order.
% Ideally, you should not use this facility. Affiliations will be numbered
% in order of appearance and this is the preferred way.
\icmlsetsymbol{equal}{*}

\begin{icmlauthorlist}
\icmlauthor{Sungnyun Kim}{kaistai}
\icmlauthor{Kangwook Jang}{kaistee}
\icmlauthor{Sangmin Bae}{kaistai}
\icmlauthor{Sungwoo Cho}{kaistai}
\icmlauthor{Se-Young Yun}{kaistai}
% \icmlauthor{Firstname6 Lastname6}{sch,yyy,comp}
% \icmlauthor{Firstname7 Lastname7}{comp}
%\icmlauthor{}{sch}
% \icmlauthor{Firstname8 Lastname8}{sch}
% \icmlauthor{Firstname8 Lastname8}{yyy,comp}
%\icmlauthor{}{sch}
%\icmlauthor{}{sch}
\end{icmlauthorlist}

\icmlaffiliation{kaistai}{KAIST AI, Republic of Korea}
\icmlaffiliation{kaistee}{School of Electrical Engineering, KAIST, Republic of Korea}
% \icmlaffiliation{sch}{School of ZZZ, Institute of WWW, Location, Country}

\icmlcorrespondingauthor{Se-Young Yun}{yunseyoung@kaist.ac.kr}
% \icmlcorrespondingauthor{Firstname2 Lastname2}{first2.last2@www.uk}

% You may provide any keywords that you
% find helpful for describing your paper; these are used to populate
% the "keywords" metadata in the PDF but will not be shown in the document
\icmlkeywords{Machine Learning, ICML}

\vskip 0.3in
]

% this must go after the closing bracket ] following \twocolumn[ ...

% This command actually creates the footnote in the first column
% listing the affiliations and the copyright notice.
% The command takes one argument, which is text to display at the start of the footnote.
% The \icmlEqualContribution command is standard text for equal contribution.
% Remove it (just {}) if you do not need this facility.

\printAffiliationsAndNotice{}  % leave blank if no need to mention equal contribution
% \printAffiliationsAndNotice{\icmlEqualContribution} % otherwise use the standard text.

%%%%%%%%%%%%%%%%%%%%%%%%%%%%%%%%%%%%%%%%%%%%%%%%%%%%%%%%%%%%%%%%%
\begin{abstract}\label{00_Abstract}
Research in the field of automated vehicles, or more generally cognitive cyber-physical systems that operate in the real world, is leading to increasingly complex systems. Among other things, artificial intelligence enables an ever-increasing degree of autonomy. In this context, the V-model, which has served for decades as a process reference model of the system development lifecycle is reaching its limits. To the contrary, innovative processes and frameworks have been developed that take into account the characteristics of emerging autonomous systems. To bridge the gap and merge the different methodologies, we present an extension of the V-model for iterative data-based development processes that harmonizes and formalizes the existing methods towards a generic framework. The iterative approach allows for seamless integration of continuous system refinement. While the data-based approach constitutes the consideration of data-based development processes and formalizes the use of synthetic and real world data. In this way, formalizing the process of development, verification, validation, and continuous integration contributes to ensuring the safety of emerging complex systems that incorporate AI. 
\end{abstract}


\begin{IEEEkeywords}
	Process Reference Model, V-Model, Continuous Integration, AI Systems, Autonomy Technology, Safety Assurance
\end{IEEEkeywords}

\section{Introduction}

% \textcolor{red}{Still on working}

% \textcolor{red}{add label for each section}


Robot learning relies on diverse and high-quality data to learn complex behaviors \cite{aldaco2024aloha, wang2024dexcap}.
Recent studies highlight that models trained on datasets with greater complexity and variation in the domain tend to generalize more effectively across broader scenarios \cite{mann2020language, radford2021learning, gao2024efficient}.
% However, creating such diverse datasets in the real world presents significant challenges.
% Modifying physical environments and adjusting robot hardware settings require considerable time, effort, and financial resources.
% In contrast, simulation environments offer a flexible and efficient alternative.
% Simulations allow for the creation and modification of digital environments with a wide range of object shapes, weights, materials, lighting, textures, friction coefficients, and so on to incorporate domain randomization,
% which helps improve the robustness of models when deployed in real-world conditions.
% These environments can be easily adjusted and reset, enabling faster iterations and data collection.
% Additionally, simulations provide the ability to consistently reproduce scenarios, which is essential for benchmarking and model evaluation.
% Another advantage of simulations is their flexibility in sensor integration. Sensors such as cameras, LiDARs, and tactile sensors can be added or repositioned without the physical limitations present in real-world setups. Simulations also eliminate the risk of damaging expensive hardware during edge-case experiments, making them an ideal platform for testing rare or dangerous scenarios that are impractical to explore in real life.
By leveraging immersive perspectives and interactions, Extended Reality\footnote{Extended Reality is an umbrella term to refer to Augmented Reality, Mixed Reality, and Virtual Reality \cite{wikipediaExtendedReality}}
(XR)
is a promising candidate for efficient and intuitive large scale data collection \cite{jiang2024comprehensive, arcade}
% With the demand for collecting data, XR provides a promising approach for humans to teach robots by offering users an immersive experience.
in simulation \cite{jiang2024comprehensive, arcade, dexhub-park} and real-world scenarios \cite{openteach, opentelevision}.
However, reusing and reproducing current XR approaches for robot data collection for new settings and scenarios is complicated and requires significant effort.
% are difficult to reuse and reproduce system makes it hard to reuse and reproduce in another data collection pipeline.
This bottleneck arises from three main limitations of current XR data collection and interaction frameworks: \textit{asset limitation}, \textit{simulator limitation}, and \textit{device limitation}.
% \textcolor{red}{ASSIGN THESE CITATION PROPERLY:}
% \textcolor{red}{list them by time order???}
% of collecting data by using XR have three main limitations.
Current approaches suffering from \textit{asset limitation} \cite{arclfd, jiang2024comprehensive, arcade, george2025openvr, vicarios}
% Firstly, recent works \cite{jiang2024comprehensive, arcade, dexhub-park}
can only use predefined robot models and task scenes. Configuring new tasks requires significant effort, since each new object or model must be specifically integrated into the XR application.
% and it takes too much effort to configure new tasks in their systems since they cannot spawn arbitrary models in the XR application.
The vast majority of application are developed for specific simulators or real-world scenarios. This \textit{simulator limitation} \cite{mosbach2022accelerating, lipton2017baxter, dexhub-park, arcade}
% Secondly, existing systems are limited to a single simulation platform or real-world scenarios.
significantly reduces reusability and makes adaptation to new simulation platforms challenging.
Additionally, most current XR frameworks are designed for a specific version of a single XR headset, leading to a \textit{device limitation} 
\cite{lipton2017baxter, armada, openteach, meng2023virtual}.
% and there is no work working on the extendability of transferring to a new headsets as far as we know.
To the best of our knowledge, no existing work has explored the extensibility or transferability of their framework to different headsets.
These limitations hamper reproducibility and broader contributions of XR based data collection and interaction to the research community.
% as each research group typically has its own data collection pipeline.
% In addition to these main limitations, existing XR systems are not well suited for managing multiple robot systems,
% as they are often designed for single-operator use.

In addition to these main limitations, existing XR systems are often designed for single-operator use, prohibiting collaborative data collection.
At the same time, controlling multiple robots at once can be very difficult for a single operator,
making data collection in multi-robot scenarios particularly challenging \cite{orun2019effect}.
Although there are some works using collaborative data collection in the context of tele-operation \cite{tung2021learning, Qin2023AnyTeleopAG},
there is no XR-based data collection system supporting collaborative data collection.
This limitation highlights the need for more advanced XR solutions that can better support multi-robot and multi-user scenarios.
% \textcolor{red}{more papers about collaborative data collection}

To address all of these issues, we propose \textbf{IRIS},
an \textbf{I}mmersive \textbf{R}obot \textbf{I}nteraction \textbf{S}ystem.
This general system supports various simulators, benchmarks and real-world scenarios.
It is easily extensible to new simulators and XR headsets.
IRIS achieves generalization across six dimensions:
% \begin{itemize}
%     \item \textit{Cross-scene} : diverse object models;
%     \item \textit{Cross-embodiment}: diverse robot models;
%     \item \textit{Cross-simulator}: 
%     \item \textit{Cross-reality}: fd
%     \item \textit{Cross-platform}: fd
%     \item \textit{Cross-users}: fd
% \end{itemize}
\textbf{Cross-Scene}, \textbf{Cross-Embodiment}, \textbf{Cross-Simulator}, \textbf{Cross-Reality}, \textbf{Cross-Platform}, and \textbf{Cross-User}.

\textbf{Cross-Scene} and \textbf{Cross-Embodiment} allow the system to handle arbitrary objects and robots in the simulation,
eliminating restrictions about predefined models in XR applications.
IRIS achieves these generalizations by introducing a unified scene specification, representing all objects,
including robots, as data structures with meshes, materials, and textures.
The unified scene specification is transmitted to the XR application to create and visualize an identical scene.
By treating robots as standard objects, the system simplifies XR integration,
allowing researchers to work with various robots without special robot-specific configurations.
\textbf{Cross-Simulator} ensures compatibility with various simulation engines.
IRIS simplifies adaptation by parsing simulated scenes into the unified scene specification, eliminating the need for XR application modifications when switching simulators.
New simulators can be integrated by creating a parser to convert their scenes into the unified format.
This flexibility is demonstrated by IRIS’ support for Mujoco \cite{todorov2012mujoco}, IsaacSim \cite{mittal2023orbit}, CoppeliaSim \cite{coppeliaSim}, and even the recent Genesis \cite{Genesis} simulator.
\textbf{Cross-Reality} enables the system to function seamlessly in both virtual simulations and real-world applications.
IRIS enables real-world data collection through camera-based point cloud visualization.
\textbf{Cross-Platform} allows for compatibility across various XR devices.
Since XR device APIs differ significantly, making a single codebase impractical, IRIS XR application decouples its modules to maximize code reuse.
This application, developed by Unity \cite{unity3dUnityManual}, separates scene visualization and interaction, allowing developers to integrate new headsets by reusing the visualization code and only implementing input handling for hand, head, and motion controller tracking.
IRIS provides an implementation of the XR application in the Unity framework, allowing for a straightforward deployment to any device that supports Unity. 
So far, IRIS was successfully deployed to the Meta Quest 3 and HoloLens 2.
Finally, the \textbf{Cross-User} ability allows multiple users to interact within a shared scene.
IRIS achieves this ability by introducing a protocol to establish the communication between multiple XR headsets and the simulation or real-world scenarios.
Additionally, IRIS leverages spatial anchors to support the alignment of virtual scenes from all deployed XR headsets.
% To make an seamless user experience for robot learning data collection,
% IRIS also tested in three different robot control interface
% Furthermore, to demonstrate the extensibility of our approach, we have implemented a robot-world pipeline for real robot data collection, ensuring that the system can be used in both simulated and real-world environments.
The Immersive Robot Interaction System makes the following contributions\\
\textbf{(1) A unified scene specification} that is compatible with multiple robot simulators. It enables various XR headsets to visualize and interact with simulated objects and robots, providing an immersive experience while ensuring straightforward reusability and reproducibility.\\
\textbf{(2) A collaborative data collection framework} designed for XR environments. The framework facilitates enhanced robot data acquisition.\\
\textbf{(3) A user study} demonstrating that IRIS significantly improves data collection efficiency and intuitiveness compared to the LIBERO baseline.

% \begin{table*}[t]
%     \centering
%     \begin{tabular}{lccccccc}
%         \toprule
%         & \makecell{Physical\\Interaction}
%         & \makecell{XR\\Enabled}
%         & \makecell{Free\\View}
%         & \makecell{Multiple\\Robots}
%         & \makecell{Robot\\Control}
%         % Force Feedback???
%         & \makecell{Soft Object\\Supported}
%         & \makecell{Collaborative\\Data} \\
%         \midrule
%         ARC-LfD \cite{arclfd}                              & Real        & \cmark & \xmark & \xmark & Joint              & \xmark & \xmark \\
%         DART \cite{dexhub-park}                            & Sim         & \cmark & \cmark & \cmark & Cartesian          & \xmark & \xmark \\
%         \citet{jiang2024comprehensive}                     & Sim         & \cmark & \xmark & \xmark & Joint \& Cartesian & \xmark & \xmark \\
%         \citet{mosbach2022accelerating}                    & Sim         & \cmark & \cmark & \xmark & Cartesian          & \xmark & \xmark \\
%         ARCADE \cite{arcade}                               & Real        & \cmark & \cmark & \xmark & Cartesian          & \xmark & \xmark \\
%         Holo-Dex \cite{holodex}                            & Real        & \cmark & \xmark & \cmark & Cartesian          & \cmark & \xmark \\
%         ARMADA \cite{armada}                               & Real        & \cmark & \xmark & \cmark & Cartesian          & \cmark & \xmark \\
%         Open-TeleVision \cite{opentelevision}              & Real        & \cmark & \cmark & \cmark & Cartesian          & \cmark & \xmark \\
%         OPEN TEACH \cite{openteach}                        & Real        & \cmark & \xmark & \cmark & Cartesian          & \cmark & \cmark \\
%         GELLO \cite{wu2023gello}                           & Real        & \xmark & \cmark & \cmark & Joint              & \cmark & \xmark \\
%         DexCap \cite{wang2024dexcap}                       & Real        & \xmark & \cmark & \xmark & Cartesian          & \cmark & \xmark \\
%         AnyTeleop \cite{Qin2023AnyTeleopAG}                & Real        & \xmark & \xmark & \cmark & Cartesian          & \cmark & \cmark \\
%         Vicarios \cite{vicarios}                           & Real        & \cmark & \xmark & \xmark & Cartesian          & \cmark & \xmark \\     
%         Augmented Visual Cues \cite{augmentedvisualcues}   & Real        & \cmark & \cmark & \xmark & Cartesian          & \xmark & \xmark \\ 
%         \citet{wang2024robotic}                            & Real        & \cmark & \cmark & \xmark & Cartesian          & \cmark & \xmark \\
%         Bunny-VisionPro \cite{bunnyvisionpro}              & Real        & \cmark & \cmark & \cmark & Cartesian          & \cmark & \xmark \\
%         IMMERTWIN \cite{immertwin}                         & Real        & \cmark & \cmark & \cmark & Cartesian          & \xmark & \xmark \\
%         \citet{meng2023virtual}                            & Sim \& Real & \cmark & \cmark & \xmark & Cartesian          & \xmark & \xmark \\
%         Shared Control Framework \cite{sharedctlframework} & Real        & \cmark & \cmark & \cmark & Cartesian          & \xmark & \xmark \\
%         OpenVR \cite{openvr}                               & Real        & \cmark & \cmark & \xmark & Cartesian          & \xmark & \xmark \\
%         \citet{digitaltwinmr}                              & Real        & \cmark & \cmark & \xmark & Cartesian          & \cmark & \xmark \\
        
%         \midrule
%         \textbf{Ours} & Sim \& Real & \cmark & \cmark & \cmark & Joint \& Cartesian  & \cmark & \cmark \\
%         \bottomrule
%     \end{tabular}
%     \caption{This is a cross-column table with automatic line breaking.}
%     \label{tab:cross-column}
% \end{table*}

% \begin{table*}[t]
%     \centering
%     \begin{tabular}{lccccccc}
%         \toprule
%         & \makecell{Cross-Embodiment}
%         & \makecell{Cross-Scene}
%         & \makecell{Cross-Simulator}
%         & \makecell{Cross-Reality}
%         & \makecell{Cross-Platform}
%         & \makecell{Cross-User} \\
%         \midrule
%         ARC-LfD \cite{arclfd}                              & \xmark & \xmark & \xmark & \xmark & \xmark & \xmark \\
%         DART \cite{dexhub-park}                            & \cmark & \cmark & \xmark & \xmark & \xmark & \xmark \\
%         \citet{jiang2024comprehensive}                     & \xmark & \cmark & \xmark & \xmark & \xmark & \xmark \\
%         \citet{mosbach2022accelerating}                    & \xmark & \cmark & \xmark & \xmark & \xmark & \xmark \\
%         ARCADE \cite{arcade}                               & \xmark & \xmark & \xmark & \xmark & \xmark & \xmark \\
%         Holo-Dex \cite{holodex}                            & \cmark & \xmark & \xmark & \xmark & \xmark & \xmark \\
%         ARMADA \cite{armada}                               & \cmark & \xmark & \xmark & \xmark & \xmark & \xmark \\
%         Open-TeleVision \cite{opentelevision}              & \cmark & \xmark & \xmark & \xmark & \cmark & \xmark \\
%         OPEN TEACH \cite{openteach}                        & \cmark & \xmark & \xmark & \xmark & \xmark & \cmark \\
%         GELLO \cite{wu2023gello}                           & \cmark & \xmark & \xmark & \xmark & \xmark & \xmark \\
%         DexCap \cite{wang2024dexcap}                       & \xmark & \xmark & \xmark & \xmark & \xmark & \xmark \\
%         AnyTeleop \cite{Qin2023AnyTeleopAG}                & \cmark & \cmark & \cmark & \cmark & \xmark & \cmark \\
%         Vicarios \cite{vicarios}                           & \xmark & \xmark & \xmark & \xmark & \xmark & \xmark \\     
%         Augmented Visual Cues \cite{augmentedvisualcues}   & \xmark & \xmark & \xmark & \xmark & \xmark & \xmark \\ 
%         \citet{wang2024robotic}                            & \xmark & \xmark & \xmark & \xmark & \xmark & \xmark \\
%         Bunny-VisionPro \cite{bunnyvisionpro}              & \cmark & \xmark & \xmark & \xmark & \xmark & \xmark \\
%         IMMERTWIN \cite{immertwin}                         & \cmark & \xmark & \xmark & \xmark & \xmark & \xmark \\
%         \citet{meng2023virtual}                            & \xmark & \cmark & \xmark & \cmark & \xmark & \xmark \\
%         \citet{sharedctlframework}                         & \cmark & \xmark & \xmark & \xmark & \xmark & \xmark \\
%         OpenVR \cite{george2025openvr}                               & \xmark & \xmark & \xmark & \xmark & \xmark & \xmark \\
%         \citet{digitaltwinmr}                              & \xmark & \xmark & \xmark & \xmark & \xmark & \xmark \\
        
%         \midrule
%         \textbf{Ours} & \cmark & \cmark & \cmark & \cmark & \cmark & \cmark \\
%         \bottomrule
%     \end{tabular}
%     \caption{This is a cross-column table with automatic line breaking.}
% \end{table*}

% \begin{table*}[t]
%     \centering
%     \begin{tabular}{lccccccc}
%         \toprule
%         & \makecell{Cross-Scene}
%         & \makecell{Cross-Embodiment}
%         & \makecell{Cross-Simulator}
%         & \makecell{Cross-Reality}
%         & \makecell{Cross-Platform}
%         & \makecell{Cross-User}
%         & \makecell{Control Space} \\
%         \midrule
%         % Vicarios \cite{vicarios}                           & \xmark & \xmark & \xmark & \xmark & \xmark & \xmark \\     
%         % Augmented Visual Cues \cite{augmentedvisualcues}   & \xmark & \xmark & \xmark & \xmark & \xmark & \xmark \\ 
%         % OpenVR \cite{george2025openvr}                     & \xmark & \xmark & \xmark & \xmark & \xmark & \xmark \\
%         \citet{digitaltwinmr}                              & \xmark & \xmark & \xmark & \xmark & \xmark & \xmark &  \\
%         ARC-LfD \cite{arclfd}                              & \xmark & \xmark & \xmark & \xmark & \xmark & \xmark &  \\
%         \citet{sharedctlframework}                         & \cmark & \xmark & \xmark & \xmark & \xmark & \xmark &  \\
%         \citet{jiang2024comprehensive}                     & \cmark & \xmark & \xmark & \xmark & \xmark & \xmark &  \\
%         \citet{mosbach2022accelerating}                    & \cmark & \xmark & \xmark & \xmark & \xmark & \xmark & \\
%         Holo-Dex \cite{holodex}                            & \cmark & \xmark & \xmark & \xmark & \xmark & \xmark & \\
%         ARCADE \cite{arcade}                               & \cmark & \cmark & \xmark & \xmark & \xmark & \xmark & \\
%         DART \cite{dexhub-park}                            & Limited & Limited & Mujoco & Sim & Vision Pro & \xmark &  Cartesian\\
%         ARMADA \cite{armada}                               & \cmark & \cmark & \xmark & \xmark & \xmark & \xmark & \\
%         \citet{meng2023virtual}                            & \cmark & \cmark & \xmark & \cmark & \xmark & \xmark & \\
%         % GELLO \cite{wu2023gello}                           & \cmark & \xmark & \xmark & \xmark & \xmark & \xmark \\
%         % DexCap \cite{wang2024dexcap}                       & \xmark & \xmark & \xmark & \xmark & \xmark & \xmark \\
%         % AnyTeleop \cite{Qin2023AnyTeleopAG}                & \cmark & \cmark & \cmark & \cmark & \xmark & \cmark \\
%         % \citet{wang2024robotic}                            & \xmark & \xmark & \xmark & \xmark & \xmark & \xmark \\
%         Bunny-VisionPro \cite{bunnyvisionpro}              & \cmark & \cmark & \xmark & \xmark & \xmark & \xmark & \\
%         IMMERTWIN \cite{immertwin}                         & \cmark & \cmark & \xmark & \xmark & \xmark & \xmark & \\
%         Open-TeleVision \cite{opentelevision}              & \cmark & \cmark & \xmark & \xmark & \cmark & \xmark & \\
%         \citet{szczurek2023multimodal}                     & \xmark & \xmark & \xmark & Real & \xmark & \cmark & \\
%         OPEN TEACH \cite{openteach}                        & \cmark & \cmark & \xmark & \xmark & \xmark & \cmark & \\
%         \midrule
%         \textbf{Ours} & \cmark & \cmark & \cmark & \cmark & \cmark & \cmark \\
%         \bottomrule
%     \end{tabular}
%     \caption{TODO, Bruce: this table can be further optimized.}
% \end{table*}

\definecolor{goodgreen}{HTML}{228833}
\definecolor{goodred}{HTML}{EE6677}
\definecolor{goodgray}{HTML}{BBBBBB}

\begin{table*}[t]
    \centering
    \begin{adjustbox}{max width=\textwidth}
    \renewcommand{\arraystretch}{1.2}    
    \begin{tabular}{lccccccc}
        \toprule
        & \makecell{Cross-Scene}
        & \makecell{Cross-Embodiment}
        & \makecell{Cross-Simulator}
        & \makecell{Cross-Reality}
        & \makecell{Cross-Platform}
        & \makecell{Cross-User}
        & \makecell{Control Space} \\
        \midrule
        % Vicarios \cite{vicarios}                           & \xmark & \xmark & \xmark & \xmark & \xmark & \xmark \\     
        % Augmented Visual Cues \cite{augmentedvisualcues}   & \xmark & \xmark & \xmark & \xmark & \xmark & \xmark \\ 
        % OpenVR \cite{george2025openvr}                     & \xmark & \xmark & \xmark & \xmark & \xmark & \xmark \\
        \citet{digitaltwinmr}                              & \textcolor{goodred}{Limited}     & \textcolor{goodred}{Single Robot} & \textcolor{goodred}{Unity}    & \textcolor{goodred}{Real}          & \textcolor{goodred}{Meta Quest 2} & \textcolor{goodgray}{N/A} & \textcolor{goodred}{Cartesian} \\
        ARC-LfD \cite{arclfd}                              & \textcolor{goodgray}{N/A}        & \textcolor{goodred}{Single Robot} & \textcolor{goodgray}{N/A}     & \textcolor{goodred}{Real}          & \textcolor{goodred}{HoloLens}     & \textcolor{goodgray}{N/A} & \textcolor{goodred}{Cartesian} \\
        \citet{sharedctlframework}                         & \textcolor{goodred}{Limited}     & \textcolor{goodred}{Single Robot} & \textcolor{goodgray}{N/A}     & \textcolor{goodred}{Real}          & \textcolor{goodred}{HTC Vive Pro} & \textcolor{goodgray}{N/A} & \textcolor{goodred}{Cartesian} \\
        \citet{jiang2024comprehensive}                     & \textcolor{goodred}{Limited}     & \textcolor{goodred}{Single Robot} & \textcolor{goodgray}{N/A}     & \textcolor{goodred}{Real}          & \textcolor{goodred}{HoloLens 2}   & \textcolor{goodgray}{N/A} & \textcolor{goodgreen}{Joint \& Cartesian} \\
        \citet{mosbach2022accelerating}                    & \textcolor{goodgreen}{Available} & \textcolor{goodred}{Single Robot} & \textcolor{goodred}{IsaacGym} & \textcolor{goodred}{Sim}           & \textcolor{goodred}{Vive}         & \textcolor{goodgray}{N/A} & \textcolor{goodgreen}{Joint \& Cartesian} \\
        Holo-Dex \cite{holodex}                            & \textcolor{goodgray}{N/A}        & \textcolor{goodred}{Single Robot} & \textcolor{goodgray}{N/A}     & \textcolor{goodred}{Real}          & \textcolor{goodred}{Meta Quest 2} & \textcolor{goodgray}{N/A} & \textcolor{goodred}{Joint} \\
        ARCADE \cite{arcade}                               & \textcolor{goodgray}{N/A}        & \textcolor{goodred}{Single Robot} & \textcolor{goodgray}{N/A}     & \textcolor{goodred}{Real}          & \textcolor{goodred}{HoloLens 2}   & \textcolor{goodgray}{N/A} & \textcolor{goodred}{Cartesian} \\
        DART \cite{dexhub-park}                            & \textcolor{goodred}{Limited}     & \textcolor{goodred}{Limited}      & \textcolor{goodred}{Mujoco}   & \textcolor{goodred}{Sim}           & \textcolor{goodred}{Vision Pro}   & \textcolor{goodgray}{N/A} & \textcolor{goodred}{Cartesian} \\
        ARMADA \cite{armada}                               & \textcolor{goodgray}{N/A}        & \textcolor{goodred}{Limited}      & \textcolor{goodgray}{N/A}     & \textcolor{goodred}{Real}          & \textcolor{goodred}{Vision Pro}   & \textcolor{goodgray}{N/A} & \textcolor{goodred}{Cartesian} \\
        \citet{meng2023virtual}                            & \textcolor{goodred}{Limited}     & \textcolor{goodred}{Single Robot} & \textcolor{goodred}{PhysX}   & \textcolor{goodgreen}{Sim \& Real} & \textcolor{goodred}{HoloLens 2}   & \textcolor{goodgray}{N/A} & \textcolor{goodred}{Cartesian} \\
        % GELLO \cite{wu2023gello}                           & \cmark & \xmark & \xmark & \xmark & \xmark & \xmark \\
        % DexCap \cite{wang2024dexcap}                       & \xmark & \xmark & \xmark & \xmark & \xmark & \xmark \\
        % AnyTeleop \cite{Qin2023AnyTeleopAG}                & \cmark & \cmark & \cmark & \cmark & \xmark & \cmark \\
        % \citet{wang2024robotic}                            & \xmark & \xmark & \xmark & \xmark & \xmark & \xmark \\
        Bunny-VisionPro \cite{bunnyvisionpro}              & \textcolor{goodgray}{N/A}        & \textcolor{goodred}{Single Robot} & \textcolor{goodgray}{N/A}     & \textcolor{goodred}{Real}          & \textcolor{goodred}{Vision Pro}   & \textcolor{goodgray}{N/A} & \textcolor{goodred}{Cartesian} \\
        IMMERTWIN \cite{immertwin}                         & \textcolor{goodgray}{N/A}        & \textcolor{goodred}{Limited}      & \textcolor{goodgray}{N/A}     & \textcolor{goodred}{Real}          & \textcolor{goodred}{HTC Vive}     & \textcolor{goodgray}{N/A} & \textcolor{goodred}{Cartesian} \\
        Open-TeleVision \cite{opentelevision}              & \textcolor{goodgray}{N/A}        & \textcolor{goodred}{Limited}      & \textcolor{goodgray}{N/A}     & \textcolor{goodred}{Real}          & \textcolor{goodgreen}{Meta Quest, Vision Pro} & \textcolor{goodgray}{N/A} & \textcolor{goodred}{Cartesian} \\
        \citet{szczurek2023multimodal}                     & \textcolor{goodgray}{N/A}        & \textcolor{goodred}{Limited}      & \textcolor{goodgray}{N/A}     & \textcolor{goodred}{Real}          & \textcolor{goodred}{HoloLens 2}   & \textcolor{goodgreen}{Available} & \textcolor{goodred}{Joint \& Cartesian} \\
        OPEN TEACH \cite{openteach}                        & \textcolor{goodgray}{N/A}        & \textcolor{goodgreen}{Available}  & \textcolor{goodgray}{N/A}     & \textcolor{goodred}{Real}          & \textcolor{goodred}{Meta Quest 3} & \textcolor{goodred}{N/A} & \textcolor{goodgreen}{Joint \& Cartesian} \\
        \midrule
        \textbf{Ours}                                      & \textcolor{goodgreen}{Available} & \textcolor{goodgreen}{Available}  & \textcolor{goodgreen}{Mujoco, CoppeliaSim, IsaacSim} & \textcolor{goodgreen}{Sim \& Real} & \textcolor{goodgreen}{Meta Quest 3, HoloLens 2} & \textcolor{goodgreen}{Available} & \textcolor{goodgreen}{Joint \& Cartesian} \\
        \bottomrule
        \end{tabular}
    \end{adjustbox}
    \caption{Comparison of XR-based system for robots. IRIS is compared with related works in different dimensions.}
\end{table*}


\section{Related Work}

In this section, we review research related to the importance and barriers to parental involvement; parental use of learning technologies; and the use of generative AI and robot in educational and parenting scenarios.

\subsection{Importance and Barriers to Parental Involvement}\label{sec-rw-2.1}

% 79 words
Early childhood is a critical period for predicting future success and well-being, with early education investments resulting in higher returns than later interventions \cite{duncan2007school, doyle2009investing}. Effective parental involvement fosters cognitive and social skills, especially in younger children \cite{blevins2016early, peck1992parent}. Parents are encouraged to prioritize home-based involvement to maximize their influence \cite{ma2016meta}, as their involvement has a greater impact on children's learning outcomes \cite{hoffner2002parents, fehrmann1987home, hill2004parent} within the family setting than partnerships with schools or communities \cite{ma2016meta, harris2008parents, fantuzzo2004multiple, sui1996effects}.

However, parents' involvement in their children's education is often constrained by practical challenges related to parents' \textit{skills}, \textit{time}, and \textit{energy}. The Hoover-Dempsey and Sandler (HDS) framework \cite{green2007parents} and the CAM framework \cite{ho2024s} both highlight these factors-- parents' perceived \textit{skills and knowledge} (capability), \textit{time} (availability), and \textit{motivation} (energy)--influence the extent of their engagement. For instance, a parent confident in math may choose to engage more in math-related tasks, while those facing inflexible schedules may participate less \cite{green2007parents}. Unlike teachers, parents often lack formal pedagogical training and may underestimate their role in supporting children's learning, particularly as young children struggle to articulate their needs \cite{hara1998parent}. The CAM framework similarly suggests parents delegate tasks to a robot when they feel less capable, have limited time, or are unmotivated. These factors reflect parents' life contexts, shaped by demographic backgrounds, occupations, and parenting responsibilities \cite{grolnick1997predictors}, highlighting the need to help parents overcome barriers to effective involvement in early education within their life contexts.

\subsection{Parental Use of Learning Technologies}\label{sec-rw-2.2}
% 207 words
Technology encourages parental involvement by facilitating parent-child engagement in learning activities while introducing risks that require active parental mediation \cite{gonzalez2022parental}. On the positive side, technology offers novel opportunities for parental engagement and enhances children's learning outcomes. For example, e-books promote interactive behaviors between parents and children better than print books \cite{korat2010new}. In addition, having access to computers at home significantly boost academic achievement of young children when parents actively mediate their use \cite{hofferth2010home, espinosa2006technology}. However, the effectiveness of these tools often depends on parents' familiarity with and attitude toward technology. Mobile applications, for instance, can improve learning outcomes but require parents to possess sufficient technology efficacy to guide their use \cite{papadakis2019parental}.

On the negative side, technology introduces risks such as excessive screen time, exposure to inappropriate content, and misinformation, which necessitate parental intervention \cite{oswald2020psychological, howard2021digital}. According to parental mediation theory, parents mitigate these risks through restrictive mediation (e.g., setting limits), active mediation (e.g., discussing content), and co-use (e.g., shared use of technology) \cite{valkenburg1999developing}. Modern technologies like video games, location-based games (\textit{e.g.,} Pokemon Go), and conversational agents (\textit{e.g.,} Alexa) also require parents to adapt their mediation strategies to ensure responsible use \cite{valkenburg1999developing, nikken2006parental, sobel2017wasn, beneteau2020parenting, yu2024parent}. Overall, parents seek to leverage technology to support their children's learning due to ite effectivenss but are also mindful of its risks. Their involvement is therefore driven by both opportunities and concerns, highlighting the need to design tools that effectively involve parents to balance benefits and risks.

\subsection{Generative AI and Companion Robots for Parenting and Education}
Generative AI and companion robots offer human-like affordances, with AI simulating human intelligence and robots providing physical human-like features. Compared to conventional models (\textit{e.g.,} machine learning) and devices (\textit{e.g.,} laptops), these emerging technologies enable natural and social interactions, creating opportunities for novel paradigms to enhance parental involvement and children's learning while introducing their unique challenges.

\subsubsection{Generative AI}
GAI offers promising support for parents by enhancing their ability to educate and engage with their children. Prior work suggested that AI-driven systems can support parenting education \cite{petsolari2024socio} and provide evidence-based advice through applications and chatbots, delivering micro-interventions such as teaching parents how to praise their children effectively \cite{davis2017parent, entenberg2023user} or offering strategies to teach complex concepts \cite{mogavi2024chatgpt, su2023unlocking}. Many parents also prefer using GAI to create educational materials tailored to their children's needs, rather than granting children direct access to these tools \cite{han2023design}. Beyond educational support, AI-based storytelling tools address practical challenges (\textit{e.g.,} time constraints) by alleviating physical labor while fostering parent-child interactions \cite{sun2024exploring}. Furthermore, GAI offers advantages to children's learning directly. It can help create personalized learning experiences by providing timely feedback and tailoring content \cite{su2023unlocking, mogavi2024chatgpt, han2024teachers}, enhancing positive learning experiences \cite{jauhiainen2023generative}. For example, a LLM-driven conversational system can teach children mathematical concepts through co-creative storytelling, achieving learning outcomes similar to human-led instructions \cite{zhang2024mathemyths}.

Despite these benefits, several concerns persist regarding the use of GAI in education. Prior work highlighted the limitations of GAI, such as its limited effectiveness in more complex learning tasks,the limited quality of the training data, and its inability to offer comprehensive educational support \cite{su2023unlocking}. There is also a significant risk of GAI producing inaccurate or biased information, discouraging independent thought among children, and threatening user privacy \cite{su2023unlocking, han2023design, han2024teachers}. Many parents are skeptical about the role of AI in their children's academic processes, concerned about the accuracy of AI-generated content, and worry that over-reliance on AI could stifle independent thinking \cite{han2023design}.

%\todo{might need to add some structural transition here}
\subsubsection{Social companion robots}
Social companion robots have proven potential to assist parents in home education settings through studies in \textit{parent-child-robot} interactions. \citet{gvirsman2020patricc} showed that the robotic system, \texttt{Patricc}, fostered more triadic interaction between parents and toddlers than a tablet, and \citet{gvirsman2024effect} found that, in a parent-toddler-robot interaction, parents tend to decrease their scaffolding affectively when the robot increases its scaffolding behavior. Similarly, \citet{chen2022designing} found that social robots enhanced parent-child co-reading activities, while \citet{chan2017wakey} demonstrated that the WAKEY robot improved morning routines and reduced parental frustration. Beyond educational support, \citet{ho2024s} uncovered that parents envisioned robots as their \textit{collaborators} to support their children's learning at home and that their collaboration patterns can be determined by the parents' capability, availability, and motivation. Although parents generally have positive attitudes toward incorporating robots into their children's learning, they remain concerned about the risk of disrupting school-based learning and potential teacher replacement \cite{tolksdorf2020parents, lee2008elementary, louie2021desire}.

In addition to parental support, social companion robots also support children in education directly through \textit{child-robot interactions}. Physically embodied robots provide adaptive assistance and verbal interaction similar to virtual or conversational agents \cite{ramachandran2019personalized, leyzberg2014personalizing, schodde2017adaptive, brown2014positive}, yet they foster greater engagement with the physical environment and encourage more advanced social behaviors during learning \cite{belpaeme2018social}, leading to improved learning outcomes \cite{leyzberg2012physical}. Prior work demonstrated that companion robots can effectively support both school-based learning (\textit{e.g.,} math \cite{lopez2018robotic}, literacy \cite{kennedy2016social, gordon2016affective}, and science \cite{davison2020working}) and home-based learning activities (\textit{e.g.,} reading \cite{michaelis2018reading, michaelis2019supporting}, number board games \cite{ho2021robomath}, and math-oriented conversations with parents \cite{ho2023designing}). For example, \citet{kennedy2016social} suggested that children can learn elements of a second language from a robot in short-term interactions, and \citet{tanaka2009use} found that children who took on the role of teaching the robot gained confidence and improved learning outcomes.

%\todo{may need to make this a separate section and explain why we propose AI-assisted robots}

% \subsubsection{Research Gap}
Parental involvement in early education is crucial and AI-assisted robots can offer promising support by helping parents overcome practical barriers (\textit{i.e.,} time, energy, and skills) and addressing concerns about technological risks. Yet, limited research has examined how technology design can simultaneously alleviate these barriers and concerns. Though \citet{zhang2022storybuddy} emphasized the importance of flexible parental involvement during reading through a system called \textit{Storybuddy}, yet they focused on a virtual chatbot rather than a physical robot, and how the flexible modes may be used in different scenarios remain unknown. Similarly, \textit{ContextQ} \cite{dietz2024contextq} presented auto-generated dialogic questions to caregivers for dialogic reading, but primarily considered situations where parents are actively involved, not scenarios where parents cannot participate fully.

In this work, we address these gaps by exploring parental involvement contexts, understanding parents' perceptions of AI-generated content, and examining how parents collaborate with AI and robots under different scenarios. In the following sections, we describe our development of  \texttt{SET}, a card-based activity, to understand parental involvement contexts (Section~\ref{sec-card}), the design of the \texttt{PAiREd} system to enable parents to co-create learning activities with an LLM (Section~\ref{sec-system}), and user study aimed to discover use patterns and understand user perceptions of the system (Section~\ref{sec-study}).
\section{Preliminaries}
\label{sec:method}

\subsection{Sparsely-gated MoE}
\label{subsec:sparse_moe}

In AVSR systems, the multimodal encoder processes a sequence of audio $\rva = [a_1, a_2, \cdots]$ and video $\rvv = [v_1, v_2, \cdots]$ data into combined audio-visual embeddings $\text{Enc}(\rva, \rvv)$. These embeddings are utilized by the decoder to predict subsequent text tokens, where the predicted token is given by $\textsl{text}_{t+1} = \text{Dec}(\text{Enc}(\rva, \rvv), \textsl{text}_t)$. Within the Transformer layer, $x_t$ is the intermediate representation of the token $\textsl{text}_t$, derived by cross-attending to the combined audio-visual embeddings from $\rva$ and $\rvv$ (see Figure\,\ref{fig:overview}).

The integration of a sparsely-gated MoE framework \citep{shazeer2017outrageously, lepikhin2021gshard} leverages $E$ experts to scale model capacity. Each token representation is routed to a selected subset of these experts through a learned gating mechanism. 
Specifically, the routing function $h(x) = W_r \cdot x$ assigns weights for each token, and the weight for expert $i$ is computed using a softmax function:
\begin{equation}
\label{eq:router_weight}
    p_i(x) = \frac{\exp(h_i(x))}{\sum_{j=1}^{E} \exp(h_j(x))},
\end{equation}
and the output $y$ is the aggregated result of computations from the top-$k$ selected experts:
\begin{equation}
y = \sum_{i \in \text{top}k(E)} \tilde{p}_i(x) E_i(x),
\end{equation}
where $\tilde{p}$ is the normalization of top-$k$ probabilities.
Note that each expert follows the same structure as a feed-forward network\,(FFN) in a Transformer block. Figure\,\ref{fig:overview} presents the overall MoE architecture and its token routing.



%%%%%%%%%%%%%%%%%%%%
\begin{figure}[!t]
    \centering
    \vspace*{-5pt}
    \includegraphics[width=\linewidth]{tex_figure/figure2.pdf}
    \vspace*{-20pt}
    \caption{Overview of sparsely-gated MoE for AVSR. A select subset of experts are activated for each token representation ($x_t$).
    }
    \label{fig:overview}
    \vspace{-5pt}
\end{figure}
%%%%%%%%%%%%%%%%%%%%


\vspace*{-8pt}
\paragraph{Load balancing.}
To mitigate the load imbalance issue commonly observed in the top-$k$ expert selection strategy, a load balancing loss has been implemented to encourage the balanced token load across all experts. Specifically, we use a differentiable load balancing loss~\citep{fedus2022switch}:
\begin{equation}
    L_B = E \cdot \sum_{i=1}^E f_i \cdot P_i,
\end{equation}
where $f_i$ denotes the frequency of expert $i$ being selected as top-1, averaged over all tokens within a batch $\mathcal{B}$,
\begin{equation}
\label{eq:expert_frequency}
    f_i = \frac{1}{T} \sum_{x\in\mathcal{B}} \mathbbm{1} \{\arg\max p(x) = i\}
\end{equation}
and $P_i$ is the average assigned probability for expert $i$,
\begin{equation}
\label{eq:expert_probability}
    P_i = \frac{1}{T} \sum_{x\in\mathcal{B}} p_i(x)
\end{equation}
with $T$ representing the total number of tokens.

An additional router z-loss~\citep{zoph2022st} is employed to stabilize the routing mechanism:
\begin{equation}
    L_Z = \frac{1}{T}\sum_{x\in\mathcal{B}} \Bigg(\log \sum_{i=1}^E \exp(h_i(x)) \Bigg)^2.
\end{equation}
This sparse MoE structure ensures that token processing is efficiently managed across multiple experts, utilizing lower compute relative to its expansive capacity.


%%%%%%%%%%%%%%%%%%%%%
\begin{figure*}[!t]
    \centering
    \includegraphics[width=0.95\linewidth]{tex_figure/figure3.pdf}
    \vspace{-5pt}
    \caption{MoE-based routing strategies for AVSR. 
    (a) A conventional MoE approach where a learned router selects the top-2 experts for each token, enforcing the balanced expert load. 
    (b) Experts are explicitly divided into audio and visual groups, with manual activation based on the input modality.
    (c) \ourmodel introduces an inter-modal router that can dynamically assign weights to modality-specific expert groups, followed by intra-modal routers that select the top-1 expert within each group. The inter-modal router is trained by the load biasing loss that guides the expert group specialization.
    }
    \label{fig:routing}
\end{figure*}
%%%%%%%%%%%%%%%%%%%%%
\subsection{Expert Group Specialization}
\label{subsec:hard_routing}


To enhance expert management within the AVSR system, a \textit{hard routing} technique can be used for expert group specialization. This approach is inspired by several practices in visual-language MoE models \citep{zhu2022uni, li2023pace, shen2023scaling, lee2025moai} where the role of experts is strictly defined by the input modality, eliminating the need for a trained router.

\vspace*{-8pt}
\paragraph{Hard routing.}
Our hard routing enforces modality-specific activation of expert groups: audio data activate only audio experts, and video data activate only visual experts.
This segregation encourages the independent development of specialized expert groups. As suggested in V/T-MoE \citep{shen2023scaling}, once the group is activated, we use an intra-modal router for the modality-specific experts.

Figure\,\ref{fig:routing}(b) visualizes the hard routing mechanism with audio and visual expert groups.
During training, audio or video sequence is randomly dropped, leading to subsets $\mathcal{A}$ and $\mathcal{V}$ within a batch $\mathcal{B}$, consisting of audio-only or video-only sequences, respectively. A token representation $x_t \in \mathcal{A}$ indicates that the cross-attention module processes the input $\textsl{text}_t$ with $\text{Enc}(\rva, \vzero)$---where the visual component is zeroed out---and vice versa for $x_t \in \mathcal{V}$.
For these, we utilize two distinct intra-modal routing networks, $W_r^A$ and $W_r^V$:
\begin{align}
\begin{split}
    h^A(x) &= W_r^A \cdot x \quad \text{for } x \in \mathcal{A}, \\
    h^V(x) &= W_r^V \cdot x \quad \text{for } x \in \mathcal{V}.
\end{split}
\end{align}
These routers calculate the weights $p^{\{A,V\}}(x)$ as in Eq.\,(\ref{eq:router_weight}) within their respective expert group, either $\{E^A\}$ for audio or $\{E^V\}$ for visual. The output for each token is then
\begin{equation}
    y = \begin{cases}
        \sum_{\text{top}k(E^A)} \tilde{p}_i^A(x) E^A_i(x) & \text{if } x \in \mathcal{A}, \\
        \sum_{\text{top}k(E^V)} \tilde{p}_i^V(x) E^V_i(x) & \text{if } x \in \mathcal{V}. \\ 
    \end{cases} 
\end{equation}
For audio-visual sequences, outputs from both groups are averaged, with the top-($k/2$) experts from each group contributing to ensure balanced processing.

\section{MoHAVE: Mixture of Hierarchical Audio-Visual Experts}
\label{sec:mohave}

Despite the benefits of hard routing in specializing expert groups according to decoupled input modality, it lacks the flexibility to autonomously determine the group utilization.
In practice, the optimal balance between audio and visual groups varies depending on ambient conditions such as noise type and intensity (more detailed in Figure\,\ref{fig:expert_load}(b)).
To address this limitation and enhance the model’s adaptability, we introduce an adaptive routing mechanism with hierarchical gating \cite{jordan1994hierarchical}, providing a more dynamic approach to manage multimodal inputs.

Our hierarchical model, \ourmodel, features a two-layer routing mechanism: \textit{inter-modal} and \textit{intra-modal} routers, where the inter-modal router learns to assign appropriate weights to each modality-specific group. Figure\,\ref{fig:routing}(c) presents the overview of \ourmodel's routing strategy.

\subsection{Hierarchical Gating Structure}
\label{subsec:hierarchical_gating}

The inter-modal router orchestrates the initial token distributions across expert groups. It generates logits through $u(x) = V_r \cdot x$ and determines the dispatch weights for group $i$ with $q_i(x) = \text{softmax}(u(x))_i$. This router dynamically selects the top-$m$ expert groups, and within those, the intra-modal routers select the top-$k$ experts, thus involving $m \times k$ experts in total. For practical efficiency, we set $k=1$ for each group and modify the intra-modal router's probability distribution to a Kronecker delta, $\tilde{p}_{ij} \rightarrow \delta_{j,\text{argmax}(p_i)}$.
The output from this layer integrates these selections:
\begin{align}
  y &= \!\!\!\!\sum_{i \in \text{top}m(G)} \!\!\tilde{q}_i(x)\!\! \sum_{j \in \text{top}k(E_i)} \!\!\tilde{p}_{ij} E_{ij}(x) \\
    &\rightarrow \!\!\!\!\sum_{i \in \text{top}m(G)} \!\!\tilde{q}_i(x)E_{ij}(x), ~~\text{where } j\!=\!\arg\max(p_i) 
\end{align}
where $\tilde{q}$ is the normalization of $q$ across top-$m$ probabilities, $G$ is the number of expert groups, and $E_{ij}(x)$ denotes the output from the $j$-th expert in the $i$-th group.

Focusing on audio-visual applications, we designate two expert groups: audio and visual. Each token $x$, regardless of its modality, is processed by the intra-modal routing networks of both groups, \ie $[h^A(x), h^V(x)] = [W_r^A, W_r^V] \cdot x$. The frequencies $f^{\{A,V\}}$ and probabilities $P^{\{A,V\}}$ for selecting experts are computed in the same manner as Eq.\,(\ref{eq:expert_frequency})--(\ref{eq:expert_probability}) for all $x \in \mathcal{B}$. Thus, the load balancing loss can be computed for both groups:
\vspace*{-5pt}
\begin{equation}
L_B = E^A \cdot \sum_{j=1}^{E^A} f_j^A \cdot P_j^A + E^V \cdot \sum_{j=1}^{E^V} f_j^V \cdot P_j^V
\end{equation}
where $f_j^A$ and $f_j^V$ denote the frequencies of token assignments to audio and visual experts, respectively.


\subsection{Group-level Load Biasing Loss}
\label{subsec:load_biasing_loss}

To autonomously manage the expert group loads without manual (de-)activation as hard routing, we introduce a biasing loss that directs the load towards a certain group. This load biasing loss encourages the inter-modal router to assign higher weights to $E^A$ experts for audio sequences and to $E^V$ experts for video sequences.
For audio sequences within a sub-batch $\mathcal{A}$, the frequency and average probability of selecting the $i$-th group is calculated as follows:
\begin{equation}
    g^A_i = \frac{1}{|\mathcal{A}|} \sum_{x\in\mathcal{A}} \mathbbm{1} \{\arg\max q(x) = i\},
\end{equation}
\begin{equation}
    Q^A_i = \frac{1}{|\mathcal{A}|} \sum_{x\in\mathcal{A}} q_i(x).
\end{equation}
Similar calculations for $g^V_i$ and $Q^V_i$ are made for video sequences $x\in\mathcal{V}$. We designate the first group as audio experts and the second group as video experts, then the load biasing loss is defined as:
\begin{equation}
    L_S = L_S^A + L_S^V = (1 - g^A_1 \cdot Q^A_1) + (1 - g^V_2 \cdot Q^V_2).
\end{equation}
Note that $L_S^A$ and $L_S^V$ are only computed over $x\in\mathcal{A}$ and $x\in\mathcal{V}$, respectively. 

For sequences containing both audio and video, we exclude them from the load biasing loss calculation but incorporate them into the load balancing.
Although these tokens are uniformly dispatched on average, the inter-modal router finds the optimal strategy for each token based on its characteristics.
Empirically, we find that \ourmodel learns to assign greater weight to the visual expert group for audio-visual inputs under high auditory noise, and to the audio expert group for less noisy inputs (see \S\ref{subsec:analysis_load} for details), demonstrating the model’s adaptability under various noisy conditions.

The overall loss function, combining the cross-entropy\,(CE) for token prediction, is formulated as:
\begin{equation}
L_{tot} = L_{CE} + c_B L_B + c_S L_S + c_Z L_Z.
\end{equation}
Here, $c_B$ and $c_Z$ are set to 1e-2 and 1e-3, respectively, in line with \citep{fedus2022switch, zoph2022st}, and $c_S$ is also set at 1e-2.

\section{Experiments and Results}
\label{sec:experiments}

\subsection{Implementation Details}
\label{sec:implementation}

\begin{table*}[t]
\resizebox{\textwidth}{!}{
\begin{tabular}{cccccccccccc}
\hline
\multirow{2}{*}{$\alpha$} & \multirow{2}{*}{Method} & \multicolumn{5}{c}{$\beta = 0.2$}                                                                                                         & \multicolumn{5}{c}{$\beta = 0.4$}                                                                                     \\ \cline{3-12} 
                          &                         & Precision                 & Recall                    & F1                        & Accuracy                  & AUC                       & Precision             & Recall                & F1                    & Accuracy              & AUC                   \\ \hline
\multirow{7}{*}{0.2}      & FedAvg                  & 53.94 $\pm$ 3.56          & 54.05 $\pm$ 3.53          & 53.21 $\pm$ 3.90          & 53.41 $\pm$ 3.71          & 69.39 $\pm$ 2.05          & 53.58 $\pm$ 2.34      & 53.80 $\pm$ 2.77      & 52.30 $\pm$ 2.32      & 52.64 $\pm$ 2.14      & 68.20 $\pm$ 2.72      \\
                          & FedProx                 & 55.81 $\pm$ 1.58          & 54.37 $\pm$ 5.19          & 54.32 $\pm$ 2.21          & 54.62 $\pm$ 2.18          & 68.29 $\pm$ 2.93          & 54.39 $\pm$ 2.69      & 54.36 $\pm$ 1.82      & 53.27 $\pm$ 2.99      & 53.52 $\pm$ 2.97      & 68.11 $\pm$ 2.74      \\
                          & FedMed-GAN              & 54.62 $\pm$ 1.02          & 55.18 $\pm$ 2.73          & 53.46 $\pm$ 0.97          & 53.52 $\pm$ 1.07          & 68.80 $\pm$ 2.16          & 53.87 $\pm$ 3.66      & 54.66 $\pm$ 3.57      & 53.35 $\pm$ 3.61      & 53.41 $\pm$ 3.63      & 68.09 $\pm$ 2.53      \\
                          & FedMI                   & 54.77 $\pm$ 2.94          & 55.22 $\pm$ 4.36          & 54.26 $\pm$ 2.82          & 54.40 $\pm$ 2.83          & 69.15 $\pm$ 2.73          & 54.03 $\pm$ 3.88      & 54.46 $\pm$ 3.56      & 53.54 $\pm$ 4.22      & 53.63 $\pm$ 4.14      & 67.61 $\pm$ 3.93      \\
                          & MFCPL                   & 54.54 $\pm$ 1.88          & \textbf{56.06 $\pm$ 4.14} & 53.88 $\pm$ 1.64          & 54.07 $\pm$ 1.93          & 69.39 $\pm$ 2.71          & 53.86 $\pm$ 5.42      & 54.01 $\pm$ 5.44      & 53.22 $\pm$ 5.16      & 53.30 $\pm$ 5.34      & 67.58 $\pm$ 3.67      \\
                          & PmcmFL                  & 53.17 $\pm$ 1.60          & 52.73 $\pm$ 3.50          & 52.58 $\pm$ 1.81          & 52.64 $\pm$ 2.03          & 67.51 $\pm$ 4.07          & 49.03 $\pm$ 1.55      & 48.01 $\pm$ 3.03      & 48.38 $\pm$ 2.00      & 48.46 $\pm$ 2.31      & 63.71 $\pm$ 2.50      \\
                          & ClusMFL                    & \textbf{57.16 $\pm$ 2.32}     & 54.73 $\pm$ 3.93              & \textbf{56.56 $\pm$ 2.36}     & \textbf{56.92 $\pm$ 2.41}     & \textbf{72.81 $\pm$ 3.64}     & \textbf{56.06 $\pm$ 1.31} & \textbf{55.44 $\pm$ 2.19} & \textbf{55.38 $\pm$ 1.07} & \textbf{55.49 $\pm$ 1.10} & \textbf{72.50 $\pm$ 2.02} \\ \hline
\multirow{7}{*}{0.4}      & FedAvg                  & 53.75 $\pm$ 3.86          & 54.26 $\pm$ 3.76          & 52.76 $\pm$ 4.64          & 53.08 $\pm$ 4.34          & 67.91 $\pm$ 3.48          & 52.60 $\pm$ 3.00      & 53.95 $\pm$ 3.66      & 51.84 $\pm$ 3.06      & 51.98 $\pm$ 3.07      & 67.03 $\pm$ 3.22      \\
                          & FedProx                 & 54.36 $\pm$ 3.50          & 54.28 $\pm$ 4.17          & 53.77 $\pm$ 3.48          & 53.96 $\pm$ 3.51          & 68.26 $\pm$ 3.53          & 52.16 $\pm$ 2.20      & 52.93 $\pm$ 3.76      & 51.35 $\pm$ 2.13      & 51.54 $\pm$ 2.07      & 67.43 $\pm$ 3.64      \\
                          & FedMed-GAN              & 52.97 $\pm$ 1.63          & 52.89 $\pm$ 2.41          & 52.43 $\pm$ 1.74          & 52.53 $\pm$ 1.63          & 67.25 $\pm$ 3.45          & 53.56 $\pm$ 3.57      & 54.02 $\pm$ 3.95      & 52.54 $\pm$ 3.97      & 52.86 $\pm$ 3.63      & 67.91 $\pm$ 2.93      \\
                          & FedMI                   & 55.40 $\pm$ 2.85          & 53.82 $\pm$ 2.45          & 54.84 $\pm$ 2.75          & 54.95 $\pm$ 2.69          & 68.59 $\pm$ 2.48          & 54.01 $\pm$ 2.07      & 52.36 $\pm$ 1.98      & 53.62 $\pm$ 2.14      & 53.85 $\pm$ 2.06      & 66.70 $\pm$ 2.63      \\
                          & MFCPL                   & 55.07 $\pm$ 4.34          & 54.66 $\pm$ 3.93          & 54.34 $\pm$ 4.37          & 54.51 $\pm$ 4.57          & 66.60 $\pm$ 2.58          & 52.84 $\pm$ 4.14      & 54.69 $\pm$ 3.92      & 51.70 $\pm$ 3.89      & 51.87 $\pm$ 3.74      & 67.81 $\pm$ 3.48      \\
                          & PmcmFL                  & 51.71 $\pm$ 5.54          & 49.80 $\pm$ 6.21          & 50.77 $\pm$ 5.21          & 50.88 $\pm$ 5.42          & 65.54 $\pm$ 6.22          & 52.45 $\pm$ 3.96      & 50.03 $\pm$ 5.74      & 50.40 $\pm$ 4.72      & 50.77 $\pm$ 4.66      & 63.31 $\pm$ 5.17      \\
                          & ClusMFL                    & \textbf{56.59 $\pm$ 4.53} & \textbf{54.68 $\pm$ 4.49} & \textbf{56.17 $\pm$ 4.08} & \textbf{56.37 $\pm$ 4.10} & \textbf{73.25 $\pm$ 4.11} & \textbf{54.83 $\pm$ 6.13} & \textbf{54.88 $\pm$ 6.09} & \textbf{54.22 $\pm$ 5.89} & \textbf{54.40 $\pm$ 6.03} & \textbf{72.22 $\pm$ 4.47} \\ \hline
\end{tabular}
}
\caption{Performance Comparison Across Different Federated Learning Methods (Mean $\pm$ Standard Deviation \%) under Different Settings.}
\label{main result}
\end{table*}


\paragraph{Datasets.}
For the robust AVSR benchmark, we utilize the LRS3 dataset~\citep{afouras2018lrs3}, which consists of 433 hours of audio-visual speech from 5,000+ speakers. Following the experimental setup of \citet{shi2022robust}, we extract audio noise samples from the MUSAN~\citep{snyder2015musan} dataset, targeting different noise types such as \textit{babble}, \textit{music}, and \textit{natural} noises, along with \textit{speech} noise from LRS3. These noises are randomly augmented into the audio data, corrupting 25\% of the training set with a signal-to-noise ratio (SNR) sampled from $\mathcal{N}(0, 5)$. We measure performance using the word error rate (WER), primarily under noisy conditions with SNRs of \{$-$10, $-$5, 0, 5, 10\}\,dB, specifically N-WER\,\citep{kim2024learning} which highlights the significance of visual cues in noise-corrupted environments.


For multilingual evaluations, the MuAViC dataset \cite{anwar2023muavic} is used, featuring 1,200 hours of audio-visual content from 8,000+ speakers across 9 languages, sourced from LRS3-TED\,\cite{afouras2018lrs3} and mTEDx\,\cite{elizabeth2021multilingual}. We use 8 languages (excluding English) for multilingual AVSR and 6 languages for X-to-English audio-visual speech-to-text translation (AVS2TT) tasks. We assess the models using WER for transcription and the BLEU score\,\cite{papineni2002bleu} for translation.

\vspace*{-8pt}
\paragraph{\ourmodel model description.}

Our \ourmodel framework is developed in two configurations: \textsc{Base} and \textsc{Large}. The \textsc{Base} model consists of 12 Transformer~\citep{vaswani2017attention} encoder layers and 6 decoder layers, while the \textsc{Large} model incorporates 24 encoder layers and 9 decoder layers. Both models’ audio-visual encoders are derived from the AV-HuBERT-\textsc{Base}/-\textsc{Large} models, pretrained on a noise-augmented corpus of LRS3 \citep{afouras2018lrs3} + VoxCeleb2 \citep{chung2018voxceleb2}. Our MoE implementation activates top-2 out of 8 experts in every FFN layer within the decoder~\citep{jiang2024mixtral}, while the hierarchical architecture engages the top-1 expert from each audio and visual group. To facilitate the expert group specialization, load biasing is used with audio or video randomly dropped in 25\% probability.

\begin{table}[t]
    \centering
    \caption{Comparison with other methods (FT). All the compared results except PAD are borrowed from TINN. Training epochs of SODA and MFNet-T are 200 and 300.}
    \label{tab:sota_comparison}
    \centering
    \scriptsize
    \setlength{\tabcolsep}{0.8mm}{
    \scalebox{1.0}{
    \begin{tabular}{l c c c c c c}
        \toprule
        Methods & Params(M) & SODA & MFNet-T \\
        \midrule
        DeepLab V3+ \citep{deeplabv3+} & 62.7  & 68.73 & 49.80  \\
        PSPNet \citep{pspnet} & 68.1 & 68.68 & 45.24 \\
        UPerNet \citep{upernet} & 72.3 & 67.45 & 48.56 \\
        SegFormer \citep{segformer} & 84.7 & 67.86 & 50.68 \\
        ViT-Adapter \citep{vitadapter} & 99.8 & 68.12 & 50.62 \\
        Mask2Former \citep{mask2former} & 216.0 & 67.58 & 51.30 \\
        MaskDINO \citep{maskdino} & 223.0 & 66.32 & 51.03 \\
        \midrule
        EC-CNN \citep{soda} & 54.5 & 65.87 & 47.56 \\
        MCNet \citep{mcnet} & 35.7 & 63.89 & 43.15 \\
        PAD (MAE-B) \citep{pad} & 164.9 & 69.71 & 50.14 \\
        TINN \citep{tinn} & 85.3 & 69.45 & 51.93  \\
        \midrule
        \rowcolor{cyan!15} UNIP-T (MAE-L) & 11.0 & 67.29 & 50.39 \\
        \rowcolor{cyan!15} UNIP-S (MAE-L) & 41.9 & \underline{71.35} & \underline{53.76} \\
        \rowcolor{cyan!15} UNIP-B (MAE-L) & 163.7 & \textbf{72.19} & \textbf{54.35} \\
        \bottomrule
    \end{tabular}}}
    \vspace{-4mm}
\end{table}

As summarized in Table\,\ref{tab:main_result}, the \textsc{Base} model of \ourmodel holds 359M parameters, and the \textsc{Large} configuration contains 1B. Despite its larger model capacity, due to the sparse activation of these parameters, only about half are active during token processing, amounting to 189M for \textsc{Base} and 553M for \textsc{Large} model. This setup ensures computational efficiency which is comparable to the smaller AV-HuBERT counterparts. For more details on the model description and computation cost, refer to Appendix\,\ref{appx:model_details} and \ref{appx:computation_cost}.

\subsection{Robust AVSR Benchmark Results}
\label{sec:results}

%%%%%%%%%%%%%%%%%%%%%%%%%%%%%%%%%%%%
\begin{figure*}[!t]
    \centering
    \includegraphics[width=\linewidth]{tex_figure/figure4_new.pdf}
    \vspace{-22pt}
    \caption{(a) Expert load distribution in \ourmodel according to input modalities, with expert selection frequencies weighted by the inter-modal router’s output probability. (b) Performance of the hard routing strategy under different weight assignments to audio expert group. The visual expert group is weighted by $p^V = 1-p^A$.
    }
    \label{fig:expert_load}
\end{figure*}

\begin{figure*}[!t]
    \centering
    \vspace{-3pt}
    \includegraphics[width=\linewidth]{tex_figure/figure5_new.pdf}
    \vspace{-22pt}
    \caption{Expert load distribution in \ourmodel for the audio group (solid bars) and visual group (dashed bars) across noisy audio-visual sequences under babble (left) and natural (right) noise. Full layer-wise results are provided in Appendix\,\ref{appx:expert_group_usage}.
    }
    \label{fig:expert_load_noise}
\end{figure*}
%%%%%%%%%%%%%%%%%%%%%%%%%%%%%%%%%%%%


Table\,\ref{tab:main_result} presents \ourmodel's robust performance on the AVSR benchmark under diverse noisy conditions, demonstrating exceptional robustness across different noise types and SNR levels: \textbf{N-WER of 5.8\% for \textsc{Base}} and \textbf{4.5\% for \textsc{Large}}. This substantiates the model's potential for effectively scaling AVSR systems without incurring significant computational costs.
%
The results also reveal that simple MoE implementations (AV-MoE in Table\,\ref{tab:main_result}), despite their larger capacity, fail to achieve remarkable gains. Instead, the key improvement stems from leveraging expert group specialization, as evidenced by the effectiveness of hard routing. By splitting experts into audio and visual groups, MoE is enabled with more targeted and effective processing of multimodal inputs, leading to substantial performance enhancements.
Without our load biasing loss, \ourmodel loses its group specialization capability, comparable to the performance of simple AV-MoEs.

Building upon this expert group strategy, \ourmodel enhances its adaptability through dynamically determining the usage of each group. This adaptive routing approach allows the model to flexibly adjust to varying audio-visual scenarios, contributing to consistent gains in robustness across the benchmark, as detailed in Table\,\ref{tab:main_result}.
An in-depth analysis of this hierarchical gating approach and its impact on token dispatching is discussed in \S\ref{subsec:analysis_load}, underscoring its critical role in advancing MoHAVE’s capabilities in various AVSR environments.


\vspace*{-12pt}
\paragraph{Comparison with state-of-the-art AVSR methods.}
Table\,\ref{tab:sota_comparison} shows how our \ourmodel decoder, when integrated with a range of audio-visual encoders, consistently improves performance compared to existing state-of-the-art methods. While BRAVEn \citep{haliassos2024braven} typically struggles in noisy multimodal scenarios---due to its original design focused on handling unimodal tasks---\ourmodel boosts its accuracy. Other recent approaches have advanced by utilizing the noise-augmented AVSR encoder\,\citep{shi2022learning}, such as additionally learning temporal dynamics with cross-modal attention modules\,(CMA) \citep{kim2024learning}. When paired with an AV-HuBERT encoder and trained through the CMA's self-supervised learning, \ourmodel achieves a remarkable performance: \textbf{N-WER of 4.2\%}.

\subsection{Expert and Group Load Analysis}
\label{subsec:analysis_load}

\begin{table*}[!t]
    \centering
    \small
    \vspace{-5pt}
    \caption{Multilingual audio-visual speech task performance, with non-English speech recognition (WER) and X-En speech-to-text translation (BLEU score), in a noisy environment with multilingual babble noise (SNR\,$=$\,0). $^\dagger$Results obtained from \citet{han-etal-2024-xlavs}. $^\ddagger$Re-implemented using the pretrained model from \citet{choi2024av2av}.}
    \label{tab:muavic}
    \vspace{5pt}
    % \addtolength{\tabcolsep}{-1pt}
    \resizebox{\textwidth}{!}{
    \begin{tabular}{llccccccccc}
    \toprule
    && \multicolumn{8}{c}{Source} & \\
    \cmidrule{3-10}
    Model & Pretrain data & Ar & De & El & Es & Fr & It & Pt & Ru & Avg \\
    \midrule
    \multicolumn{11}{c}{\textbf{\textit{Speech Recognition, Test WER $\downarrow$}}} \\
    Whisper large-v2\,\cite{radford2023robust} & \textit{680k hrs, 100+ langs} & 197.9 & 244.4 & 113.3 & 116.3 & 172.3 & 172.4 & 223.6 & 126.2 & 170.8 \\
    u-HuBERT$^\dagger$\,\cite{hsu2022u} & \textit{1.7k hrs, English} & 102.3 & 73.2 & 69.7 & 43.7 & 43.2 & 48.5 & 47.6 & 67.0 & 61.9 \\
    mAV-HuBERT\,\cite{anwar2023muavic} & \textit{1.7k hrs, English} & \textbf{82.2} & 66.9 & 62.2 & 40.7 & 39.0 & 44.3 & 43.1 & 43.1 & 52.7 \\
    XLS-R 300M$^\dagger$\,\cite{babu2022xls} & \textit{1.2k hrs, 9 langs} & 97.3 & 69.8 & 74.8 & 47.6 & 37.1 & 47.9 & 54.4 & 59.8 & 61.1 \\
    XLAVS-R 300M\,\cite{han-etal-2024-xlavs} & \textit{8.3k hrs, 100+ langs} & 91.9 & 53.5 & 49.6 & 28.8 & 29.3 & 32.2 & 32.5 & 46.1 & 45.5 \\
    XLAVS-R 2B\,\cite{han-etal-2024-xlavs} & \textit{1.2k hrs, 9 langs} & 93.5 & 58.5 & 38.6 & 23.9 & 23.5 & 24.6 & 26.1 & 41.0 & 41.2 \\
    \cdashlinelr{1-11}
    mAV-HuBERT$^\ddagger$ & \textit{7.0k hrs, 100+ langs} & 88.7 & 51.3 & 37.2 & 20.7 & 22.6 & 24.2 & 23.8 & 42.4 & 38.9 \\
    \ourmodel-\textsc{Large} (\textbf{ours}) & \textit{7.0k hrs, 100+ langs} & 92.9 & \textbf{47.3} & \textbf{35.3} & \textbf{18.7} & \textbf{21.2} & \textbf{21.6} & \textbf{21.9} & \textbf{40.6} & \textbf{37.4} \\
    \midrule
    \multicolumn{11}{c}{\textbf{\textit{X-En Speech-to-Text Translation, Test BLEU $\uparrow$}}} \\
    Whisper large-v2\,\cite{radford2023robust} & \textit{680k hrs, 100+ langs} & - & - & 0.1 & 0.4 & 0.7 & 0.1 & 0.1 & 0.2 & 0.3 \\
    mAV-HuBERT\,\cite{anwar2023muavic} & \textit{1.7k hrs, English} & - & - & 4.2 & 12.8 & 15.0 & 12.5 & 14.8 & 4.6 & 10.7 \\
    XLAVS-R 300M\,\cite{han-etal-2024-xlavs} & \textit{8.3k hrs, 100+ langs} & - & - & 13.2 & 17.4 & 23.8 & 18.7 & 21.8 & 9.4 & 17.4 \\
    XLAVS-R 2B\,\cite{han-etal-2024-xlavs} & \textit{1.2k hrs, 9 langs} & - & - & \textbf{15.7} & 19.2 & 24.6 & 20.1 & 22.3 & \textbf{10.4} & 18.7 \\
    \cdashlinelr{1-11}
    mAV-HuBERT$^\ddagger$ & \textit{7.0k hrs, 100+ langs} & - & - & 8.9 & 21.5 & 26.5 & 21.2 & 24.2 & 8.8 & 18.5 \\
    \ourmodel-\textsc{Large} (\textbf{ours}) & \textit{7.0k hrs, 100+ langs} & - & - & 11.4 & \textbf{22.3} & \textbf{27.1} & \textbf{22.1} & \textbf{25.1} & 9.2 & \textbf{19.5} \\
    \bottomrule
    \end{tabular}
    }
    \vspace{-5pt}
\end{table*}

\paragraph{\ourmodel's expert load distribution.}
Figure\,\ref{fig:expert_load}(a) illustrates the expert load distribution of \ourmodel according to input types: audio-visual, audio-only, and video-only sequences. For audio-visual inputs, all experts from both the audio and visual groups are selected at similar frequencies, with some layer-dependent variations. In contrast, when processing audio-only sequences, the model predominantly activates the audio expert group, while for video-only sequences, the visual expert group is mainly utilized. This distribution validates the effectiveness of our load biasing loss in guiding the inter-modal router to assign appropriate weights based on input modality.

\vspace*{-10pt}
\paragraph{Expert group utilization in noisy AVSR.}
To analyze the effectiveness of hierarchical gating in AVSR, we first examine the limitations of hard routing (\S\ref{subsec:hard_routing}) under noisy conditions. Since hard routing relies on manually (de-)activating the audio and visual groups, for audio-visual inputs, we assign a fixed equal weight ($p^A, p^V = 0.5$) to both groups.
However, this equal weighting may not always be optimal in varying environments, such as noise type or intensity.

As shown in Figure\,\ref{fig:expert_load}(b), increasing reliance on the audio group under babble noise degrades performance, with an optimal weight for the audio group being $0.3$.
Unlike babble noise, which confuses the model with multiple overlapping speech signals, natural noise is more distinct from speech, leading to a higher reliance on the audio group ($p^A \ge 0.5$) preferable.
These results indicate that an ideal routing strategy for audio-visual data should be dynamically adjusted.

Figure\,\ref{fig:expert_load_noise} further illustrates \ourmodel's group load distribution across different noise levels. The model adaptively adjusts its reliance between the audio and visual expert groups---under high noise conditions (low SNRs), it shifts more tokens to the visual group, while in cleaner conditions (high SNRs), the audio group is more actively utilized. This behavior also adjusts to noise types, as observed with babble and natural noise, demonstrating the MoHAVE’s adaptability and robustness.
%

\vspace{-5pt}
\subsection{Multilingual Audio-Visual Speech Tasks}
\vspace{-5pt}

MoEs have demonstrated effectiveness in multilingual speech tasks \citep{hu2023mixture, wang2023language}, as MoE is capable of enabling more diverse routing paths for different language tokens. To evaluate \ourmodel's multilingual capabilities, we train a multilingual model and assess its performance on the MuAViC benchmark~\citep{anwar2023muavic}, evaluating separately for each language.
%
Following \citet{han-etal-2024-xlavs}, we introduce multilingual babble noise at SNR 0\,dB to 50\% of the input samples during training, where the noise clips are sampled from the MuAViC train set. For inference, we apply beam search with a beam size of 5 and normalize text by punctuation removal and lower-casing before calculating WER. For AVS2TT evaluation, we use SacreBLEU\,\cite{post2018call} with its built-in \textit{13a} tokenizer. To simulate noisy test conditions, we inject babble noise sampled from the MuAViC test set.

Table\,\ref{tab:muavic} summarizes the results, where \ourmodel-\textsc{Large} achieves superior performance in both AVSR and AVS2TT. Whisper \cite{radford2023robust}, a leading multilingual ASR model, is known to perform poorly in noisy setup due to its lack of visual understanding for robustness. While multilingual AV-HuBERT \cite{anwar2023muavic} underperforms the state-of-the-art models like XLAVS-R \cite{han-etal-2024-xlavs}, we have re-implemented it using the pretrained model from \citet{choi2024av2av}, which has been pretrained on a significantly larger dataset including 7,000 hours of speech in 100+ languages. This model outperforms (38.9\% average WER) or remains competitive (18.5\% average BLEU) with much larger XLAVS-R 2B. When integrated with this version, \ourmodel further improves performance, achieving \textbf{37.4\% average WER} and \textbf{19.5\% average BLEU}, setting new benchmarks in almost every language being evaluated.



% \section{Conclusion}

The question addressed in this paper is whether it is possible to develop a \emph{general framework} for point-and-click AUIs that does not depend on task-specific heuristics or data to generate policies offline. To this end, we have introduced \marlui, a multi-agent reinforcement learning approach. Our method features a \useragent and an \interfaceagent. The \useragent aims to achieve a task-dependent goal as quickly as possible, while the \interfaceagent learns the underlying task structure by observing the interactions between the \useragent and the UI. Since the \useragent is RL-based and thus learns through trial-and-error interactions with the interface, it does not require real user data. We have evaluated our approach in simulation and with participants, by replacing the \useragent with real users, across five different interfaces and various underlying task structures. The tasks ranged from assigning items to a toolbar, handing out-of-reach objects to the user, selecting the best-performing interface, providing the correct object to the user, and enabling more efficient interaction with a hierarchical menu. 
Results show that our framework enables the development of AUIs with minimal adjustments while being able to assist real users in their task.
We believe that \marlui, and a multi-agent perspective in general, is a promising step towards tools for developing adaptive interfaces, thereby reducing the overhead of developing adaptive strategies on an interface- and task-specific basis.

% \newpage
\section*{Impact Statement}

\ourmodel addresses the scalability and robustness challenges in multimodal speech processing, enabling more efficient and adaptive expert selection. This advancement highlights the potential for further scaling AVSR models in terms of capacity, training, and datasets while extending to more diverse real-world applications without incurring excessive computational costs.

While our research contributes to the broader field of multimodal learning and robust speech recognition, we do not foresee significant ethical concerns beyond those inherent to AVSR systems, such as potential linguistic biases in speech models trained on imbalanced datasets or privacy issues when handling human speech video data. Responsible data use and fairness in model deployment remain critical in AVSR systems for ensuring equitable applications.

%%%%%%%%%%%%%%%%%%%%%%%%%%%%%%%%%%%%%%%%%%%%%%%%%%%%%%%%%%%%%%%%%

% \section*{Accessibility}
% Authors are kindly asked to make their submissions as accessible as possible for everyone including people with disabilities and sensory or neurological differences.
% Tips of how to achieve this and what to pay attention to will be provided on the conference website \url{http://icml.cc/}.

% \section*{Software and Data}

% If a paper is accepted, we strongly encourage the publication of software and data with the
% camera-ready version of the paper whenever appropriate. This can be
% done by including a URL in the camera-ready copy. However, \textbf{do not}
% include URLs that reveal your institution or identity in your
% submission for review. Instead, provide an anonymous URL or upload
% the material as ``Supplementary Material'' into the OpenReview reviewing
% system. Note that reviewers are not required to look at this material
% when writing their review.

% Acknowledgements should only appear in the accepted version.
% \section*{Acknowledgements}

% \textbf{Do not} include acknowledgements in the initial version of
% the paper submitted for blind review.

% If a paper is accepted, the final camera-ready version can (and
% usually should) include acknowledgements.  Such acknowledgements
% should be placed at the end of the section, in an unnumbered section
% that does not count towards the paper page limit. Typically, this will 
% include thanks to reviewers who gave useful comments, to colleagues 
% who contributed to the ideas, and to funding agencies and corporate 
% sponsors that provided financial support.

% \section*{Impact Statement}

% Authors are \textbf{required} to include a statement of the potential 
% broader impact of their work, including its ethical aspects and future 
% societal consequences. This statement should be in an unnumbered 
% section at the end of the paper (co-located with Acknowledgements -- 
% the two may appear in either order, but both must be before References), 
% and does not count toward the paper page limit. In many cases, where 
% the ethical impacts and expected societal implications are those that 
% are well established when advancing the field of Machine Learning, 
% substantial discussion is not required, and a simple statement such 
% as the following will suffice:

% This paper presents work whose goal is to advance the field of 
% Machine Learning. There are many potential societal consequences 
% of our work, none which we feel must be specifically highlighted here.

% The above statement can be used verbatim in such cases, but we 
% encourage authors to think about whether there is content which does 
% warrant further discussion, as this statement will be apparent if the 
% paper is later flagged for ethics review.


% In the unusual situation where you want a paper to appear in the
% references without citing it in the main text, use \nocite
% \nocite{langley00}


\bibliography{icml2025_conference}
\bibliographystyle{icml2025}


%%%%%%%%%%%%%%%%%%%%%%%%%%%%%%%%%%%%%%%%%%%%%%%%%%%%%%%%%%%%%%%%%%%%%%%%%%%%%%%
%%%%%%%%%%%%%%%%%%%%%%%%%%%%%%%%%%%%%%%%%%%%%%%%%%%%%%%%%%%%%%%%%%%%%%%%%%%%%%%
% APPENDIX
%%%%%%%%%%%%%%%%%%%%%%%%%%%%%%%%%%%%%%%%%%%%%%%%%%%%%%%%%%%%%%%%%%%%%%%%%%%%%%%
%%%%%%%%%%%%%%%%%%%%%%%%%%%%%%%%%%%%%%%%%%%%%%%%%%%%%%%%%%%%%%%%%%%%%%%%%%%%%%%
\newpage
\appendix
\onecolumn

\clearpage
\vspace{1.5em}
% \setcounter{page}{1}
{
    \centering
    \rule{\textwidth}{1.5pt}\vspace{0.6em} \\
    \vspace{0.5em}
    \Large
    \textbf{{\includegraphics[width=18pt]{desert_emoji.png}}\,MoHAVE: Mixture of Hierarchical \\
    \vspace{0.1em} Audio-Visual Experts for Robust Speech Recognition} \\
    \vspace{0.5em} Supplementary Material \\
    \rule{\textwidth}{1.5pt} \\
    \vspace{1.0em}
    % ]\
}


\section{Experimental Setup}

\subsection{Model Description}
\label{appx:model_details}

As described in \S\ref{sec:implementation}, \ourmodel is implemented in two configurations: \textsc{Base} and \textsc{Large}, following the architecture of AV-HuBERT-\textsc{Base} and AV-HuBERT-\textsc{Large}, respectively. The encoder maintains the same structure as AV-HuBERT, while the decoder incorporates MoE layers by replacing every feed-forward network (FFN) layer with expert modules. Each expert in the MoE layers follows the same bottleneck structure with FFN, consisting of two fully-connected layers with an activation function. 

To encourage expert group specialization, we apply load biasing, where either audio or video is randomly dropped with a probability of 25\%. This allows the model to learn modality-aware expert utilization. For expert selection, a router network assigns tokens to a subset of experts, ensuring efficient computation. The router probabilities are always normalized as sum to 1 when computing the output $y$. The routing networks $V_r$ and $W_r$ are parameterized as matrices, with dimensions matching the hidden dimension size by the number of experts.

For comparison, we evaluate multiple MoE-based AVSR models:
\begin{itemize}[leftmargin=10pt, label={$\circ$}]
\setlength\itemsep{-0.1em}
\vspace{-10pt}
    \item AV-MoE: A simple MoE implementation over AV-HuBERT, activating top-2 out of 4 or 8 experts per token. We follow the same implementation of sparse MoE as \citep{dai2022stablemoe, jiang2024mixtral}.
    \item AV-MoE with Hard Routing: Uses top-2 out of 4 experts for unimodal inputs (audio-only or video-only). For multimodal (audio-visual) inputs, it activates top-1 from each expert group and averages their outputs. This model does not have an explicit router for groups, but within each group, there is an intra-modal router, \ie $W_r^A$ or $W_r^V$.
    \item \ourmodel: Employs top-1 expert per group, with an inter-modal router dynamically adjusting group weight assignments and an intra-modal router uniformly dispatching the tokens to modality-specific experts.
\end{itemize}


\subsection{Computation Cost}
\label{appx:computation_cost}

\begin{wraptable}{r}{10.2cm}
    \centering
    \small
    \vspace{-17pt}
    \caption{Computational cost of AV-HuBERT and \ourmodel in FLOPs.}
    \label{tab:computation_cost}
    \vspace{5pt}
    \addtolength{\tabcolsep}{-2pt}
    \begin{tabular}{l|cc|cc}
    \toprule
    \multirow{2}{*}{Model} & Activated & Total & Compute & Compute\,/\,FFN \\
     & Params & Params & (GFLOPs\,/\,seq) & (MFLOPs\,/\,seq) \\
    \midrule
    AV-HuBERT-\textsc{Base} & 161M & 161M & 12.1 & 472 \\
    \ourmodel-\textsc{Base} & 189M & 359M & 14.8 & 921 \\
    AV-HuBERT-\textsc{Large} & 477M & 477M & 32.2 & 839 \\
    \ourmodel-\textsc{Large} & 553M & 1.0B & 39.3 & 1,630 \\
    \bottomrule
    \end{tabular}
\end{wraptable}


Table\,\ref{tab:computation_cost} summarizes the parameter sizes and computational costs of \ourmodel. \ourmodel-\textsc{Base} contains 359M parameters, while the \textsc{Large} version expands to 1B parameters. Specifically, for \ourmodel-\textsc{Base}, the encoder accounts for 103M parameters, and the decoder 256M, whereas in \textsc{Large}, the encoder holds 325M, and the decoder 681M.
Despite its larger model capacity, \ourmodel maintains computational efficiency through sparse activation, where only around half of the total parameters are active per token. This results in 189M active parameters for \textsc{Base} and 553M for \textsc{Large}.

To assess actual computation costs when processing inputs, we measure floating point operations per second (FLOPs) using an audio-visual sequence of 500 frames with 50 text tokens. The entire compute cost for AV-HuBERT-\textsc{Base} and \ourmodel-\textsc{Base} are 12.1 GFLOPs and 14.8 GFLOPs, respectively, while for \textsc{Large}, the computes are 32.2 GFLOPs and 39.3 GFLOPs. This indicates a slight increase in FLOPs for \ourmodel, primarily due to the MoE layers replacing FFNs in the decoder. Although the MoE layers require roughly twice the computation cost of standard FFNs (refer to Compute\,/\,FFN), the encoder and attention layers in the decoder remain unchanged. Consequently, the overall computational cost remains comparable to AV-HuBERT counterparts, ensuring scalability without significant computation overhead.





\subsection{LRS3 Benchmark Experiments}
\label{appx:lrs3_benchmark}

We initialize our model using the pretrained checkpoint from \citep{shi2022learning} and fine-tune it on the LRS3 train set for 120K steps. The encoder remains frozen for the first 90K steps, allowing only the AVSR decoder to be trained, after which the entire model is fine-tuned for the remaining 30K steps. Our fine-tuning setup follows the configurations from \citep{shi2022robust}. We employ a sequence-to-sequence negative log-likelihood loss for predicting the next text token, without using connectionist temporal classification (CTC) decoding \citep{watanabe2017hybrid}. The Adam optimizer \cite{kingma2014adam} is used with a learning rate of 5e-4 and a polynomial decay schedule with an initial warmup. Each training step processes 8,000 audio-visual frames, equivalent to 320 seconds of speech data.

For inference, we use beam search with a beam size of 50. The AVSR performance is evaluated using word error rate (WER) across five signal-to-noise ratio (SNR) levels: $\{-10, -5, 0, 5, 10\}$ (lower value means higher noise level). We use audio noise sampled from MUSAN (babble, music, natural) and LRS3 speech noise, ensuring no speaker overlap between training and test sets. Since Table\,\ref{tab:main_result} presents SNR-averaged results for each noise type, we provide the full results across all SNR levels in Table\,\ref{tab:lrs3_full}.


\subsection{MuAViC Benchmark Experiments}
\label{appx:muavic_benchmark}

We evaluate \ourmodel on the MuAViC benchmark \cite{anwar2023muavic} for multilingual AVSR and X-to-English AVS2TT tasks. For multilingual AVSR, the dataset includes 8 non-English languages: Arabic (Ar), German (De), Greek (El), Spanish (Es), French (Fr), Italian (It), Portuguese (Pt), and Russian (Ru), encompassing approximately 700 hours of training data from 3,700 speakers. For X-En AVS2TT, the dataset covers 6 languages: Greek, Spanish, French, Italian, Portuguese, and Russian, where each sample includes audio-visual speech with corresponding English transcriptions.

A single multilingual model is trained for each task, capable of detecting the source language and generating target transcriptions accordingly. The evaluation is conducted on each language separately, as seen in Table\,\ref{tab:muavic}. Using the pretrained multilingual AV-HuBERT from \citep{choi2024av2av}, we fine-tune the model for 120K steps, unfreezing the encoder after 10K steps. Inference is performed with beam size of 5, and the samples with empty ground-truth transcriptions are removed from the evaluation set. 


\begin{table*}[!h]
    \centering
    \small
    \vspace{10pt}
    \caption{Audio-visual speech recognition performance\,(WER\,\%) on the LRS3 dataset\,\citep{afouras2018lrs3}. The number of parameters for each model includes both encoder and decoder. For evaluation, augmented noise is sampled from the MUSAN dataset\,\citep{snyder2015musan}, while speech noise is sampled from the held-out set of LRS3. AV-MoE and \ourmodel use 8 experts.}
    \label{tab:lrs3_full}
    \vspace{5pt}
    \addtolength{\tabcolsep}{-3pt}
    \renewcommand{\arraystretch}{1.1}
    \resizebox{\textwidth}{!}{
    \begin{tabular}{l|cccccc|cccccc|cccccc|cccccc}
    \toprule
    \multirow{2}{*}{Model} & \multicolumn{6}{c|}{Babble, SNR (dB) $=$} & \multicolumn{6}{c|}{Speech, SNR (dB) $=$} & \multicolumn{6}{c|}{Music, SNR (dB) $=$} & \multicolumn{6}{c}{Natural, SNR (dB) $=$} \\
    & -10 & -5 & 0 & 5 & 10 & \!\textbf{AVG}\! & -10 & -5 & 0 & 5 & 10 & \!\textbf{AVG}\! & -10 & -5 & 0 & 5 & 10 & \!\textbf{AVG}\! & -10 & -5 & 0 & 5 & 10 & \!\textbf{AVG}\! \\
    \midrule
    AV-HuBERT-\textsc{Base} & 27.6 & 14.1 & 6.1 & 3.8 & 2.7 & 10.8 & 8.6 & 5.8 & 3.9 & 3.3 & 2.8 & 4.9 & 12.2 & 6.4 & 3.8 & 2.8 & 2.5 & 5.6 & 10.9 & 5.4 & 4.0 & 2.8 & 2.3 & 5.1 \\
    AV-MoE-\textsc{Base} & 26.3 & 13.7 & 6.2 & 3.5 & 2.7 & 10.5 & 8.4 & 5.3 & 3.4 & 2.8 & 2.3 & 4.5 & 11.5 & 6.1 & 3.7 & 2.7 & 2.5 & 5.3 & 9.7 & 6.0 & 3.4 & 2.8 & 2.5 & 4.9 \\
    ~~(+) Hard Routing & 25.2 & 13.2 & 5.6 & 3.2 & 2.4 & 9.9 & 8.2 & 5.2 & 3.4 & 2.7 & 2.3 & 4.4 & 11.0 & 5.3 & 3.6 & 2.6 & 2.2 & 5.0 & 9.4 & 5.3 & 3.2 & 2.5 & 2.3 & 4.6 \\
    \rowcolor[HTML]{e1fefe}
    \ourmodel-\textsc{Base} & 25.3 & \textbf{12.2} & \textbf{5.3} & \textbf{2.9} & \textbf{2.3} & \textbf{9.6} & \textbf{7.9} & \textbf{5.1} & \textbf{3.3} & \textbf{2.4} & \textbf{2.3} & \textbf{4.2} & \textbf{10.3} & 5.6 & \textbf{3.3} & \textbf{2.3} & \textbf{2.0} & \textbf{4.7} & 9.7 & \textbf{5.1} & \textbf{3.2} & \textbf{2.4} & \textbf{2.2} & \textbf{4.5} \\
    \rowcolor[HTML]{e1fefe}
    ~~(--) Load Biasing & 26.5 & 13.6 & 5.6 & 3.2 & 2.4 & 10.3 & 8.2 & 5.2 & 3.5 & 2.6 & 2.4 & 4.4 & 11.1 & 6.4 & 3.4 & 2.7 & 2.6 & 5.2 & 10.3 & 5.6 & 3.4 & 2.7 & 2.4 & 4.9 \\
    \midrule
    AV-HuBERT-\textsc{Large} & 27.0 & 12.4 & 4.7 & 2.4 & 1.8 & 9.7 & 11.4 & 4.6 & 2.9 & 2.2 & 1.8 & 4.6 & 10.5 & 4.9 & 2.9 & 2.0 & 1.6 & 4.4 & 9.6 & 4.7 & 2.5 & 2.0 & 1.8 & 4.1 \\
    AV-MoE-\textsc{Large} & 28.1 & 12.5 & 5.0 & 2.7 & 2.1 & 10.1 & 7.9 & 4.0 & 2.9 & 2.4 & 2.0 & 3.8 & 10.4 & 5.4 & 2.9 & 2.3 & 1.9 & 4.6 & 8.9 & 4.8 & 3.1 & 2.0 & 2.0 & 4.2 \\
    ~~(+) Hard Routing & 22.9 & 10.8 & 3.8 & 2.4 & 1.8 & 8.3 & 6.7 & 3.9 & 2.4 & 1.8 & 1.8 & 3.3 & 9.9 & 4.3 & 2.3 & 1.8 & 1.9 & 4.0 & 8.3 & 4.1 & 2.4 & 1.9 & 1.7 & 3.7 \\
    \rowcolor[HTML]{e1fefe}
    \ourmodel-\textsc{Large} & \textbf{21.0} & \textbf{9.8} & 4.1 & \textbf{2.2} & \textbf{1.6} & \textbf{7.7} & \textbf{5.0} & \textbf{3.6} & \textbf{2.3} & 2.0 & 1.9 & \textbf{3.0} & \textbf{8.2} & \textbf{4.0} & 2.6 & \textbf{1.8} & \textbf{1.8} & \textbf{3.7} & \textbf{7.3} & \textbf{3.7} & 2.6 & \textbf{1.9} & \textbf{1.6} & \textbf{3.4} \\
    \rowcolor[HTML]{e1fefe}
    ~~(--) Load Biasing & 27.8 & 12.4 & 4.5 & 2.6 & 2.0 & 9.9 & 6.7 & 4.0 & 3.1 & 2.1 & 1.9 & 3.6 & 10.6 & 5.3 & 3.0 & 2.1 & 1.8 & 4.6 & 9.9 & 4.9 & 2.8 & 2.1 & 2.0 & 4.3 \\
    \bottomrule
    \end{tabular}
    }
\end{table*}






\clearpage
\section{Additional Results}
\label{appx:additional_results}

\subsection{Expert Group Utilization}
\label{appx:expert_group_usage}

In the main paper (Figure\,\ref{fig:expert_load_noise}), we have presented expert load distribution for selected layers. Figure\,\ref{fig:appx_group_load} provides the distribution across all MoE layers, illustrating how \ourmodel dynamically adjusts expert groups based on noise conditions.

\begin{figure}[!h]
    \centering
    \includegraphics[width=\linewidth]{tex_figure/appx/appx_group_load_new.pdf}
    \vspace{-20pt}
    \caption{Expert load distribution in \ourmodel for the audio group (solid bars) and visual group (dashed bars) across noisy audio-visual sequences under babble (first row) and natural (second row) noise. The frequency of each expert has been weighted by the inter-modal router's output probability.}
    \label{fig:appx_group_load}
\end{figure}


\subsection{Multilingual Tasks in Clean Environments}

Table\,\ref{tab:muavic_clean} outlines the MuAViC benchmark results in a clean environment, without auditory noise added. The experimental setup remains consistent with Table\,\ref{tab:muavic}, utilizing the same models. Unlike the noisy setting, we observe that \ourmodel does not yield significant performance improvements in clean speech tasks. This is primarily because \ourmodel enhances AVSR under noisy conditions by dynamically adjusting the utilization of audio and visual expert groups. 

Indeed, in clean speech recognition and translation tasks, encoder capacity---particularly when pretrained on large-scale audio data---plays a more crucial role than decoder-specific training methods. In addition, visual information is less essential in noise-free environments, as demonstrated by the strong ASR performance of the Whisper-large-v2 model \cite{radford2023robust}. Even the smaller ASR model, XLS-R 300M \cite{babu2022xls}, surpasses AVSR models such as mAV-HuBERT \cite{anwar2023muavic} or u-HuBERT \cite{hsu2022u} in this setting, underscoring that the advantage of using AVSR models emerges most clearly in robust speech recognition.

\begin{table*}[!h]
    \centering
    \small
    \caption{Multilingual audio-visual speech task performance, with non-English speech recognition (WER) and X-En speech-to-text translation (BLEU score), in a clean environment without auditory noise. $^\dagger$Results obtained from \citet{han-etal-2024-xlavs}. $^\ddagger$Re-implemented using the pretrained model from \citet{choi2024av2av}.}
    \label{tab:muavic_clean}
    \vspace{5pt}
    % \addtolength{\tabcolsep}{-1pt}
    \resizebox{\textwidth}{!}{
    \begin{tabular}{llccccccccc}
    \toprule
    && \multicolumn{8}{c}{Source} & \\
    \cmidrule{3-10}
    Model & Pretrain data & Ar & De & El & Es & Fr & It & Pt & Ru & Avg \\
    \midrule
    \multicolumn{11}{c}{\textbf{\textit{Clean Speech Recognition, Test WER $\downarrow$}}} \\
    Whisper large-v2\,\cite{radford2023robust} & \textit{680k hrs, 100+ langs} & 91.5 & \textbf{24.8} & 25.4 & 12.0 & 12.7 & 13.0 & 15.5 & 31.1 & 28.2 \\
    u-HuBERT$^\dagger$\,\cite{hsu2022u} & \textit{1.7k hrs, English} & 89.3 & 52.1 & 46.4 & 17.3 & 20.5 & 21.2 & 21.9 & 44.4 & 39.1 \\
    mAV-HuBERT\,\cite{anwar2023muavic} & \textit{1.7k hrs, English} & 69.3 & 47.2 & 41.2 & 16.2 & 19.0 & 19.8 & 19.9 & 38.0 & 33.8 \\
    XLS-R 300M$^\dagger$\,\cite{babu2022xls} & \textit{1.2k hrs, 9 langs} & 85.6 & 44.0 & 34.4 & 13.2 & 15.1 & 14.3 & 16.2 & 34.4 & 32.2 \\
    XLAVS-R 300M\,\cite{han-etal-2024-xlavs} & \textit{8.3k hrs, 100+ langs} & 80.0 & 38.0 & 28.1 & 11.7 & 15.3 & 13.8 & 14.4 & 31.2 & 29.1 \\
    XLAVS-R 2B\,\cite{han-etal-2024-xlavs} & \textit{1.2k hrs, 9 langs} & 79.3 & 44.4 & \textbf{19.0} & \textbf{9.1} & \textbf{12.3} & \textbf{10.6} & \textbf{11.2} & \textbf{25.0} & \textbf{26.4} \\
    \cdashlinelr{1-11}
    mAV-HuBERT$^\ddagger$ & \textit{7.0k hrs, 100+ langs} & \textbf{78.3} & 41.4 & 25.5 & 11.9 & 16.2 & 14.8 & 14.3 & 31.6 & 29.3 \\
    \ourmodel-\textsc{Large} (\textbf{ours}) & \textit{7.0k hrs, 100+ langs} & 85.1 & 38.9 & 25.9 & 11.2 & 14.6 & 14.0 & 13.8 & 30.0 & 29.2 \\
    \midrule
    \multicolumn{11}{c}{\textbf{\textit{Clean X-En Speech-to-Text Translation, Test BLEU $\uparrow$}}} \\
    Whisper large-v2\,\cite{radford2023robust} & \textit{680k hrs, 100+ langs} & - & - & \textbf{24.2} & \textbf{28.9} & \textbf{34.5} & \textbf{29.2} & \textbf{32.6} & \textbf{16.1} & \textbf{29.9} \\
    mAV-HuBERT\,\cite{anwar2023muavic} & \textit{1.7k hrs, English} & - & - & 7.6 & 20.5 & 25.2 & 20.0 & 24.0 & 8.1 & 17.6 \\
    XLAVS-R 300M\,\cite{han-etal-2024-xlavs} & \textit{8.3k hrs, 100+ langs} & - & - & 18.3 & 23.9 & 29.8 & 25.1 & 28.9 & 12.1 & 23.0 \\
    XLAVS-R 2B\,\cite{han-etal-2024-xlavs} & \textit{1.2k hrs, 9 langs} & - & - & 21.6 & 25.1 & 30.6 & 26.6 & 29.9 & 13.9 & 24.6 \\
    \cdashlinelr{1-11}
    mAV-HuBERT$^\ddagger$ & \textit{7.0k hrs, 100+ langs} & - & - & 11.5 & 24.2 & 29.2 & 23.9 & 28.1 & 10.4 & 21.2 \\
    \ourmodel-\textsc{Large} (\textbf{ours}) & \textit{7.0k hrs, 100+ langs} & - & - & 13.8 & 24.9 & 30.8 & 25.0 & 28.7 & 10.9 & 22.4 \\
    \bottomrule
    \end{tabular}
    }
\end{table*}


\subsection{Number of Activated Experts}

By default, \ourmodel selects one expert from each group---audio and visual---activating a total of two experts per token. This design is to match the compute of standard MoE implementations, which utilizes top-2 out of 8 experts. A natural question arises: \textit{does activating more experts improve performance, or does it simply increase computational costs without substantial gains?}

Table\,\ref{tab:num_experts} presents the results when activating more experts of \ourmodel-\textsc{Base}, where top-$k^A$ experts from the audio group and top-$k^V$ experts from the visual group are selected. Interestingly, increasing the number of audio experts significantly degrades performance, implying that the model might be confused by employing another sub-optimal expert. 

% \begin{table}[!t]
\begin{wraptable}{r}{9.5cm}
    \vspace{-10pt}
    \centering
    \small
    \caption{Impact of the number of activated experts on AVSR performance.}
    \label{tab:num_experts}
    \vspace{3pt}
    \addtolength{\tabcolsep}{-1pt}
    \begin{tabular}{c|cccc|c|c}
        \toprule
        ($k^A$, $k^V$) & babble & speech & music & natural & N-WER & C-WER \\
        \midrule
        ($1, 1$) & 9.6 & 4.2 & 4.7 & 4.5 & 5.8 & 1.8 \\
        ($1, 2$) & 9.3 & 4.1 & 4.8 & 4.4 & 5.7 & 1.9 \\
        ($2, 1$) & 10.1 & 4.5 & 5.3 & 4.9 & 6.2 & 2.4 \\
        ($2, 2$) & 11.0 & 5.2 & 5.8 & 5.5 & 6.9 & 3.0 \\
        \bottomrule
    \end{tabular}
    \vspace{-5pt}
\end{wraptable}
% \end{table}
In contrast, activating two visual experts while keeping one audio expert improves performance (N-WER of 5.7\%) compared to the default setting of single visual expert. Particularly under the babble noise, WER has decreased from 9.6\% to 9.3\%. 
This suggests that adding an additional visual expert can be beneficial in noisy environments, likely due to the increased robustness from visual information in challenging audio conditions.


\subsection{Unimodal Task Results}

Table\,\ref{tab:asr_vsr} presents unimodal task results, evaluating model performance on video-only (VSR) sequences and audio-only (ASR) sequences. 
BRAVEn \cite{haliassos2024braven} and Llama-3.1-8B-AVSR \cite{cappellazzo2024large} models achieve the best VSR performance, as these models are specifically pretrained for the VSR task. While using an LLM decoder is highly effective in VSR, since LLMs are able to refine and correct recognition errors, ASR performance is largely determined by the encoder's pretraining strategy as BRAVEn and Whisper encoders.
As an adaptive audio-visual model, \ourmodel does not specialize in unimodal tasks but instead performs robustly in multimodal AVSR. It only exhibits a slight improvement in VSR over AV-HuBERT. These results indicate that unimodal performance is primarily influenced by the effectiveness of the encoder pretraining strategy rather than the MoE-based multimodal approach.

\begin{table}[!h]
    \centering
    \small
    \caption{Comparison on the unimodal ASR and VSR task performance.}
    \label{tab:asr_vsr}
    \vspace{5pt}
    \begin{tabular}{ll|cc}
    \toprule
    Method & Encoder\,+\,Decoder & V & A \\
    \midrule
    BRAVEn~\cite{haliassos2024braven} & BRAVEn\,+\,Transformer & 26.6 & 1.2 \\
    Llama3.1-8B-AVSR~\cite{cappellazzo2024large} & AV-HuBERT\,+\,LLM & 25.3 & 1.4 \\ 
    Llama3.1-8B-AVSR~\cite{cappellazzo2024large} & Whisper\,+\,LLM & - & 1.1 \\
    \midrule
    AV-HuBERT-\textsc{Large}~\cite{shi2022learning} & AV-HuBERT\,+\,Transformer & 28.6 & 1.4 \\
    \ourmodel-\textsc{Large} (\textbf{ours}) & AV-HuBERT\,+\,MoE & 28.2 & 1.4 \\
    \bottomrule
    \end{tabular}
\end{table}


\clearpage
\subsection{Variations of MoHAVE Implementations}

\paragraph{\ourmodel in the encoder.}

We have implemented \ourmodel by integrating MoE into the decoder to facilitate text token processing while enhancing multimodal fusion. Since the AVSR decoder incorporates information from both audio and visual modalities along with text tokens, the decoder-based MoHAVE is expected to be the most effective strategy. An alternative approach is to apply MoHAVE within the encoder, by pretraining the encoder using the AV-HuBERT masked prediction strategy \cite{shi2022learning}. For this, we initialize the pretrained encoder (with standard Transformers) and convert the FFN layers into MoE layers by copying the FFN parameters into all the expert modules. Since the \textsc{Base} model consists of 12 encoder layers, we convert 6 of them alternatively to match the number of MoE layers in the decoder \ourmodel. During fine-tuning, all MoE layers in the encoder are also trained following the same procedure.

There are two options for pretraining strategies: (1) pretraining only the MoE layers initialized from the FFN parameters, and (2) pretraining the entire encoder with MoE layers. As shown in Table\,\ref{tab:encoder_mohave}, the latter approach significantly outperforms the former, suggesting that encoder \ourmodel requires full pretraining for effective learning.
However, even with full pretraining, encoder \ourmodel performs inferior to decoder \ourmodel. This is because the encoder only processes audio and visual tokens, whereas the decoder directly integrates audio-visual embeddings with text, finding optimal strategies for text token dispatching that best improves speech recognition. In addition, applying \ourmodel to both the encoder and decoder leads to degraded performance despite the increased computational cost.

\vspace{-8pt}
\paragraph{Decoder uptraining.}

We also explore a successive training strategy for the decoder, referred to as uptraining \cite{ainslie2023gqa}, where the decoder \ourmodel undergoes additional training after the fine-tuning phase of standard Transformers. However, as seen in Table\,\ref{tab:encoder_mohave}, uptraining does not yield further improvements compared to training from scratch, even after an additional 120K training steps. In fact, we observed the shorter uptraining steps leading to degraded performance. This suggests that training the decoder \ourmodel requires a comprehensive learning phase rather than incremental fine-tuning, as MoE  may fundamentally alter the processing pathways of tokens.

\begin{table}[!h]
    \centering
    \small
    \caption{Performance comparison of \ourmodel applied to the encoder, decoder, and both.}
    \label{tab:encoder_mohave}
    \vspace{5pt}
    \begin{tabular}{l|cccc|c|c}
        \toprule
        Method & babble & speech & music & natural & N-WER & C-WER \\
        \midrule
        Encoder MoHAVE (pretrain only MoE) & 10.5 & 5.0 & 5.6 & 5.0 & 6.5 & 2.4 \\
        Encoder MoHAVE & 10.0 & 4.4 & 5.1 & 4.5 & 6.0 & 1.9 \\
        Encoder + Decoder MoHAVE & 10.1 & 4.7 & 5.3 & 4.7 & 6.2 & 2.0 \\
        \midrule
        Decoder MoHAVE & 9.6 & 4.2 & 4.7 & 4.5 & 5.8 & 1.8 \\
        Decoder MoHAVE (uptrain) & 9.7 & 4.2 & 4.8 & 4.5 & 5.8 & 1.8 \\        
        \bottomrule
    \end{tabular}
\end{table}

% \section{You \emph{can} have an appendix here.}

% You can have as much text here as you want. The main body must be at most $8$ pages long.
% For the final version, one more page can be added.
% If you want, you can use an appendix like this one.  

% The $\mathtt{\backslash onecolumn}$ command above can be kept in place if you prefer a one-column appendix, or can be removed if you prefer a two-column appendix.  Apart from this possible change, the style (font size, spacing, margins, page numbering, etc.) should be kept the same as the main body.
%%%%%%%%%%%%%%%%%%%%%%%%%%%%%%%%%%%%%%%%%%%%%%%%%%%%%%%%%%%%%%%%%%%%%%%%%%%%%%%
%%%%%%%%%%%%%%%%%%%%%%%%%%%%%%%%%%%%%%%%%%%%%%%%%%%%%%%%%%%%%%%%%%%%%%%%%%%%%%%


\end{document}


% This document was modified from the file originally made available by
% Pat Langley and Andrea Danyluk for ICML-2K. This version was created
% by Iain Murray in 2018, and modified by Alexandre Bouchard in
% 2019 and 2021 and by Csaba Szepesvari, Gang Niu and Sivan Sabato in 2022.
% Modified again in 2023 and 2024 by Sivan Sabato and Jonathan Scarlett.
% Previous contributors include Dan Roy, Lise Getoor and Tobias
% Scheffer, which was slightly modified from the 2010 version by
% Thorsten Joachims & Johannes Fuernkranz, slightly modified from the
% 2009 version by Kiri Wagstaff and Sam Roweis's 2008 version, which is
% slightly modified from Prasad Tadepalli's 2007 version which is a
% lightly changed version of the previous year's version by Andrew
% Moore, which was in turn edited from those of Kristian Kersting and
% Codrina Lauth. Alex Smola contributed to the algorithmic style files.
