%%%%%%%% ICML 2025 EXAMPLE LATEX SUBMISSION FILE %%%%%%%%%%%%%%%%%

\documentclass{article}

% Recommended, but optional, packages for figures and better typesetting:
\usepackage{microtype}
\usepackage{graphicx}
\usepackage{subfigure}
\usepackage{booktabs} % for professional tables
\usepackage{xcolor}
\usepackage{subcaption}

% hyperref makes hyperlinks in the resulting PDF.
% If your build breaks (sometimes temporarily if a hyperlink spans a page)
% please comment out the following usepackage line and replace
% \usepackage{icml2025} with \usepackage[nohyperref]{icml2025} above.
\usepackage{hyperref}


% Attempt to make hyperref and algorithmic work together better:
\newcommand{\theHalgorithm}{\arabic{algorithm}}

% Use the following line for the initial blind version submitted for review:
\usepackage[accepted]{icml2025}

% If accepted, instead use the following line for the camera-ready submission:
% \usepackage[accepted]{icml2025}

% Optional math commands from https://github.com/goodfeli/dlbook_notation.
%%%%% NEW MATH DEFINITIONS %%%%%

% \usepackage{amsmath,amsfonts,bm}
\usepackage{amsmath,amsfonts}

\usepackage{pifont}


\newcommand{\R}{\mathbb{R}}


\def\va{{\mathbf{a}}}
\def\vg{{\mathbf{g}}}

% Sets
\def\sR{\mathbb{R}}
\def\sC{\mathbb{C}}
\def\sZ{\mathbb{Z}}
\def\sN{\mathbb{N}}
\def\sQ{\mathbb{Q}}

\def\sS{\mathcal{S}}



% Vectors
\def\vzero{{\mathbf{0}}}
\def\vone{{\mathbf{1}}}
\def\vmu{{\mathbf{\mu}}}
\def\vtheta{{\mathbf{\theta}}}
\def\va{{\mathbf{a}}}
\def\vb{{\mathbf{b}}}
\def\vc{{\mathbf{c}}}
\def\vd{{\mathbf{d}}}
\def\ve{{\mathbf{e}}}
\def\vf{{\mathbf{f}}}
\def\vg{{\mathbf{g}}}
\def\vh{{\mathbf{h}}}
\def\vi{{\mathbf{i}}}
\def\vj{{\mathbf{j}}}
\def\vk{{\mathbf{k}}}
\def\vl{{\mathbf{l}}}
\def\vm{{\mathbf{m}}}
\def\vn{{\mathbf{n}}}
\def\vo{{\mathbf{o}}}
\def\vp{{\mathbf{p}}}
\def\vq{{\mathbf{q}}}
\def\vr{{\mathbf{r}}}
\def\vs{{\mathbf{s}}}
\def\vt{{\mathbf{t}}}
\def\vu{{\mathbf{u}}}
\def\vv{{\mathbf{v}}}
\def\vw{{\mathbf{w}}}
\def\vx{{\mathbf{x}}}
\def\vy{{\mathbf{y}}}
\def\vz{{\mathbf{z}}}
\def\vzeta{{\mathbf{\zeta}}}

% Matrix
\def\mA{{\mathbf{A}}}
\def\mB{{\mathbf{B}}}
\def\mC{{\mathbf{C}}}
\def\mD{{\mathbf{D}}}
\def\mE{{\mathbf{E}}}
\def\mF{{\mathbf{F}}}
\def\mG{{\mathbf{G}}}
\def\mH{{\mathbf{H}}}
\def\mI{{\mathbf{I}}}
\def\mJ{{\mathbf{J}}}
\def\mK{{\mathbf{K}}}
\def\mL{{\mathbf{L}}}
\def\mM{{\mathbf{M}}}
\def\mN{{\mathbf{N}}}
\def\mO{{\mathbf{O}}}
\def\mP{{\mathbf{P}}}
\def\mQ{{\mathbf{Q}}}
\def\mR{{\mathbf{R}}}
\def\mS{{\mathbf{S}}}
\def\mT{{\mathbf{T}}}
\def\mU{{\mathbf{U}}}
\def\mV{{\mathbf{V}}}
\def\mW{{\mathbf{W}}}
\def\mX{{\mathbf{X}}}
\def\mY{{\mathbf{Y}}}
\def\mZ{{\mathbf{Z}}}
\def\mBeta{{\mathbf{\beta}}}
\def\mPhi{{\mathbf{\Phi}}}
\def\mLambda{{\mathbf{\Lambda}}}
\def\mSigma{{\mathbf{\Sigma}}}


% Expectation
% \def\eE{\mathop{\mathbb{E}}\limits}
\def\eE{\mathbb{E}}

% Probability
\def\pP{\mathbb{P}}

% Tilde
\def\tf{\tilde{f}}
\def\tS{\tilde{S}}
\def\wtF{\widetilde{\mathcal{F}}}
\def\whR{\widehat{R}}
\def\tvx{\tilde{\mathbf{x}}}
\def\ty{\tilde{y}}


\def\defeq{\overset{\textup{def}}{=}}
% \def\defeq{\overset{.}{=}}
\def\defone{\overset{\text{\ding{172}}}{=}}
\def\deftwo{\overset{\text{\ding{173}}}{=}}
\def\leqone{\overset{\text{\ding{172}}}{\leq}}
\def\leqtwo{\overset{\text{\ding{173}}}{\leq}}
\def\leqthree{\overset{\text{\ding{174}}}{\leq}}
\def\leqfour{\overset{\text{\ding{175}}}{\leq}}
\def\eqone{\overset{\text{\ding{172}}}{=}}
\def\eqtwo{\overset{\text{\ding{173}}}{=}}
\def\eqthree{\overset{\text{\ding{174}}}{=}}
\def\eqfour{\overset{\text{\ding{175}}}{=}}
\def\geqfive{\overset{\text{\ding{176}}}{\geq}}
% \usepackage{microtype}
\usepackage{geometry}
% \usepackage{subfig}
\usepackage{booktabs} 
\usepackage{bbm}
\usepackage{mathtools}
% \usepackage{amsthm}
\usepackage{nccmath}
\usepackage{setspace}

\usepackage{caption}
\usepackage{subcaption}

\usepackage[linesnumbered,ruled,vlined]{algorithm2e}
% \usepackage{algorithmic}
% \usepackage{algorithm}

\SetKwInput{KwInput}{Input}                % Set the Input
\SetKwInput{KwOutput}{Output}              % set the Output
\newcommand\mycommfont[1]{\footnotesize\ttfamily\textcolor{blue}{#1}}
\SetCommentSty{mycommfont}
\newcommand{\algcapsty}[1]{\small\sffamily\bfseries{#1}}
\SetAlCapSty{algcapsty}

\usepackage[T1]{fontenc}
\usepackage{wrapfig,lipsum,booktabs}

% \usepackage{natbib}
\usepackage{soul}
\usepackage{dsfont}
\usepackage{enumerate}
\usepackage{enumitem}

% \usepackage{kotex}
% \usepackage{hyperref}
% \usepackage[hidelinks]{hyperref}
\usepackage{amsmath}
% \usepackage{amsthm}
\usepackage{amsfonts}
\usepackage{bbm}
\usepackage{dsfont}
\usepackage[Symbol]{upgreek}
\usepackage{lscape}
\usepackage{caption}
\usepackage{balance}
\usepackage{xspace}
\usepackage{float}
\usepackage{kotex}

\usepackage{wasysym}
%\usepackage[table,xcdraw,dvipsnames]{xcolor}
\usepackage{xcolor}
\usepackage{multirow}
\usepackage{array, boldline, rotating}

\usepackage{amssymb}% http://ctan.org/pkg/amssymb
\usepackage{pifont}% http://ctan.org/pkg/pifont
\newcommand{\cmark}{\ding{51}\xspace}%
\newcommand{\omark}{\textbf{$\mathcal{O}$}\xspace}%
\newcommand{\xmark}{\ding{55}\xspace}%

\newcommand{\ds}[1]{\mathds{#1}}
\newcommand{\mc}[1]{\mathcal{#1}}
\newcommand{\bb}[1]{\mathbbm{#1}}

% %%%%%% Theorem Related Things %%%%%%
% \theoremstyle{plain}
% \newtheorem{thm}{Theorem}
% \newtheorem{cor}{Corollary}
% \newtheorem{lem}{Lemma}
% \newtheorem{prop}{Proposition}

% \theoremstyle{definition}
% \newtheorem{defn}{Definition}
% \newtheorem{assum}{Assumption}



% Citation

% \let\oldeqcite\cite
% \renewcommand*\cite[1]{(\oldcite{#1})}
\let\oldeqref\eqref
\renewcommand*\eqref[1]{(\ref{#1})}

% % Highlight (incl. note)
\newcommand{\smnote}[1]{\textbf{\textcolor{Cyan}{SM: #1}}}
\newcommand{\jhnote}[1]{\textbf{\textcolor{Orange}{JH: #1}}}
\newcommand{\yes}[1]{\textcolor{blue}{[YES]}}
\newcommand{\no}[1]{\textcolor{orange}{[NO]}}
\newcommand{\na}[1]{\textcolor{gray}{[N/A]}}
%\newcommand\bg[1]{\textcolor{blue}{#1}} % JH
\newcommand\jt[1]{\textcolor{brown}{#1}} % JT
\newcommand\jh[1]{\textcolor{black}{#1}} % JH
\newcommand\sm[1]{\textcolor{blue}{#1}} % SM
\newcommand{\eg}{\emph{e.g.,~}}
\newcommand{\ie}{\emph{i.e.,~}}


% \renewcommand\jt[1]{\textcolor{black}{#1}} % JT
% \renewcommand\jh[1]{\textcolor{black}{#1}} % JH

% % Separation (paragraph)
\newcommand{\myparagraph}[1]{\vspace{0.07cm}\noindent\textbf{#1}~}

% % math op.

% % Font
\def\code#1{\texttt{#1}}
\DeclarePairedDelimiter\norm{\lVert}{\rVert}

% % % Definition
% \theoremstyle{definition}
\newcommand\scalemath[2]{\scalebox{#1}{\mbox{\ensuremath{\displaystyle #2}}}}



\newcommand{\thickhline}{\hlineB{4}}
\newcommand{\bfcode}[1]{\code{\textbf{#1}}}


\definecolor{LightCyan}{rgb}{0.88,1,1}
\definecolor{Blue}{rgb}{0, 0.5, 1}
\definecolor{Green}{rgb}{0.0, 0.8, 0.0 }
\definecolor{Red}{rgb}{0.95, 0.55, 0.6}
\definecolor{Skyblue}{rgb}{0.6, 0.6, 0.95 }



% Supplementary title
\NewDocumentEnvironment{suptitle}{ +b }{
    \twocolumn[{#1}]%
}{}

\NewDocumentCommand{\supptitle}{s}{
\begin{suptitle}
        \centering
        % \rule{\textwidth}{0.07cm}\\[-0.34cm]
        \rule{\textwidth}{0.03cm}\\[0.1cm]
        - Appendix -\\[0.2cm]
        {\Large 
            \textbf{\mytitle }
        }\\%[0.40cm]
        \rule{\textwidth}{0.03cm}\\[0.2cm]
\end{suptitle}}

% For theorems and such
\usepackage{mathtools}
\usepackage{amsthm}

% if you use cleveref..
\usepackage[capitalize,noabbrev]{cleveref}

%%%%%%%%%%%%%%%%%%%%%%%%%%%%%%%%
% THEOREMS
%%%%%%%%%%%%%%%%%%%%%%%%%%%%%%%%
\theoremstyle{plain}
\newtheorem{theorem}{Theorem}[section]
\newtheorem{proposition}[theorem]{Proposition}
\newtheorem{lemma}[theorem]{Lemma}
\newtheorem{corollary}[theorem]{Corollary}
\theoremstyle{definition}
\newtheorem{definition}[theorem]{Definition}
\newtheorem{assumption}[theorem]{Assumption}
\theoremstyle{remark}
\newtheorem{remark}[theorem]{Remark}

% Todonotes is useful during development; simply uncomment the next line
%    and comment out the line below the next line to turn off comments
%\usepackage[disable,textsize=tiny]{todonotes}
\usepackage[textsize=tiny]{todonotes}


% The \icmltitle you define below is probably too long as a header.
% Therefore, a short form for the running title is supplied here:
\icmltitlerunning{MoHAVE: Mixture of Hierarchical Audio-Visual Experts for Robust Speech Recognition}

\begin{document}

\twocolumn[
\icmltitle{{\vspace{-5pt}\includegraphics[width=18pt]{desert_emoji.png}}\,MoHAVE: Mixture of Hierarchical\\Audio-Visual Experts for Robust Speech Recognition}

% It is OKAY to include author information, even for blind
% submissions: the style file will automatically remove it for you
% unless you've provided the [accepted] option to the icml2025
% package.

% List of affiliations: The first argument should be a (short)
% identifier you will use later to specify author affiliations
% Academic affiliations should list Department, University, City, Region, Country
% Industry affiliations should list Company, City, Region, Country

% You can specify symbols, otherwise they are numbered in order.
% Ideally, you should not use this facility. Affiliations will be numbered
% in order of appearance and this is the preferred way.
\icmlsetsymbol{equal}{*}

\begin{icmlauthorlist}
\icmlauthor{Sungnyun Kim}{kaistai}
\icmlauthor{Kangwook Jang}{kaistee}
\icmlauthor{Sangmin Bae}{kaistai}
\icmlauthor{Sungwoo Cho}{kaistai}
\icmlauthor{Se-Young Yun}{kaistai}
% \icmlauthor{Firstname6 Lastname6}{sch,yyy,comp}
% \icmlauthor{Firstname7 Lastname7}{comp}
%\icmlauthor{}{sch}
% \icmlauthor{Firstname8 Lastname8}{sch}
% \icmlauthor{Firstname8 Lastname8}{yyy,comp}
%\icmlauthor{}{sch}
%\icmlauthor{}{sch}
\end{icmlauthorlist}

\icmlaffiliation{kaistai}{KAIST AI, Republic of Korea}
\icmlaffiliation{kaistee}{School of Electrical Engineering, KAIST, Republic of Korea}
% \icmlaffiliation{sch}{School of ZZZ, Institute of WWW, Location, Country}

\icmlcorrespondingauthor{Se-Young Yun}{yunseyoung@kaist.ac.kr}
% \icmlcorrespondingauthor{Firstname2 Lastname2}{first2.last2@www.uk}

% You may provide any keywords that you
% find helpful for describing your paper; these are used to populate
% the "keywords" metadata in the PDF but will not be shown in the document
\icmlkeywords{Machine Learning, ICML}

\vskip 0.3in
]

% this must go after the closing bracket ] following \twocolumn[ ...

% This command actually creates the footnote in the first column
% listing the affiliations and the copyright notice.
% The command takes one argument, which is text to display at the start of the footnote.
% The \icmlEqualContribution command is standard text for equal contribution.
% Remove it (just {}) if you do not need this facility.

\printAffiliationsAndNotice{}  % leave blank if no need to mention equal contribution
% \printAffiliationsAndNotice{\icmlEqualContribution} % otherwise use the standard text.

%%%%%%%%%%%%%%%%%%%%%%%%%%%%%%%%%%%%%%%%%%%%%%%%%%%%%%%%%%%%%%%%%
% 150words
% Replying to workplace emails that are typically long and require politeness is time-consuming and cognitively demanding.
% Replying to lengthy and polite workplace emails is often time-consuming and cognitively demanding.
% takes time to understand and reply
\red{Replying to formal emails is time-consuming and cognitively demanding, as it requires crafting polite phrasing and providing an adequate response to the sender's demands.}
% \red{Replying to formal emails, which often takes time to understand and require polite phrasing, is time-consuming and cognitively demanding.}
Although systems with Large Language Models (LLM) were designed to simplify the email replying process, users still need to provide detailed prompts to obtain the expected output.
Therefore, we proposed and evaluated an \red{LLM-powered question-and-answer (QA)-based approach} for users to reply to emails by answering a set of simple and short questions generated from the incoming email.
We developed a prototype system, \textit{ResQ}, and conducted controlled and field experiments with 12 and \red{8} participants.
Our results demonstrated that \red{the QA-based approach} improves the efficiency of replying to emails and reduces workload while maintaining email quality, compared to a conventional prompt-based approach that requires users to craft appropriate prompts to obtain email drafts.
We discuss how \red{the QA-based approach} influences the email reply process and interpersonal relationship dynamics, as well as the opportunities and challenges associated with using a QA-based approach in AI-mediated communication.

% original
% Replying to lengthy and polite workplace emails is often time-consuming and cognitively demanding.
% Although systems with Large Language Models were designed to simplify the email replying process, users still needed to provide detailed prompts to obtain the expected output.
% Therefore, we proposed and evaluated a question-and-answer-based approach for users to reply to emails by answering a set of simple and short questions generated from the incoming email.
% We developed a prototype system, \textit{ResQ}, and conducted both controlled and field experiments with 12 and 9 participants.
% Our results demonstrated that ResQ improves the efficiency of replying to emails and reduces workload while maintaining email quality compared to a conventional prompt-based approach that requires users to craft appropriate prompts to obtain email drafts.
% We discuss how ResQ influences the email reply process and interpersonal relationship dynamics, as well as the opportunities and challenges associated with using a QA-based approach in AI-mediated communication.

\section{Introduction}

\begin{figure*}
    \centering
    \includegraphics[width=\textwidth]{figures/Introduction.pdf}
    \caption{Showing the novel problem statement applied to traffic prediction use case. Multiple unstructured observations from the past are used to reconstruct a hidden traffic state from which a full traffic state is forecast with a set of query locations. }
    \label{fig:intro}
\end{figure*}

% Was sagen denn die anderen warum Traffic Prediction gut ist? 
Forecasting the traffic in the near future is an important task for city management.
Data from the near past is used to predict future traffic states with spatio-temporal Graph Neural Networks \cite{bui22}.
Accurate prediction provides the opportunity to optimize traffic flow, reduce traffic jams and increase air quality \cite{Po19}.

% Wieso ist Sparsity in allen Dimensionen wichtig.
While traffic prediction relies on the availability of data from traffic sensors, there exists a plethora of reasons why sensors may stop working temporarily, such as simple errors, energy saving, or overloaded communication systems.
Considering small- or medium-sized cities, the coverage of sensors may be low because the sensors are too expensive or not available.
Also, the sensors are typically static and do not adapt to changes in the traffic flow (e.g. caused by a construction site), which motivates moving sensors that for example could be mounted on cars. 
However, both missing and moving sensors introduce sparsity, since measurements may not be available for all locations at all times.
This sparsity must be explicitly addressed in traffic prediction for a realistic application scenario, which is illustrated in figure \ref{fig:intro}.
From one hour of data on Sunday morning, only few observations of the traffic state are available at each timestep.
The number of observations may differ throughout the observed time and the observation itself can be distributed arbitrarily in the city. 
We assume a relatively low number of sensors to account for resource saving and sensor failure in our proposed framework SUSTeR.
The task is to predict the dense traffic state one timestep after the observations at all possible sensor locations.
We study this problem on the traffic dataset Metr-LA and PEMS-BAY to test our assumption that only a fraction of the sensor values would be enough for good predictions.
By modifying an existing traffic dataset, we are able to compare our results from very sparse observations to the bottom line with all information available.
A successful study will provide insights in how sensors in new cities can be reduced before installing them and further mobile sensors would save more resources and are able to adapt to new traffic situations.
We argue that in order to be adaptable to other cities and changes in traffic flows, prior information like the road network should be neglected and just the sparse observations considered.
This comes with the added benefit of making our solution applicable in regions where no openly available road network is maintained or pathways change frequently (e.g. flood areas, animal observations). 


The aforementioned problem is novel and more challenging than the commonly considered traffic prediction problem, since there exist very few observations in each input sample.
Current works for the traffic prediction problem do not consider any missing values. \cite{Li2021, Shao22}
A common method among state of the art approaches is the usage of Graph Neural Networks on graphs that model the sensor network \cite{bui22}.
The values of a sensor are applied to the same graph node for each timestep which prohibits any non-stationary sensors . 
With fixed sensor locations, the resulting sensor network is highly correlated with the road network.
Streets connecting two intersections with sensors should be also an interesting point for correlations in the sensor network.
However, variable observations and high temporal sparsity rates can not be modeled adequately in a static network.
We show in our experiments that the road network has only a small influence on the traffic predictions.

Besides the traffic prediction for future timesteps, some works explore the field of traffic speed imputation \cite{Cini22, Cuza22} where missing sensor values are predicted.
But the amount of missing values is assumed to be at most 80\%, which on average are still over 40 given sensors in each timestep in the Metr-LA dataset with a total of 207 sensors.
We consider up to 99.9\% missing values which are on average 2.4 observations in each timestep that are used as input.
Such high sparsity rates drastically decrease the chance that multiple values are present in one input sample from the same sensor location, which makes it challenging to recognize and learn temporal correlations for each location on its own.

High sparsity rates (>95\%) result in few sensor values, but if a reconstruction of the traffic state would be possible, we question if spatio-temporal graphs require nodes for each sensor.
In SUSTeR we utilize only a small amount of graph nodes for the encoding of information and do not relate such nodes to the sensor network.
We call this the hidden graph (see figure \ref{fig:intro}), which is still able to reconstruct the complete traffic state.
Due to the reduced number of nodes SUSTeR achieves faster runtimes, as shown in the experiments.
This hidden graph is not embedded directly in the spatial domain, which is why the assignment of observations, as well as the querying of the future traffic, is done with an encoder and a decoder, implemented as neural networks.
The decoding from the hidden graph to future values depends on a set of query locations.
Figure \ref{fig:intro} shows the query locations as given from outside and in combination with the reconstructed traffic state the future values are predicted.

To construct the hidden graph we encode observations from each timestep into from multiple graphs, one for each timestep. 
The graphs are created in a residual style and information is added to the node embeddings from the previous timesteps.
We choose this method to incorporate all timesteps equally into the hidden state because the redundant information along the past is non-existing for high sparsity rates.
From the sequence of graphs where our framework inserted the observations step by step we apply STGCN \cite{Yu18}, an algorithm for traffic prediction to find and learn the spatio-temporal correlations on our small number of graph nodes.
The first future timestep of the STGCN is our hidden graph in which the traffic state is reconstructed. 

% Recent work has an implicit embedding of the graph nodes into the spatial domain as the assignment from the sensor to graph node is fixed one by one.
% Because the graph has the same structure as the road network spatio-temporal correlations can be learned between those sensors.
% We reduce the number of nodes and use a non-linear assignment learned data-driven from the observations.

We find in the experiments that SUSTeR outperforms the plain STGCN and modern traffic prediction frameworks like D2STGNN for high sparsity rates $(\geq 99\%)$.
This is equivalent to only $0.2$ to $2.4$ observation for each timestep on average.
SUSTeR uses fewer parameters than the baselines and can train faster and with less training data.
Our main contributions can be summarized as follows:
\begin{itemize}
    \item We introduce a sparse and unstructured variant of the traffic prediction problem with sparsity in all dimensions. The sensors report only a fraction of their values and are arbitrarily distributed in the spatial domain.
    \item We propose SUSTeR, a framework around the STGCN architecture, which maps sparse observations onto a dense hidden graph to reconstruct the complete traffic state.
    Our code is available at github.\footnote{https://github.com/ywoelker/SUSTeR}
    \item We conducts experiments that show that SUSTeR outperforms the baselines in very sparse situations ($\geq 95\%$) and has a competitive performance in low sparsity rates.
    % \item SUSTeR trains a third faster than the next competitor.
\end{itemize}

\section{Related Work}
\label{sec:related_work}

\subsection{Robustness of Audio-Visual Speech Recognition} 

The robustness of AVSR systems has significantly advanced by integrating auditory and visual cues to improve speech recognition, especially in noisy environments. Conventional ASR methods have evolved from relying solely on audio signals \cite{schneider2019wav2vec, gulati2020conformer, baevski2020wav2vec, hsu2021hubert, chen2022wavlm, chiu2022self, radford2023robust} to incorporating visual data from speech videos \citep{makino2019recurrent}.
The multimodal AVSR methods \citep{pan2022leveraging, shi2022learning, seo2023avformer, ma2023auto} have enhanced robustness under audio-corrupted conditions, leveraging visual details like speaker's face or lip movements as well as acoustic features of speech. These advancements have been driven by various approaches, including end-to-end learning frameworks \citep{dupont2000audio, ma2021end, hong2022visual, burchi2023audio} and self-supervised pretraining \citep{ma2021lira, qu2022lipsound2, seo2023avformer, zhu2023vatlm, kim2025multitask}, which focus on audio-visual alignment and the joint training of modalities~\citep{zhang2023self, lian2023av, haliassos2022jointly, haliassos2024braven}.


Furthermore, recent advancements in AVSR highlight the importance of visual understanding alongside audio \citep{dai2024study, kim2024learning}. While initial research primarily targeted audio disturbances \citep{shi2022robust, hu2023hearing, hu2023cross, chen2023leveraging}, latest studies increasingly focus on the visual robustness to address challenges such as real-world audio-visual corruptions~\citep{hong2023watch, wang2024restoring, kim2025multitask} or modality asynchrony~\citep{zhang2024visual, fu2024boosting, li2024unified}. These efforts remark a shift towards a more balanced use of audio and visual modalities. Yet, there has been limited exploration in scaling model capacity or introducing innovative architectural designs, leaving room for further developments in AVSR system that can meticulously balance audio and visual modalities.



\subsection{MoE for Language, Vision, and Speech Models}

Mixture-of-Experts (MoE), first introduced by \citet{jacobs1991adaptive}, is a hybrid structure incorporating multiple sub-models, \ie experts, within a unified framework. The essence of sparsely-gated MoE \cite{shazeer2017outrageously, lepikhin2021gshard, dai2022stablemoe} lies in its routing mechanism where a learned router activates only a subset of experts for processing each token, significantly enhancing computational efficiency. Initially applied within LLMs using Transformer blocks, this structure has enabled unprecedented scalability \cite{fedus2022switch, zoph2022st, jiang2024mixtral, guo2025deepseek} and has been progressively adopted in multimodal models, especially in large vision-language models (LVLMs) \cite{mustafa2022multimodal, lin2024moellava, mckinzie2025mm1}.
Among these multimodal MoEs, \citet{zhu2022uni, shen2023scaling, li2023pace, li2024uni} and \citet{lee2025moai} share the similar philosophy to ours, assigning specific roles to each expert and decoupling them based on distinct modalities or tasks. These models design an expert to focus on specialized segments of input and enhance the targeted processing.

Beyond its applications in LLMs and LVLMs, the MoE framework has also been applied for speech processing \cite{you2021speechmoe, you2022speechmoe2, hu2023mixture, wang2023language}, where it has shown remarkable effectiveness in multilingual and code-switching ASR tasks. In addition, MoE has been employed in audio-visual models \cite{cheng2024mixtures, wu2024robust}, although they primarily focus on general video processing and not specifically on human speech videos. These approaches leverage MoE to model interactions between audio and visual tokens without directly processing multimodal tokens.
Our research advances the application of the MoE framework to AVSR by designing a modality-aware hierarchical gating mechanism, which categorizes experts into audio and visual groups and effectively dispatches multimodal tokens to each expert group. 
This tailored design enhances the adaptability in managing audio-visual speech inputs, which often vary in complexity due to diverse noise conditions.

\section{Preliminaries}
\label{sec:method}

\subsection{Sparsely-gated MoE}
\label{subsec:sparse_moe}

In AVSR systems, the multimodal encoder processes a sequence of audio $\rva = [a_1, a_2, \cdots]$ and video $\rvv = [v_1, v_2, \cdots]$ data into combined audio-visual embeddings $\text{Enc}(\rva, \rvv)$. These embeddings are utilized by the decoder to predict subsequent text tokens, where the predicted token is given by $\textsl{text}_{t+1} = \text{Dec}(\text{Enc}(\rva, \rvv), \textsl{text}_t)$. Within the Transformer layer, $x_t$ is the intermediate representation of the token $\textsl{text}_t$, derived by cross-attending to the combined audio-visual embeddings from $\rva$ and $\rvv$ (see Figure\,\ref{fig:overview}).

The integration of a sparsely-gated MoE framework \citep{shazeer2017outrageously, lepikhin2021gshard} leverages $E$ experts to scale model capacity. Each token representation is routed to a selected subset of these experts through a learned gating mechanism. 
Specifically, the routing function $h(x) = W_r \cdot x$ assigns weights for each token, and the weight for expert $i$ is computed using a softmax function:
\begin{equation}
\label{eq:router_weight}
    p_i(x) = \frac{\exp(h_i(x))}{\sum_{j=1}^{E} \exp(h_j(x))},
\end{equation}
and the output $y$ is the aggregated result of computations from the top-$k$ selected experts:
\begin{equation}
y = \sum_{i \in \text{top}k(E)} \tilde{p}_i(x) E_i(x),
\end{equation}
where $\tilde{p}$ is the normalization of top-$k$ probabilities.
Note that each expert follows the same structure as a feed-forward network\,(FFN) in a Transformer block. Figure\,\ref{fig:overview} presents the overall MoE architecture and its token routing.



%%%%%%%%%%%%%%%%%%%%
\begin{figure}[!t]
    \centering
    \vspace*{-5pt}
    \includegraphics[width=\linewidth]{tex_figure/figure2.pdf}
    \vspace*{-20pt}
    \caption{Overview of sparsely-gated MoE for AVSR. A select subset of experts are activated for each token representation ($x_t$).
    }
    \label{fig:overview}
    \vspace{-5pt}
\end{figure}
%%%%%%%%%%%%%%%%%%%%


\vspace*{-8pt}
\paragraph{Load balancing.}
To mitigate the load imbalance issue commonly observed in the top-$k$ expert selection strategy, a load balancing loss has been implemented to encourage the balanced token load across all experts. Specifically, we use a differentiable load balancing loss~\citep{fedus2022switch}:
\begin{equation}
    L_B = E \cdot \sum_{i=1}^E f_i \cdot P_i,
\end{equation}
where $f_i$ denotes the frequency of expert $i$ being selected as top-1, averaged over all tokens within a batch $\mathcal{B}$,
\begin{equation}
\label{eq:expert_frequency}
    f_i = \frac{1}{T} \sum_{x\in\mathcal{B}} \mathbbm{1} \{\arg\max p(x) = i\}
\end{equation}
and $P_i$ is the average assigned probability for expert $i$,
\begin{equation}
\label{eq:expert_probability}
    P_i = \frac{1}{T} \sum_{x\in\mathcal{B}} p_i(x)
\end{equation}
with $T$ representing the total number of tokens.

An additional router z-loss~\citep{zoph2022st} is employed to stabilize the routing mechanism:
\begin{equation}
    L_Z = \frac{1}{T}\sum_{x\in\mathcal{B}} \Bigg(\log \sum_{i=1}^E \exp(h_i(x)) \Bigg)^2.
\end{equation}
This sparse MoE structure ensures that token processing is efficiently managed across multiple experts, utilizing lower compute relative to its expansive capacity.


%%%%%%%%%%%%%%%%%%%%%
\begin{figure*}[!t]
    \centering
    \includegraphics[width=0.95\linewidth]{tex_figure/figure3.pdf}
    \vspace{-5pt}
    \caption{MoE-based routing strategies for AVSR. 
    (a) A conventional MoE approach where a learned router selects the top-2 experts for each token, enforcing the balanced expert load. 
    (b) Experts are explicitly divided into audio and visual groups, with manual activation based on the input modality.
    (c) \ourmodel introduces an inter-modal router that can dynamically assign weights to modality-specific expert groups, followed by intra-modal routers that select the top-1 expert within each group. The inter-modal router is trained by the load biasing loss that guides the expert group specialization.
    }
    \label{fig:routing}
\end{figure*}
%%%%%%%%%%%%%%%%%%%%%
\subsection{Expert Group Specialization}
\label{subsec:hard_routing}


To enhance expert management within the AVSR system, a \textit{hard routing} technique can be used for expert group specialization. This approach is inspired by several practices in visual-language MoE models \citep{zhu2022uni, li2023pace, shen2023scaling, lee2025moai} where the role of experts is strictly defined by the input modality, eliminating the need for a trained router.

\vspace*{-8pt}
\paragraph{Hard routing.}
Our hard routing enforces modality-specific activation of expert groups: audio data activate only audio experts, and video data activate only visual experts.
This segregation encourages the independent development of specialized expert groups. As suggested in V/T-MoE \citep{shen2023scaling}, once the group is activated, we use an intra-modal router for the modality-specific experts.

Figure\,\ref{fig:routing}(b) visualizes the hard routing mechanism with audio and visual expert groups.
During training, audio or video sequence is randomly dropped, leading to subsets $\mathcal{A}$ and $\mathcal{V}$ within a batch $\mathcal{B}$, consisting of audio-only or video-only sequences, respectively. A token representation $x_t \in \mathcal{A}$ indicates that the cross-attention module processes the input $\textsl{text}_t$ with $\text{Enc}(\rva, \vzero)$---where the visual component is zeroed out---and vice versa for $x_t \in \mathcal{V}$.
For these, we utilize two distinct intra-modal routing networks, $W_r^A$ and $W_r^V$:
\begin{align}
\begin{split}
    h^A(x) &= W_r^A \cdot x \quad \text{for } x \in \mathcal{A}, \\
    h^V(x) &= W_r^V \cdot x \quad \text{for } x \in \mathcal{V}.
\end{split}
\end{align}
These routers calculate the weights $p^{\{A,V\}}(x)$ as in Eq.\,(\ref{eq:router_weight}) within their respective expert group, either $\{E^A\}$ for audio or $\{E^V\}$ for visual. The output for each token is then
\begin{equation}
    y = \begin{cases}
        \sum_{\text{top}k(E^A)} \tilde{p}_i^A(x) E^A_i(x) & \text{if } x \in \mathcal{A}, \\
        \sum_{\text{top}k(E^V)} \tilde{p}_i^V(x) E^V_i(x) & \text{if } x \in \mathcal{V}. \\ 
    \end{cases} 
\end{equation}
For audio-visual sequences, outputs from both groups are averaged, with the top-($k/2$) experts from each group contributing to ensure balanced processing.

\section{MoHAVE: Mixture of Hierarchical Audio-Visual Experts}
\label{sec:mohave}

Despite the benefits of hard routing in specializing expert groups according to decoupled input modality, it lacks the flexibility to autonomously determine the group utilization.
In practice, the optimal balance between audio and visual groups varies depending on ambient conditions such as noise type and intensity (more detailed in Figure\,\ref{fig:expert_load}(b)).
To address this limitation and enhance the model’s adaptability, we introduce an adaptive routing mechanism with hierarchical gating \cite{jordan1994hierarchical}, providing a more dynamic approach to manage multimodal inputs.

Our hierarchical model, \ourmodel, features a two-layer routing mechanism: \textit{inter-modal} and \textit{intra-modal} routers, where the inter-modal router learns to assign appropriate weights to each modality-specific group. Figure\,\ref{fig:routing}(c) presents the overview of \ourmodel's routing strategy.

\subsection{Hierarchical Gating Structure}
\label{subsec:hierarchical_gating}

The inter-modal router orchestrates the initial token distributions across expert groups. It generates logits through $u(x) = V_r \cdot x$ and determines the dispatch weights for group $i$ with $q_i(x) = \text{softmax}(u(x))_i$. This router dynamically selects the top-$m$ expert groups, and within those, the intra-modal routers select the top-$k$ experts, thus involving $m \times k$ experts in total. For practical efficiency, we set $k=1$ for each group and modify the intra-modal router's probability distribution to a Kronecker delta, $\tilde{p}_{ij} \rightarrow \delta_{j,\text{argmax}(p_i)}$.
The output from this layer integrates these selections:
\begin{align}
  y &= \!\!\!\!\sum_{i \in \text{top}m(G)} \!\!\tilde{q}_i(x)\!\! \sum_{j \in \text{top}k(E_i)} \!\!\tilde{p}_{ij} E_{ij}(x) \\
    &\rightarrow \!\!\!\!\sum_{i \in \text{top}m(G)} \!\!\tilde{q}_i(x)E_{ij}(x), ~~\text{where } j\!=\!\arg\max(p_i) 
\end{align}
where $\tilde{q}$ is the normalization of $q$ across top-$m$ probabilities, $G$ is the number of expert groups, and $E_{ij}(x)$ denotes the output from the $j$-th expert in the $i$-th group.

Focusing on audio-visual applications, we designate two expert groups: audio and visual. Each token $x$, regardless of its modality, is processed by the intra-modal routing networks of both groups, \ie $[h^A(x), h^V(x)] = [W_r^A, W_r^V] \cdot x$. The frequencies $f^{\{A,V\}}$ and probabilities $P^{\{A,V\}}$ for selecting experts are computed in the same manner as Eq.\,(\ref{eq:expert_frequency})--(\ref{eq:expert_probability}) for all $x \in \mathcal{B}$. Thus, the load balancing loss can be computed for both groups:
\vspace*{-5pt}
\begin{equation}
L_B = E^A \cdot \sum_{j=1}^{E^A} f_j^A \cdot P_j^A + E^V \cdot \sum_{j=1}^{E^V} f_j^V \cdot P_j^V
\end{equation}
where $f_j^A$ and $f_j^V$ denote the frequencies of token assignments to audio and visual experts, respectively.


\subsection{Group-level Load Biasing Loss}
\label{subsec:load_biasing_loss}

To autonomously manage the expert group loads without manual (de-)activation as hard routing, we introduce a biasing loss that directs the load towards a certain group. This load biasing loss encourages the inter-modal router to assign higher weights to $E^A$ experts for audio sequences and to $E^V$ experts for video sequences.
For audio sequences within a sub-batch $\mathcal{A}$, the frequency and average probability of selecting the $i$-th group is calculated as follows:
\begin{equation}
    g^A_i = \frac{1}{|\mathcal{A}|} \sum_{x\in\mathcal{A}} \mathbbm{1} \{\arg\max q(x) = i\},
\end{equation}
\begin{equation}
    Q^A_i = \frac{1}{|\mathcal{A}|} \sum_{x\in\mathcal{A}} q_i(x).
\end{equation}
Similar calculations for $g^V_i$ and $Q^V_i$ are made for video sequences $x\in\mathcal{V}$. We designate the first group as audio experts and the second group as video experts, then the load biasing loss is defined as:
\begin{equation}
    L_S = L_S^A + L_S^V = (1 - g^A_1 \cdot Q^A_1) + (1 - g^V_2 \cdot Q^V_2).
\end{equation}
Note that $L_S^A$ and $L_S^V$ are only computed over $x\in\mathcal{A}$ and $x\in\mathcal{V}$, respectively. 

For sequences containing both audio and video, we exclude them from the load biasing loss calculation but incorporate them into the load balancing.
Although these tokens are uniformly dispatched on average, the inter-modal router finds the optimal strategy for each token based on its characteristics.
Empirically, we find that \ourmodel learns to assign greater weight to the visual expert group for audio-visual inputs under high auditory noise, and to the audio expert group for less noisy inputs (see \S\ref{subsec:analysis_load} for details), demonstrating the model’s adaptability under various noisy conditions.

The overall loss function, combining the cross-entropy\,(CE) for token prediction, is formulated as:
\begin{equation}
L_{tot} = L_{CE} + c_B L_B + c_S L_S + c_Z L_Z.
\end{equation}
Here, $c_B$ and $c_Z$ are set to 1e-2 and 1e-3, respectively, in line with \citep{fedus2022switch, zoph2022st}, and $c_S$ is also set at 1e-2.

\section{Experiments and Results}
\label{sec:experiments}

\subsection{Implementation Details}
\label{sec:implementation}

\begin{table*}[t]
\centering
\footnotesize

\begin{subtable}{\textwidth}
\centering
\begin{adjustbox}{width=\textwidth, center}
\begin{tabular}{llccccccccccc}
\toprule
\multirow{2}{*}{Model} & \multirow{2}{*}{Modality} & \multicolumn{11}{c}{mAP (\%)} \\
\cmidrule(lr){3-13}
 & & 0.20s & 0.40s & 0.60s & 0.80s & 1.00s & 1.20s & 1.40s & 1.60s & 1.80s & 2.00s & Avg \\
\midrule
\multirow{4}{*}{\centering Transformer} 
 & A & 72.2 & 65.2 & 60.3 & 56.8 & 54.4 & 53.1 & 52.4 & 52.0 & 51.6 & 51.4  & 56.9 $\pm$ 0.05 
 \\
 & V & 52.0 & 51.7 & 51.6 & 51.3 & 51.1 & 50.9 & 50.8 & 50.5 & 50.3 & 50.1  & 51.0 $\pm$ 0.08 
 \\
 & A+V & 73.8 & 66.9 & 62.1 & 58.5 & 56.3 & 55.0 & 54.1 & 53.7 & 53.3 & 53.0 & 58.7 $\pm$ 0.13
 \\
 & A+V\textsuperscript{P} & 73.4 & 66.8 & 61.8 & 58.3 & 56.1 & 54.8 & 54.1 & 53.5 & 53.2 & 52.7  & 58.5 $\pm$ 0.26 
 \\
\midrule
\multirow{4}{*}{\centering GRU} 
 & A & 71.5 & 65.0 & 60.1 & 57.0 & 55.0 & 53.8 & 52.9 & 52.2 & 51.5 & 50.9  & 57.0 $\pm$ 0.30 
 \\
 & V & 53.0 & 52.7 & 52.4 & 52.0 & 51.7 & 51.6 & 51.2 & 51.1 & 50.8 & 50.6 & 51.7 $\pm$ 0.29 
 \\
 & A+V & 73.5 & 68.1 & 63.7 & 60.7 & 59.1 & 58.1 & 57.2 & 56.3 & 55.4 & 54.4 & 60.6 $\pm$ 0.17 
 \\
 & A+V\textsuperscript{P} & 70.8 & 64.9 & 60.1 & 56.9 & 55.0 & 53.8 & 53.0 & 52.4 & 51.8 & 51.4 & 57.0 $\pm$ 0.29 
 \\
\midrule
\multirow{4}{*}{\centering Mamba} 
 & A & 67.5 & 62.2 & 58.4 & 55.7 & 54.0 & 52.9 & 52.0 & 51.1 & 50.2 & 49.6 & 55.4 $\pm$ 0.62 
 \\
 & V & 52.2 & 51.8 & 51.5 & 51.1 & 50.9 & 50.7 & 50.5 & 50.4 & 50.0 & 49.7 & 50.9 $\pm$ 0.21 
 \\
 & A+V & 71.8 & 65.4 & 60.5 & 57.1 & 55.0 & 53.9 & 53.5 & 53.1 & 52.3 & 51.8 & 57.4 $\pm$ 0.26 
 \\
 & A+V\textsuperscript{P} & 68.9 & 63.2 & 59.1 & 56.0 & 54.0 & 52.7 & 51.8 & 51.4 & 50.7 & 50.1 & 55.8 $\pm$ 0.43 
 \\
\bottomrule
\end{tabular}
\end{adjustbox}
\label{tab:main_result_a}
\caption*{(a) Results on EasyCom} %
\end{subtable}


\begin{subtable}{\textwidth}
\centering
\begin{adjustbox}{width=\textwidth, center}
\begin{tabular}{llccccccccccc}
\toprule
\multirow{2}{*}{Model} & \multirow{2}{*}{Modality} & \multicolumn{11}{c}{mAP (\%)} \\
\cmidrule(lr){3-13}
 & & 0.20s & 0.40s & 0.60s & 0.80s & 1.00s & 1.20s & 1.40s & 1.60s & 1.80s & 2.00s & Avg \\
\midrule
\multirow{4}{*}{\centering Transformer} 
 & A & 78.8 & 74.9 & 71.8 & 69.7 & 68.1 & 67.0 & 66.3 & 65.7 & 65.1 & 64.7 & 69.2 $\pm$ 0.03
 \\
 & V & 58.7 & 58.5 & 58.4 & 58.2 & 58.1 & 58.0 & 57.9 & 57.8 & 57.7 & 57.7 & 58.0 $\pm$ 0.27
 \\
 & A+V & 78.1 & 74.3 & 71.5 & 69.4 & 68.0 & 67.0 & 66.3 & 65.7 & 65.3 & 64.9 & 69.0 $\pm$ 0.24
 \\
 & A+V\textsuperscript{P} & 78.4 & 74.5 & 71.5 & 69.4 & 67.9 & 66.7 & 65.9 & 65.4 & 65.0 & 64.5 & 68.9 $\pm$ 0.18
\\
\midrule
\multirow{4}{*}{\centering GRU} 
 & A & 78.6 & 74.8 & 71.8 & 69.6 & 68.1 & 66.9 & 66.2 & 65.6 & 65.2 & 64.8  & 69.2 $\pm$ 0.25
 \\
 & V & 58.6 & 58.3 & 58.1 & 57.9 & 57.8 & 57.8 & 57.7 & 57.6 & 57.5 & 57.5  & 57.9 $\pm$ 0.61
 \\
 & A+V & 76.4 & 73.0 & 70.4 & 68.5 & 67.1 & 66.3 & 65.6 & 65.2 & 64.7 & 64.4  & 68.2 $\pm$ 0.42
 \\
 & A+V\textsuperscript{P} & 76.9 & 73.4 & 70.6 & 68.6 & 67.3 & 66.3 & 65.6 & 65.1 & 64.7 & 64.4  & 68.3 $\pm$ 0.18
 \\
\midrule
\multirow{4}{*}{\centering Mamba} 
 & A & 77.4 & 73.6 & 70.5 & 68.5 & 66.9 & 65.8 & 65.0 & 64.3 & 63.9 & 63.5  & 67.9 $\pm$ 0.37
 \\
 & V & 58.2 & 58.1 & 57.9 & 57.8 & 57.6 & 57.5 & 57.5 & 57.4 & 57.4 & 57.3 & 57.7 $\pm$ 0.28
 \\
 & A+V & 76.0 & 72.5 & 69.8 & 67.9 & 66.6 & 65.6 & 64.8 & 64.2 & 63.8 & 63.5  & 67.5 $\pm$ 0.18
 \\
 & A+V\textsuperscript{P} & 74.1 & 70.8 & 68.1 & 66.2 & 64.8 & 63.9 & 63.2 & 62.7 & 62.3 & 62.0 & 65.8 $\pm$ 0.23
 \\
\bottomrule
\end{tabular}
\end{adjustbox}
\label{tab:main_result_b}
\caption*{(b) Results on Ego4D} 
\end{subtable}

\caption{Mean average precision (mAP) scores on (a)~EasyCom and (b)~Ego4D at time steps 0.20\,s to 2.00\,s for Transformer, GRU, and Mamba architectures under Audio~(A), Visual~(V), and Audio+Visual~(A+V) modalities. Models with \textsuperscript{P} are pretrained on YT-Conversation. Per-timestep values come from a single random seed, while ``Avg'' shows mean $\pm$ SE over five seeds (see \Cref{app:stat_results} for full multi-seed results). Using both A and V yields the best performance overall.}




\label{tab:main_result}
\end{table*}














\paragraph{Datasets.}
For the robust AVSR benchmark, we utilize the LRS3 dataset~\citep{afouras2018lrs3}, which consists of 433 hours of audio-visual speech from 5,000+ speakers. Following the experimental setup of \citet{shi2022robust}, we extract audio noise samples from the MUSAN~\citep{snyder2015musan} dataset, targeting different noise types such as \textit{babble}, \textit{music}, and \textit{natural} noises, along with \textit{speech} noise from LRS3. These noises are randomly augmented into the audio data, corrupting 25\% of the training set with a signal-to-noise ratio (SNR) sampled from $\mathcal{N}(0, 5)$. We measure performance using the word error rate (WER), primarily under noisy conditions with SNRs of \{$-$10, $-$5, 0, 5, 10\}\,dB, specifically N-WER\,\citep{kim2024learning} which highlights the significance of visual cues in noise-corrupted environments.


For multilingual evaluations, the MuAViC dataset \cite{anwar2023muavic} is used, featuring 1,200 hours of audio-visual content from 8,000+ speakers across 9 languages, sourced from LRS3-TED\,\cite{afouras2018lrs3} and mTEDx\,\cite{elizabeth2021multilingual}. We use 8 languages (excluding English) for multilingual AVSR and 6 languages for X-to-English audio-visual speech-to-text translation (AVS2TT) tasks. We assess the models using WER for transcription and the BLEU score\,\cite{papineni2002bleu} for translation.

\vspace*{-8pt}
\paragraph{\ourmodel model description.}

Our \ourmodel framework is developed in two configurations: \textsc{Base} and \textsc{Large}. The \textsc{Base} model consists of 12 Transformer~\citep{vaswani2017attention} encoder layers and 6 decoder layers, while the \textsc{Large} model incorporates 24 encoder layers and 9 decoder layers. Both models’ audio-visual encoders are derived from the AV-HuBERT-\textsc{Base}/-\textsc{Large} models, pretrained on a noise-augmented corpus of LRS3 \citep{afouras2018lrs3} + VoxCeleb2 \citep{chung2018voxceleb2}. Our MoE implementation activates top-2 out of 8 experts in every FFN layer within the decoder~\citep{jiang2024mixtral}, while the hierarchical architecture engages the top-1 expert from each audio and visual group. To facilitate the expert group specialization, load biasing is used with audio or video randomly dropped in 25\% probability.

\begin{table}[!t]
\centering
\caption{Comparison of \methodname~with related backdoor attacks.}
\label{table:sota_comparison}
\begin{adjustbox}{max width=\linewidth}
\begin{threeparttable}
\begin{tabular}{c|c|c|c|c|}
\cline{2-5}
 & \textbf{\begin{tabular}[c]{@{}c@{}}Provides\\ Concealed\\ Backdoor\\ Feature?\end{tabular}} & \textbf{\begin{tabular}[c]{@{}c@{}}Without\\ Modifying\\ Training \\ Process?\end{tabular}} & \textbf{\begin{tabular}[c]{@{}c@{}}Requires Victim\\Model Access\\for Data\\ Poisoining?\end{tabular}} & \textbf{\begin{tabular}[c]{@{}c@{}}Camouflaging\\ Without\\ Auxiliary\\ Data?\end{tabular}} \\ \hline
\multicolumn{1}{|c|}{TrojanNN~\cite{trojannn}} & \redcross & \greentick & $\square$ & \multirow{13}{*}{\begin{tabular}[c]{@{}c@{}}Not\\ Applicable\end{tabular}} \\ \cline{1-4}
\multicolumn{1}{|c|}{SIG~\cite{sig}} & \redcross & \greentick & \textcolor{OliveGreen}{No Access} &  \\ \cline{1-4}
\multicolumn{1}{|c|}{BadNets~\cite{badnets}} & \redcross & \greentick & \textcolor{OliveGreen}{No Access} &  \\ \cline{1-4}
\multicolumn{1}{|c|}{ReFool~\cite{refool}} & \redcross & \greentick & \textcolor{OliveGreen}{No Access} &  \\ \cline{1-4}
\multicolumn{1}{|c|}{Input-Aware~\cite{inputaware}} & \redcross & \redcross & $\square$ &  \\ \cline{1-4}
\multicolumn{1}{|c|}{Blind~\cite{blind}} & \redcross & \redcross$^{\star}$ & \textcolor{OliveGreen}{No Access} &  \\ \cline{1-4}
\multicolumn{1}{|c|}{LIRA~\cite{lira}} & \redcross & \redcross & $\square$ &  \\ \cline{1-4}
\multicolumn{1}{|c|}{SSBA~\cite{ssba}} & \redcross & \greentick & \textcolor{OliveGreen}{No Access} &  \\ \cline{1-4}
\multicolumn{1}{|c|}{WaNet~\cite{wanet}} & \redcross & \greentick & \textcolor{OliveGreen}{No Access} &  \\ \cline{1-4}
\multicolumn{1}{|c|}{LF~\cite{lf}} & \redcross & \greentick & $\square$ &  \\ \cline{1-4}
\multicolumn{1}{|c|}{FTrojan~\cite{ftrojan}} & \redcross & \greentick & \textcolor{OliveGreen}{No Access} &  \\ \cline{1-4}
\multicolumn{1}{|c|}{BppAttack~\cite{bppattack}} & \redcross & \greentick & \textcolor{OliveGreen}{No Access} &  \\ \cline{1-4}
\multicolumn{1}{|c|}{PoisonInk~\cite{poisonink}} & \redcross & \greentick & \textcolor{OliveGreen}{No Access} &  \\ \hline \hline
\multicolumn{1}{|c|}{Di et al.~\cite{DBLP:conf/nips/DiDA0S23}} & \greentick & \greentick & $\square$ & \greentick \\ \hline
\multicolumn{1}{|c|}{Liu et al.~\cite{DBLP:conf/aaai/LiuWHM24}} & \greentick & \greentick & $\blacksquare^{\dagger}$ & \greentick \\ \hline
\multicolumn{1}{|c|}{UBA-Inf~\cite{uba}} & \greentick & \greentick & $\textcolor{gray!60}{\blacksquare}^{\ddagger}$ & \redcross \\ \hline
\multicolumn{1}{|c|}{\textbf{\methodname~[Ours]}} & \greentick & \greentick & \textcolor{OliveGreen}{No Access} & \greentick \\ \hline
\end{tabular}
\begin{tablenotes}
    \item $\square$: Represents white-box model access.
    \item $\blacksquare$: Represents black-box model access.
    \item $\textcolor{gray!60}{\blacksquare}$: Represents substitute model access.
    \item $\star$: Changes the training code to maliciously modify loss value.
    \item $\dagger$: Non-data poisoning attack mode requires Black-Box model access to synthesize samples for a successful attack.
    \item $\ddagger$: Substitute model is trained on auxiliary data.
\end{tablenotes}
\end{threeparttable}
\end{adjustbox}
\end{table}

As summarized in Table\,\ref{tab:main_result}, the \textsc{Base} model of \ourmodel holds 359M parameters, and the \textsc{Large} configuration contains 1B. Despite its larger model capacity, due to the sparse activation of these parameters, only about half are active during token processing, amounting to 189M for \textsc{Base} and 553M for \textsc{Large} model. This setup ensures computational efficiency which is comparable to the smaller AV-HuBERT counterparts. For more details on the model description and computation cost, refer to Appendix\,\ref{appx:model_details} and \ref{appx:computation_cost}.

\subsection{Robust AVSR Benchmark Results}
\label{sec:results}

%%%%%%%%%%%%%%%%%%%%%%%%%%%%%%%%%%%%
\begin{figure*}[!t]
    \centering
    \includegraphics[width=\linewidth]{tex_figure/figure4_new.pdf}
    \vspace{-22pt}
    \caption{(a) Expert load distribution in \ourmodel according to input modalities, with expert selection frequencies weighted by the inter-modal router’s output probability. (b) Performance of the hard routing strategy under different weight assignments to audio expert group. The visual expert group is weighted by $p^V = 1-p^A$.
    }
    \label{fig:expert_load}
\end{figure*}

\begin{figure*}[!t]
    \centering
    \vspace{-3pt}
    \includegraphics[width=\linewidth]{tex_figure/figure5_new.pdf}
    \vspace{-22pt}
    \caption{Expert load distribution in \ourmodel for the audio group (solid bars) and visual group (dashed bars) across noisy audio-visual sequences under babble (left) and natural (right) noise. Full layer-wise results are provided in Appendix\,\ref{appx:expert_group_usage}.
    }
    \label{fig:expert_load_noise}
\end{figure*}
%%%%%%%%%%%%%%%%%%%%%%%%%%%%%%%%%%%%


Table\,\ref{tab:main_result} presents \ourmodel's robust performance on the AVSR benchmark under diverse noisy conditions, demonstrating exceptional robustness across different noise types and SNR levels: \textbf{N-WER of 5.8\% for \textsc{Base}} and \textbf{4.5\% for \textsc{Large}}. This substantiates the model's potential for effectively scaling AVSR systems without incurring significant computational costs.
%
The results also reveal that simple MoE implementations (AV-MoE in Table\,\ref{tab:main_result}), despite their larger capacity, fail to achieve remarkable gains. Instead, the key improvement stems from leveraging expert group specialization, as evidenced by the effectiveness of hard routing. By splitting experts into audio and visual groups, MoE is enabled with more targeted and effective processing of multimodal inputs, leading to substantial performance enhancements.
Without our load biasing loss, \ourmodel loses its group specialization capability, comparable to the performance of simple AV-MoEs.

Building upon this expert group strategy, \ourmodel enhances its adaptability through dynamically determining the usage of each group. This adaptive routing approach allows the model to flexibly adjust to varying audio-visual scenarios, contributing to consistent gains in robustness across the benchmark, as detailed in Table\,\ref{tab:main_result}.
An in-depth analysis of this hierarchical gating approach and its impact on token dispatching is discussed in \S\ref{subsec:analysis_load}, underscoring its critical role in advancing MoHAVE’s capabilities in various AVSR environments.


\vspace*{-12pt}
\paragraph{Comparison with state-of-the-art AVSR methods.}
Table\,\ref{tab:sota_comparison} shows how our \ourmodel decoder, when integrated with a range of audio-visual encoders, consistently improves performance compared to existing state-of-the-art methods. While BRAVEn \citep{haliassos2024braven} typically struggles in noisy multimodal scenarios---due to its original design focused on handling unimodal tasks---\ourmodel boosts its accuracy. Other recent approaches have advanced by utilizing the noise-augmented AVSR encoder\,\citep{shi2022learning}, such as additionally learning temporal dynamics with cross-modal attention modules\,(CMA) \citep{kim2024learning}. When paired with an AV-HuBERT encoder and trained through the CMA's self-supervised learning, \ourmodel achieves a remarkable performance: \textbf{N-WER of 4.2\%}.

\subsection{Expert and Group Load Analysis}
\label{subsec:analysis_load}

\begin{table*}[!t]
    \centering
    \small
    \vspace{-5pt}
    \caption{Multilingual audio-visual speech task performance, with non-English speech recognition (WER) and X-En speech-to-text translation (BLEU score), in a noisy environment with multilingual babble noise (SNR\,$=$\,0). $^\dagger$Results obtained from \citet{han-etal-2024-xlavs}. $^\ddagger$Re-implemented using the pretrained model from \citet{choi2024av2av}.}
    \label{tab:muavic}
    \vspace{5pt}
    % \addtolength{\tabcolsep}{-1pt}
    \resizebox{\textwidth}{!}{
    \begin{tabular}{llccccccccc}
    \toprule
    && \multicolumn{8}{c}{Source} & \\
    \cmidrule{3-10}
    Model & Pretrain data & Ar & De & El & Es & Fr & It & Pt & Ru & Avg \\
    \midrule
    \multicolumn{11}{c}{\textbf{\textit{Speech Recognition, Test WER $\downarrow$}}} \\
    Whisper large-v2\,\cite{radford2023robust} & \textit{680k hrs, 100+ langs} & 197.9 & 244.4 & 113.3 & 116.3 & 172.3 & 172.4 & 223.6 & 126.2 & 170.8 \\
    u-HuBERT$^\dagger$\,\cite{hsu2022u} & \textit{1.7k hrs, English} & 102.3 & 73.2 & 69.7 & 43.7 & 43.2 & 48.5 & 47.6 & 67.0 & 61.9 \\
    mAV-HuBERT\,\cite{anwar2023muavic} & \textit{1.7k hrs, English} & \textbf{82.2} & 66.9 & 62.2 & 40.7 & 39.0 & 44.3 & 43.1 & 43.1 & 52.7 \\
    XLS-R 300M$^\dagger$\,\cite{babu2022xls} & \textit{1.2k hrs, 9 langs} & 97.3 & 69.8 & 74.8 & 47.6 & 37.1 & 47.9 & 54.4 & 59.8 & 61.1 \\
    XLAVS-R 300M\,\cite{han-etal-2024-xlavs} & \textit{8.3k hrs, 100+ langs} & 91.9 & 53.5 & 49.6 & 28.8 & 29.3 & 32.2 & 32.5 & 46.1 & 45.5 \\
    XLAVS-R 2B\,\cite{han-etal-2024-xlavs} & \textit{1.2k hrs, 9 langs} & 93.5 & 58.5 & 38.6 & 23.9 & 23.5 & 24.6 & 26.1 & 41.0 & 41.2 \\
    \cdashlinelr{1-11}
    mAV-HuBERT$^\ddagger$ & \textit{7.0k hrs, 100+ langs} & 88.7 & 51.3 & 37.2 & 20.7 & 22.6 & 24.2 & 23.8 & 42.4 & 38.9 \\
    \ourmodel-\textsc{Large} (\textbf{ours}) & \textit{7.0k hrs, 100+ langs} & 92.9 & \textbf{47.3} & \textbf{35.3} & \textbf{18.7} & \textbf{21.2} & \textbf{21.6} & \textbf{21.9} & \textbf{40.6} & \textbf{37.4} \\
    \midrule
    \multicolumn{11}{c}{\textbf{\textit{X-En Speech-to-Text Translation, Test BLEU $\uparrow$}}} \\
    Whisper large-v2\,\cite{radford2023robust} & \textit{680k hrs, 100+ langs} & - & - & 0.1 & 0.4 & 0.7 & 0.1 & 0.1 & 0.2 & 0.3 \\
    mAV-HuBERT\,\cite{anwar2023muavic} & \textit{1.7k hrs, English} & - & - & 4.2 & 12.8 & 15.0 & 12.5 & 14.8 & 4.6 & 10.7 \\
    XLAVS-R 300M\,\cite{han-etal-2024-xlavs} & \textit{8.3k hrs, 100+ langs} & - & - & 13.2 & 17.4 & 23.8 & 18.7 & 21.8 & 9.4 & 17.4 \\
    XLAVS-R 2B\,\cite{han-etal-2024-xlavs} & \textit{1.2k hrs, 9 langs} & - & - & \textbf{15.7} & 19.2 & 24.6 & 20.1 & 22.3 & \textbf{10.4} & 18.7 \\
    \cdashlinelr{1-11}
    mAV-HuBERT$^\ddagger$ & \textit{7.0k hrs, 100+ langs} & - & - & 8.9 & 21.5 & 26.5 & 21.2 & 24.2 & 8.8 & 18.5 \\
    \ourmodel-\textsc{Large} (\textbf{ours}) & \textit{7.0k hrs, 100+ langs} & - & - & 11.4 & \textbf{22.3} & \textbf{27.1} & \textbf{22.1} & \textbf{25.1} & 9.2 & \textbf{19.5} \\
    \bottomrule
    \end{tabular}
    }
    \vspace{-5pt}
\end{table*}

\paragraph{\ourmodel's expert load distribution.}
Figure\,\ref{fig:expert_load}(a) illustrates the expert load distribution of \ourmodel according to input types: audio-visual, audio-only, and video-only sequences. For audio-visual inputs, all experts from both the audio and visual groups are selected at similar frequencies, with some layer-dependent variations. In contrast, when processing audio-only sequences, the model predominantly activates the audio expert group, while for video-only sequences, the visual expert group is mainly utilized. This distribution validates the effectiveness of our load biasing loss in guiding the inter-modal router to assign appropriate weights based on input modality.

\vspace*{-10pt}
\paragraph{Expert group utilization in noisy AVSR.}
To analyze the effectiveness of hierarchical gating in AVSR, we first examine the limitations of hard routing (\S\ref{subsec:hard_routing}) under noisy conditions. Since hard routing relies on manually (de-)activating the audio and visual groups, for audio-visual inputs, we assign a fixed equal weight ($p^A, p^V = 0.5$) to both groups.
However, this equal weighting may not always be optimal in varying environments, such as noise type or intensity.

As shown in Figure\,\ref{fig:expert_load}(b), increasing reliance on the audio group under babble noise degrades performance, with an optimal weight for the audio group being $0.3$.
Unlike babble noise, which confuses the model with multiple overlapping speech signals, natural noise is more distinct from speech, leading to a higher reliance on the audio group ($p^A \ge 0.5$) preferable.
These results indicate that an ideal routing strategy for audio-visual data should be dynamically adjusted.

Figure\,\ref{fig:expert_load_noise} further illustrates \ourmodel's group load distribution across different noise levels. The model adaptively adjusts its reliance between the audio and visual expert groups---under high noise conditions (low SNRs), it shifts more tokens to the visual group, while in cleaner conditions (high SNRs), the audio group is more actively utilized. This behavior also adjusts to noise types, as observed with babble and natural noise, demonstrating the MoHAVE’s adaptability and robustness.
%

\vspace{-5pt}
\subsection{Multilingual Audio-Visual Speech Tasks}
\vspace{-5pt}

MoEs have demonstrated effectiveness in multilingual speech tasks \citep{hu2023mixture, wang2023language}, as MoE is capable of enabling more diverse routing paths for different language tokens. To evaluate \ourmodel's multilingual capabilities, we train a multilingual model and assess its performance on the MuAViC benchmark~\citep{anwar2023muavic}, evaluating separately for each language.
%
Following \citet{han-etal-2024-xlavs}, we introduce multilingual babble noise at SNR 0\,dB to 50\% of the input samples during training, where the noise clips are sampled from the MuAViC train set. For inference, we apply beam search with a beam size of 5 and normalize text by punctuation removal and lower-casing before calculating WER. For AVS2TT evaluation, we use SacreBLEU\,\cite{post2018call} with its built-in \textit{13a} tokenizer. To simulate noisy test conditions, we inject babble noise sampled from the MuAViC test set.

Table\,\ref{tab:muavic} summarizes the results, where \ourmodel-\textsc{Large} achieves superior performance in both AVSR and AVS2TT. Whisper \cite{radford2023robust}, a leading multilingual ASR model, is known to perform poorly in noisy setup due to its lack of visual understanding for robustness. While multilingual AV-HuBERT \cite{anwar2023muavic} underperforms the state-of-the-art models like XLAVS-R \cite{han-etal-2024-xlavs}, we have re-implemented it using the pretrained model from \citet{choi2024av2av}, which has been pretrained on a significantly larger dataset including 7,000 hours of speech in 100+ languages. This model outperforms (38.9\% average WER) or remains competitive (18.5\% average BLEU) with much larger XLAVS-R 2B. When integrated with this version, \ourmodel further improves performance, achieving \textbf{37.4\% average WER} and \textbf{19.5\% average BLEU}, setting new benchmarks in almost every language being evaluated.



\section{Conclusion}
\red{In formal email communication, users are often required to read detailed (lengthy or complex) emails. 
Crafting appropriate responses to such emails is time-consuming and may lead to overlooked sender requests or delayed responses, causing communication issues.}
Thus, we propose \red{QA-based approach}, which leverages LLM-based question generation to help users create efficient and high-quality replies by generating multiple question-answer pairs related to the received email content.
\red{To examine the comprehensive impact of the QA-based approach on both email senders and recipients, we conducted controlled and field experiments using our prototype system, \textit{ResQ}.
Our findings demonstrate that structuring email content into question-answer pairs improves efficiency, reduces cognitive load, and lowers barriers to initiating responses. 
Additionally, this approach enhances email quality and may leave a better impression on recipients.
However, our findings also revealed challenges, including a potential reduction in user agency and an increased psychological distance in communication. 
These trade-offs emphasize the need for adaptive designs that balance efficiency with personalization and user control.
% The QA-based approach shows promise for applications beyond email communication in domains requiring structured response generation. 
Future research should investigate the long-term effects of such systems on user behavior, cross-cultural differences in adoption, and the effectiveness of the QA-based approach across varying email characteristics}
% Future research should explore its long-term impacts, cross-cultural applicability, and integration into diverse communication platforms to optimize both efficiency and authenticity in AI-mediated interactions.}

% original
% In workplace email communication, users are often required to read lengthy emails and craft appropriate responses, a time-consuming task that may lead them to overlook parts of the sender's request or delay their response, causing communication issues.
% Thus, we propose \textit{ResQ}, which leverages LLM-based question generation to help users create efficient and high-quality replies by generating multiple question-answer pairs related to the received email content.
% Our controlled and field experiments confirmed that compared to a prompt-based approach, ResQ significantly improved email replying efficiency, reduced cognitive load, and lowered the barriers to task initiation.
% Additionally, AI support was shown to improve the quality of emails and enhance the recipient's impression.
% However, we observed that ResQ decreased users' sense of agency and control and enlarged the psychological distance between email senders and receivers.
% We also discussed communication scenarios where QA-based approach might be effective for AI-mediated communication.
% \newpage
\section*{Impact Statement}

\ourmodel addresses the scalability and robustness challenges in multimodal speech processing, enabling more efficient and adaptive expert selection. This advancement highlights the potential for further scaling AVSR models in terms of capacity, training, and datasets while extending to more diverse real-world applications without incurring excessive computational costs.

While our research contributes to the broader field of multimodal learning and robust speech recognition, we do not foresee significant ethical concerns beyond those inherent to AVSR systems, such as potential linguistic biases in speech models trained on imbalanced datasets or privacy issues when handling human speech video data. Responsible data use and fairness in model deployment remain critical in AVSR systems for ensuring equitable applications.

%%%%%%%%%%%%%%%%%%%%%%%%%%%%%%%%%%%%%%%%%%%%%%%%%%%%%%%%%%%%%%%%%

% \section*{Accessibility}
% Authors are kindly asked to make their submissions as accessible as possible for everyone including people with disabilities and sensory or neurological differences.
% Tips of how to achieve this and what to pay attention to will be provided on the conference website \url{http://icml.cc/}.

% \section*{Software and Data}

% If a paper is accepted, we strongly encourage the publication of software and data with the
% camera-ready version of the paper whenever appropriate. This can be
% done by including a URL in the camera-ready copy. However, \textbf{do not}
% include URLs that reveal your institution or identity in your
% submission for review. Instead, provide an anonymous URL or upload
% the material as ``Supplementary Material'' into the OpenReview reviewing
% system. Note that reviewers are not required to look at this material
% when writing their review.

% Acknowledgements should only appear in the accepted version.
% \section*{Acknowledgements}

% \textbf{Do not} include acknowledgements in the initial version of
% the paper submitted for blind review.

% If a paper is accepted, the final camera-ready version can (and
% usually should) include acknowledgements.  Such acknowledgements
% should be placed at the end of the section, in an unnumbered section
% that does not count towards the paper page limit. Typically, this will 
% include thanks to reviewers who gave useful comments, to colleagues 
% who contributed to the ideas, and to funding agencies and corporate 
% sponsors that provided financial support.

% \section*{Impact Statement}

% Authors are \textbf{required} to include a statement of the potential 
% broader impact of their work, including its ethical aspects and future 
% societal consequences. This statement should be in an unnumbered 
% section at the end of the paper (co-located with Acknowledgements -- 
% the two may appear in either order, but both must be before References), 
% and does not count toward the paper page limit. In many cases, where 
% the ethical impacts and expected societal implications are those that 
% are well established when advancing the field of Machine Learning, 
% substantial discussion is not required, and a simple statement such 
% as the following will suffice:

% This paper presents work whose goal is to advance the field of 
% Machine Learning. There are many potential societal consequences 
% of our work, none which we feel must be specifically highlighted here.

% The above statement can be used verbatim in such cases, but we 
% encourage authors to think about whether there is content which does 
% warrant further discussion, as this statement will be apparent if the 
% paper is later flagged for ethics review.


% In the unusual situation where you want a paper to appear in the
% references without citing it in the main text, use \nocite
% \nocite{langley00}


\bibliography{icml2025_conference}
\bibliographystyle{icml2025}


%%%%%%%%%%%%%%%%%%%%%%%%%%%%%%%%%%%%%%%%%%%%%%%%%%%%%%%%%%%%%%%%%%%%%%%%%%%%%%%
%%%%%%%%%%%%%%%%%%%%%%%%%%%%%%%%%%%%%%%%%%%%%%%%%%%%%%%%%%%%%%%%%%%%%%%%%%%%%%%
% APPENDIX
%%%%%%%%%%%%%%%%%%%%%%%%%%%%%%%%%%%%%%%%%%%%%%%%%%%%%%%%%%%%%%%%%%%%%%%%%%%%%%%
%%%%%%%%%%%%%%%%%%%%%%%%%%%%%%%%%%%%%%%%%%%%%%%%%%%%%%%%%%%%%%%%%%%%%%%%%%%%%%%
\newpage
\appendix
\onecolumn

\clearpage
\vspace{1.5em}
% \setcounter{page}{1}
{
    \centering
    \rule{\textwidth}{1.5pt}\vspace{0.6em} \\
    \vspace{0.5em}
    \Large
    \textbf{{\includegraphics[width=18pt]{desert_emoji.png}}\,MoHAVE: Mixture of Hierarchical \\
    \vspace{0.1em} Audio-Visual Experts for Robust Speech Recognition} \\
    \vspace{0.5em} Supplementary Material \\
    \rule{\textwidth}{1.5pt} \\
    \vspace{1.0em}
    % ]\
}


\section{Experimental Setup}

\subsection{Model Description}
\label{appx:model_details}

As described in \S\ref{sec:implementation}, \ourmodel is implemented in two configurations: \textsc{Base} and \textsc{Large}, following the architecture of AV-HuBERT-\textsc{Base} and AV-HuBERT-\textsc{Large}, respectively. The encoder maintains the same structure as AV-HuBERT, while the decoder incorporates MoE layers by replacing every feed-forward network (FFN) layer with expert modules. Each expert in the MoE layers follows the same bottleneck structure with FFN, consisting of two fully-connected layers with an activation function. 

To encourage expert group specialization, we apply load biasing, where either audio or video is randomly dropped with a probability of 25\%. This allows the model to learn modality-aware expert utilization. For expert selection, a router network assigns tokens to a subset of experts, ensuring efficient computation. The router probabilities are always normalized as sum to 1 when computing the output $y$. The routing networks $V_r$ and $W_r$ are parameterized as matrices, with dimensions matching the hidden dimension size by the number of experts.

For comparison, we evaluate multiple MoE-based AVSR models:
\begin{itemize}[leftmargin=10pt, label={$\circ$}]
\setlength\itemsep{-0.1em}
\vspace{-10pt}
    \item AV-MoE: A simple MoE implementation over AV-HuBERT, activating top-2 out of 4 or 8 experts per token. We follow the same implementation of sparse MoE as \citep{dai2022stablemoe, jiang2024mixtral}.
    \item AV-MoE with Hard Routing: Uses top-2 out of 4 experts for unimodal inputs (audio-only or video-only). For multimodal (audio-visual) inputs, it activates top-1 from each expert group and averages their outputs. This model does not have an explicit router for groups, but within each group, there is an intra-modal router, \ie $W_r^A$ or $W_r^V$.
    \item \ourmodel: Employs top-1 expert per group, with an inter-modal router dynamically adjusting group weight assignments and an intra-modal router uniformly dispatching the tokens to modality-specific experts.
\end{itemize}


\subsection{Computation Cost}
\label{appx:computation_cost}

\begin{wraptable}{r}{10.2cm}
    \centering
    \small
    \vspace{-17pt}
    \caption{Computational cost of AV-HuBERT and \ourmodel in FLOPs.}
    \label{tab:computation_cost}
    \vspace{5pt}
    \addtolength{\tabcolsep}{-2pt}
    \begin{tabular}{l|cc|cc}
    \toprule
    \multirow{2}{*}{Model} & Activated & Total & Compute & Compute\,/\,FFN \\
     & Params & Params & (GFLOPs\,/\,seq) & (MFLOPs\,/\,seq) \\
    \midrule
    AV-HuBERT-\textsc{Base} & 161M & 161M & 12.1 & 472 \\
    \ourmodel-\textsc{Base} & 189M & 359M & 14.8 & 921 \\
    AV-HuBERT-\textsc{Large} & 477M & 477M & 32.2 & 839 \\
    \ourmodel-\textsc{Large} & 553M & 1.0B & 39.3 & 1,630 \\
    \bottomrule
    \end{tabular}
\end{wraptable}


Table\,\ref{tab:computation_cost} summarizes the parameter sizes and computational costs of \ourmodel. \ourmodel-\textsc{Base} contains 359M parameters, while the \textsc{Large} version expands to 1B parameters. Specifically, for \ourmodel-\textsc{Base}, the encoder accounts for 103M parameters, and the decoder 256M, whereas in \textsc{Large}, the encoder holds 325M, and the decoder 681M.
Despite its larger model capacity, \ourmodel maintains computational efficiency through sparse activation, where only around half of the total parameters are active per token. This results in 189M active parameters for \textsc{Base} and 553M for \textsc{Large}.

To assess actual computation costs when processing inputs, we measure floating point operations per second (FLOPs) using an audio-visual sequence of 500 frames with 50 text tokens. The entire compute cost for AV-HuBERT-\textsc{Base} and \ourmodel-\textsc{Base} are 12.1 GFLOPs and 14.8 GFLOPs, respectively, while for \textsc{Large}, the computes are 32.2 GFLOPs and 39.3 GFLOPs. This indicates a slight increase in FLOPs for \ourmodel, primarily due to the MoE layers replacing FFNs in the decoder. Although the MoE layers require roughly twice the computation cost of standard FFNs (refer to Compute\,/\,FFN), the encoder and attention layers in the decoder remain unchanged. Consequently, the overall computational cost remains comparable to AV-HuBERT counterparts, ensuring scalability without significant computation overhead.





\subsection{LRS3 Benchmark Experiments}
\label{appx:lrs3_benchmark}

We initialize our model using the pretrained checkpoint from \citep{shi2022learning} and fine-tune it on the LRS3 train set for 120K steps. The encoder remains frozen for the first 90K steps, allowing only the AVSR decoder to be trained, after which the entire model is fine-tuned for the remaining 30K steps. Our fine-tuning setup follows the configurations from \citep{shi2022robust}. We employ a sequence-to-sequence negative log-likelihood loss for predicting the next text token, without using connectionist temporal classification (CTC) decoding \citep{watanabe2017hybrid}. The Adam optimizer \cite{kingma2014adam} is used with a learning rate of 5e-4 and a polynomial decay schedule with an initial warmup. Each training step processes 8,000 audio-visual frames, equivalent to 320 seconds of speech data.

For inference, we use beam search with a beam size of 50. The AVSR performance is evaluated using word error rate (WER) across five signal-to-noise ratio (SNR) levels: $\{-10, -5, 0, 5, 10\}$ (lower value means higher noise level). We use audio noise sampled from MUSAN (babble, music, natural) and LRS3 speech noise, ensuring no speaker overlap between training and test sets. Since Table\,\ref{tab:main_result} presents SNR-averaged results for each noise type, we provide the full results across all SNR levels in Table\,\ref{tab:lrs3_full}.


\subsection{MuAViC Benchmark Experiments}
\label{appx:muavic_benchmark}

We evaluate \ourmodel on the MuAViC benchmark \cite{anwar2023muavic} for multilingual AVSR and X-to-English AVS2TT tasks. For multilingual AVSR, the dataset includes 8 non-English languages: Arabic (Ar), German (De), Greek (El), Spanish (Es), French (Fr), Italian (It), Portuguese (Pt), and Russian (Ru), encompassing approximately 700 hours of training data from 3,700 speakers. For X-En AVS2TT, the dataset covers 6 languages: Greek, Spanish, French, Italian, Portuguese, and Russian, where each sample includes audio-visual speech with corresponding English transcriptions.

A single multilingual model is trained for each task, capable of detecting the source language and generating target transcriptions accordingly. The evaluation is conducted on each language separately, as seen in Table\,\ref{tab:muavic}. Using the pretrained multilingual AV-HuBERT from \citep{choi2024av2av}, we fine-tune the model for 120K steps, unfreezing the encoder after 10K steps. Inference is performed with beam size of 5, and the samples with empty ground-truth transcriptions are removed from the evaluation set. 


\begin{table*}[!h]
    \centering
    \small
    \vspace{10pt}
    \caption{Audio-visual speech recognition performance\,(WER\,\%) on the LRS3 dataset\,\citep{afouras2018lrs3}. The number of parameters for each model includes both encoder and decoder. For evaluation, augmented noise is sampled from the MUSAN dataset\,\citep{snyder2015musan}, while speech noise is sampled from the held-out set of LRS3. AV-MoE and \ourmodel use 8 experts.}
    \label{tab:lrs3_full}
    \vspace{5pt}
    \addtolength{\tabcolsep}{-3pt}
    \renewcommand{\arraystretch}{1.1}
    \resizebox{\textwidth}{!}{
    \begin{tabular}{l|cccccc|cccccc|cccccc|cccccc}
    \toprule
    \multirow{2}{*}{Model} & \multicolumn{6}{c|}{Babble, SNR (dB) $=$} & \multicolumn{6}{c|}{Speech, SNR (dB) $=$} & \multicolumn{6}{c|}{Music, SNR (dB) $=$} & \multicolumn{6}{c}{Natural, SNR (dB) $=$} \\
    & -10 & -5 & 0 & 5 & 10 & \!\textbf{AVG}\! & -10 & -5 & 0 & 5 & 10 & \!\textbf{AVG}\! & -10 & -5 & 0 & 5 & 10 & \!\textbf{AVG}\! & -10 & -5 & 0 & 5 & 10 & \!\textbf{AVG}\! \\
    \midrule
    AV-HuBERT-\textsc{Base} & 27.6 & 14.1 & 6.1 & 3.8 & 2.7 & 10.8 & 8.6 & 5.8 & 3.9 & 3.3 & 2.8 & 4.9 & 12.2 & 6.4 & 3.8 & 2.8 & 2.5 & 5.6 & 10.9 & 5.4 & 4.0 & 2.8 & 2.3 & 5.1 \\
    AV-MoE-\textsc{Base} & 26.3 & 13.7 & 6.2 & 3.5 & 2.7 & 10.5 & 8.4 & 5.3 & 3.4 & 2.8 & 2.3 & 4.5 & 11.5 & 6.1 & 3.7 & 2.7 & 2.5 & 5.3 & 9.7 & 6.0 & 3.4 & 2.8 & 2.5 & 4.9 \\
    ~~(+) Hard Routing & 25.2 & 13.2 & 5.6 & 3.2 & 2.4 & 9.9 & 8.2 & 5.2 & 3.4 & 2.7 & 2.3 & 4.4 & 11.0 & 5.3 & 3.6 & 2.6 & 2.2 & 5.0 & 9.4 & 5.3 & 3.2 & 2.5 & 2.3 & 4.6 \\
    \rowcolor[HTML]{e1fefe}
    \ourmodel-\textsc{Base} & 25.3 & \textbf{12.2} & \textbf{5.3} & \textbf{2.9} & \textbf{2.3} & \textbf{9.6} & \textbf{7.9} & \textbf{5.1} & \textbf{3.3} & \textbf{2.4} & \textbf{2.3} & \textbf{4.2} & \textbf{10.3} & 5.6 & \textbf{3.3} & \textbf{2.3} & \textbf{2.0} & \textbf{4.7} & 9.7 & \textbf{5.1} & \textbf{3.2} & \textbf{2.4} & \textbf{2.2} & \textbf{4.5} \\
    \rowcolor[HTML]{e1fefe}
    ~~(--) Load Biasing & 26.5 & 13.6 & 5.6 & 3.2 & 2.4 & 10.3 & 8.2 & 5.2 & 3.5 & 2.6 & 2.4 & 4.4 & 11.1 & 6.4 & 3.4 & 2.7 & 2.6 & 5.2 & 10.3 & 5.6 & 3.4 & 2.7 & 2.4 & 4.9 \\
    \midrule
    AV-HuBERT-\textsc{Large} & 27.0 & 12.4 & 4.7 & 2.4 & 1.8 & 9.7 & 11.4 & 4.6 & 2.9 & 2.2 & 1.8 & 4.6 & 10.5 & 4.9 & 2.9 & 2.0 & 1.6 & 4.4 & 9.6 & 4.7 & 2.5 & 2.0 & 1.8 & 4.1 \\
    AV-MoE-\textsc{Large} & 28.1 & 12.5 & 5.0 & 2.7 & 2.1 & 10.1 & 7.9 & 4.0 & 2.9 & 2.4 & 2.0 & 3.8 & 10.4 & 5.4 & 2.9 & 2.3 & 1.9 & 4.6 & 8.9 & 4.8 & 3.1 & 2.0 & 2.0 & 4.2 \\
    ~~(+) Hard Routing & 22.9 & 10.8 & 3.8 & 2.4 & 1.8 & 8.3 & 6.7 & 3.9 & 2.4 & 1.8 & 1.8 & 3.3 & 9.9 & 4.3 & 2.3 & 1.8 & 1.9 & 4.0 & 8.3 & 4.1 & 2.4 & 1.9 & 1.7 & 3.7 \\
    \rowcolor[HTML]{e1fefe}
    \ourmodel-\textsc{Large} & \textbf{21.0} & \textbf{9.8} & 4.1 & \textbf{2.2} & \textbf{1.6} & \textbf{7.7} & \textbf{5.0} & \textbf{3.6} & \textbf{2.3} & 2.0 & 1.9 & \textbf{3.0} & \textbf{8.2} & \textbf{4.0} & 2.6 & \textbf{1.8} & \textbf{1.8} & \textbf{3.7} & \textbf{7.3} & \textbf{3.7} & 2.6 & \textbf{1.9} & \textbf{1.6} & \textbf{3.4} \\
    \rowcolor[HTML]{e1fefe}
    ~~(--) Load Biasing & 27.8 & 12.4 & 4.5 & 2.6 & 2.0 & 9.9 & 6.7 & 4.0 & 3.1 & 2.1 & 1.9 & 3.6 & 10.6 & 5.3 & 3.0 & 2.1 & 1.8 & 4.6 & 9.9 & 4.9 & 2.8 & 2.1 & 2.0 & 4.3 \\
    \bottomrule
    \end{tabular}
    }
\end{table*}






\clearpage
\section{Additional Results}
\label{appx:additional_results}

\subsection{Expert Group Utilization}
\label{appx:expert_group_usage}

In the main paper (Figure\,\ref{fig:expert_load_noise}), we have presented expert load distribution for selected layers. Figure\,\ref{fig:appx_group_load} provides the distribution across all MoE layers, illustrating how \ourmodel dynamically adjusts expert groups based on noise conditions.

\begin{figure}[!h]
    \centering
    \includegraphics[width=\linewidth]{tex_figure/appx/appx_group_load_new.pdf}
    \vspace{-20pt}
    \caption{Expert load distribution in \ourmodel for the audio group (solid bars) and visual group (dashed bars) across noisy audio-visual sequences under babble (first row) and natural (second row) noise. The frequency of each expert has been weighted by the inter-modal router's output probability.}
    \label{fig:appx_group_load}
\end{figure}


\subsection{Multilingual Tasks in Clean Environments}

Table\,\ref{tab:muavic_clean} outlines the MuAViC benchmark results in a clean environment, without auditory noise added. The experimental setup remains consistent with Table\,\ref{tab:muavic}, utilizing the same models. Unlike the noisy setting, we observe that \ourmodel does not yield significant performance improvements in clean speech tasks. This is primarily because \ourmodel enhances AVSR under noisy conditions by dynamically adjusting the utilization of audio and visual expert groups. 

Indeed, in clean speech recognition and translation tasks, encoder capacity---particularly when pretrained on large-scale audio data---plays a more crucial role than decoder-specific training methods. In addition, visual information is less essential in noise-free environments, as demonstrated by the strong ASR performance of the Whisper-large-v2 model \cite{radford2023robust}. Even the smaller ASR model, XLS-R 300M \cite{babu2022xls}, surpasses AVSR models such as mAV-HuBERT \cite{anwar2023muavic} or u-HuBERT \cite{hsu2022u} in this setting, underscoring that the advantage of using AVSR models emerges most clearly in robust speech recognition.

\begin{table*}[!h]
    \centering
    \small
    \caption{Multilingual audio-visual speech task performance, with non-English speech recognition (WER) and X-En speech-to-text translation (BLEU score), in a clean environment without auditory noise. $^\dagger$Results obtained from \citet{han-etal-2024-xlavs}. $^\ddagger$Re-implemented using the pretrained model from \citet{choi2024av2av}.}
    \label{tab:muavic_clean}
    \vspace{5pt}
    % \addtolength{\tabcolsep}{-1pt}
    \resizebox{\textwidth}{!}{
    \begin{tabular}{llccccccccc}
    \toprule
    && \multicolumn{8}{c}{Source} & \\
    \cmidrule{3-10}
    Model & Pretrain data & Ar & De & El & Es & Fr & It & Pt & Ru & Avg \\
    \midrule
    \multicolumn{11}{c}{\textbf{\textit{Clean Speech Recognition, Test WER $\downarrow$}}} \\
    Whisper large-v2\,\cite{radford2023robust} & \textit{680k hrs, 100+ langs} & 91.5 & \textbf{24.8} & 25.4 & 12.0 & 12.7 & 13.0 & 15.5 & 31.1 & 28.2 \\
    u-HuBERT$^\dagger$\,\cite{hsu2022u} & \textit{1.7k hrs, English} & 89.3 & 52.1 & 46.4 & 17.3 & 20.5 & 21.2 & 21.9 & 44.4 & 39.1 \\
    mAV-HuBERT\,\cite{anwar2023muavic} & \textit{1.7k hrs, English} & 69.3 & 47.2 & 41.2 & 16.2 & 19.0 & 19.8 & 19.9 & 38.0 & 33.8 \\
    XLS-R 300M$^\dagger$\,\cite{babu2022xls} & \textit{1.2k hrs, 9 langs} & 85.6 & 44.0 & 34.4 & 13.2 & 15.1 & 14.3 & 16.2 & 34.4 & 32.2 \\
    XLAVS-R 300M\,\cite{han-etal-2024-xlavs} & \textit{8.3k hrs, 100+ langs} & 80.0 & 38.0 & 28.1 & 11.7 & 15.3 & 13.8 & 14.4 & 31.2 & 29.1 \\
    XLAVS-R 2B\,\cite{han-etal-2024-xlavs} & \textit{1.2k hrs, 9 langs} & 79.3 & 44.4 & \textbf{19.0} & \textbf{9.1} & \textbf{12.3} & \textbf{10.6} & \textbf{11.2} & \textbf{25.0} & \textbf{26.4} \\
    \cdashlinelr{1-11}
    mAV-HuBERT$^\ddagger$ & \textit{7.0k hrs, 100+ langs} & \textbf{78.3} & 41.4 & 25.5 & 11.9 & 16.2 & 14.8 & 14.3 & 31.6 & 29.3 \\
    \ourmodel-\textsc{Large} (\textbf{ours}) & \textit{7.0k hrs, 100+ langs} & 85.1 & 38.9 & 25.9 & 11.2 & 14.6 & 14.0 & 13.8 & 30.0 & 29.2 \\
    \midrule
    \multicolumn{11}{c}{\textbf{\textit{Clean X-En Speech-to-Text Translation, Test BLEU $\uparrow$}}} \\
    Whisper large-v2\,\cite{radford2023robust} & \textit{680k hrs, 100+ langs} & - & - & \textbf{24.2} & \textbf{28.9} & \textbf{34.5} & \textbf{29.2} & \textbf{32.6} & \textbf{16.1} & \textbf{29.9} \\
    mAV-HuBERT\,\cite{anwar2023muavic} & \textit{1.7k hrs, English} & - & - & 7.6 & 20.5 & 25.2 & 20.0 & 24.0 & 8.1 & 17.6 \\
    XLAVS-R 300M\,\cite{han-etal-2024-xlavs} & \textit{8.3k hrs, 100+ langs} & - & - & 18.3 & 23.9 & 29.8 & 25.1 & 28.9 & 12.1 & 23.0 \\
    XLAVS-R 2B\,\cite{han-etal-2024-xlavs} & \textit{1.2k hrs, 9 langs} & - & - & 21.6 & 25.1 & 30.6 & 26.6 & 29.9 & 13.9 & 24.6 \\
    \cdashlinelr{1-11}
    mAV-HuBERT$^\ddagger$ & \textit{7.0k hrs, 100+ langs} & - & - & 11.5 & 24.2 & 29.2 & 23.9 & 28.1 & 10.4 & 21.2 \\
    \ourmodel-\textsc{Large} (\textbf{ours}) & \textit{7.0k hrs, 100+ langs} & - & - & 13.8 & 24.9 & 30.8 & 25.0 & 28.7 & 10.9 & 22.4 \\
    \bottomrule
    \end{tabular}
    }
\end{table*}


\subsection{Number of Activated Experts}

By default, \ourmodel selects one expert from each group---audio and visual---activating a total of two experts per token. This design is to match the compute of standard MoE implementations, which utilizes top-2 out of 8 experts. A natural question arises: \textit{does activating more experts improve performance, or does it simply increase computational costs without substantial gains?}

Table\,\ref{tab:num_experts} presents the results when activating more experts of \ourmodel-\textsc{Base}, where top-$k^A$ experts from the audio group and top-$k^V$ experts from the visual group are selected. Interestingly, increasing the number of audio experts significantly degrades performance, implying that the model might be confused by employing another sub-optimal expert. 

% \begin{table}[!t]
\begin{wraptable}{r}{9.5cm}
    \vspace{-10pt}
    \centering
    \small
    \caption{Impact of the number of activated experts on AVSR performance.}
    \label{tab:num_experts}
    \vspace{3pt}
    \addtolength{\tabcolsep}{-1pt}
    \begin{tabular}{c|cccc|c|c}
        \toprule
        ($k^A$, $k^V$) & babble & speech & music & natural & N-WER & C-WER \\
        \midrule
        ($1, 1$) & 9.6 & 4.2 & 4.7 & 4.5 & 5.8 & 1.8 \\
        ($1, 2$) & 9.3 & 4.1 & 4.8 & 4.4 & 5.7 & 1.9 \\
        ($2, 1$) & 10.1 & 4.5 & 5.3 & 4.9 & 6.2 & 2.4 \\
        ($2, 2$) & 11.0 & 5.2 & 5.8 & 5.5 & 6.9 & 3.0 \\
        \bottomrule
    \end{tabular}
    \vspace{-5pt}
\end{wraptable}
% \end{table}
In contrast, activating two visual experts while keeping one audio expert improves performance (N-WER of 5.7\%) compared to the default setting of single visual expert. Particularly under the babble noise, WER has decreased from 9.6\% to 9.3\%. 
This suggests that adding an additional visual expert can be beneficial in noisy environments, likely due to the increased robustness from visual information in challenging audio conditions.


\subsection{Unimodal Task Results}

Table\,\ref{tab:asr_vsr} presents unimodal task results, evaluating model performance on video-only (VSR) sequences and audio-only (ASR) sequences. 
BRAVEn \cite{haliassos2024braven} and Llama-3.1-8B-AVSR \cite{cappellazzo2024large} models achieve the best VSR performance, as these models are specifically pretrained for the VSR task. While using an LLM decoder is highly effective in VSR, since LLMs are able to refine and correct recognition errors, ASR performance is largely determined by the encoder's pretraining strategy as BRAVEn and Whisper encoders.
As an adaptive audio-visual model, \ourmodel does not specialize in unimodal tasks but instead performs robustly in multimodal AVSR. It only exhibits a slight improvement in VSR over AV-HuBERT. These results indicate that unimodal performance is primarily influenced by the effectiveness of the encoder pretraining strategy rather than the MoE-based multimodal approach.

\begin{table}[!h]
    \centering
    \small
    \caption{Comparison on the unimodal ASR and VSR task performance.}
    \label{tab:asr_vsr}
    \vspace{5pt}
    \begin{tabular}{ll|cc}
    \toprule
    Method & Encoder\,+\,Decoder & V & A \\
    \midrule
    BRAVEn~\cite{haliassos2024braven} & BRAVEn\,+\,Transformer & 26.6 & 1.2 \\
    Llama3.1-8B-AVSR~\cite{cappellazzo2024large} & AV-HuBERT\,+\,LLM & 25.3 & 1.4 \\ 
    Llama3.1-8B-AVSR~\cite{cappellazzo2024large} & Whisper\,+\,LLM & - & 1.1 \\
    \midrule
    AV-HuBERT-\textsc{Large}~\cite{shi2022learning} & AV-HuBERT\,+\,Transformer & 28.6 & 1.4 \\
    \ourmodel-\textsc{Large} (\textbf{ours}) & AV-HuBERT\,+\,MoE & 28.2 & 1.4 \\
    \bottomrule
    \end{tabular}
\end{table}


\clearpage
\subsection{Variations of MoHAVE Implementations}

\paragraph{\ourmodel in the encoder.}

We have implemented \ourmodel by integrating MoE into the decoder to facilitate text token processing while enhancing multimodal fusion. Since the AVSR decoder incorporates information from both audio and visual modalities along with text tokens, the decoder-based MoHAVE is expected to be the most effective strategy. An alternative approach is to apply MoHAVE within the encoder, by pretraining the encoder using the AV-HuBERT masked prediction strategy \cite{shi2022learning}. For this, we initialize the pretrained encoder (with standard Transformers) and convert the FFN layers into MoE layers by copying the FFN parameters into all the expert modules. Since the \textsc{Base} model consists of 12 encoder layers, we convert 6 of them alternatively to match the number of MoE layers in the decoder \ourmodel. During fine-tuning, all MoE layers in the encoder are also trained following the same procedure.

There are two options for pretraining strategies: (1) pretraining only the MoE layers initialized from the FFN parameters, and (2) pretraining the entire encoder with MoE layers. As shown in Table\,\ref{tab:encoder_mohave}, the latter approach significantly outperforms the former, suggesting that encoder \ourmodel requires full pretraining for effective learning.
However, even with full pretraining, encoder \ourmodel performs inferior to decoder \ourmodel. This is because the encoder only processes audio and visual tokens, whereas the decoder directly integrates audio-visual embeddings with text, finding optimal strategies for text token dispatching that best improves speech recognition. In addition, applying \ourmodel to both the encoder and decoder leads to degraded performance despite the increased computational cost.

\vspace{-8pt}
\paragraph{Decoder uptraining.}

We also explore a successive training strategy for the decoder, referred to as uptraining \cite{ainslie2023gqa}, where the decoder \ourmodel undergoes additional training after the fine-tuning phase of standard Transformers. However, as seen in Table\,\ref{tab:encoder_mohave}, uptraining does not yield further improvements compared to training from scratch, even after an additional 120K training steps. In fact, we observed the shorter uptraining steps leading to degraded performance. This suggests that training the decoder \ourmodel requires a comprehensive learning phase rather than incremental fine-tuning, as MoE  may fundamentally alter the processing pathways of tokens.

\begin{table}[!h]
    \centering
    \small
    \caption{Performance comparison of \ourmodel applied to the encoder, decoder, and both.}
    \label{tab:encoder_mohave}
    \vspace{5pt}
    \begin{tabular}{l|cccc|c|c}
        \toprule
        Method & babble & speech & music & natural & N-WER & C-WER \\
        \midrule
        Encoder MoHAVE (pretrain only MoE) & 10.5 & 5.0 & 5.6 & 5.0 & 6.5 & 2.4 \\
        Encoder MoHAVE & 10.0 & 4.4 & 5.1 & 4.5 & 6.0 & 1.9 \\
        Encoder + Decoder MoHAVE & 10.1 & 4.7 & 5.3 & 4.7 & 6.2 & 2.0 \\
        \midrule
        Decoder MoHAVE & 9.6 & 4.2 & 4.7 & 4.5 & 5.8 & 1.8 \\
        Decoder MoHAVE (uptrain) & 9.7 & 4.2 & 4.8 & 4.5 & 5.8 & 1.8 \\        
        \bottomrule
    \end{tabular}
\end{table}

% \section{You \emph{can} have an appendix here.}

% You can have as much text here as you want. The main body must be at most $8$ pages long.
% For the final version, one more page can be added.
% If you want, you can use an appendix like this one.  

% The $\mathtt{\backslash onecolumn}$ command above can be kept in place if you prefer a one-column appendix, or can be removed if you prefer a two-column appendix.  Apart from this possible change, the style (font size, spacing, margins, page numbering, etc.) should be kept the same as the main body.
%%%%%%%%%%%%%%%%%%%%%%%%%%%%%%%%%%%%%%%%%%%%%%%%%%%%%%%%%%%%%%%%%%%%%%%%%%%%%%%
%%%%%%%%%%%%%%%%%%%%%%%%%%%%%%%%%%%%%%%%%%%%%%%%%%%%%%%%%%%%%%%%%%%%%%%%%%%%%%%


\end{document}


% This document was modified from the file originally made available by
% Pat Langley and Andrea Danyluk for ICML-2K. This version was created
% by Iain Murray in 2018, and modified by Alexandre Bouchard in
% 2019 and 2021 and by Csaba Szepesvari, Gang Niu and Sivan Sabato in 2022.
% Modified again in 2023 and 2024 by Sivan Sabato and Jonathan Scarlett.
% Previous contributors include Dan Roy, Lise Getoor and Tobias
% Scheffer, which was slightly modified from the 2010 version by
% Thorsten Joachims & Johannes Fuernkranz, slightly modified from the
% 2009 version by Kiri Wagstaff and Sam Roweis's 2008 version, which is
% slightly modified from Prasad Tadepalli's 2007 version which is a
% lightly changed version of the previous year's version by Andrew
% Moore, which was in turn edited from those of Kristian Kersting and
% Codrina Lauth. Alex Smola contributed to the algorithmic style files.
