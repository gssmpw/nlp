\section{Introduction}

Humans use a rich set of action repertoire to manipulate objects: we not only pick-and-place objects but also toss laundry, slide a box, snatch a pen, hammer a nail, otherwise leverage momentum and forces to manipulate diverse objects efficiently and effectively. In contrast, most manipulators today are limited to picking, where a robot simply grasps an object to resist the frictional force. While kinematic pick-and-place is sufficient for static, controlled tasks, dynamic manipulation is necessary for building an effective general-purpose robot.

One critical reason for the lack of dynamic manipulation capability is hardware. Traditional industrial robots are strong, precise, and consistent, but their heavy and high inertia structures make them unsuitable for dynamic manipulation in human environments. Modern collaborative robots, such as the Franka Panda~\cite{haddadin2022franka}, are designed to work alongside humans by using a lighter and smaller design. However, they still have high inertia and are constrained by velocity and torque limits to ensure safety, rendering them inadequate for dynamic manipulation that requires high speed and acceleration. Furthermore, they use high gear ratio actuators which effectively ``lock'' joints, making them difficult to absorb high impact forces at contacts. Such actuators also require expensive torque sensors, as current-to-torque relationship is difficult to model due to the backlash and friction caused by multi-stage gears.

Our objective is to build and open-source a 6 degrees-of-freedom (DoF) bimanual robot that can be easily and cheaply assembled at a lab to democratize dynamic manipulation research, such as examples shown in Figure~\ref{fig:overall_illustration}. To achieve this, our design must meet the following criteria:

\begin{itemize}
    \item \textbf{Dynamic}: the robot should be able to move at high speed with acceleration to perform dynamic manipulation.
    \item \textbf{Low-cost}: the design should use inexpensive materials, sensors, and actuators while having high payload, precision, and torque necessary for versatile manipulation.
    \item \textbf{Safe}: the robot should have a low inertia to ensure safety even at the maximum speed.
    \item \textbf{Ease of assembly}: the robot should be built with off-the-shelf materials.
\end{itemize}

%One approach to meet the first three requirements is via the use of tendon drive systems~\cite{song2018development, nishii2023ultra, guist2024safe}. Tendons can transmit torques to joints located far from the motors, enabling all actuators to be at the base of the arm, which significantly reduces the arm's inertia. Additionally, the low-gear ratio nature of tendon-driven robots make them far more back-drivable than high-gear ratio actuators used in collaborative robots. This enables torque control without requiring expensive torque sensors at each joint, as it greatly reduces difficult-to-model effects like backlash and friction, which are common in high gear ratio actuators. Furthermore, the natural compliance of these systems allows for high-speed, high-impact interactions by effectively absorbing impact forces.

One potential way to achieve these is through tendon drive systems~\cite{song2018development, quigley2011low, guist2024safe}. These systems transmit torques to joints located far from motors, allowing actuators to be placed at the arm's base. This design significantly reduces inertia and enhances the robot's agility. Furthermore, tendons are highly back-drivable, making them well-suited for rapid and dynamic movements.

However, tendon-driven robots have several problems. First, they are difficult to assemble and maintain at a research lab. Assembling a tendon-wiring mechanism involves integrating multiple pulleys connecting tendons to rotors while avoiding interference with electronic connections, an intricate and complex task. Achieving the desired levels of stiffness and friction adds another layer of complexity, as it requires precise adjustment of tendon tension through a tensioner. Moreover, tendon-driven systems are susceptible to wear and tear, particularly during high-impact tasks such as hammering. In such scenarios, tendons may loosen upon impact, necessitating frequent adjustments and repairs to maintain performance.

%They are also back-drivable and have low-gear ratio, which enable quick and dynamic movements. This natural compliance allows for high-speed, high-impact interactions by effectively absorbing impact forces that frequently occurs in dynamic manipulation. Furthermore, their back-drivability facilitates precise torque control without the need for expensive torque sensors at each joint, as it minimizes backlash and friction. 



%Additionally, elastic transmission systems are far more back-drivable than collaborative robots that rely on high-gear-ratio actuators. This back-drivability enables torque control without requiring expensive torque sensors at each joint, as it mitigates difficult-to-model effects like backlash and friction, which are prevalent in high-gear actuators. Moreover, the natural compliance of these systems allows for high-speed, high-impact interactions by effectively absorbing impact forces.

% However, transmission systems utilizing elastic materials also present challenges. First, they can be difficult to assemble and maintain in a research lab setting. Assembling mechanisms such as tendon-wiring or timing-belt routing involves integrating multiple pulleys or belts with actuators while avoiding interference with electronic connections, which can be a complex task for non-experts. Additionally, achieving the desired stiffness and friction requires precise adjustment of tension in the tendons or belts, which is non-trivial. Finally, these systems are prone to wear and tear, particularly during high-impact tasks like hammering, where the tendons or belts may loosen or degrade upon impact, necessitating frequent maintenance and repairs.



%We instead take inspiration from the success of quadruped hardware in designing our robot. Recent quadrupeds demonstrate dynamic and explosive movements, such as jumping and parkour~\cite{hoeller2024anymal, cheng2024extreme}, while reliably withstanding high-impact contact forces. Their strategy involves using quasi-direct drive (QDD) actuators in combination with transmission systems that place actuators close to the body, enabling efficient and robust motion. To address the challenges associated with elastic material-based transmission, such as tendons or timing belts, we choose to use a linkage-based transmission system. Linkage systems offer easier maintenance and are free from the need to maintain precise tension or address issues of wear and tear, which are common in elastic material-based systems during high-impact tasks.

We instead take inspiration from the recent success of quadrupeds in designing our bimanual robot. Recent quadrupeds demonstrate dynamic and explosive movements, such as jumping and parkour~\cite{hoeller2024anymal, cheng2024extreme}, while reliably withstanding high-impact, high-frequency contact forces. What enables this is the use of quasi-direct drive (QDD) actuators, which have significantly lower gear ratios compared to those typically used in collaborative robots (eg., 1:10 vs. 1:100). These low-gear ratio actuators make the system highly back-drivable, allowing joints to absorb impact forces through natural compliance. Additionally, the reduced backlash and friction in these actuators simplify modeling the current-to-torque relationship, eliminating the need for expensive torque sensors.

To minimize inertia, we would ideally move all the actuators to the body, as done in quadrupeds. However, the key difference between a quadruped leg and a bimanual arm is the number of joints. While a quadruped leg typically has 3 DoF, with only the knee joint located away from the body, a robot arm requires at least 6 DoF to achieve full spatial manipulation. To address this, we mount heavier and stronger actuators on the main body to control the shoulder joints, and use smaller, lighter actuators for the elbow and wrist joints. This design results in a lighter moving mass, albeit at the cost of reduced strength in the elbow and wrist.

Figure~\ref{fig:overall_illustration}a top showcases our robot, \robot (Affordable Robot for Manipulation and Dynamic Actions), built with these design principles. Each arm weighs 1.09 kg (excluding body-mounted parts and gripper) and is constructed using off-the-shelf motors and 3D-printed links. The entire system costs \$6,100 to build. To eliminate the need for a torque sensor at each joint, we manually measured and calibrated the current-to-torque relationship\footnote{vendor's data was inaccurate}. To improve impact resistance and ease of assembly, we use a linkage-based transmission mechanism that does not require tensioners. The robot’s links are 3D-printed using polylactic acid (PLA), except for the elbow joint, which is made from aluminum to reduce deformation under high loads. 

We also develop a simple, compact jaw gripper compatible with the arm shown in Figure~\ref{fig:overall_illustration}a bottom.  By using the thermoplastic polyurethane (TPU) based finger without linkage structure, this design simplifies construction and maintenance while providing sufficient robustness for dynamic tasks. The gripper design, like the rest of \robot, is fully open-sourced.%, enabling other researchers to adopt and customize it easily.

In our experiments, we show \robot could perform several dynamic motions, such as object snatching and hammering. We also demonstrate we can train a contact-rich non-prehensile manipulation policy entirely in simulation using reinforcement learning (RL), and zero-shot transfer to the real world. Lastly, we show \robot can be used for human motion retargeting on the dynamic bimanual object throwing task. Examples from these tasks are highlighted in Figure~\ref{fig:overall_illustration}. We completely open-source our code and design.