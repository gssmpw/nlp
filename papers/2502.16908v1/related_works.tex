\section{Related Work}


\subsection{Existing manipulators in the context of dynamic manipulation}
\label{sec:existing_manipulators}

\begin{table*}[ht]
\resizebox{\textwidth}{!}{%
\centering
\begin{threeparttable}
\caption{Comparison of manipulators}
\label{tab:comparison}
\begin{tabular}{l||c|c|c|c|c|c|c|c}

                                  & \robot (Ours)    & Franka Panda~\cite{haddadin2022franka} & KUKA iiwa 7 R800 & Quigley et al.~\cite{quigley2011low} & LIMS~\cite{kim2017anthropomorphic}    & Nishii et al.~\cite{nishii2023ultra} & PAMY2~\cite{guist2024safe}    & BLUE~\cite{gealy2019quasi}              \\ \hline
DoF (one arm)                 & 6       & 7            & 7                & 7              & 7       & 6                      &  {4}        & 7\\ %\hline
Inertia (kg\ensuremath{\cdot}m\textsuperscript{2})            &  {0.234}   &  {large}           &  {large}               &  {0.083}        &  {0.599}   & ?       & ?        &  {0.75}\\ %\hline
Moving mass\tnote{1} (kg)   &  {1.09}    &  {18}           &  {22.3}             &  {2}              &  {2.24}    & {0.176}   &  {1.3}      &  {8.7}\\ %\hline
End-effector speed (m/s)    &  {6.16}    &  {1.7}          & 3.2              &  {1.5}            &  {5.35}    & ?       &  {12}       & 2.1\\ %\hline
Total cost (\$, one arm)      &  {3,040} &  {expensive}           &  {expensive}               &  {4,135}          & ?       & ?       &  {14,540}     &  {\textless{}5,000} \\ %\hline
Open-source                   &  {O}       &  {X}            &  {X}                &  {O}              &  {X}       &  {X}        &  {O}        &  {O}                 \\ %\hline
% Repeatability (mm)          & 2.626    &  {0.1}          &  {0.1}              & 3              & 0.425   & 2.2      & ?        & 3.7\\ %\hline
% Repeatability method          & ISO9283 & ISO9283      & ISO9283          & manually       & ISO9283 & manually & manually    & manually\\ %\hline
Payload (kg)                & 2.5     & 3            & 7                & 2              & 3       & 3                      & ?        & 2\\ 
\end{tabular}
\begin{tablenotes}
\item ``?" denotes information not provided in the paper.
\item[1] moving mass is defined as the arm’s mass, excluding body-mounted components and the gripper.
\end{tablenotes}
\end{threeparttable}
}
\end{table*}
Classical industrial robots, such as Unimate~\cite{devol1961programmed}, PUMA~\cite{beecher1979puma}, and SCARA~\cite{makino1980selective}, were among the earliest programmable robots designed for efficient manipulation in factory settings. They perform quasi-static tasks such as assembly, die casting, and welding, with exceptional precision and consistency. However, their high inertia structures render them unsuitable for dynamic tasks. Optimized for fixed and controlled factory environments, these robots lack the agility required for frequent contact or rapid movements required for dynamic manipulation.

Collaborative robots (cobots), such as the Franka Emika~\cite{haddadin2022franka} and KUKA LBR series~\cite{hirzinger2001new, hirzinger2002dlr, albu2007dlr}, are designed to operate in humans environments. One significant limitation of cobots is their reliance on high-gear-ratio actuators, which introduce uncertainties in joint dynamics due to gearbox backlash and friction. Because of this, to achieve precise torque control, cobots often incorporate torque sensors; however, these sensors are susceptible to damage from impacts, and are expensive~\cite{zhu2023design}. Furthermore, while cobots are smaller in size, they still have the same design structure as industrial robots, with heavy actuators (around 2 kg) placed at each joint, which results in a high-inertia arm.

The dominant approach for designing robots for dynamic manipulation is to use tendon transmissions to achieve low inertia and absorb impact forces. Barrett WAM~\cite{rooks2006harmonious} uses a cable-and-cylinder drive system to mount four motors for the shoulder and elbow joints on the robot’s body, resulting in a low-inertia 4 DoF arm capable of tasks such as batting and throwing. Similarly, Quigley et al.~\cite{quigley2011low} incorporate series-elastic actuators at the shoulder and elbow joints with tendon-based transmission to achieve low inertia. LIMS~\cite{kim2015design, song2018development} attach all seven actuators to the main body and control joints via tendons. Coupled with lightweight link designs, it can perform dynamic tasks such as swinging folding fans. PAMY2~\cite{guist2024safe} combines tendon-driven transmissions with pneumatic artificial muscles for passive compliance in a 4 DoF arm, demonstrating dynamic capabilities such as table tennis smashing. Nishii et al.~\cite{nishii2023ultra} uses a combination of timing belts and tendons. It uses a timing belt to attach four motors near the body, and a tendon-based quaternion joint mechanism for the 2 DoF wrist, achieving 6 DoF maneuverability with reduced moving mass. 

Tendon-based systems generally achieve low inertia and gear ratio, and enable quick, explosive, and dynamic movements. However, maintaining tendons requires precise tensioning to achieve proper stiffness and friction, and it is very difficult to assemble them. These systems also face wear and tear during high-impact tasks, requiring frequent repairs. We summarize the comparison among \robot and existing manipulators in Table~\ref{tab:comparison}.



\subsection{Dynamic movements in quadrupeds}
The state-of-the-art quadruped hardware, combined reinforcement learning, has shown impressive dynamic motions such as jumping and parkour recently~\cite{zhuang2023robot, cheng2024extreme, choi2023learning}. Representative quadruped hardware includes, but is not limited to, Unitree Go1, MIT Cheetah~\cite{bledt2018cheetah}, KAIST Hound~\cite{shin2022design}, and Raibo~\cite{choi2023learning}.
One common factor in all these hardware platforms is the use of quasi-direct-drive (QDD) actuators~\cite{wensing2017proprioceptive}. Since QDD actuators 
have fewer gear stages, they have much less backlash and friction compared to high-gear ratio actuators (See Figure~\ref{fig:gear_uncertainty}). Further, their back-drivability makes them naturally compliant, enabling them to absorb high-impact forces, and allows us to model the current-to-torque relationship much easier. We take inspiration from these successes, and adopt QDD in our arms.


\begin{figure}[ht]
\centering
\includegraphics[width=0.8\linewidth]{fig/gear_unncertainty_modi2.png}
\caption{Comparison of gearboxes in low-gear ratio actuators (b) vs. high-gear ratio actuators (c).  All gear mechanisms introduce backlash and friction (a) which are difficult to model. Because low gear ratios use fewer gears, it is easier to model.}
\label{fig:gear_uncertainty}
\end{figure}

Another essential design principle in recent quadrupeds is the use of low-inertia legs, typically achieved by mounting actuators on the body and connecting them to the legs via transmission systems such as timing belts or linkages. This approach minimizes the moving mass, enabling the legs to move rapidly while using less torque. In our arm, we also employ linkages as our transmission system to control the elbow joint since it is easier to assemble, and locate as many actuators as possible to the robot's body. However, unlike quadrupeds, our arm features six degrees of freedom (6 DoF) and requires additional degrees of freedom at the wrist. To address this and maintain low inertia, we use lighter, weaker actuators at the wrist since high torque is generally not required in these parts region.


\subsection{Hardware solutions for high-impact manipulation}
%Since cobots easily shut down from high external impacts, most contact-rich manipulations on cobots remain slow. Therefore, 

For dynamic and contact-rich tasks, robustness against impact forces is essential. Several works have proposed hardware solutions to address this in high-impact scenarios such as hammering. Izumi et al.~\cite{izumi1993control} utilize a flexible link to minimize impulsive forces, and Garabini et al.~\cite{garabini2011optimality} use actuators with torsional elastic springs to absorb impact while achieving high speeds. 
Both approaches model the system to control and demonstrate the hammering task but require accounting for damping forces and vibrations from elastic components, making them difficult to simulate.
% task-specific modeling,  
Instead of mitigating impact forces with elastic materials, \robot adopts QDD actuators to make the system inherently compliant which simplifies modeling and simulation.

Other designs~\cite{imran2016impulse, romanyuk2019multiple} focus on adding links or control modes to manage impact forces for hammering. Imran et al.~\cite{imran2016impulse} introduce serial linkage chains around the robot links to distribute impact forces, while Romanyuk et al.~\cite{romanyuk2019multiple} employ MRR actuators~\cite{liu2008development} with multiple working modes. By switching to a passive mode during impacts, these actuators allow free rotation to absorb impulsive forces. Although these design choices effectively absorb impacts, additional links or control modes complicate the robot’s design, making them harder to assemble and control. Instead of increasing complexity, \robot uses a simpler design with just a four-bar linkage and QDD actuators, ensuring ease of assembly and control while maintaining robustness to impacts.

\subsection{Dynamic manipulation with existing manipulators}
There have been several attempts to generate dynamic manipulation motions using existing arms, although they have been limited to one specific category of motion such as throwing, catching, or batting. For throwing, Senoo et al.~\cite{high_speed_throwing} models contact at the robot fingertip to throw with the WAM robot (4 DoF). Bombile and Billard~\cite{kuka_grabbing_tossing} formulate modulated dynamical systems of a pair of KUKA LBR to toss a box with dual manipulators. Recently, some works using learning to account for the physical properties not explicitly considered in the projectile mechanics from perceptual input, using a Tx60 (7 DoF)~\cite{throwing_visual_feedback}, a single PR2 arm (7 DoF)~\cite{dppt}, and UR5 (6 DoF)~\cite{zeng2020tossingbot}. 

For catching, Lampariello et al.~\cite{planning_for_catching}  model the motion of 7 DoF manipulator (KUKA LBR) as a dynamical system and propose real-time online optimization algorithms to catch an object tracked with a vision-based motion capture system. Kim et al.~\cite{kim2014catching} leverage expert demonstration to guide the same 7 DoF manipulator to catch more diverse objects.  B\"{a}tz et al.~\cite{batz2010dynamic} predicts the trajectory of a flying ball and plausible interception points computed offline to catch the ball with Staubli RX90B (6 DoF). 
These works focus on the light object which induces negligible impulse since the aforementioned manipulators lack compliance to cope with the large impact generated during the catching of heavier objects. Salehian et al.~\cite{salehian2016dynamical} and Yan et al.~\cite{yan2024impact} attempt to mitigate the impact in the non-compliant manipulator (KUKA LBR) by separately planning a decelerating motion after the intercept of the object.  However, additional computation is required for such motion, which often results in failing to plan the motion before the object falls. 
Unlike these robots, \robot~is naturally compliant and can avoid such additional computation.  Similarly, Kim et al.~\cite{free_flying} enables fast catching with a compliant Barret WAM (7 DoF), yet maintenance of the robot is difficult due to its tendon-based mechanism.

For batting, Senoo et al.~\cite{high_speed_batting} develop a high-frequency vision system to track a ball and bat it with a Barrett WAM (4 DoF) using handcrafted batting dynamics. Similarly, Jia et al.~\cite{jia2019batting} model 2D impact dynamics with Coloumb friction and energy-based restitution to bat an object to a desired position with the same 4 DoF manipulator. Although these methods enable dynamic batting with light objects like styrofoam or ping pong balls, it is difficult to use them with heavier objects with existing arms, as hitting heavy objects might damage the manipulator. In contrast, \robot~is naturally compliant, and is easy to maintain.


\subsection{RL-based Manipulation}

%Dynamic contact-rich manipulation involves frequent impacts and complex interactions between the robot, objects, and the environment, making precise modeling of contacts and physical properties difficult. RL overcomes these challenges by learning policies that implicitly capture contact dynamics through trial and error, with additional simulation-to-reality techniques such as domain randomization~\cite{andrychowicz2020learning, peng2018sim, tobin2017domain}. 

Recently, reinforcement learning (RL) has been successfully applied to contact-rich manipulation tasks, including in-hand manipulation~\cite{chen2022system, andrychowicz2020learning, allshire2022transferring, handa2023dextreme} and non-prehensile manipulation~\cite{zhou2023hacman, zhou2023learning, cho2024corn, kim2023pre}, addressing challenges that traditional planning techniques have struggled to solve. These advancements have been made possible through the combination of large-scale simulation and simulation-to-reality transfer techniques, such as domain randomization. However, most of these works rely on cobots, which are less suited for quick and dynamic motions involving frequent impacts limiting them to slow movements. In contrast, we demonstrate that our robot can achieve similar non-prehensile manipulation tasks using RL while exhibiting much more agile and dynamic motions.

\subsection{Human motion shadowing}

Thanks to the recent development of computer vision algorithms, several works show that we can collect human demonstrations simply by watching them. H2O~\cite{h2o} and OmniH20~\cite{omnih2o} adopts HybrIK~\cite{hybrik} as body pose predictors to achieve real-time humanoid teleoperation using a RGB camera. HumanPlus~\cite{humanplus} leverages human body~\cite{wham} and hand~\cite{hamer} pose predictors to extract human joint angles in real time and project them on the humanoid actuators. Similarly, OKAMI~\cite{okami} adopts body~\cite{slahmr} and hand~\cite{hamer} pose predictors to extract human body keypoints and solve differential inverse kinematics~\cite{pink} to imitate human motion that keeps the relative positions between the hands similar to that of human. Leveraging a single RGB camera with body pose predictor~\cite{wham} we showcase that our hardware, \robot, also is capable of shadowing human demonstrations in dynamic bimanual throwing. 
