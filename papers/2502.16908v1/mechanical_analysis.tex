\section{Mechanical Analysis}

\subsection{End-effector speed}

%To evaluate the dynamic performance and ensure safety given the robot arm's low inertia, 
To test whether \robot can perform dynamic manipulation, we measure its end-effector maximum speed. To do this, we initialize the robot with its right end-effector positioned at the opposite shoulder with the elbow completely flexed, and then fully extend the elbow downward, as shown in Figure~\ref{fig:speed_payload_impact}a. The joint trajectory is generated by interpolating between the initial and final configurations, and joint position control is used to follow the trajectory. \robot achieves an average maximum end-effector speed of 6.16 m/s with a standard deviation of 0.472 m/s across 40 repetitions, without any damage to the arm or power supply.

\begin{figure}
    \begin{subfigure}[b]{0.47\linewidth}
        \centering
        \includegraphics[width=0.92\linewidth]{fig/ee_speed_fig_V4.png}
        \caption{Speed test}
    \end{subfigure}
    \begin{subfigure}[b]{0.471\linewidth}
        \centering
        \includegraphics[width=\linewidth]{fig/payload_fig_V2.png}
        \caption{Lifting test}
    \end{subfigure}
    \begin{subfigure}[t]{\linewidth}
        \centering
        \includegraphics[width=0.99\linewidth]{fig/ee_impact.png}
        \caption{Impact test}
    \end{subfigure}
    \caption{(a) Trajectory of the end-effector speed test. Starting from the initial joint position, \robot accelerates toward the terminal position. (b) A sequence of a realistic dumbbell lifting task. The robot grasps, lifts, holds for three seconds, and places down the dumbbell. (c) We use push-pull gauge to measure the maximum impact force near the point of maximum speed.}
    \label{fig:speed_payload_impact}
\end{figure}

% \begin{figure}[hbt]
%     \centering
%     \includegraphics[width=0.5\linewidth]{fig/ee_speed_fig_V4.png}
%     \caption{Trajectory of the end-effector speed test. Starting from the initial joint position, \robot accelerates toward the terminal position.}
%     \label{fig:ee_speed}
% \end{figure}

\subsection{Impact force at the maximum velocity}
To evaluate the safety of \robot, we measure its impact force near maximum velocity using a push-pull gauge, as shown in Figure~\ref{fig:speed_payload_impact}c. Over eight repetitions, \robot records an average impact force of 50.5 N, with a maximum of 52.2 N and a standard deviation of 5.55 N. For comparison, a typical human palm strike at a similar speed generates approximately 575 N of impact force, based on an average effective hand mass of 1.39 kg~\cite{adamec2021biomechanical} and an impact duration of 13 ms~\cite{walilko2005biomechanics}, which is nearly 10 times that of \robot. While humans have soft skin and \robot does not, this demonstrates \robot can operate safely even at its maximum speed.

%These findings demonstrate that \robot operates safely within these conditions.

% \begin{figure}[hbt]
%     \centering
%     \includegraphics[width=0.99\linewidth]{fig/ee_impact.png}
%     \caption{We use push-pull gauge to measure the maximum impact force near the point of maximum speed.}
%     \label{fig:ee_impact}
% \end{figure}

\subsection{Payload}

We evaluate \robot's payload using dumbbells, as shown in Figure~\ref{fig:speed_payload_impact}b. The test involves performing bicep curls with the dumbbell.
% If the position error exceeds $5\deg$ at any of the joints, we consider the trial as a failure. 
By gradually increasing the weights, we find that \robot can hold up to 2.5 kg, compared to 3 kg for existing collaborative robots. The arm and gripper reliably grasp, lift, hold, and release the weight across ten consecutive trials without any damage or failure.

% \begin{figure}[hbt]
%     \centering
%     \includegraphics[width=0.5\linewidth]{fig/payload_fig_V2.png}
%     \caption{Sequence of a realistic dumbbell lifting task: the robot grasps the dumbbell, lifts it, holds it for three seconds, and places it down, mimicking human motion.}
%     \label{fig:ee_payload}
% \end{figure}

We analyze both mechanical stress and current to evaluate the reliability under the maximum payload. To measure mechanical stress, the robot configuration is set to position 2 in Figure~\ref{fig:speed_payload_impact}b. Each link’s stress limit is determined by its tensile strength, which is defined as the material’s resistance to breaking under tension. Using FEM analysis in Ansys, as shown in Figure~\ref{fig:payload}a, we find that the maximum stress experienced by the arm while lifting a 2.5 kg payload is 49.8 MPa, which is within the tensile strength of the PLA filament, 61 MPa. This indicates that \robot would deform, but not break and can safely operate under its maximum payload without significant risk of structural failure.

\begin{figure}[hbt]
\begin{subfigure}{\linewidth}
    \centering
    \includegraphics[width=0.7\linewidth]{fig/payload_fem_fig_V2.png}
    \caption{FEM result}
\end{subfigure}
\vfill
\begin{subfigure}{\linewidth}
    \centering
    \includegraphics[width=0.95\linewidth]{fig/payload_current_fig_V3.png}
    \caption{Current graph during payload experiment}
\end{subfigure}
\caption{(a) FEM analysis under a 2.5 kg payload in position 2 in Figure~\ref{fig:speed_payload_impact}b. Maximum deformation occurs at the gripper tip, measuring 4.60 mm, while the maximum stress is observed at the wrist link at 49.8 MPa. Note that the material of the TPU-based gripper is replaced with PLA for FEM analysis, as it is difficult to accurately simulate TPU's flexibility. (b) The average ratio of actuator current to its nominal value for each actuator and their standard deviations, measured during ten repetitions under a 2.5 kg payload.} 
\label{fig:payload}
\end{figure}

Each actuator has a nominal current limit, the maximum current it can handle without overheating. To evaluate whether the robot operates safely within this limit, we measure the input current throughout the entire payload test including lifting, holding, and releasing. Figure~\ref{fig:payload}b shows the current ratios relative to nominal values for each actuator. We can see that all currents are under their nominal values.

\begin{figure}[hbt]
    \centering
    \includegraphics[width=0.5\linewidth]{fig/repeat_iso_V2.png}
    \caption{Repeatability test setup with five end-effector points $P_1$ to $P_5$. The test plane is placed diagonally within the 250 mm cube. $P_1$ is positioned at the center of the plane, while $P_2$, $P_3$, $P_4$, and $P_5$ are near the corners, each located one-tenth of the cube's diagonal length from the center.}
    \label{fig:repeatability}
\end{figure}

\subsection{Repeatability}
To evaluate \robot's repeatability, we follow ISO 9283 standard~\cite{ISO9283:1998} to setup an experiment where the robot has to follow a designated set of points in a 250mm cube within \robot's workspace. Figure~\ref{fig:repeatability} shows a test plane placed diagonally within the cube. Five end-effector points $P_1$ to $P_5$ are marked on the plane, with $P_1$ at the center and $P_2$, $P_3$, $P_4$, and $P_5$ near the corners, each located one-tenth of the cube's diagonal length from the center. \robot sequentially moves through these five points and repeats the trajectory 30 times with its end-effector facing forward. We use 12 OptiTrack cameras monitor it simultaneously to record the end-effector's position. For each point, \( N=30 \) is the total number of measurements, and  $P_i\in \mathbb{R}^3, i=1, 2, \cdots, N $ is the measured position at the $i^{th}$ instance. The ISO 9283 standard states that the mean position of the end-effector $\bar{P} = \frac{1}{N} \sum_{i=1}^N P_i $ is calculated as the average of the measured positions $P_i$. The repeatability $R$ is computed based on the average (\( \mu \)) and the standard deviation (\( \sigma \)) of the distances between \( \bar{P} \) and $P_i$.

\begin{equation*}
    \mu = \frac{1}{N} \sum_{i=1}^N \lVert P_i - \bar{P} \rVert, \quad 
    \sigma = \sqrt{\frac{1}{N-1} \sum_{i=1}^N (\lVert P_i - \bar{P} \rVert)^2}
\end{equation*}

\begin{equation*}
    R = \mu + 3\sigma
\end{equation*} 

As shown in Table~\ref{tab:repeatability_results}, \robot achieves an average repeatability of 2.63 mm which means it is less consistent compared to the cobots with high gear ratio actuators, whose repeatability is around 0.1mm. This shows the trade-off between compliance and consistency.

\begin{table}[hbt]
\centering
\caption{ISO 9283 repeatability experiment results}
\label{tab:repeatability_results}
\resizebox{0.46\textwidth}{!}{%
\begin{tabular}{|c|c|c|c|}
\hline
Point & Average distance (mm) & Std dev. (mm) & Repeatability (mm)\\ \hline
$P_1$    & 1.048     & 0.546    & 2.687\\ %\hline
$P_2$    & 1.042     & 0.409    & 2.269\\ %\hline
$P_3$    & 0.939     & 0.689    & 3.006\\ %\hline
$P_4$    & 0.751     & 0.520    & 2.311\\ %\hline
$P_5$    & 1.104     & 0.584    & 2.857\\ \hline
Average    & 0.977     & 0.550    & 2.626\\ \hline

\multicolumn{3}{|c|}{Franka Panda~\cite{haddadin2022franka}} & 0.1 \\ 
\multicolumn{3}{|c|}{KUKA iiwa 7 R800} & 0.1 \\ 
\multicolumn{3}{|c|}{Quigley et al.~\cite{quigley2011low}} & *3 \\ 
\multicolumn{3}{|c|}{LIMS~\cite{kim2017anthropomorphic}} & 0.43 \\ 
\multicolumn{3}{|c|}{Nishii et al.~\cite{nishii2023ultra}} & *2.2 \\ 
\multicolumn{3}{|c|}{BLUE~\cite{gealy2019quasi}} & *3.7 \\ \hline
\end{tabular}%
% \begin{tablenotes}
% \item moving mass is defined as the arm’s mass, excluding body-mounted components and the gripper.
% \end{tablenotes}
}
\parbox{0.46\textwidth}{
\vspace{2mm}
\textit{``*"} indicates that repeatability is obtained manually, not by ISO 9283.

}
\end{table}


% Repeatability (mm)          & 2.626    & \textcolor{blue}{0.1}          & \textcolor{blue}{0.1}              & 3              & 0.425   & 2.2      & ?        & 3.7\\ %\hline
% Repeatability method          & ISO9283 & ISO9283      & ISO9283          & manually       & ISO9283 & manually & manually    & manually\\ %\hline

% \begin{table}[hbt]
% \centering
% \caption{Repeatability comparison}
% \label{tab:repeatability_comparison}
% \resizebox{0.46\textwidth}{!}{%
% \begin{tabular}{|c|c|c|c|}
% \hline
%                         & \robot (Ours) & Franka Panda~\cite{haddadin2022franka} & KUKA iiwa 7 R800 \\ \hline
% Repeatability (mm)      & 2.626    & 0.1    & 0.1\\ \hline
% \end{tabular}%
% }
% \begin{tablenotes}
% \item \textcolor{blue}{Blue} indicates strengths, \textcolor{red}{red} highlights weaknesses, and ``?" denotes information not provided in the paper.
% \item[1] moving mass is defined as the arm’s mass, excluding body-mounted components and the gripper.
% \end{tablenotes}
% \end{table}