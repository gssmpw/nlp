\documentclass[conference]{IEEEtran}
\usepackage{times}
\usepackage{caption}
\usepackage{subcaption}

% numbers option provides compact numerical references in the text. 
\usepackage[numbers]{natbib}
\usepackage{multicol}
\usepackage[bookmarks=true]{hyperref}
\usepackage{amsfonts}
\usepackage{amsmath}
\usepackage{comment}
\usepackage{graphicx}
\usepackage{algorithm}
\usepackage{algorithmic}
\usepackage{xcolor}
\usepackage{booktabs} % For prettier tables
\usepackage{threeparttable}
% \usepackage{appendix}
% \usepackage{enumitem} % For customizing lists
\usepackage{xspace}
\setlength{\abovedisplayskip}{3pt}
\setlength{\belowdisplayskip}{3pt}

% \title{Template paper for the \\Robotics: Science and Systems Conference}
\title{Design of a low-cost and lightweight 6 DoF bimanual arm for dynamic and contact-rich manipulation}

% \author{Jaehyung Kim$^{1}$, Jiho Kim$^{2}$, Dongryung Lee$^{1}$, Yujin Jang$^{2}$, Beomjoon Kim$^{1}$ \\

%         $^{1}$
%         Kim Jaechul Graduate School of AI at KAIST, 
%         {\tt\small\{kimjaehyung, dlee960504, beomjoon.kim\}@kaist.ac.kr} \\
%         $^{2}$
%         Mechanical System \& Design Engineering at Seoultech, 
%         {\tt\small\{wlgh3242, uzzz\}@seoultech.ac.kr}
% }

\author{
\authorblockN{Jaehyung Kim$^{1}$, Jiho Kim$^{2}$, Dongryung Lee$^{1}$, Yujin Jang$^{2}$, Beomjoon Kim$^{1}$} \\
\authorblockA{$^{1}$Kim Jaechul Graduate School of AI, KAIST\\
\{kimjaehyung, dlee960504, beomjoon.kim\}@kaist.ac.kr} \\
\authorblockA{$^{2}$Mechanical System \& Design Engineering, Seoultech\\
\{wlgh3242, uzzz\}@seoultech.ac.kr}
}



\let\oldtwocolumn\twocolumn
\renewcommand\twocolumn[1][]{%
    \oldtwocolumn[{#1}{
    \begin{center}
           \includegraphics[width=0.9\textwidth]{fig/first_figure_250122.png}
           \captionof{figure}{(a) Illustration of the ARMADA design. (b) ARMADA's lightweight 6-DoF bimanual arms enable dynamic manipulations such as object snatching and hammering. (c) The platform supports sim-to-real transfer of dynamic motions using reinforcement learning policies. (d) ARMADA also demonstrates the ability to shadow dynamic human movements.}
           \label{fig:overall_illustration}
        \end{center}
    }]
}

% Macros
\newcommand{\robot}{{ARMADA}\xspace} 

\begin{document}

% paper title


%\author{\authorblockN{Michael Shell}
%\authorblockA{School of Electrical and\\Computer Engineering\\
%Georgia Institute of Technology\\
%Atlanta, Georgia 30332--0250\\
%Email: mshell@ece.gatech.edu}
%\and
%\authorblockN{Homer Simpson}
%\authorblockA{Twentieth Century Fox\\
%Springfield, USA\\
%Email: homer@thesimpsons.com}
%\and
%\authorblockN{James Kirk\\ and Montgomery Scott}
%\authorblockA{Starfleet Academy\\
%San Francisco, California 96678-2391\\
%Telephone: (800) 555--1212\\
%Fax: (888) 555--1212}}


% avoiding spaces at the end of the author lines is not a problem with
% conference papers because we don't use \thanks or \IEEEmembership


% for over three affiliations, or if they all won't fit within the width
% of the page, use this alternative format:
% 
%\author{\authorblockN{Michael Shell\authorrefmark{1},
%Homer Simpson\authorrefmark{2},
%James Kirk\authorrefmark{3}, 
%Montgomery Scott\authorrefmark{3} and
%Eldon Tyrell\authorrefmark{4}}
%\authorblockA{\authorrefmark{1}School of Electrical and Computer Engineering\\
%Georgia Institute of Technology,
%Atlanta, Georgia 30332--0250\\ Email: mshell@ece.gatech.edu}
%\authorblockA{\authorrefmark{2}Twentieth Century Fox, Springfield, USA\\
%Email: homer@thesimpsons.com}
%\authorblockA{\authorrefmark{3}Starfleet Academy, San Francisco, California 96678-2391\\
%Telephone: (800) 555--1212, Fax: (888) 555--1212}
%\authorblockA{\authorrefmark{4}Tyrell Inc., 123 Replicant Street, Los Angeles, California 90210--4321}}


\maketitle
\thispagestyle{empty}
\pagestyle{empty}

\begin{abstract}
Dynamic and contact-rich object manipulation, such as striking, snatching, or hammering, remains challenging for robotic systems due to hardware limitations. Most existing robots are constrained by high-inertia design, limited compliance, and reliance on expensive torque sensors. To address this, we introduce ARMADA (Affordable Robot for Manipulation and Dynamic Actions), a 6 degrees-of-freedom bimanual robot designed for dynamic manipulation research. ARMADA combines low-inertia, back-drivable actuators with a lightweight design, using readily available components and 3D-printed links for ease of assembly in research labs.  The entire system, including both arms, is built for just \$6,100. Each arm achieves speeds up to 6.16m/s, almost twice that of most collaborative robots, with a comparable payload of 2.5kg. We demonstrate ARMADA can perform dynamic manipulation like snatching, hammering, and bimanual throwing in real-world environments. We also showcase its effectiveness in reinforcement learning (RL) by training a non-prehensile manipulation policy in simulation and transferring it zero-shot to the real world, as well as human motion shadowing for dynamic bimanual object throwing. ARMADA is fully open-sourced with detailed assembly instructions, CAD models, URDFs, simulation, and learning codes. We highly recommend viewing the supplementary video at \url{https://sites.google.com/view/im2-humanoid-arm}.
%It achieves a payload of 2.5 kg, a speed of 5.38 m/s, and a repeatability of 1.67 mm, whereas collaborative robots typically offer a payload of around 3 kg, speeds of 2–3 m/s, and a repeatability of 0.1 mm and are limited by their high-inertia designs and high gear ratios. 


\end{abstract}

\IEEEpeerreviewmaketitle

\section{Introduction}


\begin{figure}[t]
\centering
\includegraphics[width=0.6\columnwidth]{figures/evaluation_desiderata_V5.pdf}
\vspace{-0.5cm}
\caption{\systemName is a platform for conducting realistic evaluations of code LLMs, collecting human preferences of coding models with real users, real tasks, and in realistic environments, aimed at addressing the limitations of existing evaluations.
}
\label{fig:motivation}
\end{figure}

\begin{figure*}[t]
\centering
\includegraphics[width=\textwidth]{figures/system_design_v2.png}
\caption{We introduce \systemName, a VSCode extension to collect human preferences of code directly in a developer's IDE. \systemName enables developers to use code completions from various models. The system comprises a) the interface in the user's IDE which presents paired completions to users (left), b) a sampling strategy that picks model pairs to reduce latency (right, top), and c) a prompting scheme that allows diverse LLMs to perform code completions with high fidelity.
Users can select between the top completion (green box) using \texttt{tab} or the bottom completion (blue box) using \texttt{shift+tab}.}
\label{fig:overview}
\end{figure*}

As model capabilities improve, large language models (LLMs) are increasingly integrated into user environments and workflows.
For example, software developers code with AI in integrated developer environments (IDEs)~\citep{peng2023impact}, doctors rely on notes generated through ambient listening~\citep{oberst2024science}, and lawyers consider case evidence identified by electronic discovery systems~\citep{yang2024beyond}.
Increasing deployment of models in productivity tools demands evaluation that more closely reflects real-world circumstances~\citep{hutchinson2022evaluation, saxon2024benchmarks, kapoor2024ai}.
While newer benchmarks and live platforms incorporate human feedback to capture real-world usage, they almost exclusively focus on evaluating LLMs in chat conversations~\citep{zheng2023judging,dubois2023alpacafarm,chiang2024chatbot, kirk2024the}.
Model evaluation must move beyond chat-based interactions and into specialized user environments.



 

In this work, we focus on evaluating LLM-based coding assistants. 
Despite the popularity of these tools---millions of developers use Github Copilot~\citep{Copilot}---existing
evaluations of the coding capabilities of new models exhibit multiple limitations (Figure~\ref{fig:motivation}, bottom).
Traditional ML benchmarks evaluate LLM capabilities by measuring how well a model can complete static, interview-style coding tasks~\citep{chen2021evaluating,austin2021program,jain2024livecodebench, white2024livebench} and lack \emph{real users}. 
User studies recruit real users to evaluate the effectiveness of LLMs as coding assistants, but are often limited to simple programming tasks as opposed to \emph{real tasks}~\citep{vaithilingam2022expectation,ross2023programmer, mozannar2024realhumaneval}.
Recent efforts to collect human feedback such as Chatbot Arena~\citep{chiang2024chatbot} are still removed from a \emph{realistic environment}, resulting in users and data that deviate from typical software development processes.
We introduce \systemName to address these limitations (Figure~\ref{fig:motivation}, top), and we describe our three main contributions below.


\textbf{We deploy \systemName in-the-wild to collect human preferences on code.} 
\systemName is a Visual Studio Code extension, collecting preferences directly in a developer's IDE within their actual workflow (Figure~\ref{fig:overview}).
\systemName provides developers with code completions, akin to the type of support provided by Github Copilot~\citep{Copilot}. 
Over the past 3 months, \systemName has served over~\completions suggestions from 10 state-of-the-art LLMs, 
gathering \sampleCount~votes from \userCount~users.
To collect user preferences,
\systemName presents a novel interface that shows users paired code completions from two different LLMs, which are determined based on a sampling strategy that aims to 
mitigate latency while preserving coverage across model comparisons.
Additionally, we devise a prompting scheme that allows a diverse set of models to perform code completions with high fidelity.
See Section~\ref{sec:system} and Section~\ref{sec:deployment} for details about system design and deployment respectively.



\textbf{We construct a leaderboard of user preferences and find notable differences from existing static benchmarks and human preference leaderboards.}
In general, we observe that smaller models seem to overperform in static benchmarks compared to our leaderboard, while performance among larger models is mixed (Section~\ref{sec:leaderboard_calculation}).
We attribute these differences to the fact that \systemName is exposed to users and tasks that differ drastically from code evaluations in the past. 
Our data spans 103 programming languages and 24 natural languages as well as a variety of real-world applications and code structures, while static benchmarks tend to focus on a specific programming and natural language and task (e.g. coding competition problems).
Additionally, while all of \systemName interactions contain code contexts and the majority involve infilling tasks, a much smaller fraction of Chatbot Arena's coding tasks contain code context, with infilling tasks appearing even more rarely. 
We analyze our data in depth in Section~\ref{subsec:comparison}.



\textbf{We derive new insights into user preferences of code by analyzing \systemName's diverse and distinct data distribution.}
We compare user preferences across different stratifications of input data (e.g., common versus rare languages) and observe which affect observed preferences most (Section~\ref{sec:analysis}).
For example, while user preferences stay relatively consistent across various programming languages, they differ drastically between different task categories (e.g. frontend/backend versus algorithm design).
We also observe variations in user preference due to different features related to code structure 
(e.g., context length and completion patterns).
We open-source \systemName and release a curated subset of code contexts.
Altogether, our results highlight the necessity of model evaluation in realistic and domain-specific settings.






\section{Related Works}
\label{sec:related_works}


\noindent\textbf{Diffusion-based Video Generation. }
The advancement of diffusion models \cite{rombach2022high, ramesh2022hierarchical, zheng2022entropy} has led to significant progress in video generation. Due to the scarcity of high-quality video-text datasets \cite{Blattmann2023, Blattmann2023a}, researchers have adapted existing text-to-image (T2I) models to facilitate text-to-video (T2V) generation. Notable examples include AnimateDiff \cite{Guo2023}, Align your Latents \cite{Blattmann2023a}, PYoCo \cite{ge2023preserve}, and Emu Video \cite{girdhar2023emu}. Further advancements, such as LVDM \cite{he2022latent}, VideoCrafter \cite{chen2023videocrafter1, chen2024videocrafter2}, ModelScope \cite{wang2023modelscope}, LAVIE \cite{wang2023lavie}, and VideoFactory \cite{wang2024videofactory}, have refined these approaches by fine-tuning both spatial and temporal blocks, leveraging T2I models for initialization to improve video quality.
Recently, Sora \cite{brooks2024video} and CogVideoX \cite{yang2024cogvideox} enhance video generation by introducing Transformer-based diffusion backbones \cite{Peebles2023, Ma2024, Yu2024} and utilizing 3D-VAE, unlocking the potential for realistic world simulators. Additionally, SVD \cite{Blattmann2023}, SEINE \cite{chen2023seine}, PixelDance \cite{zeng2024make} and PIA \cite{zhang2024pia} have made significant strides in image-to-video generation, achieving notable improvements in quality and flexibility.
Further, I2VGen-XL \cite{zhang2023i2vgen}, DynamicCrafter \cite{Xing2023}, and Moonshot \cite{zhang2024moonshot} incorporate additional cross-attention layers to strengthen conditional signals during generation.



\noindent\textbf{Controllable Generation.}
Controllable generation has become a central focus in both image \citep{Zhang2023,jiang2024survey, Mou2024, Zheng2023, peng2024controlnext, ye2023ip, wu2024spherediffusion, song2024moma, wu2024ifadapter} and video \citep{gong2024atomovideo, zhang2024moonshot, guo2025sparsectrl, jiang2024videobooth} generation, enabling users to direct the output through various types of control. A wide range of controllable inputs has been explored, including text descriptions, pose \citep{ma2024follow,wang2023disco,hu2024animate,xu2024magicanimate}, audio \citep{tang2023anytoany,tian2024emo,he2024co}, identity representations \citep{chefer2024still,wang2024customvideo,wu2024customcrafter}, trajectory \citep{yin2023dragnuwa,chen2024motion,li2024generative,wu2024motionbooth, namekata2024sg}.


\noindent\textbf{Text-based Camera Control.}
Text-based camera control methods use natural language descriptions to guide camera motion in video generation. AnimateDiff \cite{Guo2023} and SVD \cite{Blattmann2023} fine-tune LoRAs \cite{hu2021lora} for specific camera movements based on text input. 
Image conductor\cite{li2024image} proposed to separate different camera and object motions through camera LoRA weight and object LoRA weight to achieve more precise motion control.
In contrast, MotionMaster \cite{hu2024motionmaster} and Peekaboo \cite{jain2024peekaboo} offer training-free approaches for generating coarse-grained camera motions, though with limited precision. VideoComposer \cite{wang2024videocomposer} adjusts pixel-level motion vectors to provide finer control, but challenges remain in achieving precise camera control.

\noindent\textbf{Trajectory-based Camera Control.}
MotionCtrl \cite{Wang2024Motionctrl}, CameraCtrl \cite{He2024Cameractrl}, and Direct-a-Video \cite{yang2024direct} use camera pose as input to enhance control, while CVD \cite{kuang2024collaborative} extends CameraCtrl for multi-view generation, though still limited by motion complexity. To improve geometric consistency, Pose-guided diffusion \cite{tseng2023consistent}, CamCo \cite{Xu2024}, and CamI2V \cite{zheng2024cami2v} apply epipolar constraints for consistent viewpoints. VD3D \cite{bahmani2024vd3d} introduces a ControlNet\cite{Zhang2023}-like conditioning mechanism with spatiotemporal camera embeddings, enabling more precise control.
CamTrol \cite{hou2024training} offers a training-free approach that renders static point clouds into multi-view frames for video generation. Cavia \cite{xu2024cavia} introduces view-integrated attention mechanisms to improve viewpoint and temporal consistency, while I2VControl-Camera \cite{feng2024i2vcontrol} refines camera movement by employing point trajectories in the camera coordinate system. Despite these advancements, challenges in maintaining camera control and scene-scale consistency remain, which our method seeks to address. It is noted that 4Dim~\cite{watson2024controlling} introduces absolute scale but in  4D novel view synthesis (NVS) of scenes.




% \begingroup
% Box for terms and concepts
\definecolor{lightbeige}{HTML}{eceae0} % Light beige color
\definecolor{darkbeige}{HTML}{b8b09a}  % Darker beige 
\definecolor{darkgray}{HTML}{696969}
\definecolor{gainsboro}{HTML}{DCDCDC}
\begin{tcolorbox}[
    colback=gainsboro,        % Light gray background
    colframe=darkgray,         % Black border
    boxrule=0.8pt,          % Border thickness
    width=\textwidth,       % Full width of text
    title=\textcolor{white}{PsychoCounsel Principles}, % Title of the box
    fonttitle=\bfseries,    % Bold title
    coltitle=black,         % Title text color
    left=2mm,               % Padding on the left
    right=2mm,              % Padding on the right
    top=2mm,                % Padding at the top
    bottom=2mm,             % Padding at the bottom
    before skip=10pt,       % Space before the box
    after skip=10pt         % Space after the box
]

% \setstretch{0.9} % Reduce line spacing for compactness
\noindent\textbf{Empathy and Emotional Understanding}: The response should convey genuine empathy, acknowledging and validating the client’s feelings and experiences. \\
\noindent\textbf{Personalization and Relevance}: The response should be tailored to the client’s unique situation, ensuring that the content is directly relevant to their concerns.\\
\noindent\textbf{Clarity and Conciseness}: The response should be clear, well-organized, and free of unnecessary jargon, making it easy for the client to understand and engage with.\\
\noindent\textbf{Avoidance of Harmful Language or Content}: The response should avoid any language or content that could potentially harm, distress, or trigger the client, ensuring the interaction is safe and supportive.\\
\noindent\textbf{Facilitation of Self-Exploration}: The response should encourage the client to reflect on their thoughts and feelings, promoting self-awareness and insight.\\
\noindent\textbf{Promotion of Autonomy and Confidence}: The response should support the client’s sense of control over their decisions and encourage confidence in their ability to make positive changes.\\
\noindent\textbf{Sensitivity to the Stage of Change}: The response should recognize the client’s current stage in the process of change and address their needs accordingly. If the client is in an early stage—uncertain or ambivalent about making a change—the response should help them explore their thoughts and motivations. If the client is in a later stage and has already made changes, the response should focus on reinforcing progress, preventing setbacks, and sustaining positive outcomes.
\end{tcolorbox}
\endgroup

\section{The \search\ Search Algorithm}
\label{sec:search}

%In traditional ML, structure changes and step (operator) changes are performed before model training, \ie, fixed to the training process, and weights are updated with SGD, because weights are continous, differentiable values, and there are significantly more weights than structure and operator changes. In workflow autotuning, all three types of cogs can be chosen with a unified search-based approach, because all of them are non-differentiable configurations and the number of cogs in different types are all small.
%Thus, \sysname\ only needs to navigate the search space of combination of cogs as the search space to produce its workflow optimization results.

%We propose, \textit{\textbf{\search}}, an adaptive hierarchical search algorithm that autotunes gen-AI workflows based on observed end-to-end workflow results. In each search iteration, \search\ selects a combination of cogs to apply to the workflow and executes the resulting workflow with user-provided training inputs. \search\ evaluates the final generation quality using the user-specified evaluator and measures the execution time and cost for each training input. These results are aggregated and serve as BO observations and pruning criteria.
%the optimizer can condition on and propose better configurations in later trials. The optimizer will also be informed about the violation of any user-specified metric thresholds. More details of this mechanism can be found in Appendix ~\ref{appdx:TPE}.

With our insights in Section~\ref{sec:theory}, we believe that search methods based on Bayesian Optimizer (BO) can work for all types of cogs in gen-AI workflow autotuning because of BO's efficiency in searching discrete search space.
A key challenge in designing a BO-based search is the limited search budgets that need to be used to search a high-dimensional cog space. 
For example, for 4 cogs each with 4 options and a workflow of 3 LLM steps, the search space is $4^{12}$. Suppose each search uses GPT-4o and has 1000 output tokens, the entire space needs around \$168K to go through. A user search budget of \$100 can cover only 0.06\% of the search space. A traditional BO approach cannot find good results with such small budgets.
%The entire search space grows exponentially with the number of cogs and the number of steps in a workflow. Moreover, different cogs and different combinations of cogs can have varying impacts on different workflows. 
%Without prior knowledge, it is difficult to determine the amount of budget to give to each cog.

To confront this challenge, we propose \textit{\textbf{\search}}, an adaptive hierarchical search algorithm that efficiently assigns search budget across cogs based on budget size and observed workflow evaluation results, as defined in Algorithms~\ref{alg:main} and \ref{alg:outer} and described below.
%autotunes gen-AI workflows based on observed end-to-end workflow results.
%\search\ includes a search layer partitioning method, a search budget initial assignment method, an evaluation-guided budget re-allocation mechanism, and a convergence-based early-exiting strategy. We discuss them in details below.

%\zijian{\search\ allows users to specify the optimization budget allowed in terms of the maximum number of search iterations. Based on the relationship between the complexity of the search space and the available budget, we will separate all tunable parameters into different layers each optimized by independent Bayesian optimization routines. Then we will decide the maximum budget each layer can get with a bottom-up partition strategy. Besides search space and resource partition, we also employ a novel allocation algorithm that integrates successive halving~\cite{successivehalving} and a convergence-based early exiting strategy to facilitate efficient usage of assigned budget.}


% The outermost layer searches and selects structures for a workflow; the middle layer searches and selects step options under the workflow structure selected in the outermost layer; the innermost layer searches and selects weights with the given workflow structure and steps. 

\begin{algorithm}[h]
    \caption{\search\ Algorithm}
    \label{alg:main}
      \small
\begin{algorithmic}[1]
\STATE \textbf{Global Value:} $R = \emptyset$ \COMMENT{Global result set}
%\STATE \textbf{Global Value:} $F = \emptyset$ \COMMENT{Global observation set}

%Reduct factor $\eta > 1$, explore width $W$
\STATE \textbf{Input:} User-specified Total Budget $TB$
\STATE \textbf{Input:} Cog set $C = \{c_{11},c_{12},...\}, \{c_{21},c_{22},...\}, \{c_{31},c_{32},...\}$

    \STATE
%\FOR{$i = 1,2,3$}
    %\COMMENT{$\alpha$ is a configurable value default to 1.1}
%\ENDFOR
%\STATE
%    \STATE \{$B_1,B_2,B_3$\} = LayerPartition($C$) \COMMENT{Calculate ideal layer budget}
    %\STATE \textbf{Glob}.budgets = budgets
%    \STATE opt\_layers = init\_opt\_routines() \COMMENT{A list of optimize routine each layer will use for search}
%\STATE
%    \FOR{$i \in L, \dots, 1$}
%        \IF{$i == L$}
 %           \STATE opt\_layers[L] = InnerLayerOpt
  %      \ELSE
   %         \STATE opt\_layers[i] = OuterLayerOpt
            %\STATE opt\_layers[i].next\_layer\_budgets = B[i+1]
            %\STATE opt\_layers[i].next\_layer\_routine = opt\_layers[i+1]
    %    \ENDIF
    %\ENDFOR
%\STATE opt\_layers[1].invoke($\emptyset$, B[1])
\STATE $U = 0$ \COMMENT{Used budget so far, initialize to 0}

\STATE \COMMENT{Perform search with 1 to 3 layers until budget runs out}
\FOR{$L = 1,2,3$} 
        \IF{$L=1$}
            \STATE $C_1 = C_1 \cup C_2 \cup C_3$ \COMMENT{Merge all cogs into a single layer}
        \ENDIF
        \IF{$L==2$}
            \STATE $C_1 = C_1 \cup C_2$ \COMMENT{Merge step and weight cogs}
            \STATE $C_2 = C_3$ \COMMENT{Architecture cog becomes the second layer}
        \ENDIF
        \STATE
    \FOR{$i = 1,..,L$}
    \STATE $NC_i = |C_i|$ \COMMENT{Total number of cogs in layer $L$} 
%    NO_i &= \sum_{L} \{\text{number of possible options in cog } c_{ij}\} \\
    \STATE $S_i = NC_i^\alpha$ \COMMENT{Estimated expected search size in layer $i$}
    \ENDFOR
    \STATE $E_L = \prod\limits_{i=1}^{L}S_i$ \COMMENT{Expected total search size in the current round}
    \STATE $E = TB - U > E_L$ ? $E_L$ : $(TB - U)$ \COMMENT{Consider insufficient budget} 
    \IF{$L==3$ and $(TB - U)$ > $E_L$}
         \STATE $E = TB - U$ \COMMENT{Spend all remaining budget if at 3 layer}
    \ENDIF
    %\STATE$TL = |N|$ \COMMENT{number of layers}
    \FOR{$i = 1,..,L$}
        \STATE $B_i =  \lfloor S_i \times \sqrt[L]{\frac{E}{E_L}}\rfloor$
        %$B$ = BudgetAssign($N$, $TL$, $TB$)
        \COMMENT{Assign budget proportionally to $S_i$}
    \ENDFOR
    \STATE
\STATE \texttt{LayerSearch} ($\emptyset$, $B$, $L$, $B_L$) \COMMENT{Hierarchical search from layer $L$}
\STATE
\STATE $U = U + E$
\IF{$U \geq TB$}
\STATE break \COMMENT{Stop search when using up all user budget}
\ENDIF
\ENDFOR
%\STATE
%\STATE $O$ = \texttt{SelectBestConfigs} ($R$)
%\IF{$L == 1$}
%    \STATE InnerLayerOpt($\emptyset$, B[1])
%\ELSE
%    \STATE OuterLayerOpt($\emptyset$, B[1], 1)
%\ENDIF
\STATE
\STATE \textbf{Output:} $O$ = \texttt{SelectBestConfigs} ($R$) \COMMENT{Return best optimizations}
\end{algorithmic}
\end{algorithm}

\subsection{Hierarchical Layer and Budget Partition}
\label{sec:ssp}

%We motivate \search's adaptive hierarchical search 
A non-hierarchical search has all cog options in a single-layer search space for an optimizer like BO to search, an approach taken by prior workflow optimizers~\cite{dspy-2-2024,gptswarm}.
With small budgets, a single-layer hierarchy allows BO-like search to spend the budget on dimensions that could potentially generate some improvements.
%While given enough budget, the single-layer space can be extensively searched to find global optimal, with little budget, 
However, a major issue with a single-layer search space is that a search algorithm like BO can be stuck at a local optimum even when budgets increase.
% (unless the budget is close to covering a very large space across dimensions).
To mitigate this issue, our idea is to perform a hierarchical search that works by choosing configurations in the outermost layer first, then under each chosen configuration, choosing the next layer's configurations until the innermost layer. 
With such a hierarchy, a search algorithm could force each layer to sample some values. Given enough budget, each dimension will receive some sampling points, allowing better coverage in the entire search space. However, with high dimensionality (\ie, many types of cogs) and insufficient budget, a hierarchical search may not be able to perform enough local search to find any good optimizations.

To support different user-specified budgets and to get the best of both approaches, we propose an adaptive hierarchical search approach, as shown in Algorithm~\ref{alg:main}.
\search\ starts the search by combining all cogs into one layer ($L=1$, line 9 in Algorithm~\ref{alg:main}) and estimating the expected search budget of this single layer to be the total number of cogs to the power of $\alpha$ (lines 16-19, by default $\alpha = 1.1$). This budget is then passed to the \texttt{LayerSearch} function (Algorithm~\ref{alg:outer}) to perform the actual cog search. When the user-defined budget is no larger than this estimated budget, we expect the single-layer, non-hierarchical search to work better than hierarchical search.
%as the budget for this single layer.

If the user-defined budget is larger, \search\ continues the search with two layers ($L=2$), combining step and weight cogs into the inner layer and architecture cogs as the outer layer (lines 11-14).
\search\ estimates the total search budget for this round as the product of the number of cogs in each of the two layers to the power of $\alpha$ (lines 16-20). It then distributes the estimated search budget between the two layers proportionally to each layer's complexity (lines 22-24) and calls the upper layer's \texttt{LayerSearch} function. Afterward, if there is still budget left, \search\ performs a last round of search using three layers and the remaining budget in a similar way as described above but with three separate layers (architecture as the outermost, step as the middle, and weight cogs as the innermost layer). Two or three layers work better for larger user-defined budgets, as they allow for a larger coverage of the high-dimensional search space.

Finally, \search\ combines all the search results to select the best configurations based on user-defined metrics (line 34).

%\search\ organizes cogs by having architecture cogs in the outer-most search layer, step cogs in the middle layer, and weight cogs in the inner-most layer (line 4 in Algorithm~\ref{alg:main}).
%This is because step cogs' input and output format are dependent on the workflow structure, and the effectiveness of weights (\eg, prompting) are dependent on steps (\eg, LLM model). 

% increases the number of layers until hitting the user-specified total search budget, $TB$

%Thus, the first step of \search\ is to determine the number of layers in its hierarchy and what cogs to include in a layer.
%Intuitively, structure cogs should be placed in the outer-most search layer to be determined first before exploring other cogs. This is because other cogs change node and edge values, and it is easier for 
%However, instead of a fixed number of layers in the hierarchy, we adapt the cog layering according to user-specified total search budgets, $TB$, and the complexity of each layer, using Algorithm~\ref{alg:main}.

% the following \texttt{LayerPartition} method.
%We begin by modeling the relationship between the expected number of evaluations and the number of cogs as well as the number of options in each layer:

%We first consider the identity of each cog in the search space. All structure-cogs will be placed in the outer-most search layer exclusively, which is similar to non-differentiable NAS in traditional ML training. This layer will fix the workflow graph and pass it to the following layer, allowing a stabilized search space for faster convergence.

%Since step-cogs will not create a changing search space, the partition of step-cogs and weight-cogs is conditioned on the search space complexity and the given total budget. Separating step-cogs out can benefit from a more flexible budget allocation strategy and broader exploration for local search at weight-cogs but performs poorly when the given budget is more constrained, in that case, we will optimize them jointly in the same layer.


%\small
%\begin{align*}
%    C &= \{c_{11},c_{12},...\}, \{c_{21},c_{22},...\}, \{c_{31},c_{32},...\} \\
%    NC_i &= \text{total number of cogs in layer i} \\
%    NO_i &= \sum_{j} \{\text{number of possible options in cog } c_{ij}\} \\
%    N_i &= max(NC_i^\alpha,NO_i) \\
%    N_i &= \sum_{j} \{\text{number of possible options in } C_{ij}\} \\
%    N_i &= max(|C_i|^\alpha, N_i) \\
%    B_j &= \prod\limits_{i=1}^{j}N_i, j \in \{1,2,3\}
%\end{align*}

%\normalsize
%where $L$ represents the total number of layers and can be 1, 2, or 3. 
%$C$ represents the entire cog search space, with each row $c_{i*}$ being one of the three types of cogs and lower layers having lower-numbered rows (\eg, $c_{1*}$ being weight cogs). $NC_i$ is the number of cogs in layer $i$, and $NO_i$ is the total number of options across all cogs in layer $i$. $N_i$ is our estimation of the complexity of layer $i$ based on $NC_i$ and $NO_i$ ($\alpha$ is a configurable weight to control the importance between $NC_i$ and $NO_i$; by default $\alpha = 1.1$). 
%$\alpha$ stands for a control parameter, setting the intensity of this scaling behavior w.r.t the number of cogs, we found that $\alpha = 1.2$ is empirically sufficient and efficient for optimizing real workloads. 
%$B_j$ is the expected total number of workflow evaluations for all the lower $j$ layers.
%After calculating $B_1$, $B_2$, and $B_3$, we compare the total budget $TB$ with them.
%When $TB \geq B_3$, we set the total number of layers, $TL$, to 3. When $B_2 \leq TB < B_3$, we set the total number of layers to 2 and merge the step and weight cogs into one layer. When $TB < B_1$, we put all cogs in one layer.
%We only create a separate layer for step-cogs when the given budget $TB$ is greater or equal to the total expected budget for three layers.

%\subsection{Seach Budget Partition}
%\label{sec:sbp}
%After determining cog layers, we distribute the total budget, $TB$, across the layers proportionally to each layer's expected budget $N_i$: , which is the \texttt{BudgetAssign} function.
%We follow a bottom-up partition strategy, where lower layers will try to greedily take the expected budget. This stems from two simple heuristics: (1) feedback to the upper layer is more accurate when the succeeding layer is trained with enough iterations, and (2) the effectiveness of a structure change depends on the setting of individual steps in the workflow (\eg, majority voting is more powerful when each LLM-agent is embedded with diverse few-shot examples or reasoning styles). In cases where the given resource exceeds the total expected budget, 
%We assign $TB$ across layers proportionally to their expected budget $N_i$. 
%The budget assigned at each layer $B_i$ given the total available number of evaluations $TB$ is obtained as follows:

%\small
%\begin{align}
%B_i &=  \lfloor N_i \times \sqrt[L]{\frac{TB}{B^*}}\rfloor
%    B_L &= \begin{cases}
%        min(N_L, TB) & TB < B^* \\
%        \lfloor N_L \times \sqrt[L]{\frac{TB}{B^*}}\rfloor & TB \geq B^*
%    \end{cases}
%    \\
%    B_i &= \begin{cases}
%        min(N_i, \lfloor\frac{TB}{\prod_{j=i+1}^L B_j}\rfloor) & TB < B^* \\
%        \lfloor N_i \times \sqrt[L]{\frac{TB}{B^*}}\rfloor & TB \geq B^*
%    \end{cases}
%\end{align}

%\normalsize


\subsection{Recursive Layer-Wise Search Algorithm}
%The calculation above pre-assigns cogs to layers and search budgets to each layer. 
We now introduce how \search\ performs the actual search in a recursive manner until the inner-most layer is searched, as presented in Algorithm~\ref{alg:outer} \texttt{LayerSearch}. 
Our overall goal is to ensure strong cog option coverage within each layer while quickly directing budgets to more promising cog options based on evaluation results.
%So far, we have determined the optimization layer structure and the maximum allowed search iteration each layer will get. Next, we introduce how the budget is consumed in each layer. The inner-most layer, where weight-cogs, and potentially step-cogs, reside, follows the conventional Bayesian optimization process, exhausting all budgets unless an early stop signal is sent. This signal will be triggered when the current optimizer witnesses $p$ consecutive iterations without any improvements above the threshold. All optimization layers use early stopping to avoid budget waste.
%Algorithm~\ref{alg:inner} describes the search happening at the inner-most (bottom) layer, and 
Specifically, every layer's search is under a chosen set of cog configurations from its upper layers ($C_{chosen}$) and is given a budget $b$. 
In the inner-most layer (lines 7-20), \search\ samples $b$ configurations and evaluates the workflow for each of them together with the configurations from all upper layers ($C_{chosen}$). The evaluation results are added to the feedback set $F$ as the return of this layer.

\begin{algorithm}[h]
  %\algsetup{linenosize=\tiny}
  \small
    \caption{\texttt{LayerSearch} Function}
    \label{alg:outer}
\begin{algorithmic}[1]
%\STATE \textbf{Global Config:} Reduct factor $\eta > 1$, explore width $W$
\STATE \textbf{Global Value:} $R$ \COMMENT{Global result set}
%\STATE \textbf{Global Value:} $F$ \COMMENT{Global observation set}
\STATE \textbf{Input:} $C_{chosen}$: configs chosen in upper layers
\STATE \textbf{Input:} $B$: Array storing assigned budgets to different layers
\STATE \textbf{Input:} $curr\_layer$: this layer's level
\STATE \textbf{Input:} $curr\_b$: this layer's assigned budget
%\STATE
%\FUNCTION{LayerSearch\hspace{0.4em}($C_{chosen}$, $B$, $curr\_layer$, $curr\_b$)}

    \STATE
    \STATE \COMMENT{Search for inner-most layer}
    \IF{curr\_layer == 1}
        \STATE $F = \emptyset$ \COMMENT{Init this layer's feedback set to empty}
        %\STATE $F^{\prime} = match(C_{chosen}, F)$ \COMMENT{Local feedback set}
        \FOR{$k = 0, \dots, curr\_b$}
            \STATE $\lambda$ = \texttt{TPESample} (1) \COMMENT{Sample one configuration using TPE}
            \STATE $f = $ \texttt{EvaluateWorkflow} ($C_{chosen} \cup \lambda$)
            \STATE $R = R \cup \{C_{chosen} \cup \lambda\}$ \COMMENT{Add configuration to global $R$}
            \IF{\texttt{EarlyStop} (f)}
            \STATE break
            \ENDIF
            \STATE $F = F \cup \{f\}$ \COMMENT{Add evaluate result to feedback $F$}
        \ENDFOR
        %\STATE $F = F \cup F^{\prime}$
        \STATE \textbf{Return} $F$
    \ENDIF
    \STATE
    \STATE \COMMENT{Search for non-inner-most layer}
    %\STATE $K = \lfloor \frac{b}{W} \rfloor$, 
    \STATE $b\_used = 0$, $TF = \emptyset$ \COMMENT{Init this layer's used budget and feedback set}
    \STATE $R = \lceil\frac{curr\_b}{\eta}\rceil$, $S = \lfloor\frac{curr\_b}{R}\rfloor$ \COMMENT{Set $R$ and $S$ based on $curr\_b$}
    \STATE
    \WHILE{$b\_{used}$ $\leq$ $curr\_b$}
        \STATE \COMMENT{Sample $W$ configs at a time until running out of $curr\_b$}
        \STATE $n = (curr\_b - b_{used})$ > $W$ ? $W$ : $(curr\_b - b_{used})$
        %\IF{$b - b_{used} < W$}
        %    \STATE $n = b_l - b_{used}$
        %\ELSE
         %   \STATE $n=W$
        %\ENDIF
        \STATE $b\_used$ += $n$
        %\STATE $n = \text{min}(W,\ b_l - kW)$ \COMMENT{Propose $W$ configs and meet $b_l$ constraint}
        \STATE $\Theta = $ \texttt{TPESample} ($n$) \COMMENT{Sample a chunk of $n$ configs in the layer} 
        %\STATE $F^{\prime} = match(C_{chosen}, F)$ \COMMENT{Per-chunk feedback set}
        \STATE $F = \emptyset$ \COMMENT{Init this layer's feedback set to empty}
        \STATE
        \FOR{$s = 0, 1, \dots, S$}
            \STATE $r_s = R\cdot \eta^s$
            \FOR{$\theta \in \Theta$}
                %\IF{$curr\_layer < max\_layer$}
                    \STATE $f =$ \texttt{LayerSearch} ($C_{chosen} \cup \{\theta\}$, $B$, curr\_layer$-1$, $r_s$)
                %\ELSE
                %    \STATE $f =$ InnerOpt($\gamma \cup \{\theta\}$, $r_s$)
                %\STATE $f$ = $opt\_layers[current\_layer+1](\gamma \cup \{\theta\}, r_s)$ \{Optimize the current config at the next layer with $r_s$ budget \}
                %\ENDIF
                \STATE $F = F \cup f$ \COMMENT{Add evaluate result to feedback}
                \IF{\texttt{EarlyStop} ($f$)}
                    \STATE $\Theta = \Theta - \{\theta\}$ \COMMENT{Skip converged configs}
                \ENDIF
            \ENDFOR
            \STATE $\Theta$ = Select top $\lfloor \frac{|\Theta|}{\eta}\rfloor$ configs from $F$ based on user-specified metrics
        \ENDFOR
        \STATE
        \IF{\texttt{EarlyStop} ($F$)}
            \STATE break \COMMENT{Skip remaining search if results converged}
        \ENDIF
        \STATE $TF = TF \cup F$
    \ENDWHILE
    %\STATE $F = F \cup TF$
        \STATE \textbf{Return} $TF$

%\ENDFUNCTION

%\STATE \textbf{Output:} Best metrics in all trials
\end{algorithmic}
\end{algorithm}

% consumption\_nextlayer\_bucket = WSR

% for s in 0, 1,...S do
%     w = W*\eta^{s}
%     r = R*\eta^{-s}

% total budget at next layer = b_l / W * WSR = b_l * SR

% b_l * SR <= b_l * B_l+1

% S = B_{l+1} / R



For a non-inner-most layer, \search\ samples a chunk ($W$) of points at a time using the TPE BO algorithm~\cite{bergstra2011tpe} until all this layer's pre-assigned budget is exhausted (lines 27-30). Within a chunk, \search\ uses a successive-halving-like approach to iteratively direct the search budget to more promising configurations within the chunk (the dynamically changing set, $\Theta$). In each iteration, \search\ calls the next-level search function for each sampled configuration in $\Theta$ with a budget of $r_s$ and adds the evaluation observations from lower layers to the feedback set $F$ for later TPE sampling to use (lines 35-37).
In the first iteration ($s=0$), $r_s$ is set to $R\cdot \eta^0=R$ (line 34). After the inner layers use this budget to search, \search\ filters out configurations with lower performance and only keeps the top $\lfloor \frac{|\Theta|}{\eta}\rfloor$ configurations as the new $\Theta$ to explore in the next iteration (line 42). In each next iteration, \search\ increases $r_s$ by $\eta$ times (line 34), essentially giving more search budget to the better configurations from the previous iteration.

The successive halving method effectively distributes the search budget to more promising configurations, while the chunk-based sampling approach allows for evaluation feedback to accumulate quickly so that later rounds of TPE can get more feedback (compared to no chunking and sampling all $b$ configurations at the same time). To further improve the search efficiency, we adopt an {\em early stop} approach where we stop a chunk or a layer's search when we find its latest few searches do not improve workflow results by more than a threshold, indicating convergence (lines 14,38,45).

%algorithm takes as input other cog settings from previous layers and the assigned budget at the current layer. It tiles the search loop into fixed-size blocks (line 4), each runs the SuccessiveHalving (SH) subroutine in the inner loop (line 7-15). In each SH iteration, only top-$1/\eta$-quantile configurations in $\Theta$ will continue in the next round with $\eta$ times larger budget consumption. As a result, exponentially more trials will be performed by more promising configurations. 

%On average, \textit{Outer-layer search} will create $K$ brackets, each granting approximately $WRS$ budget to the next layer. $R$ represents the smallest amount of resource allocated to any configurations in $\Theta$. 

% layer - 1: budget = 4
% K * W <= b\_current layer
% layer -1: itear 0: propose 2

%     SH:
%     2 config -> R
%     1 config -> 2R

%     iteration 1: propose 2 = W
%     SH:
%     2 config -> R
%     1 config -> 2R

% W configs; each has R resource

% W / eta configs; each has R * eta resource

% R -> least resource one config can get = B2 - smth
% R + R*eta + ... + R*eta\^s -> most promising = B2 + smth


% $L2$ is the middle layer where structure-cogs and step-cogs may be placed exclusively. We employ hyperband for its robustness in exploration and exploitation trace-off. If this layer exists, it will instruct $L1$ the number of search iterations to run in each invocation. Specifically, in each iteration at line 4, \sysname will propose $n$ configurations and run SuccessiveHalving (SH) subroutine (line 8-15). SH will optimize each proposal and use the search results from $L1$ to rank their performance. Each time only the top-performing $n \cdot \eta^{-i}$ can continue in the next round with a larger budget. With this strategy, exponentially more search budgets are allocated to more promising configs at $L2$.

% \input{algo-l2-search}

% $L3$ is the outer-most layer for structure-cogs only when $L2$ is created. For this layer, we employ plain SH without hyperband because of its predictable convergence behavior. This is mainly due to two factors: (1) structure change to the workflow is more significant thus different configurations are more likely to deviate after training with the following layers. (2) with the search space partition strategy in Sec ~\ref{sec:ssp}, we can assume the available budget at each layer is substantial when $L3$ exists. Given this prior knowledge, we can avoid grid searching control parameter $n$ as in the hyperband but adopt a more aggressive allocation scheme to bias towards better proposals and moderate search wastes.



%\subsubsection{Runtime Budget Adaptation}
%Using static estimation of the expected budget for each layer is not enough, we also adjust the assignment during the optimization based on real convergence behavior. Specifically, for layer $i$, we record the number of configurations evaluated in each optimize routine. We set the convergence indicator $C_{ij}$ of $j^{th}$ routine with this number if the search early exits before reaching the budget limit, otherwise 2\x of its assigned resource. Then we update $E_i$ with $\frac{\sum_{j}^M C_{ij}}{M}$. \sysname\ will update the budget partition according to Sec~\ref{sec:sbp} for any newly spawned optimizer routines. Besides controlling the proportion of budgets across layers, a smaller/larger $B_{l+1}$ will also guide the SH in Alg~\ref{alg:outer} to shrink/extend the budget $R$ for differentiating config performance.


\section{\sysname\ Design}
\label{sec:cognify}

We build \sysname, an extensible gen-AI workflow autotuning platform based on the \search\ algorithm. The input to \sysname\ is the user-written gen-AI workflow (we currently support LangChain \cite{langchain-repo}, DSPy \cite{khattab2024dspy}, and our own programming model), a user-provided workflow training set, a user-chosen evaluator, and a user-specified total search budget. \sysname\ currently supports three autotuning objectives: generation quality (defined by the user evaluator), total workflow execution cost, and total workflow execution latency. Users can choose one or more of these objectives and set thresholds for them or the remaining metrics (\eg, optimize cost and latency while ensuring quality to be at least 5\% better than the original workflow). 
\sysname\ uses the \search\ algorithm to search through the cog space.
When given multiple optimization objectives, \sysname\ maintains a sorted optimization queue for each objective and performs its pruning and final result selection from all the sorted queues (possibly with different weighted numbers).
To speed up the search process, we employ parallel execution, where a user-configurable number of optimizers, each taking a chunk of search load, work together in parallel. %Below, we introduce each type of cogs in more details.
\sysname\ returns multiple autotuned workflow versions based on user-specified objectives.
\sysname\ also allows users to continue the auto-tuning from a previous optimization result with more budgets so that users can gradually increase their search budget without prior knowledge of what budget is sufficient.
Appendix~\ref{sec:apdx-example} shows an example of \sysname-tuned workflow outputs. 
\sysname\ currently supports six cogs in three categories, as discussed below. 

%In \sysname, we call every workflow optimization technique a {\em cog}, including structure-changing cogs like task decomposition, step-changing cogs like model selection, and weight-changing cogs like adding few-shot examples to prompts. 
%\sysname\ places structure-changing cogs in the outermost layer, step cogs in the middle layer, and weight cogs in the innermost layer, because \fixme{TODO}.


\subsection{Architecture Cogs}
\label{sec:structure-cog}
%Changing the structure of a workflow can potentially improve its generation quality (\eg, by using multiple steps to attempt at a task in parallel or in chain) or reduce its execution cost and latency (\eg, by merging or removing steps).
\sysname\ currently supports two architecture cogs: task decomposition and task ensemble.
Task decomposition~\cite{khot2023decomposed} breaks a workflow step into multiple sub-steps and can potentially improve generation quality and lower execution costs, as decomposed tasks are easier to solve even with a small (cheaper) model.
There are numerous ways to perform task decomposition in a workflow. 
%, as all LM steps can potentially be decomposed and into different numbers of sub-steps in different ways. Throwing all options to the Bayesian Optimizer would drastically increase the search space for \sysname. 
To reduce the search space, we propose several ways to narrow down task decomposition options. Even though we present these techniques in the setting of task decomposition, they generalize to many other structure-changing tuning techniques.

%First, we narrow down a selected set of steps in a workflow to decompose. 
Intuitively, complex tasks are the hardest to solve and worth decomposition the most. We use a combination of LLM-as-a-judge \cite{vicuna_share_gpt} and static graph (program) analysis to identify complex steps. We instruct an LLM to give a rating of the complexity of each step in a workflow. We then analyze the relationship between steps in a workflow and find the number of out-edges of each step (\ie, the number of subsequent steps getting this step's output). More out-edges imply that a step is likely performing more tasks at the same time and is thus more complex. We multiply the LLM-produced rating and the number of out-edges for each step and pick the modules with scores above a learnable threshold as the target for task decomposition. We then instruct an LLM to propose a decomposition (\ie, generate the submodules and their prompts) for each candidate step. %We provide the LLM with few-shot examples for what proposed modules for a separate task could look like. We also add a refinement step that validates whether the proposition decomposition maintains the semantics of the original trajectory. Once candidate decompositions are generated, those are used for the entirety of the optimization.

{
\begin{figure*}[t!]
\begin{center}
\centerline{\includegraphics[width=0.95\textwidth]{Figures/big_grid.pdf}}
\vspace{-0.1in}
\mycap{Generation Quality vs Cost/Latency.}{Dashed lines show the Pareto frontier (upper left is better). Cost shown as model API dollar cost for every 1000 requests. Cognify selects models
from GPT-4o-mini and Llama-8B. DSPy and Trace do not support model selection and are given GPT-4o-mini for all steps. Trace results for Text-2-SQL and FinRobot have 0 quality and are not included.} 
\Description{Eight graphs with different shapes representing baselines compared to points on a Pareto frontier.}
\label{fig-biggrid}
\end{center}
\end{figure*}
}


The second structure-changing cog that \sysname\ supports is task ensembling. This cog spawns multiple parallel steps (or samplers) for a single step in the original workflow, as well as an aggregator step that returns the best output (or combination of outputs). By introducing parallel steps, \sysname\ can optimize these independently with step and weight cogs. This provides the aggregator with a diverse set of outputs to choose from. 
%The aggregator is prompted with the role of the samplers, as well as the inputs to each. It also receives a criteria by which it should make a decision. We choose to prompt it with a qualitative description of how it should resolve discrepancies between outputs. 


\subsection{Step Cogs}
We currently support two step-changing cogs: model selection for language-model (LM) steps and code rewriting for code steps.

For model selection, to reduce its search space, we identify ``important'' LM steps---steps that most critically impact the final workflow output to reduce the set \search\ performs TPE sampling on. Our approach is to test each step in isolation by freezing other steps with the cheapest model and trying different models on the step under testing. 
We then calculate the difference between the model yielding the best and worst workflow results as the importance of the step under testing. %For each model, we get the workflow output quality score using sampled user-supplied inputs and user-specific evaluator. We then calculate the difference between the highest and lowest scores as this module's importance. 
After testing all the steps, we choose the steps with the highest K\% importance as the ones for TPE to sample from.
%, where K is determined based on user-chosen stop criteria. We then initialize the Bayesian optimization to start with the state where important modules use the largest model and all other modules use the cheapest model. We set the TPE optimization bandwidth of each module to be the inverse of importance, \ie, the more important a module is the more TPE spends on optimizing.

The second step cog \sysname\ supports is code rewriting, where it automatically changes code steps to use better implementation. To rewrite a code step, \sysname\ finds the $k$ worst- and best-performing training data points and feeds their corresponding input and output pairs of this code step to an LLM. We let the LLM propose $n$ new candidate code pieces for the step at a time.
%in parallel to generate a set of $n$ candidates.
In subsequent trials, the optimizer dynamically updates the candidate set using feedback from the evaluator.


\subsection{Weight Cogs}
\sysname\ currently supports two weight-changing cogs: reasoning and few-shot examples.
First, \sysname\ supports adding reasoning capability to the user's original prompt, with two options: zero-shot Chain-of-Thought \cite{wei2022chain} (\ie, ``think step-by-step...'') and dynamic planning \cite{huang2022language} (\ie, ``break down the task into simpler sub-tasks...''). These prompts are appended to the user's prompt. In the case where the original module relies on structured output, we support a reason-then-format option that injects reasoning text into the prompt while maintaining the original output schema.

Second, \sysname\ supports dynamically adding few-shot examples to a prompt. At the end of each iteration, we choose the top-$k$-performing examples for an LM step in the training data and use their corresponding input-output pairs of the LM step as the few-shot examples to be appended to the original prompt to the LM step for later iterations' TPE sampling. As such, the set of few-shot examples is constantly evolving during the optimization process based on the workflow's evaluation results. 
%Few-shot examples are available to all modules, even intermediate steps in the workflow. We use the full trajectory of each request to generate examples for the intermediate steps. Furthermore, we automatically filter out examples that do not meet a user-specified threshold. 




\section{Mechanical Analysis}

\subsection{End-effector speed}

%To evaluate the dynamic performance and ensure safety given the robot arm's low inertia, 
To test whether \robot can perform dynamic manipulation, we measure its end-effector maximum speed. To do this, we initialize the robot with its right end-effector positioned at the opposite shoulder with the elbow completely flexed, and then fully extend the elbow downward, as shown in Figure~\ref{fig:speed_payload_impact}a. The joint trajectory is generated by interpolating between the initial and final configurations, and joint position control is used to follow the trajectory. \robot achieves an average maximum end-effector speed of 6.16 m/s with a standard deviation of 0.472 m/s across 40 repetitions, without any damage to the arm or power supply.

\begin{figure}
    \begin{subfigure}[b]{0.47\linewidth}
        \centering
        \includegraphics[width=0.92\linewidth]{fig/ee_speed_fig_V4.png}
        \caption{Speed test}
    \end{subfigure}
    \begin{subfigure}[b]{0.471\linewidth}
        \centering
        \includegraphics[width=\linewidth]{fig/payload_fig_V2.png}
        \caption{Lifting test}
    \end{subfigure}
    \begin{subfigure}[t]{\linewidth}
        \centering
        \includegraphics[width=0.99\linewidth]{fig/ee_impact.png}
        \caption{Impact test}
    \end{subfigure}
    \caption{(a) Trajectory of the end-effector speed test. Starting from the initial joint position, \robot accelerates toward the terminal position. (b) A sequence of a realistic dumbbell lifting task. The robot grasps, lifts, holds for three seconds, and places down the dumbbell. (c) We use push-pull gauge to measure the maximum impact force near the point of maximum speed.}
    \label{fig:speed_payload_impact}
\end{figure}

% \begin{figure}[hbt]
%     \centering
%     \includegraphics[width=0.5\linewidth]{fig/ee_speed_fig_V4.png}
%     \caption{Trajectory of the end-effector speed test. Starting from the initial joint position, \robot accelerates toward the terminal position.}
%     \label{fig:ee_speed}
% \end{figure}

\subsection{Impact force at the maximum velocity}
To evaluate the safety of \robot, we measure its impact force near maximum velocity using a push-pull gauge, as shown in Figure~\ref{fig:speed_payload_impact}c. Over eight repetitions, \robot records an average impact force of 50.5 N, with a maximum of 52.2 N and a standard deviation of 5.55 N. For comparison, a typical human palm strike at a similar speed generates approximately 575 N of impact force, based on an average effective hand mass of 1.39 kg~\cite{adamec2021biomechanical} and an impact duration of 13 ms~\cite{walilko2005biomechanics}, which is nearly 10 times that of \robot. While humans have soft skin and \robot does not, this demonstrates \robot can operate safely even at its maximum speed.

%These findings demonstrate that \robot operates safely within these conditions.

% \begin{figure}[hbt]
%     \centering
%     \includegraphics[width=0.99\linewidth]{fig/ee_impact.png}
%     \caption{We use push-pull gauge to measure the maximum impact force near the point of maximum speed.}
%     \label{fig:ee_impact}
% \end{figure}

\subsection{Payload}

We evaluate \robot's payload using dumbbells, as shown in Figure~\ref{fig:speed_payload_impact}b. The test involves performing bicep curls with the dumbbell.
% If the position error exceeds $5\deg$ at any of the joints, we consider the trial as a failure. 
By gradually increasing the weights, we find that \robot can hold up to 2.5 kg, compared to 3 kg for existing collaborative robots. The arm and gripper reliably grasp, lift, hold, and release the weight across ten consecutive trials without any damage or failure.

% \begin{figure}[hbt]
%     \centering
%     \includegraphics[width=0.5\linewidth]{fig/payload_fig_V2.png}
%     \caption{Sequence of a realistic dumbbell lifting task: the robot grasps the dumbbell, lifts it, holds it for three seconds, and places it down, mimicking human motion.}
%     \label{fig:ee_payload}
% \end{figure}

We analyze both mechanical stress and current to evaluate the reliability under the maximum payload. To measure mechanical stress, the robot configuration is set to position 2 in Figure~\ref{fig:speed_payload_impact}b. Each link’s stress limit is determined by its tensile strength, which is defined as the material’s resistance to breaking under tension. Using FEM analysis in Ansys, as shown in Figure~\ref{fig:payload}a, we find that the maximum stress experienced by the arm while lifting a 2.5 kg payload is 49.8 MPa, which is within the tensile strength of the PLA filament, 61 MPa. This indicates that \robot would deform, but not break and can safely operate under its maximum payload without significant risk of structural failure.

\begin{figure}[hbt]
\begin{subfigure}{\linewidth}
    \centering
    \includegraphics[width=0.7\linewidth]{fig/payload_fem_fig_V2.png}
    \caption{FEM result}
\end{subfigure}
\vfill
\begin{subfigure}{\linewidth}
    \centering
    \includegraphics[width=0.95\linewidth]{fig/payload_current_fig_V3.png}
    \caption{Current graph during payload experiment}
\end{subfigure}
\caption{(a) FEM analysis under a 2.5 kg payload in position 2 in Figure~\ref{fig:speed_payload_impact}b. Maximum deformation occurs at the gripper tip, measuring 4.60 mm, while the maximum stress is observed at the wrist link at 49.8 MPa. Note that the material of the TPU-based gripper is replaced with PLA for FEM analysis, as it is difficult to accurately simulate TPU's flexibility. (b) The average ratio of actuator current to its nominal value for each actuator and their standard deviations, measured during ten repetitions under a 2.5 kg payload.} 
\label{fig:payload}
\end{figure}

Each actuator has a nominal current limit, the maximum current it can handle without overheating. To evaluate whether the robot operates safely within this limit, we measure the input current throughout the entire payload test including lifting, holding, and releasing. Figure~\ref{fig:payload}b shows the current ratios relative to nominal values for each actuator. We can see that all currents are under their nominal values.

\begin{figure}[hbt]
    \centering
    \includegraphics[width=0.5\linewidth]{fig/repeat_iso_V2.png}
    \caption{Repeatability test setup with five end-effector points $P_1$ to $P_5$. The test plane is placed diagonally within the 250 mm cube. $P_1$ is positioned at the center of the plane, while $P_2$, $P_3$, $P_4$, and $P_5$ are near the corners, each located one-tenth of the cube's diagonal length from the center.}
    \label{fig:repeatability}
\end{figure}

\subsection{Repeatability}
To evaluate \robot's repeatability, we follow ISO 9283 standard~\cite{ISO9283:1998} to setup an experiment where the robot has to follow a designated set of points in a 250mm cube within \robot's workspace. Figure~\ref{fig:repeatability} shows a test plane placed diagonally within the cube. Five end-effector points $P_1$ to $P_5$ are marked on the plane, with $P_1$ at the center and $P_2$, $P_3$, $P_4$, and $P_5$ near the corners, each located one-tenth of the cube's diagonal length from the center. \robot sequentially moves through these five points and repeats the trajectory 30 times with its end-effector facing forward. We use 12 OptiTrack cameras monitor it simultaneously to record the end-effector's position. For each point, \( N=30 \) is the total number of measurements, and  $P_i\in \mathbb{R}^3, i=1, 2, \cdots, N $ is the measured position at the $i^{th}$ instance. The ISO 9283 standard states that the mean position of the end-effector $\bar{P} = \frac{1}{N} \sum_{i=1}^N P_i $ is calculated as the average of the measured positions $P_i$. The repeatability $R$ is computed based on the average (\( \mu \)) and the standard deviation (\( \sigma \)) of the distances between \( \bar{P} \) and $P_i$.

\begin{equation*}
    \mu = \frac{1}{N} \sum_{i=1}^N \lVert P_i - \bar{P} \rVert, \quad 
    \sigma = \sqrt{\frac{1}{N-1} \sum_{i=1}^N (\lVert P_i - \bar{P} \rVert)^2}
\end{equation*}

\begin{equation*}
    R = \mu + 3\sigma
\end{equation*} 

As shown in Table~\ref{tab:repeatability_results}, \robot achieves an average repeatability of 2.63 mm which means it is less consistent compared to the cobots with high gear ratio actuators, whose repeatability is around 0.1mm. This shows the trade-off between compliance and consistency.

\begin{table}[hbt]
\centering
\caption{ISO 9283 repeatability experiment results}
\label{tab:repeatability_results}
\resizebox{0.46\textwidth}{!}{%
\begin{tabular}{|c|c|c|c|}
\hline
Point & Average distance (mm) & Std dev. (mm) & Repeatability (mm)\\ \hline
$P_1$    & 1.048     & 0.546    & 2.687\\ %\hline
$P_2$    & 1.042     & 0.409    & 2.269\\ %\hline
$P_3$    & 0.939     & 0.689    & 3.006\\ %\hline
$P_4$    & 0.751     & 0.520    & 2.311\\ %\hline
$P_5$    & 1.104     & 0.584    & 2.857\\ \hline
Average    & 0.977     & 0.550    & 2.626\\ \hline

\multicolumn{3}{|c|}{Franka Panda~\cite{haddadin2022franka}} & 0.1 \\ 
\multicolumn{3}{|c|}{KUKA iiwa 7 R800} & 0.1 \\ 
\multicolumn{3}{|c|}{Quigley et al.~\cite{quigley2011low}} & *3 \\ 
\multicolumn{3}{|c|}{LIMS~\cite{kim2017anthropomorphic}} & 0.43 \\ 
\multicolumn{3}{|c|}{Nishii et al.~\cite{nishii2023ultra}} & *2.2 \\ 
\multicolumn{3}{|c|}{BLUE~\cite{gealy2019quasi}} & *3.7 \\ \hline
\end{tabular}%
% \begin{tablenotes}
% \item moving mass is defined as the arm’s mass, excluding body-mounted components and the gripper.
% \end{tablenotes}
}
\parbox{0.46\textwidth}{
\vspace{2mm}
\textit{``*"} indicates that repeatability is obtained manually, not by ISO 9283.

}
\end{table}


% Repeatability (mm)          & 2.626    & \textcolor{blue}{0.1}          & \textcolor{blue}{0.1}              & 3              & 0.425   & 2.2      & ?        & 3.7\\ %\hline
% Repeatability method          & ISO9283 & ISO9283      & ISO9283          & manually       & ISO9283 & manually & manually    & manually\\ %\hline

% \begin{table}[hbt]
% \centering
% \caption{Repeatability comparison}
% \label{tab:repeatability_comparison}
% \resizebox{0.46\textwidth}{!}{%
% \begin{tabular}{|c|c|c|c|}
% \hline
%                         & \robot (Ours) & Franka Panda~\cite{haddadin2022franka} & KUKA iiwa 7 R800 \\ \hline
% Repeatability (mm)      & 2.626    & 0.1    & 0.1\\ \hline
% \end{tabular}%
% }
% \begin{tablenotes}
% \item \textcolor{blue}{Blue} indicates strengths, \textcolor{red}{red} highlights weaknesses, and ``?" denotes information not provided in the paper.
% \item[1] moving mass is defined as the arm’s mass, excluding body-mounted components and the gripper.
% \end{tablenotes}
% \end{table}

\section{Experiments}
\label{sec:experiments}
The experiments are designed to address two key research questions.
First, \textbf{RQ1} evaluates whether the average $L_2$-norm of the counterfactual perturbation vectors ($\overline{||\perturb||}$) decreases as the model overfits the data, thereby providing further empirical validation for our hypothesis.
Second, \textbf{RQ2} evaluates the ability of the proposed counterfactual regularized loss, as defined in (\ref{eq:regularized_loss2}), to mitigate overfitting when compared to existing regularization techniques.

% The experiments are designed to address three key research questions. First, \textbf{RQ1} investigates whether the mean perturbation vector norm decreases as the model overfits the data, aiming to further validate our intuition. Second, \textbf{RQ2} explores whether the mean perturbation vector norm can be effectively leveraged as a regularization term during training, offering insights into its potential role in mitigating overfitting. Finally, \textbf{RQ3} examines whether our counterfactual regularizer enables the model to achieve superior performance compared to existing regularization methods, thus highlighting its practical advantage.

\subsection{Experimental Setup}
\textbf{\textit{Datasets, Models, and Tasks.}}
The experiments are conducted on three datasets: \textit{Water Potability}~\cite{kadiwal2020waterpotability}, \textit{Phomene}~\cite{phomene}, and \textit{CIFAR-10}~\cite{krizhevsky2009learning}. For \textit{Water Potability} and \textit{Phomene}, we randomly select $80\%$ of the samples for the training set, and the remaining $20\%$ for the test set, \textit{CIFAR-10} comes already split. Furthermore, we consider the following models: Logistic Regression, Multi-Layer Perceptron (MLP) with 100 and 30 neurons on each hidden layer, and PreactResNet-18~\cite{he2016cvecvv} as a Convolutional Neural Network (CNN) architecture.
We focus on binary classification tasks and leave the extension to multiclass scenarios for future work. However, for datasets that are inherently multiclass, we transform the problem into a binary classification task by selecting two classes, aligning with our assumption.

\smallskip
\noindent\textbf{\textit{Evaluation Measures.}} To characterize the degree of overfitting, we use the test loss, as it serves as a reliable indicator of the model's generalization capability to unseen data. Additionally, we evaluate the predictive performance of each model using the test accuracy.

\smallskip
\noindent\textbf{\textit{Baselines.}} We compare CF-Reg with the following regularization techniques: L1 (``Lasso''), L2 (``Ridge''), and Dropout.

\smallskip
\noindent\textbf{\textit{Configurations.}}
For each model, we adopt specific configurations as follows.
\begin{itemize}
\item \textit{Logistic Regression:} To induce overfitting in the model, we artificially increase the dimensionality of the data beyond the number of training samples by applying a polynomial feature expansion. This approach ensures that the model has enough capacity to overfit the training data, allowing us to analyze the impact of our counterfactual regularizer. The degree of the polynomial is chosen as the smallest degree that makes the number of features greater than the number of data.
\item \textit{Neural Networks (MLP and CNN):} To take advantage of the closed-form solution for computing the optimal perturbation vector as defined in (\ref{eq:opt-delta}), we use a local linear approximation of the neural network models. Hence, given an instance $\inst_i$, we consider the (optimal) counterfactual not with respect to $\model$ but with respect to:
\begin{equation}
\label{eq:taylor}
    \model^{lin}(\inst) = \model(\inst_i) + \nabla_{\inst}\model(\inst_i)(\inst - \inst_i),
\end{equation}
where $\model^{lin}$ represents the first-order Taylor approximation of $\model$ at $\inst_i$.
Note that this step is unnecessary for Logistic Regression, as it is inherently a linear model.
\end{itemize}

\smallskip
\noindent \textbf{\textit{Implementation Details.}} We run all experiments on a machine equipped with an AMD Ryzen 9 7900 12-Core Processor and an NVIDIA GeForce RTX 4090 GPU. Our implementation is based on the PyTorch Lightning framework. We use stochastic gradient descent as the optimizer with a learning rate of $\eta = 0.001$ and no weight decay. We use a batch size of $128$. The training and test steps are conducted for $6000$ epochs on the \textit{Water Potability} and \textit{Phoneme} datasets, while for the \textit{CIFAR-10} dataset, they are performed for $200$ epochs.
Finally, the contribution $w_i^{\varepsilon}$ of each training point $\inst_i$ is uniformly set as $w_i^{\varepsilon} = 1~\forall i\in \{1,\ldots,m\}$.

The source code implementation for our experiments is available at the following GitHub repository: \url{https://anonymous.4open.science/r/COCE-80B4/README.md} 

\subsection{RQ1: Counterfactual Perturbation vs. Overfitting}
To address \textbf{RQ1}, we analyze the relationship between the test loss and the average $L_2$-norm of the counterfactual perturbation vectors ($\overline{||\perturb||}$) over training epochs.

In particular, Figure~\ref{fig:delta_loss_epochs} depicts the evolution of $\overline{||\perturb||}$ alongside the test loss for an MLP trained \textit{without} regularization on the \textit{Water Potability} dataset. 
\begin{figure}[ht]
    \centering
    \includegraphics[width=0.85\linewidth]{img/delta_loss_epochs.png}
    \caption{The average counterfactual perturbation vector $\overline{||\perturb||}$ (left $y$-axis) and the cross-entropy test loss (right $y$-axis) over training epochs ($x$-axis) for an MLP trained on the \textit{Water Potability} dataset \textit{without} regularization.}
    \label{fig:delta_loss_epochs}
\end{figure}

The plot shows a clear trend as the model starts to overfit the data (evidenced by an increase in test loss). 
Notably, $\overline{||\perturb||}$ begins to decrease, which aligns with the hypothesis that the average distance to the optimal counterfactual example gets smaller as the model's decision boundary becomes increasingly adherent to the training data.

It is worth noting that this trend is heavily influenced by the choice of the counterfactual generator model. In particular, the relationship between $\overline{||\perturb||}$ and the degree of overfitting may become even more pronounced when leveraging more accurate counterfactual generators. However, these models often come at the cost of higher computational complexity, and their exploration is left to future work.

Nonetheless, we expect that $\overline{||\perturb||}$ will eventually stabilize at a plateau, as the average $L_2$-norm of the optimal counterfactual perturbations cannot vanish to zero.

% Additionally, the choice of employing the score-based counterfactual explanation framework to generate counterfactuals was driven to promote computational efficiency.

% Future enhancements to the framework may involve adopting models capable of generating more precise counterfactuals. While such approaches may yield to performance improvements, they are likely to come at the cost of increased computational complexity.


\subsection{RQ2: Counterfactual Regularization Performance}
To answer \textbf{RQ2}, we evaluate the effectiveness of the proposed counterfactual regularization (CF-Reg) by comparing its performance against existing baselines: unregularized training loss (No-Reg), L1 regularization (L1-Reg), L2 regularization (L2-Reg), and Dropout.
Specifically, for each model and dataset combination, Table~\ref{tab:regularization_comparison} presents the mean value and standard deviation of test accuracy achieved by each method across 5 random initialization. 

The table illustrates that our regularization technique consistently delivers better results than existing methods across all evaluated scenarios, except for one case -- i.e., Logistic Regression on the \textit{Phomene} dataset. 
However, this setting exhibits an unusual pattern, as the highest model accuracy is achieved without any regularization. Even in this case, CF-Reg still surpasses other regularization baselines.

From the results above, we derive the following key insights. First, CF-Reg proves to be effective across various model types, ranging from simple linear models (Logistic Regression) to deep architectures like MLPs and CNNs, and across diverse datasets, including both tabular and image data. 
Second, CF-Reg's strong performance on the \textit{Water} dataset with Logistic Regression suggests that its benefits may be more pronounced when applied to simpler models. However, the unexpected outcome on the \textit{Phoneme} dataset calls for further investigation into this phenomenon.


\begin{table*}[h!]
    \centering
    \caption{Mean value and standard deviation of test accuracy across 5 random initializations for different model, dataset, and regularization method. The best results are highlighted in \textbf{bold}.}
    \label{tab:regularization_comparison}
    \begin{tabular}{|c|c|c|c|c|c|c|}
        \hline
        \textbf{Model} & \textbf{Dataset} & \textbf{No-Reg} & \textbf{L1-Reg} & \textbf{L2-Reg} & \textbf{Dropout} & \textbf{CF-Reg (ours)} \\ \hline
        Logistic Regression   & \textit{Water}   & $0.6595 \pm 0.0038$   & $0.6729 \pm 0.0056$   & $0.6756 \pm 0.0046$  & N/A    & $\mathbf{0.6918 \pm 0.0036}$                     \\ \hline
        MLP   & \textit{Water}   & $0.6756 \pm 0.0042$   & $0.6790 \pm 0.0058$   & $0.6790 \pm 0.0023$  & $0.6750 \pm 0.0036$    & $\mathbf{0.6802 \pm 0.0046}$                    \\ \hline
%        MLP   & \textit{Adult}   & $0.8404 \pm 0.0010$   & $\mathbf{0.8495 \pm 0.0007}$   & $0.8489 \pm 0.0014$  & $\mathbf{0.8495 \pm 0.0016}$     & $0.8449 \pm 0.0019$                    \\ \hline
        Logistic Regression   & \textit{Phomene}   & $\mathbf{0.8148 \pm 0.0020}$   & $0.8041 \pm 0.0028$   & $0.7835 \pm 0.0176$  & N/A    & $0.8098 \pm 0.0055$                     \\ \hline
        MLP   & \textit{Phomene}   & $0.8677 \pm 0.0033$   & $0.8374 \pm 0.0080$   & $0.8673 \pm 0.0045$  & $0.8672 \pm 0.0042$     & $\mathbf{0.8718 \pm 0.0040}$                    \\ \hline
        CNN   & \textit{CIFAR-10} & $0.6670 \pm 0.0233$   & $0.6229 \pm 0.0850$   & $0.7348 \pm 0.0365$   & N/A    & $\mathbf{0.7427 \pm 0.0571}$                     \\ \hline
    \end{tabular}
\end{table*}

\begin{table*}[htb!]
    \centering
    \caption{Hyperparameter configurations utilized for the generation of Table \ref{tab:regularization_comparison}. For our regularization the hyperparameters are reported as $\mathbf{\alpha/\beta}$.}
    \label{tab:performance_parameters}
    \begin{tabular}{|c|c|c|c|c|c|c|}
        \hline
        \textbf{Model} & \textbf{Dataset} & \textbf{No-Reg} & \textbf{L1-Reg} & \textbf{L2-Reg} & \textbf{Dropout} & \textbf{CF-Reg (ours)} \\ \hline
        Logistic Regression   & \textit{Water}   & N/A   & $0.0093$   & $0.6927$  & N/A    & $0.3791/1.0355$                     \\ \hline
        MLP   & \textit{Water}   & N/A   & $0.0007$   & $0.0022$  & $0.0002$    & $0.2567/1.9775$                    \\ \hline
        Logistic Regression   &
        \textit{Phomene}   & N/A   & $0.0097$   & $0.7979$  & N/A    & $0.0571/1.8516$                     \\ \hline
        MLP   & \textit{Phomene}   & N/A   & $0.0007$   & $4.24\cdot10^{-5}$  & $0.0015$    & $0.0516/2.2700$                    \\ \hline
       % MLP   & \textit{Adult}   & N/A   & $0.0018$   & $0.0018$  & $0.0601$     & $0.0764/2.2068$                    \\ \hline
        CNN   & \textit{CIFAR-10} & N/A   & $0.0050$   & $0.0864$ & N/A    & $0.3018/
        2.1502$                     \\ \hline
    \end{tabular}
\end{table*}

\begin{table*}[htb!]
    \centering
    \caption{Mean value and standard deviation of training time across 5 different runs. The reported time (in seconds) corresponds to the generation of each entry in Table \ref{tab:regularization_comparison}. Times are }
    \label{tab:times}
    \begin{tabular}{|c|c|c|c|c|c|c|}
        \hline
        \textbf{Model} & \textbf{Dataset} & \textbf{No-Reg} & \textbf{L1-Reg} & \textbf{L2-Reg} & \textbf{Dropout} & \textbf{CF-Reg (ours)} \\ \hline
        Logistic Regression   & \textit{Water}   & $222.98 \pm 1.07$   & $239.94 \pm 2.59$   & $241.60 \pm 1.88$  & N/A    & $251.50 \pm 1.93$                     \\ \hline
        MLP   & \textit{Water}   & $225.71 \pm 3.85$   & $250.13 \pm 4.44$   & $255.78 \pm 2.38$  & $237.83 \pm 3.45$    & $266.48 \pm 3.46$                    \\ \hline
        Logistic Regression   & \textit{Phomene}   & $266.39 \pm 0.82$ & $367.52 \pm 6.85$   & $361.69 \pm 4.04$  & N/A   & $310.48 \pm 0.76$                    \\ \hline
        MLP   &
        \textit{Phomene} & $335.62 \pm 1.77$   & $390.86 \pm 2.11$   & $393.96 \pm 1.95$ & $363.51 \pm 5.07$    & $403.14 \pm 1.92$                     \\ \hline
       % MLP   & \textit{Adult}   & N/A   & $0.0018$   & $0.0018$  & $0.0601$     & $0.0764/2.2068$                    \\ \hline
        CNN   & \textit{CIFAR-10} & $370.09 \pm 0.18$   & $395.71 \pm 0.55$   & $401.38 \pm 0.16$ & N/A    & $1287.8 \pm 0.26$                     \\ \hline
    \end{tabular}
\end{table*}

\subsection{Feasibility of our Method}
A crucial requirement for any regularization technique is that it should impose minimal impact on the overall training process.
In this respect, CF-Reg introduces an overhead that depends on the time required to find the optimal counterfactual example for each training instance. 
As such, the more sophisticated the counterfactual generator model probed during training the higher would be the time required. However, a more advanced counterfactual generator might provide a more effective regularization. We discuss this trade-off in more details in Section~\ref{sec:discussion}.

Table~\ref{tab:times} presents the average training time ($\pm$ standard deviation) for each model and dataset combination listed in Table~\ref{tab:regularization_comparison}.
We can observe that the higher accuracy achieved by CF-Reg using the score-based counterfactual generator comes with only minimal overhead. However, when applied to deep neural networks with many hidden layers, such as \textit{PreactResNet-18}, the forward derivative computation required for the linearization of the network introduces a more noticeable computational cost, explaining the longer training times in the table.

\subsection{Hyperparameter Sensitivity Analysis}
The proposed counterfactual regularization technique relies on two key hyperparameters: $\alpha$ and $\beta$. The former is intrinsic to the loss formulation defined in (\ref{eq:cf-train}), while the latter is closely tied to the choice of the score-based counterfactual explanation method used.

Figure~\ref{fig:test_alpha_beta} illustrates how the test accuracy of an MLP trained on the \textit{Water Potability} dataset changes for different combinations of $\alpha$ and $\beta$.

\begin{figure}[ht]
    \centering
    \includegraphics[width=0.85\linewidth]{img/test_acc_alpha_beta.png}
    \caption{The test accuracy of an MLP trained on the \textit{Water Potability} dataset, evaluated while varying the weight of our counterfactual regularizer ($\alpha$) for different values of $\beta$.}
    \label{fig:test_alpha_beta}
\end{figure}

We observe that, for a fixed $\beta$, increasing the weight of our counterfactual regularizer ($\alpha$) can slightly improve test accuracy until a sudden drop is noticed for $\alpha > 0.1$.
This behavior was expected, as the impact of our penalty, like any regularization term, can be disruptive if not properly controlled.

Moreover, this finding further demonstrates that our regularization method, CF-Reg, is inherently data-driven. Therefore, it requires specific fine-tuning based on the combination of the model and dataset at hand.

\section{Limitation}
The use of 3D-printed PLA for structural components improves improving ease of assembly and reduces weight and cost, yet it causes deformation under heavy load, which can diminish end-effector precision. Using metal, such as aluminum, would remedy this problem. Additionally, \robot relies on integrated joint relative encoders, requiring manual initialization in a fixed joint configuration each time the system is powered on. Using absolute joint encoders could significantly improve accuracy and ease of use, although it would increase the overall cost. 

%Reliance on commercially available actuators simplifies integration but imposes constraints on control frequency and customization, further limiting the potential for tailored performance improvements.

% The 6 DoF configuration provides sufficient mobility for most tasks; however, certain bimanual operations could benefit from an additional degree of freedom to handle complex joint constraints more effectively. Furthermore, the limited torque density of commercially available proprioceptive actuators restricts the payload and torque output, making the system less suitability for handling heavier loads or high-torque applications. 

The 6 DoF configuration of the arm provides sufficient mobility for single-arm manipulation tasks, yet it shows a limitation in certain bimanual manipulation problems. Specifically, when \robot holds onto a rigid object with both hands, each arm loses 1 DoF because the hands are fixed to the object during grasping. This leads to an underactuated kinematic chain which has a limited mobility in 3D space. We can achieve more mobility by letting the object slip inside the grippers, yet this renders the grasp less robust and simulation difficult. Therefore, we anticipate that designing a lightweight 3 DoF wrist in place of the current 2 DoF wrist allows a more diverse repertoire of manipulation in bimanual tasks.

Finally, the limited torque density of commercially available proprioceptive actuators restricts the performance. Currently, all of our actuators feature a 1:10 gear ratio, so \robot can handle up to 2.5 kg of payload. To handle a heavier object and manipulate it with higher torque, we expect the actuator to have 1:20$\sim$30 gear ratio, but it is difficult to find an off-the-shelf product that meets our requirements. Customizing the actuator to increase the torque density while minimizing the weight will enable \robot to move faster and handle more diverse objects.

%These constraints highlight opportunities for improvement in future iterations, including alternative materials for enhanced rigidity, custom actuator designs for higher control precision and torque density, the adoption of absolute joint encoders, and optimized configurations to balance dexterity and weight.



% \section*{Acknowledgments}


%% Use plainnat to work nicely with natbib. 

\bibliographystyle{plainnat}
\bibliography{references}

% ~\clearpage
% \subsection{Lloyd-Max Algorithm}
\label{subsec:Lloyd-Max}
For a given quantization bitwidth $B$ and an operand $\bm{X}$, the Lloyd-Max algorithm finds $2^B$ quantization levels $\{\hat{x}_i\}_{i=1}^{2^B}$ such that quantizing $\bm{X}$ by rounding each scalar in $\bm{X}$ to the nearest quantization level minimizes the quantization MSE. 

The algorithm starts with an initial guess of quantization levels and then iteratively computes quantization thresholds $\{\tau_i\}_{i=1}^{2^B-1}$ and updates quantization levels $\{\hat{x}_i\}_{i=1}^{2^B}$. Specifically, at iteration $n$, thresholds are set to the midpoints of the previous iteration's levels:
\begin{align*}
    \tau_i^{(n)}=\frac{\hat{x}_i^{(n-1)}+\hat{x}_{i+1}^{(n-1)}}2 \text{ for } i=1\ldots 2^B-1
\end{align*}
Subsequently, the quantization levels are re-computed as conditional means of the data regions defined by the new thresholds:
\begin{align*}
    \hat{x}_i^{(n)}=\mathbb{E}\left[ \bm{X} \big| \bm{X}\in [\tau_{i-1}^{(n)},\tau_i^{(n)}] \right] \text{ for } i=1\ldots 2^B
\end{align*}
where to satisfy boundary conditions we have $\tau_0=-\infty$ and $\tau_{2^B}=\infty$. The algorithm iterates the above steps until convergence.

Figure \ref{fig:lm_quant} compares the quantization levels of a $7$-bit floating point (E3M3) quantizer (left) to a $7$-bit Lloyd-Max quantizer (right) when quantizing a layer of weights from the GPT3-126M model at a per-tensor granularity. As shown, the Lloyd-Max quantizer achieves substantially lower quantization MSE. Further, Table \ref{tab:FP7_vs_LM7} shows the superior perplexity achieved by Lloyd-Max quantizers for bitwidths of $7$, $6$ and $5$. The difference between the quantizers is clear at 5 bits, where per-tensor FP quantization incurs a drastic and unacceptable increase in perplexity, while Lloyd-Max quantization incurs a much smaller increase. Nevertheless, we note that even the optimal Lloyd-Max quantizer incurs a notable ($\sim 1.5$) increase in perplexity due to the coarse granularity of quantization. 

\begin{figure}[h]
  \centering
  \includegraphics[width=0.7\linewidth]{sections/figures/LM7_FP7.pdf}
  \caption{\small Quantization levels and the corresponding quantization MSE of Floating Point (left) vs Lloyd-Max (right) Quantizers for a layer of weights in the GPT3-126M model.}
  \label{fig:lm_quant}
\end{figure}

\begin{table}[h]\scriptsize
\begin{center}
\caption{\label{tab:FP7_vs_LM7} \small Comparing perplexity (lower is better) achieved by floating point quantizers and Lloyd-Max quantizers on a GPT3-126M model for the Wikitext-103 dataset.}
\begin{tabular}{c|cc|c}
\hline
 \multirow{2}{*}{\textbf{Bitwidth}} & \multicolumn{2}{|c|}{\textbf{Floating-Point Quantizer}} & \textbf{Lloyd-Max Quantizer} \\
 & Best Format & Wikitext-103 Perplexity & Wikitext-103 Perplexity \\
\hline
7 & E3M3 & 18.32 & 18.27 \\
6 & E3M2 & 19.07 & 18.51 \\
5 & E4M0 & 43.89 & 19.71 \\
\hline
\end{tabular}
\end{center}
\end{table}

\subsection{Proof of Local Optimality of LO-BCQ}
\label{subsec:lobcq_opt_proof}
For a given block $\bm{b}_j$, the quantization MSE during LO-BCQ can be empirically evaluated as $\frac{1}{L_b}\lVert \bm{b}_j- \bm{\hat{b}}_j\rVert^2_2$ where $\bm{\hat{b}}_j$ is computed from equation (\ref{eq:clustered_quantization_definition}) as $C_{f(\bm{b}_j)}(\bm{b}_j)$. Further, for a given block cluster $\mathcal{B}_i$, we compute the quantization MSE as $\frac{1}{|\mathcal{B}_{i}|}\sum_{\bm{b} \in \mathcal{B}_{i}} \frac{1}{L_b}\lVert \bm{b}- C_i^{(n)}(\bm{b})\rVert^2_2$. Therefore, at the end of iteration $n$, we evaluate the overall quantization MSE $J^{(n)}$ for a given operand $\bm{X}$ composed of $N_c$ block clusters as:
\begin{align*}
    \label{eq:mse_iter_n}
    J^{(n)} = \frac{1}{N_c} \sum_{i=1}^{N_c} \frac{1}{|\mathcal{B}_{i}^{(n)}|}\sum_{\bm{v} \in \mathcal{B}_{i}^{(n)}} \frac{1}{L_b}\lVert \bm{b}- B_i^{(n)}(\bm{b})\rVert^2_2
\end{align*}

At the end of iteration $n$, the codebooks are updated from $\mathcal{C}^{(n-1)}$ to $\mathcal{C}^{(n)}$. However, the mapping of a given vector $\bm{b}_j$ to quantizers $\mathcal{C}^{(n)}$ remains as  $f^{(n)}(\bm{b}_j)$. At the next iteration, during the vector clustering step, $f^{(n+1)}(\bm{b}_j)$ finds new mapping of $\bm{b}_j$ to updated codebooks $\mathcal{C}^{(n)}$ such that the quantization MSE over the candidate codebooks is minimized. Therefore, we obtain the following result for $\bm{b}_j$:
\begin{align*}
\frac{1}{L_b}\lVert \bm{b}_j - C_{f^{(n+1)}(\bm{b}_j)}^{(n)}(\bm{b}_j)\rVert^2_2 \le \frac{1}{L_b}\lVert \bm{b}_j - C_{f^{(n)}(\bm{b}_j)}^{(n)}(\bm{b}_j)\rVert^2_2
\end{align*}

That is, quantizing $\bm{b}_j$ at the end of the block clustering step of iteration $n+1$ results in lower quantization MSE compared to quantizing at the end of iteration $n$. Since this is true for all $\bm{b} \in \bm{X}$, we assert the following:
\begin{equation}
\begin{split}
\label{eq:mse_ineq_1}
    \tilde{J}^{(n+1)} &= \frac{1}{N_c} \sum_{i=1}^{N_c} \frac{1}{|\mathcal{B}_{i}^{(n+1)}|}\sum_{\bm{b} \in \mathcal{B}_{i}^{(n+1)}} \frac{1}{L_b}\lVert \bm{b} - C_i^{(n)}(b)\rVert^2_2 \le J^{(n)}
\end{split}
\end{equation}
where $\tilde{J}^{(n+1)}$ is the the quantization MSE after the vector clustering step at iteration $n+1$.

Next, during the codebook update step (\ref{eq:quantizers_update}) at iteration $n+1$, the per-cluster codebooks $\mathcal{C}^{(n)}$ are updated to $\mathcal{C}^{(n+1)}$ by invoking the Lloyd-Max algorithm \citep{Lloyd}. We know that for any given value distribution, the Lloyd-Max algorithm minimizes the quantization MSE. Therefore, for a given vector cluster $\mathcal{B}_i$ we obtain the following result:

\begin{equation}
    \frac{1}{|\mathcal{B}_{i}^{(n+1)}|}\sum_{\bm{b} \in \mathcal{B}_{i}^{(n+1)}} \frac{1}{L_b}\lVert \bm{b}- C_i^{(n+1)}(\bm{b})\rVert^2_2 \le \frac{1}{|\mathcal{B}_{i}^{(n+1)}|}\sum_{\bm{b} \in \mathcal{B}_{i}^{(n+1)}} \frac{1}{L_b}\lVert \bm{b}- C_i^{(n)}(\bm{b})\rVert^2_2
\end{equation}

The above equation states that quantizing the given block cluster $\mathcal{B}_i$ after updating the associated codebook from $C_i^{(n)}$ to $C_i^{(n+1)}$ results in lower quantization MSE. Since this is true for all the block clusters, we derive the following result: 
\begin{equation}
\begin{split}
\label{eq:mse_ineq_2}
     J^{(n+1)} &= \frac{1}{N_c} \sum_{i=1}^{N_c} \frac{1}{|\mathcal{B}_{i}^{(n+1)}|}\sum_{\bm{b} \in \mathcal{B}_{i}^{(n+1)}} \frac{1}{L_b}\lVert \bm{b}- C_i^{(n+1)}(\bm{b})\rVert^2_2  \le \tilde{J}^{(n+1)}   
\end{split}
\end{equation}

Following (\ref{eq:mse_ineq_1}) and (\ref{eq:mse_ineq_2}), we find that the quantization MSE is non-increasing for each iteration, that is, $J^{(1)} \ge J^{(2)} \ge J^{(3)} \ge \ldots \ge J^{(M)}$ where $M$ is the maximum number of iterations. 
%Therefore, we can say that if the algorithm converges, then it must be that it has converged to a local minimum. 
\hfill $\blacksquare$


\begin{figure}
    \begin{center}
    \includegraphics[width=0.5\textwidth]{sections//figures/mse_vs_iter.pdf}
    \end{center}
    \caption{\small NMSE vs iterations during LO-BCQ compared to other block quantization proposals}
    \label{fig:nmse_vs_iter}
\end{figure}

Figure \ref{fig:nmse_vs_iter} shows the empirical convergence of LO-BCQ across several block lengths and number of codebooks. Also, the MSE achieved by LO-BCQ is compared to baselines such as MXFP and VSQ. As shown, LO-BCQ converges to a lower MSE than the baselines. Further, we achieve better convergence for larger number of codebooks ($N_c$) and for a smaller block length ($L_b$), both of which increase the bitwidth of BCQ (see Eq \ref{eq:bitwidth_bcq}).


\subsection{Additional Accuracy Results}
%Table \ref{tab:lobcq_config} lists the various LOBCQ configurations and their corresponding bitwidths.
\begin{table}
\setlength{\tabcolsep}{4.75pt}
\begin{center}
\caption{\label{tab:lobcq_config} Various LO-BCQ configurations and their bitwidths.}
\begin{tabular}{|c||c|c|c|c||c|c||c|} 
\hline
 & \multicolumn{4}{|c||}{$L_b=8$} & \multicolumn{2}{|c||}{$L_b=4$} & $L_b=2$ \\
 \hline
 \backslashbox{$L_A$\kern-1em}{\kern-1em$N_c$} & 2 & 4 & 8 & 16 & 2 & 4 & 2 \\
 \hline
 64 & 4.25 & 4.375 & 4.5 & 4.625 & 4.375 & 4.625 & 4.625\\
 \hline
 32 & 4.375 & 4.5 & 4.625& 4.75 & 4.5 & 4.75 & 4.75 \\
 \hline
 16 & 4.625 & 4.75& 4.875 & 5 & 4.75 & 5 & 5 \\
 \hline
\end{tabular}
\end{center}
\end{table}

%\subsection{Perplexity achieved by various LO-BCQ configurations on Wikitext-103 dataset}

\begin{table} \centering
\begin{tabular}{|c||c|c|c|c||c|c||c|} 
\hline
 $L_b \rightarrow$& \multicolumn{4}{c||}{8} & \multicolumn{2}{c||}{4} & 2\\
 \hline
 \backslashbox{$L_A$\kern-1em}{\kern-1em$N_c$} & 2 & 4 & 8 & 16 & 2 & 4 & 2  \\
 %$N_c \rightarrow$ & 2 & 4 & 8 & 16 & 2 & 4 & 2 \\
 \hline
 \hline
 \multicolumn{8}{c}{GPT3-1.3B (FP32 PPL = 9.98)} \\ 
 \hline
 \hline
 64 & 10.40 & 10.23 & 10.17 & 10.15 &  10.28 & 10.18 & 10.19 \\
 \hline
 32 & 10.25 & 10.20 & 10.15 & 10.12 &  10.23 & 10.17 & 10.17 \\
 \hline
 16 & 10.22 & 10.16 & 10.10 & 10.09 &  10.21 & 10.14 & 10.16 \\
 \hline
  \hline
 \multicolumn{8}{c}{GPT3-8B (FP32 PPL = 7.38)} \\ 
 \hline
 \hline
 64 & 7.61 & 7.52 & 7.48 &  7.47 &  7.55 &  7.49 & 7.50 \\
 \hline
 32 & 7.52 & 7.50 & 7.46 &  7.45 &  7.52 &  7.48 & 7.48  \\
 \hline
 16 & 7.51 & 7.48 & 7.44 &  7.44 &  7.51 &  7.49 & 7.47  \\
 \hline
\end{tabular}
\caption{\label{tab:ppl_gpt3_abalation} Wikitext-103 perplexity across GPT3-1.3B and 8B models.}
\end{table}

\begin{table} \centering
\begin{tabular}{|c||c|c|c|c||} 
\hline
 $L_b \rightarrow$& \multicolumn{4}{c||}{8}\\
 \hline
 \backslashbox{$L_A$\kern-1em}{\kern-1em$N_c$} & 2 & 4 & 8 & 16 \\
 %$N_c \rightarrow$ & 2 & 4 & 8 & 16 & 2 & 4 & 2 \\
 \hline
 \hline
 \multicolumn{5}{|c|}{Llama2-7B (FP32 PPL = 5.06)} \\ 
 \hline
 \hline
 64 & 5.31 & 5.26 & 5.19 & 5.18  \\
 \hline
 32 & 5.23 & 5.25 & 5.18 & 5.15  \\
 \hline
 16 & 5.23 & 5.19 & 5.16 & 5.14  \\
 \hline
 \multicolumn{5}{|c|}{Nemotron4-15B (FP32 PPL = 5.87)} \\ 
 \hline
 \hline
 64  & 6.3 & 6.20 & 6.13 & 6.08  \\
 \hline
 32  & 6.24 & 6.12 & 6.07 & 6.03  \\
 \hline
 16  & 6.12 & 6.14 & 6.04 & 6.02  \\
 \hline
 \multicolumn{5}{|c|}{Nemotron4-340B (FP32 PPL = 3.48)} \\ 
 \hline
 \hline
 64 & 3.67 & 3.62 & 3.60 & 3.59 \\
 \hline
 32 & 3.63 & 3.61 & 3.59 & 3.56 \\
 \hline
 16 & 3.61 & 3.58 & 3.57 & 3.55 \\
 \hline
\end{tabular}
\caption{\label{tab:ppl_llama7B_nemo15B} Wikitext-103 perplexity compared to FP32 baseline in Llama2-7B and Nemotron4-15B, 340B models}
\end{table}

%\subsection{Perplexity achieved by various LO-BCQ configurations on MMLU dataset}


\begin{table} \centering
\begin{tabular}{|c||c|c|c|c||c|c|c|c|} 
\hline
 $L_b \rightarrow$& \multicolumn{4}{c||}{8} & \multicolumn{4}{c||}{8}\\
 \hline
 \backslashbox{$L_A$\kern-1em}{\kern-1em$N_c$} & 2 & 4 & 8 & 16 & 2 & 4 & 8 & 16  \\
 %$N_c \rightarrow$ & 2 & 4 & 8 & 16 & 2 & 4 & 2 \\
 \hline
 \hline
 \multicolumn{5}{|c|}{Llama2-7B (FP32 Accuracy = 45.8\%)} & \multicolumn{4}{|c|}{Llama2-70B (FP32 Accuracy = 69.12\%)} \\ 
 \hline
 \hline
 64 & 43.9 & 43.4 & 43.9 & 44.9 & 68.07 & 68.27 & 68.17 & 68.75 \\
 \hline
 32 & 44.5 & 43.8 & 44.9 & 44.5 & 68.37 & 68.51 & 68.35 & 68.27  \\
 \hline
 16 & 43.9 & 42.7 & 44.9 & 45 & 68.12 & 68.77 & 68.31 & 68.59  \\
 \hline
 \hline
 \multicolumn{5}{|c|}{GPT3-22B (FP32 Accuracy = 38.75\%)} & \multicolumn{4}{|c|}{Nemotron4-15B (FP32 Accuracy = 64.3\%)} \\ 
 \hline
 \hline
 64 & 36.71 & 38.85 & 38.13 & 38.92 & 63.17 & 62.36 & 63.72 & 64.09 \\
 \hline
 32 & 37.95 & 38.69 & 39.45 & 38.34 & 64.05 & 62.30 & 63.8 & 64.33  \\
 \hline
 16 & 38.88 & 38.80 & 38.31 & 38.92 & 63.22 & 63.51 & 63.93 & 64.43  \\
 \hline
\end{tabular}
\caption{\label{tab:mmlu_abalation} Accuracy on MMLU dataset across GPT3-22B, Llama2-7B, 70B and Nemotron4-15B models.}
\end{table}


%\subsection{Perplexity achieved by various LO-BCQ configurations on LM evaluation harness}

\begin{table} \centering
\begin{tabular}{|c||c|c|c|c||c|c|c|c|} 
\hline
 $L_b \rightarrow$& \multicolumn{4}{c||}{8} & \multicolumn{4}{c||}{8}\\
 \hline
 \backslashbox{$L_A$\kern-1em}{\kern-1em$N_c$} & 2 & 4 & 8 & 16 & 2 & 4 & 8 & 16  \\
 %$N_c \rightarrow$ & 2 & 4 & 8 & 16 & 2 & 4 & 2 \\
 \hline
 \hline
 \multicolumn{5}{|c|}{Race (FP32 Accuracy = 37.51\%)} & \multicolumn{4}{|c|}{Boolq (FP32 Accuracy = 64.62\%)} \\ 
 \hline
 \hline
 64 & 36.94 & 37.13 & 36.27 & 37.13 & 63.73 & 62.26 & 63.49 & 63.36 \\
 \hline
 32 & 37.03 & 36.36 & 36.08 & 37.03 & 62.54 & 63.51 & 63.49 & 63.55  \\
 \hline
 16 & 37.03 & 37.03 & 36.46 & 37.03 & 61.1 & 63.79 & 63.58 & 63.33  \\
 \hline
 \hline
 \multicolumn{5}{|c|}{Winogrande (FP32 Accuracy = 58.01\%)} & \multicolumn{4}{|c|}{Piqa (FP32 Accuracy = 74.21\%)} \\ 
 \hline
 \hline
 64 & 58.17 & 57.22 & 57.85 & 58.33 & 73.01 & 73.07 & 73.07 & 72.80 \\
 \hline
 32 & 59.12 & 58.09 & 57.85 & 58.41 & 73.01 & 73.94 & 72.74 & 73.18  \\
 \hline
 16 & 57.93 & 58.88 & 57.93 & 58.56 & 73.94 & 72.80 & 73.01 & 73.94  \\
 \hline
\end{tabular}
\caption{\label{tab:mmlu_abalation} Accuracy on LM evaluation harness tasks on GPT3-1.3B model.}
\end{table}

\begin{table} \centering
\begin{tabular}{|c||c|c|c|c||c|c|c|c|} 
\hline
 $L_b \rightarrow$& \multicolumn{4}{c||}{8} & \multicolumn{4}{c||}{8}\\
 \hline
 \backslashbox{$L_A$\kern-1em}{\kern-1em$N_c$} & 2 & 4 & 8 & 16 & 2 & 4 & 8 & 16  \\
 %$N_c \rightarrow$ & 2 & 4 & 8 & 16 & 2 & 4 & 2 \\
 \hline
 \hline
 \multicolumn{5}{|c|}{Race (FP32 Accuracy = 41.34\%)} & \multicolumn{4}{|c|}{Boolq (FP32 Accuracy = 68.32\%)} \\ 
 \hline
 \hline
 64 & 40.48 & 40.10 & 39.43 & 39.90 & 69.20 & 68.41 & 69.45 & 68.56 \\
 \hline
 32 & 39.52 & 39.52 & 40.77 & 39.62 & 68.32 & 67.43 & 68.17 & 69.30  \\
 \hline
 16 & 39.81 & 39.71 & 39.90 & 40.38 & 68.10 & 66.33 & 69.51 & 69.42  \\
 \hline
 \hline
 \multicolumn{5}{|c|}{Winogrande (FP32 Accuracy = 67.88\%)} & \multicolumn{4}{|c|}{Piqa (FP32 Accuracy = 78.78\%)} \\ 
 \hline
 \hline
 64 & 66.85 & 66.61 & 67.72 & 67.88 & 77.31 & 77.42 & 77.75 & 77.64 \\
 \hline
 32 & 67.25 & 67.72 & 67.72 & 67.00 & 77.31 & 77.04 & 77.80 & 77.37  \\
 \hline
 16 & 68.11 & 68.90 & 67.88 & 67.48 & 77.37 & 78.13 & 78.13 & 77.69  \\
 \hline
\end{tabular}
\caption{\label{tab:mmlu_abalation} Accuracy on LM evaluation harness tasks on GPT3-8B model.}
\end{table}

\begin{table} \centering
\begin{tabular}{|c||c|c|c|c||c|c|c|c|} 
\hline
 $L_b \rightarrow$& \multicolumn{4}{c||}{8} & \multicolumn{4}{c||}{8}\\
 \hline
 \backslashbox{$L_A$\kern-1em}{\kern-1em$N_c$} & 2 & 4 & 8 & 16 & 2 & 4 & 8 & 16  \\
 %$N_c \rightarrow$ & 2 & 4 & 8 & 16 & 2 & 4 & 2 \\
 \hline
 \hline
 \multicolumn{5}{|c|}{Race (FP32 Accuracy = 40.67\%)} & \multicolumn{4}{|c|}{Boolq (FP32 Accuracy = 76.54\%)} \\ 
 \hline
 \hline
 64 & 40.48 & 40.10 & 39.43 & 39.90 & 75.41 & 75.11 & 77.09 & 75.66 \\
 \hline
 32 & 39.52 & 39.52 & 40.77 & 39.62 & 76.02 & 76.02 & 75.96 & 75.35  \\
 \hline
 16 & 39.81 & 39.71 & 39.90 & 40.38 & 75.05 & 73.82 & 75.72 & 76.09  \\
 \hline
 \hline
 \multicolumn{5}{|c|}{Winogrande (FP32 Accuracy = 70.64\%)} & \multicolumn{4}{|c|}{Piqa (FP32 Accuracy = 79.16\%)} \\ 
 \hline
 \hline
 64 & 69.14 & 70.17 & 70.17 & 70.56 & 78.24 & 79.00 & 78.62 & 78.73 \\
 \hline
 32 & 70.96 & 69.69 & 71.27 & 69.30 & 78.56 & 79.49 & 79.16 & 78.89  \\
 \hline
 16 & 71.03 & 69.53 & 69.69 & 70.40 & 78.13 & 79.16 & 79.00 & 79.00  \\
 \hline
\end{tabular}
\caption{\label{tab:mmlu_abalation} Accuracy on LM evaluation harness tasks on GPT3-22B model.}
\end{table}

\begin{table} \centering
\begin{tabular}{|c||c|c|c|c||c|c|c|c|} 
\hline
 $L_b \rightarrow$& \multicolumn{4}{c||}{8} & \multicolumn{4}{c||}{8}\\
 \hline
 \backslashbox{$L_A$\kern-1em}{\kern-1em$N_c$} & 2 & 4 & 8 & 16 & 2 & 4 & 8 & 16  \\
 %$N_c \rightarrow$ & 2 & 4 & 8 & 16 & 2 & 4 & 2 \\
 \hline
 \hline
 \multicolumn{5}{|c|}{Race (FP32 Accuracy = 44.4\%)} & \multicolumn{4}{|c|}{Boolq (FP32 Accuracy = 79.29\%)} \\ 
 \hline
 \hline
 64 & 42.49 & 42.51 & 42.58 & 43.45 & 77.58 & 77.37 & 77.43 & 78.1 \\
 \hline
 32 & 43.35 & 42.49 & 43.64 & 43.73 & 77.86 & 75.32 & 77.28 & 77.86  \\
 \hline
 16 & 44.21 & 44.21 & 43.64 & 42.97 & 78.65 & 77 & 76.94 & 77.98  \\
 \hline
 \hline
 \multicolumn{5}{|c|}{Winogrande (FP32 Accuracy = 69.38\%)} & \multicolumn{4}{|c|}{Piqa (FP32 Accuracy = 78.07\%)} \\ 
 \hline
 \hline
 64 & 68.9 & 68.43 & 69.77 & 68.19 & 77.09 & 76.82 & 77.09 & 77.86 \\
 \hline
 32 & 69.38 & 68.51 & 68.82 & 68.90 & 78.07 & 76.71 & 78.07 & 77.86  \\
 \hline
 16 & 69.53 & 67.09 & 69.38 & 68.90 & 77.37 & 77.8 & 77.91 & 77.69  \\
 \hline
\end{tabular}
\caption{\label{tab:mmlu_abalation} Accuracy on LM evaluation harness tasks on Llama2-7B model.}
\end{table}

\begin{table} \centering
\begin{tabular}{|c||c|c|c|c||c|c|c|c|} 
\hline
 $L_b \rightarrow$& \multicolumn{4}{c||}{8} & \multicolumn{4}{c||}{8}\\
 \hline
 \backslashbox{$L_A$\kern-1em}{\kern-1em$N_c$} & 2 & 4 & 8 & 16 & 2 & 4 & 8 & 16  \\
 %$N_c \rightarrow$ & 2 & 4 & 8 & 16 & 2 & 4 & 2 \\
 \hline
 \hline
 \multicolumn{5}{|c|}{Race (FP32 Accuracy = 48.8\%)} & \multicolumn{4}{|c|}{Boolq (FP32 Accuracy = 85.23\%)} \\ 
 \hline
 \hline
 64 & 49.00 & 49.00 & 49.28 & 48.71 & 82.82 & 84.28 & 84.03 & 84.25 \\
 \hline
 32 & 49.57 & 48.52 & 48.33 & 49.28 & 83.85 & 84.46 & 84.31 & 84.93  \\
 \hline
 16 & 49.85 & 49.09 & 49.28 & 48.99 & 85.11 & 84.46 & 84.61 & 83.94  \\
 \hline
 \hline
 \multicolumn{5}{|c|}{Winogrande (FP32 Accuracy = 79.95\%)} & \multicolumn{4}{|c|}{Piqa (FP32 Accuracy = 81.56\%)} \\ 
 \hline
 \hline
 64 & 78.77 & 78.45 & 78.37 & 79.16 & 81.45 & 80.69 & 81.45 & 81.5 \\
 \hline
 32 & 78.45 & 79.01 & 78.69 & 80.66 & 81.56 & 80.58 & 81.18 & 81.34  \\
 \hline
 16 & 79.95 & 79.56 & 79.79 & 79.72 & 81.28 & 81.66 & 81.28 & 80.96  \\
 \hline
\end{tabular}
\caption{\label{tab:mmlu_abalation} Accuracy on LM evaluation harness tasks on Llama2-70B model.}
\end{table}

%\section{MSE Studies}
%\textcolor{red}{TODO}


\subsection{Number Formats and Quantization Method}
\label{subsec:numFormats_quantMethod}
\subsubsection{Integer Format}
An $n$-bit signed integer (INT) is typically represented with a 2s-complement format \citep{yao2022zeroquant,xiao2023smoothquant,dai2021vsq}, where the most significant bit denotes the sign.

\subsubsection{Floating Point Format}
An $n$-bit signed floating point (FP) number $x$ comprises of a 1-bit sign ($x_{\mathrm{sign}}$), $B_m$-bit mantissa ($x_{\mathrm{mant}}$) and $B_e$-bit exponent ($x_{\mathrm{exp}}$) such that $B_m+B_e=n-1$. The associated constant exponent bias ($E_{\mathrm{bias}}$) is computed as $(2^{{B_e}-1}-1)$. We denote this format as $E_{B_e}M_{B_m}$.  

\subsubsection{Quantization Scheme}
\label{subsec:quant_method}
A quantization scheme dictates how a given unquantized tensor is converted to its quantized representation. We consider FP formats for the purpose of illustration. Given an unquantized tensor $\bm{X}$ and an FP format $E_{B_e}M_{B_m}$, we first, we compute the quantization scale factor $s_X$ that maps the maximum absolute value of $\bm{X}$ to the maximum quantization level of the $E_{B_e}M_{B_m}$ format as follows:
\begin{align}
\label{eq:sf}
    s_X = \frac{\mathrm{max}(|\bm{X}|)}{\mathrm{max}(E_{B_e}M_{B_m})}
\end{align}
In the above equation, $|\cdot|$ denotes the absolute value function.

Next, we scale $\bm{X}$ by $s_X$ and quantize it to $\hat{\bm{X}}$ by rounding it to the nearest quantization level of $E_{B_e}M_{B_m}$ as:

\begin{align}
\label{eq:tensor_quant}
    \hat{\bm{X}} = \text{round-to-nearest}\left(\frac{\bm{X}}{s_X}, E_{B_e}M_{B_m}\right)
\end{align}

We perform dynamic max-scaled quantization \citep{wu2020integer}, where the scale factor $s$ for activations is dynamically computed during runtime.

\subsection{Vector Scaled Quantization}
\begin{wrapfigure}{r}{0.35\linewidth}
  \centering
  \includegraphics[width=\linewidth]{sections/figures/vsquant.jpg}
  \caption{\small Vectorwise decomposition for per-vector scaled quantization (VSQ \citep{dai2021vsq}).}
  \label{fig:vsquant}
\end{wrapfigure}
During VSQ \citep{dai2021vsq}, the operand tensors are decomposed into 1D vectors in a hardware friendly manner as shown in Figure \ref{fig:vsquant}. Since the decomposed tensors are used as operands in matrix multiplications during inference, it is beneficial to perform this decomposition along the reduction dimension of the multiplication. The vectorwise quantization is performed similar to tensorwise quantization described in Equations \ref{eq:sf} and \ref{eq:tensor_quant}, where a scale factor $s_v$ is required for each vector $\bm{v}$ that maps the maximum absolute value of that vector to the maximum quantization level. While smaller vector lengths can lead to larger accuracy gains, the associated memory and computational overheads due to the per-vector scale factors increases. To alleviate these overheads, VSQ \citep{dai2021vsq} proposed a second level quantization of the per-vector scale factors to unsigned integers, while MX \citep{rouhani2023shared} quantizes them to integer powers of 2 (denoted as $2^{INT}$).

\subsubsection{MX Format}
The MX format proposed in \citep{rouhani2023microscaling} introduces the concept of sub-block shifting. For every two scalar elements of $b$-bits each, there is a shared exponent bit. The value of this exponent bit is determined through an empirical analysis that targets minimizing quantization MSE. We note that the FP format $E_{1}M_{b}$ is strictly better than MX from an accuracy perspective since it allocates a dedicated exponent bit to each scalar as opposed to sharing it across two scalars. Therefore, we conservatively bound the accuracy of a $b+2$-bit signed MX format with that of a $E_{1}M_{b}$ format in our comparisons. For instance, we use E1M2 format as a proxy for MX4.

\begin{figure}
    \centering
    \includegraphics[width=1\linewidth]{sections//figures/BlockFormats.pdf}
    \caption{\small Comparing LO-BCQ to MX format.}
    \label{fig:block_formats}
\end{figure}

Figure \ref{fig:block_formats} compares our $4$-bit LO-BCQ block format to MX \citep{rouhani2023microscaling}. As shown, both LO-BCQ and MX decompose a given operand tensor into block arrays and each block array into blocks. Similar to MX, we find that per-block quantization ($L_b < L_A$) leads to better accuracy due to increased flexibility. While MX achieves this through per-block $1$-bit micro-scales, we associate a dedicated codebook to each block through a per-block codebook selector. Further, MX quantizes the per-block array scale-factor to E8M0 format without per-tensor scaling. In contrast during LO-BCQ, we find that per-tensor scaling combined with quantization of per-block array scale-factor to E4M3 format results in superior inference accuracy across models. 


\end{document}


