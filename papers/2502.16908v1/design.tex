\section{Design of \robot}

This section describes our hardware design and implementation details. Our primary goal is to achieve high-speed manipulation with low gear-ratio proprioceptive actuators while maintaining sufficient durability and ease of maintenance.

\subsection{Proprioceptive actuators}

\subsubsection{Choice of actuators}
We choose two types of off-the-shelf proprioceptive actuators: T-motor AK70-10 for shoulder and elbow and Cubemars RMD X4 V2 for wrist and forearm. Both types have 1:10 gear ratio with integrated drivers. Each arm is powered by two 24V Mean Well LRS-600-24 switch mode power supplies (SMPS). To protect the power supply from the back electromotive force generated by the actuators, SEMI-REX MD110-16 diodes are installed between the power supply and the actuators. 

\subsubsection{Sensor-free torque estimation for control}
Instead of using expensive torque sensors, we estimate the torque output from the actuators by building a custom current-to-torque function using our custom calibration process. To do this, we fix an actuator to a 3D-printed fixture connected to a standard electronic scale, measure the torques by giving a wide range of current values, and create a linear interpolation mapping between current and torque values. We extrapolate to estimate values beyond the available data range to compute the torque.

\subsection{Actuator placement and transmission}

\subsubsection{Low inertia actuator placement}

\begin{figure}[hbt]
    \centering
    \includegraphics[width=0.57\linewidth]{fig/dof_position_back.png}
    \caption{Positions of the six actuators. Four heavy and strong actuators are attached to the base. Two small actuators to rotate the wrist is attached to the elbow and wrist.}
    \label{fig:dof_position_back}
\end{figure}

\robot arm features 6 DoF: three on the shoulder, one on the elbow, and two on the wrist. We place the shoulder and elbow actuators near the base as illustrated in Figure~\ref{fig:dof_position_back}. By positioning the heavy actuators near the shoulder and transmitting the motion via the linkage system, we can effectively reduce the moment of inertia of the arm, achieving faster motion with the same amount of power from the actuators. We attach the two weaker but lightweight actuators directly on the elbow and wrist to keep the design simple.

\subsubsection{Parallelogram linkage for one-to-one torque transfer}
\label{sec:linkage}

One of the body-mounted actuators drives the elbow joints through a four-bar parallelogram linkage, as illustrated in Figure~\ref{fig:arm_assembly_and_gripper}a. This linkage transfers torque in 1:1 ratio, preserving the low gear ratio of the actuator and simplifying the model used in simulation and control.

% \begin{figure} [hbt]
%      \centering
%      \includegraphics[width=0.95\linewidth]{fig/upper_arm_detailed_modi2.png}   
%         \caption{Detailed assembly figure of the upper arm. Black parts are aluminum-machined, and white parts are 3D printed PLA.}
%         \label{fig:arm_assembly}
% \end{figure}

\subsection{Material selection}
We primarily build the arm using 3D printing with PLA. This choice of material provides several advantages. First, compared to metals, which are the common choice for existing robot arms, PLA greatly reduces the total mass of the arm. Second, 3D printing is cost-effective and enables rapid part replacement for design iteration and maintenance. One disadvantage of PLA is that it is non-rigid compared to metals. This may harm the durability of the hardware and reduce control accuracy when exposed to high loads. Therefore, for the linkage for the elbow, where high loads are applied, we use aluminum, as shown in Figure~\ref{fig:arm_assembly_and_gripper}a. This way, we minimize the complicated and costly machining process while maintaining the rigidity sufficient for manipulation. To validate that our robot attains enough rigidity, we perform finite element method (FEM) analysis. Specifically, we assess the arm's deformation under a 10 N load applied along the x-axis (leftward) and the z-axis (downward).

Figure~\ref{fig:analysis} illustrates the deformation results under z-axis and x-axis loading, respectively. Although this material change involves a trade-off with an increase in moving mass from 0.962 kg to 1.09 kg, it reduces deformation under loading conditions. For z-axis loading, the deformation decreases from 1.69 mm with PLA components to 1.58 mm with the addition of aluminum. Similarly, for x-axis loading, the deformation is reduced from 2.85 mm to 2.24 mm.
% Under z-axis loading, the deformation is decreased by approximately 7.08\% by using aluminum components, and under x-axis loading, the deformation decreases by about 31.85\%.
%with 11.0\% increase in moving mass including the gripper.

\begin{figure}[hbt]
    \centering
    \includegraphics[width=0.75\linewidth]{fig/material_fem.png}
    \caption{FEM analysis of deformation under a 10 N load applied along the (a) z-axis and (b) x-axis for configurations with all PLA and partially aluminum components. Deformation reduces from 1.69 mm to 1.58 mm (z-axis) and 2.85 mm to 2.24 mm (x-axis) with aluminum reinforcement. Deformation is exaggerated for clarity and visualization.}
    \label{fig:analysis}
\end{figure}

\subsection{Custom gripper}
We also design a compact jaw gripper driven by a Dynamixel XM430 motor. The gripper utilizes a simple direct-drive 1 DoF gear mechanism, enabling jaw manipulation while maintaining ease of assembly. The gear mechanism is made of PLA, while the gripper pads are printed from flexible TPU to accommodate objects of various shapes. The entire assembly weighs approximately 290 g. Figure~\ref{fig:arm_assembly_and_gripper}b illustrates the design of the gripper.

\begin{figure}
    \begin{subfigure}[t]{\linewidth}
        \centering
        \includegraphics[width=0.85\linewidth]{fig/upper_arm_detailed_modi2.png}
        \caption{Arm assembly}
    \end{subfigure}
    \begin{subfigure}[t]{\linewidth}
        \centering
        \includegraphics[width=0.8\linewidth]{fig/gripper_cad.png}
        \caption{Gripper}
    \end{subfigure}
    \caption{(a) Assembly of the upper arm. Black parts are aluminum-machined, and white parts are 3D printed PLA. (b) Assembly of the custom-designed gripper. The brown component is the Dynamixel XM430 motor, the white parts are PLA parts, and the green parts are flexible TPU fingers.}
    \label{fig:arm_assembly_and_gripper}
\end{figure}

% \begin{figure}[hbt]
%     \centering
%     \includegraphics[width=0.8\linewidth]{fig/gripper_cad.png}
%     \caption{Assembly diagram of the custom-designed gripper. The brown component represents the Dynamixel XM430 motor, the white parts are PLA parts, and the green parts are flexible TPU fingers.}
%     \label{fig:gripper}
% \end{figure}
