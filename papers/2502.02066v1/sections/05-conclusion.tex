\section{Conclusion and Future Work}
% In this paper, we have presented a novel framework that integrates the predictive capabilities of large language models (LLMs) with the precision of classical planners to optimize the execution of household tasks by an assistive agent. We introduce a domain of common household tasks in PDDL, and demonstrate that our agent effectively anticipates the sequence of tasks for the day, informed by the user's previous routine and preferences, and leverages the common subtasks across different tasks to craft an optimal plan. However, there remains room for enhancing the framework's efficiency and utility. 
This paper described a framework for task anticipation and action execution by an agent in complex household environments. The framework leverages the generic knowledge encoded in LLMs for high-level task anticipation based on limited prompts, and plans a sequence of finer-granularity actions based on the domain-specific knowledge encoded in PDDL to jointly accomplish the anticipated tasks. We demonstrate a substantial reduction in action execution time and plan length in comparison with a planning system that does not consider anticipated actions. In future work, we intend to explore scalability of this framework to more complex domains, incorporate probabilistic planning, and explore implementation on a physical robot assistant. We will also explore the ability of LLMs to automatically adapt task anticipation to the preferences and needs of specific individuals in the household in which the agent is operating. 

%individual as it is important for a household agent to adapt to the specific household's needs.
%In future work, we aim to address the time consumption issue in planning for long-horizon tasks, seeking ways to speed up the process without compromising the quality of the plan. Additionally, we plan to extend our model's anticipation and planning capabilities to account for specific times of the day, such as morning routines or evening chores. 
% We also aim to incorporate optimization of resource usage into our planning formulation, ensuring the agent can consider factors like water wastage when executing tasks. This aspect could be particularly beneficial for eco-conscious households or in resource-limited settings.



% For a common household agent, there are multiple tasks having common subtasks that can be done in parallel. It is important for such an intelligent agent to anticipate the possible tasks that it has to do throughout the day, and create an optimal plan taking care of the common preconditions and sub-tasks. We present a framework that uses language models for learning the pattern of tasks done in a house and anticipates future tasks based on the user preference and previous routine. We leverage the internal commonsense reasoning of large language models in understanding the tasks as well as the user preference before coming up with the list of anticipated tasks. We ground the model by providing the LLM with a list of tasks and available objects in a scene. We make use of PDDL based planners to ensure that the final plan is accurate and streamlines common subtasks between different household tasks. 

% \textbf{Future Work} Planning takes a lot of time for long-horizon tasks., specific time of the day; optimizing on materials (water wastage, etc?), real-world application, more rooms,


% \begin{table}[h]
% \centering
% \begin{tabular}{|c|c|c|c|c|c|}
% \hline
% \multicolumn{2}{|c|}{} &  \multicolumn{4}{c|}{\textbf{Number of anticipated tasks}} \\ 
% \hline
% \textbf{Task} & \textbf{Task Name} & \textbf{0} & \textbf{1} & \textbf{3} & \textbf{7} \\
% \hline
% T1 & serve veggie & 319 & \multirow{2}{*}{392}  & \multirow{4}{*}{654} & \multirow{8}{*}{1088} \\ \cline{1-3}

% T2 & clean remaining food & 73 & & & \\ \cline{1-4} 

% T3 & clean dishes & 31 & \multirow{2}{*}{272} & & \\ \cline{1-3}

% T4 & clean house & 241 & & & \\\cline{1-5}

% T5 & dust sofa & 285 & \multirow{2}{*}{447} & \multirow{4}{*}{604} & \\ \cline{1-3}

% T6 & clear trash & 232 & & & \\\cline{1-4}

% T7 & cutting plants & 151 & \multirow{2}{*}{257} & & \\\cline{1-3}

% T8 & watering plants & 76 & & & \\\cline{1-6}

% \multicolumn{2}{|c|}{\textbf{Total Cost}} & \makebox[10pt]{ \textbf{1408} }  & \textbf{1368} & \textbf{1258} & \textbf{1088} \\
% \hline
% \end{tabular}
% \caption{Example table showing a particular day's activity}
% \end{table}

% \begin{figure}[!htbp]
% \centering
% \includegraphics[width=0.5\textwidth]{sections/figures/table.png}
% \caption{Heatmap for transition probability matrix of common household tasks. The colour legend on bottom left shows the high-level sequences each task belongs to.}
% \label{fig: heatmap}
% \end{figure}
% \color{black}
