\newpage
\section{STEM Topics}
\label{stem_topics}

\textbf{Computer Science }: \\
(Introductory)  Depth-First Search\\
(Intermediate)  Back Propagation \\
(Advanced)      Recurrent Neural Networks\\

\noindent\textbf{Physics}: \\
(Introductory)  Distance and Displacement\\
(Intermediate)  Thermal Equilibrium \\
(Advanced)      Thin Film Interference\\

\noindent \textbf{Statistics}: \\
(Introductory)  Normal Distribution\\
(Intermediate)  Central Limit Theorem \\
(Advanced)      Linear Regression\\

\noindent \textbf{Civil Engineering}: \\
(Introductory)  Lattice Structure\\
(Intermediate)  Tensile Structure \\
(Advanced)      Curtain Wall System\\

\noindent \textbf{Psychology}: \\
(Introductory)  Retroactive and Proactive Interference\\
(Intermediate)  Feature Integration Theory \\
(Advanced)      Walker's Action Decrement Theory\\


\section{GPT Generation Prompts}
\label{prompts}

\subsection{EWP}

[FewShot] \footnote{EWP-NoFewShot does not have this part.}\\

\noindent Instructions:\\
Talk to a friend about the topic: [topic] in the domain: [domain].\\
Explain how the topic works.\\
Use the given example to help explain how the topic words: [example].\\
 Use the scenario to provide additional context: [scenario].\\

\noindent Output format: \\
a piece of writing with short paragraphs (280 characters for each paragraph).\\

\noindent Walkthrough Guidelines:\\
Tell the story from a first-person perspective.\\
Walk through the story timeline in a sequence. \\
Be sure to explain each dimension of the topic in detail, relating it back to the given example and scenario.\\

\noindent Emotional Guidelines:\\
Take the second-person audience on an emotional journey.\\
Add visually descriptive details in the storytelling.\\
Use emotional languages (both negative and positive).\\
Add questions that echo with the audience and spark curiosity.



\subsection{WP}
\label{WP_prompt}

Narrative: [Output from EWP]\\

\noindent Instructions:\\
You are given a science narrative that explains how [topic] works.\\
Keep the same tone and structure as the given narrative.\\
Remove the example of [example\_label] from the narrative.\\
Do not include ANY examples.\\
Only provide a technical walkthrough of [topic] following the same structure.


\subsection{EP-NoFewShot}
\label{EP_prompt}

Instructions:\\
Write a series of Tweets explaining the given topic: [topic] in the domain of [domain].\\
Make sure each Tweet is less than 280 characters.\\

\noindent Do not use technical jargon and define all technical components.\\
Do not walkthrough using timeline sequence. \\
Do not use words such as ``before", ``after", ``then", ``next", ``also", ``first", ``second", ``third", ``last", ``summary".\\
Explain a different technical component of the topic in each tweet in a non-sequential modular approach.\\
Each tweet stands alone and allows the reader to navigate through the explanations in various orders.\\


\subsection{EW}
\label{EW_prompt}

Narrative: [Output from EWP]\\

\noindent Instructions:\\
You are given a science narrative that uses emotional and engaging language to explain a concept.\\
Your task is to write a new narrative that maintains the same structure of the given narrative but removes all emotional language and all subjective language.\\
Edit the sentences to remove `you', `I', and `we' pronouns. \\
Do not use rhetorical or confirmation questions.\\
Maintain the active voice and use ``people" or ``student" or other general terms as the subject.\\
Use objective and generalizable language wherever possible.\\
Remove any extraneous descriptions and adjectives.\\
Use formal language.\\