\section{Background on Tweetorials}
Tweetorials are a form of social media science communication on Twitter, defined as a chronological series of tweets that explains a science topic \cite{symplurTweetorialsFrom, Bruggemann2020-ez}. According to Breu and Berstein, Tweetorials emerged from the medical community to continue education for other medical professionals, often containing medical jargon and not intended for everyday audiences. As the form became popularized, other communities on Twitter began to adopt the format \cite{doi:10.1056/NEJMp1906790, symplurTweetorialsFrom, Bruggemann2020-ez}. As a result, Breu's definition of a Tweetorial, ``a collection of threaded tweets aimed at teaching users who engage with them," can be applied to various domains such as biology, computer science, and economics. 

Gero et. al. identified and compared key features of Tweetorials to traditional science communication strategies \cite{10.1145/3479566}. The authors identified 3 main structural components of Tweetorials to be the hook, body, and conclusion. Previous research has explored strategies for writing engaging hooks \cite{10.1145/3643834.3661587, long2023tweetorialhooksgenerativeai}. Thus, we focus on specific strategies that appear within the body of a Tweetorial. 

\subsection{Key Strategies of the Tweetorial Body}
According to Gero et. al., the body of a Tweetorial is the most varied in length and types of detail used to explain a given topic \cite{10.1145/3479566}. The researchers found that Tweetorials contain specific techniques such as the use of an example, a step-by-step walkthrough structure, and personal language. 

\subsubsection{Example (E)}
\label{example_defintion}
Like traditional science communication, Tweetorials often contain an example that uses familiar or simpler concepts to explain the main idea \cite{10.1145/3479566}. We define this as an \textbf{example (E)} technique. Components of the example are the \textit{use case} and the \textit{scenario}. The \textit{use case} is a general application of the scientific topic and the \textit{scenario} is a specific situation that describes how the science topic was applied. For example, one Tweetorial uses the example of fingertips getting wrinkly in the bath to explain water immersion wrinkling.\footnote{\href{http://language-play.com/tech-tweets/tweetorial/4}{http://language-play.com/tech-tweets/tweetorial/4}} For this Tweetorial, the use case is when fingers get wrinkly in water. The scenario is a parent bathing their child. We use the use case and scenario for more finegrain control over the LLM-generated narratives in Section \ref{narrative_generation_strategy}. 

% Another 


% \grace{where to add descirption fo data inputs}
% The [example] data field is in the format of a short phrase. For instance: \textit{``Spider’s Web”} is the [example] for the topic Tensile Structure in Civil Engineering. The [scenario] data field contains a scenario of a personal narrative that narrates the example through a scenario and connects with the scientific topic. For instance, for the \textit{``Spider’s Web”} example, the scenario is \textit{``During our late-night camping, my adventurous friend decided to challenge a playful spider. He picked up a small twig and started slowly poking its web. Upon noticing the twig, the spider ran towards it, displaying its territorial instinct. I witnessed how the web, a miraculous tensile structure, withheld the pressure without falling apart. Thanks to the constant tension in the silk material, the web stayed steady, absorbing the additional load, distributed throughout its double-curved surface, and transmitting it to its anchor points.”}. We used the prompt seen in Figure 

\subsubsection{Walkthrough (W)}
Tweetorial structures often use a narrative and signposting to establish a narrative structure. We define these two attributes as the \textbf{walkthrough (W)} technique. Narrative is defined by a series of connected events to explain a given topic. Tweetorials signpost by using transition words like ``Firstly" and ``Secondly,'' or by using a list of questions to help frame the sequence of the Tweetorial. In a Tweetorial about selectivity metrics in college rankings, the author uses the second tweet to list out 3 driving questions for the explanation and to establish the structure of the Tweetorial: ``1. Does ``selectivity" actually tell you anything useful about how good your education will be? 2. What does ``selectivity" actually measure that is of value to a student? 3. Why do I have the feeling somebody chose this metric cause they just needed more stuff to rank by?"\footnote{\href{http://language-play.com/tech-tweets/tweetorial/14}{http://language-play.com/tech-tweets/tweetorial/14}} 
% \yy{the [] in the quote is easily confused with citations, and they are not in the actual tweetorial. This example is more aligned with motivation after the hook, rather than a body walkthrough}

\subsubsection{Personal Language (P)}
Tweetorials often use subjective, conversational, and informal language. We define these features as the \textbf{personal language (P)} technique.  Some authors might use first-person pronouns like ``I" to talk from their subjective perspective,\footnote{\href{http://language-play.com/tech-tweets/tweetorial/1}{http://language-play.com/tech-tweets/tweetorial/1}} or use the second person, ``you," to directly address the audience in conversation: ``You can think of a Hash Function like a magic fingerprint reader."\footnote{\href{http://language-play.com/tech-tweets/tweetorial/31}{http://language-play.com/tech-tweets/tweetorial/31}} Some authors might use ALLCAPS or emojis to engage in informal language and humor: ``OH MAN MY HEAD HURTS AND MY LIMBS TINGLE EVERY TIME I GO TO A CHINESE RESTAURANT, I THINK IT MAY BE ALL THE MSG THEY PUT IN THE FOOD???".\footnote{\href{http://language-play.com/tech-tweets/tweetorial/33}{http://language-play.com/tech-tweets/tweetorial/33}} 

\vspace{5mm}
Figure \ref{fig:everything} provides an annotated Tweetorial highlighting these three techniques (example (E), walkthrough (W), and personal language (P)) on the topic of Walker's Action Decrement Theory in Psychology to demonstrate how they are applied in a science explanation. We use these three techniques to ground our approach in understanding reader preferences for science communication on social media and how writers explore the design space for science writing structures and styles. 


\begin{figure}
    \centering\includegraphics[width=0.62\linewidth]{Figures/Everything_Annotate.png}
    \caption{Annotated Tweetorial on the topic of Walker's Action Decrement Theory in Psychology with color highlights corresponding to Example, Walkthrough, and Personal Language.}
    \label{fig:everything}

\end{figure}