\section{Related Work}

\subsection{HCI for Science Communication}
% \grace{add some more bits from, edit intro

% 1. motivate this more, doing a deeper indepth XYZ to see how audience respond to particular aspects of writing, generalize aspects of writing
% 2. related work defends what we do is novel [LLM section]}

% Science communication has transitioned away from traditional scholarly articles and towards more informal platforms such as blogs and social media \cite{hou2017hacking, williams2022hci}. 
% \grace{take paragraph 1, cut last sentence
% take paragraph 2, --> add to the end of paragraph 2, research find it hard ebcause of blah blah --> use these findings to support people }

Science communication is integral to engage everyday people in science advancements and help them understand the world around them. Research has explored how new forms of science communication, such as Tweetorials, are situated among other genres, how different social media platforms shape the expectations that readers have, readers' interaction behaviors with science writing, and the way readers access scientific information \cite{heyd2015digital, williams2022hci, hargittai2018young, tardy2023spread}. 

% recommendations for communication systematic review of paper

Science communication on social media is informal and often told from a personal perspective to engage a broader audience \cite{10.1145/3479566}. Previous work has compiled recommendations for effective science communication on social media based on how likely using these strategies will increase engagement: comments, likes, and shares  \cite{zhang2022no, bik2015ten, fontaine2019communicating}. But these works do not consider whether these techniques are actually effective for readers to understand the science. Our work evaluates whether these different science communication strategies on Twitter are effective in helping readers engage with and understand the science.  

For scientists seeking to engage with science communication on social media, they struggle to understand their ``audience" which consists of readers from diverse backgrounds, levels of scientific literacy, and preferences which makes it challenging for scientists to determine \textit{who} they are writing for \cite{rice2017contexts, schafer2017changing, williams2022hci}. Furthermore, each social media platform has its own set of community norms and demographics \cite{pewresearchFactsAbout, oktay2014demographic}. Research has also shown how scientists struggle to reach non-research audiences on social media and to adapt to these new genres of science communication \cite{cote2018scientists, kopke2019stepping, lorono2018responsibility, koivumaki2020social}. Even though language with strong sentiment can predict success on social media, many scientists are hesitant about such language having negative effects on their credibility \cite{yuan2020s, yuan2019should, gilbert2020run}. We use these findings to explore how providing different structure and style options to writers can help accommodate different writers' hesitancy and comfort levels with science communication on social media. 

% We provide scientists with different ways to structure their science explanations to help them consider what content to include and how to frame it. We also provide scientists with options to view science narratives with and without personal language to help writers balance their personal and professional identities. 

% we support readers and writers in navigating the different ways to structure and style science communication on social media.

% Science experts are often hesitant about engaging in the form of science communication on social media because inaccurate interpretations or misrepresentations of their work could affect public understanding and their own reputations \cite{lorono2018responsibility}. Scientists often feel conflicted over their personal and professional identities when engaging in science communication on social media \cite{koivumaki2020social}. E Despite these challenges, many scientists want to communicate their work on social media to increase public interest in science or provide useful knowledge to the public \cite{williams2022hci}. 



% \grace{a lot, cut these and retool and add to end of 2nd paragraph}
% Our work provides scientists with different narrative structures and styles to accommodate different writers' hesitancy and comfort levels with science communication on social media. 


% As such, there is a growing need to make science accessible and relatable for everyone \cite{williams2022hci, jones2019r}. But scientists struggle to understand their audience and the appropriate framing strategies for different audiences \cite{}. 

% \subsection{Comparison-based User Evaluations}

% Having users explore two or more options in parallel is grounded in HCI research and traditional design practices to evaluate end-user preferences and to support diverse ideation \cite{buxton2010sketching}. In software engineering, this process is referred to as A/B testing where end-users are exposed to different variants of a system to evaluate the merits of each design \cite{quin2024b}. A/B testing has been used to evaluate different graphical user interfaces on users' subsequent click rates, user conversions, and estimates on total user spend during a subscription period \cite{johari2017peeking, wang2019heavy, duan2021online}. 

% Comparison-based evaluations are also used to explore individual preferences for how information presented to understand how to design better for certain communities. In healthcare, A/B testing has used to understand patient preferences for computer-based applications, conversational user interfaces, and digital messaging on no show rates \cite{brennan1998improving, kocielnik2021can, berliner2020s}. Our work builds upon these findings to understand reader preferences for different narrative structures of science communication.  

% More recently, comparison-based user evaluations have also been used to in the ideation process to support writers and students evaluate LLM-generated outputs through on user interface designs to better support parallel comparisons for LLM generated outputs. Dennison et al. designed an interface with side-by-side comparisons of different LLM outputs that were generated from the same prompt to help students with contrastive learning \cite{dennison2024consumers}. Gero et al. explored different color coding and user interface design methods to help users evaluate LLM-outputs at scale, highlighting semantic similarities and differences across textual documents \cite{gero2024supporting}. Recent work has explored the intersection of writing and parallel exploration for revisions in the writing process, user preferences for choosing from multiple suggestions instead of writing external prompts for LLMs, and new interfaces for viewing persistent text selection that help writers store and compare variations in text \cite{reza2023abscribe, dang2023choice, beaudouin2000reification}. Our work explores how parallel comparison between different science narrative structures accommodates different writers' preferences for how to communicate on social media. By offering multiple options, we help writers explore the design space by alternating between perspectives as a writer and reader when exploring different narrative structures.

% extends these findings to explore how scientist can use different narrative structures as a way to explore their own comfort

% katy's sensemaking at scale
% \cite{gero2024supporting}

% for various forms of science communication. Work in the field of 


\subsection{LLM-based Writing Support}
Recent advances in AI writing capabilities have accelerated work in LLM-based writing interfaces \cite{biermann2022tool, buschek2021impact, shakeri2021saga, sommers1980revision, lee2024design}. Our contributions are most directly tied to the specific writing task of drafting (writing stage) science narrative explanations (purpose) for an everyday audience (audience) \cite{lee2024design}.

Previous research has explored how analogies can be used to translate technical content into more relatable references for people \cite{august2020writing, kim2024authors, nguyen2024simulating, hullman2018improving}. LLM-based systems have been used to explore different ways to generate engaging and relatable hooks for science narratives \cite{long2023tweetorialhooksgenerativeai, 10.1145/3643834.3661587}. LLMs have been used for the personalization of scientific information to improve comprehension, alignment to individual preferences, and accommodate different science literacy levels \cite{august2023paper, das2023balancing, ding2023fluid}. But these cases focus on generating relatable examples or translating technical jargon into everyday language and do not connect the corresponding examples to an overarching narrative to explain the science. 

LLMs have been used in narrative generation to support story writing by tailoring generated narratives to user inputs such as story elements, topics, or rough sketches of plot development \cite{calderwood2022spinning, belz2024story, park2023designing, rashkin2020plotmachines}. New methods have explored ways to improve coherence in LLM-generated long-form story content through recursive prompting and revision and outline control \cite{yang2022re3, yang2022doc, wang2023improving}. While advancements have been made in improving coherence in LLM-generated long-form stories, generalizing these methods to science narratives is hard. Science narratives need both a coherent story and an accurate representation of the science. Our work explores how LLMs can support scientists in drafting a science narrative. To do so, we bridge research in relatable example generation with story narrative generation to create science narratives that follow a cohesive narrative structure around a relatable example. We provide scientists a baseline science narrative to iterate on. 

% While previous has explored how ways to improve the coherence of LLM generated long-form narratives, to our knowledge, narrative generations do not focus on the specific task of science communication through a worked example. 

% While using LLMs to generate coherent long-form stories has made significant advances in terms of story coherence, 

% use a singular example from 

% adapting science narratives for a social media specific format has not yet been explored. Our work explores how 


% 2 paragraphs -- how do we write this in relationship to this problem, don't want to frame it as a tool, also about the writers and readers (people have already done the hook **also say in background (2 papers)**), the narratives that are challenging, want to narrow in on explaining the narrative part of science. 

% how is it unique from writing support perspective: not a lot of papers support the whole narrative that needs to be cohesive/coherent narrative, step-by-step, 

% we're related to super topic of decomposing complex tasks, BUT we provide a full science narrative that is correct and coherent and we use a consistent example to make this happen. 

% Tweetorial Hooks Paper, Topic Scoping Paper, other LLM papers

% \subsection{LLM-based Writing Support}
% Previous research has explored the different ways in which intelligent writing assistants have been used to support writers. Lee et. al. surveyed 115 papers on intelligent writing assistants to construct the design space which cover five aspects: task, user, technology, interaction, and ecosystem \cite{lee2024design}. Our research focuses specifically on the dimensions of the task, the rhetorical purpose of a written artifact, and the user, who is using the system and what are their unique requirements. As such, we explore related work in those dimensions to frame our contributions. 

% \textbf{Task:} For the task dimension, we focus on the supporting the drafting and revision stages of writing (writing stage) for personal science communication on social media (writing context) for the purpose of educating (purpose) the general public (audience) about science topics. We focus specifically on support writers explore the format of Tweetorials, which are series of tweets that are used to explain a science topic on Twitter.

% Previous work in the task dimension fo drafting and revision explores story writing to help writers co-create stories with AI \cite{chung2022talebrush}, 

% the existing work in the domain of science communication for everyday audiences on social media, narrative strategies for science communication, and LLM-based writing experiences 
% % \grace{DIS paper, longitudinal, bit of a body section, closer to writing, ICCC main}



% \subsection{Science Communication on Social Media}

% In the context of news-related Twitter posts, Rudat et. al. found that ``informational value'' in a tweet had a large predictor on the retweet volume, or popularity of a given tweet \cite{RUDAT201575}. With roots in \cite{galtung1965structure}


% Hwong et. al. investigates the psycholinguistic features that make engaging space science-related social media messages. \cite{hwong2017makes}

% \subsection{Narrative Structures}

% \subsection{Personal Language}

% Gero et. al. explored the emergence of science communication on Twitter through a format called Tweetorials to appeal to everyday audiences \cite{10.1145/3479566}. The authors curated a dataset of Tweetorials and compared Tweetorials writing strategies to traditional science writing techniques. The authors found that Tweetorials often used subjective language to reference the author's own personal experiences, conversational language to engage audiences, and informal language and humor to appeal to pop culture or social pools of knowledge. Using Gero et. al.'s findings of these unique dimensions of Tweetorials, we define the term \textit{Personal Language Style} to include dimensions of subjective, conversational, and informal language characteristics. \grace{can also talk about the writer study gero et. al. did to see that it was difficult for writers to shift into informal register??? and we extend this work.} Our r

% In the book, ``Update Culture and the Afterlife of Digital Writing," John Gallagher analyzes the interactions between different groups of writers (redditors, Amazon reviewers, digital journalists and bloggers) consider their audiences into their writing practices. Gallagher's found a tension between what writers paid close attention to as opposed to what readers were drawn to \cite{gallagher2020update}. For example, while Amazon reviewers were focused on writing helpful reviews, readers sometimes preferred amusing reviews \cite{gallagher2020update}. We extend Gallagher's work to investigate these nuances within the domain of science communication to understand the nuances of both readers and writers when engaging with STEM topics on social media. 

% Gallagher also found a tension between writers' desire for personal style when engaging audiences and ``template rhetoric'' of their respective platforms. \cite{gallagher2020update}. ``Template rhetoric'' defines ....

% Gallagher demonstrates the importance of understanding the differing values and perspectives that writers and readers to understand the dynamic nature of digital writing. Gallagher focuses specifically on established writers (such as writers with a certain level of karma on Reddit, Amazon top Review credentials). Our research focuses specifically scientists who want to share STEM topics more broadly \grace{basically, survey in depth a specific subset of people ie. scientists...}

% \grace{think about reader study?}

% \subsection{LLM-Based Writing Tools}
