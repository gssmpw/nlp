\section{Discussion}




% \grace{overall, writers, the options are good, there is a design space, this enabled them to borrow some parts of the personal language, NOT all social media or all text book, show the two extremes they are able to see/find something in the middle. writers might not warm up to the polarized view of sicence, helping scientists pick where they want to be on the spectrum is important, options give them a continuum}

% \grace{findings were mixed, for both, go back to thoughts in the introduction and work them back in. air on the side of x was better than y, overall readers showed a preference for example and personal langauge. there's no one size fits all but these techniques are good strategies, two separate paragraphs one for reader study and one for writer structure, follow same structure as paper}

\subsection{The Importance of Narratives in Science Communication}
Overall, the use of an example was helpful for most readers. An example helped frame the science and provide readers a grounding for the science explanation. For certain topics, readers also enjoyed reading multiple different examples to gain an understanding of the topic through multiple perspectives. Overall the use of one or many examples was beneficial to most readers, the effectiveness of these strategies vary by reader. Some readers replace or augment science explanations with examples from their own life when the example used is unrelatable or missing. Some readers do not want an explanation at all. Different readers need different strategies to help them engage with and understand the science. 

To account for varying audience preferences, we provide writers with different narrative structures to help them recognize which narrative strategy was most suitable for a given topic. Instead of having to recall science communication strategies, writers could rely on the generated structure options to identify the effective techniques that were used and merge techniques across different options. This finding connects with the design principle of recognition over recall and echoes findings from the reader study: writers benefit from seeing different narrative strategies \cite{du1994recognition}. This scaffolding technique for writers’ decision-making also exposed them to narrative structures they might not have considered and enables them to combine different structures to align better with their goals. Building off literature in constraint-based creativity, we provide writers with 3 distinct options to create a constraint space that encouraged writers to merge elements across different strategies \cite{TROMP2022101184, collins2024building}. We found that presenting different options for narrative structures helped some writers shift perspectives on how to communicate science on social media.

% Our findings from the reader study illustrate that the effectiveness of science communication strategies on social media vary depending on the reader's own preferences and science background. 

% While readers were split on the effectiveness of the walkthrough strategy, we found in followup interviews that many readers preferred narratives without a walkthrough because they also included multiple different examples that frame the topic in different ways. Furthermore, the reader study showed that certain topics were more suitable for narratives without a walkthrough. 





% Previous science communication has found that science explanations with a single example, step-by-step walkthrough, with personal language is important in engaging a general audience, relating a STEM concept to their own lives, and helping them understand a complex topic \cite{doi:10.1177/2056305118797720}. 

% This suggests that the effectiveness of different narrative structures may depend on the nature of the topic—some topics benefit more from multiple examples, while others may be better suited for a structured walkthrough. This finding is further reinforced by insights from our writer study. When writers were given the opportunity to choose among different narrative structures, they were better able to recognize effective science writing techniques in the generated narrative structure option and identify which strategies worked best for their given topic. This finding aligned with the design principle of recognition over recall which eases the cognitive load of requiring writers to recall which science explanation strategy would be most effective. Explicit exposure to structured examples helps writers internalize effective communication strategies, rather than relying solely on their ability to recall and implement these techniques from memory. By providing a range of narrative structures, we can support writers in making more deliberate choices that align with their intended audience and topic complexity.

% \grace{need to rework this sentence}
% \grace{first sentence needs to summarize our novel findings, not surprising and long, interesting is recognition and recall stuff... condense the other stuff to get to this part faster}

% Beyond the field of science communication, these findings can help inform the development of Human-AI interfaces that support human evaluation and judgment on AI-generated outputs.  \grace{cite contrasting cases in learning theory}

% This suggests that 



% while we found that while readers do prefer these types of narratives, we also discovered that a relatable example is not necessary for helping a reader understand a topic. The use of a relatable example mainly improves the relatability of a science narrative, and not the understandability or engagement of a reader. Instead, if the science explanation used easy-to-understand language and engaged the reader through personal language, even in the absence of a relatable example or an example at all, readers still were able to understand a topic. In our followup interviews with readers, we found that the language of the science explanation often prompts readers to reflect on their own experiences and to find a relatable example in their own lives to apply to the science explanation. Instead, we found that personal language was shown to improve both engagement and understandability. Through the use of questions, vivid language, and a conversational tone, personal language not only made science accessible to readers, but also helped readers form their own connections with the science explanation. This demonstrates that personal language is helpful not only for engaging the reader, but also in helping readers understand a concept.

\subsection{Presenting Personal Language as a Continuum and not a Binary}
% \grace{previous XYZ showed that writers, and readers do prefer but not always, therefor it is a continuum and not a binary, help writers find their voice in the mix, bring it back and expand on it. people can explore their identties with side-by-side --> speaks to creativity and cognition, cognition can help them recognize and explore more creative things for how to reach an audeince while staying true to themselves, exlaborate on exactly what the contribution is in the discussion. }

Previous research has found that scientists often feel conflicted over
their personal and professional identities when engaging in science communication on social media \cite{koivumaki2020social}. Some scientists are hesitant to use colloqiual language, fearing it would undermine their authority or affect their reputation \cite{lorono2018responsibility, 10.1145/3479566}. Overall, we found that readers do prefer science explanations with personal language, demonstrating that this science communication strategy on social media is helpful in engaging a broad audience. But not all readers liked or needed personal language to help them understand. As such, personal language in science communication on social media should be presented as a continuum: the effectiveness of personal language varies depending on the reader. 

To help writers cater to diverse audience preferences and to accommodate writers' own hesitancy around adopting this communication strategy, we provide writers with contrasting examples of science explanations with and without personal language. Some writers' beliefs about how science communication should be was challenged when they viewed side-by-side explanations with and without personal language. By evaluating examples at both extremes, writers can more effectively discern which aspects of personal language align with their own communication style. Writers can easily evaluate which elements they wanted to keep, discard, or adapt to fit their own writing style. This process reduces cognitive load by offering writers with concrete reference points for how personal language can be used. Contrasting examples also allows writers to situate their own identities within a continuum, instead of trying to adhere to one method of communication. This contrast enables them to critically assess their own preferences, recognize effective techniques they may not have considered, and refine their approach based on the needs of their intended audience. These findings can help inform the design of AI-assisted writing tools to help writers explore a range of stylistic options, better understand audience preferences, and reflect on their own preferences.


% We found that seeing examples at the two extremes of personal language helped writers situate their own voice somewhere in between and help writers identify effective communication strategies that they might have previously overlooked. 

% As such, personal language in science communication on social media should be presented as a continuum. 


% our writer study finds that 

% By presenting writers with different options, we supported  In providing contrasting examples, writers could   %Our writer study shows that by presenting two extremes, \textit{With Personal Language} and \textit{Without Personal Language}, experts could more easily situate their own preferences between the two extremes, demonstrating the importance of contrasting options in accommodating different writers' preferences. 


% Previous research has focused solely on the presence of personal language on improving science communication on social media and has highlighted the hesitance of scientists in adopting these communication strategies \cite{doi:10.1177/2056305118797720, 10.1145/3479566}. 




% Our writer study demonstrates the importance of presenting writers with science narratives with different degrees of personal language to support them in navigating their own comfort and preferences of personal language in science communication. We found that The writer study also revealed that writer preferences are flexible and not constant. 


% In the reader and writer study, both groups of participants interacted with LLM-generated personal language. For many readers, the generated personal language did not hinder their understanding of the topic. For P7 in the reader study, the generated language felt very eloquent: ``there's something almost beautiful [about the language]." This directly contrasts the actions of writers who often opted to edit, replace, or delete LLM-generated personal language. The difference between reader and writer tolerances for LLM-generated personal language illustrates the role of authorship and the expectations tied to different roles in the communication process.

% Our study reveals the importance of presenting writers with options on how to style their science explanations on social media. By presenting explanations with and without personal language, writers had the freedom to explorethe degree to which they wanted to include personal language in their writing. This is Previous research into science communication on Twitter has established the importance of personal language to engage and educate the public, but many STEM experts are hesitant to write in this style, fearing that by making science personal, subjective, and informal will undermine their authority \cite{10.1145/3479566}. By seeing drafts of a science explanation with and without personal language, helped them find a space that they are comfortable with on the spectrum of Yes - Personal Language and No - Personal Language. This continuum structure accommodates for all writers, no matter their experience or comfort incorporating personal language into science explanations. Allowing writers to communicate within their own comfort level is important in broadening the types of science communication that is shared. Our reader study also showed that readers have diverse preferences for science explanations, so it is important not just for writers, but also for readers to have the opportunity to explore different ways of science communication. 

% \grace{what is the nuance that we added, the importance is that there is a narrative, can be hard to find 

% easy to understand:
% - single example is helpful, but it is more about relateable then helping the udnerstanding

% easy to relatable:
% - go back to those, what did we learn about relateable and understadnable 

% AND personal language affects both
% - you can write tweetorials with no example + people will pay attention anyway 

% Confirmed that the relatedableness was important for readers, but ranged from helpfulness, even with the 

% personal languae was helpful

% single walkthrogh or not is elss important }

% \subsection{Reader and Writer Tolerances for LLM-Generated Personal Language}
% \grace{merge this with the 8.2}
% In the reader and writer study, both groups of participants interacted with LLM-generated personal language. For many readers, the generated personal language did not hinder their understanding of the topic. For P7 in the reader study, the generated language felt very eloquent: ``there's something almost beautiful [about the language]." This directly contrasts the actions of writers who often opted to edit, replace, or delete LLM-generated personal language. The difference between reader and writer tolerances for LLM-generated personal language illustrates the role of authorship and the expectations tied to different roles in the communication process. Readers engage with explanations primarily for understanding and engagement, and as long as the content remains clear and compelling, they may appreciate or even admire the fluency and eloquence of the language, regardless of its source. In contrast, writers, particularly those with a strong sense of ownership over their work, may feel compelled to refine or alter the generated text to align with their personal style, voice, or intentions. This discrepancy highlights how authorship influences perceptions of AI-generated language—while readers can passively appreciate it, writers experience a stronger impulse to actively shape it. Different tolerances for LLM-generated personal language can help inform future work into the design of interactive writing assistants and the preferences that writers have for certain style attributes. While our writer study showed, that many writers preferred working off narratives with LLM-generated examples there were writers who also deleted and rewrote almost all of the generated language. To account for these preferences, future work can explore personalized style preferences for writing assistants to account for variations in the types of tasks where personal langauge might play a role.

% While many writers found the LLM-generated Personal language to be unnatural and often opted to edit or delete LLM-generated phrases, we found that readers were much more tolerant of LLM-generated personal language. All science explanations that the readers saw were generated by GPT and were not edited, which demonstrates that readers still found the science narratives engaging and understandable even when writers would have personally removed or edited most instances of LLM-generated personal language. 

% \grace{- readers didn't know that the voice was 
% - may bother writers, but doesn't bother readers, because they still found it good / likeable enough
% }


% \grace{ideal number of things is 3: what did we talk about in the intro + what are the theories that given this, 

% motivation: science communication is important, scientists don't know how to do this, 

% OPTIONS are on a continuum [finding from the reader study], + helping writers realize that these are on a continuum is important


% Title: Narrative 


% \subsection{Implications for Science Communication}
% \begin{itemize}
%     \item Survey results demonstrates the importance of diverse methods of science communication in engaging different audiences

%     \item Writer study demonstrates the importance of using "remove" to prompt the writer to engage in a reflective practice and consider or re-consider alternative approaches
    
%  \subsection{Writer Study: lack of diversity in GPT generated examples}
%  - given that we let GPT generate it's own example in the first narrative condition. in the instances where writers choose the "Step-by-Step: one Example" the generated narratives all contained the same example of sorting a deck of cards. This can be harmful especially when the reader study showed that it's important to have a diversity of examples to appeal to different audience members


 





