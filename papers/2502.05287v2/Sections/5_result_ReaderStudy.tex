\section{Results: Reader Study}

% \subsubsection{Readers prefer science explanations with an example over no example (H1:Example)}
% Readers demonstrate a statistically significant preference for reading science explanations with an example (EWP+FewShot) over one without an example (EWP-RemoveE). Readers' overall score consisted of 4 Likert scale questions the measure engagement, relatability, easy-to-follow, and   with 5 possible answer choices ranging from Strongly Agree (5 points) to Strongly Disagree (1 point). Th

% for science narratives with explanations was to science narratives without,  


\subsection{Quantitative Findings on Survey Data}
% \grace{to answer H1 example, we comapred the overall rating with EWP : WP, be more explicit about RQs and }
Overall, we collected 105 ratings on 75 different science explanations. This consisted of 35 different annotators who evaluated each science explanation on 4 different Likert-scale questions. We conduct a quantitative analysis of the survey data followed by a qualitative analysis of individual reader's preferences for science communication to contextualize nuances to the quantitative data.

% \grace{update labels}
% \grace{fill in the new stats from what juna runs}
% \grace{stars next to numbers that are statistically significant}

% \begin{table}[H]
%     \centering
%     \includegraphics[width=0.9\linewidth]{Figures/reader_survey_results.png}
%     \caption{Reader preferences for different methods of science explanations and the average scores for each rating dimension (engaging, relateable, understandable, and easy-to-follow) and the average total score for each condition.}
%     \label{fig:reader_survey_results}
% \end{table}


\subsubsection{Usage of Example Preferences}
% To answer our Hypothesis 1, Readers \textbf{do} prefer explanations with examples, EWP (Everything) compared to those without an example, EWP-RemoveE.

% Results from our survey demonstrate that readers preferred reading explanations \textbf{with} a single example, EWP (Everything), with an average score of 15.8 out of 20 points. Readers rated science explanations with the \textbf{removed} example, EWP-RemoveE, an average score of 13.0, almost a 3-point difference "close to" or "about" a 3-point difference. This result is statistically significant with a p-value of 0.000003.  (See Table \ref{table:reader-results-example}), 

To evaluate H1:Example, we compare experimental condition EWP to baseline condition WP. We found that participants prefer reading explanations with an example (EWP) over without an example (WP).
The results presented in Table \ref{tab:glmm_results_H1} illustrate the effects of the two conditions on the total score of the four survey questions (max 20 points) within a Generalized Linear Mixed Model (GLMM) framework. The experimental condition, EWP, received a score of 15.831 which shows a statistically significant increase in user preference when compared to the baseline condition of 13.003 (p < 0.0001). The effect size is 2.828 which shows that readers rated narratives with an example (EWP) almost 3 points higher than narratives without an example (WP). Additionally, the random effects analysis reveals a large group variance of 2.626 suggesting the differences among individual participants play a significant role and a small fields variance at 0.062, the specific field of study has a minimal impact. 
% The intercept score of 13.003 out of 20 signifies a strong baseline response level. 
%indicating that participants typically demonstrate a substantial consistency in their preference to the baseline condition -- science explanation without using an example.
% The experimental condition EWP received a score of 15.831 out of 20.It shows a coefficient of 2.828 (p < 0.0001) suggesting that participants prefer having an example (EWP condition) significantly better than without an example (baseline condition WP). 
%gave significantly higher scores to this condition compared to the baseline condition. 
%This positive coefficient indicates that the EWP condition provides a substantial improvement, increasing participants' preference effectively.
%Additionally, the random effects analysis reveals a group variance of 2.626, indicating considerable variability among individual participants. The fields variance is minimal (0.062), suggesting that while individual differences play a significant role, the specific field of study has a lesser impact. Overall, these findings highlight the positive influence of the EWP condition on participant preference, reinforcing its effectiveness in improving science explanation.



% \yy{no p-value for intercept. Add total of intercept + condition cofficient ===> handholding readers}

\begin{table}[h]
    \centering
    \caption{GLMM Results for H1:Examples}
    \begin{tabular}{@{}lccccc@{}}
        \toprule
        \textbf{Effect}                  & \textbf{Score (max. 20)}& \textbf{Coefficient} & \textbf{Standard Error} & \textbf{z-value} & \textbf{p-value} \\ \midrule
        Intercept[WP]                        & 13.003 & 13.003               & 0.483                   & 26.937            &           \\ \midrule
        CONDITION[EWP]       & 15.831 & 2.828                & 0.546                   & 5.183             & 0.000          \\ \midrule
        \textbf{Random Effects}   &        &                       &                          &                   &                   \\ 
        Group Variance    &        & 2.626                & 0.752                   &                   &                   \\ 
        Fields Variance      &            & 0.062                & 0.543                   &                   &                   \\ \bottomrule
    \end{tabular}
    \label{tab:glmm_results_H1}
\end{table}


\subsubsection{Step-by-Step Walkthrough Preferences}
% To answer Hypothesis 2 (H2), we cannot conclude that readers prefer the step-by-step explanation structure, EWP (Everything) - NoFewShot, over the multiple unrelated examples structure, Ep - NoFewShot.

% According to Table \ref{fig:reader_survey_results}, readers preferred science narratives with a step-by-step walk through with an average score of 15.7 out of 20. Readers rated science narratives \textbf{without} a step-by-step walkthrough an average score of 14.8. The less than 1-point difference is not statically significant (p>0.05). 

To evaluate H2:Walkthrough, we compare experimental condition EWP-NoFewShot to baseline condition EP-NoFewShot. We found that readers had a slight preference for explanations with a walkthrough compared to without a walkthrough, with a considerable variance on the fields of study.
Table \ref{tab:glmm_results_H2} summarizes the findings from a GLMM analysis. The experimental condition, EWP-NoFewShot received a score of 15.965 which was not statistically significant compared to 14.729, the baseline score for EP-NoFewShot (p = 0.057). As such, there is no significant difference in how readers rated explanations with and without walkthroughs. The random effects analysis shows a minimal group variance of 0.002, indicating little variability among participants. However, the field variance is considerable at 2.566, suggesting that participants prefer having a walkthrough for some STEM topics but prefer having no walkthrough for other STEM topics. 
%with the intercept corresponding to the baseline condition EP-NoFewShot. 
% The intercept score of 14.729 out of 20 indicates a strong baseline response level. %indicating that participants overall demonstrate a high consistency in their level of preference to the baseline condition -- science explanation without a step-by-step walkthrough.
% The experimental condition EWP-NoFewShot received a score of 15.965 out of 20. Its coefficient is 0.966 (p = 0.057), which suggests that while participants prefer having a walkthrough (EWP-NoFewShot condition) better than without a walkthrough (EP-NoFewShot condition), the statistical significance is only marginal. 
%while the total response is higher in the EWP-NoFewShot condition compared to the EP-NoFewShot condition, this difference is only marginally significant.
%This indicates a potential positive impact of the EWP-NoFewShot condition on participant responses, though further investigation may be needed.
%though follow-up qualitative interviews may reveal the nuances in participants' 
%may be necessary to confirm its statistical significance.
%differences in field of study contribute significantly to variations in total responses. 
%Overall, these findings indicate that while the EWP-NoFewShot condition may enhance responses relative to the EP-NoFewShot baseline, the effect is not conclusively significant, warranting further exploration. 
Our qualitative findings (Section 5.2) provide further explanations on why some participants are divided in their preferences for walkthrough.  


\begin{table}[h]
    \centering
    \caption{GLMM Results for H2:Walkthrough}
    \begin{tabular}{@{}lccccc@{}}
        \toprule
        \textbf{Effect}                 & \textbf{Score (max. 20)} & \textbf{Coefficient} & \textbf{Standard Error} & \textbf{z-value} & \textbf{p-value} \\ \midrule
        Intercept[EP-NoFewShot]               & 14.729         & 14.729               & 0.439                   & 33.537            &        \\ \midrule
        CONDITION[EWP-NoFewShot]    & 15.695    & 0.966                & 0.508                   & 1.902             & 0.057          \\ \midrule
        \textbf{Random Effects}   &       &                       &                          &                   &                   \\ 
        Group Variance     &       & 0.002                & 0.618                   &                   &                   \\ 
        Fields Variance        &          & 2.566                & 0.506                   &                   &                   \\ \bottomrule
    \end{tabular}
    \label{tab:glmm_results_H2}
\end{table}





\subsubsection{Personal Language Preferences}
% To answer our Hypothesis 3 (H3), readers do prefer science explanations that have use personal language, EWP(Everything) over explanations that do not use personal language, EWP-RemoveP.

% According to Table \ref{fig:reader_survey_results}, readers prefer reading science narrative \textbf{with} personal language, with an average score of 15.8 out of 20, as opposed to narratives with personal language removed which scored an average of 13.9 out of 20 points. The difference is statistically significant with a p-value of 0.0007. 

% The EWP (Everything) condition outperformed the EWP – RemoveP condition by 0.5 points across all dimensions.

To evaluate H3:Personal Language, we compare experimental condition EWP to baseline condition EW. We found that participants prefer reading explanations with personal language over without personal language. 
Table \ref{tab:glmm_results_H3} summarizes the findings from a GLMM analysis. The experimental condition, EWP, received a statistically significantly higher score of 15.883 compared to 13.973 for the baseline condition, EW (p < 0.0001). The effect size is 1.910 which means that readers rate explanations with personal language almost 2 points higher than explanations without personal language. Additionally, there is a group variance of 1.392 indicating a moderate level of variability among participants and a fields variance of 1.361 showing a moderate variance among STEM topics. 

% The intercept score of 13.973 out of 20 indicates a significant baseline level for participants in the EW condition.
% The experimental condition EWP received a score of 15.883 out of 20. The coefficient is 1.910 (p < 0.0001), suggesting that participants prefer having personal language (EWP condition) better than without personal language (EW condition) reaching a statistical significance. 
%participants in the experimental condition gave significantly higher scores compared to those in the baseline condition. This positive coefficient indicates that participants exhibit substantially higher preference to the EWP condition. 

%, indicating a moderate level of variability among participants, while the field variance is also notable at 1.361. This suggests that both individual differences and variations in fields of study play significant roles in influencing total scores. 
%Overall, these findings highlight the positive impact of the EWP condition relative to the EW baseline, reinforcing our hypothesis.

\begin{table}[h]
    \centering
    \caption{GLMM Results for H3:Personal Language}
    \begin{tabular}{@{}lccccc@{}}
        \toprule
        \textbf{Effect}        & \textbf{Score(max. 20)}          & \textbf{Coefficient} & \textbf{Standard Error} & \textbf{z-value} & \textbf{p-value} \\ \midrule
        Intercept[EW]      & 13.973                  & 13.973               & 0.463                   & 30.175            &           \\ \midrule
        CONDITION[EWP]   &15.883     & 1.910                & 0.526                   & 3.632             & 0.000          \\ \midrule
        \textbf{Random Effects}   &       &                       &                          &                   &                   \\ 
        Group Variance     &       & 1.392                & 0.717                   &                   &                   \\ 
        Fields Variance       &           & 1.361                & 0.548                   &                   &                   \\ \bottomrule
    \end{tabular}
    \label{tab:glmm_results_H3}
\end{table}


% \subsubsection{Overall Reader Preference}

% \yy{add preamble and hypothesis == EWP is the best of all.}
% In addition to the three hypotheses, we want to understand how the three techniques compare to each other and whether or not using all three techniques produces a science explanation that is the most preferable to the audience (H4: All three techniques (EWP)). Table \ref{tab:glmm_results_all} shows the findings from a GLMM analyzing total scores across four conditions: WP, EP-NoFewShot and EW with the intercept representing EWP. 
%We hypothesize that EWP would have the highest score of all conditions. 
% The intercept score of 15.874 out of 20 indicates participants prefer EWP having all three techniques over other conditions. 
% Among the conditions, WP received 13.260 with a significant negative effect coefficient of -2.614 (p < 0.0001) and EW received 13.799 with a significant negative effect with a coefficient of -2.075 (p < 0.0001), indicating that participants consistently prefer having all three techniques (EWP) over science explanations without an example or without personal language. 
%suggesting that participants consistently did not prefer science explanations without an example compare to EWP(Everything). Similarly, EW also shows a significant negative effect with a coefficient of -2.075 (p < 0.001), indicating a notable decrease in preference compared to EWP. 
% In contrast, the no walkthrough condition EP-NoFewShot shows a slightly higher score at 14.784 with a coefficient of -1.090 (p = 0.075), suggesting that participants potentially did not prefer having no walkthrough compared to having all three techniques, but further investigation is needed. Our qualitative results in the next section provide some further explanations on this. 
%The condition EP-NoFewShot has a coefficient of -1.090, which approaches significance (p = 0.075), implying a potential decrease in preference without a walkthrough in the explanation that may warrant further investigation. \yy{clarify the writing}
% Conversely, the EWP-NoFewShot condition exhibits a coefficient of -0.043 (p = 0.942), indicating no significant difference in responses compared to EWP(Everything).

% \yy{do we need random effects here?}
% \revision
% {The random effects analysis indicates a group variance of 1.477, highlighting the variability in responses attributed to individual differences. The covariance terms related to field variations indicate how different fields of study may interact with the group, suggesting complex relationships between random effects. The variances associated with different fields (e.g., Computer Science, Physics, Psychology, Statistics) further emphasize the importance of field context in influencing total responses. Overall, these results underscore the negative impact of certain conditions on participant performance while highlighting the potential complexity introduced by random effects associated with PID and FIELD.}
% \begin{table}[h]
%     \centering
%     \caption{GLMM Results for all conditions}
%     \begin{tabular}{@{}lccccc@{}}
%         \toprule
        
%         \textbf{Effect} & \textbf{Score(max. 20)} & \textbf{Coefficient} & \textbf{Standard Error} & \textbf{z-value} & \textbf{p-value} \\ \midrule
%         Intercept[EWP] & 15.874 & 15.874 & 0.515 & 30.829 &  \\ \midrule
%         CONDITION[WP] &13.260 & -2.614 & 0.589 & -4.436 & 0.000 \\ 
%         % CONDITION[EWP(Everything)-NoFewShot] & -0.043 & 0.595 & -0.072 & 0.942 \\ 
%         CONDITION[EP-NoFewShot] &14.784 & -1.090 & 0.612 & -1.780 & 0.075 \\ 
%         CONDITION[EW] &13.799& -2.075 & 0.568 & -3.655 & 0.000 \\ \midrule
%         % \textbf{Random Effects} & & & & \\ 
%         % Group Variance & 1.477 & 0.587 & & \\ 
%         % FIELD[Computer Science] Variance & 3.473 & 0.892 & & \\ 
%         % FIELD[Physics] Variance & 1.609 & 2.368 & & \\ 
%         % FIELD[Psychology] Variance & 3.633 & 0.943 & & \\ 
%         % FIELD[Statistics] Variance & 2.562 & 0.995 & & \\ \bottomrule
%     \end{tabular}
%     \label{tab:glmm_results_all}
% \end{table}




\subsection{Qualitative Findings on Reader's Preferences}

% The findings from the GLMM demonstrate (1) strong audience preferences for examples in science communication with considerable variability among individuals, (2) marginal preferences for a walkthrough in science communication with little variation among individuals, and (3) substantial audience preferences for personal language in science communication with a moderate variation in individual preferences. 
% \grace{need to edit this to be reflective of the findings. people preferred science explanations with examples and personal langauge, but were split on walkthroughs}
Overall, participants preferred science explanations with an example over no example and with personal language over no personal language. However, participants were split on the their preference for narratives with and without walkthroughs. We use semi-structured interviews to gain further insights into the reasons why participants might prefer narratives with or without an example, walkthrough, and personal language.

% People preferred science explanations with an example, a step-by-step walkthrough, and personal language, but there are nuances to each reader's ratings. To explore the nuances of reader preferences, we conducted 

% We identified three themes in reader's preferences as it relates to the usage of examples, structure, and personal language when explaining science to the public:
% \grace{one-two word: sentence, Relatable Example: reader's personal experiences affect score}

% \begin{itemize}
%     \item Theme 1: Relatable example: Reader's relatedness to an example affects their engagement with the topic.
%     \item Theme 2: Walkthrough structure: Readers sometimes prefer broad overviews of a topic.
%     \item Theme 3: Personal Language: Readers sometimes prefer formal and academic explanations of a topic over personal language.
% \end{itemize}


\subsubsection{Most Readers Preferred Examples}
Overall, readers reported that having an example was helpful. 4 of 8 readers stated the example helped them understand the importance of a topic (P2, P4, P7, P8). 5 of 8 found that examples helped them reflect and understand their own experiences better (P1, P2, P3, P4, and P7). When reading a science explanation about the computer science topic of depth-first search, P2 remarked that learning how the algorithm can be applied to navigating a maze helped her engage with the topic: ``I don't really care about computer science algorithms, but I do care about how this applies to my own life."

However, not all readers needed an example to help them understand the topic. 3 of 8 participants mentioned how the example felt unnecessary and detracted from the content of the explanation (P1, P2, P8). When reading about depth-first search, P1 mentioned how the given example of navigating through the Botanical Garden felt unnecessary: ``The concept in and of itself is interesting. I would have just read about that. I don't really need any more context." For P8, when learning about thin film interface through the example of a child playing with bubbles, he remarked that ``[the example] doesn't really pertain to me in any way," demonstrating how certain examples might not resonate with particular audiences.

3 of the 8 participants mentioned how they would reference their own experiences to help ground the technical explanation (P1, P2, P4). When P2 was reading about thin film interference without an example, she used her own experience in working with glass to ground her understanding of the topic: ``[the topic] was really relevant to my life and what I'm already doing." For P4, when the given example of watching a horror film was used to explain Walker's Action Decrement Theory, she mentioned how even though the example was unrelatable, she referenced her own experiences to find an example that fit the same context. This shows that readers use their own experiences to contextualize the science for explanations that do not include an example or that include an example unrelatable to the reader.
 
 % even when an example is included in the narrative, if the example is unrelateable  when a narrative used an example that was unrelatable to the reader, This demonstrates shows how some people liked narratives without examples because it allows them to fill in their own experiences. 

 % These findings illustrate how examples can help ground the science in something tangible to help readers to understand a topic. But without an example, some readers would reference their own experiences to understand the topic demonstrating how examples are not necessary for all readers. Some readers also felt that an example was unnecessary for the complexity of a topic and convoluted the explanation showing how for some topics an example might not be neeeded. 

\subsubsection{Readers had No Preference for Narratives with Walkthroughs}
There was no statistical significance in our results comparing reader preferences for narratives with and without walkthroughs. In this section we will investigate why some readers prefer walkthroughs or no walkthroughs. 
 
Some readers reported that the walkthrough helped establish a structure for the explanation (P2, P3, P8), provide evenly paced information (P1, P3, P6), and helped them follow through with the science (P4, P6, P7). P3 appreciated the walkthrough for the topic of Walker's Action Decrement Theory in Psychology because it helped them understand how their heightened emotional state when watching a horror movie might affect their memory. By walking through 3 different stages of the narrator's emotional journey during this event and how it affects their memory, P3 mentioned how the narrative helped ``set the stage" and ``allows you to follow through with [the science explanation]." The walkthrough provided a scaffold to support the reader in processing the information as a sequence of events. 

But not all readers preferred narratives with a walkthrough. 3 participants (P2, P3, P7) reported that the walkthrough structure made the explanation feel overly explanatory or repetitive. For the topic of Walker's Action Decrement Theory, P2 said that the walkthrough narrative was ``just overly explanatory for a concept that is very intuitive," demonstrating how for certain topics a walkthrough might not be necessary.

4 participants (P2, P4, P5, P6) preferred science explanations without a walkthrough but with many examples because they provided multiple different angles to view a topic in a condensed space. P4 stated that the explanation without a walkthrough broke down the topic of the curtain wall system into smaller, self-contained chunks of information that made reading the explanation ``less intimidating." For example, one paragraph of the explanation focused specifically on the aspect of temperature control, while the subsequent paragraph explored the structural construction of the curtain wall. P4 said that having the information broken down in this way made it more ``digestible'' to learn about the topic. For P6, the ``list of facts'' structure of explanation without a walkthrough helped them gain a broad overview of the topic. This illustrates that for some topics, readers might not want an in-depth walkthrough of the science and might prefer getting a high-level overview of how the science works. 

 % Walkthroughs can be helpful for providing scaffolding for readers to read about the science, but is not always preferred by readers. Some topics might not require in-depth walkthroughs and some readers might prefer just getting a broad understanding of the topic. 

 % This demonstrates that for certain topics, readers might not want an indepth walkthrough of the science. Furthermore, science explanations without a walkthrough also helped certain readers learn about different aspects of the topic, making the reading process more engaging and less intimidating. 


 \subsubsection{Personal Language Sometimes Distracts from the Science}
 Overall, readers preferred reading explanations with personal language. Participants reported that the personal language helped establish a writer/reader connection (P1, P3, P5). But not all readers preferred narratives with personal language. 2 readers preferred reading science explanations with no personal language because they believed personal language was unnecessary (P1, P8): ``[the explanation is] full of fluff coming from a personal perspective that I didn’t really care about" (P1). P8 mentioned that he did not need personal language for engagement when reading about science, additionally, he hypothesized that other readers ``[might] need the personal aspects to get them to read something that they wouldn't otherwise be interested in." Depending on a reader's inclination towards science, personal language may or may not be necessary to help them engage with these science narratives. 
 %how when he is reading about science, he doesn't need the personal language to get him engaged and 

 
 % Without personal language, readers often found that science explanations overly technical and hard to understand (P1, P6, P7). 

 % For two readers (P1, P7), the presence of technical jargon often triggered an emotional aversion to continue reading. P1 said that ``I read ``linear regression", and I wouldn't be interested in it." Similarly, P7 said ``when I saw those bigger words that I was like, ``Oh, that's scary to read,"" demonstrating how technical jargon can trigger strong aversions in readers.  Furthermore, when technical jargon is paired with scientific concepts, P7 mentioned how the overall explanation became more intimidating to read. \grace{is this the place to situate these findings in other works?}

 

% This demonstrates that the presence of personal language in science explanations depends heavily on the reader's comfort engaging with STEM topics. Without personal language, explanations can trigger strong emotional aversions to the science for some readers, while for other readers, they would prefer only reading about the science.
 
 
 % For example, one section might focus specifically on the aspect of temperature control, while the subsequent paragraph explores the structural construction of the curtain wall. 

 
% \grace{time to shape a narrative, what is surprising and interesting things here: 

% examples: takeaway (one example where it is good, but anchor them in a positive case), what are the interesting themes around what happened generally examples are good, but there are exceptions 

% examples are bad when examples are too childish... [lydia needs to be convinced of this] (how many people said each thing)

% someitmes people liked the narratives without the examples because they could fill in their own (how many people said each thing)

% what are things that multiple people said: 
% }

% For P1, P2, P3, P4, and P7, examples helped them better understand their own experiences demonstrating how examples can help make science more relatable: ``This is a specific scenario that I can connect to and  apply this to my life, and imagine and make sense of" (P7). When examples were not present, P4 and P6 mentioned how they were unsure how the topic applies to real life: ``I wasn't really sure when you would like see this in real life" (P4). Furthermore, P4 and P5 explicitly mentioned how they wanted to have examples when none were present in the EWP-E narrative: ``I wish I had more examples instead of just 'the vibrance around light'" (P4). 

% P3 mentioned how the use of an example helped establish her understanding of the topic: the example "set the stage [for the explanation], [the example] allows you to follow through with something". P4 mentioned "I like how it has the example, like of public transportation. So it's something that I can be like, "Okay, this applies to me."" Additionally, in the EWP-RemoveE conditions, where an example is not included in the thread, P4 mentioned how "it didn't give any examples" and how they "wished" there were examples to help explain the topic. These quotes illustrate how the use of examples helps improve a reader's engagement and understanding of a topic. 

% However, the GLMM results show that there is considerable variability in individual preferences for science narratives with and without examples. To better understand the different preferences that readers have for science narratives with and without examples, we identified 2 different dimensions to these preferences:
% \begin{itemize}
%     % \item Dimension 1: Unrelateable examples don't always negatively affect readers' understanding.
%     \item Dimension 1: Readers often reference their own experiences to create examples when no explicit examples are included.
%     \item Dimension 2: Examples can hurt readers' engagement and understanding of a topic.
% \end{itemize}

% But the experiences of readers with the specific examples that are used vary in two different dimension (1) relatability of a given example to the reader and (2) the use of a given example. For example, some readers can find the example unrelatable, but still find the science explanation understandable. 

% \grace{for readers who didn't like the example, the narrative was not relateable, but still understandable. whether example is relateable, whether narrative is understandble, and whether}

% \textbf{Dimension 1: Unrelateable examples don't always negatively affect understanding}

% For P8 that did not relate to an example that was used in the EWP (Everything Condition), he still found the science explanation understandable. The reader stated that he could not relate to the given example of the narrator playing with bubbles with his child because “I do not have a kid and I do not play with bubbles.” But even though the example was unrelatable, he found that the structure of the explanation was “presented in a logical way” that helped him understand the topic.


% \textbf{Dimension 1: Readers often reference their own experiences to create examples when no examples are included.}

%  For P2 reading an EWP-RemoveE science explanation, she said that she would slightly disagree that the thread was engaging and relatable because “… because [there] weren’t any examples. I wasn’t sure when [I] would see this [phenomena].” But the lack of an engaging and relatable example did not detract from P2’s understanding of the technical components because “there was a good mix of specific concepts that was not overly technical.” Besides the engagement and relatability score which the reader rated a 2 out of 5, the reader rated the understandablility and easy-to-follow structure a 4 out of 5. 

% \grace{"a relatability score of 2 out of 5"}

% The lack of an example does not always detract from a reader’s understanding of the science topic. P4 mentioned how the use of thought-provoking questions like: ``Have you ever wondered why you sometimes forget significant events as they are happening, only to remember them more clearly later on?" (P4), allowed her to reflect on her personal experiences and form a connection with the science explanation, even when she could not relate to the given example. This demonstrates that the language used in the science narrative can prompt readers to reflect and identify an example from their own lives to use and relate to the concept.

% For P2, her own personal interest in a certain topic area made them more likely to keep reading a science explanation without an example. Even though there was no explicit example used to explain the concept of thin film interference in physics, the reader’s own experience of ``working with colored glass and other materials right now” allowed them to form more specific connections with the topic: “[the science explanation] was really relevant to my life and what I'm already doing.” This demonstrates how readers implicitly draw upon their own personal experiences to connect with the text. Even when no explicit example was provided, this reader used her own personal interest in the topic and current work as an implicit example to supplement the explanation. 

% \textbf{Dimension 2: Examples can hurt engagement and understanding of a topic.} 

% For P1, P2, and P8 the use of examples can detract from a science explanation. One science explanation for the topic of the central limit theorem in statistics, used an example of a person trying to work with skewed transit time data. For P2, she remarked that ``I have no sympathy for the narrator. It seems like they've never taken public transit before in their lives” (P2). This demonstrates how an unrelatable example can deter the reader from reading on. Additionally, P2 also remarked that the example felt unnecessary to explain the topic, ``This is like a waste of time. It was overly explanatory for the concept,” demonstrating how some topics might not need an example to help explain it. 

% P1 said how she didn't find the given science explanation on linear regression ``engaging at all. It seemed to me, [the explanation] was trying to apply, a statistical application in the real world... I think the point was that you can use data to help you make decisions, but people don't really make decisions that way. I just didn't find it relatable or predictable" (P1). This demonstrates that the use of an example in for this reader felt contrived and did not help them understand or engage with the science narrative.

% For P8, the use of the example is unnecessary and doesn't add to his understanding of the topic of thin film interference stating: "the first two paragraphs of the [science narrative] is kind of nice, but it doesn't really do anything for me...It doesn't really pertain to me in any way. I don't play bubbles and I don't have kids" (P8). This demonstrates that an unrelateable example provides unnecessary information that isn't needed to help the reader understand the science.

% Overall, most readers like science narratives that contain an example to help them understand a topic. But the importance of examples in helping improve a reader's understanding of the topic wasn't universal. In our followup interviews, we found that (1) that some people were able to draw upon their own experiences to create a relatable example, and (2) some people were deterred when a science narrative used example they did not relate to. 

% \subsubsection{Theme 2: Walkthrough structure: Readers sometimes prefer broad overviews of a topic.}
% \grace{there was no statistical signfiicance, why wasn't the walkthrough was always preferred, there were some topics that they didn't need it. one example that they didn't. 

% remove the dimensions, each heading is the takeaway, the first paragraph is really short, and then one or two paragraphs that talk about the exception}

% The GLMM model showed a slight but insignificant reader preference for science narratives with a walkthrough. In our followup interviews we explored the different reasons why people preferred each of these structures. We identified 2 different dimensions that readers mentioned when choosing between these two narrative structures:
% \begin{itemize}
%     \item Dimension 1: Readers prefer step-by-step walkthroughs to understand a topic in-depth.  
%     \item Dimension 2: Readers prefer non-sequential walkthroughs to understand a broad overview of a topic.
% \end{itemize}

% \textbf{Dimension 1: Readers prefer step-by-step explanations to understand a topic in-depth.}
% P2, P3, and P8 mentioned how the clearly established structure of step-by-step walkthroughs was helpful: ``It laid out the steps that I would have to take" (P2), ``well organized" (P8), and ``You can follow a sequence" (P6). More specifically, for P4, P6, and P7, the story-like structure of step-by-step walkthroughs to help them follow the science: ``the narrative of telling a story, going from like broad to specific. It flowed nicely" (P4) and ``Starting at the beginning with the example: ``Say your watching a movie," then at [tweet] six it's the peak, ``it acts as a trigger." This feels like the climax of the story. Then it leads to a nice ending" (P6). Finally for P1, P3, and P6, step-by-step walkthroughs provided evenly-paced information spread to help readers understand the science: ``The information is spread throughout. There's little tidbits that help us think it over" (P1) and ``There's a good lead up to the explanations and how they make you feel" (P6).

% P3 mentioned how they liked the clear sequence of the explanation: "it was very straightforward and explanatory, and sort of laid out the steps that we would have to take if we were in a maze and had to execute this process." The step-by-step process helped this participant understand the exact steps that they would need to take in order to perform depth-first-search. 

% P5 found that the step-by-step story sequence was engaging: "I think the the narrative of telling a story, going from like broad to specific. It flowed nicely, which was engaging." The sequential ordering of information not only helped them understand the topic, but also kept the reader engaged throughout the reading process.

% When a step-by-step walkthrough was not present, P5 mentioned how the narrative seemed very disjointed: ``All of the examples were sort of like one offs. It wasn't very like coherent. It felt very definition-[like]." For this participant, the non-sequential narrative only allowed her to get a surface level understanding of the topic and not a true understanding of how the topic works: ``I feel like the descriptions, while good, it could be drawn out more so that it's not just glossed over." P6 also mentioned how a EP-NoFewShot explanation ``felt more like a list of facts rather than a story, [a story] can make it easy to engage with if you're fairly new to the subject." For P6, not only did they not like the non-sequential explanation, but also wished that it had been explained through a narrative-based structure, EWP-NoFewShot.

% \textbf{Dimension 2: Readers prefer non-sequential explanations to understand a broad overview of a topic.}

% Some readers preferred the breadth of information that the non-sequential explanations, EP-NoFewShot, provided. P5 mentioned how it helps ``when [the science explanation] offers more examples to help break down a concept." P4 mentioned that even though the non-sequential narrative was ``not a story, but [there is] some sort of comparison to the real world," that made her feel engaged. P4 also mentioned how the non-sequential explanation was ``broken up more so it makes it less intimidating to read through, and is more digestible," illustrating that readers have different preferences for how much information is being presented to them at a time. For some topics, a step-by-step walkthrough might feel ``overly explanatory for a concept that is very intuitive" (P2).

% These two dimensions illustrate the trade-offs between a step-by-step and a non-sequential explanation structure. While some readers prefer the step-by-step information in truly understanding how a topic works, other readers might find the amount of information that needs to be processed overwhelming. Similarly, some readers might only be interested in a high-level understanding on how a certain topic works and might not be interested in an in-depth explanation. These findings illustrate that different explanations structures support different ways for a reader to engage with a topic. A step-by-step explanation facilitates a deeper understanding, while a non-sequential explanation helps readers understand many different aspects or applications of a topic.

% \subsubsection{Theme 3: Personal Language: Readers sometimes prefer formal and academic explanations of a topic over personal language.}

% The GLMM model shows that overall people demonstrated a substantial higher preference for science explanations with personal language, EWP (Everything), with slight variation in individual preferences. Overall, personal languages helps establish a writer/reader connection through the use of questions to prompt the reader to reflect: ``When a text  asks questions the reader may be thinking is engaging. It's like you're talking to yourself" (P5) and ``This is more fun to read because [the text is] asking me questions that I'm asking myself" (P5). 
% because it helped both engagement and understanding for science topics. P1 said that "some of the little jokes actually really helped me understand the what the topic is" and that it is a "very common way of speaking" that helped her stay engaged in the thread. P7 mentioned that the descriptive language of a person's "fiery red hair," and other "vivid language" helped her stay engaged when reading. She also mentioned how certain lines from the science explanation were so descriptive that "it's almost like literature. It's really good in that way, and that makes it really engaging."

% When personal language was removed from science explanations, readers remarked that they disliked the overly technical language (P1, P6, P7) and the academic voice (P1, P3, P6). Undefined technical language made it hard to understand and follow the science explanation: ``It was hard to follow at times because there were some terms I didn't understand in this particular field of study: phase shift. What's the exact definition of that?" (P6), ````For a unit change in the independent variable," I'm like, ``Okay, I know these words individually, but together. they don't make sense to me." I think it actually took away from my understanding to have this long explanation of the formula." (P1), and "I mean ``architectural breathing systems." What does that mean?" (P6). For P7, the technical language and scientific jargon was intimidating to read, ``I had to read it a few times to get it. Those bigger words that I was like, ``Oh, that's scary to read." Paired with science concepts, I mean, it makes it kind of intimidating" (P7), demonstrating how technical language is not only less engaging to read but also serves as an emotion trigger for some readers. Many readers also remarked that without personal language, the science explanations felt ``pedantic," (P6) ``highfalutin," (P4) and too ``academic" (P3). P3 said: ``I think some of the wording was a bit too academic, and which can be a little bit distracting if you're focusing really hard on figuring out what they're saying," showing how formal and impersonal language not only hurts a reader's engagement but also understanding of a topic. 

% The GLMM model also shows a slight variation in individual preferences. We identified one main dimensions to when the removal of personal language was preferred.
% \begin{itemize}
%     \item Dimensions 1: The use of personal language can convolute the explanation. 
%     \item Dimension 2: Some readers prefer purely technical explanations.
% \end{itemize}

% \textbf{Dimensions 1: The use of personal language felt unnatural}
% Both P8 and P1 remarked how the inclusion of personal language sometimes felt unnecessary: ``I feel like I'm wasting my time, if I'm reading the personal aspect of things" (P8) and ``full of fluff coming from a personal perspective that I didn't really care about" (P2) demonstrating how personal language can potentially convolute the science for some readers. For P2 the highly personal language made the science explanation feel like it was meant for a ``six years old. But I'm not six years old." Her comment demonstrates how sometimes overly personal language can come off as infantilizing to the reader and deter them from reading more about the topic.

% \textbf{ Dimension 2: Some readers prefer purely technical explanations}
% P8 mentioned how he liked how ``straightforward and explanatory” the language was when reading an explanation with no personal language, EWP-RemoveP.  P8 also mentioned ``I prefer to just if I'm reading something that's educational, I prefer it to just be educational" demonstrating persoanl preferences for how educational materials are presented to him (P8). This shows how some readers don't need personal language to engage them to learn about science on social media. 

% Overall, participants found that the addition of personal language helped them understand and engage with science narratives, but some participants had found the use of personal language detracted from the science explanations (P1, P8) and others didn't need personal language to help them feel engaged or understand a topic (P8). 

% \grace{overall picture: EWP is a great starting point, but doesn't work for all topics, and AI can help you brainstorm these different options, as a writer it can help you explore these different options}