\section{Conclusion}
It is crucial for science communication to engage the general public, and prior research suggests that using colloquial techniques from social media can be effective. Despite this, many scientists are hesitant to apply these techniques due to concerns about losing their authoritative voice. Our research highlights the complexity of public science communication and the need to balance readers’ and writers’ perspectives. While readers generally preferred explanations that included examples, walkthroughs, and personal language, their preferences were nuanced and context-dependent, influenced by their personal experiences and the complexity of the topic. Conversely, writers often feared that these techniques might compromise the clarity or authority of their explanations. However, when given the opportunity to explore various narrative structures and styles, writers were able to navigate their choices with greater confidence, finding a balance between colloquial and formal approaches. This suggests that effective science communication benefits from exploring diverse options, allowing writers to tailor their style to the scientific topic, their own preferences, and the needs of their audience.


% Using a mixed-methods approach, we conducted a survey of reader's preferences for how science explanations are presented. We explored the impact of an example, a step-by-step walkthrough, and personal language on a reader's engagement and understanding of a STEM topic. We found that most readers preferred science explanations that contained an example, a step-by-step walkthrough, and personal language, but there are nuances to each reader's rating. Thus, we conducted followup interviews with the survey participants to better understand the details of each reader's preferences for 3 dimensions: relateable examples, a step-by-step walkthrough, and personal language. 

% Based on these findings, we ran a writer's study to explore whether seeing different methods of structuring and styling a science explanation can help writers mitigate their hesitancy around science communication on social media. We found that offering writers different options for how to structure and style their writing helped writer choose which characteristics of each they wanted to include. Instead of presenting one method for science communication on social media, by offering writers multiple options, writers felt more comfortable exploring these options.

% % \grace{IDK FILL IN}

% Overall, our study provides insights into both reader and writer preferences for science communication and highlight a design space for supporting writers navigate a continuum of writing structures and styles.  