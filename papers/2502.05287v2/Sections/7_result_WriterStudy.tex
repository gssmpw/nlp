\section{Results: Writer Study}
% In this section, we provide quantitative and qualitative analysis on the writers' experience when they are presented with different structure and style options for science narratives. 10 writers wrote two different science explanations, which resulted in 20 total edited science explanations. 

% \grace{writers have preferrence for interface for interface or tools, }

% To understand writer's preferences for different structures and styles of communication, we report findings on the following research questions:

% RQ1: How does seeing structure options (Everything, No Example, No Walkthrough) help writers choose a direction to write in?

% \grace{with and without personal language}
% RQ2: How does seeing style options (Yes Personal Language, No Personal Language) help writers choose a direction to write in?

% \begin{itemize}
%     \item Structure: Which narrative structures (Everything (Everything, No Example, and EP-NoFewShot) do writers prefer?
%     \item Structure: Does seeing different structure options help writers choose a direction to write in?
%     \item Style: Which narrative style (Yes - Personal Language and Remove - Personal Language) do writers prefer?
%     \item Style: Does seeing different style options help writers choose a direction to write in?
% \end{itemize}


% which explanation structure do writers prefer + does showing different narrative structures help writers decide what to write in 

% \grace{in title of 6.1: showing different narrative structures help people mix and match, takeaway should be clear. one sentence bolded and first. that's when you are done. first/last doesn't really matter.}

% \subsection{Writers Reactions: Different Structures for Science Explanations}
% We generated three different ways (EWP, WP, EP) to structure science narratives. After we asked writers to choose one of the narrative structures to learn more about their preferences for how science explanations are structured. Then we observed the writer to see how they will edited, adapted, and merged their chosen narrative structure into something they are satisfied with. 
% \grace{edit the information}

\subsection{\textbf{RQ1}: Different Narrative Structures help Writers Choose Appropriate Framing Techniques}

Of the three narrative choices (\textit{One Example}, \textit{Many Examples}, \textit{No Examples}), most science explanations used the \textit{One Example} narrative to iterate on (9 of 20). The second most chosen narrative strategy was merging components between the \textit{One Example} and \textit{Many Examples} options (5 of 20). The \textit{Many Examples} strategy was chosen 4 of 20 times. The least used structure was \textit{No Examples} (2 of 20). Choices made by participants and topic are found in Table \ref{tab:writer-structure-preferences}. Writers' choices differed depending on the topic, their personal preferences for science communication, and whether or not they had a predefined explanation structure they wanted to follow. The \textit{One Example} option often provided writers with a clear sequencing of information to follow. Some participants (P2, P4, P6, P10) chose to merge between the \textit{One Example} and \textit{Many Examples} and used examples covered in the \textit{Many Examples} option to supplement certain explanations in the \textit{One Example} narrative. Occasionally, the \textit{No Example} option providing a strong technical explanation for writers to build off of.

% \grace{explain why each one was the winnner, people made these different choices because of largely depended on the topic, personal preference, add interpretation, this is consistent with the reader study, an example is a good way to structure thigns and that mansy examples have benefits. no examples can occasionally be a good choice.} We investigated how writers made their choices and how seeing different structure options was helpful to them. 

% \grace{get to the point faster, overall people picked one example, }
% To answer RQ1, we evaluate how many writers choose each structure option (\textit{One Example}, \textit{No Example}, \textit{Many Examples}). 
% \grace{how did thye amke the chocies and how the different options are helpful to them}
% \grace{need a lead in for the different ways that people use the nrrative structures}


% \grace{restate RQ1 short way: does structure help writers... make naming consistent, Italicize and capitalize the naming for the examples -- make sure this is consistent}


% \grace{put the sentences together --> merge, instead writers selected no example 2 of 20 times. 

% no worst/best language}
% \grace{topic sentences are bad need to make it a bit punchy, what is the takeaway --> should tell what the paragraph is going to say in, what is interesting + takeaway-ish, now is more so summary based... give the readers something interesting in topic sentence}

% \grace{need to follup which narrative they choose and why}
% Writers' existing science communication preferences and beliefs influence their choice of narrative structure. 
Seeing different narrative structures helped 7 of 10 writers identify their own preferences for science communication by solidifying or challenging their existing beliefs (P2, P3, P5, P7, P8, P9, P10). P9 mentioned that seeing the narrative with \textit{Many Examples} helped him solidify his initial feeling that ``one example is a good way to engage people." But P7 mentioned how seeing multiple structure options helped him re-evaluate his initial approach to explaining the topic of word embeddings. P7 initially wanted to use the \textit{One Example} narrative structure, but instead chose to use the \textit{Many Examples} narrative because the structure provided for the topic of word embeddings: ``I was able to see what other things I might actually want to include when I'm trying to explain it, and what other possibilities there are for explaining." 

Seeing multiple narrative structures helped 4 of 10 writers merge different framings of a topic (P2, P4, P6, P10). For multiplicative weights update, P10 merged elements from \textit{Many Examples} into a base narrative with \textit{One Example} to emphasize the dimensions of online learning and regret minimization which are ``important facts about the topic, but wasn't brought up in the [\textit{One Example}] option." Because the \textit{Many Examples} option provided him with different angles to motivate or contextualize the topic, the selected paragraphs from \textit{Many Examples} could be ``slotted in at the end."  

There were 4 times where writers already had an established outline in mind for how to write the science explanation for one of their two topics (P1, P3 P8, P9). But for most writers and topics, seeing different options provided 8 of 10 writers an example of what to avoid when explaining science to an everyday audience (P1, P2, P3, P5, P6, P7, P8, P9). Seeing the \textit{No Example} option helped them consider their audience's technical ability and the background needed to explain the science. P6 said \textit{Many Examples} narrative had a``rigorous depiction of what the software is, [which can be hard] for a general audience to imagine what that means." 

% P10 believed that the topic of multiplicative weights update method was ``relatively receptive to real world examples, so [he] would definitely include one" (P10). P7 mentioned how he ``tend towards" science explanations that have a walked through examples and as a result choose that narrative to iterate on. 

% \grace{cut some of this, needs to get to the point faster, what do you want people to take away from this... }



% and mentioned how seeing different explanation structures helped him determine which subtopics to include. He chose to include a paragraph from the \textit{Many Example} option that provided a succinct description of online learning and regret minimization to highlight why MWUM is so important.

% : ``[the structure] provided little nuggets of information [which] is more useful than one big example that tries to cover everything." P7 mentioned how

% \grace{this is a better topic sentence, shorten the topic, for some topics seeing mutliple narrative structures helped them explore how to explain the topic. put cut information somewhere else in the paragraph, cut sentence 2 and add number to the topic sentence}
% \grace{feels like the same thought, bring out the merging here, sometimes people didn't choose one topic, but instead merged them}
 

% Some writers used the different narratives to compare the pros/cons for each narrative and perform merges between narrative options if there were elements from another narrative that they would want to incorporate. 




% \grace{seeing no example showed readers what not to do, see how no example was too abstract. counterpoint}


% \grace{can merge, don't need example

% 4 times people didn't need any of the options. but for people who didn't have an idea it was helpful.}
% But seeing the different narrative options didn't always help writers. For certain topics, 4 of 10 writers already had clear ideas for how to communicate the science that seeing multiple options was distracting . For the topic of 4D printing, P9 said that he didn't like any of the narrative structures and already had a clear vision for how to write the narrative himself. While for the topic of merge sort, P9 had no strong idea for how to start, so seeing many different options helps ``give [him] a starting point." 

% This illustrates how writers sometimes do not need a system to help them decide what narrative structure to use. 

% P4 and P2 choose to merge aspects of two different narrative strategies together. In doing so, he took one narrative as his primary narrative structure and then incorporated certain elements from a secondary narrative structure into the primary. This demonstrates how the characteristics that differentiate the 3 narrative structures (Everything, No Example, and EP-NoFewShot) exist on a spectrum and are not mutually exclusive. 

% We found that writers have diverse preferences for how to structure their science explanations. While half of the science narratives used the Everything structure, writers also choose to use other narrative structures, demonstrating that there is not one standard method that writers like to structure their science explanations.

% \grace{make primary, secondary, and then change the color for primary and secondary}
\begin{table}[H]
\resizebox{\textwidth}{!}{%
\begin{tabular}{|c|c|c|c|c|c|c|}
\hline
{\textbf{Participant ID}} & {\textbf{Domain}} & {\textbf{Topic}} & {\textbf{One Example}} & {\textbf{No Example}} & {\textbf{Many Examples}} & \textbf{Merged}\\ \hline\hline

\multirow{2}{*}{P1} & {Computer Science} & {Merge Sort} &  &  & {\checkmark} & \\ \cline{2-7}
& { Statistics} & { Gradient Linear Additive Models} & {\checkmark} &  &  & \\ \hline

\multirow{2}{*}{P2} & {Computer Science} & {Merge Sort} & & & & {\begin{tabular}[c]{@{}c@{}}Primary: One Example\\ (Secondary: Many Examples)\end{tabular}}                          \\ \cline{2-7}
& {Computer Science} & {Gradient Descent} & {\checkmark} & & & \\ \hline
 
\multirow{2}{*}{P3} & {Computer Science} & {Merge Sort} & & {\checkmark} & &  \\ \cline{2-7}
& {Qualitative Analysis} & {Ordinal Regression} & & {\checkmark} & & \\ \hline
 
\multirow{2}{*}{\vspace{-0.21in} P4} & {Computer Science} & {Merge Sort} & & & & {\begin{tabular}[c]{@{}c@{}}Primary: One Example\\ (Secondary: Many Examples)\end{tabular}}  \\ \cline{2-7}
& {Natural Language Processing} & {Controlled Decoding} & & & & {\begin{tabular}[c]{@{}c@{}}Primary: Many Examples\\ (Secondary: One Example)\end{tabular}} \\ \hline

\multirow{2}{*}{P5} & {Computer Science} & {Merge Sort} & {\checkmark} & & & \\ \cline{2-7}
& {Natural Language Processing} & {Word Embeddings} & & & {\checkmark} & \\ \hline
 
\multirow{2}{*}{\vspace{-0.2in}P6} & {Computer Science} & {Merge Sort} & {\checkmark} & & & \\ \cline{2-7}
& {Programming Languages} & {Formal Verification} & & & & \makecell{Primary: One Example \\ (Secondary: Many Examples)} \\ \hline

\multirow{2}{*}{P7} & {Computer Science} & {Merge Sort} & {\checkmark} & & &  \\ \cline{2-7}
& {Natural Language Processing} & {Embedding Space} & & & {\checkmark} & \\ \hline
 
\multirow{2}{*}{P8} & {Computer Science} & {Merge Sort} & {\checkmark} & & & \\ \cline{2-7}
& {Quantum Algorithms} & {Grover's Algorithm} & & & {\checkmark} & \\ \hline

\multirow{2}{*}{P9} & {Computer Science} & {Merge Sort} & {\checkmark} & & & \\ \cline{2-7}
& {Tangible User Interfaces} & {4D Printing} & {\checkmark} & & & \\ \hline
 
\multirow{2}{*}{\vspace{-0.18in} P10} & {Computer Science} & {Merge Sort} & {\checkmark} & & & \\ \cline{2-7}
& {Optimization Algorithms}     & {Multiplicative Weights Update} & & & & {\begin{tabular}[c]{@{}c@{}}Primary: One Example\\ (Secondary: Many Examples)\end{tabular}}  \\ \hline\hline

& & \textbf{Total Count:} & \textbf{\begin{tabular}[c]{@{}c@{}}9\\ One Example\end{tabular}} & \textbf{\begin{tabular}[c]{@{}c@{}}2 \\ No Example\end{tabular}} & \textbf{\begin{tabular}[c]{@{}c@{}}4\\ Many Examples\end{tabular}} & \textbf{\begin{tabular}[c]{@{}c@{}}5\\ Merged\end{tabular}} \\\hline            
\end{tabular}%
}
\caption{Participants, the topic they wrote on, and the corresponding narrative structure that they chose. When merging two narratives, ``primary" denotes the narrative structure that the participant used as the base and ``secondary" means that the writer incorporated elements from this narrative into the primary narrative.}
\label{tab:writer-structure-preferences}
\end{table}

% \begin{table}[H]
% \resizebox{\textwidth}{!}{%
% \begin{tabular}{ccccccc}
% \hline
% {\textbf{Participant ID}} & {\textbf{Domain}} & {\textbf{Topic}} & {\textbf{\begin{tabular}[c]{@{}c@{}}Everything\\ EWP (Everything)\end{tabular}}} & {\textbf{\begin{tabular}[c]{@{}c@{}}No Example\\ EWP-RemoveE\end{tabular}}} & {\textbf{\begin{tabular}[c]{@{}c@{}}No Walkthrough\\ EP - NoFewShot\end{tabular}}} & {\textbf{\begin{tabular}[c]{@{}c@{}}Merged Multiple \\ Narrative Structures\end{tabular}}} \\ \hline
% {P1}                      & {Computer Science} & {Merge Sort}                      & {}                                                                                              & {}                                                                                        & {Primary}                                                                                         & {}                                                                                         \\ \hline
% {P1}                      & {Statistics} & {Gradient Linear Additive Models} & {Primary}                                                                                       & {}                                                                                        & {}                                                                                                & {}                                                                                         \\ \hline
% {P2}                      & {Computer Science} & {Merge Sort}                      & {Primary}                                                                                       & {}                                                                                        & {Seconday}                                                                                        & {Yes}                                                                                      \\ \hline
% {P2}                      & {Computer Science} & {Gradient Descent}                & {Primary}                                                                                       & {}                                                                                        & {}                                                                                                & {}                                                                                         \\ \hline
% {P3}                      & {Computer Science} & {Merge Sort}                      & {}                                                                                              & {Primary}                                                                                 & {}                                                                                                & {}                                                                                         \\ \hline
% {P3}                      & {Qualitative Analysis}        & {Ordinal Regression}              & {}                                                                                              & {Primary}                                                                                 & {}                                                                                                & {}                                                                                         \\ \hline
% {P4}                      & {Computer Science} & {Merge Sort}                      & {Primary}                                                                                       & {}                                                                                        & {Secondary}                                                                                       & {Yes}                                                                                      \\ \hline
% {P4}                      & {Natural Language Processing} & {Controlled Decoding} & {Secondary}                                                                                     & {}                                                                                        & {Primary}                                                                                         & {Yes}                                                                                      \\ \hline
% {P5}                      & {Computer Science} & {Merge Sort}                      & {Primary}                                                                                       & {}                                                                                        & {}                                                                                                & {}                                                                                         \\ \hline
% {P5}                      & {Natural Language Processing} & {Word Embeddings}                 & {}                                                                                              & {}                                                                                        & {Primary}                                                                                         & {}                                                                                         \\ \hline
%                                               &                                                  & \textbf{Total Count:} & \textbf{5 primary, 1 secondary}                                                                                    & \textbf{2 primary}                                                                                           & \textbf{3 primary, 1 secondary}                                                                                      & \textbf{2}                                                                                                   
% \end{tabular}%
% }
% \caption{Participants and the topic they wrote on and the corresponding narrative structures that they choose. Primary denotes the narrative structure that the participant choose and Secondary means that the writer incorporated elements from this narrative into Primary narrative.}
% \label{tab:writer-structure-preferences}
% \end{table}

% \subsubsection{Structure: Does seeing different structure options help writers choose a direction to write in?}

% To better understand how seeing different narrative structures helps writers choose a direction to write in, we analyzed the user study transcripts that contained conversations between the study administrator and the participant as the participant was using the user interface. We found two key themes in how seeing the different narrative generations helped writers:
% \begin{itemize}
%     \item Explore which dimensions of a topic to include,
%     \item Reflect on the audience they are writing for
% \end{itemize}


% \textbf{Explore which dimensions of the topic to include}

% For P3, she found that exploring the different narrative structures helped her identify a main structure to follow, as well as additional aspects of the topic that she might want to include: "if I want to explain different aspects about, about merge sort [in my final narrative], I would use specific [tweets] from the [EP-NoFewShot] output, but I would definitely not use the [EP-NoFewShot] as a whole because it's a bit disordered." This demonstrates that the different narrative structures provide not only structural guidance for writers, but also support for balancing the breadth and depth of a topic. 

% For P5, he was "keen to see what the different narrative structures would look like, see what the trade offs are and necessarily like. Essentially, it gave me a good vision of how you can write the same thing differently." For the topic of merge sort, seeing different narrative structures allowed him to see "what degree of scientific information is being retained with example versus without example." Once seeing the two structures side-by-side, P5 chose to continue with the Everything narrative structure because it was "layman friendly." This demonstrates that exploring different options for narrative structural helped him to assess the trade-offs between the amount of scientific information included and the engagement of an everyday audience. 

% For P2, he initially disregarded the Everything structure with a single, walkthrough example because he didn't like the example that was used, but after reading all three narrative structure he said: "maybe with merge sort, starting with an example, on second thought, might actually be better. Like a very minimal example, sorting three numbers or something." This demonstrates, how viewing different narrative structures side-by-side allows the writer to weigh the pros and cons between each narrative sturcture. Furthermore, P2 mentioned how he would like to keep some of the content covered in the EP-NoFewShot condition such as "the same amount of time [merge sort takes], regardless of the initial order and the fact that it's efficient, it's actually something that actually no the other [narrative] says: [that merge sort] efficient." This demonstrates that the different narrative structures also help the writer explore different aspects of a topic to include in their explanation.

% For P4, after seeing the different narrative generations, he choose to merge aspects of each together, for example he said that "the optimal case is mixing the [Everything] structure and the [No Walkthrough] structure" because he wanted to retain the step-by-step walkthrough aspect of the Everything structure, but expand the narrative to include "contain more information." Additionally, when he was writing another science explanation on the topic of controlled decoding, he choose the No Walkthrough Structure because it covered a breath of differnt dimensions that he was interested in exploring.

% \grace{insert quotes}


% \textbf{Reflect on the Audience they are writing for}

% For P2, after reading the EP-NoFewShot condition that doesn't contain a step-by-step walkthrough, he stated that "[EP-NoFewShot] does have more information, for interested readers, but maybe it wouldn't be best for someone that has no idea about these topics," demonstrating how seeing the different conditions helped him consider which audience he was writing for. 

% Similarly, for P4, seeing each of the different narrative conditions prompted him to to identify the different aspects of each that he liked. When seeing the Everything condition with a walkthrough, he liked the use of an example to explain a topic, but was unsatisfied with the amount of content covered. As a result, he used the No Walkthrough narrative structure as his primary structure, but then used the step-by-step walkthrough technique in a subset of tweets. He used the web interface to ask GPT to "Try to mentioning the example about cleaning room in the 3 - 8 points," showcasing how within the No Walkthrough structure, the writer still wanted to maintain a step-by-step explanation to help readers understand a particular dimension of the topic in depth, while still maintaining the breadth of topics that he wished to cover. This demonstrates how exploring different narrative structures supports writers in exploring a range of science explanation techniques.

% For one participant, P3, when asked to select a narrative structure to move forwards with said: "I think it depends on the audience like. So if the audience is a non tech, low tech general public, I will probably incorporate the example from the [Everything] narrative structure, but I would definitely use the structure in a second output." For her third topic on ordinal regression, P3 mentioned how "[No walkthrough] structure looks like a checklist. [Researchers] can check if this model is applicable to the data we have. So I think, this [narrative structure] being fragmented when explaining this concept is totally okay and to explain this topic to an audience [of researchers]."

% These findings demonstrate that by viewing different narrative structures side-by-side also helps writers reconsider who their audience is. While this study was focused on create science explanations for an everyday audience, these results demonstrate that this method of comparison also helps writers write for more technical audiences as well. 


\subsection{\textbf{RQ2}: Different Style Options Help Writers Balance Personal Preferences with Social Media Communication}
The most common narrative strategy that writers used was \textit{With Personal Language} (12 of 20). Merging elements from the \textit{With Personal Language} and \textit{Without Personal Language} style was the second most common action that writers performed (6 of 20). Only 2 of 20 narratives used narratives without personal language. Overall, writers prefer using science narratives that include personal language and editing the LLM-generated language to match their own voice. Many writers found that comparing between narratives with and without personal language helped them identify and adopt useful techniques they might not have otherwise used in their own writing.

8 of 10 participants chose to begin with the \textit{With Personal Language} option because seeing personal language provided a more accessible base narrative that writers could use to edit or replace style elements (P1, P2, P3, P5, P6, P8, P9, P10). Seeing the placement of personal language helped P6 identify where and what to edit instead of determining where and what to add if he was working from the \textit{Without Personal Language} option: ``I try and get a pretty good baseline, and the personal language involved adds to the baseline." Furthermore, seeing options with personal language helped writers appreciate language that they might not have thought of using and internalize the importance of these style choices when read in juxtaposition with a narrative without personal language. For P7, seeing GPT-generated personal language helped him adapt to the genre of science communication on social media: ``I think the enthusiasm, while it may not necessarily be the way that I would have written this, feels like it's a more engaging way of trying to explain this to people on the internet."

But for 2 of 10 writers, starting from the narrative \textit{Without Personal Language} helped them evaluate the technical content of the explanation before adding back personal language (P2, P3). P2 mentioned how he prefers the remove personal language option because it is a ``simpler version" that he can add his own writing voice to, instead of editing GPT's writing voice. P3 mentioned that she chose to start with the narrative \textit{Without Personal Language} because she valued the content explanation over the personal language: ``I do like being more objective and professional when writing explanations about scientific concept." 

% After a writer has finalized the structure of their science explanation, we ask the writer to choose between the narrative with and without personal language to understand the writer's preferences. Then we observed how the writer edited, adapted, and merged their chosen narrative into a an explanation that they are satisfied with. 
% \grace{don't ened to restate what we are doing,, we found blah, just say how a human is personal language was preferred over no personal language. 

% reconceptialize the section and the groupings}
% To understand how seeing different style options can help writers balance their professional identities with informal language strategies, we evaluate how many writers choose each style option (\textit{With Personal Language}. 

% \grace{re-conceptualize}

% But for some writers, having both options provided a reference to evaluate and revise their draft against. Furthermore, different writers used the side-by-side comparison of narratives with and without personal language in different ways to support their writing process. \grace{keep this? Writers who selected \textit{With Personal Language} often deleted or revised the LLM-generated language to match their own voice, while writers who selected \textit{Without Personal Language} often added their own voice to the explanation. }

% \grace{
% emphasize these two points: start with personal lanuage and then add it --> or start with remove and add it 
% --> why seeing both why they started with it or not, and add as color/reason
% --> drill into 2 paragraphs (3 at most)
% clarity, seeing btoh help....

% }




% P7 mentioned how the personal language makes the introduction to the problem too ``lengthy and drawn out," which can distract from the purpose of explaining the science.  P1 mentioned how seeing the narrative without personal language let him ``parse exactly what information is being presented without any frills around it," which let him come to the realization that ``maybe, it's not clear exactly what is happening" (P1). Sometimes writers would copy and paste technical sentences from the \textit{Without Personal Language}

% Seeing both style options helped 5 out of 10 participants remember the audience of everyday people on social media they were writing for (P1, P9, P7 P10, P3). For P9, seeing both options helped him ``gain clarity around what [he] did want" in terms of how much technical detail to include. P9 also mentioned how sometimes he might forget what is considered a technical term for everyday audiences, so when he reads a narrative without personal language, the ``re-introduction of more technical terms stands out more." It is helpful to him because the comparison `` reminds [him] of things to look for when [he] edits." This demonstrates the common struggle that experts have in evaluating how much contextualization is necessary to explain science topics for everyday people. 

% \grace{tighten the anlaysis for quotes}


% We found the most of the science explanations choose the Yes - Personal Language style as the primary narrative style (8/10 science explanations). Only two science explanations used the No - Personal Language as the primary narrative style (2/10). But most interestingly, \grace{there were 7 instances where writers choose to XYZ} 7 writers choose to merge elements from both the Personal and No Personal Language sections. This demonstrates that while there is a strong preference for writers to choose science narratives that contain personal language, often times writers use elements from both the personal and no personal language narrative styles. This demonstrates that individual writer preferences actually lie on a spectrum between the two extremes: Yes - Personal Language and No - Personal Language. By showing writers these two different options, writers can more effectively explore the continuum between these two extremes. To further understand, writer's perspectives on these two different style methods, we analyze the interview transcripts from the writer study. 


\begin{table}[H]
\resizebox{\textwidth}{!}{%
\begin{tabular}{|c|c|c|c|c|c|}
\hline
{\textbf{Participant ID}} & {\textbf{Domain}} & {\textbf{Topic}} & {\textbf{With Personal Language}} & {\textbf{Without Personal Language}} & {\textbf{Merged Narrative Styles}} \\ \hline\hline
  
\multirow{2}{*}{P1} & {Computer Science} & {Merge Sort} & {\checkmark} & & \\ \cline{2-6}
& {Statistics} & {Gradient Linear Additive Models} & & & Primary: Without Personal Language \\ \hline 

\multirow{2}{*}{P2} & {Computer Science} & {Merge Sort} & & {\checkmark} & \\ \cline{2-6}
& {Computer Science} & {Gradient Descent} & & & Primary: With Personal Language \\ \hline 
 
\multirow{2}{*}{P3} & {Computer Science} & {Merge Sort} & & & Primary: With Personal Language \\ \cline{2-6}
& {Qualitative Analysis} & {Ordinal Regression} & & & Primary: Without Personal Language \\ \hline 
 
\multirow{2}{*}{P4} & {Computer Science} & {Merge Sort} & {\checkmark} & & \\ \cline{2-6}
& { Natural Language Processing} & {Controlled Decoding} & {\checkmark} & & \\ \hline 

\multirow{2}{*}{P5} & {Computer Science} & {Merge Sort} & {\checkmark} & &  \\ \cline{2-6}
& {Natural Language Processing} & {Word Embeddings} & & & Primary: With Personal Language \\ \hline 
 
\multirow{2}{*}{P6} & {Computer Science} & {Merge Sort} & {\checkmark} & & \\ \cline{2-6}
& {Programming Languages} & { Formal Verification} & {\checkmark} & & \\ \hline 

\multirow{2}{*}{P7} & {Computer Science} & {Merge Sort} & & &  Primary: With Personal Language \\ \cline{2-6}
& {Natural Language Processing} & {Embedding Space} & {\checkmark} & &  \\ \hline 
 
\multirow{2}{*}{P8} & {Computer Science} & {Merge Sort} & {\checkmark} & & \\ \cline{2-6}
& { Quantum Algorithms} & { Grover's Algorithm} & & {\checkmark} & \\ \hline 

\multirow{2}{*}{P9} & {Computer Science} & {Merge Sort} & {\checkmark} & &  \\ \cline{2-6}
& {Tangible User Interfaces} & {4D Printing} & {\checkmark} & &  \\ \hline 
 
\multirow{2}{*}{P10} & {Computer Science} & {Merge Sort} & {\checkmark} & & \\ \cline{2-6}
& {Optimization Algorithms} & {Multiplicative Weights Update} & {\checkmark} & & \\ \hline \hline

& & \textbf{Total Count:} & \textbf{\begin{tabular}[c]{@{}c@{}}12\\ With Personal Language\end{tabular}} & \textbf{\begin{tabular}[c]{@{}c@{}}2 \\ Without Personal Language\end{tabular}} &  \textbf{\begin{tabular}[c]{@{}c@{}}6\\ Merged\end{tabular}} \\ \hline
\end{tabular}%
}
\caption{Participants, the topic they wrote on, and the corresponding narrative style that they chose. For merged narratives, ``primary" denotes the narrative style that the participant chose as their base narrative, and writers incorporated elements from the remaining style option into the primary narrative.}
\label{tab:writer-preferences-style}
\end{table}

% \grace{edit table to include primary and secondary}

% \begin{table}[H]
% \resizebox{\textwidth}{!}{%
% \begin{tabular}{cccccc}
% \hline
% {\textbf{Participant ID}} & {\textbf{Domain}} & {\textbf{Topic}} & {\textbf{Yes - Personal Language}} & {\textbf{Remove - Personal Language}} & {\textbf{Merged Narrative Styles}} \\ \hline
% {P1}                      & {Computer Science} & {Merge Sort}                      & {Primary} & {Secondary}                           & {Yes}                              \\ \hline
% {P1}                      & {Statistics} & {Gradient Linear Additive Models} & {Primary} & {Secondary}                           & {Yes}                              \\ \hline
% {P2}                      & {Computer Science} & {Merge Sort}                      & {Primary} & {Secondary}                           & {Yes}                              \\ \hline
% {P2}                      & {Computer Science} & {Gradient Descent}                & {Seconday} & {Primary}                             & {Yes}                              \\ \hline
% {P3}                      & {Computer Science} & {Merge Sort}                      & {Primary} & {} & {Yes}                              \\ \hline
% {P3}                      & {Qualitative Analysis}        & {Ordinal Regression}              & {Secondary}                        & {Primary}                             & {Yes}                              \\ \hline
% {P4}                      & {Computer Science} & {Merge Sort}                      & {Primary} & {} & \\ \hline
% {P4}                      & {Natural Language Processing} & {Controlled Decoding} & {Primary} & {} & \\ \hline
% {P5}                      & {Computer Science} & {Merge Sort}                      & {Primary} & {} & \\ \hline
% {P5}                      & {Natural Language Processing} & {Word Embeddings}                 & {Primary} & {Secondary}                           & {Yes}                              \\ \hline
%                                               &                                                  & \textbf{Total Count:} & \textbf{8 primary, 2 secondary}                       & \textbf{2 primary, 4 secondary} & \textbf{7 Merged}                                    
% \end{tabular}%
% }
% \caption{Participants, the topic they wrote on, and the corresponding narrative style that they choose. Primary denotes the narrative style that the participant choose and Secondary means that the writer incorporated elements from this narrative into Primary narrative style.}
% \label{tab:writer-preferences-style}
% \end{table}

% We found that different writers had different preferences and tolerances for incorporating personal language into their science explanations.

% P1, mentioned that seeing the explanation style with personal language "reading [the remove personal language generation] made it more obvious to me that those frills [the line: "this is cool, isn't it"] are very necessary posting something on on Twitter." This shows how the side-by-side comparison allowed the writer to realize the importance of personal language in making science communication engaging, by helping him think from the perspective of a reader.

% For P2, when seeing the side-by-side explanation styles with and without personal language he appreciated seeing the narrative without personal language because it helped him "stay focused on the concept and not get distracted by details in the example." P2 also mentioned how the personal language was distracting to him and it wasn't until he saw the explanation without personal language did he realize that there were still structural changes that he wanted to make: "reading it in simpler terms helped me determine the the structural changes I would want to make." This demonstrates that removing personal language also helped the writer check the structure and content of the science explanations without personal language that might distract him.

% But for P3, seeing the side-by-side comparisons of the narratives with and without personal narratives re-emphasized her own preferences for science communication. She said that "I think for scientific concept, it's nice to explain it in this non emotional way." Even though, she thought that the narrative with personal language looks "professional enough," the occurrence of phrases like "brilliant example of a theory-heavy concept" made the entire thread seem less professional. When finalizing the narrative, P3 decided to copy over the first tweet of the narrative with personal language because of its "vivid intro." This demonstrates how even when presented with the option to make the twitter thread more engaging, the writer's own preferences and habits for science communication overshadowed her willingness to try a different style of communication.

% For P5, he always chose the explanation that contained personal language, saying that "speaking in like a non personal language, in my opinion, makes it, like, less engaging, and also, from a reader's perspective, it looks too formal." When he encountered weird phrases such as, "Alexa, are you listening," he would opt to remove those phrases manually to "reflect to [his] personal like writing style." P5 would describe his writing style for science communication on social media as "the way you would describe something to a peer, using a informal manner, using examples, or using something that is very easy to grasp... finding an example that as many people can relate to," demonstrating that he already feels comfortable using and communicating science in an informal manner. 

% Comparing P3 and P5's responses, each writer's own personal preferences and ideas for what makes effective science communication shapes their willingness to use to use alternative styles of communication. For P3, her tendency to shy away from using personal language reflects her personal view that science communication should be professional in order to maintain credibility. While for P5, his own writing style is much more familiar with and open to using informal language to engage an audience.

% These findings illustrate that while writer's preferences for incorporating personal language into science communication may vary, most authors choose to find a middle ground between the two extremes: incorporating different elements from Yes and No Personal Language into their final science explanation. Presenting writers different options, allowed them to incorporate different degrees of personal language into their writing.   

% \grace{5 people merged them, becuase they felt it was a continuum, +juicy quotes, keep the writing tight}
% \grace{every two paragraphs a new thought, keep it snappy + moving along, answer is always two paragraph}

% Therefore, showing writers different options not only helps them decide which narrative structure to use, but also provide them with inspiration with other characteristics that they might want to include. Seeing different narrative structure helps participants decide (1) what to include and (2) how to include it. We will explore these dimensions further through the participant's experiences using the design probe...  \grace{find alternative work for system}





% \grace{same structure for personal language: how many chose each one + if they did merges, if they choose personal langaueg + then deleted things, which one they learned the hardest towards}
% - primary and then secondary
% \grace{help people undestand the design space, MERGING!! say merging, continuous, HOW MUCH do you lean on one example, which poeople chose merges + which merges they picked}



% \textbf{Exploring Different Voices by Working with Narratives with Generated Personal Language}
% Alternatively, for P1, having a narrative with personal language already built into the explanation structure, allowed him to experiment with writing in a conversational voice for science that he was not familiar with. Saying "if I were to write something myself, I wouldn't be necessarily good at, like, writing something like: "oh, did you ever wonder,"" P1 was more open to accepting GPT's generated language because seeing the narratives with and without personal language helped him realize when you get rid of the personal language the explanation "just reads like an academic paper, it doesn't actually read like something you would see on Twitter." Furthermore, P1 also mentioned that he doesn't "think of [his] voice as being the best suited to explain merge sort on the internet, and it feels like the voice that [explanation with personal language] uses is better for that. [The explanation with personal language] feels like this was something I could possibly read on the web. And I think if I were to just write in my own voice, it wouldn't sound like that. So I think this is better. Actually. I prefer this."

% \textbf{Working with GPT Generated Personal Language}
% For P2, when seeing the side-by-side explanation styles with and without personal language he appreciated seeing the narrative without personal language because it helped him "stay focused on the concept and not get distracted by details in the example." P2 also mentioned how the personal language was distracting to him and it wasn't until he saw the explanation without personal language did he realize that there were still structural changes that he wanted to make: "reading it in simpler terms helped me determine the the structural changes I would want to make."






% his experience with the side-by-side science explanations with and without personal language 


% - helped writers explore a dimension / register that they might not be familiar with using, but like when they see it
% - other times, the writer has strong preferences/ideas about how science communication should be ie. formal and thought that more personal language would undermine their authority, and thus always choose not to include personal language. 
% - another time, a writer merged tweets from both the yes - personal narrative and the remove personal narrative because sometimes the personal language overshadowed the technical aspects. 


% Writers have diverse preferences for how to structure their science explanations that vary depending on the topic, their comfort with using informal language in science communication, and the technical correctness and scientific clarity of each narrative structure. To understand the considerations that writers have for how they choose their preferred narrative structure, we analyze the think-aloud interview transcripts to find common themes. 

% Out of 20 science narratives that were writing, 14 narratives incorporated some aspect of the EWP narrative structure in their final output which included an example, walkthrough, and personal language. 9 of 20 narratives incorporated the EP narrative structure into their final output. 2 or 20 narratives used the WP narrative structure in their final output. 5 of 20 narratives choose to merge narrative structure. 

% Merging narrative structures means that writers select a primary narrative and incorporate additional elements from the secondary narrative. All writers who chose to merge narrative structures merged the EWP and EP condition. \grace{shoudl this just be primary narratives? currently covers if any part of a narrative is used in final it counts} 


% \grace{
% are these the two dimensions that we want to measure / disentangle
% 1) how does seeing multiple options help writers in the writing process
% 2) when do writers choose to use each narrative strategy 
% }

% \grace{Title: Seeing Different Narrative Structures Helps Writers}