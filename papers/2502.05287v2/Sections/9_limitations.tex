\section{Limitations and Future Work}
Our reader study only included undergraduate students in the US, and as such may not accurately reflect the general public. Additionally, we only conducted followup interviews with a subset of 8 participants which might not capture the diverse range of reader preferences. Instead, our work seeks to illuminate some reasons why readers may prefer certain strategies, and demonstrates that even within this population there are variations in reader preferences. Our dataset for science explanations only contained 15 different topics that covered 5 STEM topics (Physics, Computer Science, Civil Engineering, Psychology, and Statistics). Future work could expand the domain to life sciences and topics covered to help identify themes and trends in science communication techniques across different domains and complexities.

Our writer study only included researchers from the computer science field. Future studies should include other STEM scientists to understand how writing in different fields of study might result in different structures and style preferences. Additionally, 9 of 10 of the participants in the writer study were men which might not offer a complete understanding of expert preferences for science communication on social media. There was little age diversity in PhD students, so additional research will need to investigate how age differences affect a scientist's comfort with using different structure and style options for science communication on social media. When using the system, writers were not asked to actually publish their final science narratives on social media. As such, writers' choices might not fully reflect how they would have written if they were to post on social media. Finally, future studies can also evaluate the longitudinal effects of using the system to evaluate whether the novelty of seeing different structures and style options wears off with prolonged usage and whether there are any long-term benefits in seeing multiple structure and style options during the drafting process for science communication. 