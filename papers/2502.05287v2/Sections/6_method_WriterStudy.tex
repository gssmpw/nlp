\section{Writer Study: Methodology}

% \grace{can condense this into 2 sentence. we found that readers like blah blhan, previous work has shown that writers like blah blah and as such we do these RQs.}

The reader study found that overall readers had no preference between narratives with and without walkthroughs but did prefer when explanations had multiple different examples to explain the topic. Previous studies have shown that scientists often struggle with two aspects of science communication on social media (1) framing science for everyday audiences and (2) using informal language to communicate science \cite{williams2022hci, koivumaki2020social}. As such, we provide writers with different options to structure and frame their writing (\textit{One Example}, \textit{No Example}, \textit{Many Examples}) and provide options to see their narrative with and without personal language to support the writing process for social media science communication. Our research questions are:

\textbf{RQ1:} How does seeing structure options (\textit{One Example}, \textit{No Example}, \textit{Many Examples}) help writers consider different framing strategies for writing science for social media?

\textbf{RQ2:} How does seeing style options (\textit{With Personal Language}, \textit{Without Personal Language}) help writers consider different communication styles when writing science for social media?


% \grace{make it shorter} In the reader study, readers had no preference between narratives with and without a walkthrough. Some readers preferred narratives without a walkthrough because they contained multiple examples. As such, we investigate how different structure and framing options  can help scientists write science on social media. The reader study also demonstrated that while most participants preferred reading narratives with personal language, some participants preferred narratives without it, demonstrating that there is a range of possibilities for writers to choose from. Thus, we provide scientists with two different narratives with and without personal language to help them explore their comfort between these two extremes. Our research questions are

% \grace{format the same, but keep naming as RQ}


% \grace{readers had no preference for with/without walkthroughs. but the readerst that preferred walkthroughs because they had mutliple 

% had diverse preferences for narratives with and without personal language (subjective, conversational, and informal language and humor). 

% reader study there was high variance, there is a variety to pick through. need one more 
% The reader study also showed that most readers preferred narratives with personal language, but . But not always... meaning that there is a spectrum that writers can pick.} 


% In our reader study, we explore how techniques of example (E) and walkthrough (W) as separate dimensions of science communication. But the interview results from (H2:Walkthrough) demonstrate that some participants liked narratives without a walkthrough because they contained multiple different examples to help explain the topic. The presence of multiple different examples caused the structure of the explanation to be a more modular, each example illustrated a different dimension of the topic. As such, to help scientists explore different framings for science communication on social media, we evaluate one example, no example, and many example narrative structures.  

% The reader study also demonstrated that readers have

% To accommodate scientists' varying comfort levels on science communication on social media during the writing process. We provide scientists with different options for how to structure and style their writing. 


% Scientists are trained to write formally for science conferences and journals struggle to adapt to the  of social media for concerns around how their social media persona might affect their professional reputations \cite{koivumaki2020social}. Furthermore, many scientists want to communicate their work on social media for either for personal or professional benefits \cite{williams2022hci}. But even when they are taught techniques for science communication, may still feel hesitant to utilize them \cite{10.1145/3479566}.

% We want to understand how  To do so, We hypothesize that seeing multiple options helps writers navigate their own comfort level  on the spectrum of formal to informal science communication. C

% From the reader study results for (H2:Walkthrough), we found that there was no statistically significance preference for readers between the with and without walkthrough narratives. In the followup interviews,  The presence of multiple different examples caused the structure of the explanation to be a more modular, each example illustrated a different dimension of the topic. Narratives a walkthrough and one example explained different dimensions of the topic with one example using a more connected narrative approach.



% This shows how the structure of a narrative is also closely tied to the number of examples used. 

% The results from the reader study demonstrated how the use of example(s) were tied to the overall narrative structure. 


% Building on our Reader Study findings, which indicated...

% Science experts often have writers' fears and shy away from using the \grace{three social media --change} techniques when communicating science to the general public. We want to understand how their fears might be mitigated during the writing process by giving them the options of using initial drafts with and without the three techniques (Example, Walkthrough, Personal Language). 
% In particular, we explore the following questions:

% \grace{misalign with the fears + the RQ1, is RQ1 the direction we want to go in, scientists struggle with this, cite katy's paper everytime we say that scientists struggle w/ this. choose a direction to write in... set up the problem again, and bring this back in: scientists are trained to write in formal ways etc. etc. research shown that it can be hard to adopt these styles, explore whether showing writers options for informal writing can influence their adoption for informal writing, need to set up the context a bit more. do it for structure and style}


% RQ1: Does having the options to view a narrative w/ and w/o a walk-through help writers choose a direction to write in?

% RQ2: Does having the options to view a narrative with and without personal language help writers choose a direction to write in?



% \yy{move this into subsections}
% We conducted a writer study with 5 PhD-level researchers. 
%The writer participants are guided through a workflow via a web interface with LLM generations and editing functionalities. The workflow asked them to first work on the structure drafts (Example, Walkthrough), then work on the personal language style drafts, and finally make edits to the science explanations until they are publish-ready. We used think-aloud protocol and semi-structured interview questions to elicit rich insights from their experience. 
%We describe this in more detail in Section \ref{sec:writer_workflow}. 
%we  a web interface that provides a workflow that guides the writer participants to work first on the 
% Participants are asked to produce two Tweetorials. One topic is pre-assigned by the experimenter, and the other is chosen by the participant.

% Our survey on readers preferences for how science is explained to them is diverse. We want to explore whether showing writers different ways to (1) structure science narrative and (2) narratives with and without personal language will help them 


\subsection{Participants}
We recruited 10 PhD-level researchers interested in communicating their research to the public on social media. Participants are from 2 universities, an average age of 25, with a gender distribution of 9 males and 1 female (Table \ref{tab:my-table}). We advertised the study to students in research labs through school mailing lists, Slack workspaces, and snowball sampling among lab mates of the participants. Their expertise spans various CS research areas, including natural language processing, programming languages, and social computing. The study was conducted over Zoom and took approximately 2 hours. Each participant is compensated \$40 dollars total. The study was approved by our institutional IRB. 

% \yy{do we mention their experience w/ past tweetorial studies?  past experience w/ science communication?? Justification for only using CS field? }

% 6 Computer Science PhD students

\begin{table}[]
\begin{tabular}{|c|c|c|}
\hline
\textbf{ID} &  \textbf{Field of Expertise}              & \textbf{Research Experience (years)} \\ \hline
1                    & Computer Science and Journalism          & 2                            \\ \hline
2                    & Artificial Intelligence and Neuroscience & 2                            \\ \hline
3                    & Human-Computer Interaction               & 2.5                          \\ \hline
4                  & Natural Language Processing              & 2                            \\ \hline
5                    & Natural Language Processing              & 3                            \\ \hline
6                       & Programming Languages                    & 5                            \\ \hline
7                      & Computer Security                        & 2.5                          \\ \hline
8                      & Quantum Computing                        & 3                            \\ \hline
9                  & Human-Computer Interaction               & 3                            \\ \hline
10                   & Computer Science Education               & 7                            \\ \hline
\end{tabular}
\caption{Participant Demographics for Writer Study}
\label{tab:my-table}
\end{table}

% \begin{table}[H]
% \begin{tabular}{|c|c|c|c|c|c|}
% \hline
% \textbf{ID} & \textbf{Gender}        & \textbf{Field of Expertise}         & \textbf{Research Experience}                         \\ \hline
% 1           & Man                      & Computer Science and Journalism & 2 years                           \\ \hline
% 2           & Man             & Computer Science                & 2 years                                           \\ \hline
% 3           & Woman                   & Social Computing                & 2.5 years                                  \\ \hline
% 4           & Man                        & Natural Language Processing     & 2 years                           \\ \hline
% 5           & Man                      & Natural Language Processing     & 6 years                               \\ \hline
% \end{tabular}%
% \caption{Participants Demographics for Writer Study}
% \label{tab:my-table}
% \end{table}




% \begin{table}[H]
% \resizebox{\textwidth}{!}{%
% \begin{tabular}{|c|c|c|c|c|c|}
% \hline
% \textbf{ID} & \textbf{Gender} & \textbf{Education}         & \textbf{Field of Expertise}         & \textbf{Research Experience} & \textbf{Topic of Choice}                         \\ \hline
% 1           & Man             & 2nd Year Masters           & Computer Science and Journalism & 2 years                          & Statistics: Generalized Linear Additive Models   \\ \hline
% 2           & Man             & Bachelor of Science - 2023 & Computer Science                & 2 years                            & Computer Science: Gradient Descent               \\ \hline
% 3           & Woman           & 1st Year Ph.D              & Social Computing                & 2.5 years                         & Qualitative Analysis: Ordinal Regression         \\ \hline
% 4           & Man             & 1st Year Ph.D              & Natural Language Processing     & 2 years                           & Natural Language Processing: Controlled Decoding \\ \hline
% 5           & Man             & 2nd Year Ph.D              & Natural Language Processing     & 6 years                           & Natural Language Processing: Word Embeddings     \\ \hline
% \end{tabular}%
% }
% \caption{Participants for Writer Study}
% \label{tab:my-table}
% \end{table}


\subsection{Writing Study Procedure and Analysis}

\subsubsection{Study Procedure}
Each session includes a pre-study presentation, an interface demonstration, and two writing sessions with a 15-minute break in between. The experimenter used a short presentation to educate participants about science communication on social media, a background on Tweetorials, and different techniques and examples for science communication on social media. Next, the experimenter demonstrated the study web interface and explained how it worked. 

% \subsubsection{Writing Study}
When the participant was ready for the writing session, the experimenter shared with the participant a URL link to the study web interface. The experimenter started screen and audio recording upon the participant's verbal consent. Participants use the study interface once for each writing session: first to write about the predetermined topic of merge sort and then to write about a topic of their choice. The experimenter implemented think-aloud protocol and encouraged the participant to voice their thought process, reasoning behind their structure and style choices, and editing decisions. After each writing session, the experimenter conducted a semi-structured interview with the participant to understand how seeing different structures and styles influenced their writing choices. Some sample questions are: 
How did you arrive at your choice of structure/style?
How did seeing the options change your preference of structure/style?
How have reading the unselected options help you decide which direction to write in?

% The web interface used in the study provides a workflow that guides the participants to work first on the structure choices and second on the language style choices. We describe this in more detail in Section \ref{sec:writer_workflow}. 

% \grace{can cut a lot from this paragraph, check for redundancy}

% Participants are asked to produce two Tweetorials. One topic is pre-assigned by the experimenter, and the other is chosen by the participant. 
% After finishing writing the two topics, we conducted a semi-structured interview to elicit rich insights into their experience. We collected usage data and Tweetorial generations from the web interface, as well as Screen and audio recordings for further analysis.


\subsection{Study Interface}
\label{sec:writer_workflow}
% \grace{refer to figure 5 at the beginning of 6.2, change the name interface to match our naming conventions + rescreenshot interface }

We built a web interface that guides writers through a workflow with LLM generations and editing functionalities (Figure \ref{fig:step1}, Figure \ref{fig:step2}, Figure \ref{fig:step3}). The workflow includes the following 4 steps:

% \grace{need to relate back to structure and the way that we describe it to people, may need to describe that earlier, say this further up, examples are part of the structure section.... motivate the problem of why we do walkthrough and examples.... talk this out more and connect them, simple why are we doing things, not the same thing as reader study set up}
\textbf{Step 1: Structure Options} Users first enter their domain and topic to generate the 3 different structure options. After the generation, 3 columns are displayed side-by-side with the corresponding LLM prompts: \textit{One Example}, \textit{No Example}, and \textit{Many Examples} (Figure \ref{fig:step1}. The user chooses one of the three to proceed with. Writers can merge certain paragraphs from two structure options by copying and pasting between the columns before proceeding. 

\textbf{Step 2: Selected Structure Feedback and Edits} The second part of the interface (Figure \ref{fig:step2}) allows the writer to iterate on the selected structure by providing feedback instructions to an LLM or making manual edits directly in the textbox (copy, paste, delete, and type). The writer is asked to focus only on editing structural aspects of the text such as technical accuracy, content, and sequence. When they are satisfied with the structure, they proceed to the next step. 


\textbf{Step 3: Language Style Options} The third part of the interface compares different style options, it has 2 columns displayed side by side (Figure \ref{fig:step3}). On the left is the writer's selected and edited draft from the previous step which contains personal language and the right is the draft without personal language. Writers select which narrative they want to proceed with to the next step. Writers can merge certain paragraphs from both options by copying and pasting between the columns before proceeding. 


\textbf{Step 4: Final Edits}
The last part of the interface displays the draft from Step 3 (not shown). The writer refines and finalizes the writing until they are satisfied and ready to share it on social media. 
% \yy{TO BE CONTINUED}

% % % The workflow asked them to first work on the structure drafts (Example, Walkthrough), then work on the personal language style drafts, and finally make edits to the science explanations until they are publish-ready. 




% % \yy{terminology to be updated based on system screenshots. }




% Following that, the second part of the interface is for style choices. It has 2 columns displayed side by side, writer's current chosen version which has personal language vs. a version with personal language removed. Again, there is "I Like it!" button for user to choose which one to proceed with for further editing. 

\begin{figure}
    \centering
    \includegraphics[width=1.0\linewidth]{Figures/Step1.png}
    \caption{Study interface: Viewing different structure options (1st of 3 steps)}
    \label{fig:step1}
\end{figure}

\begin{figure}
    \centering
    \includegraphics[width=1.0\linewidth]{Figures/Step2.png}
    \caption{Study interface: Editing narrative structure (2nd of 3 steps)}
    \label{fig:step2}
\end{figure}

\begin{figure}
    \centering
    \includegraphics[width=1.0\linewidth]{Figures/Step3.png}
    \caption{Study interface: Viewing different style options (3rd of 3 steps)}
    \label{fig:step3}
\end{figure}




% First, participants are shown three GPT-generated Twitter threads using three structural techniques - (1) Everything(EWP), (2) No example, (3) No step-by-step walkthrough. Participants can re-generate as needed. 
% They are asked to choose one to continue in the next step. 

% Second, participants are to give ChatGPT feedback to refine the structure until they are satisfied. 

% Third, participants are shown two Twitter threads with different personal language styles - (1) their current version w/personal langauge, (2) GPT removed personal language from it. Participants are to choose which one to continue with in the next step. 

% Finally, the participants edited the Twitter thread until they think it is ready to be published. 







% \subsubsection{Semi-Structured Interview}
\subsubsection{Data Analysis}
We analyze interview transcripts and video recordings to understand quantitative and qualitative aspects of participants' choices and reasoning when presented with different options. We analyzed participants' choice of structure (\textit{One Example}, \textit{No Example}, \textit{Many Examples}) and style options (\textit{With Personal Language} and \textit{Without Personal Language}), participants' editing actions and LLM feedback prompts, iterations of writing generations and refinements, as well as the final writings. 

One researcher independently conducted a bottom-up, open-coding approach to data analysis \cite{charmaz_constructing_2006}. Then, the researcher worked with two other researchers to iterate on the codes, discuss their similarities and differences as part of a comparative analysis \cite{merriam2015qualitative}, and leveraged them in an affinity diagramming process \cite{holtzblatt2017affinity}. The researchers determined that they reached code saturation when no researcher could identify new codes or arrive at new interpretations. 



% We collected a combination of quantitative and qualitative data. These include We transcribed the semi-structured interview recordings into text for analysis. 



% The interview is audio recorded and later transcribed for analysis. 

% Each participant is asked to use the web interface twice to create two Twitter threads: one on a pre-assigned topic and the other on a topic of their choice.






% \subsubsection{Think aloud protocol}

% The experimenter implemented think aloud protocol through the user study process. We constantly encouraged the participant to voice their thought process, and reasoning behind their choices, feedback and editing decisions. 

% \subsubsection{Semi-structured interview}

% After completing two Twitter thread writing, we conducted a semi-structured interview with the participant to reflect on their experience. We asked questions to understand how seeing different structures and styles influenced their writing choices. Some sample questions are .. 

% \subsection{Data Analysis}
% \grace{6.3.2 writing study procedure: data collection and analysis and  instead of writing study + don't need 6.3.3 be it's own thing, just be a paragraph merged with the previous section

% 6.4 is short, can maybe lump it into previous 

% don't want all the tiny section. sections shouldn't be one paragraph}



%In accordance with Mcdonald et al. \cite{mcdonald_irr}, we did not compute inter-rater reliability (IRR), since we used the coding process to discover emergent themes or recurrent topics and permitted multiple possible interpretations of the meaning of the codes. After the two researchers completed their synthesis of an affinity diagram, one additional researcher reviewed the themes and provided their comments. 


% iterations of tweet generation/edits via the web interface in json

% screen recording of the writer's process

% think aloud audio recording

% interview recording and transcripts 

% Analysis -- thematic coding
