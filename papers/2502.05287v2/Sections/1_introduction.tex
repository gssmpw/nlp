\section{Introduction}
% \subsection{Background}
% \begin{itemize}
%     \item Communicating science important for the public to understand and engage in a rapidly changing world.
%     \item However, traditional science communication is formal and often inaccessible to lay audiences. There’s a growing trend on social media to make science communication more approachable.
%     \item MORE HERE.
%     \item 
% \end{itemize}
% MOTIVAITON: Public understanding of science is important


%%%%%%%%%C&C 2025 The maximum length for papers is 8,000 words excluding the title, references and figure/table captions. %%%%%%%%%%%
Science communication increases public interest in science by educating, engaging, and encouraging everyday people to participate in the sciences \cite{national2017communicating, burns2003science}. With the rise of social media, science communication has transitioned away from traditional scholarly articles and towards more informal platforms such as blogs and social media \cite{hou2017hacking, williams2022hci, bruggemann2020post}. There is a growing need for scientists to participate in these public forums to bridge the gap between science and everyday people \cite{pearson2001participation, powell2008building}. 


% Background: Science Writing Strategies on Social Media}
On social media, science communication strategies emphasize making science engaging and understandable to non-technical audiences and adapt common practices in science communication literature for a social media audience \cite{gilbert2020run}. A study of science communication on Twitter found these strategies: 1) use of a \textbf{relatable example} -- such as using the iridescence of bubbles to motivate the topic ``Thin Film Interference" in physics, 2) use of a \textbf{step-by-step walkthrough} that uses 10-12 tweets which use an example to explain every step, and 3) use of \textbf{personal language} -- the writer talking about themselves and their experience, often in a personal and conversational way, \textit{``Crazy, right? WTH was I thinking ?!?''}~\cite{gero2021makes}. 

% Findings - readers
But previous work has not evaluated whether these science communication techniques on social media (example, walkthrough, and personal language) are actually helpful in helping everyday people learn about science. As such, we conduct a study of non-technical readers' rating of STEM explanations with and without each of these three social media explanation techniques to understand how each factor impacts making science explanations engaging and understandable. To facilitate fair comparisons between each condition, we use an LLM to generate science explanations for 15 topics: 5 STEM fields, each with 3 topics at different levels of complexity. Readers evaluated them for understanding and engagement. Overall, readers preferred explanations with examples over no examples and personal language over no personal language. However, readers were split between narratives with and without walkthroughs; certain topics were more suitable for narratives without a walkthrough. Even for techniques they did prefer, readers' personal experience complicate why some preferred narratives without a given technique. There is no one-size-fits-all approach to science communication on social media. 

% Problem
As such, it's important to support STEM experts in navigating diverse audience preferences and help them engage with science communication on social media. Many STEM experts are eager to communicate to the public \cite{della2021expert}. But many scientists are not trained in how to translate complex technical topics for an everyday audience and struggle to motivate and explain topics to people beyond their expertise \cite{williams2022hci}. Furthermore, studies find that scientists are hesitant in adopting science communication strategies for social media because inaccurate interpretations or misrepresentations of their work could affect public understanding and their own reputations \cite{lorono2018responsibility, 10.1145/3479566}. Scientists often feel conflicted over their personal and professional identities when engaging in science communication on social media \cite{koivumaki2020social}. 

% Findings _ writers
We explore how to accommodate different writers' comfort levels when it comes to adopting science communication strategies for social media. We ran an in-depth 2-hour study of writers (n=10) who were given three initial draft options with different narrative structures (\textit{One Example}, \textit{No Example}, and \textit{Many Examples}) and selected one draft to move forwards with. Then, writers saw the selected narrative presented \textit{With Personal Language} and \textit{Without Personal Language}. We found that while many writers maintained their beliefs about how science should be communicated, seeing options side-by-side helped them recognize effective and ineffective science communication techniques, ultimately helping them make editorial decisions regarding what to cut, edit, and merge. Furthermore, when presented with science explanations with and without personal language, scientists often chose to merge lines of dialog between the different conditions. Overall, this shows there is continuous design space for structure and style options in public science communication. Writers can be aided by seeing points in the design space and finding a place they are comfortable with on the continuum. 

We conclude with a discussion of how to mediate between audiences' needs for understanding and engagement and accommodating writers' hesitancy around loss of authority and oversimplification of science.

% For example, an unrelatable example can make a reader uninterested in reading the explanation.  Additionally, even for techniques they did prefer, users had nuanced reasons why some sometimes preferred narratives without that technique. Many aspects of readers' personal experience complicate their preferences for explanation styles. There is no one size fits all approach to science communication on social media. \grace{the reader study seems inconclusive, make the takeaways more clear for how to support writers, last sentence feels flimsy in terms of the takeaways, }

% They worry that personal language will undermine their voice of authority or that subjective phrases will taint the objective science they are describing. They worry that leaning on an example will lead listeners to focus on the example (like bubbles), and not the generalizable principles (Thin Film Interface). Lastly, they worry that by walking through an example, the explanation will become convoluted because the narrative now has two parallel tracks - using the example to illustrate the topic, and explaining the technical scientific components clearly.  These fears are legitimate - as with most public communication, there is always a tension between informing and engaging audiences. 


% For example, writers might start with a draft with personal language, but then copying lines from the draft without personal language style when the personal language was excessively dramatic. 

% We found that whereas writers still maintained many of their fears of sounding too informal, by seeing options side by side, they o. 





% They are opposite of trad science writing in two ways. First, heavy use of examples - often examples the audience will connect with. Two, personal language in the narrative. Emotional, “I’. SAY LOTS, give examples..

% Previous research has found that science communication on social media is different from traditional science communication in two ways: the use of examples and the use of personal language and the incorporation of emotions [cite katy]. \grace{finish this}

% However, scientists are often unfamiliar with how to engage in and communicate on these new online spaces. Scientists are taught how to communicate with their peers, who are experts in their field, through research papers and conferences, but often struggle with translating these communication techniques to an everyday audience that might not understand technical jargon or have all the prerequisite knowledge to understand a complex topic [cite]. Therefore, it is important to understand social media specific techniques for science communication and how they might differ from traditional science communication techniques, in order to better support scientists in engaging effectively with the public.




% \subsection{Research Question / Problem}
% \begin{itemize}
%     \item However, when STEM experts are told to write these… they often struggle for two reasons: difficulty in producing this type of narrative. But more importantly, struggling emotionally to decide whether this type of narrative is good for readers.
%     \item By using example in writing, topics seems very narrow.
%     \item By using personal language, the author loses their authority, their sense of objectivity. And science should be objective.
%     \item So, we help STEM writers answer the question, but asking readers what they prefer.
% \end{itemize}

% The RQS are:
% \begin{enumerate}
%     \item We seek to understand what audiences prefer.
%     \item We seek to understand what writers prefer.
% \end{enumerate}

% \subsection{Approach/Theory}
% \begin{itemize}
%     \item Our results show the relationship between narrative structure and personal language is complicated.
%     \item LOTS MORE.
%     \item Our study of STEM expert writing with AI show that …. seeing various narrative structure helps with X and showing with and without example helps with Y.
% \end{itemize}



% \grace{CUT}
% For the audience impressions survey, we recruited college-age, non-STEM students to read science narrative and rate them. We chose to focus on non-STEM students because we are interested in how science communication can engage  people who might not be interested in STEM to begin with. In order to understand the role that structure and style have on readers, we used the following method to evaluate each of these approaches to science communication. 

% To understand structure, we used two different prompts for GPT-4 to elicit two different explanation structures. The first explanation structure follows a step-by-step walkthrough of a singular example to explain a topic. This explanation style is aligned with traditional science communication techniques that typically incorporate a singular narrative explanation. The second explanation structure uses a list-like approach to describe different dimensions of a topic independently–multiple examples may be used to help explain how an aspect works. This explanation style explores whether audiences might prefer an alternative method of explanation that focuses on breadth of topics covered. Because there are structural differences in the explanation method, we had to engineer two different prompts to elicit these different explanation styles. By testing these two different methods of explanations, audiences were able to express their structural preferences for science narratives. 

% In order to understand the style aspect, we used GPT-4 to generate a base narrative that included both an example and personal language. Then, we prompt GPT-4 to remove the given example from the base narrative, resulting in a technical explanation with personal language. We compare this condition to the base narrative to see whether audiences prefer explanations with or without explanations. Similarly, we prompt GPT-4 to remove the personal language from the base-narrative to generate the formal explanation that uses one example in its explanation to understand whether audiences prefer science explanations with or without personal language. Because we are only evaluating the style of explanation, we used a “Remove X” from the base narrative, where X is either the example or personal language, to maintain consistency in structure across generations. In this way, we were able to evaluate audience preferences for just the style aspect without affecting the structure of the explanation.

% Additionally, we run a study with 6 STEM experts to write their own science narratives on social media to determine whether seeing side-by-side comparisons of different structural and stylistic techniques helped experts choose a direction or approach to use when writing their own science narratives. STEM experts first saw the different structural  


% \subsection{Results}
% \begin{itemize}
%     \item Our results show the relationship between narrative structure and personal language is complicated.
%     \item LOTS MORE.
%     \item Our study of STEM expert writing with AI show that …. seeing various narrative structure helps with X and showing with and without example helps with Y.
% \end{itemize}
