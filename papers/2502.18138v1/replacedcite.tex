\section{Related Work}
\textbf{Echo Chambers Modeling.} The study of opinion dynamics and echo chambers in social networks has a rich history, with foundational models such as the Friedkin-Johnson Dynamics Model ____, which simulates opinion evolution based on intrinsic beliefs and social influence, and the Bounded Confidence Model by Deffuant et al. ____, which restricts interactions to individuals within a confidence range. More recent works, such as Sasahara et al. ____, emphasize the dynamic interplay of opinion polarization and structural adjustments, such as unfollowing, and demonstrate how these behaviors accelerate the formation of echo chambers. Our work builds on these models by incorporating LLMs to simulate user behavior and opinion dynamics, leveraging advanced natural language understanding to provide deeper insights into the formation and evolution of echo chambers.

\textbf{LLM-Driven Social Simulation.} Leveraging LLMs for social simulation has emerged as a novel research direction, enabling both downstream task facilitation and a deeper understanding of LLM capabilities. Early work, such as Generative Agents ____, explored LLM-empowered agents that simulate human behavior in interactive environments. This idea was extended to social networks with systems like S3 ____, which simulate emotions, attitudes, and interactions to study population-level phenomena like information and emotion propagation. Furthermore, there have been attempts to use LLMs for simulating echo chambers ____, which differ from our approach in two key aspects: we argue that echo chambers partially stem from changes in network structures, and we emphasize that simulating echo chambers does not imply that more polarization is always better. Instead, we advocate for validating the performance of LLM-based simulations through comparisons with real-world data. %Further, techniques like Social Simulacra ____ employed LLMs to generate posts and reply chains, simulating community behaviors under different design scenarios. These studies lay the foundation for using LLMs as realistic proxies in modeling opinion dynamics and social interactions on online platforms. Building on this foundation, our work incorporates real-world data to trace the evolution of echo chambers, providing a quantitative understanding of opinion dynamics and their progression over time.