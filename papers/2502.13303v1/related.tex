\section{ Related Works} 
\label{sec: related}

Time synchronization is a critical component for a large number of network services and applications. Physical and MAC layer protocols often rely on accurate timing to efficiently utilise a shared medium, effectively communicate, and maximize performance. Similarly, higher-level applications, such as localization and tracking applications, require time synchronization to deliver reliable services. CTI inhibits nodes from synchronizing clocks and, this, in turn, not only leads to a deterioration of quality of service, but also further exacerbate interference.

In \cite{Reitz_FMCW_interference_2024}, the authors investigate the impact of device imperfection (slight variations in chirp forms, attributable to different hardware properties such as clock drift) on mutual interference in radar sensors. The authors remark that one of the most critical challenges in the mitigation of this type of interference is the lack of synchronization in terms of both time and frequency. This challenge, compounded by variations in operational parameters (frequency slope, up, down, inter-chirp duration, sweep bandwidth, and Analog-to-Digital Converter (ADC) sampling time), leads to a complex interference behaviour. Specifically, their investigation reveals that the impact of parallel chirp interference, as minor deviations in the chirp slope, leads to a significant Intermediate Frequency (IF) interference, characterized by considerable interference power. In \cite{guo_imperfect_synchronization_2020}, the authors investigate the impacts of imperfect synchronization and channel estimation on known interference cancellation. The authors propose two different models to normalize time and frequency synchronization errors. Accordingly, the normalized time synchronization error is modelled as inter-symbol interference; and frequency and phase errors, as inter-frequency interference. The known interference rejection ratio is derived for the multipath block fading channel by jointly considering incomplete time-frequency synchronization and channel estimation as well as phase noise.

The work in \cite{Luz_lorawan_block_interference_2020} focuses on Long Range (LoRa) networks. The authors experimentally determined the extent to which a LoRa channel is vulnerable to jamming by an electromagnetic interference, and propose a model to estimate an immunity region. The study is closer to ours in that it, too, statistically analyzes link quality metrics (packet loss, RSSI, and SNR) to characterize interference and network performance. Experimental results show that RSSI values during interference are typically higher than values without interference. Our experiment results agree with theirs. In \cite{Tan_uav_tsynchronization_2020}, the authors propose a UAV-assisted, low-power consumption, time synchronization algorithm based on cross-technology communication (CTC) \cite{kim2015freebee} for a large-scale wireless sensor network. The algorithm enables the UAV to employ a high-power to broadcast time synchronization packet to the ground nodes. The implementation and field experiment consisted of 30 low-power RF-CC2430 nodes and a DJI M100 UAV on a 1 km highway and an indoor site. The experiment results suggest  that time synchronization could be achieved with a synchronization error below $30$ us. This result agrees with our experimental result for the case of a modest CTI. %, $P\{\mathbf{c} \leq 30.5 \ us\} = 0.41$.

In \cite{Gao_tsynch_cti_2023}, the authors investigate the problem of time synchronization among coexisting heterogeneous technologies such as WiFi, ZigBee, and Bluetooth. The authors propose a time synchronization strategy which relies on CTC and enables WiFi devices to assist ZigBee devices during time synchronization. Accordingly, a nearby WiFi device acts as a coordinator node and broadcasts timestamps to the ZigBee devices. The ZigBee nodes update their clocks periodically depending on these timestamps. In addition, the WiFi and ZigBee devices negotiate for spectrum allocation to reduce the impact of CTI. When receiving a timestamp from a WiFi coordinator, a low-power node enters into clock calibration; adjusts its local clock accordingly; and handles interference through negotiation. Both analytic and experiment results show that the proposed approach achieves a global time synchronization with a time error lower than $50$ us.

In \cite{Salazar_Lopez_gps_synchronization_2024}, the authors propose a combination of Pulse Per Second (PPS) synchronization with the help of a GPS  and real-time clocks (RTC) for structural health monitoring. The PPS signal provided by a GPS is used as a trigger signal for data synchronization. The authors observe that PPS, alone, however, is not sufficient, as there can be nodes in the network which achieve   exact synchronization and others which are not synchronized due to various reasons (data corruption and other environmental factors), in which case, peer-to-peer synchronization using RTC is employed to achieve network-wide synchronization. 
