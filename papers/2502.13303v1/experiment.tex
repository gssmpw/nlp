\section{The Impact of CTI on Link Quality}
\label{sec:lq}

\begin{figure}[!ht]
	\centering
	\includegraphics[width=0.45\textwidth]{wifi_Ch1_6_11_spectrum_and_Bluetooth.pdf}
	\caption{Spectrum occupation by the WiFi network using Channels 1, 6, and 11. The statistics were obtained by scanning the entire ISM Band at a rate of 100 KHz using R\&S PR100 Portable Radio Receiver. Bluetooth and IEEE 802.15.4 utilization are completely masked by the occupation of the three non-overlapping WiFi channels.}
    \label{fig:Blue_wifi_ch1_6_11_spectrum}
    \end{figure}

\begin{figure}[h!]
	\centering
	\includegraphics[width=0.45\textwidth]{wifi_Ch11_zigBee_Bluetooth_spectrals.pdf}
	\caption{Spectrum occupation by (1) the WiFi network (using Channel 11); (2) an IEEE 802.15.4 (Channels 21, 22, 23, 24) low-power network; and (3) an IEEE 802.15.1 transmitter.}
    \label{fig:Blue_wifi_ch1_11_zigBee_spectrum}
    \end{figure}

Fig. \ref{fig:Blue_wifi_ch1_6_11_spectrum} displays the power spectrum of the three widely used  none-overlapping WiFi channels (1, 6, and 11) when occupied at the same time. The figure clearly shows that the available spectral space between the occupied channels is useful only partially, because the spectral sidelobes extend into this space. It also suggests that the receivers' sensitivity of the low-power network should be appreciably high (above -90 dBm) in order to successfully receive packets destined to them. Fig.  \ref{fig:Blue_wifi_ch1_11_zigBee_spectrum} shows the spectral overlap between Channel 11 of the IEEE 802.11 network and four IEEE 802.15.4 channels (Channels 21-24), when the two networks operate simultaneously. The figure displays the spectrum  of additional, remote networks, which were not a part of the experiment setup. From these two figures it is clear that (1) CTI is a significant challenge for low-power networks and (2) dynamic channel selection should take into consideration the channel utilisation characteristics of both nearby and far away 802.11 networks.

The quality of a wireless link depends on many factors, including the ambient temperature, the noise factor of the receiver, which measures the amount of noise generated inside the receiver, and the signal-to-noise ratio (SNR) of the received signal. The thermal noise power at room temperature (25$^{\circ}$ C or 298 K) in a 1 Hz channel is -204 dBW or -174 dBm/Hz. This value is the reference for any noise power calculation when designing RF systems working at room temperature. The receiver's sensitivity, $\sigma$ (in dBm) takes all these terms into account:
\begin{equation}
   \label{eq:sensitivity}
  \sigma \; (dBm)  = kTB \; (dBm) + SNR \; (dBm) + NF \; (dBm) 
\end{equation}
where $kTB$ is the thermal noise and depends on Boltzmann’s constant, k, in Joules/K, the ambient temperature in degree Kelvin, and the bandwidth of the channel selective filter in the receiver (B). NF is the noise factor of the receiver. For a 2 MHz channel; a SNR of 6 dBm; and a noise factor of 2 dBm, the receiver's sensitivity equals -103 dBm. The sensitivity sets a limit to the minimum SNR of the received signal. In most practical cases, however, a much higher threshold is set to the received power in order to avoid the detection of spurious or corrupted signals.             

A closer examination of the statistics we established from the raw measurements reveal that $93.53\%$ of all the detected signals on the IEEE 802.15.4 Channel 11 (and $94.58\% $, on Channel 12) have RSSI values below the noise floor of the CC2538 radio. In other words, these channels are relatively free of interference. The figure is comparable for Channel 13 and 14; on the former, 4.37\%  and, on the latter, 3.2\% of all the detected signals have a received power exceeding the noise floor of the CC2538 radio. By contrast, the received power detected on Channels  21 to 24 has on average $-60$ dBm, signifying a stark interference. An interesting observation concerns channels 15, 20, 25, and 26. Theoretically, these channels should be the least affected by CTI. In practice, however, this is not the case. The spectral sidelobes of Channels 1, 6, and 11 interfere with the low-power channels. Thus, Channel 15 is affected by the sidelobes of the WiFi Channels 1 and 6; Channel 20 is affected by the sidelobes of the WiFi Channels 6 and 11; and Channels 25 and 26 are affected by the sidelobes of the WiFi Channel 11. From this it can be concluded that the CTI arising from the sidelobes of the IEEE 802.11 channels is considerable. For example, if the detection threshold of the CC2538 radio is $-100$ dBm and the WiFi Channel 11 is active, the interference probability on the low--power channels is as follows: Channel 21 ($97\%$),  Channel 22 ($96.28\%$), Channel 23 ($96.20\%$), and Channel 24 ($96.44\%$).  When the detection threshold is raised to $-83$ dBm,  CTI is reduced significantly, now the interference probability becomes as follows: Channel 21 ($1.94\%$), Channel 22 ($2.51\%$), Channel 23 ($2.67\%$), and Channel 24 ($1.94\%$). 

