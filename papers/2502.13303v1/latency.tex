\section{Network Join Time}
\label{sec:latency}
One of the desirable qualities of wireless sensor networks is self-organisation: Nodes can independently establish a multi-hop network; new nodes can join the network; and nodes, which leave the network for various reasons, can rejoin. The MAC protocol plays a key role during self-organisation. As stated above, in TSCH, the coordinator node broadcasts EB packets regularly. A node wishing to join the network listens to these packets, and upon receiving one, contends for the medium and sends a request-to-join packet using a unicast channel. Because initially no time synchronisation does take place between the coordinator and the nodes joining the network, almost certainly there is a time difference between the coordinator and these nodes. Expecting this condition, TSCH defines two types of time offsets. The first offset is intended to prevent a receiver from early sleeping in case a packet (preamble, SFD, Headers, payload, FCS) does not arrive according to the receiver's local time. The second time offset is intended for a transmitter to receive a delayed acknowledgement packet.

    \begin{figure}[h!]
        \centering
        \includegraphics[width=0.45\textwidth]{latency_no_interference.pdf}
        \caption{The histogram of network join latency during a moderate CTI.}
        \label{fig: latency_with_interference}
    \end{figure}  
            
    
    \begin{figure}[h!]
        \centering
        \includegraphics[width=0.45\textwidth]{latency_with_interference.pdf}
        \caption{The histogram of network join latency during an high CTI.}
        \label{fig: latency_no_interference}
    \end{figure}  

When the network is under the influence of a CTI, as we already discussed in Section~\ref{sec:ts}, the time drifts between the coordinator and the new nodes increase and the two time offsets are not sufficient to establish a reliable communication. This creates a join delay. We measured the join delay with and without a CTI. The distributions of these delay are given in Figs. \ref{fig: latency_with_interference} and \ref{fig: latency_no_interference}.   Without CTI, $83.3\%$  of the case, the join delay  is between $40$ and $70$ ms, whereas in the presence of CTI, $96.82\%$ of the time, the join delay is between $100$ and $200$ ms. In other words, the join delay in the presence of CTI is about five times higher than without CTI. Additionally, the maximum join delay we observed during CTI was $800$ ms, whereas it was $200$ ms when there was no CTI.     