\section{Background}
\label{sec:background}

CTI affecting low-power sensing networks has been studied mainly with respect to interference arising from IEEE 802.11 (WiFi) and IEEE 802.15.1 (Bluetooth) technologies \cite{chi2019concurrent, li2017webee}. The type of networks these technologies typically establish require transmission ranges in the order of a few hundred meters. Such networks can be found in residential, industrial, and office environments \cite{grimaldi2020autonomous, elias2014cross}. More recently, autonomous systems complying with IEEE 802.11 standards and requiring longer transmission ranges have been populating the market. These include Unmanned Aerial Vehicles (UAV) and Unmanned Surface Vessels (USV). These systems rely on wireless links to communicate with their remote control stations as well as their peers. Their transmission range far exceeds that of conventional WiFi networks. Moreover, they have stringent safety requirements, which necessitate the use of a high transmission power and reliable communication.

Interestingly, these systems can be jointly deployed with wireless sensor networks to monitor remote and vast areas. For example, in water quality monitoring, the sensor networks can be deployed on the surface of restless waters; and the UAVs and the USVs can be deployed to configure the sensor networks and to collect data from them. Therefore, investigating the effect of CTI on the coexistence and collaboration of such systems is important.  

\begin{figure}[t!]
	\centering
	\includegraphics[width=0.45\textwidth]{cti_bbc.pdf}
	\caption{A USV producing cross technology interference. Deployment: North Biscayne Bay, South Florida.}
	\label{fig:mmu}
\end{figure}

\subsection{Field Deployments}

In order to have a first hand experience of the impact of CTI, we carried out deployments involving USVs and wireless sensor networks. The first deployment took place in North Biscayne Bay, Miami, Florida. The deployment consisted of a USV\footnote{A product of SeaRobotics Corporation, which used the IEEE 802.11/b for communicating with its remote station: https://www.searobotics.com/} and a network of 6 wireless sensor nodes. The nodes established a multi-hop network using a 2.4 GHz radio (CC2538). The distance between them was about 50 m. When the USV was not around, the nodes communicated with one another with a modest amount of packet loss (less than 10\%). When, the USV was within 300 m radius or so, communication was considerably constrained; now packet loss varying from 30 to 100\%. We tested all the 16 available IEEE 802.15.4 channels to avoid CTI, but performance remained poor. 

The second deployment took place at one of the lakes on Florida International University's Main Campus. This time, we deployed 5 wireless sensor nodes on the surface of the lake and an additional sensor node on the USV itself. Both this node and the nodes deployed on the surface of the lake communicated with a gateway node placed outside the lake (ref. to Fig.~\ref{fig:mmu}). The present USV had a more complex setup than the one deployed in North Biscayne Bay. It communicated with its remote control station using a proprietary transceiver, operating in the 4.9-5.8 GHz band, but in addition, the control station was remotely controlled by a human agent using the IEEE 802.11b standard. When the USV navigated autonomously, both the node deployed on the boat and on the lake experienced no interference and the link quality was stable; as soon as a human agent interacted with the boat using the IEEE 802.11b interface, all the nodes experienced a significant CTI. The node which was affected the most was the one carried by the autonomous boat. Fig.~\ref{fig:cti_fiu} shows the link quality of this node, as reflected by the RSSI of the packets it received from  the base station.    

\begin{figure}[t!]
	\centering
	\includegraphics[width=0.45\textwidth]{mmu_cti.pdf}
	\caption{Deployment of a USV and a Wireless Sensor Network at one of the Lakes on Florida International University main campus. }
	\label{fig:mmu}
\end{figure}

\begin{figure}[t!]
	\centering
	\includegraphics[width=0.45\textwidth]{cti_fiu_rssi.pdf}
	\caption{Link quality fluctuation in the presence of CTI (Deployment on a lake on FIU's Main Campus).}
	\label{fig:cti_fiu}
\end{figure}

In order to separate the effect of CTI from the effect of the motion of water on the link quality of the low-power networks, we repeated the field experiments in a controlled environment. In this setup, the low-power network consisted of three wireless sensor nodes, one of which served as a coordinator. In addition, we had two laptops and three smartphones. One of the laptops served as a programming and configuration platform as well as a gateway to the Internet. It ran \texttt{pySerial}\footnote{\begin{url}https://pyserial.readthedocs.io/en/latest/pyserial.html\end{url}.}, for executing python scripts and facilitating a serial communication with the coordinator. We established a database management system using \texttt{phpMyAdmin}\footnote{https://www.phpmyadmin.net/} and \texttt{MySQL} to export and store link quality metrics. The second laptop and the three smartphones were used to generate and induce interference.

In both types of experiments, the sensor nodes integrated the CC2538 system-on-chip \cite{cc2538_soc_2013}. The radio on this chip complies  with the IEEE 802.15.4 specification. It has 16 available channels (numbered from 11 to 26); each channel has a bandwidth of 2 MHz and a 5 MHz guard band separates adjacent channels. The radio uses the 2.4 GHz license-free spectrum; it has an adjustable transmission power (the maximum transmission power being 7 dBm), and a transmission rate of 250 kbps. The Contiki operating system \cite{Contiki-NG-2022} was used as the operating system to manage the sensor nodes and the protocols they ran.

\begin{figure}
	\centering
	\includegraphics[width=0.45\textwidth]{tsch1.pdf}
	\caption{Illustration of slot and channel assignment in TSCH.}
	\label{fig:tsch}
\end{figure}

\subsection{Channel Assignment}

The 3 non-overlapping channels mostly used by IEEE 802.11 networks are Channels 1, 6, and 11. These channels overlap with some of the IEEE 802.15.4 channels. Channel 1 overlaps with Channels 11 to 14; Channel 6, with Channels 16 to 19; and Channel 11, with Channels 21 to 24. The three non-overlapping IEEE 802.11 channels are rarely used simultaneously, in which case, the IEEE 802.15.4 networks can employ dynamic channel selection/channel-hopping to minimize CTI. Furthermore, Channels 15, 20, 25, and 26 of the IEEE 802.15.4 networks are the least affected by IEEE 802.11 channel assignment and this knowledge can be useful during dynamic channel selection.

The Time-Slotted Channel Hopping (TSCH) MAC protocol \cite{dujovne20146tisch} is one of the widely used MAC protocols to mitigate the impact of CTI on IEEE 802.15.4 networks. It combines time-division medium access and dynamic channel hopping. In TSCH, a node may transmit a single packet only using a specific channel. Regardless of its success or failure, a transmitter selects a different channel in the next time slot for transmitting the next packet (ref. to Fig. \ref{fig:tsch}). Similarly, a node may transmit a packet only in a time slot assigned to it. It may be assigned multiple time slots in succession. TSCH requires strict time synchronisation to coordinate channel assignment and channel hopping. Fig.~\ref{fig:tsch3} illustrate the importance of time synchronization in TSCH.

\begin{figure*}[t!]
	\centering
	\includegraphics[width=0.8\textwidth]{tsch3.pdf}
	\caption{Illustration of slots and channel assignment in TSCH.}
	\label{fig:tsch3}
\end{figure*}

At the beginning of a time slot, a transmitter prepares a packet for transmission. The time elapse between the beginning of a time slot and the beginning of a clear channel assessment is called \texttt{TsCCAOffset}. If the channel is free, the transmitter switches to a transmission mode (it is on a receiving mode during clear channel assessment) and sends a packet. Then it switches back to a receiving mode to receive the acknowledgement packet. This packet has to arrive within a fixed amount of time, called Acknowledgement Guard Time or \texttt{AGT}. If it does not arrive, the transmitter considers the packet as a lost packet and awaits the next time slot to re-transmit the packet using a different channel.

 
