\section{Time Synchronization}
\label{sec:ts}

Time synchronization is a mechanism by which a common time concept amongst the sensor nodes belonging to the same network is established \cite{wu_clock_synch}. For medium access control protocols relying on the assignment of time slots, time synchronization plays a crucial role, since the nodes should have a shared understanding of the beginning and the end of a slot assignment \cite{romanov2020precise}. Similarly, higher layer services and applications often rely on it to function properly \cite{bhushan2020requirements}. 

Each individual node in a network maintains its own clock generator, which, due to imperfections in the local crystal oscillator, experiences clock offset and skew. The clock function of a sensor node with respect to a reference time can be expressed as : $C(t) = \theta \ + \beta t$, where $\theta$ is the phase difference or clock offset and $\beta$ is the frequency difference or the clock skew of the crystal oscillator. If one considers the relative clock offset and skew between a child node $n$ and a sink node $s$, the local clock  and its relationship to the clock of the sink can be expressed using equation \ref{equ:clock_offest}:

                \begin{equation}                 
                C_{n}(t) = \theta^{sn}+\beta^{sn}C_{s}(t)                                       
                \label{equ:clock_offest}         
                \end{equation}
where $C_{s}(t)$ is the sink's clock. If the two clocks are perfectly synchronized, the relative phase difference or the relative clock offset is zero ($\theta^{sn} = 0 $ and  $\beta^{sn}=1$), otherwise the child node needs to synchronize its clock using the time information (such as timestamps) it receives from its reference node. Graphically, the relationship between the reference and the local clocks can be illustrated using Fig. \ref{fig:clock_model}. The local clocks increase monotonically in accordance with their clock skew slopes, away from each other when no time synchronization occurs.

\begin{figure}[h!]
	\centering
	\includegraphics[width=0.45\textwidth]{clock_model_graph.pdf}
	\caption{A clock model for two sensing nodes (a sink and a child node) directly communicating.}
\label{fig:clock_model}
\end{figure}

When the network is stable and the wireless links are reliable, time synchronization takes place in regular intervals. When this is not the case, clock drifts become nonlinear and the time synchronization intervals are no longer regular. In TSCH, nodes exchange a so-called ``Extended Beacon'' (EB) on a regular basis. This packet contains ``information elements'' -- timing, channel assignments, time slots, slot frames and other preambles and header files. This information is critical for existing nodes to communicate with one another and for new nodes wishing to join the network to learn about channel and slot assignments. 

The time deviation between two nodes can be calculated using two different techniques. In the first, the coordinator broadcasts EB packets and a child node determines the  deviation from the timestamps embedded in these packets. In the second (so called ``two-way approach''), the coordinator determines the same from acknowledgement packets and sends the correction along with other timing information to the child  node. This is illustrated in Fig. \ref{fig: two_nodes_time_synch}. Once the network's topology is determined through a higher-level self-organization algorithm, a child node sends its time information to the coordinator at time $T_{1}$; the coordinator node records the arriving time $T_{2}$ and sends a reply  at time $T_{3}$. There is a process delay between  $T_{1}$ and  $T_{3}$. This delay is included in $T_{3}$.  Finally, the child node records the time at which it receive the correction packet at time stamp $T_{4}$. 

% Drawing the two-way time meassage exchange between two nodes
    %--------------------------------------------------------------------------------
     \begin{figure}[h!]
        \begin{center}      
            \begin{tikzpicture}
                \node at (2,2) (nodeA) {s};
                \node at (2,4) (nodeB) {n};
                \draw[red] (nodeA) circle (2mm);
                \draw[blue] (nodeB) circle (2mm);
                \tikzstyle{arrow} = [thick,->,>=stealth]
                \draw [arrow] (nodeB) -- (9,4);
                \draw [arrow] (nodeA) -- (9,2);
                \draw [arrow] (2.4,4) -- (3.2,2); % T1 to T2
                \draw [arrow] (3.4,2) -- (4.2,4); %T3 to T4
                \draw [arrow] (4.4,4) -- (5.2,2); %The second round
                \draw [arrow] (5.4,2) -- (6.2,4); % the second round                
                \draw [decoration={text along path, text={n local time},text align={center}},decorate]  (6.2,4) -- (9,4);
                \draw [decoration={text along path, text={s local time},text align={center}},decorate]  (6.2,2) -- (9,2); 
                \draw [decoration={text along path, text={---},text align={center}},decorate]  (6.2,3) -- (7.2,3); 
                \draw [decoration={text along path, text={T1},text align={center}},decorate]  (2.2,4) -- (2.8,4); 
                \draw [decoration={text along path, text={T2},text align={center}},decorate]  (2.8,1.7) -- (3.4,1.7);
                \draw [decoration={text along path, text={T3},text align={center}},decorate]  (3.4,1.7) -- (3.8,1.7);
                \draw [decoration={text along path, text={T4},text align={center}},decorate]  (3.8,4) -- (4.2,4); 
            \end{tikzpicture}        
        \end{center}
        \caption{ Illustration of a two-way timing message exchange between a sink and a child node during time synchronization measurement: $T_{2}=T_{1}+delay+offset$ and $T_{4}=T_{3}+delay-offset$. The simple time offset between the sink and the child node can be calculated as $[(T_{2}-T_{1})-(T_{4}-T_{3})]/2$}
        \label{fig: two_nodes_time_synch}
    \end{figure}  

Using Equation \ref{equ:clock_offest} and Fig.\ref{fig: two_nodes_time_synch}, the local time of the coordinator and the child node can be determined as follows \cite{wu_clock_synch},\cite{Shi_interference_wifizigBee_2017}:

   \begin{equation}
                 \begin{split}
                    T_{2} = \theta+\beta(T_{1} + \tau + \delta ): \ client \ time \ stamp \\
                    T_{3} = \theta+\beta(T_{4} - \tau - \sigma ): \ sink \ time \ stamp  \\                   
                 \end{split}                     
                \label{equ:timestamp_clock_offest}         
                \end{equation}
Where $\beta$ and  $\theta$ the are relative clock skew and offset, respectively, of the child node with respect to the coordinator; $\tau $ is a fixed network delay; $\delta$ and  $\sigma$ are variables related to transmission delay (send, receive, process, etc.) at the coordinator and the child node, respectively. The three parameters -- $\beta$, $\theta$ and $\tau $ -- are factors, which determine time synchronization frequency. When there is a significant packet loss, their values increase and the nodes make frequent but unsuccessful attempts to synchronise time.               

      \begin{figure}[h!]
            	\centering
            	\includegraphics[width=0.45\textwidth]{time_drift.pdf}
            	\caption{The histogram of the time drift of a child node with respect to a sink node between two time synchronisation intervals. The statistics are established for an observation duration of 1 hour. The WiFi network producing the CTI was occupying Channel 11. The interference reaching the low-power network had an average magnitude of $-83$ dBm.} 
            \label{fig: time_drift_interference_1}
            \end{figure}

The Contiki operating system offers a set of timer libraries. The underlying clock module provides system time as a 32-bit unsigned integer. Time drift corrections are made in terms of clock ticks at a drift rate at ppm and are expressed as a 32 bit integer. Typically 1 clock tick can vary between 1 and 10 milliseconds. TSCH sends out Extended Beacon (EB) every second. In the absence of a considerable CTI, the relative time drift between two synchronization attempts is approximately 1 rtimer tick (between $-1$ and $1$ ticks). In order to investigate the impact of CTI, we considered two power levels, namely, a high CTI, $-30$ dBm, and a low CTI,  $-86$ dBm. In the case of a high CTI, the time drift between two successful synchronization attempts increases, now extending between $-3$ and $3$ ticks. Fig.~ \ref{fig: time_drift_interference_1} shows the time drift histogram between the child nodes and the coordinator during a 1 hour observation window. In the case of a low CTI, the time drift distribution exhibited a slight improvement. Even though the shape of the distribution is similar to the previous case, the variance becomes smaller, as can be seen in Fig.~\ref{fig: time_drift_interference_2}. Hence, in the case of a low CTI, $P\{\mathbf{c} \geq 61.02 \ us\} = 0.07$, whereas in the case of a high CTI, $P\{\mathbf{c} \geq 61.02 \ us\} = 0.1$, where $\mathbf{c}$ is the relative clock drift regarded as a random variable. Similarly, in the case of a modest CTI, $P\{\mathbf{c} \leq 30.5 \ us\} = 0.41$, whereas in the case of a high CTI,  $P\{\mathbf{c} \leq 30.5 \ us\} = 0.39$. In general, in the presence of CTI, on average, $36.40\%$ additional synchronization messages were exchanged in 1 hour in an attempt to keep the relative time drift within $\pm 1$ rtimer tick. The more synchronization massages are exchanged, the higher is the energy cost of time synchronization. 

\begin{figure}[h!]
    \centering
    \includegraphics[width=0.45\textwidth]{time_drift_no_interference.pdf} 
            	\caption{The histogram of the time drift of a child node with respect to a sink node between two time synchronisation intervals. The statistics are established for an observation duration of 1 hour. The WiFi network producing the CTI was occupying Channel 11. The interference reaching the low-power network had an average magnitude of $-30$ dBm.} 
    \label{fig: time_drift_interference_2}
\end{figure}  