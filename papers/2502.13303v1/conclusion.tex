\section{Conclusion}
\label{sec:conclusion}

In this paper we investigated the impact of CTI on time synchronization and network join time of low-power sensing nodes. Medium Access Control protocols which are designed to deal with CTI often rely on time synchronization and dynamic channel hopping. In the presence of CTI, however, packets containing synchronization as well as slot and channel assignment information may be subject to corruption. This condition further exacerbates slot and channel occupation and causes further mutual interference. The study shows that in the absence of a CTI, the clock drift between two synchronization intervals (1 s) is within $\pm 1$ clock tick. when the average CTI reaching a low-power network is about $-86$ dBm, i.e., a moderate CTI, the clock drift between two synchronization intervals is between $\pm 2$ clock ticks. When the CTI magnitude increases to $-30$ dBm, the clock drift increases to $\pm 3$ clock ticks. The impact of CTI is even worse on network join latency. Nodes have to receive a join invitation from a coordinator. When the network relies on the Time-Slotted Channel-Hopping (TSCH) protocol, the invitation comes in the form of an Extended Beacon, which, a coordinator node broadcasts every second. For our case, all nodes receive this invitation directly. In the absence of a CTI, the join latency is below 5 clock ticks. In the presence of a moderate CTI, the latency increases to about 7  clock ticks. When the CTI becomes considerable, network join time increases considerably as well, now exceeding 10 clock ticks on average. Our future work is to further address this issue and improve time synchronization in low-power networks operating in the presence of CTI. 