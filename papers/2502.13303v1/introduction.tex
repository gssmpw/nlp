\section{Introduction}
\label{sec:intro}

Cross technology interference (CTI) arises as a consequence of two or more heterogeneous wireless technologies utilizing the same radio spectrum seize the medium simultaneously. These systems do not have a shared medium access control (MAC) protocol to arbitrate between them. Often, the impact of CTI is asymmetric, as the systems rely on different physical layer strategies, the ones using higher transmission powers and more advanced modulation schemes affecting those with lower transmission powers and less advanced modulation schemes more significantly \cite{dargie2024mitigating}. In the context of low-power wireless sensing networks, the most important CTI arises from IEEE 802.11 networks, since both systems rely on the 2.4 GHz, license-free, spectrum \cite{shi2017signal}. 

Different approaches have been proposed to mitigate CTI in this spectrum. Some of these enable low-power networks to estimate the transmission and channel occupation pattern of the more advanced networks to determine least affected channels and best transmission timing \cite{hithnawi2016crosszig}. Other approaches aim to establish back-channels for interfering networks to  convey transmission intentions and coordinate packet transmission \cite{yin2018explicit, guo2020zigfi}. But one of the most widely used and practical solutions is dynamic channel/frequency hopping \cite{hermeto2017scheduling}. Like many other technologies, such as the IEEE 802.15.1 technologies  \cite{golmie2003bluetooth}, dynamic channel hopping entails dynamic channel selection, which requires exact timing and coordination between communicating partners, so that both the  transmitter and the receiver transit from a present channel to a future channel in sync \cite{chang2015adaptive, duquennoy2017tsch}. This feature makes the very solution which deals with CTI, vulnerable to it, since in the presence of CTI, time synchronization becomes a serious challenge. In this paper we investigate this challenge experimentally. As the first contribution of this paper, both through field deployments and experiments in controlled environments, we investigate the impact of CTI on the link quality of low power networks, closely examining the impact in all the available channels for the case of IEEE 802.15.4 networks. As a second contribution, we closely examine the  impact of CTI on the synchronization and network join latency, assuming that the IEEE 802.15.4 networks employ the Time-Slotted and Channel Hopping (TSCH) medium access protocol \cite{tinka2010decentralized}. 

The rest of this paper is organised as follows: In Section \ref{sec: related}, we review related work. In Section \ref{sec:background}, we provide background information and discuss field deployments and experiments setups. In Section \ref{sec:lq} we address the impact of CTI On link quality in low-power networks. In Section \ref{sec:ts}, we examine the impact of CTI on time synchronization, and in  Section \ref{sec:latency}, on self-organization. Finally, in \ref{sec:conclusion}, we provide concluding remarks and outline future work.

