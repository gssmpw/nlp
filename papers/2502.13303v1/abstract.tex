\begin{abstract}   
Low-power and low-cost wireless sensor networks enable scalable and affordable sensing and can be deployed in different environments to monitor various physical parameters. In some environments, these networks may have to coexist and interact with other systems which use the same frequency spectrum for communication. This potentially results in cross-technology interference (CTI). Dynamic channel hopping is one of the mechanisms that is currently employed to deal with CTI, but its usefulness depends on the channel selection and occupation timing. In this paper, we experimentally study the impact of CTI (caused by IEEE 802.11 networks) on time synchronization and network join time. Experiment results show that CTI can increase time drift between a child and a parent node by up to $\pm 3$ clock ticks between two synchronization intervals. Likewise, CTI affects new nodes from timely joining a network. In a simple network which does not involve multi-hop communication, the time it takes for nodes to join the network in the absence of CTI is between 40 and 70 ms (83.3\% of the time). In the presence of CTI, 96.82\% of the time, the join time is between 100 and 200 ms. In other words, the join time in the presence of CTI is about five times higher. Interestingly, not only the main spectral lobes, but also the spectral sidelobes of interfering networks impact the performance of low-power networks. 
\end{abstract}