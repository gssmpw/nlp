\section{Related Work}
\subsection{Prompt Injection Attacks}
Prompt injection attacks pose a critical challenge for Large Language Models (LLMs) and have garnered significant research attention  ____. ____ explore the use of an ``ignoring prompt'' which is prepended to the injected instruction to manipulate the models. Similarly, ____ introduces a technique involving the addition of fake responses, tricking the LLMs into believing the user’s input has already been processed, thereby executing the malicious instruction instead. ____ further enhances attack effectiveness by combining multiple attack strategies. Additionally, ____ optimize suffixes to effectively mislead LLMs.

\subsection{Prompt Injection Defenses}
In response to the growing threat of prompt injection attacks, numerous defense mechanisms have been proposed  ____. ____ and ____ suggest appending reminders to reinforce adherence to the original instructions.  ____ and ____ propose using special tokens to clearly delineate the data content area. ____ address the issue by training models to specialize in specific tasks. ____ and ____ advocate fine-tuning LLMs with adversarial training ____, granting privileged status to authorized instructions. Lastly, detection methods ____ have been proposed to address direct prompt injection attacks. However, these methods overlook indirect prompt injection attacks, which are more practical and applicable.