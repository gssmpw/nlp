% This must be in the first 5 lines to tell arXiv to use pdfLaTeX, which is strongly recommended.
\pdfoutput=1
% In particular, the hyperref package requires pdfLaTeX in order to break URLs across lines.

\documentclass[11pt]{article}

% Change "review" to "final" to generate the final (sometimes called camera-ready) version.
% Change to "preprint" to generate a non-anonymous version with page numbers.
% \usepackage[review]{acl}
\usepackage[preprint]{acl}

% Standard package includes
\usepackage{times}
\usepackage{latexsym}

% For proper rendering and hyphenation of words containing Latin characters (including in bib files)
\usepackage[T1]{fontenc}
% For Vietnamese characters
% \usepackage[T5]{fontenc}
% See https://www.latex-project.org/help/documentation/encguide.pdf for other character sets

% This assumes your files are encoded as UTF8
\usepackage[utf8]{inputenc}

% This is not strictly necessary, and may be commented out,
% but it will improve the layout of the manuscript,
% and will typically save some space.
\usepackage{microtype}

% This is also not strictly necessary, and may be commented out.
% However, it will improve the aesthetics of text in
% the typewriter font.
\usepackage{inconsolata}

%Including images in your LaTeX document requires adding
%additional package(s)
\usepackage{graphicx}

%self-added packages
\usepackage{float}
\usepackage{multirow}
\usepackage{booktabs} % For \toprule, \midrule, \bottomrule
\usepackage{makecell} % For \makecell
\usepackage{array}    % For column spacing and alignment
\usepackage{xcolor}
\usepackage{amsmath}
\usepackage{colortbl}
\usepackage{amssymb}
% \usepackage{multirow}
\usepackage{tabularx}
\newcommand{\cyl}[1]{\textcolor{blue}{#1}}
% If the title and author information does not fit in the area allocated, uncomment the following
%
%\setlength\titlebox{<dim>}
%
% and set <dim> to something 5cm or larger.

\title{Can Indirect Prompt Injection Attacks Be Detected and Removed?}

% Author information can be set in various styles:
% For several authors from the same institution:
% \author{Author 1 \and ... \and Author n \\
%         Address line \\ ... \\ Address line}
% if the names do not fit well on one line use
%         Author 1 \\ {\bf Author 2} \\ ... \\ {\bf Author n} \\
% For authors from different institutions:
% \author{Author 1 \\ Address line \\  ... \\ Address line
%         \And  ... \And
%         Author n \\ Address line \\ ... \\ Address line}
% To start a separate ``row'' of authors use \AND, as in
% \author{Author 1 \\ Address line \\  ... \\ Address line
%         \AND
%         Author 2 \\ Address line \\ ... \\ Address line \And
%         Author 3 \\ Address line \\ ... \\ Address line}

\author{
 \textbf{Yulin Chen\textsuperscript{1}\thanks{Yulin Chen and Haoran Li contributed equally.}},
 \textbf{Haoran Li\textsuperscript{2}\footnotemark[1]},
 \textbf{Yuan Sui\textsuperscript{1}},
 \textbf{Yufei He\textsuperscript{1}}\\
 \textbf{Yue Liu\textsuperscript{1}},
 \textbf{Yangqiu Song\textsuperscript{2}},
 \textbf{Bryan Hooi\textsuperscript{1}}
\\
\\
 \textsuperscript{1}National University of Singapore \\
 \textsuperscript{2}The Hong Kong University of Science and Technology 
\\
 \small{
   \textbf{} \href{mailto:chenyulin28@u.nus.edu}{chenyulin28@u.nus.edu}
 }
}

%\author{
%  \textbf{First Author\textsuperscript{1}},
%  \textbf{Second Author\textsuperscript{1,2}},
%  \textbf{Third T. Author\textsuperscript{1}},
%  \textbf{Fourth Author\textsuperscript{1}},
%\\
%  \textbf{Fifth Author\textsuperscript{1,2}},
%  \textbf{Sixth Author\textsuperscript{1}},
%  \textbf{Seventh Author\textsuperscript{1}},
%  \textbf{Eighth Author \textsuperscript{1,2,3,4}},
%\\
%  \textbf{Ninth Author\textsuperscript{1}},
%  \textbf{Tenth Author\textsuperscript{1}},
%  \textbf{Eleventh E. Author\textsuperscript{1,2,3,4,5}},
%  \textbf{Twelfth Author\textsuperscript{1}},
%\\
%  \textbf{Thirteenth Author\textsuperscript{3}},
%  \textbf{Fourteenth F. Author\textsuperscript{2,4}},
%  \textbf{Fifteenth Author\textsuperscript{1}},
%  \textbf{Sixteenth Author\textsuperscript{1}},
%\\
%  \textbf{Seventeenth S. Author\textsuperscript{4,5}},
%  \textbf{Eighteenth Author\textsuperscript{3,4}},
%  \textbf{Nineteenth N. Author\textsuperscript{2,5}},
%  \textbf{Twentieth Author\textsuperscript{1}}
%\\
%\\
%  \textsuperscript{1}Affiliation 1,
%  \textsuperscript{2}Affiliation 2,
%  \textsuperscript{3}Affiliation 3,
%  \textsuperscript{4}Affiliation 4,
%  \textsuperscript{5}Affiliation 5
%\\
%  \small{
%    \textbf{Correspondence:} \href{mailto:email@domain}{email@domain}
%  }
%}

\begin{document}
\maketitle
\begin{abstract}
Prompt injection attacks manipulate large language models (LLMs) by misleading them to deviate from the original input instructions and execute maliciously injected instructions, because of their instruction-following capabilities and inability to distinguish between the original input instructions and maliciously injected instructions. 
To defend against such attacks, recent studies have developed various detection mechanisms. While significant efforts have focused on detecting direct prompt injection attacks, where injected instructions are directly from the attacker who is also the user, limited attention has been given to indirect prompt injection attacks, where injected instructions are indirectly from external tools, such as a search engine. Moreover, current works mainly investigate injection detection methods and pay less attention to the post-processing method that aims to mitigate the injection after detection.
In this paper, we investigate the feasibility of detecting and removing indirect prompt injection attacks, and we construct a benchmark dataset for evaluation. For detection, we assess the performance of existing LLMs and open-source detection models, and we further train detection models using our crafted training datasets. For removal, we evaluate two intuitive methods: (1) the \textit{segmentation removal method}, which segments the injected document and removes parts containing injected instructions, and (2) the \textit{extraction removal method}, which trains an extraction model to identify and remove injected instructions.


\end{abstract}



% Bibliography entries for the entire Anthology, followed by custom entries
%\bibliography{anthology,custom}
% Custom bibliography entries only
%!TEX root = gcn.tex
\section{Introduction}
Graphs, representing structural data and topology, are widely used across various domains, such as social networks and merchandising transactions.
Graph convolutional networks (GCN)~\cite{iclr/KipfW17} have significantly enhanced model training on these interconnected nodes.
However, these graphs often contain sensitive information that should not be leaked to untrusted parties.
For example, companies may analyze sensitive demographic and behavioral data about users for applications ranging from targeted advertising to personalized medicine.
Given the data-centric nature and analytical power of GCN training, addressing these privacy concerns is imperative.

Secure multi-party computation (MPC)~\cite{crypto/ChaumDG87,crypto/ChenC06,eurocrypt/CiampiRSW22} is a critical tool for privacy-preserving machine learning, enabling mutually distrustful parties to collaboratively train models with privacy protection over inputs and (intermediate) computations.
While research advances (\eg,~\cite{ccs/RatheeRKCGRS20,uss/NgC21,sp21/TanKTW,uss/WatsonWP22,icml/Keller022,ccs/ABY318,folkerts2023redsec}) support secure training on convolutional neural networks (CNNs) efficiently, private GCN training with MPC over graphs remains challenging.

Graph convolutional layers in GCNs involve multiplications with a (normalized) adjacency matrix containing $\numedge$ non-zero values in a $\numnode \times \numnode$ matrix for a graph with $\numnode$ nodes and $\numedge$ edges.
The graphs are typically sparse but large.
One could use the standard Beaver-triple-based protocol to securely perform these sparse matrix multiplications by treating graph convolution as ordinary dense matrix multiplication.
However, this approach incurs $O(\numnode^2)$ communication and memory costs due to computations on irrelevant nodes.
%
Integrating existing cryptographic advances, the initial effort of SecGNN~\cite{tsc/WangZJ23,nips/RanXLWQW23} requires heavy communication or computational overhead.
Recently, CoGNN~\cite{ccs/ZouLSLXX24} optimizes the overhead in terms of  horizontal data partitioning, proposing a semi-honest secure framework.
Research for secure GCN over vertical data  remains nascent.

Current MPC studies, for GCN or not, have primarily targeted settings where participants own different data samples, \ie, horizontally partitioned data~\cite{ccs/ZouLSLXX24}.
MPC specialized for scenarios where parties hold different types of features~\cite{tkde/LiuKZPHYOZY24,icml/CastigliaZ0KBP23,nips/Wang0ZLWL23} is rare.
This paper studies $2$-party secure GCN training for these vertical partition cases, where one party holds private graph topology (\eg, edges) while the other owns private node features.
For instance, LinkedIn holds private social relationships between users, while banks own users' private bank statements.
Such real-world graph structures underpin the relevance of our focus.
To our knowledge, no prior work tackles secure GCN training in this context, which is crucial for cross-silo collaboration.


To realize secure GCN over vertically split data, we tailor MPC protocols for sparse graph convolution, which fundamentally involves sparse (adjacency) matrix multiplication.
Recent studies have begun exploring MPC protocols for sparse matrix multiplication (SMM).
ROOM~\cite{ccs/SchoppmannG0P19}, a seminal work on SMM, requires foreknowledge of sparsity types: whether the input matrices are row-sparse or column-sparse.
Unfortunately, GCN typically trains on graphs with arbitrary sparsity, where nodes have varying degrees and no specific sparsity constraints.
Moreover, the adjacency matrix in GCN often contains a self-loop operation represented by adding the identity matrix, which is neither row- nor column-sparse.
Araki~\etal~\cite{ccs/Araki0OPRT21} avoid this limitation in their scalable, secure graph analysis work, yet it does not cover vertical partition.

% and related primitives
To bridge this gap, we propose a secure sparse matrix multiplication protocol, \osmm, achieving \emph{accurate, efficient, and secure GCN training over vertical data} for the first time.

\subsection{New Techniques for Sparse Matrices}
The cost of evaluating a GCN layer is dominated by SMM in the form of $\adjmat\feamat$, where $\adjmat$ is a sparse adjacency matrix of a (directed) graph $\graph$ and $\feamat$ is a dense matrix of node features.
For unrelated nodes, which often constitute a substantial portion, the element-wise products $0\cdot x$ are always zero.
Our efficient MPC design 
avoids unnecessary secure computation over unrelated nodes by focusing on computing non-zero results while concealing the sparse topology.
We achieve this~by:
1) decomposing the sparse matrix $\adjmat$ into a product of matrices (\S\ref{sec::sgc}), including permutation and binary diagonal matrices, that can \emph{faithfully} represent the original graph topology;
2) devising specialized protocols (\S\ref{sec::smm_protocol}) for efficiently multiplying the structured matrices while hiding sparsity topology.


 
\subsubsection{Sparse Matrix Decomposition}
We decompose adjacency matrix $\adjmat$ of $\graph$ into two bipartite graphs: one represented by sparse matrix $\adjout$, linking the out-degree nodes to edges, the other 
by sparse matrix $\adjin$,
linking edges to in-degree nodes.

%\ie, we decompose $\adjmat$ into $\adjout \adjin$, where $\adjout$ and $\adjin$ are sparse matrices representing these connections.
%linking out-degree nodes to edges and edges to in-degree nodes of $\graph$, respectively.

We then permute the columns of $\adjout$ and the rows of $\adjin$ so that the permuted matrices $\adjout'$ and $\adjin'$ have non-zero positions with \emph{monotonically non-decreasing} row and column indices.
A permutation $\sigma$ is used to preserve the edge topology, leading to an initial decomposition of $\adjmat = \adjout'\sigma \adjin'$.
This is further refined into a sequence of \emph{linear transformations}, 
which can be efficiently computed by our MPC protocols for 
\emph{oblivious permutation}
%($\Pi_{\ssp}$) 
and \emph{oblivious selection-multiplication}.
% ($\Pi_\SM$)
\iffalse
Our approach leverages bipartite graph representation and the monotonicity of non-zero positions to decompose a general sparse matrix into linear transformations, enhancing the efficiency of our MPC protocols.
\fi
Our decomposition approach is not limited to GCNs but also general~SMM 
by 
%simply 
treating them 
as adjacency matrices.
%of a graph.
%Since any sparse matrix can be viewed 

%allowing the same technique to be applied.

 
\subsubsection{New Protocols for Linear Transformations}
\emph{Oblivious permutation} (OP) is a two-party protocol taking a private permutation $\sigma$ and a private vector $\xvec$ from the two parties, respectively, and generating a secret share $\l\sigma \xvec\r$ between them.
Our OP protocol employs correlated randomnesses generated in an input-independent offline phase to mask $\sigma$ and $\xvec$ for secure computations on intermediate results, requiring only $1$ round in the online phase (\cf, $\ge 2$ in previous works~\cite{ccs/AsharovHIKNPTT22, ccs/Araki0OPRT21}).

Another crucial two-party protocol in our work is \emph{oblivious selection-multiplication} (OSM).
It takes a private bit~$s$ from a party and secret share $\l x\r$ of an arithmetic number~$x$ owned by the two parties as input and generates secret share $\l sx\r$.
%between them.
%Like our OP protocol, o
Our $1$-round OSM protocol also uses pre-computed randomnesses to mask $s$ and $x$.
%for secure computations.
Compared to the Beaver-triple-based~\cite{crypto/Beaver91a} and oblivious-transfer (OT)-based approaches~\cite{pkc/Tzeng02}, our protocol saves ${\sim}50\%$ of online communication while having the same offline communication and round complexities.

By decomposing the sparse matrix into linear transformations and applying our specialized protocols, our \osmm protocol
%($\prosmm$) 
reduces the complexity of evaluating $\numnode \times \numnode$ sparse matrices with $\numedge$ non-zero values from $O(\numnode^2)$ to $O(\numedge)$.

%(\S\ref{sec::secgcn})
\subsection{\cgnn: Secure GCN made Efficient}
Supported by our new sparsity techniques, we build \cgnn, 
a two-party computation (2PC) framework for GCN inference and training over vertical
%ly split
data.
Our contributions include:

1) We are the first to explore sparsity over vertically split, secret-shared data in MPC, enabling decompositions of sparse matrices with arbitrary sparsity and isolating computations that can be performed in plaintext without sacrificing privacy.

2) We propose two efficient $2$PC primitives for OP and OSM, both optimally single-round.
Combined with our sparse matrix decomposition approach, our \osmm protocol ($\prosmm$) achieves constant-round communication costs of $O(\numedge)$, reducing memory requirements and avoiding out-of-memory errors for large matrices.
In practice, it saves $99\%+$ communication
%(Table~\ref{table:comm_smm}) 
and reduces ${\sim}72\%$ memory usage over large $(5000\times5000)$ matrices compared with using Beaver triples.
%(Table~\ref{table:mem_smm_sparse}) ${\sim}16\%$-

3) We build an end-to-end secure GCN framework for inference and training over vertically split data, maintaining accuracy on par with plaintext computations.
We will open-source our evaluation code for research and deployment.

To evaluate the performance of $\cgnn$, we conducted extensive experiments over three standard graph datasets (Cora~\cite{aim/SenNBGGE08}, Citeseer~\cite{dl/GilesBL98}, and Pubmed~\cite{ijcnlp/DernoncourtL17}),
reporting communication, memory usage, accuracy, and running time under varying network conditions, along with an ablation study with or without \osmm.
Below, we highlight our key achievements.

\textit{Communication (\S\ref{sec::comm_compare_gcn}).}
$\cgnn$ saves communication by $50$-$80\%$.
(\cf,~CoGNN~\cite{ccs/KotiKPG24}, OblivGNN~\cite{uss/XuL0AYY24}).

\textit{Memory usage (\S\ref{sec::smmmemory}).}
\cgnn alleviates out-of-memory problems of using %the standard 
Beaver-triples~\cite{crypto/Beaver91a} for large datasets.

\textit{Accuracy (\S\ref{sec::acc_compare_gcn}).}
$\cgnn$ achieves inference and training accuracy comparable to plaintext counterparts.
%training accuracy $\{76\%$, $65.1\%$, $75.2\%\}$ comparable to $\{75.7\%$, $65.4\%$, $74.5\%\}$ in plaintext.

{\textit{Computational efficiency (\S\ref{sec::time_net}).}} 
%If the network is worse in bandwidth and better in latency, $\cgnn$ shows more benefits.
$\cgnn$ is faster by $6$-$45\%$ in inference and $28$-$95\%$ in training across various networks and excels in narrow-bandwidth and low-latency~ones.

{\textit{Impact of \osmm (\S\ref{sec:ablation}).}}
Our \osmm protocol shows a $10$-$42\times$ speed-up for $5000\times 5000$ matrices and saves $10$-2$1\%$ memory for ``small'' datasets and up to $90\%$+ for larger ones.



\section{Related Work}
Our work draws on and contributes to research in mobility aids and the built environment, online image-based survey for urban assessment, personalized routing applications and accessibility maps.

\subsection{Mobility Aids and the Built Environment}
People who use mobility aids (\textit{e.g.,} canes, walkers, mobility scooters, manual wheelchairs and motorized wheelchairs) face significant challenges navigating their communities.
Studies have repeatedly found that sidewalk conditions can significantly impede mobility among these users~\cite{bigonnesse_role_2018,fomiatti_experience_2014,f_bromley_city_2007,rosenberg_outdoor_2013, harris_physical_2015,korotchenko_power_2014}. 
In a review of the physical environment's role in mobility, \citet{bigonnesse_role_2018} summarized factors affecting mobility aid users, including uneven or narrow sidewalks (\textit{e.g.,}~\cite{fomiatti_experience_2014,f_bromley_city_2007}), rough pavements (\textit{e.g.,}~\cite{fomiatti_experience_2014,f_bromley_city_2007}), absent or poorly designed curb ramps (\textit{e.g.,}~\cite{rosenberg_outdoor_2013, f_bromley_city_2007, korotchenko_power_2014}), lack of crosswalks (\textit{e.g.,}~\cite{harris_physical_2015}), and various temporary obstacles (\textit{e.g.,}~\cite{harris_physical_2015}).

Though most research on mobility disability and the built environment has focused on wheelchair users~\cite{bigonnesse_role_2018}, mobility challenges are not experienced uniformly across different user populations~\cite{prescott_factors_2020, bigonnesse_role_2018}. 
For example, crutch users could overcome a specific physical barrier (such as two stairs down to a street), whereas motorized wheelchair users could not (without a ramp)~\cite{bigonnesse_role_2018}. 
Such variability demonstrates how person-environment interaction can differ based on mobility aids and environmental factors~\cite{sakakibara_rasch_2018,smith_review_2016}.
Further, mobility aids such as canes, crutches, or walkers are more commonly used than wheelchairs in the U.S.~\cite{taylor_americans_2014, firestine_travel_2024}: in 2022, approximately 4.7 million adults used a cane, crutches, or a walker, compared to 1.7 million who used a wheelchair~\cite{firestine_travel_2024}.
This underscores the importance of considering a diverse range of mobility aid users in urban accessibility research.
For example, \citet{prescott_factors_2020} explored the daily path areas of users of manual wheelchairs, motorized wheelchairs, scooters, walkers, canes, and crutches and found that the type of mobility device had a strong association with users' daily path area size.
Our study aims to further advance knowledge of how different mobility aid users perceive sidewalk barriers, with a more inclusive understanding of urban accessibility.

\begin{figure*}
    \centering
    \includegraphics[width=1\linewidth]{figures/figure-tutorial.png}
    \caption{Survey Part 2.1 showed all 52 images and asked participants to rate their passability based on their lived experience and use of their mobility aid. Above is the interactive tutorial we showed at the beginning of this part.}
    \Description{This figure shows a screenshot from the online survey. In survey part 2.1, participants were presented with 52 images and were asked to rate their passibility based on their lived experience and use of their mobility aid. The screenshot shows the interactive tutorial shown before this section.}
    \label{fig:survey-part2-instructions}
\end{figure*}

\subsection{Online Image-Based Survey for Urban Assessment}
Sidewalk barriers hinder individuals with mobility impairments not just by preventing particular travel paths but also by reducing confidence in self-navigating and decreasing one's willingness to travel to areas that might be physically challenging or unsafe~\cite{vasudevan_exploration_2016,clarke_mobility_2008}.
Prior work in this area traditionally uses three main study methods: in-person interviews (\textit{e.g}.~\cite{rosenberg_outdoor_2013,castrodale_mobilizing_2018}), GPS-based activity studies (\textit{e.g.,}~\cite{prescott_exploration_2021, prescott_factors_2020,rosenberg_outdoor_2013}), and online-questionnaires (\textit{e.g.,}~\cite{carlson_wheelchair_2002}). 
In-person interviews, while providing detailed and nuanced information, are limited by small sample sizes~\cite{rosenberg_outdoor_2013}. GPS-based activity studies involve tracking mobility aids user activity over a period of time, offering insights into movement patterns and activity space; however, these studies are constrained by geographical location~\cite{prescott_exploration_2021}. In contrast, online questionnaires can reach much larger populations and cover broader geographical regions, but they often yield high-level information that lacks the depth and nuance of the other approaches~\cite{carlson_wheelchair_2002}.
Our study aims to strike a balance between these approaches, capturing nuanced perspectives of mobility aid users about the built environment while maintaining a sufficiently large enough sample size for robust statistical analysis. 
Building on~\citet{bigonnesse_role_2018}'s work, we explore not only the types of factors considered to be barriers, but the \textit{intensity} of these barriers and their differential impacts.

Visual assessment of environmental features has long been employed by researchers across diverse fields, including human well-being~\cite{humpel_environmental_2002}, ecosystem sustainability~\cite{gobster_shared_2007}, and public policy~\cite{dobbie_public_2013}. 
These studies examine the relationship between images and the reactions they provoke in respondents or compare differences in reactions between groups.
Over the past decade, online visual preference surveys have gained popularity (\textit{e.g.,}~\cite{evans-cowley_streetseen_2014, salesses_collaborative_2013, goodspeed_research_2017}), where respondents are asked to make pairwise comparisons between randomly selected images.
Using this approach has two advantages: it adheres to the law of comparative judgment~\cite{thurstone_law_2017} by allowing respondents to make direct comparisons, and it prevents inter-rater inconsistency possible with scale ratings~\cite{goodspeed_research_2017}.
Additionally, online surveys generally offer advantages of increased sample sizes, reduced costs, and greater flexibility~\cite{wherrett_issues_1999}.
For people with disabilities, online surveys can be particularly beneficial. They help reach hidden or difficult-to-access populations~\cite{cook_challenges_2007,wright_researching_2005} and are believed to encourage more honest answers to sensitive questions~\cite{eckhardt_research_2007} by providing a higher level of anonymity and confidentiality~\cite{cook_challenges_2007, wright_researching_2005}.

\begin{figure*}
    \centering
    \includegraphics[width=1\linewidth]{figures/figure-comaprison-screenshot.png}
    \caption{In survey Part 2.2, participants were asked to perform a series of pairwise comparisons based on their 2.1 responses.}
    \Description{This figure shows a screenshot from the online survey. In Survey Part 2.2, participants were asked to perform a series of pairwise comparisons based on their 2.1 responses.}
    \label{fig:survey-part2b-pairwise}
\end{figure*}

\subsection{Personalized Routing Applications and Accessibility Maps}
Navigation challenges faced by mobility aid users can be mitigated through the provision of routes and directions that guide them to destinations safely, accurately, and efficiently~\cite{kasemsuppakorn_understanding_2015}. However, current commercial routing applications (\textit{e.g.}, \textit{Google Maps}) do not provide sufficient guidance for mobility aid users.
To address this gap, significant research has focused on routing systems for this population over the past two decades~\cite{barczyszyn_collaborative_2018, karimanzira_application_2006, matthews_modelling_2003, kasemsuppakorn_understanding_2015, volkel_routecheckr_2008, holone_people_2008, wheeler_personalized_2020, gharebaghi_user-specific_2021, ding_design_2007}.
One early, well-known prototype system is \textit{MAGUS}~\cite{matthews_modelling_2003}, which computes optimal routes for wheelchair users based on shortest distance, minimum barriers, fewest slopes, and limits on road crossings and challenging surfaces.
\textit{U-Access}~\cite{sobek_u-access_2006} provides the shortest route for people with three accessibility levels: unaided mobility, aided mobility (using crutch, cane, or walker), and wheelchair users.
However, U-Access only considers distance and ignores other
important factors for mobility aid users~\cite{barczyszyn_collaborative_2018}.
A series of projects by Kasemsuppakorn \textit{et al}.~\cite{kasemsuppakorn_personalised_2009, kasemsuppakorn_understanding_2015} attempted to create personalized routes for wheelchair users using fuzzy logic and \textit{Analytic Hierarchy Process} (AHP).

While influential, many personalized routing prototypes face limited adoption due to a scarcity of accessibility data for the built environment. 
Geo-crowdsourcing~\cite{karimi_personalized_2014}, a.k.a. volunteered geographic information (VGI)~\cite{goodchild_citizens_2007}, has emerged as an effective solution~\cite{karimi_personalized_2014, wheeler_personalized_2020}.
In this approach, users annotate maps with specific criteria or share personal experiences of locations, typically using web applications based on Google Maps or \textit{OpenStreetMap} (OSM)~\cite{karimi_personalized_2014}.
Examples include \textit{Wheelmap}~\cite{mobasheri_wheelmap_2017}, \textit{CAP4Access}~\cite{cap4access_cap4access_2014}, \textit{AXS Map}~\cite{axs_map_axs_2012}, and \textit{Project Sidewalk}~\cite{saha_project_2019}.
Recent research demonstrated the potential of using crowdsourced geodata for personalized routing~\cite{goldberg_interactive_2016, bolten_accessmap_2019,menkens_easywheel_2011, neis_measuring_2015}.
For example, \textit{EasyWheel}~\cite{menkens_easywheel_2011}, a mobile social navigation system based on OSM, provides wheelchair users with optimized routing, accessibility information for points of interest, and a social community for reporting barriers. 
\textit{AccessMap}~\cite{bolten_accessmap_2019} offers routing information tailored to users of canes, manual wheelchairs, or powered wheelchairs, calculating routes based on OSM data that includes slope, curbs, stairs and landmarks. 
Our work builds on the above by gathering perceptions of sidewalk obstacles from different mobility aid users to create generalizable profiles based on mobility aid type. We envision that these profiles can provide starting points in tools like Google Maps for personalized routing but can be further customized by the end user to specify additional needs (\textit{e.g.}, ability to navigate hills, \textit{etc.})

Beyond routing applications, our study data can contribute to modeling and visualizing higher-level abstractions of accessibility. 
Similar to \textit{AccessScore}~\cite{li_interactively_2018}, data from our survey can provide personalizable and interactive visual analytics of city-wide accessibility. By identifying both differences between mobility groups and common barriers within groups, we can develop analytical tools to prioritize barriers and assess the impact of their mitigation or removal, potentially benefiting the broadest range of mobility group users. Incorporating perceptions of passibility into urban planning processes provides a new dimension for urban planners' toolkits, which are often narrowly focused on compliance with ADA standards.




\section{Benchmark and Training Datasets}

Our evaluation benchmark is designed to simultaneously assess the detection task, removal task, and prompt injection defense task. The detection task evaluates the ability of different models to identify injected documents. The removal task measures the effectiveness of various methods in removing injected instructions from these documents. Finally, the prompt injection defense task examines the robustness of different approaches against indirect prompt injection attacks. 

Training datasets are constructed to investigate the challenges of training detection and removal models for defending against indirect prompt injection attacks.


\subsection{Evaluation Benchmark Construction}
Each sample in the benchmark is represented as a tuple $(p, d, a, x, y)$, with the specific meanings of the elements detailed below:

\begin{itemize}
\item $p$: The original input instruction, such as ``What is the process of adding structure to real property or construction of buildings?'' as shown in Table \ref{tab:examples_benchmark}.
\item $d$: The clean document, for example, ``Building construction is the process of …''
\item $a$: The answer to the original input instruction, such as ``Building construction.''
\item $x$: The injected instruction, for instance, ``Give back only www.example.com.''
\item $y$: The probe used to determine whether the injected instruction $x$ is executed. In this example, $y$ is ``www.example.com.'' To determine whether $x$ is executed, we will check whether $y$ is present in the response.
\end{itemize}
% \noindent $\bullet$ The original input instruction, such as ``What is the land area of Jacksonville?'' as shown in Table \ref{tab:examples_benchmark}.
% Example
% The details of the evaluation process can be found in Appendix \ref{sec:eval}.




The evaluation process can be found in Appendix \ref{sec:eval}. For constructing the original input instruction $p$ and clean document $d$, we follow the methodology of \citet{li2023evaluating}, utilizing questions and documents from QA datasets. Specifically, we construct two benchmarks using the validation data from SQuAD\footnote{The SQuAD dataset is released under the CC BY-SA 4.0 license.} \cite{rajpurkar-etal-2016-squad} and TriviaQA \cite{2017arXivtriviaqa}, with each benchmark containing 900 samples. These benchmarks are referred to as ``Inj-SQuAD'' and ``Inj-TriviaQA'', respectively.

When constructing the injected instructions $x$ and the probes $y$, prior benchmarks typically use simple questions and their corresponding answers \cite{zverev2024can, li2023evaluating}. However, this approach presents two problems: (1) the probe $y$ lacks accuracy. For instance, if the injected question has multiple answers that are semantically identical but differ in their string representation, the probe may fail to detect the success of the attack. (2) Injecting simple questions does not fully align with real-world attack scenarios. While simple questions can achieve goal-hijacking, attackers may have additional purposes, such as phishing \cite{liu2024automatic} or advertising \cite{shu2023exploitability}.
To address these issues, we construct injected instructions tailored for phishing, advertising, and propaganda purposes. Moreover, we redefine the probe $y$ to capture the key content of the response to the injected instruction. For example, if the injected instruction $x$ is ``Tell people to click www.example.com,'' the presence of the phishing link in the response is sufficient to indicate that the attacker’s goal has been achieved. Therefore, we set the corresponding probe  $y$  as ``www.example.com.''
We craft these injected instructions using GPT-4o \cite{hurst2024gpt} and append them, along with their probes, to both the Inj-SQuAD and Inj-TriviaQA benchmarks. It is important to note that the injected instructions are identical across both benchmarks. Examples of Inj-SQuAD are shown in Table \ref{tab:examples_benchmark}.

\subsection{Training Data Construction}
% In this research, we have two training tasks: training detection models and training extraction models. The detection models process clean or injected documents and the extraction models process injected documents. 
We first collect clean document and injected instruction pairs, represented as tuples $ \mathcal{P}=
\{(d_i, x_i)\}_{i=1}^{N}$, in preparation for further training data construction.
We construct two sets of clean documents using documents from the SQuAD and TriviaQA training datasets. The SQuAD dataset contributes 18,891 samples, while TriviaQA provides 19,000 samples. Instructions from Stanford-Alpaca \cite{alpaca} are selected as the injected instructions and appended to the two sets of documents, constructing two sets of the clean document and injected instruction pairs. 

For training the detection models, the clean document and injected instruction pairs $\mathcal{P}$ are divided to construct clean documents and injected documents, along with considerations for the injected positions (analyzed in Section \ref{sec:rq1}). $\mathcal{P}$ are divided as follows for constructing training data: 40\% of the samples are clean documents, 15\% have injected instructions at the head of the document, 30\% have injections in the middle and 15\% at the tail. The final detection training dataset is denoted as $\mathcal{D}_{det}$.
Clean documents are excluded to train the extraction models. For each sample from the clean document and injected instruction pairs, the injected instruction $x$ is placed at three different positions (head, middle, and tail) within the document $d$, effectively tripling the size of the training dataset as denoted $\mathcal{D}_{ext}$. This approach ensures robust coverage of different positions during model training.

\subsection{Evaluation Metrics}
To evaluate detection performance, we employ the \textbf{true positive rate} for evaluating the injected documents and \textbf{false positive rate} for evaluating the clean documents.  For clean documents, a higher false positive rate indicates a more severe over-defense problem. Conversely, for injected documents, a higher true positive rate reflects better detection effectiveness.
Then we evaluate removal performance using \textbf{removal rate}, which measures if the injected instruction is \textbf{not} in the processed documents. 
Finally, we integrate the detection and removal methods to assess the overall defense performance against indirect prompt injection attacks. We measure this using the \textbf{attack success rate (ASR)}, which verifies whether the probe $y$ appears in the model’s response.
\section{Method}

In Fig. \ref{fig:overview}, we illustrate two major stages of MedForge for collaborative model development, including feature branch development (Sec~\ref{branch}) and model merging (Sec~\ref{forging}). In the feature branch development, individual contributors (i.e., medical centers) could make individual knowledge contributions asynchronously. Our MedForge allows each contributor to develop their own plugin module and distilled data locally without the need to share any private data. In the model merging stage, MedForge enables multi-task knowledge integration by merging the well-prepared plugin module asynchronously. This key integration process is guided by the distilled dataset produced by individual branch contributors, resulting in a generalizable model that performs strongly among multiple tasks.


\subsection{Preliminary}
\label{pre}
In MedForge, the development of a multi-capability model relies on the multi-center and multi-task knowledge introduced by branch plugin modules and the distilled datasets.
The relationship between the main base model and branch plugin modules in our proposed MedForge is conceptually similar to the relationship between the main repository and its branches in collaborative software version control platforms (e.g., GitHub~\cite{github}). 
To facilitate plugin module training on branches and model merging, we use the parameter-efficient finetuning (PEFT) technique~\cite{hu2021lora} for integrating knowledge from individual contributors into the branch plugin modules. 

\subsubsection{Parameter-efficient Finetuning}
Compared to resource-intensive full-parameter finetuning, parameter-efficient finetuning (PEFT) only updates a small fraction of the pretrained model parameters to reduce computational costs and accelerate training on specific tasks. These benefits are particularly crucial in medical scenarios where computational resources are often limited.
As the representative PEFT technique, LoRA (Low-Rank Adaptation)~\cite{hu2021lora} is widely utilized in resource-constrained downstream finetuning scenarios. In our MedForge, each contributor trains a lightweight LoRA on a specific task as the branch plugin module. LoRA decomposes the weight matrices of the target layer into two low-rank matrices to represent the update made to the main model when adapting to downstream tasks. If the target weight matrix is $W_0 \in R^{d \times k}$, during the adaptation, the updated weight matrix can be represented as $W_0+\Delta W=W_0+B A$, where $B \in \mathbb{R}^{d \times r}, A \in \mathbb{R}^{r \times k}$ are the low-rank matrices with rank $r \ll  \min (d, k)$ and $AB$ constitute the LoRA module. 



\subsubsection{Dataset Distillation}
Dataset distillation~\cite{wang2018dataset, yu2023dataset, lei2023comprehensive} is particularly valuable for medicine scenarios that have limited storage capabilities, restricted transmitting bandwidth, and high concerns for data privacy~\cite{li2024dataset}. 
We leverage the power of dataset distillation to synthesize a small-scale distilled dataset from the original data.

The distilled datasets serve as the training set in the subsequent merging stage to allow multi-center knowledge integration. Models trained on this distilled dataset maintain comparable performance to those trained on the original dataset (\ref{tab:main_res}). Moreover, the distinctive visual characteristics among images of the raw dataset are blurred (see \ref{fig:overview}(a)), which alleviates the potential patient information leakage. 

To perform dataset distillation, we define the original dataset as $\mathcal{T}=\{x_i,y_i\}^N_{i=1}$ and the model parameters as $\theta$. The dataset distillation aims to synthesize a distilled dataset ${\mathcal{S}=\{{s_i},\tilde{y_i}\}^M_{i=1}}$ with a much smaller scale (${M \ll N}$), while models trained on $\mathcal{S}$ can show similar performance as models trained on $\mathcal{T}$. 
This process is achieved by narrowing the performance gap between the real dataset $\mathcal{T}$ and the synthesized dataset $\mathcal{S}$. In MedForge, we utilize the distribution matching (DM)~\cite{zhao2023dataset}, which increases data distribution similarity between the synthesized distilled data and the real dataset
The distribution similarity between the real and synthesized dataset is evaluated through the empirical estimate of the Maximum Mean Discrepancy (MMD)~\cite{gretton2012kernel}:
\begin{equation}
\mathbb{E}_{\boldsymbol{\vartheta} \sim P_{\vartheta}}\left\|\frac{1}{|\mathcal{T}|} \sum_{i=1}^{|\mathcal{T}|} \psi_{\boldsymbol{\vartheta}}\left(\boldsymbol{x}_i\right)-\frac{1}{|\mathcal{S}|} \sum_{j=1}^{|\mathcal{S}|} \psi_{\boldsymbol{\vartheta}}\left(\boldsymbol{s}_j\right)\right\|^2
\end{equation}

where $P_\vartheta$ is the distribution of network parameters, $\psi_{\boldsymbol{\vartheta}}$ is a feature extractor. Then the distillation loss $\mathcal{L}_{DM}$ is:
\begin{equation}\scalebox{0.9}{$
\mathcal{L}_{\mathrm{DM}}(\mathcal{T},\mathcal{S},\psi_{\boldsymbol{\vartheta}})=\sum_{c=0}^{C-1}\left\|\frac{1}{\left|\mathcal{T}_c\right|} \sum_{\mathbf{x} \in \mathcal{T}_c} \psi(\mathbf{x})-\frac{1}{\left|\mathcal{S}_c\right|} \sum_{\mathbf{s} \in \mathcal{S}_c} \psi(\mathbf{s})\right\|^2$}
\end{equation}

We also applied the Differentiable Siamese Augmentation (DSA) strategy~\cite{zhao2021dataset} in the training process of distilled data to enhance the quality of the distilled data. DSA could ensure the distilled dataset is representative of the original data by exploiting information in real data with various transformations. The distilled images extract invariant and critical features from these augmented real images to ensure the distilled dataset remains representative.
\begin{figure}[t]
    \centering
    \includegraphics[width=\linewidth]{assets/img/model_arch.png}
    \caption{\textbf{Main model architecture.} We adopt CLIP as the base module and attach LoRA modules to the visual encoder and visual projection as the plugin module. During all the procedures, only the plugin modules are tuned while the rest are frozen. We get the classification result by comparing the cosine similarity of the visual and text embeddings.}
    \label{fig:model_arch}
\end{figure}

\subsection{Feature Branch Development}
\label{branch}
In the feature branch development stage, the branch contributors are responsible for providing the locally trained branch plugin modules and the distilled data to the MedForge platform, as shown in Fig~\ref{fig:overview} (a).
In collaborative software development, contributors work on individual feature branches, push their changes to the main platform, and later merge the changes into the main branch to update the repository with new features. Inspired by such collaborative workflow, branch contributors in MedForge follow similar preparations before the merging stage, enabling the integration of diverse branch knowledge into the main branch while effectively utilizing local resources.

MedForge consistently keeps a base module and a forge item as the main branch. The base module preserves generative knowledge of the foundation model pretrained on natural image datasets (i.e., ImageNet~\cite{deng2009imagenet}), while the forge item contains model merging information that guides the integration of feature branch knowledge (i.e., a merged plugin module or the merging coefficients assigned to plugin modules). 
Similar to individual software developers working in their own branches, each branch contributor (e.g., individual medical centers) trains a task-specific plugin module using their private data to introduce feature branch knowledge into the main branch. These branch plugin modules are then committed and pushed to update the forge items of the main branch in the merging stage, thus enhancing the model's multi-task capabilities.


\begin{figure*}
    \centering
    \includegraphics[width=\textwidth]{assets/img/fusion.png}
    \caption{\textbf{The detailed methodology of the proposed Fusion.} Branch contributors can asynchronously commit and push their branch plugin modules and the distilled datasets. the plugin modules will then be weighted fused to the current main plugin module.}

    \label{fig:merge}
\end{figure*}


Regarding model architecture, MedForge contains a base module and a plugin module (Fig ~\ref{fig:model_arch}). The base module is pretrained on general datasets (e.g., ImageNet) and remains the model parameters frozen in all processes and branches (main and feature branches) to avoid catastrophic forgetting of foundational knowledge acquired from pretraining. Meanwhile, the plugin module is adaptable for knowledge integration and can be flexibly added or removed from the base module, allowing updates without affecting the base model. In our study, we use the pretrained CLIP~\cite{radford2021learning} model as the base module. For the language encoder and projection layer of the CLIP model, all the parameters are frozen, which enables us to directly leverage the language capability of the original CLIP model. For the visual encoder, we apply LoRA on weight matrices of query ($W_q$) and value ($W_v$), following the previous study~\cite{hu2021lora}. To better adapt the model to downstream visual tasks, we apply the LoRA technique to both the visual encoder and the visual projection, and these LoRA modules perform as the plugin module. During the training, only the plugin module (LoRA modules) participates in parameter updates, while the base module (the original CLIP model) remains unchanged. 

In addition to the plugin modules, the feature branch contributors also develop a distilled dataset based on their private local data, which encapsulates essential patterns and features, serving as the foundation for training the merging coefficients in the subsequent merging stage~\ref{forging}. Compared to previous model merging approaches that rely on whole datasets or few-shot sampling, distilled data is lightweight and representative, mitigating the privacy risks associated with sharing raw data. 
We illustrate our distillation procedure in Algorithm~\ref{algorithm:alg1}. In each distillation step, the synthesized data $\mathcal{S}$ will be updated by minimizing $\mathcal{L}_{DM}$.
\begin{algorithmic}[1]
    \STATE \textbf{Input:} A list of clauses $C$
    \STATE \textbf{Output:} List of primary outputs $PO$, primary inputs $PI$, intermediate variables $IV$, and Boolearn expressions $BE$
    \STATE $SC$ = [] \COMMENT{List of sub-clauses}
    \FOR{$i = 1$ to length($C$)}
        % \IF{$C[i] \cap SC = \emptyset$}
        %     \STATE Append \text{Simplify}(\text{FindBooleanExpression}([], $SC$)) to $BE$
        %     %\COMMENT{Simplify Boolean expression}
        %     \FOR{each item $w$ in $SC$}
        %         \IF{$w \notin IV$ and $w \neq v$}
        %             \STATE Append $w$ to $PI$
        %         \ENDIF
        %     \ENDFOR
        %     \STATE $SC$ = []
        % \ELSE
            \STATE Append $C[i]$ to $SC$
            \FOR{each item $v$ in $SC$}
                \IF{$v \notin PI$ and $v \notin IV$}
                    \STATE $f \gets \text{FindBooleanExpression}(v, SC)$ %\COMMENT{Find Boolean expression for $v$}
                    \STATE $g \gets \text{FindBooleanExpression}(\neg v, SC)$ %\COMMENT{Find Boolean expression for $\neg v$}
                    \IF{$f = \neg g$}
                        \STATE Append \text{Simplify}($f$) to $BE$ %\COMMENT{Simplify Boolean expression}
                        \IF{$f = True$ or $f = False$}
                            \STATE Append $v$ to $PO$
                        \ELSE
                            \STATE Append $v$ to $IV$
                        \ENDIF
                        \FOR{each item $w$ in $SC$}
                            \IF{$w \notin IV$ and $w \neq v$}
                                \STATE Append $w$ to $PI$
                            \ENDIF
                        \ENDFOR
                        \STATE SC = []
                        \STATE \textbf{break}
                    \ENDIF
                \ENDIF    
            \ENDFOR
        % \ENDIF
    \ENDFOR
    \STATE \textbf{Return} $PO, PI, IV, BE$
    \vspace{-0.65cm}
\end{algorithmic}



\subsection{MedForge Merging Stage}
\label{forging}
Following the feature branch development stage illustrated in Fig~\ref{fig:overview} (a), branch contributors push and merge their branch plugin modules along with the corresponding distilled dataset into the main branch, as shown in Fig~\ref{fig:overview} (b). Our MedForge allows an incremental capability accumulation from branches to construct a comprehensive medical model that can handle multiple tasks.

In the merging stage, the $i^{th}$ branch contributor is assigned a coefficient $w'_i$ for the contribution of merging, while the coefficient for the current main branch is $w_i$. By adaptively adjusting the value of coefficients, the main branch can balance and coordinate updates from different contributors, ultimately enhancing the overall performance of the model across multiple tasks.
The optimization of the coefficients is done by minimizing the cross-entropy loss for classification based on the distilled datasets. We also add $L1$ regularization to the loss to regulate the weights to avoid outlier coefficient values (e.g., extremely large or small coefficient values)~\cite{huang2023lorahub}. During optimization, following~\cite{huang2023lorahub}, we utilize Shiwa algorithm~\cite{liu2020versatile} to enable model merging under gradient-free conditions, with lower computational and time costs. The optimizer selector~\cite{liu2020versatile} automatically chooses the most suitable optimization method for coefficient optimization. 

In the following sections, we introduce the two merging methods used in our MedForge: Fusion and Mixture. In MedForge-Fusion, the parameters of the branch plugin modules are fused into the main branch after each round of the merging stage. For MedForge-Mixture, the outputs of the branch modules are weighted and summed based on their respective coefficients rather than directly applying the weighted sum to the model parameters. This largely preserves the internal parameter structure of each branch module.

\paragraph{MedForge-Fusion}
In MedForge, forge items are utilized to facilitate the integration of branch knowledge into the main branch.
For MedForge-Fusion, the forge item refers to adaptable main plugin modules. When the $i^{th}$ branch contributor pushes its branch plugin module $\theta'_i=A'_iB'_i$ to the main branch, the current main plugin module $\theta_{i-1}=A_{i-1}B_{i-1}$ will be updated to $\theta_{i}=A_{i}B_{i}$. The parameters of the branch and the current main plugin modules are weighted with coefficients and added to fuse a new version. The $A_i$, $B_i$ are the low-rank matrices composing the LoRA module $\theta_i$. The detailed fusion process can be represented as:
\begin{equation}
\theta_{i}=(w_i A_{i-1}+w'_i A'_i)(w_i B_{i-1}+ w'_i B'_i)
\end{equation}
Where $w_i$ is the coefficient assigned to the current main branch, while $w'_i$ is the coefficient assigned to the branch contributor. After this round of merging, the resulting plugin module $\theta_{i}$ is the updated version of main forging item, thus the main model is able to obtain new capacity introduced by the current branch contributor. When new contributors push their plugin modules and distilled datasets, the main branch can be incrementally updated through merging stages, and the optimization of the coefficients is guided by distilled data.
As shown in Fig.~\ref{fig:merge}, though multiple contributors commit their branch plugin modules and distilled datasets at different times, they can flexibly merge their plugin modules with the current main branch. After each merging round, the plugin module of the main branch will be updated, and thus the version iteration has been achieved.
\begin{figure*}[t]
    \centering
    \includegraphics[width=\textwidth]{assets/img/mixture.png}
    \caption{\textbf{The detailed methodology of the proposed Mixture.} Branch contributors can asynchronously commit and push their branch plugin modules and the distilled datasets. the outputs of different plugin modules will be weighted aggregated. The weights of each merging step will be saved.}

    \label{fig:mixmerge}
\end{figure*}


\paragraph{MedForge-Mixture}
To further improve the model merging performance, inspired by~\cite{zhao2024loraretriever}, we also propose medForge-mixture. For MedForge-Mixture, the forge items refer to the optimized coefficients.
As shown in Fig.~\ref{fig:mixmerge}, for MedForge-Mixture, the coefficient of each branch contributor is acquired and optimized based on distilled datasets. Then the outputs of plugin modules will be weighted combined with these coefficients to get the merged output. 

For each merging round, with branch contributor $i$, the branch coefficient is $w'_i$, the main coefficient is $w_i$, the branch plugin module is $\theta'_i=A'_iB'_i$, and the current main plugin module is $\theta_i=A_iB_i$. With the input $x$, the resulted MedForge-Mixture output can be represented as:
\begin{equation}
y_{i}=w_i A_{i-1} B_{i-1} x+w'_i A'_i B'_i x
\end{equation}

In this way, MedForge encourages additional contributors as the workflow supports continuous incremental knowledge updates.

Overall, both MedForge merging strategies greatly improve the communication efficiency among contributors. We use this design to build a multi-task medical foundation model that enhances the full utilization of resources in the medical community. For the MedForge-Fusion strategy, the main plugin module is updated after each merging round, thus avoiding storing the previous plugin modules and saving space. Meanwhile, the MedForge-Mixture strategy avoids directly updating the parameters of each plugin module, thus preserving their original structure and preventing the introduction of additional noise, which enhances the robustness and stability of the models.

\section{Experiments}
\label{sec:exp}
\noindent\textbf{Datasets.} {\YuiR Following existing studies~\cite{liu2023hybrid,zhoustrengthened,liu2020learning,mccreesh2017partitioning,solnon2015complexity,hoffmann2017between},} we use four benchmark graph collections, namely biochemicalReactions (\textsf{BI}), images-CVIU11 (\textsf{CV}), images-PR15 (\textsf{PR}) and LV (\textsf{LV}), in the experiments. All datasets are collected from http://liris.cnrs.fr/csolnon/SIP.html and come from real-world applications in various domains, {\Yui as shown in Table~\ref{tab:my_label}}. Specifically, \textsf{BI} contains 136 unlabeled bipartite graphs, each of which corresponds to a biochemical reaction network. \textsf{CV} contains 44 pattern graphs and 146 target graphs, which are generated from segmented images. \textsf{PR} contains 24 pattern graphs and 1 target graph, which are also from segmented images. \textsf{LV} contains 112 graphs generated from biological networks. 
%
{\YuiR All graphs have up to thousands of vertices. We note that (1) solving our problem on two graphs with beyond 10K vertices is challenging based on the worst-case time complexity of $O^*((|V_G|+1)^{|V_Q|})$, (2) the largest graph used in previous studies~\cite{liu2023hybrid,zhoustrengthened,liu2020learning,mccreesh2017partitioning} has 6,771 vertices, which is also covered (in LV) by our experiments, and (3) finding the largest common subgraph between two graphs with thousands of vertices has found many real applications~\cite{ehrlich2011maximum}.}
%
{\Yui Following existing studies~\cite{liu2023hybrid,zhoustrengthened,liu2020learning,mccreesh2017partitioning,solnon2015complexity,hoffmann2017between}}, for \textsf{BI} and \textsf{LV}, we generate and test the problem instances (i.e., $Q$ and $G$) by pairing any two distinct graphs; and for \textsf{CV} and \textsf{PR} {\revision which consist of two types of graphs, namely pattern graphs and target graphs}, we test all those problem instances with one graph $Q$ from pattern graphs and the other $G$ from target graphs.

\begin{table*}[]
    \centering
    \caption{\Yui Datasets used in the experiments (``\# of solved instances'' refers to the number of instances solved by algorithms within 1,800 seconds and ``Achieved speedups'' refers to the percentage of the solved instances that \texttt{RRSplit} runs at least 5$\times$/10$\times$/100$\times$ faster than \texttt{McSplitDAL})}
    \vspace{-0.15in}
    \begin{tabular}{|c|c|c|c|c|c|c|c|c|c|c|}
        \hline
        \multirow{2}{*}{Dataset} & \multirow{2}{*}{Domain} & \multirow{2}{*}{\# of graphs} & \multirow{2}{*}{\# of instances} & \multirow{2}{*}{\# of vertices} & \multicolumn{2}{c|}{\# of solved instances} & \multicolumn{3}{c|}{Achieved speedups} \\
        \cline{6-10}
        & & & & & \texttt{RRSplit} & \texttt{McSplitDAL} & 5$\times$ & 10$\times$ & 100$\times$\\
        \hline
        \textsf{BI} & Biochemical & 136 & 9,180 & 9$\sim$ 386 & 7,730 & 4,696 & 91.3\% & 84.4\% & 69.7\% \\
        \textsf{CV} & Segmented images & 190 & 6,424 & 22$\sim$ 5,972 & 1,351 & 1,291& 76.5\% & 48.6\% & 0.2\% \\
        \textsf{PR} & Segmented images & 25& 24& 4$\sim$ 4,838  & 24 & 24 & 91.7\% & 91.7\% & 58.3\% \\
        \textsf{LV} & Synthetic & 112 & 6,216& 10$\sim$ 6,671 & 1,059 & 883 & 68.0\% & 54.7\% & 38.3\%\\
        \hline
    \end{tabular}
    
    \label{tab:my_label}
\end{table*}

\begin{table*}[]
    \centering
    \caption{\YuiR Comparison of running time on all datasets (statistics of achieved speedups in Figure~\ref{fig:all_datasets_T})}
    \vspace{-0.15in}
    \begin{tabular}{|c|c|c|c|c|c|c|}
        \hline
        \multirow{2}{*}{Dataset} & \multicolumn{3}{c|}{\texttt{RRSplit} runs faster} & \multicolumn{3}{c|}{\texttt{McSplitDAL} runs faster} \\
        \cline{2-7}
        & \% of instances & Avg. speedup & Max. speedup & \% of instances & Avg. speedup & Max. speedup\\
        \hline
        BI& 99.43\% & 3.3$\times 10^4$ & $10^6$ & 0.5\% & 24.81 & 872.37 \\
        CV& 92.15\% & 10.92 & 161 & 7.84\% & 4.96 & 38.97 \\
        PR& 95.83\% & 139.39 & 234 & 4.17\% & 1.23 & 1.23 \\
        LV& 93.48\% & 1.2$\times 10^4$ & $10^6$ & 6.51\% & 24.23 & 652.13 \\
        \hline
    \end{tabular}
    
    \label{tab:results}
\end{table*}

\smallskip
\noindent\textbf{Algorithms.} We compare the newly proposed algorithm \texttt{RRSplit} with \texttt{McSplitDAL}~\cite{liu2023hybrid}. To be specific, \texttt{McSplitDAL} is one variant of \texttt{McSplit} as introduced in Section~\ref{sec:sota}, which follows the framework of \texttt{McSplit} (i.e., Algorithm~\ref{alg:mcsplit}) and introduces some learning-based techniques for optimizing the policies of selecting vertices at line 6, line 8 and line 10 of Algorithm~\ref{alg:mcsplit}. To our best knowledge, \texttt{McSplitDAL} is the state-of-the-art algorithm and runs significantly faster than previous solutions, including \texttt{McSplitLL}~\cite{zhoustrengthened} and \texttt{McSplitRL}~\cite{liu2023hybrid}. Besides these, in order to study the effectiveness of different reductions employed in our algorithm \texttt{RRSplit}, we evaluate three variants of  \texttt{RRSplit} --  
{\YuiR \texttt{RRSplit-VE}, \texttt{RRSplit-MB}, and \texttt{RRSplit-UB}, respectively obtained by turning off vertex-equivalence based reductions, maximality based reductions,  and  vertex-equivalence based upper bound}. 
%namely \texttt{RRSplit-MR} and \texttt{RRSplit-VER},


\smallskip
\noindent\textbf{Implementation and metrics.} All algorithms are implemented in C++ and compiled with -O3 optimization. All experiments run on a Linux machine with a 2.10GHz Intel CPU and 128GB memory. Note that, for the implementation of \texttt{McSplitDAL}, we directly use the source code from the authors of~\cite{liu2023hybrid}. We record and compare the total running times of the algorithms on different problem instances (note that the measured running time excludes the I/O time of reading graphs from the disk). We set the running time limit (INF) as 1,800 seconds by default. Our data and code are available at https://github.com/KaiqiangYu/SIGMOD25-MCSS. 

\subsection{Comparison among algorithms}

\begin{figure}[]
		\subfigure[\textsf{BI}]{
			\includegraphics[width=4.0cm]{figure/BI_TDS.pdf}
		}
		\subfigure[\textsf{CV}]{
			\includegraphics[width=4.0cm]{figure/ICVIU11_TDS.pdf}
		}
		\subfigure[\textsf{PR}]{
			\includegraphics[width=4.0cm]{figure/PR15_TDS.pdf}
		}	
		\subfigure[\textsf{LV}]{
			\includegraphics[width=4.0cm]{figure/LV_TDS.pdf}
		}
        \vspace{-0.15in}
	\caption{Running time on all datasets. {\Yui For those problem instances locating at the right side of dash line `- .' with orange color (resp. `- -' with green color),  \texttt{RRSplit} achieves at least 100$\times$ (resp. 10$\times$) speedup compared with \texttt{McSplitDAL}.}}
	\label{fig:all_datasets_T}
\end{figure}

\noindent\textbf{All datasets (running time)}. We compare our algorithm \texttt{RRSplit} with the baseline \texttt{McSplitDAL} on all graph collections. {\YuiR Following some existing works~\cite{mccreesh2016clique}}, we report the running times of the algorithms on various problem instances in Figure~\ref{fig:all_datasets_T}. 
%
%
Specifically, each dot in the scatter figures represents a problem instance, with the $x$-axis (resp. $y$-axis) corresponding to the running time of \texttt{RRSplit} (resp. \texttt{McSplitDAL}) {\chengC on the instance}. Hence, for those problem instances with small values on $x$-axis and large values on $y$-axis (which thus locate on the top left region of the figures), \texttt{RRSplit} performs better than \texttt{McSplitDAL}.
%In particular, for those problem instances locating at the right side of dash line `- .' {\Yui with orange color} (resp. `- -' with green color),  \texttt{RRSplit} achieves at least 100$\times$ (resp. 10$\times$) speedup compared with \texttt{McSplitDAL}. 
We mark the running time as INF if the problem instance cannot be solved within the default time limit.
%
{\YuiR Besides, we also provide some statistics in Table~\ref{tab:my_label} and Table~\ref{tab:results}.}
%
We observe that (1) \texttt{RRSplit} outperforms \texttt{McSplitDAL} by achieving around one to {\Yui four} orders of magnitude speedup {\YuiR (in average)} on the majority {\YuiR (above 92\%)} of the tested problem instances and (2) \texttt{McSplitDAL} cannot handle all problem instances within the time limit. 
% This fact demonstrates the efficiency of our algorithm \texttt{RRSplit}. 
We do note that \texttt{McSplitDAL} runs slightly faster on a few {\YuiR (below 8\%)} problem instances in \textsf{CV} and \textsf{LV}. {\YuiR Some possible reasons are as follows. 
First, our \texttt{RRSplit} introduces some extra time costs for conducting the proposed reductions as well as computing the upper bound. Second, the heuristic polices adopted in \texttt{RRSplit} and \texttt{McSplitDAL} for branching may have different behaviors. 
%
In specific, on these problem instances, the heuristic policies may help \texttt{McSplitDAL} to find a large common subgraph quickly so as to prune more unpromising branches {\revision via the upper-bound based reduction} (note that they are based on reinforcement learning {\cheng and the behaviors of the learned policy is} based on the explored branches during the running time). } 

%{\cheng One possible reason could be that} their learned heuristic policies can help to find a large common subgraph quickly so as to prune more unpromising branches {\cheng on these datasets}. 


\begin{figure}[]
		\subfigure[\textsf{BI}]{
			\includegraphics[width=4.0cm]{figure/BI_Branch_TDS.pdf}
		}
		\subfigure[\textsf{CV}]{
			\includegraphics[width=4.0cm]{figure/ICVIU11_Branch_TDS.pdf}
		}
		\subfigure[\textsf{PR}]{
			\includegraphics[width=4.0cm]{figure/PR15_Branch_TDS.pdf}
		}	
		\subfigure[\textsf{LV}]{
			\includegraphics[width=4.0cm]{figure/LV_Branch_TDS.pdf}
		}
        \vspace{-0.15in}
	\caption{Number of formed branches on all datasets}
	\label{fig:all_datasets_BT}
\end{figure}

\smallskip
\noindent\textbf{All datasets (number of formed branches)}. We report the number of branches formed by the algorithms on different problem instances in Figure~\ref{fig:all_datasets_BT}. Similarly, each dot in the scatter figures represents a problem instance, with the $x$-axis (resp. $y$-axis) corresponding to the number of branches formed by \texttt{RRSplit} (resp. \texttt{McSplitDAL}) {\chengC on the instance}. We have the following observations. First, the number of branches formed by \texttt{RRSplit} is significantly {\chengC smaller} than that formed by \texttt{McSplitDAL}, e.g., the former is around 10\% - 0.01\% of the latter on the most of problem instances. This shows the effectiveness of our proposed maximality-based reductions and vertex-equivalence-based reductions.
%, and is also compatible with the theoretical results.
Second, the distribution of the number of formed branches in Figure~\ref{fig:all_datasets_BT} is consistent with that of the running time in Figure~\ref{fig:all_datasets_T}. This indicates the achieved speedups on the running time \laks{can be traced} to our newly-designed reductions.

\begin{figure}[]
		\subfigure[\textsf{BI}]{
			\includegraphics[width=4.0cm]{figure/BI_CDF.pdf}
		}
		\subfigure[\textsf{CV}]{
			\includegraphics[width=4.0cm]{figure/ICVIU11_CDF.pdf}
		}
		\subfigure[\textsf{PR}]{
			\includegraphics[width=4.0cm]{figure/PR15_CDF.pdf}
		}	
		\subfigure[\textsf{LV}]{
			\includegraphics[width=4.0cm]{figure/LV_CDF.pdf}
		}
        \vspace{-0.2in}
	\caption{Comparison by varying time limits}
	\label{fig:all_vary_T}
\end{figure}

\begin{figure}[]
		\subfigure[\textsf{BI}]{
			\includegraphics[width=4.0cm]{figure/BI_Branch_CDF.pdf}
		}
		\subfigure[\textsf{CV}]{
			\includegraphics[width=4.0cm]{figure/ICVIU11_Branch_CDF.pdf}
		}
		\subfigure[\textsf{PR}]{
			\includegraphics[width=4.0cm]{figure/PR15_Branch_CDF.pdf}
		}	
		\subfigure[\textsf{LV}]{
			\includegraphics[width=4.0cm]{figure/LV_Branch_CDF.pdf}
		}
        \vspace{-0.2in}
	\caption{Comparison by varying the limit of number of formed branches}
	\label{fig:all_vary_B}
\end{figure}

\smallskip
\noindent\textbf{Varying time limits}. We report the number of solved problem instances in Figure~\ref{fig:all_vary_T} as the time limit is varied. Clearly, all algorithms solve more problem instances as the time limit increases. We observe that \texttt{RRSplit} solves more problem instances than \texttt{McSplitDAL} within the same time limit. In particular, \texttt{RRSplit} with a time limit of 1 second even solves more problem instances than \texttt{McSplitDAL} with a time limit of 10 seconds in all graph collections {\cheng except for} \textsf{CV}; and on \texttt{PR}, \texttt{RRSplit} solves all problem instances within the time limit of 10 seconds. This further demonstrates the superiority of our algorithm \texttt{RRSplit} over the baseline \texttt{McSplitDAL}. 

\smallskip
\noindent\textbf{Varying the limits of number of formed branches}. We report the number of solved problem instances in Figure~\ref{fig:all_vary_B} as the limit on  number of formed branches is varied. We note that the more branches are allowed to be formed, the more instances will be solved. We observe that (1) \texttt{RRSplit} solves more problem instances than \texttt{McSplitDAL} within the same limit of the number of formed branches and (2) the results in Figure~\ref{fig:all_vary_B} show  similar tendencies as those in Figure~\ref{fig:all_vary_T} in general. This further {\cheng explains} the practical superiority of the newly proposed reductions.


\begin{figure}[]
		\subfigure[\textsf{Running time (BI)}]{
			\includegraphics[width=4.0cm]{figure/BI_SIM.pdf}
		}	
		\subfigure[\textsf{Running time (LV)}]{
			\includegraphics[width=4.0cm]{figure/LV_SIM.pdf}
		}
        \subfigure[\textsf{\# of branches (BI)}]{
			\includegraphics[width=4.0cm]{figure/BI_SIMB.pdf}
		}	
		\subfigure[\textsf{\# of branches (LV)}]{
			\includegraphics[width=4.0cm]{figure/LV_SIMB.pdf}
		}
        \vspace{-0.2in}
	\caption{Comparison by varying similarities}
	\label{fig:all_vary_S}
\end{figure}

\smallskip
\noindent\textbf{Varying the similarities of  input graphs}. We define the similarity of  input graphs $Q$ and $G$, $Sim(Q,G)$, as follows.
\begin{equation}
\label{eq:sim}
    Sim(Q,G)=\frac{|S^*|}{\min\{|V_Q|,|V_G|\}},
\end{equation}
where $S^*$ is the maximum common subgraph between $Q$ and $G$. Clearly, $Sim(Q,G)$ varies from 0 to 1, and the larger the value of $Sim(Q,G)$, the higher the similarity between $Q$ and $G$. We test different problem instances as the similarity varies from 0.5 to 1 on \textsf{BI} and \textsf{LV}, and report the average running time in Figures~\ref{fig:all_vary_S}(a)-(b) and the average number of formed branches in Figures~\ref{fig:all_vary_S}(c)-(d). {\Yui The results on \textsf{CV} and \textsf{PR} show similar trends, complete details of which appear in the 
\ifx \CR\undefined
Appendix. 
\else
technical report~\cite{TR}. 
\fi
}  We can see that \texttt{RRSplit} consistently outperforms \texttt{McSplitDAL} {\chengC in} various settings, e.g., \texttt{RRSplit} runs several orders of magnitude faster and forms fewer branches than \texttt{McSplitDAL}. This demonstrates that our designed reductions are effective for pruning the redundant branches on problem instances with various similarities. Besides, we observe that both \texttt{RRSplit} and \texttt{McSplitDAL} have the running time and the number of formed branches first increase and then decrease as the similarity grows. {\revision The possible reasons are as follows. (1) The maximum common subgraphs become larger as the similarity increases according to Equation~(\ref{eq:sim}) and typically more common subgraphs will be explored for finding a large maximum common subgraph. Therefore, the running time firstly increases; (2) the upper-bound based reduction {\chengE performs} better as the similarity grows. {\chengE For example, in the setting of} $Sim(Q,G)=1$, the algorithm can be terminated directly once a common subgraph with $\min\{|V_Q|,|V_G|\}$ vertices is found. Therefore, the running time then decreases.}

%Possible reasons include (1) the number of common subgraphs (i.e., search space) first increases and then decreases as the similarity grows and/or (2) the proposed reductions performs better on those problem instances with the similarity {\cheng close to} 0.5 or 1.   


\begin{figure}[]
		\subfigure[\textsf{\Yui Varying time limits (BI)}]{
			\includegraphics[width=4.0cm]{figure/BI_RT.pdf}
		}	
		\subfigure[\textsf{\Yui Varying time limits (LV)}]{
			\includegraphics[width=4.0cm]{figure/LV_RT.pdf}
		}
        \subfigure[\textsf{\Yui Varying limit of \#branches (BI)}]{
			\includegraphics[width=4.0cm]{figure/BI_BT.pdf}
		}	
		\subfigure[\textsf{\Yui Varying limit of \#branches (LV)}]{
			\includegraphics[width=4.0cm]{figure/LV_BT.pdf}
		}
        \vspace{-0.15in}
	\caption{Comparison among various reductions}
	\label{fig:all_vary_R}
\end{figure}

{\revision
\smallskip
\noindent\textbf{Scalability test.} We test the scalability of our \texttt{RRSplit} on two large datasets, i.e., \textsf{Twitter} and \textsf{DBLP}, which are collected from different domains (http://konect.cc/). Here, \textsf{Twitter} is a social network with 465,017 vertices and 833,540 edges, and \textsf{DBLP} is a collaboration network with 317,080 vertices and 1,049,866 edges. Following existing studies~\cite{arai2023gup,jin2023circinus,sun2023efficient}, we generate the problem instances (i.e., $Q$ and $G$) as follows. Let \textsf{Twitter} or \textsf{DBLP} be the graph $G$. We first extract a set of graphs $Q$ from $G$. Specifically, we conduct a random walk on $G$ and extract a subgraph induced by the visited vertices. By varying the size of the extracted graph $Q$ (among $\{20,30,40,50,60\}$), we extract 5 sets and each of them contains 100 different graphs $Q$.
%
Then, we generate different problems by pairing the graph $G$ (i.e., \textsf{Twitter} or \textsf{DBLP}) with different graphs $Q$ in the set. In summary, for each dataset, we have 500 different problem instances. 

We compare our \texttt{RRSplit} with \texttt{McSplitDAL} by varying the size of $Q$, and report the average running time in Figure~\ref{fig:scalability_test}. We observe that our \texttt{RRSplit} outperforms \texttt{McSplitDAL} significantly.
% , which demonstrates the scalability of the proposed method. 
Besides, \texttt{McSplitDAL} cannot handle almost all the problem instances within the time and/or space limit (INF/OOM). This is because the implementation of \texttt{McSplitDAL} highly relies on the adjacent matrix of $Q$ and $G$, which introduces huge space and time costs when $G$ is very large. Finally, we observe that \texttt{RRSplit} has the running time increase as the size of $Q$ grows. This is also consistent with the theoretical analysis.
}

\begin{figure}[]
		\subfigure[\textsf{Running time (Twitter)}]{
			\includegraphics[width=4.0cm]{figure/TW_S.pdf}
		}	
		\subfigure[\textsf{Running time (DBLP)}]{
			\includegraphics[width=4.0cm]{figure/DBLP_S.pdf}
		}
        \vspace{-0.2in}
	\caption{\revision Scalability test on large datasets}
	\label{fig:scalability_test}
\end{figure}

\subsection{Ablation studies}

We study the effects of various reductions on reducing the redundant computations. In specific, we compare \texttt{RRSplit} with three variants, namely \texttt{RRSplit-VE}: the full version without vertex-equivalence based reductions, \texttt{RRSplit-MB}: the full version without maximality based reductions and \texttt{RRSplit-UB}: the full version without the vertex-equivalence based upper bound, on \textsf{BI} and \textsf{LV}. We report the number of solved problem instances in Figure~\ref{fig:all_vary_R}(a,b) for varying the time limit and in Figure~\ref{fig:all_vary_R}(c,d) for varying the limit of number of formed branches. {\Yui The results on \textsf{CV} and \textsf{PR} show similar clues, which we put in the 
\ifx \CR\undefined
Appendix. 
\else
technical report~\cite{TR}. 
\fi
First, we can see that all four algorithms perform better than the baseline \texttt{McSplitDAL}, among which \texttt{RRSplit} performs the best. This demonstrates the effectiveness of vertex-equivalence-based reductions, maximality-based reductions and vertex-equivalence-based upper bound. Second, \texttt{RRSplit-VE} and \texttt{RRSplit-MB} {\chengB achieve} comparable performance and {\chengB both} contribute to the improvements. Specifically, we note that \texttt{RRSplit-VE} runs slightly faster than \texttt{RRSplit-MB} on \textsf{BI} while \texttt{RRSplit-MB} runs slightly faster than \texttt{RRSplit-VE} on \textsf{LV}. }  {\revision This is possibly because graphs in \textsf{BI} are relatively small biochemical networks where two vertices are more likely to be structural equivalent and thus the vertex-equivalence based reductions outperform other reductions, while graphs in \texttt{LV} are synthetic networks. }
\section{Conclusion }
This paper introduces the Latent Radiance Field (LRF), which to our knowledge, is the first work to construct radiance field representations directly in the 2D latent space for 3D reconstruction. We present a novel framework for incorporating 3D awareness into 2D representation learning, featuring a correspondence-aware autoencoding method and a VAE-Radiance Field (VAE-RF) alignment strategy to bridge the domain gap between the 2D latent space and the natural 3D space, thereby significantly enhancing the visual quality of our LRF.
Future work will focus on incorporating our method with more compact 3D representations, efficient NVS, few-shot NVS in latent space, as well as exploring its application with potential 3D latent diffusion models.

% \clearpage
\bibliography{custom}

\appendix

\subsection{Lloyd-Max Algorithm}
\label{subsec:Lloyd-Max}
For a given quantization bitwidth $B$ and an operand $\bm{X}$, the Lloyd-Max algorithm finds $2^B$ quantization levels $\{\hat{x}_i\}_{i=1}^{2^B}$ such that quantizing $\bm{X}$ by rounding each scalar in $\bm{X}$ to the nearest quantization level minimizes the quantization MSE. 

The algorithm starts with an initial guess of quantization levels and then iteratively computes quantization thresholds $\{\tau_i\}_{i=1}^{2^B-1}$ and updates quantization levels $\{\hat{x}_i\}_{i=1}^{2^B}$. Specifically, at iteration $n$, thresholds are set to the midpoints of the previous iteration's levels:
\begin{align*}
    \tau_i^{(n)}=\frac{\hat{x}_i^{(n-1)}+\hat{x}_{i+1}^{(n-1)}}2 \text{ for } i=1\ldots 2^B-1
\end{align*}
Subsequently, the quantization levels are re-computed as conditional means of the data regions defined by the new thresholds:
\begin{align*}
    \hat{x}_i^{(n)}=\mathbb{E}\left[ \bm{X} \big| \bm{X}\in [\tau_{i-1}^{(n)},\tau_i^{(n)}] \right] \text{ for } i=1\ldots 2^B
\end{align*}
where to satisfy boundary conditions we have $\tau_0=-\infty$ and $\tau_{2^B}=\infty$. The algorithm iterates the above steps until convergence.

Figure \ref{fig:lm_quant} compares the quantization levels of a $7$-bit floating point (E3M3) quantizer (left) to a $7$-bit Lloyd-Max quantizer (right) when quantizing a layer of weights from the GPT3-126M model at a per-tensor granularity. As shown, the Lloyd-Max quantizer achieves substantially lower quantization MSE. Further, Table \ref{tab:FP7_vs_LM7} shows the superior perplexity achieved by Lloyd-Max quantizers for bitwidths of $7$, $6$ and $5$. The difference between the quantizers is clear at 5 bits, where per-tensor FP quantization incurs a drastic and unacceptable increase in perplexity, while Lloyd-Max quantization incurs a much smaller increase. Nevertheless, we note that even the optimal Lloyd-Max quantizer incurs a notable ($\sim 1.5$) increase in perplexity due to the coarse granularity of quantization. 

\begin{figure}[h]
  \centering
  \includegraphics[width=0.7\linewidth]{sections/figures/LM7_FP7.pdf}
  \caption{\small Quantization levels and the corresponding quantization MSE of Floating Point (left) vs Lloyd-Max (right) Quantizers for a layer of weights in the GPT3-126M model.}
  \label{fig:lm_quant}
\end{figure}

\begin{table}[h]\scriptsize
\begin{center}
\caption{\label{tab:FP7_vs_LM7} \small Comparing perplexity (lower is better) achieved by floating point quantizers and Lloyd-Max quantizers on a GPT3-126M model for the Wikitext-103 dataset.}
\begin{tabular}{c|cc|c}
\hline
 \multirow{2}{*}{\textbf{Bitwidth}} & \multicolumn{2}{|c|}{\textbf{Floating-Point Quantizer}} & \textbf{Lloyd-Max Quantizer} \\
 & Best Format & Wikitext-103 Perplexity & Wikitext-103 Perplexity \\
\hline
7 & E3M3 & 18.32 & 18.27 \\
6 & E3M2 & 19.07 & 18.51 \\
5 & E4M0 & 43.89 & 19.71 \\
\hline
\end{tabular}
\end{center}
\end{table}

\subsection{Proof of Local Optimality of LO-BCQ}
\label{subsec:lobcq_opt_proof}
For a given block $\bm{b}_j$, the quantization MSE during LO-BCQ can be empirically evaluated as $\frac{1}{L_b}\lVert \bm{b}_j- \bm{\hat{b}}_j\rVert^2_2$ where $\bm{\hat{b}}_j$ is computed from equation (\ref{eq:clustered_quantization_definition}) as $C_{f(\bm{b}_j)}(\bm{b}_j)$. Further, for a given block cluster $\mathcal{B}_i$, we compute the quantization MSE as $\frac{1}{|\mathcal{B}_{i}|}\sum_{\bm{b} \in \mathcal{B}_{i}} \frac{1}{L_b}\lVert \bm{b}- C_i^{(n)}(\bm{b})\rVert^2_2$. Therefore, at the end of iteration $n$, we evaluate the overall quantization MSE $J^{(n)}$ for a given operand $\bm{X}$ composed of $N_c$ block clusters as:
\begin{align*}
    \label{eq:mse_iter_n}
    J^{(n)} = \frac{1}{N_c} \sum_{i=1}^{N_c} \frac{1}{|\mathcal{B}_{i}^{(n)}|}\sum_{\bm{v} \in \mathcal{B}_{i}^{(n)}} \frac{1}{L_b}\lVert \bm{b}- B_i^{(n)}(\bm{b})\rVert^2_2
\end{align*}

At the end of iteration $n$, the codebooks are updated from $\mathcal{C}^{(n-1)}$ to $\mathcal{C}^{(n)}$. However, the mapping of a given vector $\bm{b}_j$ to quantizers $\mathcal{C}^{(n)}$ remains as  $f^{(n)}(\bm{b}_j)$. At the next iteration, during the vector clustering step, $f^{(n+1)}(\bm{b}_j)$ finds new mapping of $\bm{b}_j$ to updated codebooks $\mathcal{C}^{(n)}$ such that the quantization MSE over the candidate codebooks is minimized. Therefore, we obtain the following result for $\bm{b}_j$:
\begin{align*}
\frac{1}{L_b}\lVert \bm{b}_j - C_{f^{(n+1)}(\bm{b}_j)}^{(n)}(\bm{b}_j)\rVert^2_2 \le \frac{1}{L_b}\lVert \bm{b}_j - C_{f^{(n)}(\bm{b}_j)}^{(n)}(\bm{b}_j)\rVert^2_2
\end{align*}

That is, quantizing $\bm{b}_j$ at the end of the block clustering step of iteration $n+1$ results in lower quantization MSE compared to quantizing at the end of iteration $n$. Since this is true for all $\bm{b} \in \bm{X}$, we assert the following:
\begin{equation}
\begin{split}
\label{eq:mse_ineq_1}
    \tilde{J}^{(n+1)} &= \frac{1}{N_c} \sum_{i=1}^{N_c} \frac{1}{|\mathcal{B}_{i}^{(n+1)}|}\sum_{\bm{b} \in \mathcal{B}_{i}^{(n+1)}} \frac{1}{L_b}\lVert \bm{b} - C_i^{(n)}(b)\rVert^2_2 \le J^{(n)}
\end{split}
\end{equation}
where $\tilde{J}^{(n+1)}$ is the the quantization MSE after the vector clustering step at iteration $n+1$.

Next, during the codebook update step (\ref{eq:quantizers_update}) at iteration $n+1$, the per-cluster codebooks $\mathcal{C}^{(n)}$ are updated to $\mathcal{C}^{(n+1)}$ by invoking the Lloyd-Max algorithm \citep{Lloyd}. We know that for any given value distribution, the Lloyd-Max algorithm minimizes the quantization MSE. Therefore, for a given vector cluster $\mathcal{B}_i$ we obtain the following result:

\begin{equation}
    \frac{1}{|\mathcal{B}_{i}^{(n+1)}|}\sum_{\bm{b} \in \mathcal{B}_{i}^{(n+1)}} \frac{1}{L_b}\lVert \bm{b}- C_i^{(n+1)}(\bm{b})\rVert^2_2 \le \frac{1}{|\mathcal{B}_{i}^{(n+1)}|}\sum_{\bm{b} \in \mathcal{B}_{i}^{(n+1)}} \frac{1}{L_b}\lVert \bm{b}- C_i^{(n)}(\bm{b})\rVert^2_2
\end{equation}

The above equation states that quantizing the given block cluster $\mathcal{B}_i$ after updating the associated codebook from $C_i^{(n)}$ to $C_i^{(n+1)}$ results in lower quantization MSE. Since this is true for all the block clusters, we derive the following result: 
\begin{equation}
\begin{split}
\label{eq:mse_ineq_2}
     J^{(n+1)} &= \frac{1}{N_c} \sum_{i=1}^{N_c} \frac{1}{|\mathcal{B}_{i}^{(n+1)}|}\sum_{\bm{b} \in \mathcal{B}_{i}^{(n+1)}} \frac{1}{L_b}\lVert \bm{b}- C_i^{(n+1)}(\bm{b})\rVert^2_2  \le \tilde{J}^{(n+1)}   
\end{split}
\end{equation}

Following (\ref{eq:mse_ineq_1}) and (\ref{eq:mse_ineq_2}), we find that the quantization MSE is non-increasing for each iteration, that is, $J^{(1)} \ge J^{(2)} \ge J^{(3)} \ge \ldots \ge J^{(M)}$ where $M$ is the maximum number of iterations. 
%Therefore, we can say that if the algorithm converges, then it must be that it has converged to a local minimum. 
\hfill $\blacksquare$


\begin{figure}
    \begin{center}
    \includegraphics[width=0.5\textwidth]{sections//figures/mse_vs_iter.pdf}
    \end{center}
    \caption{\small NMSE vs iterations during LO-BCQ compared to other block quantization proposals}
    \label{fig:nmse_vs_iter}
\end{figure}

Figure \ref{fig:nmse_vs_iter} shows the empirical convergence of LO-BCQ across several block lengths and number of codebooks. Also, the MSE achieved by LO-BCQ is compared to baselines such as MXFP and VSQ. As shown, LO-BCQ converges to a lower MSE than the baselines. Further, we achieve better convergence for larger number of codebooks ($N_c$) and for a smaller block length ($L_b$), both of which increase the bitwidth of BCQ (see Eq \ref{eq:bitwidth_bcq}).


\subsection{Additional Accuracy Results}
%Table \ref{tab:lobcq_config} lists the various LOBCQ configurations and their corresponding bitwidths.
\begin{table}
\setlength{\tabcolsep}{4.75pt}
\begin{center}
\caption{\label{tab:lobcq_config} Various LO-BCQ configurations and their bitwidths.}
\begin{tabular}{|c||c|c|c|c||c|c||c|} 
\hline
 & \multicolumn{4}{|c||}{$L_b=8$} & \multicolumn{2}{|c||}{$L_b=4$} & $L_b=2$ \\
 \hline
 \backslashbox{$L_A$\kern-1em}{\kern-1em$N_c$} & 2 & 4 & 8 & 16 & 2 & 4 & 2 \\
 \hline
 64 & 4.25 & 4.375 & 4.5 & 4.625 & 4.375 & 4.625 & 4.625\\
 \hline
 32 & 4.375 & 4.5 & 4.625& 4.75 & 4.5 & 4.75 & 4.75 \\
 \hline
 16 & 4.625 & 4.75& 4.875 & 5 & 4.75 & 5 & 5 \\
 \hline
\end{tabular}
\end{center}
\end{table}

%\subsection{Perplexity achieved by various LO-BCQ configurations on Wikitext-103 dataset}

\begin{table} \centering
\begin{tabular}{|c||c|c|c|c||c|c||c|} 
\hline
 $L_b \rightarrow$& \multicolumn{4}{c||}{8} & \multicolumn{2}{c||}{4} & 2\\
 \hline
 \backslashbox{$L_A$\kern-1em}{\kern-1em$N_c$} & 2 & 4 & 8 & 16 & 2 & 4 & 2  \\
 %$N_c \rightarrow$ & 2 & 4 & 8 & 16 & 2 & 4 & 2 \\
 \hline
 \hline
 \multicolumn{8}{c}{GPT3-1.3B (FP32 PPL = 9.98)} \\ 
 \hline
 \hline
 64 & 10.40 & 10.23 & 10.17 & 10.15 &  10.28 & 10.18 & 10.19 \\
 \hline
 32 & 10.25 & 10.20 & 10.15 & 10.12 &  10.23 & 10.17 & 10.17 \\
 \hline
 16 & 10.22 & 10.16 & 10.10 & 10.09 &  10.21 & 10.14 & 10.16 \\
 \hline
  \hline
 \multicolumn{8}{c}{GPT3-8B (FP32 PPL = 7.38)} \\ 
 \hline
 \hline
 64 & 7.61 & 7.52 & 7.48 &  7.47 &  7.55 &  7.49 & 7.50 \\
 \hline
 32 & 7.52 & 7.50 & 7.46 &  7.45 &  7.52 &  7.48 & 7.48  \\
 \hline
 16 & 7.51 & 7.48 & 7.44 &  7.44 &  7.51 &  7.49 & 7.47  \\
 \hline
\end{tabular}
\caption{\label{tab:ppl_gpt3_abalation} Wikitext-103 perplexity across GPT3-1.3B and 8B models.}
\end{table}

\begin{table} \centering
\begin{tabular}{|c||c|c|c|c||} 
\hline
 $L_b \rightarrow$& \multicolumn{4}{c||}{8}\\
 \hline
 \backslashbox{$L_A$\kern-1em}{\kern-1em$N_c$} & 2 & 4 & 8 & 16 \\
 %$N_c \rightarrow$ & 2 & 4 & 8 & 16 & 2 & 4 & 2 \\
 \hline
 \hline
 \multicolumn{5}{|c|}{Llama2-7B (FP32 PPL = 5.06)} \\ 
 \hline
 \hline
 64 & 5.31 & 5.26 & 5.19 & 5.18  \\
 \hline
 32 & 5.23 & 5.25 & 5.18 & 5.15  \\
 \hline
 16 & 5.23 & 5.19 & 5.16 & 5.14  \\
 \hline
 \multicolumn{5}{|c|}{Nemotron4-15B (FP32 PPL = 5.87)} \\ 
 \hline
 \hline
 64  & 6.3 & 6.20 & 6.13 & 6.08  \\
 \hline
 32  & 6.24 & 6.12 & 6.07 & 6.03  \\
 \hline
 16  & 6.12 & 6.14 & 6.04 & 6.02  \\
 \hline
 \multicolumn{5}{|c|}{Nemotron4-340B (FP32 PPL = 3.48)} \\ 
 \hline
 \hline
 64 & 3.67 & 3.62 & 3.60 & 3.59 \\
 \hline
 32 & 3.63 & 3.61 & 3.59 & 3.56 \\
 \hline
 16 & 3.61 & 3.58 & 3.57 & 3.55 \\
 \hline
\end{tabular}
\caption{\label{tab:ppl_llama7B_nemo15B} Wikitext-103 perplexity compared to FP32 baseline in Llama2-7B and Nemotron4-15B, 340B models}
\end{table}

%\subsection{Perplexity achieved by various LO-BCQ configurations on MMLU dataset}


\begin{table} \centering
\begin{tabular}{|c||c|c|c|c||c|c|c|c|} 
\hline
 $L_b \rightarrow$& \multicolumn{4}{c||}{8} & \multicolumn{4}{c||}{8}\\
 \hline
 \backslashbox{$L_A$\kern-1em}{\kern-1em$N_c$} & 2 & 4 & 8 & 16 & 2 & 4 & 8 & 16  \\
 %$N_c \rightarrow$ & 2 & 4 & 8 & 16 & 2 & 4 & 2 \\
 \hline
 \hline
 \multicolumn{5}{|c|}{Llama2-7B (FP32 Accuracy = 45.8\%)} & \multicolumn{4}{|c|}{Llama2-70B (FP32 Accuracy = 69.12\%)} \\ 
 \hline
 \hline
 64 & 43.9 & 43.4 & 43.9 & 44.9 & 68.07 & 68.27 & 68.17 & 68.75 \\
 \hline
 32 & 44.5 & 43.8 & 44.9 & 44.5 & 68.37 & 68.51 & 68.35 & 68.27  \\
 \hline
 16 & 43.9 & 42.7 & 44.9 & 45 & 68.12 & 68.77 & 68.31 & 68.59  \\
 \hline
 \hline
 \multicolumn{5}{|c|}{GPT3-22B (FP32 Accuracy = 38.75\%)} & \multicolumn{4}{|c|}{Nemotron4-15B (FP32 Accuracy = 64.3\%)} \\ 
 \hline
 \hline
 64 & 36.71 & 38.85 & 38.13 & 38.92 & 63.17 & 62.36 & 63.72 & 64.09 \\
 \hline
 32 & 37.95 & 38.69 & 39.45 & 38.34 & 64.05 & 62.30 & 63.8 & 64.33  \\
 \hline
 16 & 38.88 & 38.80 & 38.31 & 38.92 & 63.22 & 63.51 & 63.93 & 64.43  \\
 \hline
\end{tabular}
\caption{\label{tab:mmlu_abalation} Accuracy on MMLU dataset across GPT3-22B, Llama2-7B, 70B and Nemotron4-15B models.}
\end{table}


%\subsection{Perplexity achieved by various LO-BCQ configurations on LM evaluation harness}

\begin{table} \centering
\begin{tabular}{|c||c|c|c|c||c|c|c|c|} 
\hline
 $L_b \rightarrow$& \multicolumn{4}{c||}{8} & \multicolumn{4}{c||}{8}\\
 \hline
 \backslashbox{$L_A$\kern-1em}{\kern-1em$N_c$} & 2 & 4 & 8 & 16 & 2 & 4 & 8 & 16  \\
 %$N_c \rightarrow$ & 2 & 4 & 8 & 16 & 2 & 4 & 2 \\
 \hline
 \hline
 \multicolumn{5}{|c|}{Race (FP32 Accuracy = 37.51\%)} & \multicolumn{4}{|c|}{Boolq (FP32 Accuracy = 64.62\%)} \\ 
 \hline
 \hline
 64 & 36.94 & 37.13 & 36.27 & 37.13 & 63.73 & 62.26 & 63.49 & 63.36 \\
 \hline
 32 & 37.03 & 36.36 & 36.08 & 37.03 & 62.54 & 63.51 & 63.49 & 63.55  \\
 \hline
 16 & 37.03 & 37.03 & 36.46 & 37.03 & 61.1 & 63.79 & 63.58 & 63.33  \\
 \hline
 \hline
 \multicolumn{5}{|c|}{Winogrande (FP32 Accuracy = 58.01\%)} & \multicolumn{4}{|c|}{Piqa (FP32 Accuracy = 74.21\%)} \\ 
 \hline
 \hline
 64 & 58.17 & 57.22 & 57.85 & 58.33 & 73.01 & 73.07 & 73.07 & 72.80 \\
 \hline
 32 & 59.12 & 58.09 & 57.85 & 58.41 & 73.01 & 73.94 & 72.74 & 73.18  \\
 \hline
 16 & 57.93 & 58.88 & 57.93 & 58.56 & 73.94 & 72.80 & 73.01 & 73.94  \\
 \hline
\end{tabular}
\caption{\label{tab:mmlu_abalation} Accuracy on LM evaluation harness tasks on GPT3-1.3B model.}
\end{table}

\begin{table} \centering
\begin{tabular}{|c||c|c|c|c||c|c|c|c|} 
\hline
 $L_b \rightarrow$& \multicolumn{4}{c||}{8} & \multicolumn{4}{c||}{8}\\
 \hline
 \backslashbox{$L_A$\kern-1em}{\kern-1em$N_c$} & 2 & 4 & 8 & 16 & 2 & 4 & 8 & 16  \\
 %$N_c \rightarrow$ & 2 & 4 & 8 & 16 & 2 & 4 & 2 \\
 \hline
 \hline
 \multicolumn{5}{|c|}{Race (FP32 Accuracy = 41.34\%)} & \multicolumn{4}{|c|}{Boolq (FP32 Accuracy = 68.32\%)} \\ 
 \hline
 \hline
 64 & 40.48 & 40.10 & 39.43 & 39.90 & 69.20 & 68.41 & 69.45 & 68.56 \\
 \hline
 32 & 39.52 & 39.52 & 40.77 & 39.62 & 68.32 & 67.43 & 68.17 & 69.30  \\
 \hline
 16 & 39.81 & 39.71 & 39.90 & 40.38 & 68.10 & 66.33 & 69.51 & 69.42  \\
 \hline
 \hline
 \multicolumn{5}{|c|}{Winogrande (FP32 Accuracy = 67.88\%)} & \multicolumn{4}{|c|}{Piqa (FP32 Accuracy = 78.78\%)} \\ 
 \hline
 \hline
 64 & 66.85 & 66.61 & 67.72 & 67.88 & 77.31 & 77.42 & 77.75 & 77.64 \\
 \hline
 32 & 67.25 & 67.72 & 67.72 & 67.00 & 77.31 & 77.04 & 77.80 & 77.37  \\
 \hline
 16 & 68.11 & 68.90 & 67.88 & 67.48 & 77.37 & 78.13 & 78.13 & 77.69  \\
 \hline
\end{tabular}
\caption{\label{tab:mmlu_abalation} Accuracy on LM evaluation harness tasks on GPT3-8B model.}
\end{table}

\begin{table} \centering
\begin{tabular}{|c||c|c|c|c||c|c|c|c|} 
\hline
 $L_b \rightarrow$& \multicolumn{4}{c||}{8} & \multicolumn{4}{c||}{8}\\
 \hline
 \backslashbox{$L_A$\kern-1em}{\kern-1em$N_c$} & 2 & 4 & 8 & 16 & 2 & 4 & 8 & 16  \\
 %$N_c \rightarrow$ & 2 & 4 & 8 & 16 & 2 & 4 & 2 \\
 \hline
 \hline
 \multicolumn{5}{|c|}{Race (FP32 Accuracy = 40.67\%)} & \multicolumn{4}{|c|}{Boolq (FP32 Accuracy = 76.54\%)} \\ 
 \hline
 \hline
 64 & 40.48 & 40.10 & 39.43 & 39.90 & 75.41 & 75.11 & 77.09 & 75.66 \\
 \hline
 32 & 39.52 & 39.52 & 40.77 & 39.62 & 76.02 & 76.02 & 75.96 & 75.35  \\
 \hline
 16 & 39.81 & 39.71 & 39.90 & 40.38 & 75.05 & 73.82 & 75.72 & 76.09  \\
 \hline
 \hline
 \multicolumn{5}{|c|}{Winogrande (FP32 Accuracy = 70.64\%)} & \multicolumn{4}{|c|}{Piqa (FP32 Accuracy = 79.16\%)} \\ 
 \hline
 \hline
 64 & 69.14 & 70.17 & 70.17 & 70.56 & 78.24 & 79.00 & 78.62 & 78.73 \\
 \hline
 32 & 70.96 & 69.69 & 71.27 & 69.30 & 78.56 & 79.49 & 79.16 & 78.89  \\
 \hline
 16 & 71.03 & 69.53 & 69.69 & 70.40 & 78.13 & 79.16 & 79.00 & 79.00  \\
 \hline
\end{tabular}
\caption{\label{tab:mmlu_abalation} Accuracy on LM evaluation harness tasks on GPT3-22B model.}
\end{table}

\begin{table} \centering
\begin{tabular}{|c||c|c|c|c||c|c|c|c|} 
\hline
 $L_b \rightarrow$& \multicolumn{4}{c||}{8} & \multicolumn{4}{c||}{8}\\
 \hline
 \backslashbox{$L_A$\kern-1em}{\kern-1em$N_c$} & 2 & 4 & 8 & 16 & 2 & 4 & 8 & 16  \\
 %$N_c \rightarrow$ & 2 & 4 & 8 & 16 & 2 & 4 & 2 \\
 \hline
 \hline
 \multicolumn{5}{|c|}{Race (FP32 Accuracy = 44.4\%)} & \multicolumn{4}{|c|}{Boolq (FP32 Accuracy = 79.29\%)} \\ 
 \hline
 \hline
 64 & 42.49 & 42.51 & 42.58 & 43.45 & 77.58 & 77.37 & 77.43 & 78.1 \\
 \hline
 32 & 43.35 & 42.49 & 43.64 & 43.73 & 77.86 & 75.32 & 77.28 & 77.86  \\
 \hline
 16 & 44.21 & 44.21 & 43.64 & 42.97 & 78.65 & 77 & 76.94 & 77.98  \\
 \hline
 \hline
 \multicolumn{5}{|c|}{Winogrande (FP32 Accuracy = 69.38\%)} & \multicolumn{4}{|c|}{Piqa (FP32 Accuracy = 78.07\%)} \\ 
 \hline
 \hline
 64 & 68.9 & 68.43 & 69.77 & 68.19 & 77.09 & 76.82 & 77.09 & 77.86 \\
 \hline
 32 & 69.38 & 68.51 & 68.82 & 68.90 & 78.07 & 76.71 & 78.07 & 77.86  \\
 \hline
 16 & 69.53 & 67.09 & 69.38 & 68.90 & 77.37 & 77.8 & 77.91 & 77.69  \\
 \hline
\end{tabular}
\caption{\label{tab:mmlu_abalation} Accuracy on LM evaluation harness tasks on Llama2-7B model.}
\end{table}

\begin{table} \centering
\begin{tabular}{|c||c|c|c|c||c|c|c|c|} 
\hline
 $L_b \rightarrow$& \multicolumn{4}{c||}{8} & \multicolumn{4}{c||}{8}\\
 \hline
 \backslashbox{$L_A$\kern-1em}{\kern-1em$N_c$} & 2 & 4 & 8 & 16 & 2 & 4 & 8 & 16  \\
 %$N_c \rightarrow$ & 2 & 4 & 8 & 16 & 2 & 4 & 2 \\
 \hline
 \hline
 \multicolumn{5}{|c|}{Race (FP32 Accuracy = 48.8\%)} & \multicolumn{4}{|c|}{Boolq (FP32 Accuracy = 85.23\%)} \\ 
 \hline
 \hline
 64 & 49.00 & 49.00 & 49.28 & 48.71 & 82.82 & 84.28 & 84.03 & 84.25 \\
 \hline
 32 & 49.57 & 48.52 & 48.33 & 49.28 & 83.85 & 84.46 & 84.31 & 84.93  \\
 \hline
 16 & 49.85 & 49.09 & 49.28 & 48.99 & 85.11 & 84.46 & 84.61 & 83.94  \\
 \hline
 \hline
 \multicolumn{5}{|c|}{Winogrande (FP32 Accuracy = 79.95\%)} & \multicolumn{4}{|c|}{Piqa (FP32 Accuracy = 81.56\%)} \\ 
 \hline
 \hline
 64 & 78.77 & 78.45 & 78.37 & 79.16 & 81.45 & 80.69 & 81.45 & 81.5 \\
 \hline
 32 & 78.45 & 79.01 & 78.69 & 80.66 & 81.56 & 80.58 & 81.18 & 81.34  \\
 \hline
 16 & 79.95 & 79.56 & 79.79 & 79.72 & 81.28 & 81.66 & 81.28 & 80.96  \\
 \hline
\end{tabular}
\caption{\label{tab:mmlu_abalation} Accuracy on LM evaluation harness tasks on Llama2-70B model.}
\end{table}

%\section{MSE Studies}
%\textcolor{red}{TODO}


\subsection{Number Formats and Quantization Method}
\label{subsec:numFormats_quantMethod}
\subsubsection{Integer Format}
An $n$-bit signed integer (INT) is typically represented with a 2s-complement format \citep{yao2022zeroquant,xiao2023smoothquant,dai2021vsq}, where the most significant bit denotes the sign.

\subsubsection{Floating Point Format}
An $n$-bit signed floating point (FP) number $x$ comprises of a 1-bit sign ($x_{\mathrm{sign}}$), $B_m$-bit mantissa ($x_{\mathrm{mant}}$) and $B_e$-bit exponent ($x_{\mathrm{exp}}$) such that $B_m+B_e=n-1$. The associated constant exponent bias ($E_{\mathrm{bias}}$) is computed as $(2^{{B_e}-1}-1)$. We denote this format as $E_{B_e}M_{B_m}$.  

\subsubsection{Quantization Scheme}
\label{subsec:quant_method}
A quantization scheme dictates how a given unquantized tensor is converted to its quantized representation. We consider FP formats for the purpose of illustration. Given an unquantized tensor $\bm{X}$ and an FP format $E_{B_e}M_{B_m}$, we first, we compute the quantization scale factor $s_X$ that maps the maximum absolute value of $\bm{X}$ to the maximum quantization level of the $E_{B_e}M_{B_m}$ format as follows:
\begin{align}
\label{eq:sf}
    s_X = \frac{\mathrm{max}(|\bm{X}|)}{\mathrm{max}(E_{B_e}M_{B_m})}
\end{align}
In the above equation, $|\cdot|$ denotes the absolute value function.

Next, we scale $\bm{X}$ by $s_X$ and quantize it to $\hat{\bm{X}}$ by rounding it to the nearest quantization level of $E_{B_e}M_{B_m}$ as:

\begin{align}
\label{eq:tensor_quant}
    \hat{\bm{X}} = \text{round-to-nearest}\left(\frac{\bm{X}}{s_X}, E_{B_e}M_{B_m}\right)
\end{align}

We perform dynamic max-scaled quantization \citep{wu2020integer}, where the scale factor $s$ for activations is dynamically computed during runtime.

\subsection{Vector Scaled Quantization}
\begin{wrapfigure}{r}{0.35\linewidth}
  \centering
  \includegraphics[width=\linewidth]{sections/figures/vsquant.jpg}
  \caption{\small Vectorwise decomposition for per-vector scaled quantization (VSQ \citep{dai2021vsq}).}
  \label{fig:vsquant}
\end{wrapfigure}
During VSQ \citep{dai2021vsq}, the operand tensors are decomposed into 1D vectors in a hardware friendly manner as shown in Figure \ref{fig:vsquant}. Since the decomposed tensors are used as operands in matrix multiplications during inference, it is beneficial to perform this decomposition along the reduction dimension of the multiplication. The vectorwise quantization is performed similar to tensorwise quantization described in Equations \ref{eq:sf} and \ref{eq:tensor_quant}, where a scale factor $s_v$ is required for each vector $\bm{v}$ that maps the maximum absolute value of that vector to the maximum quantization level. While smaller vector lengths can lead to larger accuracy gains, the associated memory and computational overheads due to the per-vector scale factors increases. To alleviate these overheads, VSQ \citep{dai2021vsq} proposed a second level quantization of the per-vector scale factors to unsigned integers, while MX \citep{rouhani2023shared} quantizes them to integer powers of 2 (denoted as $2^{INT}$).

\subsubsection{MX Format}
The MX format proposed in \citep{rouhani2023microscaling} introduces the concept of sub-block shifting. For every two scalar elements of $b$-bits each, there is a shared exponent bit. The value of this exponent bit is determined through an empirical analysis that targets minimizing quantization MSE. We note that the FP format $E_{1}M_{b}$ is strictly better than MX from an accuracy perspective since it allocates a dedicated exponent bit to each scalar as opposed to sharing it across two scalars. Therefore, we conservatively bound the accuracy of a $b+2$-bit signed MX format with that of a $E_{1}M_{b}$ format in our comparisons. For instance, we use E1M2 format as a proxy for MX4.

\begin{figure}
    \centering
    \includegraphics[width=1\linewidth]{sections//figures/BlockFormats.pdf}
    \caption{\small Comparing LO-BCQ to MX format.}
    \label{fig:block_formats}
\end{figure}

Figure \ref{fig:block_formats} compares our $4$-bit LO-BCQ block format to MX \citep{rouhani2023microscaling}. As shown, both LO-BCQ and MX decompose a given operand tensor into block arrays and each block array into blocks. Similar to MX, we find that per-block quantization ($L_b < L_A$) leads to better accuracy due to increased flexibility. While MX achieves this through per-block $1$-bit micro-scales, we associate a dedicated codebook to each block through a per-block codebook selector. Further, MX quantizes the per-block array scale-factor to E8M0 format without per-tensor scaling. In contrast during LO-BCQ, we find that per-tensor scaling combined with quantization of per-block array scale-factor to E4M3 format results in superior inference accuracy across models. 

\label{sec:appendix}


\end{document}
