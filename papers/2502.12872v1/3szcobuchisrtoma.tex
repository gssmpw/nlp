\theoremcobuchisrtoma*
We start by fixing a SR coB\"uchi automaton $\Ac$ throughout the proof of \cref{theorem:coBuchiHDisSR}. 
We first relabel the priorities on $\Ac$ to obtain $\Cc$ as follows. Consider the graph consisting of all states of $\Ac$ and 0 priority transitions of $\Ac$. For any 0 priority transition of $\Ac$ that is not in any strongly connected component (SCC) in this graph, we change that transition to have priority~1 in $\Cc$. This relabelling of priority does not change the acceptance of any run (\cref{prop:priority-reduction}, \cref{app:succandcompCB}), and thus,~$\Cc$ is SR and language-equivalent to~$\Ac$.  

We start by introducing notions to describe a proof sketch of \cref{theorem:coBuchiHDisSR}. 
\paragraph*{Safe-approximation}
For the automaton $\Cc$, define the safe-approximation of $\Cc$, denoted $\Cc_{\safety}$ as the safety automaton constructed as follows. The automaton $\Cc_{\safety}$ has the same states as $\Cc$ and an additional rejecting sink state. The transitions of priority $0$ in $\Cc$ are preserved as the safe transitions of $\Cc_{\safety}$, and transitions of priority~$1$ in $\Cc$ are redirected to the rejecting sink state and have priority~1. 

\paragraph*{Weak-coreachability}
We call two states  $p$ and $q$ in $\Cc$ as coreachable, denoted by $p,q\in \CR(\Cc)$, if there is a finite word $u$ on which there are runs from the initial state of $\Cc$ to $p$ and $q$. We denote the transitive closure of this relation as \emph{weak-coreachability}, which we denote by $\WCR(\Cc)$. Note that weak-coreachability is an equivalence relation.

\paragraph*{SR self-coverage} 
For two  parity automata $\Bc$ and $\Bc'$, we say that $\Bc$ SR-covers $\Bc'$, denoted by $\Bc \succ_{SR} \Bc'$, if Eve has an almost-sure winning strategy in the modified SR game as follows. Eve, similar to the SR game on $\Bc$, constructs a run in $\Bc$, but 
Eve wins a play of the game if, in that play, Eve's constructed run in $\Bc$ is accepting whenever Adam's word is in $\Lc(\Bc')$.
We say that a coB\"uchi automaton $\Bc$ has \emph{SR self-coverage} if for every state $q$ there is another state $p$ that is coreachable to $q$ in $\Cc$, such that $(\Bc_{\safety},p)$ SR-covers~$(\Bc_{\safety},q)$.  

The crux of \cref{theorem:coBuchiHDisSR} is in proving the following result.
\begin{restatable}{lemma}{lemmaSRhassafeSRcoverage}\label{lemma:coBuchiSRhassafeSRcoverage}
The coB\"uchi automaton $\Cc$  has SR self-coverage.
\end{restatable}
\begin{proof}[Proof sketch] Fix an almost-sure resolver $\Mc$ for Eve in $\Cc$. Let~$\Pc$ be the probabilistic automaton that is the resolver-product of $\Mc$ and $\Cc$. We define $\Pc_{\safety}$ as a safety probabilistic automaton that is the safe-approximation of $\Pc$, similar to how we defined $\Cc_{\safety}$. Suppose, towards a contradiction, that there is a state $q$ in $\Cc$, such that for every state $p$ coreachable to $q$ in $\Cc$, $(\Cc_{\safety},p)$ does not SR-cover $(\Cc_{\safety},q)$. In particular, for every state $(p,m)$ in $\Pc$, where $p$ is coreachable to $q$ in $\Cc$, we have that $\Lc(\Pc_{\safety},(p,m)) \subsetneq \Lc(\Cc_{\safety},q)$. We use this to show that there is a finite word $\alpha_{(p,m)}$, on which there is a run consisting of only priority 0 transitions from $q$ to $q$ in $\Cc$, while a run $\rho$ of $(\Pc,(p,m))$ on $\alpha_{(p,m)}$ contains a priority $1$ transition with probability at least $\epsilon$ for some $\epsilon>0$.    

Adam then has a strategy in the SR game on $\Cc$ against Eve's strategy $\Mc$ as follows. Adam starts by giving a finite word~$u_q$, such that there is a run of $\Cc$ from its  initial state to~$q$. Then Adam, from this point and at each \emph{reset}, selects a state~$(p,m)$ of $\Pc$ uniformly at random, such that $p$ is coreachable to $q$ in $\Cc$ and $m$ is a memory-state in $\Mc$. He then plays the letters of the word $\alpha_{(p,m)}$ in sequence, after which he \emph{resets} to select another such state and play similarly. This results in Eve constructing a run in the SR game on $\Cc$ that contains infinitely many priority $1$ transitions almost-surely, while Adam's word is in $\Lc(\Cc)$. It follows that $\Mc$ is not an almost-sure resolver for Eve, which is a contradiction.
\end{proof}

 SR-covers is a transitive relation, i.e., if $\Ac_1,\Ac_2,\Ac_3$ are nondeterministic parity automata, such that $\Ac_1 \succ_{SR} \Ac_2$ and $\Ac_2 \succ_{SR} \Ac_3$, then $\Ac_1 \succ_{SR} \Ac_3$. The following result then follows from the definition of SR self-coverage and the fact that $\Cc$ has finitely many states.% any finite directed graph in which every vertex has outdegree 1 contains a cycle.

\begin{restatable}{lemma}{lemmacobuchisometingdbp}\label{lemma:cobuchi-srselfcoverage-implies-somtingsdbp}
    For every state $q$ in $\Cc$, there is another state $p$ weakly coreachable to $q$ in $\Cc$, such that $(\Cc_{\safety},p)$ SR-covers $(\Cc_{\safety},q)$ and $(\Cc_{\safety},p)$ SR-covers $(\Cc_{\safety},p)$.  
\end{restatable}

Note that if $(\Cc_{\safety},p)$ SR-covers $(\Cc_{\safety},p)$ then $(\Cc_{\safety},p)$ is SR. Since SR automata are semantically deterministic (\cref{lemma:SR-implies-SD}) and SD safety automata are determinisable-by-pruning (\cref{lemma:sd-safety-is-dbp}), we call  such states $p$ as \emph{safe-deterministic}. 

We will build a memoryless adversarially resolvable automaton $\Hc$, whose states are the safe-deterministic states in $\Cc$. This construction is similar to the one used by Kuperberg and Skrzypczak, in 2015, for giving a polynomial time procedure to recognise HD coB\"uchi automata~\cite[Section E.7 in the full version]{KS15}.  We fix a uniform determinisation of transitions from each safe-deterministic state in $\Cc_{\safety}$ and we add these transitions in $\Hc$ with priority 0. If there are no outgoing transitions in $\Hc$ from the state $p$ on letter $a$ so far, then we add outgoing transitions from $p$ on $a$ as follows. Let $q$ be a state in $\Cc$ such that there is a transition from $p$ to $q$ on $a$ in $\Hc$. For each state $r$ that is weakly coreachable to $q$ in $\Cc$ and that is safe-deterministic, we add a transition from $q$ to $r$ in $\Hc$ with priority $1$. This concludes our construction of $\Hc$.

We show that the strategy of Eve that chooses transitions from $\Hc$ uniformly at random is an almost-surely winning strategy for Eve in the HD game, and thus, $\Hc$ is MA (\cref{lemma:cobuchi-h-is-ma},\cref{app:succandcompCB}). Both this fact and the language equivalence of $\Hc$ to $\Cc$ primarily relies on \cref{lemma:cobuchi-srselfcoverage-implies-somtingsdbp}. Since a safety automaton is DBP if and only if that automaton is HD and every HD safety automaton can be determinised in polynomial-time~\cite{BL23quantitative},  the safe-deterministic states of $\Cc$ can be identified, and outgoing safe transitions from these states can be determinised in polynomial-time. Thus, the construction of $\Hc$ takes polynomial-time overall. This completes our proof sketch for \cref{theorem:coBuchiHDisSR}.

%the run of $(\Pc_{\safety})$ from $(p,m)$ on $\alpha_{(p,m)}$ sees a rejectin


%Thus, there is a finite word $u_{(p,m)}$ such that a run of $\Pc_{\safety}$ from $(p,m)$ on $u_{p,m}$ sees a rejecting transition with positive probability, but there is a run of $\Cc_{\safety}$ on $u_{p,m}$ from $q$ to $q'$ that contains only safe transitions. Now, $q'$ and $q$ are in the same SCC in the graph consisting of  $\Cc_{\safety}$