In \cref{sec:succandcomp}, we compared the novel classes of nondeterministic automata we defined with each other and the existing notions of semantic determinism and history determinism, focusing on the questions of succinctness and how they coincide. In this section, we compare these notions in terms of expressivity. We show that, similar to history-determinism, stochastically resolvable $[i,j]$-parity automata are as expressive as deterministic $[i,j]$-parity  automata.

\begin{restatable}{theorem}{pih}\label{theorem:pih}
Stochastically resolvable $[i,j]$-parity automata recognise the same  languages as deterministic $[i,j]$-parity automata.
\end{restatable}
Since deterministic automata are trivially SR, one direction is clear. For the other direction, we will show that any $\omega$-regular language that is not recognised by any deterministic $[i,i+d]$-parity automaton  cannot be recognised by any SR $[i,i+d]$-parity automaton. To show this, we will consider the language $L_{[i+1,i+d]}$ of the $[i+1,i+d+1]$-parity condition, which is the set of infinite words over the alphabet $[i+1,i+d+1]$ in which the highest number occurring infinitely often is even. We show that there is no SR $[i,i+d]$-parity automaton recognising language $L_{[i+1,i+d+1]}$. 

To prove that no SR $[i,i+d]$-parity automaton recognising language $L_{[i+1,i+d+1]}$, consider any SR automaton $\Ac$ that recognises the language $L_{[i+1,i+d+1]}$ and has an almost-sure resolver $\sigma$. We inductively construct words $u_0,u_1,\cdots,u_d$, such that from every state $q$, a run from $q$ on the word $u_k$ in automaton $\Ac$ constructed using $\sigma$ contains a transition with priority at least $(i+k+1)$ with positive probability. This part of the proof is nontrivial and requires careful analysis of probabilities. Once we have proved this result inductively, we obtain that $\Ac$ has at least as many priorities as in the interval $[i+1,i+d+1]$, and in particular, is not an $[i,i+d]$-parity automaton, as desired.
%We remark that the proof has a similar flavour to the classical proof of Wagner from 1979 that showed the strictness of the parity index hierarchy~\cite{Wag79}. 
