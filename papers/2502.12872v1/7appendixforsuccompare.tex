\subsection{Safety, reachability and weak automata}\label{app:succandcompsafety}\label{app:succandcompWeak}
We prove the theorem that every SD automaton, and therefore every pre-SD automaton, is determinisable by pruning.
\SDsafetyisDBP*
\begin{proof}
    Let $\Sc$ be a SD safety automaton. For each state $q$ in $\Sc$ and letter $a$ in the alphabet of $\Sc$, fix a language-preserving transition on $a$ that is outgoing from $q$, and consider the deterministic automaton $\Dc$ consisting of these transitions. It suffices to show that $\Lc(\Sc) \subseteq \Lc(\Dc)$. Let $w$ be a word in $\Lc(\Sc)$, and let $\rho$ be the unique run of $\Dc$ on $w$. For any finite prefix $u$ of $w$, consider the state $q$ that is reached in $\Dc$ upon reading the word $u$. Because $\Sc$ is SD, $u^{-1}w\in \Lc(\Ac,q)$, and hence, $q$ is not the rejecting sink state. It follows that $\rho$ does not reach the rejecting sink state, and hence is an accepting run. We obtain $w\in \Lc(\Dc)$, as desired.
\end{proof}

\lemmasdweakismr*
\begin{proof}
    Let $\Ac$ be a semantically deterministic weak automaton. We will show that the resolver which selects transitions uniformly at random constructs runs that are almost-surely accepting on words in $\Lc(\Ac)$. Let $w$ be a word in $\Lc(\Ac)$, $\rho_{acc}$ an accepting run of $\Ac$ on $w$, and $\rho$ a run of $\Ac$ on $w$, where transitions are chosen uniformly at random. Then there is a finite prefix $u$ of $w$, such that the run $\rho_{acc}$ after reading $u$, and on the word $w'=u^{-1}w$ only contains accepting transitions. Let $K=n2^n$, where $n$ is the number of states of $\Ac$. The crux of our proof is to show the following claim.
 \begin{claim}\label{claim:claiminsdweakismr}
        There is a positive probability $\epsilon>0$, such that on any infix of $w'$ that has length at least $K$, the segment of the run $\rho$ on that infix contains an accepting transition with probability at least $\epsilon$.
    \end{claim}
    \begin{proof}
        Let $v$ be an infix of $w'$ such that $w'=u'vw''$, and $v$ has length $K$. Suppose that the run $\rho$ after reading $uu'$ is at the state $q$, and $\rho_{acc}$ is at the state $p$. Consider the sequence of states $p_0p_1p_2\dots p_K$ in $\rho$ from $p$ on the word $v$, and the sequence of set of states that can be reached from $q$ on reading the prefixes of $v$:
        $$\{q_0\} \xrightarrow{a_0} Q_1 \xrightarrow{a_1} Q_2 \xrightarrow{a_2} \dots \xrightarrow{a_K} Q_K,$$ where $v=a_1a_2\dots a_K$ and $Q_{l+1}$ is the set consisting of states to which there is a transition from a state in $Q_l$ on the letter $a_l$.  By the pigeonhole principle, there are numbers $i<j$, such that $(p_i,S_i)=(p_j,S_j)$. 
        
        Let $v'$ be the word $v'=a_{i} a_{i+1} \dots a_j$, and consider the word $t = a_1 a_2 \dots a_{i-1} (v')^{\omega}$. Then $t$ is in $\Lc(\Ac,p)$, and since $\Ac$ is SD, $t$ is in $\Lc(\Ac,q)$ as well. Thus, there must be a run from a state in $S_i$ on the word $v'$ that contains an accepting transition. It follows that there is a run  from $q$ on the word $v$ that contains an accepting transition. Let $\epsilon=\frac{1}{d^K}$, where $d$ is the maximum outdegree in $\Ac$. Then, the probability that a random run from $q$ on $v$ contains an accepting transition is at least $\epsilon$, as desired.    
    \end{proof}
We now use \cref{claim:claiminsdweakismr} to prove that any random run from $q$ on $w'=u^{-1}w$ is almost-surely accepting. Let $w'=v_1v_2 v_3\dots$, where each $v_i$ is of length $K=n 2^n$, and $\rho_q$ be the suffix of the run $\rho$ on the word $w'$. Note that $\rho_q$ is rejecting if and only if $\rho_q$ contains finitely many priority $2$ transitions (since $\Ac$ is weak). Thus, we have the following chain of inequalities to show that $\rho_q$ is rejecting with probability $0$. 
\begin{equation*}
    \begin{split}
                &\prob[\rho \text{ is rejecting}] = \prob[\rho_q \text{ is rejecting}]\\
                &=\prob[\bigcup_{N\in\mathbb{N}} \rho_q \text{ does not contain a transition of}\\ 
                &\quad\quad\quad\quad\text{priority 2 after prefix $v_1v_2\dots v_N$}]\\
                &=\lim_{N\to\infty}\prob[\rho_q \text{ does not contain a transition of}\\ 
                &\quad\quad\quad\quad\quad\quad\text{priority 2 after prefix $v_1v_2\dots v_N$}] \\
        &\leq \lim_{N\to\infty} \prod_{n\geq N} (1-\epsilon) = 0\\  
    \end{split}
\end{equation*}
Thus, $\rho$ is almost-surely accepting, as desired.
\end{proof}

\subsection{CoB\"uchi automata}\label{app:succandcompCB}

\lemmaHDcoBuchinotMR*
\begin{proof}
    Consider the HD coB\"uchi automaton shown in \cref{fig:HDcoBuchinotMR}, which we re-illustrate in \cref{fig:HDcoBuchinotMRappendix} for convenience. 
\begin{figure}[ht]
\centering
        \begin{tikzpicture}
        \tikzset{every state/.style = {inner sep=-3pt,minimum size =20}}

    \node[state] (q0) at (0,0) {$q_0$};
    \node[state] (q1) at (0,2) {$q_1$};
    \node[state] (d1) at (2,0) {$d_1$};
    \node[state] (d2) at (2,2) {$d_2$};
    \node[state] (d3) at (4,2) {$d_3$};
    \path[-stealth]
    (-0.5,-0.5) edge (q0)
    (q0) edge [bend left = 15] node [left] {$x$} (q1)
    (q1) edge [red, dashed, bend left = 15] node [right] {$a$} (q0)
    (q0) edge [loop left] node [left] {$x$} (q0)
    (q0) edge [red, dashed, loop below] node [left] {$b$} (q0)
    (q1) edge [loop left] node [left] {$x$} (q1)
    (q1) edge node [above] {$b$} (d2)
    (d2) edge [red, dashed, loop above] node [left] {$b$} (d2)
    (d2) edge [red, dashed, bend left = 15] node [right] {$a$} (d1)
    (d1) edge [red, dashed, bend left = 15] node [left] {$b$} (d2)
    (q0) edge node [above] {$a$} (d1)
    (d1) edge [loop right] node [right] {$x,a$} (d1)
    (d2) edge [bend left = 15] node [above] {$x$} (d3)
    (d3) edge [bend left = 15] node [below] {$b$} (d2)
    (d3) edge [loop right] node [right] {$x$} (d3)
    (d3) edge [red, dashed] node [above] {$a$} (d1)
    
;
    \end{tikzpicture}
\caption{A HD coB\"uchi automaton that is not MR. The rejecting transitions are represented by dashed arrows.}\label{fig:HDcoBuchinotMRappendix}
\end{figure} 
    The automaton $\Cc$ has nondeterminism on the letter $x$ in the initial state $q_0$. Informally, in the HD game or the SR game, Eve needs to ``guess'' whether the next sequence of letters Adam will give forms a word in $x^* a$ or in $x^+ b$. The automaton $\Cc$
    recognises the language $$L=(x+a+b)^{*} ((x)^{\omega} + (x^* a)^{\omega} + (x^+ b)^{\omega}).$$ 
    \paragraph*{$\Cc$ is HD}  If Eve's token in the HD game reaches the state $d_1,d_2$, or $d_3$, then she wins the HD game from here onwards since her transitions are deterministic. At the start of the HD game on $q_0$, or whenever she is at $q_0$ after reading $a$ or $b$ in the previous round, she decides between staying at $q_0$ until a $a$ or $b$ is seen, or moving to $q_1$ in the first $x$ as follows.
    \begin{itemize}
        \item If the word read so far has a suffix in $x^{*}a$, then she stays in $q_0$ until the next $a$ or~$b$.
        \item If the word read so far has a suffix in $x^{+}b$, then she takes the transition to $q_1$ on~$x$.
        \item Otherwise, she stays in $q_0$ till the next $a$ or $b$.
    \end{itemize}
    Due to the language of $\Cc$ being the set of words which have a suffix in $x^{\omega},(x^{*}a)^{\omega},$ or $(x^+b)^{\omega}$, the above strategy guarantees that Eve's token moves on any word in $L$ in HD game to one of $d_1$ or $d_2$, from where she wins the HD game.
    
    \paragraph*{$\Cc$ is not MR} Note that the automaton $\Cc$ does not accept the same language if any of its transitions are deleted. Therefore, consider a memoryless resolver $\Mc$ for $\Cc$ that takes the self-loop on $x$ on $q_0$ with probability $(1-p)$ and the transition to $q_1$ on $x$ with probability $p$, for some $p$ satisfying $0<p<1$. 
    
    We will show that on the word $w = x a x^2 a x^3 a \dots$, the resolver $\Mc$ constructs a rejecting run with a positive probability. Let $\rho$ be a run on $w$ constructed by $\Mc$. We denote $w$ as $v_1 a v_2 a \dots $, where $v_i=x^i$ for each $i\geq 1$.
    
    If $\rho$ is ever at the state $q_0$ after reading $v_1 a v_2 a \dots v_i$, then $\rho$ is at the state $d_1$ after reading $a$, from where $\rho$ is accepting. Thus, $\rho$ is rejecting if and only if after reading the substring $v_i$, $\rho$ is at $q_1$. The probability that $\rho$, starting from $q_0$, ends on $q_1$ after reading $v_i=x^i$ is $(1-p^i)$. Thus, 
$$
                \prob[\rho \text{ is rejecting}] = \prod_{i=1}^n\left(1-p^{i}\right).
$$
The above quantity is positive due to \cref{prop:complexanalysisConverge}, and thus $\rho$ is not almost-surely accepting, as desired.
\end{proof}

\subsubsection*{Converting SR automata to MA automata}
Next we focus on providing a detailed proof for \cref{theorem:coBuchiHDisSR}.
\theoremcobuchisrtoma*
To prove \cref{theorem:coBuchiHDisSR}, we fix a coB\"uchi automaton $\Ac$ that is stochastically resolvable. We do the following relabelling of priorities on $\Ac$ to get another coB\"uchi automaton $\Cc$, such that a run in $\Ac$ is accepting if and only if that run is accepting in~$\Cc$. 
\paragraph*{Priority-reduction} Consider the graph $G$ consisting of the states of $\Ac$ and the transitions of $\Ac$ that have priority~$0$. Consider the strongly connected components of this graph, and for any priority $0$ transition that is not in any SCC, we change its priority to $1$ in $\Cc$. The rest of the transitions in $\Cc$ have the same priority as in $\Ac$.

\begin{proposition}\label{prop:priority-reduction}
A run in $\Cc$ is accepting if and only if the corresponding run is accepting in $\Ac$.
\end{proposition}
\begin{proof}
    Let $\rho$ be a run. If $\rho$ is a rejecting run in $\Ac$, then it contains infinitely many priority $1$ transitions in $\Ac$. It is clear then that $\rho$ contains infinitely many priority $1$ transitions in $\Cc$ as well. 
    
    Otherwise, if $\rho$ is accepting in $\Ac$ then $\rho$ contains finitely many priority $1$ transitions in $\Ac$. Thus, $\rho$ eventually stays in the same SCC in $G$, and therefore, $\rho$ contains finitely many priority $1$ transitions in~$\Cc$.
\end{proof}
It follows from \cref{prop:priority-reduction} that $\Ac$ and $\Cc$ are language equivalent, and since $\Ac$ is SR, so is $\Cc$: any almost-sure resolver for $\Ac$ is also an almost-sure resolver for $\Cc$.
\begin{corollary}\label{cor:lang-c-is-lang-a}
The automaton $\Cc$ is language equivalent to $\Ac$ and stochastically resolvable.
\end{corollary}

Recall the safety-automaton $\Cc_{\safety}$ that we had defined in \cref{subsec:sac-cobuchi}. The following observation is easy to see.

\begin{proposition}\label{prop:safe-cobuchi-langcomparison}
    For every state $q$ in $\Cc$, $\Lc(\Cc_{\safety},q)\subseteq \Lc(\Cc,q)$.
\end{proposition}
\begin{proof}
    If $\rho$ is an accepting run on $w$ in $(\Cc_{\safety},q)$, then the same run $\rho$ does not contain any priority $1$ transition in $(\Cc,q)$, and therefore, is accepting.
\end{proof}

Recall the definition of SR-covers and SR self-coverage presented in \cref{subsec:sac-buchi}. We next show that $\Bc$ has SR self-coverage.
\lemmaSRhassafeSRcoverage*
\begin{proof}
We fix an almost-sure resolver $\Mc$ for Eve in $\Cc$,  and let $\Pc$ be the probabilistic automaton that is the resolver-product of $\Mc$ and $\Cc$ (see \cref{sec:prelims}). We define $\Pc_{\safety}$ as a safety probabilistic automaton that is the \emph{safe-approximation} of $\Pc$ as follows, similar to how we defined $\Cc_{\safety}$. 
The automaton $\Pc_{\safety}$ has the same states as $\Pc$, and in addition, a rejecting sink state $q_{\bot}$. 

For each transition $\delta=(q,m)\xrightarrow{a:0}(q',m')$ in $\Pc$ of priority $0$ that has probability $p$, we add the transition $(q,m)\xrightarrow{a:0} (q',m')$ in $\Pc_{\safety}$ with the same probability $p$. For each state $(q,m)$ and letter $a$, we add a transition $(q,m)\xrightarrow{a:1}q_{\bot}$ that has probability $p'$, so that the sum of  the probabilities of all outgoing transitions from $(q,m)$ on $a$ is~$1$.

To prove \cref{lemma:coBuchiSRhassafeSRcoverage}, we suppose, towards contradiction, that
there is a state $q$ in $\Cc$, such that for every state $p$ coreachable to $q$ in $\Cc$, $(\Cc_{\safety},p)$ does not SR-cover $(\Cc_{\safety},q)$. In particular, for every state $(p,m)$ in $\Pc$, where $p$ is coreachable to $q$ in $\Cc$, we have $\Lc(\Pc_{\safety},(p,m)) \subsetneq \Lc(\Cc_{\safety},q)$. 

This implies that there is a finite word $u_{(p,m)}$ such that there is a run from $q$ to some state $q'$ on $u_{(p,m)}$ in $\Cc_{\safety}$ that only consists of safe transitions, while any run of $\Pc_{\safety}$ from $(p,m)$ on $u_{(p,m)}$ reaches the rejecting sink state $q_{\bot}$ with a positive probability $\epsilon_{(p,m)}$. Recall that we modified $\Ac$ to obtain $\Cc$ so that each priority $0$ transition occurs in an SCC consisting of only priority $0$ transitions. Thus, there is a word $v_{(p,m)}$ on which there is run from $q'$ to $q$ in $\Cc_{\safety}$ that contains only safe transitions. Consider the finite word $\alpha_{(p,m)}=u_{(p,m)} v_{(p,m)}$. Then there is a run from $q$ to $q$ in $\Cc$ on $\alpha_{(p,m)}$ that contains only priority $0$ transitions, while a run from $(p,m)$ on $\alpha_{(p,m)}$ in $\Pc$ contains a transition of priority~$1$ with probability $\epsilon_{(p,m)}>0$. 

We define the words $\alpha_{(p',m')}$ and the real number $\epsilon_{(p',m')}>0$ similarly, for all states $(p',m')$ in $\Pc$, such that $p'$ is coreachable to $q$ in $\Cc$. Define $\epsilon>0$ as the quantity $$\epsilon=\min\{\epsilon_{(p,m)}\mid (p,q) \in \CR(\Cc) \}.$$  

We will describe a strategy for Adam in the SR game on $\Cc$, using which he wins almost-surely against Eve's strategy $\Mc$. Adam starts by giving a finite word $u_q$, such that there is a run of $\Cc$ from its  initial state to $q$. Then Adam, from this point and at each \emph{reset}, selects a state $(p,m)$ of $\Pc$ uniformly at random, such that $p$ is coreachable to $q$ in $\Cc$ and $m$ is a memory-state in $\Mc$. He then plays the letters of the word $\alpha_{(p,m)}$ in sequence. Adam then \emph{resets} to select another such state $(p',m')$ with $(p',q) \in \CR(\Cc)$ and plays similarly. 

Consider a play of the SR game on $\Cc$ where Eve is playing according to her strategy $\Mc$ and Adam is playing according to the strategy described above. Note that at each reset, Eve's token is at a state $p$ that is weakly coreachable to $q$ and Eve has the memory $m$. Let $\lvert\Pc\rvert$ be the number of states in the probabilistic automaton. Adam picks the state $(p,m)$ at that reset with probability at least $\frac{1}{\lvert \Pc \rvert}$, from which point Eve's run on the word $\alpha_{(p,m)}$ contains a transition of priority $1$ with probability at least $\epsilon$. Thus, between every two consecutive resets, Eve's token takes a priority $1$ transition with probability at least $\frac{\epsilon}{\lvert \Pc \rvert}$. By the second Borel-Cantelli lemma, the run on Eve's token contains infinitely many priority $1$ transitions and hence is rejecting with probability 1. Thus, $\Mc$ is not an almost-sure resolver for Eve (\cref{lemma:random-is-pure}), which is a contradiction.  
\end{proof}

We note that SR-covers is a transitive relation.
\begin{lemma}\label{lemma:sr-cover-transitivity}
If $\Ac_1,\Ac_2,\Ac_3$ are three nondeterministic parity automata such that  $\Ac_1 \succ_{SR} \Ac_2$ and $\Ac_2 \succ_{SR} \Ac_3$, then $\Ac_1 \succ_{SR} \Ac_3$.
\end{lemma}
\begin{proof}
    Observe that if $\Ac_2 \succ_{SR} \Ac_3$, then $\Lc(\Ac_2) \supseteq \Lc(\Ac_3)$. Thus, if Eve has a strategy to construct a run in $\Ac_1$ that is almost-surely accepting on any word in $\Lc(\Ac_2)$, then the same strategy constructs a run that is almost-surely accepting on any word in $\Lc(\Ac_3)$. Thus, $\Ac_1$ SR-covers $\Ac_3$, as desired. 
\end{proof}

Thus, the following result follows from the definition of SD self-coverage and \cref{lemma:sr-cover-transitivity} above.

\lemmacobuchisometingdbp*
\begin{proof}
    Consider the directed graph $H$ whose vertices are states of $\Cc$. We add an edge from $q$ to $p$ in $\Cc$ if $(q,p)$ in $\Cc$ and $(\Cc_{\safety},p)$ SR-covers $(\Cc_{\safety},q)$. Note that if there is a path from $r$ to $s$ in $H$, then $(\Cc_{\safety},s)$ SR-covers $(\Cc_{\safety},r)$ and $(r,s)\in\WCR(\Cc)$. Since $\Cc$ has SR self-coverage, we note that every vertex has outdegree $1$. Thus, for every vertex $q$, there is a vertex $p$, such that there is a path from $q$ to $p$ and a path from $p$ to $p$ in $\Hc$. The conclusion follows.
\end{proof}

Note that if for some state $p \in \Cc$,  $(\Cc_{\safety},p)$ SR-covers $(\Cc_{\safety},p)$, then $(\Cc_{\safety},p)$ is SR, and therefore, $(\Cc_{\safety},p)$ is pre-SD (\cref{lemma:SR-implies-SD}), and therefore determinisable-by-pruning (\cref{lemma:sd-safety-is-dbp}). We thus call a state $p$ of $\Cc$ as \emph{safe-deterministic} if $(\Cc_{\safety},p)$ is SR. 

\paragraph*{Construction of $\Hc$} We will construct an MA automaton $\Hc$ that is language-equivalent to $\Cc$. The states of $\Hc$ consists of states that are safe-deterministic in $\Cc$.

For the transitions of $\Hc$, we start by fixing a uniform determinisation of transitions from every state that is safe-deterministic in $\Cc$ to obtain $\Cc'$, so that for every safe-deterministic state $q$, $\Lc(\Cc_{\safety},q)=\Lc(\Cc'_{\safety},q)$. Such a determinisation exists since $(\Cc_{\safety},q)$ is DBP. 

We add the transitions of $\Cc'$ in $\Hc$. Then, for every state $p$ and letter $a$ in $\Hc$, we add priority $1$ transitions from $p$ on $a$ to every state $q$ that is safe-deterministic in $\Cc$ and weakly coreachable to an $a$-successor of $p$ in~$\Cc$. 

We let the initial state of $\Hc$ be a safe-deterministic state that is weakly coreachable to the initial state of $\Cc$. Note that such a state exists due to \cref{lemma:cobuchi-srselfcoverage-implies-somtingsdbp}. This concludes our description of $\Hc$. 

In the next two lemmas, we show that $\Lc(\Hc)=\Lc(\Cc)$ and that $\Hc$ is MA.

\begin{lemma}\label{lemma:cobuchi-lang-equivalence}
    The automata $\Hc$ and $\Cc$ are language-equivalent.
\end{lemma}
\begin{proof}
    \textit{{$\Lc(\Cc) \subseteq \Lc(\Hc)$}:} Let $w$ be a word in $\Lc(\Cc)$, and $\rho$ an accepting run of $\Cc$ on $w$. Then, there is a decomposition~$w=uw'$, such that $u \neq \epsilon$ and $\rho$ after reading the prefix $u$ does not contain any priority $1$ transition on the suffix~$w'$ of~$w$. 
    Suppose $\rho$ is at the state $q$ after reading $u$. Then there is a safe-deterministic state $p$, such that $p$ is safe-deterministic, $(p,q) \in \CR(\Cc)$, and $(\Cc_{\safety},p)$ SR-covers $(\Cc_{\safety},q)$. Thus, $\Lc(\Cc_{\safety},p)\supseteq \Lc(\Cc_{\safety},q)$. Since $w' \in \Lc(\Cc_{\safety},q)$, it follows that $w' \in \Lc(\Cc_{\safety},p)$.
    
    Observe that $(\Cc_{\safety},p)=(\Hc_{\safety},p)$, and for each state in $\Hc$ and a letter in $\Sigma$, there is at most one priority 0 transition from that state on that letter. Thus, there is a unique run from $p$ on $w'$ in $\Hc$ that contains only priority 0 transitions. Consider the run of $\Hc$ on $w$ that takes arbitrary transitions until reading the penultimate letter of $u$, and then takes the transition to $p$ on the last letter of $u$, and then follows the unique run from $p$ on $w'$ that contains only priority 0 transitions. This is an accepting run of $\Hc$ on $w$, and thus $w \in \Lc(\Hc)$.
    
    \textit{{$\Lc(\Hc) \subseteq \Lc(\Cc)$}:} Let $w$ be a word in $\Lc(\Hc)$, and $\rho$ an accepting run of $\Hc$ on $w$. Then, there is a decomposition of $w$ as $uw'$, such that $\rho$ after reading $u$ does not contain any priority 1 transition on $w'$. Suppose $\rho$ is at the state $p$ after reading $u$. Then, $w' \in \Lc(\Hc_{\safety},p) = \Lc(\Cc_{\safety},p) \subseteq \Lc(\Cc,p)$, where the last equality holds due to \cref{prop:safe-cobuchi-langcomparison}. Recall that $\Cc$ is pre-semantically deterministic (\cref{lemma:SR-implies-SD}), i.e., contains a language-equivalent SD subautomaton $\Cc'$. Let $q$ be a state in $\Cc'$ that is reached after reading the word $u$. We will show the following claim.

    
    \begin{claim}\label{claim:cobuchilangquiv}
        $\Lc(\Cc,p) \subseteq \Lc(\Cc',q).$
    \end{claim}
            Note that $p$ and $q$ are weakly coreachable in $\Cc$. Thus, there is a sequence of states $p_1,p_2,\dots,p_k$ in $\Cc$ and finite words $u_0, u_1,\dots,u_k$, such that there are runs from the initial state of $\Cc$ to both $p$ and $p_1$ on the word $u_0$, to $p_i$ and $p_{i+1}$ on the word $u_i$ for each $i \in [1,k-1]$, and to $p_k$ and $q$ on $u_k$. We can pick the states $p_1,p_2,\dots,p_k$, such that there is a run from $q_0$ to $p_i$ in $\Cc'$ on the words $u_{i-1}$ and $u_{i}$ for each $i \in [1,k-1]$. Since $\Cc'$ is SD, this implies that $\Lc(\Cc',q)=\Lc(\Cc',p_1)$. Note, due to \cref{lemma:SDautomata}, that $$\Lc(\Cc,p) \subseteq u^{-1}\Lc(\Cc)=u^{-1}\Lc(\Cc')= \Lc(\Cc',q),$$  and thus the proof of the claim follows.
   
Using \cref{claim:cobuchilangquiv}, we note that there is an accepting run $\rho'$ on $w=uw'$ in $\Cc'$ which follows a run to $q$ on the word $u$, and then since $w' \in \Lc(\Cc,p) \subseteq \Lc(\Cc',q)$, follows an accepting run from $q$ on the word $w'$ in $\Cc'$. Since $\Cc'$ is a subautomaton of $\Cc$, $\rho'$ is also an accepting run of $\Cc$ on $w$, as desired.
\end{proof}

We next show that $\Hc$ is MA.
\begin{lemma}\label{lemma:cobuchi-h-is-ma}
    The coB\"uchi automaton $\Hc$ is a memoryless-adversarially resolvable automaton.
\end{lemma}
\begin{proof}
    Consider the following memoryless strategy for Eve in the HD game on $\Hc$, where from the state $q$ on the letter $a$:
    \begin{enumerate}
        \item if there is a priority $0$ transition from $q$ on $a$, then she picks that transition (observe that such a transition is unique);
        \item Otherwise, she picks an outgoing priority $1$ transition from $q$ on $a$ uniformly at random.
    \end{enumerate}
    We claim that this strategy is winning for Eve in the HD game on $\Ac$. To see this, consider a play in which Adam in the HD game is constructing a word letter-by-letter and Eve is building a run according to the above strategy. If Adam produces a word that is not in $\Lc(\Hc)$, then Eve wins trivially. Otherwise, eventually, Adam's word must have a prefix $u$ and a run which is at a state $p$ after reading $u$ and after which the suffix $w'$ that Adam constructs in the rest of the rounds by his letters is in $\Lc(\Hc_{\safety},p)$. Observe that the run from $p$ on $w'$ in $(\Hc_{\safety},p)$ is unique.
    Suppose, Eve's token is at the state $q$ after the word $u$ is read. Consider the run of Eve from $q$ where she picks transitions according to her strategy above, while Adam builds a word such that there is a unique run from $p$ on that word consisting of only priority 0 transitions. For every finite word $v$ that is a prefix for some infinite word in $\Lc(\Hc_{\safety},p)$, let $p_v$ be the unique state to which there is a run from $p$ to $p_v$ consisting of only priority 0 transitions.
    
    Then, after a word $v$ is read and Adam chooses a letter $a$, whenever Eve's token is at a state $q_v$ and Eve has no priority $0$ transition available to her, she takes the transition to $p_{va}$ with probability at least $1-\frac{1}{\lvert \Hc \rvert}$, where $\lvert \Hc \rvert$ is the number of states of $\Hc$. If Eve's token after the word $v'$ is read is at $p_{v'}$, then it is clear that Eve wins the HD game from here on. Thus, 
    \begin{equation*}
        \begin{split}
            &\prob[\text{Eve's run is rejecting}]\\ 
            &= \prob[\text{Eve takes infinitely many priority $1$ transitions and } \\ &\text{never takes the transition to $p_v$ after the word $v$}] \\ 
            &\leq \prod_{n\in\mathbb{N}} (1-\frac{1}{|\Hc|}) = 0. 
        \end{split}
    \end{equation*}
    Thus, Eve's run is almost-surely accepting, as desired.
\end{proof}

We have thus proved so far that every SR automaton has a language-equivalent MA automaton with at most as many states. We next show that we can find such an MA automaton for every input SR automaton efficiently, thus proving \cref{theorem:coBuchiHDisSR}.
\theoremcobuchisrtoma*
\begin{proof}
    Let $\Ac$ be an SR coB\"uchi automaton. The priority-reduction procedure relabels the priorities of transitions of  $\Ac$ to obtain a coB\"uchi automaton $\Cc$, in which every priority $0$ transitions occurs in a strongly connected component consisting of only priority $1$ transitions. This procedure is efficient since SCCs can be computed in linear time~\cite{Tar72}. From \cref{cor:lang-c-is-lang-a}, the automaton $\Cc$ is language-equivalent to $\Ac$ and $\Cc$ is stochastically resolvable. We then find states $p$ in $\Cc$, such that $(\Cc_{\safety},p)$ is HD and find a pure positional strategy from all such states. These are the safe-deterministic states, since SR safety automata are determinisable-by-pruning. Such states and this strategy can be found efficiently \cite[Theorem 4.5]{BL23quantitative}. Then, construction of $\Hc$ we described takes polynomial time, since the relations of coreachability and weak-coreachability can be computed in $\ptime$. This automaton $\Hc$ has as many states as $\Cc$ and hence $\Ac$, is language-equivalent to $\Ac$ (\cref{lemma:cobuchi-lang-equivalence}), and is MA (\cref{lemma:cobuchi-h-is-ma}). This concludes our proof.
\end{proof}
\subsection{B\"uchi automata}\label{succandcompBuchi}
\buchisdnotsr*
\begin{proof}
    Consider the B\"uchi automaton $\Bc$ in \cref{fig:BuchiSDbutnotSR}, which we reillustrate in \cref{fig:BuchiSDbutnotSRAgain} for the reader's convenience.  This automaton $\Bc$ has nondeterminism on the initial state $q_0$, and it recognises the language $$((x \cdot (a+b)\cdot y)^{*}(x\cdot (a+b)\cdot z))^{\omega}.$$ It is easy to verify that $\Bc$ is SD.
    \begin{figure}[ht]
    \centering
        \begin{tikzpicture}[auto]
        \tikzset{every state/.style = {inner sep=-3pt,minimum size =15}}

    
    \node[state] (s1)  at (0,0) {$q_0$};
    \node[state]  (s2)  at (2,0) {};
    \node[state] (s3) at (1,0.8) {$q_a$};
    \node[state] (s4) at (1,-0.8) {$q_b$};

    \node[state] (f1)  at (-1.2,0) {};
    \node[state]  (f2)  at (-2.2,0.8) {};
    \node[state] (f3) at (-2.2,-0.8) {};

    \path[->]
        (f3) edge node [left] {$x$}  (f2)
        (f1) edge node [yshift=1mm] {$y$} (f3)
        (f2) edge node [xshift=-2mm] {$a,b$} (f1)
        (s2) edge [double,bend left = 8] node {$z$}  (s1)
          (s1) edge  node [below,xshift=2mm,yshift=2mm] {$x$} (s3)
         (s1) edge  node [above,xshift=2mm,yshift=-2mm] {$x$} (s4)
         
         (s3) edge  node [above] {$a$} (s2)
         (s4) edge  node [below] {$b$} (s2)
         
         (s2) edge [double,bend left = 8] node [below] {$z$} (s1)
         (s2) edge [bend right = 8] node [above] {$y$} (s1)
         (s3) edge  node [above] {$b$} (f1)
         (s4) edge node [below] {$a$} (f1)
         (f1) edge node [above,xshift=1mm,yshift=-0.5mm] {$z$} (s1)
;
    \path[->,every node/.style={sloped,anchor=south}]
        ;
    \end{tikzpicture}
    \caption{A semantically deterministic B\"uchi automaton that is not stochastically resolvable. The accepting transitions are double-arrowed, and the initial state is $q_0$.}
    \label{fig:BuchiSDbutnotSRAgain}
\end{figure} 
    
    We will describe a strategy for Adam in the SR game on $\Bc$ using which Adam wins almost-surely in the SR game. This would imply, due to \cref{lemma:random-is-pure}, that $\Bc$ is not SR. 
    
    Note that when Eve's token is at $q_0$ in the SR game, Eve needs to guess whether the next letter is going to be $a$ or $b$. If she guesses incorrectly then her token moves to the left states--- states $l_1,l_2,$ and $l_3$, where she stays until a $z$ is seen. Adam's strategy in the SR game is as follows. Let $Y$ be the regular expression $xay+xby$ and $Z$ be the regular expression $xaz+xbz$. Note that both $Y$ and $Z$ consist of two words. Adam picks a word from the set $YZY^2ZY^3ZY^4Z \dots$ in the SR game, where from each occurrence of $Y$ or $Z$, he picks one of the two words in the regular expression with half probability. We next show that the probability $p_n$ that Eve's token, starting at $q_0$, takes an accepting transition on reading a word chosen randomly from $Y^{n}Z$ is $\frac{1}{2^{n+1}}$. Indeed, note that $1-p_n$ is same as the probability that Eve's token does not reach the left states on the word $Y^n Z$. This is the case only if, whenever Eve's token is at the state $q_a$ (resp.\ $q_b$), Adam picks the letter $a$ (resp.\ $b$). Thus, the probability that Eve's token does not reach a left state on the word $Y^n Z$ is $\frac{1}{2^{n+1}}$, and hence $p_n=\frac{1}{2^{n+1}}$. Thus, in the SR game where Adam picks the word as above,

  $$
         \prob[\text{Eve's run is accepting}] = \sum_{n\geq 1} \frac{1}{2^{n+1}} = \frac{1}{2}.
$$It then follows from the Borel-Cantelli lemma (\cref{lemma:borellcantelli}) that the probability that Eve's token takes infinitely many accepting transitions in the SR game is 0, as desired. 
\end{proof} 

\HDBuchinotMR*
\begin{proof}[Proof of \cref{lemma:HDBuchinotMR}]
    Consider the B\"uchi automaton $\Bc$ shown in \cref{fig:HDBuchinotMR}, which we re-illustrate  below for convenience.
    Let $\Sigma_{\diamond}=\{a,b,c,\diamond\}$ and $\Sigma=\{a,b,c\}$. Then the B\"uchi automaton $\Bc$ recognises the language  $\left[(L_1+L_2)^*(L_1L_1+L_2L_2)\right]^\omega$, where $L_1 = {\Sigma_\diamond}^*  c^+\diamond $ and $L_2 =  
    {\Sigma_\diamond}^* a \Sigma^* b^+\diamond
    $.  Equivalently, it accepts words in $(L_1+L_2)^\omega$ that are however not in $(L_1+L_2)^*(L_1L_2)^\omega$.  
    \begin{figure}[ht]
\centering
        \begin{tikzpicture}
        \tikzset{every state/.style = {inner sep=-3pt,minimum size =15}}

    \node[state] (q0) at (-1,1.5) {};
    \node[state] (q1) at (-2.5,1.5) {};
    \node[state] (q2) at (2,1.5) {};
    \node[state] (q3) at (0.5,1.5) {};
    \path[->] (-0.7,1) edge (q0);
    
    \node[state, fill=blue!40, blue!40] (r0) at (2,0.2) {$q_B$};
    \node[state, blue] (r1) at (2,-1.3) {};
    \node[state, blue] (r2) at (3.5,-1.3) {};
    \node[state, blue] (r3) at (3.5,0.2) {};    
    \node[state, blue] (r4) at (5.25,-0.55) {};    

    \node[state, fill=red!40, red!40] (l0) at (-3,0.2) {$q_R$};
    \node[state, red] (l1) at (-3,-1.3) {};
    \node[state, red] (l2) at (-1.5,-1.3) {};
    \node[state, red] (l3) at (-1.5,0.2) {};    
    \node[state, red] (l4) at (0.25,-0.55) {};    

    \node[state, fill=red!40, red!40] (l5) at (-2.25,-2.1) {$q_R$};    
    \node[state, fill=blue!40, blue!40] (l6) at (0.25,-1.8) {$q_B$};  
    
    \node[state, fill=red!40, red!40] (r5) at (2.75,-2.1) {$q_R$};    
    \node[state, fill=blue!40, blue!40] (r6) at (5.25,-1.8) {$q_B$};    
    \node (l52) at (-2.25,-2.1) {$q_R$};    
    \node (l62) at (0.25,-1.8) {$q_B$};  
    
    \node (r51) at (2.75,-2.1) {$q_R$};    
    \node (r61) at (5.25,-1.8) {$q_B$};    
    \node (l01) at (-3,0.2) {$q_R$};
    \node (r01) at (2,0.2) {$q_B$};
    
    \path[-stealth]
%    (-0.5,-0.5) edge (q0)
    (q0) edge [loop above] node [right] {$a,b,\diamond$} (q0)
    (q0) edge [bend left = 8] node [above] {$a$} (q3)
    (q3) edge [bend left = 8] node [below] {$\diamond$} (q0)
    (q0) edge [bend left = 8] node [below] {$c$} (q1)
    (q1) edge [bend left = 8] node [above] {$b,a$} (q0)
    (q3) edge [bend right = 8] node [below] {$b$} (q2)
    (q2) edge [bend right = 8] node [above] {$c,a$} (q3)
    (q3) edge [loop above] node [right] {$a,c$} (q3)
    (q1) edge [loop above] node [left] {$c$} (q1)
    (q2) edge [loop right] node [right] {$b$} (q2)

    (l0) edge [loop above] node [above] {$b,\diamond$} (l0)
    (l0) edge [bend left = 8] node [above] {$a$} (l3)
    (l3) edge [bend left = 8] node [below] {$\diamond$} (l0)
    (l1) edge  node [below, left] {$a$} (l3)
    (l0) edge  node [left] {$c$} (l1)
    (l3) edge [bend right = 8] node [left] {$c$} (l2)
    (l2) edge [bend right = 8] node [right] {$a$} (l3)
    (l3) edge [bend right = 8] node [below=3pt,left] {$b$} (l4)
    (l4) edge [bend right = 8] node [above=3pt,right] {$a$} (l3)
    (l4) edge [bend right = 8] node [above=0.5pt] {$c$} (l2)
    (l2) edge [bend right = 8] node [below=0.5pt] {$b$} (l4)
    
    (l3) edge [in=30,out=60,loop] node [right] {$a$} (l3)
    (l1) edge [in=240,out=270,loop] node [below] {$c$} (l1)
    (l2) edge [in=-30,out=-60,loop]  node [right] {$c$} (l2)
    (l4) edge [loop above] node [above] {$b$} (l4)
    

    (r0) edge [in=165,out=195,loop] node [left] {$b,\diamond$} (r0)
    (r0) edge [bend left = 8] node [above] {$a$} (r3)
    (r3) edge [bend left = 8] node [below] {$\diamond$} (r0)
    (r1) edge  node [below, left] {$a$} (r3)
    (r0) edge  node [left] {$c$} (r1)
    (r3) edge [bend right = 8] node [left] {$c$} (r2)
    (r2) edge [bend right = 8] node [right] {$a$} (r3)
    (r3) edge [bend right = 8] node [below=3pt,left] {$b$} (r4)
    (r4) edge [bend right = 8] node [above=3pt, right] {$a$} (r3)
    (r2) edge [bend right = 8] node [below=0.5pt] {$b$} (r4)
    (r4) edge [bend right = 8] node [above=0.5pt] {$c$} (r2)

    (r3) edge [in=30,out=60,loop]  node [above] {$a$} (r3)
    (r1) edge [loop left] node [left] {$c$} (r1)
    (r2) edge [in=-30,out=-60,loop] node [right] {$c$} (r2)
    (r4) edge [loop above] node [above] {$b$} (r4)

    (q1) edge [in=50,out=-90] node [right] {$\diamond$} (l0)
    (q2) edge node [right] {$\diamond$} (r0)

    (l1) edge[ double] node [left] {$\diamond$} (l5)
    (l2) edge[ double] node [right] {$\diamond$} (l5)
    (l4) edge node [left] {$\diamond$} (l6)
    %(l4) edge [out=60,in=180] node [below] {$\diamond$} (r0)

    (r1) edge node [left] {$\diamond$} (r5)
    (r2) edge node [right] {$\diamond$} (r5)
    (r4) edge [double] node [left] {$\diamond$} (r6)
;
    \end{tikzpicture}
\caption{A HD B\"uchi automaton that is not MR. The accepting transitions are represented by double arrows. All red-filled states ($q_R$) are identified as the same state, and all blue-filled states ($q_B$) are identified as the same state.}\label{fig:HDBuchinotMRAgain}
\end{figure}
    This automaton only has runs on words of the form $(L_1+L_2)^\omega$. When viewed as a finite-state automaton restricted to red (resp.\ red) states where the B\"uchi transitions are accepting transitions and $q_R$ (resp.\ $q_B$) is the initial state, this automaton accepts words in $L_1$ (resp.\ $L_2$). 
    
    For a run to contain the accepting transitions infinitely often, observe that it must visit the state $q_R$ or $q_B$ after reading some prefix in $(L_1+L_2)^*L_1$ and $(L_1+L_2)^*L_2$, respectively. Furthermore, observe that runs of words in $L_1$ and $L_2$ that start from the states $q_R$ or $q_B$, respectively, visit an accepting transition and then end at $q_R$ and $q_B$, respectively. %Therefore, there are runs on words in  $(L_1+L_2)^*(L_1L_1)$ or $(L_1+L_2)^*(L_2L_2)$ which sees an accepting transition. Since the automaton from  $q_R$ or $q_B$ is deterministic, we can also see that there is a run starting at the red or blue state for words in $(L_1+L_2)^*(L_1L_1)$ or $(L_1+L_2)^*(L_2L_2)$, respectively, that visits an accepting state.

\paragraph*{Eve's strategy in the HD game}
The only state with nondeterminism in automaton $\Ac$ is on the state $q_0$ on the letter $a$, from where  Eve can either choose to keep her token in $q_0$ using the transition that is the self loop: $\delta_1$, or she can move her token along the $a$-transition that goes right: $\delta_2$.  
Intuitively, Eve needs to guess at $q_0$ whether the word being input from now is going to be $L_1$ or $L_2$. If she guesses incorrectly, then her token ends up at the starting state, and she can guess again. If the resolver guesses correctly, Eve's token goes to the deterministic part of the game, i.e., the red or blue state, from where she wins the HD game. 

 We claim that the following Eve's strategy in the HD game is winning, where when she is at the initial state $q$ and Adam gives the letter $a$: if the longest prefix of the input word so far is in the language $(L_1 + L_2)^*L_1$ (rather than $(L_1 + L_2)^*L_2$), then she chooses the transition $\delta_1$ on her. If the longest prefix of the input word so far is in the language $(L_1 + L_2)^*L_2$ instead, then she chooses $\delta_2$ instead. 

Observe that for any word in the language, there are infinitely many prefixes which are either in the language $(L_1 + L_2)^*L_1L_1$ or $(L_1 + L_2)^*L_2L_2$. Consider the first time that the prefix is in $(L_1 + L_2)^*L_1L_1$ (the case for the prefix in the language $(L_1 + L_2)^*L_2L_2$ is similar). Let this prefix be denoted by $w \cdot u\cdot v$, such that $w\in (L_1 + L_2)^*$ and $u$ and $v$ are both words in $L_1$. Although there are many decompositions possible, we find one which ensures that the length of $u$ and $v$ are the shortest. 

Suppose Eve's token takes the transition $\delta_1$ after reading $w$. Then on reading a word in $L_1L_1$, her token would visit an accepting state, and therefore a deterministic component. Otherwise, is she chose the transition $\delta_2$ on reading $w$, then after reading the word $w\cdot u$, the longest prefix of $w\cdot u $ is in the language $(L_1 + L_2)^*L_1$, and therefore the she would chose the transition $\delta_1$. Using this transition and continuing to read a word in $L_1$, the run on her token reaches the deterministic part of the automaton, from where Eve wins the HD game. 

\paragraph*{No memoryless stochastic resolver}
We show that any resolver $\Mc_p$ that assigns with probability $0\leq p \leq 1$, the transition $\delta_1$ and with probability $1-p$, the transition $\delta_2$,
is not an almost-sure resolver. Note that if either of $\delta_1$ or $\delta_2$ is removed from the automaton $\Bc$, then the language changes, and hence if $p=0$ or $p=1$, $\Mc_p$ is not an almost-sure resolver. We therefore suppose that $0<p<1$.

We will construct a word in the language that is accepted with probability $<1$ by the resolver-product $\Mc_p \circ \Bc$. Consider the word $w = ac\diamond a^2c\diamond a^3c\diamond\dots \diamond a^{n}c\diamond\dots$, which is in the language since $w$ is in $(L_1)^\omega$. 
A run on $w$ is accepted if and only if some finite prefix of the run on the word visits the red state. For the run to visit a red state, the transition $\delta_1$ should be chosen by the resolver at every step wh $a^kc\#$ for some $k$. 

The probability that a run of the word $a^kc\diamond$ constructed using $\Mc_p$ starting from $q_0$ and ends at the red state $q_r$ is $p^k$, since the probability of $\delta_1$ being chosen at every step on a word $a^k$ is $p^k$. 
Therefore, the probability that a run constructed using $\Mc_p$ on the word $a^kc\diamond$ starting from $q_0$ \emph{does not} ends in the red state is $1-p^k$.

The probability that on the finite word $w_n = ac\diamond a^2c\diamond a^3c\diamond\dots a^{n}c\diamond$, a run constructed using the resolver $\Mc_p$,  \emph{does not} visit the red state even once is the probability that it does not visit the red state for any substring $a^{i}c\diamond$ which is 
\begin{align*}
\Pr[\text{a run on $w_n$ constructed using }\Mc_p&\\\text{ does not visit a red state}] &= \prod_{i=1}^n\left(1-p^{i}\right)    
\end{align*} 


For the infinite word $w = ac\diamond a^2c\diamond a^3c\diamond\dots a^{n}c\diamond\dots$, again using the resolver $\Mc_p$,
\begin{align*}
    \Pr[\text{a run on $w$ using resolver }\Mc_p & \\\text{ does not } \text{visit a red state}] & = \prod_{i=1}^\infty\left(1-p^{i}\right)
\end{align*} 

Since the value $\sum_i^\infty |-p^i| <\infty$ for $0<p<1$ and each $p^i$ is positive, we obtain that $\prod_{i=1}^\infty\left(1-p^{i}\right)$ converges to a positive value (see \cref{prop:complexanalysisConverge}). We can argue further that since each of the elements in the product is strictly smaller than $1-p$, but strictly larger than $0$, therefore $0<\prod_{i=1}^\infty\left(1-p^{i}\right)<1-p$.

This shows that the word $w$ is not accepted with probability~$1$ using $\Mc_p$ as a resolver for any $0<p<1$ and therefore this automaton is not MR.
\end{proof}

