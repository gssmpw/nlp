We now turn our attention to the computational complexity for the problems of deciding if a given automaton is MR, SR, or MA, respectively. The exact complexity of deciding if a given automaton is HD is an open problem since 2006, with only recent results showing that the problem is $\NP$-hard~\cite{Pra24a} and in $\PSPACE$~\cite{LP25}. 

We discuss the key results of this section. Firstly, unlike for history-determinism where the problem of deciding if the automaton is HD has a  complexity gap for parity automata, we show that the problem for memoryless adversarial resolvability is $\NP$-complete.
\begin{restatable}{theorem}{theoremmanpcomplete}\label{theorem:manpcomplete}
    The problem of deciding if a given parity automaton is memoryless-adversarially resolvable is $\NP$-complete. 
\end{restatable}
Later, we turn our attention to problems of checking if an automaton is SR and the problem of checking if a resolver is an almost-sure resolver in the stochastically resolvable setting. 
\begin{restatable}{theorem}{complexity}\label{theorem:complexity}
\begin{enumerate}
    \item The problem of deciding if a given safety automaton is stochastically resolvable or memory\-less-stochastically  resolvable is in $\ptime$.
    \item The problem of deciding if a given reachability or weak automata is stochastically resolvable or memoryless stochastically resolvable is $\PSPACE$-complete.
    \item The problem of deciding, for a given B\"uchi or coB\"uchi automaton and a finite memory resolver for that automaton, if the resolver is an almost-sure resolver of that automaton is undecidable.
\end{enumerate}
\end{restatable}
\subsection{Memoryless adversarially resolvable automata}
We first prove \cref{theorem:manpcomplete} and show $\NP$-completeness for the problem of deciding if an automaton is MA. To prove the upper bound of $\NP$ (\cref{lemma:MAcheckNPEasy}), we show that an automaton is MA if and only if there is a specific kind of strategy in the so-called $2$-token game on that automaton. 
The $2$-token game, defined by Bagnol and Kuperberg, was introduced as a game that characterises history determinism for B\"uchi automata~\cite{BK18}. Lehtinen and Prakash have recently shown that these games also characterise history determinism for parity automata~\cite{LP25}. Using the characterisation of MA using $2$-token games, we provide an algorithm for verifying if such a specific kind of strategy is indeed a winning strategy. We then show the lower bound in \cref{lemma:MAcheckNPhard} by showing that the problem is $\NP$-hard, similar to the proof of $\NP$-hardness for deciding history-determinism~\cite{Pra24a}. 

\begin{lemma}\label{lemma:MAcheckNPEasy}
Checking if a given parity automaton is memoryless adversarially resolvable is in $\NP$.
\end{lemma}
Next, we describe a proof sketch for \cref{lemma:MAcheckNPEasy}, where we will use the following notions.

\paragraph*{$2$-token game} Informally, the 2-token game on an automaton is a game played between Adam and Eve with three tokens at the starting state of the automaton, one used by Eve and two used by Adam. The game proceeds in infinitely many rounds, where in each round, Adam selects a letter, Eve selects a transition on that letter along which she moves her token, and finally Adam moves each of his two tokens along transitions on the same letter. Therefore, in a play of this game, Eve builds a run and Adam two runs, all on the same word. A play is won by Adam if at least one of the runs on his tokens is accepting while Eve's run on her token is rejecting, and Eve wins otherwise. Given an automaton $\Ac$, we write $G2(\Ac)$ to represent the 2-token game on $\Ac$, and we use $G2(\Bc;\Ac)$ to describe a modified version of the 2-token game where Eve moves her token in the automaton $\Bc$, while Adam moves his two tokens in the automaton~$\Ac$.


 \paragraph*{Muller condition} Muller objectives on graphs are specified by a finite set of colour $C$ and a set of accepting subsets $\Fc\subseteq 2^C$, and each edge of the graph is labelled by a colour from $C$. An infinite path of this graph is accepting if the set of colours that appears infinitely often in this path is a member of~$\Fc$. Such subsets $\Fc$ are sometimes represented using \emph{Zielonka trees}~\cite{DJW97}  defined below.

\paragraph*{Zielonka tree $\Zc_{C,\Fc}$}
For a Muller objective defined by colours $C$ and accepting set $\Fc$, we construct its Zielonka tree, denoted $\Zc_{C,\Fc}$, as a labelled rooted tree. We will call the vertices of this tree as \emph{nodes}. Each node $\nu$ of the Zielonka tree is labelled by a nonempty subset of $C$. 
The root of $\Zc_{C,\Fc}$ is labelled by $C$. For a node $\nu_X$ labelled by $X$, its children are nodes $\nu_Y$ labelled by distinct maximal nonempty subsets $Y$ of $X$ such that $Y\in\Fc$ if and only if $X \notin \Fc$. If there are no such subsets $Y$, then $\nu_X$ has no children in $\Zc_{C,\Fc}$.

\paragraph*{Zielonka DAG} A Zielonka DAG~\cite{HD05} is a succinct representation of Zielonka tree, where nodes with the same labels are merged.

\noindent We show that we can characterise memoryless resolvability using the $2$-token game.
    \begin{lemma}\label{prop:rigid2Token}
        An automaton $\Ac$ is MA if and only if there is a subautomaton $\Bc$ of $\Ac$ such that Eve randomising between all available outgoing edges in $G2(\Bc;\Ac)$ is almost-surely winning for Eve.
    \end{lemma}
    \begin{proof}[Proof of \cref{prop:rigid2Token}]
        \emph{$\Ac$ is not HD.} Then $\Ac$ is clearly not MA.  Adam wins $G2(\Ac)$ and hence $G2(\Bc;\Ac)$, due to the recent result of Lehtinen and Prakash~\cite[2-token theorem]{LP25}.
        
        \emph{$\Ac$ is HD and MA.} If $\Ac$ is MA, then there is a strategy of Eve in the HD game on $\Ac$ where she only randomises between some fixed set of transitions $\Ac$. Let $\Bc$ be the sub-automaton consisting exactly of this set of transitions.  Since such a resolution always produces an accepting run almost surely on any word in $\Lc(\Ac)$, Eve also must win the 2-token game $G2(\Bc;\Ac)$ using this strategy. 
        
        \emph{$\Ac$ is HD and not MA.} Then for every Eve's memoryless strategy $\sigma$ in the HD game, Adam has a strategy $\tau_{\sigma}$ to ensure, with positive probability, that his word is accepting while Eve's run is rejecting. For every subautomaton $\Bc$ of Eve, in the $2$-token game, consider the strategy for Adam $\tau_{\Bc}$ in the $2$-token game, where he picks letters according to a strategy that ensures Adam's word is accepting while Eve's run on her token is rejecting with positive probability. He picks transitions on his token according to a winning strategy for Eve in the HD game on $\Ac$, which ensures that the runs on his tokens are accepting if his word is accepting. Thus, there is no subautomaton $\Bc$ of $\Ac$ such that Eve wins $G2(\Bc;\Ac)$ almost-surely by choosing transitions randomly in $\Bc$, as desired.  
    \end{proof}
    To check if an automaton is MA, we guess a subautomaton $\Bc$ of $\Ac$, construct the $2$-token game $G2(\Bc;\Ac)$, and verify if Eve playing randomly is an almost-sure winning strategy. Subsequently, we construct a game where Eve's vertices in are substituted with stochastic vertices in $G2(\Bc;\Ac)$, resulting in a Markov Decision Process (MDP): a single-player complete-observation stochastic game where Adam is the only player. 
    
    If $\Ac$ is MA, Adam can satisfy his objective in the 2-token game with probability $0$ in this resulting MDP. The winning condition of the 2-token game for Adam, which is that at least one of Adam's runs on his tokens is accepting while Eve's run on her token is rejecting, can be represented by a Muller objective~\cite[Page 70]{Pra25}. The Zielonka DAG of this Muller objective has size that is at most polynomial in the size of the automaton $\Ac$~\cite[Page 72]{Pra25}.

    In \cref{lemma:ZlkDAGMDP}, we show that it can be verified in polynomial time if an MDP  satisfies a Muller objective with positive probability or almost-surely where the objective is input as a Zielonka DAG. Therefore,
    we can verify if Eve can win almost-surely or Adam can win with positive probability in the MDP in polynomial time. This proves \cref{lemma:MAcheckNPEasy}. 

\begin{restatable}{theorem}{ZlkDAGMDP}\label{lemma:ZlkDAGMDP}\label{thm:ZlkDAGMDP}
        Given an MDP $\Mc$ where the Muller objective is represented as a Zielonka DAG $\Zc_{C,\Fc}$, deciding whether Adam can satisfy the Muller objective with positive probability (resp. almost-surely) in $\Mc$ can be computed in time $\Oc(|\Mc||\Zc_{C,\Fc}|)$.        
\end{restatable}

Chatterjee showed that for MDPs, deciding if there is a strategy to almost-surely (or positively) satisfy a Muller objective represented by a set of subset of colours $\Fc$ that is union closed or upward closed and succinctly represented as a ``basis condition'' is in $\P$~\cite[Section 4]{Cha07}. Since every upward-closed or union-closed condition can be represented as a simple Zielonka DAG where all its nodes other than the root are leaf nodes, \cref{thm:ZlkDAGMDP} is therefore an improvement. 

We now prove the lower bound for the problem of checking memoryless adversarial resolvability using the result of $\NP$-hardness of checking history determinism~\cite{Pra24a}, and observe, in the appendix, that the reduction in \cite{Pra24a} can also be modified to show the $\NP$-hardness of checking if an automaton is MA. 
\begin{restatable}{lemma}{MAcheckNPhard}\label{lemma:MAcheckNPhard}
    Checking if a given parity automaton is memoryless-adversarially resolvable is $\NP$-hard.
\end{restatable}

%%%%%%%SRRRRRRRRRRRRR
\subsection{Memoryless-stochastically resolvable automata and stochastically resolvable automata}

We now discuss the questions of deciding whether an automaton is MR or SR in this subsection. 
%More specifically, we prove \cref{theorem:complexity}.

\begin{proposition}\label{prop:safetyPTIME}\label{prop:reachweakPSPACE}
    Deciding if a safety automaton is stochastically resolvable is in $\ptime$, and deciding if a reachability or weak automaton is stochastically resolvable is $\PSPACE$-complete. 
\end{proposition}
\begin{proof}
    From \cref{lemma:SR-implies-SD}, an automaton is SR if and only if it is SD-by-pruning. A safety automaton is SD if and only if it is determinisable by pruning if and only if it is HD \cite[Theorem 19]{BL21}, which is decidable in $\ptime$~\cite{BL23quantitative}. 

    Deciding if a weak automata is pre-SD is  $\PSPACE$-complete~\cite[Theorem 3]{AKL21}. The automaton constructed for their lower bound is also a reachability automaton, thus providing our lowerbound. 
    The upper bound follows from our results (\cref{lemma:sdweak-is-mr,lemma:SR-implies-SD}) that a weak (or reachability) automaton is SR if and only if it is pre-semantically deterministic.
\end{proof}
\begin{question}[Resolver-(co)Buchi]
    Given a finite memory resolver~$\Rc$ for a (co)B\"uchi automaton $\Ac$, is $\Rc$ an almost-sure resolver for $\Ac$?
\end{question}
\begin{restatable}{lemma}{lemmaUndecidablecoBuchi}\label{lemma:UndecidablecoBuchi}
        The problem \textsf{Resolver-coBuchi} is undecidable.
\end{restatable}
Our reduction to prove undecidability of the problem of \textsf{Resolver-coBuchi} is by reducing an instance of the problem of checking the emptiness of probabilistic B\"uchi automaton under the positive semantics of acceptance (the word is accepted if there is an accepting run of non-zero probability). This problem was proved undecidable by Baier, Bertrand, and Gr\"o{\ss}er~\cite[Theorem~2]{BBG08}.    

The dual problem of emptiness checking for probabilistic coB\"uchi automata is decidable. Therefore, we show the following lemma using a different reduction.
% \begin{question}[Resolver-Buchi] 
%     Given a finite-memory resolver $\Rc$ for a B\"uchi automaton $\Bc$, is $\Rc$ an almost-sure resolver for $\Bc$?
% \end{question}
\begin{restatable}{lemma}{lemmaUndecidableBuchi}\label{lemma:UndecidableBuchi}
    The problem \textsf{Resolver-Buchi} is undecidable.
\end{restatable}

We reduce from the zero-isolation problem for finite probabilistic automata. We define the problem formally in the appendix, but informally, the zero-isolation problems for finite probabilistic automata asks if there are words for which probability of acceptance is positive and reaches arbitrarily close to $0$. This problem was proved undecidable by Gimbert and Oaulhadj~\cite[Theorem 4]{GO10}. 

% Note that we do not know if it is undecidable to check if a given automaton is SR for these classes of automaton. It would be interesting to settle the decidability status of the problem, but conjecture at this point is that  this problem is also undecidable. 

