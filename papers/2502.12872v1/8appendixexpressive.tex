\subsection{Parity index hierarchy is strict for SR parity automata}
We will give a complete proof of \cref{theorem:pih} below, which shows that the parity-index hierarchy on stochastically resolvable automata is strict.
\pih*
We begin by observing that SR $[i,j]$-parity automata are as expressive as MR-parity $[i,j]$  automata.

\begin{lemma}
    Every SR $[i,j]$-parity  automaton can be converted to an MR $[i,j]$-parity  automaton.
\end{lemma}
\begin{proof}
    Let $\Ac$ be an SR automaton, and let $\Mc$ be an almost-sure resolver for $\Ac$. Let $\Pc$ be the resolver-product of $\Mc$ and $\Ac$, and $\Nc$ be the underlying nondeterministic automaton of $\Pc$. Then  the memoryless resolver for $\Nc$ that picks transitions according to $\Pc$ is an almost-sure resolver, since $\Lc(\Pc) = \Lc(\Ac) = \Lc(\Nc)$.
\end{proof}

Thus, it suffices to show that MR $[i,j]$-parity automata are as expressive as deterministic $[i,j]$-parity  automata. 
Observe also that any deterministic $[i,j]$-parity  automaton is trivially an MR $[i,j]$-parity automaton. For the other direction, let $L$ be an $\omega$-regular language that is not recognised by any deterministic $[i,j]$-parity automaton. We will show that the priorities of any automaton $\Mc$ recognising $L$ include all the priorities interval $[2m+i+1,2m+j+1]$ for some integer $m$, which proves \cref{theorem:pih}. We will do this in the following two steps. 

First, in \cref{lemma:SR-and-parity-languages}, we will show that for if an MR automaton $\Nc$ recognises the \emph{$[i+1,j+1]$-parity language $L_{[i+1,j+1]}$}, which is the set of words in $[i+1,j+1]^{\omega}$ such that the highest priority occurring infinitely often is even, then the priorities of $\Nc$ include all the priorities in the interval $[2m+i+1,2m+j+1]$ for some $m$. 

Then, in \cref{lemma:index-hierarchy-reduction-to-parity-language}, we show that if $\Mc$ is an MR automaton recognising $L$---which, recall, is an $\omega$-regular language that is not recognised by any deterministic $[i,j]$-parity automaton---then there is an automaton $\Nc$ recognising the $[i+1,j+1]$-parity language whose priorities are a subset of the priorities of $\Mc$. 

Combining these two results proves \cref{theorem:pih}. We show these two lemmas next. 

\begin{lemma}\label{lemma:SR-and-parity-languages}
   For every two natural numbers $\alpha,\beta$ with $\alpha<\beta$, any MR automaton $\Nc$ recognising the $[\alpha,\beta]$-parity language must contain all the priorities in $[2m+\alpha,2m+\beta]$ for some integer $m$.
\end{lemma}
\begin{proof}
    Let $\Nc$ be an MR automaton that recognises the $[\alpha,\beta]$-parity language $L_{[\alpha,\beta]}$. Let $\Mc$ be a memoryless almost-sure resolver for $\Nc$, and $\Pc = \Mc \circ \Nc$ be the probabilistic automaton that is the resolver product of $\Mc$ and $\Nc$. We assume, without loss of generality, that $\Mc$ chooses every transition of $\Nc$ on some word with positive probability. 
    
 We start by noting that $\Nc$ contains a priority of the same parity as $\alpha$. Indeed, consider the word $\alpha^{\omega}$. If $\alpha$ is odd (resp. even), then the word $\alpha^{\omega}$ is rejected (resp. accepted) by $\Nc$, and hence $\Nc$ must contain an odd (resp. even) priority, and thus, so must $\Pc$. Thus, let $\eta$ be the lowest priority in $\Pc$ that has the same parity as $\alpha$. We shall induct on the following statement, which implies the proof of \cref{lemma:SR-and-parity-languages}.

    \begin{claim}[IH(k)]\label{claim:Ihk}
        For each natural number $k$ with $\alpha+k\leq \beta$, there is a finite word $u_k \in ([\alpha,\alpha+k])^{*}$ and a positive real $0<\theta_k\leq 1$, such that from all states $q$ in $\Pc$, a run from $q$ on $u_k$ in $\Pc$, with probability at least $\theta_k$, contains a priority that is at least $\eta+k$ and that has the same parity as $\alpha+k$.
    \end{claim} 
    
    We prove the claim by induction. For the base case where $k=0$, we distinguish between the cases of when $\alpha$ is odd and when $\alpha$ is even.
    
    \paragraph*{Base case: $\alpha$ is even} Consider the word $\alpha^{\omega}$, which is in $L_{[\alpha,\beta]}$ and thus accepted by $\Pc$ almost-surely, i.e., a random run of $\Pc$ on $\alpha^{\omega}$ is accepting with probability $1$. We assume, without loss of generality, that every state $q$ in $\Pc$ can be reached from the initial state of $\Pc$. Then the automaton $(\Pc,q)$, must accept the word $\alpha^{\omega}$ almost-surely. This implies that for each state $q$ there is be an $n_q \in \mathbb{N}$, such that there is a path from $q$ on $\alpha^{n_q}$ that sees an even priority, which by minimality of $\eta$ (recall that $\eta$ is the smallest priority in $\Pc$ that has the same parity as $\alpha$), is at least $\eta$. Thus, there is a positive probability $\theta_q$ such that a random run of $\alpha^{n_q}$ on $q$ sees an even priority at least $\eta$. Taking $u_0$ to be $\alpha^{n_0}$, where  $n_0$ is the maximum of $n_q$'s for all states $q$, we get that a random run from each state $q$ on $u_0$ sees an even priority at least $\eta$ with probability at least $\theta$, where $\theta =\min \{\theta_q \mid q \in Q\}$.

    \paragraph*{Base case: $\alpha$ is odd} Consider the word $\alpha^{\omega} \notin L_{[\alpha,\beta]}$ that is rejecting in $\Pc$ in $\Nc$. Since $\Lc(\Pc) = \Lc(\Nc)$, all possible runs of $\Pc$ on $\alpha^{\omega}$ is rejecting. Therefore, all runs from each state $q$ on $\alpha^{\omega}$ in $\Pc$ must contain at least one odd priority. This implies that for each state $q$, there is a finite word $u_q=\alpha^{n_q}$ on which there is a run from $q$ in $\Pc$ that contains an odd priority. Thus, there is a positive probability $\theta_q$ of seeing an odd priority at least as large as $\eta$ from each state $q$ on $\alpha^{n_q}$. Taking $u_0$ to be $\alpha^{n_0}$, where  $n_0$ is the maximum of $n_q$'s for all states $q$, we get that a random run from each state $q$ on $u_0$ sees an odd priority at least $\eta$ with probability that is at least the minimum of $\theta_q$'s for all states $q$. 

    This completes our base case for when $k=0$. 
    
    Let us now prove the induction step. Assume IH($k$) holds for some natural number $k<\beta-\alpha$. That is, there is a finite word $u_k \in [\alpha,\alpha+k]^{*}$ and a positive real $\theta_k$, such that from any state $q$, the probability that a run from $q$ on $u_k$ sees a priority that is at least $\eta+k$ and of the same parity as $\eta+k$ is at least $\theta_k$. We now show that this implies IH($(k+1)$).

    We once again distinguish between the cases $(\alpha+k+1)$ is even or odd.
    
    \paragraph*{Induction step: if $(\alpha+k+1)$ is even} 
    
    Fix a state $q$ and consider the finite word $v=u_k \cdot (\alpha+k+1)$. Since $u_k$ is a prefix of $v$, from every state $p$, the probability that an odd priority at least $\eta+k$ occurs in a run from $p$ on $v$ in $\Pc$ is at least $\theta_k$. The word $v^{\omega}$ is accepted almost surely from $q$ in $\Pc$, however. We shall utilise the above two remarks to show that an even priority at least $(\eta+k+1)$ occurs with positive probability on a run from $q$ on the word $v^n$ from some $n$ in $\Pc$.
    
    More concretely, let $\zeta^q_n$ denote the probability that a run of $\Pc$ from $q$ on the word $v^n$ contains an even priority that is at least $\eta+(k+1)$. The sequence $\zeta^q_0,\zeta^q_1,\zeta^q_2,\cdots$ then is nondecreasing, and since it is bounded above by 1, converges to a value $\zeta^q$. Below, we show that $\zeta^q=1$, which implies that $\zeta^q_n>0$ for some finite~$n$. This follows from the computations below.

    If $\rho$ is a run of $\Pc$ from $q$ on the word  $v^{\omega}$, then 
    \begin{equation*}
    \begin{split}
        \prob[\rho \text{ is accepting}]
        &= \prob[\max(\inf(\rho)) \text{ is even}] \\
        &= \prob[\max(\inf(\rho)) \text{ is even} \\ 
        &\quad\quad\qquad\text{ and at most $(\eta+k-1)$}] \\
        &+ \prob[\max(\inf(\rho)) \text{ is even} \\
        &\quad\quad\qquad\text{ and at least $(\eta+k+1)$}] \\
    \end{split} 
    \end{equation*}
    We first show that that $$\prob[\max(\inf(\rho)) \text{ is even and at most $(\eta+k-1)$}]=0.$$ To do so, we use the fact that from any state in $\Pc$, the probability that an odd priority at least $(\eta+k)$ occurs on a run from $\Pc$ on $v$ is at least $\theta_k$. Thus the probability that a priority at most $(\eta+k-1)$ occurs on a run in $\Pc$ from any state on $v$ is at most $(1-\theta_k)$. 

 
\begin{equation*}
    \begin{split}
        &\prob[\max(\inf(\rho)) \text{ is even and at most $(\eta+k-1)$}] \\
        &\leq  \prob[\max(\inf(\rho)) \text{ is at most $(\eta+k-1)$}]  \\ 
        &=\prob[\bigcup_{N \in \mathbb{N}} \text{(No priority that is at least $\eta+k$ }\\
        &\quad\quad\quad\quad\text{occurs after reading the $N^{th}$ $v$ in $\rho$)} ] \\
        &=\lim_{N\to\infty}\prob[\text{(No priority that is at least  $\eta+k$ }\\
        &\quad\quad\quad\quad\quad\quad\text{occurs after reading the $N^{th}$ $v$ in $\rho$)}] \\
        &\leq \lim_{N\to\infty} \prod_{n\geq N} (1-\theta_k) = 0
    \end{split}
\end{equation*}
Thus, we have, 
\begin{equation*}
    \begin{split}
        \prob[\rho \text{ is accepting}]
        &= \prob[\max(\inf(\rho)) \text{ is even }\\
        &\quad\quad\quad\quad\quad\text{and at most $(\eta+k-1)$}] \\
        &+ \prob[\max(\inf(\rho)) \text{ is even }
        \\&\quad\quad\quad\quad\quad\text{and at least $(\eta+k+1)$}] \\
        &= \prob[\max(\inf(\rho)) \text{ is even }\\
        &\quad\quad\quad\quad\quad\text{and at least $(\eta+k+1)$}] \\
        &\leq \prob[\rho \text{ contains a transition with an} 
        \\ &\qquad\text{even priority at least $(\eta+k+1)$}] \\
        &=\lim_{n\to \infty} \zeta^q_n = \zeta^q\\
    \end{split} 
\end{equation*}
    
Since $v^{\omega}$ is almost-surely accepted, $\rho$ is almost-surely an accepting run, and hence we get $1 \leq \zeta^q$, which gives us $\zeta^q$ is $1$ and hence $\zeta^q_n>0$ for some $n$. Let $n$ be large enough so that $\zeta^p_n>0$ for all states $p$. Then, any run of $\Pc$ on the word $u_{k+1}=v^n$ from any state contains an even priority at least $(\eta+k+1)$ with probability $\theta_{k+1}=\min (\zeta^p_n)_{p \in Q}$, as desired.

\paragraph*{Induction step: if $(\alpha+k+1)$ is odd} 

Consider the finite word $v=u_k \cdot (\alpha+k+1)$. Since $u_k$ is a prefix of $v$, from every state $q$, the probability that an even priority at least $\eta+k$ is seen on a run from $q$ on $v$ in $\Pc$ is at least $\theta_k$. In particular, from each state $q$, there is a run from $q$ on $v$ in $\Pc$, such that an even priority at least $\eta+k$ occurs.  

Fix a state $q$, and consider the following run from $q$ on $v^{\omega}$ in $\Pc$ in which the highest even priority occurring infinitely often is at least $(\eta+k)$. There is a run $\rho_1$ in $\Nc$ from $q$ to some state $q_1$ on $v$ such that the highest even priority occurring in $\rho_1$ is at least $(\eta+k)$. Similarly, there is a run $\rho_2$ from $q_1$ to $q_2$ such that the highest even priority occurring in $\rho_1$ is at least $(\eta+k)$. Extending this to get $\rho_3,\rho_4,\dots$ similarly, and then concatenating $\rho_1 \cdot \rho_2 \cdot \rho_3 \dots$ to get $\rho$, we see that $\rho$ is a run from $q$ on $v^{\omega}$ in which the highest even priority occurring infinitely often is at least $(\eta+k)$. Since the word $v^{\omega}$ is rejected by $\Nc$, this means that there is an odd priority at least $(\eta+k+1)$ that occurs in $\rho$ infinitely often. Let $n_q$ be the minimum natural number so that there is a run from $q$ on $v^{n_q}$ that contains a transition of an odd priority at least $(\eta+k+1)$. Thus, the probability that a run of $\Pc$ from $q$ on $v^{n_q}$ that contains a transition of an odd priority at least $(\eta+k+1)$ is positive, say $\theta_q$. Taking $u_k$ to be $v^{n}$, where  $n$ is the maximum of $n_q$'s for all states $q$, we get that a random run from each state $q$ on $u_k$ sees an odd priority at least $(\eta+k+1)$ with probability that is at least the minimum of $\theta_q$'s for all states $q$, as desired.  

This completes our proof of \cref{claim:Ihk}, and also of \cref{lemma:SR-and-parity-languages}.
\end{proof}


A consequence of the lemma above is that an MR $[i,j]$-parity automata cannot recognise the $[i+1,j+1]$-parity language.  The next lemma allows us to reduce  \cref{theorem:pih} to parity languages, from where the conclusion follows.

\begin{lemma}\label{lemma:index-hierarchy-reduction-to-parity-language}
    Let $L$ be an $\omega$-regular language, such that $L$ cannot be recognised by any deterministic $[i,j]$-parity automata. If automaton $\Ac$ is an MR automaton recognising $L$, then there is an MR automaton $\Nc$ recognising the $[i+1,j+1]$-parity language  whose priorities are a subset of the priorities of $\Ac$. 
\end{lemma}
\begin{proof}
     Let $\Dc$ be a deterministic parity automaton that recognises $L$, and $\Pc$ be a probabilistic parity automaton that is obtained by taking the resolver-product of $\Ac$ with a memoryless resolver that ensures all accepted words in $\Dc$ are accepted almost-surely in $\Pc$.
     
     Since $L$ cannot be recognised by a deterministic $[i,j]$ automaton, we know due to the flower lemma of Niwi\'nski and Walukiewicz that $\Dc$ contains an $[i+1,j+1]$-\emph{flower}~\cite[Lemma 14]{NW98}. That is, there exists a state $p$ in $\Dc$, an integer $\ell$, and finite words $u_{k}$ for each $k$ in the interval $[i+1,j+1]$, such that the unique run from $p$ on $u_k$ ends at $p$ and the highest priority seen during this run is $2\ell+k$. Let $v$ be a finite word such that there is a run from the initial state of $\Dc$ to $p$ on the word $v$.

     We build a nondeterministic automaton $\Nc$ using the automaton $\Ac$, and the finite words $v,u_{i+1},u_{i+2},\cdots,$ $u_{j+1}$. The states of the automaton $\Nc$ are same as the states of $\Pc$. The initial state of $\Nc$ is a state that can be reached in $\Pc$ from its initial state on the word $v$ with positive probability. The transitions of $\Nc$ are over the alphabet $[i+1,j+1]$, and we have a transition from $q$ to $q'$ on the letter $k$ with priority $\pi$ if there is a run from $q$ to $q'$ in $\Pc$ on the word $u_k$, where the largest priority that run contains is $\pi$. 
     
     We construct an almost sure resolver $\Pc_\Nc$ for $\Nc$, which, by construction is a memoryless resolver, utilising the probabilistic automaton $\Pc$. Concretely, the almost-sure resolver $\Pc_\Nc$, for a state $q$ and letter $k\in [i+1,j+1]$, chooses the transition to $q'$ of priority $\pi$ with probability $\zeta$, if the probability of reaching $q'$ from $q$ on the word $u_k$ such that the highest priority occurring in the run is $\pi$ is $\zeta$. Abusing the notation slightly, we let $\Pc_\Nc$ denote the probabilistic automaton that is obtained by taking the resolver product of $\Pc_\Nc$ with~$\Nc$.

     We will show that $\Nc$ recognises the $[i+1,j+1]$-parity language, and $\Pc_\Nc$ recognises all words in the $[i+1,j+1]$-parity language with probability $1$. This follows from the following chain of equivalences. 
     
     The word $w'=k_0 k_1 k_2 \cdots$ with $k_i \in [i+1,j+1]$ is in the $[i+1,j+1]$-parity language \emph{if~and~only~if}  $w=u_{k_0}\cdot u_{k_1} \cdot u_{k_2} \cdots $ is accepted by $(\Dc,p)$ \emph{if and only if} the word $v.w$ is accepted by $\Ac$ and accepted almost surely by $\Pc$ \emph{if and only if} the word $w'=k_0 k_1 k_2 \cdots$ is accepted by $\Nc$ and almost surely by $\Pc_\Nc$. This concludes our proof. 
\end{proof}

We note that \cref{lemma:index-hierarchy-reduction-to-parity-language,lemma:SR-and-parity-languages} together prove \cref{theorem:pih}. 