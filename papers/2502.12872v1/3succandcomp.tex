In this section, we compare the notions of stochastic resolvability, memoryless adversarial resolvability, memoryless stochastic resolvability with each other and the existing notions of nondeterminism in the literature.

We start by making a connection to the notion of semantic determinism. These were introduced by Kuperberg and Skrzypczak as residual automata~\cite{KS15}, but we follow the more recent works of Abu Radi, Kupferman, and Leshkowitz in calling them semantically deterministic (SD) automata instead~\cite{AKL21,AK23}.
%\ap{it might make sense to move SD to prelims. we might also want to talk about DBP in \cref{theorem:comp}.}
We call a transition $\delta$ from $p$ to $q$ on a letter $a$ in a parity automaton $\Ac$ as \emph{language-preserving} if $\Lc(\Ac,q) = a^{-1} \Lc(\Ac,p)$. We say that a parity automaton is \emph{semantically deterministic}, SD for short, if all transitions in that automaton are language-preserving. The following observation concerning semantically deterministic automata can be shown by a simple inductive argument on the length of words.

\begin{observation}\label{lemma:SDautomata}
For every semantically deterministic automaton $\Ac$, all states in $\Ac$ that can be reached from a state $q$ upon reading a finite word $u$ recognise the language $u^{-1}\Lc(\Ac,q)$.%\ap{language of state}
\end{observation}

We observe that all SR automata are SD, up to removal of some transitions.
\begin{lemma}\label{lemma:SR-implies-SD}
Every stochastically resolvable parity automaton $\Ac$ has a language-equivalent subautomaton $\Bc$ that is semantically deterministic.
\end{lemma}
\begin{proof}
Let $\Mc$ be an almost-sure resolver for $\Ac$. Consider the set of transitions of $\Ac$ that are reachable from the initial state and taken in $\Mc \circ \Ac$ with nonzero probability: call these transitions \emph{feasible}. Let $\Bc$ be the subautomaton of $\Ac$ consisting of only feasible transitions. Since $\Mc\circ \Ac$ accepts the same language as $\Ac$, so does $ \Mc\circ \Bc$, and therefore, $\Lc(\Bc) = \Lc(\Ac)$. We will argue that $\Bc$ is semantically deterministic. Indeed, let $\delta=q\xrightarrow{a:c}q'$ be any transition in $\Bc$, and let $m,m'$ be states such that in $\Tc$, we have the transition $(q,m) \xrightarrow{a:c} (q',m')$ in $\Mc \circ \Ac$ that has non-zero probability. Let $u$ be a word such that $\Mc \circ \Ac$ has a run from from $(q_0,m_0)$ to $(q,m)$. For every word $aw \in \Lc(\Bc,q)$, we must have that the word $w$ is accepting in $(\Bc,q')$, since otherwise we have that the word $uaw$ is rejected with positive probability in $\Mc\circ \Ac$. It follows that $\Bc$ is semantically deterministic, as desired. 
\end{proof}

A \emph{pre-semantically deterministic} (pre-SD) automaton is an automaton that has a language-equivalent SD subautomaton. The following result gives a comprehensive comparison between the notions of nondeterminism we have discussed so far. The results of \cref{theorem:comp} are summarised by the Venn diagram in \cref{fig:venndiagrammmmmm} in \cref{sec:intro}.  

\begin{theorem}\label{theorem:comp}
\begin{enumerate}
    \item For safety automata, the notions of pre-semantic determinism, stochastic resolvability, memoryless-adversarial resolvability, memoryless-stochastic resolvability, and history-determinism are equivalent.
    \item For reachability and weak automata, the following statements hold.
    \begin{enumerate}
        \item Pre-semantic-determinism, stochastic resolvability, and memoryless stochastic resolvability are equivalent and are strictly larger classes than history-deterministic automata.
        \item History-determinism and memoryless adversarial resolvability are equivalent notions.
    \end{enumerate}
    % \item For reachability and weak automata, the notions of semantic-determinism, stochastic resolvability, and memoryless stochastic resolvability are equivalent and encompass memoryless adversarial resolvability, with memoryless resolvability and history determinism being equivalent notions.\theju{eehhh??? I am also happy with your version because it distils the components better} 
    \item For B\"uchi, coB\"uchi, and parity automata, the following statements hold.
    \begin{enumerate}
        \item Pre-semantically deterministic automata are a strictly larger class than stochastically resolvable automata.
        \item Stochastically resolvable automata are a strictly larger class than history-deterministic automata.
        \item Stochastically resolvable automata are a strictly larger class than memoryless stochastically resolvable automata.
        \item There are history-deterministic automata that are not memoryless stochastically resolvable, and there are memoryless stochastically resolvable automata that are not history-deterministic.
        \item Both history-deterministic and memoryless-stochastically resolvable automata are strictly larger classes than memoryless-adversarially resolvable automata.
    \end{enumerate}
\end{enumerate}
\end{theorem}

%\input{3zfigvenn}

Even though the five notions of nondeterminism discussed for coB\"uchi automata are all different, we show that every stochastically resolvable coB\"uchi automaton can be efficiently converted to a language-equivalent memoryless-adversarially resolvable coB\"uchi automaton.

\begin{restatable}{theorem}{theoremcobuchisrtoma}\label{theorem:coBuchiHDisSR}
    There is a polynomial-time algorithm that converts stochastically resolvable coB\"uchi automata with $n$ states into language-equivalent memoryless-adversarially resolvable coB\"uchi automata with at most $n$ states. 
\end{restatable}
We note that the above result does not hold for SD coB\"uchi automata, since SD coB\"uchi are exponentially more succinct than HD coB\"uchi automata~\cite[Theorem 14]{AK23}. %SD B\"uchi automata are also exponentially more succinct than HD B\"uchi automata, as was shown by Abu Radi and Kupferman~\cite[Theorem 5]{AK23}. We show that the family of succinct SD B\"uchi automata that they construct are MR, thus giving us the following result. 
However, we show the succinctness of SR B\"uchi automata against HD B\"uchi automata and MA coB\"uchi automata against deterministic coB\"uchi~automata.
\begin{theorem}\label{theorem:succ}
\begin{enumerate}
    \item  There is a class of languages $L_n$ such that, there are memoryless-stochastically resolvable B\"uchi automata recognising $L_n$ with $\Oc(n)$ states, and any HD B\"uchi automaton recognising $L_n$ requires at least $2^n$ states. 
    \item  There is a class of languages $L_n'$, such that there are memoryless-adversarially resolvable coB\"uchi automata recognising $L_n'$ with $\Oc(n)$ states and any deterministic coB\"uchi automaton recognising $L_n'$ requires at least $\Omega(2^n/2n+1)$ states. 
\end{enumerate}
\end{theorem}


In the next subsections, we will prove \cref{theorem:comp,theorem:coBuchiHDisSR,theorem:succ}. We organise their proofs based on the acceptance~conditions.
\subsection{Safety automata}\label{subsec:sac-safe}
We showed in \cref{lemma:SR-implies-SD} that stochastically resolvable automata are semantically deterministic. The next result shows that every safety automaton $\Sc$ is semantically deterministic if and only if $\Sc$ is determinisable-by-pruning, that is, $\Sc$ contains a language-equivalent deterministic subautomaton, whose proof is presented in the appendix. 
\begin{restatable}[Folklore]{lemma}{SDsafetyisDBP}\label{lemma:sd-safety-is-dbp}
    Every semantically deterministic safety automaton is determinisable-by-pruning.
\end{restatable}

Observe that any determinisable-by-pruning automaton is trivially memoryless-adversarially resolvable, where Eve's strategy is to take transitions in a fixed, language-equivalent, deterministic subautomaton. Thus, \cref{lemma:sd-safety-is-dbp} implies the following result.

\begin{lemma}\label{lemma:comp-safety}
    For safety automata, the notions of pre-semantic determinism, stochastic resolvability, memoryless adversarial resolvability, memoryless stochastic resolvability, and history determinism are equivalent.
\end{lemma}

\subsection{Reachability and weak automata}\label{subsec:sac-rw}
We now compare our notions of nondeterminism on automata with reachability and weak acceptance conditions. To start with, recall that in \cref{example:reachSRbutnotHD}, we showed a MR reachability automaton that is not HD. This implies the following result.
\begin{lemma}\label{lemma:reachability-MR-not-HD}
    There is a memoryless stochastically resolvable reachability automaton that is not HD.
\end{lemma}

We next show that SD weak automata are MR. Since SR automata are pre-SD (\cref{lemma:SR-implies-SD}), we obtain that the notions of pre-semantic determinism and (memoryless) stochastic resolvability are equivalent on weak automata.  

\begin{restatable}{lemma}{lemmasdweakismr}\label{lemma:sdweak-is-mr}
    Every semantically deterministic weak automaton is memoryless-stochastically resolvable. 
\end{restatable}
\begin{proof}[Proof sketch]
Let $\Ac$ be a semantically deterministic weak automaton. We show that the resolver that selects transitions uniformly at random constructs, on any word in $\Lc(\Ac)$, a run that is almost-surely accepting. Let $w$ be a word in $\Lc(\Ac)$, and $\rho$ a run of $\Ac$ on $w$ where transitions are chosen uniformly at random. Then there is a finite prefix $u$ of $w$ and a state $p$ of $\Ac$, such that there is a run from $q_0$ to $p$ on $u$ and a run from $p$ on $u^{-1}w$ that only visits accepting states. Let $K=n2^n$, where $n$ is the number of states of $\Ac$. We show that there is a positive probability $\epsilon>0$, such that on any infix of $u^{-1}w$ that has length at least $K$, the segment of the run $\rho$ on that infix contains an accepting transition with probability at least $\epsilon$. Using this, we argue that $\rho$ contains infinitely many accepting transitions with probability $1$, and hence is almost-surely~accepting.
\end{proof}

Thus, stochastically resolvable weak automata are also memoryless-stochastically resolvable, and strictly encompass HD automata. 
History-deterministic weak automata are determinisable-by-pruning~\cite{BKS17}, and since MA automata are HD, and deterministic automata are trivially MA, we obtain the following result.
\begin{lemma}\label{lemma:reachability-hd=ma}
    The notions of determinisable-by-pruning, history-determinism, and memoryless adversarial resolvability coincide on reachability automata.
\end{lemma}

\subsection{CoB\"uchi automata}\label{subsec:sac-cobuchi}
We continue our comparison of the notions of nondeterminism, and focus on coB\"uchi automata. We will show that no two notions among SD, SR, HD, MR, and MA are equivalent for coB\"uchi automata. We will then give a polynomial-time algorithm that converts stochastically resolvable coB\"uchi automata with $n$ states into language-equivalent memoryless adversarially resolvable coB\"uchi automata with at most $n$ states. This shows that SR and MR coB\"uchi automata are no more succinct than MA (and also HD) coB\"uchi automata. 
However, this is not the case for B\"uchi automata (\cref{lemma:succinctBuchi}).

We start by showing an SD coB\"uchi automaton that is not stochastically resolvable, as shown in \cref{fig:cobuchisdbutnotsr}. 
\begin{figure}[h]
    \centering
    \begin{minipage}[b]{0.45\linewidth}
            % \begin{subfigure}[b]{\linewidth}
        \centering
            \begin{tikzpicture}
        \tikzset{every state/.style = {inner sep=-3pt,minimum size =20}}

    \node[state] (q0) at (0,0) {$q_0$};
    \node[state] (qa) at (1.5,0) {$q_a$};
    \node[state] (qb) at (-1.5,0) {$q_b$};
    \path[-stealth]
    (0,-0.75) edge (q0)

  
    (q0) edge node [above,yshift=-1mm] {$a,b$} (qa)
    (q0) edge node [above,yshift=-1mm] {$a,b$} (qb)
    (qa) edge [red, dashed, bend right = 45] node [above] {$b$} (q0) 
    (qa) edge [bend left = 45] node [below] {$a$} (q0)
    (qb) edge [red, dashed, bend left = 45] node [above] {$a$} (q0)
    (qb) edge [bend right = 45] node [below] {$b$} (q0)
    ;
    \end{tikzpicture}
    \caption{A coB\"uchi automaton that is SD but not SR. Rejecting transitions are represented by dashed arrows.}\label{fig:cobuchisdbutnotsr}
    %\end{subfigure}%
    \end{minipage}
    \hfill
    \begin{minipage}[b]{0.5\linewidth}
        %\begin{subfigure}[b]{\linewidth}
                \begin{tikzpicture}
        \tikzset{every state/.style = {inner sep=-3pt,minimum size =20}}

    \node[state] (q0) at (0,0) {$q_0$};
    \node[state] (q1) at (0,1.5) {$q_1$};
    \node[state] (d1) at (1.5,0) {$d_1$};
    \node[state] (d2) at (1.5,1.5) {$d_2$};
    \node[state] (d3) at (3,1.5) {$d_3$};
    \path[-stealth]
    (-0.5,-0.5) edge (q0)
    (q0) edge [bend left = 15] node [left] {$x$} (q1)
    (q1) edge [red, dashed, bend left = 15] node [right] {$a$} (q0)
    (q0) edge [in=-30,out=-60,loop] node [below, xshift=1mm] {$x$} (q0)
    (q0) edge [red, dashed, loop below] node [left] {$b$} (q0)
    (q1) edge [loop above] node [above] {$x$} (q1)
    (q1) edge node [above] {$b$} (d2)
    (d2) edge [red, dashed, loop above] node [left] {$b$} (d2)
    (d2) edge [red, dashed, bend left = 15] node [right] {$a$} (d1)
    (d1) edge [red, dashed, bend left = 15] node [left] {$b$} (d2)
    (q0) edge node [above] {$a$} (d1)
    (d1) edge [loop right] node [right] {$x,a$} (d1)
    (d2) edge [bend left = 15] node [above] {$x$} (d3)
    (d3) edge [bend left = 15] node [below] {$b$} (d2)
    (d3) edge [loop above] node [right] {$x$} (d3)
    (d3) edge [red, dashed] node [below] {$a$} (d1)
;
    \end{tikzpicture}
          \caption{A HD coB\"uchi automaton that is not MR. Rejecting transitions are represented by dashed arrows.}\label{fig:HDcoBuchinotMR}
    %\end{subfigure}
    \end{minipage}%\caption{}
\end{figure}

\begin{lemma}
    There is a semantically deterministic coB\"uchi automaton that is not stochastically resolvable.
\end{lemma}
\begin{proof}
    Consider the coB\"uchi automaton $\Cc$ shown in \cref{fig:cobuchisdbutnotsr} that accepts all infinite words over $\{a,b\}$. The automaton $\Cc$ has nondeterminism on the letters $a$ and $b$ in the initial state.    
    Consider the following random strategy of Adam in the SR game on~$\Cc$, where in each round, he picks the letter $a$ or $b$ with equal probability. Then whenever Eve's token is at $q_{\alpha}$ for $\alpha\in\{a,b\}$, Adam's letter is $\beta \in \{a,b\}$ with $\beta\neq \alpha$ with probability~$\frac{1}{2}$. Thus, in round $2i$ of the SR game for each $i$ in $\mathbb{N}$, the run on Eve's token takes a rejecting transition with probability~$\frac{1}{2}$. Therefore, due to the second Borel-Cantelli lemma (\cref{lemma:secondborellcantelli}), the run on Eve's token contains infinitely many rejecting transitions with probability~1, and Eve loses almost-surely. Thus, Eve has no strategy to win the SR game on $\Cc$ almost-surely, proving $\Cc$ is not SR (\cref{lemma:random-is-pure}). 
\end{proof}

In \cref{subsec:sac-rw}, we showed an MR reachability automaton that is not HD (\cref{lemma:reachability-MR-not-HD}). Since reachability automata are a subclass of coB\"uchi automata, there is a coB\"uchi automaton that is not HD. We next show a HD coB\"uchi automaton that is not MR, and hence, also not MA. 
\begin{restatable}{lemma}{lemmaHDcoBuchinotMR}\label{lemma:HDcoBuchinotMR}
      There is a history-deterministic coB\"uchi automaton that is not memoryless stochastically resolvable.
\end{restatable}
\begin{proof}[Proof sketch]
    Consider the coB\"uchi automaton $\Cc$ shown in \cref{fig:HDcoBuchinotMR}. The automaton $\Cc$ has nondeterminism on the letter~$x$ in the initial state $q_0$. Informally, Eve, from the state $q_0$ in the HD game or the SR game, needs to `guess' whether the next sequence of letters till an $a$ or $b$ is seen form a word in $x^* a$ or in $x^+ b$. The automaton $\Cc$
    recognises the language $$L=(x+a+b)^{*} ((x)^{\omega} + (x^* a)^{\omega} + (x^+ b)^{\omega}).$$ 
    \paragraph*{$\Cc$ is HD}  If Eve's token in the HD game reaches the state $d_1,d_2$, or $d_3$, then she wins the HD game from here onwards since her transitions are deterministic. At the start of the HD game on $q_0$, or whenever she is at $q_0$ after reading $a$ or $b$ in the previous round she decides between staying at $q_0$ till an $a$ or $b$ is seen, or moving to $q_1$ on the first $x$ as follows.
    \begin{itemize}
        \item If the word read so far has a suffix in $x^{*}a$, then she stays in $q_0$ till the next $a$ or~$b$.
        \item If the word read so far has a suffix in $x^{+}b$, then she takes the transition to $q_1$ on~$x$.
        \item Otherwise, she stays in $q_0$ till the next $a$ or $b$.
    \end{itemize}
    Due to the language of $\Cc$ being the set of words which have a suffix in $x^{\omega},(x^{*}a)^{\omega},$ or $(x^+b)^{\omega}$, the above strategy guarantees that Eve's token moves on any word in $L$ in HD game to one of $d_1$ or $d_2$, from where she wins the HD game.
    \paragraph*{$\Cc$ is not MR} Note that the automaton $\Cc$ does not accept the same language if any of its transitions are deleted. Consider a memoryless resolver for $\Cc$ that takes the self-loop on $x$ on $q_0$ with probability $(1-p)$ and and the transition to $q_1$ on $x$ with probability $p$, for some $p$ satisfying $0<p<1$. We show that the runs that the resolver constructs on the word $x^2 a x^3 a x^4 a \dots$, do not visit the states $d_1,d_2,$ or $d_3$ and are rejecting with positive probability. This shows that $\Cc$ has no almost-sure memoryless resolver for $\Cc$, and is not MR.
\end{proof}

We have shown so far that each of the five classes in \cref{fig:venndiagrammmmmm} are different for coB\"uchi automata. We next show that every SR coB\"uchi automaton can be converted into a language-equivalent MA coB\"uchi automaton without any additional states.

\theoremcobuchisrtoma*
We start by fixing a SR coB\"uchi automaton $\Ac$ throughout the proof of \cref{theorem:coBuchiHDisSR}. 
We first relabel the priorities on $\Ac$ to obtain $\Cc$ as follows. Consider the graph consisting of all states of $\Ac$ and 0 priority transitions of $\Ac$. For any 0 priority transition of $\Ac$ that is not in any strongly connected component (SCC) in this graph, we change that transition to have priority~1 in $\Cc$. This relabelling of priority does not change the acceptance of any run (\cref{prop:priority-reduction}, \cref{app:succandcompCB}), and thus,~$\Cc$ is SR and language-equivalent to~$\Ac$.  

We start by introducing notions to describe a proof sketch of \cref{theorem:coBuchiHDisSR}. 
\paragraph*{Safe-approximation}
For the automaton $\Cc$, define the safe-approximation of $\Cc$, denoted $\Cc_{\safety}$ as the safety automaton constructed as follows. The automaton $\Cc_{\safety}$ has the same states as $\Cc$ and an additional rejecting sink state. The transitions of priority $0$ in $\Cc$ are preserved as the safe transitions of $\Cc_{\safety}$, and transitions of priority~$1$ in $\Cc$ are redirected to the rejecting sink state and have priority~1. 

\paragraph*{Weak-coreachability}
We call two states  $p$ and $q$ in $\Cc$ as coreachable, denoted by $p,q\in \CR(\Cc)$, if there is a finite word $u$ on which there are runs from the initial state of $\Cc$ to $p$ and $q$. We denote the transitive closure of this relation as \emph{weak-coreachability}, which we denote by $\WCR(\Cc)$. Note that weak-coreachability is an equivalence relation.

\paragraph*{SR self-coverage} 
For two  parity automata $\Bc$ and $\Bc'$, we say that $\Bc$ SR-covers $\Bc'$, denoted by $\Bc \succ_{SR} \Bc'$, if Eve has an almost-sure winning strategy in the modified SR game as follows. Eve, similar to the SR game on $\Bc$, constructs a run in $\Bc$, but 
Eve wins a play of the game if, in that play, Eve's constructed run in $\Bc$ is accepting whenever Adam's word is in $\Lc(\Bc')$.
We say that a coB\"uchi automaton $\Bc$ has \emph{SR self-coverage} if for every state $q$ there is another state $p$ that is coreachable to $q$ in $\Cc$, such that $(\Bc_{\safety},p)$ SR-covers~$(\Bc_{\safety},q)$.  

The crux of \cref{theorem:coBuchiHDisSR} is in proving the following result.
\begin{restatable}{lemma}{lemmaSRhassafeSRcoverage}\label{lemma:coBuchiSRhassafeSRcoverage}
The coB\"uchi automaton $\Cc$  has SR self-coverage.
\end{restatable}
\begin{proof}[Proof sketch] Fix an almost-sure resolver $\Mc$ for Eve in $\Cc$. Let~$\Pc$ be the probabilistic automaton that is the resolver-product of $\Mc$ and $\Cc$. We define $\Pc_{\safety}$ as a safety probabilistic automaton that is the safe-approximation of $\Pc$, similar to how we defined $\Cc_{\safety}$. Suppose, towards a contradiction, that there is a state $q$ in $\Cc$, such that for every state $p$ coreachable to $q$ in $\Cc$, $(\Cc_{\safety},p)$ does not SR-cover $(\Cc_{\safety},q)$. In particular, for every state $(p,m)$ in $\Pc$, where $p$ is coreachable to $q$ in $\Cc$, we have that $\Lc(\Pc_{\safety},(p,m)) \subsetneq \Lc(\Cc_{\safety},q)$. We use this to show that there is a finite word $\alpha_{(p,m)}$, on which there is a run consisting of only priority 0 transitions from $q$ to $q$ in $\Cc$, while a run $\rho$ of $(\Pc,(p,m))$ on $\alpha_{(p,m)}$ contains a priority $1$ transition with probability at least $\epsilon$ for some $\epsilon>0$.    

Adam then has a strategy in the SR game on $\Cc$ against Eve's strategy $\Mc$ as follows. Adam starts by giving a finite word~$u_q$, such that there is a run of $\Cc$ from its  initial state to~$q$. Then Adam, from this point and at each \emph{reset}, selects a state~$(p,m)$ of $\Pc$ uniformly at random, such that $p$ is coreachable to $q$ in $\Cc$ and $m$ is a memory-state in $\Mc$. He then plays the letters of the word $\alpha_{(p,m)}$ in sequence, after which he \emph{resets} to select another such state and play similarly. This results in Eve constructing a run in the SR game on $\Cc$ that contains infinitely many priority $1$ transitions almost-surely, while Adam's word is in $\Lc(\Cc)$. It follows that $\Mc$ is not an almost-sure resolver for Eve, which is a contradiction.
\end{proof}

 SR-covers is a transitive relation, i.e., if $\Ac_1,\Ac_2,\Ac_3$ are nondeterministic parity automata, such that $\Ac_1 \succ_{SR} \Ac_2$ and $\Ac_2 \succ_{SR} \Ac_3$, then $\Ac_1 \succ_{SR} \Ac_3$. The following result then follows from the definition of SR self-coverage and the fact that $\Cc$ has finitely many states.% any finite directed graph in which every vertex has outdegree 1 contains a cycle.

\begin{restatable}{lemma}{lemmacobuchisometingdbp}\label{lemma:cobuchi-srselfcoverage-implies-somtingsdbp}
    For every state $q$ in $\Cc$, there is another state $p$ weakly coreachable to $q$ in $\Cc$, such that $(\Cc_{\safety},p)$ SR-covers $(\Cc_{\safety},q)$ and $(\Cc_{\safety},p)$ SR-covers $(\Cc_{\safety},p)$.  
\end{restatable}

Note that if $(\Cc_{\safety},p)$ SR-covers $(\Cc_{\safety},p)$ then $(\Cc_{\safety},p)$ is SR. Since SR automata are semantically deterministic (\cref{lemma:SR-implies-SD}) and SD safety automata are determinisable-by-pruning (\cref{lemma:sd-safety-is-dbp}), we call  such states $p$ as \emph{safe-deterministic}. 

We will build a memoryless adversarially resolvable automaton $\Hc$, whose states are the safe-deterministic states in $\Cc$. This construction is similar to the one used by Kuperberg and Skrzypczak, in 2015, for giving a polynomial time procedure to recognise HD coB\"uchi automata~\cite[Section E.7 in the full version]{KS15}.  We fix a uniform determinisation of transitions from each safe-deterministic state in $\Cc_{\safety}$ and we add these transitions in $\Hc$ with priority 0. If there are no outgoing transitions in $\Hc$ from the state $p$ on letter $a$ so far, then we add outgoing transitions from $p$ on $a$ as follows. Let $q$ be a state in $\Cc$ such that there is a transition from $p$ to $q$ on $a$ in $\Hc$. For each state $r$ that is weakly coreachable to $q$ in $\Cc$ and that is safe-deterministic, we add a transition from $q$ to $r$ in $\Hc$ with priority $1$. This concludes our construction of $\Hc$.

We show that the strategy of Eve that chooses transitions from $\Hc$ uniformly at random is an almost-surely winning strategy for Eve in the HD game, and thus, $\Hc$ is MA (\cref{lemma:cobuchi-h-is-ma},\cref{app:succandcompCB}). Both this fact and the language equivalence of $\Hc$ to $\Cc$ primarily relies on \cref{lemma:cobuchi-srselfcoverage-implies-somtingsdbp}. Since a safety automaton is DBP if and only if that automaton is HD and every HD safety automaton can be determinised in polynomial-time~\cite{BL23quantitative},  the safe-deterministic states of $\Cc$ can be identified, and outgoing safe transitions from these states can be determinised in polynomial-time. Thus, the construction of $\Hc$ takes polynomial-time overall. This completes our proof sketch for \cref{theorem:coBuchiHDisSR}.

%the run of $(\Pc_{\safety})$ from $(p,m)$ on $\alpha_{(p,m)}$ sees a rejectin


%Thus, there is a finite word $u_{(p,m)}$ such that a run of $\Pc_{\safety}$ from $(p,m)$ on $u_{p,m}$ sees a rejecting transition with positive probability, but there is a run of $\Cc_{\safety}$ on $u_{p,m}$ from $q$ to $q'$ that contains only safe transitions. Now, $q'$ and $q$ are in the same SCC in the graph consisting of  $\Cc_{\safety}$

We next show that MA coB\"uchi automata are exponentially more succinct when compared to deterministic coB\"uchi automata, thus proving the coBu\"chi part of \cref{theorem:succ}.
\begin{corollary}\label{lemma:succinctcoBuchi}
    There is a family $L_2,L_3,L_4,\dots$ of languages such that for every $n\geq 2$, there is a memoryless adversarially resolvable automaton recognising $L_n$ that has $2n+1$ states and any deterministic coB\"uchi automaton recognising $L_n$ needs at least $\Omega(2^n/2n+1)$ states. 
\end{corollary}
% Since 
The language family constructed in the work of Kuperberg and Skrzypczak~\cite[Theorem~1]{KS15} are accepted by $2n+1$-state HD coB\"uchi automata and are not accepted by any $\Omega(2^n/2n+1)$-state deterministic coB\"uchi automata. From \cref{theorem:coBuchiHDisSR}, since any SR---and therefore any HD---automaton has a language-equivalent MA automaton with the same number of states, \cref{lemma:succinctcoBuchi} follows. 
\subsection{B\"uchi automata}\label{subsec:sac-buchi}
Similar to coB\"uchi automata, we show that for B\"uchi automata, no two notions amongst the notions of semantic determinism, stochastic resolvability, history determinism, memoryless stochastic resolvability, and memoryless adversarial resolvability coincide. We then later show that memoryless-stochastically resolvable automata are exponentially more succinct than HD B\"uchi automata: recall that this is not the case for coB\"uchi automata (\cref{theorem:coBuchiHDisSR}).

We start by giving a SD B\"uchi automaton that is not stochastically resolvable. Consider the B\"uchi automaton as shown in \cref{fig:BuchiSDbutnotSR} below, which we show is SD but not SR.
\begin{figure}[ht]
\centering
        \begin{tikzpicture}[auto]
        \tikzset{every state/.style = {inner sep=-3pt,minimum size =15}}
    \node[state] (s1)  at (0,0) {$q_0$};
    \node[state]  (s2)  at (2,0) {};
    \node[state] (s3) at (1,0.8) {$q_a$};
    \node[state] (s4) at (1,-0.8) {$q_b$};

    \node[state] (f1)  at (-1.2,0) {};
    \node[state]  (f2)  at (-2.2,0.8) {};
    \node[state] (f3) at (-2.2,-0.8) {};

    \path[->]
        (f3) edge node [left] {$x$}  (f2)
        (f1) edge node [yshift=1mm] {$y$} (f3)
        (f2) edge node [xshift=-2mm] {$a,b$} (f1)
        (s2) edge [double,bend left = 8] node {$z$}  (s1)
          (s1) edge  node [below,xshift=2mm,yshift=2mm] {$x$} (s3)
         (s1) edge  node [above,xshift=2mm,yshift=-2mm] {$x$} (s4)
         
         (s3) edge  node [above] {$a$} (s2)
         (s4) edge  node [below] {$b$} (s2)
         
         (s2) edge [double,bend left = 8] node [below] {$z$} (s1)
         (s2) edge [bend right = 8] node [above] {$y$} (s1)
         (s3) edge  node [above] {$b$} (f1)
         (s4) edge node [below] {$a$} (f1)
         (f1) edge node [above,xshift=1mm,yshift=-0.5mm] {$z$} (s1)
;
    \path[->,every node/.style={sloped,anchor=south}]
        ;
    \end{tikzpicture}
    \caption{A semantically deterministic B\"uchi automaton that is not stochastically resolvable. The accepting transitions are double-arrowed, and the initial state is $q_0$.}
    \label{fig:BuchiSDbutnotSR}
\end{figure} 
\begin{restatable}{lemma}{buchisdnotsr}\label{lemma:buchisdnotsr}
    There is a semantically deterministic B\"uchi automaton that is not stochastically resolvable.
\end{restatable}
\begin{proof}[Proof sketch]
    Consider the B\"uchi automaton $\Bc$ shown in \cref{fig:BuchiSDbutnotSR}. This automaton $\Bc$ has nondeterminism on the initial state $q_0$, and it recognises the language $$((x \cdot (a+b)\cdot y)^{*}(x\cdot (a+b)\cdot z))^{\omega}.$$ 
    It is easy to verify that $\Bc$ is SD. We will describe a strategy for Adam in the SR game on $\Bc$ using which he wins almost surely. This would imply, due to \cref{lemma:random-is-pure}, that $\Bc$ is not SR. Note that when Eve's token is at $q_0$ in the SR game, then Eve needs to guess whether the next letter after $x$ is going to be $a$ or $b$. If she guesses incorrectly, then her token moves to the left states---states $l_1,l_2,$ and $l_3$, where she stays until a $z$ is seen. Adam's strategy in the SR game is as follows. Let $Y$ be the regular expression $xay+xby$ and $Z$ be the regular expression $xaz+xbz$. Note that both $Y$ and $Z$ consist of two words. Adam picks a word from the set $YZY^2ZY^3ZY^4Z \dots$ in the SR game, where from each occurrence of $Y$ or $Z$, he picks one of the two words in the regular expression with half probability. We show that the probability that Eve's token takes an accepting transition on reading a word chosen randomly from $Y^{n}Z$ is $\frac{1}{2^{n+1}}$. It then follows from the Borel-Cantelli lemma (\cref{lemma:borellcantelli}, \cref{app:defs}) that the probability that Eve's token takes infinitely many accepting transitions in the SR game is 0, as desired. 
\end{proof}

We showed in \cref{lemma:reachability-MR-not-HD} that MR reachability automaton are not HD. This also shows that there are MR B\"uchi automata which are not HD. We next show the other side by showing that there are HD B\"uchi automata that are not MR.
\begin{restatable}{lemma}{HDBuchinotMR}\label{lemma:HDBuchinotMR}
    There is a history-deterministic B\"uchi automaton that is not memoryless stochastically resolvable.
\end{restatable}
\begin{proof}[Proof sketch]
    Consider the automaton in \cref{fig:HDBuchinotMR}. We will only define the language here, but remark that the proof that this language is not MR is similar to that the proof of \cref{lemma:HDcoBuchinotMR}. 
    Let  $\Sigma_\diamond = \{a,b,c,\diamond\}$ and $\Sigma = \{a,b,c\}$. Let $L_1$ and $L_2$ be languages of finite words over the alphabet $\Sigma_\diamond$ where $L_1 =   {\Sigma_\diamond}^*  c^+\diamond $ and $L_2 = {\Sigma_\diamond}^* a \Sigma^* b^+\diamond$. 
    The automaton $\Ac$ accepts the language $\left[(L_1+L_2)^*(L_1L_1+L_2L_2)\right]^\omega$. Equivalently, it accepts words in $(L_1+L_2)^\omega$ that are not in  $(L_1+L_2)^*(L_1L_2)^\omega$.  
\end{proof}
\begin{figure}[ht]
\centering
        \begin{tikzpicture}
        \tikzset{every state/.style = {inner sep=-3pt,minimum size =15}}

    \node[state] (q0) at (-1,1.5) {};
    \node[state] (q1) at (-2.5,1.5) {};
    \node[state] (q2) at (2,1.5) {};
    \node[state] (q3) at (0.5,1.5) {};
    \path[->] (-0.7,1) edge (q0);
    
    \node[state, fill=blue!40, blue!40] (r0) at (2,0.2) {$q_B$};
    \node[state, blue] (r1) at (2,-1.3) {};
    \node[state, blue] (r2) at (3.5,-1.3) {};
    \node[state, blue] (r3) at (3.5,0.2) {};    
    \node[state, blue] (r4) at (5.25,-0.55) {};    

    \node[state, fill=red!40, red!40] (l0) at (-3,0.2) {$q_R$};
    \node[state, red] (l1) at (-3,-1.3) {};
    \node[state, red] (l2) at (-1.5,-1.3) {};
    \node[state, red] (l3) at (-1.5,0.2) {};    
    \node[state, red] (l4) at (0.25,-0.55) {};    

    \node[state, fill=red!40, red!40] (l5) at (-2.25,-2.1) {$q_R$};    
    \node[state, fill=blue!40, blue!40] (l6) at (0.25,-1.8) {$q_B$};  
    
    \node[state, fill=red!40, red!40] (r5) at (2.75,-2.1) {$q_R$};    
    \node[state, fill=blue!40, blue!40] (r6) at (5.25,-1.8) {$q_B$};    
    \node (l52) at (-2.25,-2.1) {$q_R$};    
    \node (l62) at (0.25,-1.8) {$q_B$};  
    
    \node (r51) at (2.75,-2.1) {$q_R$};    
    \node (r61) at (5.25,-1.8) {$q_B$};    
    \node (l01) at (-3,0.2) {$q_R$};
    \node (r01) at (2,0.2) {$q_B$};
    
    \path[-stealth]
%    (-0.5,-0.5) edge (q0)
    (q0) edge [loop above] node [right] {$a,b,\diamond$} (q0)
    (q0) edge [bend left = 8] node [above] {$a$} (q3)
    (q3) edge [bend left = 8] node [below] {$\diamond$} (q0)
    (q0) edge [bend left = 8] node [below] {$c$} (q1)
    (q1) edge [bend left = 8] node [above] {$b,a$} (q0)
    (q3) edge [bend right = 8] node [below] {$b$} (q2)
    (q2) edge [bend right = 8] node [above] {$c,a$} (q3)
    (q3) edge [loop above] node [right] {$a,c$} (q3)
    (q1) edge [loop above] node [left] {$c$} (q1)
    (q2) edge [loop right] node [right] {$b$} (q2)

    (l0) edge [loop above] node [above] {$b,\diamond$} (l0)
    (l0) edge [bend left = 8] node [above] {$a$} (l3)
    (l3) edge [bend left = 8] node [below] {$\diamond$} (l0)
    (l1) edge  node [below, left] {$a$} (l3)
    (l0) edge  node [left] {$c$} (l1)
    (l3) edge [bend right = 8] node [left] {$c$} (l2)
    (l2) edge [bend right = 8] node [right] {$a$} (l3)
    (l3) edge [bend right = 8] node [below=3pt,left] {$b$} (l4)
    (l4) edge [bend right = 8] node [above=3pt,right] {$a$} (l3)
    (l4) edge [bend right = 8] node [above=0.5pt] {$c$} (l2)
    (l2) edge [bend right = 8] node [below=0.5pt] {$b$} (l4)
    
    (l3) edge [in=30,out=60,loop] node [right] {$a$} (l3)
    (l1) edge [in=240,out=270,loop] node [below] {$c$} (l1)
    (l2) edge [in=-30,out=-60,loop]  node [right] {$c$} (l2)
    (l4) edge [loop above] node [above] {$b$} (l4)
    

    (r0) edge [in=165,out=195,loop] node [left] {$b,\diamond$} (r0)
    (r0) edge [bend left = 8] node [above] {$a$} (r3)
    (r3) edge [bend left = 8] node [below] {$\diamond$} (r0)
    (r1) edge  node [below, left] {$a$} (r3)
    (r0) edge  node [left] {$c$} (r1)
    (r3) edge [bend right = 8] node [left] {$c$} (r2)
    (r2) edge [bend right = 8] node [right] {$a$} (r3)
    (r3) edge [bend right = 8] node [below=3pt,left] {$b$} (r4)
    (r4) edge [bend right = 8] node [above=3pt, right] {$a$} (r3)
    (r2) edge [bend right = 8] node [below=0.5pt] {$b$} (r4)
    (r4) edge [bend right = 8] node [above=0.5pt] {$c$} (r2)

    (r3) edge [in=30,out=60,loop]  node [above] {$a$} (r3)
    (r1) edge [loop left] node [left] {$c$} (r1)
    (r2) edge [in=-30,out=-60,loop] node [right] {$c$} (r2)
    (r4) edge [loop above] node [above] {$b$} (r4)

    (q1) edge [in=50,out=-90] node [right] {$\diamond$} (l0)
    (q2) edge node [right] {$\diamond$} (r0)

    (l1) edge[ double] node [left] {$\diamond$} (l5)
    (l2) edge[ double] node [right] {$\diamond$} (l5)
    (l4) edge node [left] {$\diamond$} (l6)
    %(l4) edge [out=60,in=180] node [below] {$\diamond$} (r0)

    (r1) edge node [left] {$\diamond$} (r5)
    (r2) edge node [right] {$\diamond$} (r5)
    (r4) edge [double] node [left] {$\diamond$} (r6)
;
    \end{tikzpicture}
\caption{A HD B\"uchi automaton that is not MR. The accepting transitions are represented by double arrows. All red-filled states ($q_R$) are identified as the same state, and all blue-filled states ($q_B$) are identified as the same state.}\label{fig:HDBuchinotMR}
\end{figure}
% \begin{figure}[ht]
% \centering
%         \begin{tikzpicture}
%         \tikzset{every state/.style = {inner sep=-3pt,minimum size =15}}

%     \node[state,initial,initial text=] (q0) at (0,1.5) {};
%     \node[state] (q1) at (0,0) {};
%     \node[state] (q2) at (1.5,0) {};
%     \node[state] (q3) at (1.5,1.5) {};

%     \node[state, fill=blue!40, blue!40] (r0) at (2,-1) {$q_B$};
%     \node[state, blue] (r1) at (2,-2.5) {};
%     \node[state, blue] (r2) at (3.5,-2.5) {};
%     \node[state, blue] (r3) at (3.5,-1) {};    
%     \node[state, blue] (r4) at (5.25,-1.75) {};    

%     \node[state, fill=red!40, red!40] (l0) at (-3,-1) {$q_R$};
%     \node[state, red] (l1) at (-3,-2.5) {};
%     \node[state, red] (l2) at (-1.5,-2.5) {};
%     \node[state, red] (l3) at (-1.5,-1) {};    
%     \node[state, red] (l4) at (0.25,-1.75) {};    

%     \node[state, fill=red!40, red!40] (l5) at (-2.25,-3.3) {$q_R$};    
%     \node[state, fill=blue!40, blue!40] (l6) at (0.25,-3) {$q_B$};  
    
%     \node[state, fill=red!40, red!40] (r5) at (2.75,-3.3) {$q_R$};    
%     \node[state, fill=blue!40, blue!40] (r6) at (5.25,-3) {$q_B$};    
%     \node (l52) at (-2.25,-3.3) {$q_R$};    
%     \node (l62) at (0.25,-3) {$q_B$};  
    
%     \node (r51) at (2.75,-3.3) {$q_R$};    
%     \node (r61) at (5.25,-3) {$q_B$};    
%     \node (l01) at (-3,-1) {$q_R$};
%     \node (r01) at (2,-1) {$q_B$};
    
%     \path[-stealth]
% %    (-0.5,-0.5) edge (q0)
%     (q0) edge [loop above] node [left, yshift=-2mm,xshift=-1mm] {$a,b,\diamond$} (q0)
%     (q0) edge [bend left = 8] node [above] {$a$} (q3)
%     (q3) edge [bend left = 8] node [below] {$\diamond$} (q0)
%     (q0) edge [bend left = 8] node [right] {$c$} (q1)
%     (q1) edge [bend left = 8] node [left] {$b,a$} (q0)
%     (q3) edge [bend right = 8] node [left] {$b$} (q2)
%     (q2) edge [bend right = 8] node [right] {$c,a$} (q3)
%     (q3) edge [loop right] node [right] {$a,c$} (q3)
%     (q1) edge [loop left] node [left] {$c$} (q1)
%     (q2) edge [loop right] node [right] {$b$} (q2)

%     (l0) edge [loop above] node [above] {$b,\diamond$} (l0)
%     (l0) edge [bend left = 8] node [above] {$a$} (l3)
%     (l3) edge [bend left = 8] node [below] {$\diamond$} (l0)
%     (l1) edge  node [below, left] {$a$} (l3)
%     (l0) edge  node [left] {$c$} (l1)
%     (l3) edge [bend right = 8] node [left] {$c$} (l2)
%     (l2) edge [bend right = 8] node [right] {$a$} (l3)
%     (l3) edge [bend right = 8] node [below=3pt,left] {$b$} (l4)
%     (l4) edge [bend right = 8] node [above=3pt,right] {$a$} (l3)
%     (l4) edge [bend right = 8] node [above=0.5pt] {$c$} (l2)
%     (l2) edge [bend right = 8] node [below=0.5pt] {$b$} (l4)
    
%     (l3) edge [in=30,out=60,loop] node [right] {$a$} (l3)
%     (l1) edge [in=240,out=270,loop] node [below] {$c$} (l1)
%     (l2) edge [in=-30,out=-60,loop]  node [right] {$c$} (l2)
%     (l4) edge [loop above] node [above] {$b$} (l4)
    

%     (r0) edge [in=165,out=195,loop] node [left] {$b,\diamond$} (r0)
%     (r0) edge [bend left = 8] node [above] {$a$} (r3)
%     (r3) edge [bend left = 8] node [below] {$\diamond$} (r0)
%     (r1) edge  node [below, left] {$a$} (r3)
%     (r0) edge  node [left] {$c$} (r1)
%     (r3) edge [bend right = 8] node [left] {$c$} (r2)
%     (r2) edge [bend right = 8] node [right] {$a$} (r3)
%     (r3) edge [bend right = 8] node [below=3pt,left] {$b$} (r4)
%     (r4) edge [bend right = 8] node [above=3pt, right] {$a$} (r3)
%     (r2) edge [bend right = 8] node [below=0.5pt] {$b$} (r4)
%     (r4) edge [bend right = 8] node [above=0.5pt] {$c$} (r2)

%     (r3) edge [in=30,out=60,loop]  node [above] {$a$} (r3)
%     (r1) edge [loop left] node [left] {$c$} (r1)
%     (r2) edge [in=-30,out=-60,loop] node [right] {$c$} (r2)
%     (r4) edge [loop above] node [above] {$b$} (r4)

%     (q1) edge [in=60,out=210] node [above] {$\diamond$} (l0)
%     (q2) edge node [right] {$\diamond$} (r0)

%     (l1) edge[ double] node [left] {$\diamond$} (l5)
%     (l2) edge[ double] node [right] {$\diamond$} (l5)
%     (l4) edge node [left] {$\diamond$} (l6)
%     %(l4) edge [out=60,in=180] node [below] {$\diamond$} (r0)

%     (r1) edge node [left] {$\diamond$} (r5)
%     (r2) edge node [right] {$\diamond$} (r5)
%     (r4) edge [double] node [left] {$\diamond$} (r6)
% ;
%     \end{tikzpicture}
% \caption{A HD B\"uchi automaton that is not MR. The accepting transitions are represented by double arrows. All red-filled states ($q_R$) are identified as the same state, and all blue-filled states ($q_B$) are identified as the same state.}\label{fig:HDBuchinotMR}
% \end{figure} 
We now show that SR B\"uchi automata  are exponentially more succinct than HD B\"uchi automata. 
\begin{restatable}{lemma}{succinctBuchi}\label{lemma:succinctBuchi}
    There is a family $L_2,L_3,L_4,\dots$ of languages such that for every $n\geq 2$, there is a memoryless stochastically resolvable automaton recognising $L_n$  that has $3n+3$ states and any HD B\"uchi automaton recognising $L_n$ needs at least $2^n$ states. 
\end{restatable}
To prove this result, we will use the family of languages of Abu Radi and Kupferman to show the exponential succinctness of SD B\"uchi automata~\cite[Theorem 5]{AK23}. For each $n\geq 2$, consider the alphabet $\Sigma_n=\{1,2,\dots,n,\$,\#\}$, and let us denote the set $\{1,2,\dots,n\}$ by $[n]$. 
Define the language $L_n=\{\$w_0\#i_0\$w_1\#i_1\$w_2\#i_2\dots \mid \text{~there are}$ $\text{ infinitely many indices $j$ such that $i_j$ appears in $w_j \in [n]^{\omega}$}\}$. The automaton in \cref{fig:succinctSRBuchi}, which has $3n+3$ states (with a missing rejecting sink state), accepts this language. This automaton is MR, where the memoryless resolver chooses uniformly at random one of the outgoing transitions on $\$$ at state $q_0$ is an almost-sure resolver for it. For any word $w$ in $L_n$, there are infinitely many occurrences of $i$-good words for some $i \in [1,n]$, and therefore, by the second Borel-Cantelli Lemma (\cref{lemma:secondborellcantelli}), there are infinitely many positions at which the resolver chooses the transition to $s_i$ while reading a $\$$ right before an $i$-good word from $q_0$, and then visits an accepting transition. 
\begin{figure}[ht]
\centering
  
        \begin{tikzpicture}
        \tikzset{every state/.style = {inner sep=-3pt,minimum size =15}}

    \node[state,initial,initial text=] (q0)  at (-2.5,1.5) {$q_0$};
    \node[state] (r)  at (-2.5,3) {$r$};    
    % \node[draw, rectangle, minimum width=2.5cm, minimum height=1cm] (box1)  at (0,-1) {$\Ac_3$};
    % \node[draw, rectangle, minimum width=2.5cm, minimum height=1cm]  (box2)  at (0,1.5) {$\Ac_2$};
    % \node[draw, rectangle, minimum width=2.5cm, minimum height=1cm] (box3) at (0,3) {$\Ac_1$};

    \node[state] (s1)  at (-0.7,-0.7) {$s_n$};
    \node[state]  (s2)  at (-0.7,1.5) {$s_2$};
    \node[state] (s3) at (-0.7,3) {$s_1$};

    \node[state] (m1)  at (0.4,-0.7) {$m_n$};
    \node[state]  (m2)  at (0.4,1.5) {$m_2$};
    \node[state] (m3) at (0.4,3) {$m_1$};

    \node[state] (f1)  at (1.4,-0.7) {$f_n$};
    \node[state]  (f2)  at (1.4,1.5) {$f_2$};
    \node[state] (f3) at (1.4,3) {$f_1$};

    \node (d)  at (0,0.9) {$\vdots$};

    \node[state] (fin) at (3,1.5) {$q_0$};

    \path[->]
    (s3) edge node [above] {$\#$}  (r)
    (r) edge node [left] {$[n]$}  (q0);

        \path[->]
    (s1) edge[loop above] node [above] {$[n]\setminus\{i\}$}  (s1)
        (s1) edge node [above] {$i$}  (m1)
        (m1) edge [loop above] node [above] {$[n]$}  (m1)
        (m1) edge  node [above] {$\#$} (f1)
        ;
            \path[->]
    (s2) edge[loop above] node [above] {$[n]\setminus\{i\}$}  (s2)
        (s2) edge node [above] {$i$}  (m2)
        (m2) edge [loop above] node [above] {$[n]$}  (m2)
        (m2) edge  node [above] {$\#$} (f2)
        ;
            \path[->]
    (s3) edge[loop above] node [above] {$[n]\setminus\{i\}$}  (s3)
        (s3) edge node [above] {$i$}  (m3)
        (m3) edge [loop above] node [above] {$[n]$}  (m3)
        (m3) edge  node [above] {$\#$} (f3)
        ;
        \path[->]
        (q0) edge node [below] {$\$$} (s1)
        (q0) edge node [below] {$\$$} (s2)
        (q0) edge node [above] {$\$$} (s3)

        % (s1) edge node [below] {$\$$} (m1)
        % (s2) edge node [below] {$\$$} (m2)
        % (s3) edge node [below] {$\$$} (m3)

        % (s1) edge[bend right = 8] node [above, xshift=1.5mm] {$[n]$} (q0)
        % (s2) edge[bend right = 8] node [above, xshift=0.5mm,yshift=-1mm] {$[n]$} (q0)
        % (s3) edge[bend right = 8] node [above,xshift=-1mm] {$[n]$} (q0)

        (f1) edge[bend right =15, double] node [left] {$n$} (fin)
        (f1) edge[bend right = 30] node [right] {$[n]\setminus \{n\}$} (fin)
        (f2) edge[double] node [above] {$2$} (fin)
        (f2) edge[bend right =30] node [below] {$[n]\setminus \{2\}$} (fin)
        (f3) edge[double, bend left = 8] node [below, xshift=-2mm,yshift=+2mm] {$1$} (fin)
        (f3) edge[bend left = 30] node [right] {$[n]\setminus \{1\}$} (fin);
    \end{tikzpicture}
\caption{An MR B\"uchi automaton that is exponentially more succinct than any HD automaton accepting the same language. Both $q_0$s are the identified as the same state. Additionally, transitions are added from all states $s_i$ on $\#$ to state $r$. Other missing transitions go to a sink state (not pictured).}\label{fig:succinctSRBuchi}
\end{figure} 