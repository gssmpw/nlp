\vspace{-2mm}
\section{Conclusion}
\vspace{-2mm}
This survey reviews the literature on data contamination in LLM benchmarking, analyzing both static and dynamic approaches. We find that static methods, though consistent, become more vulnerable to contamination as training datasets grow. While dynamic approaches show promise, they face challenges in reliability and reproducibility. Future research should focus on standardized dynamic evaluation, and practical mitigation tools.
% As LLMs continue to advance, ensuring the integrity of evaluation benchmarks through effective contamination prevention remains crucial for the field's progress.


\section* {Limitations}

While this survey provides a comprehensive overview of static and dynamic benchmarking methods for LLMs, there are several limitations to consider. First, due to the rapidly evolving nature of LLM development and benchmarking techniques, some recent methods or tools may not have been fully covered. As benchmarking practices are still emerging, the methods discussed may not yet account for all potential challenges or innovations in the field. Additionally, our proposed criteria for dynamic benchmarking are a first step and may need further refinement and validation in real-world applications. Lastly, this survey focuses primarily on high-level concepts and may not delve into all the fine-grained technical details of specific methods, which may limit its applicability to practitioners seeking in-depth implementation guidelines.


\section* {Ethical Considerations}

Our work is rooted in the goal of enhancing the transparency and fairness of LLM evaluations, which can help mitigate the risks of bias and contamination in AI systems. However, ethical concerns arise when considering the use of both static and dynamic benchmarks. Static benchmarks, if not carefully constructed, can inadvertently perpetuate biases, especially if they rely on outdated or biased data sources. Dynamic benchmarks, while offering a more adaptive approach, raise privacy and security concerns regarding the continual collection and updating of data. Moreover, transparency and the potential for misuse of benchmarking results, such as artificially inflating model performance or selecting biased evaluation criteria, must be carefully managed. It is essential that benchmarking frameworks are designed with fairness, accountability, and privacy in mind, ensuring they do not inadvertently harm or disadvantage certain user groups or research domains. Lastly, we encourage further exploration of ethical guidelines surrounding data usage, model transparency, and the broader societal impact of AI benchmarks.