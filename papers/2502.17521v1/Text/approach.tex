\section{Dynamic Benchmarking}
\label{sec:dynamic}

Due to the inherent limitations of static benchmarking schemes, they face challenges in providing a transparent yet faithful evaluation of LLMs. To address this, \textit{dynamic} benchmarking has been proposed.  

% \begin{table*}[ht]
    \centering
    \small
    \begin{tabularx}{\textwidth}{lX}
        \toprule
        \textbf{Task} & \textbf{Benchmark} \\ 
        \midrule
        Math & LiveBench~\citep{white2024livebench}, UGMathBench~\citep{xu2025ugmathbench}, Mathador-LM~\citep{kurtic-etal-2024-mathador} \\
        Language & LiveBench~\citep{white2024livebench}, C$^2$LEVA~\citep{li2024c}, ITD~\citep{zhu-etal-2024-inference} \\
        Coding   & LiveBench~\citep{white2024livebench}, LiveCodeBench~\citep{jain2024livecodebenchholisticcontaminationfree}, ComplexCodeEval~\citep{10.1145/3691620.3695552} \\
        Reasoning & LiveBench~\citep{white2024livebench}, DyVal~\citep{zhu2024dyval}, C$^2$LEVA~\citep{li2024c}, NPHardEval~\citep{fan-etal-2024-nphardeval}, S3Eval~\citep{lei-etal-2024-s3eval}, DARG~\citep{zhang2024darg} \\
        Knowledge & C$^2$LEVA~\citep{li2024c}, ITD~\citep{zhu-etal-2024-inference}, Auto-Dataset~\citep{ying2024automating}, DyVal2~\citep{10.5555/3692070.3694661}, SciEval~\citep{sun2024scieval} \\
        Safety & C$^2$LEVA~\citep{li2024c}, FactBench~\citep{bayat2024factbench} \\
        Instruction Following & LiveBench~\citep{white2024livebench} \\
        Reading Comprehension & LatestEval~\citep{li2023avoiding}, Antileak-bench~\citep{wu2024antileak} \\
        \bottomrule 
        \end{tabularx}
    
    \caption{Summary of dynamic benchmarks.}
    \label{tab:dynamic-benchmarks}
\end{table*}


\subsection{Problem Formulation}

% We begin by formulating the problem of dynamic benchmarking for LLMs.
A dynamic benchmark is defined as  
$
\small
\mathcal{B}_{\text{dynamic}} = (\mathcal{D}, T(\cdot)), \quad 
\mathcal{D} = (\mathcal{X}, \mathcal{Y}, \mathcal{S}(\cdot))
$
where \( \mathcal{D} \) represents the static benchmark dataset. 
% consisting of input prompts \( \mathcal{X} \), expected outputs \( \mathcal{Y} \), and a scoring function \( \mathcal{S}(\cdot) \) that evaluates the quality of an LLM's outputs by comparing them against \( \mathcal{Y} \). 
The transformation function \( T(\cdot) \) modifies the data set during the benchmarking to avoid possible data contamination.
The dynamic dataset for the evaluation of an LLM can then be expressed as
$
\small
        \mathcal{D}_t = T_t(\mathcal{D}),  \quad
        \forall t \in \{1, \dots, N\}
$
where \( \mathcal{D}_t \) represents the evaluation data set at the timestamp \( t \), and \( N\) is the total timestamp number, which could be finite or infinite. % \ie $N= \infty$.
If the seed dataset $\mathcal{D}$ is empty, the dynamic benchmarking dataset will be created from scratch.




% Table generated by Excel2LaTeX from sheet 'Sheet1'
\begin{table*}[htbp]
  \centering
  \resizebox{0.98\textwidth}{!}{
    \begin{tabular}{clcccccc}
    \toprule
    \toprule
    \multirow{2}[2]{*}{\textbf{Dynamic Mechnisms}} & \multirow{2}[2]{*}{\textbf{Bechmark Name}} & \multicolumn{6}{c}{\textbf{Evaluation Creteria}} \\
          &       & \textbf{Correctness} & \textbf{Scalability} & \textbf{Collision} & \textbf{Stable of Complexity} & \textbf{Diversity} & \textbf{Interpretability} \\
    \midrule
    \multirow{6}[2]{*}{\textbf{Temporal Cutoff}} & LiveBench~\citep{white2024livebench} & \CIRCLE  & \LEFTcircle & \LEFTcircle & \Circle & \Circle & \CIRCLE  \\
          & AcademicEval~\citep{zhang2024academiceval}  & \CIRCLE  & \LEFTcircle & \LEFTcircle & \Circle & \Circle & \CIRCLE  \\
          & LiveCodeBench~\citep{jain2024livecodebenchholisticcontaminationfree} & \CIRCLE  & \LEFTcircle & \LEFTcircle & \Circle & \Circle & \CIRCLE  \\
          & LiveAoPSBench~\citep{mahdavi2025leveraging} & \CIRCLE  & \LEFTcircle & \LEFTcircle & \Circle & \Circle & \CIRCLE  \\
          & AntiLeak-Bench~\citep{wu2024antileak} & \CIRCLE  & \CIRCLE  & \LEFTcircle & \CIRCLE  & \Circle & \CIRCLE  \\
    \midrule
    \multirow{6}[2]{*}{\textbf{Rule-Based}} & S3Eval~\citep{lei-etal-2024-s3eval} & \CIRCLE  & \CIRCLE  & \LEFTcircle & \CIRCLE  & \LEFTcircle & \CIRCLE  \\
          & DyVal~\citep{zhu2024dyval}  & \CIRCLE  & \CIRCLE  & \CIRCLE  & \CIRCLE  & \LEFTcircle & \CIRCLE  \\
        & MMLU-CF~\citep{zhao2024mmlu} & \CIRCLE  & \LEFTcircle & \Circle & \CIRCLE  & \LEFTcircle & \LEFTcircle \\
          & NPHardEval~\citep{fan-etal-2024-nphardeval} & \CIRCLE  & \CIRCLE  & \CIRCLE  & \CIRCLE  & \LEFTcircle & \CIRCLE  \\
          & GSM-Symbolic~\citep{mirzadeh2025gsmsymbolic} & \CIRCLE  & \Circle & \LEFTcircle & \CIRCLE  & \LEFTcircle & \CIRCLE  \\
          & PPM~\citep{chen2024ppm}   & \CIRCLE  & \CIRCLE  & \CIRCLE  & \LEFTcircle & \LEFTcircle & \CIRCLE  \\
          & GSM-Infinite~\citep{zhou2025gsm} & \CIRCLE  & \CIRCLE  & \CIRCLE  & \CIRCLE  & \LEFTcircle & \CIRCLE  \\
    \midrule
    \multirow{7}[2]{*}{\textbf{LLM-Based}} & Auto-Dataset~\citep{ying2024automating} & \LEFTcircle & \CIRCLE  & \LEFTcircle & \CIRCLE  & \CIRCLE  & \Circle \\
          & LLM-as-an-Interviewer~\citep{kim2024llm} & \LEFTcircle & \CIRCLE  & \LEFTcircle & \LEFTcircle & \CIRCLE  & \Circle \\
          & TreeEval~\citep{li2024treeeval} & \LEFTcircle & \CIRCLE  & \LEFTcircle & \LEFTcircle & \LEFTcircle & \Circle \\
          & BeyondStatic~\citep{li2023beyond} & \LEFTcircle & \CIRCLE  & \LEFTcircle & \Circle & \CIRCLE  & \Circle \\
          & StructEval~\citep{cao-etal-2024-structeval}  & \LEFTcircle & \CIRCLE  & \CIRCLE  & \CIRCLE  & \CIRCLE  & \Circle \\
          & Dynabench~\citep{kiela2021dynabench}  & \LEFTcircle & \LEFTcircle & \LEFTcircle & \LEFTcircle & \CIRCLE  & \Circle \\
          & Self-Evolving~\citep{wang2024selfevolving} & \LEFTcircle & \CIRCLE  & \LEFTcircle & \CIRCLE  & \CIRCLE  & \Circle \\
    \midrule
    \multirow{3}[2]{*}{\textbf{Hybrid }} & DARG~\citep{zhang2024darg}  & \LEFTcircle & \CIRCLE  & \LEFTcircle & \CIRCLE  & \CIRCLE  & \LEFTcircle \\
    & LatestEval~\citep{li2023avoiding} & \CIRCLE  & \LEFTcircle & \LEFTcircle & \Circle & \Circle & \CIRCLE  \\
          & C2LEVA~\citep{li2024c} & \LEFTcircle & \CIRCLE  & \LEFTcircle & \LEFTcircle & \CIRCLE  & \LEFTcircle \\
    \bottomrule
    \bottomrule
    \end{tabular}%
    
    }
  \caption{Existing dynamic benchmarks and their quality on our summarized criteria. $\CIRCLE$ represents support, $\LEFTcircle$ represents partial support, and $\Circle$ represents no support }      
  \label{tab:dynamic}%
\end{table*}%

\subsection{Evaluation Criteria}

While many dynamic benchmarking methods have been proposed to evaluate LLMs, the evaluation criteria for assessing these benchmarks themselves remain non-standardized. To this end, we propose the following evaluation criteria to assess the quality of a dynamic benchmarking algorithm.

\subsubsection{Correctness}
The first criterion for evaluating the quality of dynamic benchmarking is \textsf{Correctness}. If the correctness of the generated dataset cannot be guaranteed, the benchmark may provide a false sense of reliability when applied to benchmarking LLMs, leading to misleading evaluations.  
We quantify the correctness of dynamic benchmarks as:  
$$
\small
    \textsf{Correctness} = \mathbb{E}_{i=1}^{N}  
    \mathcal{S} \big( \mathcal{Y}_i, \mathcal{G}(\mathcal{X}_i) \big)
$$
where \( \mathcal{X}_i \) and \( \mathcal{Y}_i \) represent the input and output of the \( i^{th} \) transformation, respectively. The function \( \mathcal{G}(\cdot) \) is an oracle that returns the ground truth  of its input, ensuring an objective reference for correctness evaluation. For example, the function \( \mathcal{G}(\cdot) \) could be a domain-specific an annotator.  
This equation can be interpreted as the expected alignment between the transformed dataset's outputs and their corresponding ground truth values, measured using the scoring function \( \mathcal{S}(\cdot) \). A higher correctness score indicates that the dynamic benchmark maintains correctness to the ground truth.



\subsubsection{Scalability}
The next evaluation criterion is scalability, which measures the ability of dynamic benchmarking methods to generate large-scale benchmark datasets. A smaller dataset can introduce more statistical errors during the benchmarking process. Therefore, an optimal dynamic benchmark should generate a larger dataset while minimizing associated costs. The scalability of a dynamic benchmark is quantified as:
$$
\small
    \textsf{Scalability} = \mathbb{E}_{i=1}^{N} \left[ \frac{\lVert T_i(\mathcal{D}) \rVert}{\lVert \mathcal{D} \rVert \times \textsf{Cost}(T_i)} \right]
$$
This represents the expectation over the entire transformation space, where \( \lVert T_i(\mathcal{D}) \rVert \) is the size of the transformed dataset, and \( \lVert \mathcal{D} \rVert \) is the size of the original dataset. The function \( \textsf{Cost}(\cdot) \) measures the cost associated with the transformation process, which could include monetary cost, time spent, or manual effort according to the detailed scenarios.
This equation could be interpreted as the proportion of data that can be generated per unit cost.



\subsubsection{Collision}

One of the main motivations for dynamic benchmarking is to address the challenge of balancing transparent benchmarking with the risk of data contamination. Since the benchmarking algorithm is publicly available, an important concern arises:  \textit{If these benchmarks are used to train LLM, can they still reliably reflect the true capabilities of LLMs?}
To evaluate the robustness of a dynamic benchmark against this challenge, we introduce the concept of \textit{collision} in dynamic benchmarking. Collision refers to the extent to which different transformations of the benchmark dataset produce overlapping data, potentially limiting the benchmark’s ability to generate novel and diverse test cases. To quantify this, we propose the following metrics:  

\[
\small
\begin{split}
    & \textsf{Collision Rate} = \mathbb{E}_{\substack{i, j = 1, \, i \neq j}}^{N}  
    \left[ \frac{\lVert \mathcal{D}_i \cap \mathcal{D}_j \rVert }{ \lVert \mathcal{D} \rVert} \right]\\
    & \textsf{Repeat} = \mathbb{E}_{i=1}^{N} \left[ k \mid k = \min \left\{ \bigcup_{j=1}^{k} \mathcal{D}_j \supseteq \mathcal{D}_i \right\} \right]
\end{split}
\]  
\textsf{Collision Rate} measures the percentage of overlap between two independently transformed versions of the benchmark dataset, indicating how much poential contamination among two trials. \textsf{Repeat Trials} quantifies the expected number of transformation trials required to fully regenerate an existing transformed dataset \( T_i(\mathcal{D}) \), providing insight into the benchmark’s ability to produce novel variations.  
These metrics help assess whether a dynamic benchmark remains effective in evaluating LLM capabilities, even when exposed to potential training data contamination.  


\subsubsection{Stable of Complexity}


Dynamic benchmarks must also account for complexity to help users determine whether a performance drop in an LLM on the transformed dataset is due to potential data contamination or an increase in task complexity. If a dynamic transformation increases the complexity of the seed dataset, a performance drop is expected, even without data contamination. However, accurately measuring the complexity of a benchmark dataset remains a challenging task. Existing work has proposed various complexity metrics, but these are often domain-specific and do not generalize well across different applications. For example, DyVal~\citep{zhu2024dyval} proposes applying graph complexity to evaluate the complexity of reasoning problems.
Formally, given a complexity measurement function \( \Psi(\cdot) \), the stability can be formulated as:  
\[
\small
    \textsf{Stability} = \text{Var}(\Psi(D_i))
\]  
This equation can be interpreted as the variance in complexity across different trials, where high variance indicates that the dynamic benchmarking method is not stable.  

% \subsubsection{Stability} 
% Even if a dynamic benchmark has a complexity fucntion for each of its generated data sample, it can also need to enure the benchmark is stable, 



\subsubsection{Diversity }
Besides the aforementioned criteria, another important factor is the diversity of the transformed dataset. This diversity can be categorized into two components: \textsf{external diversity} and \textsf{internal diversity}:  External diversity measures the variation between the transformed dataset and the seed dataset. Internal diversity quantifies the differences between two transformation trials.  
\[
    \small
    \begin{aligned}
        \textsf{External Diversity} &= \mathbb{E}_{i = 1}^{N} \Theta(\mathcal{D}_i, \mathcal{D}) \\
        \textsf{Internal Diversity} &= \mathbb{E}_{\substack{i, j = 1,  i \neq j}}^{N} \Theta(\mathcal{D}_i, \mathcal{D}_j)
    \end{aligned}
\]
where \( \Theta(\cdot) \) is a function that measures the diversity between two datasets. For example, it could be the N-gram metrics or the reference based metrics, such as BLEU scores. 





\subsubsection{Interpretability} 
Dynamic benchmarking generates large volumes of transformed data, making manual verification costly and challenging. To ensure correctness, the transformation process must be interpretable. Interpretable transformations reduce the need for extensive manual validation, lowering costs. Rule-based or manually crafted transformations are inherently interpretable, while LLM-assisted transformations depend on the model's transparency and traceability. In such cases, additional mechanisms like explainability tools, or human-in-the-loop validation may be needed to ensure reliability and correctness.


% \fakeparagraph{Augmentation ?}

% \clearpage



\subsection{Existing Work}


Building on the task formats of static benchmarks, dynamic benchmarks have been introduced to assess LLM capabilities while minimizing data contamination and ensuring fairness. 
Table~\ref{tab:static-benchmarks} summarizes recent dynamic benchmarks for LLM evaluation.
Dynamic benchmarks can be categorized into four types based on their construction process: temporal cutoff, rule-based generation, LLM-based generation, and hybrid approaches.
Temporal cutoff follows a data collection process similar to static benchmarks, with the key difference being that data is gathered from newly released information.
Rule-based and LLM-based generation approaches create novel evaluation data points using predefined rules or by leveraging the strong generative capabilities of LLMs.
Finally, hybrid approaches combine the idea of these different approaches.

\subsubsection{Temporal Cutoff} 
Since LLMs typically have a knowledge cutoff date, using data collected after this cutoff to construct dataset can help evaluate the model while mitigating data contamination. 
This approach has been widely adopted to construct reliable benchmarks that prevent contamination.
LiveBench~\citep{white2024livebench} collects questions based on the latest information source, e.g., math competitions from the past 12 months, with new questions added and updated every few months.
AntiLeak-Bench~\citep{wu2024antileak} generates queries about newly emerged knowledge that was unknown before the model's knowledge cutoff date to eliminate potential data contamination.
AcademicEval~\citep{zhang2024academiceval} designs academic writing tasks on latest arXiv papers.
LiveCodeBench~\citep{jain2024livecodebenchholisticcontaminationfree} continuously collects new human-written coding problems from online coding competition platforms like LeetCode.
LiveAoPSBench~\citep{mahdavi2025leveraging} collects live math problems from the Art of Problem Solving forum.
Forecastbench~\citep{smith2024forecastbench} updates new forecasting questions on a daily basis from different data sources, e.g., prediction markets.


\paragraph{Limitations}
The collection process typically requires significant human effort~\citep{white2024livebench,jain2024livecodebenchholisticcontaminationfree}, and continuous updates demand ongoing human involvement. Despite the popularity of temporal cutoffs, using recent information from competitions to evaluate LLMs can still lead to data contamination, as these problems are likely to be reused in future competitions~\citep{wu2024antileak}. Verification is often overlooked in these live benchmarks~\citep{white2024livebench}.

\subsubsection{Rule-Based Generation}
This method synthesizes new test cases based on predefined rules, featuring an extremely low collision probability~\citep{zhu2024dyval}.

\paragraph{Template-Based}
GSM-Symbolic~\citep{mirzadeh2025gsmsymbolic} creates dynamic math benchmarks by using query templates with placeholder variables, which are randomly filled to generate diverse problem instances.
Mathador-LM\citep{kurtic-etal-2024-mathador} generates evaluation queries by adhering to the rules of Mathador games\citep{PUMA2023105587} and varying input numbers.
MMLU-CF~\citep{zhao2024mmlu} follows the template of multiple-choice questions and generates novel samples by shuffling answer choices and randomly replacing incorrect options with "None of the other choices."


\paragraph{Table-Based}
S3Eval~\citep{lei-etal-2024-s3eval} evaluates the reasoning ability of LLMs by assessing their accuracy in executing random SQL queries on randomly generated SQL tables.

\paragraph{Graph-Based}
In this category, LLMs are evaluated with randomly generated graphs.
For instance, DyVal~\citep{zhu2024dyval} assesses the reasoning capabilities of LLMs using randomly generated directed acyclic graphs (DAGs). 
The framework first constructs DAGs with varying numbers of nodes and edges to control task difficulty. 
%For example, in arithmetic reasoning tasks, leaf nodes represent random numeric values, while edges correspond to randomly assigned arithmetic operators. 
These DAGs are then transformed into natural language descriptions through rule-based conversion. Finally, the LLM is evaluated by querying it for the value of the root node.
Similarly, NPHardEval~\citep{fan-etal-2024-nphardeval} evaluates the reasoning ability of LLMs on well-known P and NP problems, such as the Traveling Salesman Problem (TSP). 
Random graphs of varying sizes are synthesized as inputs for TSP to assess the LLM's performance.
\citet{xie2024memorization} automatically constructs Knights and Knaves puzzles with random reasoning graph.

% I think these two papars are not related.
% \zw{
% \paragraph{Structured Generation}
% Dynabench~\citep{geva2024dynabench} and Dynatask~\citep{liu2024dynatask} rethink benchmark construction by applying structured rules to create diverse test tasks.
% }

\paragraph{Limitations} 
The pre-defined rules may limit sample diversity, and publicly available rule-generated data may increase the risk of in-distribution contamination during training~\citep{tu2024dice}.



\subsubsection{LLM-Based Generation}

\paragraph{Benchmark Rewriting}
In this category, LLMs are employed to rewrite samples from existing static benchmarks, which may be contaminated.
Auto-Dataset~\citep{ying2024automating} prompts LLMs to generate two types of new samples: one that retains the stylistics and essential knowledge of the original, and another that presents related questions at different cognitive levels~\citep{bloom1956handbook}.
StructEval~\citep{cao-etal-2024-structeval} expands on examined concepts from the original benchmark by using LLMs and knowledge graphs to develop a series of extended questions.
ITD~\citep{zhu-etal-2024-inference} utilizes a contamination detector~\citep{shidetecting} to identify contaminated samples in static benchmarks and then prompts an LLM to rewrite them while preserving their difficulty levels.
VarBench~\citep{qian2024varbench} prompts LLMs to identify and replace variables in samples from existing benchmarks, generating new samples.


\paragraph{Interactive Evaluation} 
In this category, inspired by the human interview process, LLMs are evaluated through multi-round interactions with an LLM ~\citep{li2023beyond}.
LLM-as-an-Interviewer~\citep{kim2024llm} employs an interviewer LLM that first paraphrases queries from existing static benchmarks and then conducts a multi-turn evaluation by posing follow-up questions or providing feedback on the examined LLM's responses.
TreeEval~\citep{li2024treeeval} begins by generating an initial question on a given topic using an LLM. Based on the previous topic and the examined LLM’s response, it then generates follow-up subtopics and corresponding questions to further assess the model.
KIEval~\citep{yu-etal-2024-kieval} generates follow-up questions based on the evaluated model's response to an initial question from a static benchmark.

\paragraph{Multi-Agent Evaluation} 
Inspired by the recent success of multi-agents systems~\citep{guo2024large}, multi-agent collaborations are used to construct dynamic benchmarks.
Benchmark Self-Evolving~\citep{wang2024selfevolving} employs a multi-agent framework to dynamically extend existing static benchmarks, showcasing the potential of agent-based methods.
Given a task description, BENCHAGENTS~\citep{butt2024benchagents} leverages a multi-agent framework for automated benchmark creation. 
It splits the process into planning, generation, verification, and evaluation—each handled by a specialized LLM agent.
This coordinated approach, with human-in-the-loop feedback, yields scalable, diverse, and high-quality benchmarks. 



\paragraph{Limitations}
The quality of LLM-generated samples is often uncertain. For example, human annotation of LatestEval~\citep{li2023avoiding} shows that about 10\% of samples suffer from faithfulness and answerability issues, compromising evaluation reliability. Moreover, in interactive evaluation, reliability largely depends on the interviewer LLM, which generates questions and assigns scores.

\subsubsection{Hybrid Generation}
LatestEval~\citep{li2023avoiding} combines temporal cutoff and LLM-based generation to automatically generate reading comprehension datasets using LLMs on real-time content from sources such as BBC.
DARG~\citep{zhang2024darg} integrates LLM-based and graph-based generation. It first extracts reasoning graphs from existing benchmarks and then perturbs them into new samples using predefined rules.
C$^2$LEVA~\citep{li2024c} incorporates all three contamination-free construction methods to build a contamination-free bilingual evaluation.

