\begin{abstract}
NVMe Flash-based SSDs are widely deployed in data centers to cache working sets of large-scale web services. As data centers face increasing sustainability demands, such as reduced carbon emissions, efficient management of Flash overprovisioning and endurance has become crucial. Our analysis demonstrates that mixing data with different lifetimes on Flash blocks results in high device garbage collection costs, which either reduce device lifetime or necessitate host overprovisioning. Targeted data placement on Flash to minimize data intermixing and thus device write amplification shows promise for addressing this issue.

The NVMe Flexible Data Placement (FDP) proposal is a newly ratified technical proposal aimed at addressing data placement needs while reducing the software engineering costs associated with past storage interfaces, such as ZNS and Open-Channel SSDs. In this study, we explore the feasibility, benefits, and limitations of leveraging NVMe FDP primitives for data placement on Flash media in CacheLib, a popular open-source Flash cache widely deployed and used in Meta's software ecosystem as a caching building block. We demonstrate that targeted data placement in CacheLib using NVMe FDP SSDs helps reduce device write amplification, embodied carbon emissions, and power consumption with almost no overhead to other metrics. Using multiple production traces and their configurations from Meta and Twitter, we show that an ideal device write amplification of \textasciitilde1 can be achieved with FDP, leading to improved SSD utilization and sustainable Flash cache deployments.
\end{abstract}
