%%
%% This is file `sample-sigplan.tex',
%% generated with the docstrip utility.
%%
%% The original source files were:
%%
%% samples.dtx  (with options: `all,proceedings,bibtex,sigplan')
%% 
%% IMPORTANT NOTICE:
%% 
%% For the copyright see the source file.
%% 
%% Any modified versions of this file must be renamed
%% with new filenames distinct from sample-sigplan.tex.
%% 
%% For distribution of the original source see the terms
%% for copying and modification in the file samples.dtx.
%% 
%% This generated file may be distributed as long as the
%% original source files, as listed above, are part of the
%% same distribution. (The sources need not necessarily be
%% in the same archive or directory.)
%%
%%
%% Commands for TeXCount
%TC:macro \cite [option:text,text]
%TC:macro \citep [option:text,text]
%TC:macro \citet [option:text,text]
%TC:envir table 0 1
%TC:envir table* 0 1
%TC:envir tabular [ignore] word
%TC:envir displaymath 0 word
%TC:envir math 0 word
%TC:envir comment 0 0
%%
%%
%% The first command in your LaTeX source must be the \documentclass
%% command.
%%
%% For submission and review of your manuscript please change the
%% command to \documentclass[manuscript, screen, review]{acmart}.
%%
%% When submitting camera ready or to TAPS, please change the command
%% to \documentclass[sigconf]{acmart} or whichever template is required
%% for your publication.
%%
%%
\documentclass[sigplan, 10pt]{acmart}
%\settopmatter{authorsperrow=2}
%\documentclass[sigplan,review,anonymous]{acmart}
\acmSubmissionID{<751>}
%\renewcommand\footnotetextcopyrightpermission[1]{}
% Optional: Remove the ACM reference between the abstract and the main text.
%\settopmatter{printfolios=true,printacmref=false}
% Optional: Comment out the CCS concepts and keywords.
%%
%% \BibTeX command to typeset BibTeX logo in the docs
\AtBeginDocument{%
  \providecommand\BibTeX{{%
    Bib\TeX}}}

%% Rights management information.  This information is sent to you
%% when you complete the rights form.  These commands have SAMPLE
%% values in them; it is your responsibility as an author to replace
%% the commands and values with those provided to you when you
%% complete the rights form.
%\setcopyright{acmlicensed}
%\copyrightyear{2018}
%\acmYear{2018}
%\acmDOI{XXXXXXX.XXXXXXX}

%% These commands are for a PROCEEDINGS abstract or paper.
%\acmConference[Conference acronym 'XX]{Make sure to enter the correct
%  conference title from your rights confirmation email}{June 03--05,
%  2018}{Woodstock, NY}
%%
%%  Uncomment \acmBooktitle if the title of the proceedings is different
%%  from ``Proceedings of ...''!
%%
%\acmBooktitle{Woodstock '18: ACM Symposium on Neural Gaze Detection,
%  June 03--05, 2018, Woodstock, NY}
%  \acmISBN{978-1-4503-XXXX-X/18/06}


%%
%% Submission ID.
%% Use this when submitting an article to a sponsored event. You'll
%% receive a unique submission ID from the organizers
%% of the event, and this ID should be used as the parameter to this command.
%\acmSubmissionID{123-A56-BU3}

%%
%% For managing citations, it is recommended to use bibliography
%% files in BibTeX format.
%%
%% You can then either use BibTeX with the ACM-Reference-Format style,
%% or BibLaTeX with the acmnumeric or acmauthoryear sytles, that include
%% support for advanced citation of software artefact from the
%% biblatex-software package, also separately available on CTAN.
%%
%% Look at the sample-*-biblatex.tex files for templates showcasing
%% the biblatex styles.
%%

%%
%% The majority of ACM publications use numbered citations and
%% references.  The command \citestyle{authoryear} switches to the
%% "author year" style.
%%
%% If you are preparing content for an event
%% sponsored by ACM SIGGRAPH, you must use the "author year" style of
%% citations and references.
%% Uncommenting
%% the next command will enable that style.
%%\citestyle{acmauthoryear}


%%
%% end of the preamble, start of the body of the document source.
\usepackage{graphicx}
\usepackage{subfig}
\usepackage{enumitem}
\usepackage{comment}
%\usepackage{caption, subcaption}
%\usepackage{comment}
%\usepackage{svg}
%\usepackage{epsfig}
%\usepackage{adjustbox}
\usepackage{booktabs, amsmath, siunitx}
%\usepackage{paralist}
\usepackage[nameinlink,capitalize]{cleveref}
%\usepackage{textcomp}
%\usepackage{enumitem}
\usepackage{todonotes}
%\usepackage{textcomp}
\usepackage{makecell}

%\usepackage{microtype}
\newcommand{\minisec}[1]{\noindent\textbf{#1.}}
\newcommand{\minisecques}[1]{\noindent\textbf{#1?}}
%\newcommand{\edit}[1]{{\color{red}#1}}
%\newcommand{\move}[1]{{\color{orange}#1}}
\newcommand{\edit}[1]{#1}
\newcommand{\move}[1]{#1}
\newtheorem{theorem}{Theorem}
%%
%% The "title" command has an optional parameter,
%% allowing the author to define a "short title" to be used in page headers.
\title{Towards Efficient Flash Caches with Emerging NVMe Flexible Data Placement SSDs}

%%
%% The "author" command and its associated commands are used to define
%% the authors and their affiliations.
%% Of note is the shared affiliation of the first two authors, and the
%% "authornote" and "authornotemark" commands
%% used to denote shared contribution to the research.

\begin{comment}
\author{Michael Allison}
\authornote{All author names are listed in alphabetical order}
\authornotemark[0]
%\orcid{1234-5678-9012}

\author{Arun George}
%\authornotemark[1]

\author{Javier Gonzalez}
\authornotemark[1]

\author{Dan Helmick}
\authornotemark[1]

\author{Vikash Kumar}
\authornotemark[1]

\author{Roshan R Nair}
\authornotemark[1]

\author{Vivek Shah}
\authornotemark[1]


% Custom author layout: 3 authors per row without emails
\author{
    \begin{tabular}{ccccc}
        Michael Allison & Arun George & Javier Gonzalez & Dan Helmick & Vikash Kumar \\
        & Roshan Nair & & Vivek Shah & & \\
    \end{tabular}
}
\end{comment}

\author{Michael Allison, Arun George, Javier Gonzalez, Dan Helmick, Vikash Kumar, Roshan R Nair, \\Vivek Shah}
%\authornote{All author names are listed in alphabetical order}

\affiliation{
\institution{Samsung Electronics}
\city{}
\country{}
}

\authornote{All author names are listed in alphabetical order.}

%%
%% By default, the full list of authors will be used in the page
%% headers. Often, this list is too long, and will overlap
%% other information printed in the page headers. This command allows
%% the author to define a more concise list
%% of authors' names for this purpose.
\renewcommand{\shortauthors}{Shah et al.}

%%
%% The code below is generated by the tool at http://dl.acm.org/ccs.cfm.
%% Please copy and paste the code instead of the example below.
%%

\begin{CCSXML}
<ccs2012>
   <concept>
       <concept_id>10002951.10003152.10003153.10003158.10003452</concept_id>
       <concept_desc>Information systems~Flash memory</concept_desc>
       <concept_significance>500</concept_significance>
       </concept>
   <concept>
       <concept_id>10010583.10010588.10010592</concept_id>
       <concept_desc>Hardware~External storage</concept_desc>
       <concept_significance>500</concept_significance>
       </concept>
   <concept>
       <concept_id>10010520.10010521.10010537.10003100</concept_id>
       <concept_desc>Computer systems organization~Cloud computing</concept_desc>
       <concept_significance>500</concept_significance>
       </concept>
 </ccs2012>
\end{CCSXML}

\ccsdesc[500]{Information systems~Flash memory}
\ccsdesc[500]{Hardware~External storage}
\ccsdesc[500]{Computer systems organization~Cloud computing}

\keywords{Flash, SSD, Storage, Caching, FDP, Data placement, Small objects}

\begin{comment}
\ccsdesc[500]{Information systems~Flash memory}
\begin{CCSXML}
<ccs2012>
 <concept>
  <concept_id>00000000.0000000.0000000</concept_id>
  <concept_desc>Do Not Use This Code, Generate the Correct Terms for Your Paper</concept_desc>
  <concept_significance>500</concept_significance>
 </concept>
 <concept>
  <concept_id>00000000.00000000.00000000</concept_id>
  <concept_desc>Do Not Use This Code, Generate the Correct Terms for Your Paper</concept_desc>
  <concept_significance>300</concept_significance>
 </concept>
 <concept>
  <concept_id>00000000.00000000.00000000</concept_id>
  <concept_desc>Do Not Use This Code, Generate the Correct Terms for Your Paper</concept_desc>
  <concept_significance>100</concept_significance>
 </concept>
 <concept>
  <concept_id>00000000.00000000.00000000</concept_id>
  <concept_desc>Do Not Use This Code, Generate the Correct Terms for Your Paper</concept_desc>
  <concept_significance>100</concept_significance>
 </concept>
</ccs2012>
\end{CCSXML}

\ccsdesc[500]{Do Not Use This Code~Generate the Correct Terms for Your Paper}
\ccsdesc[300]{Do Not Use This Code~Generate the Correct Terms for Your Paper}
\ccsdesc{Do Not Use This Code~Generate the Correct Terms for Your Paper}
\ccsdesc[100]{Do Not Use This Code~Generate the Correct Terms for Your Paper}

%%
%% Keywords. The author(s) should pick words that accurately describe
%% the work being presented. Separate the keywords with commas.


%% A "teaser" image appears between the author and affiliation
%% information and the body of the document, and typically spans the
%% page.

    

\begin{teaserfigure}
  \includegraphics[width=\textwidth]{sampleteaser}
  \caption{Seattle Mariners at Spring Training, 2010.}
  \Description{Enjoying the baseball game from the third-base
  seats. Ichiro Suzuki preparing to bat.}
  \label{fig:teaser}
\end{teaserfigure}



\received{20 February 2007}
\received[revised]{12 March 2009}
\received[accepted]{5 June 2009}
\end{comment}
\begin{document}
\acmYear{2025}\copyrightyear{2025}
\setcopyright{rightsretained}
\acmConference[EuroSys '25]{Twentieth European Conference on Computer Systems}{March 30--April 3, 2025}{Rotterdam, Netherlands}
\acmBooktitle{Twentieth European Conference on Computer Systems (EuroSys '25), March 30--April 3, 2025, Rotterdam, Netherlands}
\acmDOI{10.1145/3689031.3696091}
\acmISBN{979-8-4007-1196-1/25/03}
\begin{abstract}  
Test time scaling is currently one of the most active research areas that shows promise after training time scaling has reached its limits.
Deep-thinking (DT) models are a class of recurrent models that can perform easy-to-hard generalization by assigning more compute to harder test samples.
However, due to their inability to determine the complexity of a test sample, DT models have to use a large amount of computation for both easy and hard test samples.
Excessive test time computation is wasteful and can cause the ``overthinking'' problem where more test time computation leads to worse results.
In this paper, we introduce a test time training method for determining the optimal amount of computation needed for each sample during test time.
We also propose Conv-LiGRU, a novel recurrent architecture for efficient and robust visual reasoning. 
Extensive experiments demonstrate that Conv-LiGRU is more stable than DT, effectively mitigates the ``overthinking'' phenomenon, and achieves superior accuracy.
\end{abstract}  

%%
%% This command processes the author and affiliation and title
%% information and builds the first part of the formatted document.
\maketitle

%%
%% The abstract is a short summary of the work to be presented in the
%% article.

\section{Introduction}
\label{sec:introduction}
The business processes of organizations are experiencing ever-increasing complexity due to the large amount of data, high number of users, and high-tech devices involved \cite{martin2021pmopportunitieschallenges, beerepoot2023biggestbpmproblems}. This complexity may cause business processes to deviate from normal control flow due to unforeseen and disruptive anomalies \cite{adams2023proceddsriftdetection}. These control-flow anomalies manifest as unknown, skipped, and wrongly-ordered activities in the traces of event logs monitored from the execution of business processes \cite{ko2023adsystematicreview}. For the sake of clarity, let us consider an illustrative example of such anomalies. Figure \ref{FP_ANOMALIES} shows a so-called event log footprint, which captures the control flow relations of four activities of a hypothetical event log. In particular, this footprint captures the control-flow relations between activities \texttt{a}, \texttt{b}, \texttt{c} and \texttt{d}. These are the causal ($\rightarrow$) relation, concurrent ($\parallel$) relation, and other ($\#$) relations such as exclusivity or non-local dependency \cite{aalst2022pmhandbook}. In addition, on the right are six traces, of which five exhibit skipped, wrongly-ordered and unknown control-flow anomalies. For example, $\langle$\texttt{a b d}$\rangle$ has a skipped activity, which is \texttt{c}. Because of this skipped activity, the control-flow relation \texttt{b}$\,\#\,$\texttt{d} is violated, since \texttt{d} directly follows \texttt{b} in the anomalous trace.
\begin{figure}[!t]
\centering
\includegraphics[width=0.9\columnwidth]{images/FP_ANOMALIES.png}
\caption{An example event log footprint with six traces, of which five exhibit control-flow anomalies.}
\label{FP_ANOMALIES}
\end{figure}

\subsection{Control-flow anomaly detection}
Control-flow anomaly detection techniques aim to characterize the normal control flow from event logs and verify whether these deviations occur in new event logs \cite{ko2023adsystematicreview}. To develop control-flow anomaly detection techniques, \revision{process mining} has seen widespread adoption owing to process discovery and \revision{conformance checking}. On the one hand, process discovery is a set of algorithms that encode control-flow relations as a set of model elements and constraints according to a given modeling formalism \cite{aalst2022pmhandbook}; hereafter, we refer to the Petri net, a widespread modeling formalism. On the other hand, \revision{conformance checking} is an explainable set of algorithms that allows linking any deviations with the reference Petri net and providing the fitness measure, namely a measure of how much the Petri net fits the new event log \cite{aalst2022pmhandbook}. Many control-flow anomaly detection techniques based on \revision{conformance checking} (hereafter, \revision{conformance checking}-based techniques) use the fitness measure to determine whether an event log is anomalous \cite{bezerra2009pmad, bezerra2013adlogspais, myers2018icsadpm, pecchia2020applicationfailuresanalysispm}. 

The scientific literature also includes many \revision{conformance checking}-independent techniques for control-flow anomaly detection that combine specific types of trace encodings with machine/deep learning \cite{ko2023adsystematicreview, tavares2023pmtraceencoding}. Whereas these techniques are very effective, their explainability is challenging due to both the type of trace encoding employed and the machine/deep learning model used \cite{rawal2022trustworthyaiadvances,li2023explainablead}. Hence, in the following, we focus on the shortcomings of \revision{conformance checking}-based techniques to investigate whether it is possible to support the development of competitive control-flow anomaly detection techniques while maintaining the explainable nature of \revision{conformance checking}.
\begin{figure}[!t]
\centering
\includegraphics[width=\columnwidth]{images/HIGH_LEVEL_VIEW.png}
\caption{A high-level view of the proposed framework for combining \revision{process mining}-based feature extraction with dimensionality reduction for control-flow anomaly detection.}
\label{HIGH_LEVEL_VIEW}
\end{figure}

\subsection{Shortcomings of \revision{conformance checking}-based techniques}
Unfortunately, the detection effectiveness of \revision{conformance checking}-based techniques is affected by noisy data and low-quality Petri nets, which may be due to human errors in the modeling process or representational bias of process discovery algorithms \cite{bezerra2013adlogspais, pecchia2020applicationfailuresanalysispm, aalst2016pm}. Specifically, on the one hand, noisy data may introduce infrequent and deceptive control-flow relations that may result in inconsistent fitness measures, whereas, on the other hand, checking event logs against a low-quality Petri net could lead to an unreliable distribution of fitness measures. Nonetheless, such Petri nets can still be used as references to obtain insightful information for \revision{process mining}-based feature extraction, supporting the development of competitive and explainable \revision{conformance checking}-based techniques for control-flow anomaly detection despite the problems above. For example, a few works outline that token-based \revision{conformance checking} can be used for \revision{process mining}-based feature extraction to build tabular data and develop effective \revision{conformance checking}-based techniques for control-flow anomaly detection \cite{singh2022lapmsh, debenedictis2023dtadiiot}. However, to the best of our knowledge, the scientific literature lacks a structured proposal for \revision{process mining}-based feature extraction using the state-of-the-art \revision{conformance checking} variant, namely alignment-based \revision{conformance checking}.

\subsection{Contributions}
We propose a novel \revision{process mining}-based feature extraction approach with alignment-based \revision{conformance checking}. This variant aligns the deviating control flow with a reference Petri net; the resulting alignment can be inspected to extract additional statistics such as the number of times a given activity caused mismatches \cite{aalst2022pmhandbook}. We integrate this approach into a flexible and explainable framework for developing techniques for control-flow anomaly detection. The framework combines \revision{process mining}-based feature extraction and dimensionality reduction to handle high-dimensional feature sets, achieve detection effectiveness, and support explainability. Notably, in addition to our proposed \revision{process mining}-based feature extraction approach, the framework allows employing other approaches, enabling a fair comparison of multiple \revision{conformance checking}-based and \revision{conformance checking}-independent techniques for control-flow anomaly detection. Figure \ref{HIGH_LEVEL_VIEW} shows a high-level view of the framework. Business processes are monitored, and event logs obtained from the database of information systems. Subsequently, \revision{process mining}-based feature extraction is applied to these event logs and tabular data input to dimensionality reduction to identify control-flow anomalies. We apply several \revision{conformance checking}-based and \revision{conformance checking}-independent framework techniques to publicly available datasets, simulated data of a case study from railways, and real-world data of a case study from healthcare. We show that the framework techniques implementing our approach outperform the baseline \revision{conformance checking}-based techniques while maintaining the explainable nature of \revision{conformance checking}.

In summary, the contributions of this paper are as follows.
\begin{itemize}
    \item{
        A novel \revision{process mining}-based feature extraction approach to support the development of competitive and explainable \revision{conformance checking}-based techniques for control-flow anomaly detection.
    }
    \item{
        A flexible and explainable framework for developing techniques for control-flow anomaly detection using \revision{process mining}-based feature extraction and dimensionality reduction.
    }
    \item{
        Application to synthetic and real-world datasets of several \revision{conformance checking}-based and \revision{conformance checking}-independent framework techniques, evaluating their detection effectiveness and explainability.
    }
\end{itemize}

The rest of the paper is organized as follows.
\begin{itemize}
    \item Section \ref{sec:related_work} reviews the existing techniques for control-flow anomaly detection, categorizing them into \revision{conformance checking}-based and \revision{conformance checking}-independent techniques.
    \item Section \ref{sec:abccfe} provides the preliminaries of \revision{process mining} to establish the notation used throughout the paper, and delves into the details of the proposed \revision{process mining}-based feature extraction approach with alignment-based \revision{conformance checking}.
    \item Section \ref{sec:framework} describes the framework for developing \revision{conformance checking}-based and \revision{conformance checking}-independent techniques for control-flow anomaly detection that combine \revision{process mining}-based feature extraction and dimensionality reduction.
    \item Section \ref{sec:evaluation} presents the experiments conducted with multiple framework and baseline techniques using data from publicly available datasets and case studies.
    \item Section \ref{sec:conclusions} draws the conclusions and presents future work.
\end{itemize}
\section{Background}\label{sec:backgrnd}

\subsection{Cold Start Latency and Mitigation Techniques}

Traditional FaaS platforms mitigate cold starts through snapshotting, lightweight virtualization, and warm-state management. Snapshot-based methods like \textbf{REAP} and \textbf{Catalyzer} reduce initialization time by preloading or restoring container states but require significant memory and I/O resources, limiting scalability~\cite{dong_catalyzer_2020, ustiugov_benchmarking_2021}. Lightweight virtualization solutions, such as \textbf{Firecracker} microVMs, achieve fast startup times with strong isolation but depend on robust infrastructure, making them less adaptable to fluctuating workloads~\cite{agache_firecracker_2020}. Warm-state management techniques like \textbf{Faa\$T}~\cite{romero_faa_2021} and \textbf{Kraken}~\cite{vivek_kraken_2021} keep frequently invoked containers ready, balancing readiness and cost efficiency under predictable workloads but incurring overhead when demand is erratic~\cite{romero_faa_2021, vivek_kraken_2021}. While these methods perform well in resource-rich cloud environments, their resource intensity challenges applicability in edge settings.

\subsubsection{Edge FaaS Perspective}

In edge environments, cold start mitigation emphasizes lightweight designs, resource sharing, and hybrid task distribution. Lightweight execution environments like unikernels~\cite{edward_sock_2018} and \textbf{Firecracker}~\cite{agache_firecracker_2020}, as used by \textbf{TinyFaaS}~\cite{pfandzelter_tinyfaas_2020}, minimize resource usage and initialization delays but require careful orchestration to avoid resource contention. Function co-location, demonstrated by \textbf{Photons}~\cite{v_dukic_photons_2020}, reduces redundant initializations by sharing runtime resources among related functions, though this complicates isolation in multi-tenant setups~\cite{v_dukic_photons_2020}. Hybrid offloading frameworks like \textbf{GeoFaaS}~\cite{malekabbasi_geofaas_2024} balance edge-cloud workloads by offloading latency-tolerant tasks to the cloud and reserving edge resources for real-time operations, requiring reliable connectivity and efficient task management. These edge-specific strategies address cold starts effectively but introduce challenges in scalability and orchestration.

\subsection{Predictive Scaling and Caching Techniques}

Efficient resource allocation is vital for maintaining low latency and high availability in serverless platforms. Predictive scaling and caching techniques dynamically provision resources and reduce cold start latency by leveraging workload prediction and state retention.
Traditional FaaS platforms use predictive scaling and caching to optimize resources, employing techniques (OFC, FaasCache) to reduce cold starts. However, these methods rely on centralized orchestration and workload predictability, limiting their effectiveness in dynamic, resource-constrained edge environments.



\subsubsection{Edge FaaS Perspective}

Edge FaaS platforms adapt predictive scaling and caching techniques to constrain resources and heterogeneous environments. \textbf{EDGE-Cache}~\cite{kim_delay-aware_2022} uses traffic profiling to selectively retain high-priority functions, reducing memory overhead while maintaining readiness for frequent requests. Hybrid frameworks like \textbf{GeoFaaS}~\cite{malekabbasi_geofaas_2024} implement distributed caching to balance resources between edge and cloud nodes, enabling low-latency processing for critical tasks while offloading less critical workloads. Machine learning methods, such as clustering-based workload predictors~\cite{gao_machine_2020} and GRU-based models~\cite{guo_applying_2018}, enhance resource provisioning in edge systems by efficiently forecasting workload spikes. These innovations effectively address cold start challenges in edge environments, though their dependency on accurate predictions and robust orchestration poses scalability challenges.

\subsection{Decentralized Orchestration, Function Placement, and Scheduling}

Efficient orchestration in serverless platforms involves workload distribution, resource optimization, and performance assurance. While traditional FaaS platforms rely on centralized control, edge environments require decentralized and adaptive strategies to address unique challenges such as resource constraints and heterogeneous hardware.



\subsubsection{Edge FaaS Perspective}

Edge FaaS platforms adopt decentralized and adaptive orchestration frameworks to meet the demands of resource-constrained environments. Systems like \textbf{Wukong} distribute scheduling across edge nodes, enhancing data locality and scalability while reducing network latency. Lightweight frameworks such as \textbf{OpenWhisk Lite}~\cite{kravchenko_kpavelopenwhisk-light_2024} optimize resource allocation by decentralizing scheduling policies, minimizing cold starts and latency in edge setups~\cite{benjamin_wukong_2020}. Hybrid solutions like \textbf{OpenFaaS}~\cite{noauthor_openfaasfaas_2024} and \textbf{EdgeMatrix}~\cite{shen_edgematrix_2023} combine edge-cloud orchestration to balance resource utilization, retaining latency-sensitive functions at the edge while offloading non-critical workloads to the cloud. While these approaches improve flexibility, they face challenges in maintaining coordination and ensuring consistent performance across distributed nodes.


\section{NVMe Flexible Data Placement (FDP)}
\label{sec:fdp}
\subsection{Overview}
The ratified NVMe Flexible Data Placement technical proposal~\cite{fdp_tp} represents an evolution in the space of SSD data placement based on lessons learned in the wild over the past decade. It is a merger of Google's SmartFTL~\cite{smartftl} and Meta's Direct Placement Mode proposals to enable data placement on Flash media without the high software engineering costs of explicit garbage collection of ZNS~\cite{bjorlingAHRMGA21} and low-level media control of Open-Channel SSD proposals~\cite{bjorling2017lightnvm}. It borrows elements from the multi-streamed SSD interface~\cite{kang2014multi} that was proposed a decade ago but did not really take off due to a lack of industry and academic interest. It has been designed with backward compatibility in mind so that applications can work unchanged with it. The choice of leveraging data placement and evaluating its costs and benefits has been left to the application. This enables investment of engineering effort in a pay-as-you-go fashion instead of an upfront cost.

\subsection{Physical Isolation in SSDs with FDP}
\subsubsection{FDP Architectural Concepts}
The Flexible Data Placement interface provides abstractions to the host to group data on the device with a similar expected lifetime (e.g., death time). The interface introduces the following concepts to expose the SSD physical architecture (see Figure \ref{fig:fdp-arch}).

\begin{figure}[!t]
  \centering
  \includegraphics[width=1.3 \linewidth]{svg_to_pdf/cns-fdpArch2.pdf}
  %\vspace{-3ex}
  \caption{Conventional SSD vs FDP SSD Architecture.} 
  \vspace{-2ex}
  \label{fig:fdp-arch}
\end{figure}

\minisec{Reclaim Unit (RU)} The NAND media is organized into a set of reclaim units where a reclaim unit consists of a set of blocks that can be written. A reclaim unit will typically consist of one or more erase blocks but no guarantees are made in the proposal towards this end. The size of an RU is decided by the SSD manufacturer. In this paper, our device has superblock-sized RUs where a superblock is a collection of erase blocks across the planes of dies in the SSD. If an SSD has 8 dies each with 2 planes and 2 erase blocks per plane, the superblock will consist of 32 erase blocks.

\minisec{Reclaim Group (RG)} A reclaim group is a set of reclaim units.

\minisec{Reclaim Unit Handles (RUH)} A reclaim unit handle is an abstraction in the device controller similar to a pointer that allows host software to point to the reclaim units in the device. Since a reclaim unit is not directly addressable by the host, the host software uses the reclaim unit handles to logically isolate data. The device manages the mapping of reclaim unit handles to a reclaim unit and has complete control over this mapping. The number of RUHs in the device determines the number of different logical locations in the NAND where the host software can concurrently place data.

\minisec{RUH Types} \move{
The FDP interface specifies two types of reclaim unit handles, each offering distinct data movement guarantees during garbage collection, along with their respective tradeoffs. During garbage collection, the RUH type is used to determine the source and destination RUs of data to be moved. FDP defines two RUH types namely,
\begin{enumerate}[leftmargin=*]
    \item \textbf{Initially Isolated} - All the reclaim units within a reclaim group pointed to by the RUHs of this type are candidates for data movement. For multiple RUHs of initially isolated type, data starts off being isolated from data written using another RUH of initially isolated type. However, upon garbage collection valid data written using these two handles can be intermixed. This type is the cheapest to implement on the SSD controller as it does not require explicit tracking of data written using RUHs and imposes the least constraints on data movement during garbage collection. 
    \item \textbf{Persistently Isolated} - All the reclaim units within a reclaim group that have been written utilizing the RUH are the only candidates for data movement upon garbage collection. This RUH type provides a stronger guarantee of data isolation but is expensive to implement on the controller as it requires explicit tracking of data written using RUHs and imposes more constraints on data movement during garbage collection.
\end{enumerate}
\minisec{Example} Consider a write pattern using two RUHs, RUH0 and RUH1 where RUH0 has written LBAs to RU0 and RU1 while RUH1 has written LBAs to RU2. For simplicity, let us assume that all the RUs belong to the same reclaim group. If RUH0 and RUH1 are of initially isolated type, then upon garbage collection valid data from RU0, RU1 and RU2 are candidates for movement and can be intermixed. If RUH0 and RUH1 are of persistently isolated type, then only data from RU0 and RU1 can be intermixed upon garbage collection while the data in RU2 is isolated from data in RU0 and RU1.
}
\begin{comment}
The FDP interface defines two types of reclaim unit handles that provide different data movement guarantees upon garbage collection with their associated tradeoffs. They are:\\
1. \textbf{Initially Isolated} - SSD is allowed to move the data written with an RUH to any Reclaim Units in the Reclaim Group. This means, upon garbage collection valid data written using different RUHs can be intermixed.\\
2. \textbf{Persistently Isolated} - SSD is only allowed to move the data written with an RUH to any Reclaim Units of the same RUH in the Reclaim Group. We cover more details and examples in Appendix~\ref{sec:appendix:fdp:details}. 
\end{comment}

\minisec{FDP Configurations}  An FDP configuration defines the RUHs, RUH type (Initially or Persistently Isolated), their association to RGs, and the RU size. \textit{The FDP configurations available on the device are predetermined by the manufacturer} and cannot be changed. This paper uses an SSD with a single FDP configuration of 8 Initially Isolated RUHs, 1RG and RU size of 6GB. A device can support multiple configurations that can be chosen by the host.

\begin{comment}
\minisec{Placement Identifier (PID)} 

\subsubsection{FDP Configurations} An FDP configuration defines the RUHs, their association to RGs, and the RU size. An SSD conforming to the FDP specification can support one or many FDP configurations that are available as a log page for selection by the host. At namespace creation, one or many RUHs can be associated to a namespace. \textit{The FDP configurations available on the device are pre-decided by the manufacturer}. If an FDP configuration is not enabled on the FDP SSD, it behaves as a conventional SSD.
\end{comment}
%\vspace{-3ex}

\subsubsection{Data Placement with RUHs}
In this section, we highlight important aspects of the FDP interface that influence data placement designs by the host.\\
\minisec{Physically Isolating Logical Blocks with RUHs}
The FDP storage interface does not introduce any new command sets to write to the device. Instead, a new data placement directive has been defined that allows each write command to specify a RUH. Thus, the host software can use the RUH to place a logical block in a RU utilizing the RUH. By allowing the host to dynamically associate a logical block with a RU, FDP enables flexible grouping of data based on varying temperature and death time (e.g., hot and cold data separation) or different data streams (e.g., large streams and small journals). This facilitates writing to different RUs in a physically isolated manner. By careful deallocation of all the data in a previously written RU, the host can achieve a DLWA of \textasciitilde1.

During namespace creation, the host software selects a list of RUHs that are accessible by the created namespace. Since FDP is backward compatible, a default RUH is chosen by the device for a namespace if it is not specified. Data is placed in this RUH in the absence of the placement directive from the host. Read operations remain unchanged as before. Writes in FDP can cross RU boundaries. If a write operation overfills an RU because the RU is written to its capacity, the device chooses a new RU and updates the mapping of the RUH to the new RU. Although this process is not visible to the host, the event is logged by the SSD in the device logs that the host can examine. 

\minisec{Managing Invalidations and Tracking RUs}
Since FDP does not focus on garbage collection but purely data placement, it does not introduce any new abstractions for erase operations. As in conventional SSDs, LBAs are invalidated or dealloacted in two ways, (1) by overwriting an LBA, (2) by explicitly using a trim operation over one or many LBAs. If all the data in a RU is invalidated, then the RU is erased for future writes and no logical blocks have to copied across RUs upon garbage collection. Since the host software can only access RUHs and not an RU, in order to perform fine-grained and targeted deallocation of RUs, the host software needs to track the LBAs that have been written to an RU together and deallocate those. The FDP specification also allows the host to query the available space in an RU which is currently referenced by the RUH.

\subsection{FDP Events and Statistics} FDP provides an elaborate set of events and garbage collection statistics for the host to track the FDP related events in the SSD. These help the host to be aware of device-level exceptions and make sure that both host and device are in sync regarding data placement.

\begin{comment}
\minisec{Reclaim Group (RG)} As shown in Figure \ref{fig:FDPARCH}, the FDP architecture allows the SSD to define a set of Reclaim Groups (RGs). Reclaim Groups enable a physical division of the NAND space. Reclaim Groups also permit the SSD to violate placement when warranty, space, or other unplanned in-drive activity demands it.  If this emergency Reclaim Group violation ever occurs within the SSD, the host software is able to discover it through querying the drive's logs.  \par
\end{comment}

\subsection{FDP and Other Major Data Placement Proposals}
\edit{
NVMe FDP technical proposal was conceived based on lessons learnt from integrating software stacks with the past data placement proposals. It has been designed to focus on data placement to allow host software stack to perform data segregation while leaving NAND media management and garbage collection to the SSD controller. In Table~\ref{tab:fdp-data-placement-tech}, we outline some of the key differences between the major data placement proposals of the past years. More details can be found in some of the recent industry presentations on FDP~\cite{sdc-fdp-dan, sdc-fdp-mike}.
}

\begin{table*}[!ht]
    \centering
    \edit{
    \begin{tabular}{|p{0.15\textwidth}|p{0.18\textwidth}|p{0.18\textwidth}|p{0.15\textwidth}|p{0.18\textwidth}|}
        \hline
    Characteristic & \textbf {Streams~\cite{kang2014multi}} & \textbf {Open-Channel~\cite{bjorling2017lightnvm}} & \textbf{ZNS~\cite{zns, bjorlingAHRMGA21}} & \textbf{FDP~\cite{fdp_tp}} \\ \hline
    Supported write patterns & Random, Sequential & Random, Sequential & Sequential & Random, Sequential \\ \hline
    Data placement primitive & Using stream identifiers & Using logical to physical address mapping by host & Using zones & Using reclaim unit handles \\ \hline
    Control of garbage collection & SSD-based without feedback to host & Host-based & Host-based & SSD-based with feedback through logs \\ \hline
    NAND media management by host & No & Yes & No & No \\ \hline
    Can run applications unchanged & Yes & No & No & Yes \\ \bottomrule
    \end{tabular}
    \caption{High-Level Comparison of Major Data Placement Proposals.}
    \label{tab:fdp-data-placement-tech}
    }
    \vspace{-2ex}
\end{table*}

\edit{
\subsection{Limitations}
\begin{enumerate}[leftmargin=*]
\item {\textbf{New and evolving technology.}} The FDP technical proposal was ratified at the end of 2022, and some devices from Samsung, such as the PM9D3a~\cite{pm9d3a} are emerging on the market with support for it, along with offerings from other vendors. Due to the relatively recent ratification, the proposal may undergo modifications over time to include extensions for desirable features.
\item {\textbf{Lack of host control over garbage collection.}} FDP was designed specifically for data placement while allowing hosts to perform random writes to LBAs, enabling the SSD to manage garbage collection. Consequently, the host has no control over the garbage collection process in the SSD, aside from invalidating LBAs by deallocating or overwriting them. Note that this limitation only applies in scenarios where the host can achieve greater performance gains by managing garbage collection more efficiently than the SSD, rather than focusing solely on data placement.
\item {\textbf{Requires device overprovisioning and mapping table in SSD.}} As in conventional SSDs today, FDP SSDs will also require a mapping table in DRAM to support transparent mapping of logical to physical addresses. Moreover, NAND overprovisioning in the device is required for acceptable performance in the absence of host-based garbage collection. This is a limitation when the proposal is viewed from the lenses of the cost of fabrication of FDP SSDs.
\end{enumerate}
}


\section{Why FDP Matters for CacheLib and Hybrid Caches?}
\label{sec:observations}
In this section, we discuss the fit of FDP and the opportunities afforded by it for CacheLib and hybrid caches based on the analysis of CacheLib's Flash Cache architecture, web service caching deployments, and workloads.

\subsection{Insights and Observations}
\begin{figure}[!t]
  \centering
  \vspace{-3ex}
  \subfloat[a][SSD cross-section without FDP]{\includegraphics[width=0.9\linewidth]{svg_to_pdf/nand_crossection_non_fdp_drawio.pdf} \label{fig:non-fdp-cross}} \hfill \\
  \vspace{-2ex}  
  \subfloat[b][SSD cross-section without FDP]{\includegraphics[width=0.9\linewidth]{svg_to_pdf/nand_crossection_fdp_drawio.pdf} \label{fig:fdp-cross}} \hfill \\   
  \caption{SSD cross-section. {\Large \textcircled{\small 1a}} shows the intermixing of LOC’s sequential and cold data with SOC’s random and hot data in SSD blocks. {\Large \textcircled{\small 1b}} shows the inefficient use of device OP by both LOC and SOC data. {\Large \textcircled{\small 2a}} shows that with SOC data being segregated, invalidation of its data can result in free SSD blocks.  {\Large \textcircled{\small 2b}} shows that with FDP, LOC data which is written sequentially will not cause DLWA. {\Large \textcircled{\small 2c}} shows the efficient use of device OP exclusively by SOC data to cushion SOC DLWA.} 
  \label{fig:ssd-cross}
  \vspace{-2ex}
\end{figure}

\minisec{\textit{Insight 1: Intermixing of SOC and LOC data leads to high DLWA}}
Large cache items are written into the LOC in a log-structured fashion utilizing a FIFO or LRU eviction policy. This results in a sequential write pattern to the SSD. Small cache items are written into SOC buckets using a uniform hash function. Each item insert causes the entire SOC bucket (size is configurable but default is 4 KB) to be written to the SSD. Contrary to LOC, SOC writes generate a random write pattern to the SSD. 

For workloads with large working set sizes and key churn, the Flash cache layer receives writes due to evictions from the RAM cache~\cite{berg2020cachelib,mcallister2021kangaroo}. For workloads dominant in small object accesses, this segregation leads to an infrequent and cold data access pattern in the LOC together with a frequent and hot data access pattern in the SOC. This leads to the intermixing of LOC's sequential and cold data with SOC's random and hot data in a single SSD block (Figure \ref{fig:non-fdp-cross} {\Large \textcircled{\small 1a}}) causing high DLWA upon garbage collection. \\

\minisec{\textit{Insight 2: The use of host overprovisioning as a control measure for DLWA is inefficient}}
As explained in Section \ref{sec:background:flash-cache-cachelib}, \textit{CacheLib deployments utilize a host overprovisioning of almost 50\% of the SSD to limit DLWA to an acceptable value of $\sim$1.3}. This is inefficient from both cost and carbon efficiency perspectives. The LOC data due to its sequential and cold access pattern does not need any host or device overprovisioning for a DLWA of 1. Without host overprovisioning the only extra space available to help control DLWA is the device overprovisioning space. The random SOC data would benefit the most from the device overprovisioned space because it is small, hot and updated frequently. However, the intermixing of SOC and LOC results in an inefficient use of the device overprovisioning space (Figure \ref{fig:non-fdp-cross} {\Large \textcircled{\small 1b}}) as both the SOC and LOC data share it causing unnecessary data movement. \\

\minisec{\textit{Insight 3: High SOC invalidation and its small size can be harnessed to control DLWA}}
A smaller SOC size on devices leads to fewer buckets and a higher rate of key collisions. Since the entire SOC bucket of 4KB is written out, a larger SOC bucket invalidation rate is SSD-friendly because it leads to more SSD page invalidation. If only invalidated SOC data resided in an SSD erase block, this would result in the entire erase block freeing itself up and not needing movement of valid data. For workloads dominant in small object accesses, a high invalidation of SOC happens but the SOC data in erase blocks is intermixed with LOC data. This prevents the SSD from taking advantage of the updates occurring in the SOC buckets over a small LBA space. \\

\begin{comment}
More collisions would mean a faster invalidation rate for the same 4KB bucket. As the SOC size increases the collisions per bucket would reduce due to a large number of SOC buckets.  \\
A larger SOC bucket invalidation rate is SSD-friendly because this leads to more SSD page invalidation within the SSD block. If only the SOC data were to reside in an SSD block, this would result in SSD blocks freeing themselves up and not needing live data migrations. However, the LOC data residing in the same SSD block prevents this behaviour from occurring. \\
\end{comment}

\minisec{\textit{Insight 4: Data placement using FDP can help CacheLib control DLWA}}
FDP can be utilized by CacheLib to separate the SOC and LOC data in the SSD using different reclaim unit handles. This allows the LOC data and SOC data to reside in mutually exclusive SSD blocks (reclaim units). Such a design will have the following benefits,
\begin{enumerate}[noitemsep, topsep=1pt, partopsep=0pt,leftmargin=*]
    \item The SSD blocks containing LOC data get overwritten sequentially resulting in minimal data movement and DLWA (Figure \ref{fig:fdp-cross} {\Large \textcircled{\small 2b}}). If LOC data resides in separate reclaim units than SOC data, the device overprovisioning space can be used exclusively by SOC data.
    \item The ideal behaviour of SOC data invalidating only itself (Insight 3) can be realized by segregating it into separate reclaim units (Figure \ref{fig:fdp-cross} {\Large \textcircled{\small 2a}}). A smaller SOC size leads to a greater invalidation rate causing most of the SOC data in the SSD erase block being invalid. This leads to minimal live data movement and DLWA. As the SOC size increases we expect an increase in DLWA even with LOC and SOC segregation across reclaim units.
    \item The ideal utilization of device overprovisioning space (Insight 2) is possible with FDP (Figure \ref{fig:fdp-cross} {\Large \textcircled{\small 2c}}). SOC data can use the overprovisioned space to cushion DLWA. When the SOC size is smaller than the device overprovisioning space we expect a DLWA of $\sim$1 since there is at least one spare block available for each block of SOC data.
    \item The separation of LOC and SOC data in the SSD does not necessitate a change in the CacheLib architecture and API. Therefore, we expect no change in the application-level write amplification (ALWA).
\end{enumerate}
\vspace{1ex}
\begin{comment}
In conclusion, with FDP, the high DLWA from SOC's random writes can be controlled efficiently due to the inherent invalidation rate of SOC data and device overprovisioning space being exclusively available to cushion data migrations. \\
\end{comment}

\minisec{\textit{Insight 5: Initially Isolated FDP devices will suffice in controlling the DLWA in CacheLib}}
With the separation of LOC and SOC data within the SSD, the only live data movement will be due to SOC data. Irrespective of whether the SSD is initially isolated or persistently isolated only SOC data would reside in reclaim units used for garbage collection. Therefore, the isolation of LOC and SOC data would be preserved regardless. \\

\minisec{\textit{Insight 6: Data placement using FDP can help reduce carbon emissions in CacheLib}}
Embodied carbon emissions account for the major chunk of carbon emissions compared to operational carbon emissions. The DLWA gains from using FDP-enabled CacheLib leads to an improved device lifetime. This results in fewer device replacements during the system lifecycle leading to reduction in embodied carbon emissions.

Fewer garbage collection operations are the reason for the DLWA gains with FDP-enabled CacheLib. For a fixed number of host operations, fewer data migrations result in fewer total device operations. The reduction in total operations requires the device to spend fewer cycles in the active state leading to a lower SSD energy consumption and reduced operational carbon footprint~\cite{ssdenergy, ssdpower}.

\begin{comment}
\begin{itemize}[noitemsep, topsep=1pt, partopsep=0pt,leftmargin=*]
    \item The DLWA gains from using FDP-enabled CacheLib leads to an improved device lifetime. This results in fewer device replacements during the system lifecycle leading to reduction in embodied carbon emissions.
    \item Fewer garbage collection operations are the reason for the DLWA gains with FDP-enabled CacheLib. For a fixed number of host operations fewer data migrations equates to fewer total device operations. The reduction in total operations requires the device to spend fewer cycles in the active state leading to a lower SSD energy consumption~\cite{ssdenergy, ssdpower} and reduced operational carbon footprint.
\end{itemize}
\end{comment}

\subsection{Theoretical Analysis of FDP-enabled CacheLib DLWA and Carbon Emissions}
\label{theorem:DLWA}
We formulate a theoretical model of DLWA and carbon emissions for SOC and LOC data segregation in CacheLib using the insights of the previous section. We assume the DLWA of LOC data is $\sim$1. Additionally, we use the fact that only SOC data will use the device overprovisioning space and item insertions to the SOC buckets follow a uniform hash function. To simplify our analysis, we assume that the uniform hash function used in CacheLib is fairly well-behaved. Modelling DLWA by estimating live data movement for a uniform random workload has been used proposed before~\cite{DayanBB15}. We extend that methodology to model the SOC DLWA that translates to the DLWA for FDP-enabled CacheLib as the LOC does not contribute to DLWA. The derivation of the theorems in this section is available in Appendix \ref{ref:appendix:theoretical-model}.
%\vspace{-2ex}
\begin{theorem}
    The DLWA for FDP-enabled CacheLib using SOC and LOC data segregation is,
    \vspace{-0.75em}
    \begin{equation*}
        \text{DLWA} = \frac{1}{1- \delta}
    \end{equation*}
    \vspace{3ex}
    
    \noindent where $\delta$ denotes the average live SOC bucket migration due to garbage collection and is given by,
    \begin{equation*}
        \delta = - \frac{\text{S}_{\text{SOC}}}{\text{S}_{\text{P-SOC}}} \times \mathcal{W} (- \frac{\text{S}_{\text{P-SOC}}}{\text{S}_{\text{SOC}}} \times e^{- \frac{\text{S}_{\text{P-SOC}}}{\text{S}_{\text{SOC}}}} )
    \end{equation*}
    where $\text{S}_{\text{SOC}}$ is the total SOC size in bytes,  $\text{S}_{\text{P-SOC}}$ is the total physical space available for SOC data including device overprovisioning in bytes and $\mathcal{W}$ denotes the Lambert W function.
\end{theorem}


\subsubsection{Modelling CO2 emissions (CO2e) for FDP-enabled CacheLib}
The total carbon footprint is the sum of embodied and operational carbon emissions.
    \begin{equation*}
        \text{Total}_{\text{CO2e}} = \text{C}_{\text{embodied}} + \text{C}_{\text{operational}}
    \end{equation*}
\label{theorem:embodied-co2e}
\begin{theorem}
    The embodied carbon emissions from using CacheLib by accounting for SSD replacement during the system lifecycle of T years and rated SSD warranty of $\text{L}_{\text{dev}}$ years is,

    \begin{equation*}
        \text{C}_{\text{embodied}} = \text{DLWA} \times \text{Device}_{\text{cap}} \times \frac{T}{\text{L}_{\text{dev}}} \times \text{C}_{\text{SSD}}
    \end{equation*}
    where $\text{Device}_{\text{cap}}$ is the physical capacity of the device. \\
    $\text{Host}_{\text{cap}} = \text{Device}_{\text{cap}} \times (1 - \text{Total}_{\text{op}})$ denotes the SSD capacity used by the host system in GB, $\text{H}_{\text{op}} \text{ and } \text{D}_{\text{op}} \in [0,1) $ is the fraction of host overprovisioning and device overprovisioning and $\text{C}_{\text{SSD}}$ is the amount of CO2e (Kg) per GB of SSD manufactured.
    
\end{theorem}

Operational Energy can be converted to CO2 emission (CO2e) using the greenhouse equivalence calculator ~\cite{greenhouse_calc}. The operational energy consumed can be modelled by estimating the time spent in idle and active states ~\cite{ssdenergy}. The time spent in active states is proportional to the total number of device operations during the period in question,

\begin{theorem}
Operational energy is proportional to the total number of garbage collection events.
\begin{equation*} 
    \mathcal{E}_{\text{operational}} \propto \mathcal{E}(\text{Host}_{\text{operations}}) + \mathcal{E}(\text{Device}_{\text{migrations}})
    \end{equation*}
    where, $\text{Device}_{\text{migrations}}$ is the number of garbage collection operations triggered in the SSD. 
\end{theorem}


%PlanGREEN

%GEN-Plan

%G- generate
%R-refine
%E- edit

%% GREEN-plan

%% PURPLE
\begin{figure*}
    \centering
    \Description{PLAID's system architecture diagram. Top part shows the database (a), and bottom part shows the interface (b). The system starts from bottom right as an instructor is interested in a programming domain, then the pipeline described in the text produces reference materials at different levels of granularity, and these are presented in the interface.}
    \includegraphics[width=\textwidth]{img/system-architecture-subgoals.png}
    \caption{PLAID's reference content is generated through an LLM pipeline
    %inspired by the practices of instructors who have successfully identified programming plans. 
    that produces output on three levels.
    First, a wide variety of use cases are generated to create example programs that focus on code's applications. Next, using LLM's explanatory comments that represent subgoals within the code, the examples are segmented into meaningful code snippets. The LLM is queried to generate other plan components for each code snippet. Finally, the code snippets are clustered to identify the most common patterns, representing plan candidates. The full programs are presented in `Programs' views of PLAID interface, whereas snippets are presented in clusters in the `Plan Creation' view.}
    \label{fig:system-pipeline}
\end{figure*}
\section{PLAID: A System for Supporting Plan Identification}
\label{sec:system-design}

Following the design goals devised from the design workshop, we refined our early prototype into PLAID: a
%LLM-powered
tool to assist instructors in their plan identification process.
PLAID synthesizes the capabilities of LLMs in code generation with interactions enabling plan identification practices observed in our studies with instructors.
As we noted in the findings of our design workshop, the LLM-generated candidate plans are not ready to be used as is in instruction, but instructors can readily adapt and correct them (\cref{sec:workshop-findings-condition2}).
PLAID enables collaboration between instructors and LLMs, enhancing the plan identification process by automating its time-intensive information-gathering tasks and facilitating instructors' ability to refine LLM-generated candidate plans based on their knowledge about pedagogy and the programming domain. 



\subsection{Practical Illustration}

To understand how instructors use can PLAID to more easily adopt plan-based pedagogies, we follow Jane, a computer science instructor using PLAID to design programming plans for her course (summarized in \cref{fig:jane-workflow}).

Jane is teaching a programming course for psychology majors and wants to introduce her students to data analysis using Pandas. As her students have limited prior programming experience and use programming for specific goals, she organizes her lecture material around programming plans to emphasize purpose over syntax. 
% that explain practical concepts to students and help them focus on the purpose behind the code they write.
% However, she realizes that all introductory computer science courses offered at her institution only teach basic programming constructs like data structures. After exploring Google Scholar for effective instructional methods to teach application-focused programming to non-computer science majors, she learned about plan-based pedagogies that help them focus on the purpose behind the code they write. In her literature review, she finds out about PLAID, a tool that can help her design domain-specific plans. She reviews the domains supported by the tool (Pandas, Pytorch, Beautifulsoup, and Django) and decides to use Pandas, a popular and powerful data analysis and manipulation library, to create her curriculum. 

She logs in to the PLAID web interface, % and takes time to explore the system's features. 
and asks PLAID to suggest a plan (\cref{fig:jane-workflow}, 1). The first plan recommended to her 
% she sees is a plan to help students learn about
is about reading CSV files. 
She thinks the topic is important and the solution code aligns with her experience; % the solution is promising and represents an important concept that students need to know about.
% She is satisfied with the given solution 
but she finds the generated name and goal to be too generic. She edits (\cref{fig:jane-workflow}, 2) these fields to provide more context that she feels is right for her students.
% She refines those fields and then 
To make this plan more abstract and appropriate for more use cases, %explain how this plan can be used for reading data from different file formats,
she marks the file path as a changeable area (\cref{fig:jane-workflow}, 3), generalizing the plan for reading data from different file formats.

Inspired by the first plan, she decides to create a plan for saving data to disk. She wants to teach the most conventional way of saving data, so she switches to the use case tab (\cref{fig:jane-workflow}, 4) and explores example programs that save data to get a sense of common practices.  %interact with the list of complete programs.
She finds a complete example where a DataFrame is created and and saved to a file. %performs cleaning tasks like deleting NaN values, and exports it.
% She realizes that something she hadn't thought of before: saving new data is almost always necessary after performing data manipulation operations!
She selects the part of the code that exports data to a file and creates a plan from that selection (\cref{fig:jane-workflow}, 5).


For the next plan, she reflects on her own experience with Pandas. She recalls that merging DataFrames was a key concept, but cannot remember the full syntax. 
% Jane reflects on her experience working with Pandas and recalls that merging DataFrames is a key operation when working with big data.
She switches to the full programs tab (\cref{fig:jane-workflow}, 6) that includes complete code examples and searches (\cref{fig:jane-workflow}, 7) for ``\texttt{.merge}'' to find different instances of merging operations. % and tries to use the search bar to find a relevant program that contains ``.merge''. 
After finding a comprehensive example, she selects the relevant section of the code and creates a plan from it.
% She again selects a part of the example, creates plan from the selection, and refines it. She engages with the system iteratively and designs twenty plans for her lecture. 

After designing a set of plans that capture the important topics, she organizes them into groups (\cref{fig:jane-workflow}, 8) 
% also grouped similar plans together
to emphasize sets of plans with similar goals but different implementations. For instance, she takes her plans about \texttt{.merge} and \texttt{.concat} and groups them together to form a category of plans that students can reference when they want to {combine data from different sources}.

% combining data using ``merge'' or ``concat''.

% the the she used plans isn't very good right now
% She exports these plans and starts preparing her lecture slides, using the plans as a way of presenting key concepts to students with minimal programming experience.
She exports these plans to support her students with minimal programming experience by preparing lecture slides that organize the sections around plan goals, using plan solutions as worked examples in class, and providing students with cheat sheets that include relevant plans for their laboratory sessions.
% using the plan goals as titles for different sections of her slides, and using the solutions as references for the examples she creates. Finally, she makes a PDF cheatsheet with all the plans for students to reference during the week's laboratory.
% The next day, she starts preparing her lecture slides and realizes that the names and goals she wrote for her plans represent key concepts in Pandas. She references the plans she created to design annotated examples that she includes on her lecture slides.

%% How does Jane actually use the plans? 
%% > Important to be careful to note that this isn't actually part of the system....
%% > She uses the generated plans to (a) as inspiration for worked examples in teh course, (b) as stems for questions that test how code should be completed
%% > She notices she now has a list of key concepts in the area


\begin{figure*}[h]
        \Description{An annotated screenshot of PLAID's `Programs' view. On the left, a list of use cases such as `Renaming columns in a Frame' and `Plotting a histogram of a column' is shown, with a scrollable list and a search bar. The latter one is selected, and on the right, we see the contents of the program in a monospaced font, with four buttons explained in the caption.}
        \includegraphics[width=\textwidth]{img/system-diagram-1-fixed.png}
        \caption{Plan Identification using PLAID: (a) list of example programs for instructors organized by natural language descriptions, (b) list of full programs of code, (c) search bar enabling easy navigation of given content to find code for specific ideas, (d) button to create a plan using the selected part of the code, (e) button to create a plan using the complete example program, (f) button to view an explanation for a selected code snippet, and (g) button for executing the selected code.}
        \label{fig:system-diagram-1}
\end{figure*}

\subsection{System Design}

At a high level, PLAID\footnote{The code for PLAID can be found at: https://github.com/yosheejain/plaid-interface.} operates on two subsystems: (1) a database of LLM-generated reference materials created through a pipeline that uses \edit{OpenAI's GPT-4o\footnote{https://openai.com/index/hello-gpt-4o/}~\cite{achiam2023gpt}}, inspired by instructors' best practices for identifying programming plans (see ~\cref{fig:system-pipeline})
%LLM for identifying plans in application-focused domains 
and (2) an interface that allows instructors to browse reference materials for relevant code snippets 
% and other plan components to achieve a goal that meets their needs. Then, they refine the candidates to mine plans 
and refine suggested content into programming plans
(see Figures~\ref{fig:system-diagram-1} and~\ref{fig:system-diagram-2}).
% In this section, we describe the implementation of the pipeline generating the reference materials and the key interface features of PLAID.



\subsubsection{Database of Reference Materials for Application-Focused Domains}

PLAID extracts information from reference materials at three levels of granularity to support each instructor's unique workflow: complete programs that address a particular use case, annotated program snippets that include goals and changeable areas, and plan candidates that cluster relevant program snippets together.

\textbf{Generating complete example programs.}
The content at the lowest level of granularity in the PLAID database are the complete programs. 
%These candidate plans were generated using a pipeline to generate \textit{plan-ful examples}, which we define as examples of programming plans in use, with all plan components identified (see Section~\ref{sec:components}). This implementation had three stages: (1) generating in-domain programs, (2) segmenting programs into plan-ful examples, and (3) clustering plan-ful examples into plans. 
\label{sec:llm-pipeline}
% \begin{figure}
% \centering
% % \includegraphics[width=0.5\textwidth]{img/pipeline-new.png}
% \includegraphics[width=\textwidth]{img/new-plan-pipeline.png}
% \caption{The three stage process for generating example programs, segmenting them with plan components, and clustering these plan-ful examples.
% %collecting and processing responses from ChatGPT into plan-ful examples}
% %\caption{The pipeline for LLM plan generation.}
% }
% \label{fig:llm-methods}
% \end{figure}
% \subsubsection{Generating In-Domain Programs}
% Informed by the insights identified in our interview study, we generated programming plans relevant to an application-focused domain: web scraping via BeautifulSoup. We utilized an LLM-based approach to generate these plans with the GPT-4 model from OpenAI using its publicly available API in an iterative workflow. 
% Our participants examined example programs and conducted literature reviews (Section \ref{sec:viewing-programs}) as key parts of their plan identification process. 
As these examples should capture a variance of use cases in the real world, we utilized an LLM trained on a large corpus of computer programs and natural language descriptions~\cite{liu2023isyourcode}.
% Inspired by this, we used Open AI's GPT-4, a state-of-the-art large language model for code generation that is trained on a large corpus of computer programs~\cite{liu2023isyourcode},
% to generate candidate programs along with its respective plan components in the programs.
We prompted\footnote{Full prompts can be found in \cref{sec:appendix-pipeline}.} the model to generate \texttt{specific use cases of <application-focused library>}, defining use case as \texttt{a task you can achieve 
with the given library} (see \cref{sec:use_case_prompt}). Subsequently, we prompted the model to \texttt{write code to do the following: <use case>}, producing a set of 100 example programs with associated tasks (see \cref{sec:code_prompt}). By generating the use cases first and generating the solution later, we avoided the problems with context windows of LLMs where the earlier input might get `forgotten', resulting in the model producing the same output repeatedly. For practical purposes, we generated 100 programs per domain. \edit{To test for potential ``hallucinations'' where the LLM generates plausible yet incorrect code~\cite{Ji_2023_hallucination}, we checked the syntactic validity of the generated programs before developing the rest of our pipeline. No more than one out of 100 generated programs included syntax errors in each of our domains, i.e., Pandas, Django, and PyTorch. Thus, we concluded that hallucinations are not a major threat to the code generation aspect of PLAID.}
%while hallucinations in LLMs are a pressing concern for systems that utilize these models,
% This collection of example programs (which we refer to as 
%dataset 
% $\mathcal{D}$) was used as our primary dataset for further analysis.

\textbf{Generating annotated program snippets.}
% \subsubsection{Segmenting Programs Into Plan-ful Examples}
% We then proceed to compile these examples with each of the plan components generated using ChatGPT. We construct a new dataset with these components, Dataset \((\mathcal{D}^{\textit{Comp}})\).
The second level of granularity in PLAID consists of small program snippets and a goal, with changeable areas annotated. 
We used the generated programs from
% \mathcal{D}$
the prior step as the input to the LLM to add subgoal labels, where we prompted the LLM to annotate subgoals (see \cref{sec:subgoals_prompt}) as comments that describe \texttt{small chunks of code that achieve a task that can be explained in natural language}. These subgoal labels were used to break the full program into shorter snippets. Each snippet was fed back to the model to generate changeable areas (see \cref{sec:ca_prompt}), defined in the prompt as \texttt{parts of the idiom that would change when it is used in different scenarios}. The subgoal label that explained a code snippet corresponded to its goal in the plan view and the list of elements assigned as changeable was used for annotations.
% (see Stage 2 in Figure~\ref{fig:llm-methods}),
%We fragmented these generated programs into smaller code pieces by generating \textit{subgoals} in the program. Then, each goal (Section \ref{sec:goal}) and the accompanying code solution (Section \ref{sec:solution}) were added as a single unit of data in our plan-ful example dataset of components, \(\mathcal{D}^{\textit{Plan-ful}}\). For each of these datapoints, we prompted the model to identify \textit{changeable areas} (Section \ref{sec:changeable}). %The name (Section \ref{sec:name}) was determined later in the pipeline (Stage 2 in Figure \ref{fig:llm-methods}).


% From the results of our qualitative study, we now know about the parts of a programming plan. In order to extract these plans automatically, we used ChatGPT. We accessed it using its publicly available API and we used the GPT-4 model. We selected 3 domains that are interesting for non-majors. This included . 

% For each of these domains, we first asked the LLM to generate 100 use cases. We then re-prompted it with the use cases it generated and asked it to generate code that would be written to accomplish that use case.
% potential for another table?
% add code metrics from stackoverflow github work for chatgpt
% With all these code pieces collected, we then asked ChatGPT to generate each of the plan parts one-by-one.

% \subsubsection*{Extracting Goals and Solutions}Generated programs 
% in \(\mathcal{D}\) 
% typically included a comment before each line, which described that line's functionality. However, these comments did not capture the high-level purpose of the code, as required by a plan goal. To generate more abstract goals for a piece of code, we defined subgoals as \texttt{short descriptions of small pieces of code that do something meaningful} in a prompt and asked the LLM to \texttt{highlight subgoals as comments in the code.} %In our query, we also added the way we define subgoals to provide the relevant context to the model. Specifically, we wrote that 
% The output from this prompt was a modified version of each program
% from \(\mathcal{D}\), 
% where blocks of code are preceded by a comment describing the goal of that block. % of code. % instead of restating functionality. 

% We split each complete program into multiple segments based on these new comments. Thus, the subgoal comments from each complete program I
% n the modified \(\mathcal{D}\) 
% became a plan goal, and the code following that comment became the associated solution. %, collected in \(\mathcal{D}^{\textit{Plan-ful}}\). % After it returned the annotated code piece, we extracted the comment and the following lines of code before the next comment. This pair acted as a subgoal-code piece. We collected all such pairs across all use cases from \(\mathcal{D}\) and added them to \(\mathcal{D}^{\textit{Plan-ful}}\).
% Each goal 
% %(Section \ref{sec:goal}) 
% and solution pair
% %(Section \ref{sec:solution}) 
% was added as a single unit of data in our plan-ful example dataset.
% , \(\mathcal{D}^{\textit{Plan-ful}}\).

% \subsubsection*{Extracting Changeable Areas}To annotate the changeable areas for a plan, we defined changeable areas as \texttt{parts of the plan that would change when it is used in a different context} in our prompt and asked the model to \texttt{return the exact part of the code from the line that would change} for all code pieces from the dataset with plan-ful examples.
% from \(\mathcal{D}^{\textit{Plan-ful}}\). 
% This data was added to \(\mathcal{D}^{\textit{Plan-ful}}\).

% to-do
% \subsubsection{Clustering Plan-ful Examples into Plans}
\textbf{Generating clustered plan candidates.}
\label{sec:clustering}
% We perform k-means clustering on the plans \(\mathcal{D}^{\textit{Plan-ful}}\) to identify clusters of similar code pieces and thus, programming plans.
The highest level of granularity provided in PLAID
%presents users with 
are
plan candidates, in the form of clusters of annotated program snippets. To compare the similarity of program snippets, we used CodeBERT embeddings following prior work~\cite{codebert} and applied Principal Component Analysis (PCA) \cite{PCAanalysis} to reduce the dimensionality of the embedding while preserving 90\% of the variance. The snippets were clustered using the K-means algorithm~\cite{kmeansclustering}, using the mean silhouette coefficient for determining optimal K~\cite{silhouettecoeff}. Each cluster is treated as a plan candidate, with the goal, code, and changeable areas from each program snippet in the cluster presented as a suggested value for the plan attributes.
% We used a clustering algorithm to group similar program snippets 
% plan-ful examples together as a programming plan. For clustering the code pieces, we used the CodeBERT model from Microsoft \cite{codebert} to obtain embeddings for each code piece in our dataset of plan-ful examples
% % in \(\mathcal{D}^{\textit{Plan-ful}}\) 
% and applied Principal Component Analysis (PCA) \cite{PCAanalysis} to reduce the dimensionality of the embedding vectors while preserving 90\% of the variance. These embeddings were clustered using the K-means algorithm~\cite{kmeansclustering}. The optimal number of clusters \(\mathcal{K}\) was determined by assessing all possible \(\mathcal{K}\) values 
% % (where \(\mathcal{K} \in [2, \texttt{length}(\mathcal{D}^{\textit{Plan-ful}})]\))
% using the mean silhouette coefficient \cite{silhouettecoeff}. We assigned each example 
% % in \(\mathcal{D}^{\textit{Plan-ful}}\) 
% to a cluster of similar code pieces. 
% \subsubsection*{Extracting Names}
For each plan candidate, a name (see \cref{sec:name_prompt}) that summarizes all snippets in the cluster was generated by prompting an LLM with the contents of the snippets and stating that it should generate \texttt{a name that reflects the code's purpose} and it should focus on \texttt{what the code is achieving and not the context}. 
% Then, all code snippets from each cluster of examples were provided as input to the LLM along with a prompt asking it to \texttt{devise a name for that cluster of plans}.

% \subsection{Interface for Refining Candidate Plans}

% %nd the back-end server relied on routes written in Flask. The domain-specific candidate plans suggested to the user are queried from the database of candidate plans generated using the LLM. Each participant was required to log in to the web page using their unique credentials, which allowed us to record their activity for analysis. While the complete details of our implementation of the web-based application are out of scope for this paper, we describe its main features in Section~\ref{sec:implementation_of_webinterface}.

% \subsubsection{\edit{Preliminary Technical Evaluation of Generated Content}}

% \edit{syntactic validity and standard code complexity metrics to determine
% their suitability for novices}


\begin{figure*}[h]
    \Description{An annotated screenshot of PLAID's Plan Creation view with three panes, with plans shown as boxes on the left. A plan is highlighted, and we see its components on the middle pane. On the rightmost pane, we see suggested values for the selected component.}
    \includegraphics[width=\textwidth]{img/system-diagram-2-new.png}        
    \caption{Plan Identification using PLAID: (h) button that suggests a domain-specific candidate plan from the system database, (i) pane enabling viewing of similar values for the selected plan component, (j) button to view the solution code as part of a complete program, (k) pane with a structured template for plan design with editable fields to refine plan components, (l) button to copy a selected plan, (m) button to mark snippets of code from the plan solution as changeable areas, and (n) a button to group plans together into a category and add a name.}
    \label{fig:system-diagram-2}
\end{figure*}

% \subsubsection{Key Characteristics}
% PLAID supports the process of plan identification in data processing with Pandas, machine learning with Pytorch, web development using Django, and web scraping using BeautifulSoup. 

\subsubsection{Interface for Designing Programming Plans}
Building on the 
%characteristics addressed in the artifact (Section~\ref{sec:design-artifact}) and 
design goals identified in the design workshop (\cref{sec:design-goals}), PLAID enables a set of key interactions to assist instructors in refining candidates to design plans for their instruction. 



\textbf{Interactions for Initial Plan Identification.}
% Initial Plan Identification with Quick Exploration of Many Authentic Programs
While instructors valued the availability of code examples in the design workshop (Section~\ref{sec:design-workshop-findings}), we observed many opportunities for scaffolding their interaction with the reference material. To this end, PLAID presents example programs in two different views \textbf{(DG1)}. 
% We saw instructors scanning examples, selecting desired code pieces, and copying them over into their plan templates in all conditions in the study. 
The ``Programs (Organized by Use Case)'' (\cref{fig:system-diagram-1}a) tab includes a list of use cases where instructors can click on an item to expand the program for that use case.
The ``Programs (Full Text)''  tab (\cref{fig:system-diagram-1}b) lists all the programs and enables instructors to scroll or search through (\cref{fig:system-diagram-1}c) all the code at once.
% presents the contents of all the programs expanded viewing a list of complete code examples, allowing instructors to look at materials they would typically search for when designing plans.
% equipping instructors with full-code programs organized in a list of short natural language descriptions of common use cases in their domain of expertise. 
Both views support directly creating a plan from the whole example (\cref{fig:system-diagram-1}e), or a selected part of it (\cref{fig:system-diagram-1}d), by copying the solution and the goal of the program into an empty plan template
% < Highlight code in full code and code pane in tab1 and make a plan (D1)
% < Add a button to add full program as a plan too (D1)
further supporting efficient use of the material \textbf{(DG3)}.
% This interaction copies over the selected code and its respective use case into the solution and name fields, respectively. 
% < Code explanation plugin for strange syntax (GPT) (D2)

To facilitate understanding unfamiliar code and syntax, we implemented a ``View Explanation'' button (\textbf{DG2}) that generates a short description of the selected line(s) of code by prompting an LLM (\cref{fig:system-diagram-1}f). 
% In this case, participants hesitated to use the suggested syntax in their plans because its functionality was unclear to them. PLAID supports a button named ``View Explanation'' where the user can select a method, function, or line of code that is unclear and click on it to understand its working \textbf{(D2)}. 
Participants also looked for code execution to validate and understand a program. However, since the code snippets instructors work with are often incomplete in this task, we implemented a ``Run Code'' feature (\textbf{DG2}) that predicts the output of a selected code snippet by prompting an LLM to walk through the code \texttt{step by step}, using Chain-of-Thought prompting~\cite{wei2022chain} (\cref{fig:system-diagram-1}g). Only the predicted output for the code is presented, ignoring other output from the LLM.

% to examine the code behavior and thus mitigate the challenge of being faced with unfamiliar syntax. Thus, using PLAID, instructors are able to run complete programs to view their output \textbf{(D2)}.
% < Search in the use cases (and full progs) (D3)
% Frequently, instructors relied on their expertise and experience to formulate ideas about goals for which they wanted to create plans. While interacting with condition C in the design workshop, interviewees suggested including a mechanism to search for specific keywords within code and  its natural language description. To facilitate the instructor-LLM collaboration, allowing users to find examples implementing their ideas, PLAID includes a search bar that helps users navigate the given use cases, complete programs, and effectively find specific examples they may be looking for \textbf{(D3)}.

\textbf{Interactions for Plan Refinement.}
% Support Plan Refinement with Comparisons of content
% Participants indicated difficulty mining plans from code examples (Section~\ref{sec:challenges_practice}). 
To provide suggestions for code patterns common enough to be potential programming plans,
%To alleviate challenges in identifying content common enough for designing plans, 
we utilize the clustered program snippets from our database. In the ``Plan Creation'' view of PLAID, instructors can ask for suggestions (\cref{fig:system-diagram-2}h) to see a candidate plan to refine (\textbf{DG3}).  \edit{This functionality allows instructors to draw on their experience to recognize common code snippets and decide if they are valuable to teach students.}
% If instructors want to demonstrate their plan as part of a complete code example, they can review these examples reducing the effort that they would need to put in to recall syntax and construct a complete example. 
\edit{This promotes recognition over recall \cite{recognition_over_recall}, thus helping reduce the cognitive effort that instructors may have to put in while designing programming plans traditionally.}
To allow instructors to better understand the context of a plan under refinement, PLAID 
also includes a button for searching for the current solution within the entire set of full programs
%, showing the code snippet in context 
%as part of a complete example
(\textbf{DG3}, \cref{fig:system-diagram-2}j).
% < Keyword search/embedding filter for potential values (D1)

As instructors edit the components of a plan, they are shown similar values from the corresponding component in that cluster (\cref{fig:system-diagram-2}i). By clicking on any suggested value, instructors can replace a plan component with a suggestion that better captures that aspect of the plan \textbf{(DG1)}. \edit{By allowing instructors to view the plan they are working on along with other related code pieces in a split screen view, we promote instructor efficiency by reducing the split-attention effect \cite{tarmizi1988guidance}. In the current plan creation process, even when using LLMs from their chat interface, instructors would have to switch between windows with code examples and their text editor which may increase the load on the instructors' working memory \cite{clark2023learning}. In PLAID, instructors can edit their plans and view similar code pieces at the same time.}

% \edit{By enabling these interactions and thus organizing ``knowledge in the world'' effectively, PLAID reduces the need for instructors to store and retrieve the ``knowledge in their head'' \cite{Norman_DOET}. Thus, PLAID optimizes the plan creation process by allowing efficient search within the ``knowledge in the world'' and reducing the cognitive load while storing and retrieving ``knowledge in the head'', minimizing the total effort required \cite{} by instructors.}
% after searching its code corpus for similar examples using a keyword search \textbf{(DG1)}.
% < Show use case button in solution (add highlighting) (D2)
% To help instructors easily consider the context of a plan as they refine it, PLAID 
% In the design workshop, few instructors emphasized the importance of presenting worked and contextualized examples to students. 

% ‘go to a use case’ button that redirects the user to the tab with full code programs and highlights the plan as part of a complete example \textbf{(D2)}.

\textbf{Interactions for Building Robust and Shareable Plan Descriptions.}
% Support robust/sharable plan descriptions
% From Section~\ref{sec:process_intro_plan_design}, instructors indicated drawing on their experience in the application-specific domain and instructional expertise to think about how to best solve a problem. 
PLAID encourages instructors to design plans in a structured template (\cref{fig:system-diagram-2}k). Moreover, PLAID reinforces the plan template by providing a dedicated method for annotating changeable areas by highlighting any part of the code (\textbf{DG3}, \cref{fig:system-diagram-2}m). Instructors can further explain the changeable areas by adding comments as text.
% \edit{The structured template view of the plan encourages instructors to articulate their mental models of how the plan would generalize to other problems, allowing the transfer of ``knowledge in the head'' to ``knowledge in the world''.}

Our design workshop showed that participants would create a plan and copy it to emphasize alternatives or modifications to the underlying idea. To support this workflow,
% In our design workshop, participants created copies of their plans to display alternative solutions to achieve the same goal, emphasizing that multiple possible solutions in code could accomplish the same goal.
% < Duplicating plans (D3)
% To accelerate this process of teaching a variety of possible solutions, 
PLAID allows users to ``duplicate'' plans on the canvas and further edit them to present alternative solutions for the same plan \textbf{(DG3}, \cref{fig:system-diagram-2}l).
% Highlight text from solution to change it to changeable areas (highlighting code itself) (D4)

% In conditions A and B, instructors highlighted the changeable areas in the code itself.
% To allow participants to emphasize the changeable areas in code in PLAID, we implemented the ``add to changeable areas'' button. After selecting the changeable piece of code, clicking on this button highlights the text in a different color and adds it to the box of changeable areas to complete the templated plan design (\textbf{D4}).
% Grouping plans into categories (D4)
% < Multiple selection of the boxes (D4)
% < Naming groups of boxes (D4)
To encourage instructors to think about organizing plans in ways that they would present them to students, PLAID provides an open canvas view for instructors that allows them to arrange plans as they prefer. In addition, PLAID supports a ``grouping'' feature (\cref{fig:system-diagram-2}n), which allows instructors to combine plans with similar goals together into one category (\textbf{DG4}).

% A handful of users postulated each plan as an example question that can be used on assessments. They intended to create multiple variants of the same question for students. They suggested that being able to visualize the different categories would be helpful. Using PLAID, users can select multiple patterns together, add them to a group, and name the group \textbf{(D4)}.  % :(

\subsubsection{System Architecture}
The pipeline to create reference materials is implemented in Python, using the state-of-the-art large language model GPT-4o (Model Version: 2024-05-13). The interface for PLAID is implemented as a web application in Python as a Flask webserver, with an SQLite database. The user-facing interface is implemented using HTML, CSS, and JavaScript, with the canvas interactions realized with the library `\textit{interact.js}'. 



\section{Experiments}
\label{sec:experiments}
The experiments are designed to address two key research questions.
First, \textbf{RQ1} evaluates whether the average $L_2$-norm of the counterfactual perturbation vectors ($\overline{||\perturb||}$) decreases as the model overfits the data, thereby providing further empirical validation for our hypothesis.
Second, \textbf{RQ2} evaluates the ability of the proposed counterfactual regularized loss, as defined in (\ref{eq:regularized_loss2}), to mitigate overfitting when compared to existing regularization techniques.

% The experiments are designed to address three key research questions. First, \textbf{RQ1} investigates whether the mean perturbation vector norm decreases as the model overfits the data, aiming to further validate our intuition. Second, \textbf{RQ2} explores whether the mean perturbation vector norm can be effectively leveraged as a regularization term during training, offering insights into its potential role in mitigating overfitting. Finally, \textbf{RQ3} examines whether our counterfactual regularizer enables the model to achieve superior performance compared to existing regularization methods, thus highlighting its practical advantage.

\subsection{Experimental Setup}
\textbf{\textit{Datasets, Models, and Tasks.}}
The experiments are conducted on three datasets: \textit{Water Potability}~\cite{kadiwal2020waterpotability}, \textit{Phomene}~\cite{phomene}, and \textit{CIFAR-10}~\cite{krizhevsky2009learning}. For \textit{Water Potability} and \textit{Phomene}, we randomly select $80\%$ of the samples for the training set, and the remaining $20\%$ for the test set, \textit{CIFAR-10} comes already split. Furthermore, we consider the following models: Logistic Regression, Multi-Layer Perceptron (MLP) with 100 and 30 neurons on each hidden layer, and PreactResNet-18~\cite{he2016cvecvv} as a Convolutional Neural Network (CNN) architecture.
We focus on binary classification tasks and leave the extension to multiclass scenarios for future work. However, for datasets that are inherently multiclass, we transform the problem into a binary classification task by selecting two classes, aligning with our assumption.

\smallskip
\noindent\textbf{\textit{Evaluation Measures.}} To characterize the degree of overfitting, we use the test loss, as it serves as a reliable indicator of the model's generalization capability to unseen data. Additionally, we evaluate the predictive performance of each model using the test accuracy.

\smallskip
\noindent\textbf{\textit{Baselines.}} We compare CF-Reg with the following regularization techniques: L1 (``Lasso''), L2 (``Ridge''), and Dropout.

\smallskip
\noindent\textbf{\textit{Configurations.}}
For each model, we adopt specific configurations as follows.
\begin{itemize}
\item \textit{Logistic Regression:} To induce overfitting in the model, we artificially increase the dimensionality of the data beyond the number of training samples by applying a polynomial feature expansion. This approach ensures that the model has enough capacity to overfit the training data, allowing us to analyze the impact of our counterfactual regularizer. The degree of the polynomial is chosen as the smallest degree that makes the number of features greater than the number of data.
\item \textit{Neural Networks (MLP and CNN):} To take advantage of the closed-form solution for computing the optimal perturbation vector as defined in (\ref{eq:opt-delta}), we use a local linear approximation of the neural network models. Hence, given an instance $\inst_i$, we consider the (optimal) counterfactual not with respect to $\model$ but with respect to:
\begin{equation}
\label{eq:taylor}
    \model^{lin}(\inst) = \model(\inst_i) + \nabla_{\inst}\model(\inst_i)(\inst - \inst_i),
\end{equation}
where $\model^{lin}$ represents the first-order Taylor approximation of $\model$ at $\inst_i$.
Note that this step is unnecessary for Logistic Regression, as it is inherently a linear model.
\end{itemize}

\smallskip
\noindent \textbf{\textit{Implementation Details.}} We run all experiments on a machine equipped with an AMD Ryzen 9 7900 12-Core Processor and an NVIDIA GeForce RTX 4090 GPU. Our implementation is based on the PyTorch Lightning framework. We use stochastic gradient descent as the optimizer with a learning rate of $\eta = 0.001$ and no weight decay. We use a batch size of $128$. The training and test steps are conducted for $6000$ epochs on the \textit{Water Potability} and \textit{Phoneme} datasets, while for the \textit{CIFAR-10} dataset, they are performed for $200$ epochs.
Finally, the contribution $w_i^{\varepsilon}$ of each training point $\inst_i$ is uniformly set as $w_i^{\varepsilon} = 1~\forall i\in \{1,\ldots,m\}$.

The source code implementation for our experiments is available at the following GitHub repository: \url{https://anonymous.4open.science/r/COCE-80B4/README.md} 

\subsection{RQ1: Counterfactual Perturbation vs. Overfitting}
To address \textbf{RQ1}, we analyze the relationship between the test loss and the average $L_2$-norm of the counterfactual perturbation vectors ($\overline{||\perturb||}$) over training epochs.

In particular, Figure~\ref{fig:delta_loss_epochs} depicts the evolution of $\overline{||\perturb||}$ alongside the test loss for an MLP trained \textit{without} regularization on the \textit{Water Potability} dataset. 
\begin{figure}[ht]
    \centering
    \includegraphics[width=0.85\linewidth]{img/delta_loss_epochs.png}
    \caption{The average counterfactual perturbation vector $\overline{||\perturb||}$ (left $y$-axis) and the cross-entropy test loss (right $y$-axis) over training epochs ($x$-axis) for an MLP trained on the \textit{Water Potability} dataset \textit{without} regularization.}
    \label{fig:delta_loss_epochs}
\end{figure}

The plot shows a clear trend as the model starts to overfit the data (evidenced by an increase in test loss). 
Notably, $\overline{||\perturb||}$ begins to decrease, which aligns with the hypothesis that the average distance to the optimal counterfactual example gets smaller as the model's decision boundary becomes increasingly adherent to the training data.

It is worth noting that this trend is heavily influenced by the choice of the counterfactual generator model. In particular, the relationship between $\overline{||\perturb||}$ and the degree of overfitting may become even more pronounced when leveraging more accurate counterfactual generators. However, these models often come at the cost of higher computational complexity, and their exploration is left to future work.

Nonetheless, we expect that $\overline{||\perturb||}$ will eventually stabilize at a plateau, as the average $L_2$-norm of the optimal counterfactual perturbations cannot vanish to zero.

% Additionally, the choice of employing the score-based counterfactual explanation framework to generate counterfactuals was driven to promote computational efficiency.

% Future enhancements to the framework may involve adopting models capable of generating more precise counterfactuals. While such approaches may yield to performance improvements, they are likely to come at the cost of increased computational complexity.


\subsection{RQ2: Counterfactual Regularization Performance}
To answer \textbf{RQ2}, we evaluate the effectiveness of the proposed counterfactual regularization (CF-Reg) by comparing its performance against existing baselines: unregularized training loss (No-Reg), L1 regularization (L1-Reg), L2 regularization (L2-Reg), and Dropout.
Specifically, for each model and dataset combination, Table~\ref{tab:regularization_comparison} presents the mean value and standard deviation of test accuracy achieved by each method across 5 random initialization. 

The table illustrates that our regularization technique consistently delivers better results than existing methods across all evaluated scenarios, except for one case -- i.e., Logistic Regression on the \textit{Phomene} dataset. 
However, this setting exhibits an unusual pattern, as the highest model accuracy is achieved without any regularization. Even in this case, CF-Reg still surpasses other regularization baselines.

From the results above, we derive the following key insights. First, CF-Reg proves to be effective across various model types, ranging from simple linear models (Logistic Regression) to deep architectures like MLPs and CNNs, and across diverse datasets, including both tabular and image data. 
Second, CF-Reg's strong performance on the \textit{Water} dataset with Logistic Regression suggests that its benefits may be more pronounced when applied to simpler models. However, the unexpected outcome on the \textit{Phoneme} dataset calls for further investigation into this phenomenon.


\begin{table*}[h!]
    \centering
    \caption{Mean value and standard deviation of test accuracy across 5 random initializations for different model, dataset, and regularization method. The best results are highlighted in \textbf{bold}.}
    \label{tab:regularization_comparison}
    \begin{tabular}{|c|c|c|c|c|c|c|}
        \hline
        \textbf{Model} & \textbf{Dataset} & \textbf{No-Reg} & \textbf{L1-Reg} & \textbf{L2-Reg} & \textbf{Dropout} & \textbf{CF-Reg (ours)} \\ \hline
        Logistic Regression   & \textit{Water}   & $0.6595 \pm 0.0038$   & $0.6729 \pm 0.0056$   & $0.6756 \pm 0.0046$  & N/A    & $\mathbf{0.6918 \pm 0.0036}$                     \\ \hline
        MLP   & \textit{Water}   & $0.6756 \pm 0.0042$   & $0.6790 \pm 0.0058$   & $0.6790 \pm 0.0023$  & $0.6750 \pm 0.0036$    & $\mathbf{0.6802 \pm 0.0046}$                    \\ \hline
%        MLP   & \textit{Adult}   & $0.8404 \pm 0.0010$   & $\mathbf{0.8495 \pm 0.0007}$   & $0.8489 \pm 0.0014$  & $\mathbf{0.8495 \pm 0.0016}$     & $0.8449 \pm 0.0019$                    \\ \hline
        Logistic Regression   & \textit{Phomene}   & $\mathbf{0.8148 \pm 0.0020}$   & $0.8041 \pm 0.0028$   & $0.7835 \pm 0.0176$  & N/A    & $0.8098 \pm 0.0055$                     \\ \hline
        MLP   & \textit{Phomene}   & $0.8677 \pm 0.0033$   & $0.8374 \pm 0.0080$   & $0.8673 \pm 0.0045$  & $0.8672 \pm 0.0042$     & $\mathbf{0.8718 \pm 0.0040}$                    \\ \hline
        CNN   & \textit{CIFAR-10} & $0.6670 \pm 0.0233$   & $0.6229 \pm 0.0850$   & $0.7348 \pm 0.0365$   & N/A    & $\mathbf{0.7427 \pm 0.0571}$                     \\ \hline
    \end{tabular}
\end{table*}

\begin{table*}[htb!]
    \centering
    \caption{Hyperparameter configurations utilized for the generation of Table \ref{tab:regularization_comparison}. For our regularization the hyperparameters are reported as $\mathbf{\alpha/\beta}$.}
    \label{tab:performance_parameters}
    \begin{tabular}{|c|c|c|c|c|c|c|}
        \hline
        \textbf{Model} & \textbf{Dataset} & \textbf{No-Reg} & \textbf{L1-Reg} & \textbf{L2-Reg} & \textbf{Dropout} & \textbf{CF-Reg (ours)} \\ \hline
        Logistic Regression   & \textit{Water}   & N/A   & $0.0093$   & $0.6927$  & N/A    & $0.3791/1.0355$                     \\ \hline
        MLP   & \textit{Water}   & N/A   & $0.0007$   & $0.0022$  & $0.0002$    & $0.2567/1.9775$                    \\ \hline
        Logistic Regression   &
        \textit{Phomene}   & N/A   & $0.0097$   & $0.7979$  & N/A    & $0.0571/1.8516$                     \\ \hline
        MLP   & \textit{Phomene}   & N/A   & $0.0007$   & $4.24\cdot10^{-5}$  & $0.0015$    & $0.0516/2.2700$                    \\ \hline
       % MLP   & \textit{Adult}   & N/A   & $0.0018$   & $0.0018$  & $0.0601$     & $0.0764/2.2068$                    \\ \hline
        CNN   & \textit{CIFAR-10} & N/A   & $0.0050$   & $0.0864$ & N/A    & $0.3018/
        2.1502$                     \\ \hline
    \end{tabular}
\end{table*}

\begin{table*}[htb!]
    \centering
    \caption{Mean value and standard deviation of training time across 5 different runs. The reported time (in seconds) corresponds to the generation of each entry in Table \ref{tab:regularization_comparison}. Times are }
    \label{tab:times}
    \begin{tabular}{|c|c|c|c|c|c|c|}
        \hline
        \textbf{Model} & \textbf{Dataset} & \textbf{No-Reg} & \textbf{L1-Reg} & \textbf{L2-Reg} & \textbf{Dropout} & \textbf{CF-Reg (ours)} \\ \hline
        Logistic Regression   & \textit{Water}   & $222.98 \pm 1.07$   & $239.94 \pm 2.59$   & $241.60 \pm 1.88$  & N/A    & $251.50 \pm 1.93$                     \\ \hline
        MLP   & \textit{Water}   & $225.71 \pm 3.85$   & $250.13 \pm 4.44$   & $255.78 \pm 2.38$  & $237.83 \pm 3.45$    & $266.48 \pm 3.46$                    \\ \hline
        Logistic Regression   & \textit{Phomene}   & $266.39 \pm 0.82$ & $367.52 \pm 6.85$   & $361.69 \pm 4.04$  & N/A   & $310.48 \pm 0.76$                    \\ \hline
        MLP   &
        \textit{Phomene} & $335.62 \pm 1.77$   & $390.86 \pm 2.11$   & $393.96 \pm 1.95$ & $363.51 \pm 5.07$    & $403.14 \pm 1.92$                     \\ \hline
       % MLP   & \textit{Adult}   & N/A   & $0.0018$   & $0.0018$  & $0.0601$     & $0.0764/2.2068$                    \\ \hline
        CNN   & \textit{CIFAR-10} & $370.09 \pm 0.18$   & $395.71 \pm 0.55$   & $401.38 \pm 0.16$ & N/A    & $1287.8 \pm 0.26$                     \\ \hline
    \end{tabular}
\end{table*}

\subsection{Feasibility of our Method}
A crucial requirement for any regularization technique is that it should impose minimal impact on the overall training process.
In this respect, CF-Reg introduces an overhead that depends on the time required to find the optimal counterfactual example for each training instance. 
As such, the more sophisticated the counterfactual generator model probed during training the higher would be the time required. However, a more advanced counterfactual generator might provide a more effective regularization. We discuss this trade-off in more details in Section~\ref{sec:discussion}.

Table~\ref{tab:times} presents the average training time ($\pm$ standard deviation) for each model and dataset combination listed in Table~\ref{tab:regularization_comparison}.
We can observe that the higher accuracy achieved by CF-Reg using the score-based counterfactual generator comes with only minimal overhead. However, when applied to deep neural networks with many hidden layers, such as \textit{PreactResNet-18}, the forward derivative computation required for the linearization of the network introduces a more noticeable computational cost, explaining the longer training times in the table.

\subsection{Hyperparameter Sensitivity Analysis}
The proposed counterfactual regularization technique relies on two key hyperparameters: $\alpha$ and $\beta$. The former is intrinsic to the loss formulation defined in (\ref{eq:cf-train}), while the latter is closely tied to the choice of the score-based counterfactual explanation method used.

Figure~\ref{fig:test_alpha_beta} illustrates how the test accuracy of an MLP trained on the \textit{Water Potability} dataset changes for different combinations of $\alpha$ and $\beta$.

\begin{figure}[ht]
    \centering
    \includegraphics[width=0.85\linewidth]{img/test_acc_alpha_beta.png}
    \caption{The test accuracy of an MLP trained on the \textit{Water Potability} dataset, evaluated while varying the weight of our counterfactual regularizer ($\alpha$) for different values of $\beta$.}
    \label{fig:test_alpha_beta}
\end{figure}

We observe that, for a fixed $\beta$, increasing the weight of our counterfactual regularizer ($\alpha$) can slightly improve test accuracy until a sudden drop is noticed for $\alpha > 0.1$.
This behavior was expected, as the impact of our penalty, like any regularization term, can be disruptive if not properly controlled.

Moreover, this finding further demonstrates that our regularization method, CF-Reg, is inherently data-driven. Therefore, it requires specific fine-tuning based on the combination of the model and dataset at hand.

%\section{Related Work}
%\label{sec:related-work}

%\subsection{Background}

%Defect detection is critical to ensure the yield of integrated circuit manufacturing lines and reduce faults. Previous research has primarily focused on wafer map data, which engineers produce by marking faulty chips with different colors based on test results. The specific spatial distribution of defects on a wafer can provide insights into the causes, thereby helping to determine which stage of the manufacturing process is responsible for the issues. Although such research is relatively mature, the continual miniaturization of integrated circuits and the increasing complexity and density of chip components have made chip-level detection more challenging, leading to potential risks\cite{ma2023review}. Consequently, there is a need to combine this approach with magnified imaging of the wafer surface using scanning electron microscopes (SEMs) to detect, classify, and analyze specific microscopic defects, thus helping to identify the particular process steps where defects originate.

%Previously, wafer surface defect classification and detection were primarily conducted by experienced engineers. However, this method relies heavily on the engineers' expertise and involves significant time expenditure and subjectivity, lacking uniform standards. With the ongoing development of artificial intelligence, deep learning methods using multi-layer neural networks to extract and learn target features have proven highly effective for this task\cite{gao2022review}.

%In the task of defect classification, it is typical to use a model structure that initially extracts features through convolutional and pooling layers, followed by classification via fully connected layers. Researchers have recently developed numerous classification model structures tailored to specific problems. These models primarily focus on how to extract defect features effectively. For instance, Chen et al. presented a defect recognition and classification algorithm rooted in PCA and classification SVM\cite{chen2008defect}. Chang et al. utilized SVM, drawing on features like smoothness and texture intricacy, for classifying high-intensity defect images\cite{chang2013hybrid}. The classification of defect images requires the formulation of numerous classifiers tailored for myriad inspection steps and an Abundance of accurately labeled data, making data acquisition challenging. Cheon et al. proposed a single CNN model adept at feature extraction\cite{cheon2019convolutional}. They achieved a granular classification of wafer surface defects by recognizing misclassified images and employing a k-nearest neighbors (k-NN) classifier algorithm to gauge the aggregate squared distance between each image feature vector and its k-neighbors within the same category. However, when applied to new or unseen defects, such models necessitate retraining, incurring computational overheads. Moreover, with escalating CNN complexity, the computational demands surge.

%Segmentation of defects is necessary to locate defect positions and gather information such as the size of defects. Unlike classification networks, segmentation networks often use classic encoder-decoder structures such as UNet\cite{ronneberger2015u} and SegNet\cite{badrinarayanan2017segnet}, which focus on effectively leveraging both local and global feature information. Han Hui et al. proposed integrating a Region Proposal Network (RPN) with a UNet architecture to suggest defect areas before conducting defect segmentation \cite{han2020polycrystalline}. This approach enables the segmentation of various defects in wafers with only a limited set of roughly labeled images, enhancing the efficiency of training and application in environments where detailed annotations are scarce. Subhrajit Nag et al. introduced a new network structure, WaferSegClassNet, which extracts multi-scale local features in the encoder and performs classification and segmentation tasks in the decoder \cite{nag2022wafersegclassnet}. This model represents the first detection system capable of simultaneously classifying and segmenting surface defects on wafers. However, it relies on extensive data training and annotation for high accuracy and reliability. 

%Recently, Vic De Ridder et al. introduced a novel approach for defect segmentation using diffusion models\cite{de2023semi}. This approach treats the instance segmentation task as a denoising process from noise to a filter, utilizing diffusion models to predict and reconstruct instance masks for semiconductor defects. This method achieves high precision and improved defect classification and segmentation detection performance. However, the complex network structure and the computational process of the diffusion model require substantial computational resources. Moreover, the performance of this model heavily relies on high-quality and large amounts of training data. These issues make it less suitable for industrial applications. Additionally, the model has only been applied to detecting and segmenting a single type of defect(bridges) following a specific manufacturing process step, limiting its practical utility in diverse industrial scenarios.

%\subsection{Few-shot Anomaly Detection}
%Traditional anomaly detection techniques typically rely on extensive training data to train models for identifying and locating anomalies. However, these methods often face limitations in rapidly changing production environments and diverse anomaly types. Recent research has started exploring effective anomaly detection using few or zero samples to address these challenges.

%Huang et al. developed the anomaly detection method RegAD, based on image registration technology. This method pre-trains an object-agnostic registration network with various images to establish the normality of unseen objects. It identifies anomalies by aligning image features and has achieved promising results. Despite these advancements, implementing few-shot settings in anomaly detection remains an area ripe for further exploration. Recent studies show that pre-trained vision-language models such as CLIP and MiniGPT can significantly enhance performance in anomaly detection tasks.

%Dong et al. introduced the MaskCLIP framework, which employs masked self-distillation to enhance contrastive language-image pretraining\cite{zhou2022maskclip}. This approach strengthens the visual encoder's learning of local image patches and uses indirect language supervision to enhance semantic understanding. It significantly improves transferability and pretraining outcomes across various visual tasks, although it requires substantial computational resources.
%Jeong et al. crafted the WinCLIP framework by integrating state words and prompt templates to characterize normal and anomalous states more accurately\cite{Jeong_2023_CVPR}. This framework introduces a novel window-based technique for extracting and aggregating multi-scale spatial features, significantly boosting the anomaly detection performance of the pre-trained CLIP model.
%Subsequently, Li et al. have further contributed to the field by creating a new expansive multimodal model named Myriad\cite{li2023myriad}. This model, which incorporates a pre-trained Industrial Anomaly Detection (IAD) model to act as a vision expert, embeds anomaly images as tokens interpretable by the language model, thus providing both detailed descriptions and accurate anomaly detection capabilities.
%Recently, Chen et al. introduced CLIP-AD\cite{chen2023clip}, and Li et al. proposed PromptAD\cite{li2024promptad}, both employing language-guided, tiered dual-path model structures and feature manipulation strategies. These approaches effectively address issues encountered when directly calculating anomaly maps using the CLIP model, such as reversed predictions and highlighting irrelevant areas. Specifically, CLIP-AD optimizes the utilization of multi-layer features, corrects feature misalignment, and enhances model performance through additional linear layer fine-tuning. PromptAD connects normal prompts with anomaly suffixes to form anomaly prompts, enabling contrastive learning in a single-class setting.

%These studies extend the boundaries of traditional anomaly detection techniques and demonstrate how to effectively address rapidly changing and sample-scarce production environments through the synergy of few-shot learning and deep learning models. Building on this foundation, our research further explores wafer surface defect detection based on the CLIP model, especially focusing on achieving efficient and accurate anomaly detection in the highly specialized and variable semiconductor manufacturing process using a minimal amount of labeled data.

\section{Conclusion}
In this work, we propose a simple yet effective approach, called SMILE, for graph few-shot learning with fewer tasks. Specifically, we introduce a novel dual-level mixup strategy, including within-task and across-task mixup, for enriching the diversity of nodes within each task and the diversity of tasks. Also, we incorporate the degree-based prior information to learn expressive node embeddings. Theoretically, we prove that SMILE effectively enhances the model's generalization performance. Empirically, we conduct extensive experiments on multiple benchmarks and the results suggest that SMILE significantly outperforms other baselines, including both in-domain and cross-domain few-shot settings.
\section{Acknowledgements}

%\newpage

%%
%% The next two lines define the bibliography style to be used, and
%% the bibliography file.
\bibliographystyle{ACM-Reference-Format}
\bibliography{cachelib}
%%
%% If your work has an appendix, this is the place to put it.
%\appendix

%\newpage
\appendix
\subsection{Lloyd-Max Algorithm}
\label{subsec:Lloyd-Max}
For a given quantization bitwidth $B$ and an operand $\bm{X}$, the Lloyd-Max algorithm finds $2^B$ quantization levels $\{\hat{x}_i\}_{i=1}^{2^B}$ such that quantizing $\bm{X}$ by rounding each scalar in $\bm{X}$ to the nearest quantization level minimizes the quantization MSE. 

The algorithm starts with an initial guess of quantization levels and then iteratively computes quantization thresholds $\{\tau_i\}_{i=1}^{2^B-1}$ and updates quantization levels $\{\hat{x}_i\}_{i=1}^{2^B}$. Specifically, at iteration $n$, thresholds are set to the midpoints of the previous iteration's levels:
\begin{align*}
    \tau_i^{(n)}=\frac{\hat{x}_i^{(n-1)}+\hat{x}_{i+1}^{(n-1)}}2 \text{ for } i=1\ldots 2^B-1
\end{align*}
Subsequently, the quantization levels are re-computed as conditional means of the data regions defined by the new thresholds:
\begin{align*}
    \hat{x}_i^{(n)}=\mathbb{E}\left[ \bm{X} \big| \bm{X}\in [\tau_{i-1}^{(n)},\tau_i^{(n)}] \right] \text{ for } i=1\ldots 2^B
\end{align*}
where to satisfy boundary conditions we have $\tau_0=-\infty$ and $\tau_{2^B}=\infty$. The algorithm iterates the above steps until convergence.

Figure \ref{fig:lm_quant} compares the quantization levels of a $7$-bit floating point (E3M3) quantizer (left) to a $7$-bit Lloyd-Max quantizer (right) when quantizing a layer of weights from the GPT3-126M model at a per-tensor granularity. As shown, the Lloyd-Max quantizer achieves substantially lower quantization MSE. Further, Table \ref{tab:FP7_vs_LM7} shows the superior perplexity achieved by Lloyd-Max quantizers for bitwidths of $7$, $6$ and $5$. The difference between the quantizers is clear at 5 bits, where per-tensor FP quantization incurs a drastic and unacceptable increase in perplexity, while Lloyd-Max quantization incurs a much smaller increase. Nevertheless, we note that even the optimal Lloyd-Max quantizer incurs a notable ($\sim 1.5$) increase in perplexity due to the coarse granularity of quantization. 

\begin{figure}[h]
  \centering
  \includegraphics[width=0.7\linewidth]{sections/figures/LM7_FP7.pdf}
  \caption{\small Quantization levels and the corresponding quantization MSE of Floating Point (left) vs Lloyd-Max (right) Quantizers for a layer of weights in the GPT3-126M model.}
  \label{fig:lm_quant}
\end{figure}

\begin{table}[h]\scriptsize
\begin{center}
\caption{\label{tab:FP7_vs_LM7} \small Comparing perplexity (lower is better) achieved by floating point quantizers and Lloyd-Max quantizers on a GPT3-126M model for the Wikitext-103 dataset.}
\begin{tabular}{c|cc|c}
\hline
 \multirow{2}{*}{\textbf{Bitwidth}} & \multicolumn{2}{|c|}{\textbf{Floating-Point Quantizer}} & \textbf{Lloyd-Max Quantizer} \\
 & Best Format & Wikitext-103 Perplexity & Wikitext-103 Perplexity \\
\hline
7 & E3M3 & 18.32 & 18.27 \\
6 & E3M2 & 19.07 & 18.51 \\
5 & E4M0 & 43.89 & 19.71 \\
\hline
\end{tabular}
\end{center}
\end{table}

\subsection{Proof of Local Optimality of LO-BCQ}
\label{subsec:lobcq_opt_proof}
For a given block $\bm{b}_j$, the quantization MSE during LO-BCQ can be empirically evaluated as $\frac{1}{L_b}\lVert \bm{b}_j- \bm{\hat{b}}_j\rVert^2_2$ where $\bm{\hat{b}}_j$ is computed from equation (\ref{eq:clustered_quantization_definition}) as $C_{f(\bm{b}_j)}(\bm{b}_j)$. Further, for a given block cluster $\mathcal{B}_i$, we compute the quantization MSE as $\frac{1}{|\mathcal{B}_{i}|}\sum_{\bm{b} \in \mathcal{B}_{i}} \frac{1}{L_b}\lVert \bm{b}- C_i^{(n)}(\bm{b})\rVert^2_2$. Therefore, at the end of iteration $n$, we evaluate the overall quantization MSE $J^{(n)}$ for a given operand $\bm{X}$ composed of $N_c$ block clusters as:
\begin{align*}
    \label{eq:mse_iter_n}
    J^{(n)} = \frac{1}{N_c} \sum_{i=1}^{N_c} \frac{1}{|\mathcal{B}_{i}^{(n)}|}\sum_{\bm{v} \in \mathcal{B}_{i}^{(n)}} \frac{1}{L_b}\lVert \bm{b}- B_i^{(n)}(\bm{b})\rVert^2_2
\end{align*}

At the end of iteration $n$, the codebooks are updated from $\mathcal{C}^{(n-1)}$ to $\mathcal{C}^{(n)}$. However, the mapping of a given vector $\bm{b}_j$ to quantizers $\mathcal{C}^{(n)}$ remains as  $f^{(n)}(\bm{b}_j)$. At the next iteration, during the vector clustering step, $f^{(n+1)}(\bm{b}_j)$ finds new mapping of $\bm{b}_j$ to updated codebooks $\mathcal{C}^{(n)}$ such that the quantization MSE over the candidate codebooks is minimized. Therefore, we obtain the following result for $\bm{b}_j$:
\begin{align*}
\frac{1}{L_b}\lVert \bm{b}_j - C_{f^{(n+1)}(\bm{b}_j)}^{(n)}(\bm{b}_j)\rVert^2_2 \le \frac{1}{L_b}\lVert \bm{b}_j - C_{f^{(n)}(\bm{b}_j)}^{(n)}(\bm{b}_j)\rVert^2_2
\end{align*}

That is, quantizing $\bm{b}_j$ at the end of the block clustering step of iteration $n+1$ results in lower quantization MSE compared to quantizing at the end of iteration $n$. Since this is true for all $\bm{b} \in \bm{X}$, we assert the following:
\begin{equation}
\begin{split}
\label{eq:mse_ineq_1}
    \tilde{J}^{(n+1)} &= \frac{1}{N_c} \sum_{i=1}^{N_c} \frac{1}{|\mathcal{B}_{i}^{(n+1)}|}\sum_{\bm{b} \in \mathcal{B}_{i}^{(n+1)}} \frac{1}{L_b}\lVert \bm{b} - C_i^{(n)}(b)\rVert^2_2 \le J^{(n)}
\end{split}
\end{equation}
where $\tilde{J}^{(n+1)}$ is the the quantization MSE after the vector clustering step at iteration $n+1$.

Next, during the codebook update step (\ref{eq:quantizers_update}) at iteration $n+1$, the per-cluster codebooks $\mathcal{C}^{(n)}$ are updated to $\mathcal{C}^{(n+1)}$ by invoking the Lloyd-Max algorithm \citep{Lloyd}. We know that for any given value distribution, the Lloyd-Max algorithm minimizes the quantization MSE. Therefore, for a given vector cluster $\mathcal{B}_i$ we obtain the following result:

\begin{equation}
    \frac{1}{|\mathcal{B}_{i}^{(n+1)}|}\sum_{\bm{b} \in \mathcal{B}_{i}^{(n+1)}} \frac{1}{L_b}\lVert \bm{b}- C_i^{(n+1)}(\bm{b})\rVert^2_2 \le \frac{1}{|\mathcal{B}_{i}^{(n+1)}|}\sum_{\bm{b} \in \mathcal{B}_{i}^{(n+1)}} \frac{1}{L_b}\lVert \bm{b}- C_i^{(n)}(\bm{b})\rVert^2_2
\end{equation}

The above equation states that quantizing the given block cluster $\mathcal{B}_i$ after updating the associated codebook from $C_i^{(n)}$ to $C_i^{(n+1)}$ results in lower quantization MSE. Since this is true for all the block clusters, we derive the following result: 
\begin{equation}
\begin{split}
\label{eq:mse_ineq_2}
     J^{(n+1)} &= \frac{1}{N_c} \sum_{i=1}^{N_c} \frac{1}{|\mathcal{B}_{i}^{(n+1)}|}\sum_{\bm{b} \in \mathcal{B}_{i}^{(n+1)}} \frac{1}{L_b}\lVert \bm{b}- C_i^{(n+1)}(\bm{b})\rVert^2_2  \le \tilde{J}^{(n+1)}   
\end{split}
\end{equation}

Following (\ref{eq:mse_ineq_1}) and (\ref{eq:mse_ineq_2}), we find that the quantization MSE is non-increasing for each iteration, that is, $J^{(1)} \ge J^{(2)} \ge J^{(3)} \ge \ldots \ge J^{(M)}$ where $M$ is the maximum number of iterations. 
%Therefore, we can say that if the algorithm converges, then it must be that it has converged to a local minimum. 
\hfill $\blacksquare$


\begin{figure}
    \begin{center}
    \includegraphics[width=0.5\textwidth]{sections//figures/mse_vs_iter.pdf}
    \end{center}
    \caption{\small NMSE vs iterations during LO-BCQ compared to other block quantization proposals}
    \label{fig:nmse_vs_iter}
\end{figure}

Figure \ref{fig:nmse_vs_iter} shows the empirical convergence of LO-BCQ across several block lengths and number of codebooks. Also, the MSE achieved by LO-BCQ is compared to baselines such as MXFP and VSQ. As shown, LO-BCQ converges to a lower MSE than the baselines. Further, we achieve better convergence for larger number of codebooks ($N_c$) and for a smaller block length ($L_b$), both of which increase the bitwidth of BCQ (see Eq \ref{eq:bitwidth_bcq}).


\subsection{Additional Accuracy Results}
%Table \ref{tab:lobcq_config} lists the various LOBCQ configurations and their corresponding bitwidths.
\begin{table}
\setlength{\tabcolsep}{4.75pt}
\begin{center}
\caption{\label{tab:lobcq_config} Various LO-BCQ configurations and their bitwidths.}
\begin{tabular}{|c||c|c|c|c||c|c||c|} 
\hline
 & \multicolumn{4}{|c||}{$L_b=8$} & \multicolumn{2}{|c||}{$L_b=4$} & $L_b=2$ \\
 \hline
 \backslashbox{$L_A$\kern-1em}{\kern-1em$N_c$} & 2 & 4 & 8 & 16 & 2 & 4 & 2 \\
 \hline
 64 & 4.25 & 4.375 & 4.5 & 4.625 & 4.375 & 4.625 & 4.625\\
 \hline
 32 & 4.375 & 4.5 & 4.625& 4.75 & 4.5 & 4.75 & 4.75 \\
 \hline
 16 & 4.625 & 4.75& 4.875 & 5 & 4.75 & 5 & 5 \\
 \hline
\end{tabular}
\end{center}
\end{table}

%\subsection{Perplexity achieved by various LO-BCQ configurations on Wikitext-103 dataset}

\begin{table} \centering
\begin{tabular}{|c||c|c|c|c||c|c||c|} 
\hline
 $L_b \rightarrow$& \multicolumn{4}{c||}{8} & \multicolumn{2}{c||}{4} & 2\\
 \hline
 \backslashbox{$L_A$\kern-1em}{\kern-1em$N_c$} & 2 & 4 & 8 & 16 & 2 & 4 & 2  \\
 %$N_c \rightarrow$ & 2 & 4 & 8 & 16 & 2 & 4 & 2 \\
 \hline
 \hline
 \multicolumn{8}{c}{GPT3-1.3B (FP32 PPL = 9.98)} \\ 
 \hline
 \hline
 64 & 10.40 & 10.23 & 10.17 & 10.15 &  10.28 & 10.18 & 10.19 \\
 \hline
 32 & 10.25 & 10.20 & 10.15 & 10.12 &  10.23 & 10.17 & 10.17 \\
 \hline
 16 & 10.22 & 10.16 & 10.10 & 10.09 &  10.21 & 10.14 & 10.16 \\
 \hline
  \hline
 \multicolumn{8}{c}{GPT3-8B (FP32 PPL = 7.38)} \\ 
 \hline
 \hline
 64 & 7.61 & 7.52 & 7.48 &  7.47 &  7.55 &  7.49 & 7.50 \\
 \hline
 32 & 7.52 & 7.50 & 7.46 &  7.45 &  7.52 &  7.48 & 7.48  \\
 \hline
 16 & 7.51 & 7.48 & 7.44 &  7.44 &  7.51 &  7.49 & 7.47  \\
 \hline
\end{tabular}
\caption{\label{tab:ppl_gpt3_abalation} Wikitext-103 perplexity across GPT3-1.3B and 8B models.}
\end{table}

\begin{table} \centering
\begin{tabular}{|c||c|c|c|c||} 
\hline
 $L_b \rightarrow$& \multicolumn{4}{c||}{8}\\
 \hline
 \backslashbox{$L_A$\kern-1em}{\kern-1em$N_c$} & 2 & 4 & 8 & 16 \\
 %$N_c \rightarrow$ & 2 & 4 & 8 & 16 & 2 & 4 & 2 \\
 \hline
 \hline
 \multicolumn{5}{|c|}{Llama2-7B (FP32 PPL = 5.06)} \\ 
 \hline
 \hline
 64 & 5.31 & 5.26 & 5.19 & 5.18  \\
 \hline
 32 & 5.23 & 5.25 & 5.18 & 5.15  \\
 \hline
 16 & 5.23 & 5.19 & 5.16 & 5.14  \\
 \hline
 \multicolumn{5}{|c|}{Nemotron4-15B (FP32 PPL = 5.87)} \\ 
 \hline
 \hline
 64  & 6.3 & 6.20 & 6.13 & 6.08  \\
 \hline
 32  & 6.24 & 6.12 & 6.07 & 6.03  \\
 \hline
 16  & 6.12 & 6.14 & 6.04 & 6.02  \\
 \hline
 \multicolumn{5}{|c|}{Nemotron4-340B (FP32 PPL = 3.48)} \\ 
 \hline
 \hline
 64 & 3.67 & 3.62 & 3.60 & 3.59 \\
 \hline
 32 & 3.63 & 3.61 & 3.59 & 3.56 \\
 \hline
 16 & 3.61 & 3.58 & 3.57 & 3.55 \\
 \hline
\end{tabular}
\caption{\label{tab:ppl_llama7B_nemo15B} Wikitext-103 perplexity compared to FP32 baseline in Llama2-7B and Nemotron4-15B, 340B models}
\end{table}

%\subsection{Perplexity achieved by various LO-BCQ configurations on MMLU dataset}


\begin{table} \centering
\begin{tabular}{|c||c|c|c|c||c|c|c|c|} 
\hline
 $L_b \rightarrow$& \multicolumn{4}{c||}{8} & \multicolumn{4}{c||}{8}\\
 \hline
 \backslashbox{$L_A$\kern-1em}{\kern-1em$N_c$} & 2 & 4 & 8 & 16 & 2 & 4 & 8 & 16  \\
 %$N_c \rightarrow$ & 2 & 4 & 8 & 16 & 2 & 4 & 2 \\
 \hline
 \hline
 \multicolumn{5}{|c|}{Llama2-7B (FP32 Accuracy = 45.8\%)} & \multicolumn{4}{|c|}{Llama2-70B (FP32 Accuracy = 69.12\%)} \\ 
 \hline
 \hline
 64 & 43.9 & 43.4 & 43.9 & 44.9 & 68.07 & 68.27 & 68.17 & 68.75 \\
 \hline
 32 & 44.5 & 43.8 & 44.9 & 44.5 & 68.37 & 68.51 & 68.35 & 68.27  \\
 \hline
 16 & 43.9 & 42.7 & 44.9 & 45 & 68.12 & 68.77 & 68.31 & 68.59  \\
 \hline
 \hline
 \multicolumn{5}{|c|}{GPT3-22B (FP32 Accuracy = 38.75\%)} & \multicolumn{4}{|c|}{Nemotron4-15B (FP32 Accuracy = 64.3\%)} \\ 
 \hline
 \hline
 64 & 36.71 & 38.85 & 38.13 & 38.92 & 63.17 & 62.36 & 63.72 & 64.09 \\
 \hline
 32 & 37.95 & 38.69 & 39.45 & 38.34 & 64.05 & 62.30 & 63.8 & 64.33  \\
 \hline
 16 & 38.88 & 38.80 & 38.31 & 38.92 & 63.22 & 63.51 & 63.93 & 64.43  \\
 \hline
\end{tabular}
\caption{\label{tab:mmlu_abalation} Accuracy on MMLU dataset across GPT3-22B, Llama2-7B, 70B and Nemotron4-15B models.}
\end{table}


%\subsection{Perplexity achieved by various LO-BCQ configurations on LM evaluation harness}

\begin{table} \centering
\begin{tabular}{|c||c|c|c|c||c|c|c|c|} 
\hline
 $L_b \rightarrow$& \multicolumn{4}{c||}{8} & \multicolumn{4}{c||}{8}\\
 \hline
 \backslashbox{$L_A$\kern-1em}{\kern-1em$N_c$} & 2 & 4 & 8 & 16 & 2 & 4 & 8 & 16  \\
 %$N_c \rightarrow$ & 2 & 4 & 8 & 16 & 2 & 4 & 2 \\
 \hline
 \hline
 \multicolumn{5}{|c|}{Race (FP32 Accuracy = 37.51\%)} & \multicolumn{4}{|c|}{Boolq (FP32 Accuracy = 64.62\%)} \\ 
 \hline
 \hline
 64 & 36.94 & 37.13 & 36.27 & 37.13 & 63.73 & 62.26 & 63.49 & 63.36 \\
 \hline
 32 & 37.03 & 36.36 & 36.08 & 37.03 & 62.54 & 63.51 & 63.49 & 63.55  \\
 \hline
 16 & 37.03 & 37.03 & 36.46 & 37.03 & 61.1 & 63.79 & 63.58 & 63.33  \\
 \hline
 \hline
 \multicolumn{5}{|c|}{Winogrande (FP32 Accuracy = 58.01\%)} & \multicolumn{4}{|c|}{Piqa (FP32 Accuracy = 74.21\%)} \\ 
 \hline
 \hline
 64 & 58.17 & 57.22 & 57.85 & 58.33 & 73.01 & 73.07 & 73.07 & 72.80 \\
 \hline
 32 & 59.12 & 58.09 & 57.85 & 58.41 & 73.01 & 73.94 & 72.74 & 73.18  \\
 \hline
 16 & 57.93 & 58.88 & 57.93 & 58.56 & 73.94 & 72.80 & 73.01 & 73.94  \\
 \hline
\end{tabular}
\caption{\label{tab:mmlu_abalation} Accuracy on LM evaluation harness tasks on GPT3-1.3B model.}
\end{table}

\begin{table} \centering
\begin{tabular}{|c||c|c|c|c||c|c|c|c|} 
\hline
 $L_b \rightarrow$& \multicolumn{4}{c||}{8} & \multicolumn{4}{c||}{8}\\
 \hline
 \backslashbox{$L_A$\kern-1em}{\kern-1em$N_c$} & 2 & 4 & 8 & 16 & 2 & 4 & 8 & 16  \\
 %$N_c \rightarrow$ & 2 & 4 & 8 & 16 & 2 & 4 & 2 \\
 \hline
 \hline
 \multicolumn{5}{|c|}{Race (FP32 Accuracy = 41.34\%)} & \multicolumn{4}{|c|}{Boolq (FP32 Accuracy = 68.32\%)} \\ 
 \hline
 \hline
 64 & 40.48 & 40.10 & 39.43 & 39.90 & 69.20 & 68.41 & 69.45 & 68.56 \\
 \hline
 32 & 39.52 & 39.52 & 40.77 & 39.62 & 68.32 & 67.43 & 68.17 & 69.30  \\
 \hline
 16 & 39.81 & 39.71 & 39.90 & 40.38 & 68.10 & 66.33 & 69.51 & 69.42  \\
 \hline
 \hline
 \multicolumn{5}{|c|}{Winogrande (FP32 Accuracy = 67.88\%)} & \multicolumn{4}{|c|}{Piqa (FP32 Accuracy = 78.78\%)} \\ 
 \hline
 \hline
 64 & 66.85 & 66.61 & 67.72 & 67.88 & 77.31 & 77.42 & 77.75 & 77.64 \\
 \hline
 32 & 67.25 & 67.72 & 67.72 & 67.00 & 77.31 & 77.04 & 77.80 & 77.37  \\
 \hline
 16 & 68.11 & 68.90 & 67.88 & 67.48 & 77.37 & 78.13 & 78.13 & 77.69  \\
 \hline
\end{tabular}
\caption{\label{tab:mmlu_abalation} Accuracy on LM evaluation harness tasks on GPT3-8B model.}
\end{table}

\begin{table} \centering
\begin{tabular}{|c||c|c|c|c||c|c|c|c|} 
\hline
 $L_b \rightarrow$& \multicolumn{4}{c||}{8} & \multicolumn{4}{c||}{8}\\
 \hline
 \backslashbox{$L_A$\kern-1em}{\kern-1em$N_c$} & 2 & 4 & 8 & 16 & 2 & 4 & 8 & 16  \\
 %$N_c \rightarrow$ & 2 & 4 & 8 & 16 & 2 & 4 & 2 \\
 \hline
 \hline
 \multicolumn{5}{|c|}{Race (FP32 Accuracy = 40.67\%)} & \multicolumn{4}{|c|}{Boolq (FP32 Accuracy = 76.54\%)} \\ 
 \hline
 \hline
 64 & 40.48 & 40.10 & 39.43 & 39.90 & 75.41 & 75.11 & 77.09 & 75.66 \\
 \hline
 32 & 39.52 & 39.52 & 40.77 & 39.62 & 76.02 & 76.02 & 75.96 & 75.35  \\
 \hline
 16 & 39.81 & 39.71 & 39.90 & 40.38 & 75.05 & 73.82 & 75.72 & 76.09  \\
 \hline
 \hline
 \multicolumn{5}{|c|}{Winogrande (FP32 Accuracy = 70.64\%)} & \multicolumn{4}{|c|}{Piqa (FP32 Accuracy = 79.16\%)} \\ 
 \hline
 \hline
 64 & 69.14 & 70.17 & 70.17 & 70.56 & 78.24 & 79.00 & 78.62 & 78.73 \\
 \hline
 32 & 70.96 & 69.69 & 71.27 & 69.30 & 78.56 & 79.49 & 79.16 & 78.89  \\
 \hline
 16 & 71.03 & 69.53 & 69.69 & 70.40 & 78.13 & 79.16 & 79.00 & 79.00  \\
 \hline
\end{tabular}
\caption{\label{tab:mmlu_abalation} Accuracy on LM evaluation harness tasks on GPT3-22B model.}
\end{table}

\begin{table} \centering
\begin{tabular}{|c||c|c|c|c||c|c|c|c|} 
\hline
 $L_b \rightarrow$& \multicolumn{4}{c||}{8} & \multicolumn{4}{c||}{8}\\
 \hline
 \backslashbox{$L_A$\kern-1em}{\kern-1em$N_c$} & 2 & 4 & 8 & 16 & 2 & 4 & 8 & 16  \\
 %$N_c \rightarrow$ & 2 & 4 & 8 & 16 & 2 & 4 & 2 \\
 \hline
 \hline
 \multicolumn{5}{|c|}{Race (FP32 Accuracy = 44.4\%)} & \multicolumn{4}{|c|}{Boolq (FP32 Accuracy = 79.29\%)} \\ 
 \hline
 \hline
 64 & 42.49 & 42.51 & 42.58 & 43.45 & 77.58 & 77.37 & 77.43 & 78.1 \\
 \hline
 32 & 43.35 & 42.49 & 43.64 & 43.73 & 77.86 & 75.32 & 77.28 & 77.86  \\
 \hline
 16 & 44.21 & 44.21 & 43.64 & 42.97 & 78.65 & 77 & 76.94 & 77.98  \\
 \hline
 \hline
 \multicolumn{5}{|c|}{Winogrande (FP32 Accuracy = 69.38\%)} & \multicolumn{4}{|c|}{Piqa (FP32 Accuracy = 78.07\%)} \\ 
 \hline
 \hline
 64 & 68.9 & 68.43 & 69.77 & 68.19 & 77.09 & 76.82 & 77.09 & 77.86 \\
 \hline
 32 & 69.38 & 68.51 & 68.82 & 68.90 & 78.07 & 76.71 & 78.07 & 77.86  \\
 \hline
 16 & 69.53 & 67.09 & 69.38 & 68.90 & 77.37 & 77.8 & 77.91 & 77.69  \\
 \hline
\end{tabular}
\caption{\label{tab:mmlu_abalation} Accuracy on LM evaluation harness tasks on Llama2-7B model.}
\end{table}

\begin{table} \centering
\begin{tabular}{|c||c|c|c|c||c|c|c|c|} 
\hline
 $L_b \rightarrow$& \multicolumn{4}{c||}{8} & \multicolumn{4}{c||}{8}\\
 \hline
 \backslashbox{$L_A$\kern-1em}{\kern-1em$N_c$} & 2 & 4 & 8 & 16 & 2 & 4 & 8 & 16  \\
 %$N_c \rightarrow$ & 2 & 4 & 8 & 16 & 2 & 4 & 2 \\
 \hline
 \hline
 \multicolumn{5}{|c|}{Race (FP32 Accuracy = 48.8\%)} & \multicolumn{4}{|c|}{Boolq (FP32 Accuracy = 85.23\%)} \\ 
 \hline
 \hline
 64 & 49.00 & 49.00 & 49.28 & 48.71 & 82.82 & 84.28 & 84.03 & 84.25 \\
 \hline
 32 & 49.57 & 48.52 & 48.33 & 49.28 & 83.85 & 84.46 & 84.31 & 84.93  \\
 \hline
 16 & 49.85 & 49.09 & 49.28 & 48.99 & 85.11 & 84.46 & 84.61 & 83.94  \\
 \hline
 \hline
 \multicolumn{5}{|c|}{Winogrande (FP32 Accuracy = 79.95\%)} & \multicolumn{4}{|c|}{Piqa (FP32 Accuracy = 81.56\%)} \\ 
 \hline
 \hline
 64 & 78.77 & 78.45 & 78.37 & 79.16 & 81.45 & 80.69 & 81.45 & 81.5 \\
 \hline
 32 & 78.45 & 79.01 & 78.69 & 80.66 & 81.56 & 80.58 & 81.18 & 81.34  \\
 \hline
 16 & 79.95 & 79.56 & 79.79 & 79.72 & 81.28 & 81.66 & 81.28 & 80.96  \\
 \hline
\end{tabular}
\caption{\label{tab:mmlu_abalation} Accuracy on LM evaluation harness tasks on Llama2-70B model.}
\end{table}

%\section{MSE Studies}
%\textcolor{red}{TODO}


\subsection{Number Formats and Quantization Method}
\label{subsec:numFormats_quantMethod}
\subsubsection{Integer Format}
An $n$-bit signed integer (INT) is typically represented with a 2s-complement format \citep{yao2022zeroquant,xiao2023smoothquant,dai2021vsq}, where the most significant bit denotes the sign.

\subsubsection{Floating Point Format}
An $n$-bit signed floating point (FP) number $x$ comprises of a 1-bit sign ($x_{\mathrm{sign}}$), $B_m$-bit mantissa ($x_{\mathrm{mant}}$) and $B_e$-bit exponent ($x_{\mathrm{exp}}$) such that $B_m+B_e=n-1$. The associated constant exponent bias ($E_{\mathrm{bias}}$) is computed as $(2^{{B_e}-1}-1)$. We denote this format as $E_{B_e}M_{B_m}$.  

\subsubsection{Quantization Scheme}
\label{subsec:quant_method}
A quantization scheme dictates how a given unquantized tensor is converted to its quantized representation. We consider FP formats for the purpose of illustration. Given an unquantized tensor $\bm{X}$ and an FP format $E_{B_e}M_{B_m}$, we first, we compute the quantization scale factor $s_X$ that maps the maximum absolute value of $\bm{X}$ to the maximum quantization level of the $E_{B_e}M_{B_m}$ format as follows:
\begin{align}
\label{eq:sf}
    s_X = \frac{\mathrm{max}(|\bm{X}|)}{\mathrm{max}(E_{B_e}M_{B_m})}
\end{align}
In the above equation, $|\cdot|$ denotes the absolute value function.

Next, we scale $\bm{X}$ by $s_X$ and quantize it to $\hat{\bm{X}}$ by rounding it to the nearest quantization level of $E_{B_e}M_{B_m}$ as:

\begin{align}
\label{eq:tensor_quant}
    \hat{\bm{X}} = \text{round-to-nearest}\left(\frac{\bm{X}}{s_X}, E_{B_e}M_{B_m}\right)
\end{align}

We perform dynamic max-scaled quantization \citep{wu2020integer}, where the scale factor $s$ for activations is dynamically computed during runtime.

\subsection{Vector Scaled Quantization}
\begin{wrapfigure}{r}{0.35\linewidth}
  \centering
  \includegraphics[width=\linewidth]{sections/figures/vsquant.jpg}
  \caption{\small Vectorwise decomposition for per-vector scaled quantization (VSQ \citep{dai2021vsq}).}
  \label{fig:vsquant}
\end{wrapfigure}
During VSQ \citep{dai2021vsq}, the operand tensors are decomposed into 1D vectors in a hardware friendly manner as shown in Figure \ref{fig:vsquant}. Since the decomposed tensors are used as operands in matrix multiplications during inference, it is beneficial to perform this decomposition along the reduction dimension of the multiplication. The vectorwise quantization is performed similar to tensorwise quantization described in Equations \ref{eq:sf} and \ref{eq:tensor_quant}, where a scale factor $s_v$ is required for each vector $\bm{v}$ that maps the maximum absolute value of that vector to the maximum quantization level. While smaller vector lengths can lead to larger accuracy gains, the associated memory and computational overheads due to the per-vector scale factors increases. To alleviate these overheads, VSQ \citep{dai2021vsq} proposed a second level quantization of the per-vector scale factors to unsigned integers, while MX \citep{rouhani2023shared} quantizes them to integer powers of 2 (denoted as $2^{INT}$).

\subsubsection{MX Format}
The MX format proposed in \citep{rouhani2023microscaling} introduces the concept of sub-block shifting. For every two scalar elements of $b$-bits each, there is a shared exponent bit. The value of this exponent bit is determined through an empirical analysis that targets minimizing quantization MSE. We note that the FP format $E_{1}M_{b}$ is strictly better than MX from an accuracy perspective since it allocates a dedicated exponent bit to each scalar as opposed to sharing it across two scalars. Therefore, we conservatively bound the accuracy of a $b+2$-bit signed MX format with that of a $E_{1}M_{b}$ format in our comparisons. For instance, we use E1M2 format as a proxy for MX4.

\begin{figure}
    \centering
    \includegraphics[width=1\linewidth]{sections//figures/BlockFormats.pdf}
    \caption{\small Comparing LO-BCQ to MX format.}
    \label{fig:block_formats}
\end{figure}

Figure \ref{fig:block_formats} compares our $4$-bit LO-BCQ block format to MX \citep{rouhani2023microscaling}. As shown, both LO-BCQ and MX decompose a given operand tensor into block arrays and each block array into blocks. Similar to MX, we find that per-block quantization ($L_b < L_A$) leads to better accuracy due to increased flexibility. While MX achieves this through per-block $1$-bit micro-scales, we associate a dedicated codebook to each block through a per-block codebook selector. Further, MX quantizes the per-block array scale-factor to E8M0 format without per-tensor scaling. In contrast during LO-BCQ, we find that per-tensor scaling combined with quantization of per-block array scale-factor to E4M3 format results in superior inference accuracy across models. 

\balance
\end{document}
\endinput
%%
%% End of file `sample-sigplan.tex'.
