\section{Related Work}
\label{section:literature}

The design of reserve systems for health rationing was first introduced by ____, where the authors developed a general theoretical framework with significant policy implications, particularly during and after the COVID-19 pandemic. They demonstrated that the classical deferred acceptance algorithm satisfies three fundamental axioms: eligibility compliance, non-wastefulness, and respect for priorities. Building on this framework, ____ introduced the first known algorithm that satisfies the fourth fundamental axiom: maximal cardinality. While their work laid the foundation for sequential reserve systems in health rationing, they did not address whether it is possible to design an algorithm that satisfies all four fundamental axioms when categories are processed sequentially. In this paper, we provide an affirmative answer to this question and introduce the Sequential Category Updating (SCU) rule, which not only achieves the four fundamental axioms but also nice properties in sequential reserve systems.

The precedence order over categories has been studied in various matching settings.  
____ investigate a matching problem with slot-specific priorities and introduce the concept of a precedence order for slots, which corresponds to categories in reserve matching. Their model is more general, as it considers agents with unit demand who are matched to branches with multiple slots, each with its own priority ranking over agents. A branch then selects agents by filling its slots sequentially according to a predetermined precedence order.  
%
____ examine the role of precedence order in admissions reserves, where different types of seats are allocated sequentially based on a specified precedence order. They show that either lowering the precedence of reserve seats at a school or increasing the school's reserve size weakly increases the number of assignments to the reserve group at that school. 
%
____ explore public misconceptions about reserve systems in affirmative action policies, highlighting how a lack of awareness about processing order leads individuals to conflate policies with different levels of affirmative action.  
%
However, achieving maximum cardinality is not the primary objective in these works.  



The reserve system has been studied in other contexts, and the healthcare rationing problem shares similarities with the literature on school choice with diversity goals. In the latter setting, schools reserve seats for different categories, such as gender, socioeconomic status, or neighborhood affiliation. ____ studied reserve systems in school choice settings, where each student belongs to a single category and a baseline priority is applied across all categories. Subsequent work expanded this framework to more general settings where agents may belong to multiple categories ____. However, a key difference lies in the nature of preferences: in healthcare rationing, agents have dichotomous preferences, whereas in school choice, students have strict preferences over schools.


More generally, this paper falls within the broader research area of matching with distribution constraints. Various forms of distributional constraints have been explored, including matching with minimum quotas ____, matching with regional quotas ____, and matching under complex constraints ____.


%%%%%%%%%%%%%%%%%%%%%%%%%
% Section: Preliminaries
%%%%%%%%%%%%%%%%%%%%%%%%%