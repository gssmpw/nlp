%% LyX 2.3.8 created this file.  For more info, see http://www.lyx.org/.
%% Do not edit unless you really know what you are doing.
\documentclass{article}
\usepackage[latin9]{inputenc}
\usepackage{prettyref}
\usepackage{float}
\usepackage{mathtools}
\usepackage{amsmath}
\usepackage{amsthm}
\usepackage{amssymb}
\usepackage{graphicx}
\usepackage{microtype}
\usepackage[unicode=true,
 bookmarks=false,
 breaklinks=false,pdfborder={0 0 1},backref=section,colorlinks=false]
 {hyperref}

\makeatletter
%%%%%%%%%%%%%%%%%%%%%%%%%%%%%% User specified LaTeX commands.
%%%%%%%% ICML 2025 EXAMPLE LATEX SUBMISSION FILE %%%%%%%%%%%%%%%%%



% Recommended, but optional, packages for figures and better typesetting:
\usepackage{subfigure}
% for professional tables

% hyperref makes hyperlinks in the resulting PDF.
% If your build breaks (sometimes temporarily if a hyperlink spans a page)
% please comment out the following usepackage line and replace
% \usepackage{icml2025} with \usepackage[nohyperref]{icml2025} above.



% Attempt to make hyperref and algorithmic work together better:
\newcommand{\theHalgorithm}{\arabic{algorithm}}

% Use the following line for the initial blind version submitted for review:
\usepackage[accepted]{./icml2025_style/icml2025}

% If accepted, instead use the following line for the camera-ready submission:
% \usepackage[accepted]{icml2025}


% For theorems and such
\usepackage{amsthm}


% if you use cleveref..
\usepackage[capitalize,noabbrev]{cleveref}


%%%%%%%%%%%%%%%%%%%%%%%%%%%%%%%%
% THEOREMS
%%%%%%%%%%%%%%%%%%%%%%%%%%%%%%%%
\theoremstyle{plain}
\newtheorem{theorem}{Theorem}[section]
\newtheorem{proposition}[theorem]{Proposition}\newtheorem{lemma}[theorem]{Lemma}\newtheorem{corollary}[theorem]{Corollary}\theoremstyle{definition}
\newtheorem{definition}[theorem]{Definition}\newtheorem{assumption}[theorem]{Assumption}\theoremstyle{remark}
\newtheorem{remark}[theorem]{Remark}

% Todonotes is useful during development; simply uncomment the next line
%    and comment out the line below the next line to turn off comments
%\usepackage[disable,textsize=tiny]{todonotes}
\usepackage[textsize=tiny]{todonotes}


\AtBeginDocument{%
\newrefformat{fig}{\hyperref[#1]{Fig.~\ref{#1}}}
}
\AtBeginDocument{%
\newrefformat{app}{\hyperref[#1]{App.~\ref{#1}}}
}

\makeatother

% The \icmltitle you define below is probably too long as a header.
% Therefore, a short form for the running title is supplied here:
\icmltitlerunning{A Multi-Scale Adaptive Theory of Feature Learning}

\begin{document}
\global\long\def\T{\mathsf{T}}%
\global\long\def\N{\mathcal{N}}%
\global\long\def\bR{\mathbb{R}}%
\global\long\def\tQ{\tilde{Q}}%
\global\long\def\I{\mathbb{I}}%
\global\long\def\const{\mathrm{const}}%
\global\long\def\tr{\mathrm{tr}}%
\global\long\def\llangle{\langle\!\langle}%
\global\long\def\rrangle{\rangle\!\rangle}%
\twocolumn[
\icmltitle{From Kernels to Features: A Multi-Scale Adaptive Theory of Feature Learning}

\icmlsetsymbol{equal}{*}

\begin{icmlauthorlist}
\icmlauthor{Noa Rubin}{equal,huji}
\icmlauthor{Kirsten Fischer}{equal,fzj,rwth_phd}
\icmlauthor{Javed Lindner}{equal,fzj,rwth,rwth_cosmo}
\icmlauthor{David Dahmen}{fzj}
\icmlauthor{Inbar Seroussi}{telaviv}
\icmlauthor{Zohar Ringel}{huji}
\icmlauthor{Michael Kr\"amer}{rwth_cosmo}
\icmlauthor{Moritz Helias}{fzj,rwth}
\end{icmlauthorlist}

\icmlaffiliation{fzj}{
Institute for Advanced Simulation (IAS-6), Computational and Systems Neuroscience, J�lich Research Centre, J�lich, Germany}
\icmlaffiliation{rwth}{Department of Physics, RWTH Aachen University, Aachen, Germany}
\icmlaffiliation{rwth_phd}{RWTH Aachen University, Aachen, Germany}
\icmlaffiliation{rwth_cosmo}{Institute for Theoretical Particle Physics and Cosmology, RWTH Aachen University, Aachen, Germany}
\icmlaffiliation{telaviv}{Department of Applied Mathematics, School of Mathematical Sciences, Tel-Aviv University, Tel-Aviv, Israel}
\icmlaffiliation{huji}{The Racah Institute of Physics, The Hebrew University of Jerusalem, Jerusalem, Israel}

\icmlcorrespondingauthor{Noa Rubin}{noa.rubin@mail.huji.ac.il}
%\icmlcorrespondingauthor{Kirsten Fischer}{ki.fischer@fz-juelich.de}
%\icmlcorrespondingauthor{Javed Linder}{javed.lindner@rwth-aachen.de}

\icmlkeywords{Machine Learning Theory}

\vskip 0.3in
]
\printAffiliationsAndNotice{\icmlEqualContribution} % leave blank if no need to mention equal contribution
\begin{abstract}
Theoretically describing feature learning in neural networks is crucial
for understanding their expressive power and inductive biases, motivating
various approaches. Some approaches describe network behavior after
training through a simple change in kernel scale from initialization,
resulting in a generalization power comparable to a Gaussian process.
Conversely, in other approaches training results in the adaptation
of the kernel to the data, involving complex directional changes to
the kernel. While these approaches capture different facets of network
behavior, their relationship and respective strengths across scaling
regimes remains an open question. This work presents a theoretical
framework of multi-scale adaptive feature learning bridging these
approaches. Using methods from statistical mechanics, we derive analytical
expressions for network output statistics which are valid across scaling
regimes and in the continuum between them. A systematic expansion
of the network's probability distribution reveals that mean-field
scaling requires only a saddle-point approximation, while standard
scaling necessitates additional correction terms. Remarkably, we find
across regimes that kernel adaptation can be reduced to an effective
kernel rescaling when predicting the mean network output of a linear
network. However, even in this case, the multi-scale adaptive approach
captures directional feature learning effects, providing richer insights
than what could be recovered from a rescaling of the kernel alone.
\end{abstract}

\section{Introduction}

A central phenomenon that is essential for explaining the power
of neural networks (NNs) is feature learning (FL), where networks
learn meaningful high-dimensional representations of the data \cite{Bengio13_1798}.
FL plays an increasingly important role in our ability to understand
and rationalize the behavior of large language models (LLMs). Sparse
autoencoders can extract so called monosemantic features from LLMs
that are given by a superposition of layer activations \cite{Bricken2023};
these features allow interpreting and even altering model behavior
\cite{Templeton2024}. Beyond interpretability, FL is essential for
efficient generalization with finite data, as it enhances informative
directions in the learned representations, reducing the complexity
of functions of these directions \cite{dandi2023_staircase,abbe2021_staircase,Paccolat_2021_staircase}.
Despite its significance, many open questions remain regarding the
theoretical mechanisms underlying the emergence of such feature directions.

A well-characterized case in NN theory is the limit of infinite-width
and finite sample size, where networks behave as Gaussian processes
(GPs) \cite{Mackay2003}, characterized by the neural network Gaussian
process (NNGP) kernel \cite{Neal96,Williams98_1203,Matthews18,Lee18}.
However, the NNGP does not capture FL, which emerges at finite network
width as well as in the proportional limit, where both network width
and sample size tend to infinity together \cite{Li21_031059}, or
in certain scaling regimes \cite{Yang24}. Multiple theoretical approaches
have emerged as to describe this phenomenon, yet there is no consensus
on how to characterize FL. A common approach is to study the change
of the network kernel, though the existing frameworks differ in their
predictions for this change.

One prominent class of theories, which are commonly referred to as
rescaling theories \cite{Li21_031059,Pacelli23_1497,Baglioni24_027301},
predicts that the average network output and variance can be described
by a rescaled NNGP kernel. Initially developed for linear networks
in the standard scaling regime\footnote{A scaling where readout weight variance scales as $1/\text{width}$.},
this framework surprisingly yields impressively accurate predictions
even in mean-field scaling\footnote{A scaling where readout weight variance scales as $\mathrm{1/\mathrm{\text{width}}^{2}}$.}.
Despite the strong FL in this regime, the average network outputs
can be obtained from an output kernel that is simply a rescaled NNGP
kernel.

\begin{figure}[H]
\centering\vskip 0.2in\includegraphics[width=1\columnwidth]{fig1_graphical_abstract_comp}

\caption{(a) The multi-scale adaptive theory bridges between rescaling and
adaptive theories of feature learning. Starting from the distribution
of network outputs for trained networks, the choice of order parameter
decides whether a rescaling (red) or adaptive (blue) theory is obtained.
The choice of order parameter recasts feature learning into either
a (i) low-dimensional minimization or (ii)\textbf{ }high-dimensional
minimization problem. An approximation of the multi-scale adaptive
theory in certain limits yields the result of the rescaling approach,
but in addition describes (iii) directional aspects of feature learning.
(b) Training (solid line) and test errors (dashed line) across scaling
regimes for different approaches. While standard scaling (green shaded
area) requires a one-loop approximation with fluctuation corrections
(Fluct. Corr.), a saddle-point or tree-level approximation (Saddle-Point)
is sufficient in mean-field scaling (orange shaded area). We show
results for the kernel rescaling theory by \cite{Li21_031059} as
reference (Rescaling). Parameters: $\gamma=1$, $P_{\text{train}}=80$,
$N=100$, $D=200$, $\kappa_{0}=1$, $P_{\text{test}}=10^{3}$, $g_{v}=g_{w}=0.5$,
$\Delta p=0.1$.}
\label{fig:graphical_abstract}
\end{figure}

%\vspace{.5cm}
\noindent\rule{\columnwidth}{0pt}

However, FL is often considered a structural phenomenon, such as the
case of Gabor filters \cite{Gabor1946,Rai2020} emerging in the latent
layers of convolutional neural networks \cite{Luan2018}. Thus the
expectation would be that the effect of FL on the output would be
directional as well. The rescaling result raises fundamental questions
about how learned features are represented in network outputs and
can be captured theoretically.

In contrast, adaptive theories\emph{ }of FL \cite{Roberts22,seroussi23_908,Bordelon23_114009,Fischer24_10761}
consider learned features, predicting that the kernel undergoes a
structural change and incorporates features explicitly. Consequentially,
these theories are able to predict phenomena in networks that stem
from FL such as a reduction in sample complexity -- the required
amount of samples to learn a given task-- relative to that of a GP
\cite{naveh2021a} as well as grokking \cite{Rubin24_iclr}. However,
adaptive theories are significantly more complex computationally than
rescaling theories, while yielding comparable predictions for quantities
such as the network loss. A fundamental open question remains: How
can two such different descriptions of FL be valid at the same time?

In this work, we address this pivotal question by systematically connecting
different FL theories and uncovering their underlying relationships.
To this end, we choose the simplest non-trivial model where both frameworks
are valid. Our main contributions are:
\begin{itemize}
\item Using methods from statistical mechanics, we recast the theoretical
description of the posterior distribution of network outputs into
a minimization problem with respect to a parameter which we call the
\textquotedblleft order parameter\textquotedblright . We find that
different theories result from different choices of order parameters,
in particular with regard to their dimensionality (see \prettyref{fig:graphical_abstract}a).
\item We derive a multi-scale adaptive theory that is valid across the full
range of scaling regimes, from mean-field to standard scaling, and
which allows us to systematically include finite-width corrections
(see \prettyref{fig:graphical_abstract}b). The latter point goes
beyond previous adaptive approaches that are only valid for certain
scaling regimes.
\item We show that for the mean network output the multi-scale adaptive
theory can be approximated in certain limits to yield an effective
rescaling of the kernel. This explains why certain FL phenomena do
not appear in rescaling theories, reconciling the two different perspectives.
\item While rescaling theories approximate the mean network output well
(especially when NNs have the same sample complexity as GPs as in
linear networks), we demonstrate for mean-field scaling that the output
covariance does not behave as predicted by rescaling theories but
rather clearly adapts to specific directions relevant to the task.
These changes are correctly predicted by our multi-scale adaptive
theory.
\end{itemize}
Overall, our findings suggest that a comprehensive understanding of
FL requires moving beyond kernel rescaling towards high-dimensional
kernel adaptation.

\section{Related works}

The limit of infinite network width and finite amount of training
data has been studied extensively, yielding among others the NNGP
kernel \cite{Neal1995BayesianLF,Williams98_1203,Lee18,Matthews18,avidan2024}.
This theory relates network behavior at initialization to training
dynamics \cite{Poole16_3360,Pennington17_04735,Schoenholz17_01232,Xiao18_05393}.
However, the NNGP cannot explain the often superior performance of
finite-width networks \cite{Li15_196,Chizat19_neurips,Lee20_ad086f59,Aitchison2020,Refinetti21_8936},
requiring the inclusion of finite-width effects in theories of FL.

Describing FL in neural networks in a Bayesian framework has lead
to concurrent views: kernel rescaling \cite{Li21_031059,Li2022,Pacelli23_1497,bassetti2024,Baglioni24_027301}
and kernel adaptation \cite{Naveh21_NeurIPS,seroussi23_908,Fischer24_10761,Rubin24_iclr}.
These differ in the choice of order parameters considered and in consequence
also in the explained phenomena.

Beyond these two views, various works study other aspects of networks
in the Bayesian framework: \citet{Canatar22_ieee} investigate experimentally
the effect of hyperparameters on adaptive FL. \citet{ZavatoneVeth21_NeurIPS_I}
study properties of the network prior, whereas we focus on the network
posterior. \citet{Hanin23} obtain a rigorous non-asymptotic description
of deep linear networks in terms of Meijer-G functions. \citet{Cui23_6468}
exploit the Nishimori conditions that hold for Bayes-optimal inference,
where student and teacher have the same architecture and the student
uses the teacher's weight distribution as a prior; the latter is assumed
Gaussian i.i.d., which allows them to use the Gaussian equivalence
principle \cite{Goldt20_14709} to obtain closed-form solutions.

Our work is distinct from perturbative approaches such as \cite{Antognini19_arxiv,Naveh21_064301,Cohen21_023034,Roberts22,Hanin24,Halverson21_035002}
for the Bayesian setting or \cite{Dyer20_ICLR,Huang20_4542,Aitken20_06687,Roberts22,Bordelon23_114009,buzaglo2024}
for gradient-based training that use the strength of non-Gaussian
cumulants of the outputs as an expansion parameter; however, we perform
an expansion in terms of fluctuations around the mean outputs, which
is able to capture phenomena that escape perturbative treatments,
such as phase transitions; this technique corresponds to an infinite
resummation of perturbative terms.

Another line of work focuses on the dynamics of FL: \citet{Saxe14_iclr}
derive exact learning dynamics for deep linear networks, while \cite{Bordelon23_114009}
use dynamical mean-field theory to describe network behavior in the
early stages of training of gradient descent training in different
scaling regimes while we consider networks at equilibrium. \citet{Yang20_14522}
consider the effect of network training dynamics and learning rate
scales in networks. \citet{day2024} study the effect of weight initialization
on generalization and training speed. A different viewpoint considers
spectral properties of FL \cite{Simon23,Yang24} as well as investigating
the effects of learned representations directly \cite{Petrini23_114003}.
\citet{Maillard24} derive polynomial scaling limits of the required
amount of training data.

\section{Single hidden-layer linear network\label{sec:setup}}

We consider the following network architecture
\begin{align}
h_{\alpha}=Vx_{\alpha},\;f_{\alpha}=w^{\T}h_{\alpha},\;y_{\alpha} & =f_{\alpha}+\xi,\label{eq:network_architecture}
\end{align}
where $\xi$ is Gaussian regularization noise $\xi\stackrel{\text{i.i.d.}}{\sim}\N(0,\kappa)$.
We consider $P$ tuples of training data $\mathcal{D}=\{(x_{\alpha},y_{\alpha})\}_{1\le\alpha\le P}$
with $x_{\alpha}\in\mathbb{\bR}^{D}$ and $y_{\alpha}\in\mathbb{\bR}$
as well as an unseen test point $(x_{*},y_{*})$ denoted by '$*$'.
Here $f_{\alpha}\in\bR$ denotes the scalar network output. We study
the Bayesian setting with Gaussian priors on the readin weights $V\in\bR^{N\times D}$
as $V_{ij}\sim\mathcal{N}(0,g_{v}/D)$ and the readout weights $w\in\bR^{N}$
as $w_{i}\sim\mathcal{N}(0,g_{w}/N^{\gamma})$. Here, we differentiate
between two cases: (a) standard scaling for $\gamma=1$ and (b) mean-field
scaling for ${\gamma=2}$. Accordingly, we scale the regularization
noise as $\kappa=\kappa_{0}N^{1-\gamma}$ so that it does not dominate
the network output in mean-field scaling. To keep the notation concise,
we use the shorthands $f_{\mathcal{D}}=(f_{\alpha})_{1\le\alpha\le P}$,
$X=(x_{\alpha})_{1\le\alpha\le P}$ and $y=(y_{\alpha})_{1\le\alpha\le P}$
in the following. Further, summations over repeated indices are implied
$V_{kl}x_{l}\equiv\sum_{l=1}^{N}V_{kl}x_{l}$.

\section{Multi-scale adaptive feature learning theory\label{sec:multi_scale_approach}}

 In this section we compute the network posterior on the test data
$(x_{*},y_{*})$ by conditioning on the training data $\mathcal{D}$
and derive a set of self-consistency equations for the average discrepancies
$\langle\Delta\rangle$ between labels $y$ and mean posterior network
outputs $\langle f_{\mathcal{D}}\rangle$ on the training data. This
description on the level of the discrepancies yields a high-resolution
picture of network behavior: it allows us to explain kernel rescaling
results in the proportional limit as well as predict directional aspects
of FL.

\subsection{Predictor statistics of the neural network\label{subsec:predictor_stats}}

We are interested in the Bayesian network posterior for the network
output on training points $f_{\mathcal{D}}$ and a test point $f_{*}$,
which corresponds to training the network with Langevin stochastic
gradient descent (LSGD) until convergence \cite{Naveh21_064301} (see
\prettyref{app:tasks_training} for details). We denote the joint
vector of outputs as ${f\coloneqq(f_{\mathcal{D}},f_{\ast})\in\mathbb{R}^{P+1}}$.
Following along the lines of \citet{Segadlo22_103401}, we may write
the joint distribution as
\begin{align}
 & p(f_{\mathcal{D}},f_{\ast},y)=\N(y\vert f_{\mathcal{D}},\kappa_{0}N^{1-\gamma})\label{eq:joint_output}\\
 & \quad\times\int\mathrm{d}\tilde{f}_{\mathcal{D}}\int\mathrm{d}\tilde{f}_{\ast}\,\exp\big[-i\tilde{f}^{\T}f+W(i\tilde{f}_{\mathcal{D}},i\tilde{f}_{\ast})\big],\nonumber 
\end{align}
with $\tilde{f}\coloneqq(\tilde{f}_{\mathcal{D}},\tilde{f_{*}})$
the conjugate fields to ${f=(f_{\mathcal{D}},f_{\ast})}$. The cumulant-generating
function $W(i\tilde{f}_{\mathcal{D}},i\tilde{f_{*}})$ of the network
prior is given by
\begin{align}
 & W(i\tilde{f}_{\mathcal{D}},i\tilde{f}_{\ast})\\
 & =\ln\left\langle \exp\left(\sum_{a=1}^{P+1}i\tilde{f_{a}}w_{j}h_{aj}\right)\right\rangle _{w_{j},h_{aj}}\label{eq:def_W_main}\\
 & =-\frac{N}{2}\Big(\ln\det\left[\mathbb{I}+\frac{g_{w}}{N^{\gamma}}C^{(xx)}\tilde{f}\,\tilde{f}^{\T}\right]\Big),
\end{align}
where the average $\langle\dots\rangle_{w_{j},h_{j}}$ is over the
prior distribution on the network parameters and the hidden representations
$h_{aj}\overset{\text{i.i.d. over j}}{\sim}\N(0,C^{(xx)})$ with $C^{(xx)}=g_{v}/D\,XX^{\T}\in\bR^{(P+1)\times(P+1)}$.
The detailed derivation can be found in \prettyref{app:theory}. The
statistics of the conjugate fields $(\tilde{f}_{\mathcal{D}},\tilde{f_{*}})$
are directly linked to the statistics of the network predictors $f_{\mathcal{D}}$
via the output discrepancies ${\Delta=y-f_{\mathcal{D}}}$ on the
training data as in \eqref{eq:mean_discrepancy}
\begin{equation}
\langle\Delta\rangle=-i\kappa_{0}N^{1-\gamma}\langle\tilde{f}_{\mathcal{D}}\rangle.\label{eq:expDelta_exptilf}
\end{equation}
To obtain the statistics of the conjugate variables $(\tilde{f}_{\mathcal{D}},\tilde{f_{*}})$
and thus also of the network outputs $(f_{\mathcal{D}},f_{\ast})$,
we define a conditional cumulant-generating function $\mathcal{W}(k,j_{\ast}|y)\coloneqq\ln\langle\exp(j_{\ast}f_{\ast}+ik^{\T}\tilde{f}_{\mathcal{D}})\rangle_{f_{\ast},\tilde{f}_{\mathcal{D}}}$
which takes the form 
\begin{align}
\mathcal{W}(k,j_{\ast}\vert y) & =\ln\int\mathrm{d}\tilde{f}\exp\big[ik^{\T}\tilde{f}_{\mathcal{D}}+\mathcal{S}(\tilde{f}_{\mathcal{D}},j_{\ast}\vert y)\big],\label{eq:cum_gen_posterior}\\
\mathcal{S}(\tilde{f}_{\mathcal{D}},j_{\ast}\vert y) & =-iy^{\T}\tilde{f}_{\mathcal{D}}-\frac{\kappa_{0}}{2}N^{1-\gamma}\tilde{f}_{\mathcal{D}}^{\T}\tilde{f}_{\mathcal{D}}\\
 & \phantom{=}+W(i\tilde{f}_{\mathcal{D}},j_{\ast}).\nonumber 
\end{align}
Here, we introduced source terms $(k,j_{\ast})$ with $k\in\mathbb{R}^{P}$,
$j_{*}\in\mathbb{R}$ from which we can obtain the statistics of $(\tilde{f}_{\mathcal{D}},f_{*})$
as their derivatives. On the training points, we have
\begin{align}
\langle\tilde{f}_{\mathcal{D}}\rangle & =-i\nabla_{k}\mathcal{W}\vert_{k,j_{\ast}=0},\:\llangle\tilde{f}_{\mathcal{D}}\tilde{f}_{\mathcal{D}}^{\T}\rrangle=-\nabla_{k}^{2}\mathcal{W}\vert_{k,j_{\ast}=0},
\end{align}
with $\llangle\dots\rrangle$ being the covariance. On the test point,
we get
\begin{align}
\langle f_{\ast}\rangle & =\partial_{j_{\ast}}\mathcal{W}\vert_{k,j_{\ast}=0},\;\llangle f_{\ast}^{2}\rrangle=\partial_{j_{\ast}}^{2}\mathcal{W}\vert_{k,j_{\ast}=0}.
\end{align}
However, the cumulant-generating function $\mathcal{W}(k,j_{\ast}\vert y)$
in \prettyref{eq:cum_gen_posterior} in general does not have an analytical
solution. Instead, we perform a systematic expansion in terms of fluctuations
of the network output using its Legendre transform
\begin{equation}
\Gamma(\bar{\tilde{f}},j_{\ast}\vert y)=\text{extr}_{k}\,ik^{\T}\bar{\tilde{f}}-\mathcal{W}(k,j_{\ast}\vert y),\label{eq:legendre_transform}
\end{equation}
where we take the extremum with respect to $k$. This transform is
a function of the mean conjugate field ${\bar{\tilde{f}}=\langle\tilde{f}_{\mathcal{D}}\rangle}$
(we drop the index $\mathcal{D}$ for readability), defined self-consistently
by the stationary condition given by
\begin{equation}
\partial_{\bar{\tilde{f}}}\Gamma(\bar{\tilde{f}},j_{\ast}\vert y)\overset{!}{=}0.\label{eq:self_consistent_f_tilde}
\end{equation}
In that way, studying the Legendre transform $\Gamma(\bar{\tilde{f}},j_{\ast}\vert y)$
is a natural way of constructing a minimization problem that yields
the quantity we are interested in. It recasts the problem of computing
the statistics of the posterior, which is the stationary solution
of the stochastic minimization problem described by the LSGD training,
into an effective deterministic optimization problem of $\Gamma$
with regard to the mean discrepancies $\bar{\tilde{f}}$; intuitively,
we may therefore think of $\Gamma$ as an effective loss function
that explicitly only depends on the mean discrepancies $\langle\Delta\rangle\propto\bar{\tilde{f}}$,
but implicitly takes fluctuations of $\Delta$ into account. Moreover,
it allows computing corrections to the mean network outputs in a systematic
manner, building on a broad foundation of methods from statistical
physics \cite{ZinnJustin96,Helias20_970}.

Using the relationship between first-order parametric derivatives
of the Legendre transform $\Gamma(\bar{\tilde{f}},j_{\ast}\vert y)$
and the cumulant-generating function $\mathcal{W}(k,j_{\ast}\vert y)$,
we obtain
\begin{equation}
\begin{alignedat}{1}\langle f_{*}\rangle & =-\partial_{j_{\ast}}\Gamma\vert_{k,j^{*}=0}.\end{alignedat}
\label{eq:mean_predictor}
\end{equation}
 In the following sections, we consider approximations of the Legendre
transform $\Gamma(\bar{\tilde{f}},j_{\ast}\vert y)$ for different
scaling regimes, and use these to systematically determine the network
output statistics.

\subsection{Saddle-point approximation in mean-field scaling\label{subsec:mf_scaling}}

\begin{figure*}[t]
\vskip 0.2in
\begin{centering}
\includegraphics[width=1\textwidth]{scatter_discrepancies_mean_field_scaling}\caption{(a) Training discrepancies $\langle\Delta\rangle=y-\langle f_{\mathcal{D}}\rangle$
and (b) test discrepancies $\langle\Delta_{\ast}\rangle=y_{\ast}-\langle f_{\ast}\rangle$
on an Ising task in mean-field scaling. We show theoretical values
for both NNGP and tree-level against empirical results, where the
gray line marks the identity. In contrast to the NNGP, the tree-level
approximation accurately matches the empirical values. Parameters:
$\gamma=2$, $P_{\text{train}}=80$, $N=100$, $D=200$, $\kappa_{0}=1$,
$P_{\text{test}}=10^{3}$, $g_{v}=g_{w}=0.5$, $\Delta p=0.1$.}
\label{fig:scatter_mean_field}
\par\end{centering}
\vskip -0.2in
\end{figure*}
In mean-field scaling, the exponent $\mathcal{S}$ of the cumulant-generating
function in \prettyref{eq:cum_gen_posterior} scales linearly with
the network width, while the fluctuations of the network output scale
as $\llangle ff^{\T}\rrangle\sim1/N$ and become negligible. Thus,
we can perform a saddle-point approximation for the integral in \prettyref{eq:cum_gen_posterior}
and obtain the tree-level approximation of the Legendre transform
\cite{Helias20_970} by replacing $\tilde{f}_{\mathcal{D}}\mapsto\bar{\tilde{f}}$
in the exponent $\mathcal{S}$, yielding
\begin{equation}
\Gamma(\bar{\tilde{f}},j_{\ast}\vert y)\approx\Gamma_{\text{TL}}(\bar{\tilde{f}},j_{\ast}\vert y)=-\mathcal{S}(\bar{\tilde{f}},j_{\ast}\vert y).
\end{equation}
 We derive this result more rigorously in \prettyref{app:theory}
using a large deviation principle \cite{Touchette09}. From the stationary
condition in \prettyref{eq:self_consistent_f_tilde}, we obtain a
self-consistency equation for $\bar{\tilde{f}}$ given by
\begin{align}
\bar{\tilde{f}} & =-i\left(\kappa_{0}N^{-1}\mathbb{I}+C_{\text{TL}}\big(\bar{\tilde{f}}\big)\,C_{\mathcal{D}\mathcal{D}}^{(xx)}\right)^{-1}y,\label{eq:self_cons_tl}\\
C_{\text{TL}}\big(\bar{\tilde{f}}\big) & =g_{w}N^{-1}\left[\mathbb{I}+\frac{g_{w}}{N^{2}}C_{\mathcal{D}\mathcal{D}}^{(xx)}\bar{\tilde{f}}\bar{\tilde{f}}^{\T}\right]^{-1},
\end{align}
where $C_{\mathcal{D}\mathcal{D}}^{(xx)}\in\mathbb{R}^{P\times P}$
refers to the training data submatrix of $C^{(xx)}$. In the remainder
of this section, $\bar{\tilde{f}}$ refers to the solution of \prettyref{eq:self_cons_tl}.
We obtain the discrepancies on the training points as
\begin{align}
\langle\Delta\rangle_{\text{TL}} & =\kappa_{0}\left(\kappa_{0}\mathbb{I}+C_{\text{TL}}\big(\bar{\tilde{f}}\big)\,C_{\mathcal{D}\mathcal{D}}^{(xx)}\right)^{-1}y.
\end{align}
For the test point, we get
\begin{align}
 & \langle f_{\ast}\rangle_{\text{TL}}\\
 & =\left[C_{\text{TL}}\big(\bar{\tilde{f}}\big)\,C_{\mathcal{D}\ast}^{(xx)}\right]^{\T}\left(\kappa_{0}N^{1-\gamma}\mathbb{I}+C_{\text{TL}}\big(\bar{\tilde{f}}\big)\,C_{\mathcal{D}\mathcal{D}}^{(xx)}\right)^{-1}y\nonumber 
\end{align}
where $C_{\mathcal{D}*}^{(xx)}:=\left\{ g_{v/D}x_{\alpha}\cdot x_{*}\right\} _{1\leq\alpha\leq P}\in\mathbb{R}^{P\times1}$,
recovering results by \citet{seroussi23_908}. In \prettyref{fig:scatter_mean_field},
we compare theoretical values for training and test discrepancies
against empirical measurements for networks trained on a linearly
separable Ising task (see \prettyref{app:tasks_training} for details).
Comparing to the NNGP as a baseline, we find that, while the NNGP
fails to match network outputs, the multi-scale adaptive theory accurately
predicts the values observed in trained networks.

\subsection{Fluctuation corrections in standard scaling\label{subsec:standard_scaling}}

In standard scaling, output fluctuations are not scaled down by
the network width $N$ and instead become non-negligible. To obtain
the leading-order fluctuation corrections, we expand the exponent
$\mathcal{S}$ of the cumulant-generating function around its saddle-point
$\bar{\tilde{f}}$ to second order as
\begin{equation}
\mathcal{S}\big(\tilde{f}_{\mathcal{D}},j_{\ast}\vert y\big)\approx\mathcal{S}\big(\bar{\tilde{f}},j_{\ast}\big)+\frac{1}{2}\big(\tilde{f}_{\mathcal{D}}-\bar{\tilde{f}}\big)^{\T}\mathcal{S}^{(2)}\big(\tilde{f}_{\mathcal{D}}-\bar{\tilde{f}}\big),
\end{equation}
where $\mathcal{S}^{(2)}$ denotes the Hessian of $\mathcal{S}(\tilde{f},j_{\ast}\vert y)$
with respect to $\tilde{f}_{\mathcal{D}}$ at the saddle-point $\bar{\tilde{f}}$.
Calculating the Gaussian integral in \prettyref{eq:cum_gen_posterior},
we obtain what is known as the one-loop approximation of the Legendre
transform \cite{Helias20_970} as
\begin{align}
\Gamma_{\text{1-Loop}}(\bar{\tilde{f}},j_{\ast}\vert y) & =-\mathcal{S}(\bar{\tilde{f}},j_{\ast})-\frac{1}{2}\log\det(-\mathcal{S}^{(2)}).\label{eq:gamma_one_loop}
\end{align}
The self-consistency equation for $\bar{\tilde{f}}_{\mathcal{D}}$
from the stationary condition in \prettyref{eq:self_consistent_f_tilde}
is then given by
\begin{equation}
\bar{\tilde{f}}_{\alpha}=\left[A\big(\bar{\tilde{f}}\big)\right]_{\alpha\beta}^{-1}\left[-iy_{\beta}-\frac{1}{2}\left[\mathcal{S}^{(2)}\right]_{\delta\epsilon}^{-1}\mathcal{S}_{\epsilon\delta\beta}^{(3)}\bigg\vert_{j^{*}=0}\right],\label{eq:self_cons_oneloop}
\end{equation}
where $A\big(\bar{\tilde{f}}\big)=\kappa_{0}\mathbb{I}+C_{\text{TL}}\big(\bar{\tilde{f}}\big)\,C_{\mathcal{D}\mathcal{D}}^{(xx)}$
and $\mathcal{S}^{(n)}$ refers to the $n$-th derivative of the exponent
with respect to $\tilde{f}_{\mathcal{D}}$ evaluated at $\bar{\tilde{f}}$
(see \prettyref{app:tree-level_app} for details). In the remainder
of this section, $\bar{\tilde{f}}$ refers to the self-consistent
solution of \prettyref{eq:self_cons_oneloop}, which is not necessarily
the same as the one of \prettyref{eq:self_cons_tl} in the previous
section. This yields for the training discrepancies $\langle\Delta\rangle_{\text{1-Loop}}=i\kappa_{0}\bar{\tilde{f}}$
as in \eqref{eq:expDelta_exptilf} and for the test point from \eqref{eq:mean_predictor}
\begin{align}
 & \langle f_{\ast}\rangle_{\text{1-Loop}}\\
 & =\kappa_{0}^{-1}C_{\ast\mathcal{D}}^{(xx)}C_{\text{TL}}\big(\bar{\tilde{f}}\big)^{\T}\langle\Delta\rangle+\frac{1}{2}(\mathcal{S}^{(2)})_{\beta\alpha}^{-1}\mathcal{S}_{\alpha\beta\ast}^{(3)}\bigg\vert_{j^{*}=0}.\nonumber 
\end{align}
In the next section, we will see how these expressions reduce to a
kernel rescaling theory in the proportional limit ${N\propto P\rightarrow\infty}$,
which in linear networks we refer to as one-loop simplified in \prettyref{fig:scatter_standard_scaling_nngp},
where we compare theoretical predictions to empirical measurements
on the Ising task. We show results for the multi-scale adaptive theory
presented here as well as the rescaling theory by \citet{Li21_031059},
which was derived for the standard scaling regime. Due to the weak
FL in standard scaling, all theories match the network behavior relatively
well. However, by taking the NNGP as a reference, the differences
between the theories become discernable: The tree-level solution shows
deviations from the other solutions, predicting overly small test
errors compared to the one-loop solution and compared to empirics.
Furthermore, predictions of the one-loop solution agree to those of
the rescaling theory by \citet{Li21_031059}.
\begin{figure*}[t]
\vskip 0.2in
\begin{centering}
\includegraphics[width=1\textwidth]{scatter_discrepancies_summary_standard_scaling}
\par\end{centering}
\begin{centering}
\caption{(a) Training discrepancies $\langle\Delta\rangle=y-\langle f_{\mathcal{D}}\rangle$
and (b) test discrepancies $\langle\Delta_{\ast}\rangle=y_{\ast}-\langle f_{\ast}\rangle$
on an Ising task in standard scaling. Upper row: theoretical values
for different theories against empirical results; gray line marks
the identity. Lower row: difference of theoretical values to the NNGP
as a baseline against NNGP predictions, indicating small-scale differences
between the different approaches. Results of the kernel approach by
\citet{Li21_031059} shown as reference (LS). Parameters: $\gamma=1$,
$P_{\text{train}}=80$, $N=100$, $D=200$, $\kappa_{0}=0.4$, $P_{\text{test}}=10^{3}$,
$g_{v}=0.5$, $g_{w}=0.2$, $\Delta p=0.1$.}
\label{fig:scatter_standard_scaling_nngp}
\par\end{centering}
\vskip -0.2in
\end{figure*}

The one-loop solution takes into account leading-order fluctuation
corrections. The latter vanish in mean-field scaling, so one expects
the one-loop approximation to converge to the tree-level result in
this scaling regime. We show this explicitly in \prettyref{fig:graphical_abstract}b,
where we demonstrate how the different theories transition between
the two scaling regimes by scaling $\kappa_{0}\mapsto\kappa_{0}/\chi$
and $g_{w}\mapsto g_{w}/\chi$ with $0.1/N<1/\chi<10$ determining
the scale of fluctuations. As expected, train and test errors decrease
for increasing FL in the mean-field regime. Due to non-negligible
fluctuations, the tree-level and one-loop solutions differ in standard
scaling. When further increasing the fluctuations scale, even the
one-loop solution does not accurately predict empirical measurements
anymore since this regime requires fluctuation corrections beyond
first order. In principle, the multi-scale adaptive approach allows
computing these higher-order correction terms \cite{Helias20_970}.
When decreasing the fluctuations towards the mean-field scaling regime,
the one-loop solution converges to the tree-level solution. Notably,
the here presented multi-scale adaptive approach accurately predicts
train and test errors across both scaling regimes, including the intermediate
regime.

\section{Kernel rescaling theory as an approximation of the multi-scale adaptive
theory\label{sec:connecting_approaches}}

Existing rescaling theories \cite{Li21_031059,Li2022,Pacelli23_1497,bassetti2024,Baglioni24_027301}
and adaptive theories \cite{Naveh21_NeurIPS,seroussi23_908,Fischer24_10761,Rubin24_iclr}
make both qualitatively and quantitatively different predictions regarding
network behavior. On the one hand, rescaling approaches predict that
the mean network output is equivalent to that obtained by a rescaled
NNGP kernel. On the other hand, adaptive approaches such as the multi-scale
adaptive theory presented here, as well as other existing approaches,
predict that the kernel adapts to the data in a richer manner, showing
changes in specific directions that are determined by the training
data's statistics. While these approaches are quite different, in
this section we expose the tight relation between them in two respects.
First, (i) we show that the adaptive and the rescaling approach can both
be derived from the same starting point; the expression for the joint
distribution of the network outputs \prettyref{eq:joint_output}.
Second, (ii) we show that the adaptive approach in the proportional
limit ${N\propto P\rightarrow\infty}$ can be approximated by a kernel
rescaling for the mean outputs.

The differences between the two viewpoints stem from different choices
of the order parameter used in the approximation of the posterior,
utilizing either a saddle-point approximation or approximations including
fluctuation corrections. Specifically, with point (i), we show in
\prettyref{app:kernel_scaling_app} that the equations obtained by
\citet{Li21_031059} can be rederived from \prettyref{eq:joint_output}
by marginalizing over the hidden representations $h$ in \eqref{eq:def_W_main}
and performing a change of variables so that the posterior is a function
of a single scalar order parameter $Q:=\|w\|^{2}$. A saddle-point
approximation with respect to this variable yields a self-consistent
equation for $Q$ and consequently expressions for the predictor statistics
on test and training points, such as the mean and fluctuations. As
the order parameter is scalar here, it is limited to describing scalar
changes to the kernel.

Conversely, the choice of the high-dimensional order parameter in
the multi-scale adaptive approach, which in mean-field scaling reproduces
equivalent equations to those of the approach in \cite{seroussi23_908},
results in structural changes to the kernel. Notably, the choice of
a high-dimensional order parameter results in the need to correct
for fluctuations that arise in standard scaling, requiring us to go
beyond the saddle-point approximation by using fluctuation corrections.

Surprisingly, as we have shown in the previous section, for a linear
network and considering only the mean predictor, the multi-scale adaptive
approach converges to that of the rescaling one, even though they
have qualitatively different kernels. This motivates point (ii) of
this section, showing that for a linear network in the proportional
limit ${N\propto P\rightarrow\infty}$, regardless of the initial
choice of order parameter, the mean network output can be obtained
from kernel regression \cite{WilliamsRasmussen06} with a rescaled
NNGP kernel.

In the kernel rescaling case, the predictor for the mean output is
obtained by replacing the NNGP kernel $K_{\text{NNGP}}=g_{w}N^{1-\gamma}C_{\mathcal{D\mathcal{D}}}^{(xx)}$
with a rescaled kernel 
\begin{equation}
K_{\text{rescaling}}=Q/(g_{w}N^{1-\gamma})\,K_{\text{NNGP}}.
\end{equation}
For the multi-scale adaptive approach presented here, the output statistics
in mean-field scaling are obtained by using the kernel 
\begin{equation}
K_{\text{adaptive, TL }}=\left[\mathbb{I}+\frac{g_{w}}{N^{\gamma}}C_{\mathcal{D\mathcal{D}}}^{\left(xx\right)}\bar{\tilde{f}}_{\text{TL}}\bar{\tilde{f}}_{\text{TL}}^{\T}\right]^{-1}K_{\text{NNGP}}.
\end{equation}
The appearing matrix product allows a non-trivial change of the NNGP
kernel in certain meaningful directions, yielding additional insights.
However, we derive an equivalent equation for the mean predictor by
simplifying \prettyref{eq:def_W_main} using the matrix-determinant-lemma,
which yields the mean output from a rescaled NNGP kernel given by
\begin{equation}
K_{\text{rescaling, TL }}=Q_{\text{TL}}\big(\bar{\tilde{f}}\big)/(g_{w}N^{1-\gamma})\,K_{\text{NNGP}},
\end{equation}
where $Q_{\text{TL}}\big(\bar{\tilde{f}}\big)=g_{w}N^{1-\gamma}/\big(1+\frac{g_{w}}{N^{\gamma}}\bar{\tilde{f}}^{\T}C^{(xx)}\bar{\tilde{f}}\big)$
and $\bar{\tilde{f}}$ satisfies \prettyref{eq:self_cons_tl}. So
even though the adaptive approach in mean-field scaling considers
a directional change to the kernel, in terms of the mean output this
is equivalent to a rescaled kernel. In standard scaling, one cannot
immediately express the mean output in terms of a rescaled kernel.
However, in the proportional limit ${N\propto P\rightarrow\infty}$,
certain fluctuation correction terms vanish, reducing the expressions
to a rescaling form again (see \prettyref{app:one_loop_app}). The
rescaling factor is given by
\begin{align}
 & Q_{\text{1-loop}}(\bar{\tilde{f}})\\
 & =Q_{\text{TL}}(\bar{\tilde{f}})-\frac{Q_{\text{TL}}^{2}(\bar{\tilde{f}})}{N}\text{Tr}\left[A^{-1}(\bar{\tilde{f}})\,C_{\mathcal{D\mathcal{D}}}^{(xx)}\right],\nonumber 
\end{align}
where $A(\bar{\tilde{f}})\coloneqq\kappa_{0}\mathbb{I}+Q_{\text{TL}}\big(\bar{\tilde{f}}\big)C_{\mathcal{D\mathcal{D}}}^{(xx)}$,
and $\bar{\tilde{f}}$ satisfies \prettyref{eq:self_cons_eq_oneloop_approx}.
We thus find that known theoretical approaches are all derived from
the same original posterior distribution by considering different
order parameters, while their resulting predictions for the mean network
output behave like a rescaled NNGP. However the rescaling behavior
of mean predictors holds only for linear networks. Other properties,
such as phase transitions in non-linear networks \cite{Rubin24_iclr}
require an adaptive approach and would escape a description by a rescaled
kernel.

\section{Directional feature learning emerges in adaptive description\label{sec:directional_fl}}

The power of NNs stems from their ability to detect high-dimensional
features in the data, implying that in the transition from the lazy
to the rich regime this would be reflected in the network output statistics
in a non-trivial manner. It is well established that the network weights
adapt during training in an anisotropic manner, detecting relevant
directions present in the training data \cite{seroussi23_908,Fischer24_10761};
yet surprisingly, for the mean output of a linear network this adaptation
seems to be equivalent to an isotropic rescaling of the NNGP kernel.

In this section, we demonstrate that the directional aspect of FL
is nonetheless present in output fluctuations, which is only captured
by the adaptive approach. Given a normalized feature direction $\hat{\phi}$,
we define a directional FL measure $\Phi\big(\hat{\phi}\big)$ that
indicates to which degree this feature is represented by the network
relative to other learned features
\begin{equation}
\Phi\big(\hat{\phi}\big)\coloneqq\frac{\hat{\phi}^{\T}\llangle ff^{\T}\rrangle\hat{\phi}}{\text{Tr}\left(\llangle ff^{\T}\rrangle\right)}.
\end{equation}
Then, $\Phi\big(\hat{\phi}\big)\rightarrow1$ indicates that the feature
direction $\hat{\phi}$ dominates the covariance, implying that this
feature has been perfectly learned, whereas $\Phi\big(\hat{\phi}\big)\ll1$
is an indication of weak directional FL. As derived in \prettyref{app:theory},
we obtain for the covariance of the network outputs on the training
data
\begin{align}
\llangle ff^{\T}\rrangle_{\text{adaptive}} & =\kappa\mathbb{I}-\kappa^{2}\big(A+\frac{2Q_{\text{TL}}^{2}}{N\kappa_{0}^{2}}F\big)^{-1},\label{eq:cov_adaptive}
\end{align}
where we observe a structural change in the covariance matrix in form
of the term $F\coloneqq C_{\mathcal{D}\mathcal{D}}^{(xx)}\langle\Delta\rangle\langle\Delta\rangle^{\T}C_{\mathcal{D}\mathcal{D}}^{(xx)}$,
which is not present in a rescaling of the NNGP, whose covariance
is
\begin{equation}
\llangle ff^{\T}\rrangle_{\text{rescaling}}=\kappa\mathbb{I}-\kappa^{2}\big(\kappa\mathbb{I}+Q\,C_{\mathcal{D}\mathcal{D}}^{(xx)}\big)^{-1}.
\end{equation}
We comment that in the original work by \citet{Li21_031059}, only
the variance but no explicit form for the full covariance matrix has
been derived. Nevertheless, the expressions derived in \prettyref{app:kernel_scaling_app}
for the rescaling approach remain consistent with their solution for
the mean network output and the variance. As evident from the expressions
for the covariance, the directional FL measure $\Phi$ differs significantly
between the two approaches, which is illustrated most easily for a
kernel $C^{(xx)}\propto\mathbb{I}$: the isotropy of the rescaling
theory then results in the same value of $\Phi\big(\hat{\phi}\big)$
independent of the direction of $\hat{\phi}$, whereas the structural
change of the covariance in the adaptive theory by the rank-one term
$\langle\Delta\rangle\langle\Delta\rangle^{\T}$ in \eqref{eq:cov_adaptive}
may yield larger values of $\Phi\big(\hat{\phi}\big)$ for features
$\hat{\phi}\parallel\langle\Delta\rangle$.

We show the directional aspect of FL in a teacher-student setting,
where the teacher is given by $y=Xw_{*}$ with $X\sim\mathcal{N}(0,\mathbb{I})$
and the student is a linear network as in \prettyref{sec:setup}.
In this setting, the teacher defines a feature direction $\hat{y}_{*}=Xw_{*}/\left|Xw_{*}\right|$,
and for comparison, we consider another feature direction $\hat{y}_{\perp}=Xw_{\perp}/\left|Xw_{\perp}\right|$,
orthogonal to the former in the sense that $w_{\perp}\perp w_{*}$.
The latter can be thought of as the direction of a randomly selected
teacher that differs in the weights of the hidden layer from that
of the actual target teacher. In \prettyref{fig:fluctuations_adaptation},
we show the relative directional FL measure $\Phi\big(\hat{y}_{*}\big)/\Phi\big(\hat{y}_{\perp}\big)$
between the target teacher and a random, orthogonal teacher direction.
While the rescaling theory does not differentiate between these directions,
the adaptive theory accurately predicts amplification of the teacher
direction when entering the mean-field regime. The choice of a high-dimensional
order parameter in the adaptive theory allows us to capture this directional
property of FL, illustrating how this theory yields a richer picture
of the network behavior.
\begin{figure}
\vskip 0.2in\includegraphics[bb=0bp 0bp 792bp 612bp,width=1\columnwidth]{fluctuations_teacher_student}\caption{Relative directional feature learning in a teacher student setting
as a function of the fluctuation scale $1/\chi$. Both NNGP and rescaling
theory fail to capture directional feature learning, while the multi-scale
adaptive theory accurately predicts network behavior. Insets show
the output distribution in different directions; a detailed version
can be found in \prettyref{fig:teacher_student_appendix}. Parameters:
$P_{\text{train}}=80$, $N=200$, $D=50$, $\kappa_{0}=2$, $g_{v}=0.01,$
$g_{w}=2$.}
\label{fig:fluctuations_adaptation}

\vskip -0.2in
\end{figure}


\section{Discussion}

In this work we present a unified theoretical framework to understand
feature learning (FL) in the Bayesian setting across scaling regimes,
from lazy to rich learning. This framework describes both effects
of data adaptation in trained networks, i.e. directional changes of
the network's output statistics in response to statistical dependencies
present in the training data, as well as output rescaling phenomena
that were described in previous works \cite{Li21_031059,Pacelli23_1497}.
Our theory thus creates links between existing and so far unconnected
previous theories. In the rich regime, the presented multi-scale adaptive
theory clearly exposes directional aspects of FL, thus going beyond
rescaling theories. We finally reconcile the apparent contradiction
between directional adaptation and rescaling by recovering the latter
as an approximation of the former on the level of the mean network
output.

Furthermore, the here presented multi-scale adaptive theory applies
to both standard and mean-field scaling and the entirety of the scaling
spectrum. The latter is possible since the presented theoretical frameworks
allows systematically computing fluctuation corrections depending
on the scaling regime.

\subparagraph{Limitations}

We study linear single-hidden-layer networks in this work to limit
the approximations needed, providing a clearer picture on the relation
between different FL theories. Extending the presented framework to
non-linear networks is straightforward and will be a direction of
future research.

\subparagraph{Outlook}

We expect directional FL to be crucial for network performance in
non-linear networks. Further, we plan to study the effect of network
depth on FL. Beyond this, it will be valuable to extend the theoretical
framework to other network architectures such as convolutional networks,
residual networks, and transformers, using the respective network
priors \cite{Garriga19,Hron20_4376,Fischer24_arxiv}. To study the
effect of noise in input data on FL \cite{Lindner23_arxiv}, we would
like to include fluctuations of the input kernel in the theoretical
framework.

\section*{Impact Statement}

This paper works towards understanding feature learning, thus aiming
to advance explainability of networks. While the latter surely has
societal impacts, these will be much further down the line.

\section*{Acknowledgements}

This work was partly funded by the Deutsche Forschungsgemeinschaft (DFG, German Research Foundation) - 368482240/GRK2416. MK would like to thank the Institute for Advanced Simulation (IAS-6) at Juelich Research Center and its directors Markus Diesmann and Sonja Gr\"un for their hospitality during regular visits.

\bibliographystyle{icml2025_style/icml2025}
\bibliography{brain,add_to_brain}

\newpage{}


\appendix
\onecolumn 

\section{Universal theory of train and test statistics\label{app:theory}}

We are interested in the training discrepancies $\langle\Delta_{\alpha}\rangle=y_{\alpha}-\langle f_{\alpha}\rangle$
with $\langle f_{\alpha}\rangle$ denoting the mean network output,
and in the mean network output $\langle f_{\ast}\rangle$ for a test
point $x_{\ast}$ after conditioning on the training data $\mathcal{D}=\{(x_{\alpha},y_{\alpha})\}_{1\le\alpha\le P}$.
For clarity, in the appendix we make all index notations explicit
instead of using $\mathcal{D}$ as in the main text, and denote summations
over training data points with Greek letters. We refer to the mean
network outputs as predictors. The joint prior distribution for $(f,f_{\ast},y)$
can be computed as in \cite{Segadlo22_accepted,Fischer24_10761} and
is given by
\begin{align}
p(f,f_{\ast},y) & =\N(y\vert f,\kappa_{0}N^{1-\gamma})\,\int d^{(P)}\tilde{f}\int\text{d}\tilde{f}_{\ast}\exp\bigg(-\sum_{\alpha=1}^{P+1}i\tilde{f}_{a}f_{a}+W(i\tilde{f},i\tilde{f}_{\ast})\bigg),\label{eq:dist_output}\\
W(\tilde{f}_{\mathcal{D}},\tilde{f}_{\ast}) & =\ln\bigg\langle\exp\bigg(\sum_{a=1}^{P+1}\tilde{f_{a}}\sum_{j=1}^{N}w_{j}h_{aj}\bigg)\bigg\rangle_{w_{i},h_{ai}},
\end{align}
where we use the shorthands $\int\text{d}^{(P)}\tilde{f}=\prod_{\alpha=1}^{P}\int_{-\infty}^{\infty}d\tilde{f}_{\alpha}/(2\pi)$,
the $P+1$ index corresponds to the test point, and $i$ is the imaginary
unit. The i.i.d. distribution of the readin weights $V_{kl}$ implies
that $h_{\alpha j}\overset{\text{i.i.d. over j}}{\sim}\N(0,C^{(xx)})$
with $C^{(xx)}=g_{v}/D\,XX^{\T}\in\bR^{P\times P}$. To keep notation
concise, summations over repeated indices on the right are implied
in the following.

We may obtain training discrepancies $\langle\Delta_{\alpha}\rangle$
and the test predictor $\langle f_{\ast}\rangle$ from the joint cumulant-generating
function $\mathcal{W}$ for the test point defined as
\begin{align}
\mathcal{W}(j_{\ast}\vert y) & =\ln\int df_{\ast}\int df\exp(j_{\ast}f_{\ast})\,p(f,f_{\ast},y).\label{eq:def_calW}
\end{align}
Taking its derivatives w.r.t. to either training labels $y_{\alpha}$
or the source term $j_{\ast}$ yields the posterior of the desired
quantities
\begin{align}
\langle\Delta_{\alpha}\rangle & =-\kappa_{0}N^{1-\gamma}\frac{\partial\mathcal{W}(j_{\ast}\vert y)}{\partial y_{\alpha}}\vert_{j_{\ast}},\label{eq:mean_discrepancy}\\
\langle f_{\ast}\rangle & =\frac{\partial\mathcal{W}(j_{\ast}\vert y)}{\partial j_{\ast}}\bigg\vert_{j_{\ast}=0},
\end{align}
because the outer derivative of the logarithm produces the normalization
by the model evidence (marginal likelihood) $1/p(y)=1/\int df\,\int df_{\ast}\,p(f,f^{\ast},y)$.

Likewise, the variances follow as
\begin{align}
\llangle\Delta_{\alpha}\Delta_{\beta}\rrangle & =\kappa_{0}N^{1-\gamma}-\kappa_{0}^{2}N^{2-2\gamma}\frac{\partial^{2}\mathcal{W}(j_{\ast}\vert y)}{\partial y_{\alpha}\partial y_{\beta}}\vert_{j_{\ast}=0},\label{eq:Var_discrepancy}\\
\llangle f_{*}^{2}\rrangle & =\frac{\partial^{2}\mathcal{W}(j_{\ast}\vert y)}{\partial(j_{\ast})^{2}}\bigg\vert_{j_{\ast}=0},
\end{align}
By inserting \eqref{eq:dist_output} into \eqref{eq:def_calW} and
performing the integration over $f$, we can rewrite $\mathcal{W}$
as
\begin{equation}
\mathcal{W}(j_{\ast}\vert y)=\ln\int d^{(P)}\tilde{f}\,\exp\big(-iy_{\alpha}\tilde{f}_{\alpha}-\frac{\kappa_{0}}{2}N^{1-\gamma}\tilde{f}_{\alpha}\tilde{f}_{\alpha}+W(i\tilde{f}_{\mathcal{D}},j_{\ast})\big).\label{eq:calW_f_integrated}
\end{equation}
Comparing \eqref{eq:mean_discrepancy}, \eqref{eq:Var_discrepancy},
and \eqref{eq:calW_f_integrated}, we note that $y$ acts as a linear
source term for $\tilde{f}_{\mathcal{D}}$, from which we see that
the physical meaning of the field $\tilde{f}_{\mathcal{D}}$ is related
to the discrepancy between target and network output
\begin{align}
\langle\Delta_{\alpha}\rangle & =\kappa_{0}N^{1-\gamma}\langle i\tilde{f}_{\alpha}\rangle,\label{eq:relation_Delta_tilde_f_stats}\\
\llangle\Delta_{\alpha}\Delta_{\beta}\rrangle & =\kappa_{0}N^{1-\gamma}\delta_{\alpha\beta}+\kappa_{0}^{2}N^{2-2\gamma}\,\llangle\tilde{f}_{\alpha}\tilde{f}_{\beta}\rrangle.
\end{align}
For computational convenience, we now introduce a source term $k$
\begin{equation}
\mathcal{W}(k,j_{\ast}\vert y)=\ln\int\text{d}\tilde{f}\exp\Big(ik_{\alpha}\tilde{f}_{\alpha}\underbrace{-iy_{\alpha}\tilde{f}_{\alpha}-\frac{\kappa_{0}}{2}N^{1-\gamma}\tilde{f}_{\alpha}\tilde{f}_{\alpha}+W(i\tilde{f},j_{\ast})}_{\mathcal{S}}\Big),\label{eq:cum_gen_full}
\end{equation}
allowing us to compute moments of $\tilde{f}$ by differentiating
by $k$ instead of $y$ and subsequently setting $k=0$. We define
the latter part of the exponent of $\mathcal{W}$ as the action
\begin{equation}
\mathcal{S}(\tilde{f},j_{\ast}\vert y)\coloneqq-iy_{\alpha}\tilde{f}_{\alpha}-\frac{\kappa_{0}}{2}N^{1-\gamma}\tilde{f}_{\alpha}\tilde{f}_{\alpha}+W(i\tilde{f},j_{\ast}\vert y).\label{eq:appendix_action}
\end{equation}

Depending on the scaling in $\gamma$, the network outputs $f$ fully
concentrate on their mean values or require corrections due to non-negligible
fluctuations. To treat both cases jointly and systematically, we introduce
the so-called effective action \cite{Helias20_970} as
\begin{equation}
\Gamma(\bar{\tilde{f}},j_{\ast}\vert y)=\text{extr}_{k}\,ik^{\T}\bar{\tilde{f}}-\mathcal{W}(k,j_{\ast}\vert y),\label{eq:Appendix_VGF_Definition}
\end{equation}
where we explicitly keep the dependence on the source term $j_{\ast}$
for the test point in order to compute parametric derivatives to obtain
test point statistics. This corresponds to the Legendre transform
of the cumulant-generating function $\mathcal{W}$; in the case that
$\mathcal{W}(k,j_{\ast}\vert y)$ has a scaling form, a large deviation
principle can be applied and the effective action corresponds to the
rate function \cite{Touchette09}.

The argument $\bar{\tilde{f}}$ is implicitly defined by the stationary
point (sometimes referred to as the equation of state)
\begin{equation}
\frac{\partial\Gamma(\bar{\tilde{f}},j_{\ast}\vert y)}{\partial\bar{\tilde{f}}_{\alpha}}=ik_{\alpha}=0,\label{eq:appendix_equation_of_state}
\end{equation}
as we set the source term $k$ to $0$ by definition. Using the definition
of $\Gamma$ in \prettyref{eq:Appendix_VGF_Definition}, the supremum
condition yields a self-consistency equation for $\bar{\tilde{f}}$
\begin{equation}
i\bar{\tilde{f}}(j_{\ast})=\frac{\partial\mathcal{W}(k,j_{\ast}\vert y)}{\partial k}.
\end{equation}
In the following we determine approximations of the Legendre transform
$\Gamma(\bar{\tilde{f}},j_{\ast}\vert y)$ to different orders of
statistical fluctuations, corresponding to different scaling regimes.
From the definition of the effective action $\Gamma$ follows as well
that we obtain the mean output on that test point from
\begin{align}
\langle f_{\ast}\rangle & =\frac{\partial\mathcal{W}(k,j_{\ast}\vert y)}{\partial j_{\ast}}\bigg\vert_{k,j_{\ast}=0}=-\frac{\partial\Gamma(\bar{\tilde{f}},j_{\ast}\vert y)}{\partial j_{\ast}}\bigg\vert_{j^{*}=0}.
\end{align}


\subsection{Cumulant-generating function $W$ of the network prior\label{app:cum_gen_fun_adapt}}

We compute the cumulant-generating function $W$ of the network prior
by first taking the average over network weights $V,w$ and subsequently
over the hidden-layer representations $h$. We have
\begin{align}
W(i\tilde{f},j_{\ast}) & =\ln\left\langle \exp\big((i\tilde{f_{\alpha}}h_{\alpha j}+j_{\ast}h_{\ast j})w_{j}\big)\right\rangle _{w_{j}\stackrel{\text{i.i.d.}}{\sim}\N(0,g_{w}/N^{\gamma}),\,h_{\circ j}\stackrel{\text{i.i.d. in }j}{\sim}\N(0,\hat{C}^{(xx)})}\\
 & =N\,\ln\left\langle \exp\big((i\tilde{f_{\alpha}}h_{\alpha}+j_{\ast}h_{\ast})w\big)\right\rangle _{w\sim\N(0,g_{w}/N^{\gamma}),\,h_{\circ}\sim\N(0,\hat{C}^{(xx)})}\\
 & =N\,\ln\left\langle \exp\left(-\frac{g_{w}}{2N^{\gamma}}\tilde{f_{\alpha}}h_{\alpha}\tilde{f_{\beta}}h_{\beta}+\frac{g_{w}}{N^{\gamma}}i\tilde{f_{\alpha}}h_{\alpha}j_{\ast}h_{\ast}+\frac{g_{w}}{2N^{\gamma}}j_{\ast}^{2}h_{\ast}^{2}\right)\right\rangle _{h\sim\N(0,\hat{C}^{(xx)})},
\end{align}
where we used from the first to the second line that $w_{j}$ and
$h_{\alpha j}$ are i.i.d. distributed over neuron index $j$. The
covariance matrix of the hidden-layer representation $h_{\alpha j}$
likewise is i.i.d. Gaussian in $j$ with covariance matrix in the
index $\alpha$ given by
\begin{equation}
\hat{C}^{\left(xx\right)}=\begin{bmatrix}C^{\left(xx\right)} & \left\{ C_{\alpha*}^{(xx)}\right\} _{\alpha=1}^{P}\\
\left\{ C_{*\alpha}^{(xx)}\right\} _{\alpha=1}^{P} & C_{**}^{\left(xx\right)}
\end{bmatrix}
\end{equation}
 where $C^{(xx)}=g_{v}/D\,XX^{\T},$and $C_{*\alpha}^{(xx)}=g_{v}/D\ x_{\alpha}\cdot x_{*}$.
We compute the Gaussian integral over the hidden-layer representations
$h$, yielding

\begin{align}
W(i\tilde{f},j_{\ast})=-\frac{N}{2}\,\ln\det\left[\mathbb{I}-\frac{g_{w}}{N^{\gamma}}\hat{C}^{(xx)}\left(\begin{array}{c}
i\tilde{f}\\
j_{\ast}
\end{array}\right)\left(\begin{array}{cc}
i\tilde{f}^{T} & j_{\ast}\end{array}\right)\right].\label{eq:cum_gen_linear}
\end{align}


\subsection{Tree-level approximation\label{app:tree-level_app}}

To compute the output statistics, one technically requires the exact
effective action $\Gamma$ in \prettyref{eq:Appendix_VGF_Definition}.
However, in general it does not have an analytical solution and we
instead determine a systematic expansion. A well-established method
from both statistical physics and quantum field theory is the loopwise
expansion \cite{Helias20_970}, expands the effective action $\Gamma(\bar{\tilde{f}})$
in terms of fluctuations of $\tilde{f}$ around its mean value $\bar{\tilde{f}}$.
The lowest-order term of the loopwise expansion is called the tree-level
approximation, which hence corresponds to a standard mean-field approximation:
one replaces $\tilde{f}_{\mathcal{D}}$ by its mean $\bar{\tilde{f}}$in
the action itself
\begin{align}
\Gamma_{\text{TL}}(\bar{\tilde{f}},j_{\ast}\vert y) & =-S(\bar{\tilde{f}},j_{\ast}\vert y)\\
 & =iy_{\alpha}\bar{\tilde{f}}_{\alpha}+\frac{\kappa_{0}}{2}N^{1-\gamma}\bar{\tilde{f}}_{\alpha}\bar{\tilde{f}}_{\alpha}-W(i\bar{\tilde{f}},j_{\ast}).
\end{align}
The average value of $\bar{\tilde{f}}_{\alpha}$ is given by the equation
of state \eqref{eq:appendix_equation_of_state} of the effective action
\begin{equation}
\frac{\partial\Gamma_{\text{TL}}(\bar{\tilde{f}},j_{\ast}\vert y)}{\partial\bar{\tilde{f}}_{\alpha}}\bigg\vert_{j_{\ast}=0}=0.\label{eq:eq_state_tilde_f}
\end{equation}
From this we obtain a self-consistency equation for $\bar{\tilde{f}}_{\alpha}$
as
\begin{align}
\bar{\tilde{f}} & =-i\left(\kappa_{0}N^{1-\gamma}\mathbb{I}+C_{\text{TL}}\big(\bar{\tilde{f}}\big)C^{\left(xx\right)}\right)^{-1}y,\\
C_{\text{TL}}\big(\bar{\tilde{f}}\big) & =g_{w}N^{1-\gamma}\left[\mathbb{I}+\frac{g_{w}}{N^{\gamma}}C^{(xx)}\bar{\tilde{f}}\bar{\tilde{f}}^{\T}\right]^{-1}.
\end{align}
 Using the relation between the statistics of the discrepancies $\Delta$
and $\tilde{f}_{\mathcal{D}}$ \eqref{eq:relation_Delta_tilde_f_stats},
we obtain for the training discrepancies
\begin{align}
\langle\Delta_{\alpha}\rangle & =-\kappa_{0}N^{1-\gamma}\frac{\partial\mathcal{W}(k,j_{\ast}\vert y)}{\partial k_{\alpha}}\vert_{j_{\ast}=0}\\
 & =\kappa_{0}\left(\kappa_{0}\mathbb{I}+C_{\text{TL}}\big(\bar{\tilde{f}}\big)C^{(xx)}\right)_{\alpha\beta}^{-1}y_{\beta}.
\end{align}
For the test point, we get
\begin{align}
\langle f_{\ast}\rangle_{\text{TL}} & =-\frac{\partial\Gamma_{\text{TL}}(\bar{\tilde{f}},j_{\ast}\vert y)}{\partial j_{\ast}}\rvert_{j_{\ast}=0}=-\frac{\partial W(i\bar{\tilde{f}},j_{\ast})}{\partial j_{\ast}}\rvert_{j_{\ast}=0}\label{eq:f_tilde_tree_level}\\
 & =\frac{g_{w}}{N^{1-\gamma}}C_{*\alpha}^{(xx)}\left(\mathbb{I}+\frac{g_{w}}{N^{\gamma}}C^{(xx)}\bar{\tilde{f}}\bar{\tilde{f}}^{\T}\right)_{\alpha\beta}^{-1}i\bar{\tilde{f}}_{\beta}.
\end{align}
Where substituting the self-consistency equation for $\bar{\tilde{f}}$,
we obtain
\begin{align}
\langle f_{*}\rangle_{\text{TL}} & =C_{\text{TL}}\big(\bar{\tilde{f}}\big)_{\delta\alpha}C_{*\delta}^{(xx)}\left[\left(\kappa_{0}N^{1-\gamma}\mathbb{I}+C_{\text{TL}}\big(\bar{\tilde{f}}\big)\right)^{-1}\right]_{\alpha\beta}y_{\beta}
\end{align}
In mean-field scaling ($\gamma=2$) and for $N\rightarrow\infty$
this result becomes exact using the G\"artner-Ellis theoreom: the
output cumulant-generating function $\mathcal{W}$ in \prettyref{eq:cum_gen_posterior}
has a scaling form as 
\begin{equation}
iy_{\alpha}\tilde{f}_{\alpha}+\frac{\kappa_{0}}{2N}\tilde{f}_{\alpha}\tilde{f}_{\alpha}-W(i\tilde{f}_{\mathcal{D}},j_{\ast})=N\lambda_{f}(\tilde{f}_{\mathcal{D}}/N)
\end{equation}
with $\lambda_{f}(k)=iy_{\alpha}k_{\alpha}+\frac{\kappa_{0}}{2}k_{\alpha}k_{\alpha}-W(ik,j_{\ast})$.
Thus, we can approximate the probability distribution of network outputs
as \cite{Touchette09}
\begin{equation}
-p(y\vert C^{(xx)})/N\approx\Gamma_{\text{TL}}(\bar{\tilde{f}},j_{\ast}\vert y).
\end{equation}
Due to the strong suppression of fluctuations in mean-field scaling
with $N\rightarrow\infty$, the tree-level approximation is sufficient
to describe the network behavior and in particular
\begin{align}
\lim_{N\rightarrow\infty}-p(y\vert C^{(xx)})/N & =\Gamma_{\mathrm{TL}}(\bar{\tilde{f}},j_{\ast}\vert y).
\end{align}
However, in the case of larger output fluctuations as in standard
scaling ($\gamma=1$), we need to take into account the output fluctuations
systematically by including higher-order corrections to the tree-level
result. We derive the leading-order correction in the following section.

\subsection{One-Loop corrections in standard scaling\label{app:one_loop_app}}

While in mean-field scaling ($\gamma=2$) the cumulant-generating
function has a scaling form and the network outputs $f$ concentrate,
we need to account for their fluctuations in standard scaling ($\gamma=1$).
In the following, we thus set $\gamma=1$. To leading order, also
called one-loop approximation, we have
\begin{equation}
\Gamma_{\text{1-Loop}}(\bar{\tilde{f}},j_{\ast}\vert y)=-\mathcal{S}(\bar{\tilde{f}},j_{\ast})-\frac{1}{2}\log\det(-\mathcal{S}^{(2)}).
\end{equation}
The self-consistency equation for $\bar{\tilde{f}}$ then becomes
\begin{equation}
\frac{\partial\Gamma_{\text{1-Loop}}(\bar{\tilde{f}},j_{\ast}\vert y)}{\partial\bar{\tilde{f}}_{\delta}}\bigg\vert_{j^{*}=0}=-\frac{\partial\mathcal{S}(\bar{\tilde{f}},j_{\ast})}{\partial\bar{\tilde{f}}_{\delta}}\bigg\vert_{j^{*}=0}-\frac{1}{2}\sum_{\alpha\beta}(-\mathcal{S}^{(2)})_{\beta\alpha}^{-1}\frac{\partial^{3}(-\mathcal{S}(\bar{\tilde{f}},j_{\ast}))}{\partial\bar{\tilde{f}}_{\alpha}\partial\bar{\tilde{f}}_{\beta}\partial\bar{\tilde{f}}_{\delta}}\bigg\vert_{j^{*}=0}\overset{!}{=}0.
\end{equation}
By applying the matrix-determinant lemma on \prettyref{eq:cum_gen_linear},
we can simplify the cumulant-generating function $W$ of the network
prior as
\begin{align}
W(i\tilde{f}_{\mathcal{D}},j_{\ast}) & =-\frac{N}{2}\left(\ln\det\left[\mathbb{I}-\frac{g_{w}}{N^{\gamma}}\hat{C}^{(xx)}\left(\begin{array}{cc}
i\tilde{f} & j_{\ast}\end{array}\right)\left(\begin{array}{c}
i\tilde{f}^{\T}\\
j_{\ast}
\end{array}\right)\right]\right)\\
 & =-\frac{N}{2}\left(\ln\left[1-\frac{g_{w}}{N^{\gamma}}\left(\begin{array}{cc}
i\tilde{f} & j_{\ast}\end{array}\right)\hat{C}^{(xx)}\left(\begin{array}{c}
i\tilde{f}\\
j_{\ast}
\end{array}\right)\right]\right)\\
 & =-\frac{N}{2}\ln\left[1-\frac{g_{w}}{N^{\gamma}}\left(\begin{array}{cc}
i\tilde{f}_{\alpha} & j_{\ast}\end{array}\right)\left(\begin{array}{cc}
C^{(xx)} & C_{\alpha*}^{(xx)}\\
C_{*\beta}^{(xx)} & C_{**}^{(xx)}
\end{array}\right)\left(\begin{array}{c}
i\tilde{f}_{\beta}\\
j_{\ast}
\end{array}\right)\right].
\end{align}
Given this form of the cumulant-generating function $W$, taking the
derivative of $\mathcal{S}$ with respect to $\bar{\tilde{f}}$ in
\eqref{eq:eq_state_tilde_f} yields a different expression than in
the previous section
\begin{align}
-\frac{\partial\mathcal{S}(\bar{\tilde{f}},j_{\ast})}{\partial\bar{\tilde{f}}_{\rho}}\bigg\vert_{j^{*}=0} & =iy_{\rho}+\kappa_{0}\bar{\tilde{f}}_{\rho}+Q_{\text{TL}}\big(\bar{\tilde{f}}\big)\,C_{\rho\delta}^{(xx)}\overline{\tilde{f}}_{\delta},\label{eq:stationary_tilde_f}\\
Q_{\text{TL}}\big(\bar{\tilde{f}}\big) & =\frac{g_{w}}{1+\frac{g_{w}}{N^{\gamma}}\bar{\tilde{f}}^{\T}C^{(xx)}\bar{\tilde{f}}}.
\end{align}
Note that in this form the tree-level equation for $\bar{\tilde{f}}$
\eqref{eq:f_tilde_tree_level} can be written as
\begin{equation}
\bar{\tilde{f}}=-i\left(\kappa_{0}N^{1-\gamma}\mathbb{I}+Q_{\text{TL}}\big(\bar{\tilde{f}}\big)\,C^{\left(xx\right)}\right)^{-1}y,
\end{equation}
thereby obtaining an expression in which the input kernel $C^{(xx)}$
is only rescaled by a scalar, which we call a kernel rescaling expression.
For the second and third derivatives, we obtain
\begin{align}
\frac{\partial^{2}(-\mathcal{S}(\bar{\tilde{f}},j_{\ast}))}{\partial\bar{\tilde{f}}_{\alpha}\partial\bar{\tilde{f}}_{\beta}}\bigg\vert_{j^{*}=0} & =\kappa_{0}\mathbb{I}+Q_{\text{TL}}\big(\bar{\tilde{f}}\big)\,C_{\alpha\beta}^{(xx)}-\frac{2}{N}Q_{\text{TL}}\big(\bar{\tilde{f}}\big)\,C_{\alpha\delta}^{(xx)}\bar{\tilde{f}}_{\delta}\,Q_{\text{TL}}\big(\bar{\tilde{f}}\big)\,C_{\beta\epsilon}^{(xx)}\bar{\tilde{f}}_{\epsilon}\\
 & =A\big(\bar{\tilde{f}}\big)_{\alpha\beta}-\frac{2}{N}Q_{\text{TL}}^{2}\big(\bar{\tilde{f}}\big)\,\left[C^{(xx)}\bar{\tilde{f}}\bar{\tilde{f}}^{\T}C^{(xx)}\right]_{\alpha\beta},\\
\frac{\partial^{3}(\mathcal{-S}(\bar{\tilde{f}},j_{\ast}))}{\partial\bar{\tilde{f}}_{\alpha}\partial\bar{\tilde{f}}_{\beta}\partial\bar{\tilde{f}}_{\text{\ensuremath{\delta}}}}\bigg\vert_{j^{*}=0} & =-\frac{2}{N}Q_{\text{TL}}^{2}\big(\bar{\tilde{f}}\big)\,\left[C_{\alpha\beta}^{(xx)}C_{\delta\epsilon}^{(xx)}\bar{\tilde{f}}_{\epsilon}+C_{\alpha\delta}^{(xx)}C_{\beta\epsilon}^{(xx)}\bar{\tilde{f}}_{\epsilon}+C_{\beta\delta}^{(xx)}C_{\alpha\epsilon}^{(xx)}\bar{\tilde{f}}_{\epsilon}\right]\\
 & \phantom{=}+\frac{8}{N^{2}}Q_{\text{TL}}^{3}\big(\bar{\tilde{f}}\big)\,C_{\alpha\alpha^{\prime}}^{(xx)}\bar{\tilde{f}}_{\alpha^{\prime}}\,C_{\beta\beta^{\prime}}^{(xx)}\bar{\tilde{f}}_{\beta^{\prime}}\,C_{\delta\delta^{\prime}}^{(xx)}\bar{\tilde{f}}_{\delta^{\prime}}\\
 & =-\frac{2}{N}Q_{\text{TL}}^{2}\big(\bar{\tilde{f}}\big)\,\left(C_{\alpha\beta}^{(xx)}\left[C^{(xx)}\bar{\tilde{f}}\right]_{\delta}+C_{\alpha\delta}^{(xx)}\left[C^{(xx)}\bar{\tilde{f}}\right]_{\beta}+C_{\beta\delta}^{(xx)}\left[C^{(xx)}\bar{\tilde{f}}\right]_{\alpha}\right)\\
 & \phantom{=}+\frac{8}{N^{2}}Q_{\text{TL}}^{3}\big(\bar{\tilde{f}}\big)\,\left[C^{(xx)}\bar{\tilde{f}}\right]_{\alpha}\left[C^{(xx)}\bar{\tilde{f}}\right]_{\beta}\left[C^{(xx)}\bar{\tilde{f}}\right]_{\delta}.
\end{align}
Here, we use the shorthand $A\big(\bar{\tilde{f}}\big)=\kappa_{0}\mathbb{I}+Q_{\text{TL}}\big(\bar{\tilde{f}}\big)\,C^{(xx)}$.
Overall, we obtain
\begin{align}
\bar{\tilde{f}}_{\delta} & =\Big[A\big(\bar{\tilde{f}}\big)^{-1}\Big]_{\delta\epsilon}\Bigg[-iy_{\epsilon}+\frac{1}{2}\sum_{\alpha\beta}(-\mathcal{S}^{(2)})_{\beta\alpha}^{-1}\frac{\partial^{3}(-\mathcal{S}(\bar{\tilde{f}},j_{\ast}))}{\partial\bar{\tilde{f}}_{\alpha}\partial\bar{\tilde{f}}_{\beta}\partial\bar{\tilde{f}}_{\epsilon}}\bigg\vert_{j^{*}=0}\Bigg].
\end{align}
Similarly to the previous section, the training discrepancies are
given by
\begin{equation}
\langle\Delta_{\alpha}\rangle=i\kappa_{0}\bar{\tilde{f}}_{\alpha}.
\end{equation}
For the test predictor, we have
\begin{align}
\langle f^{*}\rangle_{\mathrm{\text{1-Loop}}} & =-\frac{\partial\Gamma_{\text{1-Loop}}(\bar{\tilde{f}},j_{\ast}\vert y)}{\partial j^{*}}\bigg\vert_{j^{*}=0}\\
 & =-\frac{\partial W(\bar{\tilde{f}},j_{\ast})}{\partial j^{*}}\bigg\vert_{j^{*}=0}-\frac{1}{2}\sum_{\alpha\beta}(-\mathcal{S}^{(2)})_{\beta\alpha}^{-1}\frac{\partial^{3}(-\mathcal{S}(\bar{\tilde{f}},j_{\ast}))}{\partial\bar{\tilde{f}}_{\alpha}\partial\bar{\tilde{f}}_{\beta}\partial j_{\ast}}\bigg\vert_{j^{*}=0}\\
 & =Q_{\text{TL}}\big(\bar{\tilde{f}}\big)\,C_{*\alpha}^{(xx)}\bar{\tilde{f}}_{\alpha}-\frac{1}{2}\sum_{\alpha\beta}(-\mathcal{S}^{(2)})_{\beta\alpha}^{-1}\frac{\partial^{3}(-\mathcal{S}(\bar{\tilde{f}},j_{\ast}))}{\partial\bar{\tilde{f}}_{\alpha}\partial\bar{\tilde{f}}_{\beta}\partial j_{\ast}}\bigg\vert_{j^{*}=0}.
\end{align}
The appearing derivatives of the action are structurally similar but
we replace the training point $x_{\gamma}$ by the test point $x_{\ast}$,
yielding
\begin{align}
\frac{\partial^{3}(-\mathcal{S}(\bar{\tilde{f}},j_{\ast}))}{\partial\bar{\tilde{f}}_{\alpha}\partial\bar{\tilde{f}}_{\beta}\partial j_{\ast}}\bigg\vert_{j^{*}=0} & =-\frac{2}{N}Q_{\text{TL}}^{2}\big(\bar{\tilde{f}}\big)\,\Big(C_{\alpha\beta}^{(xx)}\left[C^{(xx)}\bar{\tilde{f}}\right]_{\ast}+C_{\alpha\ast}^{(xx)}\left[C^{(xx)}\bar{\tilde{f}}\right]_{\beta}+C_{\beta\ast}^{(xx)}\left[C^{(xx)}\bar{\tilde{f}}\right]_{\alpha}\Big)\\
 & \phantom{=}+\frac{8}{N^{2}}Q_{\text{TL}}^{3}\big(\bar{\tilde{f}}\big)\,\left[C^{(xx)}\bar{\tilde{f}}\right]_{\alpha}\left[C^{(xx)}\bar{\tilde{f}}\right]_{\beta}\left[C^{(xx)}\bar{\tilde{f}}\right]_{\ast}.
\end{align}
When solving these equations, we backtransform to the imaginary variables
$\bar{\tilde{f}}\mapsto i\bar{\tilde{f}}$, which changes multiple
signs and absorbs the appearing imaginary units.

\subsection{Kernel rescaling approach\label{app:kernel_scaling_app}}

We here derive the results by \citet{Li21_031059} in our multi-scale
adaptive theory including regularization noise $\kappa_{0}$. In contrast
to \prettyref{app:cum_gen_fun_adapt}, we here first take the average
over hidden layer representations $h_{\alpha j}$and subsequently
over readout weights $w_{j}$, , leading to a different approximation
of the network output posterior distribution.

Using that $h_{\alpha j}\sim\N(0,C^{(xx)})$ i.i.d. over the neuron
index $j$, we can rewrite the cumulant-generating function $W$ conditioned
on readout weights $w$ as
\begin{align}
W(\tilde{f}_{\mathcal{D}}\vert w) & =\ln\left\langle \exp(-\tilde{f_{\alpha}}w_{j}h_{\alpha j})\right\rangle _{h_{\alpha j}}=-\frac{1}{2}\tilde{f_{\alpha}}C_{\alpha\beta}^{(xx)}\tilde{f}_{\beta}\,\|w\|^{2},
\end{align}
where we drop the test point here to keep notation concise. The result
for the test point will follow naturally later. We observe that the
readout weights only appear in the form of the squared norm $\|w\|^{2}$.
The distribution of the network output is hence
\begin{align}
p(y,f|C^{(xx)}) & =\N(y|f,\kappa_{0})\,\int\mathrm{d}\tilde{f}_{\mathcal{D}}\,\big\langle\exp\big(-i\tilde{f}_{\alpha}f_{\alpha}-\frac{1}{2}\tilde{f_{\alpha}}C_{\alpha\beta}^{(xx)}\tilde{f}_{\beta}\,\|w\|^{2}\big)\big\rangle_{w_{i}\stackrel{\text{i.i.d. }}{\sim}\N(0,\frac{g_{w}}{N})}.\label{eq:Appendix_NetworkPosterior_Sompolinsky}
\end{align}
Since both, the prior measure of the weights $w\sim\N(w|0,g_{w}N^{-1})\propto\exp(N\,\|w\|^{2}/2g_{w})$
and the explicit appearance of $w$, is only in the form of $\|w\|^{2}$,
we may introduce this quantity as an auxiliary variable, which we
name $Q:=\|w\|^{2}=\sum_{i=1}^{N}w_{i}^{2}$ and which corresponds
to the Euclidian norm of the readout weight vector $w$. Note that,
given $\|w\|^{2}$, the integral over $\tilde{f}_{\mathcal{D}}$ simply
yields $f|_{\|w\|^{2}}\sim\N(0,\|w\|^{2}C^{(xx)})$, so
\begin{align}
p(y,f|C^{(xx)}) & =\N(y|f,\kappa_{0})\,\int dQ\,\N(f|0,Q\,C^{(xx)})\,p(Q).\label{eq:p_z_y_Q}
\end{align}
Here the distribution of the squared norm is 
\begin{align}
p(Q) & =\big\langle\delta[-Q+\|w\|^{2}]\big\rangle_{w_{i}\stackrel{\text{i.i.d.}}{\sim}\N(0,\frac{g_{w}}{N})}\label{eq:p_Q}\\
 & =\int_{-i\infty}^{i\infty}\frac{d\tilde{Q}}{2\pi i}\,\big\langle\exp\big(\tilde{Q}\big[-Q+\|w\|^{2}\big]\big)\big\rangle_{w_{i}\stackrel{\text{i.i.d.}}{\sim}\N(0,\frac{g_{w}}{N})}\\
 & =\int_{-i\infty}^{i\infty}\frac{d\tilde{Q}}{2\pi i}\,\exp\big(-\tilde{Q}\,Q+W(\tilde{Q})\big),
\end{align}
where $W(\tilde{Q})=\ln\,\langle\exp\big(\tQ\,\|w\|^{2})\rangle_{w_{i}\stackrel{\text{i.i.d.}}{\sim}\N(0,\frac{g_{w}}{N})}$
is the cumulant-generating function of $Q$ . Using that the $w_{j}$
are i.i.d, we get
\begin{align}
W(\tQ) & =N\,\ln\,\big\langle\exp\big(\tQ\|w\|^{2}\big)\big\rangle_{w\sim\N(0,g_{w}/N)}\\
 & =-\frac{N}{2}\,\ln\big[1-\frac{2g_{w}}{N}\tilde{Q}\big],
\end{align}
where we performed the one-dimensional Gaussian integral over $w$.
Up to here, all steps are exact

\prettyref{eq:p_z_y_Q} shows that the auxiliary variable $Q$ being
a scalar may only carry fluctuations of the overall scaling of the
kernel and hence all descriptions and approximations in terms of $Q$
can only change the scale of the kernel. .

\subsubsection{Approximation of network prior for wide networks}

One expects that $Q$ concentrates since $Q=\|w\|^{2}=\sum_{i=1}^{N}w_{i}^{2}$
for large $N$ and i.i.d. $w_{i}\sim\N(0,g_{w}/N)$. The cumulant-generating
function $W$ can be written as a scaling form $\lambda_{N}(k):=N^{-1}\,W(N\,k)=-\frac{1}{2}\,\ln\big[1-2g_{w}k\big]$
and its limit $N\to\infty$ then exists trivially, so that we may
approximate $p(Q)$ with the G\"artner-Ellis theorem \cite{Touchette09}
as
\begin{align}
\ln\,p(Q) & \simeq\sup_{\tQ}\,-Q\tQ+W(\tQ)\label{eq:GammaQ}\\
 & =-\frac{N}{2g_{w}}\,\big[1-\frac{g_{w}}{Q}\big]\,Q-\frac{N}{2}\,\ln\big[\frac{g_{w}}{Q}\big]\\
 & =-\frac{N}{2}\,\big[\frac{Q}{g_{w}}-1-\ln\,\frac{Q}{g_{w}}\big]=:-\Gamma(Q).
\end{align}
Intuitively, by the scaled cumulant-generating function of the form
$N\,W(\tilde{Q}/N)=-\frac{N}{2}\,\ln\big[1-2g_{w}\,\frac{\tilde{Q}}{N}\big]$
the mean of is of order $\langle Q\rangle=\mathcal{O}(1)$ and all
higher-order cumulants of $Q$ are being suppressed by at least $\mathcal{O}(N^{-1})$.
So on exponential scales, one may parametrize the probability by the
mean, namely one obtains the distribution of $Q$ from the rate function
as $e^{-\Gamma(Q)}$. To obtain \prettyref{eq:GammaQ}, the supremum
condition has been used $0\stackrel{!}{=}-Q+g_{w}\,\big[1-2g_{w}\frac{\tilde{Q}}{N}\big]^{-1}$,
solved for $1-\frac{2g_{w}}{N}\,\tQ=\frac{g_{w}}{Q}$ and $\tilde{Q}=\frac{N}{2g_{w}}\,\big[1-\frac{g_{w}}{Q}\big]$
and inserted into the first line of \prettyref{eq:GammaQ} to obtain
the second line. The rate function, being the Legendre transform of
$W$, obeys the equation of state
\begin{align}
\frac{d}{dQ}\Gamma(Q) & =\tQ=\frac{N}{2g_{w}}\,\big[1-\frac{g_{w}}{Q}\big].\label{eq:eos_Gamma}
\end{align}
So the final expression for the joint probability of $y$ and $f$,
the network prior, is
\begin{align}
p(y,f|C^{(xx)}) & \simeq\N(y|f,\kappa_{0})\,\int dQ\,\N(f|0,Q\,C^{(xx)})\,e^{-\Gamma(Q)},\label{eq:p_z_y_pre_saddle-1}\\
 & =\int dQ\,e^{S(Q|f,y)},\nonumber 
\end{align}
where the action $S(Q|f,y)$ is
\begin{align}
S(Q|f,y) & =-\frac{\|y-f\|^{2}}{2\kappa_{0}}-\frac{P}{2}\ln\kappa_{0}\label{eq:action_Q-2-1}\\
 & \phantom{=}-\frac{1}{2}f^{\T}\,\big(QC^{(xx)}\big)^{-1}\,f-\frac{1}{2}\ln\det\big(QC^{(xx)}\big)-\Gamma(Q).\nonumber 
\end{align}


\subsubsection{Maximum a posteriori estimate for $Q$}

To obtain the posterior distribution for $Q$ we marginalize \eqref{eq:p_z_y_pre_saddle-1}
over the network outputs $f$, which yields
\begin{align}
p(y|C^{(xx)}) & \equiv\int df\,p(y,f|C^{(xx)})\label{eq:p_z_Q-1}\\
 & =\int dQ\,\exp\big(S(Q|y)\big),\nonumber 
\end{align}
which yields the action
\begin{align}
S(Q|y) & =-\frac{1}{2}y^{\T}\,\big(QC^{(xx)}+\kappa_{0}\I\big)^{-1}\,y-\frac{1}{2}\ln\det\big(QC^{(xx)}+\kappa_{0}\I\big)-\Gamma(Q),\label{eq:S_Q_z-1}
\end{align}
and which reproduces Eq. A11 in \citet{Li21_031059} after inserting
the rate function \eqref{eq:GammaQ}. When computing the maximum a
posteriori value $Q_{\text{LS}}$, it only depends on the numerator
of
\begin{align}
p(Q|y) & =\frac{p(y|Q)\,p(Q)}{p(y)},
\end{align}
since the form of \eqref{eq:p_z_Q-1} is $p(y)=\int dQ\,p(y|Q)\,p(Q)$.
Thus, computing the $Q$-integral in saddle point approximation comprises
to the maximum a posteriori (MAP) $Q_{\text{LS}}$ as $\ln\,p(y|Q)\,p(Q)=S(Q|y)+\const$
has the same stationary point as $p(y|Q)\,p(Q)$.

The length $Q=\|w\|^{2}$ in their theory is obtained by the maximum
of \eqref{eq:S_Q_z-1}, which is given by
\begin{align}
0 & \overset{!}{=}\frac{\partial S}{\partial Q}=\frac{1}{2}y^{\T}\big(QC^{(xx)}+\kappa_{0}\I\big)^{-1}C^{(xx)}\,\big(QC^{(xx)}+\kappa_{0}\I\big)^{-1}y-\tr\,C^{(xx)}\big(QC^{(xx)}+\kappa_{0}\I\big)^{-1}-\frac{N}{2}\big(\frac{1}{g_{w}}-\frac{1}{Q}\big).\label{eq:stat_Q-1}
\end{align}
This yields the tree-level approximation for $\|w\|^{2}=Q_{\text{LS}}$.

\subsubsection{Predictor statistics\label{sec:Predictor-statistics}}

To obtain predictions beyond the length of the readout $\|w\|$, we
start from \ref{eq:p_z_Q-1}. We obtain the training discrepancies
from
\begin{align}
\frac{\partial}{\partial y_{\alpha}}\,\ln p(y|C^{(xx)})\stackrel{\text{MAP }Q}{\simeq} & \frac{d}{dy_{\alpha}}\,\sup_{Q}S(Q|y)\\
= & \frac{\partial}{\partial y_{\alpha}}\,S(Q_{\text{LS}}|y)+\underbrace{\frac{\partial S}{\partial Q}}_{=0}\,\frac{\partial Q}{\partial y_{\alpha}}\big|_{Q=Q^{\ast}},
\end{align}
where the derivative by $Q$ vanishes because $Q_{\text{LS}}$ has
been determined by the supremum condition as the stationary point
of the action. The partial derivative by $y_{\alpha}$ only acts on
$-y^{\T}\,\big(QC^{(xx)}+\kappa_{0}\I\big)^{-1}\,y/2$ in the expression
for \eqref{eq:S_Q_z-1}
\begin{align}
\langle\Delta_{\alpha}\rangle & =\kappa_{0}\,\big(Q_{\text{LS}}C^{(xx)}+\kappa_{0}\I\big)^{-1}\,y.\label{eq:mean_discrepancies}
\end{align}
In consequence, the test predictor is identical to the NNGP predictor
with a different regularizer $\kappa_{0}/Q_{\text{LS}}$
\begin{align}
\langle f_{\ast}\rangle_{\text{LS}} & =\big[C_{\ast\alpha}^{(xx)}\big]\,\big(C^{(xx)}+\kappa_{0}/Q_{\text{LS}}\I\big)_{\alpha\beta}^{-1}\,y_{\beta}.
\end{align}

To compute the variance of the predictor, we generalize\eqref{eq:Appendix_NetworkPosterior_Sompolinsky}
such that instead of the variance $\kappa_{0}\I$ in $\N(y|f,\kappa_{0}\I)$,
we insert a general covariance matrix $K$ into the Gaussian measure
$\N(y|f,K)$ and perform an integration over $f$
\begin{align}
p(y|K,C^{(xx)}) & =\int df\,\N(y|f,K)\,\big\langle\,\prod_{\alpha=1}^{P}\delta\,\big[f_{\alpha}-\sum_{i=1}^{N}w_{i}\,\phi(h_{\alpha i})\big]\big\rangle_{w_{i}\stackrel{\text{i.i.d.}}{\sim}\N(0,\frac{g_{w}}{N}),\quad h_{\alpha i}\stackrel{\text{i.i.d. over }i}{\sim}\N(0,C^{(xx)})}.\label{eq:p_y_z_given_X-1}
\end{align}
The presence of the general matrix $K$ allows us to measure the statistics
of the discrepancies $\Delta_{\alpha}=y_{\alpha}-z_{\alpha}$, because
writing the Gaussian $\N(y|f,K)\propto\exp\big(-\frac{1}{2}(y-f)^{\T}K^{-1}(y-f)+\frac{1}{2}\,\ln\,\det\,(K^{-1})\big)$
explicitly we observe that derivatives by $\big[K^{-1}\big]_{\alpha\beta}$
yield
\begin{align}
\frac{\partial}{\partial[K]_{\alpha\beta}^{-1}}\ln p(y|K,C^{(xx)})\Big|_{K=\kappa_{0}\I} & =-\frac{1}{2}\langle(y-f)_{\alpha}(y-f)_{\beta}\rangle+\frac{1}{2}\,\kappa_{0}\,\delta_{\alpha\beta}.\label{eq:gen_disc}
\end{align}
With the same manipulations that led to \eqref{eq:p_z_Q-1} one then
has
\begin{align}
p(y|K,C^{(xx)}) & \equiv\int dQ\,\exp\big(S(Q|K,z)\big),
\end{align}
where the action, corresponding to \eqref{eq:S_Q_z-1}, is
\begin{align}
S(Q|y,K) & =-\frac{1}{2}y^{\T}\,\big(C+K\big)^{-1}\,y-\frac{1}{2}\ln\det\big(C+K\big)-\frac{N}{2}\,\big(\frac{Q}{g_{w}}-\ln\,Q\Big)\big|_{C=QC^{(xx)}}.
\end{align}
So in the approximation replacing $Q$ by its MAP $Q_{\text{LS}}$
we get
\begin{align}
\frac{\partial}{\partial[K]_{\alpha\beta}^{-1}}\ln p(y|K,C^{(xx)})\Big|_{K=\kappa_{0}\I} & =\frac{d}{d[K]_{\alpha\beta}^{-1}}\,\sup_{Q}S(Q|y,K)\Big|_{K=\kappa_{0}\I}\\
 & =\frac{\partial}{\partial[K]_{\alpha\beta}^{-1}}S(Q_{\text{LS}}|y,K)\Big|_{K=\kappa_{0}\I},
\end{align}
where the inner derivative by $\partial S/\partial Q$ drops out due
to stationarity at $Q_{\text{LS}}$, which is given by the solution
of \eqref{eq:stat_Q-1}. The latter partial derivative evaluates to
\begin{align}
\frac{\partial}{\partial[K]_{\alpha\beta}^{-1}}S(Q|y,K)\Big|_{K=\kappa_{0}\I} & =\Big[-\frac{1}{2}K\,\big[C+K\big]^{-1}yy^{\T}\,\big[C+K\big]^{-1}\,K+\frac{1}{2}\,K\,(C+K)^{-1}\,K\Big]_{\alpha\beta}\Big|_{K=\kappa_{0}\I,C=Q_{\text{LS}}\,C^{(xx)}}\\
 & =\kappa_{0}^{2}\,\Big[-\frac{1}{2}\,\big[C+\kappa_{0}\I\big]^{-1}yy^{\T}\,\big[C+\kappa_{0}\I\big]^{-1}+\frac{1}{2}\,(C+\kappa_{0}\I)^{-1}\Big]_{\alpha\beta}\Big|_{C=Q_{\text{LS}}\,C^{(xx)}},
\end{align}
where we used that $\partial K_{\gamma\delta}/\partial[K]_{\alpha\beta}^{-1}=-K_{\gamma\alpha}\,K_{\beta\delta}$,
which follows by symmetry from $\partial K_{\gamma\delta}^{-1}/\partial K_{\alpha\beta}=-K_{\gamma\alpha}^{-1}\,K_{\beta\delta}^{-1}$.

So the second moment of the discrepancies with \eqref{eq:gen_disc}
is
\begin{align}
\langle\Delta_{\alpha}\Delta_{\beta}\rangle & =\kappa_{0}\delta_{\alpha\beta}+\kappa_{0}^{2}\,\Big[\,\big[C+\kappa_{0}\I\big]^{-1}yy^{\T}\,\big[C+\kappa_{0}\I\big]^{-1}-(C+\kappa_{0}\I)^{-1}\Big]_{\alpha\beta}\Big|_{C=Q_{\text{LS}}\,C^{(xx)}}\nonumber \\
 & =\langle\Delta_{\alpha}\rangle\langle\Delta_{\beta}\rangle+\kappa_{0}\delta_{\alpha\beta}-\kappa_{0}^{2}\,(C+\kappa_{0}\I)_{\alpha\beta}^{-1}\Big|_{C=Q_{\text{LS}}\,C^{(xx)}},\label{eq:second_moment_Delta}
\end{align}
where we used \eqref{eq:mean_discrepancies} in the last step. Because
$\Delta=y-f$ and $y$ as the target does not fluctuate, the latter
two terms in \eqref{eq:second_moment_Delta} are the variance
\begin{eqnarray}
\llangle\Delta_{\alpha},\Delta_{\beta}\rrangle & = & \llangle f_{\alpha},f_{\beta}\rrangle\\
 & = & \kappa_{0}\delta_{\alpha\beta}-\kappa_{0}^{2}\,(C+\kappa_{0}\I)_{\alpha\beta}^{-1}\\
 & = & C-C\,[C+\kappa_{0}\I]^{-1}\,C\Big|_{C=Q_{\text{LS}}\,C^{(xx)}},
\end{eqnarray}
which is the usual expression for the variance of the NNGP predictor
of a Gaussian process with the kernel $C=Q_{\text{LS}}\,C^{(xx)}$.

\subsection{Connecting kernel rescaling and adaptive approach\label{app:connection_adpative_kernel_app}}

While the kernel rescaling approach holds in the proportional limit${N\propto P\rightarrow\infty}$
, the one-loop approximation holds also for large but finite $P,N\gg$1.
As we have seen in \prettyref{fig:scatter_standard_scaling_nngp}
in the main text, they yield almost identical results in certain settings.
By considering the proportional limit, we may connect these two approaches:
some correction terms vanish in this limit, leaving only a scalar
term.

To this end, we look at the scaling of each correction term with both
$P$ and $N$. We have
\begin{align}
-S^{(2)} & =\kappa_{0}\mathbb{I}+Q_{\text{TL}}\,C^{(xx)}-\frac{2}{N}Q_{\text{TL}}^{2}\,C^{(xx)}\bar{\tilde{f}}\bar{\tilde{f}}^{\T}C^{(xx)}\\
 & =\kappa_{0}\mathbb{I}+Q_{\text{TL}}\,C^{(xx)}+\mathcal{O}(1/N),
\end{align}
since $(Q\,C^{(xx)}+\kappa)\,\bar{\tilde{f}}\propto y=\mathcal{O}(1)$
and thus also $C^{(xx)}\bar{\tilde{f}}=\mathcal{O}(1)$. Here, we
drop the dependence of $Q_{\text{TL}}$ on $\bar{\tilde{f}}$ for
brevity. The fluctuation correction is given by
\begin{align}
 & \frac{1}{2}\sum_{\alpha\beta}(-\mathcal{S}^{(2)})_{\beta\alpha}^{-1}\frac{\partial^{3}(-\mathcal{S}(\bar{\tilde{f}},j_{\ast}))}{\partial\bar{\tilde{f}}_{\alpha}\partial\bar{\tilde{f}}_{\beta}\partial\bar{\tilde{f}}_{\delta}}\bigg\vert_{j^{*}=0}\\
 & =-\frac{1}{N}Q_{\text{TL}}^{2}\,\sum_{\alpha\beta}(-\mathcal{S}^{(2)})_{\beta\alpha}^{-1}\,\Big(C_{\alpha\beta}^{(xx)}\left[C^{(xx)}\bar{\tilde{f}}\right]_{\gamma}+C_{\alpha\gamma}^{(xx)}\left[C^{(xx)}\bar{\tilde{f}}\right]_{\beta}+C_{\beta\gamma}^{(xx)}\left[C^{(xx)}\bar{\tilde{f}}\right]_{\alpha}\Big)\\
 & \phantom{=}+\frac{4}{N^{2}}Q_{\text{TL}}^{3}\,\sum_{\alpha\beta}(-\mathcal{S}^{(2)})_{\beta\alpha}^{-1}\,\left[C^{(xx)}\bar{\tilde{f}}\right]_{\alpha}\left[C^{(xx)}\bar{\tilde{f}}\right]_{\beta}\left[C^{(xx)}\bar{\tilde{f}}\right]_{\gamma}
\end{align}
Looking at the individual terms, we have for the first term in the
second line
\begin{equation}
\frac{1}{N}Q_{\text{TL}}^{2}\,\text{Tr}\left[(\kappa_{0}\mathbb{I}+Q_{\text{TL}}C^{(xx)}+\mathcal{O}(1/N))^{-1}\,C^{(xx)}\right]C_{\gamma\delta}^{(xx)}\bar{\tilde{f}}_{\delta}=\mathcal{O}(P/N),
\end{equation}
where the factor $P$ results from the appearing trace. Assuming the
regularization noise $\kappa_{0}$ to be small compared to the kernel
$C^{(xx)}$, we see for the other terms that they scale as
\begin{align}
\frac{1}{N}Q_{\text{TL}}^{2}\,\sum_{\alpha\beta}(\kappa_{0}\mathbb{I}+Q_{\text{TL}}C^{(xx)}+\mathcal{O}(1/N))_{\beta\alpha}^{-1}\left(C_{\alpha\gamma}^{(xx)}C_{\beta\delta}^{(xx)}\bar{\tilde{f}}_{\delta}+C_{\beta\gamma}^{(xx)}C_{\alpha\delta}^{(xx)}\bar{\tilde{f}}_{\delta}\right) & \approx\frac{4}{N}Q_{\text{TL}}C_{\gamma\delta}^{(xx)}\bar{\tilde{f}}_{\delta}=\mathcal{O}(1/N),\\
\frac{4}{N^{2}}Q_{\text{TL}}^{3}\,\sum_{\alpha\beta}(\kappa_{0}\mathbb{I}+Q_{\text{TL}}C^{(xx)}+\mathcal{O}(1/N))_{\beta\alpha}^{-1}\,\left[C^{(xx)}\bar{\tilde{f}}\right]_{\alpha}\left[C^{(xx)}\bar{\tilde{f}}\right]_{\beta}\left[C^{(xx)}\bar{\tilde{f}}\right]_{\gamma} & =\mathcal{O}(P/N^{2}).
\end{align}
In the proportional limit $P\propto N\rightarrow\infty$, only the
first term does not vanish and the self-consistency equation for $\bar{\tilde{f}}$
becomes
\begin{equation}
iy+\kappa_{0}\bar{\tilde{f}}+Q_{\text{TL}}\big(\bar{\tilde{f}}\big)\,C^{(xx)}\bar{\tilde{f}}-\frac{1}{N}Q_{\text{TL}}^{2}\big(\bar{\tilde{f}}\big)\,\text{Tr}\left[(\kappa_{0}\mathbb{I}+Q_{\text{TL}}\big(\bar{\tilde{f}}\big)\,C^{(xx)})^{-1}C^{(xx)}\right]C^{(xx)}\bar{\tilde{f}}\overset{!}{=}0,
\end{equation}
yielding
\begin{equation}
i\bar{\tilde{f}}=\Big(\kappa_{0}\mathbb{I}+Q_{\text{TL}}\big(\bar{\tilde{f}}\big)\,\Big(1-\frac{1}{N}Q_{\text{TL}}\big(\bar{\tilde{f}}\big)\,\text{Tr}\big[(\kappa_{0}\mathbb{I}+Q_{\text{TL}}\big(\bar{\tilde{f}}\big)\,C^{(xx)})^{-1}C^{(xx)}\big]\Big)C^{(xx)}\Big)^{-1}y.\label{eq:self_cons_eq_oneloop_approx}
\end{equation}
The rescaling factor is thus given by 
\begin{equation}
Q_{\text{1-Loop}}\coloneqq Q_{\text{TL}}-\frac{1}{N}Q_{\text{TL}}^{2}\text{Tr}\big[(\kappa_{0}\mathbb{I}+Q_{\text{TL}}C^{(xx)})^{-1}C^{(xx)}\big],\label{eq:rescaling_factor_oneloop}
\end{equation}
where $Q_{\text{TL}}=Q_{\text{TL}}\big(\bar{\tilde{f}}\big)$ depends
on the self-consistent solution in \prettyref{eq:self_cons_eq_oneloop_approx}.
The tree-level solution is the leading term here and receives a correction
due to the output fluctuations. We cannot directly compare the expression
for this rescaling factor to the one in \cite{Li21_031059}, since
the latter is given by the self-consistency equation \prettyref{eq:S_Q_z-1}
and the former by the self-consistency equation \prettyref{eq:self_cons_eq_oneloop_approx}
for the training discrepancies inserted into \prettyref{eq:rescaling_factor_oneloop}.
Nevertheless, \prettyref{fig:scatter_standard_scaling_nngp} in the
main text shows that these two expressions yield the same value and
thus the same predictions for the mean discrepancies numerically.

\section{Details of experiments\label{app:details_experiments}}

\subsection{Self-consistency equations for numerics}

In \prettyref{app:theory}, we derive train and test statistics in
a framework involving imaginary variables $\bar{\tilde{f}}$. To solve
the resulting self-consistency equations, we need to account for their
imaginary nature and substitute in all of the results above $\bar{\tilde{f}}\rightarrow i\bar{\tilde{f}}$,
changing various signs in the process. The final expressions read
as follows: In tree-level approximation, we have
\begin{align}
\bar{\tilde{f}} & =N^{\gamma-1}\Big(\kappa_{0}\mathbb{I}+Q_{\text{TL}}\big(\bar{\tilde{f}}_{\alpha}\big)C^{(xx)}\Big)^{-1}y,\\
Q_{\text{TL}}\big(\bar{\tilde{f}}\big) & =\frac{g_{w}}{1-\frac{g_{w}}{N^{\gamma}}\bar{\tilde{f}}^{\T}C^{(xx)}\bar{\tilde{f}}},\\
\langle\Delta\rangle_{\text{TL}} & =\kappa_{0}N^{1-\gamma}\bar{\tilde{f}}=\kappa_{0}\Big(\kappa_{0}\mathbb{I}+Q_{\text{TL}}\big(\bar{\tilde{f}}\big)\,C^{(xx)}\Big)^{-1}y,\\
\langle f_{\ast}\rangle_{\text{TL}} & =Q_{\text{TL}}\big(\bar{\tilde{f}}\big)\,C_{*\alpha}^{(xx)}\Big(\kappa_{0}\mathbb{I}+Q_{\text{TL}}\big(\bar{\tilde{f}}\big)\,C^{(xx)}\Big)_{\alpha\beta}^{-1}y_{\beta}.
\end{align}
In one-loop approximation, we have for the train discrepancies
\begin{align}
\bar{\tilde{f}}_{\delta} & =\big[A\big(\bar{\tilde{f}}\big)\big]_{\delta\epsilon}^{-1}\bigg[y_{\epsilon}+\frac{1}{2}\sum_{\alpha\beta}(-\mathcal{S}^{(2)})_{\beta\alpha}^{-1}\frac{\partial^{3}(-\mathcal{S})}{\partial\bar{\tilde{f}}_{\alpha}\partial\bar{\tilde{f}}_{\beta}\partial\bar{\tilde{f}}_{\epsilon}}\bigg],\\
A\big(\bar{\tilde{f}}\big) & =\kappa_{0}\mathbb{I}+Q_{\text{TL}}\big(\bar{\tilde{f}}\big)\,C^{(xx)},\\
\frac{\partial^{2}(-\mathcal{S})}{\partial\bar{\tilde{f}}_{\alpha}\partial\bar{\tilde{f}}_{\beta}}\bigg\vert_{j^{*}=0} & =A\big(\bar{\tilde{f}}\big)+\frac{2}{N}Q_{\text{TL}}^{2}\big(\bar{\tilde{f}}\big)\,C^{(xx)}\bar{\tilde{f}}\bar{\tilde{f}}^{\T}C^{(xx)},\\
\frac{\partial^{3}(\mathcal{-S})}{\partial\bar{\tilde{f}}_{\alpha}\partial\bar{\tilde{f}}_{\beta}\partial\bar{\tilde{f}}_{\delta}}\bigg\vert_{j^{*}=0} & =\frac{2}{N}Q_{\text{TL}}^{2}\big(\bar{\tilde{f}}\big)\,\Big(C_{\alpha\beta}^{(xx)}\left[C^{(xx)}\bar{\tilde{f}}\right]_{\delta}+C_{\alpha\delta}^{(xx)}\left[C^{(xx)}\bar{\tilde{f}}\right]_{\beta}+C_{\beta\delta}^{(xx)}\left[C^{(xx)}\bar{\tilde{f}}\right]_{\alpha}\Big)\\
 & \phantom{=}+\frac{8}{N^{2}}Q_{\text{TL}}^{3}\big(\bar{\tilde{f}}\big)\,\left[C^{(xx)}\bar{\tilde{f}}\right]_{\alpha}\left[C^{(xx)}\bar{\tilde{f}}\right]_{\beta}\left[C^{(xx)}\bar{\tilde{f}}\right]_{\delta},\nonumber \\
\langle\Delta_{\alpha}\rangle_{\text{1-Loop}} & =\kappa_{0}N^{1-\gamma}\bar{\tilde{f}}_{\alpha},
\end{align}
and for the test predictors
\begin{align}
\langle f^{*}\rangle_{\mathrm{\text{1-Loop}}} & =Q_{\text{TL}}\big(\bar{\tilde{f}}\big)\,C_{*\alpha}^{(xx)}\bar{\tilde{f}}_{\alpha}-\frac{1}{2}\sum_{\alpha\beta}(-\mathcal{S}^{(2)})_{\beta\alpha}^{-1}\frac{\partial^{3}(-\mathcal{S})}{\partial\bar{\tilde{f}}_{\alpha}\partial\bar{\tilde{f}}_{\beta}\partial j_{\ast}}\bigg\vert_{j^{*}=0}.\\
\frac{\partial^{3}(\mathcal{-S})}{\partial\bar{\tilde{f}}_{\alpha}\partial\bar{\tilde{f}}_{\beta}\partial j_{\ast}}\bigg\vert_{j^{*}=0} & =\frac{2}{N}Q_{\text{TL}}^{2}\big(\bar{\tilde{f}}\big)\,\Big(C_{\alpha\beta}^{(xx)}\left[C^{(xx)}\bar{\tilde{f}}\right]_{\ast}+C_{\alpha\ast}^{(xx)}\left[C^{(xx)}\bar{\tilde{f}}\right]_{\beta}+C_{\beta\ast}^{(xx)}\left[C^{(xx)}\bar{\tilde{f}}\right]_{\alpha}\Big)\\
 & \phantom{=}+\frac{8}{N^{2}}Q_{\text{TL}}^{3}\big(\bar{\tilde{f}}\big)\,\left[C^{(xx)}\bar{\tilde{f}}\right]_{\alpha}\left[C^{(xx)}\bar{\tilde{f}}\right]_{\beta}\left[C^{(xx)}\bar{\tilde{f}}\right]_{\ast},\nonumber 
\end{align}
Finally, in the proportional limit $P\propto N\rightarrow\infty$
this reduces to
\begin{align}
\bar{\tilde{f}} & =\Big(\kappa_{0}\mathbb{I}+Q_{\text{1-Loop}}\big(\bar{\tilde{f}}\big)\,C^{(xx)}\Big)^{-1}y,\\
Q_{\text{1-Loop}}\big(\bar{\tilde{f}}\big) & =Q_{\text{TL}}\big(\bar{\tilde{f}}\big)-\frac{1}{N}Q_{\text{TL}}^{2}\big(\bar{\tilde{f}}\big)\,\text{Tr}\big[(\kappa_{0}\mathbb{I}+Q_{\text{TL}}\big(\bar{\tilde{f}}\big)\,C^{(xx)})^{-1}C^{(xx)}\big],\\
\langle\Delta\rangle & =\kappa_{0}N^{1-\gamma}\Big(\kappa_{0}\mathbb{I}+Q_{\text{1-Loop}}\big(\bar{\tilde{f}}\big)\,C^{(xx)}\Big)^{-1}y,\\
\langle f^{*}\rangle_{\mathrm{\text{1-Loop, rescaling}}} & =\big[Q_{\text{1-Loop}}\big(\bar{\tilde{f}}\big)\,C_{\ast\alpha}^{(xx)}\big]\,\big(\kappa_{0}\I+Q_{\text{1-Loop}}\big(\bar{\tilde{f}}\big)\,C^{(xx)}\big)_{\alpha\beta}^{-1}\,y_{\beta}.
\end{align}


\subsection{Network tasks and training\label{app:tasks_training}}

\paragraph*{Ising task}

We use a linearly separable Ising task: Each pattern $x_{\alpha}$
in the Ising task is $D$-dimensional and $x_{\alpha i}\in\{\pm1\}$.
If the pattern belongs to class $-1$, each $x_{\alpha i}$ realizes
$x_{\alpha i}=+1$ with a probability of $p_{1}=0.5-\Delta p$ and
the value $x_{\alpha i}=-1$ with $p_{2}=0.5+\Delta p$. The value
for each pattern element $x_{\alpha i}$ is drawn independently. If
the pattern belongs to class $+1$, the probabilities for $x_{\alpha i}=1$
and $x_{\alpha i}=-1$ are inverted. The task complexity decreases
with larger $\Delta p$. We use $\Delta p=0.1$ throughout, corresponding
to an oracle accuracy on the classification task of $P_{\text{oracle}}=99,78\%$.

\paragraph*{Teacher-student task}

In this setting, the target is given by a $y_{\alpha}=w_{*}\cdot x_{\alpha}$,
where $x_{\alpha}\in\mathbb{R}^{D}$ is standard normally distributed
$x_{\alpha i}\sim\mathcal{N}(0,\mathbb{I})$. The teacher direction
$w_{*}\in\mathbb{R}^{D}$ is chosen to be $\hat{e}_{1}$ in the standard
basis.

\paragraph*{Network training}

We train networks using Langevin stochastic gradient descent (LSGD)
as detailed in \cite{Naveh21_064301} so that the trained networks
are effectively sampled from the posterior distribution \prettyref{eq:dist_output}.
Here evolving network parameters $\Theta$ such as weights $V,w$
with the stochastic differential equation
\begin{align}
\partial_{t}\Theta(t) & =-\rho\Theta(t)-\nabla_{\Theta}\mathcal{L}(\Theta(t);y)+\sqrt{2T}\zeta(t),\label{eq:Appendix_SingleParameter_LSGD}\\
\big\langle\zeta_{i}(t)\zeta_{j}(s)\big\rangle & =\delta_{ij}\delta(t-s),\nonumber 
\end{align}
with the squared error loss $\mathcal{L}(\Theta;y)=\sum_{\alpha=1}^{P}(f_{\alpha}(\Theta)-y_{\alpha})^{2}$,
$\zeta$ a unit variance Gaussian white noise, and $f_{\alpha}(\Theta)$
denoting the network output for sample $\alpha=1,\ldots,P$, leads
to sampling from the equilibrium distribution for $\Theta$ for large
times $t$ which reads
\begin{equation}
\lim_{t\rightarrow\infty}p\left(\Theta(t)\right)\sim\exp\left(-\frac{\rho}{2T}\|\Theta\|^{2}-\frac{1}{T}\mathcal{L}(\Theta;y)\right).\label{eq:Appendix_equilibrated_dist}
\end{equation}
Using the Fokker-Planck equation \cite{Risken96} one can derive this
density for $\Theta$. Further, this implies a distribution on the
network output
\begin{align}
p(Y|X)\propto & \int d\Theta\,\exp\big(-\frac{\rho}{2T}\,\|\Theta\|^{2}-\frac{1}{T}\,\|f-y\|^{2}\big)\\
\propto & \Big\langle\exp\big(-\frac{1}{T}\,\|f-y\|^{2}\big)\Big\rangle_{\Theta_{k}\stackrel{\text{i.i.d.}}{\sim}\N(0,T/\rho)}\nonumber \\
\propto & \N(y|f,T/2)\,\langle\delta\big[f-f(\Theta)\big]\rangle_{\Theta_{k}\stackrel{\text{i.i.d.}}{\sim}\N(0,T/\rho)},\nonumber 
\end{align}
In fact, $p(f|X)\equiv\langle\delta\big[f-f(\Theta)\big]\rangle_{\Theta_{k}\stackrel{\text{i.i.d.}}{\sim}\N(0,T/\rho)}$,
leads to the posterior in \prettyref{eq:cum_gen_posterior} if one
identifies $\kappa_{0}=T/2$ with the regularization noise and $T/\ensuremath{\rho}=g/N$
with the variance of the parameter $\Theta_{k}$. Implementing the
sampling in practice this corresponds to requiring different weight
decay $\rho$ for each parameter, as weight variances can differ in
the input and output layer.

The time discrete version of \prettyref{eq:Appendix_SingleParameter_LSGD}
is implemented in our PyTorch code as 
\begin{align}
\Theta_{t} & =\Theta_{t-1}-\eta\left(\rho\Theta_{t-1}+\nabla_{\Theta}\mathcal{L}(\Theta_{t-1};y)\right)+\sqrt{2T\eta}\,\zeta_{t},\\
\langle\zeta_{t}\zeta_{s}\rangle & =\delta_{ts},\nonumber 
\end{align}
with standard normal $\zeta_{t}$ and finite time step $\eta$, which
can also be interpreted as a learning rate. To accurately reflect
the time evolution according to \prettyref{eq:Appendix_SingleParameter_LSGD}
the learning rate $\eta$ needs to be small enough.

Hence the LSGD we implement corresponds to full-batch gradient descent
with the addition of i.i.d. distributed standard normal noise and
weight decay regularization \cite{Krogh91}. The value for $\kappa_{0}$
corresponds to a tradeoff in the optimization between the weight priors
and the likelihood in terms of the loss $\mathcal{L}$. Choosing large
$\kappa_{0}$ corresponds to large $T=2\kappa_{0}$ and hence a large
noise in the LSGD and therefore putting more emphasis on the Gaussian
parameter priors. Small regularization values $\kappa_{0}$ favor
the training data in terms of the loss in the exponent of \eqref{eq:Appendix_equilibrated_dist}.

To faithfully compare the numerical results with our theoretical results,
the LSGD needs to sample from the equilibrium distribution. For this
it needs to be ensured that the distribution is equilibrated by evolving
the networks for $10.000$ steps . We ensure uncorrelated network
samples by initializing different networks with different random seeds.

For the Ising task, we average over $N_{\text{networks}}=100$ with
different initial weights to obtain the training and test predictors.
For the teacher-student task, we average over $N_{\text{networks}}=5.000$
with different initial weights to obtain the covariance of the network
output projected onto different directions.

\section{Additional figures\label{app:add_figures}}

\begin{figure*}
\vskip 0.2in
\begin{centering}
\includegraphics[width=1\textwidth]{scatter_discrepancies_mean_field_scaling_appendix}\caption{Scatter plots of (a) training discrepancies $\langle\Delta_{\alpha}\rangle=y_{\alpha}-\langle f_{\alpha}\rangle$
and (b) test discrepancies $\langle\Delta_{\ast}\rangle=y_{\ast}-\langle f_{\ast}\rangle$
on an Ising task in mean-field scaling. We show theoretical values
for NNGP and different feature learning theories against empirical
results, where the gray line marks the identity. In contrast to the
NNGP, the tree-level approximation accurately matches the empirical
values. Further, the different feature learning theories lie on top
of one another in mean-field scaling. Parameters: $\gamma=2$, $P_{\text{train}}=80$,
$N=100$, $D=200$, $\kappa_{0}=0.4$, $P_{\text{test}}=10^{3}$,
$g_{v}=0.5$, $g_{w}=0.2$, $\Delta p=0.1$.}
\label{fig:scatter_mean_field-1}
\par\end{centering}
\vskip -0.2in
\end{figure*}
\begin{figure*}
\vskip 0.2in
\begin{centering}
\includegraphics[width=1\textwidth]{fluctuations_teacher_student_appendix}\caption{(a) Directional feature learning in a teacher-student setting as a
function of the fluctuation scale $1/\chi$. Both NNGP and rescaling
theory fail to capture directional feature learning, while the multi-scale
adaptive theory accurately predicts network behavior. Output distribution
in different directions (b) in mean field scaling ($\chi=N$) and
(c) in standard scaling ($\chi=128$) . Parameters: $P_{\text{train}}=80$,
$N=200$, $D=50$, $\kappa_{0}=2$, $g_{v}=0.01,$ $g_{w}=2$.}
\label{fig:teacher_student_appendix}
\par\end{centering}
\vskip -0.2in
\end{figure*}

\end{document}
