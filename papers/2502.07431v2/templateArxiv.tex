\documentclass{article}


\usepackage{PRIMEarxiv}

\usepackage[utf8]{inputenc} % allow utf-8 input
\usepackage[T1]{fontenc}    % use 8-bit T1 fonts
\usepackage{hyperref}       % hyperlinks
\usepackage{url}            % simple URL typesetting
\usepackage{booktabs}       % professional-quality tables
\usepackage{amsfonts}       % blackboard math symbols
\usepackage{nicefrac}       % compact symbols for 1/2, etc.
\usepackage{microtype}      % microtypography
\usepackage{lipsum}
\usepackage{fancyhdr}       % header
\usepackage{graphicx}       % graphics
\usepackage{enumitem}
\usepackage{subcaption}
\usepackage{makecell}
\usepackage{amsmath}
\usepackage{pifont}

\graphicspath{{media/}}     % organize your images and other figures under media/ folder

%Header
\pagestyle{fancy}
\thispagestyle{empty}
\rhead{ \textit{ }} 

% Update your Headers here

% \fancyhead[RE]{Firstauthor and Secondauthor} % Firstauthor et al. if more than 2 - must use \documentclass[twoside]{article}



  
%% Title
\title{ArthroPhase: A Novel Dataset and Method for Phase Recognition in Arthroscopic Video
%%%% Cite as
%%%% Update your official citation here when published 
\thanks{
\textbf{This is a preprint of a manuscript submitted to Computer Assisted Surgery. The final version may be different after peer review and acceptance.}} 
}

\author{
  Ali Bahari Malayeri, Matthias Seibold, Nicola Cavalcanti, Jonas Hein, Sascha Jecklin \\
  Research in Orthopedic Computer Science (ROCS)\\
  Balgrist University Hospital, University of Zurich \\
  Zürich, Switzerland \\
  \texttt{Corresponding author: ali.baharimalayeri@balgrist.ch} \\
   \And
  Lazaros Vlachopoulos, Sandro Fucentese, Sandro Hodel \\
  Department of Orthopedics \\
  Balgrist University Hospital, University of Zurich \\
  Zürich, Switzerland \\
   \And
  Philipp Fürnstahl \\
  Research in Orthopedic Computer Science (ROCS)\\
  Balgrist University Hospital, University of Zurich \\
  Zürich, Switzerland \\
}



\begin{document}
\maketitle


\begin{abstract}
\textbf{Purpose} \\
This study aims to advance surgical phase recognition in arthroscopic procedures, specifically Anterior Cruciate Ligament (ACL) reconstruction, by introducing the first arthroscopy dataset and developing a novel transformer-based model. We aim to establish a benchmark for arthroscopic surgical phase recognition by leveraging spatio-temporal features to address the specific challenges of arthroscopic videos including limited field of view, occlusions, and visual distortions.

\textbf{Methods} \\
We developed the ACL27 dataset, comprising 27 videos of ACL surgeries, each labeled with surgical phases. Our model employs a transformer-based architecture, utilizing temporal-aware frame-wise feature extraction through a ResNet-50 and transformer layers. This approach integrates spatio-temporal features and introduces a Surgical Progress Index (SPI) to quantify surgery progression. The model's performance was evaluated using accuracy, precision, recall, and Jaccard Index on the ACL27 and Cholec80 datasets.

\textbf{Results} \\
The proposed model achieved an overall accuracy of 72.91\% on the ACL27 dataset, with a precision of 72.86\% and Jaccard Index of 57.39\%. On the Cholec80 dataset, the model achieved a comparable performance with the state-of-the-art methods with an accuracy of 92.4\%, precision of 85.6\%, recall of 80.6\%. The SPI demonstrated an output error of 10.6\% and 9.86\% on ACL27 and Cholec80 datasets respectively, indicating reliable surgery progression estimation.

\textbf{Conclusion} \\
This study introduces a significant advancement in surgical phase recognition for arthroscopy, providing a comprehensive dataset and a robust transformer-based model. The results validate the model's effectiveness and generalizability, highlighting its potential to improve surgical training, real-time assistance, and operational efficiency in orthopedic surgery. The publicly available dataset and code will facilitate future research and development in this critical field.\end{abstract}


% keywords can be removed
\keywords{Surgical phase recognition \and orthopedics \and arthroscopy \and transformer}


\section{Introduction}
In the field of surgical interventions, automated phase recognition and analysis enable a wide range of downstream applications, i.e. detecting complications, optimise clinical workflows, documenting procedures, and generating automated surgery reports, thus emphasizing the critical importance of surgical phase recognition \cite{1.maier2017surgical}. Furthermore, understanding the surgical context is a fundamental task for the development of advanced computer-aided surgery systems that enable intelligent surgical assistance. By leveraging various data modalities such as signals \cite{21.padoy2012statistical} and video \cite{2.twinanda2016endonet}, particularly endoscopic video as one of the most promising data sources, machine learning and advanced algorithmic approaches have significantly enhanced the capabilities of surgical systems \cite{deo2015machine}.

Automated surgical workflow recognition for clinical applications such as cholecystectomy, cataract removal, and rectal resection have been extensively investigated in previous work \cite{demir2023deep}. While there has been some attempts in summarizing arthroscopic videos, such as the work by Lux et al. on keyframe-based video summarization \cite{lux2010novel}, comprehensive surgical workflow recognition for arthroscopic procedures remains largely unexplored. This gap underscores the importance and drives our research in automatic phase detection for arthroscopy. Surgical phase recognition in arthroscopy presents several unique challenges, including blurry frames, scenes obscured by liquid, floating particles, limited workspace, and inherent ambiguity in individual frames. Our approach addresses these challenges by leveraging a transformer-based model and introducing a novel dataset specifically for arthroscopic surgical phase recognition, aiming to establish a scientific benchmark and improve the accuracy and applicability of phase recognition in this domain.

The problem of surgical phase recognition has initially been tackled using methods like Hidden Markov Model (HMM) that operated on manually crafted binary signals from the OR, such as synchronized signals acquired over time, thereby setting foundational techniques in the field \cite{21.padoy2012statistical}. Later, frame-based analysis using deep learning models, particularly image-based Convolutional Neural Networks (CNNs), marked a significant advancement by classifying individual frames based on spatial features \cite{2.twinanda2016endonet}; subsequently, spatio-temporal methods were developed to not only improve feature extraction but also explicitly model the dependencies between frames in a sequence, a critical aspect in surgical video analysis, with approaches such as Long-Short-Term Memory (LSTM) \cite{17_10.1162/neco.1997.9.8.1735} and Temporal Convolutional Networks (TCN) \cite{13.lea2017temporal}, significantly enhancing the ability to capture long-range dependencies within surgical videos. More recently, Transformer-based methods have revolutionized the approach by efficiently managing these dependencies across extensive sequences using self-attention mechanisms and positional embedding \cite{14.vaswani2017attention}, allowing to handle much longer sequences.

This study presents the first comprehensive investigation into surgical phase recognition in the field of arthroscopy, introducing a novel dataset consisting of 27 videos captured from ACL surgeries including surgical phase labels. We propose a novel method that sets a scientific benchmark in the field of surgical phase recognition for arthroscopic procedures. Our approach utilizes a transformer-based model to tackle the challenges of surgical phase recognition in arthroscopy. We employ temporal-aware frame-based features essential for accurately analyzing and predicting phases in long surgical videos. These spatio-temporally coherent features are fed into two branches of transformers that differ in input length, enhancing temporal resolution and precision in identifying and classifying surgical phases. In addition to phase recognition, our model incorporates a novel parallel output termed the Surgical Progress Index (SPI), which provides a straightforward measure for the progression of the surgical procedures. The associated code and dataset will be made publicly available upon acceptance. The contributions of this work can be summarized as follows:

\begin{itemize}[leftmargin=0.8cm]
    \item We establish a foundational scientific benchmark for surgical phase recognition in arthroscopy by providing a novel publicly available dataset consisting of 27 Anterior Cruciate Ligament surgery videos with surgical phase labels.
    \item We propose an effective mechanism, the Surgical Progress Index (SPI), which implicitly integrates global temporal context and provides a valuable measure for the progress of a surgery.
    \item We propose the utilization of spatio-temporal features, provide a comprehensive ablation study and benchmark the proposed method with tate-of-the-art approaches from other domains like laparoscopy.
\end{itemize}


\section{State-of-the-art}
In the early stages of surgical phase recognition, manually recorded data on tool usage and equipment were used to interpret and classify different segments of surgery. Algorithms like Dynamic Time Warping and Hidden Markov Models were applied to process this data, particularly in cholecystectomy surgeries \cite{21.padoy2012statistical}. These methods aimed to effectively segment and recognize workflow phases \cite{21.padoy2012statistical}. Automated surgical phase recognition evolved significantly with the introduction of EndoNet \cite{2.twinanda2016endonet}, the first model to utilize Convolutional Neural Networks (CNN) for extracting spatial features from video data. This approach was further developed by integrating these features into a hierarchical Hidden Markov Model to capture temporal dynamics, establishing the groundwork for multi-task recognition within cholecystectomy surgery videos. Enhancing the foundational approach of EndoNet, Twinanda \textit{et al.} \cite{Twinanda2016SingleAM} substituted the hierarchical HMM with LSTM networks to deepen the model's understanding of temporal sequences, marking a pivotal shift towards intricate temporal analysis. This advancement paved the way for the SV-RCNet framework, which harnessed the combined power of ResNet and LSTM to holistically capture spatial and temporal features, marking a significant improvement in spatio-temporal feature representation within an end-to-end learning framework \cite{10_jin2017sv}. Up to this point, most papers used Cholec80, Cholec120, and M2CAI16 datasets which contain laparoscopic videos of cholecystectomy surgeries. Subsequently, a video dataset named Cataract-101 \cite{x1_schoeffmann2018cataract}, containing recordings of 101 cataract surgeries, was published. Yu \textit{et al.} \cite{x2_yu2019assessment} leveraged the Cataract-101 dataset, employing various deep transfer learning algorithms to classify surgery phases in cataract surgery videos, inspiring further research for this clinical application. The advancement of surgical phase recognition continued with the development of MTRCNet-CL by Jin \textit{et al.} \cite{18_jin2020multi}, which incorporated a correlation loss to leveraging the relationship between tool detection and phase recognition using Cholec80, thereby enhancing the performance of both tasks simultaneously. This model introduces the strategic use of multiple loss terms to refine learning outcomes. Addressing the limitations of LSTM in handling long sequences, Czempiel \textit{et al.} \cite{22_czempiel2020tecno} proposed TeCNO, which utilizes Temporal Convolutional Networks (TCN) for enhanced temporal feature integration in surgical videos. This method marks a significant improvement in the continuous and dynamic recognition of surgical phases, refining phase detection capabilities in complex video sequences.

Building on these achievements, the introduction of Transformer models \cite{21_vaswani2017attention}, originally developed for Natural Language Processing (NLP), has significantly advanced surgical video analysis. Their inherent ability to analyze elements of sequences in parallel, without the constraints of sequential processing, makes them particularly suited for maintaining temporal context in the extended sequences typical of surgical video, thus aiding in precise phase recognition. OperA \cite{czempiel2021opera} by Czempiel \textit{et al.} utilizes a transformer-based approach with attention regularization to significantly enhance surgical phase recognition. Trans-SVNet introduced by Gao \textit{et al.} \cite{gao2021trans} represents another significant advancement in Transformer application for surgical phase recognition. This model uniquely combines ResNet and TCN to generate spatial and temporal embeddings, which are then effectively aggregated using a Transformer framework to enhance phase recognition accuracy. Liu \textit{et al.} \cite{liu2023lovit} introduced LoViT, which enhances online surgical phase recognition by integrating a temporally-rich spatial feature extractor with a multi-scale temporal aggregator. Utilizing self-attention and ProbSparse mechanisms, LoViT efficiently processes local and global features, achieving superior performance and robustness compared to state-of-the-art methods. Additionally, Liu \textit{et al.} \cite{liu2023skit} developed SKiT, a model that uses an efficient key pooling operation to capture crucial information in lengthy surgical videos, achieving higher accuracy and constant inference time, making it suitable for real-time recognition systems. Holm \textit{et al.} \cite{holm2023dynamic} further expanded the scope of surgical phase recognition by employing dynamic scene graphs to represent surgical videos, enhancing the robustness and explainability of model predictions.















\section{Methods}
\subsection{Development and Annotation of a Comprehensive ACL Surgery Dataset}
% We introduce a unique dataset of arthroscopic video collected at Balgrist University Hospital
We introduce a unique dataset of arthroscopic video collected in our institution, using a Karl Storz Laparoscope. Specifically, we identified 27 surgery recordings focusing on Anterior Cruciate Ligament (ACL) reconstruction, collected from January 2022 until June 2023, that met our inclusion criteria: they exclusively include standard ACL procedural steps and involve cases without any additional pathologies requiring further surgical interventions. The included videos vary from 6 to 48 minutes, totaling approximately 13 hours of surgical footage. We labeled the data with surgical phase labels in close collaboration with clinical experts to ensure clinical validity and high data quality. Each surgery is segmented into five distinct phases: Preparation, Diagnosis, Femoral Tunnel Creation, Tibial Tunnel Creation, and ACL Reconstruction, with labels assigned at one label per second as detailed in Table \ref{tab:acl_phases}.



Fig. \ref{fig:phases} illustrates frames belonging to the respective phases. The figure also highlights the specific challenges of recognizing phases in arthroscopic ACL surgery videos compared to laparoscopic procedures like those in the Cholec80 dataset. Arthroscopy involves a smaller and more confined field of view, and the presence of fluids causes light reflection and refraction, creating visual distortions. Precise instrument navigation in narrow joint spaces increases the likelihood of occlusions, and the camera's proximity to the surgical site necessitates frequent adjustments. Blood and debris can quickly cloud the view, and variable lighting conditions lead to glare and shadows. The similar visual textures of joint structures make tissue differentiation difficult, and any movement can cause significant video disruptions. These challenges necessitate advanced recognition techniques to improve accuracy in surgical training and real-time assistance.












\begin{figure}[h!]
    \centering
    % Row 1
    \begin{subfigure}[t]{0.18\textwidth}
        \includegraphics[width=\textwidth]{figures/output_110.jpg}
    \end{subfigure}
    \hfill
    \begin{subfigure}[t]{0.18\textwidth}
        \includegraphics[width=\textwidth]{figures/output_303.jpg}
    \end{subfigure}
    \hfill
    \begin{subfigure}[t]{0.18\textwidth}
        \includegraphics[width=\textwidth]{figures/output_870.jpg}
    \end{subfigure}
    \hfill
    \begin{subfigure}[t]{0.18\textwidth}
        \includegraphics[width=\textwidth]{figures/output_1469.jpg}
    \end{subfigure}
    \hfill
    \begin{subfigure}[t]{0.18\textwidth}
        \includegraphics[width=\textwidth]{figures/output_1925.jpg}
    \end{subfigure}
    
    % Add vertical space between rows
    \vspace{1em}
    
    % Row 2
    \begin{subfigure}[t]{0.18\textwidth}
        \includegraphics[width=\textwidth]{Bluryfig/output_158.jpg}
        \caption*{Preparation}
    \end{subfigure}
    \hfill
    \begin{subfigure}[t]{0.18\textwidth}
        \includegraphics[width=\textwidth]{Bluryfig/output_438.jpg}
        \caption*{Diagnosis}
    \end{subfigure}
    \hfill
    \begin{subfigure}[t]{0.18\textwidth}
        \includegraphics[width=\textwidth]{Bluryfig/output_1256.jpg}
        \caption*{Femoral Tunnel}
    \end{subfigure}
    \hfill
    \begin{subfigure}[t]{0.18\textwidth}
        \includegraphics[width=\textwidth]{Bluryfig/output_1807.jpg}
        \caption*{Tibial Tunnel}
    \end{subfigure}
    \hfill
    \begin{subfigure}[t]{0.18\textwidth}
        \includegraphics[width=\textwidth]{Bluryfig/output_2242.jpg}
        \caption*{ACL Reko}
    \end{subfigure}
    
    \caption{Top row: Clear images from each phase of ACL surgery, showcasing identifiable stages of the procedure. Bottom row: Corresponding unclear or blurry frames, demonstrating the challenge of phase recognition in arthroscopy. This contrast highlights the importance of advanced recognition techniques to improve accuracy in surgical training and real-time assistance.}
    \label{fig:phases}
\end{figure}






\begin{table}[h!]
    \centering
    \begin{tabular}{|l|c|c|c|}
        \hline
        \textbf{Phase} & \textbf{Total Duration} & \makecell{\textbf{\# of Videos} \\ \textbf{Containing Phase}} & \makecell{\textbf{Average Phase} \\ \textbf{Length}} \\
        \hline
        Preparation & 50\textquotesingle :40\textquotedbl & 19 & 2\textquotesingle :40\textquotedbl \\
        Diagnosis & 208\textquotesingle :32\textquotedbl & 25 & 8\textquotesingle :20\textquotedbl \\
        Femoral Tunnel Creation & 265\textquotesingle :54\textquotedbl & 26 & 10\textquotesingle :14\textquotedbl \\
        Tibial Tunnel Creation & 214\textquotesingle :02\textquotedbl & 25 & 8\textquotesingle :34\textquotedbl \\
        ACL Reconstruction & 54\textquotesingle :38\textquotedbl & 16 & 3\textquotesingle :25\textquotedbl \\
        \hline
    \end{tabular}

    \vspace{5pt} % Adds a small gap between the table and the caption

    \caption{This table provides a detailed breakdown of the ACL surgery dataset, highlighting the duration and distribution of the different surgical phases.}
    \label{tab:acl_phases}
\end{table}

















\subsection{Temporal-aware Frame-wise Feature Extraction}
The detection of the surgical phases requires the analysis of a longer temporal context of surgical video, as high-level surgical phases are composed of several surgical actions, tool usage, and sub-tasks. However, the processing of long sequences of video data introduces a high computational complexity. Therefore, in previous work, dimensionality reduction techniques, such as the extraction of intermediate features as a preprocessing step, have been introduced \cite{2.twinanda2016endonet,czempiel2021opera}. While initially, frame-based feature extraction techniques were employed \cite{10_jin2017sv}, the integration of temporal context in the feature extraction stage has been shown to lead to higher-quality features and resulting performance in the downstream task of surgical phase prediction \cite{liu2023lovit}.

Our model, as illustrated in Fig. \ref{fig:architecture}, takes a novel approach by emphasizing temporal-aware feature extraction from the outset. Unlike traditional methods that rely solely on frame-based feature extraction, our approach begins with a ResNet-50 model to perform initial feature extraction on three-channel images of size 240x240 pixels. This step generates a 2048-dimensional feature vector for each frame. However, we further enhance these features by employing a transformer layer with 2 heads to process and refine the extracted features, thus ensuring a deeper integration of temporal data. This dual approach of combining spatial features with temporal context right from the feature extraction phase enables our model to capture and utilize the sequential dependencies within the video data more effectively, enhancing the precision and accuracy of surgical phase recognition.

Building upon this enhanced feature set, we utilize a Spatial Feature Encoder (SFE) and a Temporal Context Encoder (TCE), as illustrated in figure \ref{fig:architecture}, to extract features that take the sequential nature of video data into account. The SFE transforms each video frame, denoted as \(x_t \in \mathbb{R}^{H \times W \times C}\), into a spatial feature vector. These spatial features are then compiled into a sequence for a given window of frames \(n\), creating a contiguous representation of the surgical procedure over time. The TCE processes this sequence to merge the spatial information with the temporal context, outputting a comprehensive spatio-temporal feature vector. 
\begin{figure}[h!]
    \captionsetup{belowskip=0pt, aboveskip=5pt}
    \centering
    \includegraphics[width=\textwidth, trim=5 0pt 5pt 0pt, clip]{figures/cropped_FeatEx.pdf}
    \caption{Architecture of our spatio-temporal feature extractor showcasing the process from initial feature extraction using ResNet-50 to enhanced temporal resolution via a transformer layer. The Spatial Feature Encoder (SFE) and Temporal Context Encoder (TCE) further analyze and integrate these features to produce comprehensive spatio-temporal vectors, critical for accurate surgical phase prediction.}
    \label{fig:architecture}
\end{figure}





\subsection{Surgical Progress Index}
Traditionally, surgical phase recognition models classify discrete phases without a continuous understanding of time. However, the progression of surgery is inherently sequential and can benefit from a regression-based approach, where understanding the flow of each phase is crucial. To enhance the capabilities of our phase recognition model, we introduce the Surgical Progress Index (SPI), a novel metric designed to quantify the progression of the surgery. The SPI addresses this by providing a continuous measure that enables the model to implicitly incorporate the global temporal context of the surgical procedure. This ability to integrate global context without processing the entire video sequence makes the SPI particularly valuable. It allows the model to make more accurate predictions and offers a deeper insight into the surgical phase by bridging the gap between mere classification and holistic temporal analysis, thus enhancing the precision and utility of our surgical phase recognition system. The overall architecture of the proposed model, including the learned features and the SPI component, is illustrated in Fig. \ref{fig:architecture2}.

\begin{figure}[h!]
    \centering
    \includegraphics[width=\textwidth, trim=150pt 200pt 150pt 50pt, clip]{figures/Transformer.pdf}
    \caption{Overview of the final model for surgical phase recognition. The architecture has a transformer branch which processes an 80-second sequence of features. These features are derived from our spatio-temporal feature extractor, which combines spatial information from ResNet-50 with temporal context through transformer layers.}
    \label{fig:architecture2}
\end{figure}




rgical Progress Index (SPI) can be calculated in a straightforward manner, requiring minimal additional labelling effort. For a given surgery recording $k$, $t$ represents the elapsed time since the beginning of the surgery, which corresponds to the frame index at the last frame of the current sequence, considering that the video is recorded at a rate of 1 frame per second. \( T_k \) represents the entire duration of the surgery, corresponding to the total number of frames in the recording. Finally, the SPI is calculated as:
\[
SPI(t,k) = \frac{t}{T_k}
\]
This ratio provides a continuous measure, indicating the percentage completion of the surgery at any given time step. However, variations in surgical practices and recording methods can lead to missing phases in videos, potentially misleading the model’s predictions; to address this, we use only videos that contain all predefined surgical phases for baseline computation (12 videos in ACL27 and 65 in Cholec80). For surgeries with missing phases, we establish an average progression percentage for each phase using the complete videos. Table \ref{tab:spi_transitions} gives an overview of the statistical distribution of the coverage of phases throughout the dataset. These statistics are used to compute an average duration for missing phases.  The calculation for the SPI of a frame \( t \) in a video \( n \), which might have missing initial or final phases, is given by:

\[
SPI(t,k,i)^{\text{adjusted}} = \frac{t}{T_k} + \sum_{i \in \text{missing phases}} \text{avg\_SPI}(i) \cdot I(\text{phase } i \text{ is missing})
\]

where the indicator function \(I(\text{phase } i \text{ is missing})\) is defined as:

\[
I(\text{phase } i \text{ is missing}) = 
\begin{cases} 
1 & \text{if phase } i \text{ is missing in video } k \\
0 & \text{otherwise}
\end{cases}
\]


In this formula, \( t \) represents the frame index of the current sequence, and \( T_k \) denotes the total number of frames in the video. The term \(\text{avg\_SPI}(i)\) refers to the average SPI of the missing phase \( i \), which is determined from the analysis of videos that include all phases, as detailed in Table \ref{tab:spi_transitions}.

For instance, if a video starts from the second phase, the SPI for the first frame of this video is set to the average SPI of the first phase. This average SPI is calculated based on the analysis of videos containing all phases. By making this adjustment, the SPI more accurately reflects the surgical progress, thereby enhancing the model’s ability to generalize across various surgical recordings.





\begin{table}[h!]
    \centering
    \resizebox{\textwidth}{!}{
    \begin{tabular}{|c|c|c|c|c|c|c|c|c|}
        \hline
        & \makecell{\textbf{First} \\ \textbf{Frame}} 
        & \makecell{\textbf{First} \\ \textbf{Transition}} 
        & \makecell{\textbf{Second} \\ \textbf{Transition}} 
        & \makecell{\textbf{Third} \\ \textbf{Transition}} 
        & \makecell{\textbf{Fourth} \\ \textbf{Transition}} 
        & \makecell{\textbf{Fifth Transition} \\ \textbf{Last (ACL)}} 
        & \makecell{\textbf{Sixth} \\ \textbf{Transition}} 
        & \makecell{\textbf{Last Frame} \\ \textbf{(Cholec80)}} \\
        \hline
        \textbf{Cholec80} & 0.0 & 0.051 & 0.452 & 0.530 & 0.847 & 0.885 & 0.964 & 1.000 \\
        \hline
        \textbf{ACL27} & 0.0 & 0.073 & 0.309 & 0.534 & 0.765 & 1.000 & & \\
        \hline
    \end{tabular}
    }
    \vspace{5pt} % Small gap before caption
    \caption{Average SPI transitions for Cholec80 and ACL27 datasets.}
    \label{tab:spi_transitions}
\end{table}







\subsection{Loss Function and Training Details}
For model training, we employ a composite loss function that integrates both phase classification and progress regression. Specifically, for the classification of surgical phases, we use Sparse Categorical Cross Entropy, and for SPI regression, we use Mean Absolute Error (MAE). The combined loss function is expressed as:
\[
L(p, y) = \lambda \left( -\sum_{i} y_i \log(p_i) \right) + (1 - \lambda) \left( \frac{1}{N} \sum_{i=1}^{N} |p_i - y_i| \right)
\]
where \( y_i \) is the ground truth label, \( p_i \) is the model's predicted probability for class \( i \), \( N \) is the number of samples, and \( \lambda \) is the weight assigned to each loss component. For our model, we set \( \lambda = 0.5 \), giving equal importance to both the classification accuracy and the regression precision.

All experiments were conducted using the TensorFlow 2.10.1 framework on an NVIDIA RTX-A6000 GPU. The model was trained using the Adam optimizer with an initial learning rate of 0.000005, scheduled to decrease by a factor of ten after every 10 epochs, over a total of 30 epochs. The batch size for training was set at 32. For the cross-validation procedure, the datasets were randomly divided into five equal partitions for both the ACL27 and Cholec80 datasets. In this context, the Cholec80 dataset was split such that 40 videos were allocated for training and testing the model's performance, respectively. Similarly, the ACL27 dataset was divided with 18 videos used for training and the remaining 9 videos reserved for validation. For the experiments not involving cross-validation, the data split was executed in the following manner. This approach was adopted to facilitate comparison with the majority of other studies in this domain, which typically followed the same methodology. Consequently, for the Cholec80 dataset, the first 40 videos were selected for training, and for the ACL27 dataset, the first 18 videos were chosen for training purposes. Results marked with an uppercase plus (+) indicate that they were obtained using the non-cross-validation data split.

\section{Results}
\subsection{Evaluation Metrics}
To determine the effectiveness of our surgical phase recognition model, we use a combination of standard metrics: Accuracy, Precision, Recall, and Jaccard Index. Accuracy evaluates the model's ability to correctly identify surgical phases in videos, irrespective of their duration. Given the imbalanced nature of our dataset, with some phases being significantly shorter than others, we also focus on average Precision, Recall, and Jaccard Index for a more detailed evaluation. These metrics collectively quantify how well our model predicts surgical phases and provide a comprehensive view of the model’s performance at both the video and phase level without being influenced by phase length variations. In previous research, some studies used cross-validation \cite{22_czempiel2020tecno,czempiel2021opera} while others did not and instead reported sample-based standard deviation (STD) \cite{2.twinanda2016endonet,18_jin2020multi, liu2023skit}. To enable comparison with all these studies, we evaluated both.

\subsection{Comparative Analysis with Existing Models}
To evaluate the performance of our model against existing methods, we implemented and trained Trans\_SVNet, MTRCNet, TeCNO, and Opera on the ACL27 dataset. The results based on the evaluation without cross-validation are summarized in Table \ref{tab:comparison_results}. Our model achieved an accuracy 72.91\%, a precision of 72.86\%, and a Jaccard Index of 57.39\%. In comparison, Trans\_SVNet models, evaluated with sequence lengths of 30, 60, and 80, achieved accuracies of 66.62\%, 66.75\%, and 66.73\%, respectively, with corresponding Jaccard Indices of 47.96\%, 48.05\%, and 48.03\%. MTRCNet and TeCNO reported lower performance, with MTRCNet achieving an accuracy of 58.75\% and TeCNO achieving 58.31\%. This comparative analysis shows that our model outperforms existing methods on the new ACL27 dataset, particularly in terms of accuracy and precision.




\begin{table}[h!]
    \centering
    \begin{tabular}{|l|c|c|c|c|}
        \hline
        \textbf{Model} & \textbf{Accuracy (\%)} & \textbf{Precision (\%)} & \makecell{\textbf{Jaccard} \\ \textbf{Index (\%)}} & \textbf{F1-Score (\%)} \\
        \hline
        Trans\_SVNet\textsuperscript{+}~\cite{gao2021trans} (Seq = 30s) & 66.62 & 66.51 & 47.96 & - \\
        Trans\_SVNet\textsuperscript{+}~\cite{gao2021trans} (Seq = 60s) & 66.75 & 66.45 & 48.05 & - \\
        Trans\_SVNet\textsuperscript{+}~\cite{gao2021trans} (Seq = 80s) & 66.73 & 66.43 & 48.03 & - \\
        MTRCNet\textsuperscript{+}~\cite{18_jin2020multi} (Seq = 4s) & 58.75 & 59 & - & 58 \\
        MTRCNet\textsuperscript{+}~\cite{18_jin2020multi} (Seq = 10s) & 60.68 & 63 & - & 59 \\
        TeCNO\textsuperscript{+}~\cite{22_czempiel2020tecno} & 58.31 & 66.05 & 40.66 & - \\
        \textbf{Ours\textsuperscript{+}} & \textbf{72.91} & \textbf{72.86} & \textbf{57.39} & - \\
        \hline
    \end{tabular}
    \vspace{5pt} % Small gap before caption
    \caption{Comparison of performance metrics for different models on the ACL27 dataset.}
    \label{tab:comparison_results}
\end{table}


These results highlight the model's capability to accurately classify surgical phases and predict the progression of surgery. In addition to phase recognition, the model's performance in predicting the Surgical Progress Index (SPI) was evaluated. The SPI, which quantifies the progression of the surgery on a scale from 0 to 1, demonstrated an output error of 10.6\%. This low error rate indicates the model's effectiveness in providing a reliable estimate of the surgery's progression, thus enabling a more continuous and nuanced understanding of the surgical workflow.

\subsection{Ablation Study}

To thoroughly understand the contributions of the proposed components in our model, we conducted an ablation study. This study evaluates the inclusion of the Surgical Progress Index (SPI) and the spatio-temporal features on the performance of our model. The results of different configurations are summarized in Table \ref{tab:ablation_study}. The results were obtained using a 5-round cross-validation approach to ensure robustness and reliability of the findings.




\begin{table}[h!]
    \centering
    \begin{tabular}{|c|c|c|c|c|c|}
        \hline
        \makecell{\textbf{Spatio-temporal} \\ \textbf{Features}} & 
        \textbf{SPI} &
        \textbf{Accuracy} &
        \textbf{Precision} &
        \textbf{Recall} &
        \makecell{\textbf{Jaccard} \\ \textbf{Index}} \\
        \hline
        \ding{55} & \ding{55} & $66.38 \pm 1.71$ & $66.818 \pm 2.26$ & $65.48 \pm 2.59$ & $48.92 \pm 2.90$ \\
        \ding{55} & \ding{51} & $69.83 \pm 3.39$ & $69.67 \pm 3.66$ & $69.32 \pm 3.20$ & $53.19 \pm 3.82$ \\
        \ding{51} & \ding{55} & $71.23 \pm 2.96$ & $72.58 \pm 2.26$ & $71.12 \pm 3.41$ & $55.79 \pm 4.02$ \\
        \ding{51} & \ding{51} & $\mathbf{76.71 \pm 2.44}$ & $\mathbf{76.44 \pm 3.39}$ & $\mathbf{76.13 \pm 3.76}$ & $\mathbf{62.184 \pm 4.44}$ \\
        \hline
    \end{tabular}
    \vspace{5pt} % Small gap before caption
    \caption{Performance Metrics for Different Model Configurations on ACL Dataset}
    \label{tab:ablation_study}
\end{table}







This table summarizes the impact of incorporating the spatio-temporal features and the SPI on the model's performance across various metrics, namely Accuracy, Precision, Recall, and the Jaccard Index. A cross (\ding{55}) indicates the absence of a component, while a check mark (\ding{51}) denotes its presence. The first row of the table shows the baseline model, which lacks both spatio-temporal features and SPI. This configuration yields the lowest performance, with an accuracy of $66.38 \pm 1.71$, precision of $66.818 \pm 2.26$, recall of $65.48 \pm 2.59$, and a Jaccard Index of $48.92 \pm 2.90$. When spatio-temporal features are added (second row), there is a notable improvement across all metrics. The accuracy increases to $71.23 \pm 2.96$, precision to $72.58 \pm 2.26$, recall to $71.12 \pm 3.41$, and the Jaccard Index to $55.79 \pm 4.02$. The inclusion of only the SPI (third row) also enhances the model's performance compared to the baseline. The accuracy is $69.83 \pm 3.39$, precision is $69.67 \pm 3.66$, recall is $69.32 \pm 3.20$, and the Jaccard Index is $53.19 \pm 3.82$. The final configuration, which incorporates both spatio-temporal features and the SPI (fourth row), achieves the best performance across all metrics. The accuracy is the highest at $76.71 \pm 2.44$, precision at $76.44 \pm 3.39$, recall at $76.13 \pm 3.76$, and the Jaccard Index at $62.184 \pm 4.44$.

\subsection{Benchmark on Cholec80 Dataset}
We also benchmarked our proposed method on the standard Cholec80 dataset to compare its performance with previous work. This dataset has been extensively utilized in previous research for surgical phase recognition, making it an ideal benchmark for evaluating our model. Table \ref{tab:benchmark_cholec80} presents a comparative analysis of our model with the other state-of-the-art methods. The table is divided into two parts: the top section lists results from methods using the first 40 videos for training and the remaining 40 for validation, while the bottom section lists results from methods using cross-validation. For each approach, we also report our model's performance using the corresponding data splits.


\begin{table}[h!]
    \centering
    \begin{tabular}{|l|r|r|r|}
        \hline
        \textbf{Method} & \textbf{Accuracy} & \textbf{Precision} & \textbf{Recall}  \\
        \hline   
        EndoNet\textsuperscript{+}~\cite{2.twinanda2016endonet} & $81.7 \pm 4.2$ & $73.7$ & $79.6$\\
        PhaseNet\textsuperscript{+}~\cite{2.twinanda2016endonet} & $78.8 \pm 4.7$ & $71.3$ & $76.6$ \\
        MTRCNet\textsuperscript{+}~\cite{18_jin2020multi} & $89.2 \pm 7.6$ & $86.9$ & $88.0$ \\
        LoViT\textsuperscript{+}~\cite{liu2023lovit} & $91.5 \pm 6.1$ & $83.1$ & $86.5$ \\
        SKiT\textsuperscript{+}~\cite{liu2023skit} & \textbf{92.5 $\pm$ 5.1} & $84.6$ & \textbf{88.5} \\
        \textbf{Ours}\textsuperscript{+} & \textbf{92.4 $\pm$ 5.0} & \textbf{85.6} & $80.6$ \\
        \hline
        TeCNO~\cite{22_czempiel2020tecno} & $88.5 \pm 0.27$ & $81.6$ & $85.2$ \\
        OperA~\cite{czempiel2021opera} & $91.2 \pm 0.6$ & $82.1$ & \textbf{86.92} \\
        \textbf{Ours} & \textbf{91.9 $\pm$ 0.9} & \textbf{83.1} & $86.5$ \\
        \hline
    \end{tabular}
    \vspace{5pt} % Adds a small gap before the caption
    \caption{Benchmark Performance Comparison on Cholec80 Dataset}
    \label{tab:benchmark_cholec80}
\end{table}



Our model demonstrates competitive performance with an accuracy of \(92.4 \pm 5.0\%\), which is very close to SKiT's highest accuracy of \(92.5 \pm 5.1\%\). Our model achieves a precision of 85.6, higher than both SKiT (84.6) and LoViT (83.1), and a recall of 80.6. In the bottom part, we compare methods that employed cross-validation. Our model achieves the highest accuracy (\(91.9 \pm 0.9\%\)), and precision (83.1) demonstrating its robustness and reliability across different evaluation schemes.

In addition to these metrics, we also evaluated the model's performance in predicting the Surgical Progress Index (SPI) on the Cholec80 dataset. The SPI, which quantifies the progression of the surgery on a scale from 0 to 1, demonstrated an output error of $9.86\%$.

\section{Discussion}
Automated workflow analysis and surgical phase recognition hold significant potential for improving orthopedic surgery by enhancing precision and workflow efficiency. While these technologies have been effectively applied in various surgical domains to aid real-time decision-making and documentation, their use in arthroscopy has been limited. Arthroscopic procedures, commonly employed in orthopedics, present unique challenges such as limited visual fields and frequent occlusions, which complicate phase recognition. By addressing these challenges, advanced phase recognition systems can improve surgical training, assist surgeons during procedures, and streamline clinical workflows. This study introduces the ACL27 dataset and a novel transformer-based model designed to meet the specific needs of arthroscopic video analysis, providing a valuable resource and a new benchmark for future research in this area.

Our approach addresses these challenges by enriching the feature extraction process with temporal information directly integrated into the spatial feature extraction stage. This methodology enhances the model's ability to discern subtle transitions and similar visual appearances across different phases, which are critical in arthroscopic environments.

The integration of the Surgical Progress Index (SPI) enhances the model's capability to not only recognize but also predict the progression of surgery, an aspect crucial for real-time surgical assistance systems. The SPI provides a continuous measure of surgery progression, incorporating global temporal context, which is essential for accurately mapping surgical dynamics and predicting the remaining duration of surgery. This capability is particularly beneficial for operating room (OR) planning and management. By providing an estimate of the surgery's remaining duration, the SPI can help optimize OR scheduling, reduce patient wait times, and improve overall surgical efficiency.

Empirical results from testing our model on the newly introduced ACL27 dataset and the established Cholec80 dataset demonstrate its robustness in handling complex surgical phase recognition tasks. Our proposed method and dataset for ACL phase recognition highlight the model's adaptability to different surgical environments and its resilience against variations inherent in live surgical scenarios. The model achieved an accuracy of $72.91 \pm 11.76\%$ on the ACL27 dataset. Since recent works on other types of surgeries like cholecystectomy cannot be compared on ACL due to unavailable code, we benchmarked our model on the Cholec80 dataset, achieving an accuracy of $92.4 \pm 5.0\%$, which is comparable to existing methods.

One of the key contributions of our work is the potential clinical impact. By enabling more accurate and timely recognition of surgical phases, our model can support real-time decision-making, potentially reducing surgery times and improving patient outcomes. The SPI, in particular, offers significant practical benefits for OR management. Accurate predictions of surgical progress and remaining duration can facilitate better resource allocation, improve patient flow, and reduce the likelihood of scheduling conflicts.

While our results are promising, several limitations warrant further discussion. Firstly, the size of the ACL27 dataset, although sufficient for initial validation, could be expanded to improve generalizability. Standardizing recording protocols is crucial, as variations in camera placement, lighting, and occlusions can affect phase recognition accuracy. Additionally, the issue of missing phases in some surgical videos can mislead predictions and impact the SPI calculation. Although we used average progression percentages for missing phases, more sophisticated methods are needed. Labeling efforts pose another significant limitation. Accurate annotation requires significant expertise and is labor-intensive, introducing potential human error and variability. Developing automated or semi-automated labeling tools could streamline this process and enhance label accuracy. The computational demands of our model are substantial, potentially limiting its practical applicability. Future work should focus on optimizing the model’s efficiency to make it more accessible for real-time surgical assistance.



\section{Discussion}
\paragraph{Overall Accuracy}

Our proposed model achieved a notable accuracy of $72.91 \pm 11.76\%$ on the ACL27 dataset, specifically designed for arthroscopic surgery, and $92.4 \pm 5.0\%$ on the widely-used Cholec80 dataset. These results underscore the effectiveness of our approach across different surgical domains. On the ACL27 dataset, our model significantly outperformed existing methods. For instance, when compared to other models adapted to this dataset, our model's accuracy of 72.91\% is considerably higher than the 66.75\% accuracy achieved by the Trans-SVNet model with an 80-second sequence length. Similarly, MTRCNet and TeCNO, two other prominent models in surgical phase recognition, reported lower accuracies of 60.68\% and 58.31\%, respectively. These comparisons, as detailed in Table 3 of our paper, highlight our model's superior capability to handle the specific challenges of arthroscopic video analysis, such as limited fields of view and frequent occlusions. Moreover, our model's performance on the Cholec80 dataset, with an accuracy of 92.4\%, is comparable to other leading methods in the field. For example, SKiT and LoViT achieved accuracies of 92.5\% and 91.5\%, respectively. This demonstrates that our approach not only excels in the specialized context of arthroscopy but also maintains strong performance in more general surgical phase recognition tasks.

\paragraph{Feature Extractor}

The feature extraction process plays a critical role in the overall performance of surgical phase recognition models. In earlier works, such as the OperA model \cite{czempiel2021opera}, only spatial features were extracted, which, while effective, limited the model's ability to capture temporal dependencies essential for surgical video analysis. The LoViT model \cite{liu2023lovit} made significant strides by incorporating a temporally-rich feature extractor with a temporal aggregator. However, during the final feature extraction step, LoViT reverted to extracting only spatial features, leaving temporal dependencies less integrated into the final feature representation. Our approach advances this by directly incorporating spatio-temporal features into the feature extraction process. The effectiveness of this approach is clearly demonstrated in our ablation study (Table 4). When comparing models trained without spatio-temporal features (first two rows) to those that include them (last two rows), there is a marked improvement in performance. Specifically, the introduction of spatio-temporal features led to significant gains in accuracy, precision, recall, and Jaccard Index, highlighting the critical importance of integrating temporal information during feature extraction. Moreover, the use of spatio-temporal features has proven beneficial across various domains, including action recognition in videos \cite{carreira2017quo}, human activity recognition \cite{feichtenhofer2019slowfast}, and anomaly detection \cite{gong2019memorizing}, reinforcing its necessity for achieving high accuracy in complex tasks like surgical phase recognition.

\paragraph{Technical Aspects of SPI}
The Surgical Progress Index (SPI) is a novel contribution of our work that significantly enhances model performance by providing a continuous measure of surgery progression. By leveraging global temporal context, the SPI helps suppress incorrectly classified noisy frames, thereby improving overall accuracy, as demonstrated in our ablation study. This is particularly crucial given the high variability in surgery times and phase durations, which the SPI effectively manages, maintaining temporal coherence in phase predictions. Unlike other approaches, such as MTRC-Net \cite{18_jin2020multi}, which included additional outputs like tool detection that require extensive labeling efforts, our SPI offers a nearly effortless yet powerful enhancement to the model, making it both efficient and practical for clinical application, particularly in the context of arthroscopy.

\paragraph{Clinical Impact}
Our work presents a significant clinical impact by introducing the first tailored approach for surgical phase recognition in arthroscopy, making the procedure more quantifiable and measurable in an automated manner. This model can be integrated into video documentation systems for automated labeling and detailed workflow analysis, enhancing surgical training and real-time decision-making. The Surgical Progress Index (SPI) further contributes by offering accurate estimates of surgery duration, optimizing OR scheduling, and improving surgical efficiency. Moreover, this approach supports the development of more context-aware intelligent computer-assisted and robotic systems, as explored in recent studies on surgical guidance in complex procedures \cite{kolbinger2023artificial}.

\paragraph{Limitations}
Despite the promising results, several limitations should be acknowledged. The relatively small size of the ACL27 dataset, while sufficient for initial validation, limits the generalizability of our findings. Expanding this dataset would likely improve the robustness of the model, particularly in diverse clinical settings. The specific challenges of ACL surgeries, such as limited visual fields, frequent occlusions, and the presence of fluid and debris, further complicate phase recognition and may impact the model's accuracy. Additionally, variations in recording protocols, including camera placement, lighting, and surgeon technique, can affect the consistency of phase recognition. The issue of missing phases in some surgical videos also presents a challenge, particularly for SPI calculation, where more sophisticated methods could be explored to enhance accuracy.

Furthermore, the computational demands of our model, while necessary for achieving high accuracy, may limit its practical applicability in real-time surgical environments. Future work should focus on optimizing the model’s efficiency and robustness to ensure its feasibility for integration into clinical workflows and its adaptability across different surgical contexts.


\section{Conclusion and Outlook}

This study introduces a novel approach to surgical phase recognition in arthroscopic procedures, focusing on ACL reconstruction. We present the ACL27 dataset, which fills a critical gap in publicly available data for arthroscopy research. Our transformer-based model, leveraging temporal-aware frame-wise feature extraction demonstrates significant improvements in phase recognition accuracy and temporal awareness. The introduction of the Surgical Progress Index (SPI) provides a continuous measure of surgical progress, enhancing the model's practical applicability in real-time surgical assistance and planning. Experimental results on the ACL27 and Cholec80 datasets validate the robustness and generalizability of our approach. In conclusion, this study sets a new benchmark in surgical phase recognition for arthroscopic procedures and provides a robust framework for developing advanced surgical assistance systems, ultimately contributing to improved surgical outcomes and patient care.

\section{Acknowledgments}
This research is funded by the Innosuisse Flagship project PROFICIENCY No. PFFS-21-19. Additionally, this work has received support from the OR-X, a Swiss national research infrastructure for translational surgery, with associated funding provided by the University of Zurich and the University Hospital Balgrist.





%Bibliography
\bibliographystyle{unsrt}  
%Version 3 October 2023
% See section 11 of the User Manual for version history
%
%%%%%%%%%%%%%%%%%%%%%%%%%%%%%%%%%%%%%%%%%%%%%%%%%%%%%%%%%%%%%%%%%%%%%%
%%                                                                 %%
%% Please do not use \input{...} to include other tex files.       %%
%% Submit your LaTeX manuscript as one .tex document.              %%
%%                                                                 %%
%% All additional figures and files should be attached             %%
%% separately and not embedded in the \TeX\ document itself.       %%
%%                                                                 %%
%%%%%%%%%%%%%%%%%%%%%%%%%%%%%%%%%%%%%%%%%%%%%%%%%%%%%%%%%%%%%%%%%%%%%

%%\documentclass[referee,sn-basic]{sn-jnl}% referee option is meant for double line spacing

%%=======================================================%%
%% to print line numbers in the margin use lineno option %%
%%=======================================================%%

%%\documentclass[lineno,sn-basic]{sn-jnl}% Basic Springer Nature Reference Style/Chemistry Reference Style

%%======================================================%%
%% to compile with pdflatex/xelatex use pdflatex option %%
%%======================================================%%

%%\documentclass[pdflatex,sn-basic]{sn-jnl}% Basic Springer Nature Reference Style/Chemistry Reference Style


%%Note: the following reference styles support Namedate and Numbered referencing. By default the style follows the most common style. To switch between the options you can add or remove “Numbered” in the optional parenthesis. 
%%The option is available for: sn-basic.bst, sn-vancouver.bst, sn-chicago.bst%  
 
%%\documentclass[sn-nature]{sn-jnl}% Style for submissions to Nature Portfolio journals
%\documentclass[sn-basic]{sn-jnl}% Basic Springer Nature Reference Style/Chemistry Reference Style
\documentclass[pdflatex]{sn-jnl}% Math and Physical Sciences Numbered Reference Style 
%%\documentclass[sn-mathphys-ay]{sn-jnl}% Math and Physical Sciences Author Year Reference Style
%%\documentclass[sn-aps]{sn-jnl}% American Physical Society (APS) Reference Style
%%\documentclass[sn-vancouver,Numbered]{sn-jnl}% Vancouver Reference Style
%%\documentclass[sn-apa]{sn-jnl}% APA Reference Style 
%%\documentclass[sn-chicago]{sn-jnl}% Chicago-based Humanities Reference Style

%%%% Standard Packages
%%<additional latex packages if required can be included here>
\usepackage{lmodern}
\usepackage{anyfontsize}

\usepackage{subcaption}
%\usepackage{subcaption}
\usepackage{natbib}
\usepackage{graphicx}%
\usepackage{multirow}%
\usepackage{amsmath,amssymb,amsfonts}%
\usepackage{amsthm}%
\usepackage{mathrsfs}%
\usepackage[title]{appendix}%
\usepackage{xcolor}%
\usepackage{textcomp}%
\usepackage{manyfoot}%
\usepackage{booktabs}%
\usepackage{algorithm}%
\usepackage{algorithmicx}%
\usepackage{algpseudocode}%
\usepackage{listings}%
%%%%

%%%%%=============================================================================%%%%
%%%%  Remarks: This template is provided to aid authors with the preparation
%%%%  of original research articles intended for submission to journals published 
%%%%  by Springer Nature. The guidance has been prepared in partnership with 
%%%%  production teams to conform to Springer Nature technical requirements. 
%%%%  Editorial and presentation requirements differ among journal portfolios and 
%%%%  research disciplines. You may find sections in this template are irrelevant 
%%%%  to your work and are empowered to omit any such section if allowed by the 
%%%%  journal you intend to submit to. The submission guidelines and policies 
%%%%  of the journal take precedence. A detailed User Manual is available in the 
%%%%  template package for technical guidance.
%%%%%=============================================================================%%%%

%% as per the requirement new theorem styles can be included as shown below
%\theoremstyle{thmstyleone}%
%\newtheorem{theorem}{Theorem}%  meant for continuous numbers
%%\newtheorem{theorem}{Theorem}[section]% meant for sectionwise numbers
%% optional argument [theorem] produces theorem numbering sequence instead of independent numbers for Proposition
%\newtheorem{proposition}[theorem]{Proposition}% 
%%\newtheorem{proposition}{Proposition}% to get separate numbers for theorem and proposition etc.

%\theoremstyle{thmstyletwo}%
\newtheorem{example}{Example}%
\newtheorem{remark}{Remark}%

%\theoremstyle{thmstylethree}%
\newtheorem{definition}{Definition}%
\bibliographystyle{sn-basic}
\raggedbottom
%%\unnumbered% uncomment this for unnumbered level heads

\begin{document}

\title[Article Title]{Improving Similar Case Retrieval Ranking Performance By Revisiting RankSVM}

%%=============================================================%%
%% GivenName	-> \fnm{Joergen W.}
%% Particle	-> \spfx{van der} -> surname prefix
%% FamilyName	-> \sur{Ploeg}
%% Suffix	-> \sfx{IV}
%% \author*[1,2]{\fnm{Joergen W.} \spfx{van der} \sur{Ploeg} 
%%  \sfx{IV}}\email{iauthor@gmail.com}
%%=============================================================%%

\author[1]{\fnm{Yuqi} \sur{Liu}}\email{211224027@cupl.edu.cn}

\author*[1]{\fnm{Yan} \sur{Zheng}}\email{zhengyan@cupl.edu.cn}
%\equalcont{These authors contributed equally to this work.}

%\author[1,2]{\fnm{Third} \sur{Author}}\email{iiiauthor@gmail.com}
%\equalcont{These authors contributed equally to this work.}

\affil*[1]{\orgdiv{School of Information Management for Law}, \orgname{China University of Political Science and Law}, \orgaddress{\street{27 Fuxue Road}, \city{Beijing}, \postcode{102249}, \country{China}}}

%\affil[2]{\orgdiv{Department}, \orgname{Organization}, \orgaddress{\street{Street}, \city{City}, \postcode{10587}, \state{State}, \country{Country}}}

%\affil[3]{\orgdiv{Department}, \orgname{Organization}, \orgaddress{\street{Street}, \city{City}, \postcode{610101}, \state{State}, \country{Country}}}

%%==================================%%
%% Sample for unstructured abstract %%
%%==================================%%

\abstract{Given the rapid development of Legal AI, a lot of attention has been paid to one of the most important legal AI tasks--similar case retrieval, especially with language models to use. 
In our paper, however, we try to improve the ranking performance of current models from the perspective of learning to rank instead of language models. Specifically, we conduct experiments using a pairwise method--RankSVM as the classifier to substitute a fully connected layer, combined with commonly used language models on similar case retrieval datasets LeCaRDv1 and LeCaRDv2. We finally come to the conclusion that RankSVM could generally help improve the retrieval performance on the LeCaRDv1 and LeCaRDv2 datasets compared with original classifiers by optimizing the precise ranking. It could also help mitigate overfitting owing to class imbalance. Our code is available in \url{https://github.com/liuyuqi123study/RankSVM_for_SLR}}

%%================================%%
%% Sample for structured abstract %%
%%================================%%

% \abstract{\textbf{Purpose:} The abstract serves both as a general introduction to the topic and as a brief, non-technical summary of the main results and their implications. The abstract must not include subheadings (unless expressly permitted in the journal's Instructions to Authors), equations or citations. As a guide the abstract should not exceed 200 words. Most journals do not set a hard limit however authors are advised to check the author instructions for the journal they are submitting to.
% 
% \textbf{Methods:} The abstract serves both as a general introduction to the topic and as a brief, non-technical summary of the main results and their implications. The abstract must not include subheadings (unless expressly permitted in the journal's Instructions to Authors), equations or citations. As a guide the abstract should not exceed 200 words. Most journals do not set a hard limit however authors are advised to check the author instructions for the journal they are submitting to.
% 
% \textbf{Results:} The abstract serves both as a general introduction to the topic and as a brief, non-technical summary of the main results and their implications. The abstract must not include subheadings (unless expressly permitted in the journal's Instructions to Authors), equations or citations. As a guide the abstract should not exceed 200 words. Most journals do not set a hard limit however authors are advised to check the author instructions for the journal they are submitting to.
% 
% \textbf{Conclusion:} The abstract serves both as a general introduction to the topic and as a brief, non-technical summary of the main results and their implications. The abstract must not include subheadings (unless expressly permitted in the journal's Instructions to Authors), equations or citations. As a guide the abstract should not exceed 200 words. Most journals do not set a hard limit however authors are advised to check the author instructions for the journal they are submitting to.}

\keywords{Information Retrieval, Optimization, Natural Language Processing, Legal Intelligence}

%%\pacs[JEL Classification]{D8, H51}

%%\pacs[MSC Classification]{35A01, 65L10, 65L12, 65L20, 65L70}

\maketitle
                                
\section{Introduction}\label{sec1}

In recent years, Legal Artificial Intelligence (AI) has attracted attention from both AI researchers and legal professionals(\cite{greenleaf2018building}). Legal AI mainly means applying artificial intelligence technology to help with legal tasks. Benefiting from the rapid development of AI, especially natural language processing (NLP) techniques, Legal AI has had a lot of achievements in real law applications (\cite{zhong-etal-2020-nlp,Shao2020BERTPLIMP,surden2019artificial}). 

Among legal AI tasks, similar case retrieval (SCR) is a representative legal AI application, as the appeal to similar sentences for similar cases plays a pivotal role in promoting judicial fairness(\cite{Shao2020BERTPLIMP}). 

As demonstrated in a lot of work, there are mainly two approaches to do information retrieval. 1) Traditional IR models(e.g.BM25 which is a probabilistic retrieval model(\cite{BM25}) using keywords. 2) More advanced techniques for IR using pre-trained models with deep learning skills, and the latter has achieved promising results in some commonly used benchmark datasets(\cite{ma2021lecard}).When using pre-trained language models, there have been a lot of variations trying to incorporate the structural information in legal documents to help with the retrieval(\cite{hu2022bert_lf,Shao2020BERTPLIMP,DBLP:journals/corr/ChungGCB14,10.1145/3609796,10.1016/j.ipm.2024.103729,feng-etal-2022-legal,10.1145/3569929}) or to utilize LLMs to boost similar case retrieval~\cite{zhou2023boostinglegalcaseretrieval}.


However, compared with attention paid to language models, there is less focus on the classifier used for the final ranking. As it is pointed out in an early paper~\cite{10.1145/1148170.1148205}, there are two important factors in document retrieval and one of them is that to have high accuracy on top-ranked documents is crucial for an IR system. So we try to look into better classifiers in the hope of improving ranking performance. As a problem of ranking by criterion of relevance, however, SCR is reduced to a 2-class classification problem following a pointwise path in a lot of work, which means the related classifier is only used to produce one label for a query-candidate pair. It is also mentioned by other papers that ~\cite{10.1007/978-3-031-10986-7_43}the fine-tuned BERT model does not compare which case candidate is more similar to the query case.

In our paper, we reintroduce pairwise methods to do the final ranking. Under pairwise settings, classifiers care about the relative order between two documents, which is closer to the actual ranking task(\cite{10.1561/1500000016}). Among pairwise approaches, we pay extra attention to the Ranking Support Vector Machine(RankSVM) method as a representative method. The advantage of RankSVM is that it aims to minimize the ranking loss and can also mitigate the negative influence of the class-imbalance issue(\cite{10.5555/2981345.2981385,DBLP:journals/corr/WuZ16}) which is common for SCR tasks. It is also mentioned in some papers (\cite{WU202024}) that the setting of binary labels compared with multi labels could help mitigate the accumulated errors from thresholds learning in RankSVM. 


So we are inspired to use RankSVM to explore its performance on similar case retrieval tasks and measure the performance using NDCG as our main metric (Normalized Discounted Cumulative Gain).

In our experiments, we test different kinds of retrieval models when combined with RankSVM to examine their retrieval performance on different similar case retrieval datasets LeCaRDv1(\cite{ma2021lecard}) and LeCardv2(\cite{li2023lecardv2largescalechineselegal}). Our method is illustrated in Figure \ref{fig:GA}.
\begin{figure}[ht]
    \centering
    \includegraphics[width=1\linewidth]{Graph_abstract_svm_SLR.png}
    \caption{Illustration of Our Method}
    \label{fig:GA}
\end{figure}

\section{Materials and Methods}\label{sec2}
\subsection{Language Models for SCR}
In this paper, we mainly examine three language models: BERT model, BERT+LEVEN, and Lawformer language model, which respectively represent basic language model, knowledge incorporated model, and language model with long inputs. BERT(\cite{devlin2019bertpretrainingdeepbidirectional}) is a commonly used language model with a maximum window length of 512, and many researchers have done many implementations that have a lot to do with how they understand structural information in legal documents and concatenate them with models that could deal with limited length (\cite{Shao2020BERTPLIMP,hu2022bert_lf,10.1007/978-3-031-10986-7_43}). So in our experiments, we both examine the basic BERT model and the BERT model with structural information extracted from legal documents. We use the code from LEVEN(\cite{yao-etal-2022-leven}) as an example where it incorporates event detection. 

Lawformer is a representative model that people propose to deal with longer inputs (\cite{xiao2021lawformer})which is based on longformer architecture (\cite{DBLP:journals/corr/abs-2004-05150}). Lawformer could process long texts up to respectively 509 for query documents and 3072 for candidate documents. 

Some work(\cite{xu2024rankmambabenchmarkingmambasdocument}) moves on to examine another new pre-trained model Mamba(\cite{gu2024mambalineartimesequencemodeling}) that could deal with long inputs as well, where MAMBA achieves competitive performance compared with transformer-based models using the same training recipe in the document ranking task. So we also measure its performance on the LeCaRDv1 dataset.

\subsection{Datasets and Data Preprocessing}
In our experiments, we use two benchmark datasets that are commonly used for similar case retrieval tasks--LeCaRDv1 and LeCaRDv2. Here we mainly examine the difference between these two datasets. LeCaRDv1 dataset(A Chinese Legal Case Retrieval Dataset)(\cite{ma2021lecard}) and LeCaRDv2 dataset(\cite{li2023lecardv2largescalechineselegal}) are two datasets specifically and widely used as benchmarks for similar case retrieval tasks since their release. We first check the difference in the number of query files and candidate files, which is shown in Table \ref{tab:details}. While LeCaRDv1 contains 107 queries and 10,700 candidates, LeCaRDv2 contains more queries and candidates. At the same time, it is worth attention that while LeCaRDv1 has a separate candidate pool(folder) of size 100 for each query, there is no subfolder for LeCaRDv2 queries. Moreover, there lies some divergence in their judging criteria. While LeCaRDv1 focuses on the fact part of a case involving key circumstances and key elements, LeCaRDv2 argues that characterization, penalty, and procedure should all be taken into consideration. Even though they say in their paper that there is no explicit mapping function between the Overall Relevance and the sub-relevance, the calculation could be described as follows, see \ref{relev}.
\begin{table}[ht]
\caption{Details about LeCaRDv1 and LeCaRDv2 datasets}
\centering
\begin{tabular}{c c c} 
\toprule
datasets&\textbf{LeCaRDv1}	& \textbf{LeCaRDv2}\\
\midrule
\#candidate cases/query		& 100&55192\\
\midrule
\#average relevant cases per query&10.33&20.89\\	
\midrule
ratio of relevance candidates&0.1033&0.0004\\
\bottomrule
\end{tabular}
\label{tab:details}
\end{table}

\begin{equation}
    relevance_{v2}=relevance_{v1}+relevance_{penalty}+relevance_{procedure.}
\label{relev}
\end{equation}

As it is known, a legal document could usually be partitioned into 3 parts--Parties' Information, Facts, Holding, and Decision, which is illustrated in Figure \ref{fig:illus}. Here we follow the common practice to only take fact parts of those documents as inputs, see equation \ref{input} and \ref{input_v2}. 
\begin{equation}
    input_{v1}=(query['q'],cand['ajjbqk'])
    \label{input}
\end{equation}
\begin{equation}
    input_{v2}=(query['fact'],cand['fact'])
    \label{input_v2}
\end{equation}
\begin{figure}
    \centering
    \includegraphics[width=1\linewidth]{case_illustration.pdf}
    \caption{Illustration of Structure of A Case}
    \label{fig:illus}
\end{figure}

Moreover, we examine the fact parts in both datasets to be used as our inputs, see Figure \ref{fig:lengths of datasets} and Figure \ref{fig:lengths of datasets_v2}, from which we could tell that many of them are actually longer than the input limit of the BERT model as well as the Lawformer or MAMBA model. 
At the same time, as we use the same number of bins in the histogram for both LeCaRDv1 and LeCaRDv2 datasets,  it could be seen that while the LeCaRDv2 dataset has query files of longer lengths, it has candidate files of shorter lengths. As it is not



\begin{figure}[h]
     \centering
     \begin{subfigure}[b]{0.4\textwidth}
         \centering
         \includegraphics[width=\textwidth]{histogram_v1_candidates.pdf}
         \caption{Lengths of Candidates in LeCaRDv1 Dataset}
         \label{fig:Lengths of Candidates in LeCaRDv1}
     \end{subfigure}
     \hfill
     \begin{subfigure}[b]{0.4\textwidth}
         \centering
         \includegraphics[width=\textwidth]{histogram_v1_queries.pdf}
         \caption{Lengths of Queries in LeCaRDv1 Dataset}
         \label{fig:Lengths of Queries in LeCaRDv1 Dataset}
     \end{subfigure}
        \caption{Length of Documents in LeCaRDv1 Dataset}
        \label{fig:lengths of datasets}
\end{figure}

\begin{figure}
     \centering
     \begin{subfigure}[b]{0.4\textwidth}
         \centering
         \includegraphics[width=\textwidth]{v2_candidate.pdf}
         \caption{Lengths of Candidates in LeCaRDv2 Dataset}
         \label{fig:Lengths of Candidates in LeCaRDv2}
     \end{subfigure}
     \hfill
     \begin{subfigure}[b]{0.4\textwidth}
         \centering
         \includegraphics[width=\textwidth]{v2_query.pdf}
         \caption{Lengths of Queries in LeCaRDv2 Dataset}
         \label{fig:Lengths of Queries in LeCaRDv2 Dataset}
     \end{subfigure}
        \caption{Length of Documents in LeCaRDv2 Dataset}
        \label{fig:lengths of datasets_v2}
\end{figure}

%Compared with its precedent LeCaRDv1, LeCaRDv2 ~\cite{li2023lecardv2largescalechineselegal} has 800 queries and 55192 candidates. It is said that it covers 50 charges which are more than 20 covered by LeCaRDv1.
%Besides, LeCaRDv2 dataset changes its criterion of relevance from just basic facts to  It also argues that the pooling strategy employed in LeCaRDv1 is too naive, aimed at a different setting that emphasizes directly retrieving relevant cases from a large legal corpus so there's no subfolder in the dataset.

With different benchmark datasets, we conduct different data preprocessing. We also conduct visualization to check LeCaRDv2's length of documents. See Figure \ref{fig:lengths of datasets_v2} to check their distribution. It could be told that case documents in LeCaRDv2 dataset are shorter .However, as there is no subfolder for LeCaRDv2 dataset in its original repository \url{https://github.com/THUIR/LeCaRDv2}, out of consideration for memory and the potential risk of probable overfitting if we conduct experiments with all negative samples among the corpus, we conduct our experiments trying to follow the practice on LeCaRDv1 dataset by building a subfolder of 130 candidates file for each query, among which 30 are those that are labeled relevant which is also similar with the practice taken in LeCaRDv2 baseline where they have negative samples with the ratio of positives and negatives at 1:32. It is shown by our result that it is a wise choice. We also follow different train-test splits. For LeCaRDv1, we follow the 5-fold validation while for LeCaRDv2 we took 640 instances as our training dataset and 160 as our test dataset.

At the same time, the data preprocessing does differ from model to model. For Lawformer, we basically follow the practice in its paper to have candidate files of length 3072 and query files of length 509. For LEVEN code, what we do is we detect their queries and candidate files to get their trigger words and relevant event types. According to the original paper, there are As a result, for non-trigger tokens, it just feeds the sum of token embeddings and position embeddings into the BERT model. For trigger words, we define an event type embedding for each event type and add the corresponding embedding to each input.
\begin{equation}
    inputx_{without_event}=func_{tokentoids}('[CLS]'+query+'[SEP]'+cand+'[SEP]')
\end{equation}
\begin{equation}
    inputx_{use_event}=id_{CLS}+id_{query}+id_{sep}+id_{candidate}+id_{sep}
\end{equation}
\begin{equation}
    id_{events}=[0]+id_{query_event}+[0]+id_{cand_event}+[0]
\end{equation}

\subsection{Fine-tuning with Pretrained Model And Feature Extraction}

When we do fine-tuning, we basically follow the practice taken in LEVEN code \url{https://github.com/thunlp/LEVEN}. For BERT-related models, we add a two-class classification layer after the pooled output of BERT denoted as [CLS] and do fine-tuning with the training set. 

For Lawformer, the concrete steps are basically the same as BERT. In terms of  hyperparameters, we basically follow what is used in LEVEN code since  according to our experiments those are better than the original Lawformer code when applied to a smaller batch size. The hyperparameters used are shown in Table \ref{table:Parameters for BERT model}.

\begin{equation}
output=func_{twoclassclassification}([CLS])
\end{equation}
%As we said before, when doing fine-tuning, in the BERT model,BERT based LEVEN model, and lawformer model, we have a fully connected layer after the CLS token to act as a classifier. 

To extract features, we just use the output of CLS of every query-candidate pair as our features. According to the dimension of CLS, we have 648 features for each query-candidate pair. For further experiments, we only pick the models with the best performance on the validation dataset or test dataset for feature extraction. In the LeCaRDv1 dataset,when training on each fold,we iterate the training dataset for 5 epochs and choose the epoch with the best performance compared with other epochs on the respective validation dataset. After training and validating on each fold, we report the average value for the 5 folds. In the LeCaRDv2 dataset, we iterate the training dataset for 5 epochs and choose the epoch with the best performance on the test dataset to extract features. Our model is only trained from the training data.

The result of RankSVM is used to rank the similarity of candidate-query pairs under the same candidate files.
\begin{equation}
 features=[CLS]
\end{equation}
\subsection{RankSVM}
Since RankSVM is a relatively classic method developed from ordinal-regression SVM to do information retrieval(\cite{Joachims1998MakingLS,jakkula2006tutorial}), we try to apply RankSVM(\cite{Joachims1998MakingLS,zheng2019ranksvm}) for similar case retrieval task in this work. The mathematical formulation for RankSVM is shown below. Given n training queries ${q_i}^n_{i=1}$, their associated document pairs $(x_u^{(i)},x_v^{(i)})$ and the corresponding ground truth label $y_{u,v}^{(i)}$, where a linear scoring function is used without complicated kernel, i.e., $f(x)=w^Tx$.
%The intuitive idea of RankSVM is that it changes the classic classification setting to solve the ordinal regression problem based on the relationship of different labels.

The basic form of RankSVM is shown as follows.
\begin{equation}min\frac{1}{2}\||w\||^2+C\sum^n_{i=1}\sum_{u,v:y^{(i)}_{u,v}=1}\xi_{u,v}^{(i)}\end{equation}
\begin{equation}s.t. w^T(x_u^{(i)}-x_v^{(i)})\geq1-\xi_{u,v}^{(i)}, if y_{u,v}^{(i)}=1\end{equation},
\begin{equation}\xi_{u,v}^{(i)}\geq 0,i=1,\dots,n.
\end{equation}

In our experiment, we mainly use the method developed in the paper by Joachims(\cite{Joachims1998MakingLS}), where the algorithm just has 1 slack variable and it is solved using its dual form. Different algorithms for RankSVM are suitable for different applications. According to our experiment, the result with the algorithm we chose is better than other settings.(e.g., see this work(\cite{10.1145/1102351.1102399}) Also, we use different values of C, and they are 0.001, 0.05, 0.01, 0.02, 0.05, 0.1, 0.5, 1. We report the best result for each model.
%The main advantages of the method are 1) An efficient and effective method for selecting the working set, which is an active part of the optimization problem. 2) Successive "shrinking" of the optimization problem. This exploits the property that many SVM learning problems have i) much fewer support vectors(SV) than training examples. ii) many SVs which have an $\alpha_i$ at the upper bound C.3) computational improvements like caching and incremental updates of the gradient and the termination criteria. 

%In their method, they reformulated the problem as follows:

%\begin{equation}min:W(\alpha)=-\sum^l_{i=1}\alpha_i+\frac{1}{2}\sum^l_{i=1}\sum^l_{j=1}y_iy_j\alpha_i\alpha_jk(x_i,x_j)\end{equation}
%\begin{equation}subject\enspace to:\quad\sum^l_{i=1}y_i\alpha_i=0\end{equation}
%\begin{equation}\forall i:0\leq\alpha_i\leq C\end{equation}.

%To select a good working set, it is needed to find a steepest feasible direction $d$ of descent which has only $q$ non-zero elements; those non-zero elements will compose the current working set, satisfying
%The algorithm itself consists of a general decomposition algorithm,selecting a good working set, and shrinking.

It is worth mentioning that here we don't need a threshold and we just get results in scores and do ranking. Under the setting of predicting with a threshold, as it is said, there will be an implicit presumption that when in training and test, its input has the same data distribution and accumulated error(\cite{WU202024}) and may impede model performance.



%For LeCaRDv1, we  adopt top-k Normalized Discounted Cumulative Gain(NDCG@K) as our main evaluation metric. 

According to the original LeCaRDv1 and LeCaRDv2 paper, there are 4 classes of relevance. Here we just process it as a two-class classification. When processing labels, as long as two cases are relevant, we view the label of the pair as positive. 

\begin{table}[ht]
\caption{Parameters of Different Models}
\centering
\begin{tabular}{ c c c c}
\toprule
Model&BERT-Based&Lawformer&MAMBA\\
\midrule
  Algorithm& 1-slack algorithm (dual)& 1-slack algorithm (dual)& 1-slack algorithm (dual) \\
\midrule
Norm&l1-norm&l1-norm&l1-norm\\
\midrule
Learning Rate&1e-5&2.5e-6(-1)/5e-6(v2)&1e-5\\
\midrule
Optimizer&Adamw&Adamw&Adamw\\
\midrule
Training batch size&16&2(v1)/4(v2)&16\\
\midrule
Evaluating batch size&32&32&32\\
\midrule
Step size&1&1&1\\
\bottomrule
\end{tabular}
\label{table:Parameters for BERT model}
\end{table}

%\begin{table}[h]
%\caption{Parameters of Lawformer based model}
%\centering
%\begin{tabular}{ c c}
%\toprule
%Parameters\\
%\midrule
 % Algorithm& 1-slack algorithm (dual) \\
%\midrule
%Norm&l1-norm\\
%\midrule
%Learning Rate&2.5e-6(v1)/5e-6(v2)\\
%\midrule
%Optimizer&Adamw\\
%\midrule
%Training batch size&2(v1)/4(v2)\\
%\midrule
%Evaluating batch size&32\\
%\midrule
%Step size&1\\
%\bottomrule
%\end{tabular}
%\label{table:Parameters for Lawformer model}
%\end{table}
\section{Result}
\subsection{Results on LeCaRDv1}

To test the effectiveness of our method, we conduct our experiments mainly using BERT, BERT+Event, BERT+Event+RankSVM, BERT+RankSVM, Lawformer, Lawformer+RankSVM. The result is shown in Table \ref{table:LeCaRDv1-2}.

From the result on BERT, we could see that RankSVM helps improve performance especially on NDCG, which proves that our model is better at putting highly relevant cases earlier in a similar case list. To better understand the advantage of RankSVM, we finish visualization based on the training and testing dataset on fold 0 to see the score computed by original BERT and BERT+RankSVM. Specifically, we use features extracted by BERT as our input and compress them into two dimensions using t-SNE(\cite{JMLR:v9:vandermaaten08a}). After that, we color them according to similarity scores computed by relative models or labels. See Figure \ref{fig:v1_train} and \ref{fig:v1_test} for reference.
\begin{table}[ht]
\caption{Results on LeCaRDv1 Dataset}

\begin{tabular}{ccccccc}
\toprule
  Model&NDCG@10&NDCG@20 & NDCG@30\\
  \midrule
 BERT  &0.7896  &0.8389 & 0.9113\\
 BERT+RankSVM&\textbf{0.7963} &\textbf{0.8504} & \textbf{0.9166}\\
\midrule
BERT+LEVEN& 0.7881 &0.8435 & 0.9136 \\
BERT+LEVEN+RankSVM&\textbf{0.7931} &\textbf{0.8452} &\textbf{0.9164}\\
\midrule
MAMBA&0.7469&0.8013&0.892\\
\midrule
Lawformer&0.7592&0.8169&0.8993\\
Lawformer+RankSVM&\textbf{0.7738}&\textbf{0.8258}&\textbf{0.9073}\\
\bottomrule
\end{tabular}

\label{table:LeCaRDv1-2}

\end{table}
\begin{figure}[ht]
     \centering
     \begin{subfigure}[b]{0.3\textwidth}
         \centering
         \includegraphics[width=\textwidth]{label_train.pdf}
         \caption{Visualization of Relevance Labels}
         \label{fig:labels_v1}
     \end{subfigure}
     \hfill
     \begin{subfigure}[b]{0.3\textwidth}
         \centering
         \includegraphics[width=\textwidth]{BERT_train.pdf}
         \caption{Visualization of BERT Features}
         \label{fig:BERT_v1}
     \end{subfigure}
     \hfill
     \begin{subfigure}[b]{0.3\textwidth}
         \centering
         \includegraphics[width=\textwidth]{RankSVM_BERT.pdf}
         \caption{Visualization of Rank-SVM Scores}
         \label{fig:RankSVM_v1}
     \end{subfigure}
        \caption{Visualization of Labels/Features/Scores on LeCaRDv1 Training Dataset}
        \label{fig:v1_train}
\end{figure}

From the visualization result, we could see that RankSVM performs better by differentiating the relevance level of each pair continuously, while BERT treats the relevance level more discretely.

\begin{figure}
     \centering
     \begin{subfigure}[b]{0.3\textwidth}
         \centering
         \includegraphics[width=\textwidth]{label_test.pdf}
         \caption{Visualization of Relevance Labels}
         \label{fig:labels_v1_test}
     \end{subfigure}
     \hfill
     \begin{subfigure}[b]{0.3\textwidth}
         \centering
         \includegraphics[width=\textwidth]{BERT_test.pdf}
         \caption{Visualization of BERT Features}
         \label{fig:BERT_test_v1}
     \end{subfigure}
     \hfill
     \begin{subfigure}[b]{0.3\textwidth}
         \centering
         \includegraphics[width=\textwidth]{RankSVM_test.pdf}
         \caption{Visualization of RankSVM}
         \label{fig:RankSVM_v1_test}
     \end{subfigure}
        \caption{Visualization of Labels/Features/Scores on LeCaRDv1 Test Dataset}
        \label{fig:v1_test}
\end{figure}

What's more, the mechanism of RankSVM is actually maximizing ROC(Receiver Operating Characteristic curve)(\cite{Ataman2005OptimizingAU}). So we use the model of BERT comparing the results with and without RankSVM to show how ROC changes. For the BERT model, we use the first fold and the C is 0.01, and the visualization result is shown in Figure \ref{fig:ROC_curve_v1}. It could be seen that RankSVM helps improve the result of BERT on the AUC score, and it does an even better job in the test dataset. It proves our guess before that it could help reduce overfitting.
\begin{figure}
     \centering
     \begin{subfigure}[b]{0.47\textwidth}
         \centering
         \includegraphics[width=\textwidth]{ROC_training.pdf}
         \caption{ROC Curve for LeCaRDv1 Training Dataset}
         \label{fig:ROC_v1_train}
     \end{subfigure}
     \hfill
     \begin{subfigure}[b]{0.47\textwidth}
         \centering
         \includegraphics[width=\textwidth]{test_set_v1_roc.pdf}
         \caption{ROC Curve for LeCaRDv1 Test\\ Dataset}
         \label{fig:ROC_v1_test}
     \end{subfigure}
        \caption{Three simple graphs}
        \label{fig:ROC_curve_v1}
\end{figure}
Additionally, to explore SVM's performance on longer text for retrieval, we also benchmark its performance using Lawformer. It could also be seen that all NDCG-related metrics get improved. 

We also conduct experiments using MAMBA. As the performance is not competent enough, we don't do further combinations with RankSVM.


\subsection{Results on LeCaRDv2}
The result is shown in Table \ref{table:LeCaRDv2 Result}.According to the table, we could find that the NDCG related metrics are generally improved similar to LeCaRDv1, except for the BERT model. To understand its performance, here we also do visualization using features extracted from the BERT model following the same steps as the LeCaRDv1 dataset, see Figure \ref{fig:Visualization_on_v2_train} and \ref{fig:Visualization_on_v2_test}.
\begin{table}[ht]
\caption{Results on LeCaRDv2 Dataset}
\centering

\begin{tabular}{c c c c c} 
\toprule
 Model&NDCG@10&NDCG@20&NDCG@30\\
\midrule
 BERT&\textbf{0.8351}&\textbf{0.8764}&\textbf{0.9348}\\
\midrule
BERT+LEVEN&0.8117&0.8455&0.9184 \\
\midrule
BERT+RankSVM&0.793&0.8582&0.9247\\
\midrule
BERT+LEVEN+RankSVM&\textbf{0.8078}&0.8548&0.9227\\
\midrule
Lawformer&0.7968&0.8461&0.919\\
\midrule
Lawformer+RankSVM&\textbf{0.812}&0.8574&0.9247\\
\bottomrule
\end{tabular}
\label{table:LeCaRDv2 Result}
\end{table}
\begin{figure}
     \centering
     \begin{subfigure}[b]{0.45\textwidth}
         \centering
         \includegraphics[width=\textwidth]{train_v2.pdf}
         \caption{Visualization of Relevance Labels}
         \label{fig:v2_label}
     \end{subfigure}
     \hfill
     \begin{subfigure}[b]{0.45\textwidth}
         \centering
         \includegraphics[width=\textwidth]{train_v2_svm.pdf}
         \caption{Visualization of RankSVM Scores}
         \label{fig:v2_SVM}
     \end{subfigure}
        \caption{Visualization of Labels/Scores on LeCaRDv2 Training Dataset}
        \label{fig:Visualization_on_v2_train}
\end{figure}

\begin{figure}
     \centering
     \begin{subfigure}[b]{0.45\textwidth}
         \centering
         \includegraphics[width=\textwidth]{test_label_v2.pdf}
         \caption{Visualization of Relevance Labels}
         \label{fig:v2_label_test}
     \end{subfigure}
     \hfill
     \begin{subfigure}[b]{0.45\textwidth}
         \centering
         \includegraphics[width=\textwidth]{test_svm_v2.pdf}
         \caption{Visualization of RankSVM Scores}
         \label{fig:v2_SVM_test}
     \end{subfigure}
        \caption{Visualization of Labels/Score on LeCaRDv2 Test Dataset}
        \label{fig:Visualization_on_v2_test}
\end{figure}

It is shown by the figure with real labels that BERT features fail to distinguish these two classes explicitly this time as there are a lot of overlaps between the two classes compared with what is extracted from the LeCaRDv1 dataset, while RankSVM tries to learn from those features that are not distinguished between these two classes continuously. The failure of BERT may be owing to the changed standards of relevance in LeCaRDv2 and the shorter document length of the LeCaRDv2 dataset. It explains why RankSVM added to BERT fails to help improve NDCG metrics here.
\section{Discussion}
\subsection{Impact of Length of Text}
Even after we changed the length of the input data from 512 to 800 using the MAMBA model and Lawformer model, we could see that the interesting thing here is that there is no apparent increase in metrics compared with BERT, which is also mentioned in other works(\cite{deng2024learning}).

But we could also argue that there lies some space for hyperparameter tuning in the future.
\subsection{Error Analysis}
%Even RankSVM has demonstrated the ability to increase the performance on LeCaRDv1 and LeCaRDv2 with the NDCG metric, it fails to increase the performance shown by recall metrics which is also interestingly not taken in other papers as a metric~\cite{deng2024learning}. The concrete result of the recall metric is shown in the appendix. We could also argue that maybe it is because of the similarity standard changes in LeCaRDv2.

To explore further why RankSVM fails to improve the performance on BERT-based models regarding the LeCaRDv2 dataset, we conduct experiments using multi-class BERT classification to extract features out of our speculation that RankSVM on LeCaRDv2 needs language models that provide more information and end up getting all NDCG-related metrics improved again, see table \ref{table:Result when using multi-class classification}. So our speculation holds as RankSVM still works for BERT on the LeCaRDv2 dataset under this setting. Recently, there have been more research studies on improving RankSVM, and we leave that to explore in the future.
\begin{table}[ht]
\caption{Result on LeCaRDv2 when we do multi-class classification}
\centering
\begin{tabular}{c c c c}
\toprule
Model&NDCG@10&NDCG@20&NDCG@30\\
\midrule
  BERT Multiclass&0.791&0.8443&0.9181 \\
\midrule
BERT Multiclass+RankSVM&0.8053&0.8352&0.9222\\
\bottomrule
\end{tabular}
\label{table:Result when using multi-class classification}
\end{table}


\section{Conclusion}\label{sec13}
In our paper, we examine the performance of RankSVM when combined with other language models using binary labels to serve a similar case retrieval task on the LeCardv1 dataset and the LeCardv2 dataset. We come to the conclusion that RankSVM with binary labels could help improve the performance of models by improving their concrete ranks, and we also try to give some explanation from the feature perspective. Also, the RankSVM could help fight overfitting, which results from an imbalance class common in this task. Our findings point to the potential of improving SCR task performance from an optimization perspective. There still lies some work that needs to be done to explore the appropriate model to improve performance on the recall metric on the LeCaRDv2 dataset and how the performance of RankSVM will change when we extract features with larger batch sizes. We also leave the investigation regarding RankSVM combined with other models and datasets for future work.

\backmatter


%\begin{appendices}

%\section{Section title of first appendix}\label{secA1}



%%=============================================%%
%% For submissions to Nature Portfolio Journals %%
%% please use the heading ``Extended Data''.   %%
%%=============================================%%

%%=============================================================%%
%% Sample for another appendix section			       %%
%%=============================================================%%

%% \section{Example of another appendix section}\label{secA2}%
%% Appendices may be used for helpful, supporting or essential material that would otherwise 
%% clutter, break up or be distracting to the text. Appendices can consist of sections, figures, 
%% tables and equations etc.

%%===========================================================================================%%
%% If you are submitting to one of the Nature Portfolio journals, using the eJP submission   %%
%% system, please include the references within the manuscript file itself. You may do this  %%
%% by copying the reference list from your .bbl file, paste it into the main manuscript .tex %%
%% file, and delete the associated \verb+\bibliography+ commands.                            %%
%%===========================================================================================%%

\bibliography{sn-bibliography}% common bib file
%% if required, the content of .bbl file can be included here once bbl is generated
%%\input sn-article.bbl


\end{document}



\end{document}
