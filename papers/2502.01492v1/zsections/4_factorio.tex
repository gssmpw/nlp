

\section{Factorio as a System Engineering Testbed}
We now argue that \textit{Factorio} is an ideal environment to develop system engineering capability in AI agents. We describe the mechanics, features and extensible scope of the game and put forth a call for using \textit{Factorio} as a platform for public research. Interested readers can find a more detailed walkthrough of the game in Appendix~\ref{ref:sec_appendix}

\subsection{Overview}
\textit{Factorio} is a 2D, top-down factory-building game that centers on \textit{automation}, rendering it a uniquely rich environment for developing and evaluating AI agents with strong system engineering capabilities. Although it shares the open-sandbox approach of titles like \textit{Minecraft}, \textit{Factorio} is far better suited for this purpose as it emphasizes building systems with high throughput, efficiency, and resilience. Automating the production of goods—from early hand-assembled items to complex industrial chains—is not merely a side option but rather the heart of the gameplay. This emphasis on scaling and optimizing factories pushes agents to navigate challenges that mirror real-world engineering dilemmas: resource constraints, energy usage, logistical complexity, and even defensive measures against hostile forces.

A key metric of success is \textbf{science per minute} (SPM), a community-standard indicator of a factory’s overall efficiency in generating the science packs needed for technological progress. Because each successive tier of research unlocks new possibilities (e.g., improved assemblers, trains, robots) but also imposes heavier resource and energy demands, any small inefficiency can ripple into crippling bottlenecks. Consequently, an effective agent must maintain the appropriate degree of variety in its approach at all times, ensuring that its decision-making processes can handle the game’s growing complexity and unexpected fluctuations. SPM makes for a great summary benchmark metric, with human novice bases at $\sim$0-30 SPM, intermediate at $\sim$30-200 SPM and advanced bases at $\sim$200-1000+ SPM. 

From a VSM perspective, \textit{Factorio} initially starts players at purely System 1 activities like manually extracting coal and iron to be placed in a hand-crafted furnace. The use of automated conveyer belts with splitting and load-balancing mechanisms combined with automated inserter arms elevates design to a System 2 level. A key gameplay entity is the \textit{assembler} which can be programmed with a recipe to convert inputs to finished outputs using materials, power and space for operation. Scaling the automated production of intermediate goods setting up train cargo networks and selecting technology tree paths are all System 3, 4 and 5 functions (Table~\ref{table:vsm_table}). A pivotal late game technology is the use of \textit{automated construction robots} which can be used to rapidly bring and place materials in accordance with large, complex player-made blueprints. This capability hence focuses gameplay purely on systems-level control problems, choosing the right smelting column, railway depot, solar array configuration, etc. to evolve the base as needed.

This versatility is further magnified by \textit{Factorio}’s robust modding support, which allows researchers and the broader community to introduce new mechanics, custom APIs, or entire rebalanced rule sets. In other words, the sandbox nature of \textit{Factorio} can be extended indefinitely, enabling the environment itself to evolve and stress-test an agent’s capacity to adapt and manage variety. Such flexibility in scaling and customization makes \textit{Factorio} ideal for public research, as it encourages the development of AI agents that can grow beyond initial, narrowly-defined tasks and rise to dynamic challenges that demand integrated System 1–5 competencies.

\subsection{Challenges for Current AI Agents}

While AI agents have made remarkable progress in reasoning and multimodal interaction, there remains a sizable gap between the capabilities of frontier agents (e.g. \cite{deepmind2024mariner, openai2025operator}) and the level of sophistication needed to thrive in \textit{Factorio}—and, by extension, in complex real-world systems. For example, \textit{Factorio} uses traditional a keyboard-and-mouse interface with numerous GUI windows and features detailed real-time visualization—where every item, belt, or robot is tracked on screen from a 2D view. This is coupled with the ability to view monitoring for practically all processes, placing it at the cutting edge of current AI capabilities for handling multimodal data bandwidth and human interface use. 

Bases are commonly developed over several dozens if not hundreds of hours. There is a tremendous amount of temporal information involved in optimizing a base which would certainly test the long-context nature of frontier agents. Sophisticated memory and recall systems would undoubtedly be necessary for an LLM-based agent to succeed in an extended episode playing Factorio. Separately, planning for the future would certainly benefit from time spent reasoning, but this comes at a cost when acting in a real-time environment, hence aligning interplay of System 3 and 4 with a key compute usage trade-off.

Addressing these technical barriers also highlights the importance of multi-agent collaboration: large-scale systems often require multiple agents or human-agent teams working in sync. This necessitates coordination frameworks that facilitate shared state and efficient task delegation. Moreover, real-world complexities like supply-chain delays or hardware breakdowns call for robust decision-making under uncertainty—agents must act swiftly and safely, even with incomplete information. Nevertheless, scaling compute FLOPs and refining AI architectures are likely to improve input-output flow management to the point where agents can handle advanced simulations like \textit{Factorio} in real time (realistically the game only needs to be played at around 5 FPS), without relying on domain-specific observation and action spaces, as was common in earlier superhuman-agent research such as AlphaStar.

\subsection{Modding, Market Interactions, and the Agent-Evaluator Framework}

Factorio’s \textbf{modding ecosystem} is unusually flexible, allowing Lua scripts to fundamentally alter or extend nearly every facet of the simulation. At one end, small “Quality-of-Life” mods streamline actions like inventory management or blueprint deployment—an approach often mirrored in real-world industrial systems where specialized scripts automate repetitive tasks. At the other end, total conversion mods, such as \textit{Space Exploration} \cite{spaceExploration} or \textit{Industrial Revolution 3} \cite{industrialRevolution} introduce entirely new resources, tech trees, and production pipelines. This capacity for extensive re-parameterization means researchers can craft tailored scenarios focusing on, for example, large-scale chemical manufacturing or advanced energy grids. By doing so, Factorio can serve as a robust platform for evaluating AI agents under conditions that closely resemble real-world system engineering challenges.

An especially promising application of this modding framework involves designing \textbf{market pricing} and \textbf{multi-agent} interactions. Factorio already supports multiplayer, and community-created mods showcase how resource trading, diplomatic pacts and emergent economies can drive the game’s complexity \cite{blackMarket, diplomacy}. In a research context, introducing dynamic markets would allow agents to buy and sell resources, negotiate prices, and even form alliances or contracts—key elements of real-world logistics and supply chains. Observing how AI agents adapt to fluctuating market forces and coordinate with others could yield insights into cooperative and competitive strategies, as well as negotiation tactics and resilient system designs.

Beyond market dynamics, Factorio’s modding API also lends itself to the concept of a \textbf{Agent-Evaluator Framework}. In this paradigm, a “evaluator” agent (human or AI) orchestrates scenario constraints, random events, or objectives (Informing system 5 as per the VSM) while the “agent” attempts to build and maintain a functional factory. This setup is well-suited to self play-like reinforcement learning algorithms, where the evaluator can inject perturbations—ranging from supply shortages to power-grid failures—testing the agent’s capacity for adaptive, long-horizon decision-making. The evaluator could also coordinate multiple agents with distinct roles or goals, enabling both collaboration and competition. Such arrangements bring Factorio closer to real-world engineering environments, where teams of engineers and managers must not only design but also continually refactor systems in response to shifting requirements and unforeseen disruptions.

By blending flexible modding, multi-agent mechanics, and the Agent-Evaluator approach, Factorio becomes more than just a factory-building game. It becomes a powerful sandbox for studying how AI agents might operate in large-scale, ever-evolving ecosystems—spanning everything from supply-chain economics to self-directed adaptation and robust error handling. This versatility sets Factorio apart as a uniquely comprehensive testbed for advancing AI-driven system engineering. 

\subsection{Technical Advantages}
Beyond the near-limitless opportunities provided by mods, \textit{Factorio} offers a few key advantages that are worth highlighting. First, as a 2D game, it is far more resource efficient for the complexity of systems that can be built in it as it does not involve costly 3D graphics rendering as would be the case in other titles such as \textit{Satisfactory}. Even despite this major difference, \textit{Factorio} is well-known to be a very well-optimized game in terms of memory usage, as it has been continually refined by its dedicated team since its first public release in 2012. Furthermore, the game is platform-agnostic, running natively on Windows, Mac OS X and Linux, which is rare. It offers a free headless Linux server for supporting well-optimized multiplayer gameplay which would be crucial for human-AI and multi-AI agent experimentation. And as mentioned before, the game has exceptional support for modding, showcased by community mods which completely overhaul the tech tree, environmental mechanics and GUI systems. We believe it is quite feasible to build an API layer for control as an intermediate solution for AI usage similar to Mineflayer \cite{mineflayer2024} (used in the Voyager project \cite{wang2023voyager}). In fact, that could even be a task for an AI agent to perform as part of its introduction to the game.