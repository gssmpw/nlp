\section{Alternative Views}

Some critics argue that advancing AI system engineering is premature, given that core capabilities—like consistent reasoning, robust multimodality, and factual grounding—remain underdeveloped. They believe AI should first address these foundational weaknesses before tackling higher-level tasks. Yet proactive, orthogonal research can reveal new performance bottlenecks and drive innovation across modalities. Much as multimodality has progressed alongside unresolved text-based issues, tackling system engineering now can highlight what crucial gaps persist, helping to shape more integrated AI architectures.

Another concern is the risk of entrusting critical infrastructures to automated agents. Misaligned objectives or flawed reasoning could theoretically sabotage energy grids, supply chains, or other vital systems. While these dangers merit attention, the potential benefits—greater efficiency, cost savings, and creative solutions—are substantial. Alignment sits at the core of system engineering, which is rooted in clear requirements, continuous feedback loops, and stakeholder validation. By maintaining transparency and accountability, AI-driven engineering can strike a balance between prudence and progress.

Skeptics may also doubt whether games like \textit{Factorio} adequately reflect real-world complexities, noting they often omit granular physical laws or regulatory constraints. Yet such “unrealistic” environments highlight the essence of system engineering—resource management, strategic planning, and iterative trade-offs in efficiency, adaptability, and cost—far better than static benchmarks and without the noise associated with realistic physics simulations. Skills developed in orchestrating large-scale virtual factories can be paired with domain-specific testing to produce a fuller assessment of AI’s strengths. This integrated approach shows where AI excels (e.g., in macro-level design) and where further refinement is needed before applying these insights to physical-world applications.