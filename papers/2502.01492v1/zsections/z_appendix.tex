% resource extraction + smelting setup
% complex belts
% rail network (map and train loading)
% biters and defense
% power options 
% fluid/chemical processing
% circuit networks
% main bus design, mall design, modular base design
% assembler and recipes
% inventory GUI
% blueprint GUI
% construction robots


% https://factoriocheatsheet.com
% https://calculatorio.com/resources/
% https://forums.factorio.com/viewtopic.php?t=97410
\newpage
\appendix
\section{Visual Introduction to Factorio}
\label{ref:sec_appendix}

This appendix provides a high-level, illustrated overview of key \textit{Factorio} systems, ensuring that newcomers can grasp the fundamental mechanics of extracting resources, setting up production lines, defending against threats, and automating workflows. Each subsection introduces core concepts, from the simplest mining operations to advanced infrastructures like rail networks, circuit logic, and robot-assisted construction.

\subsection{Resource Extraction and Smelting}
% iron_mining_and_smelting
% 
The foundation of any Factorio factory is consistent raw material throughput. Players begin by placing mining drills on ore patches—such as iron or copper—where the drills extract resources at a steady rate. Ores are usually transported via conveyor belts to nearby smelters, which convert them into plates. A typical early-game setup involves an arrangement of furnaces linked by belts on both the input (ore) and output (finished plates) sides. This workflow underpins the factory’s growth: higher demand for plates necessitates expanding both mining operations and smelting capacity.

\begin{figure}[ht]
    \centering
    \includegraphics[width=\columnwidth]{zimages/resources/iron_mining_and_smelting} % Replace with your image file name
    \caption{An example of early-game resource extraction and smelting in Factorio. Box A shows mining drills extracting iron ore, Box B highlights stone furnaces which take ore and fuel and create plates, and Box C highlights belt routing and inserter mechanics. \cite{ironMiningSmelting}}
    \label{fig:iron_smelting}
\end{figure}
We first refer to Figure~\ref{fig:iron_smelting}. The automation process begins with the mining drills in \textbf{Box A}, which extract raw iron ore (blue material) from resource nodes and place it onto conveyor belts for transport downstream. These drills eliminate the need for manual mining, significantly increasing throughput and setting the foundation for automated workflows. In \textbf{Box B}, the raw iron ore is delivered to stone furnaces, where inserters (mechanical arms) feed the ore into the furnaces and remove the resulting iron plates—a critical intermediate resource—onto separate belts. This dual-belt system, fed by both raw iron ore and coal, ensures a continuous and automated smelting process. Finally, in \textbf{Box C}, the belts are routed efficiently using underground segments to avoid intersection conflicts, enabling seamless transport of resources. Yellow inserters deliver iron ore into the furnaces, while red inserters extract the smelted iron plates, which are then routed onward for further processing. This layered system of extraction, smelting, and material routing illustrates the early-game challenges of compact, efficient factory design in Factorio.


\subsection{Automation with Assemblers and Managing Complexity}
Automation through assemblers is a cornerstone of \textit{Factorio} gameplay, enabling exponential growth in productivity by trading energy and space for vastly higher throughput. The production of \textit{science packs} is central to the objective of unlocking advanced technologies. The earliest science packs are automation (red) and logistic (green) science. The dependencies for crafting these are shown in Figure~\ref{fig:red_green_dependencies}. Taking the example of logistic science, Figure~\ref{fig:green_science_recipe} illustrates that it takes 6s to assemble if the intermediate goods of transport belts and inserters are available. If only the raw materials of iron and copper plates are present then it will take 8.7s since the intermediate goods themselves need to be created. Hence, by automating intermediate goods, the factory can parallelize workflows, ensuring higher efficiency and faster output. 

\begin{figure}[ht]
    \centering
    \includegraphics[width=0.6\columnwidth]{zimages/recipes/red_green_dependencies}
    \caption{Dependency graph for red and green science packs. Inputs include both raw materials and intermediates, reflecting the growing complexity of production chains.}
    \label{fig:red_green_dependencies}
\end{figure}


There are thus many points to consider when designing assembly lines for these finished goods. The throughput of inputs and outputs should be well-matched given the ratios of materials needed in recipes. The demand for a common base resource has to be managed well across different use cases. The factory has to be actively refactored as increased scale means greater space and energy requirements. An example of a compact design which produces both red and green science is shown in Figure~\ref{fig:red_green_assembly}. While it looks efficient, issues may arise when the scale of production needs to increase, since the routing of intermediate goods would be significantly complicated. Efficient layouts must balance immediate needs with scalability, ensuring that adding new production lines or expanding capacity can be achieved without overhauling the entire factory.

\begin{figure}[ht]
    \centering
    \includegraphics[width=0.5\columnwidth]{zimages/recipes/green_science_recipe2}
    \caption{Recipe for logistic (green) science packs. Automating intermediate goods significantly reduces total crafting time from 8.7 seconds (raw) to 6 seconds.}
    \label{fig:green_science_recipe}
\end{figure}



\subsection{Science Packs and the Tech Tree}
% tech tree screenshots, also refer to assembler/recipe screenshot
The science packs produced in Figure~\ref{fig:red_green_assembly} are central to progression in \textit{Factorio}, serving as the primary currency for unlocking new technologies. Science packs are consumed by specialized \textit{laboratory} units, which convert them into research progress. Each research project requires a specific combination and quantity of science packs, introducing dependencies on a wide array of intermediate goods. This makes science packs a natural bottleneck for factory growth, as they represent the culmination of multiple production chains working in harmony.

This reliance on science packs is why science per minute (SPM) emerges as a critical metric for measuring factory productivity. A high SPM indicates that the factory has sufficient capacity not only to produce the required intermediates efficiently but also to scale them as the tech tree demands become more complex. For example, early-game science packs (red and green) require relatively simple intermediates such as gears, transport belts, and inserters, as shown in Figure~\ref{fig:green_science_tech}. However, as the factory evolves, higher-tier science packs (such as blue or utility science) introduce more advanced recipes involving fluids, electronics, and complex multi-stage production.

The tech tree in Figure~\ref{fig:green_science_tech} highlights this progression, showcasing how early-game technologies provide foundational tools like transport belts and inserters, which are then leveraged to unlock more advanced machinery such as trains and assemblers. This cascading dependency structure requires careful planning to ensure that production systems remain adaptable to increasing demands. The iterative process of unlocking technologies feeds back into the factory itself, enabling further automation and resource optimization. The science system is thus a core gameplay mechanic that ties together automation, logistics, and long-term planning, creating a continuous cycle of technological advancement and production refinement.

\begin{figure}[ht]
    \centering
    \includegraphics[width=0.7\columnwidth]{zimages/tech_tree/green_science_tech}
    \caption{Section of the tech tree which shows which technologies have logistic (green) science as a dependency. These include better transport belts, engines, trains, electric cables, circuit networks and more. Each recipe demands dozens if not hundreds of logistic science packs and so SPM becomes the bottleneck for further growth.}
    \label{fig:green_science_tech}
\end{figure}



\subsection{Power Generation Options}
% nuclear + solar panels
% https://www.reddit.com/r/factorio/comments/vmumny/my_473_mw_nuclear_power_plant_complete_with/
% power monitor
% https://forums.factorio.com/viewtopic.php?t=89602
% nuclear only: https://wiki.factorio.com/File:Nuclear_setup.png
Factorio offers diverse power solutions that evolve with the factory’s scale, closely mirroring the progression of energy systems in real-world industrial engineering. Early operations rely on steam engines fueled by coal, providing a reliable but resource-intensive solution. Coal mining introduces logistical challenges, requiring consistent supply chains and raising concerns about pollution, which in the game aggravates hostile aliens and causes them to attack the agent's base (see the next subsection). As research progresses, solar panels and accumulators become viable for renewable energy. While solar panels offer clean, sustainable power, they come with limitations tied to diurnal cycles, requiring accumulators to store excess energy for nighttime use. This trade-off between sustainability and infrastructure demands mirrors the challenges of integrating renewables into modern power grids, where storage and energy distribution systems are key bottlenecks.

Nuclear power, a late-game solution, exemplifies high-density energy production but comes with its own complexities. Players must process uranium, manage heat generation, and design safe reactor layouts to avoid catastrophic failures, echoing real-world concerns around nuclear fuel cycles, reactor safety, and waste management. The spatial footprint of energy systems also becomes a critical factor: steam and nuclear setups require compact layouts with high resource input, while sprawling solar farms demand significant land allocation. A reference for nuclear power is shown in Figure~\ref{fig:nuclear_power}

\begin{figure}[ht]
    \centering
    \includegraphics[width=\columnwidth]{zimages/power/nuclear_power.pdf}
    \caption{Nuclear power generation is actually quite realistic in \textit{Factorio}. Uranium ore is mined, the vast majority of which (99.3\%) is inert U-238. The more valuable U-235 is needed for energy-intensive applications. There is an enrichment process by which U-238 can be refined to make more U-235 provided some initial quantity of U-235. Then this is utilized in nuclear reactors which can produce steam to power turbines. \cite{nuclear}}
    \label{fig:nuclear_power}
\end{figure}

Each energy choice in Factorio presents distinct trade-offs—coal introduces pollution but offers consistency, solar minimizes pollution but requires storage solutions and space, and nuclear delivers immense power but requires advanced materials and precise management. These dynamics force players to weigh efficiency, scalability, and sustainability, capturing the essence of systems engineering in real-world energy infrastructure. By gradually introducing more advanced technologies and requiring players to adapt their power networks, Factorio illustrates the iterative process of scaling energy systems to meet growing demands while addressing environmental and logistical constraints.

\subsection{Biters and Defense}
% biters getting shot
% https://wiki.factorio.com/index.php?curid=46602
% biter base
% https://www.factorio.com/blog/post/fff-358
As factories grow and produce pollution, the indigenous alien lifeforms—commonly called Biters—become increasingly hostile, posing a persistent threat to factory operations. Pollution emitted by the factory spreads across the map, and once it reaches a Biter colony, such as the one depicted in Figure~\ref{fig:biter_colony}, it triggers aggressive behavior. Biters begin spawning in waves to attack the factory, targeting structures and resources critical to production. This introduces a dynamic tension between industrial expansion and the need to secure valuable infrastructure, reflecting real-world trade-offs in industrial development where growth often necessitates heightened security measures.

\begin{figure}[ht]
    \centering
    \includegraphics[width=\columnwidth]{zimages/defense/biters.pdf}
    \caption{Biters are alien residents of the planet where the agent has crash landed. They are docile initially but become aggravated by air pollution from the factory's hydrocarbon-powered operations. Thinking about biters is thus a core trade-off of expanding systems in Factorio.}
    \label{fig:biter_colony}
\end{figure}

\begin{figure}[ht]
    \centering
    \includegraphics[width=\columnwidth]{zimages/defense/biter_defense.pdf}
    \caption{Defense against Biters is essentially a resource sink in \textit{Factorio}. Settings and mods can be used to dramatically change the difficulty associated with defending bases from Biters.}
    \label{fig:biter_defense}
\end{figure}

\begin{figure*}[htb]
    \centering
    \includegraphics[width=0.8\textwidth]{zimages/recipes/red_green_assembly}
    \caption{A red and green science production setup. All belts are running from left to right. Iron and copper plates enter on the bottom-most belt (Box A). There are assemblers throughout the line which have certain recipes selected. For example Box B has the assembler responsible for assembling green circuits. The inserters to the right of Box B automatically pull copper wire from that assembler and the yellow inserter above Box B pulls iron plates from the belt. The red inserter above Box B places finished green circuits onto the belt one tile above the belt with iron and copper plates. Similar assembly happens for gears, belts, and inserters. Ultimately, red and green science packs are produced from the intermediate goods and are ready for further use (Box C).}
    \label{fig:red_green_assembly}
\end{figure*}

Early defenses rely on a combination of walls and gun turrets, as seen in Figure~\ref{fig:biter_defense}, where turrets gun down an approaching wave of Biters at a fortified perimeter. Gun turrets provide reliable protection during the early stages but depend on a steady supply of ammunition, which itself requires dedicated production lines. As the factory evolves, more advanced defensive structures like flamethrower turrets, laser turrets, and artillery become available. Flamethrowers are particularly effective for handling large swarms due to their area-of-effect damage, while laser turrets require no ammunition but demand significant power, introducing another layer of logistical complexity. Artillery, a late-game option, allows players to strike Biter nests at long range, proactively reducing the threat level.



Strategically fortifying perimeters and clearing nearby Biter nests becomes essential as pollution spreads farther and factory operations grow in scale. Defensive layouts must balance resource efficiency with resilience, ensuring that critical areas are well-protected without overextending the factory’s capacity to supply power, ammunition, or repairs. Additionally, players must consider choke points, turret placement, and overlapping fields of fire to maximize defensive effectiveness.

\subsection{Complex Belts and Main Factory Layouts}
% main bus 
% 
% modular design
% 
Conveyor belts are the arteries of a \textit{Factorio} base, transporting materials between production stages with speed and efficiency. While straightforward in the early game, managing belts becomes increasingly complex as factories grow. Scaling introduces challenges such as belt congestion, balancing input and output ratios, and ensuring that each production branch receives the right materials without overloading the system. Designing efficient layouts to manage this complexity is critical for avoiding “spaghetti”—a term used by the community to describe tangled, chaotic belt arrangements that hinder scalability and troubleshooting.

One popular solution to these challenges is the \textbf{main bus} design, as shown in Figure~\ref{fig:main_bus}. A main bus consists of a centralized set of parallel belts carrying essential resources like iron plates, copper plates, gears, and circuits. Branches extend from the main bus to feed production lines, ensuring that critical resources are readily available across the factory. This design prioritizes simplicity and organization, making it easier to scale production by adding new branches or extending the bus itself. However, maintaining a main bus requires careful planning to prevent bottlenecks and to allocate space for future resource additions. Players must also ensure that belts remain balanced to avoid starving downstream branches of materials.

\begin{figure}[ht]
    \centering
    \includegraphics[width=\columnwidth]{zimages/layouts/main_bus.pdf}
    \caption{The main bus design is a common choice for mid-game scaling. Branches for individual component assembly fork off the main bus using belt splitters and underground belts. \cite{mainBus}}
    \label{fig:main_bus}
\end{figure}

An alternative to the main bus approach is the \textbf{city block} design, illustrated in Figure~\ref{fig:city_block}. In this modular approach, the factory is divided into distinct “blocks,” each dedicated to a specific function, such as smelting, circuit production, or science pack assembly. These blocks are connected by train networks, allowing resources to be transported efficiently between distant sections of the factory. The city block layout offers excellent scalability, as additional blocks can be added without disrupting existing workflows. It also improves manageability, as each block operates semi-independently, reducing the risk of widespread factory failures due to localized issues.

Both layouts demonstrate distinct trade-offs. The main bus design excels in compactness and simplicity, making it ideal for medium-sized factories, but it can become unwieldy as the number of resources grows in the late game. City block layouts, while more complex to set up initially, provide unmatched flexibility and extensibility, especially when managing large-scale operations with diverse production needs. Figures~\ref{fig:main_bus} and~\ref{fig:city_block} highlight the strengths of these designs, showcasing how thoughtful belt organization and transportation planning are essential for managing complexity and ensuring smooth factory operation as production demands increase.

\begin{figure}[ht]
    \centering
    \includegraphics[width=\columnwidth]{zimages/layouts/city_blocks2.pdf}
    \caption{The city block design is ideal for late-game mega-base building. Modular base sections are linked using rail networks for loading and unloading of items. \cite{cityBlock}}
    \label{fig:city_block}
\end{figure}




\subsection{Rail Networks}
% plenty of stuff in tutorial
% GUi
% https://steamcommunity.com/sharedfiles/filedetails/?id=2737259470
% intersections and shit
% https://www.reddit.com/r/factorio/comments/8bappn/chunk_aligned_rhd_rail_blueprints_mostly_for_141/
% loading/unloading
% https://www.reddit.com/r/factorio/comments/j0ftu0/consuming_a_full_blue_belt_with_3_stack_inserters/

At mid to late stages of \textit{Factorio}, trains become a critical component of resource logistics, allowing raw materials and finished goods to be transported across vast distances. Tracks are laid out on a tile-based map, with stations configured for specific tasks such as ore pickups and deliveries to smelting or assembly sites. Trains enable players to overcome the limitations of conveyor belts, which can become cumbersome and inefficient for long-range transport, providing a scalable solution that supports factory growth.

\begin{figure}[ht]
    \centering
    \includegraphics[width=\columnwidth]{zimages/rail/train_loading2.pdf}
    \caption{Trains can move large quantities of resources long distances much faster than belts while reusing the same underlying infrastructure, making them crucial for any scalable build. \cite{trainUnloading}}
    \label{fig:_train_loading}
\end{figure}

Designing a robust railway system requires careful planning and mastery of key mechanics. Figure~\ref{fig:_train_loading} shows a typical train loading setup, where numerous inserters work in parallel to load ore into cargo carriages quickly. Efficient loading and unloading are essential to minimize train idle times and ensure smooth throughput. Multiple stations can be linked along a track network, with each station named and assigned schedules dictating when trains should arrive and depart, further streamlining the flow of resources between locations.


\begin{figure}[ht]
    \centering
    \includegraphics[width=\columnwidth]{zimages/rail/_train_intersections.pdf}
    \caption{The variety and customization associated with building rail networks is vast in Factorio. \cite{trainPatterns}}
    \label{fig:_train_intersections}
\end{figure}

Track layouts, particularly intersections and junctions, are another crucial element. Figure~\ref{fig:_train_intersections} highlights several blueprint designs for rail intersections, showcasing patterns optimized for traffic flow and collision avoidance. Proper signaling is necessary to manage multiple trains on shared tracks, with block signals and chain signals controlling which sections of track are reserved for individual trains. Complex networks can handle dozens of trains simultaneously, but poor design or inadequate signaling can lead to congestion or catastrophic collisions, disrupting the factory’s supply chains.

\begin{figure}[ht]
    \centering
    \includegraphics[width=\columnwidth]{zimages/rail/train_gui2.pdf}
    \caption{Factorio gives players the ability to observe and orchestrate train networks with high customization \cite{trainGuide}}
    \label{fig:_train_monitor}
\end{figure}

The train management system extends beyond physical tracks, as shown in Figure~\ref{fig:_train_monitor}, which displays the GUI for monitoring train activity. This interface allows players to track the status of all trains in the network, observe their current locations, and adjust schedules or routes as needed. The train monitor is an invaluable tool for diagnosing delays, optimizing routes, and ensuring that all resource flows remain balanced.

A well-designed railway system is not just a means of transport but a backbone for factory expansion, allowing new outposts and production sites to be integrated seamlessly into the larger network. By balancing efficient loading, modular track designs, and robust train management, players can scale their factories to unprecedented levels while maintaining resource flow and minimizing logistical bottlenecks.


%\subsection{Fluid and Chemical Processing}
%Oil refineries and chemical plants expand production beyond solid intermediates. Crude oil can be refined into multiple products (e.g., petroleum gas, heavy oil, light oil), each feeding into specialized recipes like plastics, sulfur, and lubricants. Pipe networks introduce new routing challenges—especially when storage tanks and by-product management complicate typical belt-based workflows. Fuel selection, cracking processes, and circuit-regulated valves are just a few examples of intricate fluid logistics.

%\subsection{Circuit Networks}
%Circuit networks allow players to program conditional behaviors for factory operations. Wires connect machines and storage, enabling control signals that can, for instance, stop a belt if storage is full or dispatch a train when a station demands resources. While optional, circuit logic can optimize resource usage, reduce power drain, or even orchestrate precise production cycles. Mastering circuits is akin to learning a lightweight scripting language for industrial automation.

\begin{figure}[ht]
    \centering
    \includegraphics[width=\columnwidth]{zimages/robots/constructionbots.pdf}
    \caption{Construction robots automate the placement of arbitrarily complex player-made blueprints. Here the blueprint has been partially constructed by robots and needs to be completed and connected to a source of power. \cite{constructionBots}}
    \label{fig:constructionBots}
\end{figure}


\subsection{Inventory, Blueprints, and Construction Robots}
Players interact with a comprehensive inventory GUI that tracks personal items, crafting queues, and equipment. This interface underpins many of the high-level systems within Factorio, ensuring that even the most complex production chains remain manageable. When testing these systems—particularly in large-scale or late-game scenarios—developers and players alike must verify that inventory updates, crafting queues, and personal equipment management work seamlessly without bottlenecking progress. Such testing is crucial because any inefficiency or bug in inventory handling can cascade throughout a massive base, undermining the player’s ability to grow their automation network.

A prime example of Factorio’s advanced systems is the \textbf{blueprint} feature, which allows users to save layouts ranging from simple assembler setups to sprawling smelter arrays. As shown in Figure \ref{fig:constructionBots}, pasting a blueprint summons construction robots to automatically assemble buildings and belts, provided that the necessary items are available and that the structures remain within the logistic network’s coverage (the robot hub range is visible in the center of the screenshot). High-level system testing involves confirming that these blueprint placements work at scale: robots must reliably build, repair, and upgrade components in the correct order and handle resource shortages gracefully. If the blueprint system or robot AI malfunctions, it can cause partial constructions or idle bots, quickly eroding the advantages of automation and frustrating the player.

\begin{figure}[ht]
    \centering
    \includegraphics[width=\columnwidth]{zimages/robots/destructionbots.pdf}
    \caption{Robots can also be used to efficiently clear out a factory and reclaim the resources. Here the section of the factory has been marked for clearance and robots will swarm it when the player finalizes the selection. \cite{destructionBots}}
    \label{fig:destructionBots}
\end{figure}

Moreover, these same construction robots facilitate large-scale deconstruction, an equally vital aspect of advanced base management. Figure \ref{fig:destructionBots} illustrates the user highlighting a section of the factory for removal—once marked, robots swarm to dismantle it, returning valuable materials to the appropriate storage points. Rigorous testing here ensures that no mismatches occur in item retrieval, that robots can safely access all structures slated for removal, and that the logistic system manages reclaimed items without jams. Essentially, blueprints and bots automate both production and the \textit{creation of production}, making the entire game experience highly recursive and reliant on flawlessly functioning high-level systems. Verifying these features in complex, large-scale conditions is critical for preserving Factorio’s hallmark sense of continual, smoothly scaling automation.
