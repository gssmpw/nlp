\section{The Importance of System Engineering}

In this section, we examine trends in complex system development and AI agents. We deduce that these will converge and we thus firmly establish the value proposition for developing system engineering capability in AI agents.

\subsection{The Ubiquity of Systems}

A system is defined as a collection of interacting components that together serve a function or purpose. Under this broad definition, the world as we know it is held together by complex systems. Key examples include:
\begin{itemize}
    \item \textbf{Transportation and Logistics.} Industries such as shipping, trucking, and ridesharing; public services like trains and buses; physical infrastructure projects including roads, bridges, and tunnels.
    \item \textbf{Energy Infrastructure.} Raw material acquisition and refinement; large-scale energy generation in specialized facilities; storage and distribution networks required for load balancing.
    \item \textbf{Modern Agriculture.} Encompassing land management (irrigation, fertilization, pest control), crop and livestock cycles, as well as packaging and distribution systems to bring products to market.
    \item \textbf{Advanced Manufacturing.} The creation of parts from raw materials; international supply chain coordination; final assembly of complex products across sectors such as computing, biotechnology, and aerospace.
    \item \textbf{Digital Ecosystems.} Physical networking infrastructure; internet hosting servers; cloud computing stacks; software frameworks, libraries, and algorithms.
\end{itemize}

\textbf{System engineering} refers to the design, implementation, and management of such large-scale systems that integrate hardware, software, and human processes. Our understanding of systems has evolved significantly over time. Early industrial breakthroughs (e.g., standardized components, assembly lines, electrification) introduced new layers of complexity by increasing production volumes, lowering costs, and extending distribution chains. In aerospace and defense projects, where massive interdisciplinary teams had to be coordinated, formal \textit{system engineering} practices emerged \cite{blanchard2010, kossiakoff2011, incose2015, buede2016, madni2018}. Over the years, these yielded iterative and agile methods emphasizing continuous integration and rapid feedback—trends that are now mainstream in digital ecosystems. 

The demand for building and scaling complex systems shows no signs of slowing down. Across the world, large projects are planned or underway to address unprecedented challenges in the form of energy needs, aging demographics, changing climate patterns, geopolitical tensions, and more \cite{mckinsey2022, deloitte2023, reshoring2023}. The competitive advancement of technology itself leads to self-reinforcing demand for systems, exemplified by staggering investments into AI-related infrastructure \cite{uscongress2022, stargate2025}. Budget overruns and timeline delays are all too common in implementing large projects, showing that human planning and system engineering has its limitations. It thus appears nearly certain that advanced AI will be crucial in tackling these challenges and implementing the solutions.

\subsection{The Rise of AI Agents}
% openai levels. level 4 innovators, level 5 organization
% industry leaders see great opportunities for widespread usage and integration into complex systems.
% Greater intelligence, greater safety
% data scarcity is emerging as an issue.
% browser is a dominant env for deploying agents and evaluating them
% for embodied/physical agents, there are physics-based simulations and some real-world robotics
% as there are more agents, multi-agent coordination will matter more.

Simultaneously, AI agents are gaining traction as the most promising framework for applying human-aligned generative AI models. Significant advances in multimodal input processing and reasoning through inference-time compute use have exposed the possibilities of autonomous agents using complex interfaces to accomplish tasks over longer time horizons. These opportunities are rapidly being realized through virtual agents which are increasingly using web browsers, code interpreters and other digital tools to automate workflows and offload other labor from humans \cite{devin2024, openai2025operator}. Progress in physical agents is also picking up steam, leaning on advances in both general-purpose foundation models and maturing robotics technology \cite{nvidia2024gr00t, musk2024optimus}. 

The tasks assigned to AI agents today may be composites of several smaller sub-tasks, but ultimately tend to be self-contained workflows. As AI agents become more reliable at executing these tasks, it will become more enticing to involve them in more system-level challenges. Systems-level expertise is always more scarce since it demands deep knowledge of choices for components, interconnections and associated trade-offs for costs, implementation time, complexity, scalability, etc. Training data for reasoning about systems is also commensurately scarce and so it will naturally present a challenge for improving AI agents. As AI agents proliferate, the challenges and opportunities associated with multi-agent coordination will also become more relevant and influence the efficacy of AI-enhanced systems. Hence, this trajectory of elevating AI agents to effectively work on system engineering seems central for achieving the long-term goals of developing AGI. 

\subsection{Implications of Superhuman System Engineering}

AI models and agents have shown superhuman ability in various domains. Historically, this is evidenced by their mastery of classically challenging board games and more complex real-time strategy video games \cite{silver2017alphazero, vinyals2019alphastar}. In applied settings, AI models have outperformed human-engineered solutions in predicting molecular structures \cite{alphafold2021}, weather patterns \cite{gencast2024}, and even designing certain GPU circuits \cite{prefixrl2021}. Recent frontier models demonstrate advances in multimodal and highly technical reasoning, suggesting that a trajectory toward general superhuman intelligence is potentially close. Given all this, it is worth considering the ramifications of successfully building superhuman AI system engineers.


We can look to current examples of top-tier system engineering by humans. These achievements share a common pattern: they redefined what was previously considered possible. For instance, when Apple replaced Intel chips with its in-house M-series CPUs, the gains in thermal performance, battery life, and software speed were widely seen as a generational leap. Similarly, SpaceX reshaped the frontier of aerospace by reducing costs by orders of magnitude and inventing reusable rocket technology. Meanwhile, the unrelenting dominance of Nvidia’s entire data center stack—from hardware to deep learning libraries—has made it (at least temporarily) the most valuable company in the world. In each of these cases, the teams in charge took ownership of the entire system, jointly optimizing it over many iteration cycles to achieve superior metrics. It is worth stressing that the same depth of expertise required for proposing a comprehensive initial design is needed for continually refactoring a system to improve its scalability, maintainability, security, and fault tolerance as development progresses


Superhuman AI agent capability in system engineering gives us a much better chance of addressing civilizational challenges, such as scaling clean energy systems, securing reliable water and food supplies, and lowering the cost of economically important finished goods. In other words, superhuman-level system engineering is the key to producing utopian abundance (provided we solve the alignment problem for such superintelligence). Full automation of physical projects will also require general-purpose robotics, which may be a bottleneck in the near term. Yet as those technologies mature, AI agents could combine high-level system design with low-level mechanical tasks, delivering a fully optimized, end-to-end engineering capability for arbitrarily complex systems. 