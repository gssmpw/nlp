\section{Conclusion}
AI agents stand on the verge of a new era where they can systematically design, optimize, and maintain complex systems in ways that rival or surpass human expertise. While LLMs have already showcased impressive capabilities for text generation, the true promise lies in the \textit{agentic paradigm}—with integrated multimodal interfaces, memory, autonomy, and adaptive planning.

We have argued that system engineering represents a high-leverage domain for such agentic AI. Whether the task is orchestrating large-scale software infrastructures or managing logistical networks, adaptability and continuous learning map naturally onto the strengths of a well-trained AI agent. Yet, to properly develop and evaluate these systems, we must look beyond static benchmarks toward open-ended simulations that reflect real-time constraints, multi-agent collaboration, and shifting objectives.

In this regard, \textit{Factorio} emerges as a compelling platform, providing a safe yet rich environment for refining agentic capabilities. Its emphasis on real-time resource management, multi-objective optimization, and large-scale factory layouts makes it a microcosm of industrial-scale challenge. Success in \textit{Factorio} would signal that agents can handle real complexity, track multiple objectives, and adapt in realistic ways.

In conclusion, the evolution from LLM-based chatbots to versatile AI agent that can tackle system engineering marks a logical next step if we hope to solve the grand challenges of our era. By leveraging automation-oriented sandbox simulations like \textit{Factorio}, we can accelerate progress toward AI systems that orchestrate research, design, and operations at scale—fundamentally reshaping how societies function and flourish in the coming decades.
