%%%%%%%% ICML 2025 EXAMPLE LATEX SUBMISSION FILE %%%%%%%%%%%%%%%%%

\documentclass{article}

% Recommended, but optional, packages for figures and better typesetting:
\usepackage{microtype}
\usepackage{graphicx}
\usepackage{subfigure}
%\usepackage[margin=1in]{geometry}
\usepackage{array}

\usepackage{booktabs} % for professional tables

% hyperref makes hyperlinks in the resulting PDF.
% If your build breaks (sometimes temporarily if a hyperlink spans a page)
% please comment out the following usepackage line and replace
% \usepackage{icml2025} with \usepackage[nohyperref]{icml2025} above.
\usepackage{hyperref}


% Attempt to make hyperref and algorithmic work together better:
\newcommand{\theHalgorithm}{\arabic{algorithm}}

% Use the following line for the initial blind version submitted for review:
%\usepackage{icml2025}

% If accepted, instead use the following line for the camera-ready submission:
\usepackage[accepted]{icml2025}

% For theorems and such
\usepackage{amsmath}
\usepackage{amssymb}


\usepackage{mathtools}
\usepackage{amsthm}

% if you use cleveref..
\usepackage[capitalize,noabbrev]{cleveref}

%%%%%%%%%%%%%%%%%%%%%%%%%%%%%%%%
% THEOREMS
%%%%%%%%%%%%%%%%%%%%%%%%%%%%%%%%
\theoremstyle{plain}
\newtheorem{theorem}{Theorem}[section]
\newtheorem{proposition}[theorem]{Proposition}
\newtheorem{lemma}[theorem]{Lemma}
\newtheorem{corollary}[theorem]{Corollary}
\theoremstyle{definition}
\newtheorem{definition}[theorem]{Definition}
\newtheorem{assumption}[theorem]{Assumption}
\theoremstyle{remark}
\newtheorem{remark}[theorem]{Remark}

% Todonotes is useful during development; simply uncomment the next line
%    and comment out the line below the next line to turn off comments
%\usepackage[disable,textsize=tiny]{todonotes}
\usepackage[textsize=tiny]{todonotes}

\begin{document}

\twocolumn[
%\icmltitle{Develop System Engineering in AI Agents with Automation Sandbox Games}
\icmltitle{Develop AI Agents for System Engineering in \textit{Factorio}}

% It is OKAY to include author information, even for blind
% submissions: the style file will automatically remove it for you
% unless you've provided the [accepted] option to the icml2025
% package.

% List of affiliations: The first argument should be a (short)
% identifier you will use later to specify author affiliations
% Academic affiliations should list Department, University, City, Region, Country
% Industry affiliations should list Company, City, Region, Country

% You can specify symbols, otherwise they are numbered in order.
% Ideally, you should not use this facility. Affiliations will be numbered
% in order of appearance and this is the preferred way.
\icmlsetsymbol{equal}{*}

\begin{icmlauthorlist}
\icmlauthor{Neel Kant}{hai}
%\icmlauthor{}{sch}
%\icmlauthor{}{sch}
\end{icmlauthorlist}

\icmlaffiliation{hai}{Hippocratic AI}

\icmlcorrespondingauthor{Neel Kant}{neel@hippocraticai.com}

% You may provide any keywords that you
% find helpful for describing your paper; these are used to populate
% the "keywords" metadata in the PDF but will not be shown in the document
\icmlkeywords{Machine Learning, Agentic AI, Systems Engineering, Reinforcement Learning, Adaptability, Open-Ended Simulations, Complex Systems, Industrial Automation, AI Infrastructure, Factorio, Continuous Learning.}

\vskip 0.3in
]

% this must go after the closing bracket ] following \twocolumn[ ...

% This command actually creates the footnote in the first column
% listing the affiliations and the copyright notice.
% The command takes one argument, which is text to display at the start of the footnote.
% The \icmlEqualContribution command is standard text for equal contribution.
% Remove it (just {}) if you do not need this facility.

\printAffiliationsAndNotice{}  % leave blank if no need to mention equal contribution
%\printAffiliationsAndNotice{\icmlEqualContribution} % otherwise use the standard text.

\begin{abstract}
Continuing advances in frontier model research are paving the way for widespread deployment of AI agents. Meanwhile, global interest in building large, complex systems in software, manufacturing, energy and logistics has never been greater. Although AI-driven system engineering holds tremendous promise, the static benchmarks dominating agent evaluations today fail to capture the crucial skills required for implementing dynamic systems, such as managing uncertain trade-offs and ensuring proactive adaptability. \textbf{This position paper advocates for training and evaluating AI agents’ system engineering abilities through automation-oriented sandbox games—particularly \textit{Factorio}.} By directing research efforts in this direction, we can equip AI agents with the specialized reasoning and long-horizon planning necessary to design, maintain, and optimize tomorrow’s most demanding engineering projects.
\end{abstract}

\section{Introduction}
Backdoor attacks pose a concealed yet profound security risk to machine learning (ML) models, for which the adversaries can inject a stealth backdoor into the model during training, enabling them to illicitly control the model's output upon encountering predefined inputs. These attacks can even occur without the knowledge of developers or end-users, thereby undermining the trust in ML systems. As ML becomes more deeply embedded in critical sectors like finance, healthcare, and autonomous driving \citep{he2016deep, liu2020computing, tournier2019mrtrix3, adjabi2020past}, the potential damage from backdoor attacks grows, underscoring the emergency for developing robust defense mechanisms against backdoor attacks.

To address the threat of backdoor attacks, researchers have developed a variety of strategies \cite{liu2018fine,wu2021adversarial,wang2019neural,zeng2022adversarial,zhu2023neural,Zhu_2023_ICCV, wei2024shared,wei2024d3}, aimed at purifying backdoors within victim models. These methods are designed to integrate with current deployment workflows seamlessly and have demonstrated significant success in mitigating the effects of backdoor triggers \cite{wubackdoorbench, wu2023defenses, wu2024backdoorbench,dunnett2024countering}.  However, most state-of-the-art (SOTA) backdoor purification methods operate under the assumption that a small clean dataset, often referred to as \textbf{auxiliary dataset}, is available for purification. Such an assumption poses practical challenges, especially in scenarios where data is scarce. To tackle this challenge, efforts have been made to reduce the size of the required auxiliary dataset~\cite{chai2022oneshot,li2023reconstructive, Zhu_2023_ICCV} and even explore dataset-free purification techniques~\cite{zheng2022data,hong2023revisiting,lin2024fusing}. Although these approaches offer some improvements, recent evaluations \cite{dunnett2024countering, wu2024backdoorbench} continue to highlight the importance of sufficient auxiliary data for achieving robust defenses against backdoor attacks.

While significant progress has been made in reducing the size of auxiliary datasets, an equally critical yet underexplored question remains: \emph{how does the nature of the auxiliary dataset affect purification effectiveness?} In  real-world  applications, auxiliary datasets can vary widely, encompassing in-distribution data, synthetic data, or external data from different sources. Understanding how each type of auxiliary dataset influences the purification effectiveness is vital for selecting or constructing the most suitable auxiliary dataset and the corresponding technique. For instance, when multiple datasets are available, understanding how different datasets contribute to purification can guide defenders in selecting or crafting the most appropriate dataset. Conversely, when only limited auxiliary data is accessible, knowing which purification technique works best under those constraints is critical. Therefore, there is an urgent need for a thorough investigation into the impact of auxiliary datasets on purification effectiveness to guide defenders in  enhancing the security of ML systems. 

In this paper, we systematically investigate the critical role of auxiliary datasets in backdoor purification, aiming to bridge the gap between idealized and practical purification scenarios.  Specifically, we first construct a diverse set of auxiliary datasets to emulate real-world conditions, as summarized in Table~\ref{overall}. These datasets include in-distribution data, synthetic data, and external data from other sources. Through an evaluation of SOTA backdoor purification methods across these datasets, we uncover several critical insights: \textbf{1)} In-distribution datasets, particularly those carefully filtered from the original training data of the victim model, effectively preserve the model’s utility for its intended tasks but may fall short in eliminating backdoors. \textbf{2)} Incorporating OOD datasets can help the model forget backdoors but also bring the risk of forgetting critical learned knowledge, significantly degrading its overall performance. Building on these findings, we propose Guided Input Calibration (GIC), a novel technique that enhances backdoor purification by adaptively transforming auxiliary data to better align with the victim model’s learned representations. By leveraging the victim model itself to guide this transformation, GIC optimizes the purification process, striking a balance between preserving model utility and mitigating backdoor threats. Extensive experiments demonstrate that GIC significantly improves the effectiveness of backdoor purification across diverse auxiliary datasets, providing a practical and robust defense solution.

Our main contributions are threefold:
\textbf{1) Impact analysis of auxiliary datasets:} We take the \textbf{first step}  in systematically investigating how different types of auxiliary datasets influence backdoor purification effectiveness. Our findings provide novel insights and serve as a foundation for future research on optimizing dataset selection and construction for enhanced backdoor defense.
%
\textbf{2) Compilation and evaluation of diverse auxiliary datasets:}  We have compiled and rigorously evaluated a diverse set of auxiliary datasets using SOTA purification methods, making our datasets and code publicly available to facilitate and support future research on practical backdoor defense strategies.
%
\textbf{3) Introduction of GIC:} We introduce GIC, the \textbf{first} dedicated solution designed to align auxiliary datasets with the model’s learned representations, significantly enhancing backdoor mitigation across various dataset types. Our approach sets a new benchmark for practical and effective backdoor defense.



\section{The Importance of System Engineering}

In this section, we examine trends in complex system development and AI agents. We deduce that these will converge and we thus firmly establish the value proposition for developing system engineering capability in AI agents.

\subsection{The Ubiquity of Systems}

A system is defined as a collection of interacting components that together serve a function or purpose. Under this broad definition, the world as we know it is held together by complex systems. Key examples include:
\begin{itemize}
    \item \textbf{Transportation and Logistics.} Industries such as shipping, trucking, and ridesharing; public services like trains and buses; physical infrastructure projects including roads, bridges, and tunnels.
    \item \textbf{Energy Infrastructure.} Raw material acquisition and refinement; large-scale energy generation in specialized facilities; storage and distribution networks required for load balancing.
    \item \textbf{Modern Agriculture.} Encompassing land management (irrigation, fertilization, pest control), crop and livestock cycles, as well as packaging and distribution systems to bring products to market.
    \item \textbf{Advanced Manufacturing.} The creation of parts from raw materials; international supply chain coordination; final assembly of complex products across sectors such as computing, biotechnology, and aerospace.
    \item \textbf{Digital Ecosystems.} Physical networking infrastructure; internet hosting servers; cloud computing stacks; software frameworks, libraries, and algorithms.
\end{itemize}

\textbf{System engineering} refers to the design, implementation, and management of such large-scale systems that integrate hardware, software, and human processes. Our understanding of systems has evolved significantly over time. Early industrial breakthroughs (e.g., standardized components, assembly lines, electrification) introduced new layers of complexity by increasing production volumes, lowering costs, and extending distribution chains. In aerospace and defense projects, where massive interdisciplinary teams had to be coordinated, formal \textit{system engineering} practices emerged \cite{blanchard2010, kossiakoff2011, incose2015, buede2016, madni2018}. Over the years, these yielded iterative and agile methods emphasizing continuous integration and rapid feedback—trends that are now mainstream in digital ecosystems. 

The demand for building and scaling complex systems shows no signs of slowing down. Across the world, large projects are planned or underway to address unprecedented challenges in the form of energy needs, aging demographics, changing climate patterns, geopolitical tensions, and more \cite{mckinsey2022, deloitte2023, reshoring2023}. The competitive advancement of technology itself leads to self-reinforcing demand for systems, exemplified by staggering investments into AI-related infrastructure \cite{uscongress2022, stargate2025}. Budget overruns and timeline delays are all too common in implementing large projects, showing that human planning and system engineering has its limitations. It thus appears nearly certain that advanced AI will be crucial in tackling these challenges and implementing the solutions.

\subsection{The Rise of AI Agents}
% openai levels. level 4 innovators, level 5 organization
% industry leaders see great opportunities for widespread usage and integration into complex systems.
% Greater intelligence, greater safety
% data scarcity is emerging as an issue.
% browser is a dominant env for deploying agents and evaluating them
% for embodied/physical agents, there are physics-based simulations and some real-world robotics
% as there are more agents, multi-agent coordination will matter more.

Simultaneously, AI agents are gaining traction as the most promising framework for applying human-aligned generative AI models. Significant advances in multimodal input processing and reasoning through inference-time compute use have exposed the possibilities of autonomous agents using complex interfaces to accomplish tasks over longer time horizons. These opportunities are rapidly being realized through virtual agents which are increasingly using web browsers, code interpreters and other digital tools to automate workflows and offload other labor from humans \cite{devin2024, openai2025operator}. Progress in physical agents is also picking up steam, leaning on advances in both general-purpose foundation models and maturing robotics technology \cite{nvidia2024gr00t, musk2024optimus}. 

The tasks assigned to AI agents today may be composites of several smaller sub-tasks, but ultimately tend to be self-contained workflows. As AI agents become more reliable at executing these tasks, it will become more enticing to involve them in more system-level challenges. Systems-level expertise is always more scarce since it demands deep knowledge of choices for components, interconnections and associated trade-offs for costs, implementation time, complexity, scalability, etc. Training data for reasoning about systems is also commensurately scarce and so it will naturally present a challenge for improving AI agents. As AI agents proliferate, the challenges and opportunities associated with multi-agent coordination will also become more relevant and influence the efficacy of AI-enhanced systems. Hence, this trajectory of elevating AI agents to effectively work on system engineering seems central for achieving the long-term goals of developing AGI. 

\subsection{Implications of Superhuman System Engineering}

AI models and agents have shown superhuman ability in various domains. Historically, this is evidenced by their mastery of classically challenging board games and more complex real-time strategy video games \cite{silver2017alphazero, vinyals2019alphastar}. In applied settings, AI models have outperformed human-engineered solutions in predicting molecular structures \cite{alphafold2021}, weather patterns \cite{gencast2024}, and even designing certain GPU circuits \cite{prefixrl2021}. Recent frontier models demonstrate advances in multimodal and highly technical reasoning, suggesting that a trajectory toward general superhuman intelligence is potentially close. Given all this, it is worth considering the ramifications of successfully building superhuman AI system engineers.


We can look to current examples of top-tier system engineering by humans. These achievements share a common pattern: they redefined what was previously considered possible. For instance, when Apple replaced Intel chips with its in-house M-series CPUs, the gains in thermal performance, battery life, and software speed were widely seen as a generational leap. Similarly, SpaceX reshaped the frontier of aerospace by reducing costs by orders of magnitude and inventing reusable rocket technology. Meanwhile, the unrelenting dominance of Nvidia’s entire data center stack—from hardware to deep learning libraries—has made it (at least temporarily) the most valuable company in the world. In each of these cases, the teams in charge took ownership of the entire system, jointly optimizing it over many iteration cycles to achieve superior metrics. It is worth stressing that the same depth of expertise required for proposing a comprehensive initial design is needed for continually refactoring a system to improve its scalability, maintainability, security, and fault tolerance as development progresses


Superhuman AI agent capability in system engineering gives us a much better chance of addressing civilizational challenges, such as scaling clean energy systems, securing reliable water and food supplies, and lowering the cost of economically important finished goods. In other words, superhuman-level system engineering is the key to producing utopian abundance (provided we solve the alignment problem for such superintelligence). Full automation of physical projects will also require general-purpose robotics, which may be a bottleneck in the near term. Yet as those technologies mature, AI agents could combine high-level system design with low-level mechanical tasks, delivering a fully optimized, end-to-end engineering capability for arbitrarily complex systems. 







\section{Empirical Evaluation}


\begin{figure*}[ht!]
    \centering
    \includegraphics[width=\textwidth]{img/DPO_Axioms_Impact.pdf}
    \caption{
        \textit{Impact of Training DPO with Individual Axioms on Others: A Comparative Evaluation.} 
        The plots illustrate the impact of training DPO to optimize a single axiom on the other alignment objectives. Each subplot corresponds to one axiom, with percentage changes in performance (relative to baseline) shown for all other objectives. For example, training on \emph{Artistic Freedom} improves it by 40\%, but causes notable declines in \emph{Cultural Sensitivity} (-30\%) and \emph{Verifiability} (-35\%), while improving \emph{Faithfulness to Prompt} (+22\%) and \emph{Originality} (+25\%). These results underscore the inherent trade-offs of single-axiom optimization and motivate the need for holistic alignment approaches like CAO.
    }
    \label{fig:dpo_axiom_impact}
\end{figure*}

\begin{figure*}[ht!]
\centering
\includegraphics[width=\textwidth]{img/DPO_vs_DPO_CAO_Impact.pdf}
\caption{Comparison of Alignment Impacts: The plot illustrates the effect of training with DPO versus CAO across six axioms: Artistic Freedom, Faithfulness to Prompt, Emotional Impact, Originality, Cultural Sensitivity, and Verifiability. While DPO exhibits uncontrolled variations in the impacts, leading to undesirable tradeoffs (e.g., +40\% Artistic Freedom but -30\% Cultural Sensitivity), CAO achieves a more balanced alignment with controlled tradeoffs (e.g., +10\% Artistic Freedom and +44\% Cultural Sensitivity). This demonstrates CAO's ability to harmonize competing axioms effectively.}
\label{fig:DPO_vs_DPO_CAO_Impact}
\end{figure*}


\textls[-10]{\textbf{Evaluation Setup and Insights}: Our evaluation examines the limitations of optimizing Directed Preference Optimization (DPO) models on individual alignment objectives. Specifically, we trained six models, each focusing on one axiom: \emph{Artistic Freedom}, \emph{Faithfulness to Prompt}, \emph{Emotional Impact}, \emph{Originality}, \emph{Cultural Sensitivity}, and \emph{Verifiability}. The impact of this single-axiom optimization on the other five objectives was measured in terms of percentage changes compared to a baseline.}

\subsection*{\textls[-10]{Impact of Training DPO with Individual Axioms}}

\begin{itemize}
    \item \textbf{Artistic Freedom}: Training for \emph{Artistic Freedom} resulted in a 40\% improvement, but at the expense of reduced \emph{Cultural Sensitivity} (-30\%) and \emph{Verifiability} (-35\%). \emph{Faithfulness to Prompt} and \emph{Originality} improved by 22\% and 25\%, respectively.
    \item \textbf{Faithfulness to Prompt}: Optimizing for \emph{Faithfulness to Prompt} led to a 40\% improvement but reduced \emph{Artistic Freedom} (-10\%) while marginally improving \emph{Originality} (+10\%) and \emph{Emotional Impact} (+5\%).
    \item \textbf{Emotional Impact}: Training on \emph{Emotional Impact} increased it by 40\%, but resulted in a 20\% decline in \emph{Faithfulness to Prompt} and a 10\% decline in \emph{Cultural Sensitivity}. \emph{Artistic Freedom} increased slightly (+15\%).
    \item \textbf{Originality}: Prioritizing \emph{Originality} improved it by 40\%, but reduced \emph{Cultural Sensitivity} (-25\%) and \emph{Verifiability} (-15\%).
    \item \textbf{Cultural Sensitivity}: Optimizing \emph{Cultural Sensitivity} led to a 40\% improvement, but reduced \emph{Verifiability} (-30\%) and \emph{Originality} (-20\%). \emph{Artistic Freedom} dropped by 15\%.
    \item \textbf{Verifiability}: Training for \emph{Verifiability} resulted in a 40\% improvement but came at the expense of \emph{Originality} (-25\%) and \emph{Cultural Sensitivity} (-30\%). \emph{Faithfulness to Prompt} and \emph{Emotional Impact} saw minor declines of 10\% and 15\%.
\end{itemize}




\textbf{Key Insights:} 
Empirical findings elucidate the inherent limitations of single-axiom DPO training, where optimization bias disrupts inter-axiom equilibria, thereby affirming the necessity of multi-objective strategies such as CAO for holistic alignment. This motivates the need for our proposed CAO, which harmonizes trade-offs across all alignment objectives.

For a detailed discussion of the optimization landscape differences between DPO and CAO, including comparative visualizations of error surfaces, refer to \cref{sec:appendix_error_surface_analysis}. The computational complexity and overhead introduced by the CAO framework, along with strategies to mitigate these challenges, are elaborated in \cref{sec:appendix_complexity_analysis}. Additionally, future avenues for reducing the computational burden of global synergy terms are explored in \cref{sec:appendix_synergy_overhead_reduction}. For an overview of the key hyperparameters, optimization strategies, and architectural configurations used in this work, see \cref{sec:appendix:hyperparams}.








\section{Generalization vs. Overfitting: Effect of Alignment}
\label{sec:HTSR_generalization}

The \textit{Weighted Alpha} metric \cite{martin2021predicting} offers a novel way to assess generalization and overfitting in LLMs without requiring training or test data. Rooted in Heavy-Tailed Self-Regularization (HT-SR) theory, it analyzes the eigenvalue distribution of weight matrices, modeling the Empirical Spectral Density (ESD) as a power-law \(\rho(\lambda) \propto \lambda^{-\alpha}\). Smaller \(\alpha\) values indicate stronger self-regularization and better generalization, while larger \(\alpha\) values signal overfitting. The Weighted Alpha \(\hat{\alpha}\) is computed as:
$\hat{\alpha} = \frac{1}{L} \sum_{l=1}^L \alpha_l \log \lambda_{\max,l}$,
where \(\alpha_l\) and \(\lambda_{\max,l}\) are the power-law exponent and largest eigenvalue of the \(l\)-th layer, respectively. This formulation highlights layers with larger eigenvalues, providing a practical metric to diagnose generalization and overfitting tendencies. Results reported in \cref{fig:htsr_generalization_main}.



\subsubsection*{Research Questions and Key Insights}
\begin{enumerate}
    \item \textbf{\ul{RQ1}: Do aligned T2I models lose generalizability and become overfitted?}  
    Alignment procedures introduce a marginal increase in overfitting, as evidenced by a generalization error drift of \(|\Delta \mathcal{E}_{\text{gen}}| \leq 0.1\), remaining within an acceptable range of \(\pm 10\%\).

    \item \textbf{\ul{RQ2}: Between DPO and CPO, which offers better generalizability?}  
    CAO is only marginally less generalized compared to DPO, demonstrating a minor increase in the generalization gap. However, CAO achieves superior alignment by addressing six complex and contradictory axioms, such as faithfulness, artistic freedom, and cultural sensitivity, which DPO alone cannot comprehensively balance. This trade-off between generalizability and alignment complexity highlights CAO's ability to maintain robust prompt adherence while handling nuanced alignment challenges effectively.
\end{enumerate}







\section{Factorio as a System Engineering Testbed}
We now argue that \textit{Factorio} is an ideal environment to develop system engineering capability in AI agents. We describe the mechanics, features and extensible scope of the game and put forth a call for using \textit{Factorio} as a platform for public research. Interested readers can find a more detailed walkthrough of the game in Appendix~\ref{ref:sec_appendix}

\subsection{Overview}
\textit{Factorio} is a 2D, top-down factory-building game that centers on \textit{automation}, rendering it a uniquely rich environment for developing and evaluating AI agents with strong system engineering capabilities. Although it shares the open-sandbox approach of titles like \textit{Minecraft}, \textit{Factorio} is far better suited for this purpose as it emphasizes building systems with high throughput, efficiency, and resilience. Automating the production of goods—from early hand-assembled items to complex industrial chains—is not merely a side option but rather the heart of the gameplay. This emphasis on scaling and optimizing factories pushes agents to navigate challenges that mirror real-world engineering dilemmas: resource constraints, energy usage, logistical complexity, and even defensive measures against hostile forces.

A key metric of success is \textbf{science per minute} (SPM), a community-standard indicator of a factory’s overall efficiency in generating the science packs needed for technological progress. Because each successive tier of research unlocks new possibilities (e.g., improved assemblers, trains, robots) but also imposes heavier resource and energy demands, any small inefficiency can ripple into crippling bottlenecks. Consequently, an effective agent must maintain the appropriate degree of variety in its approach at all times, ensuring that its decision-making processes can handle the game’s growing complexity and unexpected fluctuations. SPM makes for a great summary benchmark metric, with human novice bases at $\sim$0-30 SPM, intermediate at $\sim$30-200 SPM and advanced bases at $\sim$200-1000+ SPM. 

From a VSM perspective, \textit{Factorio} initially starts players at purely System 1 activities like manually extracting coal and iron to be placed in a hand-crafted furnace. The use of automated conveyer belts with splitting and load-balancing mechanisms combined with automated inserter arms elevates design to a System 2 level. A key gameplay entity is the \textit{assembler} which can be programmed with a recipe to convert inputs to finished outputs using materials, power and space for operation. Scaling the automated production of intermediate goods setting up train cargo networks and selecting technology tree paths are all System 3, 4 and 5 functions (Table~\ref{table:vsm_table}). A pivotal late game technology is the use of \textit{automated construction robots} which can be used to rapidly bring and place materials in accordance with large, complex player-made blueprints. This capability hence focuses gameplay purely on systems-level control problems, choosing the right smelting column, railway depot, solar array configuration, etc. to evolve the base as needed.

This versatility is further magnified by \textit{Factorio}’s robust modding support, which allows researchers and the broader community to introduce new mechanics, custom APIs, or entire rebalanced rule sets. In other words, the sandbox nature of \textit{Factorio} can be extended indefinitely, enabling the environment itself to evolve and stress-test an agent’s capacity to adapt and manage variety. Such flexibility in scaling and customization makes \textit{Factorio} ideal for public research, as it encourages the development of AI agents that can grow beyond initial, narrowly-defined tasks and rise to dynamic challenges that demand integrated System 1–5 competencies.

\subsection{Challenges for Current AI Agents}

While AI agents have made remarkable progress in reasoning and multimodal interaction, there remains a sizable gap between the capabilities of frontier agents (e.g. \cite{deepmind2024mariner, openai2025operator}) and the level of sophistication needed to thrive in \textit{Factorio}—and, by extension, in complex real-world systems. For example, \textit{Factorio} uses traditional a keyboard-and-mouse interface with numerous GUI windows and features detailed real-time visualization—where every item, belt, or robot is tracked on screen from a 2D view. This is coupled with the ability to view monitoring for practically all processes, placing it at the cutting edge of current AI capabilities for handling multimodal data bandwidth and human interface use. 

Bases are commonly developed over several dozens if not hundreds of hours. There is a tremendous amount of temporal information involved in optimizing a base which would certainly test the long-context nature of frontier agents. Sophisticated memory and recall systems would undoubtedly be necessary for an LLM-based agent to succeed in an extended episode playing Factorio. Separately, planning for the future would certainly benefit from time spent reasoning, but this comes at a cost when acting in a real-time environment, hence aligning interplay of System 3 and 4 with a key compute usage trade-off.

Addressing these technical barriers also highlights the importance of multi-agent collaboration: large-scale systems often require multiple agents or human-agent teams working in sync. This necessitates coordination frameworks that facilitate shared state and efficient task delegation. Moreover, real-world complexities like supply-chain delays or hardware breakdowns call for robust decision-making under uncertainty—agents must act swiftly and safely, even with incomplete information. Nevertheless, scaling compute FLOPs and refining AI architectures are likely to improve input-output flow management to the point where agents can handle advanced simulations like \textit{Factorio} in real time (realistically the game only needs to be played at around 5 FPS), without relying on domain-specific observation and action spaces, as was common in earlier superhuman-agent research such as AlphaStar.

\subsection{Modding, Market Interactions, and the Agent-Evaluator Framework}

Factorio’s \textbf{modding ecosystem} is unusually flexible, allowing Lua scripts to fundamentally alter or extend nearly every facet of the simulation. At one end, small “Quality-of-Life” mods streamline actions like inventory management or blueprint deployment—an approach often mirrored in real-world industrial systems where specialized scripts automate repetitive tasks. At the other end, total conversion mods, such as \textit{Space Exploration} \cite{spaceExploration} or \textit{Industrial Revolution 3} \cite{industrialRevolution} introduce entirely new resources, tech trees, and production pipelines. This capacity for extensive re-parameterization means researchers can craft tailored scenarios focusing on, for example, large-scale chemical manufacturing or advanced energy grids. By doing so, Factorio can serve as a robust platform for evaluating AI agents under conditions that closely resemble real-world system engineering challenges.

An especially promising application of this modding framework involves designing \textbf{market pricing} and \textbf{multi-agent} interactions. Factorio already supports multiplayer, and community-created mods showcase how resource trading, diplomatic pacts and emergent economies can drive the game’s complexity \cite{blackMarket, diplomacy}. In a research context, introducing dynamic markets would allow agents to buy and sell resources, negotiate prices, and even form alliances or contracts—key elements of real-world logistics and supply chains. Observing how AI agents adapt to fluctuating market forces and coordinate with others could yield insights into cooperative and competitive strategies, as well as negotiation tactics and resilient system designs.

Beyond market dynamics, Factorio’s modding API also lends itself to the concept of a \textbf{Agent-Evaluator Framework}. In this paradigm, a “evaluator” agent (human or AI) orchestrates scenario constraints, random events, or objectives (Informing system 5 as per the VSM) while the “agent” attempts to build and maintain a functional factory. This setup is well-suited to self play-like reinforcement learning algorithms, where the evaluator can inject perturbations—ranging from supply shortages to power-grid failures—testing the agent’s capacity for adaptive, long-horizon decision-making. The evaluator could also coordinate multiple agents with distinct roles or goals, enabling both collaboration and competition. Such arrangements bring Factorio closer to real-world engineering environments, where teams of engineers and managers must not only design but also continually refactor systems in response to shifting requirements and unforeseen disruptions.

By blending flexible modding, multi-agent mechanics, and the Agent-Evaluator approach, Factorio becomes more than just a factory-building game. It becomes a powerful sandbox for studying how AI agents might operate in large-scale, ever-evolving ecosystems—spanning everything from supply-chain economics to self-directed adaptation and robust error handling. This versatility sets Factorio apart as a uniquely comprehensive testbed for advancing AI-driven system engineering. 

\subsection{Technical Advantages}
Beyond the near-limitless opportunities provided by mods, \textit{Factorio} offers a few key advantages that are worth highlighting. First, as a 2D game, it is far more resource efficient for the complexity of systems that can be built in it as it does not involve costly 3D graphics rendering as would be the case in other titles such as \textit{Satisfactory}. Even despite this major difference, \textit{Factorio} is well-known to be a very well-optimized game in terms of memory usage, as it has been continually refined by its dedicated team since its first public release in 2012. Furthermore, the game is platform-agnostic, running natively on Windows, Mac OS X and Linux, which is rare. It offers a free headless Linux server for supporting well-optimized multiplayer gameplay which would be crucial for human-AI and multi-AI agent experimentation. And as mentioned before, the game has exceptional support for modding, showcased by community mods which completely overhaul the tech tree, environmental mechanics and GUI systems. We believe it is quite feasible to build an API layer for control as an intermediate solution for AI usage similar to Mineflayer \cite{mineflayer2024} (used in the Voyager project \cite{wang2023voyager}). In fact, that could even be a task for an AI agent to perform as part of its introduction to the game.
\section{Alternative Views}

Some critics argue that advancing AI system engineering is premature, given that core capabilities—like consistent reasoning, robust multimodality, and factual grounding—remain underdeveloped. They believe AI should first address these foundational weaknesses before tackling higher-level tasks. Yet proactive, orthogonal research can reveal new performance bottlenecks and drive innovation across modalities. Much as multimodality has progressed alongside unresolved text-based issues, tackling system engineering now can highlight what crucial gaps persist, helping to shape more integrated AI architectures.

Another concern is the risk of entrusting critical infrastructures to automated agents. Misaligned objectives or flawed reasoning could theoretically sabotage energy grids, supply chains, or other vital systems. While these dangers merit attention, the potential benefits—greater efficiency, cost savings, and creative solutions—are substantial. Alignment sits at the core of system engineering, which is rooted in clear requirements, continuous feedback loops, and stakeholder validation. By maintaining transparency and accountability, AI-driven engineering can strike a balance between prudence and progress.

Skeptics may also doubt whether games like \textit{Factorio} adequately reflect real-world complexities, noting they often omit granular physical laws or regulatory constraints. Yet such “unrealistic” environments highlight the essence of system engineering—resource management, strategic planning, and iterative trade-offs in efficiency, adaptability, and cost—far better than static benchmarks and without the noise associated with realistic physics simulations. Skills developed in orchestrating large-scale virtual factories can be paired with domain-specific testing to produce a fuller assessment of AI’s strengths. This integrated approach shows where AI excels (e.g., in macro-level design) and where further refinement is needed before applying these insights to physical-world applications.
\section{Conclusion}
AI agents stand on the verge of a new era where they can systematically design, optimize, and maintain complex systems in ways that rival or surpass human expertise. While LLMs have already showcased impressive capabilities for text generation, the true promise lies in the \textit{agentic paradigm}—with integrated multimodal interfaces, memory, autonomy, and adaptive planning.

We have argued that system engineering represents a high-leverage domain for such agentic AI. Whether the task is orchestrating large-scale software infrastructures or managing logistical networks, adaptability and continuous learning map naturally onto the strengths of a well-trained AI agent. Yet, to properly develop and evaluate these systems, we must look beyond static benchmarks toward open-ended simulations that reflect real-time constraints, multi-agent collaboration, and shifting objectives.

In this regard, \textit{Factorio} emerges as a compelling platform, providing a safe yet rich environment for refining agentic capabilities. Its emphasis on real-time resource management, multi-objective optimization, and large-scale factory layouts makes it a microcosm of industrial-scale challenge. Success in \textit{Factorio} would signal that agents can handle real complexity, track multiple objectives, and adapt in realistic ways.

In conclusion, the evolution from LLM-based chatbots to versatile AI agent that can tackle system engineering marks a logical next step if we hope to solve the grand challenges of our era. By leveraging automation-oriented sandbox simulations like \textit{Factorio}, we can accelerate progress toward AI systems that orchestrate research, design, and operations at scale—fundamentally reshaping how societies function and flourish in the coming decades.


\bibliography{main}
\bibliographystyle{icml2025}
\section{Additional Algorithmic  Details: \textsc{FLAG-Trader} with PPO}\label{sec:Appendix_A}



In this section, we outline a detailed procedure for training the \textsc{FLAG-Trader} architecture via PPO, where the \textsc{Policy\_Net} (actor) and the \textsc{Value\_Net} (critic) share a subset of trainable parameters from a LLM, with
$\theta = \big( \theta_{\texttt{train}}, \theta_P, \theta_V\big)$.
We define $\theta_{policy} = \big( \theta_{\texttt{train}}, \theta_P)$ and $\theta_{value} = \big( \theta_{\texttt{train}}, \theta_V)$ for simplicity.

\textbf{Advantage Estimation.}
We use the Generalized Advantage Estimation (GAE) to compute the advantage function \( A_t \):
\begin{align}
    A_t \;=\; \sum_{k=0}^{T-1} (\gamma \lambda)^k \bigl[r_{t+k} + \gamma V_{\theta_{value}}(s_{t+k+1}) - V_{\theta_{value}}(s_{t+k})\bigr],
\end{align}
where \( \gamma \) is the discount factor, and \( \lambda \) is the GAE parameter.

\textbf{Probability Ratio.}
Let \(\theta_{policy, \mathrm{old}}\) denote the parameters before the current update. The PPO probability ratio is
\begin{align}
    r_t(\theta_{policy}) \;=\; \frac{\pi_{\theta_{policy}}(a_t \mid s_t)}{\pi_{\theta_{policy,\mathrm{old}}}(a_t \mid s_t)}.
\end{align}


\textbf{PPO Clipped Objective.}
PPO clips this ratio to prevent overly large updates. The surrogate objective is
\begin{align}
    \mathcal{L}_P(\theta_{policy}) \;=\; \mathbb{E}_t \Bigl[
  \min\bigl(r_t(\theta_{policy})\,A_t,\; \text{clip}\bigl(r_t(\theta_{policy}),\,1-\varepsilon,\,1+\varepsilon\bigr)\,A_t\bigr)
\Bigr],
\end{align}
where \(\varepsilon\) is a hyperparameter.

\textbf{Value Function Loss.}
The critic (value network) is updated by minimizing the difference between the predicted value \(V_{\theta_{value}}(s_t)\) and the target return \(R_t\). A common choice is:
\begin{align}
    \mathcal{L}_{V}(\theta_{value}) \;=\; \mathbb{E}_t\Bigl[(V_{\theta_{value}}(s_t) - R_t)^2\Bigr].
\end{align}


\textbf{Combined Loss.}
We often add an entropy term to encourage exploration, yielding the overall objective:
\begin{align}
    \mathcal{L}_{\text{total}}(\theta)
\;=\;
-\,\mathcal{L}_{P}(\theta_{policy})
\;+\;
c_1\,\mathcal{L}_{V}(\theta_{value})
\;-\;
c_2\,\mathcal{H}\bigl(\pi_{\theta_{policy}}\bigr),
\end{align}
where \(c_1\) and \(c_2\) are weighting coefficients, and \(\mathcal{H}(\pi_{\theta_{policy}})\) represents the policy entropy.

\textbf{Parameter Updates.}
At each iteration, we apply gradient descent on the total loss:
\begin{align}
\theta_P       &\leftarrow \theta_P       \;-\; \eta \;\nabla_{\theta_P}\,\mathcal{L}_P, \\
\theta_V       &\leftarrow \theta_V       \;-\; \eta \;\nabla_{\theta_V}\,\mathcal{L}_V, \\
\theta_{\texttt{train}} &\leftarrow \theta_{\texttt{train}} - \beta \;\nabla_{\theta_{\text{train}}}\,\mathcal{L}_{\text{total}},
\end{align}
where \(\eta\) and \(\beta\) are learning rates for the policy head, value head, and trainable LLM layers respectively. The algorithm is summerized in Algorithm \ref{alg:flagtrader-ppo}.

\begin{algorithm}[H]
\caption{FLAG-TRADER with PPO}
\label{alg:flagtrader-ppo}
\begin{algorithmic}[1]
\STATE \textbf{Input:} Pre-trained LLM parameters $(\theta_{\texttt{frozen}}, \theta_{\texttt{train}})$; actor parameters $\theta_P$; critic parameters $\theta_V$; environment $\mathcal{E}$; discount factor $\gamma$; GAE parameter $\lambda$; PPO clip $\varepsilon$; learning rates $\eta, \beta$;
\STATE Initialize $\theta_{\text{train}}, \theta_P, \theta_V$; let $\theta_{\mathrm{old}} \leftarrow \theta$
\STATE Initialize replay buffer $B \leftarrow \emptyset$
\FOR{iteration = 1 to \text{max\_iters}}
  \STATE // \textit{Collect Rollouts}
  \FOR{t = 1 to T}
    \STATE Fetch the current state $s_t$ from the environment and construct an input prompt \texttt{lang}($s_t$);
    \STATE Pass prompt \texttt{lang}($s_t$) through LLM;
    \STATE \textsc{Policy\_Net} outputs $a_t$ from action space $\{\texttt{``buy,'' ``sell,'' ``hold''}\}$ based on \eqref{eq:masking};
    \STATE Execute action $a_t$ in the environment and observe reward $r(s_t, a_t)$ and transition to new state $s_{t+1}$;
    \STATE Store experience tuple $(s_t, a_t, r_t, s_{t+1})$ in replay buffer $B$;
    
  \ENDFOR

  \STATE // \textit{Compute Advantage and Targets}
  \FOR{\textbf{each} transition in $B$}
    \STATE Compute $V_{\theta_{value}}(s_t)$ and advantage $A_t$ (e.g., via GAE)
  \ENDFOR

  \STATE // \textit{Perform PPO Updates}
  \FOR{update\_epoch = 1 to K}
    \STATE Sample mini-batch $\mathcal{M}$ from $B$
    \STATE Compute probability ratio $r_t(\theta_{policy}) \;=\; \frac{\pi_{\theta_{policy}}(a_t \mid s_t)}{\pi_{\theta_{policy,\mathrm{old}}}(a_t \mid s_t)}$;
    \STATE Compute PPO loss $\mathcal{L}_P(\theta_{policy}) \;=\; \mathbb{E}_t \Bigl[
  \min\bigl(r_t(\theta_{policy})\,A_t,\; \text{clip}\bigl(r_t(\theta_{policy}),\,1-\varepsilon,\,1+\varepsilon\bigr)\,A_t\bigr)
\Bigr]$;
    \STATE Compute Value loss $\mathcal{L}_{V}(\theta_{value}) \;=\; \mathbb{E}_t\Bigl[(V_{\theta_{value}}(s_t) - R_t)^2\Bigr]$;
    \STATE Compute total loss $\mathcal{L}_{\text{total}}(\theta)
\;=\;
-\,\mathcal{L}_{P}(\theta_{policy})
\;+\;
c_1\,\mathcal{L}_{V}(\theta_{value})
\;-\;
c_2\,\mathcal{H}\bigl(\pi_{\theta_{policy}}\bigr)$;
    \STATE Perform gradient descent on each parameter group:
    \begin{align*}
\theta_P       &\leftarrow \theta_P       \;-\; \eta \;\nabla_{\theta_P}\,\mathcal{L}_P, \\
\theta_V       &\leftarrow \theta_V       \;-\; \eta \;\nabla_{\theta_V}\,\mathcal{L}_V, \\
\theta_{\texttt{train}} &\leftarrow \theta_{\texttt{train}} - \beta \;\nabla_{\theta_{\text{train}}}\,\mathcal{L}_{\text{total}};
\end{align*}
  \ENDFOR

  \STATE // \textit{Update old policy parameters}
  \STATE Update $\theta = \big( \theta_{\texttt{train}}, \theta_P, \theta_V\big)$ by $\theta_{\mathrm{old}} \leftarrow \theta$;
\ENDFOR
\STATE \textbf{Return:} Fine-tuned \textsc{Policy\_Net}($\theta_P$).
\end{algorithmic}
\end{algorithm}




% \begin{algorithm}
% \caption{FinRL-LLM Pipeline with PPO}
% \label{alg:FinRL-LLM-PPO}
% \begin{algorithmic}[1]
% \STATE  \textbf{Input:} Pre-trained LLM policy $\pi_{\theta}$, environment dynamics, reward function $r(s_t, a_t)$, learning parameters (replay buffer size, batch size, PPO clipping parameter $\epsilon$, discount factor $\gamma$, GAE parameter $\lambda$, learning rate $\eta$)
% \STATE  \textbf{Output:} Fine-tuned LLM policy $\pi_{\theta}$

% \STATE  Initialize LLM policy $\pi_{\theta}$ with pre-trained weights
% \STATE  Initialize experience replay buffer $B \leftarrow \emptyset$
% \FOR{each training iteration}
%     \STATE  Fetch the current state $s_t$ from the environment
%     \STATE  Convert $s_t$ into structured text $\text{lang}(s_t)$
%     \STATE  Encode $\text{lang}(s_t)$ into token embeddings $E_t \leftarrow \text{Embed}(\text{lang}(s_t))$
%     \STATE  Pass through LLM:
%     \STATE  $h_t^{(1)} \leftarrow \text{LLM}_{\text{frozen}}(E_t)$
%     \STATE  $h_t^{(2)} \leftarrow \text{LLM}_{\text{trainable}}(h_t^{(1)})$
%     \STATE  Compute action probability distribution:
%     \STATE  $\text{logits}_t \leftarrow \text{POLICY\_NET}(h_t^{(2)})$
%     \STATE  $\pi_{\theta}(a_t | s_t) \leftarrow \text{Softmax}(\text{logits}_t)$
%     \STATE  Sample action $a_t \sim \pi_{\theta}(a_t | s_t)$
%     \STATE  Execute $a_t$ in the environment and observe reward $r_t$ and transition to $s_{t+1}$
%     \STATE  Compute state value $V(s_t) \leftarrow \text{VALUE\_NET}(h_t^{(2)})$
%     \STATE  Store experience $(s_t, a_t, r_t, s_{t+1}, V(s_t), \pi_{\theta}(a_t | s_t))$ in replay buffer $B$

%     \IF{replay buffer $B$ is full}
%         \STATE  Sample a mini-batch of experiences $\{(s_i, a_i, r_i, s_{i+1}, V(s_i), \pi_{\theta_{\text{old}}}(a_i | s_i))\}$ from $B$
%         \STATE  Compute advantage estimation using GAE:
%         \STATE  $A_i = \sum_{l=0}^{T-t} (\gamma \lambda)^l (r_i + \gamma V(s_{i+1}) - V(s_i))$
%         \STATE  Compute log-likelihood ratio:
%         \STATE  $r_i(\theta) = \frac{\pi_{\theta}(a_i | s_i)}{\pi_{\theta_{\text{old}}}(a_i | s_i)}$
%         \STATE  Compute PPO clipped objective:
%         \STATE  $L_{\text{PPO}}(\theta) = \mathbb{E}_t [\min( r_i(\theta) A_i, \text{clip}(r_i(\theta), 1 - \epsilon, 1 + \epsilon) A_i )]$
%         \STATE  Compute value loss:
%         \STATE  $L_V = (V(s_i) - V_{\text{target}}(s_i))^2, \quad V_{\text{target}}(s_i) = r_i + \gamma V(s_{i+1})$
%         \STATE  Compute entropy bonus:
%         \STATE  $S(\pi_{\theta}) = - \sum_a \pi_{\theta}(a | s) \log \pi_{\theta}(a | s)$
%         \STATE  Compute total loss:
%         \STATE  $L_{\text{total}} = -L_{\text{PPO}} + c_1 L_V - c_2 S(\pi_{\theta})$
%         \STATE  Perform gradient descent on $\theta$ to minimize $L_{\text{total}}$
%         \STATE  Update $\pi_{\theta}$, VALUE\_NET, and trainable LLM layers
%         \STATE  Clear replay buffer $B$
%     \ENDIF
% \ENDFOR

% \STATE  \textbf{Return:} Fine-tuned LLM policy $\pi_{\theta}$
% \end{algorithmic}
% \end{algorithm}

% The online PPO module is responsible for training the LLM to improve its policy over time. The learning process involves the following steps:
% \begin{enumerate}
%     \item \textbf{Log-Likelihood Computation}: The LLM generates an action \(a_t\) conditioned on the state \(s_t\) and the prompt. The log-likelihood \(\log \pi_\theta(a_t | s_t, \text{prompt})\) is computed to measure the policy's performance.
%     \item \textbf{Environment Interaction}: The action \(a_t\) is executed in the environment, which transitions to a new state \(s_{t+1}\) and returns a reward \(r(s_t, a_t)\).
%     \item \textbf{Experience Replay}: The observed experience tuple \((s_t, a_t, r_t, s_{t+1})\) is stored in the replay buffer for batch updates.
%     \item \textbf{PPO Objective}: The policy is updated using the PPO objective:
%     \[
%     L_{\text{PPO}}(\theta) = \mathbb{E}_t \left[ \min \left( r_t(\theta) A_t, \text{clip}(r_t(\theta), 1-\epsilon, 1+\epsilon) A_t \right) \right],
%     \]
%     where \(r_t(\theta)\) is the probability ratio, \(A_t\) is the advantage function, and \(\epsilon\) is a hyperparameter controlling the update clipping.
% \end{enumerate}

\section{Additional Experimental Details}


% \subsection{Details on Evaluation Metrics}
% Below is a brief overview of these metrics:\par
% \noindent\textbf{Cumulative Return (CR) \%} measures the total value change of an investment over time by summing daily logarithmic returns, shown in \eqref{eq:cum_return}: 
% \begin{align}
%    \label{eq:cum_return}
%    \textbf{CR} &= \sum_{t=1}^{n} r_i = \sum_{t=1}^{n} \left[ \ln\left(\frac{p_{t+1}}{p_t}\right) \cdot \text{action}_t \right],
% \end{align}
% where $r_i$ is the logarithmic return from day $t$ to $t+1$, $p_t$ and $p_{t+1}$ are the closing prices on days $t$ and $t+1$, respectively, and $\text{action}_t$ is the model's trading decision for day $t$.
% Notice that higher values indicate better strategy effectiveness.

% \noindent\textbf{Sharpe Ratio (SR)} assesses risk-adjusted returns by dividing the average excess return ($R_p$) over the risk-free rate ($R_f$) by its volatility ($\sigma_p$), detailed in \eqref{eq:sharpe}: 
% \begin{equation}
%     \textbf{SR} = \frac{R_p - R_f}{\sigma_p}.
%     \label{eq:sharpe}
% \end{equation}  
% Notice that higher ratios signify better performance.
  
% \noindent\textbf{Annualized Volatility (AV) \% and Daily Volatility (DV) \%} quantify return fluctuations; AV is derived by scaling DV (\textit{standard deviation of daily logarithmic returns}) by the square root of the annual trading days (252), as in \eqref{eq:annuaVol}. This metric highlights potential return deviations across the year.
% \begin{align}
%    \label{eq:annuaVol}
%     \textbf{AV} &= \textbf{DV} \times \sqrt{252}. 
% \end{align} 

% \noindent\textbf{Max Drawdown (MDD) \%} calculates the largest portfolio value drop from peak to trough, as given in \eqref{eq:maxdrawdown}. Lower values indicate lesser risk and higher strategy robustness. 
%     \begin{align}
%     \label{eq:maxdrawdown}
%     \textbf{MDD} = \text{max}(\frac{P_{\text{peak}} - P_{\text{trough}}}{P_{\text{peak}}}).
%     \end{align}



\subsection*{Hyperparameters for Finetuening \textsc{FLAG-Trader} with PPO in Algorithm \ref{alg:flagtrader-ppo}}
\begin{table}[!ht]
\centering
\caption{\textsc{FLAG-Trader} with PPO Finetuning Hyperparameters and Settings.}
\label{tab:parameters:lora}
\scalebox{0.8}{
\begin{tabular}{lll}
\toprule
\textbf{Parameter} & \textbf{Default Value} & \textbf{Description} \\
\toprule
\texttt{total\_timesteps} & 13860 & Total number of timesteps \\
\texttt{learning\_rate} & \(5 \times 10^{-4}\) & Learning rate of optimizer \\
\texttt{num\_envs} & 1 & Number of parallel environments \\
\texttt{num\_steps} & 40 & Steps per policy rollout \\
\texttt{anneal\_lr} & True & Enable learning rate annealing \\
\texttt{gamma} & 0.95 & Discount factor \(\gamma\) \\
\texttt{gae\_lambda} & 0.98 & Lambda for Generalized Advantage Estimation \\
\texttt{update\_epochs} & 1 & Number of update epochs per cycle \\
\texttt{norm\_adv} & True & Advantages whitening \\
\texttt{clip\_coef} & 0.2 & Surrogate clipping coefficient \\
\texttt{clip\_vloss} & True & Clipped loss for value function \\
\texttt{ent\_coef} & 0.05 & Coefficient of entropy term \\
\texttt{vf\_coef} & 0.5 & Coefficient of value function \\
\texttt{kl\_coef} & 0.05 & KL divergence with reference model \\
\texttt{max\_grad\_norm} & 0.5 & Maximum gradient clipping norm \\
\texttt{target\_kl} & None & Target KL divergence threshold \\
\texttt{dropout} & 0.0 & Dropout rate \\
\texttt{llm} & "SmolLM2-135M-Instruct" & Model to fine-tune \\
\texttt{train\_dtype} & "float16" & Training data type \\
\texttt{gradient\_accumulation\_steps} & 8 & Number of gradient accumulation steps \\
\texttt{minibatch\_size} & 32 & Mini-batch size for fine-tuning \\
\texttt{max\_episode\_steps} & 65 & Maximum number of steps per episode \\
\bottomrule
\end{tabular}
}
\end{table}



\end{document}


\textbf{Keywords}: 
