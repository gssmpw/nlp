\section{The High-Leverage Potential of Systems Engineering}
Systems engineering focuses on the design, implementation, and management of large-scale systems that integrate hardware, software, and human processes. As many grand challenges—from clean energy to advanced manufacturing—hinge on orchestrating numerous interdependent subsystems, systems engineering represents a powerful \textit{leverage point} for AI \cite{buede2016}.

Our understanding of systems has evolved significantly over time. Early industrial breakthroughs (e.g., standardized components, assembly lines, electrification) introduced new layers of complexity by boosting production volumes, reducing costs, and extending distribution chains. In aerospace and defense projects, where massive interdisciplinary teams had to be coordinated, formal systems-engineering practices took shape \cite{madni2018}. Over the years, the rigid “waterfall” model yielded to iterative and agile methods that emphasize continuous integration and rapid feedback—trends that have become mainstream in digital ecosystems. As AI agents emerge, we anticipate yet another major shift in how systems are built and managed.

\subsection{Software System Engineering}
Software engineering epitomizes the progression from narrow, component-level tasks to broader, end-to-end management of complex systems. A new engineer typically begins by learning programming fundamentals and version control on small-scale projects. Over time, they tackle concurrency challenges and diverse integrations, eventually moving into principal engineering roles responsible for overseeing large-scale deployments, holistic performance optimizations, and reliability goals. Key practices include mastering design patterns \cite{gamma1994design, pressman2010}, routine refactoring \cite{humphrey1989, fowler1999refactoring}, and overall architecture \cite{shaw1996software, bass2012architecture}.

Translating these capabilities to AI involves building agents that maintain situational awareness across the entire software lifecycle—from initial planning to ongoing monitoring \cite{blanchard2010}. Moreover, it demands agents capable of reasoned decision-making around design trade-offs, providing solutions that align with stakeholder objectives regarding performance, cost, and reliability.

Even moderate advancements in these areas can have large-scale implications for infrastructure at both national and global levels. Yet evaluating AI’s effectiveness in system-level tasks remains challenging: requirements shift without warning, performance metrics may need continuous monitoring, and cost constraints often intersect with user satisfaction. 

Although “ML for Systems” research has achieved specialized optimizations (e.g., compiler strategies), a truly comprehensive AI agent that writes complex code, orchestrates microservices, schedules updates, and minimizes disruptions is still aspirational. Its eventual realization underscores the value of a systems-engineering mindset in developing the next generation of AI agents.

\subsection{Beyond Software: A Larger Systems View}
Systems engineering extends well beyond digital domains \cite{blanchard2010} and underpins countless aspects of modern life. The technical depth and negotiation of trade-offs in software engineering are matched—if not exceeded—by examples from:
\begin{itemize}
    \item \textbf{Transportation and Logistics.} Industries like shipping, trucking, and ridesharing; public services like trains and buses; and physical infrastructure projects such as roads, bridges, and tunnels.
    \item \textbf{Energy Infrastructure.} Spanning raw material acquisition and refinement, large-scale energy generation in specialized facilities, and the storage and distribution networks required to balance loads.
    \item \textbf{Modern Agriculture.} Covering land management (irrigation, fertilization, pest control), crop and livestock cycles, and packaging and distribution systems to deliver products to market.
    \item \textbf{Advanced Manufacturing.} Involving the creation of parts from raw materials, international supply chain coordination, and final assembly of complex products in areas like computing, biotechnology, and aerospace.
\end{itemize}

Despite their specific domains, these industries share core attributes that illustrate why systems engineering is both a universal discipline and a transferable skill set. Each involves numerous interacting components, trade-offs among cost, time, and complexity, and an undeniable need for continuous \textit{adaptation} over both short and long timescales. The importance of adaptation, in particular, is examined further in the next section.

As AI progresses, it is imperative to explore its broadest beneficial applications—particularly in physical, real-world systems, which have often been overlooked relative to software. Moravec’s Paradox notes that certain cognitive tasks are easier to automate than many physical ones. This paper does not assume an immediate shift in research priorities for embodied versus virtual agent development. Rather, it emphasizes that all systems involve both low-level and high-level engineering needs. In software, these levels revolve around text-based artifacts, enabling AI-driven engineering to operate across the entire software stack as it currently exists.

By contrast, real-world manufacturing and process-based systems often feature intricate physical components whose lower-level automation remains constrained by today’s general-purpose robotics. Consequently, AI agents in these areas may, for now, be limited to high-level roles such as system architecture and design. Still, this high-level expertise is vital to specifying dynamic arrangements of components under varying constraints and trade-offs. As robotics matures, we can anticipate AI agents that excel in both high-level system design and low-level mechanical tasks, delivering a fully optimized, end-to-end engineering capability for arbitrarily complex systems.

%The unifying aspects of systems engineering have led to standards like ISO/IEC/IEEE 15288 \cite{iso15288_2015} and the INCOSE Systems Engineering Handbook \cite{incose2015} which codify best practices for the design and maintenance of these complex environments. Model-based systems engineering (MBSE) \cite{estefan2008, weilkiens2012} further streamlines collaboration by consolidating requirements, architectures, and analytical models into a unified knowledge base (e.g., SysML \cite{friedenthal2014}). This holistic representation paves the way for AI agents to act as “co-engineers,” proposing design changes, running simulations, and updating documentation in near-real time. With the right kind of training, AI agents could potentially take moonshot projects like ``build a fusion reactor" and manage the end-to-end design, creation and maintenance of the resulting systems.
