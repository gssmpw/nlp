\section{Factorio as a Customizable Systems Engineering Testbed}

\subsection{Overview}
\textit{Factorio} is a 2D, top-down factory-building game that challenges players (or agents) to balance resource extraction, logistics, automation, and defense. Its emphasis on large-scale production resonates with real-world engineering, where efficiency gains often trade off against adaptability. Because Factorio already models supply flows and iterative design improvements, it naturally aligns with the needs of AI research aimed at complex system orchestration.

\subsection{Science per Minute as a Key Benchmark}
A well-known community metric, \textbf{science per minute} (SPM), serves as a succinct indicator of how efficiently a factory generates the science packs that fuel technological progress. Factories must coordinate mining, refining, assembling, and logistical pathways to keep SPM high. As expansions and new tech nodes unlock, maintaining or increasing SPM inevitably reveals bottlenecks in raw materials, transport routes, or defensive measures. By tracking SPM, researchers can distill an agent’s overall effectiveness in scaling and adapting the factory.

\subsection{Modding Ecosystem for Tailored Scenarios}
\textit{Factorio} supports extensive modding through Lua scripts and a well-documented API, enabling customized recipes, resource types, and assembly times. Researchers can craft entire tech trees to match specific industrial or exploratory visions, from advanced electronics to hypothetical energy sources. These changes can be as simple as tweaking production times or as radical as introducing wholly new supply chains. Such flexibility makes it possible to simulate highly specialized domains, testing how well AI agents handle novel constraints and orchestrate unique workflows.

\subsection{Market Pricing and Multi-Agent Interactions}
Beyond altering core recipes, mods can introduce pricing mechanisms or trading systems that inject economic elements into gameplay. Multiple agents—human or AI—might barter resources, form alliances, or compete for scarce deposits in a shared world. This transforms \textit{Factorio} from a single-player puzzle into a multi-agent environment, where negotiation and diplomacy become as important as throughput optimization. It also showcases how new in-game disruptions (like sudden price spikes or embargoes) require agents to adapt quickly, echoing the unpredictability of real markets.

\subsection{Proctor-Agent Framework}
One especially promising approach is to have a \textit{proctor} agent (or a human) set objectives and constraints, with a \textit{test-taking} agent striving to meet them. The proctor might demand a certain SPM level while limiting pollution, or require the factory to handle intermittent “resource sink” events from enemy “Biters.” Such scenarios reward proactive adaptation, as agents must balance long-term scaling with short-term emergency responses. Feedback loops from the proctor—much like regulatory oversight in real industries—ensure accountability and highlight suboptimal design choices.

\subsection{Training and Evaluation Strategies}
AI researchers have considerable latitude in deciding how to train agents. \textbf{Reinforcement learning} can leverage multiple episodes, starting from scratch each time or carrying over partial progress. Alternatively, a \textbf{multimodal} AI system might interact with the standard \textit{Factorio} interface via clicks and keystrokes, enabling near-human play. Notably, the game can run at reduced frame rates (e.g., 5~FPS), making it feasible to keep real-time feedback loops without excessive computational overhead. Meanwhile, minimal interface mods can simplify data retrieval or action dispatch if pixel-based perception proves too cumbersome.

\subsection{Implications for System Engineering Research}
Altogether, \textit{Factorio} can be molded into a dynamic sandbox that emphasizes both the high-level planning and the day-to-day adaptations characteristic of real engineering. By blending automation with evolving environmental challenges, it tests an agent’s skill in steering large-scale industrial operations while remaining flexible under shifting constraints. The ease of modding, support for multi-agent play, and detailed performance metrics such as SPM all converge to create an environment uniquely suited for studying how AI might design, automate, and sustain intricate systems. 