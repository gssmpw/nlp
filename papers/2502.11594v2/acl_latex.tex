% This must be in the first 5 lines to tell arXiv to use pdfLaTeX, which is strongly recommended.
\pdfoutput=1
% In particular, the hyperref package requires pdfLaTeX in order to break URLs across lines.
\documentclass[11pt]{article}
\usepackage{multirow}
\usepackage{multicol}

% Change "review" to "final" to generate the final (sometimes called camera-ready) version.
% Change to "preprint" to generate a non-anonymous version with page numbers.
% \usepackage[review]{acl}
\usepackage[preprint]{acl}
\usepackage{amsmath}  % 提供数学环境
\usepackage{amssymb}  % 提供数学符号
% Standard package includes
\usepackage{times}
\usepackage{latexsym}
\usepackage[many]{tcolorbox}
% For proper rendering and hyphenation of words containing Latin characters (including in bib files)
\usepackage[T1]{fontenc}
% For Vietnamese characters
% \usepackage[T5]{fontenc}
% See https://www.latex-project.org/help/documentation/encguide.pdf for other character sets

% This assumes your files are encoded as UTF8
\usepackage[utf8]{inputenc}

% This is not strictly necessary, and may be commented out,
% but it will improve the layout of the manuscript,
% and will typically save some space.
\usepackage{microtype}
\usepackage{subfigure}
\usepackage{booktabs}
% This is also not strictly necessary, and may be commented out.
% However, it will improve the aesthetics of text in
% the typewriter font.
\usepackage{inconsolata}

%Including images in your LaTeX document requires adding
%additional package(s)
\usepackage{graphicx}

\usepackage{xcolor}
\usepackage{colortbl}
\usepackage{makecell}
\definecolor{ours-highlight}{rgb}{0.86, 0.82, 1.0}
\definecolor{grounding-data-highlight}{HTML}{ffcfdf}
\definecolor{imoveit-data-highlight}{HTML}{e0f9b5}
\definecolor{general-data-highlight}{HTML}{a5dee5}
\newcommand{\texthighlight}[1]{\textcolor{blue}{#1}}
\usepackage{tcolorbox}
% If the title and author information does not fit in the area allocated, uncomment the following
%
%\setlength\titlebox{<dim>}
%
% and set <dim> to something 5cm or larger.
\usepackage{pifont}
\newcommand{\cmark}{\ding{51}}
\newcommand{\xmark}{\ding{55}}
\newcommand{\lgcmark}{\textcolor{green}{\cmark}}
\newcommand{\lgxmark}{\textcolor{red}{\xmark}}
\usepackage{subcaption}
\usepackage{multirow}
\usepackage{afterpage}
\newcommand{\modelname}{iMOVE }
\newcommand{\datasetname}{iMOVE-IT }

\title{\raisebox{-0.01\textheight}{\includegraphics[width=0.05\textwidth]{figures/logo-2.png}} \modelname: Instance-Motion-Aware Video Understanding}

% Author information can be set in various styles:
% For several authors from the same institution:
% \author{Author 1 \and ... \and Author n \\
%         Address line \\ ... \\ Address line}
% if the names do not fit well on one line use
%         Author 1 \\ {\bf Author 2} \\ ... \\ {\bf Author n} \\
% For authors from different institutions:
% \author{Author 1 \\ Address line \\  ... \\ Address line
%         \And  ... \And
%         Author n \\ Address line \\ ... \\ Address line}
% To start a separate ``row'' of authors use \AND, as in
% \author{Author 1 \\ Address line \\  ... \\ Address line
%         \AND
%         Author 2 \\ Address line \\ ... \\ Address line \And
%         Author 3 \\ Address line \\ ... \\ Address line}

\author{Jiaze Li\textsuperscript{1,2}, Yaya Shi\textsuperscript{1}\thanks{Corresponding author}, Zongyang Ma\textsuperscript{3}, Haoran Xu\textsuperscript{2}, Feng Cheng\textsuperscript{1} \\ \textbf{Huihui Xiao\textsuperscript{1}, Ruiwen Kang\textsuperscript{1}, Fan Yang\textsuperscript{1}, Tingting Gao\textsuperscript{1}, Di Zhang\textsuperscript{1}}  \\
  \textsuperscript{1}Kuaishou Technology  
      \textsuperscript{2}Zhejiang University \\
  \textsuperscript{3}Institute of Automation, Chinese Academy of Sciences \\
  % \texttt{Jiaze\_Li@zju.edu.cn} %\\\And
  % Second Author \\
  % Affiliation / Address line 1 \\
  % Affiliation / Address line 2 \\
  % Affiliation / Address line 3 \\
  % \texttt{Jiaze_Li@zju.edu.cn} \\
  }

%\author{
%  \textbf{First Author\textsuperscript{1}},
%  \textbf{Second Author\textsuperscript{1,2}},
%  \textbf{Third T. Author\textsuperscript{1}},
%  \textbf{Fourth Author\textsuperscript{1}},
%\\
%  \textbf{Fifth Author\textsuperscript{1,2}},
%  \textbf{Sixth Author\textsuperscript{1}},
%  \textbf{Seventh Author\textsuperscript{1}},
%  \textbf{Eighth Author \textsuperscript{1,2,3,4}},
%\\
%  \textbf{Ninth Author\textsuperscript{1}},
%  \textbf{Tenth Author\textsuperscript{1}},
%  \textbf{Eleventh E. Author\textsuperscript{1,2,3,4,5}},
%  \textbf{Twelfth Author\textsuperscript{1}},
%\\
%  \textbf{Thirteenth Author\textsuperscript{3}},
%  \textbf{Fourteenth F. Author\textsuperscript{2,4}},
%  \textbf{Fifteenth Author\textsuperscript{1}},
%  \textbf{Sixteenth Author\textsuperscript{1}},
%\\
%  \textbf{Seventeenth S. Author\textsuperscript{4,5}},
%  \textbf{Eighteenth Author\textsuperscript{3,4}},
%  \textbf{Nineteenth N. Author\textsuperscript{2,5}},
%  \textbf{Twentieth Author\textsuperscript{1}}
%\\
%\\
%  \textsuperscript{1}Affiliation 1,
%  \textsuperscript{2}Affiliation 2,
%  \textsuperscript{3}Affiliation 3,
%  \textsuperscript{4}Affiliation 4,
%  \textsuperscript{5}Affiliation 5
%\\
%  \small{
%    \textbf{Correspondence:} \href{mailto:email@domain}{email@domain}
%  }
%}

\begin{document}
\maketitle
\begin{abstract}
% Understanding fine-grained instance spatiotemporal motions is crucial for video understanding. However, current Video Large Language Models struggles to perceive detailed and complex instance motions.  To address these challenges, we make improvements from both data and model perspectives. In terms of data, we carefully curate \textbf{iMOVE-IT},  the first large-scale \textbf{i}nstance-\textbf{MO}tion-aware \textbf{V}id\textbf{E}o \textbf{I}nstruction-\textbf{T}uning dataset, which features rich instance motion annotations and spatiotemporal mutual-supervision tasks, providing comprehensive training for the model's instance-motion-awareness.
% Building upon this, we introduce \textbf{iMOVE}, an \textbf{i}nstance-\textbf{MO}tion-aware \textbf{V}id\textbf{E}o foundation model that employs Event-aware Spatiotemporal Efficient Modeling to preserve informative instance spatiotemporal motion details, and incorporates Relative Spatiotemporal Position Tokens to ensure the awareness of instance spatiotemporal positions. Evaluations show iMOVE's advantages on video temporal understanding and general video understanding tasks. In addition, iMOVE also performs exceptionally well in long video understanding.
Enhancing the fine-grained instance spatiotemporal motion perception capabilities of Video Large Language Models is crucial for improving their temporal and general video understanding. However, current models struggle to perceive detailed and complex instance motions. To address these challenges, we have made improvements from both data and model perspectives. In terms of data, we have meticulously curated \textbf{iMOVE-IT}, the first large-scale instance-motion-aware video instruction-tuning dataset. This dataset is enriched with comprehensive instance motion annotations and spatiotemporal mutual-supervision tasks, providing extensive training for the model's instance-motion-awareness. Building on this foundation, we introduce \textbf{iMOVE}, an instance-motion-aware video foundation model that utilizes Event-aware Spatiotemporal Efficient Modeling to retain informative instance spatiotemporal motion details while maintaining computational efficiency. It also incorporates Relative Spatiotemporal Position Tokens to ensure awareness of instance spatiotemporal positions. Evaluations indicate that iMOVE excels not only in video temporal understanding and general video understanding but also demonstrates significant advantages in long-term video understanding. 
%We will release the data, code, and model weights after acceptance.

\end{abstract}

\section{Introduction}


% 面向视频理解的多模态大模型,即video LLM,最近取得了快速进展,对视频整体语义的捕获和理解能力取得了全面提升。
% 在此基础上,一些工作开始探索如何让模型理解涉及时间信息的细节视频语义。
% 然而,当前video LLMs的时序理解能力还存在明显的提升空间,尤其是难以准确地把握视频中的物体级时空运动,如图xxx中展示的那样,the xxx ...。
% yaya: 
% 面向视频理解的多模态大模型,即video LLM,最近取得了快速进展,对视频整体语义的捕获取得了全面提升,在通用视频理解任务上获得了良好的性能。
% 在此基础上,一些工作开始探索如何让模型理解涉及时间信息的细节视频语义,并在时序理解任务上展现了进步。
% 然而,当前video LLMs难以准确地把握视频中的物体级时空运动,使其在通用视频理解和时序理解任务上还存在明显的提升空间如图xxx中展示的那样,the xxx ...。
% 存在的问题:video LLMs难以准确地把握视频中的物体级时空运动
% 造成的结果:其在通用视频理解任务和时序理解任务上还存在明显的提升空间
Recent advancements in Video Large Language Models~\citep{video-llama, video-chatgpt, videochat, videogpt+, zhang2024llava-video}, i.e., Video-LLMs, have led to rapid development, significantly enhancing the capture of overall video semantics and achieving remarkable performance in general video understanding tasks~\citep{videomme, mvbench}. Furthermore, some works~\citep{timechat, vtimellm, lita, wang2024groundedvideollm} have begun to explore methods to enable models to comprehend detailed video semantics involving temporal information, showing progress in video temporal understanding tasks~\citep{charades-sta,activitynet}. However, current Video-LLMs struggle to accurately perceive instance spatiotemporal motions within videos, which constrains their performance in these tasks. As demonstrated in Figure \ref{fig:intro}, VideoLLAMA2~\citep{video-llama} fails to identify the dog's spatiotemporal motions.
% However, the temporal understanding ability of current Video-LLMs still have considerable room for improvement, particularly in accurately perceiving object-level spatiotemporal motion within videos, as demonstrated in Figure XXX, where the xxx ....
% However, the lack of high-quality video-text alignment data for fine-grained object spatiotemporal motion and the absence of lossless video spatiotemporal encoding strategies leave considerable room for improvement in the temporal understanding of current Video-LLMs.
% However, current Video-LLMs still exhibit weak temporal understanding ability, with notable potential for improvement in comprehending video details involving temporal information, as demonstrated in Figure XX.

\begin{figure}
    \centering
    \includegraphics[width=0.9\linewidth]{figures/icml2025_intro_compress.pdf}
    \caption{
    % Current Video-LLMs struggle to accurately perceive instance-level spatiotemporal motions within videos.
     {\modelname excels at perceiving instance-level spatiotemporal motions within videos, outperforming the previous Video-LLM model, VideoLLAMA2.}
    }
    \label{fig:intro}
    \vspace{-0.5cm}
\end{figure}

% yaya:
% instance-level 的数据

% 不足的时序理解能力可以归结为3方面原因。 
% 首先,常用的时序理解数据集,如xx等时序定位数据集和xx等密集描述数据集,使用粗粒度事件级标注,缺少细粒度物体级时空运动的标注,因此模型难以学到对物体级动态变化的准确感知。 
% 其次,现有长视频编码方法没有专门考虑视频事件的关键信息,难以对参与事件的所有实例的时空运动进行无损保留,限制了对物体运动的感知。
% 此外,现有工作没有同时编码实例在视频中运动时所处的时空位置信息,使得模型时空位置感知能力弱。
% The insufficient instance-level motion perception can be attributed to three main reasons. First, commonly used training datasets, such as ShareGPT4Video~\citep{sharegpt4video}, rely on coarse-grained video annotations and lack fine-grained instance spatiotemporal motion annotations, making it difficult for the model to learn accurate perception of instance-level dynamic changes. Additionally, existing Video-LLMs~\citep{llama-vid, song2024moviechat} do not specifically consider the key information during visual token compression, resulting in the inability to retain the spatiotemporal motions of all instances involved in the video without loss and restricting the instance motion perception. Furthermore, current works~\citep{timechat, lita, wang2024groundedvideollm} can not simultaneously encode both the spatial and temporal position information of instances as they move in the video, leading to the weak instance position awareness.

The insufficient instance-level motion perception is primarily attributed to two aspects: data and model. On the data side, commonly used training datasets, such as ShareGPT4Video~\citep{sharegpt4video}, rely on coarse-grained video annotations and lack fine-grained instance spatiotemporal motion annotations. This deficiency makes it challenging for models to accurately perceive instance-level dynamic changes. On the model side, existing Video-LLMs~\citep{llama-vid, song2024moviechat} do not specifically consider critical spatiotemporal information during the visual token compression process, resulting in an inability to efficiently retain all instances' spatiotemporal motions in the video without loss, thereby limiting instance motion perception. Moreover, current works~\citep{timechat, lita, wang2024groundedvideollm} fail to encode both spatial and temporal position information of moving instances in the video simultaneously, leading to weak instance position awareness.

% 为了赋予对精细的视频行为的卓越把握并达到更好的视频时序理解,我们首先针对性构造了包含实例运动的指令微调数据集iMove-IT。
% iMove-IT设置了时空运动互相监督的目标,分别设置了给定运动和时间进行空间定位, 给定运行和空间进行时间定位和给定时间空间进行运动描述的任务,发挥二者的协同作用进而给出强大的感知能力,进而赋予模型卓越的通用视频理解和视频时序理解。
% To enable the model to accurately capture fine-grained object-level motion and support superior temporal understanding of videos, we propose a new .
% To endow the model with an exceptional ability to capture fine-grained video actions and support superior video temporal understanding, 
To enable exceptional capture of fine-grained instance motions and achieve superior video understanding, we first construct an  \textbf{i}nstance-\textbf{MO}tion-aware \textbf{V}id\textbf{E}o \textbf{I}nstruction-\textbf{T}uning dataset iMOVE-IT, which contains diverse, detailed and rich instance motions. iMOVE-IT defines spatiotemporal mutual-supervision goals, with tasks designed to perform spatial grounding given instance dynamic captions and time information, temporal grounding given instance dynamic captions and spatial information, and instance dynamic captioning given instance time and spatial information. These tasks work in synergy to enhance the model's instance motion awareness, thereby endowing the model with excellent temporal video understanding and general video understanding.

% Building on the instance-motion-aware dataset iMOVE-IT, we further introduce iMOVE, which is a novel \textbf{i}nstance-\textbf{MO}tion-aware \textbf{V}id\textbf{E}o foundation model. For visual encoding, iMOVE adopts an Event-aware Spatiotemporal Efficient Modeling strategy, adaptively segmenting key events in long videos while preserving the complete instance spatial appearances and high-frame-rate instance temporal motions within each event. 
% This ensures that the model’s detailed instance motion understanding is not compromised by information loss.
% For positional encoding, iMOVE introduces Relative Spatiotemporal Position Tokens to simultaneously indicate the spatiotemporal locations of instances during motion, enabling the model to obtain high sensitivity to instance position.
% The meticulous designs allow iMOVE to fully understand instance spatiotemporal motions and seamlessly integrate with the iMOVE-IT dataset, unlocking the potential of video understanding.

Building on the instance-motion-aware dataset iMOVE-IT, we further introduce iMOVE, a novel \textbf{i}nstance-\textbf{MO}tion-aware \textbf{V}id\textbf{E}o foundation model. For visual encoding, iMOVE adopts an Event-aware Spatiotemporal Efficient Modeling strategy, which adaptively segments key events in long videos while preserving the complete spatial appearances of instances and high-frame-rate temporal motions of instances within each event. This ensures the model’s detailed understanding of instance motion is not compromised by information loss. For positional encoding, iMOVE introduces Relative Spatiotemporal Position Tokens to simultaneously indicate the spatiotemporal locations of instances during motion, enabling the model to achieve high sensitivity to instance positions. These meticulous designs allow iMOVE to fully understand instance spatiotemporal motions and seamlessly integrate with the iMOVE-IT dataset, unlocking the potential of video understanding.

% We conduct comprehensive evaluations on both video temporal understanding and general video understanding benchmarks. 
% The results demonstrate that our approach not only achieves significant improvements on temporal understanding tasks but also excels in general video understanding tasks. 
% Notably, in the zero-shot setting, iMOVE achieves a 10.5\% mIOU improvement on the temporal sentence grounding task of Charades-STA~\cite{charades-sta} and a 1.1 SODA\_c score improvement on the dense video captioning task of ActivityNet\-Captions~\cite{activitynet}. 
% Furthermore, in the fine-tuning setting, iMOVE surpasses the classical fine-tuned expert models. Meanwhile, iMOVE not only improves general video understanding benchmarks such as MVbench and Video-MME but also significantly enhances the long video understanding ability.
% The contributions of this work are listed as follows:
We conduct comprehensive evaluations on both video temporal understanding and general video understanding benchmarks. 
The results demonstrate that our approach not only achieves significant improvements in temporal understanding tasks but also excels in general video understanding tasks. 
Notably, in the zero-shot setting, iMOVE achieves a 10.5\% mIOU improvement on the temporal sentence grounding task of Charades-STA~\cite{charades-sta} and a 1.1 SODA\_c score improvement on the dense video captioning task of ActivityNet-Captions~\cite{activitynet}. 
Furthermore, in the fine-tuning setting, iMOVE surpasses classical fine-tuned expert models. Meanwhile, iMOVE not only improves general video understanding benchmarks such as MVbench and Video-MME but also significantly enhances long video understanding capabilities.
The contributions of this work are listed as follows:

\begin{itemize}
    
    \item We collect iMOVE-IT, the first large-scale instance-motion-aware video instruction-tuning dataset with rich instance motions, to improve the model’s ability to perceive spatiotemporal motions of instances.

    
    \item We propose iMOVE, which employs Event-aware Spatiotemporal Efficient Modeling to encode key instance spatiotemporal motions and introduces Relative Spatiotemporal Position Tokens to represent instances' spatiotemporal positions, thereby acquiring comprehensive instance motion information.

    \item iMOVE achieves outstanding instance-motion-awareness, leading to significant performance improvements in tasks related to video temporal understanding as well as general and long-term video understanding.
    
\end{itemize}

\begin{table}

\resizebox{0.47\textwidth}{!}{
\begin{tabular}{@{}ccccccc@{}}
\toprule
 & \multicolumn{2}{c}{Spatial grounding} & \multicolumn{2}{c}{Temporal Grounding} & \multicolumn{2}{c}{Instance Dynamic Captioning} \\ \cmidrule(l){2-7} 
\multirow{-2}{*}{Method} & Grasp & Generate & Grasp & Generate & Grasp & Generate \\ 
\midrule
\midrule
% PG-Video-LLaVA~\citep{munasinghe2023pgvideollava} & \lgxmark & \lgcmark & \lgxmark & \lgxmark & \lgcmark & \lgcmark \\
Elysium~\citep{wang2024elysium} & \lgxmark & \lgcmark & \lgcmark & \lgxmark & \lgcmark & \lgcmark \\
PiTe~\citep{pite} & \lgxmark & \lgcmark & \lgxmark & \lgcmark & \lgcmark & \lgcmark \\
VideoGLaMM~\citep{munasinghe2024videoglamm} & \lgxmark & \lgcmark & \lgxmark & \lgxmark & \lgcmark & \lgcmark \\
INST-IT~\citep{peng2024instit} & \lgcmark & \lgxmark & \lgcmark & \lgxmark & \lgcmark & \lgcmark \\
ViCaS~\citep{athar2024vicas} & \lgxmark & \lgcmark & \lgxmark & \lgxmark & \lgcmark & \lgcmark \\
Sa2VA~\citep{yuan2025sa2va} & \lgxmark & \lgcmark & \lgxmark & \lgxmark & \lgcmark & \lgcmark \\
VideoRefer Suite~\citep{yuan2024videorefer} & \lgcmark & \lgxmark & \lgxmark & \lgxmark & \lgcmark & \lgcmark \\
\rowcolor{ours-highlight} iMOVE-IT(Ours) & \lgcmark & \lgcmark & \lgcmark & \lgcmark & \lgcmark & \lgcmark \\
\bottomrule
\end{tabular}
}
\caption{Comparison of Instance-Motion-Aware Tasks in Different Studies.}
\label{tab:various_instance_perception_aware_tasks}
\vspace{-0.4cm}
\end{table}



\section{Related Work}
\begin{figure*}
    \centering
    \includegraphics[width=1\linewidth]{figures/icml2025_dataset_construction_compress.pdf}
    \caption{Instance spatiotemporal motion generation pipeline.}
    \label{fig:motion_anno}
    \vspace{-0.5cm}
\end{figure*}
\subsection{Video Large Language Models}
With the rapid advancement of large language models (LLMs), Video-LLMs have gained significant attention. Early methods, such as Video-LLAMA~\citep{video-llama} and VideoChat~\citep{videochat}, captured the overall semantics of videos through tasks like video question answering~\citep{video-chatgpt, zhang2024llava-video} and video captioning~\citep{sharegpt4video, zhang2024llava-video}. Subsequently, methods like TimeChat~\citep{timechat}, VTimeLLM~\citep{vtimellm}, and HawkEye~\citep{hawkeye} integrated tasks such as temporal grounding~\citep{charades-sta}, dense video captioning~\citep{activitynet}, and highlight detection~\citep{qvhighlights}, modeling temporal understanding. Despite these advancements, these methods still encounter challenges in fine-grained, instance-level comprehension, where accurately modeling spatiotemporal motion features is crucial for precise video understanding. Therefore, in this paper, we introduce an instance-level instruction tuning task, iMOVE-IT, and an instance-motion aware model, iMOVE.


\subsection{Instance Perception for Video Understanding}

% 如表~\ref{tab:various_instance_perception_aware_tasks}所示,一些方法~\citep{munasinghe2023pgvideollava, wang2024elysium, pite, peng2024instit, yuan2025sa2va}尝试通过实例级监督任务来增强时空感知能力,这些任务分为视觉定位、时间定位和实例描述,并根据模型放置位置分为输入侧(抓取)和输出侧(生成)。Elysium~\citep{wang2024elysium}提出了ElysiumTrack-1M,支持单目标跟踪、参考跟踪和视频参考表达式生成;INST-IT~\citep{peng2024instit}创建了一个实例特定的指令调优数据集,专注于实例状态、转换和详细的问答对;VideoRefer Suite~\citep{yuan2024videorefer}贡献了VideoRefer-700K,这是一个区域级数据集,包含详细描述、简短描述和多轮问答对,用于对象级视频理解。

% 尽管现有方法涵盖了三类任务,但其覆盖范围仍有限,其仅在输入或输出的一侧进行感知。相比之下,本文提出的iMOVE-IT数据集在细粒度实例级时空感知能力建模上更具全面性,为视频理解任务提供了更丰富的支持。
As shown in Table~\ref{tab:various_instance_perception_aware_tasks}, recent methods have enhanced fine-grained spatiotemporal awareness through instance-level supervision tasks, categorized into spatial grounding, temporal grounding, and instance dynamic captioning, and further divided into input-side (grasp) and output-side (generate) based on model placement. Notable contributions include ElysiumTrack-1M~\citep{wang2024elysium} for single object tracking, referring tracking, and video referring expression generation; INST-IT~\citep{peng2024instit} with its instance-specific instruction tuning dataset focusing on states, transitions, and QA pairs; and VideoRefer-700K~\citep{yuan2024videorefer} providing region-level annotations with detailed captions and multi-round QA pairs for object-level video understanding. Compared to these existing methods involving tasks from either grasping or generating perspectives with limited coverage, 
our iMOVE-IT dataset provides comprehensive tasks for instance-level spatiotemporal perception.

\section{Method}
In this section, we outline the construction of iMOVE as follows: collect the dataset iMOVE-IT with an automated pipline in Sec \ref{sec:dataset} and build the model iMOVE in Sec \ref{sec:model}.

\subsection{\datasetname}
\label{sec:dataset}
% We designed an automated pipeline to create iMOVE-IT, the first large-scale, high-quality dataset with rich spatiotemporal motions. iMOVE-IT introduces spatiotemporal mutual-supervision tasks to fully unleash the model's potential in understanding both temporal and spatial motions.

\subsubsection{Instance Spatiotemporal Motion Generation}
As shown in Figure \ref{fig:motion_anno}, we propose an automatical pipeline for generating instance motion trajectories and corresponding dynamic captions. The specific steps are as follows:

\textbf{Step-1: Instance Recognition}
The initial frame of the given video is processed by the tagging expert RAM++ \citep{huang2023ram_plus} to identify all object categories present. Human checkers review these object categories and filter out some static categories, such as cloud, sky, road, and room.

\textbf{Step-2: Instance Bounding Box Detection.}
% For each recognized category, we employ the open-vocabulary detection expert Grounding DINO \citep{liu2023groundingdino} to find all relevant instances and their associated bounding boxes. Subsequently, we filter out unreasonable bounding boxes, such as those that are excessively large or those with low similarity scores between the sub-image and the category name as determined by CLIP\citep{clip}. The filtering thresholds are optimized based on feedback from human checkers.

For each recognized category, we employ the open-vocabulary detection expert Grounding DINO \citep{liu2023groundingdino} to find all relevant instances and their associated bounding boxes. Subsequently, we filter out unreasonable bounding boxes, specifically those that are excessively large or those that do not match the corresponding category names based on CLIP \citep{clip} similarity scores. The filtering thresholds are optimized based on feedback from human checkers.

\textbf{Step-3: Instance Motion Trajectory Creation.} Each detected instance bounding box is treated as a tracking target and fed into the visual tracking expert SAMURAI \citep{yang2024samurai}, which outputs the instance motion trajectory. We filter out unreasonable trajectories, i.e., those with bounding boxes that differ significantly in area from the initial bounding boxes, and those with low CLIP similarity scores between the bounding boxes and their corresponding category names. The filtering thresholds are determined based on feedback from human checkers who prioritize accuracy.

\begin{figure}
    \centering
    \includegraphics[width=1.0\linewidth]{figures/Ours_SFT_Tasks_v3_compress.pdf}
    % \vspace{-0.5cm}
    \caption{Examples of iMOVE-IT.}
    \label{fig:iMOVE-IT}
    \vspace{-0.5cm}
\end{figure}

\textbf{Step-4: Instance Dynamic Captioning.}
% To enhance the accuracy of data generation, we query the captioning expert InternVL2-40B\citep{InternVL2024} twice, separately generating static and dynamic captions Specifically, by combining the category name, bounding box, and motion trajectory of each instance, we first prompt InternVL2-40B to generate accurate static captions that describe the instance's attributes, such as appearance. Based on these static captions, InternVL2-40B further generates dynamic captions that describe the spatiotemporal motion of the instance. Additionally, we employ Qwen2-VL-7B-Instruct\citep{wang2024qwen2vl} as a checker to filter out erroneous and non-uniquely referring dynamic captions, thereby improving the quality of the generated data.
To enhance the accuracy of data generation, we query the captioning expert InternVL2-40B\citep{InternVL2024} twice, separately generating static and dynamic captions. Specifically, by combining the category name, bounding box, and motion trajectory of each instance, we first prompt InternVL2-40B to generate accurate static captions that describe the instance's appearance. Following this, based on the static captions, InternVL2-40B generates dynamic captions that describe the spatiotemporal motion of the instance. Additionally, we employ Qwen2-VL-7B-Instruct\citep{wang2024qwen2vl} as a checker to filter out erroneous dynamic captions and those that cannot uniquely refer to the instance, thereby improving the quality of the generated data.


\subsubsection{iMOVE-IT Construction}
As shown in Figure \ref{fig:iMOVE-IT}, by utilizing an automatical pipeline for data generation, we constructed iMOVE-IT, which encompasses three tasks: Spatial Grounding, Temporal Grounding, and Instance Dynamic Captioning.

\textbf{Spatial Grounding.} 
% This task involves predicting the bounding boxes $<$bbox1, bbox2$>$ for the relevant instance in the start and end frames, based on the given description $<$dynamic caption$>$ and time interval $<$t1, t2$>$. This improves the  spatial grounding ability, allowing it to accurately identify and track the target instance across consecutive video frames.
This task involves predicting the bounding box coordinates $<$bbox1, bbox2$>$ for the relevant instances in the start and end frames, based on the given dynamic caption $<$dynamic caption$>$ and the time interval $<$t1, t2$>$. This process enhances the spatial grounding capability, enabling precise localization of target instances across consecutive video frames.

\textbf{Temporal Grounding.} 
% This task requires predicting the time interval $<$t1, t2$>$ in which the instance appears, based on the given dynamic caption $<$dynamic caption$>$ and the bounding boxes $<$bbox1, bbox2$>$ at the start and end of the instance’s motion trajectory.This improves the model's temporal segment awareness, empowering it to precisely locate the time intervals in which the instance appears, matching the specific description and bounding boxes.
The task involves predicting the time interval $<$t1, t2$>$ corresponding to the motion trajectory of an instance based on the provided dynamic caption $<$ dynamic caption$>$ and bounding box coordinates $<$bbox1, bbox2$>$ at the start and end of the trajectory. This task enhances the model's temporal awareness of video segments, enabling it to determine time intervals corresponding to specific dynamic captions and changes in spatial positions.


\textbf{Instance Dynamic Captioning.} 
% The task aims to generate a dynamic caption based on the given time interval $<t1, t2>$ and the bounding box coordinates $<bbox1, bbox2>$ at the start and end of the instance's motion trajectory within this interval. This strengthens the model's video-text matching ability, enabling it to generate captions that reflect the specific instance's spatiotemporal motion within a given time interval.
The task aims to generate a dynamic caption based on the given time interval $<$t1, t2$>$ and the bounding box coordinates $<$bbox1, bbox2$>$, which represent the start and end of the instance's motion trajectory within this time interval. This task strengthens the model's ability to match video and text, enabling it to generate dynamic caption that accurately reflect the specific instance's spatiotemporal motion within the given time interval.

The proposed iMOVE-IT dataset consists of three meticulously designed instance-level spatiotemporal mutual-supervision tasks, which provide abundant spatiotemporal supervisory signals to enhance the model's fine-grained instance spatiotemporal motion perception capabilities from the data perspective.

\subsubsection{Statistics of the iMOVE-IT Dataset}
% iMOVE-IT comprises 114k video-instruction pairs from 68k unique videos with an average duration of 109 seconds. To ensure zero-shot testing integrity, the dataset excludes ActivityNet and Charades-STA videos, while incorporating data from 11 distinct datasets. Detailed statistics are in Appendix ~\ref{data_statics}.
iMOVE-IT comprises 114k video-instruction pairs from 68k unique videos, with an average duration of 109 seconds. To ensure the integrity of zero-shot setting, the dataset excludes videos from ActivityNet~\citep{activitynet} and Charades-STA~\citep{charades-sta}, while incorporating data from 11 distinct datasets. Detailed statistics are provided in Appendix~\ref{data_statics}.


\subsection{\modelname}
\label{sec:model}

\begin{figure*}
    \centering
    \includegraphics[width=1.0\linewidth]{figures/icml2025_model_architecture_compress.pdf}
    \caption{
    % \textcolor{red}{Model architecture of \modelname. We propose Event-aware Spatiotemporal Efficient Modeling for visual tokens achieving an optimal balance between model efficiency and feature representation integrity, and a Relative Temporal Indicator for temporal token, enhancing the model's ability to recognize timestamps.}
    % iMOVE的架构。iMOVE采用事件感知的时空高效建模来编码长视频,保留其中的关键实例时空运动信息。
    % 此外,iMOVE引入相对时空位置编码,增强了对实例运动所处视频时空位置的感知敏感度。
    Architecture of iMOVE. iMOVE employs Event-aware Spatiotemporal Efficient Modeling to encode long videos while preserving key instance motions. Additionally, iMOVE introduces Relative Spatiotemporal Positional Token, enhancing sensitivity to spatiotemporal locations of instance motions within the video.
    }
    \label{fig:model-architecture}
    \vspace{-0.5cm}
\end{figure*}

As shown in Figure \ref{fig:model-architecture}, iMOVE adapts short Video-LLMs to perceive fine-grained spatiotemporal instance motions, thereby enhancing its temporal and general video understanding capabilities, while improving long-term video understanding.

For a long video, we sample $T$ frames and encode each into $N$ visual tokens using a visual encoder. These tokens are projected into the language semantic space, creating a feature map $\mathbf{F} \in \mathbb{R}^{\hat{h} \times \hat{w} \times d}$ for each frame, where $\hat{h}$ and $\hat{w}$ represent the dimensions of the feature map, and $d$ is the token dimension size. The complete video is represented as $\mathbf{V} = (\mathbf{F}_1, \mathbf{F}_2, \dots, \mathbf{F}_T) \in \mathbb{R}^{T \times \hat{h} \times\hat{w} \times d}$.

% We further propose \textbf{Event-aware Spatiotemporal Efficient Modeling}, which reduces visual tokens in dense video encoding while preserving essential spatiotemporal motion, balancing model efficiency and video representation integrity. We also propose the \textbf{Relative Spatiotemporal Position Tokens} to enhance the model's understanding of instances' spatiotemporal positions. The above innovations will be presented in the subsequent sections.
We further propose \textbf{Event-aware Spatiotemporal Efficient Modeling}, which reduces visual tokens in dense video encoding while preserving essential spatiotemporal motion, balancing model efficiency and video representation integrity. We also propose \textbf{Relative Spatiotemporal Position Tokens} to enhance the model's understanding of instances' spatiotemporal positions. These innovations will be presented in the subsequent sections.
% within which, unlike previous methods that use relative temporal representations, we employ \textbf{Relative Temporal Indicator Token} to represent each visual token's temporal information. These tokens are interleaved with pruned visual tokens, allowing the model to accurately perceive video time information. The above innovations will be presented in the subsequent sections.


\subsubsection{Event-aware Spatiotemporal Efficient Modeling}
% For the encoded dense video frames $\mathbf{V}$, the video events are first adaptively segmented. The most informative frames in each event are then identified as the spatial and appearance representatives. Subsequently, other frames with less spatial information in the event are spatially compressed, preserving the high-frame-rate detailed temporal and motion information of the event. Notably, this efficient process does not introduce new parameters and remains training-free, thus reducing the overall model training complexity.

For the raw dense visual features $\mathbf{V}$, we propose the Event-aware Spatiotemporal Efficient Modeling method, which maintains computational efficiency while preserving the complete spatial appearances of instances and the high-frame-rate temporal motions of instances. Specifically, iMOVE first adaptively segments the video into events. The frame with the most information in each event is identified as the one retaining the most spatial information. Subsequently, other frames with less information in the event undergo greater spatial compression to preserve the high-frame-rate detailed temporal and motion information of the instance. Notably, this computationally efficient process does not introduce new parameters, thereby reducing the overall model training complexity.

\textbf{Adaptive Video Event Segmentation.} We propose a novel adaptive video event segmentation method that captures event transitions in videos by identifying significant changes in inter-frame similarity. First, we calculate the cosine similarity between adjacent frame features:

\begin{equation}
S_{i, i + 1} = \frac{\mathbf{F}_i \cdot \mathbf{F}_{i+1}}{\|\mathbf{F}_i\| \|\mathbf{F}_{i+1}\|}
\end{equation}

where \( \mathbf{F}_i \) and \( \mathbf{F}_{i+1} \) are the feature vectors of adjacent frames. Then, we compute the rate of change in similarity:

\begin{equation}
d_i = |S_{i, i + 1} - S_{i - 1, i}|
\end{equation}

Next, we sort \( d_i \) in descending order and select the largest \( K - 1 \) extreme points as event boundaries, thereby segmenting the video into \( K \) events.

\textbf{Event Spatial Representation.} 
According to information entropy theory, adjacent frames with larger similarity differences carry more information, while those with smaller differences contain redundancy.
Based on this, we consider the first frame of each event, which occurs at the moment with the highest rate of information change, to be the frame with the most instance appearance information of the event and designate it as the event's spatial representation.

\begin{table*}
% \vskip 0.1in
% \captionsetup{font=scriptsize}

\centering
\small % 调整表格字体大小
% \vskip 0.05in
\resizebox{\textwidth}{!}{%
\begin{tabular}{@{}lccccccccccc@{}}
\toprule
\multicolumn{1}{l|}{\multirow{2}{*}{\textbf{Model}}} & \multicolumn{1}{c|}{\textbf{LLM}} & \multicolumn{4}{c|}{\textbf{Charades-STA}} & \multicolumn{4}{c|}{\textbf{ActivityNet-Grounding}} & \multicolumn{2}{c}{\textbf{ActivityNet-Captions}} \\ \cmidrule(l){3-12} 
\multicolumn{1}{l|}{} & \multicolumn{1}{c|}{\textbf{Scale}} & R@0.3 & R@0.5 & \multicolumn{1}{c|}{R@0.7} & \multicolumn{1}{c|}{mIoU} & R@0.3 & R@0.5 & \multicolumn{1}{c|}{R@0.7} & \multicolumn{1}{c|}{mIoU} & SODA\_c & METEOR \\ 
\midrule
\midrule
\multicolumn{12}{c}{Zero-Shot} \\ \midrule
\multicolumn{1}{l|}{Video-ChatGPT \citep{video-chatgpt}} & \multicolumn{1}{c|}{7B} & 27.2 & 6.2 & \multicolumn{1}{c|}{1.9} & \multicolumn{1}{c|}{19.7} & 19.5 & 10.6 & \multicolumn{1}{c|}{4.8} & \multicolumn{1}{c|}{14.2} & 1.9 & 2.1 \\
% \multicolumn{1}{l|}{VideoChat2 \citep{mvbench}} & \multicolumn{1}{c|}{7B} & 38.0 & 14.3 & \multicolumn{1}{c|}{3.8} & \multicolumn{1}{c|}{24.6} & 40.8 & 27.8 & \multicolumn{1}{c|}{9.3} & \multicolumn{1}{c|}{27.9} & - & - \\
\multicolumn{1}{l|}{VideoChat \citep{videochat}} & \multicolumn{1}{c|}{7B} & 32.8 & 8.6 & \multicolumn{1}{c|}{0.0} & \multicolumn{1}{c|}{25.9} & 23.5 & 12.6 & \multicolumn{1}{c|}{6.0} & \multicolumn{1}{c|}{17.4} & 0.9 & 0.9 \\
\multicolumn{1}{l|}{Momentor \citep{momentor}} & \multicolumn{1}{c|}{7B} & 42.6 & 26.6 & \multicolumn{1}{c|}{11.6} & \multicolumn{1}{c|}{28.5} & \textbf{42.9} & \underline{23.0} & \multicolumn{1}{c|}{\textbf{12.4}} & \multicolumn{1}{c|}{\underline{29.3}} & \underline{2.3} & \underline{4.7} \\
\multicolumn{1}{l|}{TimeChat \citep{timechat}} & \multicolumn{1}{c|}{7B} & - & 32.2 & \multicolumn{1}{c|}{13.4} & \multicolumn{1}{c|}{-} & - & - & \multicolumn{1}{c|}{-} & \multicolumn{1}{c|}{-} & - & - \\
\multicolumn{1}{l|}{VTG-LLM \citep{vtgllm}} & \multicolumn{1}{c|}{7B} & - & 33.8 & \multicolumn{1}{c|}{15.7} & \multicolumn{1}{c|}{-} & - & - & \multicolumn{1}{c|}{-} & \multicolumn{1}{c|}{-} & - & - \\
\multicolumn{1}{l|}{HawkEye $^{\bigstar}$  \citep{hawkeye}} & \multicolumn{1}{c|}{7B} & 50.6 & 31.4 & \multicolumn{1}{c|}{14.5} & \multicolumn{1}{c|}{33.7} & \textcolor{gray}{49.1} & \textcolor{gray}{29.3} & \multicolumn{1}{c|}{\textcolor{gray}{10.7}} & \multicolumn{1}{c|}{\textcolor{gray}{32.7}} & - & - \\ 
% \multicolumn{1}{l|}{Seq2Time \citep{deng2024seq2timesequentialknowledgetransfer}} & \multicolumn{1}{c|}{7B} & - & 31.2 & \multicolumn{1}{c|}{13.7} & \multicolumn{1}{c|}{-} & - & - & \multicolumn{1}{c|}{-} & \multicolumn{1}{c|}{-} & - & - \\
\multicolumn{1}{l|}{PiTe$^{\bigstar}$ \citep{pite}} & \multicolumn{1}{c|}{7B} & - & - & \multicolumn{1}{c|}{-} & \multicolumn{1}{c|}{-} & \textcolor{gray}{30.4} & \textcolor{gray}{17.8} & \multicolumn{1}{c|}{\textcolor{gray}{7.8}} & \multicolumn{1}{c|}{\textcolor{gray}{22.0}} & \textcolor{gray}{5.1} & \textcolor{gray}{5.8} \\ 
\multicolumn{1}{l|}{Grounded-VideoLLM $^{\bigstar}$ \citep{wang2024groundedvideollm}} & \multicolumn{1}{c|}{4B} & 54.2 & 36.4 & \multicolumn{1}{c|}{19.7} & \multicolumn{1}{c|}{\underline{36.8}} & \textcolor{gray}{46.2} & \textcolor{gray}{30.3} & \multicolumn{1}{c|}{\textcolor{gray}{19.0}} & \multicolumn{1}{c|}{\textcolor{gray}{36.1}} & \textcolor{gray}{6.0} & \textcolor{gray}{6.8} \\ 
\multicolumn{1}{l|}{VTimeLLM $^{\bigstar}$ \citep{vtimellm}} & \multicolumn{1}{c|}{7B} & 51.0 & 27.5 & \multicolumn{1}{c|}{11.4} & \multicolumn{1}{c|}{31.2} & \textcolor{gray}{44.0} & \textcolor{gray}{27.8} & \multicolumn{1}{c|}{\textcolor{gray}{14.3}} & \multicolumn{1}{c|}{\textcolor{gray}{30.4}} & \textcolor{gray}{5.8} & \textcolor{gray}{6.8} \\
\multicolumn{1}{l|}{TIMESUITE \citep{timesuite}} & \multicolumn{1}{c|}{7B} & \underline{69.9} & \underline{48.7} & \multicolumn{1}{c|}{\underline{24.0}} & \multicolumn{1}{c|}{-} & - & - & \multicolumn{1}{c|}{-} & \multicolumn{1}{c|}{-} & - & - \\ 
\multicolumn{1}{l|}{TRACE \citep{trace}} & \multicolumn{1}{c|}{7B} & - & 40.3 & \multicolumn{1}{c|}{19.4} & \multicolumn{1}{c|}{-} & - & - & \multicolumn{1}{c|}{-} & \multicolumn{1}{c|}{-} & - & - \\ 
\multicolumn{1}{l|}{InternVL2-4B\citep{InternVL2024}} & \multicolumn{1}{c|}{4B} & 14.7 & 7.9 & \multicolumn{1}{c|}{3.3} & \multicolumn{1}{c|}{11.9} & 17.1 & 9.5 & \multicolumn{1}{c|}{4.6} & \multicolumn{1}{c|}{12.7} & 0.85 & 2.81 \\
\midrule
 \rowcolor{ours-highlight} \multicolumn{1}{l|}{iMOVE} & \multicolumn{1}{c|}{4B}  & \textbf{71.7} & \textbf{51.3} & \multicolumn{1}{c|}{\textbf{26.1}} & \multicolumn{1}{c|}{\textbf{47.3}} & \underline{42.4} & \textbf{23.1} & \multicolumn{1}{c|}{\underline{12.1}} & \multicolumn{1}{c|}{\textbf{29.7}} & \textbf{3.4} & \textbf{6.8} \\ 
 \midrule
 \midrule
\multicolumn{12}{c}{Fine-Tuning} \\ \midrule
\multicolumn{1}{l|}{Vid2Seq $^{\spadesuit}$\citep{yang2023vid2seqlargescalepretrainingvisual}} & \multicolumn{1}{c|}{-} & - & - & \multicolumn{1}{c|}{-} & \multicolumn{1}{c|}{-} & - & - & \multicolumn{1}{c|}{-} & \multicolumn{1}{c|}{-} & 5.8 & - \\
\multicolumn{1}{l|}{QD-DETR$^{\spadesuit}$\citep{moon2023querydependentvideorepresentationmoment}} & \multicolumn{1}{c|}{-} & - & 57.3 & \multicolumn{1}{c|}{32.6} & \multicolumn{1}{c|}{-} & - & - & \multicolumn{1}{c|}{-} & \multicolumn{1}{c|}{-} & - & - \\
\multicolumn{1}{l|}{UnLoc-L $^{\spadesuit}$ \citep{yan2023unlocunifiedframeworkvideo}} & \multicolumn{1}{c|}{-} & - & 60.8 & \multicolumn{1}{c|}{38.4} & \multicolumn{1}{c|}{-} & - & \underline{48.3} & \multicolumn{1}{c|}{\underline{30.2}} & \multicolumn{1}{c|}{-} & - & - \\
\multicolumn{1}{l|}{VTG-LLM \citep{vtgllm}} & \multicolumn{1}{c|}{7B} & - & 57.2 & \multicolumn{1}{c|}{33.4} & \multicolumn{1}{c|}{-} & - & - & \multicolumn{1}{c|}{-} & \multicolumn{1}{c|}{-} & - & - \\
\multicolumn{1}{l|}{HawkEye \citep{hawkeye}} & \multicolumn{1}{c|}{7B} & 72.5 & 58.3 & \multicolumn{1}{c|}{28.8} & \multicolumn{1}{c|}{\underline{49.3}} & \underline{55.9} & 34.7 & \multicolumn{1}{c|}{17.9} & \multicolumn{1}{c|}{\underline{39.1}} & - & - \\ 
\multicolumn{1}{l|}{TIMESUITE \citep{timesuite}} & \multicolumn{1}{c|}{7B} & \underline{79.4} & \underline{67.1} & \multicolumn{1}{c|}{\underline{43.0}} & \multicolumn{1}{c|}{-} & - & - & \multicolumn{1}{c|}{-} & \multicolumn{1}{c|}{-} & - & - \\ 

\multicolumn{1}{l|}{TRACE \citep{trace}} & \multicolumn{1}{c|}{7B} & - & 61.7 & \multicolumn{1}{c|}{41.4} & \multicolumn{1}{c|}{-} & - & 37.7 & \multicolumn{1}{c|}{24.0} & \multicolumn{1}{c|}{39.0} & \underline{6.0} & \underline{6.4} \\

\midrule
\rowcolor{ours-highlight} \multicolumn{1}{l|}{iMOVE+FT} & \multicolumn{1}{c|}{4B}  & \textbf{79.8} & \textbf{68.5} & \multicolumn{1}{c|}{\textbf{45.3}} & \multicolumn{1}{c|}{\textbf{57.9}} & \textbf{67.2} & \textbf{50.7} & \multicolumn{1}{c|}{\textbf{32.4}} & \multicolumn{1}{c|}{\textbf{49.3}} & \textbf{6.0} & \textbf{8.0} \\   \bottomrule
\end{tabular}%
}
\caption{Zero-Shot and Fine-Tuning results on Temporal Sentence Grounding and Dense Video Captioning tasks. \textbf{Bold} fonts highlight the best performance. \underline{Underline} highlights the second best performance. Fine-tuned expert models are marked with $^{\spadesuit}$, while non-strict zero-shot methods on ActivityNet-Captions dataset are marked with $^{\bigstar}$.}
\label{tab:results_tsg_dvc}
% \vskip -0.1in
\vspace{-0.4cm}
\end{table*}

\textbf{Event Temporal Representation.}
Non-first frames in each event provide less spatial information than the first frame but contain rich temporal information. Therefore, we use average pooling to merge dense visual tokens spatially. Assuming the first frame feature in each event is $\mathbf{\hat{F}} \in \mathbb{R}^{{h}\times {w} \times d}$, subsequent non-first frame features are compressed based on this dimension. Using $s$ as stride, each non-first frame feature is pooled into $\mathbf{\hat{F}} \in \mathbb{R}^{\frac{h}{s} \times \frac{w}{s} \times d}$, preserving high-frame-rate temporal motions. This strategy allows the model to process more temporal frames without increasing the total visual token length, effectively modeling high-frame-rate event temporal representation.

\subsubsection{Relative Spatiotemporal Position Tokens}
% Taking temporal tokens as examples, existing methods can be categorized into absolute and relative time representations. The former, such as TimeChat~\citep{timechat} and TimeMarker~\citep{chen2024timemarker}, use absolute time tokens like "2s" or "Second{8.0}". However, the broad nature of absolute time ranges and the impossibility of exhaustive enumeration make it challenging to generalize across videos of varying lengths. The latter, such as LITA~\citep{lita} and Grounded-VideoLLM~\citep{wang2024groundedvideollm}, introduce special time tokens into the tokenizer of LLMs. However, this requires modifying word embedding parameters, potentially disrupting their compatibility with the LLM's parameters and risking performance degradation, especially in small-scale fine-tuning scenarios. Therefore, in this paper, we utilize relative spatiotemporal tokens to encode bounding box coordinates and timestamp positions without adding them into the tokenizer. 
% Unlike previous methods that use relative time representations~\citep{wang2024groundedvideollm, lita}, we append a relative temporal token indicating the corresponding time to the visual tokens of each frame. These tokens are interleaved with the pruned visual tokens, enabling the model to more accurately perceive the temporal information of the video. Specifically, we use quantization and dequantization on the input and output sides to map actual values to relative tokens. 
Taking temporal tokens as examples, existing methods can be categorized into absolute and relative time representations. The former, such as TimeChat~\citep{timechat} and TimeMarker~\citep{chen2024timemarker}, use absolute time tokens like "2s" or "Second{8.0}". However, the broad nature of absolute time ranges and the impossibility of exhaustive enumeration make it challenging to generalize across videos of varying lengths. The latter, such as LITA~\citep{lita} and Grounded-VideoLLM~\citep{wang2024groundedvideollm}, introduce special time tokens into the tokenizer of LLMs. However, this requires modifying word embedding parameters, potentially disrupting their compatibility with the LLM's parameters and risking performance degradation, especially in small-scale fine-tuning scenarios. Therefore, in this paper, we utilize relative spatiotemporal tokens to encode bounding box coordinates and timestamp positions without adding them into the tokenizer. Unlike previous methods using relative time representations~\citep{wang2024groundedvideollm, lita}, we append a relative temporal token indicating the corresponding time to the visual tokens of each frame. These tokens are interleaved with pruned visual tokens, enabling the model to more accurately perceive the temporal information of the video. Specifically, we use quantization and dequantization on input and output sides to map actual values to relative tokens.




For temporal token, given a video of duration \( D \) seconds, we establish a bidirectional mapping between the timestamp \( t \) and a discrete value \( z \in [0, Z] \), where \( Z \) is an empirical value. The quantization process is:
\begin{equation}
z = \text{round}\left(\frac{t}{D} \times Z\right)
\end{equation}
and the dequantization process is:



\begin{equation}
t = \frac{z}{Z} \times D
\end{equation}

For spatial token, given a frame with width \( W \) and height \( H \), we establish a bidirectional mapping between the \( x \) coordinate and a discrete value \( \hat{x} \in [0, \hat{W}] \), as well as between the \( y \) coordinate and a discrete value \( \hat{y} \in [0, \hat{H}] \), in the same way as temporal token.



\begin{table*}
% \captionsetup{font=scriptsize}

% \vskip 0.05in
\centering
\small % 调整表格字体大小
\resizebox{0.8\textwidth}{!}{%
\begin{tabular}{@{}l|c|cc|c|cc|c@{}}
\toprule
\multirow{2}{*}{\textbf{Model}} & \textbf{LLM} & \multicolumn{3}{c|}{\textbf{MVBench}} & \multicolumn{2}{c|}{\textbf{Video-MME(w/o subs)}} & \textbf{LongVideoBench} \\ \cmidrule(l){3-8} 
 & \textbf{Scale} & AS & AP & \multicolumn{1}{c|}{Avg} & Long & Overall & val \\ 
 \midrule
 \midrule
\multicolumn{1}{l|}{Video-LLaMA~\cite{video-llama}} & 7B & 27.5 & 25.5 & \multicolumn{1}{c|}{34.1} & - & - & - \\
\multicolumn{1}{l|}{Video-ChatGPT~\cite{video-chatgpt}} & 7B & 23.5 & 26.0 & \multicolumn{1}{c|}{32.7} & - & - & - \\
\multicolumn{1}{l|}{VideoChat2~\cite{mvbench}} & 7B & 66.0 & 47.5 & \multicolumn{1}{c|}{51.1} & 33.2 & 39.5 &  -\\
\multicolumn{1}{l|}{ST-LLM~\cite{st-llm}} & 7B & 66.0 & 53.5 & \multicolumn{1}{c|}{54.9} & 31.3 & 37.9 &  -\\
\multicolumn{1}{l|}{VideoGPT+~\cite{videogpt+}} & 7B & 69.0 & 60.0 & \multicolumn{1}{c|}{58.7} & - & - &  -\\
\multicolumn{1}{l|}{MovieChat~\cite{song2024moviechat}} & 7B & - & - & \multicolumn{1}{c|}{55.1} & 33.4 & 38.2 & - \\
% \multicolumn{1}{l|}{mPLUG-Owl3~\cite{ye2024mplug}} & 8B & - &  -& \multicolumn{1}{c|}{54.5} & - & 53.5 & - \\ 
\multicolumn{1}{l|}{P-LLaVA-13B~\cite{pllava}} & 13B & 66.0 & 53.0 & \multicolumn{1}{c|}{50.1} & - & - & 45.6  \\
\multicolumn{1}{l|}{LLaVA-Next-Video-34B~\cite{llava2024}} & 34B & - & - & \multicolumn{1}{c|}{-} & - & - & 50.5  \\
% \multicolumn{1}{l|}{Grounded-VideoLLM~\cite{wang2024groundedvideollm}} & 4B & 76.0 & 75.5 & \multicolumn{1}{c|}{59.4} & - & - & -  \\
\multicolumn{1}{l|}{TIMESUITE \citep{timesuite}} & 7B & - & - & \multicolumn{1}{c|}{59.9} & \underline{41.9} & 46.3 & - \\
\multicolumn{1}{l|}{TRACE \citep{trace}} & 7B & - & - & \multicolumn{1}{c|}{48.1} & - & 43.8 & - \\
\multicolumn{1}{l|}{InternVL2-4B\citep{InternVL2024}} & 4B & \textbf{76.0} & \underline{62.5} & \multicolumn{1}{c|}{\underline{63.7}} & 39.2 & \underline{48.7} & \underline{50.7} \\ 
\midrule
\rowcolor{ours-highlight}\multicolumn{1}{l|}{iMOVE} & 4B & \underline{71.5} & \textbf{62.5} &  \multicolumn{1}{c|}{\textbf{63.9}} & \textbf{43.2} & \textbf{53.6} & \textbf{54.7} \\ 
\bottomrule
\end{tabular}
}
\caption{Zero-Shot results on General Video Understanding and Long-term Video Understanding tasks. \textbf{Bold} fonts highlight the best performance. \underline{Underline} highlights the second best performance.}
\label{tab:results_general_vid_understanding}
\vspace{-0.4cm}
\end{table*}

\section{Experiments}
The detailed experimental setup and hyper-parameters can be found in Appendix~\ref{experiment_setup}. Baseline and comparison details are provided in Appendix~\ref{appendix:baseline}. 
We have performed rigorous data filtering to ensure the zero-shot setting on Charades-STA and ActivityNet-Captions datasets.. Detailed training data composition and data filtering procedures are described in Appendix~\ref{data_filtering}. Qualitative analysys are available in Appendix~\ref{case_study}.

\subsection{Main Results}
% \textcolor{blue}{To comprehensively evaluate the video understanding capabilities of \modelname, we conduct quantitative assessments on four tasks: Temporal Video Grounding, Dense Video Captioning, Long-term Video Understanding, and General Video Understanding.}
To comprehensively assess the video understanding capabilities, we conducted quantitative evaluations across three task categories: Video Temporal Understanding(including Temporal Video Grounding and Dense Video Captioning), General Video Understanding and Long-term Video Understanding.
Details of the benchmarks and evaluation metrics refer to Appendix ~\ref{benchmark}. Notably, PiTe\citep{pite}, Grounded-VideoLLM\citep{wang2024groundedvideollm}, HawkEye\citep{hawkeye}, and VTimeLLM\citep{vtimellm} do not operate under a strict zero-shot setting on the ActivityNet-Captions dataset. 
Detailed explanations can be found in Appendix~\ref{data_leakage}.

\textbf{Temporal Video Grounding.} This task aims to identify the start and end timestamps of events described by a given query sentence. 
% As shown in Table~\ref{tab:results_tsg_dvc}, in the zero-shot setting, \textcolor{blue}{iMOVE achieves a mIOU of 29.7 on ActivityNet-Grounding and 47.3 on Charades-STA, significantly outperforming the previous state-of-the-art method Grounded-VideoLLM  by 10.5\%.} After further fine-tuning, iMOVE achieves 68.5 on R@0.5 for ActivityNet-Grounding and 50.7 for Charades-STA, significantly surpassing traditional supervised fine-tuning methods. 
As shown in Table~\ref{tab:results_tsg_dvc}, under the zero-shot setting, iMOVE achieves mIoU accuracies of 47.3\% and 29.7\% on Charades-STA and ActivityNet-Grounding, surpassing previous SOTAs, i.e., Grounded-VideoLLM and Momentor, by margins of 10.5\% and 0.4\%. With fine-tuning, iMOVE further attains mIoU accuracies of 57.9\% and 49.3\% on these benchmarks, significantly outperforming existing SOTA methods by 10.6\% and 10.2\%.
% These results collectively highlight iMOVE's superior capability in fine-grained temporal localization.
This highlights iMOVE's superior fine-grained temporal localization capability. 
% As shown in Table~\ref{tab:results_tsg_dvc}, under the zero-shot setting, iMOVE achieves mIoU accuracies of 47.3\% and 29.7\% on Charades-STA and ActivityNet-Grounding, surpassing previous SOTAs, i.e., Grounded-VideoLLM and Momentor\citep{momentor}, by margins of 10.5\% and 0.4\%. With fine-tuning, iMOVE further attains mIoU accuracies of 57.9\% and 49.3\% on these benchmarks, significantly outperforming existing SOTA methods by 10.6\% and 10.2\%. Notably, iMOVE significantly outperforms classical supervised fine-tuning methods such as QD-DETR\citep{moon2023querydependentvideorepresentationmoment} and UnLoc-L\citep{yan2023unlocunifiedframeworkvideo}. This underscores iMOVE's superior fine-grained temporal localization capability.

\textbf{Dense Video Captioning.} 
% The task requires detecting all events in the video and providing the corresponding time intervals and descriptions. 
This task requires detecting all events in videos while providing corresponding duration time intervals and descriptions.
% As shown in Table ~\ref{tab:results_tsg_dvc}, in the zero-shot setting, iMOVE achieves a SODA\_c~\cite{fujita2020soda} score of 3.4 and a METEOR~\cite{banerjee2005meteor} score of 6.8, significantly surpassing the previous state-of-the-art method Momentor\citep{momentor}, which scored 2.3 and 4.7, respectively. After further fine-tuning, iMOVE's performance exceeds that of the fine-tuned expert model Vid2Seq\citep{yang2023vid2seqlargescalepretrainingvisual}. 
As observed in Table ~\ref{tab:results_tsg_dvc}, iMOVE achieves SODA\_c~\cite{fujita2020soda} and METEOR~\cite{banerjee2005meteor} scores of 3.4 and 6.8 on ActivityNet-Captions under the zero-shot setting, exceeding the prior SOTA method Momentor~\cite{momentor} with scores of 2.3 and 4.7. 
After fine-tuning, iMOVE also outperforms the specialized model TRACE~\cite{trace}.
% This indicates that, thanks to architectural improvements, iMOVE can capture the complete storyline in videos and provide more detailed temporal descriptions.
The improvements can be attributed to iMOVE’s meticulously designed key-event retention strategy, which enables accurate capture of complete event narratives.

\textbf{General Video Understanding.}
% We utilize \textbf{Video-MME} and \textbf{MVBench}~\cite{mvbench} to evaluate the general video understanding capability of iMOVE. 
Video-MME~\cite{videomme} and MVBench~\cite{mvbench} are used to evaluate general video understanding capabilities.
% As shown in Table~\ref{tab:results_general_vid_understanding}, iMOVE achieves accuracy of 53.6 and 63.9 on Video-MME and MVBench, respectively, representing improvements of 4.9\% and 0.2\% compared to InternVL2-4B, despite using fewer visual tokens. 
According to the results in Table~\ref{tab:results_general_vid_understanding}, despite utilizing fewer visual tokens, iMOVE obtains accuracies of 53.6 and 63.9 on Video-MME and MVBench, improving by 4.9\% and 0.2\% compared to InternVL2-4B.
% Previously, MLLMs focused on temporal localization have demonstrated excellent performance on time-related tasks but have underperformed in general short video understanding. In contrast, iMOVE enhances the model's fine-grained instance spatiotemporal motions, achieving state-of-the-art performance in both temporal understanding and long video comprehension, while also bolstering the model's general video understanding capability.
While prior temporal-focused MLLMs excel at time-related tasks, they exhibit compromised performance on general video understanding tasks. 
In contrast, iMOVE performs best on both tasks, demonstrating that enhanced fine-grained instance motion modeling synergistically benefits dual capabilities.

\textbf{Long-term Video Understanding.}
% We utilize the Long subset of Video-MME~\cite{videomme} and LongVideoBench~\cite{wu2024longvideobench} as benchmarks to evaluate the long video understanding capability of iMOVE. 
The Long Video subset of Video-MME and LongVideoBench~\cite{wu2024longvideobench} serve as benchmarks for long video understanding evaluation.
% As shown in Table~\ref{tab:results_general_vid_understanding}, thanks to our proposed architectural improvements, iMOVE achieves accuracies of 54.7 and 43.2 on LongVideoBench and Video-MME long, respectively. This represents an improvement of 4\% over InternVL2-4B. Additionally, iMOVE significantly outperforms larger LLMs such as LLaVA-Next-Video-34B~\cite{llava2024} and P-LLaVA-13B~\cite{pllava}. 
The comparisons are listed in Table~\ref{tab:results_general_vid_understanding}, iMOVE acquires accuracies of 54.7 on LongVideoBench and 43.2 on Video-MME Long, both of which represent a 4\% improvement over the similarly parameter-sized Interval2-4B.
Additionally, iMOVE significantly outperforms Video-LLMs with larger parameter scales, e.g., LLaVA-Next-Video-34B~\cite{llava2024} and P-LLaVA-13B~\cite{pllava}.
% This indicates that enhancing fine-grained instance spatiotemporal motions can significantly improve a model's long video understanding capability.
This verifies that strengthening instance motion perceiving improves long video understanding.

\subsection{Ablation Study}

\begin{table}
\centering
% \captionsetup{font=scriptsize}
% \vspace{-0.2cm}
\resizebox{\columnwidth}{!}{%
\begin{tabular}{@{}cccc|cccccc@{}}
\toprule
DVC+TAL & TSG & Geneal & iMOVE-IT & \begin{tabular}[c]{@{}c@{}}C-STA\\ mIoU\end{tabular} & \begin{tabular}[c]{@{}c@{}}ANet-G\\ mIoU\end{tabular} & \begin{tabular}[c]{@{}c@{}}ANet-Cap\\ SODA\_c\end{tabular} & \begin{tabular}[c]{@{}c@{}}LVBench\\ val\end{tabular} & \begin{tabular}[c]{@{}c@{}}V-MME\\ Overall\end{tabular} & \begin{tabular}[c]{@{}c@{}}MVB\\ Avg\end{tabular} \\ \midrule \midrule
\checkmark & & & & 25.4 & 20.3 & Fail & 46.6 & 47.0 & 57.0 \\
\checkmark & \checkmark & & & 44.8 & 27.8 & 1.8 & 50.7 & 50.3 & 57.2 \\
\checkmark & \checkmark & \checkmark & & 46.1 & 27.5 & 2.8 & 53.6 & 53.8 & 63.4 \\
\rowcolor{ours-highlight} \checkmark & \checkmark & \checkmark & \checkmark & 47.3 & 29.7 & 3.4 & 54.7 & 53.6 & 63.9 \\ \bottomrule
\end{tabular}%
}
\vspace{-0.2cm}
\caption{Ablations on training data composition. 
% DVC refers to Dense Video Captioning, TAL denotes Temporal Action Localization, and TSG stands for Temporal Sentence Grounding. 
DVC, TAL, and TSG stands datasets for Dense Video Captioning, Temporal Action Localization and Temporal Sentence Grounding tasks. 
General includes datasets from three tasks: VideoQA, Classification, and Video Captioning. 
% Fail indicates the inability to follow instructions.
Fail is an inability to follow instructions.
}
\label{tab:imoveIT}
\vspace{-0.4cm}
\end{table}

% \textbf{About of iMOVE-IT.} 
\textbf{Benefits of iMOVE-IT.} 
As shown in Table \ref{tab:imoveIT}, the model performance steadily improved with the incremental inclusion of datasets from different tasks. 
The row 4 vs. row 3 comparison reveals that incorporating iMOVE-IT significantly enhances instance motion awareness, leading to improved temporal understanding, long-term video understanding, and general video understanding capabilities.
% The Row 4 vs. Row 3 comparison reveals that incorporating iMOVE-IT substantially enhances instance-level fine-grained spatiotemporal motion modeling, leading to improved holistic video understanding capabilities.



% Please add the following required packages to your document preamble:
% \usepackage{booktabs}
% \usepackage{graphicx}
\begin{table}
\centering
% \captionsetup{font=scriptsize}

\vspace{-0.2cm}
\resizebox{0.95\columnwidth}{!}{%
\begin{tabular}{@{}c|ccc|ccc|cc@{}}
\toprule
Row & Rand & Uniform & Event & K        & s & \# of tokens & \begin{tabular}[c]{@{}c@{}}C-STA\\ mIoU\end{tabular} & \begin{tabular}[c]{@{}c@{}}V-MME\\ Overall\end{tabular} \\ \midrule \midrule
1    &  \checkmark    &         &       &      24    & 2   &  2688            &   43.2    &    50.0   \\
2    &      &      \checkmark   &       &      24    & 2  &   2688       &    43.6   &  49.9     \\
\rowcolor{ours-highlight}3    &      &         &    \checkmark   & 24 & 2 & 2688         & 44.8  & 50.3  \\ \midrule
4    &      &         &    \checkmark   & 0        & 2 & 1536         & 43.8  & 49.2  \\
5    &      &         &   \checkmark    & 12       & 2 & 2112         & 45.0  & 49.8  \\
6    &      &         &   \checkmark    & 48       & 2 & 3840         & 45.1  & 51.3  \\
7    &      &         &   \checkmark    & 96       & 2 & 6144         & 46.2  & 51.1  \\ \midrule
8    &      &         &   \checkmark    & 24       & 1 & 6144         & 46.2  & 51.1  \\
9    &      &         &   \checkmark    & 24       & 4 & 1824         & 44.9  & 49.9  \\
10   &      &         &   \checkmark    & 24       & 8 & 1608         & 44.2  & 49.1  \\ \bottomrule
\end{tabular}%
}
\vspace{-0.2cm}
\caption{Effect of Event-aware Spatiotemporal Efficient Modeling. Rand denotes randomly selecting the first frame of the Event, Uniform denotes uniformly selecting frames, and Event denotes our method.}
\label{tab:about_K_r}
\vspace{-0.1cm}
\end{table}


\textbf{Effectiveness of Event-aware Spatiotemporal Efficient Modeling.} 
The first three rows of Table \ref{tab:about_K_r} demonstrate that our proposed Event-aware Spatiotemporal Efficient Modeling outperforms both random and uniform frame selection strategies. Furthermore, as observed from rows 3 to 5, increasing the number of segmented events \(K\) leads to continuous improvements on Charades-STA and Video-MME datasets. 
Rows 8 to 10 indicate that as the non-first frame pooling rate \(s\) increases, there is a loss of spatial information, which results in a decline in the model's temporal understanding and generalization capabilities. Conversely, when the pooling rate is too small (\(s = 1\)), the excessive number of tokens adversely affects the model's training and inference efficiency. Balancing information loss and efficiency, we selected a pooling rate of (\(K = 24\)) and (\(s = 2\)) as the optimal trade-off. The composition of the data for this ablation study can be found in Appendix ~\ref{dataset_composition_abl}.

% \begin{table}[t]
% \caption{Effect of Relative Temporal Position Token.}
% \label{tab:about_temporal_tokens}
% % \vskip 0.05in
% \vspace{-0.2cm}
% \begin{center}
% \resizebox{0.7\columnwidth}{!}{% 
% \begin{tabular}{l|c|c}
% \toprule
% \multirow{2}{*}{Method} & Charades-STA & ActivityNet-Grounding \\ \cmidrule(l){2-3} 
%  & mIoU & mIoU \\ 
%  \midrule
% \midrule
% \( m1 \) & 42.1 & 24.1 \\ \midrule
% \rowcolor{ours-highlight}\( m2 \) &\textbf{44.8} & \textbf{27.8} \\
% \( m3 \)  & 43.2 & 25.4  \\ % (as TimeChat)
% \( m4 \) &39.1 & 24.8 \\ % (as Grounded-VideoLLM)
% \bottomrule
% \end{tabular}
% }
% \end{center}
% \vspace{-0.3cm}
% \end{table}

\begin{table}
\centering

\vspace{-0.2cm}
\resizebox{0.95\columnwidth}{!}{%
\begin{tabular}{@{}c|cc|c|c|c|cc@{}}
\toprule
\multirow{2}{*}{Row} & \multirow{2}{*}{RT} & \multirow{2}{*}{AT} & \multirow{2}{*}{Add-VT} & \multirow{2}{*}{Add-Tokenizer} & \multirow{2}{*}{RST} & \multirow{2}{*}{\begin{tabular}[c]{@{}c@{}}C-STA\\ mIoU\end{tabular}} & \multirow{2}{*}{\begin{tabular}[c]{@{}c@{}}ANet-G\\ mIoU\end{tabular}} \\
                      &                     &                     &                         &                                &                      &                                                                       &                                                                        \\ \midrule \midrule
1                     &            \checkmark         &                     &                         &                                &                      & 44.9                                                                  & 26.3                                                                   \\
2                     &          \checkmark           &                     &        \checkmark                 &                                &                      & 46.6                                                                  & 29.2                                                                   \\
\rowcolor{ours-highlight} 3                     &         \checkmark            &                     &        \checkmark                 &                                &               \checkmark       & \textbf{47.3}                                                         & \textbf{29.7}                                                          \\
4                     &                     &        \checkmark             &         \checkmark                &                                &            \checkmark          & 45.7                                                                  & 28.4                                                                   \\
5                     &           \checkmark          &                     &         \checkmark                &                            \checkmark    &           \checkmark           & 44.4                                                                  & 27.6                                                                   \\ \bottomrule
\end{tabular}%
}
\vspace{-0.2cm}
\caption{Effect of Relative Spatiotemporal Position Tokens: RT and AT denote relative and absolute temporal representations. Add-VT appends the temporal token after each frame's visual tokens, while Add-Tokenizer incorporates temporal position tokens into the tokenizer. RST represents Relative Spatial Position Token.}
\label{tab:about_temporal_tokens}
\vspace{-0.4cm}
\end{table}


\textbf{About of Relative Spatiotemporal Position Tokens.} Table~\ref{tab:about_temporal_tokens} shows that row 2 and row1 indicate that adding corresponding temporal token after each frame's visual token is beneficial for achieving better model performance. The performance improvement of row 3 compared to row 4 demonstrates that relative temporal representation outperforms absolute temporal representation in our experiments. The performance decline of row 5 relative to row 3 suggests that introducing additional temporal markers in the LLM's tokenizer results in suboptimal outcomes. Meanwhile, the performance improvement of row 3 compared to row 2 proves the effectiveness of the Relative Spatial Position Token. In summary, these findings demonstrate the effectiveness of the Relative Spatiotemporal Position Tokens.




\begin{table}
\centering
% \captionsetup{font=scriptsize}

\vspace{-0.2cm}
\begin{center}
\resizebox{0.9\columnwidth}{!}{%
\begin{tabular}{@{}c|c|c|cc|c@{}}
\toprule
\multirow{2}{*}{Architecture} & C-STA & ANet-G & \multicolumn{2}{c|}{ANet-Cap} & LVBench \\ \cmidrule(l){2-6} 
                              & mIoU  & mIoU   & SODA\_c        & METEOR       & val     \\ \midrule \midrule
Phi3.5-V        & 45.5  & 29.2   & 3.3            & 7.2          & 53.4    \\ \midrule
\rowcolor{ours-highlight}InternVL2-4B                     & 47.3  & 29.7   & 3.4            & 6.8          & 54.7    \\ \bottomrule
\end{tabular}%
}
\end{center}
\vspace{-0.2cm}
\caption{Generalization Study. Phi3.5-V denotes Phi3.5-Vision-Instruct-3.8B.}
\label{tab:general}
\vspace{-0.4cm}
\end{table}

\textbf{Generalization Study}. To validate the generalization capability of the proposed method, we employ an image MLLM, Phi3.5-Vision-Instruct-3.8B~\citep{phi3v}, as our base MLLM. As shown in Table~\ref{tab:general}, using an image MLLM as the base achieves comparable performance to a short VideoLLM. Notably, it improves the METEOR score by 0.4 points on ActivityNet-Captions. This indicates that our model architecture and dataset not only enable short Video-LLMs to adapt to perceiving fine-grained spatiotemporal instance motions but are also applicable to image MLLMs.

\section{Conclusion}
% In this paper, we propose iMOVE for instance-level motion perception. Firstly, we construct an instance motion perception video instruction tuning dataset iMOVE-IT. This dataset includes three types of tasks: Spatial Grounding, Temporal Grounding, and Instance Dynamic Captioning. Temporal and spatial constraints are imposed on iMOVE. Additionally, we design an Event-aware Spatiotemporal Efficient Modeling strategy for visual encoding. Finally, iMOVE employs Relative Spatiotemporal Position Tokens to enhance the model's temporal capabilities. Overall, our iMOVE provides an effective design for MLLM by enhancing the model's ability to perceive fine-grained spatiotemporal instance motions, thereby improving the model's temporal understanding and long video comprehension capabilities, while also improving the model's general understanding ability. We hope that our iMOVE can provide insights for the general video understanding field.

% In this paper, we enhanced the fine-grained instance spatiotemporal motion perception in Video Large Language Models by making improvements from both data and model perspectives, thereby enhancing their capabilities in temporal and general video understanding. From the data perspective, we introduced \textbf{iMOVE-IT}, a large-scale instance-motion-aware video dataset with spatiotemporal mutual-supervision tasks, providing extensive training for the model's instance spatiotemporal motion perception.

% On the model side, we introduced \textbf{iMOVE}, an instance-motion-aware video foundation model. iMOVE employs Event-aware Spatiotemporal Efficient Modeling to retain informative instance spatiotemporal motion details while maintaining computational efficiency. Additionally, it incorporates Relative Spatiotemporal Position Tokens to ensure precise awareness of instance spatiotemporal positions.

% Our evaluations indicate that iMOVE excels in video temporal understanding and general video understanding, with significant advantages in long-term video understanding. This work provides valuable insights for future research in video understanding.

In this paper, we enhanced the fine-grained instance spatiotemporal motion perception in Video-LLMs by making improvements from both data and model perspectives, thereby boosting their capabilities in temporal and general video understanding. Data-wise, we propose \textbf{iMOVE-IT}, a large instance-motion-aware video dataset with spatiotemporal mutual-supervision tasks to enhance instance spatiotemporal motion learning. Model-wise, we develop \textbf{iMOVE} featuring Event-aware Spatiotemporal Efficient Modeling for efficiently preserving motion details, and Relative Spatiotemporal Position Tokens for accurate spatial-temporal positioning.

Evaluations indicate that iMOVE excels in video temporal understanding and general video understanding, with significant advantages in long-term video understanding, offering valuable insights for video understanding research.


% \section{Limitations}
% The selection of the number of events \(K\) in Event-aware Spatiotemporal Efficient Modeling is fixed. Although we believe that choosing 24 can cover the vast majority of videos and experiments have demonstrated its effectiveness, adapting the number of events for different videos is a promising way to enhance our method. We will explore this aspect in our future work. Secondly, the iMOVE-IT dataset proposed in this paper has achieved excellent results by enhancing the model's fine-grained instance spatiotemporal motion perception capabilities, thereby improving the model's temporal understanding and general video understanding. However, our primary focus is on the fine-grained spatiotemporal understanding task of bounding box coordinates. Exploring more fine-grained spatiotemporal tasks is a promising direction for future research.


% \section*{Acknowledgments}

% This document has been adapted
% by Steven Bethard, Ryan Cotterell and Rui Yan
% from the instructions for earlier ACL and NAACL proceedings, including those for
% ACL 2019 by Douwe Kiela and Ivan Vuli\'{c},
% NAACL 2019 by Stephanie Lukin and Alla Roskovskaya,
% ACL 2018 by Shay Cohen, Kevin Gimpel, and Wei Lu,
% NAACL 2018 by Margaret Mitchell and Stephanie Lukin,
% Bib\TeX{} suggestions for (NA)ACL 2017/2018 from Jason Eisner,
% ACL 2017 by Dan Gildea and Min-Yen Kan,
% NAACL 2017 by Margaret Mitchell,
% ACL 2012 by Maggie Li and Michael White,
% ACL 2010 by Jing-Shin Chang and Philipp Koehn,
% ACL 2008 by Johanna D. Moore, Simone Teufel, James Allan, and Sadaoki Furui,
% ACL 2005 by Hwee Tou Ng and Kemal Oflazer,
% ACL 2002 by Eugene Charniak and Dekang Lin,
% and earlier ACL and EACL formats written by several people, including
% John Chen, Henry S. Thompson and Donald Walker.
% Additional elements were taken from the formatting instructions of the \emph{International Joint Conference on Artificial Intelligence} and the \emph{Conference on Computer Vision and Pattern Recognition}.

% Bibliography entries for the entire Anthology, followed by custom entries
%\bibliography{anthology,custom}
% Custom bibliography entries only
\bibliography{custom}
\clearpage
\appendix

\section{More Details of \datasetname}
\label{sec:more_dataset_details}


\subsection{Data Statics of \datasetname}
\label{data_statics}

\begin{figure}
    \centering
    \subfigure{
        \includegraphics[width=0.45\textwidth]{figures/timeinternval_video_duration_histogram.pdf}
        \label{fig:subfig1}
    }
    \hfill
    \subfigure{
        \includegraphics[width=0.45\textwidth]{figures/timesinternval_video_time_histogram.pdf}
        \label{fig:subfig2}
    }    
    \caption{(a) Video duration visualization for iMOVE-IT. (b) Visualization of time intervals in iMOVE-IT.}
    \label{fig:subfigs}
\end{figure}

\paragraph{Diversity of Video Types}
The iMOVE-IT dataset contains 68387 unique videos from 11 different datasets, including InternVid-10M, CLEVRER, DiDeMo, NExT-QA, HACS, YT-Temporal-1B, COIN, TACoS, YouCook2, HiREST and ViTT, and does not include videos from the ActivityNet and Charades-STA datasets. The average duration of the videos is 109 seconds. The distribution of video durations in iMOVE-IT is shown in Figure ~\ref{fig:subfig1}. As can be seen from the figure, the distribution of video durations is quite broad, with the majority of videos being less than 60 seconds long. Videos longer than 60 seconds are primarily concentrated in the range of 100 to 160 seconds.


\paragraph{Diversity of Object Types} 
The iMOVE-IT dataset encompasses 114,705 objects, derived from 23,278 object categories, with an average of 5 objects per category. Specifically, the Spatial grounding task consists of 51,992 objects, the Temporal Grounding task includes 21,328 objects, and the Instance Dynamic Captioning task comprises 41,385 objects.

\paragraph{Diversity of time intervals} 
Figure ~\ref{fig:subfig2} depicts the distribution of time intervals in iMOVE-IT, with the majority of time intervals concentrated within 50 seconds. This is likely due to the fact that the video durations are primarily between 1 and 60 seconds. Additionally, the diversity in the distribution of time intervals contributes to enhancing the model's robustness.

From the above analysis, we can see that the iMOVE-IT dataset has significant advantages in terms of sample size, video types, object types, time intervals, and task instruction diversity. These characteristics make this dataset highly valuable for research and applications in related fields.


\clearpage
\subsection{Task Prompts of \datasetname}
\label{task_prompt}

% Prompts of iMOVE-TimeInterval-IT
\begin{center}
\begin{tcolorbox}[colback=gray!20, colframe=black, text width=0.9\textwidth, title={Prompts of iMOVE-IT}]
\textcolor{blue}{Spatial Grounding} \\
1. Please give the bounding box coordinates variation of the object depicted as \verb|<dynamic caption>| during the time interval from \verb|<t1>| to \verb|<t2>|. Output only the bounding box coordinates for the start and end times. \\
2. I would like to know the bounding box coordinates variation of the object described as \verb|<dynamic caption>| during the period from \verb|<t1>| to \verb|<t2>|. Just produce the bounding box coordinates for when it starts and ends. \\
3. Can you give me the bounding box coordinates change for the object depicted as \verb|<dynamic caption>| from \verb|<t1>| to \verb|<t2>|? Only the bounding box coordinates for the beginning and conclusion should be generated. \\
4. Kindly provide the variation in the bounding box coordinates of the object described as \verb|<dynamic caption>| from \verb|<t1>| to \verb|<t2>|. Simply output the bounding box coordinates that mark the start and end. \\
5. What is the change in the bounding box coordinates of the object depicted as \verb|<dynamic caption>| during the interval from \verb|<t1>| to \verb|<t2>|? Provide solely the bounding box coordinates that denote the start and end points. \\
6. Let me know the variation in the bounding box coordinates of the object described as \verb|<dynamic caption>| during the time span from \verb|<t1>| to \verb|<t2>|. Ensure the output includes only the bounding box coordinates for the start and end. \\
7. I need the bounding box coordinates change of the object depicted as \verb|<dynamic caption>| during the period from \verb|<t1>| to \verb|<t2>|. Limit the output to the bounding box coordinates of the start and end times. \\
8. Kindly tell me the change in the bounding box coordinates for the object depicted as \verb|<dynamic caption>| between \verb|<t1>| and \verb|<t2>|. You need to output solely the bounding box coordinates for the start and end times. \\

\textcolor{blue}{Temporal Grounding} \\
1. Please give the time interval when the object described as \verb|<dynamic caption>| transitions from the bounding box coordinates \verb|<bbox1>| to \verb|<bbox2>|. \\
2. Could you provide the time interval for the transition of the object, described as \verb|<dynamic caption>|, from the bounding box coordinates \verb|<bbox1>| to \verb|<bbox2>|? \\
3. What is the time interval when the object, depicted as \verb|<dynamic caption>|, shifts from the bounding box \verb|<bbox1>| to \verb|<bbox2>|? \\
4. Kindly indicate the time span for the object, described as \verb|<dynamic caption>|, transitioning from \verb|<bbox1>| to \verb|<bbox2>|. \\
5. Please let me know the duration of time when the object, with a description of \verb|<dynamic caption>|, moves from the coordinates \verb|<bbox1>| to \verb|<bbox2>|. \\
6. Identify the time interval for the object, described as \verb|<dynamic caption>|, as it transitions from \verb|<bbox1>| to \verb|<bbox2>|. \\
7. Could you specify the time period for the object, with a depiction of \verb|<dynamic caption>|, moving from \verb|<bbox1>| to \verb|<bbox2>|? \\
8. Please tell me the duration of time for the object, with a  portrayal of \verb|<dynamic caption>|, as it transitions from \verb|<bbox1>| to \verb|<bbox2>|. \\
\end{tcolorbox}
\end{center}
\clearpage

\begin{center}
\begin{tcolorbox}[colback=gray!20, colframe=black, text width=0.9\textwidth, title={Prompts of iMOVE-IT (continued)}]
\textcolor{blue}{Instance Dynamic Captioning} \\ 
1. Please describe the object within the time interval from \verb|<t1>| to \verb|<t2>|, with bounding box coordinates starting at \verb|<bbox1>| and ending at \verb|<bbox2>|. Include the object's appearance, changes, and behavior. \\
2. Could you describe the object within the time interval from \verb|<t1>| to \verb|<t2>|, with bounding box coordinates beginning at \verb|<bbox1>| and ending at \verb|<bbox2>|, and include its appearance, changes, and behavior? \\
3. Describe the object from \verb|<t1>| to \verb|<t2>|, with bounding box coordinates that begin at \verb|<bbox1>| and end at \verb|<bbox2>|, and include information on its appearance, alterations, and behavior. \\
4. Could you give a description of the object that is positioned between \verb|<t1>| and \verb|<t2>|, with bounding box coordinates commencing at \verb|<bbox1>| and finishing at \verb|<bbox2>|, and include its appearance, changes, and behavior? \\
5. Would you describe the object in the time span from \verb|<t1>| to \verb|<t2>|, with bounding box coordinates that begin at \verb|<bbox1>| and end at \verb|<bbox2>|, and mention its appearance, changes, and behavior? \\
6. Please provide a description of the object between \verb|<t1>| and \verb|<t2>|, with bounding box coordinates starting at \verb|<bbox1>| and ending at \verb|<bbox2>|, and include its appearance, changes, and behavior. \\
7. Describe the object from \verb|<t1>| to \verb|<t2>|, with bounding box coordinates initiating at \verb|<bbox1>| and concluding at \verb|<bbox2>|, and incorporate its appearance, variations, and behavior. \\
8. Could you portray the object from \verb|<t1>| to \verb|<t2>|, with bounding box coordinates that \break commence at \verb|<bbox1>| and finish at \verb|<bbox2>|, and include its appearance, changes, and \break behavior?
\end{tcolorbox}
\end{center}
\clearpage

\section{EXPERIMENTS}

\subsection{Detailed Dxperiment Setup}
\label{experiment_setup}
\begin{center}
\begin{table}[htbp]

\vskip 0.05in
\begin{tabular}{@{}lc@{}}
\toprule
Learning Rate                  & 1e-4             \\ 
LR Scheduler         & Cosine             \\
Global Batch Size              & 128              \\
Training Steps                 & 10K              \\
Warmup Ratio                   & 0.03                          \\
Trainable Modules              &     MLP Projector \& LoRA \\ 
Frame Resolution               & 448$\times$448               \\
Num Frames                & 96               \\ 
Train Epochs                  & 1 \\
Model Max Length                & 10000    \\
ZeRO Optimization              & Zero-2                       \\
Computation & 16 A800 \\
\bottomrule
\end{tabular}%
\caption{The hyper-parameters for iMOVE.}
\label{tab:training_settings}
\end{table}
\end{center}

We utilize InternViT-300M-448px~\cite{chen2024internvl} as the video encoder and Phi-3-mini-128k-instruct~\cite{abdin2024phi} as the LLM decoder. The parameters of equivalent components from InternVL2-4B, including linear projection layers, are used to initialize these models. The maximum value for temporal token quantization, denoted as \( Z \), is set to 300. For spatial tokens, the maximum quantization values for height \( \hat{H} \) and width \( \hat{W} \) are both set to 1000. We fine-tune the LLM using LoRA~\cite{lora}, while keeping the visual encoder frozen and making the MLP trainable. The LoRA parameters are set to \( r = 128 \) and \( \alpha = 256 \). We employ the AdamW~\cite{adamw} optimizer with a warm-up rate of 0.03. Initially, the video is divided into 96 segments, and one frame is randomly selected from each segment, resulting in 96 video frames. For each video, we select \( K = 24 \) events, setting the first frame feature's \( h \) and \( w \) of each event to 8, thus encoding with 64 tokens. The stride \( s \) is set to 2, meaning that non-first frames in each event are encoded with 16 tokens, resulting in a total of 2688 tokens per video, which is significantly fewer than the 4096 tokens per video encoding method in InternVL2-4B. Additional hyper-parameters can be found in Table~\ref{tab:training_settings}. All experiments are conducted on 16 A800 GPUs with a batch size of 128. The training set we used contains 1,276,977 data samples, and the experiment took a total of 41 hours to complete. Furthermore, in the construction of iMOVE-IT, we utilized the ViT-B/32 version of CLIP.


\begin{table*}

\resizebox{\textwidth}{!}{%
\begin{tabular}{@{}c|c|c@{}}
\toprule
\textbf{Task} & \textbf{\# of Entries} & \textbf{Datasets} \\ 
\midrule
\midrule
Temporal Sentence Grounding & 194K & DiDeMo, HiREST, QuerYD, VTG-IT-MR, TACOS \\
Temporal Action Localization & 45K & HACS \\
Dense Video Captioning & 61K & COIN, ViTT, YouCook2 \\
VideoQA & 493K & EgoQA, NExT-QA, Intent-QA, CLEVRER, LLAVA-Video-QA \\
Classification & 66K & SthSthV2, Kinetics \\
Video Captioning & 276K & YouCook2, WebVid-stage3, LLAVA-Video-cap \\
iMOVE-IT & 114K & self-collected \\ 
\bottomrule
\end{tabular}%
}
\caption{Dataset Composition.}
% \vskip 0.03in
\label{tab:train_datasets}
\end{table*}


\begin{table*}

\resizebox{\textwidth}{!}{%
\begin{tabular}{@{}c|c|c@{}}
\toprule
\textbf{Task} & \textbf{\# of Entries} & \textbf{Datasets} \\ 
\midrule
\midrule
Temporal Sentence Grounding & 194K & DiDeMo, HiREST, QuerYD, VTG-IT-MR, TACOS \\
Temporal Action Localization & 45K & HACS \\
Dense Video Captioning & 61K & COIN, ViTT, YouCook2 \\
\bottomrule
\end{tabular}%
}
\caption{Dataset Composition of the ablation study on Event-aware Spatiotemporal Efficient Modeling.}
% \vskip 0.03in
\label{tab:train_datasets_abla_study}
\end{table*}


\subsection{Baseline and Comparison}
\label{appendix:baseline}
For the baseline models, the comparison primarily involves two categories: the first is the zero-shot video large language models, and the second is the expert models obtained through fine-tuning. For the first type of baseline method, we choose InternVL2-4B\citep{InternVL2024}, Video-ChatGPT~\cite{video-chatgpt}, VideoChat~\cite{videochat}, Momentor~\cite{momentor}, TimeChat~\cite{timechat}, VTG-LLM~\cite{vtgllm}, HawkEye~\cite{hawkeye}, PiTe \citep{pite}, Grounded-VideoLLM \citep{wang2024groundedvideollm}, VTimeLLM~\cite{vtimellm}, TimeSuite~\cite{zeng2024timesuite}, and TRACE~\cite{trace}. We selected Vid2Seq\citep{yang2023vid2seqlargescalepretrainingvisual} as the classic supervised expert model on the Charades-STA dataset. For the ActivityNet-Captions, we chose QD-DETR\citep{moon2023querydependentvideorepresentationmoment} and UnLoc-L\citep{yan2023unlocunifiedframeworkvideo} as the supervised expert models. Additionally, we report the results of some zero-shot models after fine-tuning.


\subsection{Detailed Data Filtering and Dataset Composition}\label{data_filtering}

In terms of data quality control, we implemented stringent measures. We excluded datasets such as STAR, ANet-RTL, VCG-Plus112K, Videochatgpt-100K, Videochat2-Conv, and TextVR to ensure a strict zero-shot setting on the Charades-STA and ActivityNet-Captions datasets. Additionally, the LLAVA-Video-cap and LLAVA-Video-QA datasets we used are subsets extracted from LLaVA-Video-178K, excluding Charades, ActivityNet, and Ego4D video sources, thus preventing potential data leakage. The data we utilized includes tasks such as Temporal Sentence Grounding, Temporal Action Localization, Dense Video Captioning, VideoQA, Classification, Video Captioning, and iMOVE-IT, with the detailed composition of each task shown in Table~\ref{tab:train_datasets}.




\subsection{Detailed Benchmarks and Evaluation Metrics}
\label{benchmark}
iMOVE is comprehensively evaluated across the following four tasks:

Temporal Video Grounding: This task aims to determine the temporal boundaries of a single event based on a text description. The datasets \textbf{Charades-STA}~\citep{charades-sta} and \textbf{ActivityNet-Captions}~\citep{activitynet} are used for evaluation. For this task, we report the Intersection over Union (IoU) between the predicted timestamps by the model and the ground truth annotations. Specifically, we calculate \textbf{Recall at IoU} thresholds of \{0.3, 0.5, 0.7\} and their \textbf{mean IoU}.

Dense Video Captioning: This task is more complex, requiring the joint localization of key events and the generation of descriptions for each segment. The \textbf{ActivityNet-Captions} dataset is used for evaluation. We report \textbf{SODA\_c}~\cite{fujita2020soda}, which is specifically tailored for the video's storyline, and \textbf{METEOR}~\cite{banerjee2005meteor}, which is the average of traditional METEOR scores calculated based on matched pairs between generated events and the ground truth across IoU thresholds of \{0.3, 0.5, 0.7, 0.9\}.

General Video Understanding: This task aims to evaluate the general short-term video understanding capabilities of iMOVE. We utilize \textbf{Video-MME}~\cite{videomme} and \textbf{MVBench}~\cite{mvbench}, reporting their average accuracy.

Long-term Video Understanding: This task aims to evaluate the long-term video understanding capabilities of video models. We utilize the \textbf{LongVideoBench}~\cite{wu2024longvideobench} and the Long subset of \textbf{Video-MME}, reporting their average accuracy.

\subsection{Explanation of Methods on ActivityNet-Captions That Are Not Strictly Zero-Shot Setting
}\label{data_leakage}
Due to potential data leakage, some methods on ActivityNet-Captions do not strictly adhere to zero-shot settings. Below is a detailed explanation:

\textbf{PiTe} utilized the Video-ChatGPT~\cite{video-chatgpt} dataset in Stage 3, which was constructed using ActivityNet~\cite{activitynet} as the source. \par
\textbf{Grounded-VideoLLM} employed ANet-RTL, VCG-Plus-112K~\cite{videogpt+}, Videochatgpt-100K, and Videochat2-Conv in Stage 3, all of which were constructed using ActivityNet as the source. Additionally, the TextVR~\cite{wu2025large} used in this method also utilized videos from ActivityNet.  \par

\textbf{HawkEye} also utilizes VideoChatGPT and TextVR, resulting in non-strict zero-shot settings on the ActivityNet-Captions dataset. \par

\textbf{VTimeLLM} directly used the ActivityNet Captions~\cite{krishna2017dense} dataset as the training set in Stage 3.

\subsection{Data Composition of Ablation Study}\label{dataset_composition_abl}

 Due to computational resource considerations, we conducted the ablation experiment for Event-aware Spatiotemporal Efficient Modeling using only a subset of the entire dataset related to temporal understanding. As shown in Table~\ref{tab:train_datasets_abla_study}, this includes the tasks of Temporal Sentence Grounding, Temporal Action Localization, and Dense Video Captioning.

\section{Qualitative analysys}
\label{case_study}

\begin{figure}
    \centering
    \includegraphics[width=1.0\linewidth]{figures/temporal_grounding_case_study_compress.pdf}
    \caption{
    % Current Video-LLMs struggle to accurately perceive instance-level spatiotemporal motions within videos.
     {Qualitative comparison of the temporal grounding capabilities of iMOVE and InternVL2-4B.}
    }
    \label{fig:qua_temporal_grounding}
    \vspace{-0.5cm}
\end{figure}



We provide a detailed qualitative comparison between iMOVE and InternVL2-4B in terms of temporal grounding, dense captioning, and long-term video understanding.

\textbf{Qualitative Comparison in Temporal Grounding.} As shown in Figure~\ref{fig:qua_temporal_grounding}, iMOVE accurately identifies the time interval of the event "the person was putting the bag into the cabinet" from a video containing multiple events, whereas InternVL2-4B fails to pinpoint the specific video segment where the event occurs. This demonstrates iMOVE's strong fine-grained temporal perception capability.

\begin{figure}
    \centering
    \includegraphics[width=1.0\linewidth]{figures/dense_caption_case_study_compress.pdf}
    \caption{
    % Current Video-LLMs struggle to accurately perceive instance-level spatiotemporal motions within videos.
     {Qualitative comparison of the dense captioning capabilities of iMOVE and InternVL2-4B.}
    }
    \label{fig:qua_dense_caption}
    \vspace{-0.5cm}
\end{figure}


\textbf{Qualitative Comparison in Dense Captioning.} As illustrated in Figure~\ref{fig:qua_dense_caption}, iMOVE effectively captures the complete storyline of the video, accurately identifying the time intervals of various events and providing precise event descriptions. In contrast, InternVL2-4B struggles to correctly comprehend multiple events within the video, resulting in repetitive event descriptions.


\textbf{Qualitative Comparison in Long-term Video Understanding.} As depicted in Figure~\ref{fig:qua_long_video}, thanks to its meticulously designed architecture and dataset, iMOVE accurately answers reasoning questions in long videos. iMOVE first locates the segments described in the questions within the long video and determines the key characteristics of the man based on the video content, using them as crucial clues when he appears in another video segment. As a short video large language model, InternVL2-4B fails to provide correct answers.

\begin{figure}
    \centering
    \includegraphics[width=1.0\linewidth]{figures/long_video_case_study_compress.pdf}
    \caption{
    % Current Video-LLMs struggle to accurately perceive instance-level spatiotemporal motions within videos.
     {Qualitative comparison of the long-term video understanding capabilities of iMOVE and InternVL2-4B.}
    }
    \label{fig:qua_long_video}
    \vspace{-0.5cm}
\end{figure}

In summary, iMOVE, with its carefully designed architecture based on iMOVE-IT, adapts a short video large language model to perceive fine-grained spatiotemporal instance motions, significantly enhancing its temporal understanding, general video understanding and long-term video understanding capabilities. This improvement in general video understanding offers a substantial advantage over previous temporal Video-LLMs, which excelled in temporal understanding but lacked in general video understanding capabilities.


% \section{Potential Risks}
% Our approach relies on the feedback from existing MLLMs, which may inherit inherent biases. This could pose potential risks to users.

\end{document}
