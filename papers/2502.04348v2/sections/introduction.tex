\section{Introduction}

%\setlength{\intextsep}{10pt}

Recent advances in large language models (LLMs) have achieved remarkable success in a wide range of natural language processing tasks \cite{brown2020language,touvron2023llama,dubey2024llama}. However, significant computational requirements of LLMs pose challenges in resource-constrained environments, limiting their practicality. For example, LLaMA-3.3-70B needs 140GB of RAM to be loaded in bf16, which is often too big for memory-constrained local devices. Thus, reducing the model size is essential to make LLMs feasible for on-device applications.

Depth pruning is a versatile model compression technique that is particularly effective for on-device scenarios \cite{songsleb,kim2024shortened}. Such methods simply remove several transformer blocks (which we call ``omission set'') from the pretrained model, based on some measures of block importance computed using a small amount of calibration samples. As everything is identical except for the number of blocks, the pruned model is suitable to be deployed on any hardware without tailored supports on low-precision (\textit{e.g.}, integer cores) or fine-grained sparsity (\textit{e.g.}, 2:4 sparsity). Furthermore, as there is no extensive training involved, depth pruning can be easily done in a device-by-device manner for deployment on various devices.

\section{Introduction}
Backdoor attacks pose a concealed yet profound security risk to machine learning (ML) models, for which the adversaries can inject a stealth backdoor into the model during training, enabling them to illicitly control the model's output upon encountering predefined inputs. These attacks can even occur without the knowledge of developers or end-users, thereby undermining the trust in ML systems. As ML becomes more deeply embedded in critical sectors like finance, healthcare, and autonomous driving \citep{he2016deep, liu2020computing, tournier2019mrtrix3, adjabi2020past}, the potential damage from backdoor attacks grows, underscoring the emergency for developing robust defense mechanisms against backdoor attacks.

To address the threat of backdoor attacks, researchers have developed a variety of strategies \cite{liu2018fine,wu2021adversarial,wang2019neural,zeng2022adversarial,zhu2023neural,Zhu_2023_ICCV, wei2024shared,wei2024d3}, aimed at purifying backdoors within victim models. These methods are designed to integrate with current deployment workflows seamlessly and have demonstrated significant success in mitigating the effects of backdoor triggers \cite{wubackdoorbench, wu2023defenses, wu2024backdoorbench,dunnett2024countering}.  However, most state-of-the-art (SOTA) backdoor purification methods operate under the assumption that a small clean dataset, often referred to as \textbf{auxiliary dataset}, is available for purification. Such an assumption poses practical challenges, especially in scenarios where data is scarce. To tackle this challenge, efforts have been made to reduce the size of the required auxiliary dataset~\cite{chai2022oneshot,li2023reconstructive, Zhu_2023_ICCV} and even explore dataset-free purification techniques~\cite{zheng2022data,hong2023revisiting,lin2024fusing}. Although these approaches offer some improvements, recent evaluations \cite{dunnett2024countering, wu2024backdoorbench} continue to highlight the importance of sufficient auxiliary data for achieving robust defenses against backdoor attacks.

While significant progress has been made in reducing the size of auxiliary datasets, an equally critical yet underexplored question remains: \emph{how does the nature of the auxiliary dataset affect purification effectiveness?} In  real-world  applications, auxiliary datasets can vary widely, encompassing in-distribution data, synthetic data, or external data from different sources. Understanding how each type of auxiliary dataset influences the purification effectiveness is vital for selecting or constructing the most suitable auxiliary dataset and the corresponding technique. For instance, when multiple datasets are available, understanding how different datasets contribute to purification can guide defenders in selecting or crafting the most appropriate dataset. Conversely, when only limited auxiliary data is accessible, knowing which purification technique works best under those constraints is critical. Therefore, there is an urgent need for a thorough investigation into the impact of auxiliary datasets on purification effectiveness to guide defenders in  enhancing the security of ML systems. 

In this paper, we systematically investigate the critical role of auxiliary datasets in backdoor purification, aiming to bridge the gap between idealized and practical purification scenarios.  Specifically, we first construct a diverse set of auxiliary datasets to emulate real-world conditions, as summarized in Table~\ref{overall}. These datasets include in-distribution data, synthetic data, and external data from other sources. Through an evaluation of SOTA backdoor purification methods across these datasets, we uncover several critical insights: \textbf{1)} In-distribution datasets, particularly those carefully filtered from the original training data of the victim model, effectively preserve the model’s utility for its intended tasks but may fall short in eliminating backdoors. \textbf{2)} Incorporating OOD datasets can help the model forget backdoors but also bring the risk of forgetting critical learned knowledge, significantly degrading its overall performance. Building on these findings, we propose Guided Input Calibration (GIC), a novel technique that enhances backdoor purification by adaptively transforming auxiliary data to better align with the victim model’s learned representations. By leveraging the victim model itself to guide this transformation, GIC optimizes the purification process, striking a balance between preserving model utility and mitigating backdoor threats. Extensive experiments demonstrate that GIC significantly improves the effectiveness of backdoor purification across diverse auxiliary datasets, providing a practical and robust defense solution.

Our main contributions are threefold:
\textbf{1) Impact analysis of auxiliary datasets:} We take the \textbf{first step}  in systematically investigating how different types of auxiliary datasets influence backdoor purification effectiveness. Our findings provide novel insights and serve as a foundation for future research on optimizing dataset selection and construction for enhanced backdoor defense.
%
\textbf{2) Compilation and evaluation of diverse auxiliary datasets:}  We have compiled and rigorously evaluated a diverse set of auxiliary datasets using SOTA purification methods, making our datasets and code publicly available to facilitate and support future research on practical backdoor defense strategies.
%
\textbf{3) Introduction of GIC:} We introduce GIC, the \textbf{first} dedicated solution designed to align auxiliary datasets with the model’s learned representations, significantly enhancing backdoor mitigation across various dataset types. Our approach sets a new benchmark for practical and effective backdoor defense.




A key limitation of typical depth pruning algorithms is that their pruning decision is \textbf{\textit{static}}, \textit{i.e.}, the same omission set is removed regardless of the query given to the model. While this choice allows one to save storage (\textit{e.g.}, flash drives) by discarding the pruned parameters at the local device, it sacrifices the ability to adapt to various downstream tasks. Indeed, our empirical observations show that pruning some transformer blocks in an LLM may incur significant accuracy degradation on certain tasks, while being highly unnecessary for other tasks (see \cref{sec:observation}).


Can we make dynamic depth pruning decisions to improve the performance on various tasks? This question has not been well studied yet, especially in the context of on-device inference. A recent line of work develops effective \textbf{\textit{dynamic token routing}} mechanisms to save training/inference computation by processing each token with a limited number of transformer blocks \cite{raposo2024mixture,wangd}. However, such methods require all parameters to be loaded on high-speed memories (\textit{e.g.}, on-GPU memory); thus, the methods are appropriate for large-scale server clusters, not for on-device inference with memory constraints.

\textbf{Contribution.} To overcome the limitations, we develop a new \textbf{\textit{prompt-based depth pruning}} approach (\cref{sec:formulation}): In the pre-fill stage, based on the prompt given from the user, a limited number of transformer blocks are selected and loaded to the on-device RAM from the storage drive. This approach does not require a large memory to hold all parameters or highly repeated per-token routing, and thus can effectively accelerate inference on low-memory devices.

A na\"{i}ve way to achieve this goal might be to conduct conventional static depth pruning at each inference, using the given prompt as calibration samples. However, this approach incurs a large latency in running static pruning algorithms in every inference. Furthermore, such a method is likely to fail making a good pruning decision due to the shortage of calibration data, especially in single-batch inference cases common in on-device scenarios.

To this end, we propose a \textit{training-based} method for the prompt-based depth pruning of large langauge models (\cref{sec:method}). Our method, coined \underline{P}rompt-ro\underline{u}ted \underline{D}ynamic \underline{D}epth Prun\underline{ing} (PuDDing), works in two steps. 
\begin{enumerate}[leftmargin=*,topsep=0pt,parsep=0pt,itemsep=1.5pt]
\item \textit{Candidate omission set generation}. We construct a small yet diverse and performant family of omission sets. This is done by drawing multiple splits of calibration data from various task dataset, and then finding an omission set which achieves low loss on each split; here, we use a newly developed task-centric loss instead of perplexity.
\item \textit{Router training.} We train a lightweight router which predicts the appropriate omission set from the given prompt. This is done by generating a training dataset consisting of prompt-loss pairs for each omission set, and training the model to predict the loss from the prompt; routing can be done by choosing the minimum-loss option.
\end{enumerate}

Empirically, we find that the proposed PuDDing enjoys a clear advantage over static depth pruning algorithms, achieving more than 4\%p accuracy increase on zero-shot commonsense reasoning tasks (\cref{sec:experiments}). At the same time, as the algorithm uses the router only once per each prompt, PuDDing enjoys over 1.2$\times$ generation speedup over the dense model, similar to the static depth pruning algorithms.

Our key contributions can be summarized as follows:
\begin{itemize}[leftmargin=*,topsep=0pt,parsep=0pt,itemsep=1.5pt]
\item Our observations reveal that optimal depth pruning decisions may be highly depend on the task given at hand, underscoring the need for task-dependent depth pruning.
\item We consider the task of prompt-based depth pruning for the first time (to our knowledge), and propose a training-based strategy as a solution.
\item Comparing with static depth pruning algorithms, our algorithm achieves a much higher zero-shot accuracies on various tasks, while being competitive in terms of the computational efficiency.
\end{itemize}
