
\section{A Motivating Observation}\label{sec:observation}

Before describing the proposed framework, we briefly describe a motivating observation which demonstrate that:

\begin{tcolorbox}[boxsep=0pt,colback=black!5 , before skip=10pt, after skip=10pt]
The importance of a transformer block in a language model may be highly \textbf{\textit{task-dependent}}.
\end{tcolorbox}

\textbf{Setup.} To show this point, we have compared the zero-shot accuracies of the LLMs whose omission sets differ by a single transformer block. More concretely, we compare the performance of an omission set $(b_1,b_2,\ldots,b_{k-1},b_k)$ to another omission set $(b_1,b_2,\ldots,b_{k-1},\tilde{b}_k)$, on the LLaMA 3.1-8B model. Here, we have used the SLEB \citep{songsleb} to generate an omission set, and then replaced a single block to get another one. Then, we observe the impact of such replacement on three commonsense reasoning tasks: BoolQ, PIQA, and WinoGrande.

\textbf{Result.} \cref{Impact on pruning block 29 on BoolQ} illustrates our findings. We observe that pruning out block 29 instead of block 30 has a two-sided impact: On BoolQ, the change makes a dramatic drop in accuracy (62.2\% $\to$ 38.0\%, 62.5\% $\to$ 37.9\%). However, on PIQA and WinoGrande, we observe a slight accuracy boost. This phenomenon suggests that the block 30 may contain more knowledge relevant to answering BoolQ questions, while 29 may be more knowledgeable about PIQA and WinoGrande. This observation highlights the need to consider task variability during the selection of the omission set. To formally address such need, this paper considers an inference of task information from the prompt.




\subsection{DagRider\label{ssec:dagrider}}




\begin{figure}[h]
\vspace*{-.4cm}
    \centering
\scalebox{.875}{\begin{tikzpicture}
\tikzset{
    protocol_action/.style={
        draw,align=center,minimum height=1cm
        },
    buffer_memory/.style={
        draw,align=center,ellipse
        },
}
%
% == PROTOCOL ACTIONS
%
\node[protocol_action] (rec_tr) at (0,0) {receive\\transaction};
\node[protocol_action,right=.75cm of rec_tr] (prop_vert) {propose\\vertex};
%
\node[protocol_action,below=1.75cm of prop_vert] (add_todag) {add to\\DAG};
\node[protocol_action,above right=-.6cm and 1.5cm of add_todag] (rec_vert) {receive\\vertex};
%
%
\node[draw, right=.5cm of rec_vert] (r_deliver) {$\mathtt{rdlver}$};
\node[draw, right=3.15cm of prop_vert] (r_bcast) {$\mathtt{rbcast}$};
%
\node[protocol_action,below right=.8cm and -.8cm of rec_vert] (sel_leader) {select\\leader};
\node[protocol_action,left=.6cm of sel_leader] (get_wave) {get\\wave};
\node[protocol_action,left=.6cm of get_wave] (order_vert) {order\\wave}; 
\node[protocol_action,below=3cm of rec_tr] (del_tr) {finalize\\transactions};
%
%
\node[draw, right=.3cm of sel_leader] (c_flip) {$\mathtt{cflip}$};
%
%
% == MEMORY
%
\node[buffer_memory,below left=.55cm and .5cm of prop_vert] (bf_tr) {\footnotesize transactions\\\footnotesize mempool};
\node[buffer_memory,below left=.3cm and .2cm of r_bcast] (bf_vert) {\footnotesize buffered\\\footnotesize vertices};
\node[buffer_memory,below right=.6cm and 0cm of prop_vert] (bf_dag) {\footnotesize DAG};
%
% == OPERATIONS
%
\draw (bf_tr.150 |- rec_tr.south) edge[->,red]  node[right] {\scriptsize \ul{push}} (bf_tr.150);
\draw (prop_vert) edge[->,red] node[below right] {\scriptsize \ul{extract}} (bf_tr);
\draw (prop_vert) edge[->,red] node[right] {\scriptsize \ul{lookup}} (bf_dag);
\draw (add_todag) edge[->,red]  node[below left] {\scriptsize \ul{filter-out}} (bf_tr);
%
\draw (rec_vert.120) edge[->,red] node[right,pos=.4] {\scriptsize \ul{push}} (bf_vert.220);
\draw (add_todag) edge[->,red] node[above,pos=.75] {\scriptsize \ul{pop}} (bf_vert);
\draw (add_todag) edge[->,red] node[left] {\scriptsize \ul{add}} (bf_dag);
%
\node[inner sep=0pt] (anchor_lookup_dag_split) at ($(bf_dag.south) + (0,-1.9)$) {};
\draw (bf_dag.south) edge[<-,red] node[right,pos=.9] {\scriptsize \ul{lookup}} (anchor_lookup_dag_split.south);
\draw (anchor_lookup_dag_split) edge[red] (sel_leader.110 |- anchor_lookup_dag_split.south);
\draw (sel_leader.110) edge[red] (sel_leader.110 |- anchor_lookup_dag_split.south);
\draw (anchor_lookup_dag_split) edge[red] (order_vert.north |- anchor_lookup_dag_split.south);
\draw (order_vert.north) edge[red] (order_vert.north |- anchor_lookup_dag_split.south);
\draw (get_wave.north) edge[red] (get_wave.north |- anchor_lookup_dag_split.south);
%
% == TRIGGERS
%
\draw (rec_tr) edge[dashed,->] (prop_vert);
\draw (add_todag) edge[dashed,->] (prop_vert);
%
\draw (prop_vert) edge[->] node[above] {\scriptsize $V$} (r_bcast);
\draw (rec_vert.west |- add_todag.20) edge[dashed,->] (add_todag.20);
%
\draw (add_todag) edge[loop left,dashed,->] (add_todag);
%
\draw (sel_leader) edge[->] node[above] {\scriptsize $V$} (get_wave);
%
\draw (get_wave) edge[->] node[above] {\scriptsize $\{V\}$} (order_vert);
\draw (order_vert.west) edge[->] node[above] {\scriptsize $[X]$} (del_tr.east |- order_vert.west);
%
%
\node[inner sep=0pt] (anchor_add_to_dag_right) at (sel_leader.80 |- add_todag.330) {};
\draw (add_todag.330) edge[dashed] (anchor_add_to_dag_right);
\draw (anchor_add_to_dag_right) edge[->,dashed] (sel_leader.80);
%
%
% == OTHER LAYERS
%
%
\draw (r_deliver) edge[->,dashed] (rec_vert);
\draw (sel_leader) edge[<->,dashed] (c_flip);
%
% == EXTERNAL COMMS
%
\draw ($(r_deliver.east) + (.3,0)$) edge[->] node[above,pos=.3] {\scriptsize $V$} (r_deliver);
\draw (r_bcast.east) edge[->] node[above,pos=.7] {\scriptsize $V$} ($(r_deliver.east |- r_bcast.east) + (.3,0)$);
%
\draw ($(rec_tr.west) + (-.3,0)$) edge[->] node[above,pos=.3] {\scriptsize $X$} (rec_tr);
%
\draw (c_flip.east) edge[<->] ($(r_deliver.east |- c_flip.east) + (.3,0)$);
%
% == LAYERS COLORATION
%
\node[inner sep=0pt] (anchor_full_top) at ($(rec_tr.north) + (0,.1)$) {};
\node[inner sep=0pt] (anchor_full_left) at ($(del_tr.west) + (-.25,0)$) {};
\node[inner sep=0pt] (anchor_full_right) at ($(r_bcast.east) + (.25,0)$) {};
\node[inner sep=0pt] (anchor_full_bottom) at ($(sel_leader.south) + (0,-.1)$) {};
%
\node[inner sep=0pt] (anchor_application_right) at ($(rec_tr.east) + (.25,0)$) {};
%
\node[inner sep=0pt] (anchor_broadcast_left) at ($(r_bcast.west) + (-.125,0)$) {};
%
\node[inner sep=0pt] (anchor_ordering_top) at ($(sel_leader.north) + (0,.125)$) {};

\begin{scope}[on background layer]
\path[fill=yellow,opacity=.5] 
    (anchor_full_left |- anchor_full_top) 
    -- (anchor_application_right |- anchor_full_top) 
    -- (anchor_application_right |- anchor_full_bottom) 
    -- (anchor_full_left |- anchor_full_bottom) ;
\path[fill=green,opacity=.5] 
    (anchor_application_right |- anchor_full_top) 
    -- (anchor_broadcast_left |- anchor_full_top) 
    -- (anchor_broadcast_left |- anchor_ordering_top) 
    -- (anchor_application_right |- anchor_ordering_top)  ;
\path[fill=darkspringgreen,opacity=.5] 
    (anchor_application_right |- anchor_ordering_top) 
    -- (anchor_broadcast_left |- anchor_ordering_top) 
    -- (anchor_broadcast_left |- anchor_full_bottom) 
    -- (anchor_application_right |- anchor_full_bottom)  ;
\path[fill=blue,opacity=.5] 
    (anchor_broadcast_left |- anchor_full_top) 
    -- (anchor_full_right |- anchor_full_top) 
    -- (anchor_full_right |- anchor_ordering_top) 
    -- (anchor_broadcast_left |- anchor_ordering_top)  ;
\path[fill=orange,opacity=.5] 
    (anchor_broadcast_left |- anchor_ordering_top) 
    -- (anchor_full_right |- anchor_ordering_top) 
    -- (anchor_full_right |- anchor_full_bottom) 
    -- (anchor_broadcast_left |- anchor_full_bottom)  ;
\end{scope}
%
% == LEGEND
%
\node[align=center] (leg1) at (.3,-5.1) {\small$\square$ action, $\circ$ memory}; 
%
\node[below=.25cm of leg1.south west,anchor=west,align=center] (leg2) {\small\shortColRed{$\rightarrow$} \ul{operation}, $\dashrightarrow$ may trigger}; 
%
\node[below=.25cm of leg2.south west,anchor=west,align=center] (leg3) {\small$\xrightarrow{Z}$ calls with value of type $Z$}; 
%
\node[below=.25cm of leg3.south west,anchor=west,align=center] (leg4) {\small `$X$' transaction, `$V$' vertex}; 
%
\node[right=1.5cm of leg1,align=center] (dr_leg0) {\small\textcolor{yellow}{$\blacksquare$} \textbf{Application layer}}; 
\node[below=.25cm of dr_leg0.south west,anchor=west,align=center] (dr_leg) {\small\textcolor{green}{$\blacksquare$}/\textcolor{darkspringgreen}{$\blacksquare$} \textbf{DagRider layers}};
\node[below=.25cm of dr_leg.south west,anchor=west,align=center] (rb_leg) {\small\textcolor{blue}{$\blacksquare$} \textbf{Reliable Broadcast layer}};  
\node[below=.25cm of rb_leg.south west,anchor=west,align=center] (gc_leg) {\small\textcolor{orange}{$\blacksquare$} \textbf{Global Coin layer}};
\end{tikzpicture}}
    \caption{Description of DagRider}
    \label{fig:layers_dagrider}
\vspace*{-.4cm}
\end{figure}



In \cite{all_you_need_is_dag}, DagRider is described via pseudocode.
For concision, we rather describe it via the atomic actions a DagRider node may perform and how these actions are triggered by each other or by external stimuli (c.f.~actor model \cite{a_universal_modular_actor_formalism_for_artificial_intelligence}).
Fig.\ref{fig:layers_dagrider} describes the behavior of any individual node. 
The rectangles and ellipses resp.~correspond to atomic actions it may execute and to persistent information stored in its memory.
The red arrows describe the effects of the actions on the memory.
Plain arrows carrying a symbol $Z$ correspond to calling an action on a value of type $Z$. 
%, carrying a symbol $Z$, corresponds to the completion of an action transferring a temporary value of type $Z$ to another action which is then executed.
Dotted arrows signify that the completion of an action may trigger the execution of another.
The colored areas correspond to layers of abstraction.
%that separate responsibilities for the implementation of the node's behavior.


In the application layer, ``receive transaction'' is triggered whenever a node receives a new transaction $x$ of type $X$. 
This results in $x$ (if not duplicated) being \ul{pushed} in a ``transactions mempool'' buffer.
This may in turn trigger the proposal of a new vertex by the node $r$ if the following condition is met: the node is at column $c$ (i.e, it has already broadcast vertices for all $c'<c$ and has yet to do so for $c$) and there are at least $2*f+1$ distinct vertices at column $c-1$ in its local copy of the DAG (which can be verified via \ul{lookup} of the ``DAG''). 
In that case, ``propose vertex'' creates a new vertex $v_c^r$ that contains transactions \ul{extracted} from ``buffered transactions'', has at least $2*f+1$ strong edges and at most $f$ weak edges.
Details in Appendix \ref{anx:bug_dagrider}.
%, we further discuss weak edges and an oversight in the original DagRider paper.


In the reliable broadcast layer (in blue on Fig.\ref{fig:layers_dagrider}), ``$\mathtt{rbcast}$'' and ``$\mathtt{rdlver}$'' resp.~correspond to triggering the broadcast and delivering the vertex.
As per \cite{all_you_need_is_dag}, the broadcast must guarantee: \textbf{agreement} i.e., if a correct node delivers a vertex $v$ then, eventually, all the other correct nodes will also deliver $v$, \textbf{integrity} i.e., for each round, only one vertex can be delivered per node and \textbf{validity} i.e., if a correct node broadcast a correct vertex $v$ then, eventually, every correct node will deliver $v$.
These properties are s.t.~we may safely ignore issues related to duplicated or equivocated vertices.


Once ``propose vertex'' outputs a vertex $v$, it is reliably broadcast to the $n$ nodes of the network via $\mathtt{rbcast}$.
Then, eventually, $\mathtt{rdlver}$ triggers ``receive vertex'' which \ul{pushes} $v$ in a set of ``buffered vertices''.
If all vertices $v'$ that are targets of the edges of $v$ are already present in the node's local copy of the DAG, then ``add to DAG'' is triggered, which causes the node to remove vertices from ``buffered vertices'' and to insert them at their correct position in its local copy of the DAG.
The reception of a single vertex may have cascading effects and cause the addition of a large number of vertices to the DAG.
Whenever a vertex is added to the DAG, the transactions it contains are \ul{filtered-out} from ``transactions mempool'', so that they won't be included in subsequent vertex proposals (thus purging the mempool).


As the local copy of the DAG grows, columns are progressively filled, which allows constructing the waves.
The first step is ``select leader'', which outputs the leader of wave $w \geq 0$ (at column $1 + w*4$). 
Its determination corresponds to the selection of a node/row.
So that the adversary cannot predict it and tamper with the broadcast of the leader vertex (via e.g., Denial of Service \cite{a_framework_for_classifying_denial_of_service_attacks}), this is done via a global perfect coin. 
This protocol (orange layer on Fig.\ref{fig:layers_dagrider}) provides a ``$\mathtt{cflip}$'' primitive that eventually returns the identity of the leader once at least $f+1$ distinct nodes called $\mathtt{cflip}$. 
This may be implemented via threshold signatures \cite{practical_threshold_signatures}.


Once a leader is selected at column $c = 1 + w*4$, the algorithm first verifies that there are at least $2*f+1$ strong paths between it and vertices at column $4 + w*4$. 
If this is not the case, the leader is skipped.
Otherwise, a new wave is formed, containing all the not-yet ordered vertices in the causal sub-graph of the leader.
This corresponds to the output of ``get wave'' on Fig.\ref{fig:layers_dagrider}.
As per \cite{all_you_need_is_dag}, the vertices of the wave are then ordered using a certain deterministic order ``order wave''.
This results in a total ordering of the transactions, which are then finalized.




