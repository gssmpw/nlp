\begin{abstract}
Depth pruning aims to reduce the inference cost of a large language model without any hardware-specific complications, by simply removing several less important transformer blocks. However, our empirical findings suggest that the importance of a transformer block may be highly task-dependent---a block that is crucial for a task can be removed without degrading the accuracy on another task. Based on this observation, we develop a dynamic depth pruning algorithm, coined PuDDing (\underline{P}rompt-ro\underline{u}ted \underline{D}ynamic \underline{D}epth Prun\underline{ing}), which determines which blocks to omit from the model based on the input prompt. PuDDing operates by training a lightweight router to predict the best omission set among a set of options, where this option set has also been constructed in a data-driven manner. Empirical results on commonsense reasoning benchmarks demonstrate that PuDDing effectively accelerates the inference language models, and achieves better on-task performance than static depth pruning baselines.

\end{abstract}