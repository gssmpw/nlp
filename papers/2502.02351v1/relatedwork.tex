\section{Related Work}
\label{sec:literature}

This literature review examines the current advancements in applying AI to optimize medical protocols and explores the integration of DICOM metadata as attributes in database construction, emphasizing its potential in enhancing MRI protocol optimization. By leveraging ML to analyze DICOM-derived image information, these methodologies highlight the versatility of AI in streamlining workflows, improving diagnostic accuracy, and adapting protocols to clinical needs. Notably, no prior studies were identified that directly align with the methodology proposed in this work, underscoring its originality and potential contributions to the field.

\subsection{Optimization of Medical Protocols Using Artificial Intelligence} \label{AIinOptMedical}

AI, particularly through ML methodologies, has demonstrated remarkable potential in optimizing medical protocols and workflows. By identifying key attributes that significantly influence target outcomes, ML-based classification methods facilitate the generation of optimized responses tailored to specific clinical processes \cite{reviewOptMedical}.

A notable application of AI in healthcare optimization is the customization of treatment plans. By integrating diverse data sources such as clinical attributes, lifestyle factors, and environmental conditions, AI models can recommend personalized therapies. For instance, in chronic diabetes management, neural network architectures like Long Short-Term Memory (LSTM), evaluated using holdout and five-fold cross-validation methods, have shown promising results \cite{diabetes}.

AI's role extends to optimizing scheduling and resource allocation in healthcare facilities. Gradient Boosting (GB) and Decision Tree (DT) models, for example, have successfully reduced no-show rates by up to 19\% through predictive reminders based on scheduling attributes. These models underwent rigorous validation using nested cross-validation and an 80/20 train-test split \cite{noShows}. Similarly, Random Forest (RF) and Support Vector Machine (SVM) algorithms have been applied to surgical scheduling, improving resource utilization while minimizing patient wait times \cite{reviewOptMedical, optSchedule}. Moreover, predictive analysis powered by AI has been used to enhance resource preparation, leveraging historical patient admission, transfer, and discharge data to identify seasonal trends and better allocate resources \cite{reviewOptMedical}.

In the fields of medical physics and bioengineering, AI is increasingly utilized for predictive maintenance of medical equipment. Models trained on data from Computerized Maintenance Management Systems—such as Decision Trees (DT), K-Nearest Neighbors (KNN), Naïve Bayes (NB), Support Vector Machines (SVM), Random Forest (RF), and Artificial Neural Networks (ANN)—have shown significant promise. Validated through ten-fold cross-validation, these models have benefited from Bayesian optimization for hyperparameter tuning and achieved good performance, thereby improving the efficiency of maintenance scheduling \cite{optMaint}.

Overall, AI-based protocol optimization typically relies on interpretable ML models, particularly when analyzing structured datasets to identify critical attributes. This focus on interpretability not only ensures reliability but also enhances the practical applicability of these models in clinical settings.


\subsection{Usage of DICOM Image Metadata in AI Studies}
The increasing adoption of AI methodologies in medical imaging has been facilitated by advancements in extracting and analyzing metadata from DICOM datasets \cite{dicomExtraction}. In computed tomography imaging, AI models have been trained using DICOM metadata to optimize contrast detection and identify intravenous contrast phases. Convolutional Neural Networks (CNNs), optimized through nested cross-validation with an 80/10/10 data split (training, validation, test), have achieved accuracies exceeding 93\% \cite{dicomCT}. 

In mammography, GB and Deep Learning (DL) models have been applied to predict breast tissue deformation during compression. Attributes such as compression thickness, extracted from DICOM metadata, were critical inputs. These models achieved root mean square error values ranging from 0.47 mm to 1.70 mm \cite{dicomMammo}. DICOM data has also been used in radiological forensic imaging to optimize preprocessing steps. Metadata aids in separating and categorizing image volumes for specific processing tasks, whether in 2D or reconstructed 3D formats \cite{dicomForensics}. 

Another study applied AI to assist technologists in selecting optimal positioning and detection regions in spinal MRI. Region-based CNN models were trained and validated using a 64/26/10 data split, with hyperparameters fine-tuned through cross-validation. Approximately 55\% of suggested parameters required minor adjustments in practical use \cite{dicomScanPrescription}. Additionally, DICOM metadata has been leveraged to classify MRI acquisition protocols using Random Forest (RF) models, achieving an average precision of 86\% and an F1-score of 84\% across 16 protocol classes \cite{dicomClassifier}.

These findings underscore the versatility and relevance of DICOM metadata in AI-based medical imaging applications, providing robust datasets for diverse ML methodologies.

\subsection{Application of AI in Optimizing Magnetic Resonance Imaging Protocols}
Recent research has increasingly focused on optimizing MRI protocols to reduce scan times while maintaining diagnostic image quality. DL models have played a pivotal role in this area by reconstructing high-quality images from undersampled k-space data, effectively shortening acquisition times without compromising diagnostic value \cite{samplingMri,acceleratedMri,mriHighQuality,undersampleRec}. Further advancements in contrast techniques have utilized DL to generate optimized MRI sequences from scratch. These approaches often rely on data from simulated scanner environments to build and validate sequences both in simulated settings and \textit{in vivo} \cite{mrZero}.

In addition to sequence optimization, parameter tuning has emerged as a critical focus in MRI protocol studies. ML models such as Support Vector Regression (SVR), KNN, and RF, researchers have optimized imaging parameters based on data derived from simulated scanner environments. Standard practices, such as splitting datasets into 60/20/20 (training, validation, test) and employing ten-fold cross-validation, were used for hyperparameter tuning and model evaluation. Among the tested models, SVR and KNN performed the best, with root mean square error (RMSE) values ranging from 0.00 to 11.50 for SVR and 0.04 to 22.30 for KNN. Despite these promising results, the reliance on simulated data (phantoms) underscores the need for methodologies validated on real-world clinical datasets \cite{optimStudy}.

A complementary avenue of research has focused on assessing MRI image quality using AI. One study developed a binary classification convolutional neural network (CNN) to categorize MRI images as either "good" or "poor" quality. The training dataset incorporated DICOM images sourced from multiple MRI devices and public repositories, such as The Cancer Imaging Archive (TCIA) \cite{tcia}. Additional attributes, including scan duration, body part, and signal-to-noise ratio (SNR), were extracted from DICOM metadata and included in the dataset. Ground truth labels were determined by expert radiologists based on practical quality indicators, such as motion artifacts, signal non-uniformity, and magnetic susceptibility effects. The model achieved an F1-score of 89.2\% on the validation set and 90.0\% on the test set. However, the test specificity was notably low at 22.4\%, highlighting areas for improvement \cite{imageQualityClass}.

Collectively, these studies emphasize the growing integration of AI into MRI protocol optimization. The results demonstrate its potential to streamline image acquisition workflows, enhance diagnostic accuracy, and adapt protocols to meet specific clinical needs, paving the way for more efficient and patient-centered imaging practices.