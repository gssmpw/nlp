
We analyze the disparate vulnerability of standard canaries between the model-based attack and the data-based attack that uses a 2-gram model (as discussed in Sec~\ref{sec:baseline_results}). Figure~\ref{fig:scatter_plot} plots the RMIA scores for both attacks on the same set of canaries, which have either been included in the training dataset of the target model (\emph{member}) or not (\emph{non-member}). Note that the RMIA scores are used to distinguish members from non-members, and that a larger value corresponds to the adversary being more confident in identifying a record as a member, \ie, to the record being more \emph{vulnerable}.

First, we note that the scores across both threat models exhibit a statistically significant, positive correlation. We find a Pearson correlation coefficient between the RMIA scores (log) for both methods of \num{0.20} ($p$-value of \num{2.4e-10}) and \num{0.23} ($p$-value of \num{1.9e-13}) for SST-2 and AG News, respectively. This means that a record vulnerable to the model-based attack tends to be also vulnerable to the data-based attack, even though the attacks differ substantially. 

Second, and more interestingly, some canaries have disparate vulnerability across MIA methods. Indeed, Figure~\ref{fig:scatter_plot} shows how certain data records which are not particularly vulnerable to the model-based attack are significantly more vulnerable to the data-based attack, and vice versa. 

\begin{figure*}[t]
    \centering
    \begin{subfigure}{0.45\textwidth}
        \centering
        \includegraphics[width=\textwidth]{figures/scatter_sst2_syntheticcanary_uniformlabel.pdf}
        \caption{SST-2}
    \end{subfigure}
    \hspace{0.05\textwidth}
    \begin{subfigure}{0.45\textwidth}
        \centering
        \includegraphics[width=\textwidth]{figures/scatter_agnews_syntheticcanary_uniformlabel.pdf}
        \caption{AG News}
    \end{subfigure}
    \caption{
        RMIA scores (log) for model- and data-based MIAs on the same set of canaries. Results for both datasets SST-2 and AG News. Canaries are synthetically generated with target perplexity of $\mathcal{P}_{\textrm{target}}=250$ with a natural label, and inserted $n_\textrm{rep}=12$ times.
    } 
    \label{fig:scatter_plot}
\end{figure*} 
