\documentclass{article}

\usepackage{hyperref}

% Attempt to make hyperref and algorithmic work together better:
%\newcommand{\theHalgorithm}{\arabic{algorithm}}
\usepackage{algorithm}
\usepackage{algorithmic}

% Use the following line for the initial blind version submitted for review:
%\usepackage{icml2025}
\usepackage{arxiv,times}
\usepackage{subcaption}

% For theorems and such
\usepackage{amsmath}
\usepackage{amssymb}
\usepackage{mathtools}
\usepackage{amsthm}

% if you use cleveref..
\usepackage[capitalize,noabbrev]{cleveref}

%%%%%%%%%%%%%%%%%%%%%%%%%%%%%%%%
% THEOREMS
%%%%%%%%%%%%%%%%%%%%%%%%%%%%%%%%
\theoremstyle{plain}
\newtheorem{theorem}{Theorem}[section]
\newtheorem{proposition}[theorem]{Proposition}
\newtheorem{lemma}[theorem]{Lemma}
\newtheorem{corollary}[theorem]{Corollary}
\theoremstyle{definition}
\newtheorem{definition}[theorem]{Definition}
\newtheorem{assumption}[theorem]{Assumption}
\theoremstyle{remark}
\newtheorem{remark}[theorem]{Remark}

% Todonotes is useful during development; simply uncomment the next line
%    and comment out the line below the next line to turn off comments
%\usepackage[disable,textsize=tiny]{todonotes}
\usepackage[textsize=tiny]{todonotes}

\usepackage{multirow}
\usepackage[utf8]{inputenc} % allow utf-8 input
\usepackage[T1]{fontenc}    % use 8-bit T1 fonts
%\usepackage[pagebackref,colorlinks,citecolor=magenta]{hyperref}
\usepackage{doi}
\usepackage{url}            % simple URL typesetting
\usepackage{booktabs}       % professional-quality tables
\usepackage{amsfonts}       % blackboard math symbols
\usepackage{nicefrac}       % compact symbols for 1/2, etc.
\usepackage{microtype}      % microtypography
\usepackage{xcolor}         % colors
\usepackage{graphicx} 
\usepackage{array}
\usepackage{bm}
\usepackage{footnote}
\usepackage{siunitx}
\usepackage{tikz} % To generate the plot from csv
\usepackage{pgfplots}
\usepackage{pgf}
%\usepackage[noEnd=true,indLines=true]{algpseudocodex}
\usepackage{xspace}
\usepackage{enumitem}
\usepackage{threeparttable}
\usepackage{balance}

% Define a custom inline enumerate environment
\newlist{inlineenum}{enumerate*}{1}
\setlist[inlineenum,1]{label=(\roman*), itemjoin={{; }}, itemjoin*={{; and }}}

\setlist[enumerate,1]{label=\arabic*., leftmargin=*}

\makesavenoteenv{table}

\pgfplotsset{compat=1.5}

\newcolumntype{C}[1]{>{\centering\arraybackslash}p{#1}}

% Define colors for Class 1 (Blue shades)
\definecolor{color_1_a}{HTML}{003F5C} % Deep Blue
\definecolor{color_1_b}{HTML}{0072B2} % Medium Blue
\definecolor{color_1_c}{HTML}{66A9D9} % Light Blue
\definecolor{color_1_d}{HTML}{A4CFE3} % Very Light Blue

% Define colors for Class 2 (Orange shades)
\definecolor{color_2_a}{HTML}{8C2D04} % Deep Orange
\definecolor{color_2_b}{HTML}{D55E00} % Medium Orange
\definecolor{color_2_c}{HTML}{E89F73} % Light Orange
\definecolor{color_2_d}{HTML}{F4D0B5} % Very Light Orange

\newcommand{\lukas}[1]{\todo[inline, color=teal!30]{LW: #1}}
\newcommand{\matthieu}[1]{\todo[inline, color=teal!30]{MM: #1}}
\newcommand{\santiago}[1]{\todo[inline, color=teal!30]{SZ: #1}}

\newcommand{\synthetic}[1]{{\ensuremath{\widetilde{#1}}}}
\newcommand{\canary}[1]{{\ensuremath{\hat{#1}}}}
\newcommand{\prompt}[1]{{\ensuremath{\textsf{p}(#1)}}}

\newcommand{\ie}{i.e.\xspace}
\newcommand{\eg}{e.g.\xspace}
\newcommand{\wrt}{w.r.t.\xspace}
\newcommand{\etc}{etc.\xspace}

\renewcommand{\arraystretch}{0.9}
\setlength{\belowcaptionskip}{-5pt}

\title{The Canary’s Echo: Auditing Privacy Risks of LLM-Generated Synthetic Text}

\author{
    Matthieu Meeus\textsuperscript{1,2},
    Lukas Wutschitz\textsuperscript{1},
    Santiago Zanella-B{\'e}guelin\textsuperscript{1},
    Shruti Tople\textsuperscript{1},
    Reza Shokri\textsuperscript{1,3}\\[10pt]
    \textsuperscript{1}\textit{Microsoft} \qquad
    \textsuperscript{2}\textit{Imperial College London} \qquad
    \textsuperscript{3}\textit{National University of Singapore} 
}

\date{}

\begin{document}

\maketitle

\begin{abstract}
How much information about training samples can be gleaned from synthetic data generated by Large Language Models (LLMs)? Overlooking the subtleties of information flow in synthetic data generation pipelines can lead to a false sense of privacy. In this paper, we design membership inference attacks (MIAs) that target data used to fine-tune pre-trained LLMs that are then used to synthesize data, particularly when the adversary does not have access to the fine-tuned model but only to the synthetic data. We show that such data-based MIAs do significantly better than a random guess, meaning that synthetic data leaks information about the training data. Further, we find that canaries crafted to maximize vulnerability to model-based MIAs are sub-optimal for privacy auditing when only synthetic data is released. Such out-of-distribution canaries have limited influence on the model’s output when prompted to generate useful, in-distribution synthetic data, which drastically reduces their vulnerability. To tackle this problem, we leverage the mechanics of auto-regressive models to design canaries with an in-distribution prefix and a high-perplexity suffix that leave detectable traces in synthetic data. This enhances the power of data-based MIAs and provides a better assessment of the privacy risks of releasing synthetic data generated by LLMs.
\end{abstract}

\section{Introduction}
\section{Introduction}

Video generation has garnered significant attention owing to its transformative potential across a wide range of applications, such media content creation~\citep{polyak2024movie}, advertising~\citep{zhang2024virbo,bacher2021advert}, video games~\citep{yang2024playable,valevski2024diffusion, oasis2024}, and world model simulators~\citep{ha2018world, videoworldsimulators2024, agarwal2025cosmos}. Benefiting from advanced generative algorithms~\citep{goodfellow2014generative, ho2020denoising, liu2023flow, lipman2023flow}, scalable model architectures~\citep{vaswani2017attention, peebles2023scalable}, vast amounts of internet-sourced data~\citep{chen2024panda, nan2024openvid, ju2024miradata}, and ongoing expansion of computing capabilities~\citep{nvidia2022h100, nvidia2023dgxgh200, nvidia2024h200nvl}, remarkable advancements have been achieved in the field of video generation~\citep{ho2022video, ho2022imagen, singer2023makeavideo, blattmann2023align, videoworldsimulators2024, kuaishou2024klingai, yang2024cogvideox, jin2024pyramidal, polyak2024movie, kong2024hunyuanvideo, ji2024prompt}.


In this work, we present \textbf{\ours}, a family of rectified flow~\citep{lipman2023flow, liu2023flow} transformer models designed for joint image and video generation, establishing a pathway toward industry-grade performance. This report centers on four key components: data curation, model architecture design, flow formulation, and training infrastructure optimization—each rigorously refined to meet the demands of high-quality, large-scale video generation.


\begin{figure}[ht]
    \centering
    \begin{subfigure}[b]{0.82\linewidth}
        \centering
        \includegraphics[width=\linewidth]{figures/t2i_1024.pdf}
        \caption{Text-to-Image Samples}\label{fig:main-demo-t2i}
    \end{subfigure}
    \vfill
    \begin{subfigure}[b]{0.82\linewidth}
        \centering
        \includegraphics[width=\linewidth]{figures/t2v_samples.pdf}
        \caption{Text-to-Video Samples}\label{fig:main-demo-t2v}
    \end{subfigure}
\caption{\textbf{Generated samples from \ours.} Key components are highlighted in \textcolor{red}{\textbf{RED}}.}\label{fig:main-demo}
\end{figure}


First, we present a comprehensive data processing pipeline designed to construct large-scale, high-quality image and video-text datasets. The pipeline integrates multiple advanced techniques, including video and image filtering based on aesthetic scores, OCR-driven content analysis, and subjective evaluations, to ensure exceptional visual and contextual quality. Furthermore, we employ multimodal large language models~(MLLMs)~\citep{yuan2025tarsier2} to generate dense and contextually aligned captions, which are subsequently refined using an additional large language model~(LLM)~\citep{yang2024qwen2} to enhance their accuracy, fluency, and descriptive richness. As a result, we have curated a robust training dataset comprising approximately 36M video-text pairs and 160M image-text pairs, which are proven sufficient for training industry-level generative models.

Secondly, we take a pioneering step by applying rectified flow formulation~\citep{lipman2023flow} for joint image and video generation, implemented through the \ours model family, which comprises Transformer architectures with 2B and 8B parameters. At its core, the \ours framework employs a 3D joint image-video variational autoencoder (VAE) to compress image and video inputs into a shared latent space, facilitating unified representation. This shared latent space is coupled with a full-attention~\citep{vaswani2017attention} mechanism, enabling seamless joint training of image and video. This architecture delivers high-quality, coherent outputs across both images and videos, establishing a unified framework for visual generation tasks.


Furthermore, to support the training of \ours at scale, we have developed a robust infrastructure tailored for large-scale model training. Our approach incorporates advanced parallelism strategies~\citep{jacobs2023deepspeed, pytorch_fsdp} to manage memory efficiently during long-context training. Additionally, we employ ByteCheckpoint~\citep{wan2024bytecheckpoint} for high-performance checkpointing and integrate fault-tolerant mechanisms from MegaScale~\citep{jiang2024megascale} to ensure stability and scalability across large GPU clusters. These optimizations enable \ours to handle the computational and data challenges of generative modeling with exceptional efficiency and reliability.


We evaluate \ours on both text-to-image and text-to-video benchmarks to highlight its competitive advantages. For text-to-image generation, \ours-T2I demonstrates strong performance across multiple benchmarks, including T2I-CompBench~\citep{huang2023t2i-compbench}, GenEval~\citep{ghosh2024geneval}, and DPG-Bench~\citep{hu2024ella_dbgbench}, excelling in both visual quality and text-image alignment. In text-to-video benchmarks, \ours-T2V achieves state-of-the-art performance on the UCF-101~\citep{ucf101} zero-shot generation task. Additionally, \ours-T2V attains an impressive score of \textbf{84.85} on VBench~\citep{huang2024vbench}, securing the top position on the leaderboard (as of 2025-01-25) and surpassing several leading commercial text-to-video models. Qualitative results, illustrated in \Cref{fig:main-demo}, further demonstrate the superior quality of the generated media samples. These findings underscore \ours's effectiveness in multi-modal generation and its potential as a high-performing solution for both research and commercial applications.

\section{Background and problem statement}
\label{sec:preliminary}
\section{Preliminary} \label{sec:preliminary}
\paragraph{Random variable and distribution.} Let $\mathcal{X} = \mathcal{X}_v \times \mathcal{X}_t$ denote the input space, where $\mathcal{X}_v$ and $\mathcal{X}_t$ correspond to the visual and textual feature spaces, respectively. Similarly, let $\mathcal{Y}$ denote the response space. We define the random variables $\mathbf{X} = (X_v, X_t) \in \mathcal{X}$ and $Y \in \mathcal{Y}$, where $\mathbf{X}$ is the sequence of tokens that combine visual and text input queries, and $Y$ represents the associated response tokens. The joint population is denoted by $P_{\mathbf{X}Y}$, with marginals $P_{\mathbf{X}}$, $P_{Y}$, and the conditional distribution $P_{Y|\mathbf{X}}$. For subsequent sections, $P_{\mathbf{X}Y}$ refers to the instruction tuning data distribution which we consider as in-distribution (ID). 

\paragraph{MLLM and visual instruction tuning.} MLLM usually consists of three components: (1) a visual encoder, (2) a vision-to-language projector, and (3) an LLM that processes a multimodal input sequence to generate a valid textual output $y$ in response to an input query $\mathbf{x}$. An MLLM can be regarded as modeling a conditional distribution $P_{\theta}(y|\mathbf{x})$, where $\theta$ is the model parameters. To attain the multimodal conversation capability, MLLMs commonly undergo a phase so-called \textit{visual instruction tuning} \cite{liu2023visual, dai2023instructblip} with an autoregressive objective as follows:
{
\begin{align} \label{eq::1}
    % & \min_{\theta\in\Theta} \mathbb{E}_{\mathbf{x},y\sim P_{\mathbf{X}Y}} [-\log P_{\theta}(y|\mathbf{x})] \nonumber \\
     \min_{\theta\in\Theta} \mathbb{E}_{\mathbf{x},y\sim P_{\mathbf{X}Y}} [\sum_{l=0}^{L}-\log P_{\theta}(y_{l}|\mathbf{x},y_{<l})],
\end{align}}
where $L$ is a sequence length and $y=(y_{0},...,y_{L})$. After being trained by Eq. \eqref{eq::1}, MLLM produces a response given a query of any possible tasks represented by text.

\paragraph{Evaluation of open-ended generations.} 

(M)LLM-as-a-judge method \cite{zheng2023judging, kim2023prometheus} is commonly adopted to evaluate open-ended generation. In this paradigm, a judge model produces preference scores or rankings for the responses given a query, model responses, and a scoring rubric. Among the evaluation metrics, the \emph{win rate} (Eq. \eqref{eq:win_rate}) is one of the most widely used and representative.


\begin{definition}[\textbf{Win Rate}] Given a parametric reward function $r:\mathcal{X}\times \mathcal{Y}\rightarrow \mathbb{R}$, the 
win rate (WR) of model $P_{\theta}$ w.r.t. $P_{\mathbf{X}Y}$ are defined as follows: 
\begin{equation} \label{eq:win_rate}
\begin{split}
    &\text{WR}(P_{\mathbf{X}Y};\theta):=\mathbb{E}_{\begin{subarray}{l} \mathbf{x},y \sim P_{\mathbf{X}Y} \\ \hat{y} \sim P_{\theta}(\cdot|\mathbf{x}) \end{subarray}}[\mathbb{I}(r(\mathbf{x},\hat{y}) > r(\mathbf{x},y))],
\end{split}
\end{equation}
where $\mathbb{I}(\cdot)$ is  the indicator function.
\end{definition}
Here, the reward function $r(\cdot,\cdot)$, can be any possible (multimodal) LLMs such as GPT-4o \cite{hurst2024gpt}.



\section{Methodology}
\section{RoleMRC}
\label{sec:method}

In this section, we build RoleMRC. Figure\,\ref{fig:method} illustrates the overall pipeline of RoleMRC from top to bottom, which is divided into three steps.

\subsection{A Meta-pool of 10k Role Profiles}
\label{sec:meta_pool}
We first collect a meta-pool of 10k role profile using two open-source datasets, with Step 1 and 2.

\paragraph{Step 1: Persona Sampling.} We randomly sample 10.5k one-sentence demographic persona description from PersonaHub\,\cite{ge2024scaling}, such as ``\emph{A local business owner interested in economic trends}'', as shown at the top of Figure\,\ref{fig:method}. 

\paragraph{Step 2: Role Profile Standardization.} Next, we use a well-crafted prompt with gpt-4o\,\cite{gpt4o} to expand each sampled persona into a complete role profile, in reference to the 1-shot standardized example. Illustrated in the middle of Figure\,\ref{fig:method}, we require a standardized role profile consisting of seven components: \emph{Role Name and Brief Description}, \emph{Specific Abilities and Skills}, \emph{Speech Style}, \emph{Personality Characteristics}, \emph{Past Experience and Background}, \emph{Ability and Knowledge Boundaries} and \emph{Speech Examples}. %Setting standard specifications helps convert the generated role profiles into formatted records, which is beneficial for the post quality control. 
Standardizing these profiles ensures structured formatting, simplifying quality control. 
After manual checking and format filtering, we remove 333 invalid responses from gpt-4o, resulting in 10.2k final role profiles. We report complete persona-to-profile standardization prompt and structure tree of final role profiles in Appendix\,\ref{sec:app_prompt_1} and \,\ref{sec:app_tree}, respectively.

Machine Reading Comprehension (MRC) is one of the core tasks for LLMs to interact with human users. Consequently, we choose to synthesize fine-grained role-playing instruction-following data based on MRC. We first generate a retrieval pool containing 808.7k MRC data from the MSMARCO training set\,\cite{bajaj2016ms}. By leveraging SFR-Embedding\,\cite{SFR-embedding-2}, we perform an inner product search to identify the most relevant and least relevant MRC triplets (Passages, Question, Answer) for each role profile. For example, the middle part of Figure\,\ref{fig:method} shows that for the role \emph{Jessica Thompson, a resilient local business owner}, the most relevant question is about \emph{the skill of resiliency}, while the least relevant question is \emph{converting Fahrenheit to Celsius}. After review, we categorise the most relevant MRC triplet as within a role's knowledge boundary, and the least relevant MRC triplet as beyond their expertise.

\begin{figure}[t]
    \centering
    \includegraphics[width=1.0\linewidth]{figures/step3.png}
    \caption{The strategy of gradually synthesizing finer role-playing instructions in step 3 of Figure\,\ref{fig:method}.}
    \vspace{-1.0em}
    \label{fig:step3}
\end{figure}

\subsection{38k Role-playing Instructions}
Based on the role profiles, we then adopt \textbf{Step 3: Multi-stage Dialogue Synthesis} to generate 38k role-playing instructions, progressively increasing granularity across three categories %including three types with gradually finer granularity 
(Figure\,\ref{fig:step3}):
%\begin{itemize}
%[leftmargin=*,noitemsep,topsep=0pt]

\noindent \textbf{\underline{Free Chats.}} The simplest dialogues, free chats, are synthesized at first. Here, we ask gpt-4o to simulate and generate multi-turn open-domain conversations between the role and an imagined user based on the standardized role profile. When synthesizing the conversation, we additionally consider two factors: the \textbf{initial speaker} in the starting round of the conversation, and whether the role's speech has \textbf{a narration wrapped in brackets} at the beginning (e.g., \emph{(Aiden reviews the network logs, his eyes narrowing as he spots unusual activity) I found it!}). The narration refers to a short, vivid description of the role's speaking state from an omniscient perspective, which further strengthens the sense of role's depth and has been adopted in some role-playing datasets\,\cite{tu2024charactereval}. 

As shown on the left side of Figure\,\ref{fig:step3}, based on the aforementioned two factors, we synthesize four variations of Free Chats. In particular, when  narration is omitted, we deleted all the 
narration content in the speech examples from the role profile; %and for the case that 
when narration is allowed, we retain the narration content, and also add instructions to allow appropriate insertion of narration in the task prompt of gpt-4o. It worth to note that, in narration-allowed dialogues, not every response of the role has narration inserted to prevent overfitting. All categories of data in RoleMRC incorporate narration insertion and follow similar control mechanisms. The following sections will omit further details on narration.

\noindent \textbf{\underline{On-scene MRC Dialogues.}} The synthesis of on-scene MRC dialogues can be divided into two parts. The first part is similar to the free chats. As shown by the {\color{lightgreen}{green round rectangle}} in the upper part of Figure\,\ref{fig:step3}, we ask gpt-4o to synthesize a conversation (lower left corner of Figure\,\ref{fig:step3}) between the role and the user focusing on relevant passages. This part of the synthesis and the Free Chats share the entire meta-pool, so each consisting of 5k dialogues.

The remaining part forms eight types of single-turn role-playing Question Answering (QA). In the middle of Figure\,\ref{fig:step3}, we randomly select a group of roles and examined the most relevant MRCs they matched: if the question in the MRC is answerable, then the ground truth answer is stylized to match the role profile; otherwise, a seed script of ``unanswerable'' is randomly selected then stylized. The above process generates four groups of 1k data from type ``[1]'' to type``[4]''. According to the middle right side of Figure\,\ref{fig:step3}, we also select a group of roles and ensure that the least relevant MRCs they matched contain answerable QA pairs. Since the most irrelevant MRCs are outside the knowledge boundary of the roles, the role-playing responses to these questions are ``out-of-mind'' refusal or ``let-me-try'' attempt, thus synthesizing four groups of 1k data, from type ``[5]'' to type ``[8]''.

\noindent \textbf{\underline{Ruled Chats.}} We construct Ruled Chats by extending On-scene MRC Dialogues in categories ``[1]'' to ``[8]'' with incorporated three additional rules, as shown in the right bottom corner of Figure\,\ref{fig:step3}. For the \textbf{multi-turn rules}, we apply them to the four unanswerable scenarios ``[3]'', ``[4]'', ``[5]'', and ``[6]'', adding a user prompt that  forces the role to answer. Among them, data ``[3]'' and ``[4]'' maintain refusal since the questions in MRC are unanswerable; while ``[5]'' and ``[6]'' are transformed into attempts to answer despite knowledge limitations. For the \textbf{nested formatting rules}, we add new formatting instructions to the four categories of data ``[1]'', ``[2]'', ``[3]'', and ``[4]'', such as requiring emojis,  capitalization, specific punctuation marks, and controlling the total number of words, then modify the previous replies accordingly. For the last \textbf{prioritized rules}, we apply them to subsets ``[1]'' and ``[2]'' that contain normal stylized answers, inserting a  global refusal directive from the system, and thus creating a conflict between system instructions and the role's ability boundary.
%\end{itemize}

\begin{table}[t]
\resizebox{\columnwidth}{!}{%
  \begin{tabular}{c|c|c|c|c|c}
    \toprule
    & & \textbf{S*} & \textbf{P*} & \textbf{\#Turns} & \textbf{\#Words} \\ 
    \midrule
    \multirow{13.5}{*}{\textbf{RoleMRC}} 
    & \multicolumn{5}{c|}{\textbf{Free Chats}} \\ 
    \cmidrule(lr){2-6}
    & Chats & 5k & / & 9.47 & 38.62 \\ 
    \cmidrule(lr){2-6}
    & \multicolumn{5}{c|}{\textbf{On-scene MRC Dialogues}} \\ 
    \cmidrule(lr){2-6} 
    & On-scene Chats & 5k & / & 9.2 & 43.18 \\
    & Answer & 2k & 2k & 1 & 39.45 \\ 
    & No Answer & 2k & 2k & 1 & 47.09 \\ 
    & Refusal & 2k & 2k & 1 & 48.41 \\ 
    & Attempt & 2k & 2k & 1 & 47.92 \\ 
    \cmidrule(lr){2-6}
    & \multicolumn{5}{c|}{\textbf{Ruled Chats}} \\ 
    \cmidrule(lr){2-6}
    & Multi-turn & 2k & 2k & 2 & 42.47 \\ 
    & Nested & 1.6k & 1.6k & 1 & 46.17 \\ 
    & Prioritized & 2.4k & 2.4k & 1 & 42.65 \\ 
    \midrule
    & \textbf{Total} & 24k & 14k & 3.5 & 40.6 \\ 
    \midrule
    \multirow{3}{*}{\textbf{-mix}} 
    & RoleBench & 16k & / & 1 & 23.95 \\ 
    & RLHFlow & 40k & / & 1.39 & 111.79 \\ 
    & UltraFeedback & / & 14k & 1 & 199.28 \\ 
    \midrule
    & \textbf{Total} & 80k & 28k & 2 & 67.1 \\ 
    \bottomrule
  \end{tabular}}
  \vspace{-2mm}
  \caption{Statistics of RoleMRC. In particular, the column names S*, P*, \#Turns, and \#Words, stands for size of single-label data, size of pair-label data, average turns, and average number of words per reply, respectively. RoleMRC-mix expands RoleMRC by adding existing role-playing data.}
 \vspace{-3mm}
  \label{tab:roleMRC}
\end{table}

\subsection{Integration and Mix-up}
All the seed scripts and prioritized rules used for constructing On-scene Dialogues and Ruled Chats are reported in Appendix\,\ref{sec:app_scripts}. These raw responses are logically valid manual answers that remain unaffected by the roles' speaking styles, making them suitable as negative labels to contrast with the stylized answers. Thanks to these meticulous seed texts, we obtain high-quality synthetic data with stable output from gpt-4o. After integration, as shown in Table\,\ref{tab:roleMRC}, the final RoleMRC contains 24k single-label data for Supervised Fine-Tuning (SFT) and 14k pair-label data for Human Preference Optimization (HPO)\,\cite{ouyang2022training,rafailov2023direct,sampo,hong2024reference}. Considering that fine-tuning LLMs with relatively fixed data formats may lead to catastrophic forgetting\,\cite{kirkpatrick2017overcoming}, we create RoleMRC-mix as a robust version by incorporating external role-playing data (RoleBench\,\cite{wang2023rolellm}) and general instructions (RLHFlow\,\cite{dong2024rlhf}, UltraFeedback\,\cite{cui2023ultrafeedback}).


\section{Experimental setup}
\label{sec:exp_setup}
\textbf{Datasets.} 
We consider two datasets that have been widely used to study text classification: 
%
\begin{inlineenum}
\item the Stanford Sentiment Treebank (\textbf{SST-2})~\citep{socher-etal-2013-recursive}, which consists of excerpts from written movie reviews with a binary sentiment label
\item the \textbf{AG News} dataset~\citep{Zhang2015CharacterlevelCN}, which consists of news articles labelled by category (World, Sport, Business, Sci/Tech).
\end{inlineenum}
%
In all experiments, we remove examples with less than \num{5} words, bringing the total number of examples to \num{43296} for SST-2 and \num{120000} for AG News.

\textbf{Synthetic data generation.} 
We fine-tune the pre-trained Mistral-7B model~\citep{jiang2023mistral} using low-rank adaptation (LoRa)~\citep{hulora}. We use a custom prompt template $\prompt{\cdot}$ for each dataset (see Appendix~\ref{app:prompts}). More details on the implementation and parameters are provided in Appendix~\ref{app:implementation_details}. We sample synthetic data from the fine-tuned model $\theta$ conditioned on prompts $\prompt{\synthetic{\ell}_i}$, following the same distribution of labels in the synthetic dataset $\synthetic{D}$ as in the original dataset $D$, \ie $\ell_i = \synthetic{\ell}_i$ for $i=1,...,\synthetic{N}$. To generate synthetic sequences, we sequentially sample completions using a softmax temperature of \num{1.0} and top-$p$ (aka nucleus) sampling with $p = 0.95$, \ie we sample from a vocabulary restricted to the smallest possible set of tokens whose total probability exceeds \num{0.95}. We further ensure that the synthetic data bears high utility, and is thus realistic. For this, we consider the downstream classification tasks for which the original datasets have been designed. We fine-tune RoBERTa-base~\citep{DBLP:journals/corr/abs-1907-11692} on  $D$ and $\synthetic{D}$ and compare the performance of the resulting classifiers on held-out evaluation datasets. Details are provided in Appendix~\ref{app:utility}, for synthetic data generated with and without canaries.

\textbf{Canary injection.} 
We generate canaries $\canary{x} = (\canary{s}, \canary{\ell})$ as described in Sec.~\ref{sec:method_canaries}. Unless stated otherwise, we consider $50$-word canaries. Synthetic canaries are generated using Mistral-7B~\citep{jiang2023mistral} as $\theta_0$. We consider two ways of constructing a canary label:
%
\begin{inlineenum}
\item randomly sampling label $\canary{\ell}$ from the distribution of labels in $D$, ensuring that the class distribution among canaries matches that of $D$ (\emph{Natural}) 
\item extending the set of labels with a new artificial label ($\canary{\ell}=$"canary") only used for canaries (\emph{Artificial}). 
\end{inlineenum}

\textbf{Membership inference.} 
We compute the membership scores $\beta_{\theta}(\canary{x})$ as described in Sec.~\ref{sec:membership_method}. 
For one target model $\theta$, we consider \num{1000} canaries \canary{x}, of which on average half are included in the training dataset $n_\textrm{rep}$ times (members), while the remaining half are excluded (non-members).
%
We then use the computed RMIA scores and the ground truth for membership to construct ROC curves, from which we compute AUC and true positive rate (TPR) at low false positive rate (FPR) as measures of MIA performance.
%
Across experiments, we use $M = 4$ reference models $\theta'$, each trained on a dataset $D_{\theta'}$ consisting of the dataset $D$ used to train the target model $\theta$ with canaries inserted. Note that although practical attacks rarely have this amount of information, this is allowed by the threat model of Algorithm~\ref{alg:mia} and valid as a worst-case auditing methodology. 
%
We ensure that each canary is a member in half (\ie 2) of the reference models and a non-member in the other half. For the attacks based on synthetic data, we use $n=2$ for computing scores using an $n$-gram model and $k=25$ for computing scores based on similarity. We use Sentence-BERT~\citep{reimers-2019-sentence-bert} (\texttt{paraphrase-MiniLM-L6-v2} from \texttt{sentence-transformers}) as the embedding model. 


\section{Results}
% \begin{table}[!t]
% \centering
% \scalebox{0.68}{
%     \begin{tabular}{ll cccc}
%       \toprule
%       & \multicolumn{4}{c}{\textbf{Intellipro Dataset}}\\
%       & \multicolumn{2}{c}{Rank Resume} & \multicolumn{2}{c}{Rank Job} \\
%       \cmidrule(lr){2-3} \cmidrule(lr){4-5} 
%       \textbf{Method}
%       &  Recall@100 & nDCG@100 & Recall@10 & nDCG@10 \\
%       \midrule
%       \confitold{}
%       & 71.28 &34.79 &76.50 &52.57 
%       \\
%       \cmidrule{2-5}
%       \confitsimple{}
%     & 82.53 &48.17
%        & 85.58 &64.91
     
%        \\
%        +\RunnerUpMiningShort{}
%     &85.43 &50.99 &91.38 &71.34 
%       \\
%       +\HyReShort
%         &- & -
%        &-&-\\
       
%       \bottomrule

%     \end{tabular}
%   }
% \caption{Ablation studies using Jina-v2-base as the encoder. ``\confitsimple{}'' refers using a simplified encoder architecture. \framework{} trains \confitsimple{} with \RunnerUpMiningShort{} and \HyReShort{}.}
% \label{tbl:ablation}
% \end{table}
\begin{table*}[!t]
\centering
\scalebox{0.75}{
    \begin{tabular}{l cccc cccc}
      \toprule
      & \multicolumn{4}{c}{\textbf{Recruiting Dataset}}
      & \multicolumn{4}{c}{\textbf{AliYun Dataset}}\\
      & \multicolumn{2}{c}{Rank Resume} & \multicolumn{2}{c}{Rank Job} 
      & \multicolumn{2}{c}{Rank Resume} & \multicolumn{2}{c}{Rank Job}\\
      \cmidrule(lr){2-3} \cmidrule(lr){4-5} 
      \cmidrule(lr){6-7} \cmidrule(lr){8-9} 
      \textbf{Method}
      & Recall@100 & nDCG@100 & Recall@10 & nDCG@10
      & Recall@100 & nDCG@100 & Recall@10 & nDCG@10\\
      \midrule
      \confitold{}
      & 71.28 & 34.79 & 76.50 & 52.57 
      & 87.81 & 65.06 & 72.39 & 56.12
      \\
      \cmidrule{2-9}
      \confitsimple{}
      & 82.53 & 48.17 & 85.58 & 64.91
      & 94.90&78.40 & 78.70& 65.45
       \\
      +\HyReShort{}
       &85.28 & 49.50
       &90.25 & 70.22
       & 96.62&81.99 & \textbf{81.16}& 67.63
       \\
      +\RunnerUpMiningShort{}
       % & 85.14& 49.82
       % &90.75&72.51
       & \textbf{86.13}&\textbf{51.90} & \textbf{94.25}&\textbf{73.32}
       & \textbf{97.07}&\textbf{83.11} & 80.49& \textbf{68.02}
       \\
   %     +\RunnerUpMiningShort{}
   %    & 85.43 & 50.99 & 91.38 & 71.34 
   %    & 96.24 & 82.95 & 80.12 & 66.96
   %    \\
   %    +\HyReShort{} old
   %     &85.28 & 49.50
   %     &90.25 & 70.22
   %     & 96.62&81.99 & 81.16& 67.63
   %     \\
   % +\HyReShort{} 
   %     % & 85.14& 49.82
   %     % &90.75&72.51
   %     & 86.83&51.77 &92.00 &72.04
   %     & 97.07&83.11 & 80.49& 68.02
   %     \\
      \bottomrule

    \end{tabular}
  }
\caption{\framework{} ablation studies. ``\confitsimple{}'' refers using a simplified encoder architecture. \framework{} trains \confitsimple{} with \RunnerUpMiningShort{} and \HyReShort{}. We use Jina-v2-base as the encoder due to its better performance.
}
\label{tbl:ablation}
\end{table*}

\section{Results}
\label{sec:results}

In this section, we present detailed results demonstrating \emph{CellFlow}'s state-of-the-art performance in cellular morphology prediction under perturbations, outperforming existing methods across multiple datasets and evaluation metrics.

\subsection{Datasets}

Our experiments were conducted using three cell imaging perturbation datasets: BBBC021 (chemical perturbation)~\cite{caie2010high}, RxRx1 (genetic perturbation)~\cite{sypetkowski2023rxrx1}, and the JUMP dataset (combined perturbation)~\cite{chandrasekaran2023jump}. We followed the preprocessing protocol from IMPA~\cite{palma2023predicting}, which involves correcting illumination, cropping images centered on nuclei to a resolution of 96×96, and filtering out low-quality images. The resulting datasets include 98K, 171K, and 424K images with 3, 5, and 6 channels, respectively, from 26, 1,042, and 747 perturbation types. Examples of these images are provided in Figure~\ref{fig:comparison}. Details of datasets are provided in \S\ref{sec:data}.

\subsection{Experimental Setup}

\textbf{Evaluation metrics.} We evaluate methods using two types of metrics: (1) FID and KID, which measure image distribution similarity via Fréchet and kernel-based distances, computed on 5K generated images for BBBC021 and 100 randomly selected perturbation classes for RxRx1 and JUMP; we report both overall scores across all samples and conditional scores per perturbation class. (2) Mode of Action (MoA) classification accuracy, which assesses biological fidelity by using a trained classifier to predict a drug’s effect from perturbed images and comparing it to its known MoA from the literature.

\textbf{Baselines.} We compare our approach against two baselines, PhenDiff~\cite{bourou2024phendiff} and IMPA~\cite{palma2023predicting}, the only two baselines that incorporate control images into their model design --- a crucial setup for distinguishing true perturbation effects from artifacts such as batch effects. PhenDiff uses diffusion models to first map control images to noise and then transform the noise into target images. In contrast, IMPA employs GANs with an AdaIN layer to transfer the style of control images to target images, specifically designed for paired image-to-image mappings. Our method uses flow matching, which is tailored for distribution-to-distribution mapping, providing a more suitable solution for our problem. We reproduce these baselines with official codes.

\textbf{Training details.} \emph{CellFlow} employs a UNet-based velocity field with a four-stage design. Perturbations are encoded following IMPA~\cite{palma2023predicting}. Training is conducted for 100 epochs on 4 A100 GPUs. Details are in \S\ref{sec:experimental}.


\begin{table*}[ht!]
\centering
\footnotesize
\renewcommand{\arraystretch}{1.2}
\begin{tabular}{p{2.15cm} p{14.05cm}}
\toprule
 & \includegraphics[height=2.6cm]{figure/qualitative_example.pdf}\\
\midrule
Original \phantom{00000} Description & Bromazepam binds to the GABA receptor GABA-A, causing a conformational change and \textbf{increasing inhibitory effects of GABA}. Other neurotransmitters are not influenced. One of the \textbf{benzodiazepines} that is \textbf{used in the treatment of anxiety disorders.} \\
\midrule
User & Explain the features of the given molecule that are most distinct from those of other molecules. \\
\midrule
Mol-LLaMA (Ours) & The given molecule is a member of the \textcolor{tabblue}{benzodiazepine class}, which is characterized by a fusion of a benzene ring and a seven-membered diazepine ring. This structure is crucial for its biological activity, particularly its role as \textcolor{tabblue}{a sedative and hypnotic agent}. ...$<$omitted$>$...\\

& The given molecule is also characterized by a ketone group within the diazepine ring, which is a common feature in benzodiazepines. \textcolor{tabblue}{This functional group is important for the interaction with the GABA-A receptor, enhancing the inhibitory effects of GABA}. ...$<$omitted$>$...\\

\midrule
GPT-4o & 1. **Quinazoline Core**: The molecule contains a \textcolor{tabred}{quinazoline core}, ...$<$omitted$>$... \\
 & Overall, the combination of a \textcolor{tabred}{quinazoline} structure with unique substitution patterns, including bromine halogenation and the presence of a pyridinyl group, ...$<$omitted$>$... . \\
\midrule
LLaMo & The molecule has a \textcolor{tabblue}{benzodiazepine} structure with a bromo substituent at the 7-position and \textcolor{tabred}{a methyl group at the 1-position}. It is distinct from other molecules in that it contains a bromine atom, a nitrogen atom, and a methyl group, which are not present in the other molecules. \\
\midrule
3D-MoLM & ...$<$omitted$>$... It belongs to the class of compounds called \textcolor{tabred}{quinazolines}, which are characterized by a bicyclic structure consisting of a benzene ring fused to a pyrimidine ring. ...$<$omitted$>$...\\
& The molecule's structure suggests potential applications in medicinal chemistry, as quinazolines have been found to possess various biological activities, including \textcolor{tabred}{antitumor, antimicrobial, and anti-inflammatory properties.} \\
\midrule
Mol-Instructions & The molecule is a \textcolor{tabred}{quinoxaline derivative}. \\
\bottomrule
\end{tabular}
\vspace{-0.1in}
\caption{\small Case study to compare molecular understanding and reasoning ability. Mol-LLaMA accurately understands the molecular features, answering a correct molecular taxonomy and providing its distinct properties that are relevant to the given molecule.}
\label{tab:qualitative}
\vspace{-0.1in}
\end{table*}

\subsection{Main Results}

\textbf{\emph{CellFlow} generates highly realistic cell images.}  
\emph{CellFlow} outperforms existing methods in capturing cellular morphology across all datasets (Table~\ref{tab:results}a), achieving overall FID scores of 18.7, 33.0, and 9.0 on BBBC021, RxRx1, and JUMP, respectively --- improving FID by 21\%–45\% compared to previous methods. These gains in both FID and KID metrics confirm that \emph{CellFlow} produces significantly more realistic cell images than prior approaches.

\textbf{\emph{CellFlow} accurately captures perturbation-specific morphological changes.}  
As shown in Table~\ref{tab:results}a, \emph{CellFlow} achieves conditional FID scores of 56.8 (a 26\% improvement), 163.5, and 84.4 (a 16\% improvement) on BBBC021, RxRx1, and JUMP, respectively. These scores are computed by measuring the distribution distance for each specific perturbation and averaging across all perturbations.   
Table~\ref{tab:results}b further highlights \emph{CellFlow}’s performance on six representative chemical and three genetic perturbations. For chemical perturbations, \emph{CellFlow} reduces FID scores by 14–55\% compared to prior methods.
The smaller improvement (5–12\% improvements) on RxRx1 is likely due to the limited number of images per perturbation type.

\textbf{\emph{CellFlow} preserves biological fidelity across perturbation conditions.} 
Table~\ref{tab:ablation}a presents mode of action (MoA) classification accuracy on the BBBC021 dataset using generated cell images. MoA describes how a drug affects cellular function and can be inferred from morphology. To assess this, we train an image classifier on real perturbed images and test it on generated ones. \emph{CellFlow} achieves 71.1\% MoA accuracy, closely matching real images (72.4\%) and significantly surpassing other methods (best: 63.7\%), demonstrating its ability to maintain biological fidelity across perturbations. Qualitative comparisons in Figure~\ref{fig:comparison} further highlight \emph{CellFlow}’s accuracy in capturing key biological effects. For example, demecolcine produces smaller, fragmented nuclei, which other methods fail to reproduce accurately.

\textbf{\emph{CellFlow} generalizes to out-of-distribution (OOD) perturbations.}  
On BBBC021, \emph{CellFlow} demonstrates strong generalization to novel chemical perturbations never seen during training (Table~\ref{tab:ablation}b). It achieves 6\% and 28\% improvements in overall and conditional FID over the best baseline. This OOD generalization is critical for biological research, enabling the exploration of previously untested interventions and the design of new drugs.

\textbf{Ablations highlight the importance of each component in \emph{CellFlow}.}  
Table~\ref{tab:ablation}c shows that removing conditional information, classifier-free guidance, or noise augmentation significantly degrades performance, leading to higher FID scores. These underscore the critical role of each component in enabling \emph{CellFlow}’s state-of-the-art performance.  

\begin{figure*}[!tb]
    \centering
     \includegraphics[width=\linewidth]{imgs/interpolation.pdf}
     \vspace{-2em}
    \caption{
    \textbf{\emph{CellFlow} enables new capabilities.} 
\textit{(a.1) Batch effect calibration.}  
\emph{CellFlow} initializes with control images, enabling batch-specific predictions. Comparing predictions from different batches highlights actual perturbation effects (smaller cell size) while filtering out spurious batch effects (cell density variations).  
\textit{(a.2) Interpolation trajectory.}  
\emph{CellFlow}'s learned velocity field supports interpolation between cell states, which might provide insights into the dynamic cell trajectory. 
\textit{(b) Diffusion model comparison.}  
Unlike flow matching, diffusion models that start from noise cannot calibrate batch effects or support interpolation.  
\textit{(c) Reverse trajectory.}  
\emph{CellFlow}'s reversible velocity field can predict prior cell states from perturbed images, offering potential applications such as restoring damaged cells.
    }
    \label{fig:interpolation}
    \vspace{-1em}
\end{figure*}

\subsection{New Capabilities}

\textbf{\emph{CellFlow} addresses batch effects and reveals true perturbation effects.}  
\emph{CellFlow}’s distribution-to-distribution approach effectively addresses batch effects, a significant challenge in biological experimental data collection. As shown in Figure~\ref{fig:interpolation}a, when conditioned on two distinct control images with varying cell densities from different batches, \emph{CellFlow} consistently generates the expected perturbation effect (cell shrinkage due to mevinolin) while recapitulating batch-specific artifacts, revealing the true perturbation effect. Table~\ref{tab:ablation}d further quantifies the importance of conditioning on the same batch. By comparing generated images conditioned on control images from the same or different batches against the target perturbation images, we find that same-batch conditioning reduces overall and conditional FID by 21\%. This highlights the importance of modeling control images to more accurately capture true perturbation effects—an aspect often overlooked by prior approaches, such as diffusion models that initialize from noise (Figure~\ref{fig:interpolation}b).

\textbf{\emph{CellFlow} has the potential to model cellular morphological change trajectories.}
Cell trajectories could offer valuable information about perturbation mechanisms, but capturing them with current imaging technologies remains challenging due to their destructive nature. Since \emph{CellFlow} continuously transforms the source distribution into the target distribution, it can generate smooth interpolation paths between initial and final predicted cell states, producing video-like sequences of cellular transformation based on given source images (Figure~\ref{fig:interpolation}a). This suggests a possible approach for simulating morphological trajectories during perturbation response, which diffusion methods cannot achieve (Figure~\ref{fig:interpolation}b). Additionally, the reversible distribution transformation learned through flow matching enables \emph{CellFlow} to model backward cell state reversion (Figure~\ref{fig:interpolation}c), which could be useful for studying recovery dynamics and predicting potential treatment outcomes.


\section{Related work}
\section{Related Work}
\paragraph{LLMs for Visual Program Generation}
Visual programming systems (e.g., LabView~\cite{bitter2006labview}, XG5000~\cite{XG5000Manual}) typically feature node-based interfaces that let users visually write and modify programs. Recently, researchers have begun utilizing LLMs to generate VPLs, as they are known for their powerful text-based code generation capabilities. For example, \citet{cai-etal-2024-low-code} integrates low-code visual programming with LLM-based task execution for direct interaction with LLMs, while \citet{zhang2024benchmarking} studies generation of node-based visual dataflow languages in audio programming. Similarly, \citet{xue2024comfybenchbenchmarkingllmbasedagents,52868} investigates Machine Learning workflow generation from natural language commands and demonstrates that metaprogram-based text formats outperform other formats like JSON. However, these prompting-based methods face limitations for VPLs like Ladder Diagram, where custom I/O mapping and domain-specific syntax are crucial. Thus, we study fine-tuning approaches with domain-specific data to better capture these details.

\paragraph{LLM-based PLC code generation}
Programmable Logic Controllers (PLCs) are essential components in industrial automation and are used to control machinery and processes reliably and efficiently. Among the programming languages defined by the IEC 61131-3 standard~\cite{IEC61131-3}, Structured Text (ST) and Ladder Diagram (LD) are commonly used for programming PLCs. Research in this area has focused on utilizing LLMs to generate ST code from natural language descriptions. Recent studies have demonstrated the potential of LLMs in generating high-quality ST code~\cite{koziolek2023chatgpt, koziolek2024llm}, enhancing safety and accuracy with verification tools and user feedback~\cite{fakih2024llm4plc}, and automating code generation and verification using multi-agent frameworks~\cite{liu2024agents4plc}. Although these advances have improved PLC code generation, they primarily focus on ST, despite LD being widely used in industrial settings due to its similarity to electrical circuits~\cite{ladderlogic}. While \citet{Zhang_2024} attempts to generate LD as an ASCII art based on user instructions in a zero-shot manner, their findings show that even advanced LLMs struggle with basic LD generation. These limitations highlight the necessity of training-based methods for LD generation. In this work, we address this gap by introducing a training-based approach for LLMs to generate LD and thus pave the way for the broader adoption of AI-assisted PLC programming.

% Doesn't count towards page limit
%\section{Impact statement}
%\section*{Impact Statement}
The proposed method for training code agents has the potential to streamline software development processes by automating routine coding tasks, thereby reducing human labor and accelerating production timelines. However, these advances will also introduce bugs, which can propagate at scale if no proper quality control is in place. 

\section*{Acknowledgements}
AJ is supported by Fonds de Recherche du Québec~(FRQ), Calcul Québec, Canada CIFAR AI Chair program, and Canada Excellence Research Chairs (CERC) program.
The authors are also grateful to Mila (mila.quebec) IDT and Digital Research Alliance of Canada for computing resources.
AMR is supported in part by NSF CAREER \#2037519 and NSF \#2242302.
SC is supported in part by Google Faculty Research Award, OpenAI SuperAlignment Grant, ONR Young Investigator Award, NSF RI \#2312956, and NSF FRR\#2327973.


\section{Acknowledgements}
L.W. would like to thank Robert Sim for encouraging us to work on this topic and Huseyin Inan for fruitful discussions on private synthetic data generation.

\balance
\bibliography{bibliography.bib}
\bibliographystyle{icml2025}

\newpage
\appendix
\onecolumn

\section{Pseudo-code for MIAs based on synthetic data}
\label{app:pseudo_code}
\begin{algorithm*}
\setstretch{1.2}
  \caption{The testing loop of proposed \textbf{SOBA} method for test-time adaptation
    \label{alg:whole_loop}}
  \begin{algorithmic}[1]
   \State \textbf{Input: }{CLIP visual encoder \(E_{v}(\theta_v)\), text encoder \(E_{t}(\theta_t)\), testing dataset $D_{test}$, number of classes $N$, $N$ text descriptions $T$ of $N$ classes, original basis \({\mathcal{E}}\), dynamic queue $L$, hyper-parameter $\alpha$, queue capacity $\rm{K}$.}
  \For{each test sample \( x_{test} \) in \( D_{test} \)}
        \State {Image embedding: } \( f_{test} \) $\gets$ \(E_{v}(\theta_v, x_{test})\)
        \State {Text embeddings: } \( W_{t} \) $\gets$ \(E_{t}(\theta_t, T)\)
        \State {CLIP logits: } \( logits_{ori} \) $\gets$ $f_{test}W_{t}^{\rm{T}}$
        \State {Pseudo-label of \( x_{test} \): } \(\hat{l}\) $\gets$ \texttt{OneHot}($logits_{ori}$)
        \State $L$ $\gets$ \texttt{Update}($L, f_{test}, \hat{l}, logits_{ori}$) \Comment{See Algorithm \ref{alg:update}}
        \For{each pseudo-class \( \hat{l}_{k} \) in \( L \)}
            \State {Get prototype of class \(\hat{l}_{k}\):}
            ${\mathbf{\mu}_{k}}$ $\gets$ $\frac{\sum_{i=1}^{M_{k}} \mathbb{I}_{\hat{l}=k} f_{test,i}}{\sum_{i=1}^{M_{k}} \mathbb{I}_{\hat{l}=k}}$
        \EndFor
        \State {Get covariance $C$ of $L$:}
        $C$ $\gets$ $\frac{1}{N}\sum_{k=1}^{N} \frac{ {\textstyle \sum_{i=1}^{{M}_{k}}\mathbb{I}_{\hat{l}=k}(f_{test,i}-{\mathbf{\mu}_{k}})(f_{test,i}-{\mathbf{\mu}_{k}})^{\rm{T}}  } }{ {\textstyle \sum_{i=1}^{M_{k}}} \mathbb{I}_{\hat{l}=k}} $
        \State Space rotation: $\hat{\mathbf{\mu}}$ $\gets$ \texttt{SOBA}(${\mathbf{\mu}}, C$) \Comment{See Equation \ref{eq:main} and \ref{eq:main_mat}}
        \State {\textbf{SOBA} logits: }\( logits_{trans} \) $\gets$ \texttt{Linear}($f_{test}, \hat{\mathbf{\mu}}$)
        \State {Final inference: } $logits$ $\gets$ $logits_{ori} + {\alpha}\times logits_{\rm{trans}}$
  \EndFor
      \State \Return{$logits$} \Comment{return prediction based on the mode}
  \end{algorithmic}
\end{algorithm*} 

\section{Computation of RMIA scores}
\label{app:rmia_details}
We here provide more details on how we adapt RMIA, as originally proposed by~\citet{zarifzadeh2024low}, to our setup (see Sec.~\ref{sec:method_rmia}). In RMIA, the pairwise likelihood ratio is defined as: 

\begin{equation}
    LR_{\theta}(x, z) = \left(\frac{P(x\mid\theta)}{P(x)}\right) \left(\frac{P(z\mid\theta)}{P(z)}\right)^{-1} \; .
\end{equation}

where $\theta$ represents the target model, $x$ the target record, and $z$ the reference population. In this work, we only consider one target model $\theta$ and many target records $x$. As we are only interested in the relative value of the likelihood ratio across target records, we can eliminate the dependency on the reference population $z$,

\begin{equation}
    LR_{\theta}(x, z) = LR_{\theta}(x) = \frac{P(x\mid\theta)}{P(x)} \; .
    \label{eq:likelihood_ratio}
\end{equation}

As suggested by~\cite{zarifzadeh2024low}, we compute $P(x)$ as the empirical mean of $P(x\mid\theta')$ across reference models $\{ \theta'_i \}_{i=1}^M$,

\begin{equation}
    P(x) = \frac{1}{M}\sum_{i=1}^M P(x\mid\theta'_i) \; .
\end{equation}

To compute RMIA scores, we replace the probabilities in \eqref{eq:likelihood_ratio} by membership signals on target and reference models:

\begin{align}
  \beta_\theta(x) = \frac{\alpha_{\theta}(x)}{\frac{1}{M} \sum_{i=1}^M \alpha_{\theta'_i}(x)} \; .
  \label{eqn:rmia_score_computation}
\end{align}

Note that when we compute $\alpha_{\theta}(x)$ as a product of conditional probabilities (\eg when using the target model probability in the model-based attack or the $n$-gram probability in the data-based attack), we truly use a probability for $\alpha_{\theta}(x)$. However, in the case of the data-based attack using similarity metrics, we use the mean similarity to the $k$ closest synthetic sequences---which does not correspond to a true probability. In this case, we normalize similarities to fall in the range $[0,1]$ and use $\alpha_{\theta}(x)$ as an empirical proxy for the probability $P(x \mid \theta)$.

In practice, $P(x\mid\theta)$ can be an extremely small value, particularly when calculated as a product of token-level conditional probabilities, which can lead to underflow errors.
%
To mitigate this, we perform arithmetic operations on log-probabilities whenever possible. However, in the context of equation (\ref{eqn:rmia_score_computation}), where the denominator involves averaging probabilities, we employ quad precision floating-point arithmetic.
%
This method is sufficiently precise to handle probabilities for sequences of up to 50 words, which is the maximum we consider in our experiments.
 

\section{Prompts used to generate synthetic data}
\label{app:prompts}
\begin{lstlisting}[title={Sampling Responses During Training/Inference}]
Please reason step by step, and put your final answer within 
\boxed{}. 
Problem: {problem} 
\end{lstlisting}

\begin{lstlisting}[title={Verification Refinement}]
You are a math teacher. I will give you a math problem and an answer. 
Verify the answer's correctness without step-by-step solving. Use alternative verification methods. 
Question: {problem}
Answer: {answer}
Verification:
\end{lstlisting}

\begin{lstlisting}[title={Verification Collection}]
Refine this verification text to read as a natural self-check within a solution. Maintain logical flow and professionalism.
Key Requirements:
1. Avoid phrases like "without solving step-by-step" or "as a math teacher".
2. Treat the answer as your own prior solution.
3. Conclude with EXACTLY one of:
Therefore, the answer is correct.
Therefore, the answer is incorrect.
Therefore, the answer cannot be verified.
Original text: {verification}
\end{lstlisting}
 

\section{Implementation details}
\label{app:implementation_details}

\section{Implementation Details}
\label{sec:implementation_details}

\para{Training Details.}
We implement our networks using PyTorch~\cite{paszke2019pytorch} and run all experiments on a GPU cluster with four A100 GPUs.
%
We employ the Adam optimizer~\cite{kingma2014adam} to train our model with a learning rate of $2 \times 10^{-4}$ and an exponential decay of 0.998 every 1,000 iterations.
%
Following LION~\cite{zeng2022lion}, we use an exponential moving average (EMA) 
% \phil{TODO: no need to define EMA, if you will not use it later on in this paper}  \ed{This is a technique for training diffusion model so I mention here as implementation details}
of our model with a decay of 0.9999.
%
Specifically, we train our unconditional generative model for approximately 600,000 iterations with a batch size of 256, taking about four days to complete.
%
% \ed{Our method generally can produce reasonable results starting from 10 inference steps, which takes around XX seconds.}
% \phil{TODO: Try to make this sentenfce more specific and clear} \ednote{TODO: measure the inference time in A100.}
\rebuttal{It is noted that we use larger batch sizes and more training iterations compared to existing work, including~\cite{zhou2021pvd,wu2023psf}, to effectively compare various training paradigms (Figures~\ref{fig:main_quantitative_comp} \&~\ref{fig:main_qualitative_comp}). This choice ensures our training procedure has higher stability and converges properly.
% 
Additionally, we want to highlight that the online subsampling procedure introduces negligible overhead in the training process (merely requiring additional indexing of a cached array).}

\para{Network Architecture.}
For the network architecture, we adopt the same structure as PVD~\cite{zhou2021pvd} and employ PVCNN~\cite{liu2019point} as our vector field network for the unconditional generation task.
%
In the shape completion task, we use an additional 256-dimensional latent vector to represent the input partial point cloud, which is then injected into PVCNN.
%
% For image-conditioned generation, we extract this vector using the Dino-V2 (ViT-L/14)~\cite{oquab2023dinov2} 
% \phil{TODO: citation?} 
% image encoder followed by MLP layers.
%
To do so, we use another PVCNN follow LION~\cite{zeng2022lion} to extract the latent vector from the partial point cloud.
%
% \ednote{Yes. We should mention it!}
% \CL{Are we going to talk more about the inference (e.g. how long it takes for generating one point cloud with $N$ points; We should note we don't need the OT precomputation for inference since that is also a large overhead if needed. }

\para{Generated Shape Normalization.}
To ensure a fair evaluation among different baselines in the unconditional task, we convert the inference results of various generative methods into a common coordinate domain.
%
For all baseline methods, including PVD~\cite{zhou2021pvd}, DiT-3D~\cite{mo2023dit3d}, LION~\cite{zeng2022lion}, and PSF~\cite{wu2023psf}, we respect the original normalization adopted in their training procedures.
%
Since all these methods compute a global mean coordinate and global scale ratio across all training shapes, we use these two quantities to reverse the normalization on the generated shape, based on the values obtained from the training set.
%
This procedure aligns with existing baselines, such as~\cite{yang2019pointflow,zeng2022lion}.


\section{Detailed assumptions made for the adversary}
\label{app:adversary_assumptions}

We clarify the capabilities of adversaries in model- and data-based attacks according to the threat model specified in Section~\ref{sec:preliminary}. We note: 
%
\begin{enumerate}
    \item A model-based attack is strictly more powerful than a data-based attack. This is because with access to the fine-tuned model $\theta$ and the prompt template $\textsf{p}(\cdot)$, a model-based attack can synthesize $\synthetic{\mathcal{D}}$ for any set of synthetic labels and perfectly simulate the membership inference experiment for a data-based attack.
    \item In both threat models, the adversary can train reference models $\{ \theta'_i \}_{i=1}^M$. This assumes access to the private dataset $D$, and the training procedure of target model $\theta$, including hyperparameters. This is made clear in line 3 in Algorithm~\ref{alg:mia}.
    \item In our experiments, we consider model-based attacks that use the prompt template \prompt{\cdot} to compute the model loss for target records, as specified in Sec.~\ref{subsec:model_score}. Our data-based attacks use the prompt template \prompt{\cdot} to generate synthetic data $\synthetic{D}$ from reference models.
    \item Only the model-based attack has query-access to the target model $\theta$. The attacks used in our experiments use $\theta$ to compute token-level predicted logits for input sequences and do not use white-box features, although this is not excluded by the threat model.
    \item Only the data-based attack generates synthetic data from reference models, so only this threat model leverages the sampling procedure $\textsf{sample}(\cdot)$. 
\end{enumerate} 

Table~\ref{tab:adversary_assumptions} summarizes the adversary capabilities used in the attacks in our experiments.

\begin{table*}[h!]
\centering
\begin{tabular}{p{8cm}cc}
\toprule
Assumptions & Model-based MIA & Data-based MIA \\ 
\midrule
Knowledge of the private dataset $D$ used to fine-tune the target model $\theta$ (apart from knowledge of canaries). & \checkmark & \checkmark \\
\midrule
Knowledge of the training procedure of target model $\theta$. & \checkmark & \checkmark \\
\midrule
Knowledge of the prompt template \prompt{\ell_i} used to generate the synthetic data. & \checkmark & \checkmark \\
\midrule
Query-access to target model $\theta$, returning predicted logits. & \checkmark & -- \\ 
\midrule
Access to synthetic data $\synthetic{D}$ generated by target model $\theta$. & -- & \checkmark \\ 
\midrule
Knowledge of the decoding strategy employed to sample synthetic data $\synthetic{D}$ (\eg, temperature, top-$k$). & -- & \checkmark \\ 
\bottomrule
\end{tabular}
\caption{Adversary capabilities effectively used by attacks in our experiments.}
\label{tab:adversary_assumptions}
\end{table*}
 

\section{Synthetic data utility}
\label{app:utility}

To ensure we audit the privacy of synthetic text data in a realistic setup, the synthetic data needs to bear high utility. We measure the synthetic data utility by comparing the downstream classification performance of RoBERTa-base~\citep{DBLP:journals/corr/abs-1907-11692} when fine-tuned exclusively on real or synthetic data. We fine-tune models for binary (SST-2) and multi-class classification (AG News) for 1 epoch on the same number of real or synthetic data records using a batch size of $16$ and learning rate $\eta = \num{1e-5}$. We report the macro-averaged AUC score and accuracy on a held-out test dataset of real records. 

Table~\ref{tab:utility_no_canaries} summarizes the results for synthetic data generated based on original data which does not contain any canaries. While we do see a slight drop in downstream performance when considering synthetic data instead of the original data, AUC and accuracy remain high for both tasks. 

\begin{table}[ht]
    \centering
    \begin{tabular}{ccrr}
    \toprule
        & \multirow{2}{*}{Fine-tuning data} & \multicolumn{2}{c}{Classification} \\
        \cmidrule(lr){3-4}
        Dataset &  & AUC & Accuracy \\
        \midrule 
        \multirow{2}{*}{\parbox{2cm}{\centering SST-2}} & Real & $0.984$ & \SI{92.3}{\percent} \\ 
         & Synthetic & $0.968$ & \SI{91.5}{\percent} \\
         \midrule
        \multirow{2}{*}{\parbox{2cm}{\centering AG News}} & Real & $0.992$ & \SI{94.4}{\percent} \\ 
         & Synthetic & $0.978$ & \SI{90.0}{\percent} \\ 
        \bottomrule
    \end{tabular}
    \caption{Utility of synthetic data generated from real data \emph{without} canaries. We compare the performance of text classifiers trained on real or synthetic data---both evaluated on real, held-out test data.}
    \label{tab:utility_no_canaries}
\end{table}

We further measure the synthetic data utility when the original data contains standard canaries (see Sec.~\ref{sec:baseline_results}). Specifically, we consider synthetic data generated from a target model trained on data containing \num{500} canaries repeated $n_\textrm{rep} = 12$ times, so \num{6000} data records. When inserting canaries with an artificial label, we remove all synthetic data associated with labels not present originally when fine-tuning the RoBERTa-base model. 

\begin{table}[h]
    \centering
    \begin{tabular}{ccc@{\hskip 15pt}rr}
    \toprule
        & \multicolumn{2}{c}{Canary injection} & \multicolumn{2}{c}{Classification}\\
        \cmidrule(lr){2-3} \cmidrule(lr){4-5}
        Dataset & Source & Label & AUC & Accuracy \\
        \midrule
        \multirow{3}{*}{\parbox{1cm}{\centering SST-2}} & \multicolumn{2}{l}{In-distribution} & $0.972$ & \SI{91.6}{\percent} \\ 
        \cmidrule{2-5}
         & \multirow{2}{*}{\parbox{1.8cm}{Synthetic}} & Natural & $0.959$ & \SI{89.3}{\percent} \\ 
         & & Artificial & $0.962$ & \SI{89.9}{\percent} \\ 
        \midrule
        \multirow{3}{*}{\parbox{2cm}{\centering AG News}} & \multicolumn{2}{l}{In-distribution} & $0.978$ & \SI{89.8}{\percent}\\ 
        \cmidrule{2-5} 
         & \multirow{2}{*}{\parbox{1.8cm}{Synthetic}} & Natural & $0.977$ & \SI{88.6}{\percent} \\ 
         & & Artificial & $0.980$ & \SI{90.1}{\percent} \\         
         \bottomrule
    \end{tabular}
    \caption{Utility of synthetic data generated from real data \emph{with} canaries ($n_\textrm{rep}=12$). We compare the performance of text classifiers trained on real or synthetic data---both evaluated on real, held-out test data.}
    \label{tab:utility_canaries}
\end{table}

Table~\ref{tab:utility_canaries} summarizes the results. Across all canary injection methods, we find limited impact of canaries on the downstream utility of synthetic data. While the difference is minor, the natural canary labels lead to the largest utility degradation. This makes sense, as the high perplexity synthetic sequences likely distort the distribution of synthetic text associated with a certain real label. In contrast, in-distribution canaries can be seen as up-sampling certain real data points during fine-tuning, while canaries with artificial labels merely reduce the capacity of the model to learn from real data and do not interfere with this process as much as canaries with natural labels do.
 

\section{Additional results for MIAs using standard canaries}
\label{app:add_mia_results}
In line with the literature on MIAs against machine learning models~\citep{carlini2022membership}, we also evaluate MIAs by their true positive rate (FPR) at low false positive rates (FPR). Tables~\ref{tab:tpr_fpr_0.01} and~\ref{tab:tpr_fpr_0.1} summarize the MIA TPR at FPR=\num{0.01} and FPR=\num{0.1}, respectively. We also provide the ROC curves for the data-based MIAs for both datasets, considering canaries with natural labels in Figure~\ref{fig:results_primary}.

\begin{table}[ht]
    \centering
    \begin{tabular}{ccccccc}
    \toprule
         & \multicolumn{2}{c}{Canary injection} & \multicolumn{4}{c}{TPR@FPR=0.01}\\
        \cmidrule(lr){2-3} \cmidrule(lr){4-7}
        &  &  & Model $\mathcal{A}^\theta$ & Synthetic $\mathcal{A}^{\synthetic{D}}$ & Synthetic $\mathcal{A}^{\synthetic{D}}$& Synthetic $\mathcal{A}^{\synthetic{D}}$ \\
        Dataset & Source & Label &   & (2-gram) & ($\textsc{SIM}_\textrm{Jac}$) & ($\textsc{SIM}_\textrm{emb}$)\\
        \midrule
        \multirow{3}{*}{\parbox{1cm}{\centering SST-2}} & \multicolumn{2}{l}{In-distribution} & $0.148$ & $0.104$ & $0.029$ & $0.020$ \\ 
        \cmidrule{2-7}
         & \multirow{2}{*}{\parbox{1.8cm}{Synthetic}} & Natural & $0.972$ & $0.042$ & $0.018$ & $0.024$ \\ 
         & & Artificial & $0.968$ & $0.057$ & $0.000$ & $0.030$ \\ 
        \midrule
        \multirow{3}{*}{\parbox{1.8cm}{\centering AG News}} & \multicolumn{2}{l}{In-distribution} & $0.941$ & $0.050$ & $0.032$ & $0.016$ \\ 
        \cmidrule{2-7} 
         & \multirow{2}{*}{\parbox{1.8cm}{Synthetic}} & Natural & $0.955$ & $0.049$ & $0.006$ & $0.016$ \\ 
         & & Artificial & $0.990$ & $0.053$ & $0.041$ & $0.022$ \\         
         \bottomrule
    \end{tabular}
    \caption{True positive rate (TPR) at a false positive rate (FPR) of 0.01 for experiments using standard canaries (Sec.~\ref{sec:baseline_results}) across training datasets, threat models (model-based adversary $\mathcal{A}^\theta$ and data-based adversary $\mathcal{A}^{\synthetic{D}}$) and MIA methodologies. Canaries are synthetically generated with target perplexity $\mathcal{P}_{ \textrm{target}}=250$, with no in-distribution prefix ($F=0$) and inserted $n_\textrm{rep}=12$ times.}
    \label{tab:tpr_fpr_0.01}
\end{table}

\begin{table}[ht]
    \centering
    \begin{tabular}{ccccccc}
    \toprule
         & \multicolumn{2}{c}{Canary injection} & \multicolumn{4}{c}{TPR@FPR=0.1}\\
         \cmidrule(lr){2-3} \cmidrule(lr){4-7}
        &  &  & Model $\mathcal{A}^\theta$ & Synthetic $\mathcal{A}^{\synthetic{D}}$ & Synthetic $\mathcal{A}^{\synthetic{D}}$& Synthetic $\mathcal{A}^{\synthetic{D}}$\\
        Dataset & Source & Label &   & (2-gram) & ($\textsc{SIM}_\textrm{Jac}$) & ($\textsc{SIM}_\textrm{emb}$)\\
        \midrule
        \multirow{3}{*}{\parbox{1cm}{\centering SST-2}} & \multicolumn{2}{l}{In-distribution} & $0.795$ & $0.406$ & $0.207$ & $0.203$ \\ 
        \cmidrule{2-7}
         & \multirow{2}{*}{\parbox{1.8cm}{Synthetic}} & Natural & $0.996$ & $0.191$ & $0.114$ & $0.128$ \\ 
         & & Artificial & $1.000$ & $0.277$ & $0.142$ & $0.142$ \\ 
        \midrule
        \multirow{3}{*}{\parbox{1.8cm}{\centering AG News}} & \multicolumn{2}{l}{In-distribution} & $0.982$ & $0.314$ & $0.158$ & $0.168$ \\ 
        \cmidrule{2-7} 
         & \multirow{2}{*}{\parbox{1.8cm}{Synthetic}} & Natural & $0.990$ & $0.271$ & $0.114$ & $0.114$ \\ 
         & & Artificial & $0.996$ & $0.323$ & $0.152$ & $0.164$ \\         
         \bottomrule
    \end{tabular}
    \caption{True positive rate (TPR) at a false positive rate (FPR) of 0.1 for experiments using standard canaries (Sec.~\ref{sec:baseline_results}) across training datasets, threat models (model-based adversary $\mathcal{A}^\theta$ and data-based adversary $\mathcal{A}^{\synthetic{D}}$) and MIA methodologies. Canaries are synthetically generated with target perplexity $\mathcal{P}_{ \textrm{target}}=250$, with no in-distribution prefix ($F=0$) and inserted $n_\textrm{rep}=12$ times.}
    \label{tab:tpr_fpr_0.1}
\end{table}

\begin{figure}[t]
    \centering
    \begin{subfigure}{0.4\textwidth}
        \centering
        \resizebox{\textwidth}{!}{\begin{tikzpicture}
\begin{axis}[
  xlabel = {FPR},
  ylabel = {TPR},
  grid = both,
  grid style = {line width=.1pt, draw=gray!10},
  major grid style = {line width=.2pt,draw=gray!50},
  legend style = {at={(1,0)}, anchor=south east},
  axis equal,
  xmin=0, xmax=1,
  ymin=0, ymax=1,
  width=8cm, % Adjust the width as needed
  height=8cm, % Adjust the height as needed
  ]
  \addplot[color_1_a, line width=1.2pt] table[x=fpr, y=tpr] {data/method/sst2/roc/2_gram.tsv};
  \addlegendentry{2-gram}
  \addplot[color_1_b, line width=1.2pt] table[x=fpr, y=tpr] {data/method/sst2/roc/emb.tsv};
  \addlegendentry{$\textsc{SIM}_{emb}$ - $k=25$}
  \addplot[color_1_c, line width=1.2pt] table[x=fpr, y=tpr] {data/method/sst2/roc/jac.tsv};
  \addlegendentry{$\textsc{SIM}_{jac}$ - $k=25$}
  \addplot[dashed, color=darkgray, line width=0.8pt, mark=none, samples=2] coordinates {(0, 0) (1, 1)};
\end{axis}
\end{tikzpicture}
}
        \caption{SST-2}
    \end{subfigure}
    \hspace{0.05\textwidth}
    \begin{subfigure}{0.4\textwidth}
        \centering
        \resizebox{\textwidth}{!}{\begin{tikzpicture}
\begin{axis}[
  xlabel = {FPR},
  ylabel = {TPR},
  grid = both,
  grid style = {line width=.1pt, draw=gray!10},
  major grid style = {line width=.2pt,draw=gray!50},
  legend style = {at={(1,0)}, anchor=south east},
  axis equal,
  xmin=0, xmax=1,
  ymin=0, ymax=1,
  width=8cm, % Adjust the width as needed
  height=8cm, % Adjust the height as needed
  ]
  \addplot[color_1_a, line width=1.2pt] table[x=fpr, y=tpr] {data/method/agnews/roc/2_gram.tsv};
  \addlegendentry{2-gram}
  \addplot[color_1_b, line width=1.2pt] table[x=fpr, y=tpr] {data/method/agnews/roc/emb.tsv};
  \addlegendentry{$\textsc{SIM}_{emb}$ - $k=25$}
  \addplot[color_1_c, line width=1.2pt] table[x=fpr, y=tpr] {data/method/agnews/roc/jac.tsv};
  \addlegendentry{$\textsc{SIM}_{jac}$ - $k=25$}
  \addplot[dashed, color=darkgray, line width=0.8pt, mark=none, samples=2] coordinates {(0, 0) (1, 1)};
\end{axis}
\end{tikzpicture}}
        \caption{AG News}
    \end{subfigure}
    \caption{
        MIA ROC curves across data-based MIA methodologies for the SST-2 (left) and AG News (right) datasets.
        Canaries are synthetically generated with target perplexity of $\mathcal{P}_{\textrm{target}}=250$ with a natural label, with no in-distribution prefix ($F=0$) and inserted $n_\textrm{rep}=12$ times.
    }
\label{fig:results_primary}
\end{figure}
 

\section{Ablations for MIAs on synthetic data}
\label{app:ablation}
1. Distilling accelerated residuals from slow residuals
2. only one of them active -> either vla or m2r2 

\section{Disparate vulnerability of standard canaries}
\label{app:disparate_vulnerability}

We analyze the disparate vulnerability of standard canaries between the model-based attack and the data-based attack that uses a 2-gram model (as discussed in Sec~\ref{sec:baseline_results}). Figure~\ref{fig:scatter_plot} plots the RMIA scores for both attacks on the same set of canaries, which have either been included in the training dataset of the target model (\emph{member}) or not (\emph{non-member}). Note that the RMIA scores are used to distinguish members from non-members, and that a larger value corresponds to the adversary being more confident in identifying a record as a member, \ie, to the record being more \emph{vulnerable}.

First, we note that the scores across both threat models exhibit a statistically significant, positive correlation. We find a Pearson correlation coefficient between the RMIA scores (log) for both methods of \num{0.20} ($p$-value of \num{2.4e-10}) and \num{0.23} ($p$-value of \num{1.9e-13}) for SST-2 and AG News, respectively. This means that a record vulnerable to the model-based attack tends to be also vulnerable to the data-based attack, even though the attacks differ substantially. 

Second, and more interestingly, some canaries have disparate vulnerability across MIA methods. Indeed, Figure~\ref{fig:scatter_plot} shows how certain data records which are not particularly vulnerable to the model-based attack are significantly more vulnerable to the data-based attack, and vice versa. 

\begin{figure*}[t]
    \centering
    \begin{subfigure}{0.45\textwidth}
        \centering
        \includegraphics[width=\textwidth]{figures/scatter_sst2_syntheticcanary_uniformlabel.pdf}
        \caption{SST-2}
    \end{subfigure}
    \hspace{0.05\textwidth}
    \begin{subfigure}{0.45\textwidth}
        \centering
        \includegraphics[width=\textwidth]{figures/scatter_agnews_syntheticcanary_uniformlabel.pdf}
        \caption{AG News}
    \end{subfigure}
    \caption{
        RMIA scores (log) for model- and data-based MIAs on the same set of canaries. Results for both datasets SST-2 and AG News. Canaries are synthetically generated with target perplexity of $\mathcal{P}_{\textrm{target}}=250$ with a natural label, and inserted $n_\textrm{rep}=12$ times.
    } 
    \label{fig:scatter_plot}
\end{figure*} 
 

\section{Low FPR ROC results}
\label{app:loglogplots}
Figure~\ref{fig:loglogroc_main} shows log-log plots of the ROC curves in Figure~\ref{fig:roc_curves_main} to better examine behavior of attacks at low FPR.

\begin{figure*}[htb]
  \centering
  \begin{subfigure}{0.32\textwidth}
    \centering
    \resizebox{\textwidth}{!}{\begin{tikzpicture}
\begin{axis}[
  xlabel = {FPR},
  ylabel = {TPR},
  xmode=log,
  ymode=log,
  grid = both,
  grid style = {line width=.1pt, draw=gray!10},
  major grid style = {line width=.2pt,draw=gray!50},
  legend style = {at={(1,0)}, anchor=south east},
  axis equal,
  xmin=5e-3, xmax=1,
  ymin=5e-3, ymax=1,
  width=8cm, % Adjust the width as needed
  height=8cm, % Adjust the height as needed
  ]
  \addplot[color_1_a, line width=1.2pt] table[x=fpr, y=tpr] {data/n_rep/sst2/roc/synthetic_2.tsv};
  \addlegendentry{$\mathcal{A}^{\synthetic{D}}, n_{\textnormal{rep}}=2$}
  \addplot[color_1_b, line width=1.2pt] table[x=fpr, y=tpr] {data/n_rep/sst2/roc/synthetic_4.tsv};
  \addlegendentry{$\mathcal{A}^{\synthetic{D}}, n_{\textnormal{rep}}=4$}
  \addplot[color_1_c, line width=1.2pt] table[x=fpr, y=tpr] {data/n_rep/sst2/roc/synthetic_8.tsv};
  \addlegendentry{$\mathcal{A}^{\synthetic{D}}, n_{\textnormal{rep}}=8$}
  \addplot[color_1_d, line width=1.2pt] table[x=fpr, y=tpr] {data/n_rep/sst2/roc/synthetic_16.tsv};
  \addlegendentry{$\mathcal{A}^{\synthetic{D}}, n_{\textnormal{rep}}=16$}
  \addplot[color_2_a, line width=1.2pt] table[x=fpr, y=tpr] {data/n_rep/sst2/roc/model_1.tsv};
  \addlegendentry{$\mathcal{A}^{\theta}, n_{\textnormal{rep}}=1$}
  \addplot[color_2_b, line width=1.2pt] table[x=fpr, y=tpr] {data/n_rep/sst2/roc/model_2.tsv};
  \addlegendentry{$\mathcal{A}^{\theta}, n_{\textnormal{rep}}=2$}
  \addplot[color_2_c, line width=1.2pt] table[x=fpr, y=tpr] {data/n_rep/sst2/roc/model_4.tsv};
  \addlegendentry{$\mathcal{A}^{\theta}, n_{\textnormal{rep}}=4$}
  \addplot[dashed, color=darkgray, line width=0.8pt, mark=none, samples=2] coordinates {(5e-3, 5e-3) (1, 1)};
\end{axis}
\end{tikzpicture}}
    \caption{
        Number of canary repetitions $n_\textrm{rep}$. \\
        $\mathcal{P}_\textrm{target} = 31$, $F=0$.
    }
    \label{subfig:loglogrepetitions_sst2}
  \end{subfigure}
  \begin{subfigure}{0.32\textwidth}
    \centering
    \resizebox{\textwidth}{!}{\begin{tikzpicture}
\begin{axis}[
  xlabel = {FPR},
  ylabel = {TPR},
  xmode=log,
  ymode=log,
  grid = both,
  grid style = {line width=.1pt, draw=gray!10},
  major grid style = {line width=.2pt,draw=gray!50},
  legend style = {at={(1,0)}, anchor=south east},
  axis equal,
  xmin=5e-3, xmax=1,
  ymin=5e-3, ymax=1,
  width=8cm, % Adjust the width as needed
  height=8cm, % Adjust the height as needed
  ]
  \addplot[color_1_a, line width=1.2pt] table[x=fpr, y=tpr] {data/canary_ppl/sst2/roc/perp_10_synthetic.tsv};
  \addlegendentry{$\mathcal{A}^{\synthetic{D}}, \mathcal{P}_{\textrm{tar}}=10$}
  \addplot[color_1_b, line width=1.2pt] table[x=fpr, y=tpr] {data/canary_ppl/sst2/roc/perp_100_synthetic.tsv};
  \addlegendentry{$\mathcal{A}^{\synthetic{D}}, \mathcal{P}_{\textrm{tar}}=10^2$}
  \addplot[color_1_c, line width=1.2pt] table[x=fpr, y=tpr] {data/canary_ppl/sst2/roc/perp_1000_synthetic.tsv};
  \addlegendentry{$\mathcal{A}^{\synthetic{D}}, \mathcal{P}_{\textrm{tar}}=10^3$}
  \addplot[color_1_d, line width=1.2pt] table[x=fpr, y=tpr] {data/canary_ppl/sst2/roc/perp_10000_synthetic.tsv};
  \addlegendentry{$\mathcal{A}^{\synthetic{D}}, \mathcal{P}_{\textrm{tar}}=10^4$}
  \addplot[color_2_a, line width=1.2pt] table[x=fpr, y=tpr] {data/canary_ppl/sst2/roc/perp_10_model.tsv};
  \addlegendentry{$\mathcal{A}^{\theta}, \mathcal{P}_{\textrm{tar}}=10$}
  \addplot[color_2_b, line width=1.2pt] table[x=fpr, y=tpr] {data/canary_ppl/sst2/roc/perp_100_model.tsv};
  \addlegendentry{$\mathcal{A}^{\theta}, \mathcal{P}_{\textrm{tar}}=10^2$}
  \addplot[color_2_c, line width=1.2pt] table[x=fpr, y=tpr] {data/canary_ppl/sst2/roc/perp_1000_model.tsv};
  \addlegendentry{$\mathcal{A}^{\theta}, \mathcal{P}_{\textrm{tar}}=10^3$}
  \addplot[color_2_d, line width=1.2pt] table[x=fpr, y=tpr] {data/canary_ppl/sst2/roc/perp_10000_model.tsv};
  \addlegendentry{$\mathcal{A}^{\theta}, \mathcal{P}_{\textrm{tar}}=10^4$}
  \addplot[dashed, color=darkgray, line width=0.8pt, mark=none, samples=2] coordinates {(5e-3, 5e-3) (1, 1)};
\end{axis}
\end{tikzpicture}}
    \caption{
        Canary perplexity $\mathcal{P}_\textrm{target}$. \\
        $n_\textrm{rep}^{\theta}=4$, $n_\textrm{rep}^{\synthetic{D}}=16$, $F=0$.
    }
    \label{subfig:loglogperplexity_sst2}
  \end{subfigure}
  \begin{subfigure}{0.32\textwidth}
    \centering
    \resizebox{\textwidth}{!}{\begin{tikzpicture}
\begin{axis}[
  xlabel = {FPR},
  ylabel = {TPR},
  xmode=log,
  ymode=log,
  grid = both,
  grid style = {line width=.1pt, draw=gray!10},
  major grid style = {line width=.2pt,draw=gray!50},
  legend style = {at={(1,0)}, anchor=south east},
  axis equal,
  xmin=5e-3, xmax=1,
  ymin=5e-3, ymax=1,
  width=8cm, % Adjust the width as needed
  height=8cm, % Adjust the height as needed
  ]
  \addplot[color_1_a, line width=1.2pt] table[x=fpr, y=tpr] {data/prefix/sst2/roc/prefix_0.tsv};
  \addlegendentry{$\mathcal{A}^{\synthetic{D}}, F=0$}
  \addplot[color_1_b, line width=1.2pt] table[x=fpr, y=tpr] {data/prefix/sst2/roc/prefix_10.tsv};
  \addlegendentry{$\mathcal{A}^{\synthetic{D}}, F=10$}
  \addplot[color_1_c, line width=1.2pt] table[x=fpr, y=tpr] {data/prefix/sst2/roc/prefix_20.tsv};
  \addlegendentry{$\mathcal{A}^{\synthetic{D}}, F=20$}
  \addplot[color_1_d, line width=1.2pt] table[x=fpr, y=tpr] {data/prefix/sst2/roc/prefix_30.tsv};
  \addlegendentry{$\mathcal{A}^{\synthetic{D}}, F=30$}
  \addplot[color_2_a, line width=1.2pt] table[x=fpr, y=tpr] {data/n_rep/sst2/roc/model_4.tsv};
  \addlegendentry{$\mathcal{A}^{\theta}, F=0$}
  \addplot[dotted, color=black, line width=1.2pt] table[x=fpr, y=tpr] {data/prefix/sst2/roc/incan.tsv};
  \addlegendentry{$\mathcal{A}^{\synthetic{D}}, F=\text{max}$}
  \addplot[dashed, color=darkgray, line width=0.8pt, mark=none, samples=2] coordinates {(5e-3, 5e-3) (1, 1)};
\end{axis}
\end{tikzpicture}}
    \caption{
        Canary in-distribution prefix $F$. \\
        $\mathcal{P}_\textrm{target}=31$, $n_\textrm{rep}^{\theta}=4$, $n_\textrm{rep}^{\synthetic{D}}=16$.
    }
    \label{subfig:loglogprefix_sst2}
  \end{subfigure}
  \begin{subfigure}{0.32\textwidth}
    \centering
    \resizebox{\textwidth}{!}{\begin{tikzpicture}
\begin{axis}[
  xlabel = {FPR},
  ylabel = {TPR},
  xmode = log,
  ymode = log,
  grid = both,
  grid style = {line width=.1pt, draw=gray!10},
  major grid style = {line width=.2pt,draw=gray!50},
  legend style = {at={(1,0)}, anchor=south east},
  axis equal,
  xmin=5e-3, xmax=1,
  ymin=5e-3, ymax=1,
  width=8cm, % Adjust the width as needed
  height=8cm, % Adjust the height as needed
  ]
  \addplot[color_1_a, line width=1.2pt] table[x=fpr, y=tpr] {data/n_rep/agnews/roc/synthetic_2.tsv};
  \addlegendentry{$\mathcal{A}^{\synthetic{D}}, n_{\textnormal{rep}}=2$}
  \addplot[color_1_b, line width=1.2pt] table[x=fpr, y=tpr] {data/n_rep/agnews/roc/synthetic_4.tsv};
  \addlegendentry{$\mathcal{A}^{\synthetic{D}}, n_{\textnormal{rep}}=4$}
  \addplot[color_1_c, line width=1.2pt] table[x=fpr, y=tpr] {data/n_rep/agnews/roc/synthetic_8.tsv};
  \addlegendentry{$\mathcal{A}^{\synthetic{D}}, n_{\textnormal{rep}}=8$}
  \addplot[color_1_d, line width=1.2pt] table[x=fpr, y=tpr] {data/n_rep/agnews/roc/synthetic_16.tsv};
  \addlegendentry{$\mathcal{A}^{\synthetic{D}}, n_{\textnormal{rep}}=16$}
  \addplot[color_2_a, line width=1.2pt] table[x=fpr, y=tpr] {data/n_rep/agnews/roc/model_1.tsv};
  \addlegendentry{$\mathcal{A}^{\theta}, n_{\textnormal{rep}}=1$}
  \addplot[color_2_b, line width=1.2pt] table[x=fpr, y=tpr] {data/n_rep/agnews/roc/model_2.tsv};
  \addlegendentry{$\mathcal{A}^{\theta}, n_{\textnormal{rep}}=2$}
  \addplot[color_2_c, line width=1.2pt] table[x=fpr, y=tpr] {data/n_rep/agnews/roc/model_4.tsv};
  \addlegendentry{$\mathcal{A}^{\theta}, n_{\textnormal{rep}}=4$}
  \addplot[dashed, color=darkgray, line width=0.8pt, mark=none, samples=2] coordinates {(5e-3, 5e-3) (1, 1)};
\end{axis}
\end{tikzpicture}}
    \caption{
        Number of canary repetitions $n_\textrm{rep}$. \\
        $\mathcal{P}_\textrm{target} = 31$, $F=0$.
    }
    \label{subfig:loglogrepetitions_agnews}
  \end{subfigure}
  \begin{subfigure}{0.32\textwidth}
    \centering
    \resizebox{\textwidth}{!}{\begin{tikzpicture}
\begin{axis}[
  xlabel = {FPR},
  ylabel = {TPR},
  xmode=log,
  ymode=log,
  grid = both,
  grid style = {line width=.1pt, draw=gray!10},
  major grid style = {line width=.2pt,draw=gray!50},
  legend style = {at={(1,0)}, anchor=south east},
  axis equal,
  xmin=5e-3, xmax=1,
  ymin=5e-3, ymax=1,
  width=8cm, % Adjust the width as needed
  height=8cm, % Adjust the height as needed
  ]
  \addplot[color_1_a, line width=1.2pt] table[x=fpr, y=tpr] {data/canary_ppl/agnews/roc/perp_10_synthetic.tsv};
  \addlegendentry{$\mathcal{A}^{\synthetic{D}}, \mathcal{P}_{\textrm{tar}}=10$}
  \addplot[color_1_b, line width=1.2pt] table[x=fpr, y=tpr] {data/canary_ppl/agnews/roc/perp_100_synthetic.tsv};
  \addlegendentry{$\mathcal{A}^{\synthetic{D}}, \mathcal{P}_{\textrm{tar}}=10^2$}
  \addplot[color_1_c, line width=1.2pt] table[x=fpr, y=tpr] {data/canary_ppl/agnews/roc/perp_1000_synthetic.tsv};
  \addlegendentry{$\mathcal{A}^{\synthetic{D}}, \mathcal{P}_{\textrm{tar}}=10^3$}
  \addplot[color_1_d, line width=1.2pt] table[x=fpr, y=tpr] {data/canary_ppl/agnews/roc/perp_10000_synthetic.tsv};
  \addlegendentry{$\mathcal{A}^{\synthetic{D}}, \mathcal{P}_{\textrm{tar}}=10^4$}
  \addplot[color_2_a, line width=1.2pt] table[x=fpr, y=tpr] {data/canary_ppl/agnews/roc/perp_10_model.tsv};
  \addlegendentry{$\mathcal{A}^{\theta}, \mathcal{P}_{\textrm{tar}}=10$}
  \addplot[color_2_b, line width=1.2pt] table[x=fpr, y=tpr] {data/canary_ppl/agnews/roc/perp_100_model.tsv};
  \addlegendentry{$\mathcal{A}^{\theta}, \mathcal{P}_{\textrm{tar}}=10^2$}
  \addplot[color_2_c, line width=1.2pt] table[x=fpr, y=tpr] {data/canary_ppl/agnews/roc/perp_1000_model.tsv};
  \addlegendentry{$\mathcal{A}^{\theta}, \mathcal{P}_{\textrm{tar}}=10^3$}
  \addplot[color_2_d, line width=1.2pt] table[x=fpr, y=tpr] {data/canary_ppl/agnews/roc/perp_10000_model.tsv};
  \addlegendentry{$\mathcal{A}^{\theta}, \mathcal{P}_{\textrm{tar}}=10^4$}
  \addplot[dashed, color=darkgray, line width=0.8pt, mark=none, samples=2] coordinates {(5e-3, 5e-3) (1, 1)};
\end{axis}
\end{tikzpicture}}
    \caption{
        Canary perplexity $\mathcal{P}_\textrm{target}$. \\
        $n_\textrm{rep}^{\theta}=4$, $n_\textrm{rep}^{\synthetic{D}}=16$, $F=0$.
    }
    \label{subfig:loglogperplexity_agnews}
  \end{subfigure}
  \begin{subfigure}{0.32\textwidth}
    \centering
    \resizebox{\textwidth}{!}{\begin{tikzpicture}
\begin{axis}[
  xlabel = {FPR},
  ylabel = {TPR},
  xmode=log,
  ymode=log,
  grid = both,
  grid style = {line width=.1pt, draw=gray!10},
  major grid style = {line width=.2pt,draw=gray!50},
  legend style = {at={(1,0)}, anchor=south east},
  axis equal,
  xmin=5e-3, xmax=1,
  ymin=5e-3, ymax=1,
  width=8cm, % Adjust the width as needed
  height=8cm, % Adjust the height as needed
  ]
  \addplot[color_1_a, line width=1.2pt] table[x=fpr, y=tpr] {data/prefix/agnews/roc/prefix_0.tsv};
  \addlegendentry{$\mathcal{A}^{\synthetic{D}}, F=0$}
  \addplot[color_1_b, line width=1.2pt] table[x=fpr, y=tpr] {data/prefix/agnews/roc/prefix_10.tsv};
  \addlegendentry{$\mathcal{A}^{\synthetic{D}}, F=10$}
  \addplot[color_1_c, line width=1.2pt] table[x=fpr, y=tpr] {data/prefix/agnews/roc/prefix_20.tsv};
  \addlegendentry{$\mathcal{A}^{\synthetic{D}}, F=20$}
  \addplot[color_1_d, line width=1.2pt] table[x=fpr, y=tpr] {data/prefix/agnews/roc/prefix_30.tsv};
  \addlegendentry{$\mathcal{A}^{\synthetic{D}}, F=30$}
  \addplot[color_2_a, line width=1.2pt] table[x=fpr, y=tpr] {data/n_rep/agnews/roc/model_4.tsv};
  \addlegendentry{$\mathcal{A}^{\theta}, F=0$}
  \addplot[dotted, color=black, line width=1.2pt] table[x=fpr, y=tpr] {data/prefix/agnews/roc/incan.tsv};
  \addlegendentry{$\mathcal{A}^{\synthetic{D}}, F=\text{max}$}
  \addplot[dashed, color=darkgray, line width=0.8pt, mark=none, samples=2] coordinates {(5e-3, 5e-3) (1, 1)};
\end{axis}
\end{tikzpicture}}
    \caption{
        Canary in-distribution prefix $F$. \\
        $\mathcal{P}_\textrm{target}=31$, $n_\textrm{rep}^{\theta}=4$, $n_\textrm{rep}^{\synthetic{D}}=16$.
    }
    \label{subfig:loglogprefix_agnews}
  \end{subfigure}
  \caption{
    Log-log ROC curves of MIAs on synthetic data $\mathcal{A}^{\synthetic{D}}$ compared to model-based MIAs $\mathcal{A}^{\theta}$ on SST-2 (\ref{subfig:loglogrepetitions_sst2}--\ref{subfig:loglogprefix_sst2}) and AG News (\ref{subfig:loglogrepetitions_agnews}--\ref{subfig:loglogprefix_agnews}).
    We ablate over the number of canary insertions $n_\textrm{rep}$ in \ref{subfig:loglogrepetitions_sst2}, \ref{subfig:loglogrepetitions_agnews}, the target perplexity $\mathcal{P}_\textrm{target}$  of the inserted canaries in \ref{subfig:loglogperplexity_sst2}, \ref{subfig:loglogperplexity_agnews} and the length $F$ of the in-distribution prefix in the canary in \ref{subfig:loglogprefix_sst2}, \ref{subfig:loglogprefix_agnews}.
  }
  \label{fig:loglogroc_main}
\end{figure*}


\section{Interpretability}
\label{app:Interpretability}
\input{appendix/Interpretability}

\end{document}