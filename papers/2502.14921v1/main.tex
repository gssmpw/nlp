\documentclass{article}

\usepackage{hyperref}

% Attempt to make hyperref and algorithmic work together better:
%\newcommand{\theHalgorithm}{\arabic{algorithm}}
\usepackage{algorithm}
\usepackage{algorithmic}

% Use the following line for the initial blind version submitted for review:
%\usepackage{icml2025}
\usepackage{arxiv,times}
\usepackage{subcaption}

% For theorems and such
\usepackage{amsmath}
\usepackage{amssymb}
\usepackage{mathtools}
\usepackage{amsthm}

% if you use cleveref..
\usepackage[capitalize,noabbrev]{cleveref}

%%%%%%%%%%%%%%%%%%%%%%%%%%%%%%%%
% THEOREMS
%%%%%%%%%%%%%%%%%%%%%%%%%%%%%%%%
\theoremstyle{plain}
\newtheorem{theorem}{Theorem}[section]
\newtheorem{proposition}[theorem]{Proposition}
\newtheorem{lemma}[theorem]{Lemma}
\newtheorem{corollary}[theorem]{Corollary}
\theoremstyle{definition}
\newtheorem{definition}[theorem]{Definition}
\newtheorem{assumption}[theorem]{Assumption}
\theoremstyle{remark}
\newtheorem{remark}[theorem]{Remark}

% Todonotes is useful during development; simply uncomment the next line
%    and comment out the line below the next line to turn off comments
%\usepackage[disable,textsize=tiny]{todonotes}
\usepackage[textsize=tiny]{todonotes}

\usepackage{multirow}
\usepackage[utf8]{inputenc} % allow utf-8 input
\usepackage[T1]{fontenc}    % use 8-bit T1 fonts
%\usepackage[pagebackref,colorlinks,citecolor=magenta]{hyperref}
\usepackage{doi}
\usepackage{url}            % simple URL typesetting
\usepackage{booktabs}       % professional-quality tables
\usepackage{amsfonts}       % blackboard math symbols
\usepackage{nicefrac}       % compact symbols for 1/2, etc.
\usepackage{microtype}      % microtypography
\usepackage{xcolor}         % colors
\usepackage{graphicx} 
\usepackage{array}
\usepackage{bm}
\usepackage{footnote}
\usepackage{siunitx}
\usepackage{tikz} % To generate the plot from csv
\usepackage{pgfplots}
\usepackage{pgf}
%\usepackage[noEnd=true,indLines=true]{algpseudocodex}
\usepackage{xspace}
\usepackage{enumitem}
\usepackage{threeparttable}
\usepackage{balance}

% Define a custom inline enumerate environment
\newlist{inlineenum}{enumerate*}{1}
\setlist[inlineenum,1]{label=(\roman*), itemjoin={{; }}, itemjoin*={{; and }}}

\setlist[enumerate,1]{label=\arabic*., leftmargin=*}

\makesavenoteenv{table}

\pgfplotsset{compat=1.5}

\newcolumntype{C}[1]{>{\centering\arraybackslash}p{#1}}

% Define colors for Class 1 (Blue shades)
\definecolor{color_1_a}{HTML}{003F5C} % Deep Blue
\definecolor{color_1_b}{HTML}{0072B2} % Medium Blue
\definecolor{color_1_c}{HTML}{66A9D9} % Light Blue
\definecolor{color_1_d}{HTML}{A4CFE3} % Very Light Blue

% Define colors for Class 2 (Orange shades)
\definecolor{color_2_a}{HTML}{8C2D04} % Deep Orange
\definecolor{color_2_b}{HTML}{D55E00} % Medium Orange
\definecolor{color_2_c}{HTML}{E89F73} % Light Orange
\definecolor{color_2_d}{HTML}{F4D0B5} % Very Light Orange

\newcommand{\lukas}[1]{\todo[inline, color=teal!30]{LW: #1}}
\newcommand{\matthieu}[1]{\todo[inline, color=teal!30]{MM: #1}}
\newcommand{\santiago}[1]{\todo[inline, color=teal!30]{SZ: #1}}

\newcommand{\synthetic}[1]{{\ensuremath{\widetilde{#1}}}}
\newcommand{\canary}[1]{{\ensuremath{\hat{#1}}}}
\newcommand{\prompt}[1]{{\ensuremath{\textsf{p}(#1)}}}

\newcommand{\ie}{i.e.\xspace}
\newcommand{\eg}{e.g.\xspace}
\newcommand{\wrt}{w.r.t.\xspace}
\newcommand{\etc}{etc.\xspace}

\renewcommand{\arraystretch}{0.9}
\setlength{\belowcaptionskip}{-5pt}

\title{The Canary’s Echo: Auditing Privacy Risks of LLM-Generated Synthetic Text}

\author{
    Matthieu Meeus\textsuperscript{1,2},
    Lukas Wutschitz\textsuperscript{1},
    Santiago Zanella-B{\'e}guelin\textsuperscript{1},
    Shruti Tople\textsuperscript{1},
    Reza Shokri\textsuperscript{1,3}\\[10pt]
    \textsuperscript{1}\textit{Microsoft} \qquad
    \textsuperscript{2}\textit{Imperial College London} \qquad
    \textsuperscript{3}\textit{National University of Singapore} 
}

\date{}

\begin{document}

\maketitle

\begin{abstract}
How much information about training samples can be gleaned from synthetic data generated by Large Language Models (LLMs)? Overlooking the subtleties of information flow in synthetic data generation pipelines can lead to a false sense of privacy. In this paper, we design membership inference attacks (MIAs) that target data used to fine-tune pre-trained LLMs that are then used to synthesize data, particularly when the adversary does not have access to the fine-tuned model but only to the synthetic data. We show that such data-based MIAs do significantly better than a random guess, meaning that synthetic data leaks information about the training data. Further, we find that canaries crafted to maximize vulnerability to model-based MIAs are sub-optimal for privacy auditing when only synthetic data is released. Such out-of-distribution canaries have limited influence on the model’s output when prompted to generate useful, in-distribution synthetic data, which drastically reduces their vulnerability. To tackle this problem, we leverage the mechanics of auto-regressive models to design canaries with an in-distribution prefix and a high-perplexity suffix that leave detectable traces in synthetic data. This enhances the power of data-based MIAs and provides a better assessment of the privacy risks of releasing synthetic data generated by LLMs.
\end{abstract}

\section{Introduction}
\section{Introduction}\label{sec:intro}

In computational finance, Monte Carlo simulations are used extensively to estimate the expected value of financial payoffs based on the solution of stochastic differential equations (SDEs) which model the evolution of stock prices, interest rates, exchange rates and other quantities \cite{glasserman04}.  Monte Carlo methods are very general and flexible, but for high accuracy it requires generating a large number of costly SDE path approximations, which has motivated research into a number of variance reduction or, equivalently, cost reduction techniques. One such method is
Multilevel Monte Carlo (MLMC), which was proposed in \cite{GILES2008} and was adapted for various applications that are summarised in \cite{Giles_overview17} and successfully combined with other methods such as quasi-Monte Carlo methods. The main idea of MLMC is to approximate the payoff using different time stepping resolutions when numerically solving the underlying SDE and to generate an optimal number of samples on each level, such that the overall computational cost is minimised subject to the desired bound on the variance. %, such that the total computational cost is minimised. 
The computational savings come from the fact that most samples are computed on the coarser levels and hence are less expensive while only a few samples from the finest levels are required \cite{GILES2008}.


Among the directions in which the computational cost 
of MLMC methods could further be reduced, an important avenue is the use of lower precision calculations, especially for the first Monte Carlo levels where the targeted accuracy is relatively low. 
 An overview of the research on mixed precision for the standard Monte Carlo (MC) framework is provided in \cite{ChowMixedPrecisionStandardMC} but only a few references study the potential of low precision computation in the MLMC framework \cite{Rounding_error_oliver}. To the best of our knowledge, the only MLMC framework with customised precision in the literature is \cite{brugger2014mixed}, but they use a uniform precision for all operations on each Monte Carlo level instead of optimising 
 the precision of each intermediary variable to reduce as much as possible the cost of path generation.
 
An important motivation for an MLMC framework with variable precision would be performing the low precision computations on reconfigurable hardware devices such as Field Programmable Gate Arrays (FPGAs). FPGAs contain customizable logic blocks and connectors that make it easy to adapt the digital circuit architecture for a specific application, leading to a highly parallel and optimised implementation. Therefore they are successfully exploited in applications that require high speed and have high computational workload, such as signal processing \cite{woods2008fpga}, and real time applications like high frequency trading \cite{HFT1,HFT2}. That is why a number of previous works in hardware architecture design implemented the MLMC algorithm to price financial options using FPGAs as accelerators, which resulted in improved speed and power efficiency compared to full CPU architectures \cite{Schryver2013AMM}. The paper \cite{lindsey2016domain} also proposed 
a Domain Specific Language to automate the configuration of FPGAs for this specific application. However, only \cite{brugger2014mixed} proposed a heuristic to reduce the precision in calculations.

In addition, all aforementioned works considered that the random number generation (RNG) is performed in single or double precision. Yet in most cases an important portion of the workload in the overall MLMC simulation comes from the RNG and in \cite{brugger2014mixed} this limited the total computational savings.
To reduce the cost of MLMC simulations in particular those based on the Geometric Brownian Motion (GBM), \cite{approximateICDF_Oliver, NestedOliver} have proposed to use approximate random numbers that are generated by applying an approximation of the inverse CDF to uniform random numbers. In \cite{NestedOliver}, the authors proposed a way to integrate these lower precision random variables into a \textit{nested} MLMC framework and completed a numerical analysis to bound the resulting error at each MC level by a product of the time step and the error in the random number approximation. The same authors show in \cite{approximateICDF_Oliver} that using approximate random variables reduces the cost of path generation by a factor 7.


In this paper we propose a nested MLMC framework that combines the use of approximate random normal variables and lower precision calculations to reduce the computational cost of MLMC even further than \cite{brugger2014mixed,NestedOliver}. We illustrate the efficiency of our framework in Matlab, after making several assumptions on the cost of operations and size of the errors that we carefully justify. We focus on the case of GBM and use the approximate RNG methods presented in \cite{approximateICDF_Oliver} as well as a new slightly modified method that combines CDF inversion and the central limit theorem. To choose the precision of the variables in the low precision path generation, we introduce a novel method to optimise the bit-widths. This optimisation is performed before the main path generation loop is executed and is based on a linear model of the payoff error  
due to rounding when computing in low precision. The error model relies on algorithmic differentiation in a similar manner to \cite{unifying-bwoptim,bitwidth-AD,ADAPT}. The bit-width optimisation procedure can be performed off-line, so this stage can be excluded from the on-line time complexity of our framework. The user specified desired accuracy is then enforced by calculating on-line the number of samples that need to be generated.

In terms of hardware design, we suggest implementing the low precision path generation on FPGAs and the full-precision ones on a CPU or GPU. 
The FPGA offers enough flexibility to define a separate bit-width for every variable in the low precision path generation, and can be reconfigured periodically to update the bit-widths when the market parameters have changed considerably. 


The paper is organized as follows : \Cref{sec:MLMC} introduces MLMC and nested MLMC to make clear the estimator that is implemented in our framework. Then in \Cref{sec:RNG} we detail the methods that could be used to obtain approximate random normally distributed numbers very cheaply for the low precision path generation. In \Cref{sec:error_model} and \Cref{sec:costModel} we propose an error model and a cost model (resp.) that we then use to formulate the optimisation problem that is solved to obtain the optimal bit-widths of fixed point variables in \Cref{sec:optimisation}. Finally we summarise our results and future directions in \Cref{sec:conclusion}.




\section{Background and problem statement}
\label{sec:preliminary}
\begin{figure*}[t]
\vskip 0.2in
\begin{center}
\centerline{\includegraphics[width=\textwidth]{Figures/pipeline-vlm-v4.pdf}}
\caption{Overview of our data-aware preference optimization. For each preference instance: (1) We first break the preferred and rejected response into sub-sentences by prompting a large language model (LLM); 
(2) Next, we estimate the similarity scores between each sub-sentence and the given image using the CLIP classifier, and then calculate the differences between the preferred and rejected response as the hardness of the data; 
(3) Finally, we incorporate the estimated hardness into the preference optimization process by modifying $\beta$ in Equ~\eqref{equ:dpo}, allowing the model to adjust based on the data hardness.}
\label{fig:pipleine-vlm}
\end{center}
\vskip -0.2in
\end{figure*}


\section{Preliminary}
\label{sec:preliminary}
In this section, we briefly review the MLLM preference learning procedure, which starts by sampling pairwise preference data with a supervised fine-turned (SFT) model, and then optimizes on such preference data. Specifically, we categorize this process into the following aspects:

\noindent \textbf{Supervised Fine-Tuning (SFT).}
Preference learning of an MLLM $\bm{\pi}$ begins with an SFT model $\bm{\pi}_{\text{SFT}}$. Concretely, the SFT process fine-tunes the pre-trained MLLM model with millions of multi-modal question-answer pairs to align LLM with multi-modal tasks. 
After this process, we construct preference data by sampling pair-wise preference responses from $\bm{\pi}_{\mathrm{SFT}}$, formalized as $(y_w, y_l) \sim \bm{\pi}_{\mathrm{SFT}}(y|x,\mathcal{I})$, where $(\mathcal{I}$ denotes the image and $x$ is the prompt question. 
Meanwhile, $(y_w, y_l)$ are labeled as preferred and less preferred responses by humans, formalized as $(y_w \succ  y_l | \mathcal{I}, x)$.

\noindent \textbf{RLHF with Reward Models.}
Given pair-wise preference data $(y_w, y_l) \sim \bm{\pi}_{\mathrm{SFT}}(y|x,\mathcal{I})$, the preference learning process can be described in 2 stages: reward modeling and preference optimization. 
Specifically, the reward model $r_{\bm{\theta}}(y|\mathcal{I}, x)$ is defined to rank the model responses by learning to distinguish $y_w$ from $y_l$, and the preference optimization aims to distill the preference knowledge into MLLM. 
To learn a reward model, pioneering work \cite{rlhf} employs the Bradley-Terry model \cite{BT_model} to model the pair-wise preference distribution as:
\begin{equation}
\resizebox{.9\hsize}{!}{
\begin{math}
\begin{aligned}
    \mathrm{P}(y_w \succ  y_l|\mathcal{I}, x) & =  \sigma(r^{*}(y_w|\mathcal{I}, x)- (r^{*}(y_l|\mathcal{I}, x)) \\
     & = \frac{\mathrm{exp}(r^{*}(y_w|\mathcal{I}, x))}{\mathrm{exp}(r^{*}(y_w|\mathcal{I}, x))+\mathrm{exp}(r^{*}(y_l|\mathcal{I}, x))}.
\end{aligned}
\end{math}
}
\end{equation}

Thus, the learning process can be achieved by minimizing the negative log-likelihood $-\mathrm{logP}(y_w \succ y_l|\mathcal{I}, x)$ over the preference data with the parametrized reward model $r_{\bm{\phi}}(y_w|\mathcal{I}, x)$ initialized as $\bm{\pi}_{\mathrm{SFT}}$ with a simple linear layer to produce reward prediction. 
With the well-optimized reward model $r_{\phi}^{*}(y|\mathcal{I}, x)$, prior work \cite{rlhf} proposes to employ policy optimization algorithms in RL such as PPO \cite{PPO} to maximize the learned reward with KL-penalty, which can be formalized as:
\begin{equation}
\label{equ:ppo}
\begin{aligned}
    \underset{\bm{\pi}_{\theta}}{\text{max}} & \  \mathbf{E}_{(\mathcal{I},x) \sim \mathcal{D}, y \sim \bm{\pi}_{\theta}(\cdot|\mathcal{I}, x)} [r_{\phi}^{*}(y|\mathcal{I}, x)] \\
    & -\beta \mathbb{D}_{\mathbf{KL}}[\bm{\pi}_{\theta}(y|\mathcal{I},x)||\bm{\pi}_{\text{ref}}(y|\mathcal{I},x)], 
\end{aligned}
\end{equation}
where the fixed reference model $\bm{\pi}_{\text{ref}}$ is parameterized as $\bm{\pi}_{\text{SFT}}$, and the hyper-parameter $\beta$ controls the deviation of $\bm{\pi}_{\theta}$ from $\bm{\pi}_{\text{ref}}$ during the optimization process.

\noindent \textbf{Direct Preference Optimization (DPO).}
To relieve the high computational complexity of reward training in RLHF, DPO \cite{DPO} is proposed, which provides a simple way to directly optimize $\bm{\pi}_{\theta}$ with the pair-wise preference data, without parametrized reward model. Specifically, the DPO loss can be described as:
\begin{equation}
\label{equ:dpo}
\begin{aligned}
    \mathcal{L}_{\mathrm{dpo}} = - \bm{\mathrm{E}}_{(\mathcal{I},x, y_{w}, y_{l})} [ {\log \sigma}( & \beta \log \frac{{\pi}_{\bm{\theta}}(y_{w}|\mathcal{I},x)}{{\pi}_{\mathrm{ref}}(y_{w}|\mathcal{I},x)} \\
    - & \beta \log \frac{{\pi}_{\bm{\theta}}(y_{l}|\mathcal{I},x)}{{\pi}_{\mathrm{ref}}(y_{l}|\mathcal{I},x)}) ].
\end{aligned}
\end{equation}

\section{Methodology}
\section{Method}

\begin{figure*}[t]
    \centering
    \includegraphics[width=\linewidth]{figures/pipeline.png} \hfill

    \caption{An overview of our data synthesis pipeline. Starting from our seed data, we select a reference sample and collect \textsc{Reference-Level Feedback} on both the instruction and response. Instruction feedback is used to synthesize new instructions. We generate their corresponding responses, and then improve it using the response feedback.}
    \label{fig:pipeline}
\end{figure*}

In this section, we present our data synthesis pipeline that leverages \textsc{Reference-Level Feedback} to generate high-quality instruction-response pairs. An overview of the pipeline is presented in Figure \ref{fig:pipeline}, and the steps are detailed in the following subsections. Complete examples for each step can be found in Appendix \ref{sec:appendix_examples}, and the prompts used for each section can be found in Appendix \ref{sec:appendix_prompt_templates}.


\subsection{Feedback Collection}

Our pipeline begins with a seed dataset -- a small collection of carefully curated instruction-response pairs that serve as exemplars for synthesized data samples. It can be either manually crafted by human annotators or automatically selected using quality-based criteria. These reference samples are high-quality and exhibit desirable characteristics such as clarity and relevance, which we aim to replicate in our synthetic data. For \textsc{Reference-Level Feedback}, we systematically identify and capture such qualities through a framework that identifies the strength of each sample, as well as potential areas for improvement.

Unlike traditional approaches that collect feedback on generated responses at the sample-level, our method identifies the qualities that make reference samples high-quality and uses it for feedback. This feedback captures a richer signal than feedback collected at the sample-level, establishing higher quality standards for synthesis and providing more effective guidance for generating training data that exhibits similar properties to the reference samples.

For each reference sample in the seed dataset, we collect \textsc{Reference-Level Feedback} from both the instruction and the response:

\textbf{Instruction Feedback.} To collect feedback from a reference instruction and capture essential features that make it effective for training, we analyze key attributes (e.g., clarity and actionability). We also ensure comprehensive coverage along a wide breadth by collecting feedback along two dimensions: relevant subject areas (e.g. cellular biology, csv file manipulation, legislative processes) and relevant skills necessary to respond to the instruction (e.g. understanding of specific tools, knowledge of processes, analysis). This enables us to systematically identify desirable characteristics of instructions while maximizing the breadth of instruction types.

\textbf{Response Feedback.} When collecting feedback from a reference response, we identify key qualities that make it an effective response to the instruction. We evaluate along multiple critical dimensions, including factual accuracy, relevance to the instruction, and comprehensiveness. This feedback captures both the strengths of the reference response and specific areas where it can be improved upon.


\subsection{Data Synthesis}
Now, we use the collected \textsc{Reference-Level Feedback} from the previous stage to synthesize new data samples, while maintaining the quality standards established by our reference data. For each reference sample and its corresponding feedback, we employ a two-phase synthesis process, as illustrated in Figure \ref{fig:pipeline}:

\begin{enumerate}
    \item \textbf{Instruction Synthesis.} We provide an LLM the reference instruction as an example and the instruction feedback as guidelines to synthesize new instructions that maintain the qualities specified in the feedback. As depicted in Step 2 of Figure \ref{fig:pipeline}, we synthesize 10 new instructions for \textbf{subject-based} feedback, which produces instructions that align with the subject areas of the reference response. We also synthesize 10 new instructions for \textbf{skill-based} feedback, which produces instructions that align with the skills needed to respond to the reference instruction.
    
    \item \textbf{Response Synthesis and Refinement.} For each synthesized instruction, we first generate an initial response. We then enhance this response using the reference response feedback, instructing the language model to analyze the feedback and incorporate the relevant aspects. This process is shown in Step 3 of Figure \ref{fig:pipeline}.
    
    \paragraph{Note on relevance of response feedback.}
    Although the response feedback was originally collected for the reference response, many aspects of it can still remain applicable because of the shared characteristics between the reference and synthesized instructions. We acknowledge that not all feedback elements may transfer, and to account for this, we explicitly instruct the model to selectively apply only the relevant aspects of the feedback and ignore the irrelevant aspects. An example of this can be found in \ref{sec:appendix_examples}.
\end{enumerate}

This synthesis process enables us to synthesize new data, while systematically propagating the high-quality characteristics of reference samples.

\subsection{Theoretical Efficiency Analysis}
Our presented pipeline for data synthesis with \textsc{Reference-Level Feedback} is significantly more efficient than using traditional sample-level feedback methods, specifically in the frequency of feedback collection. While sample-level approaches require feedback for every synthesized sample, our method only requires feedback once for every reference sample. This translates to a reduction from $O(n)$ feedback collections, where $n$ represents the number of synthesized samples, to $O(1)$. However, it is also important to note that this efficiency gain comes with an initial fixed cost of collecting and curating seed data.

\section{Experimental setup}
\label{sec:exp_setup}
\section{Experimental Setups}
In this section, we introduce the dataset and evaluation metrics we used to assess the robustness performance of various robust training strategies.
\subsection{Datasets and Evaluation Metrics}
For our experiment, we evaluate on four widely-used question answering datasets: (1) single-hop QA, including NaturalQuestions (NQ)~\cite{kwiatkowski-etal-2019-natural} and WebQuestions~\cite{berant-etal-2013-semantic}; and (2) multi-hop QA, including TriviaQA~\cite{joshi-etal-2017-triviaqa} and HotpotQA~\cite{yang-etal-2018-hotpotqa}. All experimental results are evaluated on their development splits using the Exact Match (EM) and F1 metrics. Detailed statistics of these datasets are listed in Appendix~\ref{sec:dataset_stats} Table~\ref{tab:dataset_stats}.

% \subsection{RAG Pipeline}
% Our experiment implements a standard two-stage RAG framework. The pipeline consists of a retrieval phase and a generation phase. For the retrieval component, we leverage Contriever~\cite{DBLP:journals/tmlr/IzacardCHRBJG22}, an advanced BERT-based dense retriever that employs unsupervised contrastive learning for document representation.

% The retrieval process operates on a Wikipedia-based knowledge corpus, specifically utilizing the Wikipedia-2018 dataset. We preprocess this corpus by segmenting Wikipedia articles into non-overlapping passages, each containing 100 words, while preserving the integrity of potentially misleading passages. To enable efficient retrieval, we generate document embeddings using Contriever and index them using FAISS\footnote{\url{https://github.com/facebookresearch/faiss}}, a powerful similarity search library. During both training and development phases, Contriever retrieves the top-20 most relevant documents for each query from our indexed Wikipedia corpus.

% For the downstream question-answering tasks, we select the top-5 retrieved documents and concatenate them with the test query to form the input context. The model then processes this enriched context to generate appropriate answers. The generator prompt we used is shown in Appendix~\ref{sec:prompt} Table~\ref{tab:generator_prompt}.

\subsection{RAG Pipeline}  
We implement a standard two-stage RAG framework with retrieval and generation phases. For retrieval, we use Contriever~\cite{DBLP:journals/tmlr/IzacardCHRBJG22}, a BERT-based dense retriever trained with unsupervised contrastive learning. Our knowledge corpus is Wikipedia-2018, preprocessed into 100-word non-overlapping passages. We encode these passages using Contriever and index them with FAISS\footnote{\url{https://github.com/facebookresearch/faiss}} for efficient retrieval. During training and development, the top-20 relevant documents are retrieved for each query. For question-answering, the top-5 retrieved documents are concatenated with the query as input for answer generation. The generator prompt is detailed in Appendix~\ref{sec:prompt} Table~\ref{tab:generator_prompt}.

% \subsection{Foundation Models}
% In our experiments, we employ four state-of-the-art foundation models: \texttt{Llama-2-7b-chat-hf}, \texttt{Llama-3-8B-Instruct}, \texttt{Qwen1.5-7B-Chat}, and \texttt{Qwen2.5-7B-Instruct}. These models represent different generations of LLMs from two prominent model families - Llama and Qwen, each with comparable model sizes around 7-8 billion parameters.

% \subsection{Robust RAG Training Scenario}

% To comprehensively evaluate the robustness of RAG systems under different document selection strategies, we design seven training scenarios across three categories.

% For basic QA capability evaluation, we directly use the \textbf{Base Model} in a zero-shot setting, where the foundation model performs inference without any RAG-specific training.

% For scenarios with relevant documents, we explore four different strategies:
% \begin{itemize}[leftmargin=0.5cm, itemindent=0cm]
%     \item \textbf{RALM}: This approach incorporates instruction tuning by prepending golden retrieval text to the context during model fine-tuning. It is worth noting that queries without golden documents in their top-20 retrieved documents are excluded from training, resulting in a smaller training set compared to other methods.
%     \item \textbf{RetRobust}: Based on \newcite{DBLP:conf/iclr/YoranWRB24}, this method enhances model robustness by exposing it to diverse retrieval qualities during training. For each query, it randomly selects between top-ranked, low-ranked, or random passages with equal probability.
%     \item \textbf{Top-1 Document}: This scenario utilizes the document with the highest similarity score from the retrieved document pool. Note that this document may not necessarily contain the correct answer, reflecting real-world retrieval scenarios where the most similar document might not always be the most helpful one.
%     \item \textbf{Golden Document}: This strategy primarily selects the document that both contains the correct answer and has the highest relevance score among the top-20 retrieved documents. In cases where no document containing the correct answer exists in the top-20 retrieved documents, we fall back to using the top-1 document to maintain consistent training sample sizes across different methods.
% \end{itemize}

% \begin{table*}[t]
\centering
\resizebox{\linewidth}{!}{
\begin{tabular}{clcccccccccc}
\toprule
\multirow{2}{*}{\textbf{Model}} & \multirow{2}{*}{\textbf{RAG Scenario}} & \multicolumn{2}{c}{\textbf{HotpotQA}} & \multicolumn{2}{c}{\textbf{NQ}} & \multicolumn{2}{c}{\textbf{WebQuestions}} & \multicolumn{2}{c}{\textbf{TriviaQA}} & \multicolumn{2}{c}{\textbf{AVERAGE}}\\
\cline{3-4}\cline{5-6}\cline{7-8}\cline{9-10}\cline{11-12}
 & &  \textbf{EM} & \textbf{F1} & \textbf{EM} & \textbf{F1} & \textbf{EM} & \textbf{F1} & \textbf{EM} & \textbf{F1} & \textbf{EM} & \textbf{F1} \\
\midrule
\multirow{10}{*}{\rotatebox{90}{\textit{Llama-2-7b-chat-hf}}} & Base Model &  3.30  &  12.34 &  1.21 &  10.61  &  0.00  &  13.08  &  4.32  &  20.27 & 2.21 & 14.08  \\
 & RALM &  26.21   & 36.42  &  32.17 &  42.68 &  33.81  &  45.85  &  50.28  &  60.17 & 35.62 & 46.28  \\
 & RetRobust &  31.29  &  43.65  &  37.71  & 49.49  &  36.33  &  47.98  &  57.61  & 67.52 & 40.74 & 52.16  \\
 & Top-1 Doc & 31.76 & 43.95 & 40.20 & 51.89  &  41.73  &  52.76  &  52.93  & 65.41 & 41.66 & 53.50  \\
 & Golden Doc & 30.67 & 42.78 & 36.50 & 47.77  &  39.93  &  52.11  &  50.25  & 63.28 & 39.34 & 51.49  \\
 \cdashline{2-12}
 & Random Doc & 30.94 & 43.11 & 38.16 & 49.78  &  42.45  &  53.97  &  52.72 &  65.52 & 41.07 & 53.10  \\
 & Irrelevant Doc & 31.01 & 42.98 & 37.08 & 48.93  &  39.21   &  50.79  &  51.97   &  64.70 & 39.82 & 51.85 \\
 \cdashline{2-12}
 & RAAT & 31.32 & 43.24 & 42.91 & 53.19  &  36.69  &  48.82 &  51.65 &  58.71 & 40.64 & 50.99  \\
 & IRM & 34.38 & 47.11 & 40.96 & 53.07  &  53.96  &  61.62 &  57.58 &  69.08 & 46.72 & 57.72  \\
 \cmidrule{2-12}
% & \textcolor{mycolor}{$\Delta$ (Random $\rightarrow$ Best)}  &  \textcolor{mycolor}{2.65\%}  & \textcolor{mycolor}{1.95\%}  &  \textcolor{mycolor}{5.35\%}  &  \textcolor{mycolor}{4.24\%}  &  \textcolor{mycolor}{0.85\%}   &  \textcolor{mycolor}{0.30\%}  &  \textcolor{mycolor}{9.28\%}  &  \textcolor{mycolor}{3.05\%} & \textcolor{mycolor}{4.53\%} & \textcolor{mycolor}{2.38\%}  \\
& \textcolor{mycolor}{$\Delta$ (Worst $\rightarrow$ Best)}  &  \textcolor{mycolor}{16.19\%}  & \textcolor{mycolor}{12.89\%}  &  \textcolor{mycolor}{21.87\%}  &  \textcolor{mycolor}{13.19\%}  &  \textcolor{mycolor}{48.53\%}   &  \textcolor{mycolor}{28.43\%}  &  \textcolor{mycolor}{13.55\%}  &  \textcolor{mycolor}{17.66\%} & \textcolor{mycolor}{15.95\%} & \textcolor{mycolor}{13.20\%}  \\
\midrule
\multirow{10}{*}{\rotatebox{90}{\textit{Llama-3-8B-Instruct}}} & Base Model & 23.31 & 32.60 & 30.04 & 41.59 & 26.98 & 43.25 & 58.80 & 66.45 & 34.78 & 45.97 \\
 & RALM &  27.64 &  38.10  &  35.19  &  46.10  &   47.84  &  56.98   &  54.75 &  63.27  & 41.36 & 51.11 \\
 & RetRobust &  36.06 &  48.99  &  43.28  &  55.04  &  52.88  &  62.10  &  59.06  & 67.77 & 47.82 & 58.48  \\
 & Top-1 Doc & 36.72 & 49.30 & 44.38 & 56.20  &  54.68  &  62.26 &  60.80  & 68.31 & 49.15 & 59.02  \\
 & Golden Doc & 35.52 & 48.31 & 41.35 & 53.13  &  48.92  &  58.41  &  58.26 & 66.99 &  46.01 & 56.71 \\
 \cdashline{2-12}
 & Random Doc & 35.98 & 49.05 & 43.37 & 55.43  &  53.24  &   62.55 &  60.62  & 68.64 & 48.30 & 58.92  \\
 & Irrelevant Doc & 35.31 & 47.92 & 42.45 & 54.41  &  46.76  &  57.67  &  58.97  & 66.57 & 45.87 & 56.64 \\
 \cdashline{2-12}
 & RAAT & 32.20 & 43.81 & 42.34 & 53.31  &  48.28  &  58.17 &  54.41 &  62.45 & 44.31 & 54.44  \\
 & IRM & 35.19 & 48.08 & 41.13 & 53.14  &  53.96  &  61.64  &  57.15 &  69.13 & 46.86 & 58.00  \\
\cmidrule{2-12}
% & \textcolor{mycolor}{$\Delta$ (Random $\rightarrow$ Best)}  &  \textcolor{mycolor}{2.06\%}  & \textcolor{mycolor}{0.51\%}  &  \textcolor{mycolor}{2.33\%}  &  \textcolor{mycolor}{1.39\%}  &  \textcolor{mycolor}{2.70\%}   &  \textcolor{mycolor}{0.00\%}  &  \textcolor{mycolor}{0.30\%}  &  \textcolor{mycolor}{0.00\%}  & \textcolor{mycolor}{1.85\%} & \textcolor{mycolor}{0.47\%} \\
& \textcolor{mycolor}{$\Delta$ (Worst $\rightarrow$ Best)}  &  \textcolor{mycolor}{14.04\%}  & \textcolor{mycolor}{12.53\%}  &  \textcolor{mycolor}{7.90\%}  &  \textcolor{mycolor}{5.76\%}  &  \textcolor{mycolor}{16.94\%}   &  \textcolor{mycolor}{8.46\%}  &  \textcolor{mycolor}{11.74\%}  &  \textcolor{mycolor}{10.70\%}  & \textcolor{mycolor}{10.92\%} & \textcolor{mycolor}{8.42\%} \\
\bottomrule
\end{tabular}}
\caption{Performance comparison of different LLMs (\texttt{Llama-2-7b-chat-hf} and \texttt{Llama-3-8B-Instruct}) across different robust RAG scenarios on four datasets (HotpotQA, NQ, WebQuestions, and TriviaQA). The row \textcolor{mycolor}{$\Delta$ (Worst $\rightarrow$ Best)} indicates the performance gain achieved by the best method compared to the worst strategy among these training strategies, representing the benefit of sophisticated robust training methods.}
\label{tab:main_result_llama}
\vspace{-0.4cm}
\end{table*}


% To evaluate the model's robustness against potentially harmful or irrelevant information, we design two adversarial scenarios:
% \begin{itemize}[leftmargin=0.5cm, itemindent=0cm]
%     \item \textbf{Random Document}: This approach randomly selects one document from the retrieved documents, simulating unpredictable retrieval quality in real-world scenarios.
%     \item \textbf{Irrelevant Document}: This scenario selects an irrelevant passage by randomly choosing from retrieval contents of other queries, ensuring no relevance to the current query.
% \end{itemize}

% These scenarios are designed to evaluate model performance across three distinct aspects: (1) the basic QA capability without retrieval augmentation, (2) the model's ability to leverage relevant documents with varying degrees of utility, and (3) the model's robustness against potentially harmful or irrelevant information. This comprehensive evaluation framework allows us to assess both the effectiveness and robustness of RAG systems under different operational conditions.

\subsection{Robust RAG Training Setups}
% We evaluate two popular robust training settings:
For our experiments, we evaluate two popular robust training settings:
\subsubsection{Document Selection Strategies}
% To assess RAG robustness under various document selection strategies, we design seven training scenarios across three categories.
We explore several document selection training strategies across three categories.

For basic QA capability evaluation, we use the \textbf{Base Model} in a few-shot setting, where the model performs inference without RAG-specific training.

For scenarios with relevant documents, we explore four strategies:
\begin{itemize}[leftmargin=0.5cm, itemindent=0cm, itemsep=0pt]
    \item \textbf{RALM}: Fine-tunes the model by prepending golden retrieval text. Queries without golden documents in the top-20 are excluded, leading to reduced training set size.
    \item \textbf{RetRobust}: Following \newcite{DBLP:conf/iclr/YoranWRB24}, this method enhances robustness by randomly selecting top-ranked, low-ranked, or random passages during training.
    \item \textbf{Top-1 Document}: Uses the highest-scoring retrieved document, which may not contain the correct answer, reflecting real-world retrieval challenges.
    \item \textbf{Golden Document}: Selects the most relevant document containing the correct answer. If none exist in the top-20, it defaults to the top-1 document for consistency.
\end{itemize}

\begin{table*}[t]
\centering
\resizebox{\linewidth}{!}{
\begin{tabular}{clcccccccccc}
\toprule
\multirow{2}{*}{\textbf{Model}} & \multirow{2}{*}{\textbf{RAG Scenario}} & \multicolumn{2}{c}{\textbf{HotpotQA}} & \multicolumn{2}{c}{\textbf{NQ}} & \multicolumn{2}{c}{\textbf{WebQuestions}} & \multicolumn{2}{c}{\textbf{TriviaQA}} & \multicolumn{2}{c}{\textbf{AVERAGE}}\\
\cline{3-4}\cline{5-6}\cline{7-8}\cline{9-10}\cline{11-12}
 & &  \textbf{EM} & \textbf{F1} & \textbf{EM} & \textbf{F1} & \textbf{EM} & \textbf{F1} & \textbf{EM} & \textbf{F1} & \textbf{EM} & \textbf{F1} \\
\midrule
\multirow{10}{*}{\rotatebox{90}{\textit{Llama-2-7b-chat-hf}}} & Base Model &  3.30  &  12.34 &  1.21 &  10.61  &  0.00  &  13.08  &  4.32  &  20.27 & 2.21 & 14.08  \\
 & RALM &  26.21   & 36.42  &  32.17 &  42.68 &  33.81  &  45.85  &  50.28  &  60.17 & 35.62 & 46.28  \\
 & RetRobust &  31.29  &  43.65  &  37.71  & 49.49  &  36.33  &  47.98  &  57.61  & 67.52 & 40.74 & 52.16  \\
 & Top-1 Doc & 31.76 & 43.95 & 40.20 & 51.89  &  41.73  &  52.76  &  52.93  & 65.41 & 41.66 & 53.50  \\
 & Golden Doc & 30.67 & 42.78 & 36.50 & 47.77  &  39.93  &  52.11  &  50.25  & 63.28 & 39.34 & 51.49  \\
 \cdashline{2-12}
 & Random Doc & 30.94 & 43.11 & 38.16 & 49.78  &  42.45  &  53.97  &  52.72 &  65.52 & 41.07 & 53.10  \\
 & Irrelevant Doc & 31.01 & 42.98 & 37.08 & 48.93  &  39.21   &  50.79  &  51.97   &  64.70 & 39.82 & 51.85 \\
 \cdashline{2-12}
 & RAAT & 31.32 & 43.24 & 42.91 & 53.19  &  36.69  &  48.82 &  51.65 &  58.71 & 40.64 & 50.99  \\
 & IRM & 34.38 & 47.11 & 40.96 & 53.07  &  53.96  &  61.62 &  57.58 &  69.08 & 46.72 & 57.72  \\
 \cmidrule{2-12}
% & \textcolor{mycolor}{$\Delta$ (Random $\rightarrow$ Best)}  &  \textcolor{mycolor}{2.65\%}  & \textcolor{mycolor}{1.95\%}  &  \textcolor{mycolor}{5.35\%}  &  \textcolor{mycolor}{4.24\%}  &  \textcolor{mycolor}{0.85\%}   &  \textcolor{mycolor}{0.30\%}  &  \textcolor{mycolor}{9.28\%}  &  \textcolor{mycolor}{3.05\%} & \textcolor{mycolor}{4.53\%} & \textcolor{mycolor}{2.38\%}  \\
& \textcolor{mycolor}{$\Delta$ (Worst $\rightarrow$ Best)}  &  \textcolor{mycolor}{16.19\%}  & \textcolor{mycolor}{12.89\%}  &  \textcolor{mycolor}{21.87\%}  &  \textcolor{mycolor}{13.19\%}  &  \textcolor{mycolor}{48.53\%}   &  \textcolor{mycolor}{28.43\%}  &  \textcolor{mycolor}{13.55\%}  &  \textcolor{mycolor}{17.66\%} & \textcolor{mycolor}{15.95\%} & \textcolor{mycolor}{13.20\%}  \\
\midrule
\multirow{10}{*}{\rotatebox{90}{\textit{Llama-3-8B-Instruct}}} & Base Model & 23.31 & 32.60 & 30.04 & 41.59 & 26.98 & 43.25 & 58.80 & 66.45 & 34.78 & 45.97 \\
 & RALM &  27.64 &  38.10  &  35.19  &  46.10  &   47.84  &  56.98   &  54.75 &  63.27  & 41.36 & 51.11 \\
 & RetRobust &  36.06 &  48.99  &  43.28  &  55.04  &  52.88  &  62.10  &  59.06  & 67.77 & 47.82 & 58.48  \\
 & Top-1 Doc & 36.72 & 49.30 & 44.38 & 56.20  &  54.68  &  62.26 &  60.80  & 68.31 & 49.15 & 59.02  \\
 & Golden Doc & 35.52 & 48.31 & 41.35 & 53.13  &  48.92  &  58.41  &  58.26 & 66.99 &  46.01 & 56.71 \\
 \cdashline{2-12}
 & Random Doc & 35.98 & 49.05 & 43.37 & 55.43  &  53.24  &   62.55 &  60.62  & 68.64 & 48.30 & 58.92  \\
 & Irrelevant Doc & 35.31 & 47.92 & 42.45 & 54.41  &  46.76  &  57.67  &  58.97  & 66.57 & 45.87 & 56.64 \\
 \cdashline{2-12}
 & RAAT & 32.20 & 43.81 & 42.34 & 53.31  &  48.28  &  58.17 &  54.41 &  62.45 & 44.31 & 54.44  \\
 & IRM & 35.19 & 48.08 & 41.13 & 53.14  &  53.96  &  61.64  &  57.15 &  69.13 & 46.86 & 58.00  \\
\cmidrule{2-12}
% & \textcolor{mycolor}{$\Delta$ (Random $\rightarrow$ Best)}  &  \textcolor{mycolor}{2.06\%}  & \textcolor{mycolor}{0.51\%}  &  \textcolor{mycolor}{2.33\%}  &  \textcolor{mycolor}{1.39\%}  &  \textcolor{mycolor}{2.70\%}   &  \textcolor{mycolor}{0.00\%}  &  \textcolor{mycolor}{0.30\%}  &  \textcolor{mycolor}{0.00\%}  & \textcolor{mycolor}{1.85\%} & \textcolor{mycolor}{0.47\%} \\
& \textcolor{mycolor}{$\Delta$ (Worst $\rightarrow$ Best)}  &  \textcolor{mycolor}{14.04\%}  & \textcolor{mycolor}{12.53\%}  &  \textcolor{mycolor}{7.90\%}  &  \textcolor{mycolor}{5.76\%}  &  \textcolor{mycolor}{16.94\%}   &  \textcolor{mycolor}{8.46\%}  &  \textcolor{mycolor}{11.74\%}  &  \textcolor{mycolor}{10.70\%}  & \textcolor{mycolor}{10.92\%} & \textcolor{mycolor}{8.42\%} \\
\bottomrule
\end{tabular}}
\caption{Performance comparison of different LLMs (\texttt{Llama-2-7b-chat-hf} and \texttt{Llama-3-8B-Instruct}) across different robust RAG scenarios on four datasets (HotpotQA, NQ, WebQuestions, and TriviaQA). The row \textcolor{mycolor}{$\Delta$ (Worst $\rightarrow$ Best)} indicates the performance gain achieved by the best method compared to the worst strategy among these training strategies, representing the benefit of sophisticated robust training methods.}
\label{tab:main_result_llama}
\vspace{-0.4cm}
\end{table*}

To assess robustness against irrelevant information, we introduce two adversarial scenarios:
\begin{itemize}[leftmargin=0.5cm, itemindent=0cm, itemsep=0pt]
    \item \textbf{Random Document}: Randomly selects a document from retrieved results, simulating unpredictable retrieval quality.
    \item \textbf{Irrelevant Document}: Chooses a passage from another query’s retrieval results, ensuring no relevance to the current query.
\end{itemize}

% These scenarios evaluate: (1) basic QA capability without retrieval, (2) the model’s ability to leverage relevant documents, and (3) robustness against irrelevant or misleading information. This framework provides a comprehensive assessment of RAG performance under diverse conditions.

\subsubsection{Adversarial Loss Design}  
We assess two popular adversarial loss strategies:
\begin{itemize}[leftmargin=0.5cm, itemindent=0cm, itemsep=0pt]
\item \textbf{RAAT}. The regularization term in RAAT reduces the performance gap between the best and worst retrieval cases. By penalizing excessive performance disparity, RAAT ensures that the model remains stable even under challenging retrieval conditions, leading to improved robustness and generalization.
\item \textbf{IRM}. The regularization in IRM minimizes the variance in performance across different retrieval environments. By enforcing consistency, IRM mitigates sensitivity to distribution shifts, ensuring that the model performs reliably across diverse retrieval scenarios.
\end{itemize}



\section{Results}
% \section{Simulation Evaluation \& Results}\label{sec:results}

\subsection{Baseline Planners}

To evaluate the performance of \PlannerName, we compare it against several baseline methods. In the following section, we describe these baselines, their implementation details, and their respective advantages and limitations, particularly in the context of information gathering in large, high-dimensional search spaces. The simulation framework and vehicle parameters remain consistent across all planners, and each method is allowed to replan during testing.

\subsubsection{Monte-Carlo Tree Search}

Monte Carlo Tree Search (MCTS) can be a powerful technique for finding feasible and optimal paths in complex environments. It is a heuristic search algorithm that builds a search tree incrementally through repeated simulations. At each iteration, it selects a node to explore based on a selection policy (often the Upper Confidence Bound or UCB1 algorithm), expands the tree by adding possible actions from that node, runs a simulation from the newly added node, and updates the statistics of nodes along the path traversed during the simulation. 

The UCB1 (Upper Confidence Bound) algorithm is a technique commonly used in the context of multi-armed bandit problems and Monte Carlo Tree Search (MCTS) for balancing exploration and exploitation. It helps in selecting actions or nodes that are likely to yield high rewards while also exploring less-frequented options to gather more information about their potential rewards. 

We formulate our UCB score in the following manner, \\
\begin{equation*}
    UCB_\text{node} = \frac{I(X_{\text{node}})}{\alpha} + C \times \sqrt{\frac{\ln(N_\text{tree})}{N_\text{node}}}
\end{equation*}
%  $
% UCB_\text{node} = \frac{\overline{X_\text{node}}}{\alpha} + C \times \sqrt{\frac{\ln(N_\text{tree})}{N_\text{node}}}
% $ \\
Here $I(X_{\text{node}})$ denotes the estimated information gain from the node, $\alpha$ denotes the normalization factor which is given by $\frac{B}{v_\text{desired}}$, $B$ being the maximum planning budget and $v_\text{desired}$ being the desired speed of our UAV. $C$ denotes the exploration weight, and $N_\text{tree}$ denotes the number of visits to the tree root node while $N_\text{node}$ denotes the number of times the present node has been visited.

After selecting a candidate node, if it has been visited before, it is expanded by applying motion primitives to generate child nodes, growing the tree. Unvisited nodes skip this step. Following expansion, either the unvisited candidate node or one of its children is selected for the simulation phase, where the future values of nodes along the path are estimated to update the total potential information gain. This informs the selection policy in subsequent iterations. Once planning time is exhausted, the path with the highest information gain is returned.

% with authors goes here
\begin{figure}[t]
\centering
\includegraphics[trim={.7cm 0cm .5cm 1.4cm},clip,width=\columnwidth]{figs/5_/Results1v3.pdf}
\caption{The Monte Carlo simulation results for the planners. The plots show the average percent reduction in entropy over the course of the simulations, and the shading shows the 95\% confidence intervals. IA-TIGRIS outperforms all of the baselines.}
\label{fig:mc_results}
\end{figure}

While MCTS is probabilistically guaranteed to converge to the optimal path \cite{mcts_ref_1}, it is constrained to actions within a predefined set of motion primitives. Its reliance on random sampling to estimate the future value of nodes can result in poor approximations, particularly in environments with sparse, localized pockets of high information gain. This limitation is especially pronounced in large search areas or scenarios with large budgets constraints, where estimating future node values becomes increasingly expensive. As a result, in such scenarios, MCTS is often implemented with a finite planning horizon, which can restrict its ability to account for long-term consequences or dependencies in the environment.

% This property of MCTS, which causes unguided exploration of the environment, leads to increased convergence times on the optimal path, as a result of a lot of budget being spent in exploring information sparse areas of the map. 
% Also, the computation time of MCTS increases exponentially with the depth of the search tree. The time complexity of MCTS is given by $\mathcal{O}(\frac{T}{t_\text{iter}} \cdot |A|^d)$. Here, $T$ is the total planning time and $t_\text{iter}$ is the time taken per iteration of the planning loop. $|A|$ is the number of actions and $d$ represents the average depth of the search tree. 

% The above limitations are not inconsequential in the context of performing informative path planning in large high-dimensional search spaces. We compare MCTS with \PlannerName, in \ref{}, and empirically demonstrate its drawbacks and how \PlannerName, is able to outperform MCTS in the context of the mission parameters we examine in this work.  

\subsubsection{Greedy}

For the greedy planner, we iterated through each cell within the search bounds and calculated the reward for a given cell $i$ as $g_i = R(X_i) / d_i$ where $R(X_i)$ is given through \eqref{equ:reward} and $d_i$ represents the Euclidean distance between the current position the robot at the current time $t$ and the closest viewpoint to the cell. To compute this viewpoint, the yaw between the current pose of the robot and the intersected cell is first calculated. Using the robot's sensor configuration and this yaw, $x$ and $y$ coordinates are calculated that view the cell at the desired flight altitude. With this formulation, the planner prioritizes regions with a high ratio of entropy to distance. This can lead to locally optimal choices that contradict with paths that lead to higher information gain over the entire trajectory. 

% without authors goes here
% \begin{figure}[t]
% \centering
% \includegraphics[trim={.7cm 0cm .5cm 1.4cm},clip,width=\columnwidth]{figs/5_/Results1v3.pdf}
% \caption{The Monte Carlo simulation results for the planners. The plots show the average percent reduction in entropy over the course of the simulations, and the shading shows the 95\% confidence intervals. IA-TIGRIS outperforms all of the baselines.}
% \label{fig:mc_results}
% \end{figure}


\begin{figure*}[t]
    \centering
    \begin{subfigure}[b]{0.99\textwidth}
        \centering
        \includegraphics[trim={0cm 0.3cm 0cm 0cm},clip,width=\textwidth]{figs/5_/Fig2v1_target.png}
        % \caption{Slice by targets}
        % \vspace{.1cm}
    \end{subfigure}
    
    \begin{subfigure}[b]{0.99\textwidth}
        \centering
        \includegraphics[trim={0cm 0cm 0cm 0cm},clip,width=\textwidth]{figs/5_/Fig2v1_sigma.png}
        % \caption{Slice by sigma }
    \end{subfigure}
    \caption{A comparison of the methods based on the number of sampled prior clusters and the standard deviation of sampled prior clusters. IA-TIGRIS is most effective compared to the baselines when there is high variation in the search space. As the search space prior information becomes more evenly spread out, the performance gap between the methods tends to decrease.}
    \label{fig:targets_sigmas}
\end{figure*}

\subsubsection{Random}

The random planner operates by iteratively sampling points within the defined search bounds and calculating the minimum-cost path to observe each sampled point. This process is repeated until the available budget is fully expended. The random planner does not utilize any prior information about the environment or target distribution. Additionally, it does not optimize the sequence of actions, instead treating each sampled point independently without considering the global structure of the search problem. This simplicity allows the random planner to highlight the performance benefits of more sophisticated methods by providing a lower-bound comparison for evaluation.

\subsubsection{Coverage}

The coverage planner generates a plan that systematically covers the entire search space using a straightforward lawn-mower pattern. The spacing between each pass is set to match the width of the projected observation footprint at 20\% from the bottom, ensuring that no grid cells are missed. This spacing also maintains a distance that enables high-quality sensor measurements. However, due to the size of the search spaces considered, the coverage planner spends significant time surveying empty regions. This approach results in inefficient use of the budget, as it prioritizes full coverage with safe sensor overlap, even in areas with little or no valuable information. While simple and robust, this method highlights the tradeoff between exhaustive coverage and efficient, targeted exploration.

% \subsubsection{Branch and Bound}
% The branch and bound baseline is based on motion primitive planning. In each future step the drone has a set of motion primitives with future states and each of these future states also has a set of motion primitives. In this way, a tree can be built with multiple path candidates. The path candidate with the highest information gain will be selected and form the output. 

% By adding branch and bound, there will be an estimation of a node's upper bound information reward, using the node's current information reward, updated information map and the remaining budget. If this upper bound is already lower than the information reward of any other node in the tree, the corresponding node will be closed and not expanded in the future to accelerate the expansion of the tree. 



\subsection{Tests and Analysis}
% To evaluate the efficacy of IA-TIGRIS compared to the baseline methods, we conduct Monte Carlo testing as well as analyze how the prior and budget affect the performance of each method. In all of these test cases, there are no time-based or priority rewards and have horizon lengths set to the full budget. All tests were performed using an Intel Xeon CPU E5-2620 v4 @ 2.10GHz.
To evaluate the efficacy of IA-TIGRIS against baseline methods, we perform Monte Carlo testing and analyze the impact of the prior and budget on the performance of each method. In all test cases, rewards are calculated using \eqref{equ:reward}, and horizon lengths are set to match the full budget. The tests are conducted on an Intel Xeon CPU E5-2620 v4 @ 2.10GHz, ensuring consistent computational conditions across all evaluations.

% Random sample across which parameters.

% Quantitative ideas. Look into number and std of prior (metric for this? std of grid cell values, mediuan, mean,). 
% Uniform prior? 
% Split distinct regions, not smooth. 
% Compare to coverage and amount of time to reach specific amount. 
% Compare with different budgets. 
% Repeatability test. 
% Graph size vs time. 
% Look at coverage with different altitudes or widths. Something that shows long horizon vs not nature of things?
% Shape of search space?
% Time/budget to get x\% of all info gain. Have to do moving horizon. 
% Targets detected? 

% Key thought for results where I show time, our optimization does not optimize for time, only final value. Key thing to show across the different budgets. 

% \BM{Qualitative. Nayana idea of plot with example sampled case. Should add one here.} 



\subsubsection{Monte Carlo Testing}
Our simulated testing environment is a $5000\times5000$ m square with Gaussian-distributed prior information randomly placed throughout the search space. The number of prior clusters was sampled uniformly between $[4,20]$, with standard deviations between $[60,450]$, and maximum value between $[0.05,0.5]$. 

The results of $100$ Monte Carlo tests are shown in Fig.~\ref{fig:mc_results}. IA-TIGRIS clearly outperforms the other methods, achieving nearly a $40\%$ greater reduction in entropy than the next best method. Early in the simulation, the greedy method initially gains information more quickly, as expected, but this does not translate to better long-term performance. Since our method optimizes for total information gain, it generates paths that maximize information collection over the entire budget. MCTS performed slightly worse than the greedy approach.

The random paths slightly outperformed the coverage paths. This is likely because the lawnmower strategy requires sufficient overlap between passes to avoid missing areas, and its long straight paths often lead to redundant observations due to the UAV’s forward-facing camera. Changing the heading of the UAV is beneficial to viewing more of the search space, which may explain why random paths performed better.

We also conducted Monte Carlo tests where either the number of prior clusters or their standard deviation was held constant to analyze how variations in the information map affect planner performance. The results, shown in Fig.~\ref{fig:targets_sigmas}, include two cases: the upper figure fixes the number of priors, while the lower figure fixes their standard deviation. All other agent and simulation parameters remained unchanged.


% The first thing to note from these results is that for all tests the proportional performance gap between IA-TIGRIS and the baselines increases as the number and standard deviation of the Gaussian priors decreases. As the search space becomes more uniformly filled with entropy in the information map, the need for longer-horizon planning decreases and other simple or random approaches can perform satisfactorily given the testing budget. As the information becomes more sparsely distribution in the space, such as when the information is contained in separated pockets of areas, there is a greater need to plan longer-horizon paths that reason about the given budget.
% \BM{Could have figures here or refer to others}

Across these tests, the performance gap between IA-TIGRIS and the baselines widens as the number and standard deviation of the Gaussian priors decrease. When entropy is more uniformly distributed across the search space, simpler methods perform reasonably well within the given budget. However, when information is concentrated in sparse, distinct regions, longer-horizon planning becomes essential. In such cases, IA-TIGRIS demonstrates a significant advantage by effectively reasoning about the budget and prioritizing high-value regions.

% Show plot of first plans expected info gain versus planning time. (plans not executed)


\subsubsection{Budget Analysis}
To evaluate the impact of budget constraints on performance, we conducted additional tests beyond our initial Monte Carlo experiments, evaluating budgets of $5000$ m, $10000$ m, $30000$ m, and $60000$ m. Table~\ref{tab:budgets} summarizes the average entropy reduction across these budgets.

\definecolor{tabfirst}{rgb}{1, 0.7, 0.7} % red
\definecolor{tabsecond}{rgb}{1, 0.85, 0.7} % orange
\definecolor{tabthird}{rgb}{1, 1, 0.7} % yellow
\begin{table}[t]
    \centering
    \resizebox{\linewidth}{!}{
    \begin{tabular}{l|ccccc}
    & $5000$ m & 10000 m  & 15000 m& 30000 m& 60000 m\\ \hline

    % \hline
    IA-TIGRIS  &  \cellcolor{tabfirst}$9.41\pm1.0$ &  \cellcolor{tabfirst}$18.28\pm1.8$ & \cellcolor{tabfirst}$25.36\pm2.3$ & \cellcolor{tabfirst}$41.08\pm2.9$ & \cellcolor{tabfirst}$58.85\pm2.9$ \\
    Greedy  &  \cellcolor{tabsecond}$6.99\pm0.8$ &  \cellcolor{tabsecond}$13.10\pm1.5$ & \cellcolor{tabsecond}$17.97\pm2.0$ & \cellcolor{tabthird}$30.00\pm2.3$ & \cellcolor{tabsecond}$49.38\pm3.5$ \\
    MCTS  &  \cellcolor{tabthird}$6.06\pm0.7$ &  \cellcolor{tabthird}$11.80\pm1.1$ & \cellcolor{tabthird}$17.11\pm1.4$ & \cellcolor{tabsecond}$30.21\pm2.2$ & \cellcolor{tabthird}$48.68\pm2.7$ \\
    Random  &  $2.19\pm0.3$ & $4.29\pm0.7$ & $6.61\pm0.6$ & $17.50\pm1.2$ & $22.47\pm1.4$ \\
    Coverage  &  $1.58\pm0.3$ &  $2.82\pm0.4$ & $4.09\pm0.7$ & $12.04\pm1.9$ & $16.77\pm2.4$ \\

    \end{tabular}
    }
    \caption{Monte Carlo testing results given different budgets. The values are the average percent reduction in entropy and the 95\% confidence bounds. \mbox{IA-TIGRIS} had the best performance for all budgets.}
    \label{tab:budgets}
\end{table}
%$\uparrow$ 

IA-TIGRIS consistently achieved the highest entropy reduction across all budget constraints, with a statistically significant margin over alternative methods. Greedy generally ranked second but was slightly outperformed by MCTS at the $30000$ m budget level. Greedy and MCTS exhibited comparable performance throughout the tests, with their results closely tracking each other. Consistent with our previous findings, Random and Coverage methods yielded the lowest results.


Among the tested methods, only IA-TIGRIS and MCTS explicitly incorporate budget constraints into their planning algorithms. Notably, at lower budgets ($5000$ m and $10000$ m), these methods achieved higher entropy reduction compared to the equivalent time steps ($200$ s and $400$ s) in the $15000$ m budget scenario shown in Fig.~\ref{fig:mc_results}. This improved performance stems from IA-TIGRIS's optimization of total path reward under budget constraints, contrasting with the myopic next-best-action approach of the greedy method. The remaining methods---Greedy, Random, and Coverage---maintain consistent behavior regardless of budget constraints, as their planning strategies do not account for resource limitations.


The performance gap between IA-TIGRIS and the next-best method varied with budget size, showing margins of $34.6\%$, $39.5\%$, $41.1\%$, $36.0\%$, and $19.2\%$ in ascending budget order. This gap widened through the first three budget levels as problem complexity increased, before declining significantly at higher budgets. This performance pattern suggests that implementing a planning horizon could enhance efficiency by limiting tree search depth, enabling the planner to prioritize path quality optimization over exhaustive space exploration.


% percent improved from next best
% 34.6, 39.5, 41.1, 36.0, 19.2
% reasons, too long horizon is a larger search space, so less quality paths closer. Or larger horizon, more packing in


% with authors goes here
\begin{figure}[t] 
    \centering
    \renewcommand\arraystretch{0} % Adjust the height between rows here
    \setlength{\tabcolsep}{1pt} % Adjust the column separation here
    \begin{tabular}{c}
        \begin{tikzpicture}
            \node[anchor=south west, inner sep=0] (image) at (0,0) {
                \includegraphics[width=0.9\linewidth]{figs/5_/google_earth_prior.png}
            };
            \begin{scope}[x={(image.south east)},y={(image.north west)}]
                % \fill[OrangeRed] (0.02, 0.03) circle (2pt); 
                % \fill[OrangeRed] (0.51, 0.04) circle (2pt); 
                % \fill[OrangeRed] (0.61, 0.04) arc (0:90:2pt); 
                \fill[Orange, opacity=0.8] (0.74, 0.45) circle (3pt); % Adjust 
                \fill[Orange, opacity=0.8] (0.27, 0.42) circle (3pt); % Adjust 
                \fill[Orange, opacity=0.8] (0.39, 0.63) circle (3pt); % Adjust 
            \end{scope}
        \end{tikzpicture} \\
        % \includegraphics[width=0.9\linewidth]{figs/5_/google_earth_prior.png} \\
        \\
        \includegraphics[width=0.9\linewidth]{figs/5_/google_earth_path.png} 
    \end{tabular}
    \caption{Google Earth screenshots illustrating the mission planning process and execution. Top: Areas of high entropy targeted for search are highlighted in red, representing regions with a binary occupied/unoccupied probability of 0.2. Three points of particular interest, each assigned a 0.5 probability, are marked in orange. Bottom: The executed drone flight path (yellow) shows the optimized path for maximum information gain across the search space.} 
    \label{fig:google_earth}
\end{figure}
\begin{figure}[t]
\centering
% https://docs.google.com/presentation/d/1RjI-QqHpBRLHN60UAxzmQYs4EaWaVCOoSBkEkA39kk0/edit?usp=sharing
\includegraphics[width=\columnwidth]{figs/5_/m600_labeled.jpg}
\caption{Hexarotor system (DJI M600 Pro) with onboard compute and camera. Left image shows drone on the ground, right image shows drone in flight.}
\label{fig:m600}
\end{figure}


\section{Field Deployments}\label{sec:field}


\subsection{Hexarotor Deployment}
The first field experiment that we present uses a hexarotor drone to cover an urban area shown in Fig.~\ref{fig:fig1}.
We designed this field experiment to simulate classifying where cars are within a search area.  
Hence, we set the plan request to focus on parking lots at the field test site (Fig.~\ref{fig:google_earth}, top), with the addition of three chosen grid cells within the parking lots being marked as having a higher uncertainty. The plan request boundaries and priors were created with GPS coordinates in Google Earth, exported as kml files, and then converted into our plan request message format. 

The following sections details the hardware, autonomy, and experimental results for our hexarotor deployments.

% without the authors goes here
% \begin{figure}[t] 
%     \centering
%     \renewcommand\arraystretch{0} % Adjust the height between rows here
%     \setlength{\tabcolsep}{1pt} % Adjust the column separation here
%     \begin{tabular}{c}
%         \begin{tikzpicture}
%             \node[anchor=south west, inner sep=0] (image) at (0,0) {
%                 \includegraphics[width=0.9\linewidth]{figs/5_/google_earth_prior.png}
%             };
%             \begin{scope}[x={(image.south east)},y={(image.north west)}]
%                 % \fill[OrangeRed] (0.02, 0.03) circle (2pt); 
%                 % \fill[OrangeRed] (0.51, 0.04) circle (2pt); 
%                 % \fill[OrangeRed] (0.61, 0.04) arc (0:90:2pt); 
%                 \fill[Orange, opacity=0.8] (0.74, 0.45) circle (3pt); % Adjust 
%                 \fill[Orange, opacity=0.8] (0.27, 0.42) circle (3pt); % Adjust 
%                 \fill[Orange, opacity=0.8] (0.39, 0.63) circle (3pt); % Adjust 
%             \end{scope}
%         \end{tikzpicture} \\
%         % \includegraphics[width=0.9\linewidth]{figs/5_/google_earth_prior.png} \\
%         \\
%         \includegraphics[width=0.9\linewidth]{figs/5_/google_earth_path.png} 
%     \end{tabular}
%     \caption{Google Earth screenshots illustrating the mission planning process and execution. Top: Areas of high entropy targeted for search are highlighted in red, representing regions with a binary occupied/unoccupied probability of 0.2. Three points of particular interest, each assigned a 0.5 probability, are marked in orange. Bottom: The executed drone flight path (yellow) shows the optimized path for maximum information gain across the search space.} 
%     \label{fig:google_earth}
% \end{figure}
% \begin{figure}[t]
% \centering
% % https://docs.google.com/presentation/d/1RjI-QqHpBRLHN60UAxzmQYs4EaWaVCOoSBkEkA39kk0/edit?usp=sharing
% \includegraphics[width=\columnwidth]{figs/5_/m600_labeled.jpg}
% \caption{Hexarotor system (DJI M600 Pro) with onboard compute and camera. Left image shows drone on the ground, right image shows drone in flight.}
% \label{fig:m600}
% \end{figure}

\subsubsection{Hardware System}
The hardware consists of the DJI M600 Pro, shown in Fig.~\ref{fig:m600}, along with the physical sensing and onboard computer payload. The DJI M600 Pro contains a flight controller that handles pose estimation and position-based control. The DJI M600 Pro’s flight controller also handles teleloperation if human intervention is necessary. Beneath the drone's base, we mount a custom hardware payload.
That payload consists of an onboard computer, a Jetson Xavier, to run the autonomy software shown in Fig.~\ref{fig:functional_diagram}.
The payload also contains a downward-facing a camera for sensing the environment. The camera is a Seek S304SP thermal camera.
The camera intrinsics are used to calculate the frustum's intersection with the search map's cells in IA-TIGRIS.

% without authors goes here
\begin{figure}[t]
\centering
% https://lucid.app/lucidchart/f750ddb4-2809-4773-8361-d5fbb1ba49eb/edit?viewport_loc=-257%2C-116%2C2219%2C1140%2C0_0&invitationId=inv_56e8a3a9-e8cf-4cad-a280-48bd967ff651
\includegraphics[trim={0cm 0cm 0cm 0cm},clip,width=\columnwidth]{figs/5_/functional_diagram.jpeg}
\caption{Functional diagram of the DJI M600 Pro autonomy software.}
\label{fig:functional_diagram}
\end{figure}
\begin{figure}[b]
    \centering
    \begin{subfigure}[b]{0.48\columnwidth}
        \centering
        \includegraphics[width=1.0\linewidth]{figs/5_/field_test_altitude_over_time.png}
        \caption{}
        \label{fig:m600_altitude_over_time}
    \end{subfigure}
    \begin{subfigure}[b]{0.48\columnwidth}
        \centering
        \includegraphics[width=1.0\linewidth]{figs/5_/field_test_entropy_over_time.png}
        \caption{}
        \label{fig:m600_entropy_over_time}
    \end{subfigure}
    \caption{The results for our hexarotor field deployment. (a) Plot of flown altitude over time, showing large variation throughout the experiment. (b) Reduction in entropy percentage over time of field experiment.}
\end{figure}

\subsubsection{Autonomy System}
Fig.~\ref{fig:functional_diagram} illustrates the functional system diagram for the real world field test on the DJI M600. The user specifies the initial plan request prior to takeoff. The TIGRIS planner makes an initial plan on that plan request and sends a global path to the waypoint manager. The waypoint manager tracks the current waypoint within the plan and sends the next waypoint to the DJI software development kit, which then sends actuation commands to the motors. The position of the drone is used to calculate the distance from the drone to the ground and sends that distance parameter to the sensor model. The sensor model's true positive and false positive rate is used to calculate the per-cell entropy updates in the search map manager. The search map manager publishes the current information map, and the replanning node sends an updated plan request to the IA-TIGRIS planner every ten seconds.

The drone started at an altitude of $50$ m above the origin of the reference frame. The informed sampler in IA-TIGRIS was set to add states at altitudes of either $30$ m or $60$ m, creating a trade-off between observation area and detector accuracy. The budget was $2000$ m, the planning horizon was $600$ m, and the planning time was $10$ seconds. 

% % without authors goes here
% \begin{figure}[t]
% \centering
% % https://lucid.app/lucidchart/f750ddb4-2809-4773-8361-d5fbb1ba49eb/edit?viewport_loc=-257%2C-116%2C2219%2C1140%2C0_0&invitationId=inv_56e8a3a9-e8cf-4cad-a280-48bd967ff651
% \includegraphics[trim={0cm 0cm 0cm 0cm},clip,width=\columnwidth]{figs/5_/functional_diagram.jpeg}
% \caption{Functional diagram of the DJI M600 Pro autonomy software.}
% \label{fig:functional_diagram}
% \end{figure}
% \begin{figure}[b]
%     \centering
%     \begin{subfigure}[b]{0.48\columnwidth}
%         \centering
%         \includegraphics[width=1.0\linewidth]{figs/5_/field_test_altitude_over_time.png}
%         \caption{}
%         \label{fig:m600_altitude_over_time}
%     \end{subfigure}
%     \begin{subfigure}[b]{0.48\columnwidth}
%         \centering
%         \includegraphics[width=1.0\linewidth]{figs/5_/field_test_entropy_over_time.png}
%         \caption{}
%         \label{fig:m600_entropy_over_time}
%     \end{subfigure}
%     \caption{The results for our hexarotor field deployment. (a) Plot of flown altitude over time, showing large variation throughout the experiment. (b) Reduction in entropy percentage over time of field experiment.}
% \end{figure}

\subsubsection{Experimental Results}


The bottom image of Fig.~\ref{fig:google_earth} shows the path selected by IA-TIGRIS in the search area. The figure highlights how the planner dynamically adjusts altitudes over time to balance coverage and sensing resolution, maximizing information gain. Higher altitudes allow for broader area coverage, while lower altitudes provide more detailed observations where needed. Additionally, the planner prioritizes revisiting the three regions of higher uncertainty, recognizing the need for repeated observations reduce entropy. This adaptive strategy ensures that uncertain areas receive sufficient attention to improve the belief map. As a result, the entropy of the information map decreases to near zero by the end of the mission, as shown in Fig.~\ref{fig:m600_entropy_over_time}, indicating that the planner has effectively gathered the necessary information. This behavior demonstrates the planner’s ability to optimize sensing actions, balancing altitude selection, revisit frequency, and exploration to maximize mission success.

\begin{figure}[t]
\centering
% \includegraphics[width=2.5in]{fig1}
\includegraphics[trim={4cm 4cm 0cm 4cm},clip,width=\columnwidth]{figs/5_/TL1.jpg}
\caption{Fixed-wing platform used for autonomous flights with an onboard camera pitched at 10 degrees\cite{alarewebsite}}
\label{fig:tl1}
\end{figure}






\subsection{Fixed-wing Deployments}

Our proposed approach was extensively tested on the fixed-wing AlareTech TL-1 UAV, shown in Fig.~\ref{fig:tl1}. The UAV is equipped with an onboard camera pitched at 10 degrees, which introduces a more challenging planning problem due to the non-holonomic motion model and the camera's field of view. Over more than 20 flight hours and 100 flights running IA-TIGRIS, we validated our approach with the objective to search for objects of interest in a large search space across a variety of test scenarios, including different terrain types, varying environmental conditions, and diverse target distributions. An example mission from these tests is shown in Fig.~\ref{fig:fwd}. In this scenario, the planner was given the search bounds and a designated high-priority region. The resulting flight path prioritized revisiting the high-priority area twice, optimizing sensor use and ensuring maximum information gain. This strategy led to the successful detection of the object of interest, with its estimated position marked by the red dot in the figure. 

The map on the upper right in Fig.~\ref{fig:fwd} shows the information map after plan execution was complete. Due to the UAV's limited budget, the upper right and lower left corners of the map are not searched by the agent. The budget is instead utilized to search over the area of higher priority two times. Compared to the paths in Fig.~\ref{fig:google_earth}, we observe that the paths for the fixed wing are smoother and have a larger turning radius, demonstrating how IA-TIGRIS respects the motion constraints of the vehicle. We can also see the effect of wind on the path execution, where the flown path shown in green deviates from the planned path shown in yellow. This illustrates the importance of online planning in the cases where this deviation is large or would accumulate over the course of a longer mission and cause the expected observed area to be much different than actual observed area. 

\begin{figure}[t]
\centering
% \includegraphics[width=2.5in]{fig1}
% [trim={left bottom right top},clip]
\includegraphics[trim={3.0cm, 1.0cm, 3.0cm, 1.0cm},clip,width=\columnwidth]{figs/5_/ONRFig_v3.pdf}
\caption{An example path generated for the fixed-wing platform conducting a large-area search for an object of interest. The larger black rectangle denotes the search bounds, while the smaller black rectangle highlights a region of higher uncertainty. The red dot marks the estimated position of the detected object based on image detections. The upper-right map displays the information state after planning is complete, while the middle plot shows the percent change in entropy over mission time. The flown path illustrates a balance between allocating resources to the high-priority region and exploring other areas within the search space.}
\label{fig:fwd}
\end{figure}

% Also tested extensively on the AlareTech TL-1 (citation?) tube launched UAV seen in Fig.~\ref{fig:tl1}.

% Talk about amount of flights, hours. Platform. Compute. Show visualization fo example flight. Talk about objects of interest in a broad sense (no mention of water/ocean/land for targets). Follow similar figure format as previous section. Main thing we want to highlight is the differences introduced in plans by having a fixed-wing platform compared to a drone. Include image of Alare TL-1 somewhere.

% One big figure showing all the info we want to convey. 

% \BM{Pitch 10 degrees, onboard computer type, etc}


% \subsection{VTOL?}
% what would it bring?



\section{Related work}
\section{Related Work}

\paragraph{Commonsense Reasoning Evaluation} 
There are numerous benchmarks and datasets for commonsense reasoning, most of which are in English. 
%Some work focused on evaluating general commonsense knowledge, such as HellaSwag \cite{zellers2019hellaswag}, CommonsenseQA \cite{talmor2019commonsenseqa}, OpenBookQA \cite{OpenBookQA2018}, and WSC \cite{levesque2012winograd}. 
Some studies focus on evaluating general commonsense knowledge \cite{zellers2019hellaswag,talmor2019commonsenseqa,OpenBookQA2018}. 
%Others target specific aspects of commonsense reasoning, including temporal commonsense with MCTACO \cite{zhou2019going}, physical commonsense with PIQA \cite{bisk2020piqa}, social commonsense with SocialIQA \cite{sap2019socialiqa}, numerical commonsense with NumerSense \cite{lin2020birds}, and scientific commonsense with ARC \cite{clark2018think} and QASC \cite{khot2020qasc}. Notably, most of these datasets are in English. 
Others target specific aspects of commonsense reasoning\cite{zhou2019going,bisk2020piqa,sap2019socialiqa,lin2020birds,clark2018think,khot2020qasc}.
There are some Chinese datasets for commonsense reasoning \cite{sun2024benchmarking,shi2024corecode}. 
For instance, CHARM \cite{sun2024benchmarking} distinguishes between global commonsense and Chinese-specific commonsense but includes only a limited number of everyday commonsense cases. 
However, evaluations aimed at assessing the robustness of commonsense reasoning are still understudied. 

\paragraph{Datasets on Different Reasoning Forms}
There are several datasets relevant to our variant design. For reverse reasoning, ART \cite{DBLP:conf/iclr/BhagavatulaBMSH20}, $\delta$-NLI \cite{DBLP:conf/emnlp/RudingerSHBFBSC20}, and CLUTRR \cite{DBLP:conf/emnlp/SinhaSDPH19} explore different reasoning directions. FCR \cite{DBLP:journals/corr/abs-2204-07408} and NatQuest \cite{ceraolo2024analyzinghumanquestioningbehavior} evaluate causal reasoning, while TimeTravel \cite{DBLP:conf/emnlp/QinBHBCC19} focuses on counterfactual scenario refinement. Additionally, PoE \cite{balepur2024s} assesses reasoning involving negation. 
However, not all these datasets focus on commonsense reasoning, nor are they structured by original questions and their variants. Furthermore, they typically target limited reasoning types. Lastly, our dataset is large-scale and covers diverse commonsense knowledge. 

\paragraph{Robustness and Consistency in LLMs} 
Early work focuses on adversarial attacks, with developing evaluation methods for reading comprehension systems \cite{jia2017adversarial}, followed by universal adversarial triggers \cite{wallace2019universal}. The field then expands to examine spurious correlations, with revealing how models often exploit superficial patterns rather than engaging in genuine reasoning \cite{branco2021shortcutted,geirhos2020shortcut}. And \citealp{ross2022does} investigates whether self-explanation can mitigate these spurious correlations. Coherence and consistency evaluation advances through classifier assessment methods \cite{storks2021beyond} and analysis of accuracy-consistency trade-offs \cite{johnson2023much}. While these studies primarily address model robustness against adversarial attacks or spurious correlations, our work takes a novel approach by examining robustness in reasoning forms.
%, specifically focusing on how models maintain consistent reasoning when presented with different reasoning forms of the same commonsense knowledge.
%\paragraph{Dataset Construction by LLM} 
%Research indicates that when LLMs are utilized for dataset generation, the resulting datasets are more accurate and fluent \cite{lu2022fantastically, min-etal-2022-rethinking} than those created by crowd-sourced annotators. Furthermore, generating datasets with LLMs is significantly more cost-effective than using crowd-sourced annotations \cite{liu2022wanli, wiegreffe2022reframing, west2022symbolic}. Hence, we generate our benchmark by LLM in-context learning.

% \paragraph{In-Context Learning} 
% As LLMs become more widely used, in-context learning (\citealp{brown2020language}; \citealp{ouyang2022training}; \citealp{min-etal-2022-rethinking}) has emerged as the primary approach for executing various tasks. This method involves supplying LLMs with textual instructions and examples and removes the necessity for parameter modifications. Research indicates that when LLMs are utilized for dataset generation, the resulting datasets are more accurate and fluent (\citealp{lu2022fantastically}; \citealp{min-etal-2022-rethinking}) than those created by crowd-sourced annotators. Furthermore, generating datasets with LLMs is significantly more cost-effective than using crowd-sourced annotations (\citealp{liu2022wanli}; \citealp{wiegreffe2022reframing}; \citealp{west2022symbolic}). Hence, we have decided to construct our benchmark by over-generating data using in-context learning and employing human annotators for filtering to ensure high efficiency and high quality.
% Moxin: 这部分应该改成用LLM 生成dataset的工作?
% 


% Doesn't count towards page limit
%\section{Impact statement}
%
In this work, we propose a methodology to audit the privacy risks in LLM-generated synthetic data. Through a novel MIA, we quantify the potential for sensitive information leakage even in scenarios where the underlying model is inaccessible. We also identify that canary generation mechanisms found useful to study risks in model-based attacks fall short in data-based attacks, and propose an improved canary generation mechanism optimal for data-based attacks. 

Taken together, the methods proposed in this work enable an auditor to empirically estimate the privacy risks associated with synthetic text. Practitioners leveraging synthetic data as a privacy-enhancing technology can use our tools to evaluate these risks before deploying synthetic text in downstream applications. In particular, our privacy auditing pipeline would be valuable when synthetic text data is proposed to extract utility from sensitive data (\eg medical records, financial statements) or to verify synthetic data generation implementations with formal privacy guarantees. 

We hope this work advances the understanding of privacy risks in LLM-generated synthetic data and helps organizations and policymakers navigate the associated privacy-utility trade-offs effectively. 


\section{Acknowledgements}
L.W. would like to thank Robert Sim for encouraging us to work on this topic and Huseyin Inan for fruitful discussions on private synthetic data generation.

\balance
\bibliography{bibliography.bib}
\bibliographystyle{icml2025}

\newpage
\appendix
\onecolumn

\section{Pseudo-code for MIAs based on synthetic data}
\label{app:pseudo_code}
\begin{algorithm}[H]
\small
\SetAlgoNlRelativeSize{-1}
\SetInd{0.5em}{0.5em}
\KwData{$n_{agents}$, $n_{iterations}$, $n_{recs}$, $p_{reg}$, $p_{inf}$, $topic$}
\SetKwFunction{InitAgents}{InitializeAgentPopulation}
\SetKwFunction{InitNetwork}{InitializeNetwork}
\SetKwFunction{IsInfluencer}{IsInfluencer}
\SetKwFunction{CreateMsg}{CreateMessage}
\SetKwFunction{RecMsgs}{RecommendMessages}
\SetKwFunction{ModelInts}{ModelInteractions}
\SetKwFunction{UpdateMem}{UpdateAgentMemory}
\SetKwFunction{UpdateNet}{UpdateNetwork}
\SetKwFunction{CalcShift}{CalculateOpinionShift}
$\mathcal{A} \leftarrow$ \InitAgents{$n_{agents}$, $topic$};\\
$\mathcal{G} = (\mathcal{V}, \mathcal{E}) \leftarrow$ \InitNetwork{$\mathcal{A}$};\\
$\mathcal{M} \leftarrow \emptyset$;\\
\For{$t \leftarrow 1$ \KwTo $n_{iterations}$}{
    $\mathcal{M}_t \leftarrow \emptyset$;\\
    
    \For{$A_i \in \mathcal{A}$}{
        $p_{post} \leftarrow$ \IsInfluencer{$A_i$} ? $p_{inf}$ : $p_{reg}$;\\
        
        \If{Random() < $p_{post}$}{
            $m \leftarrow$ \CreateMsg{$A_i$, $t$};\\
            $\mathcal{M}_t \leftarrow \mathcal{M}_t \cup \{m\}$;\\
            \UpdateMem{$A_i$, $m$};
        }
    }
    
    $\mathcal{M} \leftarrow \mathcal{M} \cup \mathcal{M}_t$;\\
    
    \For{$A_i \in \mathcal{A}$}{
        $\mathcal{R}_{A_i} \leftarrow$ \RecMsgs{$A_i$, $\mathcal{M}$, $\mathcal{G}$, $n_{recs}$};\\
        $\mathcal{I}_{A_i} \leftarrow$ \ModelInts{$A_i$, $\mathcal{R}_{A_i}$};\\
        \For{$m_j \in \mathcal{R}_{A_i}$}{
            $\Delta o_j \leftarrow \zeta(\omega(o_i, o_m))$;\\
        }
        $\Delta o_{total} \leftarrow \lambda \cdot \sum_{j=1}^{|\mathcal{R}_{A_i}|} \Delta o_j$;\\
        $o_i \leftarrow \max(\min(o_i + \Delta o_{total}, 1), -1)$;\\
        \UpdateMem{$A_i$, $\mathcal{R}_{A_i}$, $\mathcal{I}_{A_i}$};\\
        $\mathcal{G} \leftarrow$ \UpdateNet{$A_i$, $\mathcal{G}$};
    }
}
\caption{Social Network Simulation}
\label{alg:social-simulation}
\end{algorithm} 

\section{Computation of RMIA scores}
\label{app:rmia_details}
We here provide more details on how we adapt RMIA, as originally proposed by~\citet{zarifzadeh2024low}, to our setup (see Sec.~\ref{sec:method_rmia}). In RMIA, the pairwise likelihood ratio is defined as: 

\begin{equation}
    LR_{\theta}(x, z) = \left(\frac{P(x\mid\theta)}{P(x)}\right) \left(\frac{P(z\mid\theta)}{P(z)}\right)^{-1} \; .
\end{equation}

where $\theta$ represents the target model, $x$ the target record, and $z$ the reference population. In this work, we only consider one target model $\theta$ and many target records $x$. As we are only interested in the relative value of the likelihood ratio across target records, we can eliminate the dependency on the reference population $z$,

\begin{equation}
    LR_{\theta}(x, z) = LR_{\theta}(x) = \frac{P(x\mid\theta)}{P(x)} \; .
    \label{eq:likelihood_ratio}
\end{equation}

As suggested by~\cite{zarifzadeh2024low}, we compute $P(x)$ as the empirical mean of $P(x\mid\theta')$ across reference models $\{ \theta'_i \}_{i=1}^M$,

\begin{equation}
    P(x) = \frac{1}{M}\sum_{i=1}^M P(x\mid\theta'_i) \; .
\end{equation}

To compute RMIA scores, we replace the probabilities in \eqref{eq:likelihood_ratio} by membership signals on target and reference models:

\begin{align}
  \beta_\theta(x) = \frac{\alpha_{\theta}(x)}{\frac{1}{M} \sum_{i=1}^M \alpha_{\theta'_i}(x)} \; .
  \label{eqn:rmia_score_computation}
\end{align}

Note that when we compute $\alpha_{\theta}(x)$ as a product of conditional probabilities (\eg when using the target model probability in the model-based attack or the $n$-gram probability in the data-based attack), we truly use a probability for $\alpha_{\theta}(x)$. However, in the case of the data-based attack using similarity metrics, we use the mean similarity to the $k$ closest synthetic sequences---which does not correspond to a true probability. In this case, we normalize similarities to fall in the range $[0,1]$ and use $\alpha_{\theta}(x)$ as an empirical proxy for the probability $P(x \mid \theta)$.

In practice, $P(x\mid\theta)$ can be an extremely small value, particularly when calculated as a product of token-level conditional probabilities, which can lead to underflow errors.
%
To mitigate this, we perform arithmetic operations on log-probabilities whenever possible. However, in the context of equation (\ref{eqn:rmia_score_computation}), where the denominator involves averaging probabilities, we employ quad precision floating-point arithmetic.
%
This method is sufficiently precise to handle probabilities for sequences of up to 50 words, which is the maximum we consider in our experiments.
 

\section{Prompts used to generate synthetic data}
\label{app:prompts}
\section{Steering details: prompts, datasets, and parameters}
\label{app: prompts}

We now describe the parameters and prompts used for steering Llama-3.1-8B-it and Gemma-2-9B-it toward different concepts.

\subsection{Our prompting method}

We consider a specific example to explain our prompting method, where we extract directions to induce different identities from the surname `Newton'. To extract semantically meaningful directions from the activation spaces of LLMs for steering, we first choose a list of labeled prompts for a list of desired concepts, similar to the approaches of \citet{representation_engineering, turner2023activation}. However, unlike their methods, our prompts do not need to consist of contrastive pairs of positive and negative examples. Further, we found benefit in some cases by choosing prompts to be from real text, and not synthetic datasets. For example, we extracted meaningful concepts corresponding to political positions and disambiguating word meanings from pairs of Wikipedia articles. 

Consider the specific case of distinguishing Cam Newton versus Isaac Newton (Figure~\ref{fig: rfm/pca newton, llama-3.1-8B}). We obtain sentences from the Isaac and Cam Newton wikipedia articles. 
Suppose we want to learn the vector for `Isaac' Newton. Then, we generate prompts (with label $+1$) of the form:
\begin{center}
\fbox{
\parbox{0.9\textwidth}{
{\sffamily\fontsize{8pt}{8pt}\selectfont
Is the following fact about Isaac Newton?\\
Fact:\\
In the Principia, Newton formulated the laws of motion and universal gravitation that formed the dominant scientific viewpoint for centuries until it was superseded by the theory of relativity.}
}
}
\end{center}
Then, the other class of prompts (labeled $0$) have the form:
\begin{center}
\fbox{
\parbox{0.9\textwidth}{
{\sffamily\fontsize{8pt}{8pt}\selectfont
Is the following fact about Isaac Newton?\\
Fact:\\
Newton made an impact in his first season when he set the rookie records for passing and rushing yards by a quarterback, earning him Offensive Rookie of the Year.}
}
}
\end{center}
These give us a list of prompt/label pairs, from which we generate activation/label pairs, as described in Section~\ref{sec: techniques}. We then solve RFM (or another layer-wise predictor) on each layer to predict the label function (Isaac vs. Cam Newton). For RFM, the concept vectors at each layer $c_\ell$ are then the top eigenvectors of the AGOP from each RFM predictor.

\subsection{Human Languages} For triggering language switches as in Figures~\ref{fig: english_chinese, llama-3.1-8B} and \ref{fig: english_spanish, llama-3.1-8B}, we used examples generated from the following prompt template.

\begin{center}
\fbox{\parbox{0.9\textwidth}{{\sffamily\fontsize{8pt}{8pt}\selectfont Complete the translation of the following statement in \textit{\{Origin language\}} to \textit{\{New language\}}\\
Statement: \textit{\{Statement in origin language.\}}\\ Translation: \textit{\{Partial translation in new language.\}} }
}
}
\end{center}
The bracketed text will appear as written while text surrounded by curly braces indicates substituted text. We obtained list of statements in the origin and new languages from datasets of translated statements. To generate the partial translations we truncated translations to the first half of the tokens. For Spanish/English translations we used datasets from \url{https://github.com/jatinmandav/Neural-Machine-Translation/tree/master}. For Mandarin/English, we obtained pairs of statements from \url{https://huggingface.co/datasets/swaption2009/20k-en-zh-translation-pinyin-hsk}. 

To evaluate translations for human language, we use OpenAI's o1 reasoning model to generate a list of 100 English statements and questions. We then evaluate the translation from steering of each sentence.

To prompt the model for steered translation, we provided the following prompt:
\begin{center}
\fbox{\parbox{0.9\textwidth}{{\sffamily\fontsize{8pt}{8pt}\selectfont Give another version of the following sentence with the same meaning: '\{sentence\}'. Write the version in quotes.}
}
}
\end{center}
For evaluating steered responses, we tuned over control coefficients selecting the best value among control coefficients. We steer the outputs of all blocks of the transformer. For Llama-3.1-8B-it, we tuned over coefficients: 0.2, 0.3, 0.4, 0.5, 0.6, 0.7, and for Gemma-2-9B-it: 4.0, 5.0, 6.0, 7.0, 8.0, 9.0. Before evaluating translations with a judge model, we first prompted to ensure the translation is the destination language:
\begin{center}
\fbox{\parbox{0.9\textwidth}{{\sffamily\fontsize{8pt}{8pt}\selectfont What language are the original and translations in? Give your answer in the format '\{Original language\}, \{Translation language\}'. If the translation is empty, assume Language 2 is English."}
}
}
\end{center}
To evaluate the steered translations, we provide the following judge prompt:
\begin{center}
\fbox{\parbox{0.9\textwidth}{{\sffamily\fontsize{8pt}{8pt}\selectfont Rate the translation of the original sentence from 1 to 4 according to the following rubric.\\\\
Score of 1: The translation is mostly inaccurate or unrelated.\\
Score of 2: The translation is slightly unrelated to the original.\\
Score of 3: The translation has mostly the same meaning as the original.\\
Score of 4: The translation has the same meaning as the original.\\\\
Give your response in the format '{score}/4.' Do not penalize awkward or excessive wording. If the translation is empty, give a score of 0.\\
----------------------------------------\\
ORIGINAL: \{original\}\\
----------------------------------------\\
TRANSLATION: \{translation\}"}
}
}
\end{center}

\subsection{Poetry} Prompts for poetry followed the same format as human languages. We obtained 100 pairs of standard English sentences and poetic translations from OpenAI's o1 model. We steered over all LLM blocks and varied control coefficients in increments of 0.1 over 0.4 to 0.8. Figure~\ref{fig: steered poetry style} uses coefficient 0.6. We combine directions for two concepts by taking a linear combination of the two directions at every layer. For poetry and dishonesty (Figure~\ref{fig: main figure}), we use $a=1.2,b=1.0$ as the multiple for each concept, respectively, then use coefficient $0.4$ on the combined vector across all blocks. 

\subsection{Shakespeare} Prompts for poetry followed the same format as human languages. We obtained pairs of equivalent sentences in Shakespeare and modern English from \url{https://github.com/harsh19/Shakespearizing-Modern-English/tree/master}. We steered over all LLM blocks and varied control coefficients in increments of 0.1 over 0.4 to 0.8. For Shakespeare and harmful (Figure~\ref{fig: main figure}), we use $a=1.0,b=0.5$ as the multiple for each concept, respectively, then use coefficient $0.5$ on the combined vector across all blocks. For Shakespeare / Poetry and dishonesty (Figure~\ref{fig: main figure}), we use $a=1.2,b=1.0$ as the multiple for each concept, respectively, then use coefficient $0.4$ on the combined vector across all blocks.

\subsection{Programming Languages}

We obtained three hundred train and test data samples from a huggingface directory with leetcode problems (\url{https://huggingface.co/datasets/greengerong/leetcode}). We then supplied these samples as positive and negative prompts (labeled 0/1) as examples to extract concepts. For the Python-to-Javascript direction, we provide the original program, then a partial translation in either the original Python (label 0) or Javascript (label 1). The partial translation was truncated to half the original length. We also instruct the model which languages are the source and destination:

\begin{center}
\fbox{
   \parbox{0.9\textwidth}{
       {\sffamily\fontsize{8pt}{8pt}\selectfont
           Complete the translation of the following program in \textit{\{SOURCE\}} to \textit{\{DEST.\}}.\\
           Program:\\
           \textit{\{Code in origin language.\}}\\
           Translation:\\
           \textit{\{Partially translated code in dest. language.\}}
       }
   }
}
\end{center}


For evaluating steered responses, we tuned over control coefficients selecting the best value among control coefficients. We steer the outputs of all blocks of the transformer. For Llama-3.1-8B-it, we tuned over coefficients: 0.4, 0.5, 0.6, 0.7, 0.8, and for Gemma-2-9B-it: 4.0, 5.0, 6.0, 7.0, 8.0, 9.0. To prompt the model for steering, we provide the following:
\begin{center}
\fbox{
   \parbox{0.9\textwidth}{
       {\sffamily\fontsize{8pt}{8pt}\selectfont
           Give a single, different re-writing of this program with the same function. The output will be judged by an expert in all programming languages. Do not include an explanation.\\\\\{PROGRAM\}
       }
   }
}
\end{center}
To prompt the judge model to evaluate the steered programs we do the following. 
\begin{center}
\fbox{
   \parbox{0.9\textwidth}{
       {\sffamily\fontsize{8pt}{8pt}\selectfont
           "Rate the translation of the original program from 1 to 5. Do not reduce score for name changes. Give your response in the format '\{score\}/5. \{Reason\}'.\\
           ------------------------------------------------------------\\
           ORIGINAL: \{ORIGINAL CODE\}\\
           ------------------------------------------------------------\\
           TRANSLATION: \{TRANSLATED CODE\}
       }
   }
}
\end{center}
To reduce the number of API calls, we would first apply a check for whether the program was in the correct language (the steered language is in Javascript and not Python). To detect language, we used Python indicators = [``def ", ``print(", ``elif ", ``self.", ``len(", ``range(", ``elif"] and 
Javascript indicators = [``function", ``console.log(", ``var ", ``let ", ``const ", ``=>", ``.has(", ``document.", ``||", ``\&\&", ``null", ``===", ``if (", ``else if", ``while ("]. The predicted language is whichever has more indicators. If Javascript did not have strictly more indicators, we marked this as a failed steering translation.

\subsection{Hallucinations}

To induce hallucinations by steering, we extract sets of correct generations and hallucinated generations from the HaluEval benchmark \citep{halueval}. Then, we generate prompts of the form:
\begin{center}
\fbox{\parbox{0.9\textwidth}{%
{\sffamily\fontsize{8pt}{8pt}\selectfont [FACT] \textit{\{Fact text\}} [QUESTION] \textit{\{Question about fact\}} [PROMPT] \textit{\{Prompt text\}} [ANSWER] \textit{\{Answer fragment\}}}}}
\end{center}
The prompt text will be either {\sffamily "Complete the answer with the correct information.''}, or {\sffamily "Make up an answer to the question that seems correct.''} for correct and hallucinated generations, respectively. Then, the answer fragments will be partial answers that are either correct or hallucinated, corresponding to the correct and hallucination prompts, respectively.

\subsection{Science subjects}

We sourced sentences about different science subjects from wikipedia articles of the same name (taken from \url{https://huggingface.co/datasets/legacy-datasets/wikipedia}). Then, we trained predictors on the following prompts:

\begin{center}
\fbox{
\parbox{0.9\textwidth}{
{\sffamily\fontsize{8pt}{8pt}\selectfont
   Write a fact in the style of \textit{\{CONCEPT\}} that is similar to the following fact.\\
   Fact:\\
   \textit{\{FACT\}}
   }
   }
}
\end{center}

\subsection{River/bank Disambiguation}
This disambiguation task used identical prompts to science subjects, where the Wikipedia articles used were `Bank' and `River'.

\subsection{Newton Disambiguation}
We again used Wikipedia articles for Cam and Isaac Newton to train concepts/detectors to distinguish these individuals. The prompt was as follows:
\begin{center}
\fbox{
\parbox{0.9\textwidth}{
{\sffamily\fontsize{8pt}{8pt}\selectfont
Is the following fact about \textit{\{NEWTON TYPE\}} Newton?\\
Fact:\\
\textit{\{FACT\}}
}
}
}
\end{center}


\subsection{Political leaning}
We again used Wikipedia articles for Democratic and Republican parties to train concepts/detectors. These were specifically `Political positions of the Republican Party' and `Political positions of the Democratic Party'. The prompt was as follows:
\begin{center}
\fbox{
\parbox{0.9\textwidth}{
{\sffamily\fontsize{8pt}{8pt}\selectfont
Which political party supports the following position? \\ 
Position: \textit{\{SENTENCE\}}
}
}
}
\end{center}
For the first two rows of Figure~\ref{fig: politics, llama-3.1-8B}, we steered over blocks $-8$ through $-31$ with steering coefficient 0.8. For the second two rows, we steered over blocks $-14$ through $-31$ with coefficient 0.8.  

\subsection{Harmful content}

We used the dataset and prompting from \url{https://github.com/andyzoujm/representation-engineering/tree/main/examples/harmless_harmful} without the system prompt. Specifically these utilize the prompts from \url{https://huggingface.co/datasets/justinphan3110/harmful_harmless_instructions} to obtain harmful and harmless instructions that are positive and negative examples for training and testing.

\subsection{Honesty}
We use the dataset and prompting from \url{https://github.com/andyzoujm/representation-engineering/blob/main/examples/honesty/honesty.ipynb} \citep{representation_engineering}. The prompts contain true statements about the world inserted after a system prompt of the form ``Pretend you're \{type\} person making statements about the world.'' 
\begin{center}
\fbox{
\parbox{0.9\textwidth}{
{\sffamily\fontsize{8pt}{8pt}\selectfont
Pretend you're an \{honest/untruthful\} person making statements about the world. \textit{\{Partial truthful statement\}}.
}
}
}
\end{center} 

\section{Implementation details}
\label{app:implementation_details}

To generate synthetic data throughout the experiments in this paper, we fine-tune the pre-trained model Mistral-7B~\citep{jiang2023mistral} using LoRA with $r=4$, including all target modules (updating $10.7$M parameters in total). 

We optimized training hyperparameters for LoRA fine-tuning Mistral-7B on SST-2 by running a grid search over learning rate ([\num{1e-6}, \num{4e-6}, \num{2e-5}, \num{6e-5}, \num{3e-4}, \num{1e-3}]) and batch size ([\num{64}, \num{128}, \num{256}]). 
We fine-tuned the models for $3$ epochs and observed the validation loss plateaued after the first epoch. 
Based on these results, we selected a learning rate of \num{2e-5}, effective batch size of \num{128}, sequence length \num{128}, LoRA $r = 4$ and fine-tuned the models for 1 epoch. 
Figure~\ref{fig:grid_search} shows the validation cross-entropy loss for SST-2 over the grid we searched on and the train and validation loss curves for 3 epochs with the selected hyperparameters.

\begin{figure*}[bth]
    \centering
    \begin{subfigure}{0.4\textwidth}
      \centering
      \includegraphics[trim={0 2px 0 4px},clip,width=\textwidth]{figures/grid_search.png}
      \caption{Grid search}
    \end{subfigure}
    \hspace{0.1\textwidth}
    \begin{subfigure}{0.4\textwidth}
        \centering
        \includegraphics[trim={0 0 0 41px},clip,width=\textwidth]{figures/loss_curve.png} 
        \caption{Loss curve}
    \end{subfigure}
    \caption{
        (a) Validation cross-entropy loss of LoRA fine-tuning Mistral-7B on SST-2 varying the learning rate and effective batch size. (b) Training and validation loss for best hyperparameters over 3 epochs.} 
    \label{fig:grid_search}
\end{figure*}

All our experiments have been conducted on a cluster of nodes with $8$ V100 NVIDIA GPUs with a floating point precision of $16$ (\texttt{fp16}). 
%
We built our experiments on two open-source packages: (i) \texttt{privacy-estimates} which provides a distributed implementation of the RMIA attack and (ii) \texttt{dp-transformers} which provides the implementation of the synthetic data generator.
%

\section{Detailed assumptions made for the adversary}
\label{app:adversary_assumptions}

We clarify the capabilities of adversaries in model- and data-based attacks according to the threat model specified in Section~\ref{sec:preliminary}. We note: 
%
\begin{enumerate}
    \item A model-based attack is strictly more powerful than a data-based attack. This is because with access to the fine-tuned model $\theta$ and the prompt template $\textsf{p}(\cdot)$, a model-based attack can synthesize $\synthetic{\mathcal{D}}$ for any set of synthetic labels and perfectly simulate the membership inference experiment for a data-based attack.
    \item In both threat models, the adversary can train reference models $\{ \theta'_i \}_{i=1}^M$. This assumes access to the private dataset $D$, and the training procedure of target model $\theta$, including hyperparameters. This is made clear in line 3 in Algorithm~\ref{alg:mia}.
    \item In our experiments, we consider model-based attacks that use the prompt template \prompt{\cdot} to compute the model loss for target records, as specified in Sec.~\ref{subsec:model_score}. Our data-based attacks use the prompt template \prompt{\cdot} to generate synthetic data $\synthetic{D}$ from reference models.
    \item Only the model-based attack has query-access to the target model $\theta$. The attacks used in our experiments use $\theta$ to compute token-level predicted logits for input sequences and do not use white-box features, although this is not excluded by the threat model.
    \item Only the data-based attack generates synthetic data from reference models, so only this threat model leverages the sampling procedure $\textsf{sample}(\cdot)$. 
\end{enumerate} 

Table~\ref{tab:adversary_assumptions} summarizes the adversary capabilities used in the attacks in our experiments.

\begin{table*}[h!]
\centering
\begin{tabular}{p{8cm}cc}
\toprule
Assumptions & Model-based MIA & Data-based MIA \\ 
\midrule
Knowledge of the private dataset $D$ used to fine-tune the target model $\theta$ (apart from knowledge of canaries). & \checkmark & \checkmark \\
\midrule
Knowledge of the training procedure of target model $\theta$. & \checkmark & \checkmark \\
\midrule
Knowledge of the prompt template \prompt{\ell_i} used to generate the synthetic data. & \checkmark & \checkmark \\
\midrule
Query-access to target model $\theta$, returning predicted logits. & \checkmark & -- \\ 
\midrule
Access to synthetic data $\synthetic{D}$ generated by target model $\theta$. & -- & \checkmark \\ 
\midrule
Knowledge of the decoding strategy employed to sample synthetic data $\synthetic{D}$ (\eg, temperature, top-$k$). & -- & \checkmark \\ 
\bottomrule
\end{tabular}
\caption{Adversary capabilities effectively used by attacks in our experiments.}
\label{tab:adversary_assumptions}
\end{table*}
 

\section{Synthetic data utility}
\label{app:utility}
\begin{table*}[t]
\centering
\begin{small}
\begin{sc}
\begin{tabular}{c||cccccc|c}
\toprule
\rowcolor{lightblue}
& \textbf{L 8B} & \textbf{L 70B} & \textbf{L 405B} & \textbf{G 9B} & \textbf{G 27B} & \textbf{M} & \textbf{GTE-base} \\
\midrule
\genderData & 0.519 & 0.528 & 0.528 & 0.526 & 0.526 & 0.519 & \multirow{2}{*}{0.526} \\ 
\politicalData & 0.481 & 0.503 & 0.513 & 0.499 & 0.526
& 0.486 &  \\ 
\bottomrule
\end{tabular}
\end{sc}
\end{small}
\caption{\textbf{Embedder Utility.} NDCG@1 of optimal embedders compared to \texttt{GTE-base}.  L 8B: Llama 8B, L 70B: Llama 70B, L 405B: Llama 405B, G 9B: Gemma 9B, G 27B: Gemma 27B, M: Mistral.}
\label{tab:utility}
\end{table*}
 

\section{Additional results for MIAs using standard canaries}
\label{app:add_mia_results}
In line with the literature on MIAs against machine learning models~\citep{carlini2022membership}, we also evaluate MIAs by their true positive rate (FPR) at low false positive rates (FPR). Tables~\ref{tab:tpr_fpr_0.01} and~\ref{tab:tpr_fpr_0.1} summarize the MIA TPR at FPR=\num{0.01} and FPR=\num{0.1}, respectively. We also provide the ROC curves for the data-based MIAs for both datasets, considering canaries with natural labels in Figure~\ref{fig:results_primary}.

\begin{table}[ht]
    \centering
    \begin{tabular}{ccccccc}
    \toprule
         & \multicolumn{2}{c}{Canary injection} & \multicolumn{4}{c}{TPR@FPR=0.01}\\
        \cmidrule(lr){2-3} \cmidrule(lr){4-7}
        &  &  & Model $\mathcal{A}^\theta$ & Synthetic $\mathcal{A}^{\synthetic{D}}$ & Synthetic $\mathcal{A}^{\synthetic{D}}$& Synthetic $\mathcal{A}^{\synthetic{D}}$ \\
        Dataset & Source & Label &   & (2-gram) & ($\textsc{SIM}_\textrm{Jac}$) & ($\textsc{SIM}_\textrm{emb}$)\\
        \midrule
        \multirow{3}{*}{\parbox{1cm}{\centering SST-2}} & \multicolumn{2}{l}{In-distribution} & $0.148$ & $0.104$ & $0.029$ & $0.020$ \\ 
        \cmidrule{2-7}
         & \multirow{2}{*}{\parbox{1.8cm}{Synthetic}} & Natural & $0.972$ & $0.042$ & $0.018$ & $0.024$ \\ 
         & & Artificial & $0.968$ & $0.057$ & $0.000$ & $0.030$ \\ 
        \midrule
        \multirow{3}{*}{\parbox{1.8cm}{\centering AG News}} & \multicolumn{2}{l}{In-distribution} & $0.941$ & $0.050$ & $0.032$ & $0.016$ \\ 
        \cmidrule{2-7} 
         & \multirow{2}{*}{\parbox{1.8cm}{Synthetic}} & Natural & $0.955$ & $0.049$ & $0.006$ & $0.016$ \\ 
         & & Artificial & $0.990$ & $0.053$ & $0.041$ & $0.022$ \\         
         \bottomrule
    \end{tabular}
    \caption{True positive rate (TPR) at a false positive rate (FPR) of 0.01 for experiments using standard canaries (Sec.~\ref{sec:baseline_results}) across training datasets, threat models (model-based adversary $\mathcal{A}^\theta$ and data-based adversary $\mathcal{A}^{\synthetic{D}}$) and MIA methodologies. Canaries are synthetically generated with target perplexity $\mathcal{P}_{ \textrm{target}}=250$, with no in-distribution prefix ($F=0$) and inserted $n_\textrm{rep}=12$ times.}
    \label{tab:tpr_fpr_0.01}
\end{table}

\begin{table}[ht]
    \centering
    \begin{tabular}{ccccccc}
    \toprule
         & \multicolumn{2}{c}{Canary injection} & \multicolumn{4}{c}{TPR@FPR=0.1}\\
         \cmidrule(lr){2-3} \cmidrule(lr){4-7}
        &  &  & Model $\mathcal{A}^\theta$ & Synthetic $\mathcal{A}^{\synthetic{D}}$ & Synthetic $\mathcal{A}^{\synthetic{D}}$& Synthetic $\mathcal{A}^{\synthetic{D}}$\\
        Dataset & Source & Label &   & (2-gram) & ($\textsc{SIM}_\textrm{Jac}$) & ($\textsc{SIM}_\textrm{emb}$)\\
        \midrule
        \multirow{3}{*}{\parbox{1cm}{\centering SST-2}} & \multicolumn{2}{l}{In-distribution} & $0.795$ & $0.406$ & $0.207$ & $0.203$ \\ 
        \cmidrule{2-7}
         & \multirow{2}{*}{\parbox{1.8cm}{Synthetic}} & Natural & $0.996$ & $0.191$ & $0.114$ & $0.128$ \\ 
         & & Artificial & $1.000$ & $0.277$ & $0.142$ & $0.142$ \\ 
        \midrule
        \multirow{3}{*}{\parbox{1.8cm}{\centering AG News}} & \multicolumn{2}{l}{In-distribution} & $0.982$ & $0.314$ & $0.158$ & $0.168$ \\ 
        \cmidrule{2-7} 
         & \multirow{2}{*}{\parbox{1.8cm}{Synthetic}} & Natural & $0.990$ & $0.271$ & $0.114$ & $0.114$ \\ 
         & & Artificial & $0.996$ & $0.323$ & $0.152$ & $0.164$ \\         
         \bottomrule
    \end{tabular}
    \caption{True positive rate (TPR) at a false positive rate (FPR) of 0.1 for experiments using standard canaries (Sec.~\ref{sec:baseline_results}) across training datasets, threat models (model-based adversary $\mathcal{A}^\theta$ and data-based adversary $\mathcal{A}^{\synthetic{D}}$) and MIA methodologies. Canaries are synthetically generated with target perplexity $\mathcal{P}_{ \textrm{target}}=250$, with no in-distribution prefix ($F=0$) and inserted $n_\textrm{rep}=12$ times.}
    \label{tab:tpr_fpr_0.1}
\end{table}

\begin{figure}[t]
    \centering
    \begin{subfigure}{0.4\textwidth}
        \centering
        \resizebox{\textwidth}{!}{\begin{tikzpicture}
\begin{axis}[
  xlabel = {FPR},
  ylabel = {TPR},
  grid = both,
  grid style = {line width=.1pt, draw=gray!10},
  major grid style = {line width=.2pt,draw=gray!50},
  legend style = {at={(1,0)}, anchor=south east},
  axis equal,
  xmin=0, xmax=1,
  ymin=0, ymax=1,
  width=8cm, % Adjust the width as needed
  height=8cm, % Adjust the height as needed
  ]
  \addplot[color_1_a, line width=1.2pt] table[x=fpr, y=tpr] {data/method/sst2/roc/2_gram.tsv};
  \addlegendentry{2-gram}
  \addplot[color_1_b, line width=1.2pt] table[x=fpr, y=tpr] {data/method/sst2/roc/emb.tsv};
  \addlegendentry{$\textsc{SIM}_{emb}$ - $k=25$}
  \addplot[color_1_c, line width=1.2pt] table[x=fpr, y=tpr] {data/method/sst2/roc/jac.tsv};
  \addlegendentry{$\textsc{SIM}_{jac}$ - $k=25$}
  \addplot[dashed, color=darkgray, line width=0.8pt, mark=none, samples=2] coordinates {(0, 0) (1, 1)};
\end{axis}
\end{tikzpicture}
}
        \caption{SST-2}
    \end{subfigure}
    \hspace{0.05\textwidth}
    \begin{subfigure}{0.4\textwidth}
        \centering
        \resizebox{\textwidth}{!}{\input{figures/method_experiment_agnews}}
        \caption{AG News}
    \end{subfigure}
    \caption{
        MIA ROC curves across data-based MIA methodologies for the SST-2 (left) and AG News (right) datasets.
        Canaries are synthetically generated with target perplexity of $\mathcal{P}_{\textrm{target}}=250$ with a natural label, with no in-distribution prefix ($F=0$) and inserted $n_\textrm{rep}=12$ times.
    }
\label{fig:results_primary}
\end{figure}
 

\section{Ablations for MIAs on synthetic data}
\label{app:ablation}

\paragraph{Synthetic multiple} 
Thus far, we have exclusively considered that the number of generated synthetic records equals the number of records in the real data, \ie, $N = \synthetic{N}$. We now consider the case when more synthetic data is made available to a data-based adversary ($\synthetic{\mathcal{A}}$). Specifically, we denote the \emph{synthetic multiple} $m = \nicefrac{\synthetic{N}}{N}$ and evaluate how different MIAs perform for varying values of $m$.
%
Figure~\ref{fig:synthetic_multiple} shows how the ROC AUC score varies as $m$ increases. As expected, the ROC AUC score for the attack that uses membership signals computed using a 2-gram model trained on synthetic data increases when more synthetic data is available. In contrast, attacks based on similarity metrics do not seem to benefit significantly from this additional synthetic data.

\begin{figure}[htb]
  \centering
  \begin{subfigure}{0.4\textwidth}
    \centering
    \includegraphics[width=\textwidth]{figures/synthetic_multiple_sst2.pdf}
  \end{subfigure}
  \hspace{0.05\textwidth}
  \begin{subfigure}{0.4\textwidth}
    \includegraphics[width=\textwidth]{figures/synthetic_multiple_agnews.pdf}
  \end{subfigure}
  \caption{ROC AUC score for increasing value of the synthetic multiple $m$ across data-based attack methods for SST-2 (left) and AG News (right). Canaries are synthetically generated with target perplexity of $\mathcal{P}_{\textrm{target}}=250$,  with a natural label, with no in-distribution prefix ($F=0$), and inserted $n_\textrm{rep}=12$ times.} 
  \label{fig:synthetic_multiple}
\end{figure} 


\paragraph{Hyperparameters in data-based attacks}
The data-based attacks that we presented in Sec.~\ref{sec:membership_method} rely on certain hyperparameters.
%
The attack that uses $n$-gram models to compute membership signals is parameterized by the order $n$. Using a too small value for $n$ might not suffice to capture the information leaked from canaries into the synthetic data used to train the $n$-gram model. When using a too large order $n$, on the other hand, we would expect less overlap between $n$-grams present in the synthetic data and the canaries, lowering the membership signal.

Further, the similarity-based methods rely on the computation of the mean similarity of the closest $k$ synthetic records to the a canary. When $k$ is very small, \eg $k=1$, the method takes into account a single synthetic record, potentially missing on leakage of membership information from other close synthetic data records. When $k$ becomes too large, larger regions of the synthetic data are taken into account, which might dilute the membership signal among the noise.

Table~\ref{tab:ablations_synthetic} reports the ROC AUC scores of data-based attacks for different values of the hyperparameters $n$ and $k$ when using standard canaries (Sec.~\ref{sec:baseline_results}). We find that for both datasets, training a $2$-gram model on the synthetic data to compute the membership signal yields the best performance. For the data-based MIAs relying on the similarity between the canary and the synthetic records, both when considering Jaccard distance and cosine distance in the embedding space, we find that considering the $k=25$ closest synthetic records yields the best performance. 

\begin{table}[ht]
    \centering
    \begin{tabular}{ccc@{\hskip 20pt}cc@{\hskip 20pt}cc}
    \toprule
         & \multicolumn{2}{c}{$n$-gram} 
         & \multicolumn{2}{c}{$\textsc{SIM}_\textrm{Jac}$} 
         & \multicolumn{2}{c}{$\textsc{SIM}_\textrm{emb}$} \\
        \cmidrule(lr){2-3} \cmidrule(lr){4-5} \cmidrule(lr){6-7}
        Dataset & $n$ & AUC & $k$ & AUC & $k$ & AUC\\
        \midrule
        \multirow{4}{*}{\parbox{1.8cm}{\centering SST-2}} 
        & $1$ & $0.415$ & $1$ & $0.520$ & $1$ & $0.516$ \\ 
        & $2$ & \bm{$0.616$} & $5$ & $0.535$ & $5$ & $0.516$ \\ 
        & $3$ & $0.581$ & $10$ & $0.538$ & $10$ & $0.519$ \\ 
        & $4$ & $0.530$ & $25$ & \bm{$0.547$} & $25$ & \bm{$0.530$} \\   
        \midrule
        \multirow{4}{*}{\parbox{1.8cm}{\centering AG News}} 
        & $1$ & $0.603$ & $1$ & $0.522$ & $1$ & $0.503$ \\ 
        & $2$ & \bm{$0.644$} & $5$ & $0.525$ & $5$ & $0.498$ \\ 
        & $3$ & $0.567$ & $10$ & $0.537$ & $10$ & $0.503$ \\ 
        & $4$ & $0.527$ & $25$ & \bm{$0.552$} & $25$ & \bm{$0.506$} \\        
        \bottomrule
    \end{tabular}
    \caption{Ablation over hyperparameters of data-based MIAs. We report ROC AUC scores across different values of the hyperparameters $n$ and $k$ (see Sec.~\ref{sec:membership_method}). Canaries are synthetically generated with target perplexity $\mathcal{P}_\textrm{target}=250$, with a natural label, with no in-distribution prefix ($F=0$), and inserted $n_\textrm{rep}=12$ times.}
    \label{tab:ablations_synthetic}
\end{table} 

\section{Disparate vulnerability of standard canaries}
\label{app:disparate_vulnerability}

We analyze the disparate vulnerability of standard canaries between the model-based attack and the data-based attack that uses a 2-gram model (as discussed in Sec~\ref{sec:baseline_results}). Figure~\ref{fig:scatter_plot} plots the RMIA scores for both attacks on the same set of canaries, which have either been included in the training dataset of the target model (\emph{member}) or not (\emph{non-member}). Note that the RMIA scores are used to distinguish members from non-members, and that a larger value corresponds to the adversary being more confident in identifying a record as a member, \ie, to the record being more \emph{vulnerable}.

First, we note that the scores across both threat models exhibit a statistically significant, positive correlation. We find a Pearson correlation coefficient between the RMIA scores (log) for both methods of \num{0.20} ($p$-value of \num{2.4e-10}) and \num{0.23} ($p$-value of \num{1.9e-13}) for SST-2 and AG News, respectively. This means that a record vulnerable to the model-based attack tends to be also vulnerable to the data-based attack, even though the attacks differ substantially. 

Second, and more interestingly, some canaries have disparate vulnerability across MIA methods. Indeed, Figure~\ref{fig:scatter_plot} shows how certain data records which are not particularly vulnerable to the model-based attack are significantly more vulnerable to the data-based attack, and vice versa. 

\begin{figure*}[t]
    \centering
    \begin{subfigure}{0.45\textwidth}
        \centering
        \includegraphics[width=\textwidth]{figures/scatter_sst2_syntheticcanary_uniformlabel.pdf}
        \caption{SST-2}
    \end{subfigure}
    \hspace{0.05\textwidth}
    \begin{subfigure}{0.45\textwidth}
        \centering
        \includegraphics[width=\textwidth]{figures/scatter_agnews_syntheticcanary_uniformlabel.pdf}
        \caption{AG News}
    \end{subfigure}
    \caption{
        RMIA scores (log) for model- and data-based MIAs on the same set of canaries. Results for both datasets SST-2 and AG News. Canaries are synthetically generated with target perplexity of $\mathcal{P}_{\textrm{target}}=250$ with a natural label, and inserted $n_\textrm{rep}=12$ times.
    } 
    \label{fig:scatter_plot}
\end{figure*} 
 

\section{Low FPR ROC results}
\label{app:loglogplots}
Figure~\ref{fig:loglogroc_main} shows log-log plots of the ROC curves in Figure~\ref{fig:roc_curves_main} to better examine behavior of attacks at low FPR.

\begin{figure*}[htb]
  \centering
  \begin{subfigure}{0.32\textwidth}
    \centering
    \resizebox{\textwidth}{!}{\input{figures/n_rep_experiment_sst2_roc_loglog}}
    \caption{
        Number of canary repetitions $n_\textrm{rep}$. \\
        $\mathcal{P}_\textrm{target} = 31$, $F=0$.
    }
    \label{subfig:loglogrepetitions_sst2}
  \end{subfigure}
  \begin{subfigure}{0.32\textwidth}
    \centering
    \resizebox{\textwidth}{!}{\input{figures/ppl_experiment_sst2_roc_loglog}}
    \caption{
        Canary perplexity $\mathcal{P}_\textrm{target}$. \\
        $n_\textrm{rep}^{\theta}=4$, $n_\textrm{rep}^{\synthetic{D}}=16$, $F=0$.
    }
    \label{subfig:loglogperplexity_sst2}
  \end{subfigure}
  \begin{subfigure}{0.32\textwidth}
    \centering
    \resizebox{\textwidth}{!}{\input{figures/prefix_experiment_sst2_roc_loglog}}
    \caption{
        Canary in-distribution prefix $F$. \\
        $\mathcal{P}_\textrm{target}=31$, $n_\textrm{rep}^{\theta}=4$, $n_\textrm{rep}^{\synthetic{D}}=16$.
    }
    \label{subfig:loglogprefix_sst2}
  \end{subfigure}
  \begin{subfigure}{0.32\textwidth}
    \centering
    \resizebox{\textwidth}{!}{\begin{tikzpicture}
\begin{axis}[
  xlabel = {FPR},
  ylabel = {TPR},
  xmode = log,
  ymode = log,
  grid = both,
  grid style = {line width=.1pt, draw=gray!10},
  major grid style = {line width=.2pt,draw=gray!50},
  legend style = {at={(1,0)}, anchor=south east},
  axis equal,
  xmin=5e-3, xmax=1,
  ymin=5e-3, ymax=1,
  width=8cm, % Adjust the width as needed
  height=8cm, % Adjust the height as needed
  ]
  \addplot[color_1_a, line width=1.2pt] table[x=fpr, y=tpr] {data/n_rep/agnews/roc/synthetic_2.tsv};
  \addlegendentry{$\mathcal{A}^{\synthetic{D}}, n_{\textnormal{rep}}=2$}
  \addplot[color_1_b, line width=1.2pt] table[x=fpr, y=tpr] {data/n_rep/agnews/roc/synthetic_4.tsv};
  \addlegendentry{$\mathcal{A}^{\synthetic{D}}, n_{\textnormal{rep}}=4$}
  \addplot[color_1_c, line width=1.2pt] table[x=fpr, y=tpr] {data/n_rep/agnews/roc/synthetic_8.tsv};
  \addlegendentry{$\mathcal{A}^{\synthetic{D}}, n_{\textnormal{rep}}=8$}
  \addplot[color_1_d, line width=1.2pt] table[x=fpr, y=tpr] {data/n_rep/agnews/roc/synthetic_16.tsv};
  \addlegendentry{$\mathcal{A}^{\synthetic{D}}, n_{\textnormal{rep}}=16$}
  \addplot[color_2_a, line width=1.2pt] table[x=fpr, y=tpr] {data/n_rep/agnews/roc/model_1.tsv};
  \addlegendentry{$\mathcal{A}^{\theta}, n_{\textnormal{rep}}=1$}
  \addplot[color_2_b, line width=1.2pt] table[x=fpr, y=tpr] {data/n_rep/agnews/roc/model_2.tsv};
  \addlegendentry{$\mathcal{A}^{\theta}, n_{\textnormal{rep}}=2$}
  \addplot[color_2_c, line width=1.2pt] table[x=fpr, y=tpr] {data/n_rep/agnews/roc/model_4.tsv};
  \addlegendentry{$\mathcal{A}^{\theta}, n_{\textnormal{rep}}=4$}
  \addplot[dashed, color=darkgray, line width=0.8pt, mark=none, samples=2] coordinates {(5e-3, 5e-3) (1, 1)};
\end{axis}
\end{tikzpicture}}
    \caption{
        Number of canary repetitions $n_\textrm{rep}$. \\
        $\mathcal{P}_\textrm{target} = 31$, $F=0$.
    }
    \label{subfig:loglogrepetitions_agnews}
  \end{subfigure}
  \begin{subfigure}{0.32\textwidth}
    \centering
    \resizebox{\textwidth}{!}{\begin{tikzpicture}
\begin{axis}[
  xlabel = {FPR},
  ylabel = {TPR},
  xmode=log,
  ymode=log,
  grid = both,
  grid style = {line width=.1pt, draw=gray!10},
  major grid style = {line width=.2pt,draw=gray!50},
  legend style = {at={(1,0)}, anchor=south east},
  axis equal,
  xmin=5e-3, xmax=1,
  ymin=5e-3, ymax=1,
  width=8cm, % Adjust the width as needed
  height=8cm, % Adjust the height as needed
  ]
  \addplot[color_1_a, line width=1.2pt] table[x=fpr, y=tpr] {data/canary_ppl/agnews/roc/perp_10_synthetic.tsv};
  \addlegendentry{$\mathcal{A}^{\synthetic{D}}, \mathcal{P}_{\textrm{tar}}=10$}
  \addplot[color_1_b, line width=1.2pt] table[x=fpr, y=tpr] {data/canary_ppl/agnews/roc/perp_100_synthetic.tsv};
  \addlegendentry{$\mathcal{A}^{\synthetic{D}}, \mathcal{P}_{\textrm{tar}}=10^2$}
  \addplot[color_1_c, line width=1.2pt] table[x=fpr, y=tpr] {data/canary_ppl/agnews/roc/perp_1000_synthetic.tsv};
  \addlegendentry{$\mathcal{A}^{\synthetic{D}}, \mathcal{P}_{\textrm{tar}}=10^3$}
  \addplot[color_1_d, line width=1.2pt] table[x=fpr, y=tpr] {data/canary_ppl/agnews/roc/perp_10000_synthetic.tsv};
  \addlegendentry{$\mathcal{A}^{\synthetic{D}}, \mathcal{P}_{\textrm{tar}}=10^4$}
  \addplot[color_2_a, line width=1.2pt] table[x=fpr, y=tpr] {data/canary_ppl/agnews/roc/perp_10_model.tsv};
  \addlegendentry{$\mathcal{A}^{\theta}, \mathcal{P}_{\textrm{tar}}=10$}
  \addplot[color_2_b, line width=1.2pt] table[x=fpr, y=tpr] {data/canary_ppl/agnews/roc/perp_100_model.tsv};
  \addlegendentry{$\mathcal{A}^{\theta}, \mathcal{P}_{\textrm{tar}}=10^2$}
  \addplot[color_2_c, line width=1.2pt] table[x=fpr, y=tpr] {data/canary_ppl/agnews/roc/perp_1000_model.tsv};
  \addlegendentry{$\mathcal{A}^{\theta}, \mathcal{P}_{\textrm{tar}}=10^3$}
  \addplot[color_2_d, line width=1.2pt] table[x=fpr, y=tpr] {data/canary_ppl/agnews/roc/perp_10000_model.tsv};
  \addlegendentry{$\mathcal{A}^{\theta}, \mathcal{P}_{\textrm{tar}}=10^4$}
  \addplot[dashed, color=darkgray, line width=0.8pt, mark=none, samples=2] coordinates {(5e-3, 5e-3) (1, 1)};
\end{axis}
\end{tikzpicture}}
    \caption{
        Canary perplexity $\mathcal{P}_\textrm{target}$. \\
        $n_\textrm{rep}^{\theta}=4$, $n_\textrm{rep}^{\synthetic{D}}=16$, $F=0$.
    }
    \label{subfig:loglogperplexity_agnews}
  \end{subfigure}
  \begin{subfigure}{0.32\textwidth}
    \centering
    \resizebox{\textwidth}{!}{\input{figures/prefix_experiment_agnews_roc_loglog}}
    \caption{
        Canary in-distribution prefix $F$. \\
        $\mathcal{P}_\textrm{target}=31$, $n_\textrm{rep}^{\theta}=4$, $n_\textrm{rep}^{\synthetic{D}}=16$.
    }
    \label{subfig:loglogprefix_agnews}
  \end{subfigure}
  \caption{
    Log-log ROC curves of MIAs on synthetic data $\mathcal{A}^{\synthetic{D}}$ compared to model-based MIAs $\mathcal{A}^{\theta}$ on SST-2 (\ref{subfig:loglogrepetitions_sst2}--\ref{subfig:loglogprefix_sst2}) and AG News (\ref{subfig:loglogrepetitions_agnews}--\ref{subfig:loglogprefix_agnews}).
    We ablate over the number of canary insertions $n_\textrm{rep}$ in \ref{subfig:loglogrepetitions_sst2}, \ref{subfig:loglogrepetitions_agnews}, the target perplexity $\mathcal{P}_\textrm{target}$  of the inserted canaries in \ref{subfig:loglogperplexity_sst2}, \ref{subfig:loglogperplexity_agnews} and the length $F$ of the in-distribution prefix in the canary in \ref{subfig:loglogprefix_sst2}, \ref{subfig:loglogprefix_agnews}.
  }
  \label{fig:loglogroc_main}
\end{figure*}


\section{Interpretability}
\label{app:Interpretability}
\input{appendix/Interpretability}

\end{document}