%%%%%%%% ICML 2025 EXAMPLE LATEX SUBMISSION FILE %%%%%%%%%%%%%%%%%

\documentclass{article}

% Recommended, but optional, packages for figures and better typesetting:
\usepackage{microtype}
\usepackage{graphicx}
\usepackage{subfigure}
\usepackage{booktabs} % for professional tables

% hyperref makes hyperlinks in the resulting PDF.
% If your build breaks (sometimes temporarily if a hyperlink spans a page)
% please comment out the following usepackage line and replace
% \usepackage{icml2025} with \usepackage[nohyperref]{icml2025} above.
\usepackage{hyperref}


% Attempt to make hyperref and algorithmic work together better:
\newcommand{\theHalgorithm}{\arabic{algorithm}}

% Use the following line for the initial blind version submitted for review:
\usepackage[accepted]{icml2025}
%%%%% NEW MATH DEFINITIONS %%%%%

\usepackage{amsmath,amsfonts,bm}
\usepackage{derivative}
% Mark sections of captions for referring to divisions of figures
\newcommand{\figleft}{{\em (Left)}}
\newcommand{\figcenter}{{\em (Center)}}
\newcommand{\figright}{{\em (Right)}}
\newcommand{\figtop}{{\em (Top)}}
\newcommand{\figbottom}{{\em (Bottom)}}
\newcommand{\captiona}{{\em (a)}}
\newcommand{\captionb}{{\em (b)}}
\newcommand{\captionc}{{\em (c)}}
\newcommand{\captiond}{{\em (d)}}

% Highlight a newly defined term
\newcommand{\newterm}[1]{{\bf #1}}

% Derivative d 
\newcommand{\deriv}{{\mathrm{d}}}

% Figure reference, lower-case.
\def\figref#1{figure~\ref{#1}}
% Figure reference, capital. For start of sentence
\def\Figref#1{Figure~\ref{#1}}
\def\twofigref#1#2{figures \ref{#1} and \ref{#2}}
\def\quadfigref#1#2#3#4{figures \ref{#1}, \ref{#2}, \ref{#3} and \ref{#4}}
% Section reference, lower-case.
\def\secref#1{section~\ref{#1}}
% Section reference, capital.
\def\Secref#1{Section~\ref{#1}}
% Reference to two sections.
\def\twosecrefs#1#2{sections \ref{#1} and \ref{#2}}
% Reference to three sections.
\def\secrefs#1#2#3{sections \ref{#1}, \ref{#2} and \ref{#3}}
% Reference to an equation, lower-case.
\def\eqref#1{equation~\ref{#1}}
% Reference to an equation, upper case
\def\Eqref#1{Equation~\ref{#1}}
% A raw reference to an equation---avoid using if possible
\def\plaineqref#1{\ref{#1}}
% Reference to a chapter, lower-case.
\def\chapref#1{chapter~\ref{#1}}
% Reference to an equation, upper case.
\def\Chapref#1{Chapter~\ref{#1}}
% Reference to a range of chapters
\def\rangechapref#1#2{chapters\ref{#1}--\ref{#2}}
% Reference to an algorithm, lower-case.
\def\algref#1{algorithm~\ref{#1}}
% Reference to an algorithm, upper case.
\def\Algref#1{Algorithm~\ref{#1}}
\def\twoalgref#1#2{algorithms \ref{#1} and \ref{#2}}
\def\Twoalgref#1#2{Algorithms \ref{#1} and \ref{#2}}
% Reference to a part, lower case
\def\partref#1{part~\ref{#1}}
% Reference to a part, upper case
\def\Partref#1{Part~\ref{#1}}
\def\twopartref#1#2{parts \ref{#1} and \ref{#2}}

\def\ceil#1{\lceil #1 \rceil}
\def\floor#1{\lfloor #1 \rfloor}
\def\1{\bm{1}}
\newcommand{\train}{\mathcal{D}}
\newcommand{\valid}{\mathcal{D_{\mathrm{valid}}}}
\newcommand{\test}{\mathcal{D_{\mathrm{test}}}}

\def\eps{{\epsilon}}


% Random variables
\def\reta{{\textnormal{$\eta$}}}
\def\ra{{\textnormal{a}}}
\def\rb{{\textnormal{b}}}
\def\rc{{\textnormal{c}}}
\def\rd{{\textnormal{d}}}
\def\re{{\textnormal{e}}}
\def\rf{{\textnormal{f}}}
\def\rg{{\textnormal{g}}}
\def\rh{{\textnormal{h}}}
\def\ri{{\textnormal{i}}}
\def\rj{{\textnormal{j}}}
\def\rk{{\textnormal{k}}}
\def\rl{{\textnormal{l}}}
% rm is already a command, just don't name any random variables m
\def\rn{{\textnormal{n}}}
\def\ro{{\textnormal{o}}}
\def\rp{{\textnormal{p}}}
\def\rq{{\textnormal{q}}}
\def\rr{{\textnormal{r}}}
\def\rs{{\textnormal{s}}}
\def\rt{{\textnormal{t}}}
\def\ru{{\textnormal{u}}}
\def\rv{{\textnormal{v}}}
\def\rw{{\textnormal{w}}}
\def\rx{{\textnormal{x}}}
\def\ry{{\textnormal{y}}}
\def\rz{{\textnormal{z}}}

% Random vectors
\def\rvepsilon{{\mathbf{\epsilon}}}
\def\rvphi{{\mathbf{\phi}}}
\def\rvtheta{{\mathbf{\theta}}}
\def\rva{{\mathbf{a}}}
\def\rvb{{\mathbf{b}}}
\def\rvc{{\mathbf{c}}}
\def\rvd{{\mathbf{d}}}
\def\rve{{\mathbf{e}}}
\def\rvf{{\mathbf{f}}}
\def\rvg{{\mathbf{g}}}
\def\rvh{{\mathbf{h}}}
\def\rvu{{\mathbf{i}}}
\def\rvj{{\mathbf{j}}}
\def\rvk{{\mathbf{k}}}
\def\rvl{{\mathbf{l}}}
\def\rvm{{\mathbf{m}}}
\def\rvn{{\mathbf{n}}}
\def\rvo{{\mathbf{o}}}
\def\rvp{{\mathbf{p}}}
\def\rvq{{\mathbf{q}}}
\def\rvr{{\mathbf{r}}}
\def\rvs{{\mathbf{s}}}
\def\rvt{{\mathbf{t}}}
\def\rvu{{\mathbf{u}}}
\def\rvv{{\mathbf{v}}}
\def\rvw{{\mathbf{w}}}
\def\rvx{{\mathbf{x}}}
\def\rvy{{\mathbf{y}}}
\def\rvz{{\mathbf{z}}}

% Elements of random vectors
\def\erva{{\textnormal{a}}}
\def\ervb{{\textnormal{b}}}
\def\ervc{{\textnormal{c}}}
\def\ervd{{\textnormal{d}}}
\def\erve{{\textnormal{e}}}
\def\ervf{{\textnormal{f}}}
\def\ervg{{\textnormal{g}}}
\def\ervh{{\textnormal{h}}}
\def\ervi{{\textnormal{i}}}
\def\ervj{{\textnormal{j}}}
\def\ervk{{\textnormal{k}}}
\def\ervl{{\textnormal{l}}}
\def\ervm{{\textnormal{m}}}
\def\ervn{{\textnormal{n}}}
\def\ervo{{\textnormal{o}}}
\def\ervp{{\textnormal{p}}}
\def\ervq{{\textnormal{q}}}
\def\ervr{{\textnormal{r}}}
\def\ervs{{\textnormal{s}}}
\def\ervt{{\textnormal{t}}}
\def\ervu{{\textnormal{u}}}
\def\ervv{{\textnormal{v}}}
\def\ervw{{\textnormal{w}}}
\def\ervx{{\textnormal{x}}}
\def\ervy{{\textnormal{y}}}
\def\ervz{{\textnormal{z}}}

% Random matrices
\def\rmA{{\mathbf{A}}}
\def\rmB{{\mathbf{B}}}
\def\rmC{{\mathbf{C}}}
\def\rmD{{\mathbf{D}}}
\def\rmE{{\mathbf{E}}}
\def\rmF{{\mathbf{F}}}
\def\rmG{{\mathbf{G}}}
\def\rmH{{\mathbf{H}}}
\def\rmI{{\mathbf{I}}}
\def\rmJ{{\mathbf{J}}}
\def\rmK{{\mathbf{K}}}
\def\rmL{{\mathbf{L}}}
\def\rmM{{\mathbf{M}}}
\def\rmN{{\mathbf{N}}}
\def\rmO{{\mathbf{O}}}
\def\rmP{{\mathbf{P}}}
\def\rmQ{{\mathbf{Q}}}
\def\rmR{{\mathbf{R}}}
\def\rmS{{\mathbf{S}}}
\def\rmT{{\mathbf{T}}}
\def\rmU{{\mathbf{U}}}
\def\rmV{{\mathbf{V}}}
\def\rmW{{\mathbf{W}}}
\def\rmX{{\mathbf{X}}}
\def\rmY{{\mathbf{Y}}}
\def\rmZ{{\mathbf{Z}}}

% Elements of random matrices
\def\ermA{{\textnormal{A}}}
\def\ermB{{\textnormal{B}}}
\def\ermC{{\textnormal{C}}}
\def\ermD{{\textnormal{D}}}
\def\ermE{{\textnormal{E}}}
\def\ermF{{\textnormal{F}}}
\def\ermG{{\textnormal{G}}}
\def\ermH{{\textnormal{H}}}
\def\ermI{{\textnormal{I}}}
\def\ermJ{{\textnormal{J}}}
\def\ermK{{\textnormal{K}}}
\def\ermL{{\textnormal{L}}}
\def\ermM{{\textnormal{M}}}
\def\ermN{{\textnormal{N}}}
\def\ermO{{\textnormal{O}}}
\def\ermP{{\textnormal{P}}}
\def\ermQ{{\textnormal{Q}}}
\def\ermR{{\textnormal{R}}}
\def\ermS{{\textnormal{S}}}
\def\ermT{{\textnormal{T}}}
\def\ermU{{\textnormal{U}}}
\def\ermV{{\textnormal{V}}}
\def\ermW{{\textnormal{W}}}
\def\ermX{{\textnormal{X}}}
\def\ermY{{\textnormal{Y}}}
\def\ermZ{{\textnormal{Z}}}

% Vectors
\def\vzero{{\bm{0}}}
\def\vone{{\bm{1}}}
\def\vmu{{\bm{\mu}}}
\def\vtheta{{\bm{\theta}}}
\def\vphi{{\bm{\phi}}}
\def\va{{\bm{a}}}
\def\vb{{\bm{b}}}
\def\vc{{\bm{c}}}
\def\vd{{\bm{d}}}
\def\ve{{\bm{e}}}
\def\vf{{\bm{f}}}
\def\vg{{\bm{g}}}
\def\vh{{\bm{h}}}
\def\vi{{\bm{i}}}
\def\vj{{\bm{j}}}
\def\vk{{\bm{k}}}
\def\vl{{\bm{l}}}
\def\vm{{\bm{m}}}
\def\vn{{\bm{n}}}
\def\vo{{\bm{o}}}
\def\vp{{\bm{p}}}
\def\vq{{\bm{q}}}
\def\vr{{\bm{r}}}
\def\vs{{\bm{s}}}
\def\vt{{\bm{t}}}
\def\vu{{\bm{u}}}
\def\vv{{\bm{v}}}
\def\vw{{\bm{w}}}
\def\vx{{\bm{x}}}
\def\vy{{\bm{y}}}
\def\vz{{\bm{z}}}

% Elements of vectors
\def\evalpha{{\alpha}}
\def\evbeta{{\beta}}
\def\evepsilon{{\epsilon}}
\def\evlambda{{\lambda}}
\def\evomega{{\omega}}
\def\evmu{{\mu}}
\def\evpsi{{\psi}}
\def\evsigma{{\sigma}}
\def\evtheta{{\theta}}
\def\eva{{a}}
\def\evb{{b}}
\def\evc{{c}}
\def\evd{{d}}
\def\eve{{e}}
\def\evf{{f}}
\def\evg{{g}}
\def\evh{{h}}
\def\evi{{i}}
\def\evj{{j}}
\def\evk{{k}}
\def\evl{{l}}
\def\evm{{m}}
\def\evn{{n}}
\def\evo{{o}}
\def\evp{{p}}
\def\evq{{q}}
\def\evr{{r}}
\def\evs{{s}}
\def\evt{{t}}
\def\evu{{u}}
\def\evv{{v}}
\def\evw{{w}}
\def\evx{{x}}
\def\evy{{y}}
\def\evz{{z}}

% Matrix
\def\mA{{\bm{A}}}
\def\mB{{\bm{B}}}
\def\mC{{\bm{C}}}
\def\mD{{\bm{D}}}
\def\mE{{\bm{E}}}
\def\mF{{\bm{F}}}
\def\mG{{\bm{G}}}
\def\mH{{\bm{H}}}
\def\mI{{\bm{I}}}
\def\mJ{{\bm{J}}}
\def\mK{{\bm{K}}}
\def\mL{{\bm{L}}}
\def\mM{{\bm{M}}}
\def\mN{{\bm{N}}}
\def\mO{{\bm{O}}}
\def\mP{{\bm{P}}}
\def\mQ{{\bm{Q}}}
\def\mR{{\bm{R}}}
\def\mS{{\bm{S}}}
\def\mT{{\bm{T}}}
\def\mU{{\bm{U}}}
\def\mV{{\bm{V}}}
\def\mW{{\bm{W}}}
\def\mX{{\bm{X}}}
\def\mY{{\bm{Y}}}
\def\mZ{{\bm{Z}}}
\def\mBeta{{\bm{\beta}}}
\def\mPhi{{\bm{\Phi}}}
\def\mLambda{{\bm{\Lambda}}}
\def\mSigma{{\bm{\Sigma}}}

% Tensor
\DeclareMathAlphabet{\mathsfit}{\encodingdefault}{\sfdefault}{m}{sl}
\SetMathAlphabet{\mathsfit}{bold}{\encodingdefault}{\sfdefault}{bx}{n}
\newcommand{\tens}[1]{\bm{\mathsfit{#1}}}
\def\tA{{\tens{A}}}
\def\tB{{\tens{B}}}
\def\tC{{\tens{C}}}
\def\tD{{\tens{D}}}
\def\tE{{\tens{E}}}
\def\tF{{\tens{F}}}
\def\tG{{\tens{G}}}
\def\tH{{\tens{H}}}
\def\tI{{\tens{I}}}
\def\tJ{{\tens{J}}}
\def\tK{{\tens{K}}}
\def\tL{{\tens{L}}}
\def\tM{{\tens{M}}}
\def\tN{{\tens{N}}}
\def\tO{{\tens{O}}}
\def\tP{{\tens{P}}}
\def\tQ{{\tens{Q}}}
\def\tR{{\tens{R}}}
\def\tS{{\tens{S}}}
\def\tT{{\tens{T}}}
\def\tU{{\tens{U}}}
\def\tV{{\tens{V}}}
\def\tW{{\tens{W}}}
\def\tX{{\tens{X}}}
\def\tY{{\tens{Y}}}
\def\tZ{{\tens{Z}}}


% Graph
\def\gA{{\mathcal{A}}}
\def\gB{{\mathcal{B}}}
\def\gC{{\mathcal{C}}}
\def\gD{{\mathcal{D}}}
\def\gE{{\mathcal{E}}}
\def\gF{{\mathcal{F}}}
\def\gG{{\mathcal{G}}}
\def\gH{{\mathcal{H}}}
\def\gI{{\mathcal{I}}}
\def\gJ{{\mathcal{J}}}
\def\gK{{\mathcal{K}}}
\def\gL{{\mathcal{L}}}
\def\gM{{\mathcal{M}}}
\def\gN{{\mathcal{N}}}
\def\gO{{\mathcal{O}}}
\def\gP{{\mathcal{P}}}
\def\gQ{{\mathcal{Q}}}
\def\gR{{\mathcal{R}}}
\def\gS{{\mathcal{S}}}
\def\gT{{\mathcal{T}}}
\def\gU{{\mathcal{U}}}
\def\gV{{\mathcal{V}}}
\def\gW{{\mathcal{W}}}
\def\gX{{\mathcal{X}}}
\def\gY{{\mathcal{Y}}}
\def\gZ{{\mathcal{Z}}}

% Sets
\def\sA{{\mathbb{A}}}
\def\sB{{\mathbb{B}}}
\def\sC{{\mathbb{C}}}
\def\sD{{\mathbb{D}}}
% Don't use a set called E, because this would be the same as our symbol
% for expectation.
\def\sF{{\mathbb{F}}}
\def\sG{{\mathbb{G}}}
\def\sH{{\mathbb{H}}}
\def\sI{{\mathbb{I}}}
\def\sJ{{\mathbb{J}}}
\def\sK{{\mathbb{K}}}
\def\sL{{\mathbb{L}}}
\def\sM{{\mathbb{M}}}
\def\sN{{\mathbb{N}}}
\def\sO{{\mathbb{O}}}
\def\sP{{\mathbb{P}}}
\def\sQ{{\mathbb{Q}}}
\def\sR{{\mathbb{R}}}
\def\sS{{\mathbb{S}}}
\def\sT{{\mathbb{T}}}
\def\sU{{\mathbb{U}}}
\def\sV{{\mathbb{V}}}
\def\sW{{\mathbb{W}}}
\def\sX{{\mathbb{X}}}
\def\sY{{\mathbb{Y}}}
\def\sZ{{\mathbb{Z}}}

% Entries of a matrix
\def\emLambda{{\Lambda}}
\def\emA{{A}}
\def\emB{{B}}
\def\emC{{C}}
\def\emD{{D}}
\def\emE{{E}}
\def\emF{{F}}
\def\emG{{G}}
\def\emH{{H}}
\def\emI{{I}}
\def\emJ{{J}}
\def\emK{{K}}
\def\emL{{L}}
\def\emM{{M}}
\def\emN{{N}}
\def\emO{{O}}
\def\emP{{P}}
\def\emQ{{Q}}
\def\emR{{R}}
\def\emS{{S}}
\def\emT{{T}}
\def\emU{{U}}
\def\emV{{V}}
\def\emW{{W}}
\def\emX{{X}}
\def\emY{{Y}}
\def\emZ{{Z}}
\def\emSigma{{\Sigma}}

% entries of a tensor
% Same font as tensor, without \bm wrapper
\newcommand{\etens}[1]{\mathsfit{#1}}
\def\etLambda{{\etens{\Lambda}}}
\def\etA{{\etens{A}}}
\def\etB{{\etens{B}}}
\def\etC{{\etens{C}}}
\def\etD{{\etens{D}}}
\def\etE{{\etens{E}}}
\def\etF{{\etens{F}}}
\def\etG{{\etens{G}}}
\def\etH{{\etens{H}}}
\def\etI{{\etens{I}}}
\def\etJ{{\etens{J}}}
\def\etK{{\etens{K}}}
\def\etL{{\etens{L}}}
\def\etM{{\etens{M}}}
\def\etN{{\etens{N}}}
\def\etO{{\etens{O}}}
\def\etP{{\etens{P}}}
\def\etQ{{\etens{Q}}}
\def\etR{{\etens{R}}}
\def\etS{{\etens{S}}}
\def\etT{{\etens{T}}}
\def\etU{{\etens{U}}}
\def\etV{{\etens{V}}}
\def\etW{{\etens{W}}}
\def\etX{{\etens{X}}}
\def\etY{{\etens{Y}}}
\def\etZ{{\etens{Z}}}

% The true underlying data generating distribution
\newcommand{\pdata}{p_{\rm{data}}}
\newcommand{\ptarget}{p_{\rm{target}}}
\newcommand{\pprior}{p_{\rm{prior}}}
\newcommand{\pbase}{p_{\rm{base}}}
\newcommand{\pref}{p_{\rm{ref}}}

% The empirical distribution defined by the training set
\newcommand{\ptrain}{\hat{p}_{\rm{data}}}
\newcommand{\Ptrain}{\hat{P}_{\rm{data}}}
% The model distribution
\newcommand{\pmodel}{p_{\rm{model}}}
\newcommand{\Pmodel}{P_{\rm{model}}}
\newcommand{\ptildemodel}{\tilde{p}_{\rm{model}}}
% Stochastic autoencoder distributions
\newcommand{\pencode}{p_{\rm{encoder}}}
\newcommand{\pdecode}{p_{\rm{decoder}}}
\newcommand{\precons}{p_{\rm{reconstruct}}}

\newcommand{\laplace}{\mathrm{Laplace}} % Laplace distribution

\newcommand{\E}{\mathbb{E}}
\newcommand{\Ls}{\mathcal{L}}
\newcommand{\R}{\mathbb{R}}
\newcommand{\emp}{\tilde{p}}
\newcommand{\lr}{\alpha}
\newcommand{\reg}{\lambda}
\newcommand{\rect}{\mathrm{rectifier}}
\newcommand{\softmax}{\mathrm{softmax}}
\newcommand{\sigmoid}{\sigma}
\newcommand{\softplus}{\zeta}
\newcommand{\KL}{D_{\mathrm{KL}}}
\newcommand{\Var}{\mathrm{Var}}
\newcommand{\standarderror}{\mathrm{SE}}
\newcommand{\Cov}{\mathrm{Cov}}
% Wolfram Mathworld says $L^2$ is for function spaces and $\ell^2$ is for vectors
% But then they seem to use $L^2$ for vectors throughout the site, and so does
% wikipedia.
\newcommand{\normlzero}{L^0}
\newcommand{\normlone}{L^1}
\newcommand{\normltwo}{L^2}
\newcommand{\normlp}{L^p}
\newcommand{\normmax}{L^\infty}

\newcommand{\parents}{Pa} % See usage in notation.tex. Chosen to match Daphne's book.

\DeclareMathOperator*{\argmax}{arg\,max}
\DeclareMathOperator*{\argmin}{arg\,min}

\DeclareMathOperator{\sign}{sign}
\DeclareMathOperator{\Tr}{Tr}
\let\ab\allowbreak


% If accepted, instead use the following line for the camera-ready submission:
% \usepackage[accepted]{icml2025}

% For theorems and such
\usepackage{amsmath}
\usepackage{amssymb}
\usepackage{mathtools}
\usepackage{amsthm}

% if you use cleveref..
\usepackage[capitalize,noabbrev]{cleveref}

%%%%%%%%%%%%%%%%%%%%%%%%%%%%%%%%
% THEOREMS
%%%%%%%%%%%%%%%%%%%%%%%%%%%%%%%%
\theoremstyle{plain}
\newtheorem{theorem}{Theorem}[section]
\newtheorem{proposition}[theorem]{Proposition}
\newtheorem{lemma}[theorem]{Lemma}
\newtheorem{corollary}[theorem]{Corollary}
\theoremstyle{definition}
\newtheorem{definition}[theorem]{Definition}
\newtheorem{assumption}[theorem]{Assumption}
\theoremstyle{remark}
\newtheorem{remark}[theorem]{Remark}

% Todonotes is useful during development; simply uncomment the next line
%    and comment out the line below the next line to turn off comments
%\usepackage[disable,textsize=tiny]{todonotes}
\usepackage[textsize=tiny]{todonotes}


% The \icmltitle you define below is probably too long as a header.
% Therefore, a short form for the running title is supplied here:
\icmltitlerunning{Improving LLM General Preference Alignment via Optimistic Online Mirror Descent}

\begin{document}

\twocolumn[
\icmltitle{Improving LLM General Preference Alignment \\ via Optimistic Online Mirror Descent}

\icmlsetsymbol{equal}{*}

\begin{icmlauthorlist}
\icmlauthor{Yuheng Zhang}{uiuc}
\icmlauthor{Dian Yu}{tencent}
\icmlauthor{Tao Ge}{tencent}
\icmlauthor{Linfeng Song}{tencent}
\icmlauthor{Zhichen Zeng}{uiuc}
\icmlauthor{Haitao Mi}{tencent}
\icmlauthor{Nan Jiang}{uiuc}
\icmlauthor{Dong Yu}{tencent}
\end{icmlauthorlist}

\icmlaffiliation{uiuc}{University of Illinois Urbana-Champaign}
\icmlaffiliation{tencent}{
Tencent AI Lab, Bellevue}

\icmlcorrespondingauthor{Yuheng Zhang}{yuhengz2@illinois.edu}


% You may provide any keywords that you
% find helpful for describing your paper; these are used to populate
% the "keywords" metadata in the PDF but will not be shown in the document
\icmlkeywords{LLM Alignment, RLHF}

\vskip 0.3in
]

% this must go after the closing bracket ] following \twocolumn[ ...

% This command actually creates the footnote in the first column
% listing the affiliations and the copyright notice.
% The command takes one argument, which is text to display at the start of the footnote.
% The \icmlEqualContribution command is standard text for equal contribution.
% Remove it (just {}) if you do not need this facility.

\printAffiliationsAndNotice{}  % leave blank if no need to mention equal contribution


\begin{abstract}
Reinforcement learning from human feedback (RLHF) has demonstrated remarkable effectiveness in aligning large language models (LLMs) with human preferences. Many existing alignment approaches rely on the Bradley-Terry (BT) model assumption, which assumes the existence of a ground-truth reward for each prompt-response pair. However, this assumption can be overly restrictive when modeling complex human preferences. In this paper, we drop the BT model assumption and study LLM alignment under general preferences, formulated as a two-player game. Drawing on theoretical insights from learning in games, we integrate optimistic online mirror descent into our alignment framework to approximate the Nash policy. Theoretically, we demonstrate that our approach achieves an $\mathcal{O}(T^{-1})$ bound on the duality gap, improving upon the previous $\mathcal{O}(T^{-1/2})$ result. More importantly, we implement our method and show through experiments that it outperforms state-of-the-art RLHF algorithms across multiple representative benchmarks.
\end{abstract}


\section{Introduction}


\begin{figure}[t]
\centering
\includegraphics[width=0.6\columnwidth]{figures/evaluation_desiderata_V5.pdf}
\vspace{-0.5cm}
\caption{\systemName is a platform for conducting realistic evaluations of code LLMs, collecting human preferences of coding models with real users, real tasks, and in realistic environments, aimed at addressing the limitations of existing evaluations.
}
\label{fig:motivation}
\end{figure}

\begin{figure*}[t]
\centering
\includegraphics[width=\textwidth]{figures/system_design_v2.png}
\caption{We introduce \systemName, a VSCode extension to collect human preferences of code directly in a developer's IDE. \systemName enables developers to use code completions from various models. The system comprises a) the interface in the user's IDE which presents paired completions to users (left), b) a sampling strategy that picks model pairs to reduce latency (right, top), and c) a prompting scheme that allows diverse LLMs to perform code completions with high fidelity.
Users can select between the top completion (green box) using \texttt{tab} or the bottom completion (blue box) using \texttt{shift+tab}.}
\label{fig:overview}
\end{figure*}

As model capabilities improve, large language models (LLMs) are increasingly integrated into user environments and workflows.
For example, software developers code with AI in integrated developer environments (IDEs)~\citep{peng2023impact}, doctors rely on notes generated through ambient listening~\citep{oberst2024science}, and lawyers consider case evidence identified by electronic discovery systems~\citep{yang2024beyond}.
Increasing deployment of models in productivity tools demands evaluation that more closely reflects real-world circumstances~\citep{hutchinson2022evaluation, saxon2024benchmarks, kapoor2024ai}.
While newer benchmarks and live platforms incorporate human feedback to capture real-world usage, they almost exclusively focus on evaluating LLMs in chat conversations~\citep{zheng2023judging,dubois2023alpacafarm,chiang2024chatbot, kirk2024the}.
Model evaluation must move beyond chat-based interactions and into specialized user environments.



 

In this work, we focus on evaluating LLM-based coding assistants. 
Despite the popularity of these tools---millions of developers use Github Copilot~\citep{Copilot}---existing
evaluations of the coding capabilities of new models exhibit multiple limitations (Figure~\ref{fig:motivation}, bottom).
Traditional ML benchmarks evaluate LLM capabilities by measuring how well a model can complete static, interview-style coding tasks~\citep{chen2021evaluating,austin2021program,jain2024livecodebench, white2024livebench} and lack \emph{real users}. 
User studies recruit real users to evaluate the effectiveness of LLMs as coding assistants, but are often limited to simple programming tasks as opposed to \emph{real tasks}~\citep{vaithilingam2022expectation,ross2023programmer, mozannar2024realhumaneval}.
Recent efforts to collect human feedback such as Chatbot Arena~\citep{chiang2024chatbot} are still removed from a \emph{realistic environment}, resulting in users and data that deviate from typical software development processes.
We introduce \systemName to address these limitations (Figure~\ref{fig:motivation}, top), and we describe our three main contributions below.


\textbf{We deploy \systemName in-the-wild to collect human preferences on code.} 
\systemName is a Visual Studio Code extension, collecting preferences directly in a developer's IDE within their actual workflow (Figure~\ref{fig:overview}).
\systemName provides developers with code completions, akin to the type of support provided by Github Copilot~\citep{Copilot}. 
Over the past 3 months, \systemName has served over~\completions suggestions from 10 state-of-the-art LLMs, 
gathering \sampleCount~votes from \userCount~users.
To collect user preferences,
\systemName presents a novel interface that shows users paired code completions from two different LLMs, which are determined based on a sampling strategy that aims to 
mitigate latency while preserving coverage across model comparisons.
Additionally, we devise a prompting scheme that allows a diverse set of models to perform code completions with high fidelity.
See Section~\ref{sec:system} and Section~\ref{sec:deployment} for details about system design and deployment respectively.



\textbf{We construct a leaderboard of user preferences and find notable differences from existing static benchmarks and human preference leaderboards.}
In general, we observe that smaller models seem to overperform in static benchmarks compared to our leaderboard, while performance among larger models is mixed (Section~\ref{sec:leaderboard_calculation}).
We attribute these differences to the fact that \systemName is exposed to users and tasks that differ drastically from code evaluations in the past. 
Our data spans 103 programming languages and 24 natural languages as well as a variety of real-world applications and code structures, while static benchmarks tend to focus on a specific programming and natural language and task (e.g. coding competition problems).
Additionally, while all of \systemName interactions contain code contexts and the majority involve infilling tasks, a much smaller fraction of Chatbot Arena's coding tasks contain code context, with infilling tasks appearing even more rarely. 
We analyze our data in depth in Section~\ref{subsec:comparison}.



\textbf{We derive new insights into user preferences of code by analyzing \systemName's diverse and distinct data distribution.}
We compare user preferences across different stratifications of input data (e.g., common versus rare languages) and observe which affect observed preferences most (Section~\ref{sec:analysis}).
For example, while user preferences stay relatively consistent across various programming languages, they differ drastically between different task categories (e.g. frontend/backend versus algorithm design).
We also observe variations in user preference due to different features related to code structure 
(e.g., context length and completion patterns).
We open-source \systemName and release a curated subset of code contexts.
Altogether, our results highlight the necessity of model evaluation in realistic and domain-specific settings.





\putsec{related}{Related Work}

\noindent \textbf{Efficient Radiance Field Rendering.}
%
The introduction of Neural Radiance Fields (NeRF)~\cite{mil:sri20} has
generated significant interest in efficient 3D scene representation and
rendering for radiance fields.
%
Over the past years, there has been a large amount of research aimed at
accelerating NeRFs through algorithmic or software
optimizations~\cite{mul:eva22,fri:yu22,che:fun23,sun:sun22}, and the
development of hardware
accelerators~\cite{lee:cho23,li:li23,son:wen23,mub:kan23,fen:liu24}.
%
The state-of-the-art method, 3D Gaussian splatting~\cite{ker:kop23}, has
further fueled interest in accelerating radiance field
rendering~\cite{rad:ste24,lee:lee24,nie:stu24,lee:rho24,ham:mel24} as it
employs rasterization primitives that can be rendered much faster than NeRFs.
%
However, previous research focused on software graphics rendering on
programmable cores or building dedicated hardware accelerators. In contrast,
\name{} investigates the potential of efficient radiance field rendering while
utilizing fixed-function units in graphics hardware.
%
To our knowledge, this is the first work that assesses the performance
implications of rendering Gaussian-based radiance fields on the hardware
graphics pipeline with software and hardware optimizations.

%%%%%%%%%%%%%%%%%%%%%%%%%%%%%%%%%%%%%%%%%%%%%%%%%%%%%%%%%%%%%%%%%%%%%%%%%%
\myparagraph{Enhancing Graphics Rendering Hardware.}
%
The performance advantage of executing graphics rendering on either
programmable shader cores or fixed-function units varies depending on the
rendering methods and hardware designs.
%
Previous studies have explored the performance implication of graphics hardware
design by developing simulation infrastructures for graphics
workloads~\cite{bar:gon06,gub:aam19,tin:sax23,arn:par13}.
%
Additionally, several studies have aimed to improve the performance of
special-purpose hardware such as ray tracing units in graphics
hardware~\cite{cho:now23,liu:cha21} and proposed hardware accelerators for
graphics applications~\cite{lu:hua17,ram:gri09}.
%
In contrast to these works, which primarily evaluate traditional graphics
workloads, our work focuses on improving the performance of volume rendering
workloads, such as Gaussian splatting, which require blending a huge number of
fragments per pixel.

%%%%%%%%%%%%%%%%%%%%%%%%%%%%%%%%%%%%%%%%%%%%%%%%%%%%%%%%%%%%%%%%%%%%%%%%%%
%
In the context of multi-sample anti-aliasing, prior work proposed reducing the
amount of redundant shading by merging fragments from adjacent triangles in a
mesh at the quad granularity~\cite{fat:bou10}.
%
While both our work and quad-fragment merging (QFM)~\cite{fat:bou10} aim to
reduce operations by merging quads, our proposed technique differs from QFM in
many aspects.
%
Our method aims to blend \emph{overlapping primitives} along the depth
direction and applies to quads from any primitive. In contrast, QFM merges quad
fragments from small (e.g., pixel-sized) triangles that \emph{share} an edge
(i.e., \emph{connected}, \emph{non-overlapping} triangles).
%
As such, QFM is not applicable to the scenes consisting of a number of
unconnected transparent triangles, such as those in 3D Gaussian splatting.
%
In addition, our method computes the \emph{exact} color for each pixel by
offloading blending operations from ROPs to shader units, whereas QFM
\emph{approximates} pixel colors by using the color from one triangle when
multiple triangles are merged into a single quad.


\section{Preliminaries}
\label{sec:prelim}
\label{sec:term}
We define the key terminologies used, primarily focusing on the hidden states (or activations) during the forward pass. 

\paragraph{Components in an attention layer.} We denote $\Res$ as the residual stream. We denote $\Val$ as Value (states), $\Qry$ as Query (states), and $\Key$ as Key (states) in one attention head. The \attlogit~represents the value before the softmax operation and can be understood as the inner product between  $\Qry$  and  $\Key$. We use \Attn~to denote the attention weights of applying the SoftMax function to \attlogit, and ``attention map'' to describe the visualization of the heat map of the attention weights. When referring to the \attlogit~from ``$\tokenB$'' to  ``$\tokenA$'', we indicate the inner product  $\langle\Qry(\tokenB), \Key(\tokenA)\rangle$, specifically the entry in the ``$\tokenB$'' row and ``$\tokenA$'' column of the attention map.

\paragraph{Logit lens.} We use the method of ``Logit Lens'' to interpret the hidden states and value states \citep{belrose2023eliciting}. We use \logit~to denote pre-SoftMax values of the next-token prediction for LLMs. Denote \readout~as the linear operator after the last layer of transformers that maps the hidden states to the \logit. 
The logit lens is defined as applying the readout matrix to residual or value states in middle layers. Through the logit lens, the transformed hidden states can be interpreted as their direct effect on the logits for next-token prediction. 

\paragraph{Terminologies in two-hop reasoning.} We refer to an input like “\Src$\to$\brga, \brgb$\to$\Ed” as a two-hop reasoning chain, or simply a chain. The source entity $\Src$ serves as the starting point or origin of the reasoning. The end entity $\Ed$ represents the endpoint or destination of the reasoning chain. The bridge entity $\Brg$ connects the source and end entities within the reasoning chain. We distinguish between two occurrences of $\Brg$: the bridge in the first premise is called $\brga$, while the bridge in the second premise that connects to $\Ed$ is called $\brgc$. Additionally, for any premise ``$\tokenA \to \tokenB$'', we define $\tokenA$ as the parent node and $\tokenB$ as the child node. Furthermore, if at the end of the sequence, the query token is ``$\tokenA$'', we define the chain ``$\tokenA \to \tokenB$, $\tokenB \to \tokenC$'' as the Target Chain, while all other chains present in the context are referred to as distraction chains. Figure~\ref{fig:data_illustration} provides an illustration of the terminologies.

\paragraph{Input format.}
Motivated by two-hop reasoning in real contexts, we consider input in the format $\bos, \text{context information}, \query, \answer$. A transformer model is trained to predict the correct $\answer$ given the query $\query$ and the context information. The context compromises of $K=5$ disjoint two-hop chains, each appearing once and containing two premises. Within the same chain, the relative order of two premises is fixed so that \Src$\to$\brga~always precedes \brgb$\to$\Ed. The orders of chains are randomly generated, and chains may interleave with each other. The labels for the entities are re-shuffled for every sequence, choosing from a vocabulary size $V=30$. Given the $\bos$ token, $K=5$ two-hop chains, \query, and the \answer~tokens, the total context length is $N=23$. Figure~\ref{fig:data_illustration} also illustrates the data format. 

\paragraph{Model structure and training.} We pre-train a three-layer transformer with a single head per layer. Unless otherwise specified, the model is trained using Adam for $10,000$ steps, achieving near-optimal prediction accuracy. Details are relegated to Appendix~\ref{app:sec_add_training_detail}.


% \RZ{Do we use source entity, target entity, and mediator entity? Or do we use original token, bridge token, end token?}





% \paragraph{Basic notations.} We use ... We use $\ve_i$ to denote one-hot vectors of which only the $i$-th entry equals one, and all other entries are zero. The dimension of $\ve_i$ are usually omitted and can be inferred from contexts. We use $\indicator\{\cdot\}$ to denote the indicator function.

% Let $V > 0$ be a fixed positive integer, and let $\vocab = [V] \defeq \{1, 2, \ldots, V\}$ be the vocabulary. A token $v \in \vocab$ is an integer in $[V]$ and the input studied in this paper is a sequence of tokens $s_{1:T} \defeq (s_1, s_2, \ldots, s_T) \in \vocab^T$ of length $T$. For any set $\mathcal{S}$, we use $\Delta(\mathcal{S})$ to denote the set of distributions over $\mathcal{S}$.

% % to a sequence of vectors $z_1, z_2, \ldots, z_T \in \real^{\dout}$ of dimension $\dout$ and length $T$.

% Let $\mU = [\vu_1, \vu_2, \ldots, \vu_V]^\transpose \in \real^{V\times d}$ denote the token embedding matrix, where the $i$-th row $\vu_i \in \real^d$ represents the $d$-dimensional embedding of token $i \in [V]$. Similarly, let $\mP = [\vp_1, \vp_2, \ldots, \vp_T]^\transpose \in \real^{T\times d}$ denote the positional embedding matrix, where the $i$-th row $\vp_i \in \real^d$ represents the $d$-dimensional embedding of position $i \in [T]$. Both $\mU$ and $\mP$ can be fixed or learnable.

% After receiving an input sequence of tokens $s_{1:T}$, a transformer will first process it using embedding matrices $\mU$ and $\mP$ to obtain a sequence of vectors $\mH = [\vh_1, \vh_2, \ldots, \vh_T] \in \real^{d\times T}$, where 
% \[
% \vh_i = \mU^\transpose\ve_{s_i} + \mP^\transpose\ve_{i} = \vu_{s_i} + \vp_i.
% \]

% We make the following definitions of basic operations in a transformer.

% \begin{definition}[Basic operations in transformers] 
% \label{defn:operators}
% Define the softmax function $\softmax(\cdot): \real^d \to \real^d$ over a vector $\vv \in \real^d$ as
% \[\softmax(\vv)_i = \frac{\exp(\vv_i)}{\sum_{j=1}^d \exp(\vv_j)} \]
% and define the softmax function $\softmax(\cdot): \real^{m\times n} \to \real^{m \times n}$ over a matrix $\mV \in \real^{m\times n}$ as a column-wise softmax operator. For a squared matrix $\mM \in \real^{m\times m}$, the causal mask operator $\mask(\cdot): \real^{m\times m} \to \real^{m\times m}$  is defined as $\mask(\mM)_{ij} = \mM_{ij}$ if $i \leq j$ and  $\mask(\mM)_{ij} = -\infty$ otherwise. For a vector $\vv \in \real^n$ where $n$ is the number of hidden neurons in a layer, we use $\layernorm(\cdot): \real^n \to \real^n$ to denote the layer normalization operator where
% \[
% \layernorm(\vv)_i = \frac{\vv_i-\mu}{\sigma}, \mu = \frac{1}{n}\sum_{j=1}^n \vv_j, \sigma = \sqrt{\frac{1}{n}\sum_{j=1}^n (\vv_j-\mu)^2}
% \]
% and use $\layernorm(\cdot): \real^{n\times m} \to \real^{n\times m}$ to denote the column-wise layer normalization on a matrix.
% We also use $\nonlin(\cdot)$ to denote element-wise nonlinearity such as $\relu(\cdot)$.
% \end{definition}

% The main components of a transformer are causal self-attention heads and MLP layers, which are defined as follows.

% \begin{definition}[Attentions and MLPs]
% \label{defn:attn_mlp} 
% A single-head causal self-attention $\attn(\mH;\mQ,\mK,\mV,\mO)$ parameterized by $\mQ,\mK,\mV \in \real^{{\dqkv\times \din}}$ and $\mO \in \real^{\dout\times\dqkv}$ maps an input matrix $\mH \in \real^{\din\times T}$ to
% \begin{align*}
% &\attn(\mH;\mQ,\mK,\mV,\mO) \\
% =&\mO\mV\layernorm(\mH)\softmax(\mask(\layernorm(\mH)^\transpose\mK^\transpose\mQ\layernorm(\mH))).
% \end{align*}
% Furthermore, a multi-head attention with $M$ heads parameterized by $\{(\mQ_m,\mK_m,\mV_m,\mO_m) \}_{m=1}^M$ is defined as 
% \begin{align*}
%     &\Attn(\mH; \{(\mQ_m,\mK_m,\mV_m,\mO_m) \}_{m\in[M]}) \\ =& \sum_{m=1}^M \attn(\mH;\mQ_m,\mK_m,\mV_m,\mO_m) \in \real^{\dout \times T}.
% \end{align*}
% An MLP layer $\mlp(\mH;\mW_1,\mW_2)$ parameterized by $\mW_1 \in \real^{\dhidden\times \din}$ and $\mW_2 \in \real^{\dout \times \dhidden}$ maps an input matrix $\mH = [\vh_1, \ldots, \vh_T] \in \real^{\din \times T}$ to
% \begin{align*}
%     &\mlp(\mH;\mW_1,\mW_2) = [\vy_1, \ldots, \vy_T], \\ \text{where } &\vy_i = \mW_2\nonlin(\mW_1\layernorm(\vh_i)), \forall i \in [T].
% \end{align*}

% \end{definition}

% In this paper, we assume $\din=\dout=d$ for all attention heads and MLPs to facilitate residual stream unless otherwise specified. Given \Cref{defn:operators,defn:attn_mlp}, we are now able to define a multi-layer transformer.

% \begin{definition}[Multi-layer transformers]
% \label{defn:transformer}
%     An $L$-layer transformer $\transformer(\cdot): \vocab^T \to \Delta(\vocab)$ parameterized by $\mP$, $\mU$, $\{(\mQ_m^{(l)},\mK_m^{(l)},\mV_m^{(l)},\mO_m^{(l)})\}_{m\in[M],l\in[L]}$,  $\{(\mW_1^{(l)},\mW_2^{(l)})\}_{l\in[L]}$ and $\Wreadout \in \real^{V \times d}$ receives a sequence of tokens $s_{1:T}$ as input and predict the next token by outputting a distribution over the vocabulary. The input is first mapped to embeddings $\mH = [\vh_1, \vh_2, \ldots, \vh_T] \in \real^{d\times T}$ by embedding matrices $\mP, \mU$ where 
%     \[
%     \vh_i = \mU^\transpose\ve_{s_i} + \mP^\transpose\ve_{i}, \forall i \in [T].
%     \]
%     For each layer $l \in [L]$, the output of layer $l$, $\mH^{(l)} \in \real^{d\times T}$, is obtained by 
%     \begin{align*}
%         &\mH^{(l)} =  \mH^{(l-1/2)} + \mlp(\mH^{(l-1/2)};\mW_1^{(l)},\mW_2^{(l)}), \\
%         & \mH^{(l-1/2)} = \mH^{(l-1)} + \\ & \quad \Attn(\mH^{(l-1)}; \{(\mQ_m^{(l)},\mK_m^{(l)},\mV_m^{(l)},\mO_m^{(l)}) \}_{m\in[M]}), 
%     \end{align*}
%     where the input $\mH^{(l-1)}$ is the output of the previous layer $l-1$ for $l > 1$ and the input of the first layer $\mH^{(0)} = \mH$. Finally, the output of the transformer is obtained by 
%     \begin{align*}
%         \transformer(s_{1:T}) = \softmax(\Wreadout\vh_T^{(L)})
%     \end{align*}
%     which is a $V$-dimensional vector after softmax representing a distribution over $\vocab$, and $\vh_T^{(L)}$ is the $T$-th column of the output of the last layer, $\mH^{(L)}$.
% \end{definition}



% For each token $v \in \vocab$, there is a corresponding $d_t$-dimensional token embedding vector $\embed(v) \in \mathbb{R}^{d_t}$. Assume the maximum length of the sequence studied in this paper does not exceed $T$. For each position $t \in [T]$, there is a corresponding positional embedding  







\begin{algorithm}[h!]
\caption{Gait-Net-augmented Sequential CMPC}
\label{alg:gaitMPC}
\begin{algorithmic}[1]
\Require $\mathbf q, \: \dot{\mathbf q}, \: \mathbf q^\text{cmd}, \: \dot{\mathbf q}^\text{cmd}$
\State \textbf{intialize} $\bm x_0 = f_\text{j2m}(\mathbf q, \: \dot{\mathbf q}), \: \bm u^0 =\bm u_\text{IG}, \: dt^0 = 0.05$ 
\State $\{ \mathbf q^\text{ref},\:\dot{\mathbf q}^\text{ref},\:\bm p_f^\text{ref}\} = f_\text{ref} \big(\mathbf q, \: \dot{\mathbf q}, \: \mathbf q^\text{cmd}, \: \dot{\mathbf q}^\text{cmd} \big)$
\State $\bm x^\text{ref} = f_\text{j2m}(\mathbf q^\text{ref},\:\dot{\mathbf q}^\text{ref},\:\bm p_f^\text{ref})$
\State $ j = 0$ 
\While{$j \leq j_\text{max} \:\text{and}\: \bm \eta \leq \delta \bm u  $} 
\State $\delta \bm u^{j} = \texttt{cmpc}(\bm x^\text{ref},\:\bm p_f^\text{ref},\:\bm p_c^\text{ref},\: \bm x_0,\: dt^j, \: \bm u^j)$
\State $\bm u^{j+1} = \bm u^j + \delta \bm u^j$ 
\State $dt^{j+1} = \Pi_\text{GN}(\mathbf q, \: \dot{\mathbf q},\: \bm p_f^{j})$
\State $\{ \bm x^\text{ref},\:\bm p_f^\text{ref}\}= f_\text{IK}(\bm p_f^{j},\:\bm p_c^{j},\: dt^{j+1})$
\State $j=j+1$
\EndWhile \\
\Return $\bm u^{j+1} $
\end{algorithmic}
\end{algorithm}
\section{Discussion of Assumptions}\label{sec:discussion}
In this paper, we have made several assumptions for the sake of clarity and simplicity. In this section, we discuss the rationale behind these assumptions, the extent to which these assumptions hold in practice, and the consequences for our protocol when these assumptions hold.

\subsection{Assumptions on the Demand}

There are two simplifying assumptions we make about the demand. First, we assume the demand at any time is relatively small compared to the channel capacities. Second, we take the demand to be constant over time. We elaborate upon both these points below.

\paragraph{Small demands} The assumption that demands are small relative to channel capacities is made precise in \eqref{eq:large_capacity_assumption}. This assumption simplifies two major aspects of our protocol. First, it largely removes congestion from consideration. In \eqref{eq:primal_problem}, there is no constraint ensuring that total flow in both directions stays below capacity--this is always met. Consequently, there is no Lagrange multiplier for congestion and no congestion pricing; only imbalance penalties apply. In contrast, protocols in \cite{sivaraman2020high, varma2021throughput, wang2024fence} include congestion fees due to explicit congestion constraints. Second, the bound \eqref{eq:large_capacity_assumption} ensures that as long as channels remain balanced, the network can always meet demand, no matter how the demand is routed. Since channels can rebalance when necessary, they never drop transactions. This allows prices and flows to adjust as per the equations in \eqref{eq:algorithm}, which makes it easier to prove the protocol's convergence guarantees. This also preserves the key property that a channel's price remains proportional to net money flow through it.

In practice, payment channel networks are used most often for micro-payments, for which on-chain transactions are prohibitively expensive; large transactions typically take place directly on the blockchain. For example, according to \cite{river2023lightning}, the average channel capacity is roughly $0.1$ BTC ($5,000$ BTC distributed over $50,000$ channels), while the average transaction amount is less than $0.0004$ BTC ($44.7k$ satoshis). Thus, the small demand assumption is not too unrealistic. Additionally, the occasional large transaction can be treated as a sequence of smaller transactions by breaking it into packets and executing each packet serially (as done by \cite{sivaraman2020high}).
Lastly, a good path discovery process that favors large capacity channels over small capacity ones can help ensure that the bound in \eqref{eq:large_capacity_assumption} holds.

\paragraph{Constant demands} 
In this work, we assume that any transacting pair of nodes have a steady transaction demand between them (see Section \ref{sec:transaction_requests}). Making this assumption is necessary to obtain the kind of guarantees that we have presented in this paper. Unless the demand is steady, it is unreasonable to expect that the flows converge to a steady value. Weaker assumptions on the demand lead to weaker guarantees. For example, with the more general setting of stochastic, but i.i.d. demand between any two nodes, \cite{varma2021throughput} shows that the channel queue lengths are bounded in expectation. If the demand can be arbitrary, then it is very hard to get any meaningful performance guarantees; \cite{wang2024fence} shows that even for a single bidirectional channel, the competitive ratio is infinite. Indeed, because a PCN is a decentralized system and decisions must be made based on local information alone, it is difficult for the network to find the optimal detailed balance flow at every time step with a time-varying demand.  With a steady demand, the network can discover the optimal flows in a reasonably short time, as our work shows.

We view the constant demand assumption as an approximation for a more general demand process that could be piece-wise constant, stochastic, or both (see simulations in Figure \ref{fig:five_nodes_variable_demand}).
We believe it should be possible to merge ideas from our work and \cite{varma2021throughput} to provide guarantees in a setting with random demands with arbitrary means. We leave this for future work. In addition, our work suggests that a reasonable method of handling stochastic demands is to queue the transaction requests \textit{at the source node} itself. This queuing action should be viewed in conjunction with flow-control. Indeed, a temporarily high unidirectional demand would raise prices for the sender, incentivizing the sender to stop sending the transactions. If the sender queues the transactions, they can send them later when prices drop. This form of queuing does not require any overhaul of the basic PCN infrastructure and is therefore simpler to implement than per-channel queues as suggested by \cite{sivaraman2020high} and \cite{varma2021throughput}.

\subsection{The Incentive of Channels}
The actions of the channels as prescribed by the DEBT control protocol can be summarized as follows. Channels adjust their prices in proportion to the net flow through them. They rebalance themselves whenever necessary and execute any transaction request that has been made of them. We discuss both these aspects below.

\paragraph{On Prices}
In this work, the exclusive role of channel prices is to ensure that the flows through each channel remains balanced. In practice, it would be important to include other components in a channel's price/fee as well: a congestion price  and an incentive price. The congestion price, as suggested by \cite{varma2021throughput}, would depend on the total flow of transactions through the channel, and would incentivize nodes to balance the load over different paths. The incentive price, which is commonly used in practice \cite{river2023lightning}, is necessary to provide channels with an incentive to serve as an intermediary for different channels. In practice, we expect both these components to be smaller than the imbalance price. Consequently, we expect the behavior of our protocol to be similar to our theoretical results even with these additional prices.

A key aspect of our protocol is that channel fees are allowed to be negative. Although the original Lightning network whitepaper \cite{poon2016bitcoin} suggests that negative channel prices may be a good solution to promote rebalancing, the idea of negative prices in not very popular in the literature. To our knowledge, the only prior work with this feature is \cite{varma2021throughput}. Indeed, in papers such as \cite{van2021merchant} and \cite{wang2024fence}, the price function is explicitly modified such that the channel price is never negative. The results of our paper show the benefits of negative prices. For one, in steady state, equal flows in both directions ensure that a channel doesn't loose any money (the other price components mentioned above ensure that the channel will only gain money). More importantly, negative prices are important to ensure that the protocol selectively stifles acyclic flows while allowing circulations to flow. Indeed, in the example of Section \ref{sec:flow_control_example}, the flows between nodes $A$ and $C$ are left on only because the large positive price over one channel is canceled by the corresponding negative price over the other channel, leading to a net zero price.

Lastly, observe that in the DEBT control protocol, the price charged by a channel does not depend on its capacity. This is a natural consequence of the price being the Lagrange multiplier for the net-zero flow constraint, which also does not depend on the channel capacity. In contrast, in many other works, the imbalance price is normalized by the channel capacity \cite{ren2018optimal, lin2020funds, wang2024fence}; this is shown to work well in practice. The rationale for such a price structure is explained well in \cite{wang2024fence}, where this fee is derived with the aim of always maintaining some balance (liquidity) at each end of every channel. This is a reasonable aim if a channel is to never rebalance itself; the experiments of the aforementioned papers are conducted in such a regime. In this work, however, we allow the channels to rebalance themselves a few times in order to settle on a detailed balance flow. This is because our focus is on the long-term steady state performance of the protocol. This difference in perspective also shows up in how the price depends on the channel imbalance. \cite{lin2020funds} and \cite{wang2024fence} advocate for strictly convex prices whereas this work and \cite{varma2021throughput} propose linear prices.

\paragraph{On Rebalancing} 
Recall that the DEBT control protocol ensures that the flows in the network converge to a detailed balance flow, which can be sustained perpetually without any rebalancing. However, during the transient phase (before convergence), channels may have to perform on-chain rebalancing a few times. Since rebalancing is an expensive operation, it is worthwhile discussing methods by which channels can reduce the extent of rebalancing. One option for the channels to reduce the extent of rebalancing is to increase their capacity; however, this comes at the cost of locking in more capital. Each channel can decide for itself the optimum amount of capital to lock in. Another option, which we discuss in Section \ref{sec:five_node}, is for channels to increase the rate $\gamma$ at which they adjust prices. 

Ultimately, whether or not it is beneficial for a channel to rebalance depends on the time-horizon under consideration. Our protocol is based on the assumption that the demand remains steady for a long period of time. If this is indeed the case, it would be worthwhile for a channel to rebalance itself as it can make up this cost through the incentive fees gained from the flow of transactions through it in steady state. If a channel chooses not to rebalance itself, however, there is a risk of being trapped in a deadlock, which is suboptimal for not only the nodes but also the channel.

\section{Conclusion}
This work presents DEBT control: a protocol for payment channel networks that uses source routing and flow control based on channel prices. The protocol is derived by posing a network utility maximization problem and analyzing its dual minimization. It is shown that under steady demands, the protocol guides the network to an optimal, sustainable point. Simulations show its robustness to demand variations. The work demonstrates that simple protocols with strong theoretical guarantees are possible for PCNs and we hope it inspires further theoretical research in this direction.

%%%%%%%%%%%%%%%%%%%%%%%%%%%%%%%%%%%%%%%%%%%%%%%%%%%%%%%%%%%%%%%%%%%%%%%%%%%%%%%%%%%%%%%%%%%%%%%%%%%%%%

%%%%%%%%%%%%%%%%%%%%%%%%%%%%%%%%%%%%%%%%%%%%%%
\begin{table*}[t]
\setlength{\tabcolsep}{3pt}
\centering
\renewcommand{\arraystretch}{1.1}
\tabcolsep=0.2cm
\begin{adjustbox}{max width=\textwidth}  % Set the maximum width to text width
\begin{tabular}{c| cccc ||  c| cc cc}
\toprule
General & \multicolumn{3}{c}{Preference} & Accuracy & Supervised & \multicolumn{3}{c}{Preference} & Accuracy \\ 
LLMs & PrefHit & PrefRecall & Reward & BLEU & Alignment & PrefHit & PrefRecall & Reward & BLEU \\ 
\midrule
GPT-J & 0.2572 & 0.6268 & 0.2410 & 0.0923 & Llama2-7B & 0.2029 & 0.803 & 0.0933 & 0.0947 \\
Pythia-2.8B & 0.3370 & 0.6449 & 0.1716 & 0.1355 & SFT & 0.2428 & 0.8125 & 0.1738 & 0.1364 \\
Qwen2-7B & 0.2790 & 0.8179 & 0.1593 & 0.2530 & Slic & 0.2464 & 0.6171 & 0.1700 & 0.1400 \\
Qwen2-57B & 0.3086 & 0.6481 & 0.6854 & 0.2568 & RRHF & 0.3297 & 0.8234 & 0.2263 & 0.1504 \\
Qwen2-72B & 0.3212 & 0.5555 & 0.6901 & 0.2286 & DPO-BT & 0.2500 & 0.8125 & 0.1728 & 0.1363 \\ 
StarCoder2-15B & 0.2464 & 0.6292 & 0.2962 & 0.1159 & DPO-PT & 0.2572 & 0.8067 & 0.1700 & 0.1348 \\
ChatGLM4-9B & 0.2246 & 0.6099 & 0.1686 & 0.1529 & PRO & 0.3025 & 0.6605 & 0.1802 & 0.1197 \\ 
Llama3-8B & 0.2826 & 0.6425 & 0.2458 & 0.1723 & \textbf{\shortname}* & \textbf{0.3659} & \textbf{0.8279} & \textbf{0.2301} & \textbf{0.1412} \\ 
\bottomrule
\end{tabular}
\end{adjustbox}
\caption{Main results on the StaCoCoQA. The left shows the performance of general LLMs, while the right presents the performance of the fine-tuned LLaMA2-7B across various strong benchmarks for preference alignment. Our method SeAdpra is highlighted in \textbf{bold}.}
\label{main}
\vspace{-0.2cm}
\end{table*}
%%%%%%%%%%%%%%%%%%%%%%%%%%%%%%%%%%%%%%%%%%%%%%%%%%%%%%%%%%%%%%%%%%%%%%%%%%%%%%%%%%%%%%%%%%%%%%%%%%%%
\begin{table}[h]
\centering
\renewcommand{\arraystretch}{1.02}
% \tabcolsep=0.1cm
\begin{adjustbox}{width=0.48\textwidth} % Adjust table width
\begin{tabularx}{0.495\textwidth}{p{1.2cm} p{0.7cm} p{0.95cm}p{0.95cm}p{0.7cm}p{0.7cm}}
     \toprule
    \multirow{2}{*}{\small \textbf{Dataset}} & \multirow{2}{*}{\small Model} & \multicolumn{2}{c}{\small Preference} & \multicolumn{2}{c}{\small Acc } \\ 
    & & \small \textit{PrefHit} & \small \textit{PrefRec} & \small \textit{Reward} & \small \textit{Rouge} \\ 
    \midrule
    \multirow{2}{*}{\small \textbf{Academia}}   & \small PRO & 33.78 & 59.56 & 69.94 & 9.84 \\ 
                                & \small \textbf{Ours} & 36.44 & 60.89 & 70.17 & 10.69 \\ 
    \midrule
    \multirow{2}{*}{\small \textbf{Chemistry}}  & \small PRO & 36.31 & 63.39 & 69.15 & 11.16 \\ 
                                & \small \textbf{Ours} & 38.69 & 64.68 & 69.31 & 12.27 \\ 
    \midrule
    \multirow{2}{*}{\small \textbf{Cooking}}    & \small PRO & 35.29 & 58.32 & 69.87 & 12.13 \\ 
                                & \small \textbf{Ours} & 38.50 & 60.01 & 69.93 & 13.73 \\ 
    \midrule
    \multirow{2}{*}{\small \textbf{Math}}       & \small PRO & 30.00 & 56.50 & 69.06 & 13.50 \\ 
                                & \small \textbf{Ours} & 32.00 & 58.54 & 69.21 & 14.45 \\ 
    \midrule
    \multirow{2}{*}{\small \textbf{Music}}      & \small PRO & 34.33 & 60.22 & 70.29 & 13.05 \\ 
                                & \small \textbf{Ours} & 37.00 & 60.61 & 70.84 & 13.82 \\ 
    \midrule
    \multirow{2}{*}{\small \textbf{Politics}}   & \small PRO & 41.77 & 66.10 & 69.52 & 9.31 \\ 
                                & \small \textbf{Ours} & 42.19 & 66.03 & 69.74 & 9.38 \\ 
    \midrule
    \multirow{2}{*}{\small \textbf{Code}} & \small PRO & 26.00 & 51.13 & 69.17 & 12.44 \\ 
                                & \small \textbf{Ours} & 27.00 & 51.77 & 69.46 & 13.33 \\ 
    \midrule
    \multirow{2}{*}{\small \textbf{Security}}   & \small PRO & 23.62 & 49.23 & 70.13 & 10.63 \\ 
                                & \small \textbf{Ours} & 25.20 & 49.24 & 70.92 & 10.98 \\ 
    \midrule
    \multirow{2}{*}{\small \textbf{Mean}}       & \small PRO & 32.64 & 58.05 & 69.64 & 11.51 \\ 
                                & \small \textbf{Ours} & \textbf{34.25} & \textbf{58.98} & \textbf{69.88} & \textbf{12.33} \\ 
    \bottomrule
\end{tabularx}
\end{adjustbox}
\caption{Main results (\%) on eight publicly available and popular CoQA datasets, comparing the strong list-wise benchmark PRO and \textbf{ours with bold}.}
\label{public}
\end{table}



%%%%%%%%%%%%%%%%%%%%%%%%%%%%%%%%%%%%%%%%%%%%%%%%%%%%%
\begin{table}[h]
\centering
\renewcommand{\arraystretch}{1.02}
\begin{tabularx}{0.48\textwidth}{p{1.45cm} p{0.56cm} p{0.6cm} p{0.6cm} p{0.50cm} p{0.45cm} X}
\toprule
\multirow{2}{*}{Method} & \multicolumn{3}{c}{Preference \((\uparrow)\)} & \multicolumn{3}{c}{Accuracy \((\uparrow)\)} \\ \cmidrule{2-4} \cmidrule{5-7}
& \small PrefHit & \small PrefRec & \small Reward & \small CoSim & \small BLEU & \small Rouge \\ \midrule
\small{SeAdpra} & \textbf{34.8} & \textbf{82.5} & \textbf{22.3} & \textbf{69.1} & \textbf{17.4} & \textbf{21.8} \\ 
\small{-w/o PerAl} & \underline{30.4} & 83.0 & 18.7 & 68.8 & \underline{12.6} & 21.0 \\
\small{-w/o PerCo} & 32.6 & 82.3 & \underline{24.2} & 69.3 & 16.4 & 21.0 \\
\small{-w/o \(\Delta_{Se}\)} & 31.2 & 82.8 & 18.6 & 68.3 & \underline{12.4} & 20.9 \\
\small{-w/o \(\Delta_{Po}\)} & \underline{29.4} & 82.2 & 22.1 & 69.0 & 16.6 & 21.4 \\
\small{\(PerCo_{Se}\)} & 30.9 & 83.5 & 15.6 & 67.6 & \underline{9.9} & 19.6 \\
\small{\(PerCo_{Po}\)} & \underline{30.3} & 82.7 & 20.5 & 68.9 & 14.4 & 20.1 \\ 
\bottomrule
\end{tabularx}
\caption{Ablation Results (\%). \(PerCo_{Se}\) or \(PerCo_{Po}\) only employs Single-APDF in Perceptual Comparison, replacing \(\Delta_{M}\) with \(\Delta_{Se}\) or \(\Delta_{Po}\). The bold represents the overall effect. The underlining highlights the most significant metric for each component's impact.}
\label{ablation}
% \vspace{-0.2cm}
\end{table}

\subsection{Dataset}

% These CoQA datasets contain questions and answers from the Stack Overflow data dump\footnote{https://archive.org/details/stackexchange}, intended for training preference models. 

Due to the additional challenges that programming QA presents for LLMs and the lack of high-quality, authentic multi-answer code preference datasets, we turned to StackExchange \footnote{https://archive.org/details/stackexchange}, a platform with forums that are accompanied by rich question-answering metadata. Based on this, we constructed a large-scale programming QA dataset in real-time (as of May 2024), called StaCoCoQA. It contains over 60,738 programming directories, as shown in Table~\ref{tab:stacocoqa_tags}, and 9,978,474 entries, with partial data statistics displayed in Figure~\ref{fig:dataset}. The data format of StaCoCoQA is presented in Table~\ref{fig::stacocoqa}.

The initial dataset \(D_I\) contains 24,101,803 entries, and is processed by the following steps:
(1) Select entries with "Questioner-picked answer" pairs to represent the preferences of the questioners, resulting in 12,260,106 entries in the \(D_Q\).
(2) Select data where the question includes at least one code block to focus on specific-domain programming QA, resulting in 9,978,474 entries in the dataset \(D_C\).
(3) All HTML tags were cleaned using BeautifulSoup \footnote{https://beautiful-soup-4.readthedocs.io/en/latest/} to ensure that the model is not affected by overly complex and meaningless content.
(4) Control the quality of the dataset by considering factors such as the time the question was posted, the size of the response pool, the difference between the highest and lowest votes within a pool, the votes for each response, the token-level length of the question and the answers, which yields varying sizes: 3K, 8K, 18K, 29K, and 64K. 
The controlled creation time variable and the data details after each processing step are shown in Table~\ref{tab:statistics}.

To further validate the effectiveness of SeAdpra, we also select eight popular topic CoQA datasets\footnote{https://huggingface.co/datasets/HuggingFaceH4/stack-exchange-preferences}, which have been filtered to meet specific criteria for preference models \cite{askell2021general}. Their detailed data information is provided in Table~\ref{domain}.
% Examples of some control variables are shown in Table~\ref{tab:statistics}.
% \noindent\textbf{Baselines}. 
% Following the DPO \cite{rafailov2024direct}, we evaluated several existing approaches aligned with human preference, including GPT-J \cite{gpt-j} and Pythia-2.8B \cite{biderman2023pythia}.  
% Next, we assessed StarCoder2 \cite{lozhkov2024starcoder}, which has demonstrated strong performance in code generation, alongside several general-purpose LLMs: Qwen2 \cite{qwen2}, ChatGLM4 \cite{wang2023cogvlm, glm2024chatglm} and LLaMA serials \cite{touvron2023llama,llama3modelcard}.
% Finally, we fine-tuned LLaMA2-7B on the StaCoCoQA and compared its performance with other strong baselines for supervised learning in preference alignment, including SFT, RRHF \cite{yuan2024rrhf}, Silc \cite{zhao2023slic}, DPO, and PRO \cite{song2024preference}.
%%%%%%%%%%%%%%%%%%%%%%%%%%%%%%%%%%%%%%%%%%%%%%%%%%%%%%%%%%%%%%%%%%%%%%%%%%%%%%%%%%%%%%%%%%%%%%%%%%%%%%%%%%%%%%%%%%%%%%%%%%%%%%%%%%

% For preference evaluation, traditional win-rate assessments are costly and not scalable. For instance, when an existing model \(M_A\) is evaluated against comparison methods \((M_B, M_C, M_D)\) in terms of win rates, upgrading model \(M_A\) would necessitate a reevaluation of its win rates against other models. Furthermore, if a new comparison method \(M_E\) is introduced, the win rates of model \(M_A\) against \(M_E\) would also need to be reassessed. Whether AI or humans are employed as evaluation mediators, binary preference between preferred and non-preferred choices or to score the inference results of the modified model, the costs of this process are substantial. 
% Therefore, from an economic perspective, we propose a novel list preference evaluation method. We utilize manually ranking results as the gold standard for assessing human preferences, to calculate the Hit and Recall, referred to as PrefHit and PrefRecall, respectively. Regardless of whether improving model \(M_A\) or expanding comparison method \(M_E\), only the calculation of PrefHit and PrefRecall for the modified model is required, eliminating the need for human evaluation. 
% We also employ a professional reward model\footnote{https://huggingface.co/OpenAssistant/reward-model-deberta-v3-large}
% for evaluation, denoted as the Reward metric.

% \subsection{Baseline} 
% Following the DPO \cite{rafailov2024direct}, we evaluated several existing approaches aligned with human preference, including GPT-J \cite{gpt-j} and Pythia-2.8B \cite{biderman2023pythia}.  
% Next, we assessed StarCoder2 \cite{lozhkov2024starcoder}, which has demonstrated strong performance in code generation, alongside several general-purpose LLMs: Qwen2 \cite{qwen2}, ChatGLM4 \cite{wang2023cogvlm, glm2024chatglm} and LLaMA serials \cite{touvron2023llama,llama3modelcard}.
% Finally, we fine-tuned LLaMA2-7B on the StaCoCoQA and compared its performance with other strong baselines for supervised learning in preference alignment, including SFT, RRHF \cite{yuan2024rrhf}, Silc \cite{zhao2023slic}, DPO, and PRO \cite{song2024preference}.
\subsection{Evaluation Metrics}
\label{sec: metric}
For preference evaluation, we design PrefHit and PrefRecall, adhering to the "CSTC" criterion outlined in Appendix \ref{sec::cstc}, which overcome the limitations of existing evaluation methods, as detailed in Appendix \ref{metric::mot}.
In addition, we demonstrate the effectiveness of thees new evaluation from two main aspects: 1) consistency with traditional metrics, and 2) applicability in different application scenarios in Appendix \ref{metric::ana}.
Following the previous \cite{song2024preference}, we also employ a professional reward.
% Following the previous \cite{song2024preference}, we also employ a professional reward model\footnote{https://huggingface.co/OpenAssistant/reward-model-deberta-v3-large} \cite{song2024preference}, denoted as the Reward.

For accuracy evaluation, we alternately employ BLEU \cite{papineni2002bleu}, RougeL \cite{lin2004rouge}, and CoSim. Similar to codebertscore \cite{zhou2023codebertscore}, CoSim not only focuses on the semantics of the code but also considers structural matching.
Additionally, the implementation details of SeAdpra are described in detail in the Appendix \ref{sec::imp}.
\subsection{Main Results}
We compared the performance of \shortname with general LLMs and strong preference alignment benchmarks on the StaCoCoQA dataset, as shown in Table~\ref{main}. Additionally, we compared SeAdpra with the strongly supervised alignment model PRO \cite{song2024preference} on eight publicly available CoQA datasets, as presented in Table~\ref{public} and Figure~\ref{fig::public}.

\textbf{Larger Model Parameters, Higher Preference.}
Firstly, the Qwen2 series has adopted DPO \cite{rafailov2024direct} in post-training, resulting in a significant enhancement in Reward.
In a horizontal comparison, the performance of Qwen2-7B and LLaMA2-7B in terms of PrefHit is comparable.
Gradually increasing the parameter size of Qwen2 \cite{qwen2} and LLaMA leads to higher PrefHit and Reward.
Additionally, general LLMs continue to demonstrate strong capabilities of programming understanding and generation preference datasets, contributing to high BLEU scores.
These findings indicate that increasing parameter size can significantly improve alignment.

\textbf{List-wise Ranking Outperforms Pair-wise Comparison.}
Intuitively, list-wise DPO-PT surpasses pair-wise DPO-{BT} on PrefHit. Other list-wise methods, such as RRHF, PRO, and our \shortname, also undoubtedly surpass the pair-wise Slic.

\textbf{Both Parameter Size and Alignment Strategies are Effective.}
Compared to other models, Pythia-2.8B achieved impressive results with significantly fewer parameters .
Effective alignment strategies can balance the performance differences brought by parameter size. For example, LLaMA2-7B with PRO achieves results close to Qwen2-57B in PrefHit. Moreover, LLaMA2-7B combined with our method SeAdpra has already far exceeded the PrefHit of Qwen2-57B.

\textbf{Rather not Higher Reward, Higher PrefHit.}
It is evident that Reward and PrefHit are not always positively correlated, indicating that models do not always accurately learn human preferences and cannot fully replace real human evaluation. Therefore, relying solely on a single public reward model is not sufficiently comprehensive when assessing preference alignment.

% In conclusion, during ensuring precise alignment, SeAdpra will focuse on PrefHit@1, even though the trade-off between PrefHit and PrefRecall is a common issue and increasing recall may sometimes lead to a decrease in hit rate. The positive correlation between Reward and BLEU, indicates that improving the quality of the generated text typically enhances the Reward. 
% Most importantly, evaluating preferences solely based on reward is clearly insufficient, as a high reward does not necessarily correspond to a high PrefHit or PrefRecall.
%%%%%%%%%%%%%%%%%%%%%%%%%%%%%%%%%%%%%%%%%%%
%%%%%%%%%%%%
\begin{figure}
  \centering
  \begin{subfigure}{0.49\linewidth}
    \includegraphics[width=\linewidth]{latex/pic/hit.png}
    \caption{The PrefHit}
    \label{scale:hit}
  \end{subfigure}
  \begin{subfigure}{0.49\linewidth}
    \includegraphics[width=\linewidth]{latex/pic/Recall.png}
    \caption{The PrefRecall}
    \label{scale:recall}
  \end{subfigure}
  \medskip
  \begin{subfigure}{0.48\linewidth}
    \includegraphics[width=\linewidth]{latex/pic/reward.png}
    \caption{The Reward}
    \label{scale:reward}
  \end{subfigure}
  \begin{subfigure}{0.48\linewidth}
    \includegraphics[width=\linewidth]{latex/pic/bleu.png}
    \caption{The BLEU}
    \label{scale:bleu}
  \end{subfigure}
  \caption{The performance with Confidence Interval (CI) of our SeAdpra and PRO at different data scales.}
  \label{fig:scale}
  % \vspace{-0.2cm}
\end{figure}
%%%%%%%%%%%%%%%%%%%%%%%%%%%%%%%%%%%%%%%%%%%%%%%%%%%%%%%%%%%%%%%%%%%%%%%%%%%%%%%%%%%%%%%%%%%%%%%%%%%%%%%%%%%%%%%%

\subsection{Ablation Study}

In this section, we discuss the effectiveness of each component of SeAdpra and its impact on various metrics. The results are presented in Table \ref{ablation}.

\textbf{Perceptual Comparison} aims to prevent the model from relying solely on linguistic probability ordering while neglecting the significance of APDF. Removing this Reward will significantly increase the margin, but PrefHit will decrease, which may hinder the model's ability to compare and learn the preference differences between responses.

\textbf{Perceptual Alignment} seeks to align with the optimal responses; removing it will lead to a significant decrease in PrefHit, while the Reward and accuracy metrics like CoSim will significantly increase, as it tends to favor preference over accuracy.

\textbf{Semantic Perceptual Distance} plays a crucial role in maintaining semantic accuracy in alignment learning. Removing it leads to a significant decrease in BLEU and Rouge. Since sacrificing accuracy recalls more possibilities, PrefHit decreases while PrefRecall increases. Moreover, eliminating both Semantic Perceptual Distance and Perceptual Alignment in \(PerCo_{Po}\) further increases PrefRecall, while the other metrics decline again, consistent with previous observations.


\textbf{Popularity Perceptual Distance} is most closely associated with PrefHit. Eliminating it causes PrefHit to drop to its lowest value, indicating that the popularity attribute is an extremely important factor in code communities.

% In summary, each module has a varying impact on preference and accuracy, but all outperform their respective foundation models and other baselines, as shown in Table \ref{main}, proving their effectiveness.


\subsection{Analysis and Discussion}

\textbf{SeAdpra adept at high-quality data rather than large-scale data.}
In StaCoCoQA, we tested PRO and SeAdpra across different data scales, and the results are shown in Figure~\ref{fig:scale}.
Since we rely on the popularity and clarity of questions and answers to filter data, a larger data scale often results in more pronounced deterioration in data quality. In Figure~\ref{scale:hit}, SeAdpra is highly sensitive to data quality in PrefHit, whereas PRO demonstrates improved performance with larger-scale data. Their performance on Prefrecall is consistent. In the native reward model of PRO, as depicted in Figure~\ref{scale:reward}, the reward fluctuations are minimal, while SeAdpra shows remarkable improvement.

\textbf{SeAdpra is relatively insensitive to ranking length.} 
We assessed SeAdpra's performance on different ranking lengths, as shown in Figure 6a. Unlike PRO, which varied with increasing ranking length, SeAdpra shows no significant differences across different lengths. There is a slight increase in performance on PrefHit and PrefRecall. Additionally, SeAdpra performs better at odd lengths compared to even lengths, which is an interesting phenomenon warranting further investigation.


\textbf{Balance Preference and Accuracy.} 
We analyzed the effect of control weights for Perceptual Comparisons in the optimization objective on preference and accuracy, with the findings presented in Figure~\ref{para:weight}.
When \( \alpha \) is greater than 0.05, the trends in PrefHit and BLEU are consistent, indicating that preference and accuracy can be optimized in tandem. However, when \( \alpha \) is 0.01, PrefHit is highest, but BLEU drops sharply.
Additionally, as \( \alpha \) changes, the variations in PrefHit and Reward, which are related to preference, are consistent with each other, reflecting their unified relationship in the optimization. Similarly, the variations in Recall and BLEU, which are related to accuracy, are also consistent, indicating a strong correlation between generation quality and comprehensiveness. 

%%%%%%%%%%%%%%%%%%%%%%%%%%%%%%%%%%%%%%%%%%%%%%%%%%%%%%%%%%%%%%%%%%%%%%%%%%%%%%%%%
\begin{figure}
  \centering
  \begin{subfigure}{0.475\linewidth}
    \includegraphics[width=\linewidth]{latex/pic/Rank1.png}
    \caption{Ranking length}
    \label{para:rank}
  \end{subfigure}
  \begin{subfigure}{0.475\linewidth}
    \includegraphics[width=\linewidth]{latex/pic/weights1.png}
    \caption{The \(\alpha\) in \(Loss\)}
    \label{para:weight}
  \end{subfigure}
  \caption{Parameters Analysis. Results of experiments on different ranking lengths and the weight \(\alpha\) in \(Loss\).}
  \label{fig:para}
  % \vspace{-0.2cm}
\end{figure}
%%%%%%%%%%%%%%%%%%%%%%%%%%%%%%%%%%%%%%%%%%%%
\begin{figure*}
  \centering
  \begin{subfigure}{0.305\linewidth}
    \includegraphics[width=\linewidth]{latex/pic/se2.pdf}
    \caption{The \(\Delta_{Se}\)}
    \label{visual:se}
  \end{subfigure}
  \begin{subfigure}{0.305\linewidth}
    \includegraphics[width=\linewidth]{latex/pic/po2.pdf}
    \caption{The \(\Delta_{Po}\)}
    \label{visual:po}
  \end{subfigure}
  \begin{subfigure}{0.305\linewidth}
    \includegraphics[width=\linewidth]{latex/pic/sv2.pdf}
    \caption{The \(\Delta_{M}\)}
    \label{visual:sv}
  \end{subfigure}
  \caption{The Visualization of Attribute-Perceptual Distance Factors (APDF) matrix of five responses. The blue represents the response with the highest APDF, and SeAdpra aligns with the fifth response corresponding to the maximum Multi-APDF in (c). The green represents the second response that is next best to the red one.}
  \label{visual}
  % \vspace{-0.2cm}
\end{figure*}
%%%%%%%%%%%%%%%%%%%%%%%%%%%%%%%%%%%%%%%%%
\textbf{Single-APDF Matrix Cannot Predict the Optimal Response.} We randomly selected a pair with a golden label and visualized its specific iteration in Figure~\ref{visual}.
It can be observed that the optimal response in a Single-APDF matrix is not necessarily the same as that in the Multi-APDF matrix.
Specifically, the optimal response in the Semantic Perceptual Factor matrix \(\Delta_{Se}\) is the fifth response in Figure~\ref{visual:se}, while in the Popularity Perceptual Factor matrix \(\Delta_{Po}\) (Figure~\ref{visual:po}), it is the third response. Ultimately, in the Multiple Perceptual Distance Factor matrix \(\Delta_{M}\), the third response is slightly inferior to the fifth response (0.037 vs. 0.038) in Figure~\ref{visual:sv}, and this result aligns with the golden label.
More key findings regarding the ADPF are described in Figure \ref{fig::hot1} and Figure \ref{fig::hot2}.
\section{Conclusion}
In this work, we propose a simple yet effective approach, called SMILE, for graph few-shot learning with fewer tasks. Specifically, we introduce a novel dual-level mixup strategy, including within-task and across-task mixup, for enriching the diversity of nodes within each task and the diversity of tasks. Also, we incorporate the degree-based prior information to learn expressive node embeddings. Theoretically, we prove that SMILE effectively enhances the model's generalization performance. Empirically, we conduct extensive experiments on multiple benchmarks and the results suggest that SMILE significantly outperforms other baselines, including both in-domain and cross-domain few-shot settings.















\section*{Impact Statement}
This paper presents work whose goal is to advance the field of 
Machine Learning. There are many potential societal consequences 
of our work, none of which we feel must be specifically highlighted here.


\bibliography{ref}
\bibliographystyle{icml2025}


%%%%%%%%%%%%%%%%%%%%%%%%%%%%%%%%%%%%%%%%%%%%%%%%%%%%%%%%%%%%%%%%%%%%%%%%%%%%%%%
%%%%%%%%%%%%%%%%%%%%%%%%%%%%%%%%%%%%%%%%%%%%%%%%%%%%%%%%%%%%%%%%%%%%%%%%%%%%%%%
% APPENDIX
%%%%%%%%%%%%%%%%%%%%%%%%%%%%%%%%%%%%%%%%%%%%%%%%%%%%%%%%%%%%%%%%%%%%%%%%%%%%%%%
%%%%%%%%%%%%%%%%%%%%%%%%%%%%%%%%%%%%%%%%%%%%%%%%%%%%%%%%%%%%%%%%%%%%%%%%%%%%%%%
\newpage
\appendix
\onecolumn

%%%%%%%%%%%%%%%%%%%%%%%%%%%%%%%%%%%%%%%%%%%%%%%%%%%%%%%%%


%%%%%%%%%%%%%%%%%%%%%%%%%%%%%%%%%%%%%%%%%%%%%%%%%%%%%%%%%

%	\section{Auxiliary results}
%	
%	\begin{corollary}
%		Assume gradient Lipchitz $\GradLip$.
%		\begin{align*}
%			\scalarvalue(\policyt{t+1})
%			\; \geq \; \scalarvalue(\policyt{t}) \, + \, \frac{1}{2 \GradLip} \, \norm[\big]{\gradtheta \scalarvalue(\policyt{t})}_2^2 \, - \, \frac{\Partitionthetabar}{2 \GradLip \parabeta^2} \, \norm{\Term_2}_2^2
%		\end{align*}
%	\end{corollary}
%	
%	\begin{proposition}
%		\label{thm:Hess}
%		Suppose that for any \mbox{$\reward \in \RewardSp$}, \mbox{$0 \leq \reward(\prompt, \response) \leq \Radius$}. Moreover, suppose $\hesstheta \reward_{\paratheta}(\prompt, \response)$ is positive semidefinite for any $(\prompt, \response) \in \PromptSp \times \ResponseSp$.
%		Then both terms
%		\begin{align*}
%			\Exp_{\prompt \sim \promptdistr, \, \response \sim \policy_{\paratheta}(\cdot \mid \prompt)} \big[ \rewardstar(\context, \response) \big]
%			\qquad \mbox{and} \qquad
%			\kull{\policytheta}{\policyref}
%		\end{align*}
%		are convex in parameter $\paratheta$.
%		Therefore, the objective function $\scalarvalue(\policytheta)$ is a difference-of-convex function.
%	\end{proposition}

%%%%%%%%%%%%%%%%%%%%%%%%%%%%%%%%%%%%%%%%%%%%%%%%%%%%%%%%%

	\section{Proof of Main Results \yaqidone}
    \label{app:proof:main}

        
    
		This section provides the proofs of the main results from \Cref{sec:theory}, covering both optimization and statistical aspects.
		In \Cref{sec:proof:thm:grad}, we prove \Cref{thm:grad}, which establishes the gradient alignment property. For the statistical results, \Cref{sec:proof:thm:stat} begins with the proofs of \Cref{thm:asymp_full,thm:asymp}, which derive the asymptotic distribution of the estimated parameter $\parathetahat$, and concludes with the proof of \Cref{lemma:hess_scalarvalue}, analyzing the asymptotic behavior of the value gap~\mbox{$\scalarvalue(\policystar) - \scalarvalue(\policyhat)$}.
	
	\subsection{Optimization Considerations: Proof of Theorem~\ref{thm:grad} \yaqidone}
	\label{sec:proof:thm:grad}

        % \yaqitbd

        We begin by presenting a rigorous restatement of \Cref{thm:grad}, formally detailed in \Cref{thm:grad_full} below.

        \begin{theorem}[Gradient structure in DPO training]
			\label{thm:grad_full}
			Consider the expected loss function $\Loss(\paratheta)$ during the DPO training phase. Using data collected from our poposed response sampling scheme $ \responsedistr $, the gradient of $ \Loss(\paratheta) $ satisfies
			\begin{align*}
				\gradtheta \Loss(\paratheta) \; = \;
				- \, \frac{\parabeta}{\Partitionthetabar} \, \gradtheta \scalarvalue(\policytheta) \, + \, \Term_2 \, ,
			\end{align*}
			where the constant $ \Partitionthetabar $ is defined in equation~\eqref{eq:weight}, and the term $ \Term_2% = \bigO( \norm{\rewardtheta - \rewardstar}^2 ) 
			$ represents a second-order error.
			
			To control term $ \Term_2 $, assume the following uniform bounds: 
            \begin{itemize}
                \item[(i)] \mbox{$\!\supnorm{\rewardstar} \leq \Radius$}.
                \item[(ii)] For any policy \mbox{$\policytheta \in \PolicySp$}, the induced reward $\rewardtheta$ satisfies 
                \begin{align*}
                    \supnorm{\rewardtheta} \leq \Radius \qquad \mbox{and} \qquad \sup\nolimits_{\prompt, \response} \, \norm{\gradtheta \rewardtheta (\prompt, \response)}_2 \leq \RadiusGrad \, .
                \end{align*}
            \end{itemize}
			Under these conditions, $ \Term_2 $ is bounded as
			\vspace{-.5em}
			\begin{align*}
                \norm{\Term_2}_2 \leq 
                \Const{} \, \cdot \, \Exp_{\prompt \sim \promptdistr, \, \responseone, \responsetwo \sim \policytheta(\cdot \mid \prompt)}
				\bigg[ \, \Big\{ \big( \rewardstar(\context, \responseone) - \rewardstar(\context, \responsetwo) \big)
                - \big( \rewardtheta(\context, \responseone) - \rewardtheta(\context, \responsetwo) \big) \Big\}^2 \bigg] \, ,
			\end{align*}
			where the constant $\Const{}$ is given by $\Const{} = 0.1 \, (1 + e^{2\Radius}) \, \RadiusGrad \big/ \Partitionthetabar$.
		\end{theorem}
	
	The proof of \Cref{thm:grad_full} is structured into three sections. In \Cref{sec:proof:thm:grad_1}, we lay the foundation by presenting the key components, including the explicit expressions for the gradients $\gradtheta \scalarvalue(\policytheta)$ and $\gradtheta \Loss(\paratheta)$, as well as for the sampling density~$\responsedistravg$.
	Then \Cref{sec:proof:thm:grad_2} establishes the connection between $\gradtheta \scalarvalue(\policytheta)$ and $\gradtheta \Loss(\paratheta)$ by leveraging these results, completing the proof of \Cref{thm:grad}. 
	Finally, in \Cref{sec:proof:thm:grad_3}, we provide a detailed derivation of the form of density function~$\responsedistravg$.
	
	\subsubsection{Building Blocks \yaqidone}
	\label{sec:proof:thm:grad_1}
	
	To establish \Cref{thm:grad}, which uncovers the relationship between the gradients of the expected value $\scalarvalue(\policytheta)$ and the negative log-likelihood function $\Loss(\paratheta)$, the first step is to derive explicit expressions for the gradients of both functions. The results are presented in \Cref{lemma:grad_scalarvalue,lemma:grad_loss}, with detailed proofs provided in \Cref{sec:proof:lemma:grad_scalarvalue,sec:proof:lemma:grad_loss}, respectively.
	\begin{lemma}[Gradient of value $\scalarvalue(\policytheta)$]
		\label{lemma:grad_scalarvalue}
		For any $\policytheta$ in the parameterized policy class $\PolicySp$, the gradient of the expected value~$\scalarvalue(\policytheta)$ satisfies
		%			\begin{subequations}
			\begin{multline}
				\label{eq:grad_scalarvalue}
				\gradtheta \scalarvalue(\policytheta)
				\; = \; \frac{1}{2 \parabeta} \, \Exp_{\prompt \sim \promptdistr; \; \responseone, \responsetwo \sim \policytheta(\cdot \mid \prompt)} 
				\bigg[ \Big\{ \big( \rewardstar(\context, \responseone) - \rewardstar(\context, \responsetwo) \big) - \big( \rewardtheta(\context, \responseone) - \rewardtheta(\context, \responsetwo) \big) \Big\} \\ 
				\cdot \big\{ \gradtheta \rewardtheta(\prompt, \responseone) - \gradtheta \rewardtheta(\prompt, \responsetwo) \big\} \bigg] \, .
			\end{multline}
			%			\end{subequations}
	\end{lemma}
	
	
	
	\begin{lemma}[Gradient of the loss function $\Loss(\paratheta)$]
		\label{lemma:grad_loss}
		For any $\policytheta$ in the parameterized policy class $\PolicySp$ and any sampling distribution $\responsedistr$ of the responses, the gradient of the negative log-likelihood function $\Loss(\paratheta)$ is given by
		\begin{subequations}
			\begin{multline}
				\label{eq:gradLoss_BT_0}
				\gradtheta \Loss(\paratheta) \; = \; - \, \Exp_{\prompt \sim \promptdistr; \; (\responseone, \, \responsetwo) \sim \responsedistravg(\cdot \mid \prompt)}
				\bigg[ \, \weight(\prompt) \cdot \Big\{ \sigmoid \big( \rewardstar(\context, \responseone) - \rewardstar(\context, \responsetwo) \big) - \sigmoid \big( \rewardtheta(\context, \responseone) - \rewardtheta(\context, \responsetwo) \big) \Big\} \\ 
				\cdot \big\{ \gradtheta \rewardtheta(\prompt, \responseone) - \gradtheta \rewardtheta(\prompt, \responsetwo) \big\} \bigg] \, ,
			\end{multline}
			where the average density $\responsedistravg$ is defined as
			\begin{align}
				\label{eq:def_responsedistravg_0}
				\responsedistravg(\responseone, \responsetwo \mid \prompt) 
				\; \defn \; \frac{1}{2} \, \big\{ \responsedistr(\responseone, \responsetwo \mid \prompt) + \responsedistr(\responsetwo, \responseone \mid \prompt) \big\}
			\end{align}
		\end{subequations}
			as previously introduced in \cref{eq:def_responsedistravg}.
	\end{lemma}
	
	In \Cref{lemma:grad_loss}, we observe that the gradient $\gradtheta \Loss(\paratheta)$ is expressed as an expectation over the probability distribution $\responsedistravg$. By applying the sampling scheme outlined in \Cref{sec:sampling}, we can derive a more detailed representation of $\gradtheta \Loss(\paratheta)$. This refined form will reveal its close relationship to the gradient $\gradtheta \scalarvalue(\policytheta)$ given in expression \eqref{eq:grad_scalarvalue}.
	
	Before moving forward, it is crucial for us to first derive the explicit form of $\responsedistravg$. Specifically, we claim that the distribution~$\responsedistravg$ satisfies the following property
	\begin{align}
		\label{eq:responsedistravg}
		\frac{\responsedistravg ( \responseone, \responsetwo \mid \prompt )}{\policytheta(\responseone \mid \prompt) \, \policytheta(\responsetwo \mid \prompt)} 
		& \; = \; \frac{1}{2 \, \{ 1 + \Partitionthetapos(\prompt) \, \Partitionthetaneg(\prompt) \}}
		\cdot \frac{1}{\divsigmoid \big( \rewardtheta(\prompt, \responseone) - \rewardtheta(\prompt, \responsetwo) \big)} \, ,
	\end{align}
	where $\divsigmoid$ denotes the derivative of the sigmoid function $\sigmoid$, given by
	\begin{align}
		\label{eq:divsigmoid}
		\divsigmoid(z) \; = \; \frac{1}{( 1 + \exp(-z) )( 1 + \exp(z) )} \; = \; \sigmoid(z) \, \sigmoid(-z)
		\qquad \mbox{for any $z \in \Real$}  \, .
	\end{align}
	With these key components in place, we are now prepared to prove \Cref{thm:grad}.
	
	
	\subsubsection{Derivation of Theorem~\ref{thm:grad} \yaqidone}
	\label{sec:proof:thm:grad_2}
	
	With the tools provided by \Cref{lemma:grad_scalarvalue,lemma:grad_loss} and the sampling density expression in \eqref{eq:responsedistravg}, we are now ready to prove \Cref{thm:grad}.
	
	We begin by applying \Cref{lemma:grad_loss} and reformulating equation~\eqref{eq:gradLoss_BT_0} as
	\begin{align}
		\gradtheta \Loss(\paratheta) \; = \; - \, \Exp_{\prompt \sim \promptdistr; \; \responseone, \, \responsetwo \sim \policytheta(\cdot \mid \prompt)}
		\bigg[ \, & \weight(\prompt) \cdot \frac{\responsedistravg ( \responseone, \responsetwo \mid \prompt )}{\policytheta(\responseone \mid \prompt) \, \policytheta(\responsetwo \mid \prompt)} \notag \\
		& \cdot \Big\{ \sigmoid \big( \rewardstar(\context, \responseone) - \rewardstar(\context, \responsetwo) \big) - \sigmoid \big( \rewardtheta(\context, \responseone) - \rewardtheta(\context, \responsetwo) \big) \Big\} \notag \\ 
		& \cdot \big\{ \gradtheta \rewardtheta(\prompt, \responseone) - \gradtheta \rewardtheta(\prompt, \responsetwo) \big\} \bigg] \,.
		\label{eq:gradLoss}
	\end{align}
	Substituting the density ratio from equation~\eqref{eq:responsedistravg} into expression \eqref{eq:gradLoss} and incorporating the weight function $\weight(\prompt)$ defined in equation \eqref{eq:weight}, we obtain 
	\begin{align}
		\gradtheta \Loss(\paratheta) \; = \; - \frac{1}{2 \, \Partitionthetabar} \, \Exp_{\prompt \sim \promptdistr; \; \responseone, \, \responsetwo \sim \policytheta(\cdot \mid \prompt)}
		\Bigg[ \, & 
		\frac{\sigmoid \big( \rewardstar(\context, \responseone) - \rewardstar(\context, \responsetwo) \big) - \sigmoid \big( \rewardtheta(\context, \responseone) - \rewardtheta(\context, \responsetwo) \big)}{\divsigmoid \big( \rewardtheta(\prompt, \responseone) - \rewardtheta(\prompt, \responsetwo) \big)}  \notag  \\
		& \qquad \qquad \qquad \cdot \big\{ \gradtheta \rewardtheta(\prompt, \responseone) - \gradtheta \rewardtheta(\prompt, \responsetwo) \big\} \Bigg] \, .  \label{eq:gradLoss_0}
	\end{align}
	Using the intuition that the first-order Taylor expansion
	\begin{align*}
		\frac{\sigmoid(z^{\star}) - \sigmoid(z)}{\divsigmoid(z)} \; = \; (z^{\star} - z) + \bigO\big((z^{\star} - z)^2\big)
	\end{align*}
	is valid when $z \to z^\star$, with $z^\star \defn \rewardstar(\context, \responseone) - \rewardstar(\context, \responsetwo)$ and $z \defn \rewardtheta(\context, \responseone) - \rewardtheta(\context, \responsetwo)$, we find that
	\begin{align*}
		& \frac{\sigmoid \big( \rewardstar(\context, \responseone) - \rewardstar(\context, \responsetwo) \big) - \sigmoid \big( \rewardtheta(\context, \responseone) - \rewardtheta(\context, \responsetwo) \big)}{\divsigmoid \big( \rewardtheta(\prompt, \responseone) - \rewardtheta(\prompt, \responsetwo) \big)}  \\
		& \; = \; \Big\{ \big( \rewardstar(\context, \responseone) - \rewardstar(\context, \responsetwo) \big) - \big( \rewardtheta(\context, \responseone) - \rewardtheta(\context, \responsetwo) \big) \Big\} \; + \; \mbox{second-order term}.
	\end{align*}
	Reformulating equation~\eqref{eq:gradLoss_0} in this context, we rewrite it as
    \begin{align}
		\gradtheta \Loss(\paraphi) 
		& = - \, \frac{1}{2 \Partitionthetabar} \, \Exp_{\, \begin{subarray}{l} \\ \prompt \sim \promptdistr; \\ \responseone, \responsetwo \sim \policytheta(\cdot \mid \prompt) \end{subarray}}
		\Bigg[ \, \Big\{ \big( \rewardstar(\context, \responseone) - \rewardstar(\context, \responsetwo) \big) - \big( \rewardtheta(\context, \responseone) - \rewardtheta(\context, \responsetwo) \big) \Big\} \notag  \\
		& \qquad \qquad \qquad \qquad \qquad \qquad \qquad \qquad \quad \cdot \big\{ \gradtheta \rewardtheta(\prompt, \responseone) - \gradtheta \rewardtheta(\prompt, \responsetwo) \big\} \Bigg]
		+ \Term_2 \, , \label{eq:gradLoss_1}
	\end{align}
	where $\Term_2$ represents the second-order residual term related to the estimation error $\rewardtheta - \rewardstar$.
	By applying \Cref{lemma:grad_scalarvalue}, we observe that the primary term in equation~\eqref{eq:gradLoss_1} aligns with the direction of $\gradtheta \scalarvalue(\policytheta)$, resulting in
	\begin{align}
		\label{eq:gradLoss_final}
		\gradtheta \Loss(\paraphi) 
		& = - \, \frac{\parabeta}{\Partitionthetabar} \, \gradtheta \scalarvalue(\policytheta)
		+ \Term_2 \, .
	\end{align}

	
	Next, we proceed to control the second-order term $\Term_2$.
	The conditions
	\begin{align*}
		\supnorm{\rewardstar}, \supnorm{\rewardtheta} \leq \Radius
		\qquad \mbox{and} \qquad \sup\nolimits_{(\prompt, \response) \in \PromptSp \times \ResponseSp} \norm{\gradtheta \rewardtheta (\prompt, \response)}_2 \leq \RadiusGrad,
	\end{align*}
	lead to the bound
	\begin{align*}
		\abs[\Big]{ \, \frac{\sigmoid(z^{\star}) - \sigmoid(z)}{\divsigmoid(z)} - (z^{\star} - z) }
		\; \leq \;  0.1 \, (1 + e^{2\Radius}) \cdot (z^{\star} - z)^2 \, ,
	\end{align*}
	which in turn implies
	\begin{align}
		& \norm{\Term_2}_2
        \notag \\
        \label{eq:gradLoss_Term2}
        & \; \leq \;  \frac{0.1 \, (1 + e^{2\Radius}) \, \RadiusGrad}{\Partitionthetabar} \, \Exp_{\prompt \sim \promptdistr; \; \responseone, \responsetwo \sim \policytheta(\cdot \mid \prompt)} 
        \bigg[ \, \Big\{ \big( \rewardstar(\context, \responseone) - \rewardstar(\context, \responsetwo) \big) - \big( \rewardtheta(\context, \responseone) - \rewardtheta(\context, \responsetwo) \big) \Big\}^2 \bigg] \, .
	\end{align}
	
	Finally, combining equation~\eqref{eq:gradLoss_Term2} with equation~\eqref{eq:gradLoss_final}, we conclude the proof of \Cref{thm:grad}.
	
	
%%%%%%%%%%%%%%%%%%%%%%%%%%%%%%%%%%%%%%%%%%%%%%%%%%%%%%%%%%%%
		
		\subsubsection{Proof of Claim~\eqref{eq:responsedistravg}}
		\label{sec:proof:thm:grad_3}
		
		The remaining step in the proof of \Cref{thm:grad} is to verify the expression for the density ratio in equation~\eqref{eq:responsedistravg}.
		
		Based on the sampling scheme described in \Cref{sec:sampling}, we find that the sampling distribution for the response satisfies
		\begin{align}
			\label{eq:responsedistr_0}
			\responsedistr \big( \responseone, \responsetwo \bigm| \prompt \big)
			& \; = \; \{ 1 - \sampleprob(\prompt) \} \cdot \policytheta(\responseone \mid \prompt) \,  \policytheta(\responsetwo \mid \prompt)
			\, + \, \sampleprob(\prompt) \cdot \policythetapos(\responseone \mid \prompt) \,  \policythetaneg(\responsetwo \mid \prompt) \, ,
		\end{align}
		where the probability $\sampleprob(\prompt)$ is defined as
		\begin{align*}
			\sampleprob(\prompt) = \Partitionthetapos(\prompt) \, \Partitionthetaneg(\prompt) / \{1 + \Partitionthetapos(\prompt) \, \Partitionthetaneg(\prompt) \}
		\end{align*}
		and the policies $\policythetapos$ and $\policythetaneg$ are specified in equations~\eqref{eq:def_policythetapos}~and~\eqref{eq:def_policythetaneg}, respectively.
		This allows us to simplify equation~\eqref{eq:responsedistr_0} to
		\begin{align*}
			\responsedistr \big( \responseone, \responsetwo \bigm| \prompt \big)
			& \; = \; \frac{\policytheta(\responseone \mid \prompt) \, \policytheta(\responsetwo \mid \prompt)}{1 + \Partitionthetapos(\prompt) \, \Partitionthetaneg(\prompt)} \, \Big\{ 1 + \exp\big\{ \rewardtheta(\prompt, \responseone) - \rewardtheta(\prompt, \responsetwo) \big\} \Big\} \, .
		\end{align*}
		Similarly, we derive an expression for $\responsedistr ( \responsetwo, \responseone \mid \prompt )$.
		By averaging the two expressions, for $\responsedistr ( \responseone, \responsetwo \mid \prompt )$ and $\responsedistr ( \responsetwo, \responseone \mid \prompt )$, we obtain
		\begin{align*}
%			\label{eq:responsedistravg_ratio}
			& \frac{\responsedistravg ( \responseone, \responsetwo \mid \prompt )}{\policytheta(\responseone \mid \prompt) \, \policytheta(\responsetwo \mid \prompt)}  \\
			& = \frac{\policytheta(\responseone \mid \prompt) \, \policytheta(\responsetwo \mid \prompt)}{2 \, \{ 1 + \Partitionthetapos(\prompt) \, \Partitionthetaneg(\prompt) \}} \, \Big\{ 2 + \exp\big\{ \rewardtheta(\prompt, \responseone) - \rewardtheta(\prompt, \responsetwo) \big\} + \exp\big\{ \rewardtheta(\prompt, \responsetwo) - \rewardtheta(\prompt, \responseone) \big\} \Big\} \, .
		\end{align*}
		Rewriting this expression using the formula for $\divsigmoid$ in equation~\eqref{eq:divsigmoid}, we arrive at
		\begin{align*}
			& \big\{ 1 + \Partitionthetapos(\prompt) \, \Partitionthetaneg(\prompt) \big\} \cdot \frac{\responsedistravg ( \responseone, \responsetwo \mid \prompt )}{\policytheta(\responseone \mid \prompt) \, \policytheta(\responsetwo \mid \prompt)}  \\
			& \; = \; \frac{1}{2} \, \Big\{ 1 + \exp\big\{ \rewardtheta(\prompt, \responsetwo) - \rewardtheta(\prompt, \responseone) \big\} \Big\}  \Big\{ 1 + \exp\big\{ \rewardtheta(\prompt, \responseone) - \rewardtheta(\prompt, \responsetwo) \big\} \Big\}  \\
			& \; = \; \frac{1}{2 \, \divsigmoid \big( \rewardtheta(\prompt, \responseone) - \rewardtheta(\prompt, \responsetwo) \big)} \, .
		\end{align*}
		Finally, rearranging terms, we recover equation~\eqref{eq:responsedistravg}, completing this part of the proof.

%%%%%%%%%%%%%%%%%%%%%%%%%%%%%%%%%%%%%%%%%%%%%%%%%%%%%%%%%%%%%%%%%%%%%%%%%%%%

	\subsection{Statistical Considerations \yaqidone}
	\label{sec:proof:thm:stat}



        In this section, we present the proofs for \Cref{thm:asymp,lemma:hess_scalarvalue,thm:asymp_full} from \Cref{sec:theory_stat}. 
        We start with the proof of \Cref{thm:asymp_full} in \Cref{sec:proof:thm:asymp_full}, with a rigorous restatement provided in \Cref{thm:asymp_full_full} below.
    		\begin{theorem}
			\label{thm:asymp_full_full}
%			We take $\weight(\prompt) \equiv 1$.
			Assume the reward model $\rewardstar$ in the BT model~\eqref{eq:BT} satisfies $\rewardstar = \reward_{\parathetastar}$ for some parameter $\parathetastar$.
			Assume that $\parathetahat$ minimizes the loss function $\Losshat(\paratheta)$ in the sense that $\sqrt{\numobs} \, \gradtheta \Losshat (\parathetahat) \convergep \veczero$ and that $\parathetahat \convergep \parathetastar$ as the sample size $\numobs \rightarrow \infty$.
			Additionally, suppose the reward function $\rewardtheta(\prompt, \response)$, its gradient $\gradtheta \rewardtheta(\prompt, \response)$ and its Hessian $\hesstheta \rewardtheta(\prompt, \response)$ are uniformly bounded and Lipchitz continuous with respect to $\paratheta$, for all $(\prompt, \response) \in \PromptSp \times \ResponseSp$.
			
			Under these conditions, the estimate $\parathetahat$ asymptotically follows a Gaussian distribution
			\begin{align*}
				\sqrt{\numobs} \; ( \parathetahat - \parathetastar)
				\; \stackrel{d}{\longrightarrow} \; \Gauss( \veczero, \CovOmega )
				\qquad \mbox{as $\numobs \rightarrow \infty$} \, .
			\end{align*}
			We have an estimate of the covariance matrix $\CovOmega$:
            \begin{align*}
                \CovOmega \; \preceq \; \supnorm{\weight} \cdot \CovOpstar^{-1} \, .
            \end{align*}
            For a general sampling scheme $\responsedistr$ chosen, the matrix~$\CovOpstar$ is given by
			\begin{align*}
                % \label{eq:def_CovOpstar_simple}
				\CovOpstar \; \defn \;
				& \Exp_{\prompt \sim \promptdistr, \, (\responseone, \, \responsetwo) \sim \responsedistravg(\cdot \mid \prompt)}
			\Big[ \, \weight(\prompt) \cdot \Var\big(\indicator\{\responseone = \responsewin\} \bigm| \prompt, \responseone, \responsetwo \big) \cdot \grad \, \grad^{\top} \Big] \, ,
				%\label{eq:def_CovOpstar}
			\end{align*}
			where the expectation is taken over the distribution
			%\vspace{-.3em}
			\begin{subequations}
				\begin{align*}
					%\label{eq:def_responsedistravg}
					\responsedistravg(\responseone, \responsetwo \mid \prompt) 
					\defn \frac{1}{2} \, \big\{ \responsedistr(\responseone, \responsetwo \mid \prompt) + \responsedistr(\responsetwo, \responseone \mid \prompt) \big\} \, .
				\end{align*} %~ \vspace{-1.8em} \\
			The variance term is specified as
				\begin{align*}
					& \Var\big(\indicator\{\responseone \; = \; \responsewin\} \mid \prompt, \responseone, \responsetwo \big)
					%\label{eq:def_var}
					= \sigmoid\big( \rewardstar(\prompt, \responseone) - \rewardstar(\prompt, \responsetwo) \big) \, \sigmoid\big( \rewardstar(\prompt, \responsetwo) - \rewardstar(\prompt, \responseone) \big)
					%\notag
				\end{align*}
			and the gradient difference $\grad$ is defined as
				\begin{align*}
					%\label{eq:def_grad}
					\grad \; \defn \; \gradtheta \rewardstar(\prompt, \responseone) - \gradtheta \rewardstar(\prompt, \responsetwo) \, .
				\end{align*}
			\end{subequations}
		\end{theorem}

    \Cref{thm:asymp_full_full} establishes the asymptotic distribution of the estimated parameter $\parathetahat$, which serves as the foundation for the subsequent results. 
	Next, we show that \Cref{thm:asymp} directly follows as a corollary of \Cref{thm:asymp_full_full}, with the detailed derivation provided in \Cref{sec:proof:thm:asymp}. Finally, in \Cref{sec:proof:lemma:hess_scalarvalue}, we prove \Cref{lemma:hess_scalarvalue}, which describes the asymptotic behavior of the value gap $\scalarvalue(\policystar) - \scalarvalue(\policyhat)$.
		
	\subsubsection{Proof of Lemma~\ref{thm:asymp_full} (Theorem~\ref{thm:asymp_full_full}) \yaqidone}
	\label{sec:proof:thm:asymp_full}
	
%	\paragraph{(a) Proof of \Cref{thm:asymp_full}:}

	In this section, we analyze the asymptotic distribution of the estimated parameter $\parathetahat$ for a general sampling distribution $\responsedistr$. The parameter $\parathetahat$ is obtained by solving the optimization problem
	\begin{align*}
		{\rm minimize}_{\paratheta} \quad
		\Losshat(\paratheta) \; \defn \;
		- \frac{1}{\numobs} \sum_{i=1}^{\numobs} \, \weight(\prompti{i}) \cdot \log \sigmoid \Big( \rewardtheta\big(\prompti{i}, \responsewini{i}\big) - \rewardtheta\big(\prompti{i}, \responselosei{i}\big) \Big) \, .
	\end{align*}
	We assume the optimization is performed to sufficient accuracy such that $\gradtheta \Losshat(\parathetahat) = \smallop\big(\numobs^{-\frac{1}{2}}\big)$.
	Under this condition, $\parathetahat$ qualifies as a $Z$-estimator.
	To study its asymptotic behavior, we use the master theorem for $Z$-estimators \citep{kosorok2008introduction}, the formal statement of which is provided in \Cref{thm:master} in \Cref{sec:master}.
	
	To apply the master theorem, we set $\Psi \defn \gradtheta \Loss$ and $\Psi_{\numobs} \defn \gradtheta \Losshat$ and verify the conditions. In particular, the smoothness condition~\eqref{eq:master_cond} in \Cref{thm:master} translates to the following equation in our context:
    \begin{align}
    	\label{eq:master_cond_proof}
    	& \sqrt{n} \, \big\{ \gradtheta \Losshat (\parathetahat) - \gradtheta \Loss(\parathetahat) \big\} - \sqrt{n} \, \big\{ \gradtheta \Losshat (\parathetastar) - \gradtheta \Loss (\parathetastar) \big\}  
    	\; = \; \smallop \big( 1 + \sqrt{n} \, \norm{ \parathetahat - \parathetastar }_2 \big) \, .
    \end{align}
    This condition follows from the second-order smoothness of the reward function $\rewardtheta$ with respect to $\paratheta$. A rigorous proof is provided in \Cref{sec:proof:eq:master_cond_proof}.
    

	We now provide the explicit form of the derivative $\dot{\Psi}_{\parathetastar} = \hesstheta \Loss(\parathetastar)$, as captured in the following lemma. The proof of this result can be found in \Cref{sec:proof:lemma:hess_loss}.
	\begin{lemma}
		\label{lemma:hess_loss}
		The Hessian matrix of the population loss $\Loss(\paratheta)$ at $\paratheta = \parathetastar$ is
		\begin{align}
			\label{eq:hess_loss}
			\hesstheta \Loss(\parathetastar) \; = \; \CovOpstar \, ,
		\end{align}
		where the matrix $\CovOpstar$ is defined in equation~\eqref{eq:def_CovOpstar}.
	\end{lemma}

	
	Next, we analyze the asymptotic behavior of the gradient $\gradtheta \Losshat(\parathetastar)$.
	The proof is deferred to \Cref{sec:proof:lemma:grad_loss_stat}.
	\begin{lemma}
		\label{lemma:grad_loss_stat}
		The gradient of the empirical loss $\Losshat(\paratheta)$ at $\paratheta = \parathetastar$ satisfies
		\begin{subequations}
		\begin{align}
			\sqrt{\numobs} \, \big( \gradtheta \Losshat(\parathetastar) - \gradtheta \Loss(\parathetastar) \big)
			\; \stackrel{d}{\longrightarrow} \; \Gauss(\veczero, \CovOptil)
			\qquad \mbox{as $\numobs \rightarrow \infty$},
		\end{align}
        where the covariance matrix $\CovOptil \in \Real^{\Dim \times \Dim}$ is bounded as follows:
        \begin{align}
        	\label{eq:CovOptil}
        	\CovOptil \; \preceq \; \supnorm{\weight} \cdot \CovOpstar \, ,
        \end{align}
        \end{subequations}
        with $\CovOpstar$ defined in equation~\eqref{eq:def_CovOpstar}.
	\end{lemma}
	
	Combining these results, and assuming $\CovOpstar$ is nonsingular, the master theorem (\Cref{thm:master}) yields the asymptotic distribution of $\parathetahat$:
	\begin{align*}
		\sqrt{\numobs} \, \big( \parathetahat - \parathetastar \big)
		\; \converged \; \Gauss\big( \veczero, \CovOpstar^{-1} \CovOptil \CovOpstar^{-1} \big) \, .
	\end{align*}
	Furthermore, from the bound~\eqref{eq:CovOptil}, the covariance matrix $\CovOmega ; \defn \CovOpstar^{-1} \CovOptil \CovOpstar^{-1}$ satisfies
	\begin{align*}
		 \CovOmega \; = \CovOpstar^{-1} \CovOptil \CovOpstar^{-1}  \; \preceq \; \supnorm{\weight} \cdot \CovOpstar^{-1} \, .
	\end{align*}
	Therefore, we have established the asymptotic distribution of $\parathetahat$, completing the proof of \Cref{thm:asymp_full}.
	
	
%%%%%%%%%%%%%%%%%%%%%%%%%%%%%%%%%%%%%%%%%%%%%%%%%%%%%%%%%%%%%%%%%%%%%%%%%%%%%%

%	\paragraph{(b) Proof of \eqref{eq:def_CovOpstar_simple}}
	\subsubsection{Proof of Theorem~\ref{thm:asymp}}
	\label{sec:proof:thm:asymp}
	
	\Cref{thm:asymp} is a direct corollary of \Cref{thm:asymp_full}, using our specific choice of sampling distribution $\responsedistr$. To establish this, we demonstrate how the general covariance matrix $\CovOpstar$ in equation~\eqref{eq:def_CovOpstar} simplifies to the form in equation~\eqref{eq:def_CovOpstar_simple} under our proposed sampling scheme.

    To establish the result in this section, we impose the following regularity condition:
    There exists a constant $\Const{} \geq 1$ satisfying
    \begin{align}
        \label{eq:last_cond}
        \Var_{\rewardtheta}\big(\indicator\{\responseone = \responsewin\} \bigm| \prompt, \responseone, \responsetwo \big)
        \; \leq \; \Const{} \cdot \Var_{\rewardstar}\big(\indicator\{\responseone = \responsewin\} \bigm| \prompt, \responseone, \responsetwo \big) 
    \end{align}
    for any prompt $\prompt \in \PromptSp$ and responses $\responseone, \responsetwo \in \ResponseSp$.
    Here $\Var_{\rewardtheta}\big(\indicator\{\responseone = \responsewin\} \bigm| \prompt, \responseone, \responsetwo \big)$ denotes the conditional variance under the BT model~\eqref{eq:BT}, when the implicit reward function $\rewardstar$ is replaced by $\rewardtheta$. The term \mbox{$\Var_{\rewardstar}\big(\indicator\{\responseone = \responsewin\} \bigm| \prompt, \responseone, \responsetwo \big)
    \equiv$} \mbox{$\Var\big(\indicator\{\responseone = \responsewin\} \bigm| \prompt, \responseone, \responsetwo \big) $} represents the conditional variance under the ground-truth BT model, where the reward function is given by $\rewardstar$.
	
	We begin by leveraging the property of the sampling distribution $\responsedistr$ from equation~\eqref{eq:responsedistravg} and the derivative $\divsigmoid$ of the sigmoid function $\sigmoid$, given in equation~\eqref{eq:divsigmoid}. Specifically, we find that
	\begin{align*}
		%\label{eq:responsedistravg2_original}
		& \frac{\responsedistravg ( \responseone, \responsetwo \mid \prompt )}{\policytheta(\responseone \mid \prompt) \, \policytheta(\responsetwo \mid \prompt)} \notag  \\
		& 
		\; = \; \frac{1}{2 \, \{ 1 + \Partitionthetapos(\prompt) \, \Partitionthetaneg(\prompt) \}}
		\cdot \frac{1}{\sigmoid \big( \rewardtheta(\prompt, \responseone) - \rewardtheta(\prompt, \responsetwo) \big) \, \sigmoid \big( \rewardtheta(\prompt, \responsetwo) - \rewardtheta(\prompt, \responseone) \big)}  \\
        & \; = \; \frac{1}{2 \, \{ 1 + \Partitionthetapos(\prompt) \, \Partitionthetaneg(\prompt) \}}
		\cdot \frac{1}{\Var_{\rewardtheta}\big(\indicator\{\responseone = \responsewin\} \bigm| \prompt, \responseone, \responsetwo \big)} \, .
	\end{align*}
    We then apply condition~\eqref{eq:last_cond} and derive
        \begin{equation} 
		 \frac{\responsedistravg ( \responseone, \responsetwo \mid \prompt )}{\policytheta(\responseone \mid \prompt) \, \policytheta(\responsetwo \mid \prompt)} \; \geq \; \frac{\Const{}^{-1}}{2 \, \{ 1 + \Partitionthetapos(\prompt) \, \Partitionthetaneg(\prompt) \}} \cdot \frac{1}{\Var_{\rewardstar}\big(\indicator\{\responseone = \responsewin\} \bigm| \prompt, \responseone, \responsetwo \big)} \, .
         \label{eq:responsedistravg2}
	\end{equation}
	Next, substituting this result~\eqref{eq:responsedistravg2} into equation~\eqref{eq:def_CovOpstar}, alongside the weight function $\weight(\prompt)$ from equation~\eqref{eq:weight}, we reform $\CovOpstar$ as
	\begin{align}
		\CovOpstar
		& \; = \; \Exp_{\prompt \sim \promptdistr; \; \responseone, \, \responsetwo \sim \policytheta(\cdot \mid \prompt)}
		\bigg[ \, \frac{\responsedistravg ( \responseone, \responsetwo \mid \prompt )}{\policytheta(\responseone \mid \prompt) \, \policytheta(\responsetwo \mid \prompt)} \cdot \weight(\prompt) \cdot \Var\big(\indicator\{\responseone = \responsewin\} \bigm| \prompt, \responseone, \responsetwo \big) \cdot \grad \, \grad^{\top} \bigg]  \notag \\
		\label{eq:def_CovOpstar_2}
		& \; \succeq \; \frac{1}{2 \, \Const{} \, \Partitionthetabar} \, \Exp_{\prompt \sim \promptdistr; \; \responseone, \, \responsetwo \sim \policytheta(\cdot \mid \prompt)}
		\big[ \, \grad \, \grad^{\top} \big] \, .
	\end{align}
	The conditional expectation of $\grad \grad^\top$ simplifies as
    \begin{align*}
    	& \Exp_{\responseone, \, \responsetwo \sim \policytheta(\cdot \mid \prompt)}
    	\big[ \, \grad \grad^{\top} \bigm| \prompt\big]  \\
    	& \; = \; \Exp_{\responseone, \, \responsetwo \sim \policytheta(\cdot \mid \prompt)}
    	\Big[ \big\{ \gradtheta \rewardstar(\prompt, \responseone) - \gradtheta \rewardstar(\prompt, \responsetwo) \big\} \big\{ \gradtheta \rewardstar(\prompt, \responseone) - \gradtheta \rewardstar(\prompt, \responsetwo) \big\}^{\top} \Bigm| \prompt\Big]  \\
    	& \; = \; 2 \cdot \Exp_{\response \sim \policytheta(\cdot \mid \prompt)}
    	\Big[ \, \gradtheta \rewardstar(\prompt, \response) \, \gradtheta \rewardstar(\prompt, \response)^{\top} \Bigm| \prompt\Big] \\
        & \qquad \qquad - 2 \cdot \Exp_{\response \sim \policytheta(\cdot \mid \prompt)} \big[ \, \gradtheta \rewardstar(\prompt, \response) \bigm| \prompt\big]  \, \Exp_{\response \sim \policytheta(\cdot \mid \prompt)}
    	\big[ \,\gradtheta \rewardstar(\prompt, \response) \bigm| \prompt \big]^{\top}  \\
    	& \; = \; 2 \cdot \Cov_{\response \sim \policytheta(\cdot \mid \prompt)} \big[ \gradtheta \rewardstar(\prompt, \response) \bigm| \prompt \big] \, .
    \end{align*}
	Substituting this result into equation~\eqref{eq:def_CovOpstar_2}, we arrive at the conclusion that
	\begin{align*}
		\CovOpstar \; \succeq \; \frac{1}{\Const{} \, \Partitionphibar} \, \Exp_{\prompt \sim \promptdistr} \Big[ \Cov_{\response \sim \policystar(\cdot \mid \prompt)} \big[ \gradtheta \rewardstar(\prompt, \response) \bigm| \prompt \big] \Big] \, ,
	\end{align*}
	which matches the simplified form in equation~\eqref{eq:def_CovOpstar_simple} as stated in \Cref{thm:asymp}.

		
		
		
%%%%%%%%%%%%%%%%%%%%%%%%%%%%%%%%%%%%%%%%%%%%%%%%%%%%%%%%%%%%%%%%%%%%%%%%%%%%
	
	
    \subsubsection{Proof of Theorem~\ref{lemma:hess_scalarvalue} \yaqidone}
    \label{sec:proof:lemma:hess_scalarvalue}

	\paragraph{Gradient $ \gradtheta \scalarvalue(\policystar) $ and Hessian $\hesstheta \scalarvalue(\policystar)$:}
	
    The equality $ \gradtheta \scalarvalue(\policystar) = 0$ follows directly from the gradient expression~\eqref{eq:grad_scalarvalue0} for $ \gradtheta \scalarvalue(\policytheta) $, evaluated at $ \paratheta = \parathetastar $ with~\mbox{$ \rewardtheta = \rewardstar $}.
    
    The proof of the Hessian result, $ \hesstheta \scalarvalue(\policystar) = - (1 / \parabeta) \cdot \CovOpstar $, involves straightforward but technical differentiation of equation~\eqref{eq:grad_scalarvalue0}. For brevity, we defer this proof to \Cref{sec:proof:eq:hessscalarvalue}.
  
  	\paragraph{Asymptotic Distribution of Value Gap $ \scalarvalue(\policystar) - \scalarvalue(\policyhat) $:}
    To understand the behavior of the value gap $ \scalarvalue(\policystar) - \scalarvalue(\policyhat) $, we start by applying a Taylor expansion of $ \scalarvalue(\policytheta) $ around $ \parathetastar $. This gives
	\begin{align*}
%		\label{eq:Taylor_scalarvalue}
		\scalarvalue(\policystar) - \scalarvalue(\policyhat)
		\; = \; \gradtheta \scalarvalue(\policystar)^{\top} (\parathetastar - \parathetahat) - \frac{1}{2} (\parathetastar - \parathetahat)^{\top} \hesstheta \scalarvalue(\policystar) (\parathetastar - \parathetahat) + \smallo\big( \norm{\parathetastar - \parathetahat}_2^2 \big) \, .
	\end{align*}
	By substituting $ \gradtheta \scalarvalue(\policystar) = \veczero $ (a direct result of the optimality of $ \policystar $), the linear term vanishes. Introducing the shorthand $ \HessMt \defn -\hesstheta \scalarvalue(\policystar) = (1 / \parabeta) \cdot \CovOpstar $, the expression simplifies to
	\begin{align}
		\label{eq:Taylor_scalarvalue}
		\scalarvalue(\policystar) - \scalarvalue(\policyhat)
		\; = \; \frac{1}{2} \, (\parathetahat - \parathetastar)^{\top} \HessMt \, (\parathetahat - \parathetastar) + \smallo\big( \norm{\parathetahat - \parathetastar}_2^2 \big) \, .
	\end{align}
	When the sample size $ \numobs $ is sufficiently large, $ \parathetahat $ approaches $ \parathetastar $, making the higher-order term negligible. Therefore, the value gap is dominated by the quadratic form.
	
	From \Cref{thm:asymp}, we know the parameter estimate $ \parathetahat $ satisfies
	\begin{align*}
	\sqrt{\numobs} \, (\parathetahat - \parathetastar)
	\;\stackrel{d}{\longrightarrow}\;
	\Gauss(\veczero, \CovOmega).
	\end{align*}
	Substituting this result into the quadratic approximation of the value gap, we find that the scaled value gap has the asymptotic distribution
	\begin{align}
		\label{eq:gap_distr}
		\numobs \cdot \{ \scalarvalue(\policystar) - \scalarvalue(\policyhat) \}
		 \; \stackrel{d}{\longrightarrow} \; \frac{1}{2} \, \vecz^{\top} \CovOmega^{\frac{1}{2}} \HessMt \CovOmega^{\frac{1}{2}} \vecz 
		 \; \nfed \bX
		 \qquad \mbox{where $\vecz \sim \Gauss(\veczero, \IdMt)$}.
	\end{align}
	This approximation provides a clear intuition: the value gap is asymptotically driven by a weighted chi-squared-like term involving the covariance structure $ \CovOmega $ and the Hessian-like matrix $ \HessMt $.
	
	To rigorously establish this result, we will apply Slutsky’s theorem. The full proof is presented in \Cref{sec:proof:gap_distr}.
	
	\paragraph{Bounding the Chi-Square Distribution:}
	
	To bound the random variable $ \bX $, we first leverage the estimate of the covariance matrix $ \CovOmega $ provided by \Cref{thm:asymp}:
	\begin{align*}
		\CovOmega \; \preceq \; \Const{} \, \Partitionthetabar \, \supnorm{\weight} \cdot \CovOpstar^{-1},
	\end{align*}
    where the constant $\Const$ comes from condition~\eqref{eq:last_cond}.
	It follows that the matrix $ \CovOmega^{\frac{1}{2}} \HessMt \CovOmega^{\frac{1}{2}} $ appearing in equation~\eqref{eq:gap_distr} can be bounded as
	\begin{align*}
		\CovOmega^{\frac{1}{2}} \HessMt \CovOmega^{\frac{1}{2}} 
		\; \preceq \;  \Const \, \supnorm{\weight} \cdot \CovOpstar^{-\frac{1}{2}} \HessMt \CovOpstar^{-\frac{1}{2}} \; = \; \Const \cdot \frac{\Partitionthetabar \, \supnorm{\weight}}{\parabeta} \cdot \IdMt
		\; = \; \Const \cdot \frac{1 + \supnorm{\Partitionthetapos \Partitionthetaneg}}{\parabeta}
		\cdot \IdMt \, .
	\end{align*}
	Here the last equality uses the definition of the weight function $ \weight $ from equation~\eqref{eq:weight}. Substituting this bound into the quadratic form, we derive
	\begin{align*}
		\bX
		\; = \; \frac{1}{2} \, \vecz^{\top} \CovOmega^{\frac{1}{2}} \HessMt \CovOmega^{\frac{1}{2}} \vecz 
		\; \leq \; \Const \cdot \frac{1 + \supnorm{\Partitionthetapos \Partitionthetaneg}}{2\parabeta}
		\cdot \vecz^{\top} \vecz \, ,
	\end{align*}
	where $ \vecz \sim \Gauss(\veczero, \IdMt) $.
	Since $ \vecz^{\top} \vecz $ follows a chi-square distribution with $ \Dim $ degrees of freedom, $ \bX $ is stochastically dominated by a rescaled chi-square random variable 
	\begin{align*}
		\Const \cdot \frac{1 + \supnorm{\Partitionthetapos \Partitionthetaneg}}{2\parabeta} \cdot \chisquare_{\Dim}.
	\end{align*}
	Equivalently, we can express this dominance as
	\begin{align}
		\label{eq:gap_bd0}
		\limsup_{\numobs \rightarrow \infty} \; \Prob \bigg\{ \numobs \, \{ \scalarvalue(\policystar) - \scalarvalue(\policyhat) \} > \Const \cdot \frac{1 + \supnorm{\Partitionthetapos \Partitionthetaneg}}{2\parabeta} \cdot t \bigg\}
		\; \leq \; \Prob\big\{ \chisquare_{\Dim} > t \big\}
		\qquad \mbox{for any $t > 0$}.
	\end{align}
	This inequality, given in equation~\eqref{eq:gap_bd0}, corresponds to the first bound in equation~\eqref{eq:gap_bd}.
	
	The second inequality in equation~\eqref{eq:gap_bd} provides a precise tail bound for $\chisquare_{\Dim}$. As its proof involves more technical details, we defer it to \Cref{sec:proof:chisqtail}.
	

	


	
	
%%%%%%%%%%%%%%%%%%%%%%%%%%%%%%%%%%%%%%%%%%%%%%%%%%%%%%%%%%%%%%%%%%%%%%%%%%%%

	\section{Proof of Auxiliary Results \yaqidone}
    \label{app:aux}
	
	This section provides proofs of auxiliary results supporting the main theorems and lemmas. In \Cref{sec:proof:aux:thm:grad}, we present the auxiliary results required for \Cref{thm:grad}. \Cref{sec:proof:thm:asymp_aux} details the proofs of supporting results for \Cref{thm:asymp}. Finally, in \Cref{sec:proof:lemma:hess_scalarvalue_aux}, we establish the auxiliary results necessary for \Cref{lemma:hess_scalarvalue}.

	\subsection{Proof of Auxiliary Results for Theorem~\ref{thm:grad} \yaqidone}
	\label{sec:proof:aux:thm:grad}
	
		In this section, we provide the proofs of several auxiliary results that support the proof of \Cref{thm:grad}. Specifically,
		\Cref{sec:proof:lemma:grad_policy} presents the forms of the gradients of the policy~$\policytheta$ and the reward $\rewardtheta$, which serve as fundamental building blocks for deriving the lemmas.
		\Cref{sec:proof:lemma:grad_scalarvalue} analyzes the gradient of the return function $\scalarvalue(\policytheta)$, as defined in equation~\eqref{eq:objective}.
		\Cref{sec:proof:lemma:grad_loss} focuses on deriving expressions for the gradient of the negative log-likelihood function $\Loss(\paratheta)$.
	
		\subsubsection{Gradients of Policy $\policytheta$ and Reward $\rewardtheta$}
		\label{sec:proof:lemma:grad_policy}
		
		In this part, we introduce results for the gradients of policy $\policytheta$ and reward~$\rewardtheta$ with respsect to parameter~$\paratheta$, which lay the foundation of our calculations.
		
			\begin{lemma}[Gradients of policy $\policytheta$ and reward function $\rewardtheta$]
			\label{lemma:grad_policy}
			The gradients of the policy $\policytheta$ and the reward function $\rewardtheta$ can be expressed in terms of each other as follows
			\begin{subequations}
				\begin{align}
					\label{eq:gradpolicy}
					\gradtheta \policytheta(\diff \response \mid \prompt)
					& \; = \;  \policytheta(\diff \response \mid \prompt) \cdot \frac{1}{\parabeta} \,
					\Big\{ \gradtheta \rewardtheta(\prompt, \response) - \Exp_{\responsenew \sim \policytheta(\cdot \mid \prompt)}\big[ \gradtheta \rewardtheta(\prompt, \responsenew) \big] \Big\} \, ,  \\
					\label{eq:gradreward}
					\gradtheta \rewardtheta (\prompt, \response)
					& \; = \; \parabeta \cdot \frac{\gradtheta \policytheta(\response \mid \prompt)}{\policytheta(\response \mid \prompt)} \, .
				\end{align}
			\end{subequations}
		\end{lemma}
		
		We now proceed to prove \Cref{lemma:grad_policy}.  \\
		
		To begin, recall our definition of the reward function $\rewardtheta$ as given in equation~\eqref{eq:def_reward}.
		It directly follows that
		\begin{align*}
			\gradtheta \rewardtheta (\prompt, \response)
			\; = \; \parabeta \cdot \frac{\gradtheta \policytheta(\response \mid \prompt)}{\policytheta(\response \mid \prompt)} \, .
		\end{align*}
		This result confirms equation~\eqref{eq:gradreward} as stated in \Cref{lemma:grad_policy}.
		
		Next, we express the policy $\policytheta(\diff \response \mid \prompt)$ in terms of the reward function $\rewardtheta(\prompt, \response)$. By reformulating equation~\eqref{eq:def_reward}, we obtain
		\begin{subequations}
		\begin{align}
			\label{eq:policyfromreward}
			\policytheta(\diff \response \mid \prompt)
			\; = \; \frac{1}{\Partitiontheta (\prompt)} \, \policyref(\diff \response \mid \prompt)
			\exp \Big\{ \frac{1}{\parabeta} \, \rewardtheta(\prompt, \response) \Big\} \, ,
		\end{align}
		where $\Partitiontheta (\prompt)$ is the partition function defined as
		\begin{align}
			\label{eq:def_Partition}
			\Partitiontheta (\prompt)
			& \; = \; \int_{\ResponseSp} \, \policyref(\diff \response \mid \prompt)
			\exp \Big\{ \frac{1}{\parabeta} \, \rewardtheta(\prompt, \response) \Big\} \, .
		\end{align}
		\end{subequations}
		
		We then compute the gradient of $\policytheta(\diff \response \mid \prompt)$ with respect to $\paratheta$. Applying the chain rule, we get
		\begin{align}
			\gradtheta \policytheta(\diff \response \mid \prompt)
			& \; = \; \frac{1}{\Partitiontheta (\prompt)} \, \policyref(\diff \response \mid \prompt)
			\exp \Big\{ \frac{1}{\parabeta} \, \rewardtheta(\prompt, \response) \Big\}
			\cdot \frac{1}{\parabeta} \, \gradtheta \rewardtheta(\prompt, \response)  \notag  \\
			\label{eq:gradtheta1}
			& \quad - \frac{1}{\Partitiontheta^2(\prompt)} \, \policyref(\diff \response \mid \prompt)
			\exp \Big\{ \frac{1}{\parabeta} \, \rewardtheta(\prompt, \response) \Big\}
			\cdot \gradtheta \Partitiontheta(\prompt) \, .
		\end{align}
		We need the gradient of the partition function $\Partitiontheta(\prompt)$:
		\begin{align}
			\gradtheta \Partitiontheta (\prompt)
			& \; = \; \int_{\ResponseSp} \, \policyref(\diff \response \mid \prompt)
			\exp \Big\{ \frac{1}{\parabeta} \, \rewardtheta(\prompt, \response) \Big\}
			\cdot \frac{1}{\parabeta} \, \gradtheta \rewardtheta(\prompt, \response)  \notag   \\
			& \; = \; \Partitiontheta (\prompt) \cdot \int_{\ResponseSp} \, \policytheta(\diff \response \mid \prompt) \cdot \frac{1}{\parabeta} \, \gradtheta \rewardtheta(\prompt, \response)  \notag   \\
			\label{eq:gradPartition}
			& \; = \; \Partitiontheta (\prompt) \cdot \frac{1}{\parabeta} \, \Exp_{\response \sim \policytheta(\cdot \mid \prompt)} \big[ \gradtheta \rewardtheta(\prompt, \response) \big] \, .
		\end{align}
		Substituting equation~\eqref{eq:gradPartition} back into equation~\eqref{eq:gradtheta1}, we simplify the expression for the gradient of $\policytheta(\diff \response \mid \prompt)$:
		\begin{align*}
			& \gradtheta \policytheta(\diff \response \mid \prompt)  \\
			& \; = \; \frac{1}{\Partitiontheta (\prompt)} \, \policyref(\diff \response \mid \prompt)
			\exp \Big\{ \frac{1}{\parabeta} \, \rewardtheta(\prompt, \response) \Big\}
			\cdot \frac{1}{\parabeta} \, \Big\{ \gradtheta \rewardtheta(\prompt, \response) - \Exp_{\responsenew \sim \policytheta(\cdot \mid \prompt)} \big[ \gradtheta \rewardtheta(\prompt, \responsenew) \big] \Big\} \, .
		\end{align*}
		This matches equation~\eqref{eq:gradpolicy} from \Cref{lemma:grad_policy}, thereby completing the proof.
		
		
		
%%%%%%%%%%%%%%%%%%%%%%%%%%%%%%%%%%%%%%%%%%%%%%%%%%%%%%%%%%%%%%%%%%%%%%%%%%%%%%%%%%%%%%%%%%%%
	
	
		\subsubsection{Proof of Lemma~\ref{lemma:grad_scalarvalue} \yaqidone}
		\label{sec:proof:lemma:grad_scalarvalue}
		
		Equality \eqref{eq:grad_scalarvalue} in \Cref{lemma:grad_scalarvalue} can be derived as a consequence of a more detailed result. We state it in \Cref{lemma:grad_scalarvalue_full}.
		
		\begin{lemma}
			\label{lemma:grad_scalarvalue_full}
			\begin{subequations}
			For a policy $\policytheta$, the gradients with respect to the parameter $\paratheta$ of its expected return $\Exp_{\prompt \sim \promptdistr, \, \response \sim \policytheta(\cdot \mid \prompt)} \big[ \rewardstar(\context, \response) \big] $ and its KL divergence from a reference policy $\kull{\policytheta}{\policyref}$ are given by
			\begin{align}
				& \gradtheta \Exp_{\prompt \sim \promptdistr, \, \response \sim \policytheta(\cdot \mid \prompt)} \big[ \rewardstar(\context, \response) \big]  \notag \\
				\label{eq:grad_return}
				& 
				\qquad  = \; \frac{1}{\parabeta} \, \Exp_{\prompt \sim \promptdistr, \,  \response \sim \policytheta(\cdot \mid \prompt)}
				\bigg[ \rewardstar(\prompt, \response)
				\Big\{ \gradtheta \rewardtheta(\prompt, \response) - \Exp_{\responsenew \sim \policytheta(\cdot \mid \prompt)}\big[ \gradtheta \rewardtheta(\prompt, \responsenew) \big] \Big\} \bigg] \, , \\
				& \gradtheta \kull{\policytheta}{\policyref}  \notag  \\
				\label{eq:grad_KL}
				& \qquad = 
				\frac{1}{\parabeta^2} \, \Exp_{\prompt \sim \promptdistr, \, \response \sim \policytheta(\cdot \mid \prompt)}
				\bigg[ \rewardtheta(\prompt, \response)
				\Big\{ \gradtheta \rewardtheta(\prompt, \response) - \Exp_{\responsenew \sim \policytheta(\cdot \mid \prompt)}\big[ \gradtheta \rewardtheta(\prompt, \responsenew) \big] \Big\} \bigg] \, .
			\end{align}
			\end{subequations}
		\end{lemma}
		
		Recall that the scalar value $\scalarvalue(\policytheta)$ of the policy is defined as
		\begin{align*}
			\scalarvalue(\policytheta) \; = \;
			\Exp_{\prompt \sim \promptdistr, \, \response \sim \policytheta(\cdot \mid \prompt)} \big[ \rewardstar(\context, \response) \big] \, - \,
			\parabeta \, \kull{\policytheta}{\policyref} \, .
		\end{align*}
		Using \Cref{lemma:grad_scalarvalue_full}, we derive the gradient of $\scalarvalue(\policytheta)$ as
		\begin{align}
			& \gradtheta \scalarvalue(\policytheta) \; = \; \gradtheta \Exp_{\prompt \sim \promptdistr, \, \response \sim \policytheta(\cdot \mid \prompt)} \big[ \rewardstar(\context, \response) \big] \, - \,
			\parabeta \, \gradtheta \kull{\policytheta}{\policyref}  \notag  \\
			\label{eq:grad_scalarvalue0}
			& \; = \; \frac{1}{\parabeta} \, \Exp_{\prompt \sim \promptdistr, \,  \response \sim \policytheta(\cdot \mid \prompt)}
			\bigg[ \big\{ \rewardstar(\prompt, \response) - \rewardtheta(\prompt, \response) \big\}
			\Big\{ \gradtheta \rewardtheta(\prompt, \response) - \Exp_{\responsenew \sim \policytheta(\cdot \mid \prompt)}\big[ \gradtheta \rewardtheta(\prompt, \responsenew) \big] \Big\} \bigg] \, .
		\end{align}
		We rewrite the expression in equation \eqref{eq:grad_scalarvalue0} in two equivalent forms by exchanging the roles of $\responseone$ and $\responsetwo$:
		\begin{subequations}
		\begin{align}
			& \gradtheta \scalarvalue(\policytheta) \notag \\ 
			\label{eq:grad_scalarvalue1}
			& \; = \; \frac{1}{\parabeta} \, \Exp_{\prompt \sim \promptdistr, \,  \responseone \sim \policytheta(\cdot \mid \prompt)}
			\bigg[ \big\{ \rewardstar(\prompt, \responseone) - \rewardtheta(\prompt, \responseone) \big\} \Big\{ \gradtheta \rewardtheta(\prompt, \responseone) - \Exp_{\responsetwo \sim \policytheta(\cdot \mid \prompt)}\big[ \gradtheta \rewardtheta(\prompt, \responsetwo) \big] \Big\} \bigg] \, ,  \\
			& \gradtheta \scalarvalue(\policytheta) \notag \\ 
			\label{eq:grad_scalarvalue2}
			& \; = \; \frac{1}{\parabeta} \, \Exp_{\prompt \sim \promptdistr, \,  \responsetwo \sim \policytheta(\cdot \mid \prompt)}
			\bigg[ \big\{ \rewardstar(\prompt, \responsetwo) - \rewardtheta(\prompt, \responsetwo) \big\} \Big\{ \gradtheta \rewardtheta(\prompt, \responsetwo) - \Exp_{\responseone \sim \policytheta(\cdot \mid \prompt)}\big[ \gradtheta \rewardtheta(\prompt, \responseone) \big] \Big\} \bigg] \, .
		\end{align}
		\end{subequations}
		By taking the average of the two equivalent formulations above, we obtain equality \eqref{eq:grad_scalarvalue} and complete the proof of \Cref{lemma:grad_scalarvalue}.  \\
		
		We now proceed to prove \Cref{lemma:grad_scalarvalue_full}, tackling equalities \eqref{eq:grad_return} and \eqref{eq:grad_KL} one by one.
		
		\paragraph{Proof of Equality~\eqref{eq:grad_return} from \Cref{lemma:grad_scalarvalue_full}:}
		We begin by expressing the expected return as
		\begin{align*}
			\Exp_{\prompt \sim \promptdistr, \, \response \sim \policytheta(\cdot \mid \prompt)} \big[ \rewardstar(\context, \response) \big]
			& \; = \; \Exp_{\prompt \sim \promptdistr} \bigg[ \int_{\ResponseSp} \rewardstar(\prompt, \response) \, \policytheta(\diff \response \mid \prompt) \bigg] \, .
		\end{align*}
		Taking the gradient of both sides with respect to $\paratheta$, we have
		\begin{align}
			\label{eq:grad_return0}
			\gradtheta \Exp_{\prompt \sim \promptdistr, \, \response \sim \policytheta(\cdot \mid \prompt)} \big[ \rewardstar(\context, \response) \big]
			& \; = \; \Exp_{\prompt \sim \promptdistr} \bigg[ \int_{\ResponseSp} \rewardstar(\prompt, \response) \, \gradtheta \policytheta(\diff \response \mid \prompt) \bigg] \, .
		\end{align}
		Using the expression for the policy gradient $\gradtheta \policytheta$ provided in \Cref{lemma:grad_policy}, the right-hand side of \eqref{eq:grad_return0} simplifies to
		\begin{align*}
			\mbox{RHS of \eqref{eq:grad_return0}}
			& \; = \; \Exp_{\prompt \sim \promptdistr} \bigg[ \int_{\ResponseSp} \rewardstar(\prompt, \response) \, \policytheta(\diff \response \mid \prompt) \cdot \frac{1}{\parabeta} \,
			\Big\{ \gradtheta \rewardtheta(\prompt, \response) - \Exp_{\responsenew \sim \policytheta(\cdot \mid \prompt)}\big[ \gradtheta \rewardtheta(\prompt, \responsenew) \big] \Big\} \bigg]   \\
			& \; = \; \frac{1}{\parabeta} \,\Exp_{\prompt \sim \promptdistr, \, \response \sim \policytheta(\cdot \mid \prompt)} \bigg[ \rewardstar(\prompt, \response) 
			\Big\{ \gradtheta \rewardtheta(\prompt, \response) - \Exp_{\responsenew \sim \policytheta(\cdot \mid \prompt)}\big[ \gradtheta \rewardtheta(\prompt, \responsenew) \big] \Big\} \bigg] \, .
		\end{align*}
		This completes the verification of equation~\eqref{eq:grad_return} from \Cref{lemma:grad_scalarvalue}.
		
		
		
		\paragraph{Proof of Equality~\eqref{eq:grad_KL} from \Cref{lemma:grad_scalarvalue_full}:}
		
		Recall the definition of the KL divergence
		\begin{align*}
			\kull{\policytheta}{\policyref}
			\; = \; \Exp_{\prompt \sim \promptdistr} 
			\bigg[ \int_{\ResponseSp} \policytheta(\diff \response \mid \prompt)
			\log \bigg( \frac{\policytheta(\response \mid \prompt)}{\policyref(\response \mid \prompt)} \bigg) \bigg] \, .
		\end{align*}
		Applying the chain rule, we obtain
		\begin{align}
			\gradtheta \kull{\policytheta}{\policyref}
			& \, = \, \Exp_{\prompt \sim \promptdistr}  \bigg[ \int_{\ResponseSp} \gradtheta \policytheta(\diff \response \mid \prompt) \,
			\log \bigg( \frac{\policytheta(\response \mid \prompt)}{\policyref(\response \mid \prompt)}\bigg) \bigg]  
			\label{eq:grad_KL2}
			+ \Exp_{\prompt \sim \promptdistr}  \bigg[ \int_{\ResponseSp} 
			\gradtheta \policytheta(\diff \response \mid \prompt) \bigg] \, .
		\end{align}
		
		Since the policy integrates to $1$, i.e., $\int_{\ResponseSp} 
		\policytheta(\diff \response \mid \prompt) = 1$, it always holds that
		\begin{align}
			\label{eq:int_grad_policy}
			\int_{\ResponseSp} 
			\gradtheta \policytheta(\diff \response \mid \prompt)
			\; = \; \gradtheta \int_{\ResponseSp} 
			\policytheta(\diff \response \mid \prompt)
			\; = \; 0 \, ,
		\end{align}
		i.e., the second term on the right-hand side of \eqref{eq:grad_KL2} is zero.
		Using the expression \eqref{eq:policyfromreward}, we take the logarithm
		\begin{align}
			\label{eq:grad_KL0}
			\log \bigg( \frac{\policytheta(\response \mid \prompt)}{\policyref(\response \mid \prompt)} \bigg)
			\; = \; \frac{1}{\parabeta} \, \rewardtheta(\prompt, \response) - \log \Partitiontheta (\prompt) \, .
		\end{align}
		Combining equations~\eqref{eq:int_grad_policy} and \eqref{eq:grad_KL0}, we get
		\begin{align}
			& \int_{\ResponseSp} \gradtheta \policytheta(\diff \response \mid \prompt) \,
			\log \bigg( \frac{\policytheta(\response \mid \prompt)}{\policyref(\response \mid \prompt)}\bigg)  \notag  \\
			& \; = \; \frac{1}{\parabeta} \int_{\ResponseSp} \rewardtheta(\prompt, \response) \, \gradtheta \policytheta(\diff \response \mid \prompt) \; - \; \log \Partitiontheta(\prompt) \int_{\ResponseSp} \gradtheta \policytheta(\diff \response \mid \prompt)  \notag  \\
			\label{eq:grad_KL1}
			& \; = \; \frac{1}{\parabeta} \int_{\ResponseSp} \rewardtheta(\prompt, \response) \, \gradtheta \policytheta(\diff \response \mid \prompt) \, .
		\end{align}
		
		Now, similar to the proof of equation \eqref{eq:grad_return}, we derive
		\begin{align*}
			\mbox{RHS of \eqref{eq:grad_KL2}}
			& \; = \; \frac{1}{\parabeta} \, \Exp_{\prompt \sim \promptdistr} \bigg[ \int_{\ResponseSp} \rewardtheta(\prompt, \response) \, \gradtheta \policytheta(\diff \response \mid \prompt) \bigg]  \\
			& \; = \; \frac{1}{\parabeta^2} \,\Exp_{\prompt \sim \promptdistr, \, \response \sim \policytheta(\cdot \mid \prompt)} \bigg[ \rewardtheta(\prompt, \response) 
			\Big\{ \gradtheta \rewardtheta(\prompt, \response) - \Exp_{\responsenew \sim \policytheta(\cdot \mid \prompt)}\big[ \gradtheta \rewardtheta(\prompt, \responsenew) \big] \Big\} \bigg] \, ,
		\end{align*}
		which verifies equality~\eqref{eq:grad_KL} from \Cref{lemma:grad_scalarvalue_full}.
		

		
		
		
		
%%%%%%%%%%%%%%%%%%%%%%%%%%%%%%%%%%%%%%%%%%%%%%%%%%%%%%%%%%%%%%%%%%%%%%%%%%%%%%%%%%%%%%%%%%%%



	\subsubsection{Proof of Lemma~\ref{lemma:grad_loss} \yaqidone}
	\label{sec:proof:lemma:grad_loss}
	
	In this section, we prove a full version of \Cref{lemma:grad_loss} as stated in \Cref{lemma:grad_loss_full} below. Equation~\eqref{eq:gradLoss_BT_0} from \Cref{lemma:grad_loss} follows directly as a straightforward corollary.
	
	In \Cref{lemma:grad_loss_full}, we consider a general class of distributions parameterized by $\paratheta$ that models the binary preference \mbox{$\Probtheta(\responseone \succ \responsetwo \mid \prompt)$}. The negative log-likelihood function is defined as
	\begin{align*}
		\Loss(\theta) = - \Exp_{\prompt \sim \promptdistr; \; (\responseone, \responsetwo) \sim \responsedistr(\cdot \mid \prompt)} \Big[ \weight(\prompt) \cdot \log \Probtheta( \responsewin \succ \responselose \bigm| \prompt) \Big] \, .
	\end{align*}
	The Bradley-Terry (BT) model described in equation~\eqref{eq:BT} and the corresponding loss function~$\Loss(\paratheta)$ in equation~\eqref{eq:Loss0} represent a special case of this general framework.
	
	\begin{lemma}[Gradient of the loss function $\Loss(\paratheta)$, full version]
		\label{lemma:grad_loss_full}
		\begin{subequations}
			For a general distribution class $\{ \Probtheta \}$, the gradient of $\Loss(\paratheta)$ with respect to $\paratheta$ is given by
			\begin{multline}
				\label{eq:gradLoss_general}
				\gradtheta \Loss(\paratheta) \; = \; - \, \Exp_{ \prompt \sim \promptdistr; \; (\responseone, \responsetwo) \sim \responsedistravg(\cdot \mid \prompt) }
				\bigg[ \, \weight(\prompt) \cdot \Big\{ \Prob\big( \responseone \succ \responsetwo \bigm| \prompt \big) - \Probtheta \big( \responseone \succ \responsetwo \bigm| \prompt \big) \Big\} \\
				\cdot \frac{\gradtheta \Probtheta( \responseone \succ \responsetwo \mid \prompt )}{\Probtheta( \responseone \succ \responsetwo \mid \prompt ) \, \Probtheta( \responsetwo \succ \responseone \mid \prompt )} \, \bigg] \, ,
			\end{multline}
			where $\responsedistravg$ is the average distribution defined in equation~\eqref{eq:def_responsedistravg_0}.
			Specifically, for the Bradley-Terry (BT) model where
			\begin{align*}
				\Probtheta \big( \responseone \succ \responsetwo \bigm| \prompt \big)
				\; = \; \sigmoid \big( \rewardtheta(\prompt, \responseone) - \rewardtheta(\prompt, \responsetwo) \big)
				\; = \; \bigg\{ 1 + \bigg( \frac{(\policytheta/\policyref)(\responsetwo \mid \prompt)}{(\policytheta/\policyref)(\responseone \mid \prompt)} \bigg)^{\parabeta} \bigg\}^{-1} \, ,
			\end{align*}
			the gradient of $\Loss(\paratheta)$ becomes
			\begin{multline}
				\label{eq:gradLoss_BT}
				\gradtheta \Loss(\paratheta) \; = \; - \, \Exp_{\prompt \sim \promptdistr; \; (\responseone, \, \responsetwo) \sim \responsedistravg(\cdot \mid \prompt)}
				\bigg[ \, \weight(\prompt) \cdot \Big\{ \sigmoid \big( \rewardstar(\context, \responseone) - \rewardstar(\context, \responsetwo) \big) - \sigmoid \big( \rewardtheta(\context, \responseone) - \rewardtheta(\context, \responsetwo) \big) \Big\} \\ 
				\cdot \big\{ \gradtheta \rewardtheta(\prompt, \responseone) - \gradtheta \rewardtheta(\prompt, \responsetwo) \big\} \bigg] \, .
			\end{multline}
		\end{subequations}
	\end{lemma}
	
	
	For notational simplicity, we focus on the proof for the case where the weight function $\weight(\prompt) = 1$. The results for a general weight function $\weight(\prompt) > 0$ can be derived in a similar manner.
	
	Recall that the negative log-likelihood function $\Loss(\paratheta)$ is defined as
	\begin{align*}
		\Loss(\paratheta) & \; = \;
		\Exp \Big[ - \log \Probtheta\big( \responsewin \succ \responselose \bigm| \prompt \big) \Big] \, .
	\end{align*}
	Based on the data generation mechanism, we can expand the expectation in $\Loss(\paratheta)$ as
%	\begin{subequations}
	\begin{align}
		\Loss(\paratheta)
		& \; = \; \Exp_{\prompt \sim \promptdistr; \; (\responseone, \, \responsetwo) \sim \responsedistr(\cdot \mid \prompt)}
		\Big[ \, \Prob\big( \responseone \succ \responsetwo \bigm| \prompt \big) \cdot \big\{ - \log \Probtheta \big( \responseone \succ \responsetwo \bigm| \prompt \big) \big\}  \notag  \\
		\label{eq:Loss0}
		& \qquad \qquad \qquad \qquad \qquad + \Prob\big( \responsetwo \succ \responseone \bigm| \prompt \big) \cdot \big\{ - \log \Probtheta\big( \responsetwo \succ \responseone \bigm| \prompt \big) \big\} \Big] \, .
	\end{align}
	Notice that we can exchange the roles of $\responseone$ and $\responsetwo$ in the expectation above. This means that we can equivalently express the expectation using the pair $(\responsetwo, \responseone) \sim \responsedistr(\cdot \mid \prompt)$.
	This symmetry allows us to replace $\responsedistr$ in equation~\eqref{eq:Loss0} with the average distribution $\responsedistravg$ as defined in equation~\eqref{eq:def_responsedistravg_0}. \\
	
	Next, we take the gradient of the loss function $\Loss(\paratheta)$ with respect to the parameter $\paratheta$ and obtain
	\begin{align*}
		\gradtheta \Loss(\paratheta)
		& \; = \; \Exp_{\prompt \sim \promptdistr, \, (\responseone, \, \responsetwo) \sim \responsedistravg(\cdot \mid \prompt)}
		\bigg[ \, \frac{\Prob( \responseone \succ \responsetwo \mid \prompt )}{\Probtheta ( \responseone \succ \responsetwo \mid \prompt )} \cdot \big\{ - \gradtheta \Probtheta( \responseone \succ \responsetwo \mid \prompt ) \big\}   \\
		& \qquad \qquad \qquad \qquad \qquad + \frac{\Prob( \responsetwo \succ \responseone \mid \prompt )}{\Probtheta( \responsetwo \succ \responseone \mid \prompt )} \cdot \big\{ - \gradtheta \Probtheta( \responsetwo \succ \responseone \mid \prompt ) \big\} \, \bigg] \, .
	\end{align*}
	Note that $\Prob\big( \responsetwo \succ \responseone \bigm| \prompt\big) = 1 - \Prob\big( \responseone \succ \responsetwo \bigm| \prompt\big)$ and $\Probtheta \big( \responsetwo \succ \responseone \bigm| \prompt\big) = 1 - \Probtheta \big( \responseone \succ \responsetwo \bigm| \prompt\big)$.
	Using this, we can rewrite the gradient as
	\begin{align*}
		& \gradtheta \Loss(\paratheta)  \\
		& \; = \;
		\Exp_{\prompt \sim \promptdistr; \; (\responseone, \, \responsetwo) \sim \responsedistravg(\cdot \mid \prompt)}
		\bigg[ \bigg\{ \frac{1 - \Prob( \responseone \succ \responsetwo \mid \prompt )}{1 - \Probtheta( \responseone \succ \responsetwo \mid \prompt )} - \frac{\Prob( \responseone \succ \responsetwo \mid \prompt )}{\Probtheta ( \responseone \succ \responsetwo \mid \prompt )} \bigg\} \cdot \gradtheta \Probtheta\big( \responseone \succ \responsetwo \bigm| \prompt \big) \bigg] \, .
	\end{align*}
	We simplify the expression further to obtain
	\begin{multline*}
		\gradtheta \Loss(\paratheta)
		\; = \;
		\Exp_{\prompt \sim \promptdistr; \; (\responseone, \, \responsetwo) \sim \responsedistravg(\cdot \mid \prompt)}
		\bigg[ \Big\{ \Probtheta \big( \responseone \succ \responsetwo \bigm| \prompt \big) - \Prob \big( \responseone \succ \responsetwo \bigm| \prompt \big) \Big\} \\ \cdot \frac{\gradtheta \Probtheta( \responseone \succ \responsetwo \mid \prompt )}{\Probtheta( \responseone \succ \responsetwo \mid \prompt ) \, \Probtheta( \responsetwo \succ \responseone \mid \prompt )} \bigg] \, .
	\end{multline*}
	This establishes equation~\eqref{eq:gradLoss_general} from \Cref{lemma:grad_loss}. \\
	
	As for the Bradley-Terry (BT) model, we use the equality
	\begin{align*}
		\divsigmoid(z) \; = \; \frac{1}{(1 + \exp(-z))(1 + \exp(z))} \; = \; \sigmoid(z) \, \sigmoid(-z)
		\qquad \mbox{for any $z \in \Real$}
	\end{align*}
	to derive the following expression
	\begin{align}
		\label{eq:grad_reward}
		\frac{\gradtheta \Probtheta( \responseone \succ \responsetwo \mid \prompt )}{\Probtheta( \responseone \succ \responsetwo \mid \prompt ) \, \Probtheta( \responsetwo \succ \responseone \mid \prompt )}
		\; = \; \gradtheta \rewardtheta(\prompt, \responseone) - \gradtheta \rewardtheta(\prompt, \responsetwo) \, .
	\end{align}
	By substituting this gradient expression from equation~\eqref{eq:grad_reward} into equation~\eqref{eq:gradLoss_general}, we directly obtain equation~\eqref{eq:gradLoss_BT}, thereby completing the proof of \Cref{lemma:grad_loss}.
	
	
%%%%%%%%%%%%%%%%%%%%%%%%%%%%%%%%%%%%%%%%%%%%%%%%%%%%%%%%%%%%%%%%%%%%%%%%%%%%%%%

	\subsection{Proof of Auxiliary Results for Theorem~\ref{thm:asymp} \yaqidone}
	\label{sec:proof:thm:asymp_aux}
	
	In this section, we present the detailed proofs of the supporting lemmas used in the proof of \Cref{thm:asymp}. 
	We begin in \Cref{sec:proof:eq:master_cond_proof} by establishing condition~\eqref{eq:master_cond_proof}, which is crucial for the valid application of the master theorem for $Z$-estimators. Following this, in \Cref{sec:proof:lemma:hess_loss}, we compute the Hessian matrix $\hesstheta \Loss(\parathetastar)$ explicitly. Finally, in \Cref{sec:proof:lemma:grad_loss_stat}, we derive the asymptotic distribution of the gradient~$\gradtheta \Losshat(\parathetastar)$.
	
	
%%%%%%%%%%%%%%%%%%%%%%%%%%%%%%%%%%%%%%%%%%%%%%%%%%%%%%%%%%%%%%%

	\subsubsection{Proof of Condition~\eqref{eq:master_cond_proof}}
	\label{sec:proof:eq:master_cond_proof}
	We begin by rewriting the left-hand side of equation~\eqref{eq:master_cond_proof} as follows:
	\begin{align}
		\Delta
		& \; \defn \; \sqrt{n} \, \big\{ \gradtheta \Losshat (\parathetahat) - \gradtheta \Loss(\parathetahat) \big\} - \sqrt{n} \, \big\{ \gradtheta \Losshat (\parathetastar) - \gradtheta \Loss (\parathetastar) \big\}   \notag  \\
		& \; = \; \sqrt{n} \, \big\{ \gradtheta \Losshat (\parathetahat) - \gradtheta \Losshat(\parathetastar) \big\} - \sqrt{n} \, \big\{ \gradtheta \Loss (\parathetahat) - \gradtheta \Loss (\parathetastar) \big\} \, .
		\label{eq:master_0}
	\end{align}
	We then leverage the smoothness properties of the function $\rewardtheta$, which guarantee the following approximations:
	\begin{subequations}
		\begin{align}
			\label{eq:gradLosshat_smooth}
			\gradtheta \Losshat(\parathetahat) - \gradtheta \Losshat(\parathetastar) & \; = \; \hesstheta \Losshat(\parathetastar) \, (\parathetahat - \parathetastar) + \smallop \big( \norm{\parathetahat - \parathetastar}_2 \big) \, ,  \\
			\label{eq:gradLoss_smooth}
			\gradtheta \Loss(\parathetahat) - \gradtheta \Loss(\parathetastar) & \; = \; \hesstheta \Loss(\parathetastar) \, (\parathetahat - \parathetastar) + \smallop \big( \norm{\parathetahat - \parathetastar}_2 \big) \, .
		\end{align}
	\end{subequations}
	Assuming these equalities~\eqref{eq:gradLosshat_smooth} and~\eqref{eq:gradLoss_smooth} hold, we substitute them into equation~\eqref{eq:master_0}, leading to
	\begin{align}
		\Delta
		& \; = \; \sqrt{n} \, \big\{ \hesstheta \Losshat (\parathetastar) \, (\parathetahat - \parathetastar) + \smallop( \norm{\parathetahat - \parathetastar}_2 ) \big\} - \sqrt{n} \, \big\{ \hesstheta \Loss (\parathetastar) \, (\parathetahat - \parathetastar) + \smallop( \norm{\parathetahat - \parathetastar}_2 ) \big\}  \notag \\
		& \; = \; \sqrt{n} \, \big\{ \hesstheta \Losshat (\parathetastar) - \hesstheta \Loss (\parathetastar) \big\} (\parathetahat - \parathetastar) + \smallop \big( 1 + \sqrt{n} \, \norm{ \parathetahat - \parathetastar }_2 \big) \, .
		\label{eq:master_1}
	\end{align}
	Using the law of large numbers, we know that $\hesstheta \Losshat (\parathetastar) \convergep \hesstheta \Loss (\parathetastar)$, which implies
	\begin{align*}
		\sqrt{n} \, \big\{ \hesstheta \Losshat (\parathetastar) - \hesstheta \Loss (\parathetastar) \big\} (\parathetahat - \parathetastar) \; = \; \smallop \big( \sqrt{n} \, \norm{ \parathetahat - \parathetastar }_2 \big) \, .
	\end{align*}
	Therefore, we conclude that
	\begin{align*}
		\Delta \; = \; \smallop \big( 1 + \sqrt{n} \, \norm{ \parathetahat - \parathetastar }_2 \big)
	\end{align*}
	as claimed in equation~\eqref{eq:master_cond_proof}.
	
	The only remaining task is to establish the validity of equalities~\eqref{eq:gradLosshat_smooth} and~\eqref{eq:gradLoss_smooth}.
	
	
	\paragraph{Proof of Equalities~\eqref{eq:gradLosshat_smooth}~and~\eqref{eq:gradLoss_smooth}:}
	
	We express the loss function $\Losshat(\paratheta)$ in the form
	\begin{align*}
		\Losshat(\paratheta) \; \defn \;
		\frac{1}{\numobs} \sum_{i=1}^{\numobs} \weight(\prompti{i}) \cdot \lliketheta\big(\prompti{i}, \responsewini{i}, \responselosei{i}\big) \, ,
	\end{align*}
	where the function $\lliketheta$ is defined as
	\begin{align*}
		\lliketheta(\prompt, \responsei{1}, \responsei{2})
		\; = \; - \log \sigmoid \big( \rewardtheta(\prompt, \responsei{1}) - \rewardtheta(\prompt, \responsei{2}) \big) \, .
	\end{align*}
	We then calculate the gradient $\gradtheta \lliketheta$ and $\hesstheta \lliketheta$ as follows:
	\begin{align*}
		\gradtheta \lliketheta(\prompt, \responsei{1}, \responsei{2})
		& \; = \; \sigmoid\big( \rewardtheta(\prompt, \responsei{2}) - \rewardtheta(\prompt, \responsei{1}) \big) \cdot \big\{ \gradtheta \rewardtheta(\prompt, \responsei{2}) - \gradtheta \rewardtheta(\prompt, \responsei{1})  \big\} \qquad \mbox{and}  \\
		\hesstheta \lliketheta(\prompt, \responsei{1}, \responsei{2})
		& \; = \; \divsigmoid\big( \rewardtheta(\prompt, \responsei{2}) - \rewardtheta(\prompt, \responsei{1}) \big) \\
        & \qquad \quad
        \cdot \big\{ \gradtheta \rewardtheta(\prompt, \responsei{2}) - \gradtheta \rewardtheta(\prompt, \responsei{1}) \big\} \big\{ \gradtheta \rewardtheta(\prompt, \responsei{2}) - \gradtheta \rewardtheta(\prompt, \responsei{1}) \big\}^{\top}  \\
		& \quad + \sigmoid\big( \rewardtheta(\prompt, \responsei{2}) - \rewardtheta(\prompt, \responsei{1}) \big) \cdot \big\{ \hesstheta \rewardtheta(\prompt, \responsei{2}) - \hesstheta \rewardtheta(\prompt, \responsei{1})  \big\} \, .
	\end{align*}
	When the reward function $\rewardtheta(\prompt, \response)$, along with its gradient $\gradtheta \rewardtheta(\prompt, \response)$ and Hessian $\hesstheta \rewardtheta(\prompt, \response)$, is uniformly bounded and Lipschitz continuous with respect to $\paratheta$ for all $(\prompt, \response) \in \PromptSp \times \ResponseSp$, it guarantees that the Hessian of the loss function, $\hesstheta \lliketheta$, is also Lipschitz continuous. This holds with some constant $\Liphess > 0$ across all $(\prompt, \response) \in \PromptSp \times \ResponseSp$, as demonstrated below:
	\begin{align*}
		\norm[\big]{\hesstheta \lliketheta (\prompt, \responsei{1}, \responsei{2}) - \hesstheta \llikethetastar (\prompt, \responsei{1}, \responsei{2})}_2
		\; \leq \; \Liphess \cdot \norm{\paratheta - \parathetastar}_2 \, .
	\end{align*}
	From this Lipschitz property, we deduce
	\begin{align*}
		\norm[\big]{\gradtheta \lliketheta (\prompt, \responsei{1}, \responsei{2}) - \gradtheta \llikethetastar (\prompt, \responsei{1}, \responsei{2}) - \hesstheta \llikethetastar (\prompt, \responsei{1}, \responsei{2}) \, (\paratheta - \parathetastar)}_2 \; \leq \; \frac{\Liphess}{2} \cdot \norm{\paratheta - \parathetastar}_2^2
	\end{align*}
	and further derive
	\begin{align*}
		\norm[\big]{\gradtheta \Losshat(\paratheta) - \gradtheta \Losshat(\parathetastar) - \hesstheta \Losshat(\parathetastar) \, (\paratheta - \parathetastar)}_2 & \; \leq \; \frac{\Liphess \, \supnorm{\weight}}{2} \cdot \norm{\paratheta - \parathetastar}_2^2 \, ,  \\
		\norm[\big]{\gradtheta \Loss(\paratheta) - \gradtheta \Loss(\parathetastar) - \hesstheta \Loss(\parathetastar) \, (\paratheta - \parathetastar)}_2 & \; \leq \; \frac{\Liphess \, \supnorm{\weight}}{2} \cdot \norm{\paratheta - \parathetastar}_2^2 \, .
	\end{align*}
	Finally, under the condition that $\parathetahat \convergep \parathetastar$, these results simplify to the expressions given in equations~\eqref{eq:gradLosshat_smooth} and~\eqref{eq:gradLoss_smooth}, as previously claimed.
	
	
%%%%%%%%%%%%%%%%%%%%%%%%%%%%%%%%%%%%%%%%%%%%%%%%%%%%%%%%%%%%%%%
	
	\subsubsection{Proof of Lemma~\ref{lemma:hess_loss}, Explicit Form of Hessian $\hesstheta \Loss(\parathetastar)$}
	\label{sec:proof:lemma:hess_loss}
	
	From equation~\eqref{eq:gradLoss_BT_0} in \Cref{lemma:grad_loss}, we recall the explicit formula for the gradient $\gradtheta \Loss(\paratheta)$. Taking the derivative of both sides of equation~\eqref{eq:gradLoss_BT_0}, we obtain
	\begin{align}
		& \begin{aligned} 
		\hesstheta \Loss(\paratheta) \; = \; \Exp_{\prompt \sim \promptdistr; \; (\responseone, \, \responsetwo) \sim \responsedistravg(\cdot \mid \prompt)}
		& \Big[ \, \weight(\prompt) \cdot \divsigmoid \big( \rewardtheta(\context, \responseone) - \rewardtheta(\context, \responsetwo) \big) \\ 
		& \cdot \big\{ \gradtheta \rewardtheta(\prompt, \responseone) - \gradtheta \rewardtheta(\prompt, \responsetwo) \big\} \big\{ \gradtheta \rewardtheta(\prompt, \responseone) - \gradtheta \rewardtheta(\prompt, \responsetwo) \big\}^{\top} \Big] \end{aligned}   \notag  \\
		& \qquad \qquad \quad
		\begin{aligned} 
		- \, \Exp_{\prompt \sim \promptdistr; \; (\responseone, \, \responsetwo) \sim \responsedistravg(\cdot \mid \prompt)}
		\bigg[ \, \weight(\prompt) & \cdot \Big\{ \sigmoid \big( \rewardstar(\context, \responseone) - \rewardstar(\context, \responsetwo) \big) - \sigmoid \big( \rewardtheta(\context, \responseone) - \rewardtheta(\context, \responsetwo) \big) \Big\} \\ 
		& \cdot \big\{ \hesstheta \rewardtheta(\prompt, \responseone) - \hesstheta \rewardtheta(\prompt, \responsetwo) \big\} \bigg] \, .
		\end{aligned}
		\label{eq:hessLoss_0}
	\end{align}
	When we set $\paratheta = \parathetastar$, it follows that $\rewardtheta = \rewardstar$. This simplification eliminates the second term in expression~\eqref{eq:hessLoss_0}, reducing the Hessian matrix to
	\begin{multline*}
		\hesstheta \Loss(\parathetastar) \; = \; \Exp_{\prompt \sim \promptdistr; \; (\responseone, \, \responsetwo) \sim \responsedistravg(\cdot \mid \prompt)}
		\Big[ \, \weight(\prompt) \cdot \divsigmoid \big( \rewardstar(\context, \responseone) - \rewardstar(\context, \responsetwo) \big) \\ 
		\cdot \big\{ \gradtheta \rewardstar(\prompt, \responseone) - \gradtheta \rewardstar(\prompt, \responsetwo) \big\} \big\{ \gradtheta \rewardstar(\prompt, \responseone) - \gradtheta \rewardstar(\prompt, \responsetwo) \big\}^{\top} \Big] \, .
	\end{multline*}
	Substituting the derivative $\divsigmoid$ with its explicit form, $\divsigmoid(z) = \sigmoid(z) \, \sigmoid(-z)$ for any $z \in \Real$, we refine the expression to
	\begin{align*}
		\hesstheta \Loss(\parathetastar) \; = \; \CovOpstar \, ,
	\end{align*}
	where the covariance matrix $\CovOpstar$ is defined in equation~\eqref{eq:def_CovOpstar}.
	This completes the proof of expression~\eqref{eq:hess_loss} from \Cref{lemma:hess_loss}.
	
%%%%%%%%%%%%%%%%%%%%%%%%%%%%%%%%%%%%%%%%%%%%%%%%%%%%%%%%%%%%%%%%%%%%%%%%%%%
	
	\subsubsection{Proof of Lemma~\ref{lemma:grad_loss_stat}, Asymptotic Distribution of Graident $\gradtheta \Losshat(\parathetastar)$}
	\label{sec:proof:lemma:grad_loss_stat}
	
	In this section, we analyze the asymptotic distribution of the gradient $\gradtheta \Losshat(\paratheta)$ at $\paratheta = \parathetastar$, where the loss function $\Losshat(\paratheta)$ is defined as
	\begin{align*}
		\Losshat(\paratheta) \; = \;
		- \frac{1}{\numobs} \sum_{i=1}^{\numobs} \, \weight(\prompt) \cdot \log \sigmoid \Big( \rewardtheta\big(\prompti{i}, \responsewini{i}\big) - \rewardtheta\big(\prompti{i}, \responselosei{i}\big) \Big) \, .
	\end{align*}
	Using the definition of the sigmoid function $\sigmoid$, we calculate that
	\begin{align*}
		( \log \sigmoid(z) )' = \divsigmoid(z) / \sigmoid(z) = \sigmoid(z) \, \sigmoid(-z) / \sigmoid(z) = \sigmoid(-z) \qquad \mbox{for any real number $z \in \Real$}.
	\end{align*}
	This allows us to reformulate $\gradtheta \Losshat(\paratheta)$ as the average of $\numobs$ i.i.d. vectors $\{ \vecgi{i} \}_{i=1}^{\numobs}$:
	\begin{align}
		\label{eq:gradLosshat}
		\gradtheta \Losshat(\paratheta)
		\; = \; \frac{1}{\numobs} \sum_{i=1}^{\numobs} \, \vecgi{i} \, .
	\end{align}
	Here each vector $\vecgi{i} \in \Real^{\Dim}$ is defined as
	\begin{align*}
		\vecgi{i} \; \defn \; \weight(\prompt) \cdot \sigmoid \big( \rewardtheta(\prompti{i}, \responselosei{i}) - \rewardtheta(\prompti{i}, \responsewini{i}) \big) \cdot \big\{ \gradtheta \rewardtheta(\prompti{i}, \responselosei{i}) - \gradtheta \rewardtheta(\prompti{i}, \responsewini{i}) \big\} \, .
	\end{align*}
%	(consistently with expression~\eqref{eq:def_grad}).
	At $\paratheta = \parathetastar$, we denote $\vecgi{i}$ as $\vecgstari{i}$ and $\gradi{i}$ as $\gradstari{i}$. Notably, vector $\vecgi{i}$ can be rewritten as
	\begin{align}
		\label{eq:vecgi2}
		\vecgi{i} 
		& \; = \; \weight(\prompt) \cdot \big\{ \sigmoid \big( \rewardtheta(\prompti{i}, \responseonei{i}) - \rewardtheta(\prompti{i}, \responsetwoi{i}) \big) - \indicator\{ \responseonei{i} = \responsewini{i}, \responsetwoi{i} = \responselosei{i} \} \big\}
		\cdot \gradi{i} \, ,
	\end{align}
	where $\gradi{i}$ is given by
	\begin{align*}
		\gradi{i} \defn \gradtheta \rewardtheta(\prompti{i}, \responseonei{i}) - \gradtheta \rewardtheta(\prompti{i}, \responsetwoi{i}) \, .
	\end{align*}
	From the structure of the BT model, it holds that
	\begin{align*}
		\Exp\big[ \indicator \{ \responseonei{i} = \responsewini{i}, \responsetwoi{i} = \responselosei{i} \} \bigm| \prompti{i} \big] \; = \; \sigmoid \big( \rewardstar(\prompti{i}, \responseonei{i}) - \rewardstar(\prompti{i}, \responsetwoi{i}) \big) \, ,
	\end{align*}
	which implies $\Exp[\vecgstari{i}] = \veczero$.
	
	
	To analyze the asymptotic distribution of $\gradtheta \Losshat(\parathetastar)$, we apply the central limit theorem (CLT) to its empirical form given in equation~\eqref{eq:gradLosshat}. 
%	\yaqiadd{Check the conditions for CLT.}
	By the CLT, we have
	\begin{align}
		\label{eq:gradLoss_CLT}
		\sqrt{\numobs} \, \big( \gradtheta \Losshat(\parathetastar) - \gradtheta \Loss(\parathetastar) \big)
		\; \stackrel{d}{\longrightarrow} \; \Gauss\big(\veczero, \CovOptil \big) \, ,
		\qquad \numobs \rightarrow \infty \, ,
	\end{align}
	where the covariance matrix $\CovOptil \in \Real^{\Dim \times \Dim}$ is given by
	\begin{align*}
		\CovOptil \; \defn \; \Cov(\vecgstari{1}) \; = \; \Exp\big[ \vecgstari{1} (\vecgstari{1})^{\top} \big] \, .
	\end{align*}
	Here we have used the property $\Exp[\vecgstari{i}] = \veczero$ in the second equality.
	
	We now compute the explicit form of the covariance matrix $\CovOptil$. Using the definition of $\vecgi{i}$ from expression~\eqref{eq:vecgi2}, we find that
	\begin{align}
		& \CovOptil \; = \; \Exp\big[ \vecgstari{1} (\vecgstari{1})^{\top} \big] \notag  \\
		& = \; \Exp_{\, \begin{subarray}{l} ~ \\ \prompt \sim \promptdistr; \\ (\responseone, \responsetwo) \sim \responsedistravg(\cdot \mid \prompt)\end{subarray}} \Big[ \, \weight^2(\prompt) \cdot \big\{ \sigmoid \big( \rewardstar(\prompti{1}, \responseonei{1}) - \rewardstar(\prompti{1}, \responsetwoi{1}) \big) - \indicator\{ \responseonei{1} = \responsewini{1}, \responsetwoi{1} = \responselosei{1} \} \big\}^2 \cdot \gradstari{1} (\gradstari{1})^{\top} \Big] \, .
		\label{eq:CovOptil_2}
	\end{align}
	Taking the conditional expectation over the outcomes of winners and losers, and using the relation
	\begin{align*}
		&  \Exp\Big[
		\big\{ \sigmoid \big( \rewardstar(\prompti{1}, \responseonei{1}) - \rewardstar(\prompti{1}, \responsetwoi{1}) \big) - \indicator\{ \responseonei{1} = \responsewini{1}, \responsetwoi{1} = \responselosei{1} \} \big\}^2 \Bigm| \prompti{1}, \responseonei{1}, \responsetwoi{1} \Big]  \\
		& \; = \; \Var \Big( \indicator\{ \responseonei{1} = \responsewini{1}, \responsetwoi{1} = \responselosei{1} \} \Bigm|  \prompti{1}, \responseonei{1}, \responsetwoi{1} \Big)  \\
		& \; = \; \sigmoid \big( \rewardstar(\prompti{i}, \responseonei{i}) - \rewardstar(\prompti{i}, \responsetwoi{i}) \big) \, \sigmoid \big( \rewardstar(\prompti{i}, \responsetwoi{i}) - \rewardstar(\prompti{i}, \responseonei{i}) \big) \, ,
	\end{align*}
	we reduce equation~\eqref{eq:CovOptil_2} to
	\begin{align*}
		\CovOptil
		& \; = \; \Exp_{\prompt \sim \promptdistr; \; (\responseone, \responsetwo) \sim \responsedistravg(\cdot \mid \prompt)} \Big[ \, \weight^2(\prompt) \cdot \Var \big( \indicator\{ \responseonei{1} = \responsewini{1}, \responsetwoi{1} = \responselosei{1} \} \bigm|  \prompti{1}, \responseonei{1}, \responsetwoi{1} \big) \cdot \gradstari{1} (\gradstari{1})^{\top} \Big] \, .
	\end{align*}
	Bounding the weight function $\weight(\prompt)$ by its uniform bound $\supnorm{\weight}$, we simplify further:
	\begin{align*}
		\CovOptil
        & \; \preceq \; \supnorm{\weight} \cdot \Exp\Big[ \, \weight(\prompt) \cdot \Var \big( \indicator\{ \responseonei{1} = \responsewini{1}, \responsetwoi{1} = \responselosei{1} \} \bigm|  \prompti{1}, \responseonei{1}, \responsetwoi{1} \big) \cdot \gradstari{1} (\gradstari{1})^{\top} \Big] \, .
    \end{align*}
    This ultimately reduces to
    \begin{align}
    	\label{eq:CovOptil_ub}
         \CovOptil & \; \preceq \; \supnorm{\weight} \cdot \CovOpstar
	\end{align}
	where $\CovOpstar$ is defined in equation~\eqref{eq:def_CovOpstar}.
    
    Finally, by combining equations~\eqref{eq:gradLoss_CLT} and~\eqref{eq:CovOptil_ub}, we establish the asymptotic normality of $\gradtheta \Losshat(\parathetastar)$ and complete the proof of \Cref{lemma:grad_loss_stat}.
    
    
%%%%%%%%%%%%%%%%%%%%%%%%%%%%%%%%%%%%%%%%%%%%%%%%%%%%%%%%%%%%%%
	
	\subsection{Proof of Auxiliary Results for Theorem~\ref{lemma:hess_scalarvalue} \yaqidone}
	\label{sec:proof:lemma:hess_scalarvalue_aux}
	
	This section contains the proofs of the auxiliary results supporting \Cref{lemma:hess_scalarvalue}. In \Cref{sec:proof:eq:hessscalarvalue}, we derive the explicit form of the Hessian $ \hesstheta \scalarvalue(\policystar) $. \Cref{sec:proof:gap_distr} rigorously establishes the asymptotic distribution of the value gap (equation~\eqref{eq:gap_distr}). Finally, \Cref{sec:proof:chisqtail} proves the tail bound~\eqref{eq:gap_bd} on the chi-square distribution $ \chisquare_{\Dim} $.
	
	\subsubsection{Proof of Equation~\eqref{eq:hessscalarvalue} from Theorem~\ref{lemma:hess_scalarvalue}, Explicit Form of Hessian $\hesstheta \scalarvalue(\policystar)$}
	\label{sec:proof:eq:hessscalarvalue}
	
	We begin by differentiating expression~\eqref{eq:grad_scalarvalue0} for the gradient $\gradtheta \scalarvalue(\policytheta)$ to obtain the Hessian matrix $\hesstheta \scalarvalue(\policytheta)$. The resulting expression can be written as
	\begin{align*}
		\hesstheta \scalarvalue(\policytheta)
		\; = \; \GammaMt_1 + \GammaMt_2 + \GammaMt_3 \, ,
	\end{align*}
	where the terms are defined as follows:
	\begin{align*}
		\GammaMt_1
		& \; \defn \; \frac{1}{\parabeta} \, \Exp_{\prompt \sim \promptdistr}
		\bigg[ \int_{\ResponseSp} \big\{ \rewardstar(\prompt, \response) - \rewardtheta(\prompt, \response) \big\} \\
		& \qquad \qquad \qquad \qquad \quad
        \cdot \Big\{ \gradtheta \rewardtheta(\prompt, \response) - \Exp_{\responsenew \sim \policytheta(\cdot \mid \prompt)}\big[ \gradtheta \rewardtheta(\prompt, \responsenew) \big] \Big\} \, \gradtheta \policytheta(\diff \response \mid \prompt)^{\top} \bigg] \, ,  \\
		\GammaMt_2
		& \; \defn \; - \frac{1}{\parabeta} \, \Exp_{\prompt \sim \promptdistr, \,  \response \sim \policytheta(\cdot \mid \prompt)}
		\bigg[ \Big\{ \gradtheta \rewardtheta(\prompt, \response) - \Exp_{\responsenew \sim \policytheta(\cdot \mid \prompt)}\big[ \gradtheta \rewardtheta(\prompt, \responsenew) \big] \Big\} \, \gradtheta \rewardtheta(\prompt, \response)^{\top} \bigg] \, ,  \\
		\GammaMt_3
		& \defn \frac{1}{\parabeta} \, \Exp_{\prompt \sim \promptdistr, \,  \response \sim \policytheta(\cdot \mid \prompt)}
		\bigg[ \big\{ \rewardstar(\prompt, \response) - \rewardtheta(\prompt, \response) \big\} \Big\{ \hesstheta \rewardtheta(\prompt, \response) - \gradtheta \Exp_{\responsenew \sim \policytheta(\cdot \mid \prompt)}\big[ \gradtheta \rewardtheta(\prompt, \responsenew) \big] \Big\} \bigg] \, .
	\end{align*}
	
	At the point $\paratheta = \parathetastar$, we know that $\rewardtheta = \rewardstar$. This simplifies the expression significantly:
	\begin{align*}
	\GammaMt_1 = \veczero \quad \text{and} \quad \GammaMt_3 = \veczero.
	\end{align*}
	Therefore, only term $\GammaMt_2$ contributes to the Hessian, and it further reduces to
	\begin{align*}
		\GammaMt_2
		& \; = \; - \frac{1}{\parabeta} \, \Exp_{\prompt \sim \promptdistr, \,  \response \sim \policytheta(\cdot \mid \prompt)}
		\Big[ \gradtheta \rewardtheta(\prompt, \response) \, \gradtheta \rewardtheta(\prompt, \response)^{\top} \Big]  \\
		& \quad + \frac{1}{\parabeta} \, \Exp_{\prompt \sim \promptdistr}
		\Big[ \Exp_{\responsenew \sim \policytheta(\cdot \mid \prompt)}\big[ \gradtheta \rewardtheta(\prompt, \responsenew) \big] \, \Exp_{\response \sim \policytheta(\cdot \mid \prompt)}\big[\gradtheta \rewardtheta(\prompt, \response)\big]^{\top} \Big] \\
		& \; = \; - \frac{1}{\parabeta} \, \Exp_{\prompt \sim \promptdistr}
		\Big[ \Cov_{\response \sim \policytheta(\cdot \mid \prompt)} \big[ \gradtheta \rewardtheta(\prompt, \response) \bigm| \prompt \big] \Big]  \, .
	\end{align*}
	From this simplification, we deduce
	\begin{align*}
		\hesstheta \scalarvalue(\policystar) \; = \;
		- \frac{1}{\parabeta} \, \Exp_{\prompt \sim \promptdistr} \Big[ \Cov_{\response \sim \policystar(\cdot \mid \prompt)} \big[ \gradtheta \rewardstar(\prompt, \response) \bigm| \prompt \big] \Big] \, ,
	\end{align*}
	which establishes equation~\eqref{eq:hessscalarvalue} as stated in \Cref{lemma:hess_scalarvalue}.


%%%%%%%%%%%%%%%%%%%%%%%%%%%%%%%%%%%%%%%%%%%%%%%%%%%%%%%%%%%%%%%%%%%%%%%
	
	\subsubsection{Proof of the Asymptotic Distribution in Equation~\eqref{eq:gap_distr}}
	\label{sec:proof:gap_distr}
	
	The goal of this part is to establish the asymptotic distribution of $\numobs \{ \scalarvalue(\policystar) - \scalarvalue(\policyhat) \}$, as stated in equation~\eqref{eq:gap_distr} from \Cref{sec:proof:lemma:hess_scalarvalue}. To achieve this, we first recast the value gap into the product of two terms and then invoke Slutsky’s theorem.
	
	We start by writing
	\begin{align}
		\numobs \cdot \{ \scalarvalue(\policystar) - \scalarvalue(\policyhat) \}
		\; = \;  \underbrace{\numobs \cdot (\parathetahat - \parathetastar)^{\top} \HessMt \, (\parathetahat - \parathetastar)}_{\Un}
		\cdot \underbrace{\frac{\scalarvalue(\policystar) - \scalarvalue(\policyhat)}{(\parathetahat - \parathetastar)^{\top} \HessMt \, (\parathetahat - \parathetastar)}}_{\Vn} \, .
	\end{align}
	By isolating \(\Un\) and \(\Vn\) in this way, we can handle their limiting behaviors separately:
	\begin{subequations}
	\begin{align}
		& \Un \; \converged \; \vecz^{\top} \CovOmega^{\frac{1}{2}} \HessMt \CovOmega^{\frac{1}{2}} \vecz \qquad \mbox{with $\vecz \sim \Gauss(\veczero, \IdMt)$},  \label{eq:Un_distr} \\
		& \Vn \; \convergep \; \frac{1}{2} \, .  \label{eq:Vn_distr}
	\end{align}
	\end{subequations}
	If these two results are established, the desired asymptotic distribution of the value gap, as given in equation~\eqref{eq:gap_distr}, follows directly from Slutsky’s theorem.
	
	To complete the proof, we proceed to verify equations~\eqref{eq:Un_distr} and~\eqref{eq:Vn_distr}. It is worth noting that equation~\eqref{eq:Un_distr} is a straightforward corollary of \Cref{thm:asymp}, so the main task is to establish the convergence result in equation~\eqref{eq:Vn_distr}.
	
	
	\paragraph{Proof of Equation~\eqref{eq:Vn_distr}:}
	
	Since $\CovOpstar$ is nonsingular, the matrix $\HessMt = (\Partitionthetabar / \parabeta) \cdot \CovOpstar$ is also nonsingular.
	From equation~\eqref{eq:Taylor_scalarvalue}, we know that for any $\varepsilon \in (0, 1)$, there exists a threshold $\eta(\varepsilon) > 0$ such that whenever $\norm{\paratheta - \parathetastar}_2 \leq \eta(\varepsilon)$, the following inequality holds:
	\begin{align*}
		\Big( \frac{1}{2} - \varepsilon \Big) \, (\paratheta - \parathetastar)^{\top} \HessMt \, (\paratheta - \parathetastar)
		\; \leq \; \scalarvalue(\policystar) - \scalarvalue(\policytheta)
		\; \leq \; \Big( \frac{1}{2} + \varepsilon \Big) \, (\paratheta - \parathetastar)^{\top} \HessMt \, (\paratheta - \parathetastar) \, .
	\end{align*}
	This can be reformulated as
	\begin{align*}
		\abs[\Big]{\Vn - \frac{1}{2}} \; \leq \; \varepsilon \, .
	\end{align*}
	Next, under the condition that $\parathetahat \convergep \parathetastar$, for any $\delta > 0$, there exists an integer $N(\varepsilon, \delta) \in \Intpos$ such that for any $\numobs \geq N(\varepsilon, \delta)$,
	\begin{align*}
		\Prob \big\{ \norm{\parathetahat - \parathetastar}_2 > \eta(\varepsilon) \big\} \leq \delta \, .
	\end{align*} 
	Therefore, for any $\numobs \geq N(\varepsilon, \delta)$, we can conclude
	\begin{align*}
		\Prob \bigg\{ \abs[\Big]{\Vn - \frac{1}{2}} \; > \; \varepsilon \bigg\} \; \leq \; \delta \, .
	\end{align*}
	In simpler terms, $\Vn \convergep \frac{1}{2}$, which establishes equation~\eqref{eq:Vn_distr}.
	


%%%%%%%%%%%%%%%%%%%%%%%%%%%%%%%%%%%%%%%%%%%%%%%%%%%%%%%%%%%%%%%%%%%%%%%

	\subsubsection{Proof of the Tail Bound in Equation~\eqref{eq:gap_bd}}
	\label{sec:proof:chisqtail}
	
	We now establish the tail bound
	\begin{align}
		\label{eq:chi_tail}
		\Prob\big\{ \chisquare_\Dim > (1 + \varepsilon) \, \Dim \big\}
		\;\leq\;
		\exp\Big\{-\frac{\Dim}{2} \bigl(\varepsilon - \log(1 + \varepsilon)\bigr)\Big\},
	\end{align}
	as stated in equation~\eqref{eq:gap_bd}.
	
	We first note that the moment-generating function (MGF) of distribution $\chisquare_\Dim$ is given by
	\begin{align*}
		\MMt(t) = (1 - 2t)^{-\frac{\Dim}{2}}, \quad \mbox{for any $t < \frac{1}{2}$}.
	\end{align*}
	Using Markov’s inequality, for any $t > 0$, we have
	\begin{align}
		\label{eq:chi_MMt}
		\Prob\big\{\chisquare_{\Dim} > (1 + \varepsilon) \, \Dim\big\}
		\;\leq\; \exp\{-t(1 + \varepsilon)\Dim\} \cdot \MMt(t)
		\; = \; \exp\{-t(1 + \varepsilon)\Dim\} \cdot (1 - 2t)^{-\frac{\Dim}{2}}
	\end{align}
    for any $t < \frac{1}{2}$.
	We optimize the bound by choosing $t$ to minimize the exponent $-t(1 + \varepsilon)\Dim - \frac{\Dim}{2}\log(1 - 2t)$.
	Solving for the optimal $t$, we obtain
	\begin{align*}
		t \; = \; \frac{\varepsilon}{2(1 + \varepsilon)} \, .
	\end{align*}
	Substituting $t$ back into inequality~\eqref{eq:chi_MMt}, the bound simplifies to the desired inequality~\eqref{eq:chi_tail}.
	

	
	
%%%%%%%%%%%%%%%%%%%%%%%%%%%%%%%%%%%%%%%%%%%%%%%%%%%%%%%%%%%%%%

	\section{Supporting Theorem: \\ Master Theorem for $Z$-Estimators}
	\label{sec:master}
	
	In this section, we provide a brief introduction to the master theorem for $Z$-estimators for the convenience of the readers.
	
	Let the parameter space be $\Theta$, and consider a data-dependent function $\Psi_n: \Theta \to \mathds{L}$, where $\mathds{L}$ is a metric space with norm~$\|\cdot\|_{\mathds{L}}$. Assume that the parameter estimate $\widehat{\theta}_n \in \Theta$ satisfies $\|\Psi_n(\widehat{\theta}_n)\|_{\mathds{L}} \convergep 0$, making $\widehat{\theta}_n$ a $Z$-estimator. The function~$\Psi_n$ is an estimator of a fixed function $\Psi: \Theta \to \mathds{L}$, where $\Psi(\theta_0) = 0$ for some parameter of interest $\theta_0 \in \Theta$.
	
	\begin{theorem}[Theorem~2.11 in \citet{kosorok2008introduction}, master theorem for $Z$-estimators]
		\label{thm:master}
		Suppose the following conditions hold:
		\begin{enumerate}
			\item $\Psi(\theta_0) = 0$, where $\theta_0$ lies in the interior of $\Theta$.
			\item $\sqrt{n} \, \Psi_n(\widehat{\theta}_n) \convergep 0$ and $\|\widehat{\theta}_n - \theta_0\| \convergep 0$ for the sequence of estimators $\{\widehat{\theta}_n\} \subset \Theta$.
			\item $\sqrt{n} (\Psi_n - \Psi)(\theta_0) \converged Z$, where $Z$ is a tight\footnote{A random variable $Z$ is tight if, for any $\epsilon > 0$, there exists a compact set $K \subset \Real$ such that $\Prob(Z \notin K) < \epsilon$.} random variable.
			\item The following smoothness condition is satisfied:
			\begin{align}
				\label{eq:master_cond}
				\frac{\big\| \sqrt{n} \big(\Psi_n(\widehat{\theta}_n) - \Psi(\widehat{\theta}_n)\big) - \sqrt{n} \big(\Psi_n(\theta_0) - \Psi(\theta_0)\big) \big\|_{\mathds{L}}}{1 + \sqrt{n} \, \| \widehat{\theta}_n - \theta_0 \|} \; \convergep \; 0 \, .
			\end{align}
		\end{enumerate}
		
		Additionally, assume that $\theta \mapsto \Psi(\theta)$ is Fréchet differentiable\footnote{Fréchet differentiability: A map $\phi: \mathds{D} \to \mathds{L}$ is Fréchet differentiable at $\theta$ if there exists a continuous, linear map $\phi_{\theta}': \mathds{D} \to \mathds{L}$ such that
		${\| \phi(\theta + h_n) - \phi(\theta) - \phi_{\theta}'(h_n) \|_{\mathds{L}}}/{\|h_n\|} \rightarrow 0$
		for all sequences $\{h_n\} \subset \mathds{D}$ with $\|h_n\| \to 0$ and $\theta + h_n \in \Theta$ for all $n \geq 1$.} at $\theta_0$
		with derivative $\dot{\Psi}_{\theta_0}$, and that $\dot{\Psi}_{\theta_0}$ is continuously invertible\footnote{Continuous invertibility: A map $A: \Theta \to \mathds{L}$ is continuously invertible if $A$ is invertible, and there exists a constant $c > 0$ such that $\|A(\theta_1) - A(\theta_2)\|_{\mathds{L}} \geq c \|\theta_1 - \theta_2\|$ for all $\theta_1, \theta_2 \in \Theta$.}.
		Then
		\begin{align*}
			\big\| \sqrt{n} \dot{\Psi}_{\theta_0}(\widehat{\theta}_n - \theta_0) + \sqrt{n} (\Psi_n - \Psi)(\theta_0) \big\|_{\mathds{L}} \convergep 0
		\end{align*}
		and therefore
		\begin{align*}
			\sqrt{n} \, \big(\widehat{\theta}_n - \theta_0\big) \; \converged \; - \dot{\Psi}_{\theta_0}^{-1} \, Z \, .
		\end{align*}
	\end{theorem}
	
	


	


%%%%%%%%%%%%%%%%%%%%%%%%%%%%%%%%%%%%%%%%%%%%%%%%%%%%%%%%%%%%%%
	
%	\tableofcontents

\section{Experimental Details}
\label{sec:app_exp}
In this section, we provide a detailed description of the evaluation process, divided into three parts: the construction and distribution details of \ourdataset~\ref{sec:app_exp_ifbench}, the evaluation dataset settings~\ref{sec:app_exp_evaluation}, and additional experimental results~\ref{sec:app_exp_more_res}.

\subsection{\ourdataset Details}
\label{sec:app_exp_ifbench}

\ourdataset is a benchmark designed to evaluate reward models for multi-constraint instruction-following. The dataset comprises $444$ carefully curated instances, each containing: an instruction with $3$ to $5$ multi-constraints, a chosen response satisfying all constraints, and a rejected response violating specific constraints. All instances were constructed using \texttt{gpt-4o-2024-11-20} version through the following systematic pipeline.

\looseness=-1
\paragraph{Instruction Construction} We sampled $500$ initial instructions from the Open Assistant~\cite{kopf2023openassistant}. To ensure clarity and simplicity, we constrained the initial instruction length to $5$ to $20$ words. Subsequently, we employed GPT-4o to generate five distinct categories of constraints for each initial instruction. It then autonomously selected $3$ to $5$ constraints and paraphrased them into $1$ to $2$ sentences. The paraphrased constraints were integrated into the initial instruction. Finally, we use GPT-4o to evaluate the final instructions and filter out those with internal contradictions, resulting in a final set of $444$ instructions.

\looseness=-1
\begin{itemize}
    \item {\bf Content Constraints: } Specify conditions governing response, including topic focus, detail depth, and content scope limitations.
    \item {\bf Style Constraints: } Control linguistic characteristics such as tone, sentiment polarity, empathetic expression, and humor.
    \item {\bf Length Constraints: } Dictate structural requirements including word counts, paragraph composition, and specific opening phrases.
    \item {\bf Keyword Constraints: } Enforce lexical constraints through keyword inclusion, prohibited terms, or character-level specifications.
    \item {\bf Format Constraints: } Define presentation standards that include specific formats such as JSON, Markdown, or Python, along with section organization and punctuation rules.
\end{itemize}


\paragraph{Response Construction} For each instruction, we generated $8$ candidate responses using GPT-4o with temperature $1.0$ to maximize diversity. The chosen response was selected as the unique candidate satisfying all constraints through automated verification. Rejected responses were systematically selected to ensure balanced distributions of unsatisfied constraint (UC) categories and counts. As shown in Figure~\ref{fig:IFbench}, instances are stratified by difficulty: simple (\#UC$\geq$3), normal (\#UC$=$2), and hard (\#UC$=$1), with detailed information of UC category distributions. Specifically, (a) shows the distribution by the number of unsatisfied constraints in the rejected responses, where the sum of all parts equals the total number of instances. (b) presents the distribution by the categories of all unsatisfied constraints, where the sum of all parts equals the total number of unsatisfied constraints.

\begin{figure}[!ht]
    \centering
    \subfigure[]{
    \includegraphics[width=0.45\linewidth]{figures/IFBench_F1.pdf} }
    \subfigure[]{
    \includegraphics[width=0.45\linewidth]{figures/IFBench_F2.pdf} }
    \caption{Proportion (\%) of data in \ourdataset based on the number of unsatisfied constraints per instance and the categories of all unsatisfied constraints. }
    \label{fig:IFbench}
\end{figure}

\begin{figure*}
    \centering
    \includegraphics[width=0.98\linewidth]{figures/gpt4o_best_of_n.pdf}
    \caption{Best-of-n results (\%) on TriviaQA, IFEval, and CELLO using the base reward model ArmoRM and \ourmethod to search. ``+Oracle'' denotes using the oracle setting of \ourmethod as mentioned in \cref{sec:exp_analysis}.}
    \label{fig:gpt4o_best_of_n}
\end{figure*}


\subsection{Evaluation Details}
\label{sec:app_exp_evaluation}

% 表3中各个数据集的evaluation setting,metric 
\paragraph{Best-of-N} For the TriviaQA, we sample $500$ instances from the validation split in \texttt{rc.nocontext} version. The model is prompted to generate direct answers, and we report the exact match accuracies. For the IFEval, we report the average accuracy across the strict prompt, strict instruction, loose prompt, and loose instruction settings. For the CELLO, we report the average score based on the official evaluation script. All three tasks are conducted under a zero-shot setting.

\paragraph{DPO Training}
For MT-Bench and CELLO, we employ FastChat\footnote{\url{https://github.com/lm-sys/FastChat/tree/main/fastchat/llm_judge}} and the official evaluation script respectively, to conduct the evaluations and report the average scores.
For the other tasks, we use the \texttt{lm-evaluation-harness}\footnote{\url{https://github.com/EleutherAI/lm-evaluation-harness}} for evaluation. Specifically, we adopt a 5-shot setting for the MMLU and MMLU-Pro tasks, while using a zero-shot setting for TriviaQA and TruthfulQA. Notably, for TruthfulQA, we use the \texttt{truthfulqa\_gen} setting. 

\subsection{More Results on Best-of-N}
\label{sec:app_exp_more_res}
We conduct best-of-n search experiments using \texttt{gpt-4o-2024-11-20} as the policy model, with the results presented in Figure~\ref{fig:gpt4o_best_of_n}. The results demonstrate that \ourmethod significantly improves best-of-n performance compared to the base reward model ArmoRM, even when applied to a more powerful policy model than \ourmethod.




\begin{table*}
    \centering
    \small
    \begin{adjustbox}{max width=1\linewidth}
    {
    \begin{tabular}{p{\linewidth}}
    \toprule
    % \textbf{Prompt For Router} \\
    % \midrule
   Given the following instruction, determine whether the following check in needed. \\
    \\
        \text{[Instruction]} \\
        \{instruction\} \\
    \\
        \text{[Checks]} \\
       \{ 
            ``name'': ``constraint check'', 
            ``desp'': ``A `constraint check' is required if the instruction contains any additional constraints or requirements on the output, such as length, keywords, format, number of sections, frequency, order, etc.'', 
            ``identifier'': ``[[A]]'' 
        \}, 
        \{  
            ``name'': ``factuality check'', 
            ``desp'': ``A `factuality check' is required if the generated response to the instruction potentially contains claims about factual information or world knowledge.'', 
            ``identifier'': ``[[B]]'' 
        \} \\
        \\
        If the instruction requires some checks, please output the corresponding identifiers (such as [[A]], [[B]]). \\
        Please do not output other identifiers if the corresponding checkers not needed. \\
    \bottomrule
    \end{tabular}
    }
    \end{adjustbox}
    \caption{Our prompt for the router, where the \{instruction\} part varies based on the input. }
    \label{tab:planner}
\end{table*}

\begin{table*}
    \centering
    \small
    \begin{adjustbox}{max width=1\linewidth}
    {
    \begin{tabular}{p{\linewidth}}
    \toprule
    % \textbf{Prompt For Difference Proposal} \\
    % \midrule
    \textbf{Prompt For Difference Proposal} \\
        \text{[Answers]} \\
        \{formatted\_answers\} \\
        \\
        \text{[Your Task]} \\
        Given the above responses, please identify and summarize one key points of contradiction or inconsistency between the claims. \\
        \\
        \text{[Requirements]} \\
        1. Return a Python list containing only the most significant differences between the two answers. \\
        2. Do not include any additional explanations, only output the list. \\
        3. If there are no inconsistencies, return an empty list. \\
    \midrule
    \textbf{Prompt For Query Generation} \\
    \text{[Original question that caused the inconsistency]} \\
        \{instruction\} \\
\\
        \text{[Inconsistencies]} \\
        \{inconsistencies\} \\
        \\
        \text{[Your Task]} \\
        To resolve the inconsistencies, We need to query search engine. For each contradiction, please generate a corresponding query that can be used to retrieve knowledge to resolve the contradiction.  \\
        \\
        \text{[Requirements]} \\
        1. Each query should be specific and targeted, aiming to verify or disprove the conflicting points.  \\
        2. Provide the queries in a clear and concise manner, returning a Python list of queries corrresponding to the inconsistencies. \\
        3. Do not provide any additional explanations, only output the list. \\
        \midrule
    \textbf{Prompt For Verification} \\
    Evaluate which of the two answers is more factual based on the supporting information. \\
        \text{[Support knowledge sources]}: \\
        \{supports\} \\
        \\
        \text{[Original Answers]}: \\
        \{formatted\_answers\} \\
        \\
        \text{[Remeber]} \\
        For each answer, provide a score between 1 and 10, where 10 represents the highest factual accuracy. Your output should only consist of the following: \\
        Answer A: [[score]] (Wrap the score of A with [[ and ]]) \\
        Answer B: <<score>> (Wrap the score of B with << and >>) \\
        Please also provide a compact explanation. \\
    \bottomrule
    \end{tabular}
    }
    \end{adjustbox}
    \caption{Our prompt for assessing factuality in verification agents, with the \{formatted\_answers\}, \{supports\}, \{inconsistencies\}, \{instruction\} and \{supports\} parts varying based on the input. }
    \label{tab:factuality_agent}
\end{table*}


\begin{table*}
    \centering
    \small
    \begin{adjustbox}{max width=1\linewidth}
    {
    \begin{tabular}{p{\linewidth}}
    \toprule
    % \textbf{Prompt For Difference Proposal} \\
    % \midrule
    \textbf{Prompt For Constraint Parsing} \\
       You are an expert in natural language processing and constraint checking. Your task is to analyze a given instruction and identify which constraints need to be checked. \\
        \\
        The `instruction' contains a specific task query along with several explicitly stated constraints. Based on the instructions, you need to return a list of checker names that should be applied to the constraints. \\
        \\
        Task Example: \\  
        Instruction: Write a 300+ word summary of the Wikipedia page ``https://en.wikipedia.org/wiki/Raymond\_III,\_Count\_of\_Tripol''. Do not use any commas and highlight at least 3 sections that have titles in markdown format, for example, *highlighted section part 1*, *highlighted section part 2*, *highlighted section part 3*.\\
        Response: \\
        NumberOfWordsChecker: 300+ word \\
        HighlightSectionChecker: highlight at least 3 sections that have titles in markdown format\\
        ForbiddenWordsChecker: Do not use any commas \\
        \\
        Task Instruction: \\
        \{instruction\} \\
        \\
        \#\#\# Your task: \\
        - Generate the appropriate checker names with corresponding descriptions from the original instruction description. \\
        - Return the checker names with their descriptions separated by `\textbackslash n'  \\
        - Focus only on the constraints explicitly mentioned in the instruction (e.g., length, format, specific exclusions).  \\
        - Do **not** generate checkers for the task query itself or its quality. \\
        - Do **not** infer or output constraints that are implicitly included in the instruction (e.g., general style or unstated rules). \\
        - Each checker should be responsible for checking only one constraint. \\
    \midrule
    \textbf{Prompt For Code Generation} \\
    You are tasked with implementing a Python function `check\_following' that determines whether a given `response' satisfies a constraint defined by a checker. The function should return `True' if the constraint is satisfied, and `False' otherwise. \\
\\
        \text{[Instruction to check]}: \\
        \{instruction\} \\
\\
        \text{[Specific Checker and Description]}: \\
        \{checker\_name\} \\
\\
        Requirements: \\
        - The function accepts only one parameter: `response' which is a Python string. \\
        - The function must return a boolean value (`True' or `False') based on whether the `response' adheres to the constraint described by the checker. \\
        - The function must not include any I/O operations, such as `input()' or `ArgumentParser'. \\
        - The Python code for each checker should be designed to be generalizable, e.g., using regular expressions or other suitable techniques. \\
        - Only return the exact Python code, with no additional explanations. \\
    \bottomrule
    \end{tabular}
    }
    \end{adjustbox}
    \caption{Our prompt for assessing instruction-following in verification agents, with the \{instruction\} and \{checker\_name\} parts varying based on the input. }
    \label{tab:if_agent}
\end{table*}



\end{document}



