\vspace{-4mm}
\section{Result \& Discussion}
The implementation of the NCDW  successfully integrated clinical, demographic, and environmental datasets, demonstrating its ability to manage multidimensional data and support real-time healthcare analytics. Our findings can be generalized into two main types. First, we will discuss our findings on the load estimation, storage requirements, and performance comparison of our proposed system. Then we will share our dengue data mart specific findings. Together, they highlight the capacity of the system and its role in data-driven public health interventions.



\subsection{Infrastructure and Performance Findings}
This one focuses on the system’s infrastructure and performance, assessing its ability to integrate clinical datasets, manage storage, and process data efficiently.



\subsubsection{Load Estimation and Storage Requirements}
\label{subsec:load_estimation}

Estimating the daily data load and total storage requirements of the NCDW is crucial for efficient management. The daily load is calculated by aggregating record entries from various healthcare facilities, using the formula:
\begin{equation}
     R =\bar{r}\times  \left \{ \sum_{j=1}^{n} \left ( \frac{s_j*hospital\,count}{\sum_{}^{total\,bed}s} \right ) \right \}\times d(days)
\end{equation}

where $R$ represents daily record entries, $\bar{r}$ is the average records per hospital, $s_j$ denotes hospital weight based on bed count, $n$ is the total hospitals, and $d$ is the number of days.

The total NCDW size is estimated as:
\begin{equation} 
\label{eqn2}
\text{Size} = R \times d \times S_r 
\end{equation}

where $S_r$ is the average record size in kilobytes. These calculations help in planning storage needs for different periods (e.g., one to five years), ensuring scalability through cloud-based solutions.

By applying these formulas, future data growth can be anticipated, enabling proactive infrastructure planning.




% \subsubsection{Load Estimation}
% To assess the system’s scalability, the daily record entry
% load and storage requirements were estimated:
\begin{table}[!h]
    \centering
    \caption{Average number of daily records entered in a hospital over a week.}
    \label{tab:daily-avg-record}
    \setlength{\tabcolsep}{2pt}
    \begin{tabular}{c|cccccccc}
        \toprule
         \textbf{Day} &\textbf{Sun}& \textbf{Mon}& \textbf{Tue} &\textbf{Wed} &\textbf{Thu} &\textbf{Fri} &\textbf{Sat}  \\
         \textbf{Avg Record, \(r_i\)} &10072 &9976 &10132 &9931 &8973 &5294 &11799\\
         \midrule
        \textbf{Avg Record Per Day, \(\bar{r}\) } & \multicolumn{6}{c}{9456}\\
         \bottomrule
    \end{tabular}
\end{table}

Table~\ref{tab:daily-avg-record} provides average daily
entries (\(r\)) for a single hospital or diagnostic center. Saturday
and Sunday generally show higher traffic, while Friday tends
to be lowest due to reduced staffing , as shown in Fig.~\ref{fig:Day-wise Weekly Entry Count}. From Table~\ref{tab:daily-avg-record}, the average per-hospital/center daily record is \(\bar{r} = 9456\). Scaling to the national level requires accounting for 552 government hospitals (secondary/tertiary), 8000 private diagnostic centers, and different “weights” (\(w\)) based on each hospital’s bed capacity.
%figure 5
\begin{figure}[!h]
    \vspace{-3mm}
    \centering
    \includegraphics[width=1.0\linewidth]{figures/Day-wise_Weekly_Entry_Count.pdf}
    \caption{Day-Wise Weekly Entry Count}
    \label{fig:Day-wise Weekly Entry Count}
\end{figure}




% \begin{table}[!ht]
% \centering
% \caption{Daily Average Record Entry in a Single Hospital/Diagnostic Center}
% \label{tab:daily-avg-record}
% \begin{tabular}{|c|c|c|c|}
% \hline
% \textbf{SL} & \textbf{Day} & \textbf{Avg Record, \(r_i\)} & \textbf{Per Day Avg} \\
% \hline
% 1 & Sun & 10072 & \multirow{7}{*}{\textbf{9456}} \\
% \cline{1-3}
% 2 & Mon & 9976 &  \\
% \cline{1-3}
% 3 & Tue & 10132 & \\
% \cline{1-3}
% 4 & Wed & 9931 & \\
% \cline{1-3}
% 5 & Thu & 8973 & \\
% \cline{1-3}
% 6 & Fri & 5294 & \\
% \cline{1-3}
% 7 & Sat & 11799 & \\
% \hline
% \end{tabular}
% \end{table}




Table~\ref{tab:seat-hospital-weight} shows how hospital bed-capacity weights translate into daily record estimates. Thus, roughly \(19.04\) million records flow into the NCDW each day.

\begin{table}[!h]
\centering
\caption{Load estimation of dfferent Hospital.}
\label{tab:seat-hospital-weight}
\begin{tabular}{c|c|c|c}
\toprule
\textbf{Seat No} (\(s\)) & \textbf{Hospital Count} (\(n\)) & \textbf{Weight} (\(w\)) & \(\textbf{w} \times n \times \bar{r}\) \\
\hline
10 & 17 & 0.0030 & 486 \\
20 & 32 & 0.0060 & 1828 \\
31 & 266 & 0.0093 & 23551 \\
50 & 158 & 0.0151 & 22562 \\
100 & 31 & 0.0302 & 8854 \\
150 & 1 & 0.0453 & 429 \\
200 & 1 & 0.0604 & 572 \\
250 & 26 & 0.0755 & 18564 \\
500 & 2 & 0.1510 & 2856 \\
500 & 11 & 0.1510 & 15708 \\
1500 & 7 & 0.4530 & 29988 \\
\hline
\multicolumn{3}{c|}{\textbf{Total for 552 Govt. Hospitals}} & 125398 \\
\hline
\multicolumn{3}{c|}{\textbf{Diagnostic Centers (8000)}} & 18912000 \\
\hline
\multicolumn{3}{c|}{\textbf{Overall Daily Total (8552)}} & 19037398 \\
\bottomrule
\end{tabular}
\end{table}



% \subsubsection{Storage Requirements}
Now assuming each record requires \(0.1\,\mathrm{KB}\), the estimated daily storage usage is given 
\(\mathrm{Daily\ Size} \approx R \times 0.1\,\mathrm{KB}\) according to Eq. ~\ref{eqn2}. 
Table~\ref{table:size-estimate} presents projected storage volumes over different periods. 
The daily storage requirement is approximately \(19\,\mathrm{GB}\), increasing to nearly \(7\,\mathrm{TB}\) annually and \(34\,\mathrm{TB}\) over five years.

\begin{table}[!h]
\centering
\caption{Estimated size of NCDW for different time spans}
\label{table:size-estimate}
\begin{tabular}{c|c|c}
\toprule
\textbf{Total Records/Day, \(R\)} & \textbf{Time Span, \(d\)} & \textbf{Estimated Size} \\
\hline
\multirow{3}{*}{19,037,398} 
  & 1 day & 19.04 GB  \\ \cline{2-3}
  & 1 year (365 days) & 6.79 TB  \\ \cline{2-3}
  & 5 years (5$\times$365 days) & 33.95 TB  \\ 
\bottomrule
\end{tabular}
\end{table}



%-----------------Figure For Environmental Correlation ----------

    \begin{figure*}[!t]
    % \vspace{-3mm}
        \centering
        \includegraphics[width=1.0\textwidth,trim={0 150 0 60}, clip]{figures/rainfall_humidity_temperature.pdf}
        \vspace{-15pt}
        \begin{center}
           (a) Rainfall vs Positive Cases \hspace{1.5cm} 
           (b) Humidity vs Positive Cases \hspace{1.5cm} 
           (c) Temperature vs Positive Cases
        \end{center}
        \vspace{-4pt}
        \caption{Impact of environmental factors on dengue cases. Rainfall (a) and humidity (b) exhibit a significant correlation with dengue cases, whereas temperature (c) shows no clear correlation.}
        \label{fig:rainfallvshum}
        \vspace{-5mm}
    \end{figure*}



%-----------------Figure For Environmental Correlation ----------











\subsubsection{Runtime and Performance Comparison (PostgreSQL vs. HBase) }
To evaluate the performance of PostgreSQL and HBase for the NCDW, cube operations were conducted on datasets of varying sizes (100K, 200K, 500K, and 1M), and the execution times for cube sizes 3 and 4 were measured for both systems, as summarized in Table~\ref{tab:cube34}.

\begin{table}[!h]
\centering
\caption{Execution Times (in minutes) for Cube Sizes 3 and 4}
\begin{tabular}{c|c|c|c|c}
\toprule
\multirow{2}{*}{\textbf{Aggregation Size}} & 
\multicolumn{2}{c|}{\textbf{Cube Size 3}} & 
\multicolumn{2}{c}{\textbf{Cube Size 4}} \\ 
% \cline{2-5}
 & \textbf{PostgreSQL} & \textbf{HBase} & \textbf{PostgreSQL} & \textbf{HBase} \\ 
 \hline
% \midrule
100,000   & 5    & 3    & 92   & 40   \\ \hline
200,000   & 49   & 15   & 150  & 70   \\ \hline
500,000   & 150  & 60   & 350  & 180  \\ \hline
1,000,000 & 400  & 180  & 780  & 410  \\ 
\bottomrule
\end{tabular}
\label{tab:cube34}
\end{table}


%Cube Size 3
%100,000 : 40.00%
%200,000 : 69.39%
%500,000 : 60.00%
%1,000,000 : 55.00%
%Cube Size 4
%100,000 : 56.52%
%200,000 : 53.33%
%500,000 : 48.57%
%1,000,000 : 47.44%


Fig.~\ref{fig:posthbs} illustrates that HBase outperforms PostgreSQL, with execution times improved by 40–69\% across different cube and dataset sizes.

\begin{figure}[h]
    \centering
    \includegraphics[width=0.5\textwidth,trim={10 60 0 70}, clip]{figures/postvshbase.pdf}
    \begin{center}
       (a) Cube Size 3 \hspace{1.9cm} (b) Cube Size 4
    \end{center}
    \vspace{-4pt}
    \caption{Execution time comparison for cube size 3 (a) and cube size 4 (b) between PostgreSQL and HBase.}
    \label{fig:posthbs}
\end{figure}



\begin{figure}
\vspace{-4mm}
    \centering
    \includegraphics[width=0.9\linewidth]{figures/Day-wise_Weekly_Test_Count.pdf}
    \caption{Day-wise Weekly Test Count(Avg)}
    \label{fig:day_wise_weekly_test}
    \vspace{-6.5mm}
\end{figure}


\subsection{ Data  Mart Specific Insights}
The second highlights the dengue data mart case study, demonstrating how the NCDW derives epidemiological insights from integrated health and environmental data.

\subsubsection{Environmental Correlation with Dengue Cases}
Environmental factors such as rainfall, humidity, and temperature play a critical role in dengue incidence. Rainfall fosters mosquito breeding by creating stagnant water, with dengue cases peaking in August, coinciding with the highest rainfall levels, as shown in Fig.~\ref{fig:rainfallvshum}a. Humidity also exhibited a strong correlation with dengue cases, reaching its peak in August, as illustrated in Fig.~\ref{fig:rainfallvshum}b, suggesting its potential as an early outbreak indicator. In contrast, temperature fluctuations had no significant impact on dengue incidence, as demonstrated in Fig.~\ref{fig:rainfallvshum}c. These insights highlight the importance of environmental monitoring for effective dengue prevention and control.

































\begin{figure}[!h]
    \centering
    \includegraphics[width=0.9\linewidth,trim={0 50 0 50},clip]{figures/demhraphic_age_month.pdf}
    \begin{center}
     (a) Monthly distribution \hspace{1cm}  (b) Age-wise distribution
    \end{center}
    \caption{Dengue-positive cases by month and age group. (a) Peak incidence from July to October. (b) Higher vulnerability among individuals aged 0–40 years.}

    \label{fig:demographic}
\end{figure}

\subsubsection{Temporal and Demographic Trends of Dengue Incidence}
Dengue cases followed a seasonal pattern, peaking between July and October, with the highest incidence in August, aligning with the monsoon season and increased mosquito breeding, as shown in Fig.~\ref{fig:demographic}a. Weekly trends indicate peak data entry on Saturdays and Sundays, with a noticeable drop on Fridays, likely due to staffing variations, as illustrated in Fig.~\ref{fig:day_wise_weekly_test}. Additionally, 77.4\% of cases were reported in the 0–40 age group, emphasizing the need for targeted prevention efforts, as depicted in Fig.~\ref{fig:demographic}b.
\begin{figure}[]
    \vspace{-3.5mm}
    \centering
    \includegraphics[width=0.95\linewidth]{figures/graph/dengue/Infected_vs_Total_Dengue_Tests.png}
    \caption{Outbreak Detection}
    \label{fig:outbreak-detection}
    \vspace{-6mm}
\end{figure}
\subsubsection{Outbreak Detection}
Tracking monthly case counts enabled early detection of a surge in infections beginning in June, reaching its highest point in August, and declining thereafter, as shown in Fig.~\ref{fig:outbreak-detection}. These findings confirm a July–October outbreak period and highlight the system’s capability to identify such surges in near-real time, supporting proactive healthcare measures.

\begin{figure}[]
    \vspace{-3.5mm}
    \centering
    \includegraphics[width=1\linewidth]{figures/graph/Top_4_TestCount_Per_Month.png}
    \caption{Distribution of the most frequently conducted diagnostic tests for dengue-positive patients, illustrating the high reliance on CBC, Dengue NS1, Electrolyte, and creatinine serum tests for disease monitoring and management.}
    \label{fig:dengue-test-count}
    \vspace{-6mm}
\end{figure}
Fig.~\ref{fig:dengue-test-count} presents the top four most frequently conducted laboratory tests for dengue-positive cases. CBC, Dengue NS1, and electrolyte serum tests dominate, reflecting their critical role in dengue diagnosis and patient monitoring.





