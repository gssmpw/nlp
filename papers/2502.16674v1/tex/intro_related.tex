\section{Introduction \& Related Works}



The accelerated digitization of healthcare services globally has led to an unprecedented surge in Electronic Health Records (EHRs) and other clinical data. In principle, data warehouses can integrate diverse healthcare information into a centralized repository, which forms the basis for advanced analytics, population health management, and evidence-based policymaking \cite{evans2016electronic}. Developed nations have successfully implemented Clinical Data Warehouses (CDWs) to enhance healthcare delivery and research. For instance, a study examining 32 French regional and university hospitals found that 14 had fully operational CDWs, with implementations accelerating since 2011 \cite{France}. In the United States, institutions have developed CDWs to support clinical decision-making and secondary data use by integrating heterogeneous data sources \cite{Usa}. However, in developing countries such as Bangladesh, where healthcare systems are fragmented, record-keeping is inconsistent, and infrastructure is limited, fully reaching this potential remains a challenge~\cite{wisniewski2003development, 7400708}.

A National Clinical Data Warehouse (NCDW) offers a viable approach to consolidating multi-institutional data into a privacy-preserving environment. By unifying patient information, laboratory results, and clinical observations, an NCDW can enhance large-scale health analytics and support more informed policy decisions \cite{adewole2024systematic}. Data warehouse architectures often incorporate specialized data marts tailored to specific domains or diseases by integrating clinical, geographic, and environmental data~\cite{heinonen2019data}. For instance, in dengue-endemic regions like Bangladesh, where seasonal outbreaks pose significant healthcare challenges, a dedicated dengue-focused data mart can enhance the opportunity to detect outbreaks, plan resource allocation, and conduct epidemiological research \cite{kayesh2023increasing, mutsuddy2019dengue}. Data mining techniques such as clustering and association can also uncover hidden patterns in health data, aiding outbreak detection~\cite{han2012data, park2005design}.

The choice of SQL (e.g., PostgreSQL) vs. NoSQL (e.g., HBase) is also critical for managing healthcare data. SQL ensures ACID compliance, while NoSQL offers better scalability for semi-structured data. A recent study in Bangladesh suggests hybrid approaches improve performance for large-scale health analytics~\cite{khan2015development}.

However, several key challenges limit the effectiveness of NCDWs in resource-constrained settings. Initially, the absence of unique patient identifiers, often due to low literacy rates, variable name spellings, and minimal official documentation, complicates record linkage. Secondly, selecting an appropriate database platform (e.g., SQL vs. NoSQL) proves critical for efficiently storing and querying large-scale health data under infrastructural constraints. Thirdly, integrating environmental or demographic data remains essential for analyzing and predicting infectious diseases such as dengue, yet many existing solutions lack a mechanism to incorporate these external factors. Lastly, machine learning (ML) algorithms for real-time outbreak detection require consistent, high-quality data, which can be difficult to obtain in developing contexts.


In earlier work \cite{ncdwProblemsIssues}, existing national health data management systems have been analyzed, along with the design and limitations of Bangladesh’s NCDW, and the need to strengthen its performance, security, and integration of heterogeneous data has been explored. Our previous work~\cite{mia2022privacy,ksrl} proposed a privacy-preserving NCDW architecture that leveraged the key-based algorithm to anonymize patient data while enabling secure linkages across fragmented sources. Although this solution addressed fundamental privacy concerns, limitations were identified in areas such as scalability, interoperability, and the incorporation of external datasets for comprehensive analysis. To overcome these challenges, this study presents an enhanced NCDW architecture with the following key improvements:
\begin{enumerate}
    \item \textbf{A wrapper-based secure data integration from heterogeneous sources.}
    \item \textbf{ External data (i.e., environmental and demographic)  integration in our CDW.}
    \item \textbf{A name-matching algorithm for improved record linkage.}
    \item \textbf{A specialized data mart for disease-specific analysis.}
    \item \textbf{SQL vs. NoSQL performance evaluation to enhance system scalability and performance.}
\end{enumerate}
% \begin{itemize}
%     \item \textbf{Environmental and demographic data integration} for comprehensive analytics.
%     \item \textbf{A wrapper-based data acquisition layer} for secure data integration.
%     \item \textbf{A name-matching algorithm} for improved record linkage.
%     \item \textbf{A specialized data mart} for disease-specific analysis.
%     \item \textbf{SQL vs. NoSQL performance evaluation} to enhance system scalability and performance.
% \end{itemize}

% \begin{description}
%     \item[\textbf{Environmental and Demographic Data Integration}] Enables comprehensive analytics for better decision-making.
%     \item[\textbf{Wrapper-Based Data Acquisition Layer}] Ensures secure and efficient data integration.
%     \item[\textbf{Name-Matching Algorithm}] Enhances record linkage and data consistency.
%     \item[\textbf{Specialized Data Mart}] Supports disease-specific analysis for targeted healthcare insights.
%     \item[\textbf{SQL vs. NoSQL Performance Evaluation}] Optimizes system scalability and performance.
% \end{description}

