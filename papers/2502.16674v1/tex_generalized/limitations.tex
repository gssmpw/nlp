\section{Limitations and Future Work}
Although the proposed NCDW framework is designed for scalability, its effectiveness depends on the consistency and availability of data sources. The limited inclusion of behavioral or personal patient traits poses a challenge in generating comprehensive patient insights. Future research should explore methods to incorporate patient lifestyle data to enhance predictive analytics and disease surveillance.

Additionally, while our framework enables efficient disease-specific analytics, it currently does not support complex data types such as test reports and medical imaging (e.g., X-rays, MRIs). Expanding the system’s storage capabilities to accommodate such data will be essential for holistic patient record management.

Another key area for improvement is the database architecture. Our SQL vs. NoSQL comparison revealed significant performance differences, suggesting that a deeper investigation into NoSQL-based solutions could optimize the handling of large-scale EHR data. Further research should evaluate hybrid approaches that balance structured (clinical records) and semi-structured (environmental, genomic) data storage. 

By addressing these challenges, the NCDW framework can evolve into a more robust, flexible, and predictive system, supporting evidence-based public health strategies and efficient resource allocation across diverse healthcare environments.