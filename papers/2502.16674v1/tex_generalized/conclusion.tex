\section{Conclusion}
Our study presents a framework for a National Clinical Data Warehouse (NCDW) that is designed to address the scalability, interoperability, and privacy challenges inherent in resource-constrained healthcare systems. The proposed wrapped-based architecture ensures secure and anonymized data integration, while our proposed name-matching algorithm effectively resolves patient identity inconsistencies in the absence of unique identifiers.
% After that, data is standardized and loaded into disease-specific datamarts, which are used to visualize key analytics on the dashboard, enabling stakeholders to make quick and informed decisions for resource allocation and outbreak response.

To evaluate the feasibility and impact of the proposed framework, we conducted a case study using healthcare data from Bangladesh. Our prototype deployed in our institution's server estimated processing 19 million daily records across 8552 hospitals. Additionally, the dengue-specific data mart revealed a clear correlation between dengue outbreaks and environmental factors such as rainfall and humidity, showcasing the potential of NCDW-driven analytics in epidemiological surveillance and outbreak prediction. 
% We were also able to successfully detect the 2019 dengue outbreak.

While our case study is focused on Bangladesh, the framework is designed to be adaptable to other developing nations facing similar challenges. By incorporating secure data acquisition and smart storage solutions, our approach provides a generalizable model that can be tailored to different national healthcare infrastructures. Our proposed star schema can also be adapted to incorporate more data types (i.e., patient demographics,
% test reports,
MRI/X-ray images). Future work should focus on expanding this framework to support additional diseases, integrating predictive models for outbreak forecasting, and refining storage solutions to accommodate imaging and genomic data. By implementing this framework, developing nations can enhance their public health decision-support systems, optimize resource allocation, and improve patient care on a national scale.


% I will talk about future prospect and scope of expansion here

% Limitation
% add points about how this can be applied to other third world countries with limited resources.
