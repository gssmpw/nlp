
\section{Conclusions}
\label{sec:conclusions}
We tackle the task of Astronaut Photography Localization (APL), obtaining a robust APL model by leveraging a huge set of weakly annotated astronaut photographs, for which we first computed full localization labels.
We propose two ways to enhance our model through this data, first adopting them (paired with their satellite counterparts) in an ad-hoc pairwise loss, and second using them to mine ideal training data to pass to our model.
We thus obtain a new model, AstroLoc, which shows impressive results on all APL datasets; furthermore, we find that AstroLoc can be utilized in other related tasks, like the lost-in-space problem and the localization of historical space imagery.
Finally, we note that we have already used AstroLoc (in conjunction with EarthMatch's post-processing) to provide localization for a staggering 500k photos, and anticipate that in a few months the backlog of non-localized photos will be nearly empty for the first time since the launch of the ISS.


\paragraph{Acknowledgements.}
\small{We acknowledge the Cineca award under the Iscra initiative, for the availability of high performance computing resources.
This work was supported by CINI.
Project supported by ESA Network of Resources Initiative.
This study was carried out within the project FAIR - Future Artificial Intelligence Research - and received funding from the European Union Next-GenerationEU (Piano nazionale di ripresa e resilienza (PNRR) – missione 4 componente 2, investimento 1.3 – D.D. 1555 11/10/2022, PE00000013). This manuscript reflects only the authors’ views and opinions, neither the European Union nor the European Commission can be considered responsible for them.
European Lighthouse on Secure and Safe AI – ELSA, Horizon EU Grant ID: 101070617
}
