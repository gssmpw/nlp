
\begin{table}
\begin{center}
\begin{adjustbox}{width=0.9\columnwidth}
\begin{tabular}{c|c|ccccc}
\toprule
Method & \# Params (M) & R@1 & R@5 & R@10 & R@20 & R@100 \\
\midrule
AnyLoc        & 1136 &4.4 &  9.6 & 13.4 & 19.0 & 40.0 \\
EarthLoc      & \underline{27.6} &3.3 &  7.0 &  8.8 & 12.0 & 27.6 \\
AstroLoc      & 105 &\textbf{52.7} & \textbf{63.9} & \textbf{68.7} & \textbf{73.3} & \textbf{83.7}\\
AstroLoc-tiny & \textbf{27.2} &\underline{36.7} & \underline{49.4} & \underline{55.3} & \underline{61.6} & \underline{74.5} \\
\bottomrule
\end{tabular}
\end{adjustbox}
\end{center}
\vspace{-5mm}
\caption{\textbf{Results of Lost-in-Space satellite problem.} Using the dataset from VINSat \cite{McCleary_2024_vinsat} as queries, we show AstroLoc can also achieve good results on this additional task. AstroLoc-tiny is a tiny version more suitable for deployment on a nanosatellite, with 27M model parameters and 512 dimensional features.}
\vspace{-5mm}
\label{tab:vinsat}
\end{table}
