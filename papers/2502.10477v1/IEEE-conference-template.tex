\documentclass[conference]{IEEEtran}
\IEEEoverridecommandlockouts
% The preceding line is only needed to identify funding in the first footnote. If that is unneeded, please comment it out.
%Template version as of 6/27/2024

\usepackage{cite}
\usepackage{amsmath,amssymb,amsfonts}
\usepackage{algorithmic}
\usepackage{graphicx}
\usepackage{textcomp}
\usepackage{url} 
\usepackage{xcolor}
\def\BibTeX{{\rm B\kern-.05em{\sc i\kern-.025em b}\kern-.08em
    T\kern-.1667em\lower.7ex\hbox{E}\kern-.125emX}}
\begin{document}

\title{Knowledge Integration Strategies in Autonomous Vehicle Prediction and Planning: A Comprehensive Survey\\
%{\footnotesize \textsuperscript{*}Note: Sub-titles are not captured for https://ieeexplore.ieee.org  and
%should not be used}
%\thanks{Identify applicable funding agency here. If none, delete this.}
}

\author{
    \IEEEauthorblockN{Kumar Manas}
    \IEEEauthorblockA{\textit{Department of Computer Science and Mathematics} \\
    \textit{Freie Universität Berlin}\\
    and\\
    \textit{Continental Automotive Technologies}\\
    Germany\\
    kumar.manas@fu-berlin.de}
    \and
    \IEEEauthorblockN{Adrian Paschke}
    \IEEEauthorblockA{\textit{Department of Computer Science and Mathematics} \\
    \textit{Freie Universität Berlin}\\
    and\\
    \textit{Fraunhofer FOKUS}\\
    Germany\\
    adrian.paschke@fu-berlin.de}
}


\maketitle

\begin{abstract}
This comprehensive survey examines the integration of knowledge-based approaches into autonomous driving systems, with a focus on trajectory prediction and planning. We systematically review methodologies for incorporating domain knowledge, traffic rules, and commonsense reasoning into these systems, spanning purely symbolic representations to hybrid neuro-symbolic architectures. In particular, we analyze recent advancements in formal logic and differential logic programming, reinforcement learning frameworks, and emerging techniques that leverage large foundation models and diffusion models for knowledge representation. Organized under a unified literature survey section, our discussion synthesizes the state-of-the-art into a high-level overview, supported by a detailed comparative table that maps key works to their respective methodological categories. This survey not only highlights current trends—including the growing emphasis on interpretable AI, formal verification in safety-critical systems, and the increased use of generative models in prediction and planning—but also outlines the challenges and opportunities for developing robust, knowledge-enhanced autonomous driving systems.
\end{abstract}


\begin{IEEEkeywords}
Formal Logic, Knowledge Representation, Knowledge Integration, Conflict Resolution, Literature Survey, Planning, Prediction, LLM, Survey
\end{IEEEkeywords}

\maketitle

\section{Introduction}


Autonomous driving has evolved into a complex and multifaceted research field. Vehicles' safe and efficient operation relies on accurately anticipating and planning motions in real-time. While significant progress has been made in simulation environments, the true challenge lies in transitioning automated vehicles from controlled simulations to the unpredictable real world. In real-world scenarios, vehicles must contend with a wide variety of situations, including unforeseen behaviors and novel circumstances.
While effective in structured environments, rule-based systems often require adaptation to accommodate dynamic traffic conditions. On the other hand, machine learning (ML)-based solutions, though powerful, typically demand extensive amounts of data and can struggle when deployed in out-of-distribution scenarios or unfamiliar domains. Unlike machines, human drivers excel in such situations by leveraging a combination of rules, knowledge, and general driving intuition to make informed decisions. Integrating such human-like reasoning—infusing rules and domain knowledge—into the autonomous driving pipeline could enhance the system’s accuracy, robustness, and ability to handle new or unexpected scenarios. Two critical components in this domain are trajectory prediction and trajectory planning. In our context, \textbf{trajectory prediction} refers to the process of forecasting the future paths of other traffic agents based on observed historical data and contextual information such as road geometry, dynamic interactions, and regulatory constraints. \textbf{Trajectory planning}, on the other hand, involves generating a feasible and optimal motion path for the controlled (ego) vehicle that satisfies safety, legal, and comfort constraints while taking into account the anticipated behavior of other road users.

Despite numerous surveys addressing various aspects of autonomous driving—from computer vision and object detection to deep reinforcement learning and semantic segmentation—there remains a notable gap in comprehensive reviews focusing specifically on integrating explicit knowledge representations into trajectory prediction and planning. In particular, there is a lack of surveys examining recent foundation and diffusion model–based methodologies. These emerging approaches have gained traction due to their ability to leverage vast amounts of unstructured data and advanced generative techniques to produce precise and adaptable predictions. They can embed traffic rules and environmental constraints either implicitly within learned representations or explicitly through integrated symbolic reasoning, thereby enhancing both interpretability and performance in complex driving scenarios.

This survey is driven by several key observations:

\begin{itemize} 
    \item \textbf{Fragmented Literature:} Previous reviews have explored trajectory prediction and planning \cite{surevey_predictor_no_KR}. However, they rarely holistically integrate these tasks under frameworks combining traditional methods with modern generative techniques and extensive domain knowledge. We examine how knowledge is integrated into the automated driving domain in various ways, providing a unified perspective.
    \item \textbf{Emerging Methodologies:} Recent advances, including foundation model-based approaches (e.g., large language models and retrieval-augmented generation) and diffusion model-based methods, offer promising new avenues for integrating explicit regulatory and spatial constraints into prediction and planning. These methods can revolutionize how autonomous systems interpret contextual information and generate compliant, safe trajectories. 
    \item \textbf{Practical and Future Relevance:} In real-world autonomous driving, it is crucial that prediction and planning modules not only operate accurately but also adhere to traffic rules and adapt dynamically to complex environments. The integration of explicit knowledge—whether embedded implicitly through data-driven models or explicitly via hybrid symbolic frameworks—can significantly enhance system robustness and safety. \end{itemize}

We focus exclusively on works that explicitly integrate rules or knowledge. For instance, \cite{Chauffeurnet} utilized road boundary information in a CNN model but did not explicitly incorporate traffic rules, so we exclude such works. By offering a detailed taxonomy and critical analysis of state-of-the-art methods—including classical approaches and the latest foundation and diffusion model–based techniques—this survey aims to provide researchers and practitioners with a unified perspective on achieving precise, rule-compliant trajectory prediction and planning.

By critically evaluating both symbolic and hybrid methodologies, we seek to offer insights that can drive the development of next-generation autonomous systems—systems that are not only capable of understanding and predicting dynamic traffic scenarios but are also inherently safe, interpretable, and aligned with regulatory requirements.
Autonomous driving relies on robust understanding and reasoning about driving environments, traffic rules, and commonsense knowledge. This survey explores knowledge representation methods for prediction and planning in automated driving, covering knowledge-graph and ontology-based, hybrid, reinforcement learning, LLM-based, and formal logic-based approaches. We aim to provide a comprehensive overview for researchers and practitioners, fostering advancements in knowledge-driven autonomous driving.

\begin{table*}[ht]
\centering
\caption{Overview of Knowledge and Traffic Rules Integration Methodology in Trajectory Prediction and Planning}
\label{tab:overview_all}
\scriptsize
\begin{tabular}{p{2cm} p{3cm} p{6cm} p{5cm}}
\hline
\textbf{Method Category} & \textbf{Representative Works} & \textbf{Key Contributions \& Approach} & \textbf{Advantages / Limitations} \\
\hline
\textbf{Knowledge-Graph and Ontology-Based} &
\begin{itemize}
    \item \cite{Geng_Liang_Yu_Zhao_He_Huang_2017,zhao_intersection_ontology_2015,Balakirsky2004KnowledgeRA,Sun_Wang_Halilaj_Luettin_2024,Regele_2008,Jr_Jurmain_Ragazzi_2024,Fang_2019_ontology_reasoning_ITSC,zhang2022uti,Huang_2019_ontology,onto_crtical_westhofen,R_Uma_2018_ontology_survey,Giret_Julian_Carrascosa_Rebollo_2018,Bhuyan_2024_Ontology_Development_for_Sustainable_Intelligent,Kommineni_König-Ries_Samuel}
\end{itemize} &
\begin{itemize}
    \item Formalizes road elements, interactions, and regulatory constraints via OWL, ontologies, and KGs.
    \item Some works integrate probabilistic reasoning and hierarchical real-time decision–making.
\end{itemize} &
\begin{itemize}
    \item \textbf{Advantages:} High interpretability and explicit rule encoding.
    \item \textbf{Limitations:} Scalability issues and extensive manual engineering.
\end{itemize} \\
\hline
\textbf{Reinforcement Learning (RL)} &
\begin{itemize}
    \item \cite{Peiss_Wohlgemuth_Xue_Meyer_Gressenbuch_Althoff_2023,Lin_Zhou_Wang_Cao_Yu_Zhao_Zhao_Yang_Li_2022,Yuan_2024_Evolutionary_Decision_Making_and_Planning,Li2024Let,tang_2023_Personalized_Decision_Making_and_Control,pan2017virtual,kiran_Deep_Reinforcement_Learning}
\end{itemize} &
\begin{itemize}
    \item Embeds traffic rules in state representations and reward functions.
    \item Employs hybrid strategies (e.g., imitation learning, formal logic) to improve safety and exploration.
\end{itemize} &
\begin{itemize}
    \item \textbf{Advantages:} Adaptive policy learning from dynamic environments.
    \item \textbf{Limitations:} Reward engineering complexity and sample inefficiency.
\end{itemize} \\
\hline
\textbf{LLM \& RAG Methods} &
\begin{itemize}
    \item \cite{Li_2024_CVPR,Cai_Liu_Zhou_Ma_Zhao_Wu_Ma_2024,tr2mtl,xu2024drivegpt4,Mao2023ALA,hwang2024emmaendtoendmultimodalmodel,Cui2024DriveLLM:,sha2023languagempclargelanguagemodels,Fu_Li_Wen_Dou_Cai_Shi_Qiao_2024,Yuan_Sun_Omeiza_Zhao_Newman_Kunze_Gadd,LLMAssistClosedLoopPlanning}
\end{itemize} &
\begin{itemize}
    \item Uses unstructured text and regulatory documents to adapt driving behavior.
    \item Translates natural language rules into formal (temporal) logic and generates context-aware decisions.
    \item Often combined with multimodal inputs (e.g., DriveGPT4).
\end{itemize} &
\begin{itemize}
    \item \textbf{Advantages:} Dynamic adaptation and enhanced interpretability.
    \item \textbf{Limitations:} Latency concerns and risk of LLM hallucinations.
\end{itemize} \\
\hline
\textbf{Formal Logic-Based Methods} &
\begin{itemize}
    \item \cite{Karimi_Duggirala_2020,auto_discern,patrikar2024rulefuser,suchan2023assessing,Manzinger2021Using,Koschi2021Set-Based,Loos_Platzer_Nistor_2011,Bhuiyan_Governatori_Bond_Demmel_Badiul_2020,Chan_Li_Lu_Lin_Bundy_2025,esterle2020,coogan2014,raman2017,Sadraddini2016,Lin2022ModelPR}
\end{itemize} &
\begin{itemize}
    \item Formalizes traffic regulations via temporal logic, ASP, etc.
    \item Integrated into verification, MPC, and optimization frameworks for provable safety.
\end{itemize} &
\begin{itemize}
    \item \textbf{Advantages:} Strong safety guarantees and high interpretability.
    \item \textbf{Limitations:} High computational complexity and challenges with real-time integration.
\end{itemize} \\
\hline
\textbf{Hybrid Symbolic \& Neural Methods} &
\begin{itemize}
    \item \cite{auto_discern,Koschi2021Set-Based} (also in Formal Logic),
    \item \cite{Bahari_Nejjar_Alahi_2021,Li_Rosman_Gilitschenski_DeCastro_Vasile_Karaman_Rus_2021,wen2024diluknowledgedrivenapproachautonomous,li2023towards,sormoli2024survey,badreddine2022logic,manhaeve2018deepproblog,manas2022robust,kamale_Cautious_Planning_with_Incremental,Li2021Vehicle}
    \item \cite{Fu_Li_Wen_Dou_Cai_Shi_Qiao_2024} (also in LLM)
  
\end{itemize} &
\begin{itemize}
    \item Fuses deep learning–based perception with explicit symbolic reasoning.
    \item Incorporates symbolic constraints in loss functions or via LLM–generated rules.
\end{itemize} &
\begin{itemize}
    \item \textbf{Advantages:} Balances flexibility with rule compliance.
    \item \textbf{Limitations:} Increased integration complexity and synchronization challenges.
\end{itemize} \\
\hline
\textbf{Diffusion \& Other Learning–Based Approaches} &
\begin{itemize}
    \item \cite{stl_diffusion,CoBL_diffusion}
\end{itemize} &
\begin{itemize}
    \item Leverages diffusion processes to generate multimodal, rule–compliant trajectories.
    \item Blends stochastic exploration with explicit constraint encoding.
\end{itemize} &
\begin{itemize}
    \item \textbf{Advantages:} Captures diverse trajectory distributions.
    \item \textbf{Limitations:} Balancing exploration with strict rule adherence can be challenging.
\end{itemize} \\
\hline
\end{tabular}
\vspace{-0.2cm}
\end{table*}

\section{Literature Survey}
This section presents a high-level overview of the diverse methodologies proposed for integrating domain knowledge, traffic rules, and commonsense reasoning into autonomous driving systems. Our survey is organized into several subsections—ranging from Knowledge-Graph and Ontology-Based Methods, Reinforcement Learning (RL) methods, and Large Language Model (LLM) approaches to Formal Logic-Based and Hybrid Neuro-Symbolic Methods—that collectively cover the spectrum of current research. To guide the reader through these varied approaches, Table~\ref{tab:overview_all} provides a compact, comparative summary of the key works discussed in this survey. The table maps each representative study to its corresponding methodological category, highlights its core contributions, and outlines its advantages and limitations. Also, note that some works fall into multiple categories due to their interdisciplinary nature. This high-level synthesis facilitates an understanding of how these diverse approaches interconnect and serves as a roadmap for the detailed discussions that follow in the subsequent sections.


\subsection{Knowledge-Graph and Ontology-Based Methods}
\label{ontology_and_KG}
Recent advancements in autonomous driving have highlighted the importance of structured knowledge representation using ontologies and knowledge graphs (KGs) to model complex urban environments, enhance prediction accuracy, and enable rule-compliant planning. These frameworks provide explicit semantic representations of static road elements (e.g., lanes, traffic signs), dynamic interactions (e.g., vehicle-pedestrian relationships), and regulatory constraints, enabling interpretable reasoning across heterogeneous data sources.

Early works demonstrated the potential of ontology-based reasoning for driving scenarios. \cite{Geng_Liang_Yu_Zhao_He_Huang_2017} pioneered a scenario-adaptive approach using Web Ontology Language (OWL) to formalize spatial states and vehicle-road relationships, integrating these with hidden Markov models for maneuver prediction. At the same time, their method improves prediction horizons through domain knowledge and manual ontology engineering limits scalability. Similarly, \cite{zhao_intersection_ontology_2015} developed intersection-specific ontologies with rule-based decision-making, and \cite{Balakirsky2004KnowledgeRA} established a framework for querying static and dynamic traffic semantics in real-time. These studies underscore the trade-off between interpretability and adaptability: rule-based ontologies excel in structured environments like intersections \cite{zhao_intersection_ontology_2015} but struggle with unseen scenarios.

SemanticFormer \cite{Sun_Wang_Halilaj_Luettin_2024} improves autonomous driving trajectory prediction by considering traffic participants, road topology, and signs, enhancing performance by 4-5\% when added to existing trajectory predictors. Hybrid methods combining ontologies with probabilistic models address uncertainty inherent in traffic interactions. \cite{Regele_2008} merged high-level traffic coordination ontologies with low-level trajectory models, while \cite{Jr_Jurmain_Ragazzi_2024} integrated description logic with probabilistic semantics to handle incomplete information. \cite{Fang_2019_ontology_reasoning_ITSC} extended this by encoding interaction probabilities for long-term behavior prediction, though computational delays hindered real-time control. \cite{zhang2022uti} augmented scene understanding by integrating KGs extracted from textual corpora, enabling causal reasoning for novel scenarios. However, offline knowledge curation in their approach limits responsiveness to dynamic environments.

Ontologies also support real-time decision-making through hierarchical reasoning. \cite{Huang_2019_ontology} leveraged driving experience knowledge bases for situation estimation, while \cite{onto_crtical_westhofen} modeled critical factors via joint description logic and rule-based reasoners. These methods enhance transparency in regulatory compliance, but face scalability challenges as ontology complexity grow \cite{Balakirsky2004KnowledgeRA}. A comprehensive survey by \cite{R_Uma_2018_ontology_survey} emphasizes that while ontological frameworks improve interoperability and rule awareness, their rigidity complicates adaptation to unexpected events.

Current research focuses on mitigating manual engineering burdens through standardization and automation. Domain-specific ontologies like ASAM OpenXOntology\footnote{\url{https://www.asam.net/standards/asam-openxontology/}} provide reusable templates for traffic concepts, reducing development efforts \cite{Giret_Julian_Carrascosa_Rebollo_2018, Bhuyan_2024_Ontology_Development_for_Sustainable_Intelligent}. Promisingly, Large Language Models (LLMs) are being explored for semi-automated KG generation \cite{Kommineni_König-Ries_Samuel}, potentially enabling dynamic knowledge integration with sensor data. Such advances aim to balance the interpretability of symbolic reasoning with the adaptability of data-driven methods, addressing long-standing challenges in real-time performance and scalability \cite{R_Uma_2018_ontology_survey, Fang_2019_ontology_reasoning_ITSC}.




\subsection{Reinforcement Learning (RL) Methods}
\label{RL_approach}
Reinforcement learning (RL) has emerged as a powerful paradigm for developing autonomous driving systems, enabling sequential decision-making through dynamic environment interactions. By integrating safety constraints, regulatory compliance, and structured domain knowledge into reward functions and state representations, RL frameworks excel in learning adaptive policies for complex driving scenarios. Recent advancements in this domain focus on curriculum learning, safety-centric reward design, and sim-to-real transfer while addressing challenges such as sample inefficiency and reward engineering.

A key trend in RL-based methods is using structured state encodings to embed traffic rules and safety priors. For instance, \cite{Peiss_Wohlgemuth_Xue_Meyer_Gressenbuch_Althoff_2023} propose a graph-based RL framework that employs curriculum learning to increase the complexity of highway driving scenarios progressively. By integrating explicit traffic-rule information into the state representation using graph neural networks, their method accelerates learning and improves collision avoidance and lane-change efficiency in dense traffic. Similarly, \cite{Lin_Zhou_Wang_Cao_Yu_Zhao_Zhao_Yang_Li_2022} address the challenge of adapting to evolving traffic regulations by designing a law-adaptive backup policy. Their approach translates natural language traffic laws into formal constraints, enabling the RL agent to recognize potential violations and trigger re-planning. While these methods enhance compliance and safety, they introduce reward engineering and policy evaluation complexity.

Safety-centric RL frameworks often enforce rule adherence through reward shaping and hybrid architectures. \cite{Yuan_2024_Evolutionary_Decision_Making_and_Planning} augment the traditional RL reward function with explicit penalties for traffic violations, balancing goal achievement with safety metrics. This evolutionary RL framework emphasizes safe and rational exploration, guiding the agent toward policies that adhere to safety standards. In a similar vein, \cite{Li2024Let} combines deep RL with Hybrid A* for local trajectory planning, using linear temporal logic (LTL) to formalize traffic rule compliance in the reward function. Imitation learning hybrids, such as the generative adversarial imitation learning framework proposed by \cite{tang_2023_Personalized_Decision_Making_and_Control}, refine policies through adversarial feedback on expert demonstrations. These approaches yield human-like behaviors while ensuring rule compliance, though they require careful calibration to avoid over-constraining exploration.

Bridging the gap between simulation and real-world deployment remains a critical challenge in RL-based autonomous driving. \cite{pan2017virtual} investigate the sim-to-real transfer of RL policies, training agents in virtual environments and adapting them to real-world conditions. Their work highlights the potential of RL for robust performance in heterogeneous scenarios but also underscores the challenges of achieving reliable generalization. Surveys such as \cite{kiran_Deep_Reinforcement_Learning} provide a broader perspective, categorizing RL-based approaches by state representation (e.g., raw sensor inputs vs. graph abstractions) and reward formulations. These surveys emphasize the trade-offs between sample efficiency and safety guarantees, offering insights into the design of scalable and reliable RL frameworks.

Despite their advantages, RL-based methods face several challenges. Reward engineering remains a non-trivial task, as balancing safety, compliance, and efficiency in reward design is critical for policy performance. Sample inefficiency is another significant bottleneck, as complex environments demand extensive training data, prolonging development cycles. The sim-to-real gap further complicates deployment, with discrepancies between simulated and real-world dynamics hindering reliable policy transfer. Additionally, the "black-box" nature of RL policies complicates verification in safety-critical contexts, necessitating interpretable and verifiable representations.

Emerging directions in RL-based autonomous driving focus on addressing these challenges. Evolutionary strategies, such as those proposed by \cite{Yuan_2024_Evolutionary_Decision_Making_and_Planning}, offer efficient exploration mechanisms for safe policy learning. Formal methods like linear temporal logic (LTL) \cite{Li2024Let} provide verifiable compliance guarantees, while hybrid architectures blending imitation learning with RL \cite{tang_2023_Personalized_Decision_Making_and_Control} enhance human-like behavior and adaptability. Future research should prioritize scalable reward automation, robust sim-to-real pipelines, and interpretable policy representations to advance the real-world applicability of RL-based autonomous driving systems.

\subsection{Large Language Model and Retrieval-Augmented Generation Methods}
\label{LLM_and_KG}
Large language models (LLMs) and retrieval-augmented generation (RAG) techniques have emerged as transformative tools in autonomous driving. They use unstructured text and regulatory data to improve decision-making, adaptability, and interpretability. These methods enable systems to incorporate vast knowledge repositories dynamically, generate human-readable explanations, and adapt to region-specific traffic laws, addressing key challenges in prediction and planning.


A prominent application of LLMs is their ability to interpret and adapt to traffic regulations in real-time. LLaDA (Large Language Driving Assistant) \cite{Li_2024_CVPR} exemplifies this capability, using LLMs to adapt driving behavior to new environments, customs, and laws. By interpreting traffic rules from local driver handbooks, LLaDA provides context-aware driving instructions and demonstrates zero-shot generalization. Similarly, \cite{Cai_Liu_Zhou_Ma_Zhao_Wu_Ma_2024} propose a framework where a Traffic Regulation Retrieval Agent employs RAG to query regional traffic documents, and an LLM-based reasoning module evaluates maneuvers for compliance and safety. While this approach dynamically adapts to region-specific laws and generates rule-referenced explanations, it faces latency and LLM hallucination challenges. The TR2MTL framework \cite{tr2mtl} further advances this paradigm by using LLMs in a chain-of-thought learning paradigm to translate natural language traffic rules into metric temporal logic formulas, reducing manual rule specification and enabling scalable regulatory updates.


LLMs are increasingly integrated with multimodal data to enhance perception and decision-making. DriveGPT4 \cite{xu2024drivegpt4} represents an end-to-end system that generates driving decisions directly from sensory inputs, providing natural language explanations for its trajectory predictions. This approach improves interpretability and generalization in novel environments. Mao et al. \cite{Mao2023ALA} further contribute by developing a language agent that fuses textual reasoning with visual perception, retrieving relevant rule excerpts and synthesizing actionable driving strategies. Industry efforts, such as Waymo’s EMMA model \cite{hwang2024emmaendtoendmultimodalmodel}, highlight the potential of multimodal LLMs like Google’s Gemini to process sensor data and predict future trajectories, signaling a shift toward LLM-centric autonomous systems.


Recent works explore using LLMs as central decision-making components in autonomous driving. DriveLLM \cite{Cui2024DriveLLM:} charts a comprehensive roadmap for integrating LLMs into fully autonomous driving systems, emphasizing their role in high-level reasoning, planning, and interaction with dynamic environments. By leveraging LLMs' generative and contextual understanding capabilities, DriveLLM demonstrates how these models can enhance adaptability and interpretability in complex driving scenarios. Similarly, LanguageMPC \cite{sha2023languagempclargelanguagemodels} proposes a framework where LLMs serve as decision-makers within a model predictive control (MPC) pipeline. This approach combines the reasoning capabilities of LLMs with the robustness of traditional control methods, enabling the system to generate human-like driving strategies while ensuring safety and compliance. Both frameworks highlight the potential of LLMs to bridge the gap between high-level reasoning and low-level control. However, challenges such as real-time performance and integration with existing systems remain.


LLMs and RAG techniques are also employed to augment reasoning with commonsense and regulatory knowledge. \cite{zhang2022uti} utilize dense retrieval mechanisms and LLMs to answer domain-specific questions about traffic scenarios, implicitly integrating background knowledge into decision-making. The “Drive Like a Human” framework \cite{Fu_Li_Wen_Dou_Cai_Shi_Qiao_2024} enhances generalization and interpretability by combining LLM-generated high-level context descriptions with neural perception modules. RAG-Driver \cite{Yuan_Sun_Omeiza_Zhao_Newman_Kunze_Gadd} extends this concept by employing a multimodal LLM for driving explanations and control predictions, leveraging retrieval-augmented in-context learning to achieve strong zero-shot generalization. Additionally, LLM-Assist \cite{LLMAssistClosedLoopPlanning} demonstrates the synergistic potential of LLMs in hybrid planning architectures, using them alongside conventional rule-based planners to improve performance in complex scenarios.


Despite their promise, LLM and RAG-based methods face several challenges. Real-time performance remains a critical bottleneck, as the computational demands of LLMs can hinder their deployment in latency-sensitive applications. The risk of hallucinations—incorrect or nonsensical outputs—poses safety concerns, particularly in dynamic driving environments. Effective fusion of multimodal sensory data with language-based reasoning also requires further research to ensure robust and reliable decision-making. Ongoing efforts aim to address these issues through advancements in model efficiency, hallucination mitigation, and multimodal integration.

LLM and RAG-based methods offer a promising pathway to adaptive, transparent, and compliant autonomous driving systems. By dynamically incorporating regulatory knowledge, generating interpretable explanations, and enabling zero-shot generalization, these approaches address key limitations of traditional methods. However, realizing their full potential will require overcoming real-time performance, reliability, and multimodal fusion challenges, paving the way for next-generation autonomous systems capable of navigating complex and rapidly changing environments.


\subsection{Formal Logic-Based Methods}
\label{formal_logic_method}
Formal logic-based methods have become a cornerstone in autonomous driving research, offering rigorous frameworks for monitoring, verification, prediction, and planning through mathematically grounded representations of traffic rules and safety requirements. A central theme in this work is the formalization of traffic regulations using temporal logic and answer set programming (ASP). For instance, \cite{Karimi_Duggirala_2020} formalizes traffic rules at uncontrolled intersections by encoding yielding and right–of–way as temporal logic formulas, ensuring that vehicle behaviors adhere to safe bounds even though extending this approach to complex urban scenarios remains challenging. Similarly, AUTO-DISCERN \cite{auto_discern} employs ASP via the goal-directed s(CASP) system to simulate commonsense reasoning in driving decisions. By encoding driving rules as explicit logical constraints and generating proof trees to justify maneuvers, AUTO-DISCERN achieves high interpretability while grappling with scalability issues and sensor noise.

Another emerging direction involves integrating formal logic with data-driven techniques to enhance robustness in trajectory prediction. RuleFuser \cite{patrikar2024rulefuser} exemplifies this trend by combining neural predictors with classical rule-based predictors; here, formal logic is used to represent traffic rules that guide predictions in out–of–distribution scenarios. In a related effort, the TR2MTL framework \cite{tr2mtl} leverages large language models (LLMs) to automatically translate unstructured natural language traffic rules into metric temporal logic (MTL) formulas. By employing a chain–of–thought in–context learning strategy, TR2MTL reduces manual rule formalization, although its performance is sensitive to prompt design and input quality.

Formal logic is also employed to enhance situational awareness and decision–making. \cite{suchan2023assessing} apply ASP–based reasoning to model and assess driver situation awareness in semi-autonomous vehicles, formalizing scene interpretation as logical rules that verify correct perception and reaction to dynamic traffic conditions. Complementing these approaches works such as \cite{Manzinger2021Using,Koschi2021Set-Based} integrate set–based reachability analysis with convex optimization to plan safe trajectories that comply with legal driving standards, albeit with challenges in accurately modeling the behavior of other road users in unpredictable environments.

Further reinforcing the role of formal methods, safety verification techniques based on differential dynamic logic \cite{Loos_Platzer_Nistor_2011} have been applied to prove the safety of vehicle control systems under varied driving conditions. Such methods, along with survey works like \cite{kiran_Deep_Reinforcement_Learning}, illustrate that while formal logic-based approaches provide strong guarantees and a high level of interpretability, they often face challenges in terms of computational complexity, scalability, and real-time integration of continuous changing sensor inputs.

Defeasible Deontic Logic (DDL) has been applied to formalize traffic rules, enabling autonomous vehicles to reason about obligations, permissions, and prohibitions in dynamic driving environments. By encoding traffic regulations using DDL \cite{Bhuiyan_Governatori_Bond_Demmel_Badiul_2020,Chan_Li_Lu_Lin_Bundy_2025}, vehicles can predict trajectories that comply with legal and ethical standards while also accommodating exceptions and resolving conflicts among rules. This approach enhances the interpretability and adaptability of trajectory prediction systems, allowing for more nuanced decision-making in complex traffic scenarios. However, implementing DDL in real-time applications poses challenges due to its computational complexity and the need for comprehensive rule databases. Balancing the expressiveness of DDL with the efficiency required for real-time trajectory prediction remains a critical area for further research.

In practice, many of these formal logic-based methods are integrated into hybrid and optimization–based models. Formal logic offers a versatile toolkit for ensuring provably safe and compliant autonomous driving behaviors, whether employed as constraints in loss functions, integrated into reinforcement learning reward structures, or used in optimization layers for planning. Despite their promise, ongoing research continues to address the trade-offs between the rigor of formal methods and the need for real-time, scalable performance in dynamic driving environments.

\subsubsection{Formal Logic in MPC and Optimization}

Model Predictive Control (MPC) is a widely adopted optimization-based control strategy in autonomous driving. It solves an optimal control problem over a finite receding horizon to generate trajectories that account for both current sensor data and predicted future states. This approach translates traffic rules into mathematical inequalities or cost functions directly embedded into optimization-based planning frameworks. By formulating safety rules as constraints within MPC, recent research has increasingly focused on integrating formalized traffic rules into these optimization frameworks to enhance both performance and safety. Embedding traffic constraints as mathematical inequalities and temporal logic conditions allows MPC-based planners to enforce rigorous safety and regulatory requirements while optimizing for efficiency.

One line of work translates human–interpretable traffic rules into machine–readable specifications. For example, \cite{esterle2020} proposes a methodology that converts natural language traffic rules into Linear Temporal Logic (LTL) specifications. This translation facilitates systematic testing and verification of autonomous behaviors and provides a formal basis for integrating these rules directly into planning algorithms. In parallel, \cite{coogan2014} develop control frameworks for traffic networks where signalized intersections are managed by synthesizing control policies that satisfy temporal logic constraints. Their approach leverages formal methods and model checking to generate strategies that guarantee compliance with complex operational constraints, thereby improving safety and efficiency in urban traffic management.

Another influential approach merges MPC with Signal Temporal Logic (STL) specifications. In \cite{raman2017}, STL constraints are encoded as mixed–integer linear constraints within an MPC framework. This formulation allows the controller to systematically enforce temporal properties during receding horizon control, which is particularly beneficial for applications requiring stringent timing and safety adherence. Extending these ideas to the urban context, \cite{Sadraddini2016} apply MPC to optimize traffic signal operations under temporal logic constraints, ensuring that predefined temporal conditions are met to reduce congestion and improve traffic flow. Further advancing robustness, \cite{Lin2022ModelPR} integrates predictive robustness into STL–based MPC formulations, providing enhanced reliability against uncertainties in dynamic environments.

Collectively, these studies illustrate a significant trend: the integration of formal traffic rules into optimization–based control strategies yields control systems that are both flexible and provably safe. The combined use of MPC and temporal logic—whether through LTL or STL—offers a promising pathway toward designing controllers capable of real-time adaptation while rigorously satisfying safety and regulatory requirements. This hybridization of formal logic and optimization advances autonomous driving and paves the way for future developments in intelligent transportation systems.
\subsection{Hybrid Approaches Combining Symbolic and Neural Methods}
\label{hybrid approach}

Hybrid approaches for autonomous driving have evolved by fusing the adaptability of neural networks with the rigor and interpretability of symbolic reasoning. Such methods aim to leverage the pattern recognition power of deep learning while enforcing explicit, rule-based constraints, ultimately improving safety, robustness, and explainability. These approaches can be broadly grouped by their integration strategy.

One prominent strand couples deep neural perception pipelines with symbolic reasoning modules. For example, AUTO–DISCERN \cite{auto_discern} processes sensory inputs through deep networks whose features feed into an answer set programming–based module (via the goal-directed s(CASP) system) to apply commonsense rules and enforce driving behaviors. Similarly, the framework in \cite{manas_legal_2023} combines explicitly encoded traffic rules with learning–based trajectory predictors; here, the symbolic component acts as a constraint, guiding the neural network towards rule–compliant predictions while reducing the data requirements.

A second category focuses on integrating symbolic knowledge directly into the learning process. In \cite{Bahari_Nejjar_Alahi_2021}, symbolic constraints are injected into the loss functions and network architectures, enhancing robustness in rare or hazardous scenarios. This idea is further developed in \cite{Li_Rosman_Gilitschenski_DeCastro_Vasile_Karaman_Rus_2021}, where formal rules expressed as temporal logic is incorporated in a differentiable manner, penalizing non–compliant predictions. In a related effort, \cite{Li2021Vehicle} employs signal temporal logic and syntax trees as features within a generative adversarial network framework to improve trajectory prediction accuracy without biasing the system away from lawful behavior.

A recent trend leverages large language models (LLMs) to bridge high–level symbolic reasoning and sensor-driven perception. For instance, \cite{Fu_Li_Wen_Dou_Cai_Shi_Qiao_2024} utilizes an LLM to generate high–level symbolic descriptions of the driving context, which are then fused with deep learning–based sensor interpretations. Building on this concept, DILU \cite{wen2024diluknowledgedrivenapproachautonomous} automatically generates symbolic rules from unstructured data via LLMs and integrates them into the decision–making pipeline. Although promising, these systems still face challenges such as LLM hallucinations, integration latency, and the synchronization of subsymbolic and symbolic processing streams.

Complementing these technical approaches, several surveys and alternative frameworks provide a broader perspective on the hybrid paradigm. Reviews in \cite{li2023towards} and the survey on hybrid motion planning \cite{sormoli2024survey} underscore the potential of embedding domain knowledge into learning–based systems to enhance safety and interpretability. Additional work such as \cite{Koschi2021Set-Based} employs set–based prediction methods—using formalized traffic rules and nondeterministic motion models—to perform reachability analysis for collision avoidance, while neural–symbolic methods like Logic Tensor Networks \cite{badreddine2022logic} and DeepProbLog \cite{manhaeve2018deepproblog} offer frameworks for integrating logical constraints into deep learning models.

Finally, specialized systems have been designed to address real-time adaptability and conflict resolution. For example, \cite{manas2022robust} formalizes explicit traffic rules to rule out infeasible maneuvers, and \cite{kamale_Cautious_Planning_with_Incremental} combines incremental symbolic perception with reactive control synthesis to generate verified, rule–compliant driving maneuvers in dynamic environments.

In summary, hybrid approaches that integrate symbolic and neural methods offer significant advantages by combining explicit rule-based reasoning with the flexibility of deep learning. While these methods enhance interpretability and safety in complex driving scenarios, challenges remain in managing computational complexity, ensuring real-time performance, and effectively synchronizing heterogeneous components.


\subsubsection{Diffusion and Other Learning-Based Prediction Incorporating Traffic Rules}

Diffusion-based models have recently shown promising results in generating rule-compliant trajectories while adhering to user-defined constraints. These models can encode traffic rules as STL within the diffusion process \cite{stl_diffusion} or integrate control barrier functions and Lupanov function-based reachability encoding \cite{CoBL_diffusion}. Diffusion models are particularly effective at capturing multimodal distributions and generating diverse trajectories while maintaining constraint satisfaction. However, they often require careful design to balance stochastic exploration with rule adherence.

Beyond purely formal and optimization-based approaches, learning-based trajectory prediction has increasingly incorporated explicit traffic rules and regulatory constraints. These methods enhance deep neural models by embedding formal rules into loss functions or input representations, ensuring predicted trajectories optimize performance while complying with safety and legal standards. For example, RuleFuser \cite{patrikar2024rulefuser} demonstrates that combining symbolic reasoning with data-driven predictors improves robustness in out-of-distribution scenarios. Similarly, approaches that integrate temporal logic constraints into the training process \cite{Li_Rosman_Gilitschenski_DeCastro_Vasile_Karaman_Rus_2021} yield dynamically feasible and rule-compliant trajectories.

Since we focus on representing traffic rules and integrating them into learning-based methods, our approach aligns closely with hybrid methods that blend formal constraints with data-driven models.

\section{Discussion and Future Directions}  

This survey systematically reviews methodologies for trajectory prediction and planning in autonomous driving. We define trajectory prediction as forecasting agents’ future paths using historical and contextual data and trajectory planning as generating safe, collision-free paths for the ego vehicle under regulatory constraints. Our analysis spans classical methods (e.g., geometric models, optimization-based strategies) to modern hybrid frameworks merging symbolic reasoning with deep neural networks.  

A central focus lies in emerging foundation model- and diffusion model-based approaches, which leverage unstructured data and generative mechanisms to encode traffic dynamics or explicitly integrate domain knowledge implicitly. These methods show promise in enhancing prediction accuracy and planning robustness in complex scenarios, signaling a paradigm shift in autonomous systems. Our review equips researchers with methodologies to address prediction-planning challenges while balancing safety, efficiency, and goal achievement.  

Critical challenges persist. Future work must resolve real-time performance in dynamic environments, improve the interpretability of deep learning systems, and unify heterogeneous data streams (e.g., sensor inputs and traffic rules). The scalability and adaptability of foundation/diffusion models in diverse driving conditions require further validation. Multimodal reasoning and formal safety guarantees—particularly for handling nuanced, conflicting rules and hierarchical regulatory frameworks—will grow in importance.  

In summary, by synthesizing state-of-the-art methods across different paradigms, this survey highlights the current capabilities and limitations in trajectory prediction and planning. It lays out a roadmap for future investigations. Advancements in integrating explicit knowledge—whether through symbolic reasoning, hybrid models, or generative approaches—will be crucial in developing the next generation of autonomous driving systems that are both safe and efficient in real-world operations. By diligently advancing knowledge representation techniques and fostering synergistic combinations of different approaches, we can pave the way for more intelligent, trustworthy, and ultimately safer autonomous vehicles that can seamlessly and reliably navigate the full spectrum of real-world driving complexities.

\section{Acknowledgments}
This work has been partially funded by the German Federal Ministry for Economic Affairs and Climate Action within the project “nxtAIM”. We sincerely thank Dr. Stefan Zwicklbauer for his constant support and mentorship throughout the work.
\bibliographystyle{IEEEtran}
\bibliography{reference}
\end{document}

