%%%%%%%% ICML 2023 EXAMPLE LATEX SUBMISSION FILE %%%%%%%%%%%%%%%%%

\documentclass{article}

%%%%% NEW MATH DEFINITIONS %%%%%

\usepackage{amsmath,amsfonts,bm}
\usepackage{derivative}
% Mark sections of captions for referring to divisions of figures
\newcommand{\figleft}{{\em (Left)}}
\newcommand{\figcenter}{{\em (Center)}}
\newcommand{\figright}{{\em (Right)}}
\newcommand{\figtop}{{\em (Top)}}
\newcommand{\figbottom}{{\em (Bottom)}}
\newcommand{\captiona}{{\em (a)}}
\newcommand{\captionb}{{\em (b)}}
\newcommand{\captionc}{{\em (c)}}
\newcommand{\captiond}{{\em (d)}}

% Highlight a newly defined term
\newcommand{\newterm}[1]{{\bf #1}}

% Derivative d 
\newcommand{\deriv}{{\mathrm{d}}}

% Figure reference, lower-case.
\def\figref#1{figure~\ref{#1}}
% Figure reference, capital. For start of sentence
\def\Figref#1{Figure~\ref{#1}}
\def\twofigref#1#2{figures \ref{#1} and \ref{#2}}
\def\quadfigref#1#2#3#4{figures \ref{#1}, \ref{#2}, \ref{#3} and \ref{#4}}
% Section reference, lower-case.
\def\secref#1{section~\ref{#1}}
% Section reference, capital.
\def\Secref#1{Section~\ref{#1}}
% Reference to two sections.
\def\twosecrefs#1#2{sections \ref{#1} and \ref{#2}}
% Reference to three sections.
\def\secrefs#1#2#3{sections \ref{#1}, \ref{#2} and \ref{#3}}
% Reference to an equation, lower-case.
\def\eqref#1{equation~\ref{#1}}
% Reference to an equation, upper case
\def\Eqref#1{Equation~\ref{#1}}
% A raw reference to an equation---avoid using if possible
\def\plaineqref#1{\ref{#1}}
% Reference to a chapter, lower-case.
\def\chapref#1{chapter~\ref{#1}}
% Reference to an equation, upper case.
\def\Chapref#1{Chapter~\ref{#1}}
% Reference to a range of chapters
\def\rangechapref#1#2{chapters\ref{#1}--\ref{#2}}
% Reference to an algorithm, lower-case.
\def\algref#1{algorithm~\ref{#1}}
% Reference to an algorithm, upper case.
\def\Algref#1{Algorithm~\ref{#1}}
\def\twoalgref#1#2{algorithms \ref{#1} and \ref{#2}}
\def\Twoalgref#1#2{Algorithms \ref{#1} and \ref{#2}}
% Reference to a part, lower case
\def\partref#1{part~\ref{#1}}
% Reference to a part, upper case
\def\Partref#1{Part~\ref{#1}}
\def\twopartref#1#2{parts \ref{#1} and \ref{#2}}

\def\ceil#1{\lceil #1 \rceil}
\def\floor#1{\lfloor #1 \rfloor}
\def\1{\bm{1}}
\newcommand{\train}{\mathcal{D}}
\newcommand{\valid}{\mathcal{D_{\mathrm{valid}}}}
\newcommand{\test}{\mathcal{D_{\mathrm{test}}}}

\def\eps{{\epsilon}}


% Random variables
\def\reta{{\textnormal{$\eta$}}}
\def\ra{{\textnormal{a}}}
\def\rb{{\textnormal{b}}}
\def\rc{{\textnormal{c}}}
\def\rd{{\textnormal{d}}}
\def\re{{\textnormal{e}}}
\def\rf{{\textnormal{f}}}
\def\rg{{\textnormal{g}}}
\def\rh{{\textnormal{h}}}
\def\ri{{\textnormal{i}}}
\def\rj{{\textnormal{j}}}
\def\rk{{\textnormal{k}}}
\def\rl{{\textnormal{l}}}
% rm is already a command, just don't name any random variables m
\def\rn{{\textnormal{n}}}
\def\ro{{\textnormal{o}}}
\def\rp{{\textnormal{p}}}
\def\rq{{\textnormal{q}}}
\def\rr{{\textnormal{r}}}
\def\rs{{\textnormal{s}}}
\def\rt{{\textnormal{t}}}
\def\ru{{\textnormal{u}}}
\def\rv{{\textnormal{v}}}
\def\rw{{\textnormal{w}}}
\def\rx{{\textnormal{x}}}
\def\ry{{\textnormal{y}}}
\def\rz{{\textnormal{z}}}

% Random vectors
\def\rvepsilon{{\mathbf{\epsilon}}}
\def\rvphi{{\mathbf{\phi}}}
\def\rvtheta{{\mathbf{\theta}}}
\def\rva{{\mathbf{a}}}
\def\rvb{{\mathbf{b}}}
\def\rvc{{\mathbf{c}}}
\def\rvd{{\mathbf{d}}}
\def\rve{{\mathbf{e}}}
\def\rvf{{\mathbf{f}}}
\def\rvg{{\mathbf{g}}}
\def\rvh{{\mathbf{h}}}
\def\rvu{{\mathbf{i}}}
\def\rvj{{\mathbf{j}}}
\def\rvk{{\mathbf{k}}}
\def\rvl{{\mathbf{l}}}
\def\rvm{{\mathbf{m}}}
\def\rvn{{\mathbf{n}}}
\def\rvo{{\mathbf{o}}}
\def\rvp{{\mathbf{p}}}
\def\rvq{{\mathbf{q}}}
\def\rvr{{\mathbf{r}}}
\def\rvs{{\mathbf{s}}}
\def\rvt{{\mathbf{t}}}
\def\rvu{{\mathbf{u}}}
\def\rvv{{\mathbf{v}}}
\def\rvw{{\mathbf{w}}}
\def\rvx{{\mathbf{x}}}
\def\rvy{{\mathbf{y}}}
\def\rvz{{\mathbf{z}}}

% Elements of random vectors
\def\erva{{\textnormal{a}}}
\def\ervb{{\textnormal{b}}}
\def\ervc{{\textnormal{c}}}
\def\ervd{{\textnormal{d}}}
\def\erve{{\textnormal{e}}}
\def\ervf{{\textnormal{f}}}
\def\ervg{{\textnormal{g}}}
\def\ervh{{\textnormal{h}}}
\def\ervi{{\textnormal{i}}}
\def\ervj{{\textnormal{j}}}
\def\ervk{{\textnormal{k}}}
\def\ervl{{\textnormal{l}}}
\def\ervm{{\textnormal{m}}}
\def\ervn{{\textnormal{n}}}
\def\ervo{{\textnormal{o}}}
\def\ervp{{\textnormal{p}}}
\def\ervq{{\textnormal{q}}}
\def\ervr{{\textnormal{r}}}
\def\ervs{{\textnormal{s}}}
\def\ervt{{\textnormal{t}}}
\def\ervu{{\textnormal{u}}}
\def\ervv{{\textnormal{v}}}
\def\ervw{{\textnormal{w}}}
\def\ervx{{\textnormal{x}}}
\def\ervy{{\textnormal{y}}}
\def\ervz{{\textnormal{z}}}

% Random matrices
\def\rmA{{\mathbf{A}}}
\def\rmB{{\mathbf{B}}}
\def\rmC{{\mathbf{C}}}
\def\rmD{{\mathbf{D}}}
\def\rmE{{\mathbf{E}}}
\def\rmF{{\mathbf{F}}}
\def\rmG{{\mathbf{G}}}
\def\rmH{{\mathbf{H}}}
\def\rmI{{\mathbf{I}}}
\def\rmJ{{\mathbf{J}}}
\def\rmK{{\mathbf{K}}}
\def\rmL{{\mathbf{L}}}
\def\rmM{{\mathbf{M}}}
\def\rmN{{\mathbf{N}}}
\def\rmO{{\mathbf{O}}}
\def\rmP{{\mathbf{P}}}
\def\rmQ{{\mathbf{Q}}}
\def\rmR{{\mathbf{R}}}
\def\rmS{{\mathbf{S}}}
\def\rmT{{\mathbf{T}}}
\def\rmU{{\mathbf{U}}}
\def\rmV{{\mathbf{V}}}
\def\rmW{{\mathbf{W}}}
\def\rmX{{\mathbf{X}}}
\def\rmY{{\mathbf{Y}}}
\def\rmZ{{\mathbf{Z}}}

% Elements of random matrices
\def\ermA{{\textnormal{A}}}
\def\ermB{{\textnormal{B}}}
\def\ermC{{\textnormal{C}}}
\def\ermD{{\textnormal{D}}}
\def\ermE{{\textnormal{E}}}
\def\ermF{{\textnormal{F}}}
\def\ermG{{\textnormal{G}}}
\def\ermH{{\textnormal{H}}}
\def\ermI{{\textnormal{I}}}
\def\ermJ{{\textnormal{J}}}
\def\ermK{{\textnormal{K}}}
\def\ermL{{\textnormal{L}}}
\def\ermM{{\textnormal{M}}}
\def\ermN{{\textnormal{N}}}
\def\ermO{{\textnormal{O}}}
\def\ermP{{\textnormal{P}}}
\def\ermQ{{\textnormal{Q}}}
\def\ermR{{\textnormal{R}}}
\def\ermS{{\textnormal{S}}}
\def\ermT{{\textnormal{T}}}
\def\ermU{{\textnormal{U}}}
\def\ermV{{\textnormal{V}}}
\def\ermW{{\textnormal{W}}}
\def\ermX{{\textnormal{X}}}
\def\ermY{{\textnormal{Y}}}
\def\ermZ{{\textnormal{Z}}}

% Vectors
\def\vzero{{\bm{0}}}
\def\vone{{\bm{1}}}
\def\vmu{{\bm{\mu}}}
\def\vtheta{{\bm{\theta}}}
\def\vphi{{\bm{\phi}}}
\def\va{{\bm{a}}}
\def\vb{{\bm{b}}}
\def\vc{{\bm{c}}}
\def\vd{{\bm{d}}}
\def\ve{{\bm{e}}}
\def\vf{{\bm{f}}}
\def\vg{{\bm{g}}}
\def\vh{{\bm{h}}}
\def\vi{{\bm{i}}}
\def\vj{{\bm{j}}}
\def\vk{{\bm{k}}}
\def\vl{{\bm{l}}}
\def\vm{{\bm{m}}}
\def\vn{{\bm{n}}}
\def\vo{{\bm{o}}}
\def\vp{{\bm{p}}}
\def\vq{{\bm{q}}}
\def\vr{{\bm{r}}}
\def\vs{{\bm{s}}}
\def\vt{{\bm{t}}}
\def\vu{{\bm{u}}}
\def\vv{{\bm{v}}}
\def\vw{{\bm{w}}}
\def\vx{{\bm{x}}}
\def\vy{{\bm{y}}}
\def\vz{{\bm{z}}}

% Elements of vectors
\def\evalpha{{\alpha}}
\def\evbeta{{\beta}}
\def\evepsilon{{\epsilon}}
\def\evlambda{{\lambda}}
\def\evomega{{\omega}}
\def\evmu{{\mu}}
\def\evpsi{{\psi}}
\def\evsigma{{\sigma}}
\def\evtheta{{\theta}}
\def\eva{{a}}
\def\evb{{b}}
\def\evc{{c}}
\def\evd{{d}}
\def\eve{{e}}
\def\evf{{f}}
\def\evg{{g}}
\def\evh{{h}}
\def\evi{{i}}
\def\evj{{j}}
\def\evk{{k}}
\def\evl{{l}}
\def\evm{{m}}
\def\evn{{n}}
\def\evo{{o}}
\def\evp{{p}}
\def\evq{{q}}
\def\evr{{r}}
\def\evs{{s}}
\def\evt{{t}}
\def\evu{{u}}
\def\evv{{v}}
\def\evw{{w}}
\def\evx{{x}}
\def\evy{{y}}
\def\evz{{z}}

% Matrix
\def\mA{{\bm{A}}}
\def\mB{{\bm{B}}}
\def\mC{{\bm{C}}}
\def\mD{{\bm{D}}}
\def\mE{{\bm{E}}}
\def\mF{{\bm{F}}}
\def\mG{{\bm{G}}}
\def\mH{{\bm{H}}}
\def\mI{{\bm{I}}}
\def\mJ{{\bm{J}}}
\def\mK{{\bm{K}}}
\def\mL{{\bm{L}}}
\def\mM{{\bm{M}}}
\def\mN{{\bm{N}}}
\def\mO{{\bm{O}}}
\def\mP{{\bm{P}}}
\def\mQ{{\bm{Q}}}
\def\mR{{\bm{R}}}
\def\mS{{\bm{S}}}
\def\mT{{\bm{T}}}
\def\mU{{\bm{U}}}
\def\mV{{\bm{V}}}
\def\mW{{\bm{W}}}
\def\mX{{\bm{X}}}
\def\mY{{\bm{Y}}}
\def\mZ{{\bm{Z}}}
\def\mBeta{{\bm{\beta}}}
\def\mPhi{{\bm{\Phi}}}
\def\mLambda{{\bm{\Lambda}}}
\def\mSigma{{\bm{\Sigma}}}

% Tensor
\DeclareMathAlphabet{\mathsfit}{\encodingdefault}{\sfdefault}{m}{sl}
\SetMathAlphabet{\mathsfit}{bold}{\encodingdefault}{\sfdefault}{bx}{n}
\newcommand{\tens}[1]{\bm{\mathsfit{#1}}}
\def\tA{{\tens{A}}}
\def\tB{{\tens{B}}}
\def\tC{{\tens{C}}}
\def\tD{{\tens{D}}}
\def\tE{{\tens{E}}}
\def\tF{{\tens{F}}}
\def\tG{{\tens{G}}}
\def\tH{{\tens{H}}}
\def\tI{{\tens{I}}}
\def\tJ{{\tens{J}}}
\def\tK{{\tens{K}}}
\def\tL{{\tens{L}}}
\def\tM{{\tens{M}}}
\def\tN{{\tens{N}}}
\def\tO{{\tens{O}}}
\def\tP{{\tens{P}}}
\def\tQ{{\tens{Q}}}
\def\tR{{\tens{R}}}
\def\tS{{\tens{S}}}
\def\tT{{\tens{T}}}
\def\tU{{\tens{U}}}
\def\tV{{\tens{V}}}
\def\tW{{\tens{W}}}
\def\tX{{\tens{X}}}
\def\tY{{\tens{Y}}}
\def\tZ{{\tens{Z}}}


% Graph
\def\gA{{\mathcal{A}}}
\def\gB{{\mathcal{B}}}
\def\gC{{\mathcal{C}}}
\def\gD{{\mathcal{D}}}
\def\gE{{\mathcal{E}}}
\def\gF{{\mathcal{F}}}
\def\gG{{\mathcal{G}}}
\def\gH{{\mathcal{H}}}
\def\gI{{\mathcal{I}}}
\def\gJ{{\mathcal{J}}}
\def\gK{{\mathcal{K}}}
\def\gL{{\mathcal{L}}}
\def\gM{{\mathcal{M}}}
\def\gN{{\mathcal{N}}}
\def\gO{{\mathcal{O}}}
\def\gP{{\mathcal{P}}}
\def\gQ{{\mathcal{Q}}}
\def\gR{{\mathcal{R}}}
\def\gS{{\mathcal{S}}}
\def\gT{{\mathcal{T}}}
\def\gU{{\mathcal{U}}}
\def\gV{{\mathcal{V}}}
\def\gW{{\mathcal{W}}}
\def\gX{{\mathcal{X}}}
\def\gY{{\mathcal{Y}}}
\def\gZ{{\mathcal{Z}}}

% Sets
\def\sA{{\mathbb{A}}}
\def\sB{{\mathbb{B}}}
\def\sC{{\mathbb{C}}}
\def\sD{{\mathbb{D}}}
% Don't use a set called E, because this would be the same as our symbol
% for expectation.
\def\sF{{\mathbb{F}}}
\def\sG{{\mathbb{G}}}
\def\sH{{\mathbb{H}}}
\def\sI{{\mathbb{I}}}
\def\sJ{{\mathbb{J}}}
\def\sK{{\mathbb{K}}}
\def\sL{{\mathbb{L}}}
\def\sM{{\mathbb{M}}}
\def\sN{{\mathbb{N}}}
\def\sO{{\mathbb{O}}}
\def\sP{{\mathbb{P}}}
\def\sQ{{\mathbb{Q}}}
\def\sR{{\mathbb{R}}}
\def\sS{{\mathbb{S}}}
\def\sT{{\mathbb{T}}}
\def\sU{{\mathbb{U}}}
\def\sV{{\mathbb{V}}}
\def\sW{{\mathbb{W}}}
\def\sX{{\mathbb{X}}}
\def\sY{{\mathbb{Y}}}
\def\sZ{{\mathbb{Z}}}

% Entries of a matrix
\def\emLambda{{\Lambda}}
\def\emA{{A}}
\def\emB{{B}}
\def\emC{{C}}
\def\emD{{D}}
\def\emE{{E}}
\def\emF{{F}}
\def\emG{{G}}
\def\emH{{H}}
\def\emI{{I}}
\def\emJ{{J}}
\def\emK{{K}}
\def\emL{{L}}
\def\emM{{M}}
\def\emN{{N}}
\def\emO{{O}}
\def\emP{{P}}
\def\emQ{{Q}}
\def\emR{{R}}
\def\emS{{S}}
\def\emT{{T}}
\def\emU{{U}}
\def\emV{{V}}
\def\emW{{W}}
\def\emX{{X}}
\def\emY{{Y}}
\def\emZ{{Z}}
\def\emSigma{{\Sigma}}

% entries of a tensor
% Same font as tensor, without \bm wrapper
\newcommand{\etens}[1]{\mathsfit{#1}}
\def\etLambda{{\etens{\Lambda}}}
\def\etA{{\etens{A}}}
\def\etB{{\etens{B}}}
\def\etC{{\etens{C}}}
\def\etD{{\etens{D}}}
\def\etE{{\etens{E}}}
\def\etF{{\etens{F}}}
\def\etG{{\etens{G}}}
\def\etH{{\etens{H}}}
\def\etI{{\etens{I}}}
\def\etJ{{\etens{J}}}
\def\etK{{\etens{K}}}
\def\etL{{\etens{L}}}
\def\etM{{\etens{M}}}
\def\etN{{\etens{N}}}
\def\etO{{\etens{O}}}
\def\etP{{\etens{P}}}
\def\etQ{{\etens{Q}}}
\def\etR{{\etens{R}}}
\def\etS{{\etens{S}}}
\def\etT{{\etens{T}}}
\def\etU{{\etens{U}}}
\def\etV{{\etens{V}}}
\def\etW{{\etens{W}}}
\def\etX{{\etens{X}}}
\def\etY{{\etens{Y}}}
\def\etZ{{\etens{Z}}}

% The true underlying data generating distribution
\newcommand{\pdata}{p_{\rm{data}}}
\newcommand{\ptarget}{p_{\rm{target}}}
\newcommand{\pprior}{p_{\rm{prior}}}
\newcommand{\pbase}{p_{\rm{base}}}
\newcommand{\pref}{p_{\rm{ref}}}

% The empirical distribution defined by the training set
\newcommand{\ptrain}{\hat{p}_{\rm{data}}}
\newcommand{\Ptrain}{\hat{P}_{\rm{data}}}
% The model distribution
\newcommand{\pmodel}{p_{\rm{model}}}
\newcommand{\Pmodel}{P_{\rm{model}}}
\newcommand{\ptildemodel}{\tilde{p}_{\rm{model}}}
% Stochastic autoencoder distributions
\newcommand{\pencode}{p_{\rm{encoder}}}
\newcommand{\pdecode}{p_{\rm{decoder}}}
\newcommand{\precons}{p_{\rm{reconstruct}}}

\newcommand{\laplace}{\mathrm{Laplace}} % Laplace distribution

\newcommand{\E}{\mathbb{E}}
\newcommand{\Ls}{\mathcal{L}}
\newcommand{\R}{\mathbb{R}}
\newcommand{\emp}{\tilde{p}}
\newcommand{\lr}{\alpha}
\newcommand{\reg}{\lambda}
\newcommand{\rect}{\mathrm{rectifier}}
\newcommand{\softmax}{\mathrm{softmax}}
\newcommand{\sigmoid}{\sigma}
\newcommand{\softplus}{\zeta}
\newcommand{\KL}{D_{\mathrm{KL}}}
\newcommand{\Var}{\mathrm{Var}}
\newcommand{\standarderror}{\mathrm{SE}}
\newcommand{\Cov}{\mathrm{Cov}}
% Wolfram Mathworld says $L^2$ is for function spaces and $\ell^2$ is for vectors
% But then they seem to use $L^2$ for vectors throughout the site, and so does
% wikipedia.
\newcommand{\normlzero}{L^0}
\newcommand{\normlone}{L^1}
\newcommand{\normltwo}{L^2}
\newcommand{\normlp}{L^p}
\newcommand{\normmax}{L^\infty}

\newcommand{\parents}{Pa} % See usage in notation.tex. Chosen to match Daphne's book.

\DeclareMathOperator*{\argmax}{arg\,max}
\DeclareMathOperator*{\argmin}{arg\,min}

\DeclareMathOperator{\sign}{sign}
\DeclareMathOperator{\Tr}{Tr}
\let\ab\allowbreak


% Recommended, but optional, packages for figures and better typesetting:
\usepackage{microtype}
\usepackage{graphicx}
\usepackage{subfigure}
\usepackage{booktabs} % for professional tables

% hyperref makes hyperlinks in the resulting PDF.
% If your build breaks (sometimes temporarily if a hyperlink spans a page)
% please comment out the following usepackage line and replace
% \usepackage{icml2023} with \usepackage[nohyperref]{icml2023} above.
\usepackage{hyperref}


% Attempt to make hyperref and algorithmic work together better:
\newcommand{\theHalgorithm}{\arabic{algorithm}}

% Use the following line for the initial blind version submitted for review:
%\usepackage{icml2023}

% If accepted, instead use the following line for the camera-ready submission:
 \usepackage[accepted]{icml2023}

% For theorems and such
\usepackage{amsmath}
\usepackage{amssymb}
\usepackage{mathtools}
\usepackage{amsthm}

% if you use cleveref..
\usepackage[capitalize,noabbrev]{cleveref}

% Todonotes is useful during development; simply uncomment the next line
%    and comment out the line below the next line to turn off comments
%\usepackage[disable,textsize=tiny]{todonotes}
\usepackage[textsize=tiny]{todonotes}
\usepackage{cancel}

\newcommand{\fix}{\marginpar{FIX}}
\newcommand{\new}{\marginpar{NEW}}
\def\defeq{\triangleq} % defined equal to
\newcommand{\iid}{\textrm{i.i.d.}\xspace}
\newcommand{\para}[1]{\textbf{#1}\ \ }
\newcommand{\indic}[1]{\mathbf{1}(#1)}
\newcommand{\qtext}[1]{\quad\text{#1}\quad} 
\newcommand{\evec}{\boldsymbol{e}}
\newcommand{\x}[1]{x^{(#1)}}
\newcommand{\z}[1]{z^{(#1)}}
\newcommand{\X}[1]{X^{(#1)}}
\newcommand{\bz}{\boldsymbol{z}}
\newcommand{\bx}{\boldsymbol{x}}
\newcommand{\bu}{\textbf{u}}
\newcommand{\bbf}{\boldsymbol{f}}

\newcommand{\hsnorm}[1]{\norm{#1}_\mathrm{HS}}
\newcommand{\llnorm}[1]{\norm{#1}_{\nu}}
\newcommand{\set}[1]{\left \{ #1 \right \}}
\newcommand{\tr}{\text{tr}} % trace
\newcommand{\divger}{\mathrm{div}} 
\newcommand{\rbf}{\mathrm{rbf}} 
\newcommand{\imq}{\mathrm{imq}}
\newcommand{\fin}{\bbf_{\|}} 
\newcommand{\fperp}{\bbf_{\perp}}
\newcommand{\uperp}{\bu_{\perp}}
\newcommand{\vperp}{\bv_{\perp}}
\newcommand{\ppperp}{p_{\perp}}
\newcommand{\Kuu}{\bK_{\bu\bu}}
\newcommand{\Kuf}{\bK_{\bu\bbf}}
\newcommand{\Kfu}{\bK_{\bbf\bu}}
\newcommand{\Kff}{\bK_{\bbf\bbf}}
\newcommand{\Qff}{\bQ_{\bbf\bbf}}
\newcommand{\bQ}{\textbf{Q}}
\newcommand{\bK}{\textbf{K}}
\newcommand{\bX}{\textbf{X}}
\newcommand{\Real}{\mathbb{R}}
\newcommand{\bI}{\textbf{I}}
\newcommand{\bZ}{\textbf{Z}}
\newcommand{\bC}{\textbf{C}}
\newcommand{\bV}{\textbf{V}}
\newcommand{\bS}{\textbf{S}}
\newcommand{\bA}{\textbf{A}}

\newcommand{\K}{\textbf{K}}
\newcommand{\f}{\textbf{f}}
\newcommand{\y}{\textbf{y}}
\newcommand{\bk}{\textbf{k}}
\newcommand{\m}{\textbf{m}}
\newcommand{\bfmu}{\boldsymbol{\mu}}
\newcommand{\bLambda}{\boldsymbol{\Lambda}}

\newcommand{\datadim}{L}
\newcommand{\ind}{i}
\newcommand{\orderdistr}{order-policy }

\newcommand{\methodname}{LO-ARM}
\newcommand{\comm}[1]{}

%%%%%%%%%%%%%%%%%%%%%%%%%%%%%%%%
% THEOREMS
%%%%%%%%%%%%%%%%%%%%%%%%%%%%%%%%
\theoremstyle{plain}
\newtheorem{theorem}{Theorem}[section]
\newtheorem{proposition}[theorem]{Proposition}
\newtheorem{lemma}[theorem]{Lemma}
\newtheorem{corollary}[theorem]{Corollary}
\theoremstyle{definition}
\newtheorem{definition}[theorem]{Definition}
\newtheorem{assumption}[theorem]{Assumption}
\theoremstyle{definition}
\newtheorem{remark}[theorem]{Remark}


% The \icmltitle you define below is probably too long as a header.
% Therefore, a short form for the running title is supplied here:
% \icmltitlerunning{Submission and Formatting Instructions for ICML 2023}

\begin{document}

\twocolumn[
\icmltitle{New Bounds for Sparse Variational Gaussian Processes %without Imposing the Conditional Prior Approximation
}

% It is OKAY to include author information, even for blind
% submissions: the style file will automatically remove it for you
% unless you've provided the [accepted] option to the icml2023
% package.

% List of affiliations: The first argument should be a (short)
% identifier you will use later to specify author affiliations
% Academic affiliations should list Department, University, City, Region, Country
% Industry affiliations should list Company, City, Region, Country

% You can specify symbols, otherwise they are numbered in order.
% Ideally, you should not use this facility. Affiliations will be numbered
% in order of appearance and this is the preferred way.
%\icmlsetsymbol{equal}{*}

\begin{icmlauthorlist}
\icmlauthor{Michalis K. Titsias}{comp}
\end{icmlauthorlist}

\icmlaffiliation{comp}{Google DeepMind}

\icmlcorrespondingauthor{Michalis K. Titsias}{mtitsias@google.com}

% You may provide any keywords that you
% find helpful for describing your paper; these are used to populate
% the "keywords" metadata in the PDF but will not be shown in the document
\icmlkeywords{Machine Learning, ICML}

\vskip 0.3in
]

% this must go after the closing bracket ] following \twocolumn[ ...

% This command actually creates the footnote in the first column
% listing the affiliations and the copyright notice.
% The command takes one argument, which is text to display at the start of the footnote.
% The \icmlEqualContribution command is standard text for equal contribution.
% Remove it (just {}) if you do not need this facility.

%\printAffiliationsAndNotice{}  % leave blank if no need to mention equal contribution
\printAffiliationsAndNotice{\icmlEqualContribution} % otherwise use the standard text.

\begin{abstract}
Sparse variational Gaussian processes (GPs) construct tractable posterior approximations to GP  models. At the core of these methods is the assumption that the true posterior distribution over training function values $\f$ and inducing variables $\bu$ is approximated by a variational distribution that incorporates the conditional GP prior $p(\f | \bu)$ in its factorization. While  this assumption is considered as fundamental, 
%Imposing this conditional prior in the approximation is believed to be a fundamental  requirement to obtain scalable GPs. 
we show that for model training we can relax it through the use of a more general variational distribution $q(\f | \bu)$ that depends on $N$ extra parameters, where  $N$ is the number of training examples. In GP regression, we can analytically optimize
the evidence lower bound over the extra parameters and express a tractable collapsed bound that is tighter than the previous bound. The new bound is also amenable to stochastic %gradient 
optimization and its implementation requires minor modifications to existing sparse GP code. 
Further, we also describe extensions to non-Gaussian likelihoods. 
On several %regression 
datasets we demonstrate that our method can reduce  bias when learning the %model
hyperpaparameters and can lead to better predictive performance. 
%such as applications to GP Poisson regression. 
\end{abstract}

\section{Introduction}

Node classification is a fundamental task in graph analysis, with a wide range of applications such as item tagging \cite{Mao2020ItemTF}, user profiling \cite{Yan2021RelationawareHG}, and financial fraud detection \cite{Zhang2022eFraudComAE}. Developing effective algorithms for node classification is crucial, as they can significantly impact commercial success. For instance, US banks lost 6 billion USD to fraudsters in 2016. Therefore, even a marginal improvement in fraud detection accuracy could result in substantial financial savings.

Given its practical importance, node classification has been a long-standing research focus in both academia and industry. The earliest attempts to address this task adopted techniques such as Laplacian regularization \cite{belkin2006manifold}, graph embeddings \cite{yang2016revisiting}, and label propagation \cite{zhu2003semi}. Over the past decade, GNN-based methods have been developed and have quickly become prominent due to their superior performance, as demonstrated by works such as \citet{kipf2017GCN}, \citet{velickovic2018GAT}, and \citet{hamilton2017SAGE}. Additionally, the incorporation of encoded textual information has been shown to further complement GNNs' node features, enhancing their effectiveness \cite{jin2023patton, zhao2022GLEM}.

Inspired by the recent success of LLMs, there has been a surge of interest in leveraging LLMs for node classification \cite{li2023survey}. LLMs, pre-trained on extensive text corpora, possess context-aware knowledge and superior semantic comprehension, overcoming the limitations of the non-contextualized shallow embeddings used by traditional GNNs. Typically, supervised methods fall into three categories: Encoder, Reasoner, and Predictor. In the Encoder paradigm, LLMs employ their vast parameters to encode nodes' textual information, producing more expressive features that surpass shallow embeddings \cite{Zhu2024ENGINE}. The Reasoner approach utilizes LLMs' reasoning capabilities to enhance node attributes and the task descriptions with a more detailed text \cite{chen2024exploring, he2023TAPE}. This generated text augments the nodes' original information, thereby enriching their attributes. Lastly, the Predictor role involves LLMs integrating graph context through graph encoders, enabling direct text-based predictions  \cite{chen23llaga,tang2023graphgpt,chai2023graphllm,Huang2024GraphAdapter}. For zero-shot learning with LLMs, methods can be categorized into two types: Direct Inference and Graph Foundation Models (GFMs). Direct Inference involves guiding LLMs to directly perform classification tasks via crafted prompts \cite{Huang2023CanLE}. In contrast, GFMs entail pre-training on extensive graph corpora before applying the model to target graphs, thereby equipping the model with specialized graph intelligence \cite{li2024zerog}. An illustration of these methods is shown in Figure \ref{fig:llm_role}. 

Despite tremendous efforts and promising results, the design principles for LLM-based node classification algorithms remain elusive. Given the significant training and inference costs associated with LLMs, practitioners may opt to deploy these algorithms only when they provide substantial performance enhancements compared to costs. This study, therefore, seeks to identify \textbf{(1) the most suitable settings for each algorithm category, and (2) the scenarios where LLMs surpass traditional LMs such as BERT}. While recent work like GLBench \cite{Li2024GLBench} has evaluated various methods using consistent data splits in semi-supervised and zero-shot settings, differences in backbone architectures and implementation codebases still hinder fair comparisons and rigorous conclusions. To address these limitations, we introduce a new benchmark that further standardizes backbones and codebases. Additionally, we extend GLBench by incorporating three new E-Commerce datasets relevant to practical applications and expanding the evaluation settings. Specifically, we assess the impact of supervision signals (e.g., supervised, semi-supervised), different language model backbones (e.g., RoBERTa, Mistral, LLaMA, GPT-4o), and various prompt types (e.g., CoT, ToT, ReAct). These enhancements enable a more detailed and reliable analysis of LLM-based node classification methods. In summary, our contributions to the field of LLMs for graph analysis are as follows:


% A fair comparison necessitates a benchmark that evaluates all methods using consistent data splitting ratios, learning paradigms, backbone architectures, and implementation codebases. A very recent work, GLBench~\cite{Li2024GLBench}, tested various methods on several datasets in a semi-supervised/zero-shot setting, maintaining the same data splits. However, differences in the underlying backbones and implementation codebases still pose challenges for a fair comparison and drawing rigorous conclusions of the above questions. This paper introduces a benchmark that further standardizes the backbones and implementation codebases. Moreover, we expand upon GLBench by providing additional datasets and evaluation settings. Specifically, we include three new datasets from the E-Commerce sector, which are more relevant for practical commercial applications. We also assess the influence of supervision signals (e.g., supervised or semi-supervised), various language model backbones (e.g., RoBERTa, Mistral, GPT-4o), and prompts (e.g., CoT, ToT, and ReAct). These datasets and settings enable a detailed analysis of the aforementioned questions. 



% However, existing works lack the necessary standardization for such comparisons. An algorithm that performs exceptionally well in its original paper might underperform when used as a baseline in subsequent studies. This discrepancy often arises from variations in data splitting, learning paradigms, backbone architectures, and implementation codebases.  The backbone architecture and implementations are adopted from the original papers, which 

% To address this issue, this paper introduces a testbed for LLM-based node classification algorithms and conducts extensive experiments to derive insights and guidelines. 

\begin{itemize}
    \item \textbf{A Testbed:} We release LLMNodeBed, a PyG-based testbed designed to facilitate reproducible and rigorous research in LLM-based node classification algorithms. The initial release includes ten datasets, eight LLM-based algorithms, and three learning configurations. LLMNodeBed allows for easy addition of new algorithms or datasets, and a single command to run all experiments, and to automatically generate all tables included in this work.
    
    \item \textbf{Comprehensive Experiments:} By training and evaluating over 2,200 models, we analyzed how the learning paradigm, homophily, language model type and size, and prompt design impact the performance of each algorithm category.
    
    \item \textbf{Insights and Tips:} Detailed experiments were conducted to analyze each influencing factor. We identified the settings where each algorithm category performs best and the key components for achieving this performance. Our work provides intuitive explanations, practical tips, and insights about the strengths and limitations of each algorithm category.
\end{itemize}




%It has been a research focus in both academia and industry due to its wide range of applications, including item tagging \cite{Mao2020ItemTF}, user profiling \cite{Yan2021RelationawareHG}, and financial fraud detection \cite{Zhang2022eFraudComAE}. 


%Building effective algorithms for node classification is a long-standing topic as it has a direct impact on commercial success \cite{Lo2022InspectionLSG}.

%Before the popularity of LLMs, node classification is typically tackled by graph neural networks (GNNs) or language models (LMs) such as BERT \cite{Devlin2019BERTPO}. GNNs \cite{kipf2017GCN,velickovic2018GAT,hamilton2017SAGE} enhance node representations by aggregating information from neighboring nodes, thereby capturing the structural context essential for accurate classification. In contrast, LMs \cite{Wang2022e5-large, Liu2019roberta} focus on semantic representations by encoding the textual information associated with each node, transforming the node classification into a text classification task. The encoded textual information can further complement GNNs' node features \cite{jin2023patton, zhao2022GLEM}. Yifei: I think the current intro is too long, to move it to related works

%Over the past decade, we have witnessed great progress in node classification algorithms. The classical ones include Graph Neural Networks (GNNs) \cite{kipf2017GCN,velickovic2018GAT,hamilton2017SAGE} and additional language modeling to enhance the node features \cite{jin2023patton, zhao2022GLEM}. Recently, there has been a surge of interest in applying LLMs for node classification \cite{li2023survey}. In these studies, the roles performed by LLMs can be primarily 


% Despite the importance of this area, the literature of LLM-based node classification is scattered: the algorithms are evaluated under different datasets, learning paradigms, baselines, and implementation codebases. The purpose of this work is to perform rigorous comparisons among algorithms, as well as to open-source our software for anyone to replicate and extend our analysis. This manuscript investigates the question: \emph{How useful are LLMs for node classification under a fair setting?}

% To answer this question, we implement and tune eight LLM-based node classification algorithms, to compare them across ten datasets and three learning paradigms.  There are four major takeaways from our investigations: (1) \textbf{LLM-as-Encoder is effective for low-homophily graphs:} These methods outperform classic LM counterparts on low-homophily graphs, with the advantages being more obvious under limited supervision.
% (2) \textbf{LLM-as-Reasoner is the most effective when LLMs have prior knowledge of the target graph:} These methods achieve superior performance on datasets where the LLMs possess prior knowledge like academic and web link datasets, and benefit from more powerful models like GPT-4o. 
% (3) \textbf{LLM-as-Predictor methods is highly effective when labeled data is abundant}: Predictor methods require extensive supervision for model training, with their performance improving as larger LLMs adhering to scaling laws \cite{Kaplan2020ScalingLF} are utilized. Among different LLMs, Mistral-7B \cite{Jiang2023Mistral7B} consistently serves as a robust backbone. (4) \textbf{Zero-shot methods are most effective when neighbor information is injected:} Although Graph Foundation Models (GFMs) \cite{liu2023one, li2024zerog, Zhu2024GraphCLIPET} outperform open-source LLMs in zero-shot settings, they still lag behind advanced models like GPT-4o. The most effective zero-shot approaches involve injecting neighbor information to guide LLMs for direct inference.

% As a result of this paper, we release LLMNodeBed, a PyTorch-based testbed designed to facilitate reproducible and rigorous research in node classification algorithms. The initial release includes ten datasets, eight algorithms, three learning configurations, and the infrastructure to run all experiments. Our experimental framework can be easily extended to include new methods and datasets. We are committed to updating this repository with new algorithms and datasets and welcome pull requests from fellow researchers to ensure its ongoing development.


%While a myriad of algorithms exists, diverse datasets, architectures, learning configurations, and implementation codebases, rendering fair and realistic comparisons difficult and conclusions inconsistent. Inspired by standardized benchmarks in computer vision like ImageNet, this paper conducts a rigorous comparison of various LLM-based node classification methods to assess the true efficacy of LLMs. This investigation addresses the following research question:

%\textit{Under What Circumstances do LLMs Help Node Classification Task?}

%At a first step, we implement LLMNodeBed, a codebase and testbed for node classification with LLMs. It includes ten multi-domain graph datasets with varying scales and levels of homophily, supports eight representative algorithms that represent diverse LLM roles, and offers three learning configurations: semi-supervised, fully-supervised, and zero-shot. Through extensive experiments, we provide empirical insights into when LLMs contribute to node classification performance: 



% In summary, we make the following contributions: 

% \begin{enumerate}
%     \item \textbf{LLMNodeBed:} We introduce LLMNodeBed, a comprehensive and extensible testbed for evaluating LLM-based node classification algorithms. It comprises ten datasets, eight representative algorithms, and three learning scenarios, and can easily accommodate new datasets, methods, and backbones.
%     \item \textbf{Comprehensive Evaluation:} We conduct extensive empirical analysis across different datasets, algorithms, and learning settings to elucidate the efficacy of different LLM roles in node classification performance. 
%     \item \textbf{Practical Guidelines:} Based on our findings, we provide actionable guidelines for effectively applying LLMs to diverse real-world node classification tasks, enhancing their performance and applicability in various scenarios.
% \end{enumerate}

% \section{Background}

% In this work, we focus on two different model families: random Fourier features (RFFs) and deep neural networks (DNNs) for transfer learning with informative priors.
% What these model families have in common is that they can be overparameterized.

%\subsection{Random Fourier features}

% MCH: MOVED TO CASE A

%\subsection{Transfer learning with informative priors}

% MCH: MOVED TO CASE B



\section{Proposed Method: Tighter Bounds
\label{sec:proposedmethod}
}

Remark \ref{remark1} suggests that 
it would be useful to tighten the collapsed bound 
%in (\ref{eq:collapsedbound_old}) 
in order to reduce underfitting bias 
and match better exact GP training. 
Remark \ref{remark2} suggests that one way to tighten the bound is to replace %the conditional GP 
$p(\f | \bu)$, in the variational approximation in (\ref{eq:pfuqu}), with another distribution 
that can better approximate 
$p(\f | \bu, \y)$. Next we develop a method that does this while keeping the cost unchanged. 

Let us write the 
exact form of $p(\f | \bu, \y)$. By noting that this quantity is the exact posterior over $\f$ in a GP regression model with joint $p(\y | \f) p(\f | \bu)$ 
%(where $p(\f | \bu)$ is now the effective GP prior) 
we conclude that this %posterior 
is 
$$
p(\f | \bu, \y) = \mathcal{N}\left(\f| \m(\y,\bu), 
(\widetilde{\bK}_{\f \f}^{-1} + \frac{1}{\sigma^2} \bI)^{-1} \right),
$$
where $\m(\y,\bu)  = \E[\f | \bu] + \widetilde{\bK}_{\f \f} (\widetilde{\bK}_{\f \f} + \sigma^2 \bI)^{-1} (\y - \E[\f | \bu]) $
with $\E[\f | \bu] = \bK_{\f \bu} \bK_{\bu \bu}^{-1} \bu$ and $\widetilde{\bK}_{\f \f} = \bK_{\f \f} - \bQ_{\f \f}$. Note that under this notation, 
$p(\f | \bu) = \mathcal{N}(\f | \E[\f | \bu], \widetilde{\bK}_{\f \f})$. We will construct a new $q(\f | \bu)$ 
that keeps the same mean $\E[\f | \bu]$ 
as $p(\f | \bu)$ but it replaces $\widetilde{\bK}_{\f \f}$ with a closer approximation to the 
% exact 
covariance 
% matrix 
$(\widetilde{\bK}_{\f \f}^{-1} + \frac{1}{\sigma^2} \bI)^{-1}$ of $p(\f | \bu, \y)$. We first 
write this %latter 
matrix as 
\begin{equation}
(\widetilde{\bK}_{\f \f}^{-1} + \frac{1}{\sigma^2} \bI)^{-1}
= \widetilde{\bK}_{\f \f}^{\frac{1}{2}}
( \bI + \frac{1}{\sigma^2} \widetilde{\bK}_{\f \f})^{-1} 
\widetilde{\bK}_{\f \f}^{\frac{1}{2}}.
\label{eq:exact_cov_pfuy}
\end{equation}
Then we approximate the inverse 
$( \bI + \frac{1}{\sigma^2} \widetilde{\bK}_{\f \f})^{-1}$ by a diagonal matrix $\bV = \text{diag}(v_1, \ldots,v_N)$ of $N$ variational parameters $v_i > 0$. In other words,  in the initial $q(\f, \bu) = p(\f|\bu)q(\bu)$ we will replace $p(\f|\bu)$ by 
\begin{equation}
q(\f|\bu) = \mathcal{N}(\f | \bK_{\f \bu} \bK_{\bu \bu}^{-1} \bu, (\bK_{\f \f} - \bQ_{\f \f})^{\frac{1}{2}} \bV
(\bK_{\f \f} - \bQ_{\f \f})^{\frac{1}{2}}).
\label{eq:qfu}
\end{equation}
The ELBO now is written as 
\begin{align} 
 & \int q(\f | \bu) q(\bu) \log \frac{p(\y | \f) p(\f | \bu) p(\bu)}{q(\f | \bu) q(\bu)} d \f d \bu = \nonumber \\ 
& \int \! \!  q(\bu) \! \left\{ \! \log \frac{e^{\E_{q(\f | \bu)}[\log p(\y | \f)]} p(\bu)}{q(\bu)} \! - \! \text{KL}[q(\f | \bu) || p(\f | \bu)] 
\! \right\} \! d \bu \nonumber 
\end{align}
and the challenge is to see whether 
$\text{KL}[q(\f | \bu) || p(\f | \bu)]$ 
and $\E_{q(\f | \bu)}[\log p(\y | \f)]$ 
are computable in $\mathcal{O}(N M^2)$ time. 
We have the following results (proofs are in  \Cref{app:detailsNewbounds}).
\begin{lemma}
\label{lem:KLqfupfu}
\emph{$\text{KL}[q(\f | \bu) || p(\f | \bu)] 
= \frac{1}{2} \sum_{i=1}^N (v_i - \log v_i - 1)$}.
\end{lemma}
\begin{lemma} 
\label{lem:Expqfu_loglik}
Let us denote the diagonal elements of \emph{$\bK_{\f \f} - \bQ_{\f \f}$} as 
\emph{$k_{ii} - q_{ii}$} for \emph{$i=1,\ldots,N$}. Then  
\emph{\begin{align}
& \E_{q(\f | \bu)}[\log p(\y | \f)] \nonumber \\ 
& \! = \! \log \mathcal{N}(\y | \bK_{\f \bu}
\bK_{\bu \bu }^{-1} \bu, \sigma^2 \bI)
- \frac{1}{2 \sigma^2} 
\sum_{i=1}^N v_i (k_{ii} - q_{ii}). 
\end{align}}
\end{lemma}
By combining the two lemmas the full bound is written as 
\begin{align} 
& \int \! \!  q(\bu) \log \frac{  \mathcal{N}(\y | \bK_{\f \bu}
\bK_{\bu \bu }^{-1} \bu, \sigma^2 \bI) p(\bu)}{q(\bu)}  d \bu \nonumber \\
& - \frac{1}{2} 
\sum_{i=1}^N \left\{  v_i \left(1 + \frac{k_{ii} - q_{ii}}{\sigma^2}\right) - \log v_i -1 \right\}.  
\label{eq:newcollapsedbound_with_vis}
\end{align}
\begin{proposition}%[new collapsed bound]
Maximizing the bound in (\ref{eq:newcollapsedbound_with_vis}) with respect to \emph{$q(\bu)$}
and each \emph{$v_i$} results in the 
optimal settings \emph{$q^*(\bu) \propto  \mathcal{N}(\y | \bK_{\f \bu}
\bK_{\bu \bu }^{-1} \bu, \sigma^2 \bI) p(\bu)$} and  
\emph{$v_i^* = \left(1 + \frac{k_{ii} - q_{ii}}{\sigma^2} \right)^{-1}$}. By substituting these values 
back to (\ref{eq:newcollapsedbound_with_vis}) we obtain
\emph{\begin{equation} 
\mathcal{F}_{new} \! = \! \log  \mathcal{N}(\y |{\bf 0},   \bQ_{\f \f} + \sigma^2 \bI) 
 - \frac{1}{2}  
\sum_{i=1}^N \log \left(\! 1 + \frac{k_{ii} - q_{ii}}{\sigma^2} \! \right).   
\label{eq:newcollapsedbound}
\end{equation}
}
\label{prop:newbound}
\end{proposition}
The first term is the 
DTC log likelihood as in the original bound in (\ref{eq:collapsedbound_old}),  
but the regularization term 
makes the bound tighter, 
i.e., $\log p(\y) \geq \mathcal{F}_{new} \geq \mathcal{F}$, due to the inequality $\log(a + 1) \leq a$. Also since $\log(a + 1) < a$ for all $a>0$, if $\bK_{\f \f} \neq \bQ_{\f \f}$ 
(so there is at least one $k_{ii} - q_{ii} > 0$), then $\mathcal{F}_{new} > \mathcal{F}$. This means that $\mathcal{F}_{new}$ is strictly better than $\mathcal{F}$ unless both bounds match exactly the log marginal likelihood. 

Clearly, $\mathcal{F}_{new}$ has $\mathcal{O}(N M^2)$ cost and its implementation requires a minor modification to the initial bound. The optimal $q^*(\bu)$
is the same as in the initial SVGP method, while 
an interpretation of the optimal $v_i^*$ values
is the following.  
%\begin{remark}
%Recall that $\log(a + 1) < a$ for $ \forall a>0$. Thus, if $\bK_{\f \f} \neq \bQ_{\f \f}$ (so there is at least one $k_{ii} - q_{ii} > 0$),  $\mathcal{F}_{new} > \mathcal{F}$ which means that $\mathcal{F}_{new}$ is strictly better than $\mathcal{F}$ unless both bounds match exactly the log marginal likelihood. 
%\end{remark}

\begin{remark}
The diagonal matrix $\bV^*$ (with the optimal $v_i^*$ values in its diagonal) is the inverse obtained after zeroing out the off-diagonal elements of $\bI + \frac{1}{\sigma^2}(\bK_{\f \f} - \bQ_{\f \f})$, 
i.e., $\bV^* = \text{diag}[\bI + \frac{1}{\sigma^2}(\bK_{\f \f} - \bQ_{\f \f})]^{-1}$ which approximates 
$(\bI + \frac{1}{\sigma^2}(\bK_{\f \f} - \bQ_{\f \f}))^{-1}$ 
in \Cref{eq:exact_cov_pfuy}. %Also note that in the ordering of positive definite matrices it holds $\bV^* \leq \bI$, from which it follows that $q(\f | \bu)$ has smaller covariance than $p(\f | \bu)$ and more accurately approximates the covariance of $p(\f | \bu, \y)$. 
%The latter %,  as implied by \Cref{eq:exact_cov_pfuy}, 
%has also smaller covariance than $p(\f | \bu)$. 
\end{remark}

%Finally, as we discuss in related work our bound is also better than the recent bound on the log determinant by \citet{artemevburt2021cglb}.  

\subsection{Predictions
\label{sec:predictions}
} 

To perform 
predictions we will be using 
the same predictive posterior 
from \Cref{eq:variational_posteriorGP}, i.e., 
$
q(\f_* | \y) =
\int p(\f_* | \bu) q(\bu) d \bu, 
$
where the optimal $q^*(\bu)$ (see \Cref{app:detailsSVGP}) 
is exactly the same as in 
the standard SVGP method. The alternative expression (and strictly speaking more appropriate 
since our variational approximation is $q(\f | \bu) q(\bu)$) is given by 
\begin{equation}
q_{high\_cost}(\f_* | \y) 
\! = \! \! \int p(\f_* | \f, \bu) q(\f |  \bu) q(\bu)  d \f d \bu. 
\end{equation}
But this is expensive since it has cost $\mathcal{O}(N^3)$. The reason is that $\int p(\f_* | \f, \bu) q(\f |  \bu) d \f$ does not simplify anymore since $q(\f |  \bu)$ 
is not the conditional GP, which 
means that $p(\f_* | \f, \bu)$ and $q(\f | \bu)$ are not consistent 
under the GP prior. 
Nevertheless, 
$q(\f_* | \y)$ and $q_{high\_cost}(\f_* | \y)$ have exactly the same mean,  since $q(\f | \bu)$ and $p(\f | \bu)$ have the 
same mean.
%but the tractable $q$ will give higher variances than  $q_{high\_cost}$. 

\subsection{Stochastic Minibatch Training
\label{sec:stochasticopt}}

The initial SVGP method \cite{titsias2009variational} does the training in a batch mode where all data are used in each optimization step. Stochastic optimization using minibatches was proposed by \citet{hensman2013gaussian}.  
Here, we apply our new approximation to this stochastic method. 

We start from  \Cref{eq:newcollapsedbound_with_vis},
and substitute only the optimal values for each $v_i$
without using the optimal setting for $q(\bu)$. This results in the uncollapsed
bound
\begin{align} 
& \sum_{i=1}^N \biggl\{  \E_{q(\bu)} [\log \mathcal{N}(y_i | \bk_{f_i \bu}
\bK_{\bu \bu }^{-1} \bu, \sigma^2 )]  \nonumber \\
& - \frac{1}{2}  \log\left(1+\frac{k_{ii} - q_{ii}}{\sigma^2} \right)  \biggr\} 
  - \text{KL}[q(\bu) || p(\bu)],
\label{eq:newuncollapsedbound}
\end{align}
where  $\bk_{f_i \bu}$ is the $1 \times M$ vector of all kernel 
values between the training input $\bx_i$ and the inducing inputs $\bZ$, while 
the expectation under $q(\bu)$ in the first line is 
analytic; see \citet{hensman2013gaussian}. %and \Cref{app:detailsNewbounds} for details.  
Then, we can apply stochastic gradient methods to optimize 
$q(\bu)$ and the hyperparameters  by subsampling 
data minibatches to deal with the  sum over the $N$ training points.  
Clearly, the above bound is strictly better than 
the previous uncollapsed bound in \citet{hensman2013gaussian},
since $- \frac{1}{2 \sigma^2} (k_{ii} - q_{ii}) \leq -\frac{1}{2} \log\left(1 + \frac{k_{ii} - q_{ii}}{\sigma^2} \right)$. 

The most common parametrization of $q(\bu)$
is $q(\bu) = \mathcal{N}(\bu | \m, \bS)$ where  the mean vector $\m$ and covariance matrix $\bS$ are    
variational parameters.  Another popular 
parametrization, for instance used as the default in GPflow \cite{GPflow17}, is the whiten 
parametrization that we consider in our experiments. %see \Cref{app:whiten} for a review.  
For any choice of
$q(\bu)$,  the above bound is always tighter than its corresponding 
 previous uncollapsed bound and requires minor modifications to existing 
 implementations.
% , i.e., to replace the previous  term $- \frac{1}{2 \sigma^2} (k_{ii} - q_{ii})$  by $-\frac{1}{2} \log\left(\frac{k_{ii} - q_{ii}}{\sigma^2} + 1 \right)$. 
        

\subsection{Non-Gaussian Likelihoods
\label{sec:nongaussian}}

Consider a factorized  likelihood $p(\y | \f) = \prod_{i=1}^N p(y_ i | f_i)$ 
where  $p(y_ i | f_i)$ is non-Gaussian, e.g., Bernoulli  for binary outputs  
or Poisson for counts.  
In this non-conjugate setting the sparse 
variational GP approximation imposes the same form for the variational distribution, i.e., $q(\f, \bu) = p(\f | \bu) q(\bu)$ 
where $p(\f | \bu)$ is the 
conditional GP prior. As shown in several works \cite{Chai12,hensman2015scalable,lloyd15,Dezfouli15,Sheth15}, this leads to the bound 
 \begin{equation}
 \sum_{i=1}^N 
\E_{q(f_i)} [\log p(y_i | f_i)]   - \text{KL}[q(\bu) || p(\bu)],
\label{eq:standard_nonconjugate_bound}
 \end{equation} 
 where $q(f_i) = \int p(\f  | \bu) q(\bu ) d \f_{-i} d \bu$ is the marginal  over
 $f_i = f(\bx_i)$  with respect to the approximate posterior $q(\f, \bu)$. Given 
 that  $q(\bu)$ is  Gaussian with mean $\m$ and covariance 
 $\bS$,  $q(f_i)$ can be computed fast in $\mathcal{O}(M^2)$ time (after precomputing the Cholesky factorization of
 $\bK_{\bu \bu}$) as follows 
 \begin{equation}
 q(f_i)  = \mathcal{N}(f_i | \bk_{f_i \bu} \bK_{\bu \bu}^{-1} \m, k_{ii} - q_{ii} + \bk_{f_i \bu} \bK_{\bu \bu}^{-1} \bS \bK_{\bu \bu}^{-1} \bk_{\bu f_i}). 
 \end{equation}
For the discussion next it is useful to observe that the efficiency when computing $q(f_i)$ comes from $p(\f | \bu)$ being a conditional GP prior, so 
expressing $p(f_i | \bu)$ is trivial. 

Suppose now that we wish to impose the more structured variational 
approximation $q(\f, \bu) = q(\f | \bu)  q(\bu)$ where
$q(\bu) = \mathcal{N}(\bu | \m, \bS)$ and $q(\f | \bu)$ 
is given by 
\Cref{eq:qfu}. The bound %(see \Cref{app:nonGaussian}) 
can be written as
\begin{align}
& \sum_{i=1}^N 
\E_{q(f_i)} [\log p(y_i | f_i)] - 
 \frac{1}{2} \sum_{i=1}^N (v_i - \log v_i - 1)
\nonumber \\
& - \text{KL}[q(\bu) || p(\bu)],
\label{eq:nonGaussian_bound_intractable}
\end{align}
where we used the fact that 
$\text{KL}[q(\f|\bu) || p(\f|\bu)]$ is obtained from  \Cref{lem:KLqfupfu}. The above bound
is not computationally efficient since 
the marginal $q(f_i) = \int q(\f  | \bu) q(\bu ) d \f_{-i} d \bu$ 
has $\mathcal{O}(N^3)$ cost. This  is because
the marginalization $q(f_i | \bu) = \int q(\f  | \bu) d \f_{-i}$ cannot be trivially expressed, due to the complex structure of the covariance
$(\bK_{\f \f} - \bQ_{\f \f})^{\frac{1}{2}} \bV
(\bK_{\f \f} - \bQ_{\f \f})^{\frac{1}{2}}$ in $q(\f | \bu)$. To overcome this,  we will use a simplified version of $q(\f | \bu)$, in which we choose a spherical $\bV = v \bI$ with $v > 0$. Then, things become tractable. 

\begin{proposition} Let \emph{$q(\f|\bu) = \mathcal{N}(\f | \bK_{\f \bu} \bK_{\bu \bu}^{-1} \bu, v (\bK_{\f \f} - \bQ_{\f \f}))$} for \emph{$v>0$}. Then (\ref{eq:nonGaussian_bound_intractable}) is computed in \emph{$\mathcal{O}(N M^2)$} time as 
\emph{\begin{align}
& \sum_{i=1}^N 
\E_{q(f_i)} [\log p(y_i | f_i)] -  
 \frac{N}{2} (v - \log v - 1)
\nonumber \\ & - \text{KL}[q(\bu) || p(\bu)],
\label{eq:nonGaussian_bound_tractable}
\end{align}}

\noindent where the marginal is \emph{$q(f_i)  = \mathcal{N}(f_i | \bk_{f_i \bu} \bK_{\bu \bu}^{-1} \m, v (k_{ii} - q_{ii}) + \bk_{f_i \bu} \bK_{\bu \bu}^{-1} \bS \bK_{\bu \bu}^{-1} \bk_{\bu f_i})$}. 
\end{proposition}
%\begin{remark}
The parameter $v$ multiplies the term 
$k_{ii} - q_{ii}$  inside the variance of 
$q(f_i)$, and it also appears in the regularization term
$-\frac{N}{2} (v - \log v - 1)$. If $v=1$ the bound 
in (\ref{eq:nonGaussian_bound_tractable}) reduces to 
(\ref{eq:standard_nonconjugate_bound}), while by
optimizing over $v$ it can become a tighter bound. 
The optimization 
of $v$ is done jointly  
with the remaining parameters $\m,\bS, \bZ, \theta$ using gradient-based methods. Stochastic gradients can also be used 
by subsampling minibathes 
to deal with the sum  
$\sum_{i=1}^N 
\E_{q(f_i)} [\log p(y_i | f_i)]$.
%and reduce the complexity to $\mathcal{O}(M^3)$. 
%\end{remark}

\section{Related Work
\label{sec:relatedwork}
}


Several recent works on sparse GPs focus on constructing  efficient inducing points, such as works that place inducing points on a grid \citep{wilson2015kernel,evans2018scalable,gardner2018product}, construct inter-domain Fourier features \cite{ hensmanetal2018}, provide Bayesian treatments to inducing inputs \cite{rossi21a}
or use nearest neighbor 
sparsity structures
\cite{tran21a, wu22h}.
There exist also algorithms that  allow to increase the number of inducing points using the decoupled method \citep{cheng2017variational, havasi2018deep}
and the related orthogonally decoupled approaches 
\citep{salimbeni2018orthogonally, shietal2020, sun2021, tiao2023}. 
Our contribution is orthogonal to these previous methods since 
we relax the conditional GP
prior
%, $p(\f | \bu)$, 
assumption in the posterior variational approximation. This means that our method could be used to improve previous variational sparse GP approaches,
as the  ones mentioned above as well as earlier schemes that select inducing points from the training inputs \cite{Cao2013,Chai12,Schreiter2016}. 

\citet{XinranZhu2023} proposed 
inducing points GP approximations that change the conditional GP $p(\f | \bu)$ in the variational approximation to a modified conditional GP that uses different kernel hyperparameters in its mean vector. 
Note that our method differs since
our $q(\f | \bu)$ directly tries to 
construct a better approximation to the exact posterior
$p(\f | \bu, \y)$, using the extra $\bV$ variational parameters, 
without changing the kernel hyperparameters; see \Cref{sec:proposedmethod}. More 
importantly, our method has $\mathcal{O}(N M^2)$ cost, while the ELBO in  \citet{XinranZhu2023} (see Section 3.1 and Appendix A.1 in their paper) has cubic cost $\mathcal{O}(N^3)$ since it depends
on the inverse of $\bK_{\f \f} - \bQ_{\f \f}$ (denoted as $\tilde{\bK}_{nn}$ in their paper).  %\citet{XinranZhu2023}). 

\citet{artemevburt2021cglb}
%, by applying linear algebra operations, 
derived an upper bound on the log determinant $\log |\bK_{\f \f} + \sigma^2 \bI|$ in the exact GP log marginal likelihood and obtained the following tighter upper bound to the initial trace 
regularization term $-\frac{1}{2 \sigma^2} \text{tr}\left(  \bK_{\f\f} - \bQ_{\f\f} \right)$: 
\begin{equation}
- \frac{N}{ 2} \log\left( 1 + 
\frac{\text{tr}(\bK_{\f\f} - \bQ_{\f\f})}{N \sigma^2} \right).
\label{eq:artemevbound}
\end{equation}
Our bound is tighter since 
from Jensen's inequality 
it holds $ - \frac{N}{ 2} \log\left( 1 + 
\frac{\text{tr}(\bK_{\f\f} - \bQ_{\f\f})}{N \sigma^2} \right)
\leq - \frac{1}{2} \sum_{i=1}^N \log\left( 1 + 
\frac{k_{ii} - q_{ii}}{\sigma^2} \right)$. Further, the above regularization term
can be interpreted as a restricted special case of our method, obtained through a $q(\f | \bu)$ from \Cref{eq:qfu} where the diagonal matrix $\bV$ is constrained to be spherical $\bV = v  \bI$; see \Cref{app:artemevbound}. Finally note, 
that unlike (\ref{eq:artemevbound}) 
(where the sum is inside the logarithm) our bound allows to apply stochastic optimization
as described in \Cref{sec:stochasticopt}.

Finally, 
\citet{Buietal2017}
used  power expectation 
propagation that minimizes  $\alpha$-divergence and derived an approximation 
to the log marginal likelihood that interpolates between the FITC ($\alpha=1$) log marginal 
likelihood \cite{Snelson2006,candela-rasmussen-05} and the standard collapsed 
variational bound in (\ref{eq:collapsedbound_old})
($\alpha \rightarrow 0$).
This approximation uses the regularization 
term 
\begin{equation}
-\frac{1-\alpha}{2 \alpha}
\sum_{i=1}^N \log\left( 1 + 
\alpha \frac{k_{ii} - q_{ii}}{\sigma^2} \right). 
\label{eq:Buiregularization}
\end{equation}
This is different from 
ours since
there is no value of $\alpha$ 
such that the two regularization terms will become equal. 
For example, note that for $\alpha \rightarrow 0$, \Cref{eq:Buiregularization} 
reduces to $-\frac{1}{2 \sigma^2} \text{tr}\left(  \bK_{\f\f} - \bQ_{\f\f} \right)$ as discussed 
%in Section 3.6 
in \citet{Buietal2017}. 

\section{Experiments
\label{sec:experiments}
}

\begin{figure*}[t]
\centering
\begin{tabular}{ccc}
\includegraphics[scale=0.29]
{toy_prediction_exact.pdf} &
\includegraphics[scale=0.29]
{toy_prediction_trace.pdf} &
\includegraphics[scale=0.29]
{toy_prediction_log.pdf} \\
(a) & (b) & (c) \\                
\includegraphics[scale=0.29]
{toy_all_predictions.pdf} &
\includegraphics[scale=0.25]
{toy_all_losses.pdf} &
\includegraphics[scale=0.25]
{toy_all_variances.pdf} \\
(d) & (e) & (f)              
\end{tabular}
\caption{First row shows posterior predictions (means with 2-standard deviations) after
  fitting the exact GP (a), and the sparse GPs with either the standard collapsed SGPR bound (b) or the proposed SGPR-new collapsed bound (c). In panels (b),(c) the seven inducing points are intiliazed to the same random locations (shown on top with crosses) while the optimized values are shown at the bottom.
  Panel (d) superimposes all predictions in order to provide a more comparative visualization.
  Finally, panel (e) shows the ELBO (or exact log marginal likelihood for the exact GP) values across optimization steps while (f) shows the corresponding values for the noise variance $\sigma^2$.}
\label{fig:toy}
\end{figure*}


\subsection{Illustration in 1-D Regression}

In the first regression experiment we consider the 1-D  Snelson dataset \cite{Snelson2006}. We took a subset of 40 examples of this dataset and we fitted the exact GP with the squared exponential kernel $k(x, x') = \sigma_f^2 \exp( - \frac{ (x - x')^2}{2 \ell^2})$. We also fitted sparse variational GPs %, denoted as SGPR, 
with either the standard collapsed bound \cite{titsias2009variational} from \Cref{eq:collapsedbound_old} (SGPR) or the new collapsed bound from \Cref{eq:newcollapsedbound} (SGPR-new).
Both sparse GP methods use seven inducing points initialized at the
same values as shown in Figure \ref{fig:toy}. All methods are initialized to the same hyperparameter values; see \Cref{app:furtherresults}.

Figure \ref{fig:toy} shows the results. %Specifically,
Note that both SGPR and SGPR-new find similar inducing point locations. But SGPR-new,  as a tighter bound (see panel (e)), is able to reduce some bias when estimating
the hyperparameters since it finds a noise variance $\sigma^2$ closer to the one by exact GP (see panel (f)).  
This results in better predictions that match better the exact GP, as shown by the comparative visualization in panel (d). From panel (d), observe that both the mean and variances of SGPR-new are closer to the exact GP than SGPR.  


\subsection{Medium Size Regression Datasets
\label{sec:mediumregress}
}

To further investigate the findings from the previous section, we consider three medium size real-world UCI regression datasets (Pol, Bike, and Elevators)
with roughly 10k training data points each, and for which we can still run the exact GP. We choose the ARD squared exponential kernel $k(\bx, \bx') = \sigma_f^2 \exp( - \sum_{i=1}^d \frac{(x_i - x_i')^2}{2 \ell_i^2})$.
We run all three previous methods (Exact GP, SGPR, SGPR-new) five times with different random train-test splits;
see \Cref{app:furtherresults} for experimental details. We also include
in the comparison a fourth method (discussed in Related Work)
which is the \citet{artemevburt2021cglb}'s bound  (SGPR-artemev) that does training using the collapsed bound from \Cref{eq:artemvecollapsedbound} in \Cref{app:artemevbound}. 
All sparse GP methods use $M=1024$ or $M=2048$ inducing points initialized by k-means.  Figure \ref{fig:mediumsize1024} shows the objective function and the noise variance $\sigma^2$ across $10k$ optimization steps using Adam with base learning rate $0.01$ and for $M=1024$.  \Cref{fig:mediumsize2048} in \Cref{app:mediumsizeRegress} shows the corresponding plots for $M=2048$.  We observe that for Pol and Bike, SGPR-new matches closer the exact GP training than SGPR and SGPR-artemev. Specifically, SGPR-new gives higher ELBO and estimates the noise variance with reduced underfitting bias.
For the Elevators dataset, $M=1024$ inducing points were enough for sparse GP methods to closely match exact GP training. This happens because in this case $\bQ_{\f \f}$ accurately approximates $\bK_{\f \f}$, i.e., the elements $k_{ii} - q_{ii}$ get close to zero. Table \ref{table:smalldatasetsTestLL} reports test log-likelihood predictions which show that 
SGPR-new outperforms SGPR and SGPR-artemev.  

\begin{table}[t]
  \caption{Average test log-likelihoods for the medium size regression datasets.
  The numbers in parentheses are standard errors.
    %The SGPR methods used $M=1024$ inducing points.
  }
\label{table:smalldatasetsTestLL}
\vskip 0.15in
%\begin{small}
\begin{center}
%  \begin{sc}
\resizebox{\linewidth}{!}{%
\begin{tabular}{lcccr}
\toprule
& Pol  & Bike & Elevators \\
\midrule
Exact GP & $1.089(0.011)$ & $3.105(0.022)$ & $-0.386(0.001)$ \\
% Exact GP & $1.089(0.011)$ & $3.105(0.022)$ & $-0.386(0.001)$  \\
\midrule
 $M=1024$ & & & \\
SGPR & $0.821(0.008)$ & $2.176(0.020)$ & $-0.387(0.001)$\\
% SGPR-trace & $0.958(0.008)$  & $2.337(0.030)$ & $-0.387(0.001)$ \\
SGPR-artemev & $0.859(0.007)$ & $2.199(0.024)$ & $-0.387(0.001)$  \\
SGPR-new & $0.920(0.006)$ & $2.326(0.026)$  & $-0.387(0.001)$  \\
%SGPR-log & $0.998(0.008)$  & $2.511(0.021)$ & $-0.387(0.001)$ \\
\midrule
$M=2048$ & & & \\
% SGPR-trace & $0.821(0.008)$ & $2.176(0.020)$ & $-0.387(0.001)$\\
SGPR & $0.958(0.008)$  & $2.337(0.030)$ & $-0.387(0.001)$ \\
% SGPR-log & $0.920(0.006)$ & $2.326(0.026)$  & $-0.387(0.001)$  \\
SGPR-artemev & $0.976(0.008)$ & $2.356(0.029)$ & $-0.387(0.001)$  \\
SGPR-new & $0.998(0.008)$  & $2.511(0.021)$ & $-0.387(0.001)$ \\
\bottomrule
\end{tabular}}
%\end{sc}
%\end{small}
\end{center}
\vskip -0.1in
\end{table}


\begin{figure*}[t]
\centering
\begin{tabular}{ccc}
\includegraphics[scale=0.25]
{smallscale_elbo_pol_1024.pdf} &
\includegraphics[scale=0.25]
{smallscale_elbo_bike_1024.pdf} &
\includegraphics[scale=0.25]
{smallscale_elbo_elevators_1024.pdf} \\
\includegraphics[scale=0.25]
{smallscale_sigma2_pol_1024.pdf} &
\includegraphics[scale=0.25]
{smallscale_sigma2_bike_1024.pdf} &
\includegraphics[scale=0.25]
{smallscale_sigma2_elevators_1024.pdf} 
\end{tabular}
\caption{The two plots in each column correspond to the same dataset: first row shows the ELBO (or log-likelihood)
 for all four methods (Exact GP, SGPR, SGPR-new and SGPR-artemev) with the number of iterations, and the plot in the second row shows the
  corresponding values for $\sigma^2$. SGPR methods use $M=1024$ inducing points initialized by k-means. For each line we plot the mean and standard error
  after repeating the experiment five times with different train-test dataset splits; see \Cref{app:furtherresults} for further experimental details.       
}
\label{fig:mediumsize1024}
\end{figure*}


\subsection{Large Scale Regression Datasets
\label{sec:largeregress}
}

\begin{table*}[t]
\caption{Test log-likelihoods for the large scale regression datasets with standard errors in parentheses. Best mean values are highlighted.} 
% Uses random 80\% / 20\% training and test splits, repeated 5 times. }
\label{table:largescaleTestLL}
\makebox[\textwidth][c]{
\resizebox{1.02\textwidth}{!}{
\setlength\tabcolsep{2pt}
\begin{tabular}{ l l cc cc cc cc}
\toprule
& & Kin40k &  Protein & \footnotesize KeggDirected & KEGGU &  3dRoad & Song &  Buzz & \footnotesize HouseElectric \\
\cmidrule(lr){3-10}
& $N$ & 25,600 & 29,267 & 31,248 & 40,708 & 278,319 & 329,820 & 373,280 & 1,311,539  \\
& $d$ & 8 & 9 & 20 & 27 & 3 & 90 & 77 & 9  \\
\midrule
%\multirow{2}{*}{SVGP}
%& $1024$  
%& 0.094(0.003) & -0.963(0.006) & 0.967(0.005) & 0.678(0.004) & -0.698(0.002) & -1.193(0.001) & -0.079(0.002) & 1.304(0.002)  \\
%& $1536$  
%& 0.129(0.003) & -0.949(0.005) & 0.944(0.006) & 0.673(0.004) & -0.674(0.003) & -1.193(0.001) & -0.079(0.002) & 1.304(0.003) \\
%\midrule
From \citet{shietal2020} \\ 
ODVGP & $1024+1024$ 
& 0.137(0.003) & -0.956(0.005) & -0.199(0.067) & 0.105(0.033) & -0.664(0.003) & -1.193(0.001) & -0.078(0.001) & 1.317(0.002) \\
& $1024+8096$  
& 0.144(0.002) & -0.946(0.005) & -0.136(0.063) & 0.109(0.033) & -0.657(0.003) & -1.193(0.001) & -0.079(0.001) & 1.319(0.004) \\
SOLVE-GP & $1024 + 1024$ & 0.187(0.002) & -0.943(0.005) &  0.973(0.003) &  0.680(0.003) & -0.659(0.002) & -1.192(0.001) &  -0.071(0.001) & 1.333(0.003) \\
%\midrule
%SVGP
% \\
%& $2048$
%& 0.137(0.003) & {\bf -0.940}(0.005) & 0.907(0.003) & 0.665(0.004) & -0.669(0.002) & {\bf -1.192}(0.001) & -0.079(0.002) & 1.304(0.003) \\
\midrule
SVGP [ours] & 1024 & $0.108(0.002)$ & $-0.969(0.006)$ & $1.042(0.009)$ & $0.699(0.005)$ & $-0.704(0.003)$ & $-1.192(0.001)$ & $-0.069(0.002)$ & $1.383(0.002)$ \\
& 2048 & $0.237(0.002)$ & $-0.944(0.006)$ & ${\bf 1.050}(0.009)$ & ${\bf 0.703}(0.005)$ & ${\bf -0.650}(0.003)$ & ${\bf -1.190}(0.001)$ & $-0.063(0.001)$ & $1.419(0.002)$ \\
SVGP-new [ours]  & 1024 & $0.152(0.003)$ & $-0.965(0.006)$ & $1.044(0.009)$ & $0.699(0.005)$ & $-0.701(0.003)$ & $-1.192(0.001)$ & $-0.065(0.002)$ & $1.387(0.003)$ \\
 & 2048 & ${\bf 0.286}(0.002)$ & ${\bf -0.938}(0.006)$ & $1.051(0.009)$ & ${\bf 0.703}(0.005)$ & $-0.651(0.004)$ & ${\bf -1.190}(0.001)$ & ${\bf -0.060}(0.001)$ & ${\bf 1.421}(0.002)$ \\
\bottomrule 
\end{tabular}
}
}
\end{table*}


We consider 8 UCI regression datasets, with training data sizes ranging from tens of thousands to millions. 
%Results of exact GP regression have been reported on these datasets with distributed training~\citep{wang2019exact}. 
We implemented the stochastic optimization versions of the two scalable sparse GP methods: (i) the one that trains using the previous uncollapsed bound from
 \citet{hensman2013gaussian} (SVGP) and (ii) our new bound from    
\Cref{eq:newuncollapsedbound} (SVGP-new). We denote these stochastic optimization versions by SVGP to distinguish them from the corresponding
SGPR methods that use the more expensive collapsed bounds. We run the SVGP methods with $M=1024$ and $2048$ inducing points, Matern3/2 kernel with common lengthscale, minibatch size $1024$, Adam with base learning rate $0.01$ and $100$ epochs. These experimental settings match the ones in \citet{wang2019exact} and \citet{shietal2020} as further described  in \Cref{app:largescaleRegress}. Table \ref{table:largescaleTestLL} reports the test log likelihood scores
for all datasets. In the comparison we also included two strong baselines from Table 2 in \citet{shietal2020}, i.e., SOLVE-GP and ODVGP \cite{salimbeni2018orthogonally}.


\begin{figure*}
\centering
\begin{tabular}{ccc}
\includegraphics[scale=0.24]
{poisson_toy_all_predictions.pdf} &
\includegraphics[scale=0.24]
{poisson_toy_all_losses.pdf} &
\includegraphics[scale=0.24]
{poisson_elbo_nybicycle_16.pdf} \\
% (a) & (b) & (c)
\end{tabular}
\caption{({\bf left}) shows the % posterior 
predictions (means with 2-standard deviations) over counts (black dots) in the artificial data example  after
  fitting the Full GP, and the two SVGPs. This plot superimposes all predictions in order to provide a comparative visualization.
  %; see \Cref{app:poisson} for individual plots. 
  ({\bf middle})  shows the ELBO  across optimization steps for the artificial data example. ({\bf right}) shows the ELBO for the NYBikes dataset and $M=16$.}
\label{fig:poisson}
\end{figure*}


From the predictive log likelihood scores in Table \ref{table:largescaleTestLL} and also the corresponding Root Mean Squared Error (RMSE)
scores reported in  \Cref{table:largescaleRMSE} in \Cref{app:largescaleRegress}, we can conclude that training with the new SVGP-new variational bound
provides a clear improvement compared to training with the previous SVGP bound. Note that this improvement requires no change in the computational
cost, and in fact there is only a minor modification needed to be done in an existing SVGP implementation in order to run SVGP-new.  

\vspace{-1mm}

\subsection{Poisson Regression
  \label{sec:poisson}
}

\vspace{-1mm}

We consider a non-Gaussian likelihood example where the output data are counts modeled  by a Poisson likelihood 
$p(\y | \f) = \prod_{i=1}^N \frac{e^{f_i}}{y_i !} e^{-e^{f_i}}$  where the log intensities values follow a GP prior. For such 
case the new variational approximation includes a single additional variational parameter denoted by $v$, which is optimized together 
with the remaining parameters; see \cref{sec:nongaussian}. We will compare training with the new ELBO 
 from \Cref{eq:nonGaussian_bound_tractable}  (we denote this method by SVGP-new) with the standard ELBO that is obtained by restricting  $v=1$  (SVGP). 
 
Firstly, we consider an artificial example of $50$ observations with 1-D inputs placed in the grid $[-10, 10]$ where counts are
generated using Poisson intensities given by $\lambda(x) = 3.5 + 3  \sin(x)$. We train the GP model with the SVGP bound and the proposed SVGP-new bound using $6$ inducing points initialized to the same values for both methods; see \Cref{app:poisson}. 
\Cref{fig:poisson}(left) shows the 
% observed counts together with the 
predictions obtained by SVGP, SVGP-new 
and non-sparse %or full 
variational % inference 
GP (Full GP). From
this figure and from the
ELBO values, 
we observe that SVGP-new
remains closer to Full GP.  

Secondly, we consider a real dataset (NYBikes) about bicycles crossings going over bridges in New York City\footnote{This dataset is freely available from
\url{https://www.kaggle.com/datasets/new-york-city/nyc-east-river-bicycle-crossings}.}.
This dataset is a daily record of the number of bicycles crossing into or out of Manhattan via one of the East River bridges over a period 9 months. The data contains $210$  points and we randomly choose $90\%$ for training and $10\%$ for test.   
We apply GP Poisson regression for the Brooklyn bridge counts where the input vector $\bx$ is taken to be two-dimensional consisted of 
 maximum and minimum daily temperatures.  We train the sparse GPs with either SVGP or SVGP-new and with $M=8,16,32$ 
 inducing points initialized by k-means.  Since the dataset is small  we also run the non-sparse  Full GP. The ELBO across iterations in \Cref{fig:poisson} (right) and the test log likelihood scores (\Cref{table:poisson_nybikes} 
 in \Cref{app:poisson})
 indicate that  SVGP-new provides a better approximation than SVGP.  
 
  
   
   









\vspace{-2mm}
\section{Conclusion}\label{sec:conclusion}
In this paper, we presented RecDreamer, a novel approach to mitigating the Multi-Face Janus problem in text-to-3D generation. Our solution introduces a rectification function to modify the prior distribution, ensuring that the resulting joint distribution achieves uniformity across poses. By expressing the modified data distribution as the product of the original density and the rectification function, we seamlessly integrate this adjustment into the score distillation algorithm. This allows us to derive a particle optimization framework for uniform score distillation. Additionally, we developed a pose classifier and implemented reliable approximations and simulations to enhance the particle optimization process. Extensive experiments on both 2D and 3D synthesis tasks demonstrate the effectiveness of our approach in addressing the Multi-Face Janus problem, resulting in more consistent geometries and textures across different views.

\textbf{Limitations.} While our method significantly reduces bias in prior distributions, further exploration of 3D modeling with multi-view priors could improve geometric and texture consistency. Extending our approach through deeper research into conditional control presents another promising avenue for addressing these challenges in future work. 

%\section*{Accessibility}
%Authors are kindly asked to make %their submissions as accessible as %possible for everyone including %people with disabilities and sensory %or neurological differences.
%Tips of how to achieve this and what to pay attention to will be provided on the conference website \url{http://icml.cc/}.

%\section*{Software and Data}
%If a paper is accepted, we strongly encourage the publication of software and data with the
%camera-ready version of the paper whenever appropriate. This can be
%done by including a URL in the camera-ready copy. However, \textbf{do not}
%include URLs that reveal your institution or identity in your
%submission for review. Instead, provide an anonymous URL or upload
%the material as ``Supplementary Material'' into the CMT reviewing
%system. Note that reviewers are not required to look at this material
%when writing their review.

%% Acknowledgements should only appear in the accepted version.
%\section*{Acknowledgements}
%
%\textbf{Do not} include acknowledgements in the initial version of
%the paper submitted for blind review.

%If a paper is accepted, the final camera-ready version can (and
%probably should) include acknowledgements. In this case, please
%place such acknowledgements in an unnumbered section at the
%end of the paper. Typically, this will include thanks to reviewers
%who gave useful comments, to colleagues who contributed to the ideas,
%and to funding agencies and corporate sponsors that provided financial support.



% In the unusual situation where you want a paper to appear in the
% references without citing it in the main text, use \nocite \nocite{langley00}

\bibliography{library}
\bibliographystyle{icml2023}


%%%%%%%%%%%%%%%%%%%%%%%%%%%%%%%%%%%%%%%%%%%%%%%%%%%%%%%%%%%%%%%%%%%%%%%%%%%%%%%
%%%%%%%%%%%%%%%%%%%%%%%%%%%%%%%%%%%%%%%%%%%%%%%%%%%%%%%%%%%%%%%%%%%%%%%%%%%%%%%
% APPENDIX
%%%%%%%%%%%%%%%%%%%%%%%%%%%%%%%%%%%%%%%%%%%%%%%%%%%%%%%%%%%%%%%%%%%%%%%%%%%%%%%
%%%%%%%%%%%%%%%%%%%%%%%%%%%%%%%%%%%%%%%%%%%%%%%%%%%%%%%%%%%%%%%%%%%%%%%%%%%%%%%
\newpage
\appendix
\onecolumn
\subsection{Lloyd-Max Algorithm}
\label{subsec:Lloyd-Max}
For a given quantization bitwidth $B$ and an operand $\bm{X}$, the Lloyd-Max algorithm finds $2^B$ quantization levels $\{\hat{x}_i\}_{i=1}^{2^B}$ such that quantizing $\bm{X}$ by rounding each scalar in $\bm{X}$ to the nearest quantization level minimizes the quantization MSE. 

The algorithm starts with an initial guess of quantization levels and then iteratively computes quantization thresholds $\{\tau_i\}_{i=1}^{2^B-1}$ and updates quantization levels $\{\hat{x}_i\}_{i=1}^{2^B}$. Specifically, at iteration $n$, thresholds are set to the midpoints of the previous iteration's levels:
\begin{align*}
    \tau_i^{(n)}=\frac{\hat{x}_i^{(n-1)}+\hat{x}_{i+1}^{(n-1)}}2 \text{ for } i=1\ldots 2^B-1
\end{align*}
Subsequently, the quantization levels are re-computed as conditional means of the data regions defined by the new thresholds:
\begin{align*}
    \hat{x}_i^{(n)}=\mathbb{E}\left[ \bm{X} \big| \bm{X}\in [\tau_{i-1}^{(n)},\tau_i^{(n)}] \right] \text{ for } i=1\ldots 2^B
\end{align*}
where to satisfy boundary conditions we have $\tau_0=-\infty$ and $\tau_{2^B}=\infty$. The algorithm iterates the above steps until convergence.

Figure \ref{fig:lm_quant} compares the quantization levels of a $7$-bit floating point (E3M3) quantizer (left) to a $7$-bit Lloyd-Max quantizer (right) when quantizing a layer of weights from the GPT3-126M model at a per-tensor granularity. As shown, the Lloyd-Max quantizer achieves substantially lower quantization MSE. Further, Table \ref{tab:FP7_vs_LM7} shows the superior perplexity achieved by Lloyd-Max quantizers for bitwidths of $7$, $6$ and $5$. The difference between the quantizers is clear at 5 bits, where per-tensor FP quantization incurs a drastic and unacceptable increase in perplexity, while Lloyd-Max quantization incurs a much smaller increase. Nevertheless, we note that even the optimal Lloyd-Max quantizer incurs a notable ($\sim 1.5$) increase in perplexity due to the coarse granularity of quantization. 

\begin{figure}[h]
  \centering
  \includegraphics[width=0.7\linewidth]{sections/figures/LM7_FP7.pdf}
  \caption{\small Quantization levels and the corresponding quantization MSE of Floating Point (left) vs Lloyd-Max (right) Quantizers for a layer of weights in the GPT3-126M model.}
  \label{fig:lm_quant}
\end{figure}

\begin{table}[h]\scriptsize
\begin{center}
\caption{\label{tab:FP7_vs_LM7} \small Comparing perplexity (lower is better) achieved by floating point quantizers and Lloyd-Max quantizers on a GPT3-126M model for the Wikitext-103 dataset.}
\begin{tabular}{c|cc|c}
\hline
 \multirow{2}{*}{\textbf{Bitwidth}} & \multicolumn{2}{|c|}{\textbf{Floating-Point Quantizer}} & \textbf{Lloyd-Max Quantizer} \\
 & Best Format & Wikitext-103 Perplexity & Wikitext-103 Perplexity \\
\hline
7 & E3M3 & 18.32 & 18.27 \\
6 & E3M2 & 19.07 & 18.51 \\
5 & E4M0 & 43.89 & 19.71 \\
\hline
\end{tabular}
\end{center}
\end{table}

\subsection{Proof of Local Optimality of LO-BCQ}
\label{subsec:lobcq_opt_proof}
For a given block $\bm{b}_j$, the quantization MSE during LO-BCQ can be empirically evaluated as $\frac{1}{L_b}\lVert \bm{b}_j- \bm{\hat{b}}_j\rVert^2_2$ where $\bm{\hat{b}}_j$ is computed from equation (\ref{eq:clustered_quantization_definition}) as $C_{f(\bm{b}_j)}(\bm{b}_j)$. Further, for a given block cluster $\mathcal{B}_i$, we compute the quantization MSE as $\frac{1}{|\mathcal{B}_{i}|}\sum_{\bm{b} \in \mathcal{B}_{i}} \frac{1}{L_b}\lVert \bm{b}- C_i^{(n)}(\bm{b})\rVert^2_2$. Therefore, at the end of iteration $n$, we evaluate the overall quantization MSE $J^{(n)}$ for a given operand $\bm{X}$ composed of $N_c$ block clusters as:
\begin{align*}
    \label{eq:mse_iter_n}
    J^{(n)} = \frac{1}{N_c} \sum_{i=1}^{N_c} \frac{1}{|\mathcal{B}_{i}^{(n)}|}\sum_{\bm{v} \in \mathcal{B}_{i}^{(n)}} \frac{1}{L_b}\lVert \bm{b}- B_i^{(n)}(\bm{b})\rVert^2_2
\end{align*}

At the end of iteration $n$, the codebooks are updated from $\mathcal{C}^{(n-1)}$ to $\mathcal{C}^{(n)}$. However, the mapping of a given vector $\bm{b}_j$ to quantizers $\mathcal{C}^{(n)}$ remains as  $f^{(n)}(\bm{b}_j)$. At the next iteration, during the vector clustering step, $f^{(n+1)}(\bm{b}_j)$ finds new mapping of $\bm{b}_j$ to updated codebooks $\mathcal{C}^{(n)}$ such that the quantization MSE over the candidate codebooks is minimized. Therefore, we obtain the following result for $\bm{b}_j$:
\begin{align*}
\frac{1}{L_b}\lVert \bm{b}_j - C_{f^{(n+1)}(\bm{b}_j)}^{(n)}(\bm{b}_j)\rVert^2_2 \le \frac{1}{L_b}\lVert \bm{b}_j - C_{f^{(n)}(\bm{b}_j)}^{(n)}(\bm{b}_j)\rVert^2_2
\end{align*}

That is, quantizing $\bm{b}_j$ at the end of the block clustering step of iteration $n+1$ results in lower quantization MSE compared to quantizing at the end of iteration $n$. Since this is true for all $\bm{b} \in \bm{X}$, we assert the following:
\begin{equation}
\begin{split}
\label{eq:mse_ineq_1}
    \tilde{J}^{(n+1)} &= \frac{1}{N_c} \sum_{i=1}^{N_c} \frac{1}{|\mathcal{B}_{i}^{(n+1)}|}\sum_{\bm{b} \in \mathcal{B}_{i}^{(n+1)}} \frac{1}{L_b}\lVert \bm{b} - C_i^{(n)}(b)\rVert^2_2 \le J^{(n)}
\end{split}
\end{equation}
where $\tilde{J}^{(n+1)}$ is the the quantization MSE after the vector clustering step at iteration $n+1$.

Next, during the codebook update step (\ref{eq:quantizers_update}) at iteration $n+1$, the per-cluster codebooks $\mathcal{C}^{(n)}$ are updated to $\mathcal{C}^{(n+1)}$ by invoking the Lloyd-Max algorithm \citep{Lloyd}. We know that for any given value distribution, the Lloyd-Max algorithm minimizes the quantization MSE. Therefore, for a given vector cluster $\mathcal{B}_i$ we obtain the following result:

\begin{equation}
    \frac{1}{|\mathcal{B}_{i}^{(n+1)}|}\sum_{\bm{b} \in \mathcal{B}_{i}^{(n+1)}} \frac{1}{L_b}\lVert \bm{b}- C_i^{(n+1)}(\bm{b})\rVert^2_2 \le \frac{1}{|\mathcal{B}_{i}^{(n+1)}|}\sum_{\bm{b} \in \mathcal{B}_{i}^{(n+1)}} \frac{1}{L_b}\lVert \bm{b}- C_i^{(n)}(\bm{b})\rVert^2_2
\end{equation}

The above equation states that quantizing the given block cluster $\mathcal{B}_i$ after updating the associated codebook from $C_i^{(n)}$ to $C_i^{(n+1)}$ results in lower quantization MSE. Since this is true for all the block clusters, we derive the following result: 
\begin{equation}
\begin{split}
\label{eq:mse_ineq_2}
     J^{(n+1)} &= \frac{1}{N_c} \sum_{i=1}^{N_c} \frac{1}{|\mathcal{B}_{i}^{(n+1)}|}\sum_{\bm{b} \in \mathcal{B}_{i}^{(n+1)}} \frac{1}{L_b}\lVert \bm{b}- C_i^{(n+1)}(\bm{b})\rVert^2_2  \le \tilde{J}^{(n+1)}   
\end{split}
\end{equation}

Following (\ref{eq:mse_ineq_1}) and (\ref{eq:mse_ineq_2}), we find that the quantization MSE is non-increasing for each iteration, that is, $J^{(1)} \ge J^{(2)} \ge J^{(3)} \ge \ldots \ge J^{(M)}$ where $M$ is the maximum number of iterations. 
%Therefore, we can say that if the algorithm converges, then it must be that it has converged to a local minimum. 
\hfill $\blacksquare$


\begin{figure}
    \begin{center}
    \includegraphics[width=0.5\textwidth]{sections//figures/mse_vs_iter.pdf}
    \end{center}
    \caption{\small NMSE vs iterations during LO-BCQ compared to other block quantization proposals}
    \label{fig:nmse_vs_iter}
\end{figure}

Figure \ref{fig:nmse_vs_iter} shows the empirical convergence of LO-BCQ across several block lengths and number of codebooks. Also, the MSE achieved by LO-BCQ is compared to baselines such as MXFP and VSQ. As shown, LO-BCQ converges to a lower MSE than the baselines. Further, we achieve better convergence for larger number of codebooks ($N_c$) and for a smaller block length ($L_b$), both of which increase the bitwidth of BCQ (see Eq \ref{eq:bitwidth_bcq}).


\subsection{Additional Accuracy Results}
%Table \ref{tab:lobcq_config} lists the various LOBCQ configurations and their corresponding bitwidths.
\begin{table}
\setlength{\tabcolsep}{4.75pt}
\begin{center}
\caption{\label{tab:lobcq_config} Various LO-BCQ configurations and their bitwidths.}
\begin{tabular}{|c||c|c|c|c||c|c||c|} 
\hline
 & \multicolumn{4}{|c||}{$L_b=8$} & \multicolumn{2}{|c||}{$L_b=4$} & $L_b=2$ \\
 \hline
 \backslashbox{$L_A$\kern-1em}{\kern-1em$N_c$} & 2 & 4 & 8 & 16 & 2 & 4 & 2 \\
 \hline
 64 & 4.25 & 4.375 & 4.5 & 4.625 & 4.375 & 4.625 & 4.625\\
 \hline
 32 & 4.375 & 4.5 & 4.625& 4.75 & 4.5 & 4.75 & 4.75 \\
 \hline
 16 & 4.625 & 4.75& 4.875 & 5 & 4.75 & 5 & 5 \\
 \hline
\end{tabular}
\end{center}
\end{table}

%\subsection{Perplexity achieved by various LO-BCQ configurations on Wikitext-103 dataset}

\begin{table} \centering
\begin{tabular}{|c||c|c|c|c||c|c||c|} 
\hline
 $L_b \rightarrow$& \multicolumn{4}{c||}{8} & \multicolumn{2}{c||}{4} & 2\\
 \hline
 \backslashbox{$L_A$\kern-1em}{\kern-1em$N_c$} & 2 & 4 & 8 & 16 & 2 & 4 & 2  \\
 %$N_c \rightarrow$ & 2 & 4 & 8 & 16 & 2 & 4 & 2 \\
 \hline
 \hline
 \multicolumn{8}{c}{GPT3-1.3B (FP32 PPL = 9.98)} \\ 
 \hline
 \hline
 64 & 10.40 & 10.23 & 10.17 & 10.15 &  10.28 & 10.18 & 10.19 \\
 \hline
 32 & 10.25 & 10.20 & 10.15 & 10.12 &  10.23 & 10.17 & 10.17 \\
 \hline
 16 & 10.22 & 10.16 & 10.10 & 10.09 &  10.21 & 10.14 & 10.16 \\
 \hline
  \hline
 \multicolumn{8}{c}{GPT3-8B (FP32 PPL = 7.38)} \\ 
 \hline
 \hline
 64 & 7.61 & 7.52 & 7.48 &  7.47 &  7.55 &  7.49 & 7.50 \\
 \hline
 32 & 7.52 & 7.50 & 7.46 &  7.45 &  7.52 &  7.48 & 7.48  \\
 \hline
 16 & 7.51 & 7.48 & 7.44 &  7.44 &  7.51 &  7.49 & 7.47  \\
 \hline
\end{tabular}
\caption{\label{tab:ppl_gpt3_abalation} Wikitext-103 perplexity across GPT3-1.3B and 8B models.}
\end{table}

\begin{table} \centering
\begin{tabular}{|c||c|c|c|c||} 
\hline
 $L_b \rightarrow$& \multicolumn{4}{c||}{8}\\
 \hline
 \backslashbox{$L_A$\kern-1em}{\kern-1em$N_c$} & 2 & 4 & 8 & 16 \\
 %$N_c \rightarrow$ & 2 & 4 & 8 & 16 & 2 & 4 & 2 \\
 \hline
 \hline
 \multicolumn{5}{|c|}{Llama2-7B (FP32 PPL = 5.06)} \\ 
 \hline
 \hline
 64 & 5.31 & 5.26 & 5.19 & 5.18  \\
 \hline
 32 & 5.23 & 5.25 & 5.18 & 5.15  \\
 \hline
 16 & 5.23 & 5.19 & 5.16 & 5.14  \\
 \hline
 \multicolumn{5}{|c|}{Nemotron4-15B (FP32 PPL = 5.87)} \\ 
 \hline
 \hline
 64  & 6.3 & 6.20 & 6.13 & 6.08  \\
 \hline
 32  & 6.24 & 6.12 & 6.07 & 6.03  \\
 \hline
 16  & 6.12 & 6.14 & 6.04 & 6.02  \\
 \hline
 \multicolumn{5}{|c|}{Nemotron4-340B (FP32 PPL = 3.48)} \\ 
 \hline
 \hline
 64 & 3.67 & 3.62 & 3.60 & 3.59 \\
 \hline
 32 & 3.63 & 3.61 & 3.59 & 3.56 \\
 \hline
 16 & 3.61 & 3.58 & 3.57 & 3.55 \\
 \hline
\end{tabular}
\caption{\label{tab:ppl_llama7B_nemo15B} Wikitext-103 perplexity compared to FP32 baseline in Llama2-7B and Nemotron4-15B, 340B models}
\end{table}

%\subsection{Perplexity achieved by various LO-BCQ configurations on MMLU dataset}


\begin{table} \centering
\begin{tabular}{|c||c|c|c|c||c|c|c|c|} 
\hline
 $L_b \rightarrow$& \multicolumn{4}{c||}{8} & \multicolumn{4}{c||}{8}\\
 \hline
 \backslashbox{$L_A$\kern-1em}{\kern-1em$N_c$} & 2 & 4 & 8 & 16 & 2 & 4 & 8 & 16  \\
 %$N_c \rightarrow$ & 2 & 4 & 8 & 16 & 2 & 4 & 2 \\
 \hline
 \hline
 \multicolumn{5}{|c|}{Llama2-7B (FP32 Accuracy = 45.8\%)} & \multicolumn{4}{|c|}{Llama2-70B (FP32 Accuracy = 69.12\%)} \\ 
 \hline
 \hline
 64 & 43.9 & 43.4 & 43.9 & 44.9 & 68.07 & 68.27 & 68.17 & 68.75 \\
 \hline
 32 & 44.5 & 43.8 & 44.9 & 44.5 & 68.37 & 68.51 & 68.35 & 68.27  \\
 \hline
 16 & 43.9 & 42.7 & 44.9 & 45 & 68.12 & 68.77 & 68.31 & 68.59  \\
 \hline
 \hline
 \multicolumn{5}{|c|}{GPT3-22B (FP32 Accuracy = 38.75\%)} & \multicolumn{4}{|c|}{Nemotron4-15B (FP32 Accuracy = 64.3\%)} \\ 
 \hline
 \hline
 64 & 36.71 & 38.85 & 38.13 & 38.92 & 63.17 & 62.36 & 63.72 & 64.09 \\
 \hline
 32 & 37.95 & 38.69 & 39.45 & 38.34 & 64.05 & 62.30 & 63.8 & 64.33  \\
 \hline
 16 & 38.88 & 38.80 & 38.31 & 38.92 & 63.22 & 63.51 & 63.93 & 64.43  \\
 \hline
\end{tabular}
\caption{\label{tab:mmlu_abalation} Accuracy on MMLU dataset across GPT3-22B, Llama2-7B, 70B and Nemotron4-15B models.}
\end{table}


%\subsection{Perplexity achieved by various LO-BCQ configurations on LM evaluation harness}

\begin{table} \centering
\begin{tabular}{|c||c|c|c|c||c|c|c|c|} 
\hline
 $L_b \rightarrow$& \multicolumn{4}{c||}{8} & \multicolumn{4}{c||}{8}\\
 \hline
 \backslashbox{$L_A$\kern-1em}{\kern-1em$N_c$} & 2 & 4 & 8 & 16 & 2 & 4 & 8 & 16  \\
 %$N_c \rightarrow$ & 2 & 4 & 8 & 16 & 2 & 4 & 2 \\
 \hline
 \hline
 \multicolumn{5}{|c|}{Race (FP32 Accuracy = 37.51\%)} & \multicolumn{4}{|c|}{Boolq (FP32 Accuracy = 64.62\%)} \\ 
 \hline
 \hline
 64 & 36.94 & 37.13 & 36.27 & 37.13 & 63.73 & 62.26 & 63.49 & 63.36 \\
 \hline
 32 & 37.03 & 36.36 & 36.08 & 37.03 & 62.54 & 63.51 & 63.49 & 63.55  \\
 \hline
 16 & 37.03 & 37.03 & 36.46 & 37.03 & 61.1 & 63.79 & 63.58 & 63.33  \\
 \hline
 \hline
 \multicolumn{5}{|c|}{Winogrande (FP32 Accuracy = 58.01\%)} & \multicolumn{4}{|c|}{Piqa (FP32 Accuracy = 74.21\%)} \\ 
 \hline
 \hline
 64 & 58.17 & 57.22 & 57.85 & 58.33 & 73.01 & 73.07 & 73.07 & 72.80 \\
 \hline
 32 & 59.12 & 58.09 & 57.85 & 58.41 & 73.01 & 73.94 & 72.74 & 73.18  \\
 \hline
 16 & 57.93 & 58.88 & 57.93 & 58.56 & 73.94 & 72.80 & 73.01 & 73.94  \\
 \hline
\end{tabular}
\caption{\label{tab:mmlu_abalation} Accuracy on LM evaluation harness tasks on GPT3-1.3B model.}
\end{table}

\begin{table} \centering
\begin{tabular}{|c||c|c|c|c||c|c|c|c|} 
\hline
 $L_b \rightarrow$& \multicolumn{4}{c||}{8} & \multicolumn{4}{c||}{8}\\
 \hline
 \backslashbox{$L_A$\kern-1em}{\kern-1em$N_c$} & 2 & 4 & 8 & 16 & 2 & 4 & 8 & 16  \\
 %$N_c \rightarrow$ & 2 & 4 & 8 & 16 & 2 & 4 & 2 \\
 \hline
 \hline
 \multicolumn{5}{|c|}{Race (FP32 Accuracy = 41.34\%)} & \multicolumn{4}{|c|}{Boolq (FP32 Accuracy = 68.32\%)} \\ 
 \hline
 \hline
 64 & 40.48 & 40.10 & 39.43 & 39.90 & 69.20 & 68.41 & 69.45 & 68.56 \\
 \hline
 32 & 39.52 & 39.52 & 40.77 & 39.62 & 68.32 & 67.43 & 68.17 & 69.30  \\
 \hline
 16 & 39.81 & 39.71 & 39.90 & 40.38 & 68.10 & 66.33 & 69.51 & 69.42  \\
 \hline
 \hline
 \multicolumn{5}{|c|}{Winogrande (FP32 Accuracy = 67.88\%)} & \multicolumn{4}{|c|}{Piqa (FP32 Accuracy = 78.78\%)} \\ 
 \hline
 \hline
 64 & 66.85 & 66.61 & 67.72 & 67.88 & 77.31 & 77.42 & 77.75 & 77.64 \\
 \hline
 32 & 67.25 & 67.72 & 67.72 & 67.00 & 77.31 & 77.04 & 77.80 & 77.37  \\
 \hline
 16 & 68.11 & 68.90 & 67.88 & 67.48 & 77.37 & 78.13 & 78.13 & 77.69  \\
 \hline
\end{tabular}
\caption{\label{tab:mmlu_abalation} Accuracy on LM evaluation harness tasks on GPT3-8B model.}
\end{table}

\begin{table} \centering
\begin{tabular}{|c||c|c|c|c||c|c|c|c|} 
\hline
 $L_b \rightarrow$& \multicolumn{4}{c||}{8} & \multicolumn{4}{c||}{8}\\
 \hline
 \backslashbox{$L_A$\kern-1em}{\kern-1em$N_c$} & 2 & 4 & 8 & 16 & 2 & 4 & 8 & 16  \\
 %$N_c \rightarrow$ & 2 & 4 & 8 & 16 & 2 & 4 & 2 \\
 \hline
 \hline
 \multicolumn{5}{|c|}{Race (FP32 Accuracy = 40.67\%)} & \multicolumn{4}{|c|}{Boolq (FP32 Accuracy = 76.54\%)} \\ 
 \hline
 \hline
 64 & 40.48 & 40.10 & 39.43 & 39.90 & 75.41 & 75.11 & 77.09 & 75.66 \\
 \hline
 32 & 39.52 & 39.52 & 40.77 & 39.62 & 76.02 & 76.02 & 75.96 & 75.35  \\
 \hline
 16 & 39.81 & 39.71 & 39.90 & 40.38 & 75.05 & 73.82 & 75.72 & 76.09  \\
 \hline
 \hline
 \multicolumn{5}{|c|}{Winogrande (FP32 Accuracy = 70.64\%)} & \multicolumn{4}{|c|}{Piqa (FP32 Accuracy = 79.16\%)} \\ 
 \hline
 \hline
 64 & 69.14 & 70.17 & 70.17 & 70.56 & 78.24 & 79.00 & 78.62 & 78.73 \\
 \hline
 32 & 70.96 & 69.69 & 71.27 & 69.30 & 78.56 & 79.49 & 79.16 & 78.89  \\
 \hline
 16 & 71.03 & 69.53 & 69.69 & 70.40 & 78.13 & 79.16 & 79.00 & 79.00  \\
 \hline
\end{tabular}
\caption{\label{tab:mmlu_abalation} Accuracy on LM evaluation harness tasks on GPT3-22B model.}
\end{table}

\begin{table} \centering
\begin{tabular}{|c||c|c|c|c||c|c|c|c|} 
\hline
 $L_b \rightarrow$& \multicolumn{4}{c||}{8} & \multicolumn{4}{c||}{8}\\
 \hline
 \backslashbox{$L_A$\kern-1em}{\kern-1em$N_c$} & 2 & 4 & 8 & 16 & 2 & 4 & 8 & 16  \\
 %$N_c \rightarrow$ & 2 & 4 & 8 & 16 & 2 & 4 & 2 \\
 \hline
 \hline
 \multicolumn{5}{|c|}{Race (FP32 Accuracy = 44.4\%)} & \multicolumn{4}{|c|}{Boolq (FP32 Accuracy = 79.29\%)} \\ 
 \hline
 \hline
 64 & 42.49 & 42.51 & 42.58 & 43.45 & 77.58 & 77.37 & 77.43 & 78.1 \\
 \hline
 32 & 43.35 & 42.49 & 43.64 & 43.73 & 77.86 & 75.32 & 77.28 & 77.86  \\
 \hline
 16 & 44.21 & 44.21 & 43.64 & 42.97 & 78.65 & 77 & 76.94 & 77.98  \\
 \hline
 \hline
 \multicolumn{5}{|c|}{Winogrande (FP32 Accuracy = 69.38\%)} & \multicolumn{4}{|c|}{Piqa (FP32 Accuracy = 78.07\%)} \\ 
 \hline
 \hline
 64 & 68.9 & 68.43 & 69.77 & 68.19 & 77.09 & 76.82 & 77.09 & 77.86 \\
 \hline
 32 & 69.38 & 68.51 & 68.82 & 68.90 & 78.07 & 76.71 & 78.07 & 77.86  \\
 \hline
 16 & 69.53 & 67.09 & 69.38 & 68.90 & 77.37 & 77.8 & 77.91 & 77.69  \\
 \hline
\end{tabular}
\caption{\label{tab:mmlu_abalation} Accuracy on LM evaluation harness tasks on Llama2-7B model.}
\end{table}

\begin{table} \centering
\begin{tabular}{|c||c|c|c|c||c|c|c|c|} 
\hline
 $L_b \rightarrow$& \multicolumn{4}{c||}{8} & \multicolumn{4}{c||}{8}\\
 \hline
 \backslashbox{$L_A$\kern-1em}{\kern-1em$N_c$} & 2 & 4 & 8 & 16 & 2 & 4 & 8 & 16  \\
 %$N_c \rightarrow$ & 2 & 4 & 8 & 16 & 2 & 4 & 2 \\
 \hline
 \hline
 \multicolumn{5}{|c|}{Race (FP32 Accuracy = 48.8\%)} & \multicolumn{4}{|c|}{Boolq (FP32 Accuracy = 85.23\%)} \\ 
 \hline
 \hline
 64 & 49.00 & 49.00 & 49.28 & 48.71 & 82.82 & 84.28 & 84.03 & 84.25 \\
 \hline
 32 & 49.57 & 48.52 & 48.33 & 49.28 & 83.85 & 84.46 & 84.31 & 84.93  \\
 \hline
 16 & 49.85 & 49.09 & 49.28 & 48.99 & 85.11 & 84.46 & 84.61 & 83.94  \\
 \hline
 \hline
 \multicolumn{5}{|c|}{Winogrande (FP32 Accuracy = 79.95\%)} & \multicolumn{4}{|c|}{Piqa (FP32 Accuracy = 81.56\%)} \\ 
 \hline
 \hline
 64 & 78.77 & 78.45 & 78.37 & 79.16 & 81.45 & 80.69 & 81.45 & 81.5 \\
 \hline
 32 & 78.45 & 79.01 & 78.69 & 80.66 & 81.56 & 80.58 & 81.18 & 81.34  \\
 \hline
 16 & 79.95 & 79.56 & 79.79 & 79.72 & 81.28 & 81.66 & 81.28 & 80.96  \\
 \hline
\end{tabular}
\caption{\label{tab:mmlu_abalation} Accuracy on LM evaluation harness tasks on Llama2-70B model.}
\end{table}

%\section{MSE Studies}
%\textcolor{red}{TODO}


\subsection{Number Formats and Quantization Method}
\label{subsec:numFormats_quantMethod}
\subsubsection{Integer Format}
An $n$-bit signed integer (INT) is typically represented with a 2s-complement format \citep{yao2022zeroquant,xiao2023smoothquant,dai2021vsq}, where the most significant bit denotes the sign.

\subsubsection{Floating Point Format}
An $n$-bit signed floating point (FP) number $x$ comprises of a 1-bit sign ($x_{\mathrm{sign}}$), $B_m$-bit mantissa ($x_{\mathrm{mant}}$) and $B_e$-bit exponent ($x_{\mathrm{exp}}$) such that $B_m+B_e=n-1$. The associated constant exponent bias ($E_{\mathrm{bias}}$) is computed as $(2^{{B_e}-1}-1)$. We denote this format as $E_{B_e}M_{B_m}$.  

\subsubsection{Quantization Scheme}
\label{subsec:quant_method}
A quantization scheme dictates how a given unquantized tensor is converted to its quantized representation. We consider FP formats for the purpose of illustration. Given an unquantized tensor $\bm{X}$ and an FP format $E_{B_e}M_{B_m}$, we first, we compute the quantization scale factor $s_X$ that maps the maximum absolute value of $\bm{X}$ to the maximum quantization level of the $E_{B_e}M_{B_m}$ format as follows:
\begin{align}
\label{eq:sf}
    s_X = \frac{\mathrm{max}(|\bm{X}|)}{\mathrm{max}(E_{B_e}M_{B_m})}
\end{align}
In the above equation, $|\cdot|$ denotes the absolute value function.

Next, we scale $\bm{X}$ by $s_X$ and quantize it to $\hat{\bm{X}}$ by rounding it to the nearest quantization level of $E_{B_e}M_{B_m}$ as:

\begin{align}
\label{eq:tensor_quant}
    \hat{\bm{X}} = \text{round-to-nearest}\left(\frac{\bm{X}}{s_X}, E_{B_e}M_{B_m}\right)
\end{align}

We perform dynamic max-scaled quantization \citep{wu2020integer}, where the scale factor $s$ for activations is dynamically computed during runtime.

\subsection{Vector Scaled Quantization}
\begin{wrapfigure}{r}{0.35\linewidth}
  \centering
  \includegraphics[width=\linewidth]{sections/figures/vsquant.jpg}
  \caption{\small Vectorwise decomposition for per-vector scaled quantization (VSQ \citep{dai2021vsq}).}
  \label{fig:vsquant}
\end{wrapfigure}
During VSQ \citep{dai2021vsq}, the operand tensors are decomposed into 1D vectors in a hardware friendly manner as shown in Figure \ref{fig:vsquant}. Since the decomposed tensors are used as operands in matrix multiplications during inference, it is beneficial to perform this decomposition along the reduction dimension of the multiplication. The vectorwise quantization is performed similar to tensorwise quantization described in Equations \ref{eq:sf} and \ref{eq:tensor_quant}, where a scale factor $s_v$ is required for each vector $\bm{v}$ that maps the maximum absolute value of that vector to the maximum quantization level. While smaller vector lengths can lead to larger accuracy gains, the associated memory and computational overheads due to the per-vector scale factors increases. To alleviate these overheads, VSQ \citep{dai2021vsq} proposed a second level quantization of the per-vector scale factors to unsigned integers, while MX \citep{rouhani2023shared} quantizes them to integer powers of 2 (denoted as $2^{INT}$).

\subsubsection{MX Format}
The MX format proposed in \citep{rouhani2023microscaling} introduces the concept of sub-block shifting. For every two scalar elements of $b$-bits each, there is a shared exponent bit. The value of this exponent bit is determined through an empirical analysis that targets minimizing quantization MSE. We note that the FP format $E_{1}M_{b}$ is strictly better than MX from an accuracy perspective since it allocates a dedicated exponent bit to each scalar as opposed to sharing it across two scalars. Therefore, we conservatively bound the accuracy of a $b+2$-bit signed MX format with that of a $E_{1}M_{b}$ format in our comparisons. For instance, we use E1M2 format as a proxy for MX4.

\begin{figure}
    \centering
    \includegraphics[width=1\linewidth]{sections//figures/BlockFormats.pdf}
    \caption{\small Comparing LO-BCQ to MX format.}
    \label{fig:block_formats}
\end{figure}

Figure \ref{fig:block_formats} compares our $4$-bit LO-BCQ block format to MX \citep{rouhani2023microscaling}. As shown, both LO-BCQ and MX decompose a given operand tensor into block arrays and each block array into blocks. Similar to MX, we find that per-block quantization ($L_b < L_A$) leads to better accuracy due to increased flexibility. While MX achieves this through per-block $1$-bit micro-scales, we associate a dedicated codebook to each block through a per-block codebook selector. Further, MX quantizes the per-block array scale-factor to E8M0 format without per-tensor scaling. In contrast during LO-BCQ, we find that per-tensor scaling combined with quantization of per-block array scale-factor to E4M3 format results in superior inference accuracy across models. 

% \section{You \emph{can} have an appendix here.}

%You can have as much text here as you want. The main body must be at most $8$ pages long. For the final version, one more page can be added. If you want, you can use an appendix like this one, even using the one-column format.
%%%%%%%%%%%%%%%%%%%%%%%%%%%%%%%%%%%%%%%%%%%%%%%%%%%%%%%%%%%%%%%%%%%%%%%%%%%%%%%
%%%%%%%%%%%%%%%%%%%%%%%%%%%%%%%%%%%%%%%%%%%%%%%%%%%%%%%%%%%%%%%%%%%%%%%%%%%%%%%


\end{document}


% This document was modified from the file originally made available by
% Pat Langley and Andrea Danyluk for ICML-2K. This version was created
% by Iain Murray in 2018, and modified by Alexandre Bouchard in
% 2019 and 2021 and by Csaba Szepesvari, Gang Niu and Sivan Sabato in 2022.
% Modified again in 2023 by Sivan Sabato and Jonathan Scarlett.
% Previous contributors include Dan Roy, Lise Getoor and Tobias
% Scheffer, which was slightly modified from the 2010 version by
% Thorsten Joachims & Johannes Fuernkranz, slightly modified from the
% 2009 version by Kiri Wagstaff and Sam Roweis's 2008 version, which is
% slightly modified from Prasad Tadepalli's 2007 version which is a
% lightly changed version of the previous year's version by Andrew
% Moore, which was in turn edited from those of Kristian Kersting and
% Codrina Lauth. Alex Smola contributed to the algorithmic style files.
