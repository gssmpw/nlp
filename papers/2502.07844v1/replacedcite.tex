\section{Related Works}
\label{Sec:related}
\subsection{Human Model}
In current researches on dynamic spine modeling, the parametric method of human body models plays an important role. The SMPL model is a widely used parametric human body model that effectively describes and generates 3D human shapes through a small number of parameters. This model uses the Linear Blended Skin (LBS) algorithm to combine pose parameters with shape parameters, thereby achieving a flexible representation of human morphology. This makes SMPL widely used in various computer vision and graphics applications, especially in tasks such as human pose estimation and motion capture ____.
In addition to this classic model of SMPL, in recent years, the field of parametric human body modeling has ushered in a new wave. In particular, Neural Radiance Fields (NeRF) ____ and its innovative applications on human subjects have pushed this field to a new height. Human NeRF methods proposed by Chung-Yi Weng et al. ____, for example, are remarkable for their extraordinary ability to directly encode human geometry and appearance into neural networks and achieve realistic rendering. However, while these methods pursue extreme realism, they also face the dual challenges of computational efficiency and training complexity.
\par It is worth mentioning that in the vast field of parametric human body modeling, there is another method that is also worthy of attention, that is, 3D Gaussian human body modeling, which uses the characteristics of Gaussian distribution to parametrically represent human shape. By dividing the human body into multiple parts and describing each part with a 3D Gaussian distribution, a fine depiction of human shape can be achieved. For example, GauHuman ____, as an efficient 3D human body modeling algorithm, takes its articulated Gaussian splatting representation as the core and is specially designed for fast training and rendering. This method encodes Gaussian splatting in the canonical space and maps it to the pose space. Using the LBS weights of the SMPL model as a starting point, it predicts the corresponding LBS weight offsets through a multilayer perceptron (MLP), thereby achieving fast and accurate modeling of human shape. Similar to Human NeRF, while pursuing efficient modeling, GauHuman also fully utilizes the characteristics of 3D Gaussian distribution, enabling it to quickly build a 3D human body model in just one to two minutes.

\subsection{Spine Model}
Currently, approaches to spine modeling have focused on multibody dynamics and finite element method (FEM). For example, Lu et al. used Mimics and Ansys software to establish a 3D finite element model of the thoracolumbar segment T12 - L2 based on the CT scan data of the T12 - L2 vertebral body of a 40-year-old healthy female volunteer____. They verified the effectiveness of the model by simulating multiple working conditions and analyzing the displacement and stress distribution of the vertebral body and intervertebral disc. Similarly, Ibrahim El Bojairami et al. developed and validated a 3D comprehensive finite element spine model that includes multiple tissues and physiological effects ____. They used a new meshing technique to improve computational efficiency and proved its reliability through various verifications, providing an effective tool for spine-related research. However, although 3D FEM has the advantages of simplicity, speed and economy, and can simulate and calculate the mechanical properties of various materials under complex conditions, and the experiment has the advantages of being easy to adjust and repeatable, FEM still has limitations. For example, it is difficult to accurately simulate certain nonlinear properties, and it is susceptible to human influence when dealing with the parameters of the model. In addition, it is often difficult to combine the FEM with the human body's action postures in biomechanical analysis. In addition, related research also has limitations such as limited simulation range, only using finite element analysis software, and failing to effectively simulate some human anatomical structure ____.
\par We can attempt to address these issues by applying computer vision methods to develop a technique for constructing a human spine model. In ____, the authors constructed 3D spine by using CNN to fit a statistical spine model to images, which demonstrated good performance in terms of landmark location accuracy and clinical parameter extraction, with a short reconstruction time, thereby providing valuable support for 3D measurements in clinical practice. Chen et al. proposed a generative adversarial network (GAN) framework called ReVerteR ____, which aimed to automatically reconstruct a 3D model of the spine from orthogonal biplane X-ray maps. The framework utilized digitally reconstructed X-ray (DRR) images from coronal and sagittal planes for 3D reconstruction and developed a deep learning-driven transformation module to improve the accuracy of the reconstruction. In addition, they introduced an automatic center-of-mass annotation module, which enhanced the overall performance of the 3D reconstruction process.
In____, the BX2S-Net deep learning framework was developed for reconstructing 3D spine structures from biplane X-ray maps. The network takes individual vertebrae in anterior-posterior (AP) and lateral (LAT) views as input and output 3D models in voxel grid format. BX2S-Net employs an incremental decoding process combined with feature fusion and attention mechanisms to optimize the 3D reconstruction results.
\par The above studies demonstrate recent advances in the field of reconstructing spine models from X-rays, employing different deep learning approaches and providing high-quality 3D spine models, which provide important references in addressing the limitations of traditional methods. 

\begin{figure*}
	\centering
     \includegraphics[width=\textwidth]{Fig2.pdf}
	\caption{In the given frame, sampling operations are carried out on the human body surface to form a point cloud, and at the same time, the corresponding positions of these points are marked in the UV position map. After that, the UV position map is fed into the pose encoder to generate the corresponding pose features. During this period, the optimizable feature tensor is precisely aligned with the pose features in a specific way, with the aim of more effectively capturing the overall appearance of the human body. These aligned features are input into the Gaussian parameter decoder, which can predict the offset $\delta x$, color c and scale s of each point. And these predicted results, together with the fixed rotation q and opacity $\alpha$, jointly form an animatable 3D Gaussian distribution in the canonical space.}
    \label{network}
\end{figure*}

%%%%%%%%%%%%%%%%%%%%%%%%%%%%%%%%%%%%%%%%%%%%%%%%%%%%%
%%%%%%%%%%%%%%%%%%%%%%%%%%%%%%%%