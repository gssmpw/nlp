\documentclass[11pt]{article}

%%% ready for submission %%%
%%%%% NEW MATH DEFINITIONS %%%%%

% \usepackage{amsmath,amsfonts,bm}
\usepackage{amsmath,amsfonts}

\usepackage{pifont}


\newcommand{\R}{\mathbb{R}}


\def\va{{\mathbf{a}}}
\def\vg{{\mathbf{g}}}

% Sets
\def\sR{\mathbb{R}}
\def\sC{\mathbb{C}}
\def\sZ{\mathbb{Z}}
\def\sN{\mathbb{N}}
\def\sQ{\mathbb{Q}}

\def\sS{\mathcal{S}}



% Vectors
\def\vzero{{\mathbf{0}}}
\def\vone{{\mathbf{1}}}
\def\vmu{{\mathbf{\mu}}}
\def\vtheta{{\mathbf{\theta}}}
\def\va{{\mathbf{a}}}
\def\vb{{\mathbf{b}}}
\def\vc{{\mathbf{c}}}
\def\vd{{\mathbf{d}}}
\def\ve{{\mathbf{e}}}
\def\vf{{\mathbf{f}}}
\def\vg{{\mathbf{g}}}
\def\vh{{\mathbf{h}}}
\def\vi{{\mathbf{i}}}
\def\vj{{\mathbf{j}}}
\def\vk{{\mathbf{k}}}
\def\vl{{\mathbf{l}}}
\def\vm{{\mathbf{m}}}
\def\vn{{\mathbf{n}}}
\def\vo{{\mathbf{o}}}
\def\vp{{\mathbf{p}}}
\def\vq{{\mathbf{q}}}
\def\vr{{\mathbf{r}}}
\def\vs{{\mathbf{s}}}
\def\vt{{\mathbf{t}}}
\def\vu{{\mathbf{u}}}
\def\vv{{\mathbf{v}}}
\def\vw{{\mathbf{w}}}
\def\vx{{\mathbf{x}}}
\def\vy{{\mathbf{y}}}
\def\vz{{\mathbf{z}}}
\def\vzeta{{\mathbf{\zeta}}}

% Matrix
\def\mA{{\mathbf{A}}}
\def\mB{{\mathbf{B}}}
\def\mC{{\mathbf{C}}}
\def\mD{{\mathbf{D}}}
\def\mE{{\mathbf{E}}}
\def\mF{{\mathbf{F}}}
\def\mG{{\mathbf{G}}}
\def\mH{{\mathbf{H}}}
\def\mI{{\mathbf{I}}}
\def\mJ{{\mathbf{J}}}
\def\mK{{\mathbf{K}}}
\def\mL{{\mathbf{L}}}
\def\mM{{\mathbf{M}}}
\def\mN{{\mathbf{N}}}
\def\mO{{\mathbf{O}}}
\def\mP{{\mathbf{P}}}
\def\mQ{{\mathbf{Q}}}
\def\mR{{\mathbf{R}}}
\def\mS{{\mathbf{S}}}
\def\mT{{\mathbf{T}}}
\def\mU{{\mathbf{U}}}
\def\mV{{\mathbf{V}}}
\def\mW{{\mathbf{W}}}
\def\mX{{\mathbf{X}}}
\def\mY{{\mathbf{Y}}}
\def\mZ{{\mathbf{Z}}}
\def\mBeta{{\mathbf{\beta}}}
\def\mPhi{{\mathbf{\Phi}}}
\def\mLambda{{\mathbf{\Lambda}}}
\def\mSigma{{\mathbf{\Sigma}}}


% Expectation
% \def\eE{\mathop{\mathbb{E}}\limits}
\def\eE{\mathbb{E}}

% Probability
\def\pP{\mathbb{P}}

% Tilde
\def\tf{\tilde{f}}
\def\tS{\tilde{S}}
\def\wtF{\widetilde{\mathcal{F}}}
\def\whR{\widehat{R}}
\def\tvx{\tilde{\mathbf{x}}}
\def\ty{\tilde{y}}


\def\defeq{\overset{\textup{def}}{=}}
% \def\defeq{\overset{.}{=}}
\def\defone{\overset{\text{\ding{172}}}{=}}
\def\deftwo{\overset{\text{\ding{173}}}{=}}
\def\leqone{\overset{\text{\ding{172}}}{\leq}}
\def\leqtwo{\overset{\text{\ding{173}}}{\leq}}
\def\leqthree{\overset{\text{\ding{174}}}{\leq}}
\def\leqfour{\overset{\text{\ding{175}}}{\leq}}
\def\eqone{\overset{\text{\ding{172}}}{=}}
\def\eqtwo{\overset{\text{\ding{173}}}{=}}
\def\eqthree{\overset{\text{\ding{174}}}{=}}
\def\eqfour{\overset{\text{\ding{175}}}{=}}
\def\geqfive{\overset{\text{\ding{176}}}{\geq}}
%%% REVIEW
\newcommand{\tocite}{{\color{red}CITE} }
\newcommand{\toref}{{\color{red}REF} }

%%% LOGO
\newcommand{\usc}{\raisebox{-1pt}{\includegraphics[height=0.8em]{figures/usc_logo.png}}}
\newcommand{\vuam}{\raisebox{-1pt}{\includegraphics[height=0.8em]{figures/vu_logo.png}}}

%%% SIGNS and SYMBOLS
\newcommand{\grad}{\texttt{grad-CROP}}
\newcommand{\att}{\texttt{att-CROP}}
\newcommand{\seg}{\texttt{seg}}
\newcommand{\clip}{\texttt{clip-CROP}}
\newcommand{\sam}{\texttt{sam-CROP}}
\newcommand{\yolo}{\texttt{yolo-CROP}}
\newcommand{\hc}{\texttt{human-CROP}}
\newcommand{\zsvqa}{\texttt{ZSVQA}}
\newcommand{\vic}{\textbf{ViCrop}}
\newcommand{\xmark}{\text{\ding{55}}}
\newcommand{\cmark}{\text{\ding{51}}}
\newcommand{\success}{\texttt{\color{green} \cmark}}
\newcommand{\failure}{\texttt{\color{red} \xmark}}
\newcommand{\rel}{\texttt{rel-att}}
\newcommand{\gra}{\texttt{grad-att}}
\newcommand{\pgra}{\texttt{pure-grad}}
\newcommand{\relh}{\texttt{rel-att$^h$}}
\newcommand{\grah}{\texttt{grad-att$^h$}}
\newcommand{\pgrah}{\texttt{pure-grad$^h$}}


%%% Text Abb.
\makeatletter
\DeclareRobustCommand\onedot{\futurelet\@let@token\@onedot}
\def\@onedot{\ifx\@let@token.\else.\null\fi\xspace}

\def\aka{\emph{a.k.a}\onedot} \def\Eg{\emph{E.g}\onedot}
\def\eg{\emph{e.g}\onedot} \def\Eg{\emph{E.g}\onedot}
\def\ie{\emph{i.e}\onedot} \def\Ie{\emph{I.e}\onedot}
\def\cf{\emph{c.f}\onedot} \def\Cf{\emph{C.f}\onedot}
\def\etc{\emph{etc}\onedot} \def\vs{\emph{vs}\onedot}
\def\wrt{w.r.t\onedot} \def\dof{d.o.f\onedot}
\def\etal{\emph{et al}\onedot}
\makeatletter



\definecolor{myred}{HTML}{FF8577}
\definecolor{mygreen}{HTML}{0FA958}
\definecolor{myblue}{HTML}{1982C4}
\definecolor{codegreen}{rgb}{0,0.5,0}
\definecolor{codegray}{rgb}{0.5,0.5,0.5}
\definecolor{codepurple}{rgb}{0.07,0,0.53}
\definecolor{codered}{RGB}{189,41,0}
\definecolor{codecomment}{RGB}{153,153,153}
\definecolor{backcolour}{rgb}{0.96,0.96,0.96}
\definecolor{royalblue}{rgb}{0.0, 0.14, 0.4}
\definecolor{egyptianblue}{rgb}{0.06, 0.2, 0.65}
\definecolor{royalazure}{rgb}{0.0, 0.22, 0.66}
\definecolor{portlandorange}{rgb}{1.0, 0.35, 0.21}
\definecolor{sienna}{RGB}{183,105,68}
\definecolor{saddlebrown}{RGB}{139,69,19}
\definecolor{mediumbrown}{RGB}{83,41,11}
\definecolor{darkbrown}{RGB}{58,28,7}
\hypersetup{
    colorlinks=true,
    linkcolor=sienna,
    urlcolor=royalblue,
    citecolor=royalblue,
}
\usepackage[utf8]{inputenc} % allow utf-8 input
\usepackage[T1]{fontenc}    % use 8-bit T1 fonts
\usepackage{tgpagella}
\usepackage{tcolorbox}
\usepackage{mathtools}
\usepackage{hyperref}
\usepackage{subcaption}
\usepackage{svg}

\usepackage{enumitem}
\usepackage{authblk}
\usepackage{natbib}
\hypersetup{
    colorlinks=true,
    linkcolor=blue,
    citecolor=magenta,      
    urlcolor=cyan,
    pdfpagemode=FullScreen,
}
\usepackage{url}
\usepackage{subcaption}
\usepackage{graphicx}
\usepackage{lipsum}  
\usepackage{cleveref}
\usepackage{xcolor}  
\usepackage{wrapfig}% colors
%\usepackage{natbib}
\usepackage{multirow}
% \bibliographystyle{achemso}

\newcommand{\conv}{\circledast}
\newcommand{\cconv}{  \boxasterisk }
\newcommand{\mb}{\mathbf}
\newcommand{\mc}{\mathcal}
\newcommand{\mf}{\mathfrak}
\newcommand{\md}{\mathds}
\newcommand{\bb}{\mathbb}
\newcommand{\magnitude}[1]{ \left| #1 \right| }
\newcommand{\set}[1]{\left\{ #1 \right\}}
\newcommand{\condset}[2]{ \left\{ #1 \;\middle|\; #2 \right\} }
\newcommand{\tbW}{\widetilde{\bm{W}}}
\newcommand{\bWcomp}{\bm{W}^{\textrm{comp}}}
\newcommand{\DS}[1]{{\color{magenta} (DS: #1)}}
\newcommand{\ZK}[1]{{\color{cyan} [{\em Zekai:} #1]}}
\newcommand{\SMK}[1]{{\color{red} [{\em SMK:} #1]}}

\newcommand{\qq}[1]{\textcolor{blue}{\bf [{\em Qing:} #1]}}

\newcommand\blfootnote[1]{%
  \begingroup
  \renewcommand\thefootnote{}\footnote{#1}%
  \addtocounter{footnote}{-1}%
  \endgroup
}

\newcommand{\fix}{\marginpar{FIX}}
\newcommand{\new}{\marginpar{NEW}}
%%% to compile a preprint version, e.g., for submission to arXiv, add the [preprint] option %%%
%\usepackage[preprint]{cpal_2024}
%\newcommand{\ag}[1]{{\color{blue} [\textbf{AG:} #1]}}

%%% to compile a camera-ready version, add the [final] option %%%
%\usepackage[final]{cpal_2024}
\usepackage[top=1in, bottom=1in, left=1in, right=1in]{geometry}


%add packages
\usepackage{url}

%\title{Compressing Overparameterized Deep Models by Harnessing Low-Dimensional Learning Dynamics}
\title{Learning Dynamics of Deep Linear Networks Beyond the Edge of Stability}

%\author{%
%  Soo Min Kwon\textsuperscript{1}\thanks{The first two authors have contributed equally to this work. Correspondence to \texttt{kwonsm@umich.edu}.}, ~Zekai Zhang\textsuperscript{2}$^*$, ~Dogyoon Song\textsuperscript{1}, ~Laura Balzano\textsuperscript{1}, ~Qing Qu\textsuperscript{1} \\
%  \vspace{12pt}
%  \textsuperscript{1} Department of EECS, University of Michigan \\
%  \vspace{3pt}
%  \textsuperscript{2} Department of Automation, Tsinghua University \\
%}
\author[1]{Avrajit Ghosh\footnote{The first two authors contributed to this work equally. Correspondence to \texttt{ghoshavr@msu.edu}, \texttt{kwonsm@umich.edu}. Code to reproduce the results is available at \href{https://github.com/soominkwon/dln-at-eos}{\texttt{https://github.com/soominkwon/dln-at-eos}}.}}
\author[2]{Soo Min Kwon$^*$}
\author[1]{Rongrong Wang}
\author[1]{\\Saiprasad Ravishankar}
\author[2]{Qing Qu}

\newcommand\CoAuthorMark{\footnotemark[\arabic{footnote}]}
\affil[1]{Computational Mathematics Science and Engineering, Michigan State University}
\affil[2]{Department of Electrical Engineering \& Computer Science, University of Michigan}


%\date{\today}

\begin{document}


\maketitle


\begin{abstract}
Deep neural networks trained using gradient descent with a fixed learning rate $\eta$ often operate in the regime of ``edge of stability'' (EOS), where the largest eigenvalue of the Hessian equilibrates about the stability threshold $2/\eta$. In this work, we present a fine-grained analysis of the learning dynamics of (deep) linear networks (DLNs) within the deep matrix factorization loss beyond EOS. For DLNs, loss oscillations beyond EOS follow a period-doubling route to chaos.
We theoretically analyze the regime of the 2-period orbit and show that the loss oscillations occur within a small subspace, with the dimension of the subspace precisely characterized by the learning rate.
The crux of our analysis lies in showing that the symmetry-induced conservation law for gradient flow, defined as the balancing gap among the singular values across layers, breaks at EOS and decays monotonically to zero.
Overall, our results contribute to explaining two key phenomena in deep networks: (i) shallow models and simple tasks do not always exhibit EOS~\citep{cohen2021gradient}; and (ii) oscillations occur within top features~\citep{zhu2023catapults}. We present experiments to support our theory, along with examples demonstrating how these phenomena occur in nonlinear networks and how they differ from those which have benign landscape such as in DLNs.
  

%Lastly, we experiment with using low-rank adaptors to fine-tune deep networks with large learning rates, showing that the resulting catapult dynamics in the loss can potentially improve generalization, which may be of independent interest.
 
%Lastly, we also demonstrate that fine-tuning deep networks using low-rank adaptation with large learning rates induces catapult dynamics in the loss, which has the potential to improve generalization.
 
 

%In this work, we investigate the training dynamics of deep networks with a large learning rate $\eta$, commonly used in machine learning practice for improved empirical performance.
%In this work, we provide a fine-grained analysis of the training dynamics of weight matrices with a large learning rate $\eta$, commonly used in machine learning practice for improved empirical performance. 
%This regime is also known as the edge of stability, where the largest eigenvalue of the Hessian hovers around $2/\eta$, and the training loss oscillates yet decreases over long timescales. Within this regime, we observe an intriguing phenomenon: the oscillations in the training loss are artifacts of the oscillations of \emph{only} a few leading singular values of the weight matrices within a tiny
%small invariant
%subspace. Theoretically, we analyze this behavior based on the deep matrix factorization problem, showing that this oscillation behavior closely follows that of its nonlinear counterparts. 
%Our analysis reveals that for $\eta$ within a specific range,
%We provably show that for $\eta$ within a specific range, 
%\edit{the leading subspaces of each weight matrix oscillate} within a 2-period fixed orbit. \edit{Empirically, we corroborate our theory by demonstrating that these oscillations occur not only in both linear and nonlinear networks but also during the fine-tuning of low-rank adaptors, and we discuss the practical implications of these findings.}
%\edit{}

%We extensively corroborate our theory with empirical justifications, namely in that (i) deep linear and nonlinear networks share many properties in their learning dynamics and (ii) our model captures the nuances that occur at the edge of stability which other models do not, providing deeper insights into this phenomenon. 

\end{abstract}



\tableofcontents


\section{Introduction}
\label{sec:intro}

\begin{figure*}[tb]
    \centering
    \includegraphics[width=0.848\linewidth]{figs/circuitnn.pdf} 
    \caption{Illustration of differentiable CircuitNN. CircuitNN is designed based on differentiable NAND gates. After DAS is guided by PI and PO pairs of the truth table, CircuitNN can get the precise circuit architecture logic equivalent to the truth table.}
    \label{fig:circuitnn}
\end{figure*}

% 1. Describe the importance of logic synthesis
% 2. Existing Problems
% (a) Neural Architecture Search: Unstable, Predefined Setting, etc.
% (b) Circuit Generation: Probabilistic Model, Logic Equivalence

With the rapid advancement of technology, the scale of integrated circuits (ICs) has expanded exponentially. 
This expansion has introduced significant challenges in chip manufacturing, particularly concerning power and area metrics.
A primary objective in IC design is achieving the same circuit function with fewer transistors, thereby reducing power usage and area occupancy.

Logic synthesis~\cite{hachtel2005logicsynth}, a critical step in electronic design automation (EDA), transforms behavioral-level circuit designs into optimized gate-level circuits, ultimately yielding the final IC layout. 
The primary goal of logic synthesis is to identify the physical implementation with the fewest gates for a given circuit function. 
This task constitutes a challenging NP-hard combinatorial optimization problem. 
Current logic synthesis tools~\cite{brayton2010abc, wolf2013yosys} rely on human-designed heuristics, often leading to sub-optimal outcomes.

Differentiable architecture search (DAS) techniques~\cite{liu2018darts, chu2020darts} offer novel perspectives on addressing challenges in this problem.
Circuit functions can be represented through truth tables, which map binary inputs to their corresponding outputs. 
Truth tables provide a precise representation of input-output relationships, ensuring the design of functionally equivalent circuits.
Inspired by this, researchers~\cite{deepmind2024ai4sys, wang2024tnet} have begun exploring the application of DAS to synthesize circuits directly from truth tables.
Specifically, \citet{deepmind2024ai4sys} proposed CircuitNN, a framework that learns differentiable connection structures with logic gates, enabling the automatic generation of logic circuits from truth tables.
This approach significantly reduces the complexity of traditional circuit generation. 
Building on this, \citet{wang2024tnet} introduced T-Net, a triangle-shaped variant of CircuitNN, incorporating regularization techniques to enhance the efficiency of DAS.

Despite these advancements, several challenges remain. 
The computational complexity of DAS grows quadratically with the number of gates, posing scalability issues.
Although triangle-shaped architecture~\cite{wang2024tnet} partially mitigates this problem, redundancy persists. 
%Additionally, DAS is susceptible to converging to local optima, limiting the ability to search architectures that satisfy the given truth tables~\cite{liu2018darts}. 
%Furthermore, hyperparameters (network depth and layer width) require extensive searches, introducing complexity and prolonging the synthesis process. 
Additionally, DAS is susceptible to converging to local optima~\cite{liu2018darts} and hyperparameters (network depth and layer width) require extensive searches. 
The challenges arise from the vast search space in DAS. 
% Even with predefined settings for CircuitNN, finding a configuration that meets the truth table requires extensive trial and error during the DAS process. 
Intuitively, limiting the search space through predefined parameters (network depth, gates per layer, and connection probabilities) can significantly reduce the complexity.

Recent advances~\cite{openai2023gpt4, abramson2024alphafold3, esser2024sd3, li2024mar} in conditional generative models have demonstrated remarkable performance across language, vision, and graph generation tasks. 
Motivated by these developments, we propose a novel approach to circuit generation that generates preliminary circuit structures to guide DAS in generating refined circuits matching specified truth tables. 
Firstly, we introduce CircuitVQ, a tokenizer with a discrete codebook for circuit tokenization. 
Built upon our Circuit AutoEncoder framework~\cite{hou2022graphmae,li2023maskgae,wu2025mgvga}, CircuitVQ is trained through a circuit reconstruction task. 
Specifically, the CircuitVQ encoder encodes input circuits into discrete tokens using a learnable codebook, while the decoder reconstructs the circuit adjacency matrix based on these tokens.
Subsequently, the CircuitVQ encoder serves as a circuit tokenizer for CircuitAR pretraining, which employs a masked autoregressive modeling paradigm~\cite{chang2022maskgit, li2023mage}. 
In this process, the discrete codes function as supervision signals. 
After training, CircuitAR can generate discrete tokens progressively, which can be decoded into initial circuit structures by the decoder of the CircuitVQ. 
These prior insights can guide DAS in producing refined circuits that match the target truth tables precisely.

Our key contributions can be summarized as follows:
\begin{itemize}
\item We introduce CircuitVQ, a circuit tokenizer that facilitates graph autoregressive modeling for circuit generation, based on our Circuit AutoEncoder framework;
\item Develop CircuitAR, a model trained using masked autoregressive modeling, which generates initial circuit structures conditioned on given truth tables;
\item Propose a refinement framework that integrates differentiable architecture search to produce functionally equivalent circuits guided by target truth tables;
\item Comprehensive experiments demonstrating the scalability and capability emergence of our CircuitAR and the superior performance of the proposed circuit generation approach.
\end{itemize}

% Motivation
% (a) Diffusion (Vision, Graph), Autoregressive (Language, Vision)
% (b) Circuit Generation for Predefined Setting
% (c) Neural Architecture Search for Strict Logic Equivalence

% Contribution
% (a) Circuit Tokenizer (new transformer arch, training strategy)
% (b) CircuitAR (train and gen strategies, post-ar strategy)
% (c) Extensive Evaluation including BitD (Bit Distance) for Scalability

\section{Experiments}
\label{sec:Experiments} 

We conduct several experiments across different problem settings to assess the efficiency of our proposed method. Detailed descriptions of the experimental settings are provided in \cref{sec:apendix_experiments}.
%We conduct experiments on optimizing PINNs for convection, wave PDEs, and a reaction ODE. 
%These equations have been studied in previous works investigating difficulties in training PINNs; we use the formulations in \citet{krishnapriyan2021characterizing, wang2022when} for our experiments. 
%The coefficient settings we use for these equations are considered challenging in the literature \cite{krishnapriyan2021characterizing, wang2022when}.
%\cref{sec:problem_setup_additional} contains additional details.

%We compare the performance of Adam, \lbfgs{}, and \al{} on training PINNs for all three classes of PDEs. 
%For Adam, we tune the learning rate by a grid search on $\{10^{-5}, 10^{-4}, 10^{-3}, 10^{-2}, 10^{-1}\}$.
%For \lbfgs, we use the default learning rate $1.0$, memory size $100$, and strong Wolfe line search.
%For \al, we tune the learning rate for Adam as before, and also vary the switch from Adam to \lbfgs{} (after 1000, 11000, 31000 iterations).
%These correspond to \al{} (1k), \al{} (11k), and \al{} (31k) in our figures.
%All three methods are run for a total of 41000 iterations.

%We use multilayer perceptrons (MLPs) with tanh activations and three hidden layers. These MLPs have widths 50, 100, 200, or 400.
%We initialize these networks with the Xavier normal initialization \cite{glorot2010understanding} and all biases equal to zero.
%Each combination of PDE, optimizer, and MLP architecture is run with 5 random seeds.

%We use 10000 residual points randomly sampled from a $255 \times 100$ grid on the interior of the problem domain. 
%We use 257 equally spaced points for the initial conditions and 101 equally spaced points for each boundary condition.

%We assess the discrepancy between the PINN solution and the ground truth using $\ell_2$ relative error (L2RE), a standard metric in the PINN literature. Let $y = (y_i)_{i = 1}^n$ be the PINN prediction and $y' = (y'_i)_{i = 1}^n$ the ground truth. Define
%\begin{align*}
%    \mathrm{L2RE} = \sqrt{\frac{\sum_{i = 1}^n (y_i - y'_i)^2}{\sum_{i = 1}^n y'^2_i}} = \sqrt{\frac{\|y - y'\|_2^2}{\|y'\|_2^2}}.
%\end{align*}
%We compute the L2RE using all points in the $255 \times 100$ grid on the interior of the problem domain, along with the 257 and 101 points used for the initial and boundary conditions.

%We develop our experiments in PyTorch 2.0.0 \cite{paszke2019pytorch} with Python 3.10.12.
%Each experiment is run on a single NVIDIA Titan V GPU using CUDA 11.8.
%The code for our experiments is available at \href{https://github.com/pratikrathore8/opt_for_pinns}{https://github.com/pratikrathore8/opt\_for\_pinns}.


\subsection{2D Allen Cahn Equation}
\begin{figure*}[t]
    \centering
    \includegraphics[scale=0.38]{figs/Burgers_operator.pdf}
    \caption{1D Burgers' Equation (Operator Learning): Steady-state solutions for different initializations $u_0$ under varying viscosity $\varepsilon$: (a) $\varepsilon = 0.5$, (b) $\varepsilon = 0.1$, (c) $\varepsilon = 0.05$. The results demonstrate that all final test solutions converge to the correct steady-state solution. (d) Illustration of the evolution of a test initialization $u_0$ following homotopy dynamics. The number of residual points is $\nres = 128$.}
    \label{fig:Burgers_result}
\end{figure*}
First, we consider the following time-dependent problem:
\begin{align}
& u_t = \varepsilon^2 \Delta u - u(u^2 - 1), \quad (x, y) \in [-1, 1] \times [-1, 1] \nonumber \\
& u(x, y, 0) = - \sin(\pi x) \sin(\pi y) \label{eq.hom_2D_AC}\\
& u(-1, y, t) = u(1, y, t) = u(x, -1, t) = u(x, 1, t) = 0. \nonumber
\end{align}
We aim to find the steady-state solution for this equation with $\varepsilon = 0.05$ and define the homotopy as:
\begin{equation}
    H(u, s, \varepsilon) = (1 - s)\left(\varepsilon(s)^2 \Delta u - u(u^2 - 1)\right) + s(u - u_0),\nonumber
\end{equation}
where $s \in [0, 1]$. Specifically, when $s = 1$, the initial condition $u_0$ is automatically satisfied, and when $s = 0$, it recovers the steady-state problem. The function $\varepsilon(s)$ is given by
\begin{equation}
\varepsilon(s) = 
\left\{\begin{array}{l}
s, \quad s \in [0.05, 1], \\
0.05, \quad s \in [0, 0.05].
\end{array}\right.\label{eq:epsilon_t}
\end{equation}

Here, $\varepsilon(s)$ varies with $s$ during the first half of the evolution. Once $\varepsilon(s)$ reaches $0.05$, it remains fixed, and only $s$ continues to evolve toward $0$. As shown in \cref{fig:2D_Allen_Cahn_Equation}, the relative $L_2$ error by homotopy dynamics is $8.78 \times 10^{-3}$, compared with the result obtained by PINN, which has a $L_2$ error of $9.56 \times 10^{-1}$. This clearly demonstrates that the homotopy dynamics-based approach significantly improves accuracy.

\subsection{High Frequency Function Approximation }
We aim to approximate the following function:
$u=    \sin(50\pi x), \quad x \in [0,1].$
The homotopy is defined as $H(u,\varepsilon) = u - \sin(\frac{1}{\varepsilon}\pi x), $
where $\varepsilon \in [\frac{1}{50},\frac{1}{15}]$.

\begin{table}[htbp!]
    \caption{Comparison of the lowest loss achieved by the classical training and homotopy dynamics for different values of $\varepsilon$ in approximating $\sin\left(\frac{1}{\varepsilon} \pi x\right)$
    }
    \vskip 0.15in
    \centering
    \tiny
    \begin{tabular}{|c|c|c|c|c|} 
    \hline 
    $ $ & $\varepsilon = 1/15$ & $\varepsilon = 1/35$ & $\varepsilon = 1/50$ \\ \hline 
    Classical Loss                & 4.91e-6     & 7.21e-2     & 3.29e-1       \\ \hline 
    Homotopy Loss $L_H$                      & 1.73e-6     & 1.91e-6     & \textbf{2.82e-5}       \\ \hline
    \end{tabular}
    % On convection, \al{} provides 14.2$\times$ and 1.97$\times$ improvement over Adam or \lbfgs{} on L2RE. 
    % On reaction, \al{} provides 1.10$\times$ and 1.99$\times$ improvement over Adam or \lbfgs{} on L2RE.
    % On wave, \al{} provides 6.32$\times$ and 6.07$\times$ improvement over Adam or \lbfgs{} on L2RE.}
    \label{tab:loss_approximate}
\end{table}

As shown in \cref{fig:high_frequency_result}, due to the F-principle \cite{xu2024overview}, training is particularly challenging when approximating high-frequency functions like $\sin(50\pi x)$. The loss decreases slowly, resulting in poor approximation performance. However, training based on homotopy dynamics significantly reduces the loss, leading to a better approximation of high-frequency functions. This demonstrates that homotopy dynamics-based training can effectively facilitate convergence when approximating high-frequency data. Additionally, we compare the loss for approximating functions with different frequencies $1/\varepsilon$ using both methods. The results, presented in \cref{tab:loss_approximate}, show that the homotopy dynamics training method consistently performs well for high-frequency functions.





\subsection{Burgers Equation}
In this example, we adopt the operator learning framework to solve for the steady-state solution of the Burgers equation, given by:
\begin{align}
& u_t+\left(\frac{u^2}{2}\right)_x - \varepsilon u_{xx}=\pi \sin (\pi x) \cos (\pi x), \quad x \in[0, 1]\nonumber\\
& u(x, 0)=u_0(x),\label{eq:1D_Burgers} \\
& u(0, t)=u(1, t)=0, \nonumber 
\end{align}
with Dirichlet boundary conditions, where $u_0 \in L_{0}^2((0, 1); \mathbb{R})$ is the initial condition and $\varepsilon \in \mathbb{R}$ is the viscosity coefficient. We aim to learn the operator mapping the initial condition to the steady-state solution, $G^{\dagger}: L_{0}^2((0, 1); \mathbb{R}) \rightarrow H_{0}^r((0, 1); \mathbb{R})$, defined by $u_0 \mapsto u_{\infty}$ for any $r > 0$. As shown in Theorem 2.2 of \cite{KREISS1986161} and Theorems 2.5 and 2.7 of \cite{hao2019convergence}, for any $\varepsilon > 0$, the steady-state solution is independent of the initial condition, with a single shock occurring at $x_s = 0.5$. Here, we use DeepONet~\cite{lu2021deeponet} as the network architecture. 
The homotopy definition, similar to ~\cref{eq.hom_2D_AC}, can be found in \cref{Ap:operator}. The results can be found in \cref{fig:Burgers_result} and \cref{tab:loss_burgers}. Experimental results show that the homotopy dynamics strategy performs well in the operator learning setting as well.


\begin{table}[htbp!]
    \caption{Comparison of loss between classical training and homotopy dynamics for different values of $\varepsilon$ in the Burgers equation, along with the MSE distance to the ground truth shock location, $x_s$.}
    \vskip 0.15in
    \centering
    \tiny
    \begin{tabular}{|c|c|c|c|c|} 
    \hline  
    $ $ & $\varepsilon = 0.5$ & $\varepsilon = 0.1$ & $\varepsilon = 0.05$ \\ \hline 
    Homotopy Loss $L_H$                &  7.55e-7     & 3.40e-7     & 7.77e-7       \\ \hline 
    L2RE                      & 1.50e-3     & 7.00e-4     & 2.52e-2       \\ \hline
        MSE Distance $x_s$                      & 1.75e-8     & 9.14e-8      & 1.2e-3      \\ \hline
    \end{tabular}
    % On convection, \al{} provides 14.2$\times$ and 1.97$\times$ improvement over Adam or \lbfgs{} on L2RE. 
    % On reaction, \al{} provides 1.10$\times$ and 1.99$\times$ improvement over Adam or \lbfgs{} on L2RE.
    % On wave, \al{} provides 6.32$\times$ and 6.07$\times$ improvement over Adam or \lbfgs{} on L2RE.}
    \label{tab:loss_burgers}
\end{table}



% \begin{itemize}
%     \item Relate the curvature in the problem to the differential operator. Use this to demonstrate why the problem is ill-conditioned
%     \item Give an argument for why using Adam + L-BFGS is better than just using L-BFGS outright. The idea is that Adam lowers the errors to the point where the rest of the optimization becomes convex \ldots
%     \item Show why we need second-order methods. We would like to prove that once we are close to the optimum, second-order methods will give condition-number free linear convergence. Specialize this to the Gauss-Newton setting, with the randomized low-rank approximation.
%     % \item Show that it is not possible to get superlinear convergence under the interpolation assumption with an overparameterized neural network. This should be true b/c the Hessian at the optimum will have rank $\min(n, d)$, and when $d > n$, this guarantees that we cannot have strong convexity.
% \end{itemize}
\section{Experiments: Planning outperforms Heuristics}
\label{sec:experiment}

We begin our empirical demonstrations by showcasing the effectiveness of our planning framework on both synthetic and real datasets. We focus on the simplest planning algorithm, 1-step lookaheads (Algorithm~\ref{alg:complete}), and show that even basic planning can hold great promise. 
We illustrate our framework using two uncertainty quantification modules---GPs and 
\ensembles/ \ensembleplus. 

Throughout this section, we focus on evaluating the mean squared error of 
a regression model $\model$,  and develop adaptive policies that minimize uncertainty on $g(f)$ defined in~\eqref{eqn:l2-g-f}.
When GPs provide a valid model of uncertainty, 
our experiments show that our planning framework significantly outperforms other baselines. 
We further demonstrate that our conceptual framework extends to deep learning-based uncertainty quantification methods such as  \ensembleplus while highlighting computational challenges that need to be resolved in order to scale our ideas. 
For simplicity, we assume a naive predictor, i.e., $\psi(\cdot) \equiv 0$. However, we emphasize that this problem is just as complex as if we were using a sophisticated model $\psi(.)$. The performance gap between the algorithms 
primarily depends
on the level  of uncertainty in our prior beliefs.

To evaluate the performance of our algorithm, we benchmark it against several baselines. 
%Active learning baselines use an acquisition function $\ac$ to select points that have the highest   function value: $X\opt_t \in \argmax_{X \in \xpoolj{t}} \ac({X})$ at every step $t$. These methods may also need an UQ module, which we simply use the same UQ module as in our algorithm, and it  outputs $V(X)$ that measures the the uncertainty of each point $X \in \xpoolj{t}$.
Our first set of baselines are from active learning~\citep{AggarwalKoGuHaPh14}:
\\ % \noindent\textbf{Active Learning Heuristics:} 
\textbf{(1)} 
\textsf{Uncertainty Sampling (Static):}  In this approach, we query the samples for which the model is least certain about. Specifically, we estimate the variance of the latent output $f(X)$ for each $X \in \xpool$ using the UQ module and select the top-$K$ points with the highest uncertainty. \\
\textbf{(2)} \textsf{Uncertainty Sampling (Sequential):} This is a greedy heuristic that sequentially selects the points with the highest uncertainty within a batch, while updating the posterior beliefs using pseudo labels from the current posterior state. Unlike \textsf{Uncertainty Sampling (Static)}, this method takes into account the information gained from each point within batch, and hence tries to diversify the selected points within a batch. 

 
We also compare our approach to the  \textbf{(3)} \textsf{Random Sampling}, which selects each batch uniformly at random from the pool. Additionally, we compare solving the planning problem using  \textsf{REINFORCE}-based policy gradients with   $\mathsf{Smoothed\text{-}Autodiff}$ policy gradients.\footnote{Our code repository is available at
  \url{https://github.com/namkoong-lab/adaptive-labeling}.}
%Detailed experimental setups are provided in Section \ref{sec:details-experiments}.

%We repeat all experiments with 10 random seeds.




\begin{figure}[t]
\centering
\begin{minipage}[b]{0.49\textwidth}
\centering
\includegraphics[width=\textwidth, height=5cm]{figures/original_scale/Var_of_l_2_loss.pdf}
\caption{(Synthetic data) Variance of mean squared loss evaluated through the posterior belief $\mu_t$ at each horizon $t$. This is the objective that policy gradient methods like \textsf{REINFORCE} and $\ouralgo$ optimizes. 1-step lookaheads are surprisingly effective even in long horizons.}
\label{fig:var-l2-sim}
\end{minipage}
\hfill
\begin{minipage}[b]{0.49\textwidth}
\centering \includegraphics[width=\textwidth, height=5cm]{figures/original_scale/Error_of_estimated_model_l_2_loss.pdf}
\caption{(Synthetic data) Error between MSE calculated based on collected data $\mc{D}^{0:T}$ vs. population oracle MSE over $\mc{D}_{\rm eval} \sim P_X$. Reducing uncertainty over posteriors directly leads to better OOD evaluations. 1-step lookaheads significantly outperform active learning heuristics in small horizons.}
\label{fig:mean-l2-sim}
\end{minipage}
%\caption{Simulated data for GPs}
%\label{fig:both_plots}
\end{figure}

\subsection{Planning with Gaussian processes}
\label{sec:experiment-plan-GP}
We now briefly describe the data generation process for the GP experiments,  deferring a more detailed discussion of the dataset generation to Section~\ref{sec:details-experiments}. 
We use both the synthetic data and the real data to test our methodology.
For the \emph{simulated data},  we construct a setting where the general population is distributed across \emph{51 non-overlapping clusters} while the initial labeled data $\dtrain$ just comes from one cluster. In contrast, both $\dpool \defeq (\xpool,\ypool),\deval \defeq (\xeval,\yeval)$ are generated   from all the clusters. 
We begin with a low-dimensional scenario, generating a one-dimensional regression setting using a GP. %Gaussian Process (GP).
Although the data-generating process is not known to the algorithms,  we assume that the GP hyperparameters are known to all the algorithms
to ensure fair comparisons. This can be viewed as a setting where our prior is well-specified, allowing us to isolate the effects
of different policy optimization approaches
 without any concerns about the misspecified priors. We select $10$ batches, each of size $K=5$ across $T = 10$ time horizons.

To examine the robustness of our method against the distributional assumptions made  in the simulated case, we then move to a real dataset where the correct prior is not known. We simulate selection bias from the eICU dataset~\citep{PollardJoRaCeMaBa18}, which contains real-world patient data with in-hospital mortality outcomes. 
We conduct a $k$-means clustering to generate 51 clusters and then select data from those clusters. We view this to be a credible replication of practice, as severe distribution shifts are common due to selection bias in clinical labels.  To convert the binary mortality labels into a regression setting, we train a  random forest classifier and fit a GP on predicted scores, which serves as the UQ module for all the algorithms. As before, the task is to select 10 batches, each consisting of 5 samples, across 10 time horizons.

 In Figures~\ref{fig:var-l2-sim} and~\ref{fig:mean-l2-sim}, we present results for the simulated data. 
Figure~\ref{fig:var-l2-sim} shows the variance of $\ell_2$ loss, and Figure~\ref{fig:mean-l2-sim} presents the error in the estimated $\ell_2$ loss using $\mu_t$ (relative to true $\ell_2$ loss, that is unknown to the algorithm). 
As we can see from these plots, our method one-step lookahead  gives substantial improvements  over active learning baselines and random sampling. In addition,
compared to the one-step lookahead planning approach using \textsf{REINFORCE}-based policy gradients, 
we observe that $\mathsf{Smoothed\text{-}Autodiff}$-based policy gradients provide significantly more robust performance over all horizons.

In Figures~\ref{fig:var-l2-real}~and~\ref{fig:mean-l2-real}, we observe similar findings on the eICU data. We see that planning policies (\textsf{REINFORCE} and $\mathsf{Smoothed\text{-}Autodiff}$) consistently outperform other heuristics by a large margin.  Active learning baselines perform poorly in these small-horizon batched problems and can sometimes be even worse than the random search baselines.  Overall, our results show the importance of careful planning in adaptive labeling for reliable model evaluation. 

We offer some intuition as to why one-step lookahead planning may outperform other heuristic algorithms. 
 First,  \textsf{Uncertainty sampling (Static)} while myopically selects the
 top-$K$ inputs with the highest uncertainty, it fails to consider 
the overlap in information content among the ``best” instances; see \citep{AggarwalKoGuHaPh14} for more details. 
In other words,  it might acquire points from the same region with high uncertainty while failing to induce diversity among the batch.
Although \textsf{Uncertainty Sampling (Sequential)} somewhat addresses the issue of information overlap, a significant drawback of 
this algorithm
is the disconnect between the objective we aim to optimize and the algorithm. For example, it might sample from a region with high uncertainty but very low density. 

\begin{figure}[t]
\centering
\begin{minipage}[b]{0.48\textwidth}
\centering
\includegraphics[width=\textwidth, height=5cm]{figures/original_scale/Var_of_l_2_loss_real.pdf}
\caption{(Real-world eICU data) Variance of mean squared loss evaluated through the posterior belief $\mu_t$ at each horizon $t$. Even 1-step lookaheads are extremely effective planners, and auto-differentiation-based pathwise policy gradients provide a reliable optimization algorithm based on low-variance gradient estimates.}
\label{fig:var-l2-real}
\end{minipage}
\hfill
\begin{minipage}[b]{0.48\textwidth}
\centering \includegraphics[width=\textwidth, height=5cm]{figures/original_scale/Error_of_estimated_model_l_2_loss_real.pdf}
\caption{(Real-world eICU data) Error between MSE calculated based on collected data $\mc{D}^{0:T}$ vs. population oracle MSE over $\mc{D}_{\rm eval} \sim P_X$. Reducing uncertainty over posteriors directly leads to better OOD evaluations. Our method significantly outperforms active learning-based heuristics, and random sampling.}
\label{fig:mean-l2-real}
\end{minipage}
%\caption{Real data for GPs}
\end{figure}
 
%\vspace{-1.5cm}
% \begin{wrapfigure}{r}{.32\columnwidth}
%   \vspace{-.5cm} 
%   \centering
% \includegraphics[scale=.29]{figures/Var of l2l_2 loss.pdf}
%   \vspace{-0.2cm}
%   \caption{Results of GP}
% \label{fig:var-l2-gp}
%   \vspace{-0.1cm}
% \end{wrapfigure}


% Attempts have been made  in the past to address these  drawbacks heuristically  (see \citep{AggarwalKoGuHaPh14}). We give a unified computational framework while approaching the problem in a more principled manner and solving it more optimally.




\subsection{Planning with  neural network-based uncertainty quantification methods ($\ensembleplus$)}


We now provide a proof-of-concept that shows the generalizability of our conceptual framework  to the deep learning-based UQ modules, specifically focusing on $\ensembleplus$ due to their previously observed superior performance~\citep{OsbandWenAsDwIbLuRo23}. Recall that implementing our framework with deep learning-based UQ modules  requires us to retrain the model across multiple possible random actions $\bm{a}(\theta)$ sampled from the current policy $\pi_\theta$.
This requires significant computational resources, in sharp contrast to the GPs where the posteriors are in closed form and can be readily updated and differentiated. 

Due to the computational constraints, we test $\ensembleplus$ on a toy setting to demonstrate the generalizability of our framework. We consider a setting where the general population consists of four clusters, while the initial labeled data only comes from one cluster. Again we generate data using GPs.  The task is to select a batch of 2 points in one horizon. We detail the $\ensembleplus$ architecture in Section \ref{sec:details-experiments}, and we assume prior uncertainty to be large (depends on the scaling of the prior generating functions). 
The results are summarized in the Table~\ref{tab:UQ_ensemble}.

% \begin{table}[H]
% \vspace{-10pt}
% \caption{Performance under \ensembleplus as UQ module}
%     \centering
%     \begin{tabular}{|m{3cm}|m{2.5cm}|m{2cm}|} 
%     \hline
%       Algorithm   & Variance of $\loss_2$ loss estimate & Error of $\loss_2$ loss estimate  \\ \hline Random Sampling 
%          & $1710.9 \pm 1352.1$ & $8.67\pm6.62$ 
%       \\ \hline \ouralgo & $1.30 \pm 0.68$ & $0.91\pm0.25$ \\ \hline
%     \end{tabular}
%     \label{tab:UQ_ensemble}
%     %\vspace{-10pt}
% \end{table}




\begin{table}[h]
\vspace{-10pt}
\caption{Performance under \ensembleplus as the UQ module}
\centering
\begin{tabular}{|l|l|l|}
\hline
Algorithm   & Variance of $\loss_2$ loss estimate & Error of $\loss_2$ loss estimate  \\
\hline
\textsf{Random sampling} & 7129.8 $\pm$ 1027.0 & 136.2 $\pm$ 8.28 \\ \hline
\textsf{Uncertainty sampling (Static)} & 10852 $\pm$ 0.0 & 162.156 $\pm$ 0.0 \\ \hline
\textsf{Uncertainty sampling (Sequential)} & 8585.5 $\pm$ 898.9 & 144 $\pm$ 6.93 \\ \hline
\textsf{REINFORCE} & 1697.1 $\pm$ 0.0 & 45.27 $\pm$ 0.0 \\ \hline
\ouralgo & 1697.1 $\pm$ 0.0 & 45.27 $\pm$ 0.0 \\ \hline
\end{tabular}
%\caption{Comparison of different algorithms based on variance   and   error in $\ell_2$ loss estimation with Ensemble $+$ as the UQ module. Our results demonstrate that {\ouralgo} and REINFORCE outperformthe other active learning based heuristics, confirming the benefits of our MDP formulation for the adaptive labeling problem, as also demonstrated in Section 4.\\
%\footnotesize{Experimental details: We use Gaussian Processes as our data generating process, GP parameters are the same as in Section D.3.  The task is to select a batch of 2 points along one horizon.The marginal distribution $p_X$ has 4 \textit{non-overlapping} clusters. Initial data comes from one cluster, while pool and evaluation points comes from all the clusters. We have $20$ initial labeled data points, $10$ pool points, and $252$ evaluation points.  Training procedures are similar to the one in Section D.3.} }
\label{tab:UQ_ensemble}
\end{table}



% We faced  issues in scaling up these experiments which will be our focus in the future. 





% \begin{itemize}
%     \item Posteriors should be consistent. Two dimensions: even with less training,  
%     \item the inference should be  fast enough
% \end{itemize}


% Potential research directions for uncertainty quantification

% In this section we consider a simple setting We consider a simpler setting and 


% For synthetic dataset generation, we use ...... For real datasets, we use ...... We compare our methodolgy to several baselines ()    This Section is structured as follows:
% \begin{itemize}
%     \item \textbf{GPs, square loss objective} (Section \ref{}): 
%     %the broad aim of the experiments  in this section is to isolate the performance of our methodology without any concerns for the inefficiencies induced due to a mis-specified prior or imperfect posterior inference. To accomplish this we generate synthetic datasets using GPs (detailed later). We use the well specified prior (GPs - with same hyperparameter setting) as our UQ module.   
%      As GPs provide differentaible posterior inference - any errors induced due to imperfect posterior updates are also isolated. We note that under this setting
%      \item In Section\ref{} we demonstrate why our methodology performs better than other baselines - by devising various synthetic experiments ()
%     \item  \textbf{UQ Benchmarking }(Section \ref{}): Before diving into the experiments using $\ensembleplus$ and ENNs,  we showcase our benchmarking experiments in Section \ref{}. We use real datasets We observe that ENNs perform better
%      \item \textbf{Ensemble $+$}, objective: recall, accuracy
%     \item \textbf{ENN}, objective: recall, accuracy
% \end{itemize}




% In Section {}, we test 
% \subsection{Experimental details}

% \begin{itemize}
%     \item UQ methodologies - GPs, ENNs
%     \item Objectives - Recall,  ATE
%     \item Datasets - ATE-synthetic datasets, Recall-synthetic, real datasets
%     \item Baselines - 
%     \begin{itemize}
%         \item Random sampling
%         \item Active learning - Uncertainty based sampling - In regression setting almost all of the 
%         \item Myopic greedy - Greedy Batch based sampling
%         \item Policy Gradient
%     \end{itemize}
    
% \end{itemize}

% \subsection{Experiments}
%     \begin{itemize}
%     \item GPs with square loss
%     \item Benchmarking ENN
%         \item ENNs with ATE
%         \item ENNs with Recall
%     \end{itemize}

% \subsection{Benefits over other algorithms - intuition and experiments}

%Active learning - Myopic greedy / Don't rely on the objective rather some entropy version.


%%% Local Variables:
%%% mode: latex
%%% TeX-master: "main"
%%% End:


%\clearpage
% Reference
% For natbib users:
%\bibliography{reference}

\clearpage
{\small 
\bibliography{main}
\bibliographystyle{iclr2025_conference}
}


%%%%%%%%%%%%%%%%%%%%%%%%%%%%%%%%%%%%%%%%%%%%%%%%%%%%%%%%%%%%


\clearpage



\onecolumn
\par\noindent\rule{\textwidth}{1pt}
\begin{center}
{\Large \bf Appendix}
\end{center}
\vspace{-0.1in}
\par\noindent\rule{\textwidth}{1pt}
\appendix

\section{Discussion on Related Work}
\label{sec:discussion}

\paragraph{Implicit Bias of Edge of Stability.} Edge of stability was first coined by \cite{cohen2021gradient}, where they showed that the Hessian of the training loss plateaus around $2/\eta$ when deep models were trained using GD. However, \cite{jastrzebski2020break,jastrzkebski2018relation} previously demonstrated that the step size influences the sharpness along the optimization trajectory. Due to the important practical implications of the edge of stability, there has been an explosion of research dedicated to understanding this phenomenon and its implicit regularization properties. Here, we survey a few of these works.~\cite{damian2023selfstabilization} explained edge of stability through a mechanism called ``self-stabilization'', where they showed that during the momentary divergence of the iterates along the sharpest eigenvector direction of the Hessian, the iterates also move along the negative direction of the gradient of the curvature, which leads to stabilizing the sharpness to $2/\eta$. \cite{agarwala2022second} proved that second-order regression models (the simplest class of models after the linearized NTK model) demonstrate progressive sharpening of the NTK eigenvalue towards a slightly different value
than $2/\eta$.
\cite{arora2022understanding} mathematically analyzed the edge of stability, where they showed that the GD updates evolve along some deterministic flow on the manifold of the minima. 
\cite{lyu2022understanding} showed that the normalization layers had an important role in the edge of stability -- they showed that these layers encouraged GD to reduce the sharpness of the loss surface and enter the EOS regime. \cite{ahn2024learning} established the phenomenon in two-layer networks and find phase transitions for step-sizes in which networks fail to learn ``threshold'' neurons.~\cite{wang2022analyzing} also analyze a two-layer network, but provide a theoretical proof for the change in sharpness across four different phases. \cite{even2024s} analyzed the edge of stability in diagonal linear networks and found that oscillations occur on the sparse support of the vectors. Lastly,~\cite{wu2024implicit} analyzed the convergence at the edge of stability for constant step size GD for logistic regression on linearly separable data. 


\paragraph{Edge of Stability in Toy Functions.}

To analyze the edge of stability in slightly simpler settings, many works have constructed scalar functions to analyze the prevalence of this phenomenon. For example,~\cite{chen2023edge} studied a certain class of scalar functions and identified conditions in which the function enters the edge of stability through a two-step convergence analysis.~\cite{wang2023good} showed that the edge of stability occurs in specific scalar functions, which satisfies certain regularity conditions and developed a global convergence theory for a family of non-convex functions without globally Lipschitz continuous gradients.  \cite{minimal_eos} analyzed local oscillatory behaviors for 4-layer scalar networks with balanced initialization. \cite{song2023trajectory,kalra2023universal} provide analyses of learning dynamics at the EOS in simplified settings such as two-layer networks. \cite{zhu2022quadratic,chen2023stability} study GD dynamics for quadratic models in large learning rate regimes.
Overall, all of these works showed that the necessary condition for the edge of stability to occur is that the second derivative of the loss function is non-zero, even though they assumed simple scalar functions. Our work takes one step further to analyze the prevalence of the edge of stability in DLNs. Although our loss simplifies to a loss in terms of the singular values, they precisely characterize the dynamics of the DLNs for the deep matrix factorization problem.


%Although there exists a rich body of literature analyzing EOS, none of the works are close to the practical setting of deeper networks (either scalar functions or two linear networks are analyzed). In this work, we provide an extensive characterization of EOS in deep linear networks through the deep matrix factorization problem where target matrix is low rank $r$. We theoritically analyze in deep linear models, that GD preserves invariances by aligning the singular vectors of each factor layers and has a balancing effect on the $r$ significant singular values. These implicit biases of GD further allows us to prove the oscillatory phenomenon in EOS for deep linear networks.  Recent work in \cite{zhu2023catapults} \textit{empirically} demonstrated that catapaults in training loss for GD/SGD occur in a low-dimensional subspace spanned by the top eigenvectors of the tangent kernel. Our work theoritically analyzes this phenomenon in deep linear networks trained with GD. 

\paragraph{Deep Linear Networks.} 
Over the past decade, many existing works have analyzed the learning dynamics of DLNs as a surrogate for deep nonlinear networks to study the effects of depth and implicit regularization~\citep{saxe2014exact, arora2018optimization, implicit_dmf,zhang2024structure}. Generally, these works focus on unveiling the dynamics of a phenomenon called ``incremental learning'', where small initialization scales induce a greedy singular value learning approach~\citep{kwon, gissin2020the, saxe2014exact}, analyzing the learning dynamics via gradient flow~\citep{saxe2014exact, CHOU2024101595, implicit_dmf}, or showing that the DLN is biased towards low-rank solution~\citep{yaras2024compressible, implicit_dmf, kwon}, amongst others.
However, these works do not consider the occurence of the edge of stability in such networks. 
On the other hand, while works such as those by~\cite{yaras2024compressible} and~\cite{kwon} have similar observations in that the weight updates occur within an invariant subspace as shown by Proposition~\ref{prop:one_zero_svs_set}, they do not analyze the edge of stability regime.


\section{Additional Results}
\label{sec:additional_exp}

%In Section~\ref{sec:exp_details}, we provide experimental details regarding the experiments in the main text. 

\subsection{Experimental Details}
\label{sec:extra_details}

\paragraph{Bifurcation Plot.}
In this section, we provide additional details regarding the experiments used to generate the figures in the main text. For Figure~\ref{fig:bifurcation}, we consider a rank-3 target matrix $\mbf{M}_\star \in \mbb{R}^{5\times 5}$ with ordered singular values $10, 6, 3$. We use a $3$-layer DLN to fit the target matrix. Since $\sigma_{\star, 1} = 10$, the network enters the EOS regime at
\begin{align*}
    \eta = \frac{2}{L\sigma_{\star, 1}^{2- 2/L}} = 0.0309.
\end{align*}
We show that there exists a two-period orbit after $0.0309 / 2 = 0.0154$, as we do not have a scaling of $1/2$ in the objective function for the code used to generate the figures.

\paragraph{Contour Plots.}
In Figure~\ref{fig:contour}, we considered the toy example 
$$f(\sigma_1, \sigma_2) = \frac{1}{2}(\sigma_2 \cdot \sigma_1 - \sigma_{*})^2,$$
which corresponds to a scalar two-layer network. By Lemma~\ref{lemma:hessian_eigvals}, the stability limit is computed as $\eta = 0.2$, as $L=2$ and $\sigma_{\star} = 5$. To this end, for GD beyond EOS, we use a learning rate of $\eta = 0.2010$, where as we use a learning rate of $\eta = 0.1997$ for GD at EOS. For GF, we plot the conservation flow, and use a learning rate of $\eta = 0.1800$ for stable GD.

\paragraph{DLN and Holder Table Function Plots.}
In Figure~\ref{fig:landscape} and~\ref{fig:figure_grid}, we compared the landscape of DLNs with that of a more complicated non-convex function such as the Holder table function.
To mimic the DLN, we considered the loss function
\begin{align}
\label{eqn:2d_example}
    z = L(x, y) = (x^{4}-0.8)^2 + (y^{4}-1)^2,
\end{align}
which corresponds to a 4-layer network.
Here the eigenvector of the Hessian at the global minima coincides with the $x, y$-axis. We calculate the eigenvalues $\lambda_{1}$ and $\lambda_{2}$ at the minimum $(0.8^{0.25},1)$ 
and plot the dynamics of the iterates for step size range $\frac{2}{\lambda_{2}}> \eta >  \frac{2}{\lambda_{1}}$ and $\eta >  \frac{2}{\lambda_{2}}$. When $\frac{2}{\lambda_{2}}> \eta >  \frac{2}{\lambda_{1}}$ the $x$-coordinate stays fixed at the minima $0.8^{0.25}$ and the $y$-coordinate oscillates around its minimum at $y=1$. This is evident in the landscape figure. Similarly, when $\eta >  \frac{2}{\lambda_{2}}$, oscillations occur in both the $x$ and $y$ direction. The loss landscape $z =L(x,y)$ does not have spurious local minima, so sustained oscillations take place in the loss basin. 

\begin{figure}[t!]
    \centering
    % First Row
    \caption*{Oscillation along Y-axis: $2/\lambda_2>\eta > 2/\lambda_1$}
    \begin{subfigure}{0.245\textwidth}
        \centering
        \includegraphics[width=\linewidth]{figures/linear_layers/007.png}
        %\caption{Loss Landscape}
    \end{subfigure}
    \hfill
    \begin{subfigure}{0.37\textwidth}
        \centering
        \includegraphics[width=\linewidth]{figures/sharpness_plot_dln_gd_h00.1_post10_lr0.075.png}
        %\caption{Sharpness}
    \end{subfigure}
    \hfill
    \begin{subfigure}{0.37\textwidth}
        \centering
        \includegraphics[width=\linewidth]{figures/x_trajectory_plot_dln_gd_h00.1_post10_lr0.075.png}
        %\caption{Oscillatory Components}
    \end{subfigure}
    
    \vspace{0.3cm} % Space between rows
    
    % Second Row
    \caption*{Oscillation along both X and Y-axis: $\eta > 2/\lambda_2$}
    \begin{subfigure}{0.25\textwidth}
        \centering
        \includegraphics[width=\linewidth]{figures/linear_layers/008.png}
        \caption*{Loss Landscape}
    \end{subfigure}
    \hfill
    \begin{subfigure}{0.3675\textwidth}
        \centering
        \includegraphics[width=\linewidth]{figures/linear_layers/sharpness_plot_dln_gd_h00.1_post10_lr0.088.png}
        \caption*{Sharpness}
    \end{subfigure}
    \hfill
    \begin{subfigure}{0.3675\textwidth}
        \centering
        \includegraphics[width=\linewidth]{figures/linear_layers/x_trajectory_plot_dln_gd_h00.1_post10_lr0.088.png}
        \caption*{Oscillatory Components}
    \end{subfigure}
    
    \caption{Demonstration of the EOS dynamics of a 2-dimensional depth-4 scalar network as shown in Equation~(\ref{eqn:2d_example}). $X, Y$ axes are the eigenvectors of the Hessian with eigenvalues $\lambda_{1}$ and $\lambda_{2}$ respectively. Top: when $\eta > 2/\lambda_1$, the $X$ component remains fixed, while the $Y$ component oscillates with a  periodicity of 2. Bottom: for $\eta > 2/\lambda_2$, the iterates oscillation in both directions.}
    \label{fig:figure_grid}
\end{figure}


\begin{figure}[h!]
    \centering
    \begin{subfigure}[t!]{0.45\textwidth}
        \centering
        \includegraphics[width=\linewidth]{figures/new_lr_cycle.png}
        
    \end{subfigure}%
    \hfill
    \begin{subfigure}[t!]{0.45\textwidth}
        \centering
        \includegraphics[width=\linewidth]{figures/holder1.png}
        %\caption{Iterates catapults out of a local basin whenever $\eta$ is increased and jumps out to surface where sharpness is just about $\frac{2}{\eta}$.}
    \end{subfigure}
    
    \caption{EOS dynamics at various step learning rates from the Holder table function. Left: plot of the learning rate steps and sharpness, showing that 
        sharpness follows the EOS limit $2/\eta$. Right: Plot showing that the iterates catapult out of a local basin when the learning rate is increased and jumps out to a surface where the sharpness is about $2/\eta$.}
    \label{fig:two_figures}
\end{figure}




For the non-convex landscape as shown in Figure~\ref{fig:landscape} and~\ref{fig:two_figures}, we consider the Holder table function: 
\begin{align*}
    f(x, y) = - \left| \sin(x) \cos(y) \exp \left( 1 - \frac{\sqrt{x^2 + y^2}}{\pi} \right) \right|.
\end{align*}
By observation, we initialize near a sharp minima and run GD with an increasing learning rate step size as shown in the lefthand side of Figure~\ref{fig:two_figures}.
When the learning rate is fixed, we observe that oscillations take place inside the local valley, but when learning rate is increased, it jumps out of the local valley to find a flatter basin. Similar to the observations by \cite{cohen2021gradient}, the sharpness of the GD iterates are ``regulated'' by the threshold $2/\eta$, as it seems to closely follow this value as shown in Figure~\ref{fig:two_figures}.

Overall, these examples aim to highlight the difference in linear and complex loss landscapes. The former consists of \emph{only} saddles and global minima, and hence (stably) oscillate about the global minimum. However, in more complicated non-convex landscapes, sharpness regularization due to large learning rates enable catapulting to flatter loss basins, where sharpness is smaller than $2/\eta$.



\subsection{Initialization Outside Singular Vector Invariant Set}


In this section, we present an initialization example that is outside the singular vector stationary set. We consider the following initialization:
\begin{align}
    \mbf{W}_L(0) = \mbf{0}, \quad \quad\quad \mbf{W}_\ell(0) = \alpha \mbf{P}_\ell, \quad \forall \ell \in [L-1],
\end{align}
where $\mbf{P}_\ell \in \mbb{R}^{d\times d}$ is an orthogonal matrix. Note that here for $\ell>1$, the singular vectors do not align and lies outside the SVS set we defined in Proposition~\ref{prop:one_zero_svs_set}. 
We consider the deep matrix factorization problem with a target matrix $\mbf{M}_\star \in \mbb{R}^{d\times d}$, where $d=100$, $r=5$, and $\alpha = 0.01$. We empirically obtain that the decomposition after convergence admits the form:
    \begin{align}
        \mbf{W}_L(t) &= 
        \mbf{U}^\star
        \begin{bmatrix}
            \mbf{\Sigma}_L(t) & \mbf{0} \\
            \mbf{0} & \mbf{0}
        \end{bmatrix} \left[\left(\prod_{i=L-1}^1{\mbf{P}_{i}}\right)\mbf{V^\star}\right]^{\top}, \\
        \mbf{W}_{\ell}(t) &= \left[\left(\prod_{i=\ell}^1{\mbf{P}_{i}}\right)\mbf{V^\star}\right]
        \begin{bmatrix}
            \mbf{\Sigma}_{\ell}(t) & \mbf{0} \\
            \mbf{0} & \alpha\mbf{I}_{d-r}
        \end{bmatrix} \left[\left(\prod_{i=\ell-1}^1{\mbf{P}_{i}}\right)\mbf{V^\star}\right]^{\top},
        \quad \forall \ell \in [2, L-1], \\
        \mbf{W}_{1}(t) &= \mbf{P}_{1}\mbf{V}^{\star} \begin{bmatrix}
            \mbf{\Sigma}_{1}(t) & \mbf{0} \\
            \mbf{0} & \alpha\mbf{I}_{d-r}
        \end{bmatrix} \mbf{V}^{\star\top},
    \end{align} 
    where  
    $\mbf{W}_L(0) = \mbf{0}$ and $\mbf{W}_{\ell}(0) = \alpha \mbf{P}_{l}$, $\forall\ell \in [L-1]$.
    The decomposition after convergence lies in the SVS set as the singular vectors now align with each other. This demonstrates an example where even when the initialization is made outside the SVS set, GD aligns the singular vectors such that after certain iterations it lies in the SVS set.
    


\begin{figure}[h!]
    \centering
     \begin{subfigure}[b]{0.495\textwidth}
         \centering
        \includegraphics[width=\textwidth]{figures/left_svec_error.png}
         \caption*{Left Singular Vectors}
     \end{subfigure}
         \begin{subfigure}[b]{0.495\textwidth}
         \centering
        \includegraphics[width=\textwidth]{figures/right_svec_error.png}
         \caption*{Right Singular Vectors}
     \end{subfigure}
    \caption{Empirical verification of the decomposition for initialization with orthogonal matrices (lying outside SVS set) in that after some GD iterations, the singular vectors of the intermediate matrices align to lie within SVS set, displaying singular vector invariance.}
    \label{fig:verify_conj}
\end{figure}



\begin{figure}[h!]
    \centering
     \begin{subfigure}[b]{0.315\textwidth}
         \centering
        \includegraphics[width=\textwidth]{figures/eos_balance_001.png}
         \caption*{$\alpha = 0.01$}
     \end{subfigure}
     \begin{subfigure}[b]{0.315\textwidth}
         \centering
        \includegraphics[width=\textwidth]{figures/eos_balance_010.png}
         \caption*{$\alpha = 0.10$}
     \end{subfigure}
         \begin{subfigure}[b]{0.315\textwidth}
         \centering
        \includegraphics[width=\textwidth]{figures/eos_balance_030.png}
         \caption*{$\alpha = 0.30$}
     \end{subfigure}
    \caption{Observing the balancedness between the singular value initialized to $0$ and a singular value initialized to $\alpha$. The scattered points are successive GD iterations (going left to right). The initial gap between the two values is larger for a larger $\alpha$, but quickly gets closer over more GD iterations.}
    \label{fig:assumption}
\end{figure}

\begin{figure}[t!]
    \centering
     \begin{subfigure}[t!]{\textwidth}
         \centering
        \includegraphics[width=0.95\textwidth]{figures/eos_init_0_01_lr3.pdf}
     \end{subfigure}
          \newline
     \centering
     \begin{subfigure}[t!]{\textwidth}
         \centering
         \includegraphics[width=0.95\textwidth]{figures/eos_init_0_01_lr1.pdf}
         
     \end{subfigure}
     \newline
     \centering
     \begin{subfigure}[t!]{\textwidth}
         \centering
         \includegraphics[width=0.95\textwidth]{figures/eos_init_0_01_lr2.pdf}
         
     \end{subfigure}
    \caption{Plots of the training loss, singular value magnitude, and the balancing gap over iterations for different learning rates: $\eta = 0.030, 0.032, 0.034$ (top to bottom). When the learning rate is stable ($\eta < 0.031$ since the top singular value is $\sigma_{\star, 1} = 10$), the balancing gap plateaus, whereas the balancing gap goes strictly to zero when the oscillations occur. }
    \label{fig:oscillations_period}
\end{figure}



\subsection{Additional Experiments for Balancing, Singular Vector Invariance, and Theory}
\label{sec:extra_balance_svs}

Our theory relied on two tools and assumptions: balancing of singular values and stationarity of the singular vectors. In this section, we investigate how the dynamics at EOS are affected if these two assumptions do not hold.

\paragraph{Balancing.}
First, we present additional experimental results on Proposition~\ref{prop:balancing} and how close the iterates become for different initialization scales. To this end, we consider the same setup from the previous section, where we have a target matrix $\mbf{M}_\star \in \mbb{R}^{d\times d}$, where $d=100$, $r=5$, and varying initialization $\alpha$.  In Figure~\ref{fig:assumption}, we observe that for larger values of $\alpha$, the balancing quickly occurs, whereas for smaller values of $\alpha$, the balancing is almost immediate. This is to also highlight that our bound on $\alpha$ in Proposition~\ref{prop:balancing} may be an artifact of our analysis, and can choose larger values of $\alpha$ in practice.

To this end, we also investigate how large $\alpha$ can be until Proposition~\ref{prop:balancing} no longer holds. We consider the dynamics of a $3$-layer DLN to fit a target matrix $\mbf{M}_\star \in \mbb{R}^{10 \times 10}$ of rank-3 with ordered singular values $10, 8, 6$. We use a learning rate of $\eta = 0.0166$, which corresponds to oscillations in the top-2 singular values. In Figure~\ref{fig:no_balance_hold}, we show the dynamics of when the initialization scale is $\alpha = 0.01$ and $\alpha = 0.5$, where balancing holds theoretically for the former but not for the latter. Clearly, we observe that balancing does not hold for $\alpha = 0.5$. However, examining the middle plots reveals that the oscillations in the singular values still have the same amplitude in both cases and for both singular values. 


\paragraph{Singular Vector Stationarity.} 


Throughout this paper, we considered two initializations in Equations~(\ref{eqn:balanced_init}) and~(\ref{eqn:unbalanced_init}), where balancing holds immediately and one where balancing holds for a sufficiently small initialization scale. In this section, we investigate different initializations with aim to observe (i) if they do not converge to the SVS set and (ii) how they affect the oscillations if they do not belong to the SVS set. To this end, we consider the following:
\begin{align}
    &\mbf{W}_L(0) = \mbf{0},  \quad \mbf{W}_\ell(0) = \alpha \mbf{I}_d, \quad \forall \ell \in [L-1],\tag{Original} \\
    &\mbf{W}_L(0) = \mbf{0},  \quad \mbf{W}_\ell(0) = \alpha \mbf{P}_\ell, \quad \forall \ell \in [L-1],\tag{Orthogonal} \\
    &\mbf{W}_L(0) = \mbf{0},  \quad \mbf{W}_\ell(0) = \alpha \mbf{H}_\ell, \quad \forall \ell \in [L-1],\tag{Random}
\end{align}
where $\mbf{P}_\ell$ is an orthogonal matrix and $\mbf{H}_\ell$ is a random matrix with Gaussian entries. 
For all of these initialization schemes, we consider the same setup as in the balancing case, with an initialization scale of $\alpha = 0.01$. To observe if singular vector stationarity holds, we consider the subspace distance as follows:
\begin{align}
\label{eqn:subs_dist}
    \mathrm{Subspace \,\, Distance} = \|\mbf{U}_{\ell-1, r}^\top \mbf{V}_{\ell, r} - \mbf{I}_r\|_{\mathsf{F}},
\end{align}
where $\mbf{U}_{\ell,r}$ and $\mbf{V}_{\ell, r}$ are the top-$r$ left and right singular vectors of layer $\mbf{W}_\ell$, respectively. Since Proposition~\ref{prop:svs_set} implies that the intermediate singular vectors cancel, the initialization converges to the SVS set if the subspace distance goes to zero. 
In Figure~\ref{fig:svs_set_test}, we plot the dynamics for all of the initializations. Generally, we observe that the subspace distance for all cases go to zero, validating the use of the SVS set for analysis purposes.

\begin{figure}[t!]
    \centering
     \begin{subfigure}[t!]{\textwidth}
         \centering
        \includegraphics[width=\textwidth]{figures/eos_init_0_01.pdf}
     \end{subfigure}
     \newline
     \centering
     \begin{subfigure}[t!]{\textwidth}
         \centering
         \includegraphics[width=\textwidth]{figures/eos_init_0_9.pdf}
         
     \end{subfigure}
    \caption{Top: EOS dynamics of a 3-layer DLN with initialization scale $\alpha=0.01$, where balancing theoretically holds. Bottom: EOS dynamics of the DLN with initialization scale $\alpha = 0.5$. While the balancing does not hold for $\alpha=0.5$, the oscillations in the singular values are still prevalent, with the same amplitude.}
\label{fig:no_balance_hold}
\end{figure}



\begin{figure}[t!]
    \centering
     \begin{subfigure}[t!]{\textwidth}
         \centering
        \includegraphics[width=0.75\textwidth]{figures/eos_init_eye.pdf}
     \end{subfigure}
          \newline
     \centering
     \begin{subfigure}[t!]{\textwidth}
         \centering
         \includegraphics[width=0.75\textwidth]{figures/eos_init_ortho.pdf}
         
     \end{subfigure}
     \newline
     \centering
     \begin{subfigure}[t!]{\textwidth}
         \centering
         \includegraphics[width=0.75\textwidth]{figures/eos_init_rand.pdf}
         
     \end{subfigure}
    \caption{EOS dynamics of a 3-layer DLN for different initializations where it all converges to the SVS set. The subspace distance is defined in Equation~(\ref{eqn:subs_dist}). Top: Dynamics with the original identity initialization. Middle: Dynamics with orthogonal initialization. Bottom: Dynamics with random initialization.}
    \label{fig:svs_set_test}
\end{figure}



\paragraph{Additional Results.}

In this section, we provide more experimental results to corroborate our theory. Recall that in Lemma~\ref{lemma:hessian_eigvals}, we proved that the learning rate needed to enter the EOS is a function of the depth, and that deeper networks can enter EOS using a smaller learning rate. To verify this claim, we provide an additional experiment where the target matrix is $\mbf{M}_\star \in \mbb{R}^{5\times 5}$ with the top singular value set to $\sigma_{\star, 1} = 0.5$. We use an initialization scale of $\alpha = 0.01$. In Figure~\ref{fig:depth_lr}, we can clearly see that shallower networks need a larger learning rate, and vice versa to enter EOS. Here, black refers to stable learning and white refers to regions in which oscillations occur (EOS regime).

\begin{figure}[h!]
    \centering
    \includegraphics[width=0.5\linewidth]{figures/depth_vs_lr.pdf}
    \caption{Demonstrating that deeper networks requires a smaller learning rate to enter the EOS regime for DLNs, as implied by Proposition~\ref{prop:balancing}, for a target matrix with top singular value $\sigma_{\star,1} = 0.5$ and initialization $\alpha = 0.01$. Black refers to stable learning and white refers to regions in which oscillations in the loss and singular values occur. The EOS limit exactly matches $\eta = 2/L \sigma^{2-\frac{2}{L}}_{\star,i} $.}
    \label{fig:depth_lr}
\end{figure}





\subsection{Periodic and Free Oscillations}

In this section, we present additional experiments on oscillation and catapults in both deep linear and nonlinear networks to supplement the results in the main paper. First, we consider a 3-layer MLP without bias terms for the weights, with each hidden layer consisting of 1000 units. The network is trained using MSE loss with a learning rate of $\eta = 4$, along with random weights scaled by $\alpha = 0.01$ and full-batch gradient descent on a 5K subset of the MNIST dataset, following~\cite{cohen2021gradient}. The motivation for omitting bias terms comes from the findings of~\cite{zhang2024when}, where they provably show that a ReLU network without bias terms behaves similarly to a linear network. With this in mind, we aimed to investigate how oscillations manifest in comparison to deep linear networks (DLNs). In Figure~\ref{fig:mlp_bias_free}, we plot the training loss, top-5 singular values, and sharpness throughout training. Interestingly, despite the non-convexity of the loss landscape, the oscillations appear to be almost periodic across all three plots. It would be of great interest to theoretically study the behavior of EOS for this network architecture and determine whether our analyses extend to this case as well.




\begin{figure}[h!]
    \centering
    \includegraphics[width=\textwidth]{figures/nonlinear/fig_relu_mnist.png}
    \caption{Plot of the training loss, singular values, and sharpness for an MLP network with no bias. Similar to the DLN case, there are oscillations in each of the plots throughout iterations.}
    \label{fig:mlp_bias_free}
\end{figure}

Next, we consider the DLN setting to corroborate our result from Theorem~\ref{thm:align_thm}. We consider modeling rank-3 target matrix with singular values $\sigma_{\star, i} = \{10, 9, 8\}$ with a 3-layer DLN with initialization scale $\alpha = 0.1$. By computing the sharpness under these settings, notice that $2 / \lambda_1 = L\sigma_{\star, 1}^{2 - \frac{2}{L}} \approx 0.01547$ and $2/\lambda_2 \approx 0.01657$. In Figure~\ref{fig:progressive_eta_dln}, we use learning rates near these values, and plot the oscillations in the singular values. Here, we can see that the oscillations follow exactly our theory. 








Lastly, we provide additional experiments demonstrating stronger oscillation in feature directions as measured by the singular values. To this end, we consider a 4-layer MLP with ReLU activations with hidden layer size in each unit of 200 for classification on a subsampled 20K set on MNIST and CIFAR-10. In Figure~\ref{fig:non-lin}, we show that the oscillations in the training loss are artifacts of jumps only in the top singular values, which is also what we observe in the DLN setting.


\begin{figure}[h!]
    \centering
    \begin{subfigure}[b]{0.485\textwidth}
        \centering
        \includegraphics[width=\textwidth]{figures/nonlinear/complete_fig_data_mnist-20k_lr_0.8_arch_fc-relu-depth4.png}
        \caption*{MNIST Dataset with 4-Layer MLP}
        \label{fig:first_image}
    \end{subfigure}
    \hfill
    \begin{subfigure}[b]{0.485\textwidth}
        \centering
        \includegraphics[width=\textwidth]{figures/nonlinear/complete_fig_data_cifar10-20k_lr_0.8_arch_fc-relu-depth4.png}
        \caption*{CIFAR-10 Dataset with 4-Layer MLP}
        \label{fig:second_image}
    \end{subfigure}
    \caption{Prevalence of oscillatory behaviors in top subspaces in $4$-layer networks with ReLU activations on two different datasets.}
    \label{fig:non-lin}
\end{figure}



\begin{figure}[ht]
    \centering
    %\captionsetup{justification=centering}
    % Specify the filename for each figure
    \begin{subfigure}[b]{0.24\textwidth}
        \centering
        \includegraphics[width=\linewidth]{figures/layerwise_oscs/train_loss_lr_030.pdf}
        \caption*{\footnotesize  Train Loss ($\eta = 0.0300$) }
        \label{fig:1}
    \end{subfigure}\hfill
    \begin{subfigure}[b]{0.24\textwidth}
        \centering
        \includegraphics[width=\linewidth]{figures/layerwise_oscs/layer1_lr_030.pdf}
        \caption*{\footnotesize   Layer 1 $\sigma_i$ ($\eta = 0.0300$)}
        \label{fig:2}
    \end{subfigure}\hfill
    \begin{subfigure}[b]{0.24\textwidth}
        \centering
        \includegraphics[width=\linewidth]{figures/layerwise_oscs/layer2_lr_030.pdf}
        \caption*{\footnotesize  Layer 2 $\sigma_i$ ($\eta = 0.0300$)}
        \label{fig:3}
    \end{subfigure}\hfill
    \begin{subfigure}[b]{0.24\textwidth}
        \centering
        \includegraphics[width=\linewidth]{figures/layerwise_oscs/layer3_lr_030.pdf}
        \caption*{\footnotesize  Layer 3 $\sigma_i$ ($\eta = 0.0300$)}
        \label{fig:4}
    \end{subfigure}

    \par\bigskip % Adds space between the rows
    \begin{subfigure}[b]{0.24\textwidth}
        \centering
        \includegraphics[width=\linewidth]{figures/layerwise_oscs/train_loss_lr_031.pdf}
        \caption*{\footnotesize  Train Loss ($\eta = 0.031$) }
        \label{fig:1}
    \end{subfigure}\hfill
    \begin{subfigure}[b]{0.24\textwidth}
        \centering
        \includegraphics[width=\linewidth]{figures/layerwise_oscs/layer1_lr_031.pdf}
        \caption*{\footnotesize  Layer 1 $\sigma_i$ ($\eta = 0.0310$)}
        \label{fig:2}
    \end{subfigure}\hfill
    \begin{subfigure}[b]{0.24\textwidth}
        \centering
        \includegraphics[width=\linewidth]{figures/layerwise_oscs/layer2_lr_031.pdf}
        \caption*{\footnotesize  Layer 2 $\sigma_i$ ($\eta = 0.0310$)}
        \label{fig:3}
    \end{subfigure}\hfill
    \begin{subfigure}[b]{0.24\textwidth}
        \centering
        \includegraphics[width=\linewidth]{figures/layerwise_oscs/layer3_lr_031.pdf}
        \caption*{\footnotesize  Layer 3 $\sigma_i$ ($\eta = 0.0310$)}
        \label{fig:4}
    \end{subfigure}
        \par\bigskip % Adds space between the rows
        
    \begin{subfigure}[b]{0.24\textwidth}
        \centering
        \includegraphics[width=\linewidth]{figures/layerwise_oscs/train_loss_lr_0325.pdf}
        \caption*{\footnotesize  Train Loss ($\eta = 0.0325$) }
        \label{fig:1}
    \end{subfigure}\hfill
    \begin{subfigure}[b]{0.24\textwidth}
        \centering
        \includegraphics[width=\linewidth]{figures/layerwise_oscs/layer1_lr_0325.pdf}
        \caption*{\footnotesize  Layer 1 $\sigma_i$ ($\eta = 0.0325$)}
        \label{fig:2}
    \end{subfigure}\hfill
    \begin{subfigure}[b]{0.24\textwidth}
        \centering
        \includegraphics[width=\linewidth]{figures/layerwise_oscs/layer2_lr_0325.pdf}
        \caption*{\footnotesize  Layer 2 $\sigma_i$ ($\eta = 0.0325$)}
        \label{fig:3}
    \end{subfigure}\hfill
    \begin{subfigure}[b]{0.24\textwidth}
        \centering
        \includegraphics[width=\linewidth]{figures/layerwise_oscs/layer3_lr_0325.pdf}
        \caption*{\footnotesize  Layer 3 $\sigma_i$ ($\eta = 0.0325$)}
        \label{fig:4}
    \end{subfigure}


        \par\bigskip % Adds space between the rows
    \begin{subfigure}[b]{0.24\textwidth}
        \centering
        \includegraphics[width=\linewidth]{figures/layerwise_oscs/train_loss_lr_0335.pdf}
        \caption*{\footnotesize  Train Loss ($\eta = 0.0335$) }
        \label{fig:1}
    \end{subfigure}\hfill
    \begin{subfigure}[b]{0.24\textwidth}
        \centering
        \includegraphics[width=\linewidth]{figures/layerwise_oscs/layer1_lr_0335.pdf}
        \caption*{\footnotesize  Layer 1 $\sigma_i$ ($\eta = 0.0335$)}
        \label{fig:2}
    \end{subfigure}\hfill
    \begin{subfigure}[b]{0.24\textwidth}
        \centering
        \includegraphics[width=\linewidth]{figures/layerwise_oscs/layer2_lr_0335.pdf}
        \caption*{\footnotesize  Layer 2 $\sigma_i$ ($\eta = 0.0335$)}
        \label{fig:3}
    \end{subfigure}\hfill
    \begin{subfigure}[b]{0.24\textwidth}
        \centering
        \includegraphics[width=\linewidth]{figures/layerwise_oscs/layer3_lr_0335.pdf}
        \caption*{\footnotesize  Layer 3 $\sigma_i$ ($\eta = 0.0335$)}
        \label{fig:4}
    \end{subfigure}

            \par\bigskip % Adds space between the rows
    \begin{subfigure}[b]{0.24\textwidth}
        \centering
        \includegraphics[width=\linewidth]{figures/layerwise_oscs/train_loss_lr_035.pdf}
        \caption*{\footnotesize  Train Loss ($\eta = 0.0350$) }
        \label{fig:1}
    \end{subfigure}\hfill
    \begin{subfigure}[b]{0.24\textwidth}
        \centering
        \includegraphics[width=\linewidth]{figures/layerwise_oscs/layer1_lr_035.pdf}
        \caption*{\footnotesize  Layer 1 $\sigma_i$ ($\eta = 0.0350$)}
        \label{fig:2}
    \end{subfigure}\hfill
    \begin{subfigure}[b]{0.24\textwidth}
        \centering
        \includegraphics[width=\linewidth]{figures/layerwise_oscs/layer2_lr_035.pdf}
        \caption*{\footnotesize  Layer 2 $\sigma_i$ ($\eta = 0.0350$)}
        \label{fig:3}
    \end{subfigure}\hfill
    \begin{subfigure}[b]{0.24\textwidth}
        \centering
        \includegraphics[width=\linewidth]{figures/layerwise_oscs/layer3_lr_035.pdf}
        \caption*{\footnotesize  Layer 3 $\sigma_i$ ($\eta = 0.0350$)}
        \label{fig:4}
    \end{subfigure}

            \par\bigskip % Adds space between the rows
    \begin{subfigure}[b]{0.24\textwidth}
        \centering
        \includegraphics[width=\linewidth]{figures/layerwise_oscs/train_loss_lr_036.pdf}
        \caption*{\footnotesize  Train Loss ($\eta = 0.0360$) }
        \label{fig:1}
    \end{subfigure}\hfill
    \begin{subfigure}[b]{0.24\textwidth}
        \centering
        \includegraphics[width=\linewidth]{figures/layerwise_oscs/layer1_lr_036.pdf}
        \caption*{\footnotesize  Layer 1 $\sigma_i$ ($\eta = 0.0360$)}
        \label{fig:2}
    \end{subfigure}\hfill
    \begin{subfigure}[b]{0.24\textwidth}
        \centering
        \includegraphics[width=\linewidth]{figures/layerwise_oscs/layer2_lr_036.pdf}
        \caption*{\footnotesize  Layer 2 $\sigma_i$ ($\eta = 0.0360$)}
        \label{fig:3}
    \end{subfigure}\hfill
    \begin{subfigure}[b]{0.24\textwidth}
        \centering
        \includegraphics[width=\linewidth]{figures/layerwise_oscs/layer3_lr_036.pdf}
        \caption*{\footnotesize  Layer 3 $\sigma_i$ ($\eta = 0.0360$)}
        \label{fig:4}
    \end{subfigure}
    
    \caption{Depiction of the training loss and the singular values of each weight matrix for fitting a rank-3 matrix with singular values $10, 9.5, 9$. The weights enter the EOS regime based on the learning rate $\eta > 2/K$, where $K = L\sigma_{\star, i}^{2-2/L}$ and $L=3$. For a sufficiently large learning rate (e.g., $\eta = 0.04$), the singular values start to enter a period-4 orbit. 
    %When $\eta=2/L\sigma^{2-\frac{2}{L}}_{\star,1} \approx 0.0154$, oscillation occur on the first singular value. When $\eta=2/\sum_{\ell=0}^{L-1} \left(\sigma_{\star, 1}^{1-\frac{1}{L} - \frac{1}{L}\ell} \cdot \sigma_{\star, 2}^{\frac{1}{L}\ell}\right)^2 \approx 0.0165$, oscillation occur on second singular value and so on.
    }
    \label{fig:progressive_eta_dln}
\end{figure}


\section{Deferred Proofs}
\label{sec:proofs}

In this section, we present the deferred proofs from the main manuscript.


\subsection{Proofs for Singular Vector Stationarity}


\subsubsection{Proof of Proposition~\ref{prop:svs_set}}


\begin{proof}

Let us consider the dynamics of $\mbf{W}_\ell(t)$ in terms of its SVD with respect to time:
\begin{align}
\label{eqn:svd_rynamics}
    \dot{\mbf{W}}_\ell(t) &= \dot{\mbf{U}}_\ell(t) \mbf{\Sigma}_\ell(t) \mbf{V}_\ell^\top(t) + \mbf{U}_\ell(t) \dot{\mbf{\Sigma}}_\ell(t) \mbf{V}_\ell^\top(t) + \mbf{U}_\ell(t) \mbf{\Sigma}_\ell(t) \dot{\mbf{V}}_\ell^\top(t).
\end{align}
By left multiplying by \(\mbf{U}_\ell^\top(t)\) and right multiplying by \(\mbf{V}_\ell(t)\), we have
\begin{align}
    \mbf{U}_\ell^\top(t) \dot{\mbf{W}}_\ell(t) \mbf{V}_\ell(t) &= \mbf{U}_\ell^\top(t) \dot{\mbf{U}}_\ell(t) \mbf{\Sigma}_\ell(t) + \dot{\mbf{\Sigma}}_\ell(t) + \mbf{\Sigma}_\ell(t) \dot{\mbf{V}}_\ell^\top(t) \mbf{V}_\ell(t), 
\end{align}
where we used the fact that \(\mbf{U}_\ell(t)\) and \(\mbf{V}_\ell(t)\) have orthonormal columns. Now, note that we also have
\begin{align*}
    \mbf{U}_\ell^\top(t) \mbf{U}_\ell(t) = \mbf{I}_r \implies \dot{\mbf{U}}_\ell^\top(t) \mbf{U}_\ell(t) + \mbf{U}_\ell^\top(t) \dot{\mbf{U}}_\ell(t) = \mbf{0},
\end{align*}
which also holds for $\mbf{V}_\ell(t)$. This implies that $\dot{\mbf{U}}_\ell^\top(t) \mbf{U}_\ell(t)$ is a skew-symmetric matrix, and hence have zero diagonals. 
Since \(\mbf{\Sigma}_\ell(t)\) is diagonal, \(\mbf{U}_\ell^\top(t) \dot{\mbf{U}}_\ell(t) \mbf{\Sigma}_\ell(t)\) and \(\mbf{\Sigma}_\ell(t) \dot{\mbf{V}}_\ell^\top(t) \mbf{V}_\ell(t)\) have zero diagonals as well. On the other hand, since \(\dot{\mbf{\Sigma}}_\ell(t)\) is a diagonal matrix, we can write
\begin{align}
\label{eqn:diag_inv}
    \hat{\mbf{I}}_r \odot \left(\mbf{U}_\ell^\top(t) \dot{\mbf{W}}_\ell(t) \mbf{V}_\ell(t)\right) &= \mbf{U}_\ell^\top(t) \dot{\mbf{U}}_\ell(t) \mbf{\Sigma}_\ell(t) + \mbf{\Sigma}_\ell(t) \dot{\mbf{V}}_\ell^\top(t) \mbf{V}_\ell(t), 
\end{align}
where \(\odot\) stands for the Hadamard product and \(\hat{\mbf{I}}_r\) is a square matrix holding zeros on its diagonal and ones elsewhere. Taking transpose of Equation~(\ref{eqn:diag_inv}), while recalling that \(\mbf{U}_\ell^\top(t) \dot{\mbf{U}}_\ell(t)\) and \(\mbf{V}_\ell^\top(t) \dot{\mbf{V}}_\ell(t)\) are skew-symmetric, we have
\begin{align}
\label{eqn:diag_inv_transpose}
    \hat{\mbf{I}}_{r} \odot \left(\mbf{V}_\ell^\top(t) \dot{\mbf{W}}_\ell^\top(t) \mbf{U}_\ell(t)\right) &= -\mbf{\Sigma}_\ell(t) \mbf{U}_\ell^\top(t) \dot{\mbf{U}}_\ell(t) - \dot{\mbf{V}}_\ell^\top(t) \mbf{V}_\ell(t) \mbf{\Sigma}_\ell(t). 
\end{align}
Then, by right multiplying Equation~(\ref{eqn:diag_inv}) by \(\mbf{\Sigma}_\ell(t)\), left-multiply Equation~(\ref{eqn:diag_inv_transpose}) by \(\mbf{\Sigma}_\ell(t)\), and by adding the two terms, we get
\begin{align*}
    \hat{\mbf{I}}_{r} \odot \biggl(\mbf{U}_\ell^\top(t) \dot{\mbf{W}}_\ell(t) \mbf{V}_\ell(t) \mbf{\Sigma}_\ell(t) + \mbf{\Sigma}_\ell(t) \mbf{V}_\ell^\top(t) &\dot{\mbf{W}}_\ell^\top(t) \mbf{U}_\ell(t)\biggr) \\
    &= \mbf{U}_\ell^\top(t) \dot{\mbf{U}}_\ell(t) \mbf{\Sigma}_\ell^2(t) - \mbf{\Sigma}_\ell^2(t) \dot{\mbf{V}}_\ell^\top(t) \mbf{V}_\ell(t). 
\end{align*}
Since we assume that the singular values of $\mbf{M}_\star$ are distinct, the top-$r$ diagonal elements of \(\mbf{\Sigma}_{\ell}^2(t)\) are also distinct (i.e., $\Sigma^2_{r}(t) \neq \Sigma^2_{r'}(t) \text{ for } r \neq r'$). This implies that
\begin{align*}
    \mbf{U}_{\ell}^\top(t) \dot{\mbf{U}}_{\ell}(t) &= \mbf{H}(t) \odot \left[\mbf{U}_{\ell}^\top(t) \dot{\mbf{W}}_{\ell}(t) \mbf{V}_{\ell}(t) \mbf{\Sigma}_{\ell}(t) + \mbf{\Sigma}_{\ell}(t) \mbf{V}_{\ell}^\top(t) \dot{\mbf{W}}_{\ell}^\top(t) \mbf{U}_{\ell}(t)\right], 
\end{align*}

where the matrix \(\mbf{H}(t) \in \mathbb{R}^{d\times d}\) is defined by:
\begin{align}
    H_{r,r'}(t) := 
    \begin{cases}
    \left(\Sigma^2_{r'}(t) - \Sigma^2_r(t)\right)^{-1}, & r \neq r', \\
    0, & r = r'.
    \end{cases}
\end{align}

Then, multiplying from the left by \(\mbf{U}_{\ell}(t)\) yields
\begin{align}
    \mbf{P}_{\mbf{U}_{\ell}(t)} \dot{\mbf{U}}_{\ell}(t) &= \mbf{U}_{\ell}(t) \left(\mbf{H}(t) \odot \left[\mbf{U}_{\ell}^\top(t) \dot{\mbf{W}}_{\ell}(t) \mbf{V}_{\ell}(t) \mbf{\Sigma}_{\ell}(t) + \mbf{\Sigma}_{\ell}(t) \mbf{V}_{\ell}^\top(t) \dot{\mbf{W}}_{\ell}^\top(t) \mbf{U}_{\ell}(t)\right]\right), 
\end{align}
with \(\mbf{P}_{\mbf{U}_{\ell}(t)} := \mbf{U}_{\ell}(t) \mbf{U}_{\ell}^\top(t)\) being the projection onto the subspace spanned by the (orthonormal) columns of \(\mbf{U}_{\ell}(t)\). Denote by \(\mbf{P}_{\mbf{U}_{{\ell}\perp}(t)}\) the projection onto the orthogonal complement ( i.e., $\mbf{P}_{\mbf{U}_{\ell\perp}(t)} := \mbf{I}_r - \mbf{U}_{\ell}(t) \mbf{U}_{\ell}^\top(t)$). Apply \(\mbf{P}_{\mbf{U}_{\ell\perp}(t)}\) to both sides of Equation~(\ref{eqn:svd_rynamics}):
\begin{align}
    \mbf{P}_{\mbf{U}_{\ell\perp}(t)}\dot{\mbf{U}}_{\ell}(t)  = \mbf{P}_{\mbf{U}_{\ell\perp}(t)} \dot{\mbf{U}}_{\ell}(t) \mbf{\Sigma}_\ell(t) \mbf{V}_{\ell}^\top(t) &+ \mbf{P}_{\mbf{U}_{\ell\perp}(t)} \mbf{U}_\ell(t) \dot{\mbf{\Sigma}}_{\ell}(t) \mbf{V}_{\ell}^\top(t)\\ &+ \mbf{P}_{\mbf{U}_{\ell\perp}(t)} \mbf{U}_\ell(t) \mbf{\Sigma}_\ell(t) \dot{\mbf{V}}_{\ell}^\top(t). 
\end{align}

Note that \(\mbf{P}_{\mbf{U}_{\ell\perp}(t)} \mbf{U}_\ell(t) = 0\), and multiply from the right by \(\mbf{V}_\ell(t) \mbf{\Sigma}_{\ell}^{-1}(t)\) (the latter is well-defined since we have the compact SVD and the top-$r$ elements are non-zero):
\begin{align}
    \mbf{P}_{\mbf{U}_{\ell\perp}(t)} \dot{\mbf{U}}_\ell(t) &= \mbf{P}_{\mbf{U}_{\ell\perp}(t)} \dot{\mbf{W}}_\ell(t) \mbf{V}_\ell(t) \mbf{\Sigma}_\ell^{-1}(t) = (\mbf{I}_r - \mbf{U}_\ell(t)\mbf{U}^\top(t)) \dot{\mbf{W}}(t) \mbf{V}_\ell(t) \mbf{\Sigma}_\ell^{-1}(t). 
\end{align}
Then by adding the two equations above, we obtain an expression for \(\dot{\mbf{U}}(t)\):
\begin{align}
    \dot{\mbf{U}}_\ell(t) &= \mbf{P}_{\mbf{U}_\ell(t)} \dot{\mbf{U}}_\ell(t) + \mbf{P}_{\mbf{U}_{\ell\perp}(t)} \dot{\mbf{U}}_\ell(t) \nonumber \\
    &= \mbf{U}_\ell(t)\left(\mbf{H}(t) \odot \left[\mbf{U}_\ell^\top(t) \dot{\mbf{W}}_\ell(t) \mbf{V}_\ell(t) \mbf{\Sigma}_\ell(t) + \mbf{\Sigma}_\ell(t) \mbf{V}_\ell^\top(t) \dot{\mbf{W}}_\ell^\top(t) \mbf{U}_\ell(t)\right]\right) \nonumber \\
    &\quad + (\mbf{I}_r - \mbf{U}_\ell(t) \mbf{U}_\ell^\top(t)) \dot{\mbf{W}}(t) \mbf{V}_\ell(t) \mbf{\Sigma}_\ell^{-1}(t). 
\end{align}
We can similarly derive the dynamics for $\dot{\mbf{V}}_\ell(t)$ and $\dot{\mbf{\Sigma}}_\ell(t)$:
\begin{align}
\dot{\mbf{V}}_\ell(t) = \mbf{V}_\ell(t)\left(\mbf{H}(t) \odot \left[\mbf{\Sigma}_\ell(t) \mbf{U}^\top_\ell(t) \dot{\mbf{W}_{\ell}}(t) \mbf{V}_\ell(t) + \mbf{V}^\top_\ell(t) \dot{\mbf{W}_{\ell}}^\top(t) \mbf{U}_\ell(t) \mbf{\Sigma}_\ell(t)\right]\right) \\
+ \left(\mbf{I}_{r} - \mbf{V}_\ell(t)\mbf{V}^\top_\ell(t)\right) \dot{\mbf{W}_{\ell}}^\top(t) \mbf{U}_\ell(t) \mbf{\Sigma}_\ell^{-1}(t), \label{vdiff}
\end{align}
\begin{align*}
   \dot{\mbf{\Sigma}}_\ell(t) = \mbf{I}_r \odot \left[ \mbf{U}^\top_\ell(t) \dot{\mbf{W}}_\ell(t) \mbf{V}_\ell(t) \right].
\end{align*}

Now, we will left multiply $\dot{\mbf{U}}_\ell(t)$ and $\dot{\mbf{V}}_\ell(t)$ with $\mbf{U}_\ell^\top(t)$ and $\mbf{V}_\ell^\top(t)$, respectively, to obtain
\begin{align*}
    \mbf{U}^\top_\ell(t) \dot{\mbf{U}}_\ell(t) &= -\mbf{H}(t) \odot \left[\mbf{U}^\top_\ell(t)\nabla_{\mbf{W}_{\ell}} f(\mbf{\Theta}) \mbf{V}_\ell(t) \mbf{\Sigma}_\ell(t) + \mbf{\Sigma}_\ell(t) \mbf{V}^\top_\ell(t) \nabla_{\mbf{W}_{\ell}} f(\mbf{\Theta}) \mbf{U}_\ell(t)\right], \\
    \mbf{V}^\top_\ell(t) \dot{\mbf{V}}_\ell(t) &= -\mbf{H}(t) \odot \left[\mbf{\Sigma}_\ell(t) \mbf{U}^\top_\ell(t) \nabla_{\mbf{W}_{\ell}} f(\mbf{\Theta}) \mbf{V}_\ell(t) + \mbf{V}^\top_\ell(t) \nabla_{\mbf{W}_{\ell}} f(\mbf{\Theta}) \mbf{U}_\ell(t) \mbf{\Sigma}_\ell(t)\right],
\end{align*}
where we replaced $\dot{\mbf{W}}_\ell(t) \coloneqq -\nabla_{\mbf{W}_{\ell}} f(\mbf{\Theta})$, as $\dot{\mbf{W}}_\ell(t)$ is the gradient of $f(\mbf{\Theta})$ with respect to $\mbf{W}_\ell$ by definition. By rearranging and multiplying by $\mbf{\Sigma}_\ell(t)$, we have
\begin{align}
\label{eqn:diagonal_grad}
      \mbf{U}^\top_\ell(t) \dot{\mbf{U}}_\ell(t) \mbf{\Sigma}_\ell(t) -   \mbf{\Sigma}_\ell(t) \mbf{V}^T (t) \dot{\mbf{V}}_\ell(t) = -  \hat{\mbf{I}}_{r} \odot [\mbf{U}^\top_\ell(t) \nabla_{\mbf{W}_{\ell}} f(\mbf{\Theta}) \mbf{V}_\ell(t)].
\end{align}
Hence, when $\dot{\mbf{U}}_\ell(t)=0$ and $\dot{\mbf{V}}_\ell(t)=0$, it must be that the left-hand side is zero and so $\mbf{U}^\top_\ell(t) \nabla_{\mbf{W}_{\ell}} f(\mbf{\Theta}) \mbf{V}_\ell(t)$ is a diagonal matrix. 

Now, notice that for the given loss function $f(\mbf{\Theta})$, we have
\begin{align*}
   -\dot{\mbf{W}}_\ell(t) = \nabla_{\mbf{W}_{\ell}} f(\mbf{\Theta}(t)) = \mbf{W}^{\top}_{L:\ell+1}(t) \cdot \left(\mbf{W}_{L:1}(t) - \mbf{M}_\star \right) \cdot \mbf{W}^{\top}_{\ell-1:1}(t). 
\end{align*}
Then, from Equation~(\ref{eqn:diagonal_grad}), when the singular vectors are stationary, we have
\begin{align*}
    \mbf{U}_\ell^\top(t)\mbf{W}^{\top}_{L:\ell+1}(t) \cdot \left(\mbf{W}_{L:1}(t) - \mbf{M}_\star \right) \cdot \mbf{W}^{\top}_{\ell-1:1}(t)\mbf{V}_\ell(t)
\end{align*}
must be a diagonal matrix for all $\ell \in [L]$. The only solution to the above should be (since the intermediate singular vectors need to cancel to satisfy the diagonal condition), is the set
\begin{align*}
\mathrm{SVS}(f(\mbf{\Theta})) = 
\begin{cases}
    (\mbf{U}_L, \mbf{V}_L) &= (\mbf{U}_\star, \mbf{Q}_L), \\
    (\mbf{U}_\ell, \mbf{V}_\ell) &= (\mbf{Q}_{\ell+1}, \mbf{Q}_\ell), \quad\forall \ell \in [2, L-1], \\
    (\mbf{U}_1, \mbf{V}_1) &= (\mbf{Q}_2, \mbf{V}_\star),
\end{cases}
\end{align*}
where \(\{\mbf{Q}_\ell\}_{\ell=2}^{L}\) are any set of orthogonal matrices. Then, notice that when the singular vectors are stationary, the dynamics become isolated on the singular values: \begin{align*}
   \dot{\mbf{\Sigma}}_\ell(t) = \mbf{I}_r \odot \left[ \mbf{U}^\top_\ell(t) \dot{\mbf{W}}_\ell(t) \mbf{V}_\ell(t) \right],
\end{align*} 
since $\left[ \mbf{U}^\top_\ell(t) \dot{\mbf{W}}_\ell(t) \mbf{V}_\ell(t) \right]$ is diagonal. This completes the proof.


\end{proof}


\subsubsection{Supporting Results}


\begin{proposition}
    
\label{prop:one_zero_svs_set}
    Let $\mbf{M}_\star = \mbf{U}_\star\mbf{\Sigma}_\star \mbf{V}_\star^\top$ denote the SVD of the target matrix. The initialization in Equation~(\ref{eqn:unbalanced_init}) is a member of the singular vector stationary set in Proposition~\ref{prop:svs_set}, where $\mbf{Q}_L = \ldots = \mbf{Q}_2 = \mbf{V}_\star$.
\end{proposition}
\begin{proof}
Recall that the initialization is given by
    \begin{align*}
        \mbf{W}_L(0) = 0 \quad \text{and} \quad \mbf{W}_\ell(0) = \alpha\mbf{I}_d \quad \forall \ell \in [L-1].
    \end{align*}
    We will show that under this initialization, each weight matrix admits the following decomposition for all $t \geq 1$:
    \begin{align}
        \mbf{W}_L(t) = \mbf{U}_\star \begin{bmatrix}
            \widetilde{\mbf{\Sigma}}_L(t) & \mbf{0} \\
            \mbf{0} & \mbf{0}
        \end{bmatrix} \mbf{V}_\star^\top,
        \quad\quad
        \mbf{W}_{\ell}(t) = \mbf{V}_\star \begin{bmatrix}
            \widetilde{\mbf{\Sigma}}(t) & \mbf{0} \\
            \mbf{0} & \alpha\mbf{I}_{d-r}
        \end{bmatrix} \mbf{V}_\star^\top,
        \quad \forall \ell \in [L-1],
    \end{align}
where
\begin{align*}
    \widetilde{\mbf{\Sigma}}_L(t) &= \widetilde{\mbf{\Sigma}}_L(t-1) - \eta \cdot\left(\widetilde{\mbf{\Sigma}}_L(t-1) \cdot \widetilde{\mbf{\Sigma}}^{L-1}(t-1) - \mbf{\Sigma}_{\star,r}\right)\cdot \widetilde{\mbf{\Sigma}}^{L-1}(t-1) \\
    \widetilde{\mbf{\Sigma}}(t) &= \widetilde{\mbf{\Sigma}}(t-1)\cdot  \left(\mbf{I}_r- \eta\cdot\widetilde{\mbf{\Sigma}}_L(t-1)\cdot\left(\widetilde{\mbf{\Sigma}}_L(t-1) \cdot \widetilde{\mbf{\Sigma}}^{L-1}(t-1) - \mbf{\Sigma}_{\star,r}\right)\cdot \widetilde{\mbf{\Sigma}}^{L-3}(t-1)\right),
\end{align*}
where $\widetilde{\mbf{\Sigma}}_L(t), \widetilde{\mbf{\Sigma}}(t) \in \mbb{R}^{r\times r}$ is a diagonal matrix with $\widetilde{\mbf{\Sigma}}_L(1) = \eta \alpha^{L-1}\cdot \mbf{\Sigma}_{r,\star}$ and $\widetilde{\mbf{\Sigma}}(1) = \alpha \mbf{I}_r$. 

This will prove that the singular vectors are stationary with $\mbf{\Sigma}_L = \ldots =\mbf{\Sigma}_2 = \mbf{V}_\star$. We proceed with mathematical induction. 

\paragraph{Base Case.} For the base case, we will show that the decomposition holds for each weight matrix at $t=1$. The gradient of $f(\mbf{\Theta})$ with respect to $\mbf{W}_{\ell}$ is
\begin{align*}
    \nabla_{\mbf{W}_{\ell}} f(\mbf{\Theta}) = \mbf{W}^{\top}_{L:\ell+1} \cdot \left(\mbf{W}_{L:1} - \mbf{M}_\star \right) \cdot \mbf{W}^{\top}_{\ell-1:1}. 
\end{align*}
For $\mbf{W}_L(1)$, we have
\begin{align*}
    \mbf{W}_L(1) &= \mbf{W}_L(0) - \eta \cdot \nabla_{\mbf{W}_{L}} f(\mbf{\Theta}(0)) \\
    &= \mbf{W}_L(0) - \eta \cdot \left(\mbf{W}_{L:1}(0) - \mbf{M}_\star \right) \cdot \mbf{W}^{\top}_{L-1:1}(0)\\
    &= \eta \alpha^{L-1}\mbf{\Sigma}_\star \\
    &= \mbf{U}_\star \cdot \left( \eta \alpha^{L-1} \cdot \mbf{\Sigma}_\star \right) \cdot \mbf{V}_\star^\top \\
    &= \mbf{U}_\star
    \begin{bmatrix}
        \widetilde{\mbf{\Sigma}}_L(1) & \mbf{0} \\
        \mbf{0} & \mbf{0}
    \end{bmatrix}
    \mbf{V}_\star^\top.
\end{align*}
Then, for each $\mbf{W}_{\ell}(1)$ in $\ell \in [L-1]$, we have
\begin{align*}
\mbf{W}_{\ell}(1)&= \mbf{W}_{\ell}(0) - \eta \cdot \nabla_{\mbf{W}_{\ell}}f(\mbf{\Theta}(0)) \\
&= \alpha\mbf{I}_d,
\end{align*}
where the last equality follows from the fact that $\mbf{W}_L(0) = \mbf{0}$. Finally, we have
\begin{align*}
    \mbf{W}_{\ell}(1) = \alpha \mbf{V}_\star \mbf{V}_\star^\top = \mbf{V}_\star \begin{bmatrix}
        \widetilde{\mbf{\Sigma}}(1) & \mbf{0} \\
        \mbf{0} & \alpha\mbf{I}_{d-r}
     \end{bmatrix}\mbf{V}_\star^\top, \quad \forall \ell \in [L-1].
\end{align*}

\paragraph{Inductive Step.} By the inductive hypothesis, suppose that the decomposition holds. Then, notice that we can simplify the end-to-end weight matrix to
\begin{align*}
    \mbf{W}_{L:1}(t) = \mbf{U}_\star
    \begin{bmatrix}
        \widetilde{\mbf{\Sigma}}_L(t) \cdot \widetilde{\mbf{\Sigma}}^{L-1}(t) & \mbf{0} \\
        \mbf{0} & \mbf{0}
    \end{bmatrix}
    \mbf{V}_\star^\top,
\end{align*}
for which we can simplify the gradients to
\begin{align*}
    \nabla_{\mbf{W}_{L}} f(\mbf{\Theta}(t)) &= \left(\mbf{U}_\star \begin{bmatrix}
        \widetilde{\mbf{\Sigma}}_L(t) \cdot \widetilde{\mbf{\Sigma}}^{L-1}(t) - \mbf{\Sigma}_{\star,r} & \mbf{0} \\
        \mbf{0} & \mbf{0}
    \end{bmatrix} \mbf{V}_\star^\top\right) \cdot  \mbf{V}_\star \begin{bmatrix}
        \widetilde{\mbf{\Sigma}}^{L-1}(t) & \mbf{0} \\
        \mbf{0} & \mbf{0}
    \end{bmatrix}\mbf{V}_\star^\top \\
    &= \mbf{U}_\star \begin{bmatrix}
        \left(\widetilde{\mbf{\Sigma}}_L(t) \cdot \widetilde{\mbf{\Sigma}}^{L-1}(t) - \mbf{\Sigma}_{\star,r}\right)\cdot \widetilde{\mbf{\Sigma}}^{L-1}(t) & \mbf{0} \\
        \mbf{0} & \mbf{0}
    \end{bmatrix} \mbf{V}_\star^\top,
\end{align*}
for the last layer matrix, and similarly,
\begin{align*}
     \nabla_{\mbf{W}_{\ell}} f(\mbf{\Theta}(t)) &= \mbf{V}_\star \begin{bmatrix}
        \widetilde{\mbf{\Sigma}}_L(t)\cdot\left(\widetilde{\mbf{\Sigma}}_L(t) \cdot \widetilde{\mbf{\Sigma}}^{L-1}(t) - \mbf{\Sigma}_{\star,r}\right)\cdot \widetilde{\mbf{\Sigma}}^{L-2}(t)  & \mbf{0} \\
        \mbf{0} & \mbf{0}
    \end{bmatrix} \mbf{V}_\star^\top, \quad \ell \in [L-1],
\end{align*}
for all other layer matrices. Thus, for the next GD iteration, we have
\begin{align*}
    \mbf{W}_L(t+1) &= \mbf{W}_{L}(t) - \eta \cdot \nabla_{\mbf{W}_L}(\mbf{\Theta}(t)) \\
    &= \mbf{U}_\star \begin{bmatrix}
        \widetilde{\mbf{\Sigma}}_L(t) - \eta \cdot\left(\widetilde{\mbf{\Sigma}}_L(t) \cdot \widetilde{\mbf{\Sigma}}^{L-1}(t) - \mbf{\Sigma}_{\star,r}\right)\cdot \widetilde{\mbf{\Sigma}}^{L-1}(t) & \mbf{0} \\
        \mbf{0} & \mbf{0}
    \end{bmatrix} \mbf{V}_\star^\top \\
    &= \mbf{U}_\star \begin{bmatrix}
        \widetilde{\mbf{\Sigma}}_L(t+1) & \mbf{0} \\
        \mbf{0} & \mbf{0}
    \end{bmatrix} \mbf{V}_\star^\top.
\end{align*}
Similarly, we have
\begin{align*}
    \mbf{W}_\ell(t+1) &= \mbf{W}_{\ell}(t) - \eta \cdot \nabla_{\mbf{W}_\ell}(\mbf{\Theta}(t)) \\
    &= \mbf{V}_\star \begin{bmatrix}
        \widetilde{\mbf{\Sigma}}(t) - \eta\cdot\widetilde{\mbf{\Sigma}}_L(t)\cdot\left(\widetilde{\mbf{\Sigma}}_L(t) \cdot \widetilde{\mbf{\Sigma}}^{L-1}(t) - \mbf{\Sigma}_{\star,r}\right)\cdot \widetilde{\mbf{\Sigma}}^{L-2}(t)  & \mbf{0} \\
        \mbf{0} & \alpha \mbf{I}_{d-r}
    \end{bmatrix} \mbf{V}_\star^\top \\
     &= \mbf{V}_\star \begin{bmatrix}
        \widetilde{\mbf{\Sigma}}(t)\cdot  \left(\mbf{I}_r- \eta\cdot\widetilde{\mbf{\Sigma}}_L(t)\cdot\left(\widetilde{\mbf{\Sigma}}_L(t) \cdot \widetilde{\mbf{\Sigma}}^{L-1}(t) - \mbf{\Sigma}_{\star,r}\right)\cdot \widetilde{\mbf{\Sigma}}^{L-3}(t)\right)  & \mbf{0} \\
        \mbf{0} & \alpha \mbf{I}_{d-r}
    \end{bmatrix} \mbf{V}_\star^\top \\
    &= \mbf{V}_\star \begin{bmatrix}
        \widetilde{\mbf{\Sigma}}(t+1) & \mbf{0} \\
        \mbf{0} & \alpha \mbf{I}_{d-r}
    \end{bmatrix} \mbf{V}_\star^\top,
\end{align*}
for all $\ell \in [L-1]$. This completes the proof.
\end{proof}




\begin{proposition}
\label{prop:balanced_svs_set}

 Let $\mbf{M}_\star = \mbf{V}_\star\mbf{\Sigma}_\star \mbf{V}_\star^\top \in \mbb{R}^{d\times d}$ denote the SVD of the target matrix. The balanced initialization in Equation~(\ref{eqn:balanced_init}) is a member of the singular vector stationary set in Proposition~\ref{prop:svs_set}, where  $\mbf{U}_L = \mbf{Q}_L = \ldots = \mbf{Q}_2 = \mbf{V}_1 = \mbf{V}_\star$.
 
\end{proposition}

\begin{proof}
    
Using mathematical induction, we will show that with the balanced initialization in Equation~(\ref{eqn:balanced_init}), each weight matrix admits a decomposition of the form
\begin{align}
    \mbf{W}_\ell(t) = \mbf{V}_\star \mbf{\Sigma}_\ell(t) \mbf{V}_\star^\top,
\end{align}
which implies that the singular vectors are stationary for all $t$ such that $\mbf{U}_L = \mbf{Q}_L = \ldots = \mbf{Q}_2 = \mbf{V}_1 = \mbf{V}_\star$.

\paragraph{Base Case.} Consider the weights at iteration $t=0$. By the initialization scheme, we can write each weight matrix as
\begin{align*}
    \mbf{W}_\ell(0) = \alpha \mbf{I}_d \implies  \mbf{W}_\ell(0) = \alpha \mbf{V}_\star \mbf{V}_\star^\top,
\end{align*}
which implies that $\mbf{W}_\ell(0) = \mbf{V}_\star \mbf{\Sigma}_\ell(0)\mbf{V}_\star^\top$ with $\mbf{\Sigma}_\ell(0) = \alpha \mbf{I}_d$.

\paragraph{Inductive Step.} By the inductive hypothesis, assume that the decomposition holds for all $t \geq 0$. We will show that it holds for all iterations $t+1$. Recall that the gradient of $f(\mbf{\Theta})$ with respect to $\mbf{W}_{\ell}$ is
\begin{align*}
    \nabla_{\mbf{W}_{\ell}} f(\mbf{\Theta}) = \mbf{W}^{\top}_{L:\ell+1} \cdot \left(\mbf{W}_{L:1} - \mbf{M}_\star \right) \cdot \mbf{W}^{\top}_{\ell-1:1}. 
\end{align*}
Then, for $\mbf{W}_\ell(t+1)$, we have
\begin{align*}
    \mbf{W}_\ell(t+1) &= \mbf{W}_\ell(t) - \eta \cdot \nabla_{\mbf{W}_{L}} f(\mbf{\Theta}(t)) \\
    &=  \mbf{V}_\star \mbf{\Sigma}_\ell(t)\mbf{V}_\star^\top - \eta \mbf{W}^\top_{L:\ell+1}(t) \cdot \left(\mbf{W}_{L:1}(t) - \mbf{M}_\star \right) \cdot \mbf{W}^{\top}_{\ell-1:1}(t)\\
    &=  \mbf{V}_\star \mbf{\Sigma}_\ell(t)\mbf{V}_\star^\top - \eta \mbf{V}_\star\cdot \left(  \mbf{\Sigma}^{L-\ell}_\ell(t)\cdot \left(\mbf{\Sigma}_\ell^{L}(t) - \mbf{\Sigma}_\star \right)\cdot \mbf{\Sigma}_\ell^{\ell-1}(t) \right) \cdot\mbf{V}^{\top}_{\star}\\
    &=  \mbf{V}_\star\cdot \left(\mbf{\Sigma}_\ell(t) - \eta\cdot  \mbf{\Sigma}^{L-\ell}_\ell(t)\cdot \left(\mbf{\Sigma}_\ell^{L}(t) - \mbf{\Sigma}_\star \right)\cdot \mbf{\Sigma}_\ell^{\ell-1}(t) \right) \cdot\mbf{V}^{\top}_{\star}\\
    &= \mbf{V}_\star
   \mbf{\Sigma}(t)
    \mbf{V}_\star^\top,
\end{align*}
where $\mbf{\Sigma}(t) = \mbf{\Sigma}_\ell(t) - \eta\cdot  \mbf{\Sigma}^{L-\ell}_\ell(t)\cdot \left(\mbf{\Sigma}_\ell^{L}(t) - \mbf{\Sigma}_\star \right)\cdot \mbf{\Sigma}_\ell^{\ell-1}(t)$. This completes the proof.
\end{proof}





\subsection{Proofs for Balancing}

In this section, we present our proof of Proposition~\ref{prop:balancing} along with supporting results. Throughout these results, we use the notion of the gradient flow solution (GFS) and the GFS sharpness as presented by~\cite{kreisler2023gradient}, which we briefly recap. \\

\noindent Consider minimizing a smooth loss function  $\mathcal{L}:\mbb{R}^d \to \mbb{R}$ using gradient flow (GF):
\begin{align*}
    \dot{\mbf{w}}(t) = -\nabla\mathcal{L}(\mbf{w}(t)).
\end{align*}
The GFS denoted by $S_{\mathrm{GF}}(\mbf{w})$ is the limit of the gradient flow trajectory when initialized at $\mbf{w}$. Furthermore, the GFS sharpness denoted by $\psi(\mbf{w})$ is defined to be the sharpness of $S_{\mathrm{GF}}(\mbf{w})$, i.e., the largest eigenvalue of $\nabla^2 \mathcal{L}\left( S_{\mathrm{GF}}(\mbf{w}) \right)$. 




\subsubsection{Supporting Lemmas}

\begin{lemma}[Conservation of Balancedness in GF]
\label{gf-unbalanced}
    Consider the singular value scalar loss         \begin{align*} 
        \mathcal{L}\left(\{\sigma_\ell\}_{\ell=1}^L\right)
 = \frac{1}{2} \left( \prod_{\ell=1}^L \sigma_{\ell} - \sigma_{\star} \right)^2.
    \end{align*}  
    Under gradient flow, the balancedness between two singular values defined by $\sigma^2_{ \ell} (t) - \sigma^2_{m} (t)$ for all $m, \ell \in [L]$ is constant for all $t\geq 0$.
\end{lemma}

\begin{proof}
Notice that the result holds specifically for gradient flow and not descent. The dynamics of each scalar factor for gradient flow can be written as
    \begin{align*}
        \dot{\sigma}_{\ell}(t) = - \left(\prod_{\ell=1}^L \sigma_{ \ell} (t) - \sigma_{\star} \right)\cdot \prod_{i\neq \ell}^L \sigma_{i}(t)
    \end{align*}
Then, the time derivative of balancing is given as
\begin{align*}
  & \frac{\partial}{\partial t} (\sigma^2_{ \ell} (t) - \sigma^2_{m} (t)) = \sigma_{ \ell} (t)\dot{\sigma}_{\ell}(t)  - \sigma_{m} (t)\dot{\sigma}_{m}(t)  \\
  & = - \sigma_{ \ell} (t)\left(\prod_{\ell=1}^L \sigma_{ \ell} (t) - \sigma_{\star} \right)\cdot \prod_{i\neq \ell}^L \sigma_{i}(t) + \sigma_{m} (t)\left(\prod_{m=1}^L \sigma_{ \ell} (t) - \sigma_{\star} \right)\cdot \prod_{j\neq m}^L \sigma_{j}(t). \\
  & = 0.
\end{align*}
Hence, the quantity $\sigma^2_{ \ell} (t) - \sigma^2_{m} (t) $ remains constant for all time $t\geq 0$, hence preserving balancedness. 
\end{proof}

\begin{lemma}[Sharpness at Minima]
\label{1d-sharp}
    Consider the singular value scalar loss     \begin{align*} 
        \mathcal{L}(\{\sigma_i\}_{i=1}^d)
 = \frac{1}{2} \left( \prod_{i=1}^L \sigma_{i} - \sigma_{\star} \right)^2,
    \end{align*}
    The sharpness at the global minima is given as $\| \nabla^2 \mathcal{L} \|_{2} = \sum_{i=1}^{L} \frac{\sigma^2_{\star}}{\sigma^2_{i}}$.
\end{lemma}

\begin{proof}
The gradient is given by
\begin{align*}
    \nabla_{\sigma_{i}}  \mathcal{L} = \left(\prod_{\ell=1}^L \sigma_{ \ell} (t) - \sigma_{\star} \right) \prod_{j\neq i}^L \sigma_{j}(t).
\end{align*}
Then, 
\begin{align*}
     \nabla_{\sigma_{j}}  \nabla_{\sigma_{i}}  \mathcal{L} =  \prod_{\ell\neq i}^L \sigma_{\ell}(t)  \prod_{\ell\neq j}^L \sigma_{\ell}(t) + \left(\prod_{\ell=1}^L \sigma_{ \ell} (t) - \sigma_{\star} \right)  \prod_{\ell\neq j, \ell \neq i}^L \sigma_{\ell}(t)
\end{align*}
 Let $\pi(t)=  \prod_{i=1}^L \sigma_{i}(t)$. Then, at the global minima, we have
\begin{align*}
     \nabla_{\sigma_{j}}  \nabla_{\sigma_{i}}  \mathcal{L} =  \frac{\pi^2}{\sigma_{i} \sigma_{j}} = \frac{\sigma_{\star}^2}{\sigma_{i} \sigma_{j}}
\end{align*}
Thus, the sharpness of the largest eigenvalue is given as $ \| \nabla^2 \mathcal{L} \|_{2} = \sum_{i=1}^{L} \frac{\sigma^2_{\star}}{\sigma^2_{i}}$. 
\end{proof}

\begin{lemma}[Balanced Minima is the Flattest]
\label{lemma:flattest}
Consider the singular value scalar loss \begin{align*} 
        \mathcal{L}\left(\{\sigma_i\}_{i=1}^L\right)
 = \frac{1}{2} \left( \prod_{i=1}^L \sigma_{i} - \sigma_{\star} \right)^2.
    \end{align*} 
The balanced minimum (i.e., $\sigma_i = \sigma_\star^{1/L}$ for all $i \in [L]$)  has the smallest sharpness amongst all global minima with a value of $\|\nabla^2 \mathcal{L}\|_2 = L\sigma_\star^{2-2/L}$.
\end{lemma}
\begin{proof}
    By Lemma~\ref{1d-sharp}, recall that the sharpness at the global minima is given in the form
    \begin{align*}
        \|\nabla^2 \mathcal{L}\|_2 = \sum_{i=1}^L \frac{\sigma_\star^2}{\sigma_i^2}.
    \end{align*}
    To show that the balanced minimum is the flattest (i.e., it has the smallest sharpness amongst all global minima), we will show that KKT stationarity condition of the constrained objective
    \begin{align*}
        \underset{\{\sigma_i\}_{i=1}^L }{\mathrm{min}} \,  \sum_{i=1}^L \frac{\sigma_\star^2}{\sigma_i^2} \quad\,\mathrm{s.t.} \,\, \prod_{i=1}^L \sigma_i = \sigma_\star,
    \end{align*}
    are only met at the balanced minimum, 
    which gives us the sharpness value $\| \nabla^2 \mathcal{L} \|_{2} = L\sigma_\star^{2-2/L}$.
    The Lagrangian is given by
    \begin{align*}
        L(\sigma_1, \ldots, \sigma_L, \mu) = \sum_{i=1}^L \frac{\sigma_\star^2}{\sigma_i^2} + \mu\left( \prod_{i=1}^L \sigma_i - \sigma_\star \right).
    \end{align*}
Then, the stationary point conditions of the Langrangian is given by 
    \begin{align}
    \label{eqn:station1}
        \frac{\partial L}{\partial \sigma_i} &= -\frac{2\sigma_\star^2}{\sigma_i^3} + \mu \prod_{j\neq i} \sigma_j = 0, \\
    \label{eqn:station2}
    \frac{\partial L}{\partial \mu} &= \prod_{i=1}^L \sigma_i - \sigma_\star = 0.
    \end{align}
    From Equation~(\ref{eqn:station1}), the solution of the stationary point gives
    \begin{align*}
        \frac{2\sigma_\star^2}{\sigma_i^3} = \mu \prod_{j\neq i} \sigma_j \implies \mu =  \frac{2\sigma_\star^2}{\sigma_i^3 \prod_{j\neq i} \sigma_j} = \frac{2\sigma_\star^2}{\sigma_i^2 \sigma_\star} = \frac{2\sigma_\star}{\sigma_i^2}.
    \end{align*}
    This also indicates that at the stationary point, $\sigma_{i} = \sqrt{\frac{2\sigma_\star}{\mu}}$ for all $i \in [L]$, which means that the condition is \emph{only} satisfied at the balanced minimum, i.e, $\sigma_{i} = \sigma_\star^{1/L}$. Furthermore, notice that
    \begin{align*}
        \nabla^2 f(\sigma_i) = 6 \sigma_\star^2 \cdot \diag\left(\frac{1}{\sigma^4_{i}}\right) \succ \mathbf{0},
    \end{align*}
    where $f(\sigma_i) = \sum_{i=1}^L \frac{\sigma_\star^2}{\sigma_i^2}$, indicating that $f$ only has a minimum. Notice that Equation~(\ref{eqn:station2}) holds immediately. Thus, the balanced minimum has the smallest shaprness (flattest), which plugging into $f$ gives a sharpness of $\|\nabla^2 \mathcal{L}\|_2 = L\sigma_\star^{2-2/L}$. 

    
    % Then, plugging in the balanced minimum and $\mu^*$ into Equation~(\ref{eqn:station1}), we have
    % \begin{align*}
    %     -\frac{2\sigma_\star^2}{\sigma_\star^{3/L}} + \frac{2\sigma_\star}{\sigma_\star^{2/L}} \cdot \left(\sigma_\star^{1/L}\right)^{L-1} = -2\sigma_\star^{\frac{2L-3}{L}} + 2\sigma_\star^{\frac{2L-3}{L}} = 0.
    % \end{align*}
    % Notice that Equation~(\ref{eqn:station2}) holds immediately. Thus, the balanced minimum $\sigma_i = \sigma_\star^{1/L}$ satisfies both the stationary point conditions, which completes the proof.
\end{proof}




\begin{lemma}
\label{GFS-3}
    Let $\mbf{s} \coloneqq \begin{bmatrix}
        \sigma_1  & \sigma_2 & \ldots & \sigma_L
    \end{bmatrix} \in \mbb{R}^L$ and define the singular value scalar loss as 
    \begin{align*} 
        \mathcal{L}(\mbf{s})
 = \frac{1}{2} \left( \prod_{i=1}^L \sigma_{i} - \sigma_{\star} \right)^2,
    \end{align*} 
    for some $\sigma_\star > 0$. If $\sigma \in \mbb{R}^L$ are initialized such that
    \begin{align*}
        \sigma_L(0) = 0 \quad \text{and} \quad \sigma_\ell(0) = \alpha, \quad \forall \ell \in [L-1],
    \end{align*}
    where $0<\alpha < \left( \ln\left( \frac{2\sqrt{2}}{\eta L \sigma_{\star}^{2 - \frac{2}{L}}} \right) \cdot \frac{ \sigma_{\star}^{\frac{4}{L}}}{L^2 \cdot 2^{\frac{2L-3}{L}}} \right)^{\frac{1}{4}}$ and $\eta > 0$, then
    the GFS sharpness satisfies $\psi(\mbf{s}) \leq \frac{2\sqrt{1+c}}{\eta}$ for some $0<c<1$.
\end{lemma} 


\begin{proof}
    We will show that the necessary condition for the GFS sharpness 
    to satisfy $\psi(\mbf{s}) \leq \frac{2\sqrt{1+c}}{\eta}$ for some $\eta >0$ and $0<c<1$ to hold is that the initialization scale $\alpha$ must satisfy $0<\alpha < \left( \ln\left( \frac{2\sqrt{2}}{\eta L \sigma_{\star}^{2 - \frac{2}{L}}} \right) \cdot \frac{ \sigma_{\star}^{\frac{4}{L}}}{L^2 \cdot 2^{\frac{2L-3}{L}}} \right)^{\frac{1}{4}}$. \\
    
    Since the singular values $\sigma_\ell$ for all $\ell \in [L-1]$ are initialized to $\alpha$, note that they all follow the same dynamics. Then, let us define the following for simplicity in exposition:
    \begin{align*}
        y \coloneqq \sigma_1 = \ldots = \sigma_{L-1} \quad \text{and} \quad x \coloneqq \sigma_L,
    \end{align*}
    and so $\prod_{\ell=1}^L \sigma_\ell = xy^{L-1}$.
    Then, note that the  gradient flow (GF) solution is the intersection between
    \begin{align*}
       xy^{L-1}=\sigma_{\star} \quad \text{and} \quad x^{2} - y^{2}= -\alpha^2,
    \end{align*}
    where the first condition comes from convergence and the second comes from the conservation flow law of GF from in Lemma \ref{gf-unbalanced}.  
    Then, if we can find a solution at the intersection such that
    \begin{align}
    \label{constraint-set}
        (\hat{x}(\alpha),\hat{y}(\alpha)) = \begin{cases}
            xy^{L-1}=\sigma_{\star} \\
            x^{2} - y^{2}= -\alpha^2,
        \end{cases} 
    \end{align}
    solely in terms of $\alpha$, we can plug in $(\hat{x}(\alpha),\hat{y}(\alpha))$ into the GFS\footnote{Note that throughout the proof $(\hat{x}(\alpha),\hat{y}(\alpha))$ denotes the gradient flow solution as function of $\alpha$. It does
not refer to the GF trajectory.}:
    \begin{align}
\label{eqn:psi_diff}
        \psi(\hat{x}(\alpha),\hat{y}(\alpha)) 
 = \psi(\mbf{s}) \stackrel{(i)}{=}\sum_{i=1}^{L} \frac{\sigma^2_{\star}}{\sigma^2_{i}} = \sigma^2_{\star}\left(\frac{1}{\hat{x}(\alpha)^2} + \frac{L-1}{\hat{y}(\alpha)^2}\right) < \frac{2\sqrt{2}}{\eta},
    \end{align}
and solve to find an upper bound in terms of $\alpha$, where (i) comes from Lemma \ref{1d-sharp}.
The strict inequality ensures that we can find a $c$ in $c \in [0,1)$ such that $ \psi(\mbf{s}) \leq\frac{2\sqrt{1+c}}{\eta} $. However, the intersection $(\hat{x}(\alpha),\hat{y}(\alpha))$ is a $2L$-th order polynomial in $\hat{y}(\alpha)$ which does not have a straightforward closed-form solution solely in terms of $\alpha$. 
To this end, we aim to find a more tractable upper bound on $\psi(\hat{x}(\alpha), \hat{y}(\alpha))$ by using variational calculus, and use that to find a bound on $\alpha$ instead. Specifically, we will compute the differential $d\psi$, upper bound $d\psi$ with a tractable function, and then integrate to obtain our new function $\psi'$ for which we use to set  $\psi' < \frac{2\sqrt{2}}{\eta}$.

\paragraph{Computing the Differentials $d\hat{x}$ and $d\hat{y}$.} 

Before computing the differential $d\psi$, we need to derive the differentials of $\hat{x}(\alpha)$ and $\hat{y}(\alpha)$. We drop the $\alpha$ notation and use $\hat{x}$ and $\hat{y}$ where applicable. 
%For now, we assume that $\alpha < \sigma_\star$, but we will see later that our bound on $\alpha$ already encompasses this case.
By plugging in $\hat{x}$ into Equation~(\ref{constraint-set}), the solution $\hat{y}$ satisfies
\begin{align*}
    \hat{y}^{2L} - \alpha^2 \hat{y}^{2L-2} =  \sigma^2_{\star}.
\end{align*}
Then, by differentiating the relation with respect to $\alpha$, we obtain the following variational relation:
\begin{align}
   &2L  \hat{y}^{2L-1}d \hat{y} - \alpha^2 2 (L-1) \hat{y}^{2L-3}d\hat{y} - 2\alpha \hat{y}^{2L-2} d\alpha = 0 \notag  \\ 
   & \implies \hat{y}^{2L-3} (\hat{y}^2 L - \alpha^2 (L-1)) d\hat{y} = \alpha \hat{y}^{2(L-1)} d\alpha \notag \\ \
   & \implies d\hat{y} = \frac{\hat{y} \alpha}{ (\hat{y}^2 L -\alpha^2 (L-1))} d\alpha,
\end{align}
where we used Lemma~3.10 of~\cite{kreisler2023gradient} to deduce that $\hat{y} > 0$ and so $\hat{y}^{2L-2} > 0$. Then, notice that we have $\hat{y} > \sqrt{\frac{L-1}{L}} \alpha$ from initialization, and so we
$\frac{d\hat{y}}{d \alpha}>0$, (i.e., $\hat{y}(\alpha)$ is an increasing function of $\alpha$). Then, we also have
\begin{align*}
    \underset{\alpha \to 0}{\lim} \, \hat{y}(\alpha) = \sigma_\star^{1/L} \quad \text{and} \quad \underset{\alpha \to 0}{\lim} \, \hat{x}(\alpha) = \sigma_\star^{1/L}, 
\end{align*}
as it corresponds to exact balancing. Hence, as $\alpha$ increases from 0, $\hat{y}(\alpha)$ increases from $\sigma_\star^{1/L}$.

Similarly, the intersection at the global minima satisfies the following relation for $ \hat{x}$:
\begin{align}
    & \hat{x}^{\left(2+ \frac{2}{L-1} \right)} + \hat{x}^{\frac{2}{L-1}} \alpha^2 = \sigma^{\frac{2}{L-1}}_{\star} \notag \\ 
    & \implies \left(2+\frac{2}{L-1} \right) \hat{x}^{\left({ \frac{2}{L-1}+1}\right)} d\hat{x} + \left(\frac{2}{L-1}\right)\alpha^2 \hat{x}^{\left(\frac{2}{L-1}-1\right)} d\hat{x}  + 2\alpha\hat{x}^{\frac{2}{L-1}}d\alpha = 0 \notag \\
    & \implies d\hat{x} = \frac{-\alpha}{\left(\frac{L\hat{x}}{L-1} + \frac{\alpha^2}{(L-1)\hat{x}}\right)} d\alpha.
\end{align}

Note that since $\hat{x}>0$, we have $\frac{dx}{d\alpha}<0$. This implies that as $\alpha$ increases from 0, $\hat{x}(\alpha)$ decreases from $\sigma_{\star}^{1/L}$.



\paragraph{Computing the Differential $d\psi$.} Now we are position to derive the differential $d\psi$. Let us define $\Psi(\alpha) \coloneqq \psi(\hat{x}(\alpha),\hat{y}(\alpha)) $ as we ultimately want the behavior in terms of $\alpha$. Let us simplify $\Psi(\alpha)$ first:
\begin{align}
   \Psi(\alpha) \coloneqq  \psi(\hat{x}(\alpha),\hat{y}(\alpha)) &=  \sigma_{\star}^2 \left(\frac{1}{\hat{x}(\alpha)^2} + \frac{L-1}{\hat{y}(\alpha)^2} \right)  \tag{From Equation~(\ref{eqn:psi_diff})} \\ 
    &= \sigma_{\star}^2 \left(\frac{\hat{y}(\alpha)^2 + (L-1)\hat{x}(\alpha)^2}{\hat{x}(\alpha)^2\hat{y}(\alpha)^2} \right) \\
\label{eqn:psi_alpha_simplify}&\implies \frac{\hat{y}^2}{L} + \left(1-\frac{1}{L} \right) \hat{x}^2 = \frac{\Psi(\alpha) \hat{x}^2 \hat{y}^2}{L \sigma_{\star}^2}.
\end{align}
Then, computing the differential, we have the following:
\begin{align}
   d \Psi &= \sigma^2_{\star} \left(-\frac{2 }{\hat{x}^3} d\hat{x} - \frac{2 (L-1) }{\hat{y}^3}d\hat{y} \right)  \\ 
   &=  \frac{1}{\hat{x}^3} \left[\frac{2 \alpha \sigma^2_{\star}}{\frac{L\hat{x}}{L-1} +\frac{\alpha^2}{(L-1)\hat{x}}} \right] d\alpha 
   - \left[\frac{(L-1)}{\hat{y}^3} \frac{2 \alpha \hat{y} \sigma^2_{\star}}{(\hat{y}^2L - \alpha^2 (L-1))} \right] d \alpha \tag{Substitute $d\hat{x}, d\hat{y}$} \\
   &= \left[\frac{1}{\hat{x}^4 + \left(\frac{\alpha^2}{L}\right)\hat{x}^2 } - \frac{1}{\hat{y}^4 - \alpha^2 \hat{y}^2\left(\frac{L-1}{L}\right)} \right] \cdot \frac{2\alpha(L-1)\sigma^2_{\star}}{L} d \alpha \\
    &=   \left[\frac{\hat{y}^4 - \hat{x}^4 -\alpha^2 \left(\frac{\hat{x}^2}{L} + \left(1 - \frac{1}{L} \right)\hat{y}^2  \right) }{\left(\hat{x}^4 + \frac{\alpha^2}{L}\hat{x}^2 \right)\cdot\left(\hat{y}^4 - \alpha^2 \hat{y}\left(\frac{L-1}{L}\right) \right) } \right] \cdot \frac{2\alpha(L-1)\sigma^2_{\star}}{L} d \alpha. %\notag 
    %\\&=   \frac{\alpha^2  \left( \frac{\hat{y}^2}{L} + (1-\frac{1}{L})\hat{x}^2\right) }{\left(\hat{x}^4 + \frac{\alpha^2}{L}\hat{x}^2 \right)\cdot\left(\hat{y}^4 - \alpha^2 \hat{y}\left(\frac{L-1}{L}\right) \right) } 2 \left(1-\frac{1}{L}\right) \sigma^2_{\star} \alpha d \alpha  
\end{align}
Then, recall the intersection constraint:
\begin{align}
\label{eqn:fourth_order}
\hat{y}^2 - \hat{x}^2 = \alpha^2 &\implies (\hat{y}^2 - \hat{x}^2)(\hat{y}^2 + \hat{x}^2) = \alpha^2(\hat{y}^2 + \hat{x}^2) \\
\label{eqn:fourth_order}
&\implies \hat{x}^4 - \hat{y}^4 = \alpha^2 \cdot (\hat{x}^2 + \hat{y}^2).
\end{align}
By substituting in Equation~(\ref{eqn:fourth_order}), we can simplify further:
\begin{align*}
    d \Psi =   \left[\frac{\alpha^2  \left( \frac{\hat{y}^2}{L} + \left(1-\frac{1}{L} \right)\hat{x}^2\right) }{(\hat{x}^4 + \frac{\alpha^2}{L}\hat{x}^2 )(\hat{y}^4 - \alpha^2 \hat{y}^2\left(\frac{L-1}{L}\right)) } \right] \cdot \frac{2\alpha(L-1)\sigma^2_{\star}}{L} d \alpha.
\end{align*}
Now, we can plug in Equation~(\ref{eqn:psi_alpha_simplify}) into the numerator:
\begin{align*}
    d \Psi &=   \left[\frac{\alpha^2  \left( \frac{\hat{y}^2}{L} + \left(1-\frac{1}{L} \right)\hat{x}^2\right) }{(\hat{x}^4 + \frac{\alpha^2}{L}\hat{x}^2 )(\hat{y}^4 - \alpha^2 \hat{y}^2\left(\frac{L-1}{L}\right)) } \right] \cdot \frac{2\alpha(L-1)\sigma^2_{\star}}{L} d \alpha\\
    &= \left[\frac{\alpha^2  \left( \frac{\Psi(\alpha) \hat{x}^2\hat{y}^2}{L\sigma_\star^2}\right) }{(\hat{x}^4 + \frac{\alpha^2}{L}\hat{x}^2 )(\hat{y}^4 - \alpha^2 \hat{y}^2\left(\frac{L-1}{L}\right)) } \right] \cdot \frac{2\alpha(L-1)\sigma^2_{\star}}{L} d \alpha \\
    &= \left[\frac{\alpha^2\Psi(\alpha) \hat{x}^2\hat{y}^2}{L\sigma_\star^2(\hat{x}^4 + \frac{\alpha^2}{L}\hat{x}^2 )(\hat{y}^4 - \alpha^2 \hat{y}^2\left(\frac{L-1}{L}\right)) } \right] \cdot \frac{2\alpha(L-1)\sigma^2_{\star}}{L} d \alpha \\
    &= \left[\frac{2\Psi(\alpha)}{(\hat{x}^2 + \frac{\alpha^2}{L} )(\hat{y}^2 - \alpha^2 \left(\frac{L-1}{L}\right)) } \right] \cdot\left(\frac{1}{L} - \frac{1}{L^2} \right)\alpha^{3} \, d \alpha. 
\end{align*}
Finally, notice that from the conservation flow, we also have
\begin{align*}
     \hat{y}^2 - \hat{x}^2 = \alpha^2 \implies \hat{x}^2 + \frac{\alpha^2}{L} =\hat{y}^2 - \alpha^2 \left(\frac{L-1}{L}\right),
\end{align*}
and so
\begin{align*}
    d\Psi = \left[\frac{2\Psi(\alpha)}{(\hat{x}^2 + \frac{\alpha^2}{L})^2} \right] \cdot\left(\frac{1}{L} - \frac{1}{L^2} \right)\alpha^{3} \, d \alpha \implies \frac{d\Psi}{\Psi(\alpha)} &= \underbrace{\left[\frac{2}{(\hat{x}^2 + \frac{\alpha^2}{L})^2} \right] \cdot\left(\frac{1}{L} - \frac{1}{L^2} \right)}_{\eqqcolon P (\alpha)}\alpha^{3} \, d \alpha \\
    &= P(\alpha)\alpha^3 \, d\alpha.
\end{align*}

\paragraph{Upper Bounding the Differential.}
Note that it is difficult to directly solve for $\alpha$ from $P(\alpha)$, as $\hat{x}$ is also a function of $\alpha$. Hence, we can upper bound $P(\alpha)$ by a function $F(\alpha)$ such that $F(\alpha) \geq P(\alpha)$ for all $\alpha > 0$, and use this to solve for $\alpha$. We proceed by looking at the derivative of $P(\alpha)$:
\begin{align*}
    P'(\alpha) &= \left[\frac{-4}{(\hat{x}^2 + \frac{\alpha^2}{L})^3 } \right]   \left(\frac{1}{L} - \frac{1}{L^2}\right) \left(2\hat{x} \frac{d \hat{x}}{d \alpha} + \frac{2 \alpha}{L} \right) \\
    &= \left[\frac{-4}{(\hat{x}^2 + \frac{\alpha^2}{L})^3 } \right]   \left(\frac{1}{L} - \frac{1}{L^2}\right) \left(\frac{2 \alpha}{L} -  \frac{2\hat{x}\alpha}{\frac{L\hat{x}}{L-1} + \frac{\alpha^2}{(L-1)\hat{x}}} \right) \\
    &= \frac{8 \alpha}{(\hat{x}^2 + \frac{\alpha^2}{L})^3 }  \left(\frac{1}{L} - \frac{1}{L^2}\right) \left( \frac{L-1}{L+ \frac{\alpha^2}{\hat{x}^2}} - \frac{1}{L}\right)
\end{align*}
\paragraph{Case 1: $L=2$.} Consider the case when $L=2$. Then, notice that for all $\alpha > 0$, $P'(\alpha) < 0$. Thus, we can choose $F(\alpha)$ as such:
\begin{align*}
    F = \underset{\alpha \to 0}{\lim} \, P(\alpha) = \frac{2}{\sigma_\star^{4/L}} \left(\frac{1}{L} - \frac{1}{L^2} \right),
\end{align*}
which is constant in $\alpha$ that upper bounds $P(\alpha)$.

\paragraph{Case 2: $L>2$.} Now consider the general case. Notice that 
\begin{align*}
    \hat{x}(\alpha) = \frac{\alpha}{\sqrt{L(L-2)}}
\end{align*}
is the only critical point of $P(\alpha)$ (since $\hat{x} > 0$). Furthermore, we have
\begin{align*}
    \hat{x}(\alpha) < \frac{\alpha}{\sqrt{L(L-2)}} \implies P'(\alpha) < 0,
\end{align*}
implying that $P(\alpha)$ is decreasing. Then, since $\hat{x}(\alpha)$ is also a decreasing function in $\alpha$, this means that there exists an $\alpha_{\text{crit}}$ such that for all $\alpha > \alpha_{\text{crit}}$, $P(\alpha)$ is always decreasing. We can find 
$\alpha_{\text{crit}}$ as such:
\begin{align*}
    \hat{x}(\alpha_{\text{crit}}) = \frac{\alpha_{\text{crit}}}{\sqrt{L(L-2)}} \implies \hat{y}(\alpha_{\text{crit}}) = \alpha_{\text{crit}} \sqrt{1 + \frac{1}{L(L-2)}}.
\end{align*}
By plugging these into our constraint set, we obtain
\begin{align*}
    &\left(\frac{\alpha_{\text{crit}}}{\sqrt{L(L-2)}} \right) \left(\alpha_{\text{crit}} \sqrt{1 + \frac{1}{L(L-2)}} \right)^{L-1} = \sigma_\star \\
    &\implies \alpha_{\text{crit}}^L\left( \sqrt{1 + \frac{1}{L(L-2)}} \right)^{L-1} = \sigma_\star\sqrt{L(L-2)} \\
    &\implies \alpha_{\text{crit}}^L = \frac{\sigma_\star\sqrt{L(L-2)}}{\left( \sqrt{1 + \frac{1}{L(L-2)}} \right)^{L-1}} \\
    &\implies \alpha_{\text{crit}} = \frac{\sigma_{*}^{1/L}}{\left(\frac{1}{\sqrt{L(L-2)}} \left(1+\frac{1}{L(L-2)}\right)^{\frac{L-1}{2}}\right)^{1/L}}.
\end{align*}
Next, also note that for any $\alpha < \alpha_{\text{crit}}$, $P'(\alpha) > 0$, and so $P(\alpha)$ is increasing. Hence, $P(\alpha_{\text{crit}})$ corresponds to the maximum value of $P$. Therefore, we can choose $F = P(\alpha_{\text{crit}})$ as a constant function that upper bounds $P(\alpha)$. This leads to
\begin{align*}
    F = P(\alpha_{\text{crit}}) &= \left[\frac{2}{(\hat{x}(\alpha_{\text{crit}})^2 + \frac{\alpha_{\text{crit}}^2}{L})^2} \right] \cdot\left(\frac{1}{L} - \frac{1}{L^2} \right) \\
    &= \left[\frac{2}{\left(\frac{\alpha_{\text{crit}}^2}{L(L-2)} + \frac{\alpha_{\text{crit}}^2}{L} \right)^2} \right] \cdot\left(\frac{1}{L} - \frac{1}{L^2} \right) \\
    &= \left[\frac{2}{\left(\frac{(L-1)\alpha_{\text{crit}}^2}{L(L-2)}\right)^2} \right] \cdot\left(\frac{1}{L} - \frac{1}{L^2} \right) \\
    &= \frac{2}{\sigma_\star^{4/L}}  \cdot\underbrace{\left(\frac{1}{L} - \frac{1}{L^2} \right) \left(\frac{L(L-2)}{L-1} \right)^2 \left(\frac{1}{\sqrt{L(L-2)}} \left(1+\frac{1}{L(L-2)}\right)^{\frac{L-1}{2}}\right)^{4/L}}_{\eqqcolon h(L)} \\
    &= \frac{2h(L)}{\sigma_\star^{4/L}}.
\end{align*}

\paragraph{Combining Both Cases.} To avoid using two separate functions $F$ for different values of $L$, we can upper bound the function $h(L)$ to encompass both cases. This yields the following upper bound:
\begin{align*}
    h(L) \leq L^2 \cdot \left(\left(1 + \frac{1}{L(L-2)}\right)^{\frac{L-1}{2}}\right)^{\frac{4}{L}} \leq L^2\cdot 2^{\frac{2(L-1)}{L}} \eqqcolon g(L).
\end{align*}
Finally, we are left with the new differential
\begin{align*}
    \frac{d\Psi}{\Psi(\alpha)} = \frac{2g(L)}{\sigma_\star^{4/L}}\alpha^3 \, d\alpha.
\end{align*}



\begin{figure}[t!]
    \centering
     \begin{subfigure}[t!]{0.325\textwidth}
         \centering
        \includegraphics[width=\textwidth]{figures/P_alpha_L2.pdf}
     \caption*{$L=2$}
     \end{subfigure}
 \begin{subfigure}[t!]{0.325\textwidth}
         \centering
        \includegraphics[width=\textwidth]{figures/P_alpha_L3.pdf}
     \caption*{$L=3$}
     \end{subfigure}
 \begin{subfigure}[t!]{0.325\textwidth}
         \centering
        \includegraphics[width=\textwidth]{figures/P_alpha_L4.pdf}
     \caption*{$L=4$}
     \end{subfigure}
    \caption{Plot of $P(\alpha)$ along with its upper bound evaluated at $F = P(\alpha_{\text{crit}})$ for different depths. The critical point occurs exactly at the computed value of $\alpha_{\text{crit}}$ and the function $F \geq P(\alpha)$ for all $\alpha > 0$.}
    
    \label{fig:p_upper_bound}
\end{figure}


\paragraph{Finding Upper Bound on $\alpha$.} Firstly, we integrate the new differential:
\begin{align*}
    \int\frac{d\Psi}{\Psi(\alpha)} = \frac{2g(L)}{\sigma_\star^{4/L}}\int\alpha^3 \, d\alpha &\implies \mathrm{ln}\left(\frac{\Psi}{\Psi_{0}} \right)  = \frac{g(L)\alpha^4}{2\sigma_\star^{4/L}} \\
    &\implies \Psi = \Psi_{0} \exp\left( \frac{g(L)\alpha^4}{2\sigma_\star^{4/L}} \right),
\end{align*} 
where $\Psi_{0} = \underset{\alpha \to 0}{\lim} \, \Psi(\alpha) = L\sigma^{2-\frac{2}{L}}_{\star} $. Now, we can solve for $\alpha$:
\begin{align*}
    L\sigma^{2-\frac{2}{L}}_{\star}\exp\left( \frac{g(L)\alpha^4}{2\sigma_\star^{4/L}} \right) < \frac{2\sqrt{2}}{\eta} &\implies \exp\left( \frac{g(L)\alpha^4}{2\sigma_\star^{4/L}} \right) < \frac{2\sqrt{2}}{\eta L\sigma^{2-\frac{2}{L}}_{\star}} \\
    &\implies \alpha < \left( \ln\left( \frac{\frac{2\sqrt{2}}{\eta}}{L \sigma_{\star}^{2 - \frac{2}{L}}} \right) \cdot \frac{2 \sigma_{\star}^{4/L}}{g(L)} \right)^{1/4} \\
    &\implies \alpha < \left( \ln\left( \frac{2\sqrt{2}}{\eta L \sigma_{\star}^{2 - \frac{2}{L}}} \right) \cdot \frac{2 \sigma_{\star}^{4/L}}{L^2 \cdot 2^{\frac{2(L-1)}{L}}} \right)^{1/4} 
\end{align*}
Simplifying further, we obtain 
\begin{align*}
    \alpha < \left( \ln\left( \frac{2\sqrt{2}}{\eta L \sigma_{\star}^{2 - \frac{2}{L}}} \right) \cdot \frac{ \sigma_{\star}^{4/L}}{L^2 \cdot 2^{\frac{2L-3}{L}}} \right)^{1/4}, 
\end{align*}
which gives us the desired bound. This completes the proof.
\end{proof}


\begin{lemma}
\label{GFS-1}
    Let $\pi(\mbf{s}) \coloneqq \prod_{\ell=1}^L \sigma_\ell$ denote the end-to-end product of $\mbf{s} \in \mbb{R}^L$ and suppose that each $\sigma_\ell > 0$.
    If the GFS sharpness $\psi(\mbf{s}) \leq \frac{2\sqrt{1+c}}{\eta}$ for some $c \in (0, 1]$, then  
    \begin{align*}
        \sum_{i=1}^{\min\{2,L-1\}} \frac{\eta^2 (\pi(\mbf{s})-\sigma_{\star})^2 \pi^2(\mbf{s})}{\sigma_{[L-i]}^2 \sigma_{[D]}^2} \leq 1+c.
    \end{align*}
\end{lemma}

\begin{proof}

We consider two cases: (i) $\pi(\mbf{s}) \in [0, \sigma_\star)$ and (ii) $\pi(\mbf{s}) > \sigma_\star$. Note that we ignore the case of $\pi(\mbf{s}) = \sigma_\star$ as this occurs with probability zero at EoS.
%\footnote{We ignore the case $\pi(\mbf{s}(t)) =\sigma_{\star}$ when we get $b_{i,j}(t+1) = b_{i,j}(t)$. Since the occurence $\pi(\mbf{s}(t)) =\sigma_{\star}$ holds with a probability of zero. } .\\

\paragraph{Case 1 $(\pi(\mbf{s}) \in [0, \sigma_\star))$.} For this case, notice that we have
\[
\sum_{i=1}^{\min\{2,L-1\}} \frac{\eta^2 (\pi(\mbf{s}) - \sigma_{\star} )^2 \pi^2(\mbf{s})}{\sigma_{L-i}^2 \sigma_{L}^2 } 
\leq \frac{\eta^2 \pi^2(\mbf{s})}{\sigma_{L-i}^2 \sigma_{L}^2}. \tag{$\pi(\mbf{s}) < \sigma_\star$}
\]
Then, note that the GFS sharpness is constant for all weights on the GF trajectory, as it is defined to be the sharpness at the limit of the GF trajectory (i.e., the GFS). 
Hence, we can focus on the weights at the solution, or global minima.

Define the GFS as $\mbf{z} \coloneqq S_{\mathrm{GF}}(\mbf{s})$. By Lemma~\ref{gf-unbalanced}, each coordinate in $\mbf{z} \in \mbb{R}^L$ (and hence $\mbf{s} \in \mbb{R}^L$) is balanced across layers under GF, and so we have that
\begin{align*}
    \sigma^2_{\ell} - \sigma^2_{m} = z^2_{\ell} - z^2_{m} \quad\quad \forall \ell, m \in [L].
\end{align*}
Hence, it is suffices to show that 
\begin{align*}
    \sum_{i=1}^{\min\{2,L-1\}} \frac{\eta^2 \pi(\mbf{z})^2}{z_{L-i}^2 z_{L}^2} \leq 1+c \implies\sum_{i=1}^{\min\{2,L-1\}} \frac{\eta^2 \pi^2(\mbf{s})}{\sigma_{L-i}^2 \sigma_{L}^2} \leq 1+c.
\end{align*}
Then, note that $\pi(\mbf{z})=\sigma_{\star}$, since it lies on the global minima, and so we have
\begin{align}
    \sum_{i=1}^{\min\{2,L-1\}} \frac{\eta^2 \pi^2(\mbf{z})}{z^{2}_{L-i} z^2_{L}} 
= \sum_{i=1}^{\min\{2,L-1\}} \frac{\eta^2 \sigma^2_{\star}}{z^{2}_{L-i} z^2_{L}}.
\end{align}
From Lemma~\ref{1d-sharp}, the sharpness at the global minima is given as
\begin{align}
\label{eqn:helper1}
\psi(\mbf{s})=\left\| \nabla^2 \mathcal{L}(\mbf{z}) \right\| = \sum_{i=1}^{L} \frac{ \sigma^2_{\star}}{z_i^2}.
\end{align}
This immediately implies that \(\frac{\sigma^2_{\star}}{z^2_{L}} \leq \psi(\mbf{s})\) and equivalently, \(\exists \beta \in [0,1]\) such that $\frac{\sigma^2_{\star}}{z^2_{L}} = \beta \psi(\mbf{s})$.
Therefore, we have
\begin{align}
\label{eqn:helper2}
    \sum_{i=1}^{\min\{2,L-1\}} \frac{\sigma^2_{\star}}{z^2_{L-i}} \leq (1 - \beta) \psi(\mbf{s}).
\end{align}
Substituting Equations~(\ref{eqn:helper1}) and~(\ref{eqn:helper2}) into the expression we aim to bound, we obtain

\[
\sum_{i=1}^{\min\{2,L-1\}} \frac{\eta^2 (\pi(\mbf{s}) - \sigma^2_{\star})^2 \pi^2(\mbf{s})}{\sigma_{L-i}^2 \sigma_{L}^2}
= \sum_{i=1}^{\min\{2,L-1\}} \frac{\eta^2 \sigma^2_{\star}}{z^{2}_{L-i} z^2_{L}} 
\leq \eta^2 \beta (1 - \beta) \psi^2(\mbf{s}) \leq \frac{\eta^2}{4} \psi^2(\mbf{s}) \leq 1+c,
\]
where we used the fact that the maximum of $\beta(1-\beta) $ is $\frac{1}{4}$ when \(\beta = \frac{1}{2}\) and \(\psi(\mbf{s}) \leq \frac{2\sqrt{1+c}}{\eta}\).
Thus, if \(\psi(\mbf{s}) \leq \frac{2\sqrt{1+c}}{\eta}\), then for every weight $\mbf{s} \in \mbb{R}^L$ lying on its GF trajectory, we have
\begin{align*}
    \sum_{i=1}^{\min\{2,L-1\}} \frac{\eta^2 (\pi(\mbf{s}) - \sigma_{\star})^2 \pi^2(\mbf{s})}{\sigma_{L-i}^2 \sigma_{L}^2} \leq 1+c.
\end{align*}


\paragraph{Case 2 $(\pi(\mbf{s}) > \sigma_{\star})$.} Consider the case in which $\pi(\mbf{s}) > \sigma_{\star}$. 
%We already have that $\sigma>0$ throughout the trajectory (refer to Lemma 3.11 in \cite{kreisler2023gradient}) and so  $\pi(\mbf{s})>0$. 
By assumption, note that we have $\sigma_i > 0$, which implies that each GD update will also remain positive:
\begin{align*}
    \sigma_{i}-\eta (\pi(\mbf{s}) - \sigma_{\star})\pi(\mbf{s}) \frac{1}{\sigma_{i}} >0.
\end{align*}
From this, we get 
\begin{align*}
  2 >  \frac{\eta (\pi(\mbf{s}) - \sigma_{\star})\pi(\mbf{s})}{\sigma^2_{i}}>0,
\end{align*}
This implies that
$$\sum_{i=1}^{\min\{2,L-1\}} \frac{\eta^2 (\pi(\mbf{s}) - \sigma_{\star})^2 \pi^2(\mbf{s})}{\sigma_{L-i}^2 \sigma_{L}^2} \leq (1+c),$$
with $c=1$. 

Putting both cases together, we have that 
$$\sum_{i=1}^{\min\{2,L-1\}} \frac{\eta^2 (\pi(\mbf{s}) - \sigma_{\star})^2 \pi^2(\mbf{s})}{\sigma_{L-i}^2 \sigma_{L}^2} \leq (1+c),$$
for $c \in (0, 1]$, which completes the proof.

\end{proof}



\subsubsection{Proof of Proposition~\ref{prop:balancing}}
\label{sec:proof_of_balancing}

\begin{proof}
    Consider the $i$-th index of the simplified loss in~(\ref{eqn:simplified_loss}):
\begin{align*}
    \frac{1}{2} \left(\prod_{\ell=1}^L \sigma_{\ell, i} - \sigma_{\star, i}  \right)^2 \eqqcolon \frac{1}{2} \left(\prod_{\ell=1}^L \sigma_{\ell} - \sigma_{\star}  \right)^2  ,
\end{align*}
and omit the dependency on $i$ for ease of exposition. Our goal is to show that the $L$-th singular value $\sigma_L$ initialized to zero become increasingly balanced to $\sigma_\ell$ which are initialized to $\alpha > 0$.
%So, first we show that in a singular value scalar loss, for a learning rate $\eta > \frac{2}{S_{i}}$, oscillations occur provably for any initialization range $0<\alpha$. We do this in the following steps: 1) show that the sharpness around the global minima is the least when iterates are balanced, 2) By using the descent lemma, if $\eta>\frac{2}{S_{i}}$, then oscillations start occuring at the balanced point, 3) By lemma-1, if oscillations can occur at balanced point, then oscillation can occur anywhere around the global minima. So, $\eta>\frac{2}{S_{i}}$ is a sufficient condition to show oscillation. \footnote{Oscillations can also start occuring for some $\eta<\frac{2}{S_{i}}$ but will eventually subside once it reaches close to balanced minima. }
To that end, let us define the balancing dynamics between $\sigma_i$ and $\sigma_j$ as $b_{i,j}(t+1) \coloneqq \left(\sigma_i^{(t+1)}\right)^2 - \left(\sigma_j^{(t+1)}\right)^2$ and $\pi(\mbf{s}(t)) \coloneqq \prod_{\ell=1}^L \sigma_{\ell}(t)$ for the product of singular values at iteration $t$.
Then, we can simplify the balancing dynamics as such:
% \begin{align*}
%     \pi^{(t+1)} = \prod_{i=1}^{L} \sigma_{i}^{(t+1)} 
%     &= \prod_{i=1}^{D} \left(\sigma_i(t) - \eta \left(\pi(\mbf{s}(t)) - 1 \right)\frac{\pi(\mbf{s}(t))}{\sigma_i(t)} \right) \\
%     &= \pi(\mbf{s}(t)) \prod_{i=1}^{L} \left(1 - \eta \left(\pi(\mbf{s}(t)) - 1 \right) \frac{\pi(\mbf{s}(t))}{\left(\sigma_i(t)\right)^2}\right) \\
%     &= \pi(\mbf{s}(t)) + \sum_{m=1}^{L} \eta^m (1 - \pi(\mbf{s}(t)))^m \pi(\mbf{s}(t)) \sigma_m(\sigma^{(t)}).
% \end{align*}

% Defining  to be the balance,
% The dynamic of the balances then becomes:
\begin{align}
    b_{i,j}(t+1) &= \left(\sigma_i(t+1)\right)^2 - \left(\sigma_j(t+1)\right)^2 \\
    &= \left(\sigma_i(t) - \eta\left(\pi(\mbf{s}(t)) - \sigma_{\star}\right)\frac{\pi(\mbf{s}(t))}{\sigma_i(t)}\right)^2 - \left(\sigma_j(t) - \eta\left(\pi(\mbf{s}(t)) - \sigma_{\star}\right)\frac{\pi(\mbf{s}(t))}{\sigma_j(t)}\right)^2 \\
    &= \left(\sigma_i(t)\right)^2 - \left(\sigma_j(t)\right)^2 + \eta^2 \left(\pi(\mbf{s}(t)) - \sigma_{\star}\right)^2 \left(\frac{\pi^2(\mbf{s}(t))}{\left(\sigma_i(t)\right)^2} - \frac{\pi^2(\mbf{s}(t))}{\left(\sigma_j(t)\right)^2}\right) \\
    &= \left(\left(\sigma_i(t)\right)^2 - \left(\sigma_j(t)\right)^2 \right) \left( 1 - \eta^2 (\pi(\mbf{s}(t)) - \sigma_{\star})^2 \frac{\pi^2(\mbf{s}(t))}{\left(\sigma_i(t)\right)^2 \left(\sigma_j(t)\right)^2} \right) \\
    \label{eqn:simplified_balanced}
    &= b_{i,j}(t) \left( 1 - \eta^2 (\pi(\mbf{s}(t)) - \sigma_{\star})^2 \frac{\pi^2(\mbf{s}(t))}{\left(\sigma_i(t)\right)^2 \left(\sigma_j(t)\right)^2} \right).
\end{align}
Then, in order to show that $ \left|b_{i,j}(t+1)\right| < c\left|b_{i,j}(t)\right|$ for some $0<c \leq 1$, we need to prove that
\begin{align*}
    \left | 1 - \eta^2 (\pi(\mbf{s}(t)) -\sigma_{\star} )^2 \frac{\pi^2(\mbf{s}(t))}{\left(\sigma_i(t)\right)^2 \left(\sigma_j(t)\right)^2} \right| < c,
\end{align*}
 for all iterations $t$. Note that for our case, it is sufficient to show the result for $i=L$ and $j=\ell$ for any $\ell \neq L$. WLOG, suppose that the $\sigma$ are sorted such that $\sigma_1 \geq \sigma_2 \geq \ldots \geq \sigma_L$. By assumption, since our initialization scale satisfies
\begin{align*}
    0<\alpha < \left( \ln\left( \frac{2\sqrt{2}}{\eta L \sigma_{\star}^{2 - \frac{2}{L}}} \right) \cdot \frac{ \sigma_{\star}^{4/L}}{L^2 \cdot 2^{\frac{2L-3}{L}}} \right)^{1/4},
\end{align*}
by Lemma~\ref{GFS-3}, we have that the GFS sharpness $\psi(\cdot)$ for positive $\mbf{s} = \begin{bmatrix}
    \sigma_1 & \ldots & \sigma_L
\end{bmatrix} \in \mbb{R}^L$ (i.e., each element $\sigma_\ell > 0$) satisfies $\psi(\mbf{s}) < \frac{2\sqrt{2}}{\eta}$. Then, by Lemma~\ref{GFS-1}, we have
\begin{align}
\label{eqn:conseq_gfs1}
    \sum_{i=1}^{\min\{2,L-1\}} \frac{\eta^2 (\pi(\mbf{s})-\sigma_{\star})^2 \pi^2(\mbf{s})}{\sigma_{[L-i]}^2 \sigma_{[D]}^2} \leq 1+c,
\end{align}
for some $c \in [0, 1)$. Then, notice that Equation~(\ref{eqn:conseq_gfs1}) implies that \begin{align}
        \frac{\eta^2 (\pi(\mbf{s}) - \sigma_{\star})^2 \pi^2(\mbf{s})}{\sigma_{L-1}^2 \sigma_{L}^2} < 1+c \quad \text{and} \quad 
        \frac{\eta^2 (\pi(\mbf{s}) - \sigma_{\star})^2 \pi^2(\mbf{s})}{\sigma_{i}^2 \sigma_{j}^2} < \frac{1+c}{2},
\end{align}
for all $i \in [L]$, $j \in [L-2] $ and $ i < j$. Notice that the latter inequality comes from the fact that 
\begin{align*}
    \frac{\eta^2 (\pi(\mbf{s}) - \sigma_{\star})^2 \pi^2(\mbf{s})}{\sigma_{L-2}^2 \sigma_{L}^2} + \frac{\eta^2 (\pi(\mbf{s}) - \sigma_{\star})^2 \pi^2(\mbf{s})}{\sigma_{L-2}^2 \sigma_{L}^2} &< \frac{\eta^2 (\pi(\mbf{s}) - \sigma_{\star})^2 \pi^2(\mbf{s})}{\sigma_{L-1}^2 \sigma_{L}^2} + \frac{\eta^2 (\pi(\mbf{s}) - \sigma_{\star})^2 \pi^2(\mbf{s})}{\sigma_{L-2}^2 \sigma_{L}^2} \\
    &< 1+c,
\end{align*}
which implies that
\begin{align*}
    2\frac{\eta^2 (\pi(\mbf{s}) - \sigma_{\star})^2 \pi^2(\mbf{s})}{\sigma_{L-2}^2 \sigma_{L}^2} < 1+c \implies \frac{\eta^2 (\pi(\mbf{s}) - \sigma_{\star})^2 \pi^2(\mbf{s})}{\sigma_{L-2}^2 \sigma_{L}^2} < \frac{1+c}{2},
\end{align*}
and since $\sigma$ are sorted, it holds for all other $\sigma$.
Therefore from Equation~(\ref{eqn:simplified_balanced}), we have for all $i \in [L-2]$,
\begin{align*}
    b_{i,i+1}(t+1) < c \cdot b_{i,i+1}(t) \quad \text{and} \quad b(t+1)_{L-2,L} < c \cdot b_{L-2,L}(t),
\end{align*}
as well as 
\begin{align*}
    -c \cdot b_{L-1,L} (t) <b_{L-1,L}(t+1) < c \cdot b_{L-1,L}(t).
\end{align*}
%Now, we prove balancing considering two cases where $b(t+1)_{L-1,L} \geq 0 $ and $b(t+1)_{L-1,L} \leq 0 $. 
%First, consider the case $b(t+1)_{L-1,L} \geq 0 $ then using the fact for all $i \in [L-2]$,
%$b(t+1)_{i,i+1} <  b(t)_{i,i+1}$, $b(t+1)_{i,i+1} <  b(t)_{i,i+1}$.
Then, notice that since we initialized all of the singular values $\sigma_\ell$ for $\ell \in [L-1]$ to be the same, they follow the same dynamics. Since we already showed that $|b_{L-1,L}(t+1)| < c \cdot|b_{L-1,L}(t)|$, it must follow that
\begin{align*}
    \left|b_{i,j}(t+1)\right| < c \cdot \left|b_{i,j}(t)\right|, \quad \forall \, i,j \in [L].
\end{align*}
This completes the proof.


\end{proof}




\subsection{Proofs for Periodic Orbits}

Before presenting our proof for Theorem~\ref{thm:align_thm}, we first show that the required condition from~\cite{chen2023edge} for stable oscillations to occur (see Lemma~\ref{lemma:chen-bruna}) is also satisfied for DLNs beyond the EOS, as shown in Appendix~\ref{subsec:supporting_lemmas_orbits}.

\subsubsection{Supporting Lemmas}
\label{subsec:supporting_lemmas_orbits}

\begin{lemma}
[Stable Subspace Oscillations]

Define $S_p\coloneqq L \sigma^{2-\frac{2}{L}}_{\star,p}$ and $K'_p \coloneqq \mathrm{max} \left\{ S_{p+1},\frac{S_p}{2\sqrt{2}}\right\}$.
%If we run GD on the deep matrix factorization loss with initialization scale $\alpha < \alpha'$ from Theorem~\ref{prop:balancing} and learning rate $\eta = \frac{2}{K}$, where $K'_p < K< S_p$, then at the steady state limit $t \rightarrow \infty$, 
If we run GD on the deep matrix factorization loss in~(\ref{eqn:deep_mf}) with learning rate $\eta = \frac{2}{K}$, where $K'_p < K< S_p$, then $2$-period orbit oscillation occurs in the direction of $\Delta_{S_p}$, where $\Delta_{S_p}$ denotes the eigenvector associated with the eigenvalue $S_p$ of the Hessian at the balanced minimum.
    \label{thm:stable_sub}
\end{lemma}
\begin{proof}

Define $f_{\Delta_i}$ as the 1-D function at the cross section of the loss landscape and the line
following the direction of $\Delta_i$ passing the (balanced) minima, where $\Delta_i$ is the $i$-th eigenvector of the training loss Hessian at the balanced minimum. To prove the result, we will invoke Lemma~\ref{lemma:chen-bruna}, which states that two-period orbit oscillation occurs in the direction of $\Delta_i$ if the minima of $f_{\Delta_i}$ satisfies $f_{\Delta_i}^{(3)}>0$ and $3[f_{\Delta_i}^{(3)}]^2 - f_{\Delta_i}^{(2)}f_{\Delta_i}^{(4)} > 0$, for $\eta>\frac{2}{\lambda_{i}}$. We show that while the condition holds for all of the eigenvector directions, the oscillations can only occur specifically in the directions of $\Delta_{S_i}$.
%Recall that the initialization condition is an artifact of Proposition~\ref{prop:balancing}, which allows us to consider the balanced minimum. 


First, we will derive the eigenvectors of the Hessian of the training loss at convergence (i.e., $\mbf{M}_\star = \mbf{W}_{L:1}$).
    To obtain the eigenvectors of the Hessian of parameters $(\mbf{W}_L, \ldots, \mbf{W}_2, \mbf{W}_1)$, consider a small perturbation of the parameters:
    \begin{align*}
        \mbf{\Theta} \coloneqq \left(\Delta \mbf{W}_\ell +  \mbf{W}_\ell \right)_{\ell=1}^L =  (\mbf{W}_L + \Delta \mbf{W}_L, \ldots, \mbf{W}_2+ \Delta \mbf{W}_2, \mbf{W}_1+ \Delta \mbf{W}_1).
    \end{align*}

    
    Given that $\mbf{W}_{L:1} = \mbf{M}_\star$, consider and evaluate the loss function at this minima: 
    \begin{align}
        \mathcal{L}(\mbf{\Theta}) = \frac{1}{2} \biggl\| &\sum_{\ell} \mbf{W}_{L:\ell+1} \Delta \mbf{W}_\ell \mbf{W}_{\ell-1:1} \\
        &+ \sum_{\ell<m} \mbf{W}_{L:\ell+1} \Delta \mbf{W}_\ell \mbf{W}_{\ell-1:m+1} \Delta \mbf{W}_{m} \mbf{W}_{m-1:1}   +   \ldots   +  \Delta \mbf{W}_{L:1}\biggr\|^2_{\mathsf{F}}.
    \end{align}
    By expanding each of the terms and splitting by the orders of $\Delta \mbf{W}_\ell$ (perturbation), we get that the second-order term is equivalent to
    \begin{align*}
        \Theta&\left(\sum_{\ell=1}^L\|\Delta \mbf{W}_\ell\|^2\right): \,\, \frac{1}{2} \biggl\| \sum_{\ell} \mbf{W}_{L:\ell+1} \Delta \mbf{W}_\ell \mbf{W}_{\ell-1:1} \biggr\|^2_{\mathsf{F}} \\
        \Theta&\left(\sum_{\ell=1}^L\|\Delta \mbf{W}_\ell\|^3\right): \,\, \mathrm{tr}\left[\left(\sum_{\ell} \mbf{W}_{L:\ell+1} \Delta \mbf{W}_\ell \mbf{W}_{\ell-1:1} \right)^\top \left( \sum_{\ell<m} \mbf{W}_{L:\ell+1} \Delta \mbf{W}_\ell \mbf{W}_{\ell-1:m+1} \Delta \mbf{W}_{m} \mbf{W}_{m-1:1} \right)\right] \\
        \Theta&\left(\sum_{\ell=1}^L\|\Delta \mbf{W}_\ell\|^4\right): \,\, \frac{1}{2} \| \sum_{\ell<m} \mbf{W}_{L:\ell+1} \Delta \mbf{W}_\ell \mbf{W}_{\ell-1:m+1} \Delta \mbf{W}_{m} \mbf{W}_{m-1:1}\|^2_{\mathsf{F}}\\
        &+ \mathrm{tr}\left[\sum_{l} \left(\mbf{W}_{L:\ell+1}\Delta \mbf{W}_\ell \mbf{W}_{\ell-1:1} \right)^\top \left(\sum_{l<m<p} \mbf{W}_{L:\ell+1} \Delta \mbf{W}_\ell \mbf{W}_{\ell-1:m+1} \Delta \mbf{W}_{m} \mbf{W}_{m-1:p+1} \Delta \mbf{W}_{p} \mbf{W}_{p-1:1} \right)\right]
    \end{align*}

The direction of the steepest change in the loss at the minima correspond to the largest eigenvector direction of the Hessian. Since higher order terms such as $ \Theta\left(\sum_{\ell=1}^L\|\Delta \mbf{W}_\ell\|^3\right)$ are insignifcant compared to the second order terms $  \Theta\left(\sum_{\ell=1}^L\|\Delta \mbf{W}_\ell\|^2\right)$, finding the direction that maximizes the second order term leads to finding the eigenvector of the Hessian.
    Then, the eigenvector corresponding to the maximum eigenvalue of  $\nabla^2 \mathcal{L}$ is the solution of 
    \begin{align}
        \Delta_{1} \coloneqq \mathrm{vec}(\Delta \mbf{W}_L, \ldots \Delta \mbf{W}_1) = \underset{\|\Delta \mbf{W}_L\|^2_{\mathsf{F}} + \ldots + \|\Delta \mbf{W}_1\|^2_{\mathsf{F}} = 1}{\mathrm{arg max}} \, f\left(\Delta \mbf{W}_L, \ldots, \Delta \mbf{W}_1 \right),\label{max-eig}
    \end{align}
    where 
    \begin{align}
        f(\Delta \mbf{W}_L, \ldots, \Delta \mbf{W}_1) \coloneqq \frac{1}{2} \|\Delta \mbf{W}_L \mbf{W}_{L-1:1} + \ldots + \mbf{W}_{L:3}\Delta \mbf{W}_2 \mbf{W}_{1} + \mbf{W}_{L:2} \Delta \mbf{W}_1\|^2_{\mathsf{F}}.
    \end{align}

While the solution of Equation~(\ref{max-eig}) gives the maximum eigenvector direction of the Hessian, $\Delta_{1}$, the other eigenvectors can be found by solving
\begin{align}
    \Delta_{r} \coloneqq  \underset{
    \substack{
    \|\Delta \mbf{W}_L\|^2_{\mathsf{F}} + \ldots + \|\Delta \mbf{W}_1\|^2_{\mathsf{F}} = 1, \\
    \Delta_{r}\perp \Delta_{r-1},.., \Delta_{r}\perp \Delta_{1}
    }
    }{\mathrm{argmax}} \, f\left(\Delta \mbf{W}_L, \ldots, \Delta \mbf{W}_1 \right).\label{other-eig}
\end{align}

    By expanding $f(\cdot)$, %subject to the constraint ${\|\Delta \mbf{W}_L\|^2_{\mathsf{F}}} + \ldots + \|\Delta \mbf{W}_1\|^2_{\mathsf{F}} = 1$ ,
    we have that % we don't expand subject to a constraint
    \begin{align}
        f(\Delta \mbf{W}_L, &\ldots, \Delta \mbf{W}_1) = \|\Delta\mbf{W}_L \mbf{W}_{L-1:1}\|^2_{\mathsf{F}} +\ldots+ \|\mbf{W}_{L:3}\Delta \mbf{W}_2 \mbf{W}_{1}\|^2_{\mathsf{F}}  + \|\mbf{W}_{L:2} \Delta \mbf{W}_1\|^2_{\mathsf{F}} \notag \\
        &+ \mathrm{tr}\left[\left(\Delta\mbf{W}_L \mbf{W}_{L-1:1} \right)^\top \left(\mbf{W}_{L:3}\Delta \mbf{W}_2 \mbf{W}_{1} + \ldots +\mbf{W}_{L:2} \Delta \mbf{W}_1 \right)\right] + \ldots + \notag \\
        &\mathrm{tr}\left[\left(\mbf{W}_{L:2} \Delta \mbf{W}_1\right)^\top \left(\mbf{W}_{L:3}\Delta \mbf{W}_2 \mbf{W}_{1} + \ldots +\mbf{W}_{L:3}\Delta \mbf{W}_2 \mbf{W}_{1} \right)\right].     \label{expansion}
    \end{align}

We can solve Equation~(\ref{max-eig}) by maximizing each of the terms, which can be done in two steps:
\begin{enumerate}[label=(\roman*)]
\item 
Each Frobenius term in the expansion is maximized when the left singular vector of $\Delta \mbf{W}_{\ell}$ aligns with $\mbf{W}_{L:\ell+1}$ and the right singular vector aligns with $\mbf{W}_{\ell-1:1}$. This is a result of Von Neumann's trace inequality~\citep{mirsky1975trace}. Similarly, each term in the trace is maximized when the singular vector of the perturbations align with the products. 
\item Due to the alignment, Equation~(\ref{max-eig}) can be written in just the singular values. Let $\Delta s_{\ell, i}$ denote the $i$-th singular value of the perturbation matrix $\Delta\mbf{W}_\ell$. Recall that all of the singular values of $\mbf{M}_\star$ are distinct (i.e., $\sigma_{\star, 1} > \ldots>\sigma_{\star, r}$). Hence, it is easy to see that
Equation~(\ref{max-eig}) is maximized when $\Delta s_{\ell,i} = 0$ (i.e, all the weight goes to $\Delta s_{\ell,1}$). Thus, each perturbation matrix must be rank-$1$.
\end{enumerate}
Now since each perturbation is rank-$1$, we can write each perturbation as 
    \begin{align}
        \Delta\mbf{W}_{\ell} = \Delta s_{\ell} \Delta \mbf{u}_{\ell} \Delta\mbf{v}_\ell^\top, \quad \forall \ell \in [L],
    \end{align}
    % Then, notice that since the top-$r$ singular values of $\mbf{M}_\star$ are unique (i.e., $\sigma_{\star, 1} > \ldots > \sigma_{\star, r}$), the singular values of each $\mbf{W}_{\ell}$ are also unique, and hence the solution for each $\Delta\mbf{W}_{\ell}$ will be rank-$1$. We can express each solution as \
    for $\Delta s_{\ell} > 0$ and orthonormal vectors $\Delta \mbf{u}_{\ell} \in \mbb{R}^d$ and  $\Delta \mbf{v}_{\ell} \in \mbb{R}^d$ with $\sum_{\ell=1}^L \Delta s^2_{\ell} = 1$.
    Plugging this in each term, we obtain:
   \begin{align*}
        \|\mbf{W}_{L:\ell+1} \Delta_1 \mbf{W}_{\ell} \mbf{W}_{\ell-1:1}\|_2^2 = \Delta_1 s_\ell^2\cdot \biggl\|\underbrace{\mbf{V}_\star \mbf{\sigma}^{\frac{L-\ell}{L}}_\star \mbf{V}^\top_\star \Delta \mbf{u}_\ell}_{\eqqcolon \mbf{a}}\underbrace{ \Delta \mbf{v}_\ell^\top \mbf{V}_\star \mbf{\sigma}^{\frac{\ell-1}{L}}_\star \mbf{V}^\top_\star}_{\eqqcolon \mbf{b}^\top}\biggr\|_2^2.
    \end{align*}

    Since alignment maximizes this expression as discussed in first point, we have:
        % We first derive the leading eigenvector of the Hessian, denoted by $\Delta_1$. This makes repeated use of Von-Neumann trace inequality~, in that $\Delta \mbf{W}_\ell$ will have their singular vectors align.
    

    % To maximize $f(\cdot)$, we can find $\Delta \mbf{W}_\ell$ that maximizes each term in its expansion. Consider the Frobenius norm terms (e.g., $\|\mbf{W}_{L:\ell+1} \Delta \mbf{W}_{\ell} \mbf{W}_{\ell-1:1}\|^2_{\mathsf{F}}$). Since each $\Delta\mbf{W}_\ell$ solution is rank-$1$, it follows that each one of the matrices within the norm must also be at most rank-$1$. Then, we have
    % \begin{align*}
    %     \|\mbf{W}_{L:\ell+1} \Delta_1 \mbf{W}_{\ell} \mbf{W}_{\ell-1:1}\|^2_{\mathsf{F}} = \|\mbf{W}_{L:\ell+1} \Delta_1 \mbf{W}_{\ell} \mbf{W}_{\ell-1:1}\|^2_{2} &\leq \|\mbf{W}_{L:\ell+1}\|_2^2 \cdot \|\Delta_1 \mbf{W}_{\ell}\|_2^2 \cdot \|\mbf{W}_{\ell-1:1}\|_2^2 \\
    %     &= \Delta_1 s_{\ell}^{2} \cdot \sigma_{\star, 1}^{2 - \frac{2}{L}}.
    % \end{align*}
    % To obtain this upper bound, consider the following:
    % \begin{align*}
    %     \|\mbf{W}_{L:\ell+1} \Delta_1 \mbf{W}_{\ell} \mbf{W}_{\ell-1:1}\|_2^2 = \Delta_1 s_\ell^2\cdot \biggl\|\underbrace{\mbf{V}_\star \mbf{\sigma}^{\frac{L-\ell}{L}}_\star \mbf{V}^\top_\star\mbf{u}_\ell}_{\eqqcolon \mbf{a}}\underbrace{\mbf{v}_\ell^\top \mbf{V}_\star \mbf{\sigma}^{\frac{\ell-1}{L}}_\star \mbf{V}^\top_\star}_{\eqqcolon \mbf{b}^\top}\biggr\|_2^2.
    % \end{align*}
     $\mbf{u}_\ell =\mbf{v}_\ell = \mbf{v}_{\star, 1}$ for all $\ell \in [2, L-1]$, then
    \begin{align*}
        \mbf{a} = \sigma_{\star, 1}^{\frac{L-\ell}{L}}\mbf{v}_{\star, 1} \quad \text{and} \quad \mbf{b}^\top = \sigma_{\star, 1}^{\frac{\ell - 1}{L}}\mbf{v}_{\star, 1}^\top \implies \mbf{ab}^\top = \sigma_{\star, 1}^{1 - \frac{1}{L}} \cdot \mbf{v}_{\star, 1}\mbf{v}_{\star, 1}^\top.
    \end{align*}
    The very same argument can be made for the trace terms.
    Hence, in order to maximize $f(\cdot)$, we must have
    \begin{align*}
        \mbf{v}_L &= \mbf{v}_{\star, 1}, \quad \text{and} \quad \mbf{u}_1 = \mbf{v}_{\star, 1}, \\
        \mbf{u}_\ell &= \mbf{v}_\ell = \mbf{v}_{\star, 1}, \quad \forall \ell \in [2, L-1].
    \end{align*}
    To determine $\mbf{u}_L$ and $\mbf{v}_1$, we can look at one of the trace terms:
    \begin{align*}
    \mathrm{tr}\left[\left(\Delta_1\mbf{W}_L \mbf{W}_{L-1:1} \right)^\top \left(\mbf{W}_{L:3}\Delta_1 \mbf{W}_2 \mbf{W}_{1} + \ldots +\mbf{W}_{L:2} \Delta_1 \mbf{W}_1 \right)\right] \leq \left(\frac{L-1}{L} \right)\cdot\sigma_{\star, 1}^{2 - \frac{2}{L}}.
    \end{align*}
    To reach the upper bound, we require $\mbf{u}_L = \mbf{u}_{\star, 1}$ and $\mbf{v}_1 = \mbf{v}_{\star, 1}$. Finally, as the for each index, the singular values are balanced, we will have  $\Delta_1 s_{\ell} = \frac{1}{\sqrt{L}}$ for all $\ell \in [L]$ to satisfy the constraint. Finally, we get that the leading eigenvector is
    \begin{align*}
        \Delta_1 \coloneqq \mathrm{vec}\left(\frac{1}{\sqrt{L}}\mbf{u}_1 \mbf{v}_1^\top, \frac{1}{\sqrt{L}}\mbf{v}_1 \mbf{v}_1^\top, \ldots, \frac{1}{\sqrt{L}}\mbf{v}_1 \mbf{v}_1^\top \right).
    \end{align*}
    Notice that we can also verify that $f(\Delta_1) = L\sigma_{\star, 1}^{2- \frac{2}{L}}$, which is the leading eigenvalue (or sharpness) derived in Lemma~\ref{lemma:hessian_eigvals}.  

    To derive the remaining eigenvectors, we need to find all of the vectors in which $\Delta_i^\top \Delta_j = 0$ for $i\neq j$, where
    \begin{align*}
        \Delta_i = \mathrm{vec}(\Delta_i \mbf{W}_L, \ldots \Delta_i \mbf{W}_1),
    \end{align*}
    and $f(\Delta_i) = \lambda_i$, where $\lambda_i$ is the $i$-th largest eigenvalue. By repeating the same process as above, we find that the eigenvector-eigenvalue pair as follows:
    \begin{align*}
        \Delta_1 &= \mathrm{vec}\left(\frac{1}{\sqrt{L}}\mbf{u}_1 \mbf{v}_1^\top, \frac{1}{\sqrt{L}}\mbf{v}_1 \mbf{v}_1^\top, \ldots, \frac{1}{\sqrt{L}}\mbf{v}_1 \mbf{v}_1^\top \right) , \quad\lambda_{1} =  L\sigma_{\star, 1}^{2- \frac{2}{L}} \\
        \Delta_2 &= \mathrm{vec}\left(\frac{1}{\sqrt{L}}\mbf{u}_1 \mbf{v}_2^\top, \frac{1}{\sqrt{L}}\mbf{v}_1 \mbf{v}_2^\top, \ldots, \frac{1}{\sqrt{L}}\mbf{v}_1 \mbf{v}_2^\top \right), \quad\lambda_{2} =  \left(\sum_{i=0}^{L-1} \sigma_{\star, 1}^{1-\frac{1}{L}-\frac{1}{L}i} \cdot \sigma_{\star, 2}^{\frac{1}{L}i} \right) \\
        \Delta_3 &= \mathrm{vec}\left(\frac{1}{\sqrt{L}}\mbf{u}_2 \mbf{v}_1^\top, \frac{1}{\sqrt{L}}\mbf{v}_2 \mbf{v}_1^\top, \ldots, \frac{1}{\sqrt{L}}\mbf{v}_2 \mbf{v}_1^\top \right), \quad\lambda_{3} =  \left(\sum_{i=0}^{L-1} \sigma_{\star, 1}^{1-\frac{1}{L}-\frac{1}{L}i} \cdot \sigma_{\star, 2}^{\frac{1}{L}i} \right) \\
        \quad\quad\quad&\vdots \\
         \Delta_d &= \mathrm{vec}\left(\frac{1}{\sqrt{L}}\mbf{u}_2 \mbf{v}_2^\top, \frac{1}{\sqrt{L}}\mbf{v}_2 \mbf{v}_2^\top, \ldots, \frac{1}{\sqrt{L}}\mbf{v}_2 \mbf{v}_2^\top \right), \quad\lambda_{d} =  L\sigma_{\star, 2}^{2- \frac{2}{L}} \\     
        \quad\quad\quad&\vdots \\
        \Delta_{dr+r} &= \mathrm{vec}\left(\frac{1}{\sqrt{L}}\mbf{u}_d \mbf{v}_r^\top, \frac{1}{\sqrt{L}}\mbf{v}_d \mbf{v}_r^\top, \ldots, \frac{1}{\sqrt{L}}\mbf{v}_d \mbf{v}_r^\top \right),
    \end{align*}
    which gives a total of $dr + r$ eigenvectors.

    Second, equipped with the eigenvectors, let us consider the 1-D function $f_{\Delta_i}$ generated by the cross-section of the loss landscape and each eigenvector $\Delta_i$ passing the minima:
    \begin{align*}
        f_{\Delta_i}(\mu) &= \mathcal{L}(\mbf{W}_L + \mu\Delta \mbf{W}_L, \ldots, \mbf{W}_2+ \mu\Delta \mbf{W}_2, \mbf{W}_1+ \mu\Delta \mbf{W}_1), \\
        &= \mu^2\cdot \frac{1}{2} \|\Delta \mbf{W}_L \mbf{W}_{L-1:1} + \ldots + \mbf{W}_{L:3}\Delta \mbf{W}_2 \mbf{W}_{1} + \mbf{W}_{L:2} \Delta \mbf{W}_1\|^2_{\mathsf{F}}\\
        \quad&+\mu^3 \cdot \sum_{\ell=1, \ell< m}^L\mathrm{tr}\left[\left(\mbf{W}_{L:\ell+1}\Delta \mbf{W}_\ell \mbf{W}_{\ell-1:1} \right)^\top \left( \mbf{W}_{L:\ell+1} \Delta \mbf{W}_\ell \mbf{W}_{\ell-1:m+1} \Delta \mbf{W}_{m} \mbf{W}_{m-1:1} \right)\right] \\
        \quad&+\mu^4\cdot \frac{1}{2} \left\|  \left( \sum_{\ell<m} \mbf{W}_{L:\ell+1} \Delta \mbf{W}_\ell \mbf{W}_{\ell-1:m+1} \Delta \mbf{W}_{m} \mbf{W}_{m-1:1} \right) \right\|^2_{\mathsf{F}}\\
        &+ \mu^4 \cdot\sum_{\ell<m<p} ^L \mathrm{tr}\left[\left(\mbf{W}_{L:\ell+1}\Delta \mbf{W}_\ell \mbf{W}_{\ell-1:1} \right)^\top  \left(\mbf{W}_{L:\ell+1} \Delta \mbf{W}_\ell \mbf{W}_{\ell-1:m+1} \Delta \mbf{W}_{m} \mbf{W}_{m-1:p+1} \Delta \mbf{W}_{p} \mbf{W}_{p-1:1} \right)\right].
    \end{align*}
    Then, the several order derivatives of $f_{\Delta_i}(\mu)$ at $\mu = 0$ can be obtained from Taylor expansion as
    \begin{align*}
        f_{\Delta_i}^{(2)}(0) &= \|\Delta_i \mbf{W}_L \mbf{W}_{L-1:1} + \ldots + \mbf{W}_{L:3}\Delta_i \mbf{W}_2 \mbf{W}_{1} + \mbf{W}_{L:2} \Delta_i \mbf{W}_1\|^2_{\mathsf{F}} = \lambda^2_{i}\\
        f_{\Delta_i}^{(3)}(0) &= 6\sum_{\ell=1}^L\mathrm{tr}\left[\left(\mbf{W}_{L:\ell+1}\Delta_i \mbf{W}_\ell \mbf{W}_{\ell-1:1} \right)^\top \left(\mbf{W}_{L:\ell+2}\Delta_i \mbf{W}_{\ell+1} \mbf{W}_\ell\Delta_i\mbf{W}_{\ell-1} \mbf{W}_{\ell-2:1} \right)\right] \\
        & = 6 \biggl\| \sum_{\ell} \mbf{W}_{L:\ell+1}\Delta_i \mbf{W}_\ell \mbf{W}_{\ell-1:1} \biggr\|_{\mathsf{F}}\cdot \biggl\| \left( \sum_{\ell<m} \mbf{W}_{L:\ell+1} \Delta \mbf{W}_\ell \mbf{W}_{\ell-1:m+1} \Delta \mbf{W}_{m} \mbf{W}_{m-1:1} \right) \biggr\|_{\mathsf{F}}\\
        & \coloneqq 6  \lambda_{i}\cdot\beta_{i} \\
        f_{\Delta_i}^{(4)}(0) &= 12\|\Delta_i \mbf{W}_L \Delta_i\mbf{W}_{L-1}\mbf{W}_{L-2:1} + \ldots + \mbf{W}_{L:4}\Delta_i \mbf{W}_3 \mbf{W}_{2}\Delta_i\mbf{W}_1 + \mbf{W}_{L:3}\Delta_i\mbf{W}_2 \Delta_i \mbf{W}_1\|^2_{\mathsf{F}} \\
        &+ 24\sum_{\ell=1}^L \mathrm{tr}\left[\left(\mbf{W}_{L:\ell+1}\Delta_i \mbf{W}_\ell \mbf{W}_{\ell-1:1} \right)^\top \left(\sum_{l<m<p} \mbf{W}_{L:\ell+1} \Delta \mbf{W}_\ell \mbf{W}_{\ell-1:m+1} \Delta \mbf{W}_{m} \mbf{W}_{m-1:p+1} \Delta \mbf{W}_{p} \mbf{W}_{p-1:1} \right)\right] \\
        &\coloneqq 12\beta^2_{i} + 24\lambda_{i}\cdot\delta_{i},
    \end{align*}
  
    % \begin{align*}
    %     3[f_{\Delta_i}^{(3)}]^2 - f_{\Delta_i}^{(2)}f_{\Delta_i}^{(4)} &= 108\beta\|\Delta_i \mbf{W}_L \mbf{W}_{L-1:1} + \ldots + \mbf{W}_{L:3}\Delta_i \mbf{W}_2 \mbf{W}_{1} + \mbf{W}_{L:2} \Delta_i \mbf{W}_1\|^2_{\mathsf{F}} \\
    %     &- 12\beta\|\Delta_i \mbf{W}_L \mbf{W}_{L-1:1} + \ldots + \mbf{W}_{L:3}\Delta_i \mbf{W}_2 \mbf{W}_{1} + \mbf{W}_{L:2} \Delta_i \mbf{W}_1\|^2_{\mathsf{F}} \\
    %     &- 24\gamma\|\Delta_i \mbf{W}_L \mbf{W}_{L-1:1} + \ldots + \mbf{W}_{L:3}\Delta_i \mbf{W}_2 \mbf{W}_{1} + \mbf{W}_{L:2} \Delta_i \mbf{W}_1\|^2_{\mathsf{F}} \\
    %     &= 96\beta\|\Delta_i \mbf{W}_L \mbf{W}_{L-1:1} + \ldots + \mbf{W}_{L:3}\Delta_i \mbf{W}_2 \mbf{W}_{1} + \mbf{W}_{L:2} \Delta_i \mbf{W}_1\|^2_{\mathsf{F}} \\
    %     &- 24\gamma\|\Delta_i \mbf{W}_L \mbf{W}_{L-1:1} + \ldots + \mbf{W}_{L:3}\Delta_i \mbf{W}_2 \mbf{W}_{1} + \mbf{W}_{L:2} \Delta_i \mbf{W}_1\|^2_{\mathsf{F}}.
    % \end{align*}

    where we defined 
    \begin{align*}
        & \lambda_{i} = \biggl\| \sum_{\ell} \mbf{W}_{L:\ell+1} \Delta_{i} \mbf{W}_\ell \mbf{W}_{\ell-1:1}  \biggr\|_{\mathsf{F}} \quad \tag{Total $L\choose 1$ terms}\\
        & \beta_{i} =\biggl\| \left( \sum_{\ell<m} \mbf{W}_{L:\ell+1} \Delta \mbf{W}_\ell \mbf{W}_{\ell-1:m+1} \Delta \mbf{W}_{m} \mbf{W}_{m-1:1} \right)\biggr\|_{\mathsf{F}} \quad \tag{Total $L\choose 2$ terms}\\
        & \delta_{i} = \biggl\| \left(\sum_{l<m<p} \mbf{W}_{L:\ell+1} \Delta \mbf{W}_\ell \mbf{W}_{\ell-1:m+1} \Delta \mbf{W}_{m} \mbf{W}_{m-1:p+1} \Delta \mbf{W}_{p} \mbf{W}_{p-1:1} \right) \biggr\|_{\mathsf{F}}, \quad \tag{Total $L\choose 3$ terms}
    \end{align*}
and used the fact that $\mathrm{tr}(\mbf{A}^\top \mbf{B}) = \| \mbf{A} \|_{\mathsf{F}}\cdot \| \mbf{B}\|_{\mathsf{F}}$ under singular vector alignment.

Then, since $\beta_{i} $ has $L\choose 2$ terms inside the sum, when the Frobenium term is expanded, it will have $\frac{{L\choose 2}\left({L\choose 2}+1\right)}{2}$ number of terms. 
Under alignment and balancedness, $\beta^{2}_{i} = \Delta s^2_{\ell} \sigma^{2-\frac{4}{L}}_{i}  \times \frac{{L\choose 2}\left({L\choose 2}+1\right)}{2}$ and $\lambda_{i} \delta_{i} =  \Delta s^2_{\ell} \sigma^{2-\frac{4}{L}}_{i}  \times {L\choose 3} L$. Thus, we have the expression
\begin{align*}
   2\beta^{2}_{i} - \lambda_{i} \delta_{i} &= \Delta s^2_{\ell} \sigma^{2-\frac{4}{L}}_{i} \left( 2 \frac{\binom{L}{2}\left(\binom{L}{2} + 1\right)}{2} -  \binom{L}{3} L \right) \\
   &= \Delta s^2_{\ell} \sigma^{2-\frac{4}{L}}_{i}  \binom{L}{3} L \times \left( \frac{3\left(\frac{L(L-1)}{2}+1\right)}{L(L-2)} -1 \right) \\
   & =  \Delta s^2_{\ell} \sigma^{2-\frac{4}{L}}_{i} \frac{2 \binom{L}{3} L }{L(L-2)} \times \left( (L-1)^2  + 5\right) > 0,\\
\end{align*}
for any depth $L>2$. Finally, the condition of stable oscillation of 1-D function is
\begin{align*}
       &  3[f_{\Delta_i}^{(3)}]^2 - f_{\Delta_i}^{(2)}f_{\Delta_i}^{(4)} =   108 \lambda^2_{i}\beta^{2}_{i} -  ( \lambda^{2}_{i})( 12\beta^{2}_{i} + 24(2\lambda_{i})(\delta_{i})) = 48 \lambda^{2}_{i} ( 2\beta^{2}_{i} - \lambda_{i} \delta_{i}  ) > 0,
\end{align*}
which we have proven to be positive for any depth $L>2$, for all the eigenvector directions corresponding to the non-zero eigenvalues. 
Lastly, by Proposition~\ref{prop:one_zero_svs_set}, notice that we can write the vectorized weights in the form
\begin{align*}
    \widetilde{\Delta} &\coloneqq \mathrm{vec}\left( \mbf{W}_L, \mbf{W}_{L-1}, \ldots, \mbf{W}_1\right)\\
    &= \mathrm{vec}\left( \mbf{U}_\star\mbf{\Sigma}_L \mbf{V}_\star^\top, \mbf{V}_\star\mbf{\Sigma}_{L-1} \mbf{V}_\star^\top, \ldots, \mbf{V}_\star\mbf{\Sigma}_1 \mbf{V}_\star^\top\right)\\
    &=\sum_{i=1}^d \mathrm{vec}\left(\sigma_{L, i}\cdot\mbf{u}_{\star, i} \mbf{v}_{\star, i}^\top,\sigma_{L-1, i}\cdot \mbf{v}_{\star, i}  \mbf{v}_{\star, i}^\top, \ldots, \sigma_{1, i}\cdot\mbf{v}_{\star, i}  \mbf{v}_{\star, i}^\top \right).
\end{align*}
Then, $\Delta_i^\top \widetilde{\Delta} \neq 0$ only in the eigenvector directions that correspond to the eigenvalues of the form $S_i = L\sigma_{\star,i}^{2 - 2/L}$. Hence, the oscillations can only occur in the direction of $\Delta_{S_i}$, where $\Delta_{S_i}$ are the eigenvectors corresponding to the eigenvalues $S_i$.
This completes the proof. 
    % Then, notice that from the first part of the proof,
    % \begin{align*}
    %     \|\Delta_i \mbf{W}_L \mbf{W}_{L-1:1} + \ldots + \mbf{W}_{L:3}\Delta_i \mbf{W}_2 \mbf{W}_{1} + \mbf{W}_{L:2} \Delta_i \mbf{W}_1\|^2_{\mathsf{F}} = 2\lambda_i > 0,
    % \end{align*}
    
\end{proof}



\subsubsection{Proof of Lemma~\ref{lemma:hessian_eigvals}}
\label{sec:proof_of_hess_eigvals}

\begin{proof}
By Proposition~\ref{prop:one_zero_svs_set}, notice that we can re-write the loss in Equation~(\ref{eqn:deep_mf}) as 
    \begin{align*}
        \frac{1}{2} \left\|\mbf{W}_{L:1} - \mbf{M}_\star\right\|^2_{\mathsf{F}} = \frac{1}{2} \|\mbf{\Sigma}_{L:1} - \mbf{\Sigma}_\star\|^2_{\mathsf{F}},
    \end{align*}
    where $\mbf{\Sigma}_{L:1}$ are the singular values of $\mbf{W}_{L:1}$. We will first show that the eigenvalues of the Hessian with respect to the weight matrices $\mbf{W}_\ell$ are equivalent to those of the Hessian taken with respect to its singular values $\mbf{\Sigma}_\ell$.
    To this end, consider the vectorized form of the loss:
    \begin{align*}
        f(\mbf{\Theta}) \coloneqq \frac{1}{2}\|\mbf{W}_{L:1} - \mbf{M}_\star\|^2_{\mathsf{F}} = \frac{1}{2}\|  \text{vec}(\mbf{W}_{L:1}) - \text{vec}(\mbf{M}_\star)\|^2_2.
    \end{align*}
    Then, each block of the Hessian $ \nabla_{\mbf{\Theta}}^2 f(\mbf{\Theta}) \in \mbb{R}^{d^2 L \times d^2 L}$ with respect to the vectorized parameters is given as
    \begin{align*}
    \left[\nabla_{\mbf{\Theta}}^2 f(\mbf{\Theta})\right]_{m, \ell} = \nabla_{\text{vec}(\mbf{W}_{m})} \nabla^\top_{\text{vec}(\mbf{W}_{\ell})} f(\mbf{\Theta}) \in \mbb{R}^{d^2 \times d^2}.
    \end{align*}
    %Here, notice that 
%Note that each of the Hessian wrt product combinatinos $\nabla_{\text{vec}(\mbf{W_{l}})} \nabla_{\text{vec}(\mbf{W}_{m})^\top} f(\mbf{\Theta})  \in \mathbf{R}^{n^2 \times n^2} $ which makes  $  \nabla_{\mbf{\Theta}}^2 f(\mbf{\Theta}) \in \mathbf{R}^{n^2 L \times n^2 L}$.
By the vectorization trick, each vectorized layer matrix has an SVD of the form $\text{vec}(\mbf{W}_{\ell}) = \text{vec} (\mbf{U}_{\ell} \mbf{\Sigma}_{\ell} \mbf{ V}^\top_{\ell}) = (\mbf{V}_{\ell} \otimes \mbf{U}_{\ell}) \cdot \text{vec}(\mbf{\Sigma}_{\ell})$.
Then, notice that we have
\begin{align*}
    \nabla_{\text{vec}(\mbf{W}_{\ell})} f(\mbf{\Theta}(t)) = (\mbf{ V}_{\ell} \otimes \mbf{U}_{\ell})  \cdot \nabla_{\text{vec}(\mbf{ \Sigma}_{\ell})}f(\mbf{\Theta}(t)),
\end{align*}
which gives us that 
each block of the Hessian is given by  
%Similarly, calculating the $ml$ block product matrix of the Hessian, we have:
\begin{align*}
    \left[\nabla_{\mbf{\Theta}}^2 f(\mbf{\Theta})\right]_{m, \ell} &= \nabla_{\text{vec}(\mbf{W}_{m})} \nabla^\top_{\text{vec}(\mbf{W}_{\ell})} f(\mbf{\Theta})\\
  &=  (\mbf{V}_{m} \otimes \mbf{U}_{m})\cdot \underbrace{\nabla_{\text{vec}(\mbf{\Sigma}_{m})}   \nabla^\top_{\text{vec}(\mbf{\Sigma}_{\ell})} f(\mbf{\Theta})}_{\eqqcolon \mbf{H}_{m, \ell}}\cdot (\mbf{V}_{\ell} \otimes \mbf{U}_\ell)^\top.
\end{align*}
Then, since the Kronecker product of two orthogonal matrices is also an orthogonal matrix by Lemma~\ref{lemma:kronecker_ortho}, we can write the overall Hessian matrix as  
    \begin{align*}
        \widetilde{\mbf{H}} =
        \begin{bmatrix}
            \mbf{R}_1\mbf{H}_{1, 1}\mbf{R}_1 & \mbf{R}_1\mbf{H}_{1, 2}\mbf{R}_2 & \ldots & \mbf{R}_1\mbf{H}_{1, L}\mbf{R}_L \\
            \mbf{R}_2\mbf{H}_{2, 1}\mbf{R}_1& \mbf{R}_2 \mbf{H}_{2, 2}\mbf{R}_2 & \ldots & \mbf{R}_2\mbf{H}_{2, L} \mbf{R}_L\\
            \vdots & \vdots & \ddots & \vdots \\
             \mbf{R}_L\mbf{H}_{L, 1}\mbf{R}_1 & \mbf{R}_L\mbf{H}_{L, 2}\mbf{R}_2 & \ldots & \mbf{R}_L\mbf{H}_{L, L}\mbf{R}_L
        \end{bmatrix},
    \end{align*}
 for orthogonal matrices $\{\mbf{R}_\ell\}_{\ell=1}^L$. Then, by Lemma~\ref{lem:relationship_lemma}, the eigenvalues of $\widetilde{\mbf{H}}$ are the same as those of $\mbf{H}$, where $\mbf{H} \in \mbb{R}^{d^2 L \times d^2 L}$ is the Hessian matrix with respect to the vectorized $\mbf{\Sigma}_\ell$:
\begin{align*}
    \mbf{H} = \begin{bmatrix}
            \mbf{H}_{1,1} & \mbf{H}_{1, 2} & \hdots &\mbf{H}_{L, 1}\\
            \mbf{H}_{2,1} & \mbf{H}_{2,2} & \hdots & \mbf{H}_{L, 2} \\
            \vdots & \vdots & \ddots & \vdots \\
            \mbf{H}_{1, L} & \mbf{H}_{2, L} & \hdots &\mbf{H}_{L, L}
        \end{bmatrix}.
\end{align*}
   Now, we can consider the following vectorized loss:
    \begin{align*}
        f(\mbf{\Theta}) = \frac{1}{2} \|\mbf{\Sigma}_{L:1} - \mbf{\Sigma}_\star\|_\mathsf{F}^2 &= \frac{1}{2} \left\|\mathrm{vec}\left(\mbf{\Sigma}_{L:1} - \mbf{\Sigma}_\star\right)\right\|_2^2 \\&= \frac{1}{2} \| \underbrace{\left(\mbf{\Sigma}^\top_{\ell-1:1} \otimes \mbf{\Sigma}_{L:\ell+1} \right)}_{\eqqcolon \mbf{A}_{\ell}}\cdot\mathrm{vec}(\mbf{\Sigma}_{\ell}) - \mathrm{vec}(\mbf{\Sigma}_\star) \|_2^2. 
    \end{align*}
    Then, the gradient with respect to $\mathrm{vec}(\mbf{\Sigma}_{\ell})$ is given by
    \begin{align*}
        \nabla_{\mathrm{vec}(\mbf{\Sigma}_{\ell})} f(\mbf{\Theta}) = \mbf{A}_{\ell}^\top \left( \mbf{A}_{\ell}\cdot \mathrm{vec}(\mbf{\Sigma}_{\ell}) - \mathrm{vec}(\mbf{\Sigma}_\star)\right).
    \end{align*}
   
    Then, for $m=\ell$, we have
    \begin{align*}
        \mbf{H}_{\ell, \ell} = \nabla^2_{\mathrm{vec}(\mbf{\Sigma}_{\ell})} f(\mbf{\Theta}) &= \mbf{A}_{\ell}^{\top}\mbf{A}_{\ell}. 
    \end{align*}
For $m\neq \ell$, we have
\begin{align*}
   & \mbf{H}_{m, \ell} = \nabla_{\mathrm{vec}(\mbf{\Sigma}_{m})}  \nabla_{\mathrm{vec}(\mbf{\Sigma}_{\ell})} f(\mbf{\Theta}) =  \nabla_{\mathrm{vec}(\mbf{\Sigma}_{m})} \left[\mbf{A}_{\ell}^\top (\mbf{A}_{\ell} \mathrm{vec}(\mbf{\Sigma}_{\ell}) - \mathrm{vec}(\mbf{M}^{\star})) \right] \\
   & = \nabla_{\mathrm{vec}(\mbf{\Sigma}_{m})} \mbf{A}_{\ell}^\top \cdot \underbrace{(\mbf{A}_\ell \mathrm{vec}(\mbf{\Sigma}_{\ell}) - \mathrm{vec}(\mbf{M}^{\star}))}_{=0 \text{ 
 at convergence}} + \mbf{A}_{\ell}^{\top} \cdot  \nabla_{\mathrm{vec}(\mbf{\Sigma}_{m})} (\mbf{A}_{\ell} \mathrm{vec}(\mbf{\Sigma}_{\ell}) - \mathrm{vec}(\mbf{M}^{\star})) \\
   & = \mbf{A}_{\ell}^\top \mbf{A}_{m},
\end{align*}
where we have used the product rule along with the fact that $\mbf{A}_{\ell} \mathrm{vec}(\mbf{\Sigma}_{\ell}) = \mbf{A}_m \mathrm{vec}(\mbf{\Sigma}_{m})$.

Overall, the Hessian at convergence for GD is given by
\begin{align*}
    \mbf{H} =
    \begin{bmatrix}
        \mbf{A}_{1}^\top \mbf{A}_{1} & \mbf{A}_{1}^\top \mbf{A}_{2} & \ldots & \mbf{A}_{1}^\top \mbf{A}_{L} \\
        \mbf{A}_{2}^\top \mbf{A}_{1} & \mbf{A}_{2}^\top \mbf{A}_{2} & \ldots & \mbf{A}_{2}^\top \mbf{A}_{L} \\
        \vdots & \vdots & \ddots & \vdots \\
        \mbf{A}_{L}^\top \mbf{A}_{1} & \mbf{A}_{L}^\top \mbf{A}_{2} & \ldots & \mbf{A}_{L}^\top \mbf{A}_{L}
    \end{bmatrix}
\end{align*}
Now, we can derive an explicit expression for each $\mbf{A}_{m, \ell}$ by considering the implicit balancing effect of GD in Proposition~\ref{prop:balancing}. Under balancing and Proposition~\ref{prop:one_zero_svs_set}, we have that at convergence,
    \begin{align*}
        \mbf{\Sigma}_{L:1} = \mbf{\Sigma}_\star \implies \mbf{\Sigma}_{\ell} = \begin{bmatrix}
            \mbf{\Sigma}^{1/L}_{\star, r} & \mbf{0} \\
            \mbf{0} & \alpha \cdot \mbf{I}_{d-r}
        \end{bmatrix}, \quad \forall \ell \in [L-1], \quad \text{and} \,\,\, \mbf{\Sigma}_L = \mbf{\Sigma}^{1/L}_{\star}.
    \end{align*}
    Thus, we have
    \begin{align*}
        \mbf{H}_{m, \ell} = \begin{cases}
            \mbf{\Sigma}_{\ell}^{2(\ell -1)} \otimes \mbf{\Sigma}_{\star}^{\frac{2(L-\ell)}{L}} \quad& \text{for } \,m=\ell, \\
            \mbf{\Sigma}_\ell^{m+\ell - 2} \otimes \mbf{\Sigma}_{\star}^{2L -m-\ell} \quad& \text{for }\, m\neq\ell. \\
        \end{cases}
    \end{align*}
        Now, we are left with computing the eigenvalues of $\mbf{H} \in \mbb{R}^{d^2 L \times d^2 L}$. To do this, let us block diagonalize $\mbf{H}$ into $\mbf{H} = \mbf{PCP}^\top$, where $\mbf{P}$ is a permutation matrix and 
    \begin{align*}
        \mbf{C} = 
        \begin{bmatrix}
            \mbf{C}_{1} & & \\
            & \ddots & \\
            &&\mbf{C}_{d^2}
        \end{bmatrix} \in \mbb{R}^{d^2 L \times d^2 L},
    \end{align*}
    where each $(i,j)$-th entry of $\mbf{C}_k \in \mbb{R}^{L \times L}$ is the $k$-th diagonal element of $\mbf{H}_{i, j}$. Since $\mbf{C}$ and $\mbf{H}$ are similar matrices, they have the same eigenvalues.
    Then, since $\mbf{C}$ is a block diagonal matrix, its eigenvalues (and hence the eigenvalues of $\mbf{H}$) are the union of each of the eigenvalues of its blocks. 

    By observing the structure of $\mbf{H}_{m, \ell}$, notice that each $\mbf{C}_k$ is a rank-$1$ matrix. Hence, when considering the top-$r$ diagonal elements of $\mbf{H}_{m, \ell}$ corresponding to each Kronecker product to construct $\mbf{C}_k$, each $\mbf{C}_k$ can be written as an outer product $\mbf{uu}^{\top}$, where $\mbf{u} \in \mbb{R}^L$ is
    \begin{align}
        \mbf{u}^{\top} = \begin{bmatrix}
            \sigma_{\star, i}^{1 - \frac{1}{L}} \sigma_{\star, j}^{0} & \sigma_{\star, i}^{1 - \frac{2}{L}} \sigma_{\star, j}^{\frac{1}{L}} & \sigma_{\star, i}^{1 - \frac{3}{L}} \sigma_{\star, j}^{\frac{2}{L}} & \ldots & \sigma_{\star, i}^{0} \sigma_{\star, j}^{1 - \frac{1}{L}} 
        \end{bmatrix}^{\top}.
    \end{align}
    Then, the non-zero eigenvalue of this rank-$1$ matrix is simply $\|\mbf{u}\|_2^2$, which simplifies to 
    \begin{align*}
        \|\mbf{u}\|_2^2 = \sum_{\ell=0}^{L-1} \left(\sigma_{\star, i}^{1-\frac{1}{L} - \frac{1}{L}\ell} \cdot \sigma_{\star, j}^{\frac{1}{L}\ell}\right)^2.
    \end{align*}
    Next, we can consider the remaining $d-r$ components of each Kronecker product of $\mbf{H}_{m, \ell}$. Notice that for $m = \ell = L$, we have
    \begin{align*}
        \mbf{H}_{L, L} = \begin{bmatrix}
            \sigma_{\star, 1}^{\frac{2(L-1)}{L}} \cdot \mbf{I}_d & & & \\
            & \ddots & & \\
            & & \sigma_{\star, r}^{\frac{2(L-1)}{L}} \cdot \mbf{I}_d  & \\
            & & & \alpha^{2(L-1)}\mbf{I}_{d-r} \otimes \mbf{I}_d
        \end{bmatrix}. 
    \end{align*}
    This amounts to a matrix $\mbf{C}_k$ with a single element  $\sigma_{\star, i}^{\frac{2(L-1)}{L}}$ and $0$ elsewhere. This gives an eigenvalue $\sigma_{\star, i}^{\frac{2(L-1)}{L}}$  for all $i \in [r]$, with multiplicity $d-r$. 

    Lastly, we can consider the diagonal components of $\mbf{H}_{m, \ell}$ that is a function of the initialization scale $\alpha$. For this case, each $\mbf{C}_k$ can be written as an outer product $\mbf{vv}^{\top}$, where 
    \begin{align}
        \mbf{v}^{\top} = \begin{bmatrix}
            \sigma_{\star, i}^{1 - \frac{1}{L}} \alpha^{0} & \sigma_{\star, i}^{1 - \frac{2}{L}} \alpha& \sigma_{\star, i}^{1 - \frac{3}{L}} \alpha^{2} & \ldots & \sigma_{\star, i}^{0} \alpha^{L-1}
        \end{bmatrix}^{\top}.
    \end{align}
    Similarly, the non-zero eigenvalue is simply $\|\mbf{v}\|_2^2$, which corresponds to
    \begin{align*}
        \|\mbf{v}\|_2^2 = \sum_{\ell=0}^{L-1} \left(\sigma_{\star, k}^{1-\frac{1}{L} - \frac{1}{L}\ell} \cdot \alpha^{\ell}\right)^2.
    \end{align*}
    This completes the proof.
\end{proof}


\subsubsection{Proof of Theorem~\ref{thm:align_thm}}
\label{sec:proof_of_orbits}


\begin{proof}

To prove the result, we will consider the GD step on the $i$-th singular value and show that the $2$-period orbit condition holds given the learning rate $\eta = \frac{2}{K}$.
For ease of exposition, let us denote the $i$-th singular value of each $\mbf{W}_\ell$ as $\sigma_{i} \coloneqq \sigma_{\ell, i}$. Under balancing, consider the two-step GD update on the first singular value:
\begin{align*}
    \sigma_i(t+1) &= \sigma_i(t) + \eta L \cdot \left(\sigma_{\star, i} - \sigma_i^L(t)\right)\cdot \sigma^{L-1}_{i}(t) \\
      \sigma_i(t) = \sigma_i(t+2) &= \sigma_i(t+1) + \eta L \cdot \left(\sigma_{\star, i} - \sigma_i^L(t+1)\right)\cdot \sigma^{L-1}_{i}(t+1). \tag{By 2-period orbit}
\end{align*}
Define $z \coloneqq \left(1 + \eta L \cdot \left(\sigma_{\star, i} - \sigma_i^L(t)\right)\cdot \sigma^{L-2}_{i}(t) \right)$ and by plugging in $\sigma_i(t+1)$ for $\sigma_i(t)$, we have
\begin{align*}
    \sigma_i(t) &= \sigma_i(t) z + \eta L \cdot \left(\sigma_{\star, i} - \sigma_i^L(t)z^L \right) \cdot \sigma_i^{L-1}(t)z^{L-1} \\
    \implies 1 &= z + \eta L \cdot \left(\sigma_{\star, i} - \sigma_i^L(t)z^L \right) \cdot \sigma_i^{L-2}(t)z^{L-1} \\
    \implies 1 &= \left(1 + \eta L \cdot \left(\sigma_{\star, i} - \sigma_i^L(t)\right)\cdot \sigma^{L-2}_{i}(t) \right) + \eta L \cdot \left(\sigma_{\star, i} - \sigma_i^L(t)z^L \right) \cdot \sigma_i^{L-2}(t)z^{L-1} \\
    \implies 0 &= \left(\sigma_{\star, i} - \sigma_i^L(t)\right) + \left(\sigma_{\star, i} - \sigma_i^L(t)z^L \right) \cdot z^{L-1}
\end{align*}
Simplifying this expression further, we have
\begin{align*}
    &0 = \sigma_{\star, i} - \sigma_i^L(t) + \sigma_{\star, i} z^{L-1} - \sigma_i^L(t) z^{2L-1} \\
    \implies &\sigma_i^L(t) + \sigma_i^L(t) z^{2L-1} =  \sigma_{\star, i} + \sigma_{\star, i} z^{L-1} \\
    \implies &\sigma_i^L(t)\cdot\left(1 + z^{2L - 1} \right) = \sigma_{\star, i}\cdot\left(1 + z^{L - 1} \right) \\
    \implies &\sigma_i^L(t)\frac{\left(1 + z^{2L - 1} \right)}{\left(1 + z^{L - 1} \right)} = \sigma_{\star, i},
\end{align*}
and by defining $\rho_i \coloneqq \sigma_i(t)$, we obtain the polynomial
\begin{align*}
    \sigma_{\star, i} = \rho_i^L\frac{1+z^{2L-1}}{1+z^{L-1}}, \quad \text{where  } \, z \coloneqq \left(1 + \eta L(\sigma_{\star, i} - \rho_i^L)\cdot \rho_i^{L-2} \right).
\end{align*}
Next, we show the existence of (real) roots within the ranges for $\rho_{i,1}$ and $\rho_{i, 2}$. We note that these roots only exist within the EOS regime.
First, consider $\rho_{i, 1} \in \left(0, \sigma_{\star, i}^{1/L} \right)$. We will show that for two values within this range, there is a sign change for all $L \geq 2$. More specifically, we show that there exists $\rho_i \in \left(0, \sigma_{\star, i}^{1/L} \right)$ such that
\begin{align*}
     \rho_i^L\frac{1+z^{2L-1}}{1+z^{L-1}} - \sigma_{\star, i} > 0 \quad \text{and} \quad \rho_i^L\frac{1+z^{2L-1}}{1+z^{L-1}} - \sigma_{\star, i} < 0.
\end{align*}
For the positive case, consider $\rho_i = (\frac{1}{2}\sigma_{\star, i})^{1/ L}$. We need to show that 
\begin{align*}
    \frac{1+z^{2L-1}}{1+z^{L-1}}  = \frac{1 + \left(1+\eta L\cdot\left(\frac{\sigma_{\star, i}}{2}\right)\frac{\sigma_{\star, i}^{1-\frac{2}{L}}}{2^{1 - \frac{2}{L}}}\right)^{2L-1}}{1 + \left(1+\eta L\cdot\left(\frac{\sigma_{\star, i}}{2}\right)\frac{\sigma_{\star, i}^{1-\frac{2}{L}}}{2^{1 - \frac{2}{L}}}\right)^{L-1}} > 2.
\end{align*}
To do this, we will plug in the smallest possible value of $\eta = \frac{2}{L\sigma_{\star, i}^{2 - \frac{2}{L}}}$ to show that the fraction is still greater than $2$, which gives us
\begin{align}
\label{eqn:first_range_pos}
    u(L) \coloneqq \frac{1 + \left(1+\frac{1}{ 2^{1 - \frac{2}{L}}} \right)^{2L-1}}{1 + \left(1+\frac{1}{ 2^{1 - \frac{2}{L}}} \right)^{L-1}},
\end{align}
which is an increasing function of $L$ for all $L\geq 2$. Since $u(2) > 2$, Equation~(\ref{eqn:first_range_pos}) must always be greater than $2$. For the negative case, we can simply consider $\rho_i = 0$.
Hence, since the polynomial is continuous, by the Intermediate Value Theorem (IVT), there must exist a root within the range $\rho_i \in \left(0, \sigma_{\star, i}^{1/L} \right)$.


Next, consider the range $\rho_{i, 2} \in \left(\sigma_{\star, i}^{1/L}, (2\sigma_{\star, i})^{1/L}\right)$. Similarly, we will show sign changes for two values in $\rho_{i, 2}$.
For the positive case, consider $\rho_i = \left(\frac{3}{2} \sigma_{\star, i}\right)^{1/L}$. For $\eta$, we can plug in the smallest possible value within the range to show that this value of $\rho_i$  provides a positive quantity. Specifically, we need to show that
\begin{align*}
    \frac{1+z^{2L-1}}{1+z^{L-1}} > \frac{2}{3} \implies \frac{1+\left(1+\frac{2}{\sigma_{\star, i}^{2- \frac{2}{L}}}\cdot(\sigma_{\star, i} - \frac{3}{2}\sigma_{\star, i})\cdot \left(\frac{3}{2}\sigma_{\star, i}\right)^{1 - \frac{2}{L}} \right)^{2L-1}}{1+\left(1+\frac{2}{\sigma_{\star, i}^{2- \frac{2}{L}}}\cdot(\sigma_{\star, i} - \frac{3}{2}\sigma_{\star, i})\cdot \left(\frac{3}{2}\sigma_{\star, i}\right)^{1 - \frac{2}{L}} \right)^{L-1}} > \frac{2}{3}.
\end{align*}
We can simplify the fraction as follows:
\begin{align*}
    \frac{1+\left(1+\frac{2}{\sigma_{\star, 1}^{2- \frac{2}{L}}}\cdot(\sigma_{\star, 1} - \frac{3}{2}\sigma_{\star, 1})\cdot \left(\frac{3}{2}\sigma_{\star, 1}\right)^{1 - \frac{2}{L}} \right)^{2L-1}}{1+\left(1+\frac{2}{\sigma_{\star, 1}^{2- \frac{2}{L}}}\cdot(\sigma_{\star, 1} - \frac{3}{2}\sigma_{\star, 1})\cdot \left(\frac{3}{2}\sigma_{\star, 1}\right)^{1 - \frac{2}{L}} \right)^{L-1}} = 
    \frac{1+\left(1-(\frac{3}{2})^{1 - \frac{2}{L}} \right)^{2L-1}}{1+\left(1-(\frac{3}{2})^{1 - \frac{2}{L}} \right)^{L-1}}.
\end{align*}
Then, since we are subtracting by $(\frac{3}{2})^{1 - \frac{2}{L}}$, we can plug in its largest value for $L\geq 2$, which is $3/2$. This gives us 
\begin{align*}
    \frac{1+\left(-0.5\right)^{2L-1}}{1+\left(-0.5 \right)^{L-1}} > \frac{2}{3},
\end{align*}
as for odd values of $L$, the function increases to $1$ starting from $L=2$, and decreases to $1$ for even $L$. 
To check negativity, let us define
\begin{align*}
    h(\rho) \coloneqq \frac{f(\rho)}{g(\rho)} \coloneqq \frac{\rho^L \left(1 + z^{2L-1} \right)}{1 + z^{L-1}}.
\end{align*}
We will show that $h'\left(\sigma_{\star, i}^{1/L} \right) < 0$:
\begin{align*}
h'\left(\sigma_{\star, i}^{1/L} \right) &= \frac{f'\left(\sigma_{\star, i}^{1/L} \right)g\left(\sigma_{\star, i}^{1/L} \right) - f\left(\sigma_{\star, i}^{1/L} \right)g'\left(\sigma_{\star, i}^{1/L} \right)}{g^2\left(\sigma_{\star, i}^{1/L} \right)} \\
&= \frac{f'\left(\sigma_{\star, i}^{1/L} \right) - \sigma_{\star, i}\cdot g'\left(\sigma_{\star, i}^{1/L} \right)}{2} \\
&= \frac{L\sigma_{\star, i}^{1 - \frac{1}{L}} - \sigma_{\star, i}(2L-1)\left(\eta L^2 \sigma_{\star, i}^{2 -\frac{3}{L}} \right) - \sigma_{\star, i}(L-1)\left(\eta L^2 \sigma_{\star, i}^{2 -\frac{3}{L}} \right) }{2} \\
&= \frac{L\sigma_{\star, i}^{1 - \frac{1}{L}} - (3L-2)\left(\eta L^2 \sigma_{\star, i}^{3 -\frac{3}{L}} \right) }{2} < 0,
\end{align*}
as otherwise we need $\eta \leq \frac{\sigma_{\star, i}^{2/L - 2}}{3L^2 - 2L}$, which is out of the range of interest. Since $h'(\rho)< 0$, it follows that there exists a $\delta > 0$ such that $h(\rho) > h(x)$ for all $x$ such that $\rho < x < \rho+\delta$. Lastly, since $h(\rho) - \sigma_{\star, i} = 0$ for $\rho = \sigma_{\star, i}^{1/L}$, it follows that $h(\rho) - \sigma_{\star, i}$ must be negative at $\rho + \delta$.
Similarly, by IVT, there must exist a root within the range 
$\rho_{i,2} \in \left(\sigma_{\star, i}^{1/L}, (2\sigma_{\star, i})^{1/L}\right)$. This proves that the $i$-th singular value undergoes a two-period orbit with the roots $\rho_{i, 1}$ and $\rho_{i, 2}$. Then, notice that if the learning rate is large enough to induce oscillations in the $i$-th singular value, then it is also large enough to have oscillations in all singular values from $1$ to the $(i-1)$-th singular value (assuming that it is not large enough for divergence). Finally, at the (balanced) minimum, we can express the dynamics as 
\begin{align}
    \mbf{W}_{L:1} = \underbrace{\sum_{i=1}^p\rho_{i, j}^L \cdot \mbf{u}_{\star, i}\mbf{v}_{\star, i}^{\top} }_{\text{oscillation subspace}}+ \underbrace{\sum_{k=p+1}^d \sigma_{\star, k}\cdot \mbf{u}_{\star, k}\mbf{v}_{\star, k}^{\top}}_{\text{stationary subspace}}, \quad j \in \{1,2\}, \quad \forall\ell \in [L-1].
\end{align}
This completes the proof.


\end{proof}


\subsection{Auxiliary Results}


\begin{lemma}
\label{lem:relationship_lemma} 
    Let $\{\mbf{R}_{\ell}\}_{\ell=1}^L \in \mathbb{R}^{n\times n}$ be orthogonal matrices and $\mbf{H}_{i, j} \in \mathbb{R}^{n^2 \times n^2}$ be diagonal matrices. Consider the two following block matrices:
    \begin{align*}
       \mbf{H} &= \begin{bmatrix}
            \mbf{H}_{1,1} & \mbf{H}_{1, 2} & \hdots &\mbf{H}_{L, 1}\\
            \mbf{H}_{2,1} & \mbf{H}_{2,2} & \hdots & \mbf{H}_{L, 2} \\
            \vdots & \vdots & \ddots & \vdots \\
            \mbf{H}_{1, L} & \mbf{H}_{2, L} & \hdots &\mbf{H}_{L, L}
        \end{bmatrix} \\  \widetilde{\mbf{H}} &=
      \begin{bmatrix}
        \mbf{R}_{L}\mbf{H}_{1,1}\mbf{R}_{L}^{\top} & \mbf{R}_{L}\mbf{H}_{1, 2}\mbf{R}_{L-1}^{\top} & \hdots & \mbf{R}_{L}\mbf{H}_{1, L}\mbf{R}_{1}^{\top}\\
        \mbf{R}_{L-1}\mbf{H}_{2,1}\mbf{R}_{L}^{\top} & \mbf{R}_{L-1}\mbf{H}_{2,2}\mbf{R}_{L-1}^{\top} & \hdots & \mbf{R}_{L-1}\mbf{H}_{2, L}\mbf{R}_{1}^{\top}\\
        \vdots & \vdots & \ddots & \vdots \\
        \mbf{R}_{1}\mbf{H}_{L,1}\mbf{R}_{L}^{\top} & \mbf{R}_{1}\mbf{H}_{L,2}\mbf{R}_{L-1}^{\top} & \hdots & \mbf{R}_{1}\mbf{H}_{L,L}\mbf{R}_{1}^{\top}
    \end{bmatrix}.
    \end{align*}
    Then, the two matrices $\mbf{H}$ and $\widetilde{\mbf{H}}$ are similar, in the sense that they have the same eigenvalues.
\end{lemma}


\begin{proof}
It suffices to show that $\mbf{H}$ and $ \widetilde{\mbf{H}}$ have the same characteristic polynomials. Let us define 
    \begin{align*}
        \widetilde{\mbf{H}} \coloneqq \begin{bmatrix}
            \mbf{A} & \mbf{B} \\
            \mbf{C} & \mbf{D}
        \end{bmatrix},
    \end{align*}
    where 
    \begin{alignat}{3}
        &\mbf{A} \coloneqq  \mbf{R}_{L}\mbf{H}_{1,1}\mbf{R}_{L}^{\top} \quad\quad\quad &\mbf{B} &\coloneqq \begin{bmatrix}
            \mbf{R}_{L}\mbf{H}_{1, 2}\mbf{R}_{L-1}^{\top} & \hdots & \mbf{R}_{L}\mbf{H}_{1, L}\mbf{R}_{1}^{\top}
        \end{bmatrix} \\
        &\mbf{C} \coloneqq \begin{bmatrix}
            \mbf{R}_{L-1}\mbf{H}_{2,1}\mbf{R}_{L}^{\top} \\
            \vdots \\
           \mbf{R}_{1}\mbf{H}_{L,1}\mbf{R}_{L}^{\top}
        \end{bmatrix}  \quad\quad\quad
        &\mbf{D} &\coloneqq \begin{bmatrix}
           \mbf{R}_{L-1}\mbf{H}_{2,2}\mbf{R}_{L-1}^{\top} & \hdots & \mbf{R}_{L-1}\mbf{H}_{2, L}\mbf{R}_{1}^{\top}\\ \\
            \vdots & \ddots & \vdots \\
           \mbf{R}_{1}\mbf{H}_{L,2}\mbf{R}_{L-1}^{\top} & \hdots & \mbf{R}_{1}\mbf{H}_{L,L}\mbf{R}_{1}^{\top}
        \end{bmatrix}.
    \end{alignat}
    Then, we have
    \begin{align*}
        \det(\widetilde{\mbf{H}} - \lambda \mbf{I}) &= \det\left(\begin{bmatrix}
            \mbf{A} - \lambda\mbf{I} & \mbf{B} \\
            \mbf{C} & \mbf{D} - \lambda\mbf{I}
        \end{bmatrix}\right) \\
        &= \det(\mbf{A} - \lambda \mbf{I}) \cdot \det((\mbf{D} - \lambda \mbf{I}) - \mbf{C}(\mbf{A} - \lambda \mbf{I})^{-1}\mbf{B}),
    \end{align*}
    where the second equality is by the Schur complement. Notice that
    \begin{align*}
        (\mbf{A} - \lambda \mbf{I})^{-1} = (\mbf{R}_{L}\mbf{H}_{1,1}\mbf{R}_{L}^\top  - \lambda \mbf{I})^{-1} &= (\mbf{R}_{L}\mbf{H}_{1,1}\mbf{R}_{L}^\top  - \lambda \mbf{\mbf{R}_{L}\mbf{R}_{L}}^\top)^{-1} \\&= \mbf{R}_{L} \cdot (\mbf{H}_{1,1} - \lambda\mbf{I})^{-1} \cdot \mbf{R}_{L}^\top.
    \end{align*}
    Then, we also see that, 
    \begin{align*}
        \mbf{C}(\mbf{A} - \lambda \mbf{I})^{-1}\mbf{B} = \underbrace{\begin{bmatrix}
            \mbf{R}_{L-1} & & \\
            & \ddots & \\
            & & \mbf{R}_{1}
        \end{bmatrix}}_{\eqqcolon \widehat{\mbf{V}}}\cdot\, 
        \mbf{E}\cdot
        \underbrace{\begin{bmatrix}
            \mbf{R}_{L-1}^\top & & \\
            & \ddots & \\
            & & \mbf{R}_{1}^\top
        \end{bmatrix}}_{\eqqcolon \widehat{\mbf{V}}^\top}.
    \end{align*}
    where
    \begin{align*}
        \mbf{E}\coloneqq
        \begin{bmatrix}
            \mbf{H}_{2, 1} \cdot (\mbf{H}_{1,1} - \lambda \mbf{I})^{-1} \cdot \mbf{H}_{1,2} & \hdots & \mbf{H}_{2,1}\cdot (\mbf{H}_{1,1} - \lambda \mbf{I})^{-1} \cdot \mbf{H}_{1, L} \\
            \vdots & \ddots & \vdots \\
            \mbf{H}_{L, 1}\cdot (\mbf{H}_{1,1} - \lambda \mbf{I})^{-1} \cdot \mbf{H}_{1, 2} & \hdots & \mbf{H}_{L, 1}\cdot (\mbf{H}_{1, 1} - \lambda \mbf{I})^{-1} \cdot \mbf{H}_{1, L}
        \end{bmatrix}.
    \end{align*}
    Similarly, we can write $\mbf{D}$ as 
    \begin{align*}
        \mbf{D} = \widehat{\mbf{V}}
        \underbrace{\begin{bmatrix}
            \mbf{H}_{2,2} & \hdots & \mbf{H}_{2, L} \\
            \vdots & \ddots & \vdots \\
            \mbf{H}_{L, 2} & \hdots & \mbf{H}_{L, L}
        \end{bmatrix}}_{\eqqcolon \mbf{F}}
        \widehat{\mbf{V}}^\top.
    \end{align*}
    Then, we have
    \begin{align*}
        \det(\widetilde{\mbf{H}} - \lambda \mbf{I}) &= \det(\mbf{R}_{L}\cdot (\mbf{H}_{1,1} - \lambda \mbf{I})\cdot\mbf{R}_{L}^\top) \cdot \det\left(\widehat{\mbf{V}} \cdot (\mbf{E} - \mbf{F})\cdot \widehat{\mbf{V}}^\top \right) \\
       &= \det(\mbf{H}_{1,1} - \lambda \mbf{I}) \cdot \det(\mbf{E} - \mbf{F}),
    \end{align*}
    which is not a function of $\mbf{U}, \mbf{V},\{\mbf{R}_{\ell}\}_{\ell=1}^L$. By doing the same for $\mbf{H}$, we can show that both $\widetilde{\mbf{H}}$ and $\mbf{H}$ have the same characteristic polynomials, and hence the same eigenvalues. This completes the proof.


\end{proof}


\begin{lemma}
\label{lemma:kronecker_ortho}
   Let $\mbf{A}, \mbf{B} \in \mbb{R}^{d\times d}$ be two orthogonal matrices. Then, the Kronecker product of $\mbf{A}$ and $\mbf{B}$ is also an orthogonal matrix:
   \begin{align*}
       (\mbf{A} \otimes \mbf{B})^\top (\mbf{A} \otimes \mbf{B}) = (\mbf{A} \otimes \mbf{B})(\mbf{A} \otimes \mbf{B})^\top = \mbf{I}_{d^2}.
   \end{align*}
\end{lemma}

\begin{proof}
We prove this directly by using properties of Kronecker products:
\begin{align*}
    (\mbf{A} \otimes \mbf{B})^\top (\mbf{A} \otimes \mbf{B}) &= \mbf{A}^\top \mbf{A} \otimes \mbf{B}^\top \mbf{B} \\
    &= \mbf{I}_d \otimes \mbf{I}_d = \mbf{I}_{d^2}.
\end{align*}
Similarly, we have
\begin{align*}
    (\mbf{A} \otimes \mbf{B}) (\mbf{A} \otimes \mbf{B})^\top &= \mbf{A} \mbf{A}^\top \otimes \mbf{B} \mbf{B}^\top \\
    &= \mbf{I}_d \otimes \mbf{I}_d = \mbf{I}_{d^2}.
\end{align*}
This completes the proof.
\end{proof}

\begin{lemma}
\label{lemm:seq_converge}
    Let $\{a(t)\}_{t=1}^N$ be a sequence such that $a(t) \geq 0$ for all $t$. 
    If there exists a constant $c \in (0,1)$ such that $a(t+1) < c \cdot a(t)$ for all $t$, 
    then $\lim_{t \to \infty} a(t) = 0$.

\end{lemma}

\begin{proof}
   We prove this by direct reasoning. 
    From the assumption $a(t+1) < c \cdot a(t)$ for some $c \in (0,1)$, we can iteratively expand this inequality:
    \[
    a(t+1) < c \cdot a(t), \quad a(t+2) < c \cdot a(t+1) < c^2 \cdot a(t),
    \]
    and, more generally, by induction:
    \[
    a(t+k) < c^k \cdot a(t), \quad \text{for all } k \geq 0.
    \]
    Since $c \in (0,1)$, the sequence $\{c^k\}_{k=0}^\infty$ converges to $0$ as $k \to \infty$. Hence:
    \[
    0\leq \lim_{k \to \infty} a(t+k) \leq \lim_{k \to \infty} c^k \cdot a(t) = 0.
    \]
    Therefore, by the squeeze theorem, the sequence $\{a(t)\}$ converges to $0$ as $t \to \infty$.
\end{proof}





\begin{lemma}
[\cite{chen2023edge}]
\label{lemma:chen-bruna}
Consider any 1-D differentiable function $f(x)$ around a local minima $\bar{x}$, satisfying (i) $f^{(3)}(\bar{x}) \neq 0$, and (ii) $3[f^{(3)}]^2 - f'' f^{(4)} > 0$ at $\bar{x}$. Then, there exists $\epsilon$ with sufficiently small $|\epsilon|$ and $\epsilon \cdot f^{(3)} > 0$ such that: for any point $x_0$ between $\bar{x}$ and $\bar{x} - \epsilon$, there exists a learning rate $\eta$ such that $F_{\eta}^2(x_0) = x_0$, and
\end{lemma}

\[
\frac{2}{f''(\bar{x})} < \eta < \frac{2}{f''(\bar{x}) - \epsilon \cdot f^{(3)}(\bar{x})}.
\]






\end{document}
