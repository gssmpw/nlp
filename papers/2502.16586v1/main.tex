% This must be in the first 5 lines to tell arXiv to use pdfLaTeX, which is strongly recommended.
\pdfoutput=1
% In particular, the hyperref package requires pdfLaTeX in order to break URLs across lines.

\documentclass[11pt]{article}

% Change "review" to "final" to generate the final (sometimes called camera-ready) version.
% Change to "preprint" to generate a non-anonymous version with page numbers.
\usepackage[preprint]{acl}

% Standard package includes
\usepackage{times}
\usepackage{latexsym}

% For proper rendering and hyphenation of words containing Latin characters (including in bib files)
\usepackage[T1]{fontenc}
% For Vietnamese characters
% \usepackage[T5]{fontenc}
% See https://www.latex-project.org/help/documentation/encguide.pdf for other character sets

% This assumes your files are encoded as UTF8
\usepackage[utf8]{inputenc}

% This is not strictly necessary, and may be commented out,
% but it will improve the layout of the manuscript,
% and will typically save some space.
\usepackage{microtype}

% This is also not strictly necessary, and may be commented out.
% However, it will improve the aesthetics of text in
% the typewriter font.
\usepackage{inconsolata}

%Including images in your LaTeX document requires adding
%additional package(s)
\usepackage{graphicx}
\usepackage{amsmath,amsfonts}
\usepackage{algorithmic}
\usepackage{algorithm}
\usepackage{array}
\usepackage[caption=false,font=normalsize,labelfont=sf,textfont=sf]{subfig}
\usepackage{booktabs}
\usepackage{multirow}
\usepackage{textcomp}
\usepackage{stfloats}
\usepackage{url}
\usepackage{verbatim}
\usepackage{graphicx}
\usepackage{colortbl} % For cell colors

% \usepackage{cite}
% updated with editorial comments 8/9/2021
\usepackage{tcolorbox} % 引入tcolorbox宏包
% \usepackage{natbib}
% \bibliographystyle{plain}

\usepackage{pifont}


% If the title and author information does not fit in the area allocated, uncomment the following
%
%\setlength\titlebox{<dim>}
%
% and set <dim> to something 5cm or larger.

\title{Multimodal Large Language Models for Text-rich Image Understanding: A Comprehensive Review}


\author{Pei Fu$^{1}$, Tongkun Guan$^{2}$, Zining Wang$^{1}$, Zhentao Guo$^{3}$, Chen Duan$^{1}$, \\
\textbf{Hao Sun$^{4}$, Boming Chen$^{1}$, Jiayao Ma$^{1}$, Qianyi Jiang$^{1}$, Kai Zhou$^{1}$, Junfeng Luo$^{1}$ } \\
  $^{1}$Meituan, $^{2}$ Shanghai Jiao Tong University, $^{3}$Beijing Institute of Technology, \\
  $^{4}$MAIS \& NLPR, Institute of Automation, Chinese Academy of Sciences\\
  \texttt{\{fupei,duanchen02,wangzining03,chenboming,majiayao02\}@meituan.com} \\
  \texttt{\{jiangqianyi02,zhoukai03,luojunfeng\}@meituan.com} \\
\texttt{gtk0615@sjtu.edu.cn,hao.sun@cripac.ia.ac.cn,zt\_guo1230@163.com}\
  }

\begin{document}
% \begin{titlepage}
\twocolumn[{
\renewcommand\twocolumn[1][]{#1}
\maketitle
\vspace{-1em}
\begin{center}
    \captionsetup{type=figure}
    \includegraphics[width=0.98\textwidth]{Figure/1.pdf}
    \captionof{figure}{The development timeline of TIU MLLMs. 
    % \textcolor{red}{We mark the five-pointed star as the midpoint to divide the development of TIU into the pre-LLM era (2019-2022) and the post-LLM era (2023-present).}
    }
    \label{fig:Development}
\end{center}
\vspace{1em}
}]


\begin{abstract}


The choice of representation for geographic location significantly impacts the accuracy of models for a broad range of geospatial tasks, including fine-grained species classification, population density estimation, and biome classification. Recent works like SatCLIP and GeoCLIP learn such representations by contrastively aligning geolocation with co-located images. While these methods work exceptionally well, in this paper, we posit that the current training strategies fail to fully capture the important visual features. We provide an information theoretic perspective on why the resulting embeddings from these methods discard crucial visual information that is important for many downstream tasks. To solve this problem, we propose a novel retrieval-augmented strategy called RANGE. We build our method on the intuition that the visual features of a location can be estimated by combining the visual features from multiple similar-looking locations. We evaluate our method across a wide variety of tasks. Our results show that RANGE outperforms the existing state-of-the-art models with significant margins in most tasks. We show gains of up to 13.1\% on classification tasks and 0.145 $R^2$ on regression tasks. All our code and models will be made available at: \href{https://github.com/mvrl/RANGE}{https://github.com/mvrl/RANGE}.

\end{abstract}

 
\section{Introduction}
Backdoor attacks pose a concealed yet profound security risk to machine learning (ML) models, for which the adversaries can inject a stealth backdoor into the model during training, enabling them to illicitly control the model's output upon encountering predefined inputs. These attacks can even occur without the knowledge of developers or end-users, thereby undermining the trust in ML systems. As ML becomes more deeply embedded in critical sectors like finance, healthcare, and autonomous driving \citep{he2016deep, liu2020computing, tournier2019mrtrix3, adjabi2020past}, the potential damage from backdoor attacks grows, underscoring the emergency for developing robust defense mechanisms against backdoor attacks.

To address the threat of backdoor attacks, researchers have developed a variety of strategies \cite{liu2018fine,wu2021adversarial,wang2019neural,zeng2022adversarial,zhu2023neural,Zhu_2023_ICCV, wei2024shared,wei2024d3}, aimed at purifying backdoors within victim models. These methods are designed to integrate with current deployment workflows seamlessly and have demonstrated significant success in mitigating the effects of backdoor triggers \cite{wubackdoorbench, wu2023defenses, wu2024backdoorbench,dunnett2024countering}.  However, most state-of-the-art (SOTA) backdoor purification methods operate under the assumption that a small clean dataset, often referred to as \textbf{auxiliary dataset}, is available for purification. Such an assumption poses practical challenges, especially in scenarios where data is scarce. To tackle this challenge, efforts have been made to reduce the size of the required auxiliary dataset~\cite{chai2022oneshot,li2023reconstructive, Zhu_2023_ICCV} and even explore dataset-free purification techniques~\cite{zheng2022data,hong2023revisiting,lin2024fusing}. Although these approaches offer some improvements, recent evaluations \cite{dunnett2024countering, wu2024backdoorbench} continue to highlight the importance of sufficient auxiliary data for achieving robust defenses against backdoor attacks.

While significant progress has been made in reducing the size of auxiliary datasets, an equally critical yet underexplored question remains: \emph{how does the nature of the auxiliary dataset affect purification effectiveness?} In  real-world  applications, auxiliary datasets can vary widely, encompassing in-distribution data, synthetic data, or external data from different sources. Understanding how each type of auxiliary dataset influences the purification effectiveness is vital for selecting or constructing the most suitable auxiliary dataset and the corresponding technique. For instance, when multiple datasets are available, understanding how different datasets contribute to purification can guide defenders in selecting or crafting the most appropriate dataset. Conversely, when only limited auxiliary data is accessible, knowing which purification technique works best under those constraints is critical. Therefore, there is an urgent need for a thorough investigation into the impact of auxiliary datasets on purification effectiveness to guide defenders in  enhancing the security of ML systems. 

In this paper, we systematically investigate the critical role of auxiliary datasets in backdoor purification, aiming to bridge the gap between idealized and practical purification scenarios.  Specifically, we first construct a diverse set of auxiliary datasets to emulate real-world conditions, as summarized in Table~\ref{overall}. These datasets include in-distribution data, synthetic data, and external data from other sources. Through an evaluation of SOTA backdoor purification methods across these datasets, we uncover several critical insights: \textbf{1)} In-distribution datasets, particularly those carefully filtered from the original training data of the victim model, effectively preserve the model’s utility for its intended tasks but may fall short in eliminating backdoors. \textbf{2)} Incorporating OOD datasets can help the model forget backdoors but also bring the risk of forgetting critical learned knowledge, significantly degrading its overall performance. Building on these findings, we propose Guided Input Calibration (GIC), a novel technique that enhances backdoor purification by adaptively transforming auxiliary data to better align with the victim model’s learned representations. By leveraging the victim model itself to guide this transformation, GIC optimizes the purification process, striking a balance between preserving model utility and mitigating backdoor threats. Extensive experiments demonstrate that GIC significantly improves the effectiveness of backdoor purification across diverse auxiliary datasets, providing a practical and robust defense solution.

Our main contributions are threefold:
\textbf{1) Impact analysis of auxiliary datasets:} We take the \textbf{first step}  in systematically investigating how different types of auxiliary datasets influence backdoor purification effectiveness. Our findings provide novel insights and serve as a foundation for future research on optimizing dataset selection and construction for enhanced backdoor defense.
%
\textbf{2) Compilation and evaluation of diverse auxiliary datasets:}  We have compiled and rigorously evaluated a diverse set of auxiliary datasets using SOTA purification methods, making our datasets and code publicly available to facilitate and support future research on practical backdoor defense strategies.
%
\textbf{3) Introduction of GIC:} We introduce GIC, the \textbf{first} dedicated solution designed to align auxiliary datasets with the model’s learned representations, significantly enhancing backdoor mitigation across various dataset types. Our approach sets a new benchmark for practical and effective backdoor defense.



\section{Background}
% \begin{tcolorbox}[simplebox]
% We first formally define the problem and highlight its challenge. 
% Then we present an EM approach to address this challenge. 
% \end{tcolorbox}
% \vspace{-0.3cm}
% \subsection{Problem Statement }\label{sec_ps}

% Here’s a polished and enriched version of your problem formulation section, with improved clarity, precision, and academic tone:

% ---
\begin{figure}[t]
    \centering % Center the figure
    \includegraphics[width=\linewidth]{figs/example.pdf} % Include the figure
    \caption{\small \textbf{Example of Autonomous Code Integration.} \small We aim to enable LLMs to determine tool-usage strategies
based on their own capability boundaries. In the example, the model write code to solve the problem that demand special tricks, strategically bypassing its inherent limitations.} 
    \label{fig_example}
    \vspace{-0.2cm}
\end{figure}
\textbf{Problem Statement.} Modern tool-augmented language models address mathematical problems \( x_q \in \mathcal{X}_Q \) by generating step-by-step solutions that interleave natural language reasoning with executable Python code (Fig.~\ref{fig_example}). Formally, given a problem \( x_q \), a model \( \mathcal{M}_\theta \) iteratively constructs a solution \( y_a = \{y_1, \dots, y_T\} \) by sampling components \( y_t \sim p(y_t | y_{<t}, x_q) \), where \( y_{<t} \) encompasses both prior reasoning steps, code snippets and execution results \( \mathbf{e}_t \) from a Python interpreter. The process terminates upon generating an end token, and the solution is evaluated via a binary reward \( r(y_a,x_q) = \mathbb{I}(y_a \equiv y^*) \) indicating equivalence to the ground truth \( y^* \). The learning objective is formulated as:
\[
\max_{\theta} \mathbb{E}_{x_q \sim \mathcal{X}_Q} \left[r(y_a, x_q) \right]
\]

\noindent\textbf{Challenge and Motivation.} Developing autonomous code integration (AutoCode) strategies poses unique challenges, as optimal tool-usage behaviors must dynamically adapt to a model's intrinsic capabilities and problem-solving contexts. While traditional supervised fine-tuning (SFT) relies on imitation learning from expert demonstrations, this paradigm fundamentally limits the emergence of self-directed tool-usage strategies. Unfortunately, current math LLMs predominantly employ SFT to orchestrate tool integration~\citep{mammoth, tora, dsmath, htl}, their rigid adherence to predefined reasoning templates therefore struggles with the dynamic interplay between a model’s evolving problem-solving competencies and the adaptive tool-usage strategies required for diverse mathematical contexts.

Reinforcement learning (RL) offers a promising alternative by enabling trial-and-error discovery of autonomous behaviors. Recent work like DeepSeek-R1~\citep{dsr1} demonstrates RL's potential to enhance reasoning without expert demonstrations. However, we observe that standard RL methods (e.g., PPO~\cite{ppo}) suffer from a critical inefficiency (see Sec.~\ref{sec_ablation}): Their tendency to exploit local policy neighborhoods leads to insufficient exploration of the vast combinatorial space of code-integrated reasoning paths, especially when only given a terminal reward in mathematical problem-solving.

To bridge this gap, we draw inspiration from human metacognition -- the iterative process where learners refine tool-use strategies through deliberate exploration, outcome analysis, and belief updates. A novice might initially attempt manual root-finding via algebraic methods, observe computational bottlenecks or inaccuracies, and therefore prompting the usage of calculators. Through systematic reflection on these experiences, they internalize the contextual efficacy of external tools, gradually forming stable heuristics that balance reasoning with judicious tool invocation. 


To this end, \emph{our focus diverges from standard agentic tool-use frameworks~\citep{agentr}}, which merely prioritize successful tool execution. Instead, \emph{we aim to instill \emph{human-like metacognition} in LLMs, enabling them to (1) determine tool-usage based on their own capability boundaries (see the analysis in Sec.~\ref{sec_ablation}), and (2) dynamically adapt tool-usage strategies as their reasoning abilities evolve (via our EM framework).}
% For instance, while an LLM might solve a combinatorics problem via CoT alone, it should autonomously invoke code for eigenvalue calculations in linear algebra where symbolic computations are error-prone. Achieving this requires models to \emph{jointly optimize} their reasoning and tool-integration policies in a mutually reinforcing manner.


% Mirroring this metacognitive cycle, we propose an Expectation-Maximization (EM) framework that allows LLMs to develop AutoCode strategies via guided exploration (the E-step) and self-refinement (the M-step).


% \vspace{-0.3cm}
\section{Methodology}

Inspired by human metacognitive processes, we introduce an Expectation-Maximization (EM) framework that trains LLMs for autonomous code integration (AutoCode) through alternations (Fig.~\ref{fig_overview}):

\begin{enumerate}[leftmargin=0.5cm,topsep=1pt,itemsep=0pt,parsep=0pt]
    \item \emph{Guided Exploration (E-step):} Identifies high-potential code-integrated solutions by systematically probing the model's inherent capabilities.
\item \emph{Self-Refinement (M-step):} Optimizes the model's tool-usage strategy and chain-of-thought reasoning using curated trajectories from the E-step.
\end{enumerate}


\begin{figure*}[t]
    \centering
    \includegraphics[width=\linewidth]{figs/overview.pdf}
    \caption{\small \textbf{Method Overview.} \small (Left) shows an overview for the EM framework, which alternates between finding a reference strategy for guided exploration (E-step) and off-policy RL (M-step). (Right) shows the data curation for guided exploration. We generate \(K\) rollouts, estimate values of code-triggering decisions and subsample the initial data with sampling weights per Eq.~\ref{eq_sampling}.}
    \label{fig_overview}
\end{figure*}

\subsection{The EM Framework for AutoCode}

A central challenge in AutoCode lies in the code triggering decisions, represented by the binary decision \(c \in \{0, 1\}\).  While supervised fine-tuning (SFT) suffers from missing ground truth for these decisions, standard reinforcement learning (RL) struggles with the combinatorial explosion of code-integrated reasoning paths. Our innovation bridges these approaches through systematic exploration of both code-enabled (\(c=1\)) and non-code (\(c=0\)) solution paths, constructing reference decisions for policy optimization.

We formalize this idea within a maximum likelihood estimation (MLE) framework. Let \( P (r=1 | x_q;\theta\) denote the probability of generating a correct response to query \( x_q \) under model \(\mathcal{M}_\theta\). Our objective becomes:
\begin{align}
    \mathcal{J}_{\mathrm{MLE}}(\theta) \doteq \log P(r=1 | x_q; \theta) \label{eq_mle}
\end{align}
This likelihood depends on two latent factors: (1) the code triggering decision \(\pi_\theta(c | x_q)\) and (2) the solution generation process \(\pi_\theta(y_a | x_q, c)\). Here, for notation-wise clarity, we consider  code-triggering decision at a solution's beginning (\( c\) following \(x_q\) immediately). We show generalization to mid-reasoning code integration in Sec.~\ref{sec_impl}.

The EM framework provides a principled way to optimize this MLE objective in the presence of latent variables~\cite{prml}. We derive the evidence lower bound (ELBO): \( \mathcal{J}_{\mathrm{ELBO}}(s, \theta) \doteq \)
\begin{align}
    % \mathcal{J}_{\mathrm{MLE}}(\theta) &
    % \ge 
    \mathbb{E}_{s(c | x_q)}\left[\log \frac{\pi_\theta(c | x_q) \cdot P(r=1 | c, x_q; \theta)}{s(c | x_q)}\right] 
    % \\
     \label{eq_elbo}
\end{align}
where \(s(c | x_q)\) serves as a surrogate distribution approximating optimal code triggering strategies. It is also considered as the reference decisions for code integration. 

\noindent\textbf{E-step: Guided Exploration}  computes the reference strategy \(s(c | x_q)\) by maximizing the ELBO, equivalent to minimizing the KL-divergence: \( \max_s \mathcal{J}_{\mathrm{ELBO}}(s, \theta) = \)
\begin{align}
     - \mathrm{D_{KL}}\left(s(c | x_q) \| P(r=1, c | x_q; \theta)\right) \label{eq_estep}
\end{align}

The reference strategy \(s(c | x_q)\) thus approximates the posterior distribution over code-triggering decisions \(c\) that maximize correctness, i.e., \(P(r=1, c | x_q; \theta)\).  Intuitively, it guides exploration by prioritizing decisions with high potential: if decision \(c\) is more likely to lead to correct solutions, the reference strategy assigns higher probability mass to it, providing guidance for the subsequent RL procedure.

\noindent\textbf{M-step: Self-Refinement } updates the model parameters \(\theta\) through a composite objective:
\begin{multline}
\max_\theta \mathcal{J}_{\mathrm{ELBO}}(s, \theta) =\mathbb{E}_{\substack{c \sim s(c|x_q) \\ y_a \sim \pi_\theta(y_a|x_q, c)}} \Big[ r(x_q, y_a) \Big] \\- \mathcal{CE}\Big(s(c|x_q) \,\|\, \pi_\theta(c|x_q)\Big)\label{eq_mstep}
\end{multline}
The first term implements reward-maximizing policy gradient updates for solution generation, while while the second aligns native code triggering with reference strategies through cross-entropy minimization (see Fig.~\ref{fig_overview} for an illustration of the optimization). This dual optimization jointly enhances both tool-usage policies and reasoning capabilities.



\subsection{Practical Implementation}\label{sec_impl}
In the above EM framework, we alternate between finding a reference strategy \( s \) for code-triggering decisions  in the E-step, and perform reinforcement learning under the guidance from \( s \) in the M-step. We implement this framework through an iterative process of offline data curation and off-policy RL.

\noindent\textbf{Offline Data Curation.} We implement the E-step through Monte Carlo rollouts and subsampling. For each problem \(x_q\), we estimate the reference strategy as an energy distribution: 
\begin{equation}
    s^\ast(c | x_q)  = \frac{\exp\left(\alpha\cdot \pi_\theta(c | x_q) Q(x_q,c;\theta)\right)}{Z(x_q)}.\label{eq_sampling}
\end{equation}
where \( Q(x_q,c;\theta)\) estimates the expected value through \( K \) rollouts per decision, \(\pi_\theta(c|x_q) \) represents the model's current prior and the \( Z(x_q) \) is the partition function to ensure normalization. Intuitively, the strategy will assign higher probability mass to the decision \( c \) that has higher expected value \( Q(x_q,c;\theta)\) meanwhile balancing its intrinsic preference \( \pi_\theta(c|x_q)\). 

Our curation pipeline proceeds through: 
\begin{itemize}[leftmargin=0.5cm,topsep=1pt,itemsep=0pt,parsep=0pt]
\item Generate \(K\) rollouts for \(c=0\) (pure reasoning) and \(c=1\) (code integration), creating candidate dataset \(\mathcal{D}\).  
\item Compute \(Q(x_q,c)\) as the expected success rate across rollouts for each pair \((x_q,c)\).  
\item Subsample \(\mathcal{D}_{\text{train}}\) from \(\mathcal{D}\) using importance weights according to Eq.~\ref{eq_sampling}.  
\end{itemize}

To explicitly probe code-integrated solutions, we employ prefix-guided generation -- e.g., prepending prompts like \texttt{``Let’s first analyze the problem, then consider if python code could help''} -- to bias generations toward free-form code-reasoning patterns.

 This pipeline enables guided exploration by focusing on high-potential code-integrated trajectories identified by the reference strategy, contrasting with standard RL’s reliance on local policy neighborhoods. As demonstrated in Sec.~\ref{sec_ablation}, this strategic data curation significantly improves training efficiency by shaping the exploration space.





\noindent\textbf{Off-Policy RL.}
To mitigate distributional shifts caused by mismatches between offline data and the policy, we optimize a clipped off-policy RL objective. The refined M-step (Eq.~\ref{eq_mstep}) becomes:
\begin{multline}
    % \max_\theta 
    \underset{(x_q,y_a)}{\mathbb{E}}\left[
\text{clip}\left(\frac{\pi_\theta(y_a|x_q)}{\pi_{\text{ref}}(y_a|x_q)},1-\epsilon,1+\epsilon\right)\cdot A\right]
\\-\mathbb{E}_{(x_q,c)}\Big[\log \pi_\theta(c|x_q) \Big]\label{eq_finalm}
\end{multline}
where  \( (x_q, c, y_a) \) is sampled from the dataset \( \mathcal{D}_{\text{train}} \). The importance weight \(\frac{\pi_\theta(y_a|x_q)}{\pi_{\text{ref}}(y_a|x_q)}\) accounts for off-policy correction with PPO-like clipping. The advantage function \(A(x_q,y_a)\) is computed via query-wise reward normalization~\cite{ppo}. 

\noindent\textbf{Generalizing to Mid-Reasoning Code Integration.} Our method extends to mid-reasoning code integration by initiating Monte Carlo rollouts from partial solutions \((x_q, y_{<t})\). Notably, we observe emergence of mid-reasoning code triggers after initial warm-up with prefix-probed solutions. Thus, our implementation requires only two initial probing strategies: explicit prefix prompting for code integration and vanilla generation for pure reasoning, which jointly seed diverse mid-reasoning code usage in later iterations.

\section{Training Strategy}
\begin{table}[h!]
\centering
\resizebox{\textwidth}{!}{
\begin{tabular}{ccccccc}
\hline training stage & dataset & bs/node & learning rate & \#iters & \#seen samples \\
\hline 
\hline
    \multirow{3}{*}{Step-1: T2I Pre-training (256px)} & $\mathcal{O}(1) \mathrm{B}$ images & 40 & 1e-4 & 53k & 0.8B \\
     & $\mathcal{O}(1) \mathrm{B}$ images & 40 & 1e-4 & 200k & 3B \\
     \cline{2-6}
     & \textbf{Total} &  &  &  \textbf{253k} & \textbf{3.8B} \\
\hline 
\hline
    \multirow{4}{*}{Step-2: T2VI Pre-training (192px)} & $\mathcal{O}(1) \mathrm{B}$ video clips & 4 & 6e-5 & 171k & 256M\\
    & $\mathcal{O}(100) \mathrm{M}$ video clips & 4 & 6e-5 & 101k & 151M \\
    & $\mathcal{O}(100) \mathrm{M}$ video clips & 4 & 6e-5 & 158k & 237M \\
    \cline{2-6}
    & \textbf{Total} &  &  &  \textbf{430k} & \textbf{644M} \\
\hline
\hline
    \multirow{4}{*}{Step-2: T2VI Pre-training (540px)} & $\mathcal{O}(100) \mathrm{M}$ video clips & 2 & 2e-5 & 23k & 17.3M\\
    & $\mathcal{O}(10) \mathrm{M}$ video clips & 2 & 1e-5 & 17k & 8.5M \\
    & $\mathcal{O}(1) \mathrm{M}$ video clips & 1 & 1e-5 & 6k & 1.5M \\
    \cline{2-6}
    & \textbf{Total} &  &  &  \textbf{46k} & \textbf{27.3M} \\
\hline
\end{tabular}
}
\caption{Pre-training details of Step-Video-T2V. 256px, 192px, and 540px denote resolutions of 256x256, 192x320, and 544x992, respectively.}
\label{trainingrecipe}
\end{table}



\begin{figure}[h] 
    \centering
    \includegraphics[width=0.5\textwidth]{figure/v2_training_loss.png}  
    \caption{Training curve of different training stages, where $s_{i}$ denotes the $i^{th}$ dataset used in the corresponding stage.} 
    \label{fig:training curve}  
\end{figure}

A cascaded training strategy is employed in Step-Video-T2V, which mainly includes four steps: text-to-image (T2I) pre-training, text-to-video/image (T2VI) pre-training, text-to-video (T2V) fine-tuning, and direct preference optimization (DPO) training. The pre-training recipe is summarized in Table~\ref{trainingrecipe}.

\paragraph{Step-1: T2I Pre-training} In the initial step, we begin by training Step-Video-T2V with a T2I pre-training approach from scratch. We intentionally avoid starting with T2V pre-training directly, as doing so will significantly slow down model convergence. This conclusion stems from our early experiments with the T2V pre-training from scratch on the 4B model, where we observed that the model struggled to learn new concepts and was much slower to converge. By first focusing on T2I, the model can establish a solid foundation in understanding visual concepts, which can later be expanded to handle temporal dynamics in the T2V phase.

\paragraph{Step-2: T2VI Pre-training} After acquiring spatial knowledge from T2I pre-training in Step-1, Step-Video-T2V progresses to a T2VI joint training stage, where both T2I and T2V are incorporated. This step is further divided into two stages. In the first stage, we pre-train Step-Video-T2V using low-resolution (192x320, 192P) videos, allowing the model to primarily focus on learning motion-related knowledge rather than fine details. In the second stage, we increase the video resolution to 544x992 (540P) and continue pre-training to enable the model to learn more intricate details. We observed that during the first stage, the model concentrates on learning motion, while in the second stage, it shifts its focus more toward learning fine details. Based on these observations, we allocate more computational resources to the first stage in Step-2 to better capture motion knowledge.

\paragraph{Step-3: T2V Fine-tuning} Due to the diversity in pre-training video data across different domains and qualities, using a pre-trained checkpoint usually introduces artifacts and varying styles in the generated videos. To mitigate these issues, we continue the training pipeline with a T2V fine-tuning step. In this stage, we use a small number of text-video pairs and remove T2I, allowing the model to fine-tune and adapt specifically to text-to-video generation.


Similar to Movie Gen Video, we found that averaging models fine-tuned with different SFT datasets improves the quality and stability of the generated videos, outperforming the Exponential Moving Average (EMA) method. Even averaging checkpoints from the same data source enhances stability and reduces distortions. Additionally, we select model checkpoints based on the period after the gradient norm peaks, ensuring both the gradient norm and loss have decreased for improved stability.

\paragraph{Step-4: DPO Training}
As described in \S\ref{dpo}, video-based DPO training is employed to enhance the visual quality of the generated videos and ensure better alignment with user prompts.



\paragraph{Hierarchical Data Filtering}
\begin{figure*}[t]
    \centering
    \includegraphics[width=1.3\textwidth, center, trim=0 0 0 0, clip]{figure/data/data_filter.png}
    \caption{Hierarchical data filtering for pre-training and post-training.}
    \label{fig:data_filter}
    %\vspace{-6mm}
\end{figure*}


We apply a series of filters to the data, progressively increasing their thresholds to create six pre-training subsets for Step-2: T2VI Pre-training, as shown in Table~\ref{trainingrecipe}. The final SFT dataset is then constructed through manual filtering. Figure~\ref{fig:data_filter} illustrates the key filters applied at each stage, with gray bars representing the data removed by each filter, and colored bars indicating the remaining data at each stage.



\paragraph{Observations from Pre-training Curve}
%We use a funnel-style filtering method to progressively refine the training dataset throughout the pre-training stage. 
During pre-training, we observe a notable reduction in loss, which correlates with the improved quality of the training data, as illustrated in Figure \ref{fig:training curve}.

Additionally, a sudden drop in loss occurs as the quality of the training dataset improves. This improvement is not directly driven by supervision through a loss function during model training, but rather follows human intuition (e.g., filtering via CLIP scores, aesthetic scores, etc.). While the flow matching algorithm does not impose strict requirements on the distribution of the model’s input data, adjusting the training data to reflect what is considered higher-quality by humans results in a significant, stepwise reduction in training loss. This suggests that, to some extent, the model’s learning process may emulate human cognitive patterns.


\paragraph{Bucketization for Variable Duration and Size}

To accommodate varying video lengths and aspect ratios during training, we employed variable-length and variable-resolution strategies~\cite{chen2023pixartalphafasttrainingdiffusion, opensora}. We defined four length buckets (1, 68, 136, and 204 frames) and dynamically adjusted the number of latent frames based on the video length. Additionally, we grouped videos into three aspect ratio buckets—landscape, portrait, and square—according to the closest height-to-width ratio.




\section{NovelSpecies Dataset}
\label{sec:novel_dataset}

Proprietary LMMs like GPT4o~\cite{hurst2024gpt4o} and Gemini~\cite{team2023gemini} are trained on vast online text-image data and proprietary data, both non-public and impossible to inspect. Some open-source and open-data LMMs such as LLaVA~\cite{liu2024improved, liu2024visual} are trained on publicly available image-text datasets. However, the text encoders used by such models are often not open-data, for example LLaVA-1.6 34B uses the closed-data Yi-34B model as its language backbone. Even in the rare cases where both image-text training data and text encoder training data are publicly available, it is still difficult to ascertain whether concepts in your benchmark were seen by your LMM through indirect data leakage (i.e. partial / paraphrased mentions). Due to the above issues, it is difficult to evaluate true novel concept recognition ability with existing datasets. 
% \footnote{Knowledge cutoff date: Dec 2023}

One way to bypass this problem with 100\% guaranteed success is to use a dataset that only contains concepts created / discovered after the LMM's knowledge cutoff, i.e. the latest knowledge cutoff date among all of its textual / visual sub-components. Based on this idea, we curate \textbf{NovelSpecies}, a dataset of novel animal species discovered in each recent year, starting with 2023 and 2024. We provide detailed information for each species, including time of discovery, latin name, common name, family category, textual description, and more. Data will be released upon publication.
% Details are described in Sec.\ref{subsec:NovelSpecies_details}.

To create \textbf{NovelSpecies}, we start by collecting the list of species first described in each year by Wikidata~\cite{wikidata}. Then, to make sure we can curate a visual benchmark of novel species, we manually annotate and filter out extinct species and species with too few publicly available images. After filtering, we end up with a dataset of 64 new species, each consisting of 35 human-verified image instances, thus a total of 2240 images. The images are split into training, validation, and test sets. For each specie, there are 5 training images, 15 validation images, and 15 test images. This data split is consistent with our goal of creating a benchmark dataset for novel concept recognition, where the maximum number of training instances for a completely unseen concept can range from 1 to 5.







% and 2170 images in total, which consist of train, validation, and test sets of equal proportion for all species. Finally, all the images are 















% \section{Datasets}
% \label{sec:dataset}


% \subsection{Confusing Pair Extraction}
% Our focus on confusing pairs arises from the need to strengthen the model's performance in distinguishing between visually similar species—a challenge where LLaVA currently shows limitations. Confusing pairs represent instances where the model's classification often fails, typically due to subtle visual cues or shared features among species within similar taxonomic groups. We designed strategy to extract confusing pair for each dataset.

% \paragraph{INaturalist and Novel Species Dataset} We extract confusing pairs with three-steps as following: 

% \begin{enumerate}
%     \item \textbf{Iterative Subset Selection:} We select a random subset of species in each iteration, sampling across different supercategories. This strategy allows us to identify confusing pairs without overloading the system, progressively building a collection of challenging cases from each subset.
%     \item \textbf{Evaluate Classification Patters:} For each species within a subset, we create prompts in a multiple-choice format, incorporating the image and a randomized list of options from all the species in the subset. Based on the response from LLaVA, we are able to highlight specific species that are commonly mistaken for one another, guiding us in selecting pairs for further analysis. The process is repeated across new subsets, incrementally building an ample dataset of confusing pairs.
%     \item \textbf{Identification of confusing pairs: } We choose a threshhold of 0.2. If class A is misclassified into class B with frequency more than 0.2 in the above multiple-choice setting, we consider the pair to be confusing. 
% \end{enumerate}

% \paragraph{SUN Dataset} We adapted the above methodology for scene classification with minor modification on the subset selection process. Instead of taxonomic groupings, we created subsets by selecting a target scene and the nine most similar scenes based on shared object occurrence. The subsequent steps—classification pattern analysis and confusing pair definition—remained consistent with the species datasets.









% \subsection{Curated INaturalist Dataset}
% In this study, we utilize a random sample of 15 classes from the "Mammals" supercategory of the iNaturalist dataset. Below, we outline the reasoning behind our dataset selection and sampling approach.
% \paragraph{iNaturalist Dataset}
% The iNaturalist dataset is known for its complexity and has proven to be a challenging benchmark for many vision-language models. Due to the extensive diversity and fine-grained nature of the categories, most models do not achieve perfect performance on this dataset, leaving ample room for further improvements.
% \paragraph{Sampling Strategy}
% Given the scale of the iNaturalist dataset, which contains approximately 10,000 classes with 50 images per class, it is necessary to reduce its size for practical purposes. Additionally, current models, such as LLaVA, have limitations in handling an excessive number of options. Therefore, we have opted to sample the dataset to manage the number of classes and reduce the computational load.
% \paragraph{Random Sampling Justification}
% Initially, we considered sampling all species from a single order, family, or genus. However, this approach resulted in classes that were too similar, making the classification task more challenging than our models could handle. By employing random sampling, the selected classes that are likely sufficiently distinct from each other, with only a few potentially confusing cases.

% Random sampling also reduces the risk of introducing human bias into the selection process, making it a more defensible approach compared to sampling based on performance metrics. 

% \subparagraph{Data Filtering}
% iNaturalist dataset contains a large number of noisy or low quality images. To ensure the quality of the dataset, we implemented an automatic filtering process to eliminate low-quality images. This step is crucial to prevent noise from negatively impacting model performance. Common issues in low-quality images include:

% 1. \textbf{Blurriness}: Images where the main subject is not in focus.
% 2. \textbf{Species Not Present}: Instances where the species is not visible (e.g., only showing its nest or footprint).
% 3. \textbf{Incomplete Specimen}: Images depicting only parts of deceased animals or broken bodies.
% 4. \textbf{Obstructions}: Cases where the species is almost entirely blocked by objects, making identification impossible.

% To improve image quality, we use CLIP score to select the images with top scores. Scores are calculated by evaluating similarity score with [
%         "a photo of an animal",
%         f"a photo of a \{common\_name\}"
%     ]. We rank the images according to this score and selected top 100 images. We randomly split the images to obtain 50 images for training, 20 images for validation and 30 images for testing. 



%展望未来及 MLLM for VDU 的应用场景
\section{Challenges and Trends}
\label{sec:summary}

%According to the evolutionary trends of model architectures shown in Table \ref{tab:mllm_summary}, it can be observed that the model structures are beginning to evolve in two directions: 1. The number of model parameters is decreasing. For instance, Mini-monkey, with 2B parameters, has achieved results comparable to those of 7B models on multiple TIU tasks. 2. Models are capable of processing longer inputs. For example, DocOwl2.0, by compressing visual tokens, can handle more input images while maintaining the same performance metrics. In terms of training strategies, in addition to MA and IA, PA is also being introduced to enhance the model's question-answering capabilities and overall performance. Moreover, it is conceivable that with the ongoing popularity of reinforcement learning sparked by DeepSeek R1, training strategies using GRPO will also emerge during the PA phase in the MLLM field.

As shown in Table \ref{tab:mllm_summary}, we calculated the average scores from four popular and widely used evaluation datasets, which can basically reflect the performance of MLLMs on TIU tasks. The top five models are Qwen2-VL-72B (88.70), InternVL2.5-78B (87.73), InternVL2.5-38B (87.45), InternVL2.5-26B (85.85), and DeepSeek-VL2-27B (85.40). This indicates that the most state-of-the-art (SOTA) MLLMs currently employ OCR-free encoders, which avoids redundant tokens and complex model architectures. Despite the promising and significant progress made by current MLLMs, the field still faces considerable challenges that require further research and innovation:

% the latest MLLM model architectures predominantly adopt an OCR-free encoder and  Qwen LLM, achieving significant progress on mainstream benchmarks. For instance, the metric on DocVQA has reached a peak of 96.5\%. However, the field still faces significant challenges that demand further research and innovation:

\noindent \textbf{Computational Efficiency and Model Compression}. The computational demands of current MLLMs remain a critical bottleneck, primarily due to two factors: (1) the necessity of processing high-resolution document images, which imposes substantial computational resource requirements, and (2) the prevalent use of 7-billion-parameter architectures, while delivering state-of-the-art performance, incur high deployment costs and latency. These challenges underscore the importance of developing more efficient MLLM architectures that balance performance with reduced computational overhead. Encouragingly, recent advancements, as illustrated in Table~\ref{tab:mllm_summary}, demonstrate promising trends toward model miniaturization. For instance, Mini-monkey~\cite{Huang2024ARXIV_Mini_Monkey_Alleviating} achieves performance comparable to 7B-parameter models on multiple TIU tasks while utilizing only 2B parameters, highlighting the potential for lightweight yet powerful architectures.

\noindent \textbf{Optimization of Visual Feature Representation}. A persistent challenge in MLLMs is the disproportionate length of image tokens compared to text tokens, which significantly increases computational complexity and degrades inference efficiency. Addressing this issue requires innovative approaches to compress image tokens without sacrificing model performance. Promising directions include the development of efficient visual encoders, adaptive token compression mechanisms, and advanced techniques for cross-modal feature fusion. Crucially, these methods must preserve the semantic richness of document content during compression. As shown in Table~\ref{tab:mllm_summary}, recent architectural innovations, such as mPLUG-DocOwl2‘s ~\cite{Hu2024ARXIV_mPLUG_DocOwl2_High} visual token compression, have made strides in this direction by enabling the processing of larger input images while maintaining benchmark performance.

\noindent \textbf{Long Document Understanding Capability}. While MLLMs excel at single-page document understanding, their performance on multi-page or long-document tasks remains suboptimal. Key challenges include modeling long-range dependencies, maintaining contextual coherence across pages, and efficiently processing extended sequences. The emergence of specialized benchmarks for long-document understanding~\cite{Ma2024NEURIPS_MMLongBench_Doc_Benchmarking}, as highlighted in Table~\ref{tab:datawithbenchmark}, is expected to drive significant progress in this field by providing standardized evaluation frameworks and fostering targeted research efforts.

\noindent \textbf{Multilingual Document Understanding}. Current MLLMs are predominantly optimized for English and a limited set of high-resource languages, resulting in inadequate performance in multilingual and low-resource language scenarios. Addressing this limitation requires the development of comprehensive multilingual datasets that encompass diverse linguistic and cultural contexts. Recent initiatives, such as MT-VQA~\cite{tang2024mtvqa} and CC-OCR~\cite{yang2024cc} (referenced in Table~\ref{tab:datawithbenchmark}), represent important steps forward by introducing TIU tasks specifically designed to evaluate multilingual capabilities. These efforts, coupled with advances in cross-lingual transfer learning, are expected to significantly enhance the inclusivity and applicability of MLLMs in global contexts.

%\noindent \textbf{Evaluation Standards and New Benchmarks}. The current evaluation standards for TIU tasks are predominantly based on limited benchmark datasets, which may not fully capture the model's performance in real-world scenarios, especially the reasoning ablity. Furthermore, the existing evaluation metrics are often overly simplistic, lacking the ability to comprehensively assess the model's multidimensional capabilities. To address these limitations, it is crucial to develop more comprehensive evaluation benchmarks that encompass a diverse range of document types, languages, and domains.
\section{Limitations}

This paper provides a systematic review of multimodal large language models (MLLMs) in the field of Text-rich Image Understanding (TIU). While the research team has conducted comprehensive retrieval and integration of core literature prior to the submission deadline, certain minor studies may still remain uncovered. It should be particularly noted that due to publisher formatting requirements, the exposition of existing technical approaches and benchmark datasets in this work maintains essential conciseness. For complete algorithmic implementation details and experimental parameter configurations, researchers are strongly recommended to consult the original publications.

%按场景介绍TIU 领域常用的 benchmark, eval metrics,
%分析当前 benchmark 存在的问题后,提出新的 benchmark,并提供当前已知可测试的 MLLM 模型在这个 benchmark 的结果





% \subsection{Evaluation metrics}
% Evaluation metrics play a critical role in assessing the performance of MLLM across various tasks and benchmarks. They include Precision, Recall, F1 Score, Accuracy, MAP, and ANLS. More details will be introduced in Supplemental Martial.
% Specifically, we introduce the important ANLS metric in detail:

% The key evaluation metrics are as follows:

% \textcolor{red}{\subsubsection{Precision, Recall, F1 Score, and Accuracy}}

% Precision, Recall, F1 Score, and Accuracy are pivotal in assessing model performance across diverse tasks.

% \noindent \textbf{Precision} measures the model's ability to correctly identify positive instances within multi-modal data. This metric is calculated as follows:
% \begin{gather}
%     \mathrm{Precision}=\frac{\text{TP}}{\text{TP}+\text{FP}}
% \end{gather}

% \noindent \textbf{Recall} evaluates the model's capability to identify all relevant positive instances in multi-modal tasks. This metric is calculated as follows:
% \begin{gather}
%     \mathrm{Recall}=\frac{\text{TP}}{\text{TP}+\text{FN}}
% \end{gather}

% \noindent \textbf{F1 Score}, as the harmonic mean of precision and recall, is particularly valuable in scenarios where a balance between these two metrics is desired. This metric is calculated as follows:
% \begin{gather}
%     \mathrm{F1~Score}=\frac{2\times\mathrm{Precision}\times\mathrm{Recall}}{\mathrm{Precision}+\mathrm{Recall}}
% \end{gather}

% \noindent \textbf{Accuracy} represents the proportion of correctly predicted observations and is the most straightforward performance metric. This metric is calculated as follows:
% \begin{gather}
%     \mathrm{Accuracy}=\frac{\text{TP}+\text{TN}}{\text{Total Samples}}
% \end{gather}
% Where, TP (TP) are the correctly predicted positive samples. FP (FP) are the incorrectly predicted positive samples. FN (FN) are the incorrectly predicted negative samples. TN (TN) are the correctly predicted negative samples. Total Samples is the total number of samples
% in the dataset.

% \subsubsection{Mean Average Precision (MAP)}
% MAP is employed to evaluate the precision of retrieval tasks involving multi-modal data, such as retrieving the most relevant images from a database based on a given text description. MAP is employed to evaluate the precision of retrieval tasks involving multi-modal data, such as retrieving the most relevant images from a database based on a given text description.  MAP effectively measures the model's average performance across multiple queries, ensuring the model excels not only in isolated instances but consistently. MAP effectively measures the model's average performance across multiple queries, ensuring the model excels not only in isolated instances but consistently. The formula for calculating MAP is as follows:
% \begin{gather}
% \mathrm{MAP}=\frac{1}{Q}\sum_{q=1}^Q\mathrm{AP}(q)
% \end{gather}
% Where, $Q$ is the total number of queries. $\mathrm{AP}(q)$ is the Average Precision for query $q$, calculated as the average precision of relevant documents at each position in the ranked list:
% \begin{gather}
% \mathrm{AP}(q)=\frac{1}{N_q}\sum_{k=1}^{N_q}\mathrm{Precision}(k)\times\mathrm{Relevance}(k)
% \end{gather}
% Where, $N_q$ is the total number of retrieved documents for query $q$. $\mathrm{Precision}(k)$ is the precision at position $k$ in the ranked list. $\mathrm{Relevance}(k)$ is an indicator function that
% takes the value 1 if the document at position
% $k$ is relevant to query $q$, and 0 otherwise.

% \textcolor{red}{\subsubsection{Average Normalized Levenshtein Similarity (ANLS)}}
% ANLS~\cite{biten2019scene} captures the OCR mistakes applying a slight penalization in case of correct intended responses, but badly recognized. It also makes use of a threshold of value 0.5 that dictates whether the output of the metric will be the ANLS if its value is equal to or bigger than 0.5 or 0 otherwise. The key point of this threshold is to determine if the answer has been correctly selected but not properly recognized, or on the contrary, the output is a wrong text selected from the options and given as an answer.

% More formally, the ANLS between the net output and the ground truth answers is given by the following equation. Where $N$ is the total number of questions, $M$ total number of GT answers per question, $a_{ij}$ the ground truth answers where $i = {0, ..., N}$, and $j = {0, ..., M}$, and $o_{qi}$ be the network’s answer for the ith question $q_i$:
% \begin{gather}
% \mathrm{ANLS}=\frac{1}{N} \sum_{i=0}^N\left(\max _j s\left(a_{i j}, o_{q i}\right)\right)
% \end{gather}
% Where, $s\left(a_{i j}, o_{q i}\right)$ is defined as follows:
% \begin{gather}
% s\left(a_{i j}, o_{q i}\right)= \begin{cases}1-\mathrm{NL}\left(a_{i j}, o_{q i}\right), & \text { if } \mathrm{NL}\left(a_{i j}, o_{q i}\right)< \tau\\ 0, & \text { if } \mathrm{NL}\left(a_{i j}, o_{q i}\right) \geq\tau \end{cases}
% \end{gather}



\bibliography{main}
\end{document}

\begin{abstract}
This document is a supplement to the general instructions for *ACL authors. It contains instructions for using the \LaTeX{} style files for ACL conferences.
The document itself conforms to its own specifications, and is therefore an example of what your manuscript should look like.
These instructions should be used both for papers submitted for review and for final versions of accepted papers.
\end{abstract}

\section{Introduction}

These instructions are for authors submitting papers to *ACL conferences using \LaTeX. They are not self-contained. All authors must follow the general instructions for *ACL proceedings,\footnote{\url{http://acl-org.github.io/ACLPUB/formatting.html}} and this document contains additional instructions for the \LaTeX{} style files.

The templates include the \LaTeX{} source of this document (\texttt{acl\_latex.tex}),
the \LaTeX{} style file used to format it (\texttt{acl.sty}),
an ACL bibliography style (\texttt{acl\_natbib.bst}),
an example bibliography (\texttt{custom.bib}),
and the bibliography for the ACL Anthology (\texttt{anthology.bib}).

\section{Engines}

To produce a PDF file, pdf\LaTeX{} is strongly recommended (over original \LaTeX{} plus dvips+ps2pdf or dvipdf). Xe\LaTeX{} also produces PDF files, and is especially suitable for text in non-Latin scripts.

\section{Preamble}

The first line of the file must be
\begin{quote}
\begin{verbatim}
\documentclass[11pt]{article}
\end{verbatim}
\end{quote}

To load the style file in the review version:
\begin{quote}
\begin{verbatim}
\usepackage[review]{acl}
\end{verbatim}
\end{quote}
For the final version, omit the \verb|review| option:
\begin{quote}
\begin{verbatim}
\usepackage{acl}
\end{verbatim}
\end{quote}

To use Times Roman, put the following in the preamble:
\begin{quote}
\begin{verbatim}
\usepackage{times}
\end{verbatim}
\end{quote}
(Alternatives like txfonts or newtx are also acceptable.)

Please see the \LaTeX{} source of this document for comments on other packages that may be useful.

Set the title and author using \verb|\title| and \verb|\author|. Within the author list, format multiple authors using \verb|\and| and \verb|\And| and \verb|\AND|; please see the \LaTeX{} source for examples.

By default, the box containing the title and author names is set to the minimum of 5 cm. If you need more space, include the following in the preamble:
\begin{quote}
\begin{verbatim}
\setlength\titlebox{<dim>}
\end{verbatim}
\end{quote}
where \verb|<dim>| is replaced with a length. Do not set this length smaller than 5 cm.

\section{Document Body}

\subsection{Footnotes}

Footnotes are inserted with the \verb|\footnote| command.\footnote{This is a footnote.}

\subsection{Tables and figures}

See Table~\ref{tab:accents} for an example of a table and its caption.
\textbf{Do not override the default caption sizes.}

\begin{table}
  \centering
  \begin{tabular}{lc}
    \hline
    \textbf{Command} & \textbf{Output} \\
    \hline
    \verb|{\"a}|     & {\"a}           \\
    \verb|{\^e}|     & {\^e}           \\
    \verb|{\`i}|     & {\`i}           \\
    \verb|{\.I}|     & {\.I}           \\
    \verb|{\o}|      & {\o}            \\
    \verb|{\'u}|     & {\'u}           \\
    \verb|{\aa}|     & {\aa}           \\\hline
  \end{tabular}
  \begin{tabular}{lc}
    \hline
    \textbf{Command} & \textbf{Output} \\
    \hline
    \verb|{\c c}|    & {\c c}          \\
    \verb|{\u g}|    & {\u g}          \\
    \verb|{\l}|      & {\l}            \\
    \verb|{\~n}|     & {\~n}           \\
    \verb|{\H o}|    & {\H o}          \\
    \verb|{\v r}|    & {\v r}          \\
    \verb|{\ss}|     & {\ss}           \\
    \hline
  \end{tabular}
  \caption{Example commands for accented characters, to be used in, \emph{e.g.}, Bib\TeX{} entries.}
  \label{tab:accents}
\end{table}

As much as possible, fonts in figures should conform
to the document fonts. See Figure~\ref{fig:experiments} for an example of a figure and its caption.

Using the \verb|graphicx| package graphics files can be included within figure
environment at an appropriate point within the text.
The \verb|graphicx| package supports various optional arguments to control the
appearance of the figure.
You must include it explicitly in the \LaTeX{} preamble (after the
\verb|\documentclass| declaration and before \verb|\begin{document}|) using
\verb|\usepackage{graphicx}|.

\begin{figure}[t]
  \includegraphics[width=\columnwidth]{example-image-golden}
  \caption{A figure with a caption that runs for more than one line.
    Example image is usually available through the \texttt{mwe} package
    without even mentioning it in the preamble.}
  \label{fig:experiments}
\end{figure}

\begin{figure*}[t]
  \includegraphics[width=0.48\linewidth]{example-image-a} \hfill
  \includegraphics[width=0.48\linewidth]{example-image-b}
  \caption {A minimal working example to demonstrate how to place
    two images side-by-side.}
\end{figure*}

\subsection{Hyperlinks}

Users of older versions of \LaTeX{} may encounter the following error during compilation:
\begin{quote}
\verb|\pdfendlink| ended up in different nesting level than \verb|\pdfstartlink|.
\end{quote}
This happens when pdf\LaTeX{} is used and a citation splits across a page boundary. The best way to fix this is to upgrade \LaTeX{} to 2018-12-01 or later.

\subsection{Citations}

\begin{table*}
  \centering
  \begin{tabular}{lll}
    \hline
    \textbf{Output}           & \textbf{natbib command} & \textbf{ACL only command} \\
    \hline
    \citep{Gusfield:97}       & \verb|\citep|           &                           \\
    \citealp{Gusfield:97}     & \verb|\citealp|         &                           \\
    \citet{Gusfield:97}       & \verb|\citet|           &                           \\
    \citeyearpar{Gusfield:97} & \verb|\citeyearpar|     &                           \\
    \citeposs{Gusfield:97}    &                         & \verb|\citeposs|          \\
    \hline
  \end{tabular}
  \caption{\label{citation-guide}
    Citation commands supported by the style file.
    The style is based on the natbib package and supports all natbib citation commands.
    It also supports commands defined in previous ACL style files for compatibility.
  }
\end{table*}

Table~\ref{citation-guide} shows the syntax supported by the style files.
We encourage you to use the natbib styles.
You can use the command \verb|\citet| (cite in text) to get ``author (year)'' citations, like this citation to a paper by \citet{Gusfield:97}.
You can use the command \verb|\citep| (cite in parentheses) to get ``(author, year)'' citations \citep{Gusfield:97}.
You can use the command \verb|\citealp| (alternative cite without parentheses) to get ``author, year'' citations, which is useful for using citations within parentheses (e.g. \citealp{Gusfield:97}).

A possessive citation can be made with the command \verb|\citeposs|.
This is not a standard natbib command, so it is generally not compatible
with other style files.

\subsection{References}

\nocite{Ando2005,andrew2007scalable,rasooli-tetrault-2015}

The \LaTeX{} and Bib\TeX{} style files provided roughly follow the American Psychological Association format.
If your own bib file is named \texttt{custom.bib}, then placing the following before any appendices in your \LaTeX{} file will generate the references section for you:
\begin{quote}
\begin{verbatim}
\bibliography{custom}
\end{verbatim}
\end{quote}

You can obtain the complete ACL Anthology as a Bib\TeX{} file from \url{https://aclweb.org/anthology/anthology.bib.gz}.
To include both the Anthology and your own .bib file, use the following instead of the above.
\begin{quote}
\begin{verbatim}
\bibliography{anthology,custom}
\end{verbatim}
\end{quote}

Please see Section~\ref{sec:bibtex} for information on preparing Bib\TeX{} files.

\subsection{Equations}

An example equation is shown below:
\begin{equation}
  \label{eq:example}
  A = \pi r^2
\end{equation}

Labels for equation numbers, sections, subsections, figures and tables
are all defined with the \verb|\label{label}| command and cross references
to them are made with the \verb|\ref{label}| command.

This an example cross-reference to Equation~\ref{eq:example}.

\subsection{Appendices}

Use \verb|\appendix| before any appendix section to switch the section numbering over to letters. See Appendix~\ref{sec:appendix} for an example.

\section{Bib\TeX{} Files}
\label{sec:bibtex}

Unicode cannot be used in Bib\TeX{} entries, and some ways of typing special characters can disrupt Bib\TeX's alphabetization. The recommended way of typing special characters is shown in Table~\ref{tab:accents}.

Please ensure that Bib\TeX{} records contain DOIs or URLs when possible, and for all the ACL materials that you reference.
Use the \verb|doi| field for DOIs and the \verb|url| field for URLs.
If a Bib\TeX{} entry has a URL or DOI field, the paper title in the references section will appear as a hyperlink to the paper, using the hyperref \LaTeX{} package.

\section*{Acknowledgments}

This document has been adapted
by Steven Bethard, Ryan Cotterell and Rui Yan
from the instructions for earlier ACL and NAACL proceedings, including those for
ACL 2019 by Douwe Kiela and Ivan Vuli\'{c},
NAACL 2019 by Stephanie Lukin and Alla Roskovskaya,
ACL 2018 by Shay Cohen, Kevin Gimpel, and Wei Lu,
NAACL 2018 by Margaret Mitchell and Stephanie Lukin,
Bib\TeX{} suggestions for (NA)ACL 2017/2018 from Jason Eisner,
ACL 2017 by Dan Gildea and Min-Yen Kan,
NAACL 2017 by Margaret Mitchell,
ACL 2012 by Maggie Li and Michael White,
ACL 2010 by Jing-Shin Chang and Philipp Koehn,
ACL 2008 by Johanna D. Moore, Simone Teufel, James Allan, and Sadaoki Furui,
ACL 2005 by Hwee Tou Ng and Kemal Oflazer,
ACL 2002 by Eugene Charniak and Dekang Lin,
and earlier ACL and EACL formats written by several people, including
John Chen, Henry S. Thompson and Donald Walker.
Additional elements were taken from the formatting instructions of the \emph{International Joint Conference on Artificial Intelligence} and the \emph{Conference on Computer Vision and Pattern Recognition}.

% Bibliography entries for the entire Anthology, followed by custom entries
%\bibliography{anthology,custom}
% Custom bibliography entries only
\bibliography{custom}

\appendix

\section{Example Appendix}
\label{sec:appendix}

This is an appendix.

\end{document}
