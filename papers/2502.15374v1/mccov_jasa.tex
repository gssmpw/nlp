\documentclass[12pt]{article}

\pdfoutput=1
\usepackage{amsmath}
\usepackage{graphicx}
\usepackage{enumerate}
\usepackage{natbib}
\usepackage{url} % not crucial - just used below for the URL 
\usepackage{amsthm}
\usepackage[normalem]{ulem}


\RequirePackage{amsthm,amsmath,amsfonts,amssymb}
% \RequirePackage[numbers]{natbib}
% \RequirePackage[authoryear]{natbib}%% uncomment this for author-year citations
% \RequirePackage[colorlinks,citecolor=blue,urlcolor=blue]{hyperref}
\usepackage[colorlinks, linkcolor=blue, citecolor=blue, urlcolor=blue]{hyperref}

\usepackage{tikz-cd, pgfplots, tikz}
\usepackage{color}
\usepackage{mathrsfs}
\usepackage{url}
\usepackage{caption}
\usepackage{mathtools}
\usepackage{lipsum}
\usepackage{amssymb}
\usepackage{multirow}
\usepackage{float}
\usepackage{algorithmicx}
\usepackage[ruled,vlined]{algorithm2e}
\usepackage{bbm}
\usepackage{bm}
\usepackage{booktabs}
\usepackage{stfloats}
\usepackage{enumitem}
\usepackage{xr}
\usepackage{subfig}
\usepackage[doublespacing]{setspace}
\usepackage{enumerate}
\usepackage{enumitem}
\usepackage{xr}
\usepackage{authblk}

\externaldocument{../JASA-Supplement/mccov_jasa}
\usepackage[utf8]{inputenc}
\usepackage[ruled,vlined]{algorithm2e}

\DeclareMathOperator*{\argmin}{argmin}
\DeclareMathOperator*{\argmax}{argmax}
\DeclareMathOperator*{\diam}{diam}
\DeclareMathOperator*{\Vol}{Vol}
\DeclareMathOperator*{\ddim}{ddim}
\DeclareMathOperator*{\Var}{Var}
\newcommand*{\perm}[2]{{}^{#1}\!P_{#2}}
\newcommand{\ii}{\mathrm{i}}
\newcommand{\rdohne}{{\mathbb{R}^d\setminus\{0\}}}
\newcommand{\indep}{\perp\hspace*{-6.2pt}\perp}
\newcommand{\CCOV}{\text{CCOV}}
\newcommand{\FCCov}{\text{FCCov}}
\newcommand{\FC}{\text{FC}}

% 图片目录
\graphicspath{{./}} % 或者直接移除 \graphicspath{}

%\pdfminorversion=4
% NOTE: To produce blinded version, replace "0" with "1" below.
\newcommand{\blind}{0}

% DON'T change margins - should be 1 inch all around.
\addtolength{\oddsidemargin}{-.5in}%
\addtolength{\evensidemargin}{-1in}%
\addtolength{\textwidth}{1in}%
\addtolength{\textheight}{1.7in}%
\addtolength{\topmargin}{-1in}%

% Theorem
\theoremstyle{plain}
\newtheorem{axiom}{Axiom}
\newtheorem{claim}[axiom]{Claim}
\newtheorem{theorem}{Theorem}[section]
\newtheorem{lemma}[theorem]{Lemma}
\newtheorem{corollary}[theorem]{Corollary}
\newtheorem{assumption}{Assumption} 
\newtheorem*{assumption*}{Assumption} 
\newtheorem{remark}[theorem]{Remark}
\newtheorem{definition}[theorem]{Definition}
\newtheorem*{example}{Example}
\newtheorem*{fact}{Fact}

\newcounter{asscounter}[asscounter]

\newcommand{\N}{\mathbb{N}}
\newcommand{\n}{{(n)}}
\newcommand{\R}{\mathbb{R}}
\newcommand{\F}{{\cal F}}
\newcommand{\Z}{\mathbb{Z}}
\newcommand{\Rd}{\mathbb{R}^d}
\newcommand{\IND}{\mathbb{I}}
\newcommand{\PROB}{{\mathbf P}}
\newcommand{\ci}{\mathrel{\text{\scalebox{1.07}{$\perp\mkern-10mu\perp$}}}}
\newcommand{\mA}{\mathcal{A}}
\newcommand{\mB}{\mathcal{B}}
\newcommand{\mC}{\mathcal{C}}
\newcommand{\mD}{\mathcal{D}}
\newcommand{\mE}{\mathcal{E}}
\newcommand{\mF}{\mathcal{F}}
\newcommand{\mG}{\mathcal{G}}
\newcommand{\mH}{\mathcal{H}}
\newcommand{\mI}{\mathcal{I}}
\newcommand{\mJ}{\mathcal{J}}
\newcommand{\mL}{\mathcal{L}}
\newcommand{\mN}{\mathcal{N}}
\newcommand{\mS}{\mathcal{S}}
\newcommand{\mU}{\mathcal{U}}
\newcommand{\mV}{\mathcal{V}}
\newcommand{\mM}{\mathcal{M}}
\newcommand{\mO}{\mathcal{O}}
\newcommand{\mP}{\mathcal{P}}
\newcommand{\mQ}{\mathcal{Q}}
\newcommand{\mW}{\mathcal{W}}
\newcommand{\mX}{\mathcal{X}}
\newcommand{\mY}{\mathcal{Y}}
\newcommand{\mZ}{\mathcal{Z}}
\newcommand{\bzero}{0}
\newcommand{\bbH}{\mathbb{H}}
\newcommand{\compp}{\mathbb{C}}
\newcommand{\ba}{\mathbf{a}}
\newcommand{\bb}{\mathbf{b}}
\newcommand{\bc}{\mathbf{c}}
\newcommand{\bC}{\mathbf{C}}
\newcommand{\bd}{\mathbf{d}}
\newcommand{\bD}{\mathbf{D}}
\newcommand{\be}{\mathbf{e}}
\newcommand{\mf}{\boldsymbol{f}}
\newcommand{\mg}{\boldsymbol{g}}
\newcommand{\mh}{\boldsymbol{h}}
\newcommand{\bi}{\mathbf{i}}
\newcommand{\bh}{\mathbf{h}}
\newcommand{\bp}{\mathbf{p}}
\newcommand{\bP}{\mathbf{P}}
\newcommand{\bq}{\mathbf{q}}
\newcommand{\bs}{s}
\newcommand{\bt}{\mathbf{t}}
\newcommand{\bu}{\mathbf{u}}
\newcommand{\bv}{\mathbf{v}}
\newcommand{\bw}{\mathbf{w}}
\newcommand{\bk}{\mathbf{k}}
\newcommand{\bj}{\mathbf{j}}
\newcommand{\bx}{\mathbf{x}}
\newcommand{\bz}{\mathbf{z}}
\newcommand{\bZ}{Z}
\newcommand{\balpha}{\bm{\alpha}}
\newcommand{\bbeta}{\bm{\beta}}
\newcommand{\bdelta}{\bm{\delta}}
\newcommand{\btheta}{\bm{\theta}}
\newcommand{\bgamma}{\bm{\gamma}}
\newcommand{\blambda}{\bm{\lambda}}
\newcommand{\bLambda}{\bm{\Lambda}}
\newcommand{\bell}{\bm{\ell}}


% 是否显示作者信息
\def\blind{1}

\begin{document}
\date{}

%\bibliographystyle{natbib}

\def\spacingset#1{\renewcommand{\baselinestretch}%
{#1}\small\normalsize} \spacingset{1}


%%%%%%%%%%%%%%%%%%%%%%%%%%%%%%%%%%%%%%%%%%%%%%%%%%%%%%%%%%%%%%%%%%%%%%%%%%%%%%

\if1\blind
{
  \title{\bf Fr\'echet Cumulative Covariance Net for Deep Nonlinear Sufficient Dimension Reduction with Random Objects}
%   \author{Hang Yuan\\
%    %\thanks{
%     %The authors gratefully acknowledge \textit{please remember to list all relevant funding sources in the unblinded version}}\hspace{.2cm}\\
%     School of Statistics, East China Normal University,\\
%     Dan Wang \\
%     New York University Shanghai,\\
%     Zhou Yu \\
%     School of Statistics, East China Normal University}
    % 使用 \author 和 \affil 来定义作者和单位
    \author[1]{BY Hang Yuan}
    \author[2]{Christina Dan Wang}
    \author[3]{Zhou Yu}

    \affil[1]{\small School of Statistics, East China Normal University, \href{mailto:52274404018@stu.ecnu.edu.cn}{52274404018@stu.ecnu.edu.cn}}
    \affil[2]{\small New York University Shanghai, \href{mailto:christina.wang@nyu.edu}{christina.wang@nyu.edu}}
    \affil[3]{\small School of Statistics, East China Normal University, \href{mailto:zyu@stat.ecnu.edu.cn}{zyu@stat.ecnu.edu.cn}}
  \maketitle
} \fi

\if0\blind
{
  \bigskip
  \bigskip
  \bigskip
  \begin{center}
    {\LARGE\bf Title}
\end{center}
  \medskip
} \fi

\bigskip
\begin{abstract}
Nonlinear sufficient dimension reduction\citep{libing_generalSDR}, which constructs nonlinear low-dimensional representations to summarize essential features of high-dimensional data, is an important branch of representation learning.  
However, most existing methods are not applicable when the response variables are complex non-Euclidean random objects, which are frequently encountered in many recent statistical applications.
%In recent years, nonlinear sufficient dimension reduction (SDR) methods based on neural networks have experienced rapid development, demonstrating significant advantages over traditional kernel-based approaches.
%Statistical analysis is increasingly challenged by complex data arising from metric spaces.
%However, current mainstream methods struggle to handle complex metric space-valued responses alongside high-dimensional structured Euclidean predictors.
In this paper, we introduce a new statistical dependence measure termed Fr\'echet Cumulative Covariance (FCCov) and develop a novel nonlinear SDR framework based on FCCov. Our approach is not only applicable to complex non-Euclidean data, but also exhibits robustness against outliers.
We further incorporate Feedforward Neural Networks (FNNs) and Convolutional Neural Networks (CNNs) to estimate nonlinear sufficient directions in the sample level. 
Theoretically, we prove that our method with squared Frobenius norm regularization achieves unbiasedness at the $\sigma$-field level. Furthermore, we establish non-asymptotic convergence rates for our estimators based on FNNs and ResNet-type CNNs, which match the minimax rate of nonparametric regression up to logarithmic factors.
Intensive simulation studies verify the performance of our methods in both Euclidean and non-Euclidean settings.
We apply our method to facial expression recognition datasets and the results underscore more realistic and broader applicability of our proposal.

\end{abstract}

\noindent%
{\it Keywords:} Sliced inverse Regression, Sliced average variance estimation, sufficient dimension reduction, Cumulative covariance, Fr\'echet regression, Neural networks
\vfill

\newpage
\spacingset{1.9} % DON'T change the spacing!

\section{Introduction}\label{sec:intro}
Sufficient dimension reduction (SDR) serves as a prominent framework in supervised learning, aiming to find an intrinsic low-dimensional representations of high-dimensional data \( X \) while preserving essential information related to the responses \( Y \). Since the introduction of sliced inverse regression (SIR) \citep{li1991sliced} for the purpose of linear sufficient dimension reduction, various linear SDR methods have been developed. These include sliced average variance estimates (SAVE) \citep{cook1991discussion}, contour regression (CR) \citep{li2005contour}, directional regression (DR) \citep{li2007directional}, and many others. However, linear functions may not adequately represent high-dimensional complex data, such as images and natural languages, due to the inherent nonlinearity of the data. 
To address these limitations,
\citet{cook2007fisher} further proposed nonlinear sufficient dimension reduction.
The integration of the kernel trick 
with SDR concepts inspires a series of important developments towards nonlinear 
SDR \citep{wu2008,hsing2009rkhs, yeh2009, li2011principal,libing_generalSDR,ying2022frechet}.
However, kernel approaches are computationally intractable, scaling as \(\mathcal{O}(n^3)\) for a dataset of sample size \(n\).

As data collection methods continue to diversify, statisticians face increasingly complex data types. Non-Euclidean data, as a significant and widely applicable category, has gradually become a focal point of research in the fields of machine learning and artificial intelligence.
Examples of non-Euclidean data include, but are not limited to, shapes,  %\cite{chang2015shapenet,small2012statistical}, 
graphs, %\cite{scarselli2008graph}, 
symmetric positive definite (SPD) matrices,
%\cite{moakher2005differential,jayasumana2013kernel}, 
Riemannian manifold structures,
and random densities. The seminal work of \citet{petersen2016functional} introduced a general Fr\'echet regression framework for data with Euclidean predictors and non-Euclidean responses. \citet{qiu2024random} further leveraged random forests to develop a weighted local nonparametric Fr\'echet regression method. Despite these advancements, these approaches share common limitations. Firstly, they are incapable of handling structured predictors, such as images and natural languages. Secondly, their performance is prone to degradation in the presence of high-dimensional predictor variables, underscoring the need for sufficient dimension reduction methods designed for non-Euclidean and complex data structures. 


Our motivating example is the facial expression recognition study based on the the JAFFE (Japanese Female Facial Expression) dataset. This dataset comprises $213$ images depicting various facial expressions from 10 distinct Japanese female subjects. Each image has a resolution of $256 \times 256$ pixels. Each subject was instructed to perform seven facial expressions, including six basic emotions and a neutral expression. These images were then annotated with average semantic ratings for each basic emotion by 60 annotators. The ratings range from 1 to 5, where 5 signifies the highest intensity of emotion and 1 signifies the lowest.
Subsequently, the average scores for each emotion were calculated for each expression image, enabling the determination of the emotional distribution for each image. Then the label corresponding to each $256\times 256$ facial expression image is a  6-dimensional vectors representing probabilistic distributions of six emotions: happiness, sadness, surprise, anger, disgust, and fear. Statistical analysis of metric space valued responses with high-dimensional structured predictors is a difficult and challenging task. 

\begin{figure}[h]
        \centering
        \subfloat{
        \includegraphics[width=0.4\textwidth]{image_combination.png}
        }
        \subfloat{
        \includegraphics[width=0.43\textwidth]{dis_combination_2.png}
        }
        \caption{Typical facial expressions as predictors with emotion distributions as responses.}
        \label{RAF_ML}
    \end{figure}

%For emotion distributions, we can still treat them as Euclidean variables, although this approach may overlook the structural characteristics inherent in non-Euclidean spaces, which could lead to undesirable effects. In the case of SPD matrices or random densities, it is particularly challenging to model them as Euclidean variables. Therefore, it is crucial to model non-Euclidean elements directly. In this regard, many researchers have made significant contributions. For example, \cite{xing2016logistic} modeled the emotion distribution vector $Y$ as a member of the exponential family. Additionally, \cite{petersen2019frechet} introduced a general Fr\'echet regression framework for data with Euclidean predictors and non-Euclidean responses. Building on this framework, \cite{qiu2024random} leveraged random forests to develop a weighted local Fr\'echet regression method.
%Despite these advancements, these approaches share common limitations. Firstly, they are incapable of handling structured predictors, such as images and natural languages. Secondly, their performance is prone to degradation in the presence of high-dimensional predictor variables, underscoring the need for effective dimension reduction methods.

To this end, we in this article investigate nonlinear SDR for complex metric space valued \( Y \) and high dimensional Euclidean predictors \( X \), formalized as  
\begin{equation}\label{nonlinear_setting} 
  Y \indep X \mid \mf(X),
\end{equation}  
where \(\mf(\cdot): \mathbb{R}^p \rightarrow \mathbb{R}^d\) denotes an unknown vector-valued function of \( X \). 

Owing to the representational power and high computational efficiency, deep neural networks achieved great success in various learning tasks.
Recently, a growing number of SDR methods utilizing deep neural networks have been proposed. \citet{huang2024deep} utilized deep neural networks based on distance covariance for characterizing
nonlinear SDR.
\citet{chen2024deep} developed a nonlinear SDR framework using the generalized martingale difference divergence (GMDD) loss, optimized through neural network architectures. They also proposed two alternative optimization strategies: successive stepwise optimization and single-step optimization with Frobenius norm regularization. However, both strategies in \citet{chen2024deep} exhibit limitations. The stepwise approach is computationally inefficient due to the need for iterative updates of functional components, while the Frobenius norm regularization introduces nonsmoothness into the loss landscape, complicating gradient-based optimization.
% \citet{chen2024deep} built a nonlinear sufficient dimension reduction (SDR) framework based on the generalized martingale difference divergence (GMDD) loss, optimized within a neural network class. Their framework presented two distinct optimization strategies: successive stepwise optimization and single-step optimization with Frobenius norm regularization. However, both approaches in \citet{chen2024deep} have limitations: stepwise optimization suffers from computational inefficiency due to iterative functional updates, while Frobenius norm regularization introduces non-smoothness into the loss landscape, thereby hindering gradient-based optimization. 
Current deep SDR methods are primarily designed for Euclidean data and struggle to generalize to non-Euclidean domains.
Moreover, the existing theoretical investigations of deep SDR methods are insufficient for handling structured predictors, such as images structures in the MNIST data and JAFFE data.

To characterize the conditional independence between non-Euclidean random objects valued responses and Euclidean predictors, we first introduce a novel measure, termed Fr\'echet Cumulative Covariance (FCCov). By leveraging FCCov, we reformulate nonlinear SDR into a constrained optimization framework. We further develop a computationally efficient deep SDR framework that incorporates both deep fully-connected networks and ResNet-type convolutional neural networks \citep{oono2019approximation} to capture nonlinear relationships in sufficient predictors. The inclusion of ResNet architectures \citep{he2016deep} is justified by their theoretical approximation guarantees and empirical success in modeling structured data, such as images or text. Extensive simulations and real applications validate the performance of our proposal. 

%To tackle the challenge of non-Euclidean responses with high-dimensional and structured predictors, 

%we develop a computationally efficient deep SDR method, using both deep fully-connected and ResNet-type convolutional neural networks \cite{oono2019approximation} to capture the nonlinearity in the sufficient predictors. 
%The reason we utilize ResNet \cite{he2016deep} type CNNs is that it has strong approximation capability and extensive applications.
%To characterize the conditional independence between non-Euclidean objects and Euclidean predictors, we first introduce a novel measure, termed Fr\'echet Cumulative Covariance (FCCov). 
%In light of identifiability, as exemplified in equation \eqref{nonlinear_setting}, the conditional independence still holds under a bijective transformation of $\mf$. By leveraging FCCov, we reformulate the search for sufficient predictors into a constrained optimization framework. Subsequently, by introducing an squared Frobenius norm to regularize the sufficient predictors to have independent components, we have converted it into a more tractable unconstrained problem. 
%Additionally, we offer an effective empirical method for dimension selection. Due to space limitations, we direct readers to the supplementary material for further details.
%Extensive simulations and real data analyses validate the performance of our proposed method. 

The major contributions of our proposed method are summarized as follows.
\begin{itemize}
    \item %Building on the cumulative covariance (CCOV) measure proposed by \cite{zhou2020model}, 
    We develop a novel statistical metric, termed FCCov, to quantify conditional mean independence for non-Euclidean response objects and derive a corresponding fast computation algorithm. The new dependence measure is of independent interest.

    % \item Motivated by the cumulative covariance (CCOV) that is introduced by  \cite{zhou2020model}, we develop a measure, termed FCCov, for assessing conditional mean independence of non-Euclidean objects and derive a corresponding fast computation algorithm.
    
    \item We further propose a nonlinear SDR method based on FCCov that is capable of addressing scenarios where responses are complex random objects in a metric space and predictors reside in Euclidean spaces. Theoretically, we demonstrate that our method is able to achieve unbiasedness at the $\sigma$-field level.
    
    \item Inheriting the appealing property of cumulative covariance \citep{zhou2020model, li2023testing}, our proposal with indicator functions involved demonstrates robustness to outliers. The robustness of our proposal has been substantiated through comprehensive comparisons with existing methods \citep{chen2024deep,huang2024deep}.
    % Inherited by appealing property of the cumulative covariance, our method shows robustness to outliers, and we also prove it when responses are in ordinary Euclidean spaces with thorough simulations. 

    \item  In contrast to \citet{chen2024deep}, which employs the Frobenius norm, we propose substituting it with squared Frobenius norm regularization, implemented via a Lagrange multiplier to reduce computational complexity.  We prove that minimizing FCCov loss equipped with squared Frobenius norm regularization can achieve unbiasedness of nonlinear SDR at the $\sigma$-field level. 
    The one-step optimization scheme with smoothed regularizers, achieves both numerical stability and computational efficiency.

    
    \item Last but not least, we establish the nonasymptotic convergence rate for our proposed nonlinear SDR estimators based on both
    deep fully-connected and ResNet-type convolutional neural networks.
    The convergence rate matches the minimax rate up to logarithmic terms. Existing theoretical studies \citep{chen2024deep, huang2024deep} did not consider CNNs-based estimators. Our theoretical investigations fill this gap. 
    
    % align with the minimax optimal rate of nonparametric regression, up to a logarithmic factor of $n$.
    
    % Furthermore, within a general class of approximation functions, we demonstrate that the nonasymptotic convergence rates of our estimator align with the minimax optimal rate of nonparametric regression, up to a logarithmic factor of \( n \). Subsequently, we extend this rates to both deep fully-connected and convolutional neural networks class.
    

    % Additionally, we elucidate the relationship between the optimal solutions of the constrained and unconstrained problems.
    
\end{itemize}

We organized the rest of this paper as follows. In Section \ref{sec:preliminary}, we present the necessary notations and provide a brief introduction to neural networks. In Section \ref{sec:measure}, we develop a new measure FCCov of conditional mean independence and derive a corresponding fast computation algorithm.
In Section \ref{sec:method}, we introduce the FCCov-Net, which minimizes the FCCov type loss function facilitated by neural networks, for the purpose of nonlinear SDR. Section \ref{sec:theory} establishes the nonasymptotic error bound of the proposed estimator.
In Sections \ref{sec:simulation} and \ref{sec:real_data}, we examine the proposed method through extensive numerical experiments and real data applications, respectively. Section \ref{sec:summary} discusses some future directions.

%a methodology for identifying nonlinear sufficient dimension reduction (SDR) predictors by solving an optimization objective based on a dependency measure. Subsequently, by incorporating the squared Frobenius norm, we transform this objective into a computationally tractable form. Additionally, we provide a detailed procedure for sample estimation. 



% \section{Introduction}\label{sec:intro}

% As data collection methods continue to diversify, statisticians face increasingly complex data types. Non-Euclidean data, as a significant and widely applicable category, has gradually become a focal point of research in the fields of machine learning and artificial intelligence.
% Examples of non-Euclidean data include, but are not limited to, shapes \cite{chang2015shapenet,small2012statistical}, graphs \cite{scarselli2008graph}, symmetric positive definite (SPD) matrices \cite{moakher2005differential,jayasumana2013kernel}, and random densities \cite{petersen2016functional}.

% In this article, we explore nonlinear sufficient dimension reduction \eqref{nonlinear_setting} in the scenario where the response objects are complex random elements and the predictor variables are standard Euclidean. 
% A typical example of this scenario can be found in the facial expression recognition study conducted by \cite{li2005facial}, where emotion labels \( Y \)'s are represented as probability distribution vectors and images \( X \)'s are treated as Euclidean variables, as illustrated in Figure \ref{RAF_ML}.

% \begin{figure}[h]
%         \centering
%         \subfloat{
%         \includegraphics[width=0.4\textwidth]{image_combination.png}
%         }
%         \subfloat{
%         \includegraphics[width=0.43\textwidth]{dis_combination_2.png}
%         }
%         \caption{Some typical expressions from the JAFFE database together 
%   with emotion distributions.}
%         \label{RAF_ML}
%     \end{figure}

% For emotion distributions, we can still treat them as Euclidean variables, although this approach may overlook the structural characteristics inherent in non-Euclidean spaces, which could lead to undesirable effects. In the case of SPD matrices or random densities, it is particularly challenging to model them as Euclidean variables. Therefore, it is crucial to model non-Euclidean elements directly. In this regard, many researchers have made significant contributions. For example, \cite{xing2016logistic} modeled the emotion distribution vector $Y$ as a member of the exponential family. Additionally, \cite{petersen2019frechet} introduced a general Fr\'echet regression framework for data with Euclidean predictors and non-Euclidean responses. Expanding on this framework, \cite{qiu2024random} incorporated machine learning techniques to develop a weighted local Fr\'echet regression method based on random forests.
% Despite these advancements, these approaches share common limitations. Firstly, they are incapable of handling structured predictors, such as images and natural languages. Secondly, their performance is prone to degradation in the presence of high-dimensional predictor variables, underscoring the need for effective dimension reduction methods. 

% Sufficient dimension reduction (SDR) serves as a prominent approach in supervised learning, aiming to derive low-dimensional representations of high-dimensional data \( X \) while preserving essential information related to the responses \( Y \). Since the introduction of sliced inverse regression (SIR) \cite{li1991sliced} for estimating linear sufficient functions of the original data \( X \), various linear SDR methods have been developed. These include parametric Inverse Regression \cite{bura2001estimating}, sliced Average Variance Estimates (SAVE) \cite{cook1991discussion}, contour regression (CR) \cite{li2005contour}, and directional regression (DR) \cite{li2007directional}. However, linear functions may not adequately represent high-dimensional complex data, such as images and natural languages, due to the inherent nonlinearity of the data.
% To address these limitations,
% \cite{cook2007fisher} further proposed the formulation of nonlinear sufficient dimension reduction as
% \begin{align}\label{nonlinear_setting} 
%   Y \indep X | \mf(X).
% \end{align}
% where $\mf(\cdot): \mathbb{R}^p \rightarrow \mathbb{R}^d$
% is an unknown vector-valued function of $X$.
% The integration of the kernel trick 
% with linear SDR methods provides a significant approach to tackling nonlinear 
% SDR problems (\cite{wu2008, report2008, hsing2009rkhs, yeh2009, li2011principal,ying2022frechet}).
% However, kernel approaches are computationally intractable, scaling as \(\mathcal{O}(n^3)\) for a dataset of size \(n\).

% \cite{huang2024deep} proposes an adversarial framework employing a representer network and a discriminator network to identify sufficient representations. Separately, \cite{chen2024deep} introduces a nonlinear sufficient dimension reduction (SDR) framework based on the generalized martingale difference divergence (GMDD) loss, optimized within a neural network architecture. Chen’s framework presents two distinct optimization strategies: successive stepwise optimization and single-step optimization with Frobenius norm regularization. However, both approaches in \cite{chen2024deep} exhibit limitations: stepwise optimization suffers from computational inefficiency due to iterative functional updates, while Frobenius norm regularization introduces non-smoothness into the loss landscape, thereby hindering gradient-based optimization. They are also inadequate for structured predictors, such as images or natural language.
% Furthermore, most contemporary deep sufficient dimension reduction (SDR) methods are tailored for Euclidean data and fail to generalize to non-Euclidean data structures. 

% To tackle the challenge of non-Euclidean responses with high-dimensional and structured predictors, we develop a computationally efficient deep SDR method, using both deep fully-connected and ResNet-type convolutional neural networks \cite{oono2019approximation} to capture the nonlinearity in the sufficient predictors. 
% The reason we utilize ResNet \cite{he2016deep} type CNNs is that it has strong approximation capability and extensive applications.
% To characterize the conditional independence of non-Euclidean objects, first we introduce a novel measure, termed Fr\'echet Cumulative Covariance (FCCov). 
% In light of identifiability, as exemplified in equation \eqref{nonlinear_setting}, the conditional independence still holds under a bijective transformation of $\mf$. By leveraging FCCov, we reformulate the search for sufficient predictors into a constrained optimization framework. Subsequently, by introducing an squared Frobenius norm to regularize the sufficient predictors to have independent components, we have converted it into a more tractable unconstrained problem. 
% Additionally, we offer an effective empirical method for dimension selection. Due to space limitations, we direct readers to the supplementary material for further details.
% Extensive simulations and real data analyses validate the performance of our proposed method. 


% In summary, there are four major contributions of this work:
% \begin{itemize}
%     \item Building on the cumulative covariance (CCOV) measure proposed by \cite{zhou2020model}, we develop a novel statistical metric, termed FCCov, to quantify conditional mean independence for non-Euclidean response objects and derive a corresponding fast computation algorithm.
    
%     \item We propose a nonlinear SDR method based on FCCov that is capable of addressing scenarios where responses are complex random objects in a metric space and predictors reside in Euclidean spaces. Theoretically, we demonstrate that our framework’s optimal solutions achieve unbiasedness at the $\sigma$-field level.
    
%     \item Inheriting the appealing property of cumulative covariance, our method demonstrates robustness to outliers. Furthermore, we have substantiated this robustness through thorough simulations when responses are in Euclidean spaces. 

%     \item  In contrast to \cite{chen2024deep}, which employs the Frobenius norm, we propose substituting it with squared Frobenius norm regularization, implemented via a Lagrange multiplier to enhance computational tractability. Furthermore, our one-step optimization procedure, facilitated by the squared norm’s inherent smoothness, achieves both numerical stability and computational efficiency.
%     At the sample level, we employ both deep fully-connected and ResNet-type convolutional neural networks to estimate sufficient predictors. Furthermore, we establish that the nonasymptotic convergence rates of our estimator align with the minimax optimal rate of nonparametric regression, up to a logarithmic factor of $n$.
    
    
% \end{itemize}

% We organized the rest of this work as follows. In Section \ref{sec:preliminary}, we present the necessary notations and provide a brief introduction to neural networks. In Section \ref{sec:measure}, we develop a new measure for conditional mean independence and derive a corresponding fast computation algorithm.
% In Section \ref{sec:method}, we introduce a methodology for identifying nonlinear sufficient dimension reduction (SDR) predictors by solving an optimization objective based on a dependency measure. Subsequently, by incorporating the squared Frobenius norm, we transform this objective into a computationally tractable form. Additionally, we provide a detailed procedure for sample estimation. 
% Section \ref{sec:theory} establishes the convergence rates of the estimator, specifically for the estimation process utilizing a suitably chosen class of neural networks.
% In Sections \ref{sec:simulation} and \ref{sec:real_data}, we validate the proposed method through extensive numerical experiments and real data examples, respectively.
% Finally, we conclude the study and recommend future
% research directions in Section \ref{sec:summary}.

\section{Preliminaries}\label{sec:preliminary}
\subsection{Notation}
When $X$ represents a vector, we denote its components using lowercase letters $x_i$.
Consider a category of scalar-valued functions denoted by $\mF_n$, which varies with the sample size $n$. 
Unless otherwise specified, we assume that any function $f$ has an input dimensionality of $p$. For matrix inputs, such as images, the matrix is flattened into a vector, where the dimension $p$ corresponds to the dimension of the resulting vector.
Let $\{\mF_n\}^d = \{\mf = (f_1, \ldots, f_d)^{\top} : f_i \in \mF_n, i = 1, \ldots, d\}$ represent the $d$-fold Cartesian product of $\mF_n$.
As an example, when a $\mathbb{R}^{d}$-valued function $\mf$ belongs to $\{L_2(Q)\}^d$, 
it implies that each $f_i$ is a member of $L_2(Q)$ for all $1 \leq i \leq d$, where $L_2(Q)$ represents the space of all square-integrable functions with respect to the measure $Q$.
$I_A$ represents the indicator function of the set $A$.
For any function \(\mf\) of random elements \(X\), we define \(\dot{\mf}(X) = \mf(X) - \mathbb{E}[\mf(X)]\).
For two functions $\mf_1:\R^p \rightarrow \R^d$ and $\mf_2:\R^d\rightarrow \R^l$,
the composition of $\mf_1$ and $\mf_2$ is represented as $\mf_2\circ \mf_1$.
We denote two sets $\mA_{n, k}$ and $\mC_{n, k}$ as follows:
\begin{equation}\label{PAC} %permutation and combination
    \begin{split}
        \mA_{n, k}&=\left\{\left(\tau_{1}, \ldots, \tau_{k}\right) \in\{1, \ldots, n\}^{k}: \tau_{j} \neq \tau_{l}, \text { for } j \neq l\right\},\\
        \mC_{n, k}&=\left\{\left(\tau_{1}, \ldots, \tau_{k}\right) \in\{1, \ldots, n\}^{k}: \tau_{j} < \tau_{l}, \text { for } j < l\right\}.
    \end{split}
\end{equation}

\subsection{Neural Networks}
Feedforward neural networks (FNNs), commonly referred to as Multi-layer perceptrons, are composed of a series of fully-connected layers. Convolutional neural networks (CNNs) are structured as a sequence that integrates both convolutional and fully-connected layers. These architectures are fundamental and widely utilized in deep learning. 
Given \(W \in \mathbb{R}^{p_2 \times p_1}\) and \(b \in \mathbb{R}^{p_2}\), a fully connected layer \(\FC^{\sigma}_{W,b}: \mathbb{R}^{p_1} \rightarrow \mathbb{R}^{p_2}\) is expressed as \(\FC^{\sigma}_{W,b}(a) = \sigma\left(Wa - b\right)\). 
In this paper, we utilize the ReLU activation function, defined as $\sigma(x) = \max(x, 0)$, which is applied element-wise when $x$ is not a scalar. Additionally, $id(x)=x$ represents the identity function.
For any positive integer $N\in \N_{+}$, we define $[N]\coloneqq \{1,\ldots,N\}$. The function class of FNNs can then be defined as follows:
\begin{definition}[Fully-connected Neural Networks (FNNs)]
        \label{def_FNN}
        Let $\mathcal{L}, \mathcal{N}, \mathcal{S} \in \mathbb{N}_+$ denote the depth, maximum width and total parameters of a FNN. For each layer $i \in [\mL]$, let $k_i$ denote width,
        i.e., weight $W_i\in \R^{k_{i}\times k_{i-1}}$ and intercept vector $b_i\in \R^{k_{i}}$.
        Then \( \mathcal{N} = \max\{k_1, \ldots, k_\mathcal{L}\} \) and \( \mathcal{S} = \sum_{i=0}^{\mathcal{L}} k_{i+1} \times (k_i + 1) \). Let $k_0=p$ and $k_{\mathcal{L}+1}=1$ represent the widths of the input and output layers, respectively.
        We define the class of FNNs by 
        $$\F_{FNN}=\left\{f\mid f=\FC^{id}_{W_\mathcal{L},b_\mathcal{L}}\circ\cdots\circ\FC^\sigma_{W_0,b_0}:\R^p\rightarrow \R\right\}.$$
    \end{definition}

Let $K,H,H^\prime\in \N_{+}$ denote a filter size, input channel size, and output
channel size, respectively.
Then, for a filter $w=$ $\left(w_{n, j, i}\right)_{n \in[K], j \in\left[H^{\prime}\right], i \in[H]}$ $ \in \mathbb{R}^{K \times H^{\prime} \times H}$, 
we define the one-sided padding and stride-one convolution %$\footnote{
%\cite{oono2019approximation} detailly discuss the difference between one-sided padding and two-sided padding, which we do not expand it.}$ 
as $4$-th order tensor
$L_p^w=\left(\left(L_p^w\right)_{\alpha, i}^{\beta, j}\right) \in \mathbb{R}^{p \times p \times H^{\prime} \times H}$,
$$
\left(L_p^w\right)_{\alpha, i}^{\beta, j}:= \begin{cases}w_{(\alpha-\beta+1), j, i} & \text { if } 0 \leq \alpha-\beta \leq K-1, \\ 0 & \text { otherwise, }\end{cases}
$$
where $i$ (resp. $j$ ) runs through 1 to $H$ (resp. $H^{\prime}$ ) and $\alpha$ and $\beta$ through 1 to $p$. 
Given a fixed input dimension \(p\), we denote the linear mapping as \(L^w\) (omitting subscript \(p\) for simplicity). This mapping operates from \(\mathbb{R}^{p \times H}\) to \(\mathbb{R}^{p \times H^\prime}.\) For \(x = (x_{\alpha,i}) \in \mathbb{R}^{p \times H}\), \(L^w\) transforms \(x\) into \(y = (y_{\beta,j}) \in \mathbb{R}^{p \times H^\prime}\) via:  
$y_{\beta,j} = \sum_{i,\alpha} \left(L^w\right)_{\alpha,i}^{\beta,j} x_{\alpha,i}$.
For a weight tensor \(w \in \mathbb{R}^{K \times H^\prime \times H}\) and a bias vector \(b \in \mathbb{R}^{H\prime}\), the convolutional layer \(\operatorname{Conv}_{w, b}^{\sigma}: \mathbb{R}^{p \times H} \rightarrow \mathbb{R}^{p \times H^\prime}\) is defined as:
$
\operatorname{Conv}_{w, b}^{\sigma}(x) = \sigma\left(L^w(x) - \mathbf{1}_p \otimes b\right),
$
where \(\mathbf{1}_p\) is a \(p\)-dimensional vector of ones and \(\otimes\) denotes the outer product. The ResNet-type CNN is constructed by sequentially concatenating one convolution block, \(\mM\) residual blocks and one fully connected layer.
\begin{definition}[ResNet-type Convolutional Neural Networks(CNNs)]
        \label{def_CNN}
        Let $\mM, \mD, \mH, \mathcal{K}\in \N_{+}$ represent the number of residual 
        blocks and depth, channel size and filter size of blocks, respectively.
        For $m \in[\mM]$ and $l \in[\mL]$, 
        $w_m^{(l)} \in \mathbb{R}^{\mathcal{K} \times \mH \times \mH}$ and $b_m^{(l)} \in \mathbb{R}^\mH$
        denote a weight tensor and bias of the l-th layer of the m-th block in the convolution part, respectively. 
        Additionally, $W \in \mathbb{R}^{1 \times p\mH}$ and $b \in \mathbb{R}$ denote $a$ weight matrix and a bias for the fully-connected layer part, respectively. 
        ResNet-type CNNs is defined as,
        $$
        \begin{aligned}
            \F_{CNN}=\{f\mid f=\mathrm{FC}_{W, b}^{\mathrm{id}} & \circ\left(\operatorname{Conv}_{\boldsymbol{w}_\mM, \boldsymbol{b}_\mM}^\sigma+\mathrm{id}\right) \circ \cdots \\
        & \circ\left(\operatorname{Conv}_{\boldsymbol{w}_1, \boldsymbol{b}_1}^\sigma+\mathrm{id}\right) \circ P:\R^p\rightarrow \R\}
        \end{aligned}
        $$
        Here, $\operatorname{Conv}_{\boldsymbol{w}_m, \boldsymbol{b}_m}^\sigma =  \operatorname{Conv}_{w_m^{(L)}, b_m^{(L)}}^\sigma \circ \cdots \circ \operatorname{Conv}_{w_m^{(1)}, b_m^{(1)}}^\sigma$ and $P$ is a padding operation that aligns the number of channels by adding zeros: $P: \mathbb{R}^p \to \mathbb{R}^{p \times \mH},\; x \mapsto [x\; 0\; \cdots\; 0]$.  
        % where $\operatorname{Conv}_{\boldsymbol{w}_m, \boldsymbol{b}_m}^\sigma 
        % := 
        % \operatorname{Conv}_{w_m^{(L)}, b_m^{(L)}}^\sigma \circ \cdots \circ \operatorname{Conv}_{w_m^{(1)}, b_m^{(1)}}^\sigma$ and 
        % P is a padding operation that aligns the number of channels by adding zeros, i.e. , $P: \mathbb{R}^p \rightarrow \mathbb{R}^{p \times C}, \quad x \mapsto \begin{bmatrix} x & 0 & \cdots & 0 \end{bmatrix}$.
        
        % $\footnote{Although in this definition $f$ has a fully-connected layer,
        % we refer to a stack of convolutional layers both with or without the
        % final fully-connect layer as a CNN in this paper.}$.
    \end{definition}


\section{FCCov: A New Dependence Measure}\label{sec:measure}
   In this section, we introduce Fr\'echet cumulative covariance as a new conditional mean independence measure and corresponding properties and fast computation algorithms.
    
\subsection{Fr\'echet Cumulative Covariance}

    Let \( U \in \mathbb{R} \) and \( V \)  be a random object that takes values in a metric space $(\mV, d)$. Let \( (\bar{U}, \bar{V}) \) and \( (\tilde{U}, \tilde{V}) \) be independent copies of \( (U, V) \), with \( 0 < \operatorname{var}(U) < \infty \) and $E d^2(V,\Tilde V) < \infty$.
    If $E(U \mid V)=E(U)$, then for all $v_0, v_1 \in \operatorname{supp}(V)$, we can derive that
    \begin{equation}\label{equ:equivalent}
        E\left(U \mid d(V,v_0)<d(v_1,v_0)\right)= E\left[E\left(U \mid V\right)\mid d(V,v_0)<d(v_1,v_0)\right] = E(U),
    \end{equation}
    where the equations follow from the smoothness property of conditional expectation and the law of iterated expectations, and  $\operatorname{supp}(V)$ stands for the support of the conditioning variable $V$.
    Inspired by the cumulative covariance (CCov) introduced by \citet{zhou2020model, li2023testing}, we further observe that, for all $v_0, v_1 \in \operatorname{supp}(V)$
    \[
    \begin{aligned}
    E(U | V)=E(U)
    \Longrightarrow & E\left(U \mid d(V,v_0) <d(v_1,v_0)\right)=E(U) \\
    \Longleftrightarrow  &\operatorname{cov}\left\{U, I\left[d(V,v_0)<d(v_1,v_0)\right]\right\}=0 \\
    \Longleftrightarrow  &E\left[\operatorname{cov}^2\left\{U, I[d(V,\Tilde{V})<d(\Bar{V}, \Tilde{V})] \mid \Bar{V}, \Tilde{V}\right\}\right]=0.
    \end{aligned}
    \]
 %   where $\operatorname{supp}(V)$ stand for the support of the conditioning variable $V$.
    % $$
    % \begin{aligned}
    % & E(U \mid V)=E(U) \text { almost surely }, \\
    % \Leftrightarrow & E\left(U \mid d(V,v_0)<d(v_1,v_0)\right)=E(U), \text { for all } v_0, v_1 \in \operatorname{supp}(V), \\
    % \Leftrightarrow & \operatorname{cov}\left\{U, I\left[d(V,v_0)<d(v_1,v_0)\right]\right\}=0, \text { for all } v_0, v_1 \in \operatorname{supp}(V), \\
    % \Leftrightarrow & E\left[\operatorname{cov}^2\left\{U, I[d(V,\Tilde{V})<d(\Bar{V}, \Tilde{V})] \mid \Bar{V}, \Tilde{V}\right\}\right]=0,
    % \end{aligned}
    % $$
    This motivates us to define the Fr\'echet Cumulative Covariance (FCCov) as follows.
    \begin{definition}\label{def_FCCov}
    Let $(\mV, d)$ be a separable metric space.
    For any random variables $V\in \mV$ and $U\in \R$
    with $\Var(U)<\infty$ and $E d^2(V,\Tilde V) < \infty$.
    The Fr\'echet cumulative covariance \normalfont{FCCov}$(U\mid V)$ is defined by
    \begin{equation}
        \label{def:FCCov}
        \FCCov(U\mid V) = E\left[\operatorname{cov}^2\left\{U, I[d(V,\Tilde{V})<d(\Bar{V},\Tilde{V})] \mid \Bar{V},\Tilde{V}\right\}\right]. 
    \end{equation}
    \end{definition}
    
    % \begin{remark}
    % % The separability assumption of metric spaces is a common hypothesis for metric spaces. 
    % % This assumption ensures that open balls can generate the Borel $\sigma$-field of $\mV$ \cite{van2003probability} 
    % % Since the rank of $d(V, \Tilde{V})$ is used in the definition of $\FCCov(U|V)$,
    % % this indicates that it is robust to outliers in the conditioning variable $V$.
    % % The robustness inherent in our approach is also evident in our advanced sufficient dimension reduction techniques. 
    % % These techniques ensure that our method consistently yields effective dimension reduction functions, even in scenarios where the original data is compromised or contaminated. 
    % \end{remark}

    The separability assumption is a common hypothesis for metric spaces, ensuring that open balls generate the Borel \(\sigma\)-field of \(\mV\) \citep{van2003probability}.
    One of the most important properties of the CCov$(U\mid V)$ \citep{zhou2020model} is that it is nonnegative and equals zero if and only if $E(U \mid V)=E(U)$. Although the above discussions brings FCCov closer to this property, it remains challenging to deduce $E(U \mid V)=E(U)$ from equation \eqref{equ:equivalent}.
    Let
    \begin{align*}
    S_2 &= \big\{B\left(v_0, d(v_0,v_1)\right): \text{ for all } v_0, 
    v_1 \in \operatorname{supp}(V)\big\}, \\
    \mathscr{A} &= \big\{\text{finite unions and intersections of sets in }S_2\big\}.
    \end{align*}
    We observe that, given the separability of the space, \( S_2 \) generates the Borel \(\sigma\)-field of the metric space. And we define \(\Pi\) which consists of the sets \( A \in \mathscr{A} \) satisfying \( E\left(U \cdot I\{V \in A\}\right) = E(U) \cdot E(I\{V \in A\}) \). We define \(\Lambda\) to consist of the sets \( C \in \sigma(V) \) that satisfy \( E\left(U \cdot I\{V \in C\}\right) = E(U) \cdot E(I\{V \in C\}) \). Then we demonstrate that \(\Pi\) is a \(\pi\)-system and \(\Lambda\) is a \(\lambda\)-system. Utilizing Dynkin's \(\pi\)-\(\lambda\) theorem, we establish the following important theorem for FCCov.
    \begin{theorem}\label{prop_FCCov}
        Let $U$ and $V$ be the random variables defined in Definition \ref{def_FCCov}. 
        Then, \begin{enumerate}
            \item[(a).] $E(U \mid V) \overset{a.s.}{=}E(U) 
            \Longleftrightarrow
            E\left(U \mid d(V,v_0)<d(v_1,v_0)\right)=E(U)$.
            %\text { for all } v_0, v_1 \in \operatorname{supp}(V);$

            \item[(b).] \normalfont{FCCov}$(U\mid V) \ge 0$. Moreover, $E(U\mid V) \overset{a.s.}{=} E(U) \Longleftrightarrow \text{\normalfont{FCCov}}(U\mid V)=0$. 
        \end{enumerate}
        
    \end{theorem}

    Theorem \ref{prop_FCCov}(b) establishes that \(\FCCov(U \mid V)\) provides a new measure of independence between Euclidean \(U\) and metric space valued \(V\) from a conditional mean perspective.
     %This has greatly inspired us to introduce FCCov into nonlinear SDR.
    Observing the indicator function involved in FCCov, the robustness is another valuable property of FCCov inherited from CCov. This property also enables our nonlinear SDR method introduced in Section \ref{sec:method} to exhibit robustness against outliers.
    % The definition of \(\FCCov(U \mid V)\) inherents two essential properties from CCOV.
    % First, the rank of \(d(V, \tilde{V})\) utilized in the definition of \(\FCCov(U \mid V)\) guarantees robustness to outliers in the conditioning variable \(V\). 
    % Furthermore, the following second property has greatly inspired us to introduce FCCov into nonlinear SDR, 

   % \begin{remark}
   %  We can extend the definition of FCCov to a vector-valued random vector \(U = (u_1, u_2, \ldots, u_p)^\top \in \mathbb{R}^p\) by defining \(\text{FCCov}(U \mid V) = \sum_{s=1}^{p} \text{FCCov}(u_s \mid V)\).  This extension facilitates the derivation of an efficient computation algorithm and establishes its relationship with conditional mean independence. Specifically, we have  
   %   \begin{align*}
   %       E(U \mid V) = E(U) & \Leftrightarrow E(u_s \mid V) = E(u_s) \text{ for all } s = 1, \ldots, p \\ 
   %       & \Leftrightarrow \sum_{s=1}^{p} \text{FCCov}(u_s \mid V) = 0.
    %  \end{align*} 
    %\end{remark}
    

    For the estimation of  \(\FCCov(U \mid V)\) based on random samples $\{(U_i,V_i),i=1,\ldots n\}$ drawn independently from $(U,V)$, we assume $E[U]=0$ without loss of generality and consider the following estimator.
    \begin{equation}\label{FCCov_sample_form}
        \widehat{\FCCov}(U\mid V) = \frac{1}{(n)_4}\sum_{(i,j,k,l)\in\mathcal{A}_{n,4}}
        U_iU_j\phi(V_i,V_k,V_l)\phi(V_j,V_k,V_l),
    \end{equation}
    where $\phi(V_1, V_2, V_3) = I[d(V_1,V_3)<d(V_2,V_3)]$ and $(n)_m = n(n-1)\cdots (n-m+1), 1 \leq m \leq n$. $\widehat{\FCCov}(U\mid V)$ is an unbiased 4th-order U-statistics. Please refer to Proposition \textcolor{blue}{S.1} in the supplementary material for the detailed derivation. 
    
    % a natural estimator for $\FCCov(U|V)$ can be defined as follows
    % \begin{equation*}
    %     W_n = \frac{1}{n^4}\sum_{i=1}^{n}\sum_{j=1}^{n}\left[
    %         \sum_{k=1}^{n}U_{k}\bigl(\phi(V_k,V_i,V_j)
    %         - F_n(V_i,V_j)\bigr)
    %     \right]^2,
    % \end{equation*}
    % where $\phi(V_k, V_i, V_j) = I[d(V_k,V_j)<d(V_i,V_j)]$ and $F_n(V_i, V_j) = \frac{1}{n}\sum_{k=1}^{n}\phi(V_k, V_i, V_j)$.
    % This biased sample version $W_n$ includes numerous redundant terms that remain asymptotically non-negligible, especially when $U$ is high-dimensional random variable, and  thereby degrading performance.
    % To address this issue, we instead consider,
    % \begin{equation}\label{U_sample_form}
    %     T_n = \frac{1}{4(n)_6}\sum_{(i,j,k,l,r,t)\in \mA_{n,6}}(U_i-U_j)(U_k-U_l)
    %     \psi(V_i,V_j,V_r,V_t)\psi(V_k,V_l,V_r,V_t)
    % \end{equation}
    % where $(n)_m = n(n-1)\cdots (n-m+1), 1 \leq m \leq n$ and
    % \begin{align*}
    %     \psi(V_i,V_j,V_r,V_t) &= \phi(V_i,V_r,V_t) - \phi(V_j,V_r,V_t) \\
    %     &= I[d(V_i,V_t)<d(V_r,V_t)] - I[d(V_j,V_t)<d(V_r,V_t)].
    % \end{align*} 
    % By proposition 1 in supplementary materials,
    % $T_n$ is an unbiased $6$th-order U-statisic of $\FCCov(U\mid V)$ and thus excludes the non-negligible redundant term. 
    % The original 6th-order U-statistic in equation \eqref{U_sample_form} has a complex form, which will complicate our analysis. 
    % Consequently, we can assume $E(U)=0$ and simplify it to a 4th-order U-statistic:
    % \begin{equation}
    %     \label{simple_U_form}
    %     T_n^\prime = \frac{1}{(n)_4}\sum_{(i,j,r,t)\in \mA_{n,4}}U_iU_j\phi(V_i,V_r,V_t)
    %     \phi(V_j,V_r,V_t).
    % \end{equation}


    \subsection{Fast Computation Algorithm}\label{sec:computation}
    Although we can leverage many desirable statistical properties of U-statistics, the computational complexity of the naive estimator of $\widehat{\FCCov}(U\mid V)$, \(O(n^4)\), makes it impractical for large-scale applications. To address this computational challenge, we propose an efficient algorithm for estimating \({\FCCov}(U \mid V)\) based on the following two preprocessing steps.
    \begin{itemize}
      \item[1] For each \(t = 1, \ldots, n\), we sort all samples based on their distance to \(V_t\), resulting in the ordered sequence, $d(V_{(1)t}, V_t) < d(V_{(2)t}, V_t) < \cdots < d(V_{(n)t}, V_t).$
       \item[2] For each \(V_{(i)t}\), define the centered predictors as \(\dot{U}_{(i)t} = U_{(i)t} - \bar{U}_n\) and \(\dot{U}_t = U_t - \bar{U}_n\). 
    \end{itemize} 
    The following theorem shows how to compute $\widehat{\FCCov}(U \mid V)$ more efficiently.
    \begin{theorem}\label{fast_alg}
      For i.i.d. samples $\{U_i,V_i\}_{i=1}^n$ drawn from the joint distribution of
      $(U,V)$, 
      \begin{equation}
        \begin{split}
        (n)_4\cdot \widehat{\text{\normalfont{FCCov}}}(U \mid V) =& \sum_{t=1}^{n}\sum_{r=1}^{n}\left[\bigl(\sum_{i=1}^{r-1}U_{(i)t}\bigr)^2-\sum_{i=1}^{r-1}U_{(i)t}^2 - 2U_t\cdot \sum_{i=1}^{r-1}U_{(i)t} \right] \\
        &+ 2(n-1)\sum_{i=1}^{n}U_i^2.
        \end{split}
      \end{equation}
    %Disregarding the cost of computing distances in the metric space, 
    \end{theorem}
    Theorem \ref{fast_alg} guarantees that $\widehat{\FCCov}$ can be computed in only $\mO(n^2 \log n)$ operations.
    % 先排序后计算的方法能够有效降低计算复杂度的原因有两个,一是示性
    % 函数可以有效的减少求和项,第二个原因是因为$T_n$的某些项可以递归求解,如下所示。
    % In a similar manner, we can express the efficient computation algorithm for the U-statistic in equation \eqref{simple_U_form} as follows:
    % \begin{align*}
    %   (n)_4\cdot T_n^\prime = \sum_{t=1}^{n}\sum_{r=1}^{n}\left[\bigl(\sum_{i=1}^{r-1}U_{(i)t}\bigr)^2-\sum_{i=1}^{r-1}U_{(i)t}^2
    %   - 2U_t\cdot \sum_{i=1}^{r-1}U_{(i)t}
    %   \right] + 2(n-1)\sum_{i=1}^{n}U_i^2.
    % \end{align*}
    % The sample form of $T_n^\prime$ is notably simpler and more computationally efficient than that of $T_n$. 
    % Moreover, we can leverage $T_n^\prime$ for fast calculations in simulations 
    % if we replace $U_i$ with $\dot{U}_i$.

    % \begin{remark}
    %   With the definition of FCCov$(U\mid V)$ for a scalar random variable \(U \in \mathbb{R}\), we can readily extend this definition to vector-valued random vector \(U = (u_1, u_2, \ldots, u_p)^\top \in \mathbb{R}^p\) as follows:  
    %   \[
    %   \text{FCCov}(U \mid V) = \sum_{s=1}^{p} \text{FCCov}(u_s \mid V).
    %   \]  
    %   This formulation facilitates the derivation of an efficient computation algorithm and establishes its relationship with conditional mean independence. For instance, we have:  
    %   \begin{align*}
    %       E(U \mid V) = E(U) & \Leftrightarrow E(u_s \mid V) = E(u_s) \text{ for all } s = 1, \ldots, p \\ 
    %       & \Leftrightarrow \sum_{s=1}^{p} \text{FCCov}(u_s \mid V) = 0.
    %   \end{align*}  
    % % 特别地,对于定理\ref{fast_alg}中$T_n$的快速算法而言,当$U$是一个随机向量
    % % 我们仅需对$\{V\}_{i=1}^n$的每个样本做一次排序,得到排序后对应的的$U_{(i)l}$
    % % 在计算$\sum_{s=1}^{p}\FCCov(\bu_s|V)$的样本估计时,每个分量的排序结果相同,把每个分量的
    % % $T_n$值求和即可。
    % \end{remark}

\section{FCCov-Net for Nonlinear SDR}\label{sec:method}
    % In this section, we introduce the general framework for nonlinear SDR at \(\sigma\)-field level and propose our method within this framework, formulated as a variance constrained optimization problem. 
    % However, the variance constraint introduces significant statistical challenges, motivating the use of a Lagrange multiplier approach. 
    % This reformulation transforms the original problem into an unconstrained optimization problem with an \(F\)-norm regularization term. 
    % Finally, we present the empirical U-statistic formulation of the proposed method.

    \subsection{The FCCov Based Objective Function}
    Let \((\Omega, \mG, P)\) be a probability space and  \((\Omega_X, \mG_X), (\Omega_Y, \mG_Y)\) and $(\Omega_{XY}, \mathcal{G}_{XY})$ be measurable spaces, where \(\Omega_X \subset \mathbb{R}^p\) and \(\Omega_Y \subset \mathcal{Y}\), with \((\mathcal{Y}, d_{\mY})\) being a separable metric space. 
    Assume that \(\Omega_{XY} = \Omega_X \times \Omega_Y\) and \(\mG_{XY} = \mG_X \times \mG_Y\). Let \(X\), \(Y\), and \((X, Y)\) be random elements taking values in \(\Omega_X\), \(\Omega_Y\), and \(\Omega_{XY}\), with distributions \(P_X\), \(P_Y\) and \(P_{XY}\). 
    The \(\sigma\)-field generated by \(X\) is defined as \(\sigma(X) = X^{-1}(\mG_X)\). The conditional distribution of \(X\mid Y\) is denoted by \(P_{X\mid Y}(\cdot \mid \cdot): \mG_X \times \Omega_Y \to \mathbb{R}\).  
    We consider following nonlinear SDR problem
    \begin{equation}\label{conditional_indep}
        Y \indep X \mid \mG_{Y\mid X},
    \end{equation}
    where $\mG_{Y | X}$ is a sub-$\sigma$-field satisfying the conditional independence. \citet{libing_generalSDR} demonstrated that, under mild conditions, the $\sigma$-field $\mG_{Y |X}$ exists and is unique. 
    % \citet{libing_generalSDR} has shown that under mild conditions the existence and uniqueness of the $\sigma$-field $\mG_{Y |X}$. 
    Assume that  family of probability measures $\{P_{X|Y} (\cdot|y) : y\in \Omega_Y\}$ is dominated by a $\sigma$-finite measure. Following \citet{libing_generalSDR}, 
     there exists a unique minimal sufficient $\sigma$-field (or central $\sigma$-field) $\mG_{Y |X}$ such that \eqref{conditional_indep} holds true. Equivalently speaking, there exists $\mf_0(X)\in\R^d (d<p)$ with $E\big\{{\mf_0}(X)\big\} = 0$ and $\operatorname{Var}\big\{{\mf_0}(X)\big\} = I_d$ satisfying $\sigma\{\mf_0(X)\}=\mG_{Y |X}$. Here, we assume that $d$ is known. However, for practical applications, we propose an empirically effective algorithm to estimate the intrinsic dimension $d$.
    

    %\begin{assumption*}
	%	Assume that the family of probability measures $\{P_{X|Y} (\cdot|y) : y\in \Omega\}$ is dominated by a $\sigma$-finite measure. Then there exists a unique minimal sufficient $\sigma$-field (or central $\sigma$-field) $\mG_{Y |X}$ such that \eqref{conditional_indep}. Moreover, there exists sufficient dimension reduction of X, $\mf_0(X)\in\R^d (d<p)$ with $E\big\{{\mf_0}(X)\big\} = 0$ and $\operatorname{Var}\big\{{\mf_0}(X)\big\} = I_d$ satisfying $\sigma\{\mf_0(X)\}=\mG_{Y |X}$.
    %\end{assumption*}
    
    Replacing \(\mf_0\) with any one-to-one function transformation does not alter the $\sigma$-field generated by
    \(\mf_0\). Consequently, \(\mf_0\) itself is unidentifiable. 
    %However, the $\sigma-$filed generated by \(\mf_0\) is unique and identifiable.
    %To simplify the analysis, we assume that 
    %\(E\left[\mf_0(X)\right] = 0\) and \(\operatorname{Var}\left[\mf_0(X)\right] = I_d\).
    Let \(\mathfrak{S}_{\mG_{Y \mid X}}\) be the central class comprising all square-integrable and \(\mG_{Y \mid X}\)-measurable functions, which is unique and identifiable.
    %Then for any non-degenerate function $\bf g(X)$ belonging to \(\mathfrak{S}_{\mG_{Y \mid X}}\), it will satisfy the conditional in 
    Following \citet{libing_generalSDR, chen2024deep}, we consider the orthogonal direct sum decomposition of $L_2(P_X)=\mathfrak{S}_{\mG_{Y\mid X}} \oplus \mathfrak{S}^\perp_{\mG_{Y\mid X}}$,  where $\mathfrak{S}^\perp_{\mG_{Y\mid X}}$ be the orthogonal complement of $\mathfrak{S}_{\mG_{Y\mid X}}$. 
    %where $\mathfrak{S}^\perp_{\mG_{Y\mid X}}$ is the orthogonal complement of $\mathfrak{S}_{\mG_{Y\mid X}}$.
 For any $f(X)\in L_2(P_X)$ and $\dot f(X)=f(X)-E(f(X))$, Theorem \ref{prop_FCCov} indicates that
    \[
    \FCCov(f(X)\mid Y) = E\left(E\left\{\dot{f}(X)k(I)|
        \Bar{Y},\Tilde{Y}\right\}E\left\{\dot{f}(X)k(I)|
        \Bar{Y},\Tilde{Y}\right\}\right) \geq 0
    \]
    where $k(I)=I\left(d(Y,\Tilde{Y})<d(\Bar{Y},\Tilde{Y})\right)-E\left[I\left(d(Y,\Tilde{Y})<d(\Bar{Y},\Tilde{Y})\right)\mid \Bar{Y},\Tilde{Y}\right]$. 
    %For any $g \in \mathfrak{S}^\perp_{\mG_{Y\mid X}}$, we apply Lemma 1 in \citet{libing_generalSDR} Theorem \ref{prop_FCCov} and further conclude that $E(g(X)\mid Y)=E(g(X))$ and  $FCCov(g(X)\mid Y) =0$. 
    %Inspired the non-negativity of FCCov as Theorem \ref{prop_FCCov} and Lemma 1 in \citet{libing_generalSDR}, we reveal that FCCov can be adopted for nonlinear SDR.
    The next theorem reveals the fundamental relationship between FCCov and the two function classes $\mathfrak{S}_{\mG_{Y\mid X}}$ and  $\mathfrak{S}^\perp_{\mG_{Y\mid X}}$, providing a novel guiding principle for nonlinear SDR.

 %    However, direct estimation of the central $\sigma$-field $\sigma\{\mf_0(X)\}$  remains abstract and analytically intractable.
 %    Following \cite{libing_generalSDR}, let $\mathfrak{S}_{\sigma\{\mf_0(X)\}}$ denote the central 
 %    class corresponding to the central $\sigma$-field $\sigma\{\mf_0(X)\}$.
 %    This class comprises all square-integrable, $\sigma\{\mf_0(X)\}$-measurable functions.
 %    Equivalently, $\mathfrak{S}_{\sigma\{\mf_0(X)\}}$ forms a subset of $L_2(P_X)$
 %    comprising functions that can be expressed as transformations of $\mf_0$.

 %    Given that $\mathfrak{S}_{\sigma\{\mf_0(X)\}}$ is a closed subspace of the Hilbert space $L_2(P_X)$, $L_2(P_X)$ 
 %    can be decomposed into the orthogonal direct sum
	% $$L_2(P_X)=\mathfrak{S}_{\sigma\{\mf_0(X)\}} \oplus \mathfrak{S}^\perp_{\sigma\{\mf_0(X)\}},$$
	% where $\mathfrak{S}^\perp_{\sigma\{\mf_0(X)\}}$ is the orthogonal complement of $\mathfrak{S}_{\sigma\{\mf_0(X)\}}$. 
 %    For any function $g \in \mathfrak{S}^\perp_{\sigma\{\mf_0(X)\}}$, it holds that
	% \begin{align*}
	% 	E \big\{g(X) \big| Y\big\} = E\big[E\big\{g(X) \big| \mf_0(X), Y\big\} \big| Y\big] = E\big[E\big\{g(X) \big| \mf_0(X)\big\} \big| Y\big]  \stackrel{a.s}{=} E\big\{g(X)\big\}, 
	% \end{align*}
	% where the second equality holds because $Y \indep X | \sigma\{\mf_0(X)\}$; the third equality follows 
 %    from $g \in \mathfrak{S}^\perp_{\sigma\{\mf_0(X)\}}$ (see Lemma 1 of \cite{libing_generalSDR}). 
 %    By the non-negative property of FCCov, it implies
 % \begin{theorem}\label{constrained}
 %     Let $g$ lies in $\mathfrak{S}^\perp_{\mG_{Y\mid X}}$, then $\FCCov\big(g(X)\mid Y\big)=0$.
 %     Consequently, let $m\in \mathbb{N}^+$ be any positive integer and $\mf^*$ be an optimal solution of the following objective fuction
 %    \begin{equation}
 %        \label{prob1}
 %        \begin{split}
 %            \max\limits_{\mf\in \{L_2(P_X)\}^m} & E\left(E\left\{\dot{\mf}(X)k(I)|
 %        \Bar{Y},\Tilde{Y}\right\}^\topE\left\{\dot{\mf}(X)
 %        k(I)|\Bar{Y},\Tilde{Y}\right\}\right) \\
 %            \text{subject to} \quad &\Var\{\mf(X)\} = I_m
 %        \end{split}
 %    \end{equation}
 %    then $f_j^*\in \mathfrak{S}_{\mG_{Y\mid X}}$, for all $1\leq j\leq m$. 
 % \end{theorem}

\begin{theorem}\label{constrained}
    \begin{itemize}
       \item[(a).]  For any $g \in \mathfrak{S}^\perp_{\mG_{Y\mid X}}$, \normalfont{FCCov}$\big(g(X)\mid Y\big)=0$.
       \item[\emph{(b).}] \emph{Let $m\in \mathbb{N}^+$ be any positive integer. Then the optimal solution $\mf^*=(f_1^*,\ldots, f_m^*)^\top$ of the following objective function}
        \begin{equation}
            \label{prob1}
            \begin{split}
                &\max\limits_{f_t\in \{L_2(P_X)\}} \sum_{t=1}^m {\FCCov}(f_j(X) \mid Y), \\
                &\text{subject to} \quad \Var\{\mf(X)\} = \textbf{I}_m.
            \end{split}
        \end{equation}
        \emph{satisfies $f_j^*\in \mathfrak{S}_{\mG_{Y\mid X}}$, for all $1\leq j\leq m$.} 
    \end{itemize}
\end{theorem}

\begin{remark}
Note that multiplying any solution to \eqref{prob1} by an orthogonal matrix yields another valid solution. Therefore, without loss of generality, we assume that \(\mf^*\) is the solution satisfying 
\[
-E\left[ E\left\{ \dot{f}^*_i(X) k(I) \mid \bar{Y}, \tilde{Y} \right\} E\left\{ \dot{f}^*_j(X) k(I) \mid \bar{Y}, \tilde{Y} \right\} \right] = l_i^*\cdot I(i=j)
\]
Specifically, we reorder the components of \(\mf^*\) using an orthogonal matrix such that for all \(1 \leq i \leq j \leq m\), \(l_i^* \leq l_j^*\). In the context of linear SDR, the solution \(f_1^*\) can be regarded as the first sufficient direction associated with the maximum eigenvalue of certain SDR matrix.

% its FCCov matrix, i.e.,
% \[
% \mathbb{E}\left[ \mathbb{E}\left\{ \dot{\mf}(X) k(I) \mid \bar{Y}, \tilde{Y} \right\} \mathbb{E}\left\{ \dot{\mf}(X) k(I) \mid \bar{Y}, \tilde{Y} \right\}^\top \right]
% \]
% is diagonal, and its negative diagonal elements (i.e., the negative FCCov values of \(f_j^*\)) are non-decreasing. Specifically, for \(1 \leq i \leq j \leq m\),
% \[
% l_i^* = -\mathbb{E}\left( \mathbb{E}^2\left\{\dot{f}_i(X) k(I) \mid \bar{Y}, \tilde{Y} \right\} \right) \leq -\mathbb{E}\left( \mathbb{E}^2\left\{\dot{f}_j(X) k(I) \mid \bar{Y}, \tilde{Y} \right\} \right) = l_j^*.
% \]
% In the context of linear sufficient dimension reduction, the first diagonal element of the FCCov matrix is analogous to the largest eigenvalue of the SDR matrix. The solution $f_1^*$ corresponds to the eigenvector associated with this eigenvalue.
\end{remark}

Theorem \ref{constrained} guarantees that $\sigma(\mf^*)\subseteq \mathcal{G}_{Y|X}$, which is called the unbiasedness in the literature of nonlinear SDR \citep{libing_generalSDR}. This implies that each component of the solution \(\mf^*\) preserves information about \(\mf_0(X)\).
Keep in mind that our primary objective is to identify nonlinear mappings that reduce the dimensionality of $X$ while retaining as much information about $Y$ as possible, rather than precisely recovering $\mf_0$. Compared to the work of \cite{chen2024deep} and \cite{huang2024deep}, our proposal is not only suitable for general metric space valued responses, but is also expected to be more robust against outliers when responses are Euclidean. This is because the objective function of our loss function involves only the computation of the distances $d(Y,\tilde Y)$ and $d(\bar Y, \tilde Y)$, along with the corresponding indicator function. However, the constraint $\Var\{\mf(X)\} = \textbf{I}_m$ of \eqref{prob1} is difficult for implementation, which needs further refinement. 
% Additionally, since the central class may be an infinite-dimensional space, our theorem holds for any arbitrary \(m\).
 

    % \begin{lemma}\label{orthogonal}
    %     Let $g$ be any square-integrable function that lies in the orthogonal complement 
    %     space of $\mathfrak{S}_{\sigma\{\mf_0(X)\}}$, then $\FCCov\big(g(X)|Y\big)=0$.
    % \end{lemma}
    % Since we know that for any $f \in L_2(P_X)$,
    % \begin{align*}
    %     &E\left[\operatorname{cov}^2\left\{f(X), I[d(Y,\Tilde{Y})<d(\Bar{Y},\Tilde{Y})] \mid 
    %     \Bar{Y}, \tilde{Y}\right\}\right] \\
    %     = &E\Big(
    %         E^2\big\{\left[f(X)-E\bigl(f(X)\bigr)\right]
    %         \big[I\big(d(Y,\Tilde{Y})<d(\Bar{Y},\Tilde{Y})\big) \\
    %         &-E(I\big(d(Y,\Tilde{Y})<d(\Bar{Y},\Tilde{Y})\big)|\Bar{Y},\Tilde{Y})\big]|\Bar{Y},\Tilde{Y}\big\}
    %         \Big) \\
    %     & \geq 0
    % \end{align*}
    % Inspired by this relationship and Lemma \ref{orthogonal}, we hypothesize that maximizing the
    % FCCov over $f \in L_2(P_x)$ might draw $f$ away from the orthogonal complement space of $\mathfrak{S}_{\sigma\{\mf_0(X)\}}$.
    % The subsequent theorem provides confirmation of this hypothesis.
    % \begin{theorem}\label{constrained}
    %     Let $m\in \mathbb{N}^+$ be any positive integer and $\mf^*$ be an
    %     optimal solution of the following objective fuction
    %     \begin{equation}
    %         \label{prob1}
    %         \begin{split}
    %             \max\limits_{\mf\in \{L_2(P_X)\}^m} & E\left(E\left\{\dot{\mf}(X)k(I)|
    %         \Bar{Y},\Tilde{Y}\right\}^\topE\left\{\dot{\mf}(X)
    %         k(I)|\Bar{Y},\Tilde{Y}\right\}\right) \\
    %             \text{subject to} \quad &\Var\{\mf(X)\} = I_m
    %         \end{split}
    %     \end{equation}
    %     Then $f_j^*\in 
    %     \mathfrak{S}_{\sigma{\{\mf_0(X)\}}}$, for all $1\leq j\leq m$.
    %     With loss of generality, we rearrange $f_j^*$ such
    %     that its negative FCCov value is nondecreasing, that is, for 
    %     $1 \leq i \leq j \leq m$,
    %     $$l_i^* = -E\left(E^2\left\{\dot{f}_i(X)k(I)|
    %     \Bar{Y},\Tilde{Y}\right\}\right) \leq -E\left(E^2\left\{\dot{f}_j(X)k(I)|
    %     \Bar{Y},\Tilde{Y}\right\}\right) = l_j^*, $$
    %     where we denote $k(I)=I\left(d(Y,\Tilde{Y})<d(\Bar{Y},\Tilde{Y})\right)
    %     -E\left(I\left(d(Y,\Tilde{Y})<d(\Bar{Y},\Tilde{Y})\right)|\Bar{Y},\Tilde{Y}
    %     \right)$ and $\dot{f}_j(X) =  f_j(X)-E(f_j(X))$.
    % \end{theorem}

    % \begin{remark}
    %     Note that multiplying any solution to Equation \ref{prob1} by an orthogonal matrix yields another valid solution. Therefore, without loss of generality, we assume that $\mf^*$
    %     is the solution satisfying its FCCov matrix i.e.
    %     \begin{equation}\label{equ:FCCov matrix}
    %         E\left[ E\left\{ \dot{\mf}(X) k(I) \mid \bar{Y}, \tilde{Y} \right\} E\left\{ \dot{\mf}(X) k(I) \mid \bar{Y}, \tilde{Y} \right\}^\top \right]
    %     \end{equation}
    %     is diagonal. 
    %     In the context of linear sufficient dimension reduction, the first diagonal element of the FCCov matrix is analogous to the largest eigenvalue of the SDR matrix. The solution $f_1^*$ corresponds to the eigenvector associated with this eigenvalue.
    % \end{remark}

    
    % 定理\ref{constrained}说明了,限制优化问题的解的每一个分量都在central sigma fields中。
    % 这说明了每个解f的分量在$\mathfrak{S}_{\sigma\{\mf_0(X)\}}$空间正交的意义下
    % 分别保留了关于$\mf_0(X)$的信息(降维后关于Y的所有信息)。
    % 这里需要特意强调的是,我们的目的是为了找到一个降维函数使得它对X降维后,可以保留X
    % 尽可能多的关于Y的信息。而不是去精确的找到f0。
    % 还有一点需要注意的是,因为central class是一个无穷维函数空间,所以我们的定理
    % 是对于任意的m都成立。
    % Theorem \ref{constrained} establishes that each component of the solution to the constrained optimization problem belongs to the function class of the central class \(\mathfrak{S}_{\sigma\{\mf_0(X)\}}\). 
    % This implies that every component of the solution \(\mf\) retains information about \(\mf_0(X)\). 
    % It is crucial to emphasize that our primary objective is to identify a dimension reduction function that, while reducing the dimensionality of \(X\), preserves as much information about \(Y\) as possible, rather than precisely recovering \(\mf_0\). 
    % Additionally, it is important to note that since the central class may constitute an infinite-dimensional function space, our theorem is applicable for any arbitrary \(m\).
    
    \subsection{Tractable Objective Function With Regularization}
    Our Theorem \ref{constrained} holds for any arbitrary $m$. However, noting that $\mf_0$ is an $\R^d$-valued mapping, it is reasonable to choose the solution $\mf^*$ with $m = d$ as an alternative to $\mf_0$.
    To address the intractable variance constraints in \eqref{prob1}, we incorporate the squared Frobenius norm of the difference between \(\Var\{\mf(X)\}\) and the identity matrix \(I_d\), thereby reformulating the problem into an unconstrained Lagrangian objective.
    % The variance constraints in \eqref{prob1} present significant challenges for analyzing statistical properties. The classical augmented Lagrange method (ALM) provides a promising solution through penalization for constraint violations. Specifically, ALM incorporate a penalty term that quantifies the degree of constraint violation into the objective function. 
    % It is crucial to note that the constraint \(\Var{\mf(X)} = I_d\) is equivalent to \(\|\Var{\mf(X)} - I_d\|_F^2 = 0\). Therefore, we can formally define the ALM version of \eqref{prob1} as follows:
    % \begin{equation}\label{unconstrained}
    %   \begin{split}
    %   L_F(\lambda, \mf) =
    %   & -E\left(E\left\{\dot{\mf}(X)k(I) | \Bar{Y}, \Tilde{Y}\right\}^\top E\left\{\dot{\mf}(X) k(I) | \Bar{Y}, \Tilde{Y}\right\}\right) \\
    %   & + \lambda \|\text{Var}\{\mf(X)\} - I_d\|_F^2,
    %   \end{split}
    % \end{equation}
    \begin{equation}\label{unconstrained}
      L_F(\lambda, \mf) = -\sum_{t=1}^d \FCCov(f_t(X)\mid Y) + \lambda \|\text{Var}\{\mf(X)\} - \textbf{I}_d\|_F^2,
    \end{equation}
    where  \(\lambda > 0\) is the regularization parameter. 
    To establish the relationship between the solution of the unconstrained objective function \eqref{unconstrained} and that of the constrained optimization problem \eqref{prob1}, we state the following assumption.
    \begin{assumption}
        \label{ass_eigengap}
        There is a strict gap among between $l_i^*$ and $l_j^*$  for $1 \leq i < j \leq d+1$,
	\[l_i^* = -\text{\normalfont{FCCov}}(f^*_i(X)\mid Y)<-\text{\normalfont{FCCov}}(f^*_j(X)\mid Y) = l_j^*,\] 
    %for $1 \leq i < j \leq m$. 
    \end{assumption}
    This condition is parallel to commonly used assumption that the nonzero eigenvalues are different in the literature of  linear SDR, which is essential in theoretical analyses involving eigen-decomposition.
    We then present the following theorem, which establishes the equivalence between the solutions of  \eqref{prob1} and \eqref{unconstrained} and also confirms the unbiasedness of soluton to \eqref{unconstrained} for nonlinear SDR.

    \begin{theorem}\label{thm_unconstrained} Let $\lambda>0$ and \(\mf_{\lambda}^*=(f^*_{1,\lambda}, f^*_{2,\lambda}, \ldots, f^*_{d,\lambda})^\top\) be the minimizer of \eqref{unconstrained} and  \(L = \text{diag}(l_1^*, \dots, l_d^*)\)  .  Then
    \begin{itemize}
    \item[(a).] $\mf^*_\lambda = V(\textbf{I}_d-\frac{L}{2\lambda})^{\frac{1}{2}}\mf^*$, where $V$ is an $m\times m$ orthogonal matrix.
    \item[(b).] \(f_{j, \lambda}^* \in \mathfrak{S}_{\sigma \{\mf_0(X)\}}\),  for all \(1 \leq j \leq d\).
    \end{itemize}
        %Let \(\lambda > 0\), \(\mf^*\) be defined as in Theorem \ref{constrained}, and \(\mf_{\lambda}^*\) be an optimal solution to the following optimization problem:
        %\begin{equation}\label{equ:population}
        %    \min_{\mf \in \{L_2(P_X)\}^d} L_F(\lambda, \mf)
        %\end{equation}
        %Then, $\mf^*_\lambda = V(\textbf{I}_d-\frac{L}{2\lambda})^{\frac{1}{2}}\mf^*$, which implies that \(\mf_{j, \lambda}^* \in \mathfrak{S}_{\sigma \{\mf_0(X)\}}\) for all \(1 \leq j \leq d\). Here, $V$ are orthogonal matrices in \(\mathbb{R}^{d \times d}\), and \(L = \text{diag}(l_1^*, \dots, l_d^*)\) is the diagonal matrix of FCCov values of the optimal solution to (\ref{prob1}).
    \end{theorem}
    % 与定理2.2类似,我们同样证明了无限制优化问题即(\ref{unconstrained})的解的每一个
    % 分量均在central class这个函数类中。
    % Similar to Theorem \ref{constrained}, we have also demonstrated that 
    % each component of the solution to the unconstrained optimization 
    % problem, as specified in (\ref{unconstrained}), belongs to the 
    % the central class $\mathfrak{M}_{\sigma\{\mf_0(X)\}}$.
    % 但需要注意的是,由于我们未对定理2.2约束问题的解加任何限制,$\mf^*$的任意正交
    % 变化$U\mf^*$也是问题的解,其中U为$\R^{d\times d}$的正交矩阵。解的不唯一性会
    % 对我们的数值求解及理论分析造成影响,因此我们做出如下假设
  %   Consequently, we make the following assumption:
  %   \begin{assumption}
  %       \label{ass_eigengap}
  %       \textcolor{red}{May reduce to for $l^*_d < l^*_{d+1}$}
  %       There is a strict gap among $l_i^*$ in Theorem \eqref{constrained} for $m=d+1$, \emph{i.e.},
		% $$l_i^* = -E\left(E^2\left\{\dot{f}_i(X)k(I)|
  %       \Bar{Y},\Tilde{Y}\right\}\right) < -E\left(E^2\left\{\dot{f}_j(X)k(I)|
  %       \Bar{Y},\Tilde{Y}\right\}\right) = l_j^*$$
		% for $1 \leq i < j \leq d+1$. 
  %   \end{assumption}
  %   This condition ensures that each nonzero eigenvalue is associated with a unique eigenvector, an essential precondition in theoretical analyses involving eigendecomposition. 
  %   Given the infinite dimensionality of the central class $\mathfrak{M}_{\sigma\{\mf_0(X)\}}$
  %   the $(d+1)$th eigenvalue is nonzero.
  %   Consequently, it becomes necessary to expand from $d$ to $d+1$ 
  %   distinct eigenvalues to ensure the uniqueness of $f_1^*, \cdots, f_d^*$
  %   as described in Theorem 3.2. This expansion is crucial for 
  %   establishing the necessary identifiability for further analysis. 
  %   A subsequent corollary elucidates the relationship between
  %   $\mf_\lambda^*$ and $\mf^*$ under assumption \ref{ass_eigengap}.

  %   \begin{corollary}\label{relation}
  %       Under the assumption \ref{ass_eigengap}, Let $\mf_\lambda^*$ and $\mf^*$
  %       defined as above. Then they differ only by the product of a linear transformation matrix, 
  %       i.e., 
  %       \begin{equation}
  %           \mf^*_\lambda = V(I_d-\frac{L}{2\lambda})^{\frac{1}{2}}\mf^*
  %       \end{equation}
  %       where $V \in \mathbb{R}^{d\times d}$ is any orthogonal matrix and
  %       $L=diag(l_1^*,...l_d^*)$.
  %   \end{corollary}


 \cite{huang2024deep} verified that the solution to the objective function of distance covariance with a normality constraint is able to recover part of $\mathfrak{S}_{\sigma \{\mf_0(X)\}}$. They further proposed to regularize certain divergences to replace the normality assumption for real implementations. However, the theoretical property of their regularized approach for nonlinear SDR remains unclear. \cite{chen2024deep}  proposed two regularized schemes for their proposed GMDD-Net approach. They only provided a thorough theoretical analysis for the sequential regularization scheme, which is computational intensive compared to our method. Similar to \cite{huang2024deep}, the theoretical properties of the alternative regularization scheme in \cite{chen2024deep} is not fully explored.
    \subsection{Sample Version}
    For the unconstrained objective \eqref{unconstrained}, direct optimization is infeasible due to the unknown distribution of \(Z = (X, Y)\). Therefore, given the dataset \(\mathscr{D}_n = \{ Z_1, Z_2, \dots, Z_n \}\), it is crucial to design an efficient sample estimator to support our optimization algorithm and control the excess risk bound of the solutions.
    We assume that $E \mf(X) = 0$ without loss of generality. Then the unbiased U-statistic form of $L_F(\lambda,\mf)$ can be expressed as follows:
    \begin{equation}\label{4th_sample_form}
        \begin{split}
            L_n(\lambda,\bm{f}) = \frac{1}{(n)_4}\left(\sum_{t=1}^d 
            \widehat{\FCCov}(f_t(X)|Y) + \lambda \sum_{(i,j,k,l)\in
            \mathcal{A}_{n,4}} h_{2,\bm{f}}(Z_{i},Z_{j},Z_{k},Z_{l})
            \right).
        \end{split}
    \end{equation}
    where $h_{2,\bm{f}}$ is the kernel of the unbiased U-statistic for the Lagrange term 
	$\|\text{Var}\{\mf(X)\} - I_d\|_F^2$. And it has the following expression.
	\begin{align*}
	h_{2,\bm{f}}(Z_1,Z_2,Z_3,Z_4)
    &=  \sum_{t=1}^{d}\left[
    \frac{1}{12}\sum_{(i,j)\in \mathcal{A}_{4,2}}f_t^2(X_i)f_t^2(X_j)
    - \frac{1}{2}\sum_{(i)\in\mathcal{A}_{4,1}}f_t^2(X_i)
    \right] + d\\
    &+ \sum_{1\leq s<t\leq d}\frac{1}{6}\sum_{(i,j)\in\mathcal{A}_{4,2}}
    f_s(X_i)f_s(X_j)f_t(X_i)f_t(X_j).
	\end{align*}
    The expression for \(\widehat{\FCCov}\) is provided in \eqref{FCCov_sample_form}. Clearly $L_n(\lambda,\mf)$ is a U-statistic with an implicit kernel. For theoretical convenience, we also derive its explicit kernel representation, which is provided in the supplementary material.
    Furthermore, the fast computational algorithm for \(\widehat{\FCCov}\) can be applied to accelerate the optimization process.
    We further utilize a neural networks function class $\mathcal{F}_n$, such as $\mF_{FNN}$ or $\mF_{CNN}$ as described in Section \ref{sec:preliminary}, to approximate smooth functions in $L_2(P_X)$. And our final FCCov-Net estimator for nonlinear SDR is defined as
    \begin{equation}\label{equ:sample}
        \hat{\mf}_\lambda=\argmin\limits_{\mf\in\{\mathcal{F}_n\}^d}  L_n(\lambda, \mf).
    \end{equation}

    \section{Nonasymptotic Convergence Rate}\label{sec:theory}
    This section investigates the non-asymptotic properties of our proposed nonlinear SDR method.
    We begin by considering the excess risk defines as
    \begin{equation*}
        R(\lambda, \Hat{\mf}_\lambda) = L_F(\lambda, \Hat{\mf}_\lambda)
        - L_F(\lambda, \mf_\lambda^*),
    \end{equation*}
    where $\mf_\lambda^* = (f_{1,\lambda}^*,...,f_{d,\lambda}^*)$ and $\Hat{\mf}_\lambda = (\Hat{f}_{1,\lambda},...,\Hat{f}_{d,\lambda})$ denote the minimizers of the objective function \eqref{unconstrained} and sample-based problem \eqref{equ:sample}, respectively.
    The excess risk quantifies the discrepancy between the estimated directions and their corresponding ideal counterparts, reflecting the accuracy of the estimated directions. Keep in mind that multiplying $\mf_\lambda^*$ by an $d\times d$ orthonormal matrix from the left will also satisfy the unbiased estimation property, we also define the following risk for theoretical investigation,
      \begin{equation*}
 \kappa^2(\Hat{\mf}_\lambda,{\mf}^*_\lambda) = \min\limits_{Q\in O(d)}\Vert \Hat{\mf}_\lambda -Q {\mf}^*_\lambda
    \Vert^2,
    \end{equation*}
where $O(d)$ is the class consisting of all orthonormal matrices in $\mathbb{R}^{d\times d}$. Before presenting the non-asymptotic results, we introduce some necessary assumptions.
    \begin{assumption}
        \label{ass:bounded}
		There exists an absolute constant $B>1$ such that $\|{f}^{*}_{j, \lambda}\|_{\infty} \leq B$ and $\|f\|_{\infty} \leq B$ for any $1 \leq j \leq d$ and $f \in \mF_n$. % and $y, \tilde{y} \in \mathbb{R}^q$.
    \end{assumption}
    
    \begin{assumption}
        \label{ass:holder} ${f}^{*}_{j, \lambda}$ is $\beta$-H\"older function for all $1 \leq j \leq m$, where a $\beta$-H\"older function $f$ is defined as  
        \[
        \|f\|_\beta := \sum_{0 \leq |\alpha| < \lfloor \beta \rfloor} \|\partial^\alpha f\|_\infty + \sum_{|\alpha| = \lfloor \beta \rfloor} \sup_{x \neq y} \frac{|\partial^\alpha f(x) - \partial^\alpha f(y)|}{|x - y|^{\beta - \lfloor \beta \rfloor}} < \infty,
        \]
        Here, $\alpha=\left(\alpha_1, \ldots, \alpha_D\right)$ represents a multi-index, and $\lfloor \beta \rfloor$ denotes the greatest integer less than or equal to $\beta$. Specifically, $\partial^\alpha f:=$ $\frac{\partial^{|\alpha|} f}{\partial x_1^{\alpha_1} \ldots \partial x_D^{\alpha_D}}$, where $|\alpha|:=\sum_{i=1}^D \alpha_i$.
    \end{assumption}

    \begin{assumption}\label{ass:neural network} The neural networks functional classes $\mF_{CNN}$ and $\mF_{FNN}$ satisfy
    \begin{itemize}
    \item[$\diamond$] $\mF_{CNN}$ with depth \(\mD = \mathcal{O}(\log \mM)\), number of channels \(\mH = \mathcal{O}(1)\), and convolutional kernel size \(\mathcal{K} \in \{2, \ldots, p\}\). We choose $\mM = n^{\frac{p}{2\beta + p}}$ and $n>\mM \log(\mM)$.
    \item[$\diamond$] $\mF_{FNN}$ with width $\mathcal{N} = 3^{p+3} \max\big(p \lfloor N^{1/p} \rfloor, N+1\big)$ where $N\in\mathbb{N}_+$ is any positive integer. We choose depth $\mathcal{L} = 12n^{\frac{p}{2(p + 2\beta)}} + 14+ 2p$ and $n>\mL \log(\mL)$.
    \end{itemize}
    \end{assumption}
    Assumptions \ref{ass:bounded} and \ref{ass:holder} are standard conditions frequently employed in the theoretical analysis of neural network error bounds, as demonstrated in \citet{Schmidt_Hieber_2020}. For the $\beta$-H\"older class, the parameter $\beta$ characterizes the smoothness of the function, with larger values of $\beta$ indicating smoother functions. To effectively approximate $\beta$-H\"older continuous functions while balancing the statistical error, we introduce Assumption \ref{ass:neural network}. This condition is also adopted in \citet{shen_app, nearly_tight} and \citet{oono2019approximation}, for studying the approximation capabilities and complexities of FNNs and CNNs. Moreover,  $\mL \log(\mL)$ and $\mM \log(\mM)$ are actually the VC dimensions of FNNs and CNNs respectively.
    %he approximation capabilities and complexities, such as the covering number and VC dimension, of FNNs have been comprehensively studied in  In parallel, \citet{oono2019approximation} study the approximation properties and non-parametric estimation of ResNet-type CNNs. 

    By applying the Hoeffding decomposition of U-statistics, we can divide $R(\lambda, \Hat{\mf}_\lambda) $ into several terms.
    The leading term is a sum of independent terms, which can be rigorously analyzed using classical empirical process theory. Additionally, we can bound the variance of this term using the previously defined orthogonal invariant distance $\kappa$.
    Under the aforementioned assumptions, the remainders of the Hoeffding decomposition become negligible compared to the first term, as ensured by Theorem 8.3 in \citet{massart2007concentration}.
    The following lemma establishes the upper bound of $R(\lambda, \Hat{\mf}_\lambda)$.
    \begin{lemma}
        \label{fast_rate}
        Under assumptions\ref{ass_eigengap}-\ref{ass:neural network}, let $\lambda>0$, then for any $\delta>0$, the following bounds hold with probability at least $1 - \delta$.
        %the following bounds hold for $L_F(\lambda, \Hat{\bm{f}}_\lambda) - L_F(\lambda, \bm{f}_\lambda^*)$ and $\kappa^2(\Hat{\bm{f}}_\lambda, \bm{f}_\lambda^*)$:
        \begin{itemize}
            \item[(a).] For $\mF_n=\mF_{FNN}$,
            \[R(\lambda, \Hat{\mf}_\lambda)=
            \mathcal{O} \left( \big(\mathcal{N}\mathcal{L}\big)^{-\frac{4 \beta}{p}}+\frac{\mathcal{S} \mathcal{L}\log\left(\mathcal{S}\right)}{n}\log\frac{n}{\mathcal{S} \mathcal{L}\log\left(\mathcal{S}\right)}  + \frac{\mathcal{S} \mathcal{L}\log\left(\mathcal{S}\right) \log(1/\delta)}{n} \right).
            \]
            \item[(b).] For $\mF_n=\mF_{CNN}$,
            \[R(\lambda, \Hat{\mf}_\lambda)=
            \mO\left(\mM^{-\frac{2\beta}{p}} + \frac{\mM\log \mM}{n}\log \frac{n}{\mM\log \mM} + \frac{\mM\log \mM}{n}\log(1/\delta)\right).
            \]
        \end{itemize}
        
    \end{lemma}
    
    % Assumption \ref{ass:neural network} focuses on the neural networks structure. 
    % Some studies, such as \citet{petersen2018optimal}, have demonstrated that for the H\"older class, the approximation and estimation error rates achieved by CNNs are minimax optimal. However, the optimal CNN architecture may be unrealistically wide, posing practical challenges in terms of computational efficiency and implementation.
    % Assumption \ref{ass:neural network} defines a constant-width CNN, where the depth increases with the sample size $n$, while the number of channels $\mH$ and the filter size $\mathcal{K}$ remain constant with respect to $n$. 
    % Constant-width CNNs are critical building blocks not only in ResNet but also in various modern CNNs such as \citet{DenseNet}. Under Assumption \ref{ass:neural network} for CNNs, we have the following approximation error for function in H\"older class.
    % \[
    % \sup_{f \in \mathcal{F}_{CNN}, \, h \in C^{\beta}([0, 1]^p)} \|f - h\|_{\infty} = \mathcal{O}\left(\mM^{-\frac{\beta}{p}}\right).
    % \]
    % To effectively approximate $\beta$-H\"older continuous functions, we refer to \citet{shen_app} and introduce Assumption \ref{ass:neural network} for FNNs. Under this assumption, the approximation error is bounded as follows:
    % \[
    % \sup_{f \in \mathcal{F}_{FNN}; \, h \in C^{\beta}([0, 1]^p)} \|f - h\|_{\infty} = \mathcal{O}\left( (\mathcal{N} \mathcal{L})^{-\frac{2 \beta}{p}} \right),
    % \]
    
    % Correspondingly, \citet{oono2019approximation} demonstrates that ResNet-type CNNs achieve minimax-optimal estimation error rates for the H\"older class and establish corresponding approximation error bounds with respect to structural parameters.
    % According to \citet{shen_app, oono2019approximation}, the approximation error of $\mF_{FNN}$ and $\mF_{CNN}$ for $\beta$-H\"older continuous functions can be characterized in terms of the network width $\mathcal{N}$, depth $\mathcal{L}$ for FNNs, and the number of residual blocks $\mM$ for CNNs, i.e.,
    % \[
    % \sup_{f \in \mathcal{F}_{FNN}; h \in C^{\beta}([0, 1]^p)}\|f - h\|_{\infty} = \mathcal{O}\left( (\mathcal{N} \mathcal{L})^{-\frac{2 \beta}{p}}\right).
    % \]
    % If the convolutional network satisfies the following conditions: depth of residual blocks \(\mD = \mathcal{O}(\log \mM)\), number of channels \(\mH = \mathcal{O}(1)\), and convolutional kernel size \(\mathcal{K} \in \{2, \ldots, p\}\), we derive the following approximation error bounds:
    % \[
    % \sup_{f \in \mathcal{F}_{CNN}, \, h \in C^{\beta}([0, 1]^p)} \|f - h\|_{\infty} = \mathcal{O}\left(\mM^{-\frac{\beta}{p}}\right)
    % \]

    Finally, by integrating Lemma \ref{fast_rate} and selecting the sample size $n$ to balance the approximation error and the statistical error, we achieve the nonasymptotic error rate for our proposed FCCov-Net, as stated in Theorem \ref{cor_NN}.
    %This trade-off is crucial for ensuring that the model achieves optimal performance while avoiding issues of overfitting or underfitting.

\begin{theorem}\label{cor_NN} Under assumptions \ref{ass_eigengap}-\ref{ass:neural network}, let $\lambda>0$, then for any \(\delta > 0\), we have
    \begin{itemize}
    \item[(a).]
      $R(\lambda, \Hat{\mf}_\lambda) = \mathcal{O}\left(n^{-\frac{2 \beta}{p + 2 \beta}} \{1 + \log(1/\delta)\} \log^2 n\right)$ holds true with probability greater than \(1 - \delta\).
      % \rho^2\left(\hat{\bm{f}}, \bm{f}_{\lambda}^*\right) = \mathcal{O}\left(n^{-\frac{2 \beta}{p + 2 \beta}} \{1 + \log(1/\delta)\} \log^2 n\right).
     \item [(b).]    $E \kappa^2\left(\hat{\bm{f}_\lambda}, \bm{f}_{\lambda}^*\right)=
      \mathcal{O}\left(n^{-\frac{2 \beta}{p+2 \beta}} \log^2 n\right).$
    % $E \min\limits_{Q\in O(d)}\Vert\hat{\bm{f}}-Q\bm{f}_{\lambda}^* \Vert^2 =
      %\mathcal{O}\left(n^{-\frac{2 \beta}{p+2 \beta}} \log^2 n\right).
      \end{itemize}
\end{theorem}

%In the proof of Theorem \ref{cor_NN}, we also derive a probabilistic upper bound for  $ \rho^2\left(\hat{\bm{f}}, \bm{f}_{\lambda}^*\right)$ with the same order of excess risk. Additionally, we obtain the convergence rate in expectation, given by:
%    $$
%      E \rho^2\left(\hat{\bm{f}}, \bm{f}_{\lambda}^*\right)=
%      \mathcal{O}\left(n^{-\frac{2 \beta}{p+2 \beta}} \log^2 n\right).
%    $$
    
    % The implication of Corollary \ref{cor_NN} is
    %   $$
    %   E \rho^2\left(\hat{\bm{f}}, \bm{f}_{\lambda}^*\right)=
    %   \mathcal{O}\left(n^{-\frac{2 \beta}{p+2 \beta}} \log^2 n\right).
    %   $$
    
    Compared to existing theoretical results of nonlinear Fr\'echet SDR methods, such as that of \cite{ying2022frechet}, our theory offers several advantages. First, we derive a non-asymptotic excess risk convergence rate, while \cite{ying2022frechet} only provided the asymptotic results.
    We also obtain a probabilistic upper bound on the distance between the estimated and true directions.
    This provides theoretical insights into the sample size required to achieve the desired accuracy in practice.
    Second, our method achieves a convergence rate of $n^{-\frac{2\beta}{p + 2\beta}}\log^2 n$, which is comparable to the minimax rate for estimating H\"older continuous functions in nonparametric regression with Euclidean response. 
    Compared to the theoretical results of \cite{chen2024deep}, which are limited to Euclidean responses and FNNs, our theoretical framework extends to both Euclidean and non-Euclidean responses, as well as to both FNNs and CNNs, demonstrating significantly broader applicability.
    % Compared to the theoretical result of \cite{chen2024deep} with Euclidean response and FNNs structure, our theoretical developments, for both Euclidean response and non-Euclidean response, and for both FNNS and CNNs, are certainly more general.

    
      % It is worth noting that the minimax rate for nonparametric estimation 
      % of H\"older continuous functions is established at 
      % $n^{-\frac{2\beta}{p + 2 \beta}}$. Remarkably, our nonlinear 
      % SDR estimator, leveraging deep neural networks, achieves this 
      % minimax optimal rate up to a logarithmic factor of $n$.
      
    %%% ————————————————————————————————————分割线————————————————————————————————————————————————————

  %   \section{Nonasymptotic Analysis}\label{sec:theory} 
  %   This section investigates the non-asymptotic properties of our proposed nonlinear SDR method.
  %   We begin by considering a general function class $\mF_n$ and, under mild assumptions, we establish a minimax optimal upper bound (up to a logarithmic factor) for the excess risk defined as:
  %   \begin{equation*}
  %       R(\lambda, \Hat{\mf}) = L_F(\lambda, \Hat{\mf})
  %       - L_F(\lambda, \mf_\lambda^*),
  %   \end{equation*}
  %   where $\mf_\lambda^* = (f_{1,\lambda}^*,...,f_{d,\lambda}^*)$ and $\Hat{\mf} = (\Hat{f}_{1,\lambda},...,\Hat{f}_{d,\lambda})$ denote the minimizers of the population problem \eqref{equ:population} and sample-based problem \eqref{equ:sample}, respectively.
  %   We then restrict $\mF_n$ to a neural network class such as $\mF_{FNN}$ or $\mF_{CNN}$.
  %   By appropriately selecting the parameters of the network framework, we derive the corresponding risk bound.
  %   The primary theoretical tool used is U-processes \cite{de2012decoupling}.

  %   \subsection{Non-asymptotic Excess Risk Bound}
  %   Before delving into the non-asymptotic results, we introduce the necessary assumptions.
  %   \begin{assumption}
  %       \label{ass:bounded}
		% There exists an absolute constant $B>1$ such that $\|{f}^{*}_{j, \lambda_j}\|_{\infty} \leq B$ and $\|f\|_{\infty} \leq B$ for any $1 \leq j \leq d$, $f \in \mF_n$, and $y, \tilde{y} \in \mathbb{R}^q$.
  %   \end{assumption}
    
  %   \begin{assumption}
  %       \label{ass:holder} ${f}^{*}_{j, \lambda}$ is $\beta$-H\"older function for all $1 \leq j \leq d$, \emph{i.e.}, $f$ is called a $\beta$-H\"older function if 
  %       \[
  %       \|f\|_\beta := \sum_{0 \leq |\alpha| < \lfloor \beta \rfloor} \|\partial^\alpha f\|_\infty + \sum_{|\alpha| = \lfloor \beta \rfloor} \sup_{x \neq y} \frac{|\partial^\alpha f(x) - \partial^\alpha f(y)|}{|x - y|^{\beta - \lfloor \beta \rfloor}} < \infty,
  %       \]
  %       Here, $\alpha=\left(\alpha_1, \ldots, \alpha_D\right)$ represents a multi-index, and $\lfloor \beta \rfloor$ denotes the greatest integer less than or equal to $\beta$. Specifically, $\partial^\alpha f:=$ $\frac{\partial^{|\alpha|} f}{\partial x_1^{\alpha_1} \ldots \partial x_D^{\alpha_D}}$, where $|\alpha|:=\sum_{d=1}^D \alpha_d$.
  %   \end{assumption}
    
  %   \begin{assumption}
  %       \label{ass:vcbound} There exist an universal constant $C$ and a parameter $V$ depending on $\mF_n$ such that 
		% $$ \log N(\epsilon, \mF_n, \|\cdot\|_{L_2(Q)}) \leq C V \big\{1 + \log (1/\epsilon)\big\},$$ 
  %        where $Q$ is the probability measure $P_X$ or $P_n$, the empirical probability measure of $X$.
  %   \end{assumption}
  %   Assumptions \ref{ass:bounded} and \ref{ass:holder} are commonly used conditions in theoretical analyses 
  %   of neural network error bounds, as exemplified in \cite{Schmidt_Hieber_2020}.
  %   As for Assumption \ref{ass:vcbound}, it is always satisfied for any VC class $\mF_n$
  %   with parameter $V$ as its VC dimension, such as in the cases of FNNs and CNNs (refer to \cite{vaart1996weak, oono2019approximation, shen2022approximation}).
    
  %   By applying the Hoeffding decomposition, we break down $R(\lambda, \mf)$ into two components. The first component is a sum of independent terms, which can be rigorously analyzed using classical empirical process theory. The second component becomes asymptotically negligible under the aforementioned assumptions.
  %   Finally, by controlling the gap with \(L_n\), we arrive at the following conclusion:
    
  %   \begin{theorem}
  %       \label{fast_rate}
  %       Under the assumption \ref{ass_eigengap}, \ref{ass:bounded} and \ref{ass:vcbound}, 
  %       the following holds for any $\delta>0$ with probability at least $1-\delta$,
  %       \begin{align*}
  %           L_F(\lambda,\Hat{\bm{f}}) - L_F(\lambda,\bm{f}_\lambda^*) = \mathcal{O}
  %           \left(\frac{V}{n} \log \frac{n}{V} + \frac{V\text{log}(1/\delta)}{n}+
  %           \inf\limits_{\bm{f}\in \{\mathcal{F}_n\}^d}\Vert\bm{f}-\bm{f}_\lambda^*\Vert^2_\infty
  %           \right),\\
  %           \rho^2(\Hat{\bm{f}},\bm{f}_\lambda^*) = \mathcal{O}
  %           \left(\frac{V}{n} \log \frac{n}{V} + \frac{V\text{log}(1/\delta)}{n}+
  %           \inf\limits_{\bm{f}\in \{\mathcal{F}_n\}^d}\Vert\bm{f}-\bm{f}_\lambda^*\Vert^2_\infty
  %           \right),
  %       \end{align*}
  %       Where $\Vert\bm{f}-\bm{f}_\lambda^*\Vert^2_\infty = \max_i \Vert f_i-f_{i,\lambda}^*\Vert^2_\infty$.
  %   \end{theorem}

  %   \subsection{Neural Network Approximation}
  %   We now consider the upper bound of the excess risk when  $\mathcal{F}_n$ is defined as the class of neural network functions. 
  %   The approximation capabilities and complexities, such as the covering number and VC dimension, of FNNs have been comprehensively studied in \cite{shen_app, nearly_tight}.
  %   CNNs are among the most widely used architectures, with diverse applications in fields such as computer vision, natural language processing, and sequence analysis in bioinformatics.
  %   In recent years, theoretical research on CNNs has advanced significantly. For example, \cite{oono2019approximation} has demonstrated that ResNet-type Convolutional Neural Networks (RCNNs) can achieve minimax-optimal approximation and estimation error rates.
    
  %   In this section, we focus on RCNNs and FNNs with ReLU activation functions, as described in Section \ref{sec:notation}, as our function class \(\mathcal{F}_n\). 
  %   We also denote the function class of RCNNs as $\mF_{CNN}$. Let $M\in \N_{+}$ represent the number of residual blocks. We define the ResNet-type CNN as a sequential concatenation of one convolution block $M$ residual blocks, and one fully-connected layer. The detailed definition of residual blocks and RCNNs is provided in the supplementary material for brevity.

  %   By exploiting the connection between the log covering number and the parameters of neural networks, we can easily demonstrate that assumption \ref{ass:vcbound} holds with with \(V = \mathcal{S} \mathcal{L} \log(\mathcal{S})\) for \(\mathcal{F}_{FNN}\) and \(V = M \log(M)\) for \(\mathcal{F}_{CNN}\). \cite{shen_app, oono2019approximation} have shown the approximation error bounds between the H\"older class and $\mF_{FNN}$ and $\mF_{CNN}$, respectively, with respect to their structural parameters. Combining these with Theorem \ref{fast_rate}, we obtain the following corollary.

  %   \begin{corollary}\label{cor_NN}
  %      Let \(\mathcal{F}_{CNN}\) and \(\mathcal{F}_{FNN}\) denote the neural networks class. 
  %      Under the assumptions in \ref{ass_eigengap}, \ref{ass:bounded}, and \ref{ass:holder}, we consider the case where \(n > V\) for appropriately chosen network parameters. 
  %      Here, \(V\) is defined as \(V = \mS \mL \log(\mS)\) for FNNs and \(V = M \log(M)\) for CNNs. 
  %      Then, for any \(\delta > 0\), the following equations hold for both \(\mathcal{F}_{CNN}\) and \(\mathcal{F}_{FNN}\) with a probability of at least \(1 - \delta\):
  %     \[
  %     \begin{gathered}
  %     L_F\left(\lambda, \hat{\bm{f}}\right) - L_F\left(\lambda, \bm{f}_{\lambda}^*\right) = \mathcal{O}\left(n^{-\frac{2 \beta}{p + 2 \beta}} \{1 + \log(1/\delta)\} \log^2 n\right), \\
  %     \rho^2\left(\hat{\bm{f}}, \bm{f}_{\lambda}^*\right) = \mathcal{O}\left(n^{-\frac{2 \beta}{p + 2 \beta}} \{1 + \log(1/\delta)\} \log^2 n\right).
  %     \end{gathered}
  %     \]
  %   where $$\rho^2(\bm{f},\bm{g}) = C_1\min\limits_{Q\in O(d)}\Vert\bm{f}-Q\bm{g}
  %   \Vert^2_{\{L_2(P_X)\}^d}$$ is a distance on $\{L_2(P_X)\}^d$, $O(d)$ is the class consisting of all orthogonal matrices in $\mathbb{R}^{d\times d}$ 
  %   and $C_1$ is an appropriate constant.
  %   \end{corollary}
  %     The implication of Corollary \ref{cor_NN} is
  %     $$
  %     E \rho^2\left(\hat{\bm{f}}, \bm{f}_{\lambda}^*\right)=
  %     \mathcal{O}\left(n^{-\frac{2 \beta}{p+2 \beta}} \log^2 n\right).
  %     $$
  %     It is worth noting that the minimax rate for nonparametric estimation 
  %     of H\"older continuous functions is established at 
  %     $n^{-\frac{2\beta}{p + 2 \beta}}$. Remarkably, our nonlinear 
  %     SDR estimator, leveraging deep neural networks, achieves this 
  %     minimax optimal rate up to a logarithmic factor of $n$.

  % ————————————————————————————————————————分割线——————————————————————————————————————————————————————

    
    \section{Simulation}\label{sec:simulation}
    %In the previous section, we proposed using neural networks to obtain an estimator for the sample-based problem \eqref{equ:sample}. 
    %For convenience, we refer to our method as FCCov-Net.
    In this section, we evaluate the performance of our proposed FCCov-Net through comprehensive simulation studies where the responses are either in Euclidean space or complex random objects such as probability distributions, symmetric positive definite matrices, and spherical data. For comparison in the Euclidean scenario, we include the Generalized Martingale Difference Divergence based on the Frobenius norm (GMDDNet-F) \citep{chen2024deep}, Deep Dimension Reduction (DDR) \citep{huang2024deep}, and the Generalized Sliced Inverse Regression (GSIR) \citep{libing_generalSDR}.
    We exclude the successive optimization method proposed in \citet{chen2024deep} owing to its marginal performance difference relative to GMDDNet-F and its computational inefficiency.
    GMDDNet-F and DDR are neural network based methods. GSIR, on the other hand, is a nonlinear SDR method based on reproducing-kernel Hilbert Space (RKHS). 
    Additionally, we introduce outliers into some Euclidean data to examine robustness of each method. For scenarios involving complex random objects, we compare our method with the nonlinear Weighted Inverse Regression Ensemble (WIRE) method introduced by \citet{ying2022frechet}, which is also a kernel-based approach.
    To evaluate the computational efficiency of the neural network approach, we select several scenarios and compared the time consumption of our method compared with GSIR and WIRE across varying sample sizes. The results indicate that the neural network approach FCCov-Net significantly outperforms traditional kernel methods in terms of computational speed, substantially reducing the computation time. For detailed results, please refer to the supplementary material.

    To effectively train neural networks in PyTorch, we set the hyperparameters as follows: the batch size is 100, and each epoch consists of 100 iterations. We use "Adam" with default parameters for optimization. The learning rate is set as $0.001$.
    Our method employs a simple fully connected network architecture with layer widths of $p,2^{\lfloor\log p\rfloor+1},2^{\lfloor\log p\rfloor+2},2^{\lfloor\log p\rfloor+1},$ $\ldots,2^4,d$, where $p$ and $d$ represent input and output dimension respectively.
    Unless otherwise specified, all training procedures are conducted based on these settings.

    \subsection{Euclidean Responses} 
   % Owing to space limitations, the simulation results for Model I $(d = 1)$ and Model IV $(d = 2)$ under settings (A), (B), and (C) are included in the supplementary material.
    We consider the following two models:
    \begin{flalign*}
        &\textbf{Model I:} \quad Y = \left(\frac{x_1^2}{1+(0.1+0.5x_2)^2}, 0, 0\right)^\top
        + \bm{\varepsilon}_1, & \\
        &\textbf{Model II:} \quad Y = \left(\frac{x_3}{x_4+2}, x_1^2\right)^\top + \bm{\varepsilon}_2, &
    \end{flalign*}
    where $\bm{\varepsilon}_1 \sim 0.25\cdot N(\bm{0}, \textbf{I}_3)$ and $\bm{\varepsilon}_2\sim 0.1\cdot N(\bm{0}, \textbf{I}_2)$
    are independent of $X$.
    Additionally, we consider three different distributional scenarios for the $p$-dimensional predictor vector $X = (x_1, \ldots, x_p)^\top$:
    \begin{flalign*}
        &\textbf{(A):} \quad  X \sim N(\bm{0}, \textbf{I}_p), &\\
        &\textbf{(B):} \quad  X \sim N(\bm{0}, \Sigma), \text{where $\Sigma=(0.5^{|i-j|})$ 
        for $i,j=1,\ldots,p$}, &\\
        &\textbf{(C):} \quad  X \sim U([-2,2]^p), \text{where $[-2,2]^p=\underbrace{[-2,2]\times[-2,2]\times\cdots\times[-2,2]}_{p}$ }, &
        % &\textbf{(C):} \quad  X \sim t(\textbf{0}_p,\Sigma , 2) \text{where $\Sigma=(0.5^{|i-j|})$ 
        % for $i,j=1,\ldots,p$}, &\\
    \end{flalign*}
    where $\textbf{I}_p$ is the identity matrix and $U([-2,2]^p)$ is multivariate uniform distribution.
    The true sufficient predictors for models II and III are 
    \[
        \frac{x_1^2}{1+(0.1+0.5x_2)^2} \text{\hspace{1cm} and \hspace{1cm}} \begin{pmatrix}
            \frac{x_3}{x_4+2} \\
            x_1^2
        \end{pmatrix}
    \] respectively. Thus, the structural dimensions of the nonlinear SDR are $d = 1$ for Model I and $d = 2$ for Model II.
    % 对于GMDDNet-F,我们同样选择正则化项为F范数平方,正则化参数通过网格搜索选取。
    % For GMDDNet-F, we also opt for the squared Frobenius norm as the regularization term, 
    % with the regularization parameter selected through a grid search approach.
    % Similarly, the hyperparameter of GSIR is chosen by generalized cross validation criterion
    % in section 13.7 of \cite{li2018sufficient}.
    % The parameters of GMDDNet-F and GSIR are set to their default values.
    % The network architecture of GMDDNet-F is identical to that of our proposed method. 
    % In the DDR framework, we adopt a 4-layer network for the Generator 
    % and a 3-layer network for the Discriminator, both utilizing Leaky 
    % ReLU activation functions, consistent with the architecture used 
    % in \cite{huang2024deep}.
    
    % For each combination of models and scenarios, we generate high-dimensional training 
    % samples to compare our method with GMDDNet-F, GSIR and DDR. Specifically, we conducted experiments 
    % to evaluate the performance of each method using different combinations of  $p=10, 20$
    % and training samples $n=500, 800, 1000$. 

    For each combination of settings and distributional scenarios, we conducted experiments under $p = 10, 20$ and $n = 500, 800, 1000$. To assess the performance of these methods, we utilize the distance correlation $\rho\{f(X), \hat{f}(X)\}$ proposed by \citet{szekely2007measuring} between the true sufficient predictors and their estimators. A larger value of $\rho$ indicates better estimation. To compute the distance correlations, an independent testing sample comprising 20\% of the training sample size is generated. To mitigate the effects of randomness, this procedure is repeated 100 times. The mean and standard deviation of the distance correlations from these 100 repetitions are computed as the evaluation criteria.
    
    \begin{table}[htbp]
        \centering
        \small
        \renewcommand{\arraystretch}{0.58} % Increase the row spacing
        \caption{Mean (standard deviation) of distance correlations with true predictor based on 100 repetitions.}
        \label{tab:Euclidean}
        \scalebox{0.9}{
        \begin{tabular}{@{\hspace{10pt}}c@{\hspace{10pt}}c@{\hspace{10pt}}c@{\hspace{5pt}}c@{\hspace{15pt}}c@{\hspace{15pt}}c@{\hspace{15pt}}c@{\hspace{15pt}}c@{\hspace{15pt}}}
            \toprule
            \multicolumn{3}{c}{Models} & & \multicolumn{4}{c}{Distance correlation with true predictor} \\ 
            \cmidrule(r){1-3} \cmidrule(r){5-8} 
            $X$ & $Y|X$ & $(n,p)$ & & FCCov-Net & GMDDNet-F & GSIR & DDR \\  % 添加新列标题
            \midrule
            A  & I & (500, 10) & & \textbf{0.930(0.016)}  & 0.905(0.102)  & 0.929(0.021)  & 0.878(0.159)\\
                & & (800, 10) & & \textbf{0.941(0.009)}  & 0.938(0.010)  & 0.947(0.016)  & 0.924(0.129)\\
                & & (1000, 10) & & 0.945(0.009)  & 0.944(0.008)  & 0.952(0.011)  & \textbf{0.957(0.036)}\\
                & & (500, 20) & & \textbf{0.807(0.035)}  & 0.701(0.177)  & 0.712(0.074)  & 0.792(0.208)\\
                & & (800, 20) & & 0.841(0.025)  & 0.850(0.072)  & 0.787(0.052)  & \textbf{0.904(0.112)}\\
                & & (1000, 20) & & 0.855(0.021)  & 0.880(0.040)  & 0.815(0.041)  & \textbf{0.929(0.088)}\\
            \noalign{\vskip 5pt}
           & II & (500, 10) & & \textbf{0.775(0.131)}  & 0.765(0.186)  & 0.743(0.130)  & 0.694(0.190)\\
            & & (800, 10) & & \textbf{0.763(0.134)}  & 0.758(0.178)  & 0.731(0.129)  & 0.751(0.157)\\
            & & (1000, 10) & & 0.763(0.153)  & 0.741(0.208)  & 0.732(0.149)  & \textbf{0.788(0.171)}\\
            & & (500, 20) & & \textbf{0.666(0.123)}  & 0.690(0.185)  & 0.589(0.106)  & 0.562(0.182)\\
            & & (800, 20) & & \textbf{0.719(0.102)}  & 0.729(0.191)  & 0.631(0.091)  & 0.699(0.172)\\
            & & (1000, 20) & & 0.698(0.151)  & 0.689(0.224)  & 0.611(0.134)  & \textbf{0.710(0.195)}\\
        \noalign{\vskip 5pt}
            \midrule
        B  & I & (500, 10) & & \textbf{0.937(0.014)}  & 0.916(0.019)  & 0.825(0.043)  & 0.864(0.179)\\
            & & (800, 10) & & \textbf{0.947(0.009)}  & 0.943(0.009)  & 0.859(0.027)  & 0.922(0.151)\\
            & & (1000, 10) & & 0.950(0.010)  & 0.948(0.009)  & 0.862(0.028)  & \textbf{0.951(0.077)}\\
            & & (500, 20) & & \textbf{0.823(0.036)}  & 0.749(0.159)  & 0.577(0.080)  & 0.791(0.224)\\
            & & (800, 20) & & 0.854(0.022)  & 0.862(0.058)  & 0.636(0.067)  & \textbf{0.922(0.083)}\\
            & & (1000, 20) & & 0.869(0.019)  & 0.895(0.030)  & 0.670(0.052)  & \textbf{0.913(0.142)}\\
        \noalign{\vskip 5pt}
        & II & (500, 10) & & \textbf{0.774(0.118)}  & 0.708(0.190)  & 0.713(0.103)  & 0.696(0.147)\\
        & & (800, 10) & & \textbf{0.716(0.198)}  & 0.638(0.243)  & 0.658(0.178)  & 0.664(0.211)\\
        & & (1000, 10) & & \textbf{0.717(0.186)}  & 0.685(0.216)  & 0.653(0.164)  & 0.695(0.191)\\
        & & (500, 20) & & \textbf{0.703(0.105)}  & 0.678(0.162)  & 0.632(0.087)  & 0.539(0.154)\\
        & & (800, 20) & & \textbf{0.721(0.118)}  & 0.679(0.178)  & 0.630(0.099)  & 0.621(0.167)\\
        & & (1000, 20) & & \textbf{0.714(0.135)}  & 0.664(0.186)  & 0.621(0.110)  & 0.677(0.171)\\
        \noalign{\vskip 5pt}
        \midrule
        C  & I & (500, 10) & & \textbf{0.951(0.009)}  & 0.876(0.155)  & 0.855(0.035)  & 0.877(0.179)\\
            & & (800, 10) & & \textbf{0.968(0.006)}  & 0.920(0.129)  & 0.897(0.017)  & 0.934(0.122)\\
            & & (1000, 10) &  & \textbf{0.973(0.004)}  & 0.956(0.006)  & 0.911(0.015)  & 0.969(0.009)\\
            & & (500, 20) & & \textbf{0.831(0.040)}  & 0.595(0.159)  & 0.477(0.084)  & 0.797(0.177)\\
            & & (800, 20) & & 0.885(0.021)  & 0.799(0.132)  & 0.543(0.065)  & \textbf{0.910(0.063)}\\
            & & (1000, 20) & & 0.900(0.017)  & 0.862(0.085)  & 0.604(0.052)  & \textbf{0.935(0.077)}\\
        \noalign{\vskip 5pt}
        & II & (500, 10) & & \textbf{0.663(0.149)}  & 0.553(0.171)  & 0.625(0.123)  & 0.566(0.114)\\
        & & (800, 10) & & \textbf{0.661(0.162)}  & 0.507(0.172)  & 0.623(0.153)  & 0.561(0.131)\\
        & & (1000, 10) & & \textbf{0.667(0.154)}  & 0.538(0.174)  & 0.616(0.136)  & 0.592(0.117)\\
        & & (500, 20) & & 0.525(0.110)  & 0.496(0.150)  & \textbf{0.551(0.100)}  & 0.465(0.107)\\
        & & (800, 20) & & \textbf{0.593(0.134)}  & 0.489(0.158)  & 0.557(0.113)  & 0.502(0.133)\\
        & & (1000, 20) & & \textbf{0.590(0.141)}  & 0.489(0.161)  & 0.536(0.119)  & 0.517(0.130)\\
    \noalign{\vskip 5pt}
        \bottomrule
        \end{tabular}
        }
    \end{table}

    From Table \ref{tab:Euclidean}, our proposed method consistently outperforms 
    the other three approaches in the majority of cases.
    Next, to illustrate the robustness of our method in the presence of outliers, we follow \citet{zhang2019robust, zhang2021robust} to modify the distributions of 
    $X$ and the error $\bm{\varepsilon}$ term , which are stated as follows:

    \textbf{Outlier case-1}
    We generate \( n \) samples of \( X \sim N(0, \Sigma) \), where \( \Sigma = (\sigma_{ij}) \) with \( \sigma_{ij} = 0.5^{|i - j|} \) for \( i, j = 1, 2, \ldots, p \). Then, we randomly replace \( r\% \) of the observations of \( x_1 \) with \( 2 \cdot t(1) \), where \( t(1) \) denotes the t-distribution with one degree of freedom. 

    \textbf{Outlier case-2}
    Let $\bm{\varepsilon}_i \sim 0.25\cdot N(\bm{0},\textbf{I})$, $i=1,2$ and $X$ satisfies \textbf{(B)}.
    Subsequently, we introduce outliers 
    by randomly replacing the distribution of the error 
    term \( \varepsilon \) with \( r\% \) of values drawn from a uniform 
    distribution \( U(-50, 50) \).

    Where $r\%$ is set to be $10\%$, $30\%$ and $50\%$ and the the means and standard errors of 
    distance correlations of three methods are given in Table \ref{tab:outliers_1} and \ref{tab:outliers_2},
    \begin{table}[htbp]
        \centering
        \small
        \renewcommand{\arraystretch}{0.6} % Increase the row spacing
        \caption{Distance correlation for \textbf{outlier case-1} when predictors have outliers.}
        \label{tab:outliers_1}
        \scalebox{0.9}{
        \begin{tabular}{@{\hspace{8pt}}c@{\hspace{8pt}}c@{\hspace{8pt}}c@{\hspace{5pt}}c@{\hspace{12pt}}c@{\hspace{12pt}}c@{\hspace{12pt}}c@{\hspace{12pt}}c@{\hspace{12pt}}}
            \toprule
            \multicolumn{3}{c}{Models} & & \multicolumn{4}{c}{Distance correlation with true predictor} \\ 
            \cmidrule(r){1-3} \cmidrule(r){5-8} 
            outlier(r\%) & $Y|X$ & $(n,p)$ & & FCCov-Net & GMDDNet-F & GSIR & DDR \\ % 添加新列标题
            \midrule
            10\% & I & (1000, 10) & & \textbf{0.940(0.097)}  & 0.297(0.098)  & 0.553(0.224)  & 0.633(0.202)\\ % 每行添加新数据
            & I & (3000, 20) & & \textbf{0.883(0.163)}  & 0.148(0.062)  & 0.351(0.192)  & 0.691(0.188)\\
            & II & (1000, 10) & & \textbf{0.667(0.185)}  & 0.329(0.086)  & 0.585(0.199)  & 0.591(0.146)\\
            & II & (1000, 20) & & \textbf{0.662(0.142)}  & 0.301(0.091)  & 0.521(0.191)  & 0.526(0.153)\\
            \noalign{\vskip 5pt} % Adjust spacing before the more compact section
            30\% & I & (1000, 10) & & \textbf{0.905(0.180)}  & 0.171(0.071)  & 0.240(0.155)  & 0.254(0.106)\\
              & I & (3000, 20) & & \textbf{0.906(0.148)}  & 0.111(0.051)  & 0.134(0.083)  & 0.255(0.124)\\
              & II & (1000, 10) & & \textbf{0.632(0.228)}  & 0.293(0.090)  & 0.381(0.155)  & 0.326(0.101)\\
              & II & (1000, 20) & & \textbf{0.548(0.166)}  & 0.202(0.045)  & 0.339(0.142)  & 0.276(0.081)\\
            \noalign{\vskip 5pt} % Adjust spacing before the more compact section
            50\% & I & (1000, 10) & & \textbf{0.946(0.083)}  & 0.162(0.059)  & 0.170(0.067)  & 0.172(0.065)\\
            & I & (1000, 20) & & \textbf{0.824(0.145)}  & 0.132(0.038)  & 0.150(0.061)  & 0.147(0.044)\\
            & II & (1000, 20) & & \textbf{0.573(0.203)}  & 0.208(0.051)  & 0.265(0.087)  & 0.231(0.053)\\
            & II & (3000, 20) & & \textbf{0.524(0.264)}  & 0.183(0.056)  & 0.184(0.049)  & 0.195(0.070)\\
            \bottomrule
          \end{tabular}
        }
    \end{table}
    
    \begin{table}[htbp]
        \centering
        \small
        \renewcommand{\arraystretch}{0.6} % Increase the row spacing
        \caption{Distance correlation for \textbf{outlier case-2} when predictors have outliers.}
        \label{tab:outliers_2}
        \scalebox{0.9}{
        \begin{tabular}{@{\hspace{8pt}}c@{\hspace{8pt}}c@{\hspace{8pt}}c@{\hspace{5pt}}c@{\hspace{12pt}}c@{\hspace{12pt}}c@{\hspace{12pt}}c@{\hspace{12pt}}c@{\hspace{12pt}}}
            \toprule
            \multicolumn{3}{c}{Models} & & \multicolumn{4}{c}{Distance correlation with true predictor} \\ 
            \cmidrule(r){1-3} \cmidrule(r){5-8} 
            outlier(r\%) & $Y|X$ & $(n,p)$ & & FCCov-Net & GMDDNet-F & GSIR & DDR \\ % 添加新列标题
            \midrule
            10\% & I & (1000, 10) & & \textbf{0.930(0.010)}  & 0.512(0.270)  & 0.851(0.025)  & 0.254(0.106)\\ % 每行添加新数据
            & I & (3000, 20) & & \textbf{0.812(0.022)}  & 0.252(0.096)  & 0.723(0.026)  & 0.224(0.094)\\
            & II & (1000, 10) & & \textbf{0.702(0.160)}  & 0.586(0.184)  & 0.662(0.151)  & 0.343(0.089)\\
            & II & (3000, 10) & & \textbf{0.734(0.138)}  & 0.634(0.197)  & 0.682(0.131)  & 0.371(0.114)\\
            \noalign{\vskip 5pt} % Adjust spacing before the more compact section
            30\% & I & (1000, 10) & & \textbf{0.887(0.019)}  & 0.230(0.096)  & 0.701(0.069)  & 0.206(0.080)\\
              & I & (3000, 10) & & \textbf{0.918(0.010)}  & 0.384(0.136)  & 0.817(0.028)  & 0.247(0.122)\\
              & II & (1000, 10) & & \textbf{0.568(0.128)}  & 0.346(0.076)  & 0.498(0.101)  & 0.322(0.083)\\
              & II & (3000, 10) & & \textbf{0.636(0.161)}  & 0.397(0.149)  & 0.581(0.146)  & 0.308(0.090)\\
            \noalign{\vskip 5pt} % Adjust spacing before the more compact section
            50\% & I & (1000, 10) & & \textbf{0.700(0.105)}  & 0.184(0.077)  & 0.272(0.123)  & 0.201(0.078)\\
            & I & (3000, 10) & & \textbf{0.852(0.019)}  & 0.196(0.079)  & 0.566(0.101)  & 0.187(0.109)\\
            & II & (1000, 10) &  & 0.320(0.075)  & 0.259(0.062)  & \textbf{0.332(0.076)}  & 0.305(0.080)\\
            & II & (3000, 10) & & \textbf{0.399(0.120)}  & 0.290(0.097)  & 0.390(0.101)  & 0.290(0.093)\\
            \bottomrule
          \end{tabular}
        }
    \end{table}

    GMDDNet-F and DDR nearly entirely lose their ability to identify the nonlinear SDR directions when outliers are present in $X$.
    In contrast, the GSIR method demonstrates relatively strong performance across certain models, though it still suffers some reduction in statistical power. Our method, however, remains relatively stable with outliers, exhibiting high robustness.
    We also assess the effectiveness of our proposed dimensionality determination algorithm, which accurately identifies the structural dimension in nine out of ten trials. Further details are provided in the supplementary material.

    \subsection{Complex Metric Space Valued Responses}
    In this section, we compare our method with nonlinear WIRE across two Fr\'echet regression scenarios. %Due to space constraints, the spherical responses are provided in the supplementary material. 
    We generate \( n \) i.i.d. samples \( \{(X_k, Y_k)\}_{k=1}^n \), where \( X_k \) is a \( p \)-dimensional random vector with independent components uniformly distributed on \( [0, 1] \). The responses \( Y_k \) are generated according to following two scenarios.

    \subsubsection{Distributions responses}
    Let $(\Omega, d_w)$ be the metric space of probability distributions on $\R$ with 
    finite second order moments, equipped with the quadratic Wasserstein metric $d_w$. 
    For distributions $Y_1, Y_2 \in \Omega$, the squared Wasserstein distance is defined as
    \begin{equation}
        \label{Wasserstein}
        d_w^2(Y_1,Y_2) = \int_{0}^{1} \left(Y_1^{-1}(t)-Y_2^{-1}(t)\right) dt
    \end{equation}
    where $Y_1^{-1}$ and $Y_2^{-1}$ are their respective quantile functions.

    We generate random normal distribution $Y$ with quantile function
    $Q_Y(t) = \mu_Y + \sigma_Y\Phi(t)$, where $\Phi(\cdot)$ is the standard normal cumulative distribution function.
    Let $\beta_1 = (0.75,0.25,0,...,0)^\top$, $\beta_2=(0.25, 0.75, 0,...,0)^\top$.
    
    \textbf{Setting I-1}: 
    $\mu_Y\mid X \sim N\{D(X), 0.1^2\}$ and $\sigma_Y = 1$,
    where $D(X) = \sin\Big(4\pi\beta_1^\top X \cdot (2\beta_2^\top X-1)\Big)$.
    % 直接用D_2(X)效果都很好
    \textbf{Setting I-2}: 
    $\mu_Y\mid X \sim N\{D_1(X), 0.1^2\}$ and $\sigma_Y = |D_2(X)| $,
    where $D_1(X) = \left(x_1^2+x_2^2\right)^{\frac{1}{2}} \log\left(x_1^2+x_2^2\right)^{\frac{1}{2}}$,
    $D_2(X) = \sin\left(0.1\pi(x_4+x_5)\right) + x_4^2$.
    
    In Setting I-1, only the mean $\mu_X$ is related to $X$, whereas in Setting I-2, both the mean $\mu_X$ and standard deviation $\sigma_X$ depend on $X$. These settings correspond to structural dimensions $d=1$ and $d=2$, respectively. For the computation of Wasserstein distance \eqref{Wasserstein}, each quantile function of $Y_i$ is discretized into $21$ equispaced points on $[0, 1]$. %From Equation \eqref{Wasserstein}, the Wasserstein distance between two distributions is equivalent to the Euclidean distance between their quantile points. 
    The simulation results in Table \ref{distribution} demonstrate that our method outperforms the WIRE.

    \begin{table}[htbp]
        \centering
        \renewcommand{\arraystretch}{0.6} % Increase the row spacing
        \caption{Distance correlations for models with distributional responses.}
        \label{distribution}
        \scalebox{0.9}{
        \begin{tabular}{@{\hspace{25pt}}c@{\hspace{25pt}}c@{\hspace{10pt}}c@{\hspace{25pt}}c@{\hspace{25pt}}c@{\hspace{25pt}}}
            \toprule
            Models & $(p,n)$ &  & FCCov-Net & WIRE \\
            \midrule
            & (10, 1000) & & \textbf{0.723(0.010)} & 0.450(0.002)   \\
            & (10, 2000) & & \textbf{0.922(0.001)} & 0.466(0.001) \\
            \textbf{Setting I-1}& (10, 5000) &  & \textbf{0.976(0.001)} & 0.467(0.000) \\
            & (30, 1000) &  & 0.296(0.005) & \textbf{0.394(0.005)} \\
            & (30, 2000) &  & \textbf{0.524(0.022)} & 0.405(0.001) \\
            & (30, 5000) &  & \textbf{0.841(0.079)} & 0.433(0.000) \\
            \\
            & (10, 1000) & & \textbf{0.937(0.000)} & 0.928(0.000)   \\
            & (10, 2000) & & \textbf{0.938(0.000)} & 0.927(0.000)   \\
            \textbf{Setting I-2}& (10, 5000) &  & \textbf{0.935(0.000)} & 0.926(0.000) \\
            & (30, 1000) &  & \textbf{0.933(0.000)} & 0.922(0.000) \\
            & (30, 2000) &  & \textbf{0.935(0.000)} & 0.923(0.000) \\
            & (30, 5000) &  & \textbf{0.937(0.000)} & 0.920(0.000) \\
            \bottomrule
          \end{tabular}
        }
    \end{table}
    

    \subsubsection{Symmetric positive-definite matrices responses}
    Let $(\Omega, d_\omega)$ be the metric space $\mS_m^+$ of $m\times m$ symmetric positive-definite matrices endowed with metric $d_\omega$. 
    Although many metrics exist, this section primarily focuses on the Log-Cholesky metric \citep{lin2019riemannian} and the affine-invariant metric \citep{moakher2005differential}.

    To define the aforementioned metrics, we first introduce the exponential and logarithmic operations for matrices.
    For an \( m \times m \) symmetric matrix \( A \), the matrix exponential is defined as \( \exp(A) = I_m + \sum_{j=1}^{\infty} \frac{1}{j!} A^j \), which yields a symmetric positive-definite matrix. Conversely, for a symmetric positive-definite matrix \( Y \), the matrix logarithm is defined as \( \log(Y) = A \), such that \( \exp(A) = Y \). Therefore, for two symmetric positive-definite matrices \( Y_1 \) and \( Y_2 \), the affine-invariant distance can be formulated as follows: 
    \begin{equation} \label{Affine} 
    d_A(Y_1, Y_2) = \left\Vert \log\left(Y_1^{-1/2} Y_2 Y_1^{-1/2}\right) \right\Vert_F \end{equation}

    Similarly, for $Y_1$ and $Y_2$, we denote their Cholesky decompositions as \( Y_i = P_i P_i^\top \) for \( i = 1, 2 \).
    Let \( \lfloor Y \rfloor \) denote the strictly lower triangular part of \( Y \), and let \( \mathbb{D}(Y) \) denote the diagonal part of \( Y \).
    Then, we define the Log-Cholesky distance as
    \begin{equation}
    \label{Log-Cholesky}
    d_L(Y_1, Y_2) = \left\{ \left\| \lfloor P_1 \rfloor - \lfloor P_2 \rfloor \right\|_F^2 + \left\| \log \mathbb{D}(P_1) - \log \mathbb{D}(P_2) \right\|_F^2 \right\}^{1/2}
    \end{equation}

    
    We generate \( Y \) from a symmetric matrix-variate normal distribution \citep{zhang2021dimension}. In the simplest case, an \( m \times m \) symmetric matrix \( A \) follows a matrix-variate normal distribution \( A \sim N_{mm}(M, \sigma^2) \) if \( A = \sigma Z + M \), where \( M \) is an \( m \times m \) symmetric matrix and \( Z \) is an \( m \times m \) symmetric random matrix with independent \( N(0, 1) \) diagonal elements and \( N(0, 1/2) \) off-diagonal elements. Let \( \beta = (0.5, 0.5, 0, \ldots, 0)^\top \). We consider the following settings where \( Y \) is a symmetric positive-definite matrix.

    \textbf{Setting II-1}:
    \[\log(Y) \sim N_{mm}(\log\{D(X)\},0.2^2) \]
    where $D(X)=\exp\begin{pmatrix}
        1 & \zeta(X) \\ \zeta(X) & 1
    \end{pmatrix}$ and $\zeta(X)= \sin\Big(4\pi\beta^\top X \cdot (2\beta^\top X-1)\Big)$.
    % $\rho(X)=\left(x_1^2+x_2^2\right)^{\frac{1}{2}} 
    % \log\left(x_1^2+x_2^2\right)^{\frac{1}{2}}$.

    \textbf{Setting II-2}:
    \[\log(Y) \sim N_{mm}(\log\{D(X)\},0.2^2) \]
    where $D(X)=\exp\begin{pmatrix}
        1 & \zeta_1(X) & \zeta_2(X) \\ 
        \zeta_1(X) & 1 & \zeta_1(X) \\
        \zeta_2(X) & \zeta_1(X) & 1  
    \end{pmatrix}$ and $\zeta_1(X) = x_1/\left(1+|x_2|^\frac{1}{2}\right)$, 
    $\zeta_2(X) = \sin\left(x_3^2\right) + \exp(x_4^2)$.
    The two settings correspond to cases where the structural dimension is $d=1$ and $d=2$, respectively.

    The results in Tables \ref{SPD_1} and \ref{SPD_2} show that our method outperforms WIRE in terms of the accuracy of estimated sufficient predictors for SPD responses across different metrics, demonstrating its robustness.
    % Additionally, the computational complexity comparison in the supplementary material shows that our method is significantly faster than WIRE.
    \begin{table}[htbp]
        \centering
        \renewcommand{\arraystretch}{0.6} % Increase the row spacing
        \caption{Distance correlations for models with SPD responses and Log-Cholesky metric.}
        \label{SPD_1}
        \scalebox{0.9}{
        \begin{tabular}{@{\hspace{25pt}}c@{\hspace{25pt}}c@{\hspace{10pt}}c@{\hspace{25pt}}c@{\hspace{25pt}}c@{\hspace{25pt}}}
            \toprule
            Models & $(p,n)$ &  & FCCov-Net & WIRE \\
            \midrule
            & (10, 1000) &  & \textbf{0.832(0.023)} & 0.645(0.002)  \\
            & (10, 2000) &  & \textbf{0.905(0.026)} & 0.658(0.001) \\
            \textbf{Setting II-1}& (10, 5000) &  & \textbf{0.979(0.000)} & 0.673(0.000) \\
            & (30, 1000) &  & \textbf{0.626(0.022)} & 0.572(0.001)  \\
            & (30, 2000) &  & \textbf{0.759(0.017)} & 0.604(0.001) \\
            & (30, 5000) &  & \textbf{0.835(0.072)} & 0.643(0.000) \\
            \\
            & (10, 1000) &  & \textbf{0.897(0.000)} & 0.893(0.000)  \\
            & (10, 2000) &  & \textbf{0.897(0.000)} & 0.890(0.000) \\
            \textbf{Setting II-2}& (10, 5000) &  & \textbf{0.913(0.000)} & 0.893(0.000) \\
            & (30, 1000) &  & 0.842(0.025) & \textbf{0.874(0.000)}  \\
            & (30, 2000) &  & \textbf{0.881(0.000)} & 0.876(0.000) \\
            & (30, 5000) &  & \textbf{0.891(0.000)} & 0.876(0.000) \\
            \bottomrule
          \end{tabular}
        }
    \end{table}

    \begin{table}[htbp]
        \centering
        \renewcommand{\arraystretch}{0.6} % Increase the row spacing
        \caption{Distance correlations for models with SPD responses and affine invariant metric.}
        \label{SPD_2}
        \scalebox{0.9}{
        \begin{tabular}{@{\hspace{25pt}}c@{\hspace{25pt}}c@{\hspace{10pt}}c@{\hspace{25pt}}c@{\hspace{25pt}}c@{\hspace{25pt}}}
            \toprule
            Models & $(p,n)$ &  & FCCov-Net & WIRE \\
            \midrule
            & (10, 1000) & & \textbf{0.804(0.026)} & 0.645(0.001) \\
            & (10, 2000) &  & \textbf{0.948(0.001)} & 0.656(0.001)  \\
            \textbf{Setting II-1}& (10, 5000) &  & \textbf{0.982(0.000)} & 0.666(0.000)  \\
            & (30, 1000) &  & \textbf{0.644(0.017)} & 0.622(0.002)  \\
            & (30, 2000) &  & \textbf{0.809(0.001)} & 0.643(0.001)  \\
            & (30, 5000) &  & \textbf{0.949(0.000)} & 0.659(0.000)  \\
            \\
            & (10, 1000) &  & \textbf{0.929(0.000)} & 0.895(0.000) \\
            & (10, 2000) &  & \textbf{0.934(0.000)} & 0.894(0.000) \\
            \textbf{Setting II-2}& (10, 5000) &  & \textbf{0.938(0.000)} & 0.894(0.000)  \\
            & (30, 1000) &  & \textbf{0.896(0.000)} & 0.871(0.000) \\
            & (30, 2000) &  & \textbf{0.930(0.000)} & 0.894(0.000)  \\
            & (30, 5000) &  & \textbf{0.925(0.000)} & 0.878(0.000)  \\
            \bottomrule
          \end{tabular}
        }
    \end{table}

    \section{Real Data}\label{sec:real_data}

    %To evaluate the effectiveness of our proposed FCCov-Net method in practical applications, we applied our approach to a real-world expression recognition task using the Japanese Female Facial Expression (JAFFE) database \citep{lyons1998coding} and an image classification task using the MNIST dataset \citep{lecun2010mnist}. Due to space constraints, the concrete 
    %network architecture and analysis of MNIST are provided in the supplementary material.

    % Facial Expression Recognition (FER) \cite{li2005facial} involves recognizing and classifying various emotional states, such as happiness, sadness, anger, surprise, disgust, fear, and neutrality, by analyzing facial features and muscle movements. According to Plutchik's wheel of emotions theory \cite{plutchik1980general}, there is a limited number of basic emotions, and all other emotions emerge as combinations, mixtures, or compounds of these fundamental emotions. Therefore, it is reasonable to associate each facial expression with an emotion distribution rather than a specific expression class label.

    % Therefore, the single-label learning methods, which recognize only 
    % one basic emotion per expression, may not be suitable for real-life 
    % facial expression recognition applications where an expression rarely 
    % conveys just one basic emotion.

    %The JAFFE (Japanese Female Facial Expression) dataset comprises 213 images depicting various facial expressions from 10 distinct Japanese female subjects. Each image has a resolution of $256 \times 256$ pixels. Each subject was instructed to perform seven facial expressions, including six basic emotions and a neutral expression. These images were then annotated with average semantic ratings for each facial expression by 60 annotators. The ratings range from 1 to 5, where 5 signifies the highest intensity of emotion and 1 signifies the lowest.
    %Subsequently, the average scores for each emotion were calculated for each expression image, enabling the determination of the emotional distribution for each image. 
    
    % Figure \ref{YM_anger} illustrates a typical expression from the JAFFE database, along with its corresponding scores and emotion distribution.
    % \begin{figure}[htbp]
    %     \centering
    %     \includegraphics[width=\textwidth]{YM_anger.png}
    %     \caption{Typical expressions from the JAFFE database together 
    %     with scores and emotion distributions.}
    %     \label{YM_anger}
    % \end{figure}

    
    % The JAFFE dataset
    % contains 213 grayscale images of 7 facial expressions 
    % (neutral, happiness, sadness, surprise, anger, disgust, and fear) posed by 10 Japanese female models. 
    % Each image has a resolution of $256\times 256$ pixels. 
    % The images have been rated based on six emotional adjectives by 
    % 60 Japanese participants. The ratings range from 1 to 5, with 
    % 5 representing the highest emotion intensity and 1 representing 
    % the lowest emotion intensity. 
    % Then the average scores for each emotion on each expression image were obtained. From these scores, the emotional distribution for each image can be determined.
    % Fig \ref{YM_anger} is a typical expressions from the JAFFE database together with scores and emotion distributions.
    

    %The images are manually cropped to ensure that the eyes are aligned. These images are subsequently resized to $100 \times 100$ pixels to serve as the predictor variable, $X$. The dataset includes two types of labels:
    To evaluate the effectiveness of our proposed FCCov-Net method in practical applications, we applied our method to the JAFFE dataset. We randomly divided the original dataset into a training set comprising 170 samples and a test set containing 43 samples.
   
    
    %\begin{itemize}
    %    \item Emotion Distribution Labels: These are six-dimensional vectors representing probabilistic distributions of six emotions: happiness, sadness, surprise, anger, disgust, and fear. This distribution serves as the response in SDR and Fr\'echet regression tasks.
     %   \item Expression Category Labels: These labels indicate the specific expression category to which each image belongs. They are utilized as label data in classification tasks.
    %\end{itemize}
    
    % The images are cropped manually so that the eyes
    % are at the same positions, and then the cropped images are
    % resized to $100\times 100$ pixels as the predictor $X$.
    % The dataset contains two types of labels. 
    % The first type is a 6-dimensional emotion distribution label, 
    % representing the probabilities that the facial expression in the 
    % image conveys each of six emotions (happiness, sadness, surprise, 
    % anger, disgust, and fear). This distribution is used as the response 
    % $Y$ in our method. The second type of label indicates the specific 
    % expression category to which the image belongs, and this 
    % can be used as the label data for subsequent classification tasks.
    % 该数据有两种标签,一种是6-dimensional emotion distribution标签表示该图片表情
    % 蕴含6种感情(happiness, sadness, surprise, anger, disgust, and fear)的概率,
    % 此标签作为我们方法的responses Y,另一种是该图片属于哪种表情,此标签可以作为
    % 我们后续分类任务的标签数据。

     

    \begin{figure}[htbp]
        \centering
        \includegraphics[width=0.68\textwidth]{2d_feature_1.png}
        \caption{Scatter plots of the first two sufficient predictors estimated by FCCov-Net.}
        \label{2d_feature}
    \end{figure}
    Figure \ref{2d_feature} presents the scatter plots of the first two sufficient
    predictors estimated by FCCov-Net with Hellinger distance based on the training data.
The scatter plot depicts a vertical gradient of facial expressions, ranging from positive at the bottom, neutral in the middle, to negative at the top.
    Below the neutral points, happiness images are located on the left and surprise ones on the right. Above the neutral points, anger images appear on the left and fear ones on the right. Disgust images are clustered at the top-center, while sadness ones are positioned at the bottom-center. We select some representative points and show their facial images.

    
    Subsequently, we compared FCCov-Net and WIRE using Hellinger distance (HD) and Total Variation (TV) distance as metrics for responses being probability vectors.  Other than the emotion distribution labels, this dataset also have expression category labels. These labels indicate the specific expression category to which each image belongs. They are utilized as label data in classification tasks.
    We employ $\hat{\mf}(X)$ to perform Fr\'echet regression with distributional responses and classification with categorical labels, where $\hat{\mf} \in \mathbb{R}^d$ represents nonlinear SDR based on the training set. % And we compare their performance on the test set. For classification tasks, we employ three classifiers logitstic Regression (LR), random forests (RF) and support vector machines (SVM).
    \begin{table}[htbp]
        \centering
        \caption{The percentage of correct classifications for the testing group based on $\hat{f}(\cdot)$}
        \label{tab:jaffe_acc}
        \begin{tabular}{@{}lccccccccccc@{}}
        \toprule
        & \multicolumn{3}{c}{\( d = 1 \)} & \multicolumn{3}{c}{\( d = 2 \)} & \multicolumn{3}{c}{\( d = 3 \)} \\
        \cmidrule(r){2-4} \cmidrule(lr){5-7} \cmidrule(lr){8-10}
        Algorithm & LR & RF & SVM & LR & RF & SVM & LR & RF & SVM \\
        \midrule
        FCCovNet-HD & 0.35 & 0.46 & 0.51 & 0.53 & 0.58 & 0.67 & 0.72 & 0.77 & 0.81  \\
        WIRE-HD & 0.27 & 0.35 & 0.10 & 0.29 & 0.58 & 0.14 & 0.53 & 0.63 & 0.28  \\
        FCCovNet-TV & 0.4 & 0.38 & 0.44 & 0.52 & 0.54 & 0.58 & 0.70 & 0.73 & 0.73 \\
        WIRE-TV & 0.18 & 0.34 & 0.12 & 0.37 & 0.58 & 0.17 & 0.45 & 0.70 & 0.20 \\
        \bottomrule
        \end{tabular}
    \end{table}

    For classification, we employ three algorithms: logistic regression (LR), random forests (RF), and support vector machines (SVM). We then calculate the accuracy of these algorithms on the test set, with the results presented in Table \ref{tab:jaffe_acc}. As illustrated in the table, FCCov-Net demonstrates higher classification accuracy than WIRE across different $d$. 
    
    In addition to expression category labels, emotion distributions are crucial for reflecting the effects of nonlinear SDR methods. Consequently, leveraging local Fr\'echet regression with Hellinger distance as described in \citet{petersen2019frechet}, we utilize $d$-dimensional features to predict the corresponding label distributions. Various measures of similarity between predicted and true label distributions are summarized in Table \ref{tab:measures}. The results for Fr\'echet regression based on $\hat\mf(X)$, using different distribution distance metrics for test data are presented in Table \ref{tab:LDL}. 
    Both Table \ref{tab:jaffe_acc} and \ref{tab:LDL} demonstrate that our method obtains more accurate sufficient predictors while preserving sufficient information and remains robust across different distance metrics.
    
   From the results of Table \ref{tab:LDL}, we see that the test error with $d=3$ is the smallest, which indicates that we can select $d$ as $3$. In addition, the two-dimensional scatter plots exhibit overlapping expressions. In contrast, three-dimensional scatter plots viewed from the top as shown in Figure \ref{fig:3d_feature} clearly distinguish different emotion categories.

    \begin{figure}[ht]
        \centering
        \subfloat[Front view]{%
            \includegraphics[width=0.4\textwidth]{3d_angle1.png}
            \label{fig:sub1}
        }
        \subfloat[Top view]{%
            \includegraphics[width=0.35\textwidth]{3d_angle3.png}
            \label{fig:sub3}
        }
        \subfloat{%
            \raisebox{1cm}{\includegraphics[width=0.2\textwidth]{colorbar.png}}
            \label{fig:colorbar}
        }
        \caption{Scatter plots of the first three sufficient predictors estimated by FCCov-Net.}
        \label{fig:3d_feature}
    \end{figure}

    \begin{table}[ht]
        \centering
        \caption{Distance metrics for Fr\'echet regression with responses beging probability vectors.}
        \label{tab:measures}
        \begin{tabular}{|c|c|}
            \hline Name & Formula \\
            \hline \hline Kullback-Leibler(K-L) &$\operatorname{Dis}_1=\sum_{j=1}^C P_j \ln \frac{P_j}{Q_j}$ \\
            Euclidean & $\operatorname{Dis}_2=\sqrt{\sum_{j=1}^C\left(P_j-Q_j\right)^2}$ \\
            Sørensen & $\operatorname{Dis}_3=\frac{\sum_{j=1}^C\left|P_j-Q_j\right|}{\sum_{j=1}^C\left(P_j+Q_j\right)}$ \\
            Squared $X^2$ & $\operatorname{Dis}_4=\sum_{j=1}^C \frac{\left(P_j-Q_j\right)^2}{P_j+Q_j}$ \\
            \hline
        \end{tabular}
    \end{table}

    % 分析图表
    \begin{table}[htbp]
        \centering
        \caption{Distances between the predicted distributions 
        and the true distributions.}
        \label{tab:LDL}
        \resizebox{\columnwidth}{!}{
        \begin{tabular}{@{}lcccccccccccc@{}}
        \toprule
        & \multicolumn{4}{c}{\( d = 1 \)} & \multicolumn{4}{c}{\( d = 2 \)} & \multicolumn{4}{c}{\( d = 3 \)} \\
        \cmidrule(r){2-5} \cmidrule(lr){6-9} \cmidrule(l){10-13}
        Algorithm & $\operatorname{Dis}_1$ & $\operatorname{Dis}_2$ & $\operatorname{Dis}_3$ & $\operatorname{Dis}_4$
        & $\operatorname{Dis}_1$ & $\operatorname{Dis}_2$ & $\operatorname{Dis}_3$ & $\operatorname{Dis}_4$
        & $\operatorname{Dis}_1$ & $\operatorname{Dis}_2$ & $\operatorname{Dis}_3$ & $\operatorname{Dis}_4$ \\
        \midrule
        FCCovNet-HD & 0.0968 & 0.0321 & 0.0918 & 1.3429 & 0.0779 & 0.0226 & 0.0761 & 0.9542 & 0.0574 & 0.0113 & 0.0571 & 0.4880 \\
        WIRE-HD & 0.1082 & 0.0347 & 0.1037 & 1.4781 & 0.0860 & 0.0233 & 0.0851 & 0.9976 & 0.0718 & 0.0162 & 0.0690 & 0.6999 \\
        FCCovNet-TV & 0.1035 & 0.0367 & 0.0978 & 1.5513 & 0.0868 & 0.0304 & 0.0838 & 1.2437 & 0.0644 & 0.0160 & 0.0635 & 0.6782 \\
        WIRE-TV & 0.1119 & 0.0369 & 0.1077 & 1.5855 & 0.0913 & 0.0269 & 0.0897 & 1.1415 & 0.0871 & 0.0233 & 0.0865 & 1.0064 \\
        \bottomrule
        \end{tabular}
        }
    \end{table}
    
\section{Conclusion}\label{sec:summary}
    %This paper considers Fr\'echet nonlinear SDR  with non-Euclidean responses. %Modeling $ Y $ as Euclidean can distort its intrinsic geometry, limiting the effectiveness of existing methods. To address this, 

    In this paper, we propose a novel method for nonlinear SDR
    with general metric space valued responses based on a new measure 
    to characterize statistical dependence. We conduct a systematic analysis of the theoretical properties of our proposal, including ubiasedness in the population level, feasibility from the computational perspective, and nonasymptotic property of the estimator based on neural networks. 
    
   In light of the significant advancements in the research on measuring statistical dependence in recent years  (e.g., \citet{JMLR:v23:20-682, liu2022model, tong2023model}), the integration of effective dependence measures into our framework also warrants further investigation. Moreover, adopting advanced neural architectures like transformers could further enhance the performance of nonlinear SDR for complex dependent data structures. We leave this for future research.

    
    %We introduce FCCov, a novel dependence measure that is suitable  , and reformulate nonlinear SDR estimation as a constrained optimization problem. We relax the constraints using Frobenius regularization, converting it into an unconstrained task. For estimation, we provide an unbiased U-statistic for FCCov; however, due to its high computational complexity of $\mathcal{O}(n^4)$, we develop a faster algorithm with $\mathcal{O}(n^2 \log n)$ complexity to ensure scalability. Theoretically, we prove unbiasedness at the population level and derive non-asymptotic error rates nearly matching the minimax optimal rate of nonparametric regression. Simulations on Euclidean data show robust performance, while those on non-Euclidean data highlight our method's superiority over traditional kernel approaches.
    % In this paper, we study the problem of sufficient dimension reduction (SDR) in a setting where the responses \( Y \) are non-Euclidean variables and the predictors \( X \) are standard Euclidean variables.
    % Directly modeling \( Y \) as a Euclidean variable in such scenarios can disrupt the intrinsic geometry of the non-Euclidean space, rendering many state-of-the-art methods ineffective.
    % Moreover, even modern neural network architectures, which have gained widespread popularity, lack suitable loss functions tailored to this type of data. To tackle this challenge, we first propose a novel measure, termed FCCov, which quantifies the conditional mean dependence between \( Y \) and \( X \). Based on this measure, we reformulate the estimation of sufficient representative directions in nonlinear SDR as a constrained optimization problem. Subsequently, to facilitate computation, we relax the constraints by introducing a square Frobenius regularization term, transforming the original problem into an unconstrained optimization task.
    % For empirical estimation, we provide an unbiased U-statistic estimator for FCCov. However, due to its computational complexity of \(\mathcal{O}(n^4)\), we further develop a fast computational algorithm that reduces the complexity to \(\mathcal{O}(n^2 \log n)\). This advancement makes the proposed methodology scalable to larger datasets while preserving the theoretical rigor required for accurate estimation in non-Euclidean response settings.
    % Theoretically, at the population level, we establish the unbiasedness of our nonlinear SDR method. At the sample level, we derive non-asymptotic error rates that align with the minimax optimal rate of nonparametric regression, up to a logarithmic factor of $n$. 
    % The simulation on Euclidean data demonstrates that our method not only performs well in most scenarios but also exhibits robustness.
    % Furthermore, the simulation on non-Euclidean data underscores the superiority of our approach compared to traditional kernel methods.

   % Future research should explore several key directions. First, adopting advanced neural architectures like transformers could extend applications to sequential data. Second, integrating sophisticated dependence measures into the SDR framework is crucial, particularly given recent advancements in statistical independence research (e.g., \citet{JMLR:v23:20-682, liu2022model, Tong02102023}). Further investigation into effective dependence measures remains a priority.  
    % In future research, several promising directions can be explored. First, we can consider leveraging more advanced neural network architectures, such as transformer-based models, as the estimation function. This would allow us to generalize our application scenarios to sequential data types. Second, we can explore the integration of more sophisticated dependence measures into our SDR framework. 
    % Measuring statistical independence is a critical research direction in statistics. In recent years, numerous significant advancements have been made in this area, including works by \citet{JMLR:v23:20-682, liu2022model, Tong02102023}.
    % The integration of effective dependence measures into our framework also warrants further investigation.
    


\bibliographystyle{apalike}
% \bibliographystyle{Chicago}

\bibliography{jasaref}

\end{document}
