\section{INTRODUCTION}
\label{section: introduction}
A fundamental ability of autonomous robots is to navigate towards a goal while avoiding collisions along the way~\cite{Xiao2020}. 
However, in complex and cluttered environments, such as domestic settings where obstacles like chairs and boxes may obstruct the path to the goal, finding collision-free paths often becomes impractical.
In such cases, traditional navigation methods often fail and Navigation Amongst Movable Obstacles~(NAMO) becomes essential. 


NAMO involves actively interacting with the environment by relocating obstacles to clear a path, enabling successful navigation to the target. 
The concept was initially explored by \citeauthor{Reif1985} \cite{Reif1985}, and later formally introduced as a category of research by \citeauthor{Stilman2004} \cite{Stilman2004}. These works demonstrated that solving NAMO in a two-dimensional workspace with movable obstacles is NP-hard, and NP-complete in simplified cases with unit-square obstacles, due to the large search space and the numerous possible configurations \cite{Reif1985, Demaine2000}. As a result, most solutions are approximations rather than exact solutions, and practical applications remain limited~\cite{Ellis2022}. Research often focuses on simplified versions of the problem, such as $LP_1$, a linear program where a single obstacle blocks the path, or $LP_2$, which involves two obstacles \cite{Stilman2004, Renault2020}.

\begin{figure}[tb!]
    \includegraphics[width=0.49\textwidth]{figures/3-method/img_simplified_svg_mppi_overview.jpg}
    \caption{An overview of the proposed SVG-MPPI architecture where the SVG provides a weighted graph with efficient node placement around movable obstacles along which a lowest-effort path can be found. The generated set of waypoints guides the MPPI control strategy to efficiently sample rollouts around movable obstacles. If during interaction an obstacle is considered non-movable, the movability estimation gets updated and the path is replanned. Snapshots of a real-world example are shown on the left where the red star indicates the goal location and the masses of the obstacles are (A): 25 kg, (B): 20 kg, (C): 5 kg.}
    \label{fig:overview_of_svg_mppi}
    \vspace{-4.5mm}
\end{figure}

Existing NAMO approaches \cite{Demaine2000, Stilman2004, Stilman2010, Nieuwenhuisen2006, Mueggler2014, Castaman2016, Moghaddam2016, Meng2018, Ellis2022} and sequential nonprehensile manipulation strategies~\cite{Muguira2023Visibility, vieira2022persistent} consider only binary movability information, i.e., whether obstacles are movable or not, neglecting the varying levels of effort required to move different obstacles and thus lacking efficiency.
Furthermore, these works rely on task planners to decide when the robot should move without displacing an obstacle (transit) or when it should displace an obstacle (transfer) and where to place the obstacle. 
However, obstacle placement planning itself is a difficult problem and these works typically function only in simplified scenarios, such as the $LP_1$ and $LP_2$ scenarios. How to realize efficient NAMO in more cluttered and random environments remains an open problem.

This work incorporates continuous movability information into the NAMO-solving process, improving efficient navigation around obstacles with contact force minimization while moving to the goal. 
Moreover, inspired by the framework in~\cite{Ellis2022}, we use a physics engine to model state transitions of the obstacles in the environment. We introduce a framework that can rapidly simulate the interaction outcomes of each rollout first before selecting the best one, without the need for explicit obstacle placement planning, in contrast to~\cite{Ellis2022}. These rollouts are planned using an SVG-MPPI approach, which tackles global planning with movability-aware node placement and local planning with the consideration of contact force minimization using a Model Predictive Path Integral (MPPI) control strategy.
Fig. \ref{fig:overview_of_svg_mppi} shows the framework of our system and snapshots of a real-world experiment.

The main contributions of this work are:
\begin{itemize}
    \item \textit{Integration of Continuous Quantified Movability:} We consider obstacle movability on a continuous scale rather than as a binary property, leading to more efficient planning in cluttered environments.

    \item \textit{Contact-Force Minimization in MPPI control:} Our approach minimizes contact forces between the robot and the environment using the movability data and a physics engine, reducing the effort required to move objects.

    \item \textit{Elimination of Explicit Obstacle Placement:} We eliminate the requirement for explicit obstacle relocation, enabling applicability in more cluttered random environmental setups.
\end{itemize}
