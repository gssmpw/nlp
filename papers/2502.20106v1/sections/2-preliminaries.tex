\section{PRELIMINARIES}
\label{chapter:preliminaries}

This section presents the key concepts underlying our approach. We begin by defining the necessary notations, followed by an overview of the Visibility Graph (VG) method which provides the basis for the SVG. Lastly, we describe the MPPI control strategy, which is integrated with a physics engine to model obstacle interactions.

\subsection{Notation}
\label{sec: notation}

In this paper, a graph is represented as \( G = (V, E) \), where \( V = \{v_1, v_2, \dots, v_{n}\} \) is the set of nodes with $n$ the number of nodes, and \( E \) the edges. Each obstacle is described as a polygon \( O_i \), with mass \( m_i \). The distance between a node \( v_i \) and an obstacle is denoted as \( d_i \), and the robot's safety margin is given by \( r \). \textit{Free-space nodes}, \( V_{\text{free}} \), originate from the visibility graph and do not require manipulation, while \textit{passage nodes}, \( V_{\text{passage}} \), involve obstacle manipulation. 
The robot’s state is denoted \( \mathbf{x}_t \), its velocity \( \mathbf{\dot{x}}_t \), and control inputs are given by \( \mathbf{v}_t \).
A set of waypoints is represented by \( \mathcal{P} = \{\mathbf{p_1}, \mathbf{p_2}, ..., \mathbf{p}_{n_{p}} \} \) with $n_{p}$ the number of positions $\mathbf{p}$.

\subsection{Visibility Graphs (VG)} \label{sec: VG}

VG is a key tool in path planning, representing direct line-of-sight connections between significant points in an environment \cite{Berg2000}. These connections, or edges, create a network that allows algorithms such as Dijkstra’s \cite{Dijkstra1959} or A* \cite{Nilsson1968} to compute optimal paths between start and goal nodes. VG construction involves identifying key points, typically at obstacle corners, and connecting them with edges wherever visibility is unobstructed. To ensure safe navigation, obstacle boundaries are inflated by a margin \( r \), resulting in a graph \( G = (V_{\text{free}}, E) \) that facilitates efficient path planning, particularly when obstacles are static. Fig.~\ref{fig:simple_svg_graph}a demonstrates a basic visibility graph applied to a straightforward environment, with inflated obstacles and points placed in the remaining free space. However, Fig.~\ref{fig:simple_svg_graph}a also highlights a limitation of VGs: when the goal is blocked by movable obstacles, certain locations are not mutually visible, creating gaps in the visibility graph. Consequently, VG alone becomes insufficient for generating a path to the goal in such scenarios.

\subsection{Model Predictive Path Integral (MPPI)} 
\label{sec: MPPI}

MPPI is a sampling-based control method used for navigating dynamic environments by generating multiple control input sequences and evaluating them using a cost function~\cite{Williams2017}. 
MPPI involves sampling a total of \( K \) input sequences \( V_k\) that generate the state trajectories \( Q_k, \ \forall k\in K\), often referred to as rollouts.
Each rollout \( Q_k \) represents a series of state vectors \( \mathbf{x}_{t,k} \), meaning that each rollout \( k \) consists of prediction steps \( t \) over a total horizon of \( T \). Using a cost function \( C_k \), each rollout is evaluated where rollouts that violate the constraints are disregarded. A weighted sum of all rollouts determines the next action for the robot. In the work by \citeauthor{Pezzato2023}, MPPI is integrated with IsaacGym \cite{Makoviychuck2021}, a physics engine that simulates the robot’s environment in parallel using GPU power, allowing for rapid computation of multiple trajectories and the elimination of an explicit model of obstacle interactions~\cite{Pezzato2023}.

\begin{figure}[tb!]
    \centering
    \begin{minipage}{0.145\textwidth}
        \centering
        \includegraphics[width=1.1\textwidth]{figures/3-method/img_vg_nodes.png}
        \caption*{\hspace{14mm} (a)}
    \end{minipage}\hfill
    \begin{minipage}{0.145\textwidth}
        \centering
        \includegraphics[width=1.1\textwidth]{figures/3-method/img_vg_and_passage_nodes.png}
        \caption*{\hspace{14mm} (b)}
    \end{minipage}\hfill
    \begin{minipage}{0.191\textwidth}
        \centering
        \includegraphics[width=1.1\textwidth]{figures/3-method/img_weighted_svg.png}
        \caption*{\hspace{14mm} (c)}
    \end{minipage}
    \caption{Comparison of different visibility graph strategies: (a) The normal Visibility Graph (VG), (b) A VG enhanced with Passage Nodes, and (c) A Semantic Visibility Graph (SVG) where node costs reflect obstacle manipulation effort.}
    \label{fig:simple_svg_graph}
    \vspace{-3mm}
\end{figure}
