\section{Experiments and Results}
\label{section:experiments_and_results}

This section outlines the experimental setup and results, validating the proposed SVG-MPPI algorithm in both simulated and real-world environments. The algorithm's performance is evaluated based on its ability to navigate toward obstructed goal positions, and we compare it against other established methods.

\subsection{Experimental Framework}
The experiments are conducted using a Dingo-O robot, a holonomic, velocity-controlled platform developed by Clearpath Robotics. In real-world trials, precise localization of the robot and obstacles is achieved through a Vicon motion capture system, with poses updated in Isaac-Gym at 25 Hz to ensure accurate MPPI rollouts. A maximum pushable mass of 30 kg is enforced, along with a 0.3-meter safety margin based on the robot’s width (517 mm) to prevent collisions. The waypoint interpolation interval is set to 0.5 meters, striking a balance between computational efficiency and path smoothness. Replanning is triggered if the conditions outlined in Section~\ref{sec: movability evaluation} persist for more than 30 seconds. MPPI control operates with a time step of 0.08 seconds and a prediction horizon of 25 steps, covering 2 seconds of movement. Further experimental details are available in our \href{https://github.com/tud-amr/SVG-MPPI}{GitHub repository}.

We evaluate the algorithm in a simulated 4x8 meter cluttered room, where 20\% of the space is occupied by movable obstacles and 5\% by stationary obstacles where obstacles are randomized with respect to their positions, dimensions and masses (see Fig.~\ref{fig:path_planning_strategies_per_planner}). This randomly structured environment challenges the robot to navigate tight spaces, where pushing is often required if avoidance is unfeasible. The randomness ensures robust testing by simulating real-world conditions, making it an effective benchmark for adaptability, efficiency, and real-time replanning capabilities.

\begin{figure}[tb!]
    \centering
    \begin{minipage}{0.49\textwidth}
        \includegraphics[width=\textwidth]{figures/4-experiments/img_planner_legend.png}
    \end{minipage}\hfill
    \begin{minipage}{0.1\textwidth}
        \centering
        \includegraphics[width=\textwidth]{figures/4-experiments/img_nvg_planner.png}
        \caption*{\hspace{7mm} (a)}
    \end{minipage}\hfill
    \begin{minipage}{0.1\textwidth}
        \centering
        \includegraphics[width=\textwidth]{figures/4-experiments/img_bvg_planner.png}
        \caption*{\hspace{7mm} (b)}
    \end{minipage}\hfill
    \begin{minipage}{0.1\textwidth}
        \centering
        \includegraphics[width=\textwidth]{figures/4-experiments/img_rrt_planner.png}
        \caption*{\hspace{7mm} (c)}
    \end{minipage}\hfill
    \begin{minipage}{0.1\textwidth}
        \centering
        \includegraphics[width=\textwidth]{figures/4-experiments/img_svg_planner.png}
        \caption*{\hspace{7mm} (d)}
    \end{minipage}\hfill
    \begin{minipage}{0.051\textwidth}
        \raisebox{5mm}{\includegraphics[width=\textwidth]{figures/4-experiments/img_mass_bar.png}}
    \end{minipage}
    \caption{Comparison of the different path planning strategies: (a) NVG, (b) BVG, (c) B-RRT, and (d) SVG. Each figure shows the planned path and the robot’s trajectory using MPPI as the local motion planner. For NVG, no path could be found as all obstacles are considered non-movable.}
    \label{fig:path_planning_strategies_per_planner}
    \vspace{-4mm}
\end{figure}

\subsection{Benchmark Algorithms} \label{sec: benchmarks}
To the best of our knowledge, no existing algorithm directly mirrors the SVG-MPPI approach. Therefore, we developed three comparative methods to benchmark our system, focusing on how obstacle movability is interpreted in both global path planning and local control strategies.

The first method, NVG (Normal Visibility Graph), treats all obstacles as stationary. The second, BVG (Binary Visibility Graph), uses a binary classification, distinguishing between movable and immovable obstacles. The third method, B-RRT (Binary Rapidly-exploring Random Tree), also applies a binary classification of movability but uses an RRT-based path planning algorithm. In this method, sampling within the safety margin of obstacles is permitted if their mass is below 30 kg, though no additional node costs are applied. The RRT approach follows the traditional structure as outlined in \cite{Lavalle1998}. Fig.~\ref{fig:path_planning_strategies_per_planner}a-d illustrates how each planner interprets movability and demonstrates a sample scenario where a path is calculated from the starting position to the goal.

In addition to comparing global path planning methods, we assess the impact of different movability interpretations on local control strategies by pairing each path planning method with MPPI. NVG-MPPI does not account for movability and treats all obstacles as stationary, while BVG-MPPI and B-RRT-MPPI apply a binary movability distinction in both the path and trajectory planner. In contrast, SVG-MPPI employs a continuous approach to movability. This comparison allows us to evaluate how different interpretations of movability affect both path planning and control performance.

\subsection{Performance Metrics}
Performance metrics were collected over 50 simulation runs, with obstacle masses randomly assigned between 4 and 36 kg. The path planners, NVG, BVG, B-RRT and SVG, are compared on their \textit{Path Planning Success} indicating the percentage of paths found from start to goal and \textit{Planner time} expressing the average computation time to generate the path. All planners are combined with MPPI as local motion planning method and analyzed on their \textit{Execution Success} providing the percentage of scenarios where the simulated robot reaches the goal and its corresponding average \textit{Execution time}. In addition, we include the cumulative force averaged over the 50 randomly generated scenarios to analyze the contact forces between the robot and obstacles over all successful trials. 

\begin{table*}[ht]
    \caption{\small{Experimental results of the compared methods for 50 scenarios with objects of randomized size, placement, and mass.}}
    \label{table:combined}
    
    \renewcommand{\arraystretch}{1.2}
    \resizebox{\textwidth}{!}{
    % The entire first column will have light gray background
    \begin{tabular}{!{\color{lightgray}\vrule width 2pt}>{\columncolor{lightgray}}c c c}
        \toprule
        \textbf{Path} & \textbf{Path Planning} & \textbf{Planner Time (s)} \\
        \textbf{Planner} & \textbf{Success (\%)}& \textbf{(Mean $\pm$ SE)} \\
        \midrule
        NVG  & 16.4 & 0.169 $\pm$ 0.014    \\ 
        BVG  & 98.2  & 0.195 $\pm$ 0.006  \\ 
        B-RRT & 92.7 & 1.120 $\pm$ 0.483  \\ 
        \textbf{SVG}  & \textbf{98.2} & \textbf{0.196 $\pm$ 0.006} \\ 
        \bottomrule
    \end{tabular}
    
    \hspace{1mm}

    % The entire first column of the control strategy table will have light gray background
    \begin{tabular}{!{\color{lightgray}\vrule width 2pt}>{\columncolor{lightgray}}c c c c}
        \toprule
        \textbf{Control} & \textbf{Execution to Goal} & \textbf{Execution Time (s)} & \textbf{Cumulative Force (kN)} \\
        \textbf{Strategy} &  \textbf{Success (\%)} & \textbf{(Mean $\pm$ SE)} & \textbf{(Mean $\pm$ SE)} \\
        \midrule
         NVG-MPPI & 12.7  & 109 $\pm$ 18 & 39 $\pm$ 23 \\ 
         BVG-MPPI & 49.1  & 129 $\pm$ 13 & 165 $\pm$ 418 \\ 
         RRT-MPPI & 50.9 & 130 $\pm$ \ 9 & 175 $\pm$ 31   \\ 
         \textbf{SVG-MPPI} & \textbf{54.5}  & \textbf{120 $\pm$ 11} & \textbf{97 $\pm$ 17} \\ 
        \bottomrule
    \end{tabular}
    }
\vspace{-3mm}
\end{table*}
\normalsize


\subsection{Experimental analysis}
\label{sec: experimental analysis}

In Table~\ref{table:combined}, the performance metrics for SVG-MPPI are compared against benchmark planners as introduced in Sec.~\ref{sec: benchmarks} across 50 randomized scenarios. The proposed path planner SVG consistently outperforms NVG and B-RRT in path planning success rates with 98.2\% for SVG and 16.4\% and 92.7\% for NVG and B-RRT respectively.  
By integrating the continuous movability within the path planner SVG and the local planner MPPI, we obtain an improved execution success rate, 54.5\%, compared to a binary classification method, 49.1\% and 50.9\% for BVG and B-RRT respectively, with a mildly improved execution time. 
Most importantly, SVG-MPPI also demonstrates significantly lower cumulative force compared to B-RRT-MPPI and BVG-MPPI, emphasizing its efficiency in minimizing push actions during navigation. Its strategic node placement, effort-based path selection, and considered continuous movability in both the local and global planner, lead to smoother interactions with the environment, making it particularly effective in scenarios requiring multiple obstacle manipulations. 
Note that NVG-MPPI can only solve scenarios where a collision-free path can be constructed. The minimal cumulative force in Table~\ref{table:combined} results from unintended pushing due to suboptimal turns and path-following.

An illustrative example can be observed in Fig.\ref{fig:path_planning_strategies_per_planner} expressing the paths and trajectories of all compared methods. By considering continuous movability, SVG-MMPI selects a path near the lighter obstacles that lead to a small displacement of the obstacles, e.g. a low push effort over the trajectory. B-RRT-MPPI, on the other hand, pushes one obstacle across the room, and both BVG-MPPI and B-RRT-MPPI push one obstacle against another obstacle resulting in additional push effort. NVG-MPPI is unable to generate a path in this environment as all obstacles are considered non-movable. SVG-MPPI applied replanning three times without success, but in two cases, it enabled the robot to reach the goal. Other methods did not include replanning.
Moreover, SVG-MPPI’s planner time remains highly efficient, 0.196 $\pm$ 0.006, supporting real-time replanning in dynamic environments. This combination of efficiency and adaptability highlights SVG-MPPI’s robustness, particularly in environments with high obstacle density and frequent replanning needs.

\begin{figure}[tb!]
    \centering
    \begin{minipage}{0.42\textwidth}
        \centering
        \includegraphics[width=\textwidth]{figures/4-experiments/img_replanning_legend.png}
    \end{minipage}\hfill
    \begin{minipage}{0.23\textwidth}
        \centering
        \includegraphics[width=\textwidth]{figures/4-experiments/img_without_replanning.png}
        \caption*{\hspace{22mm} (a)}
    \end{minipage}\hfill
    \begin{minipage}{0.23\textwidth}
        \centering
        \includegraphics[width=\textwidth]{figures/4-experiments/img_with_replanning.png}
        \caption*{\hspace{22mm} (b)}
    \end{minipage}
    \caption{Comparison between (a) the trajectory without replanning, where the robot moves a single obstacle to create a path, and (b) the trajectory with replanning, triggered when the estimated movability is updated and the robot reroutes through another path.}
    \label{fig:movability_evaluation}
    \vspace{-3mm}
\end{figure}

\subsubsection{Online Replanning} \label{sec: replanning}
Fig.~\ref{fig:movability_evaluation} demonstrates two scenarios where in scenario (a) the obstacle mass is as expected and in scenario (b) the obstacle is observed to be non-movable when encountered. 
In scenario (a), the robot moves a single obstacle to create a pathway along the upper side of the room. In scenario (b), the robot is unable to move the obstacle which triggers the replanning process as described in Sec.~\ref{sec: movability evaluation}. 
This allowed the robot to adapt by rerouting through the lower side of the room, demonstrating the ability of SVG-MPPI to update movability online and replan accordingly.

\subsubsection{Real-World Experiments}
In the real-world test, the Dingo-O robot navigates a hallway with movable and immovable obstacles to confirm the practical effectiveness of the SVG-MPPI algorithm. The robot successfully created a passage by relocating obstacles, reaching the goal despite the physical constraints of the environment, as shown in Fig.~\ref{fig:overview_of_svg_mppi} and the accompanying video. During the real-world tests, the robot encounters three movable obstacles where the lightest obstacle, i.e. obstacle C, is moved forward and the robot navigates in the created free space between obstacles B and C. During the real-world experiments, obstacle B is slightly pushed due to model mismatch and additional disturbances in the real-world tests compared to the simulated scenario. 

