%%%%%%%%%%%%%%%%%%%%%%%%%%%%%%%%%%%%%%%%%%%%%%%%%%%%%%%%%%%%%%%%%%%%%%%%%%%%%%%%
%2345678901234567890123456789012345678901234567890123456789012345678901234567890
%        1         2         3         4         5         6         7         8

\documentclass[letterpaper, 10 pt, conference]{ieeeconf}  % Comment this line out if you need a4paper

%\documentclass[a4paper, 10pt, conference]{ieeeconf}      % Use this line for a4 paper

\IEEEoverridecommandlockouts                              % This command is only needed if 
                                                          % you want to use the \thanks command

\overrideIEEEmargins                                      % Needed to meet printer requirements.

%In case you encounter the following error:
%Error 1010 The PDF file may be corrupt (unable to open PDF file) OR
%Error 1000 An error occurred while parsing a contents stream. Unable to analyze the PDF file.
%This is a known problem with pdfLaTeX conversion filter. The file cannot be opened with acrobat reader
%Please use one of the alternatives below to circumvent this error by uncommenting one or the other
%\pdfobjcompresslevel=0
%\pdfminorversion=4

% See the \addtolength command later in the file to balance the column lengths
% on the last page of the document

\RequirePackage{mathtools}  % Mathematical tools to use with amsmath
\RequirePackage{amssymb}    % Extended symbol collection
\RequirePackage{siunitx}    % Comprehensive (SI) units package

\RequirePackage{tabularx}   % Tabulars with adjustable-width columns
\RequirePackage{booktabs}   % Publication quality tables
\RequirePackage{longtable}  % Allow tables to flow over page boundaries
\RequirePackage{multirow}   % Create tabular cells spanning multiple rows
\RequirePackage{colortbl}
\RequirePackage{xcolor}

\RequirePackage{graphicx}   % Enhanced support for images
\RequirePackage{float}      % Improved interface for floating objects
\RequirePackage[labelfont=bf,justification=centering, footnotesize]{caption} % Captions
\RequirePackage{subcaption}         % Support for sub-captions
\RequirePackage{algpseudocode}      % Include algorithm pseudocode
\RequirePackage{algorithm}          % Include algorithm
\usepackage{hyperref}
\captionsetup{justification   = justified,
              singlelinecheck = false}

\usepackage{soul,xcolor}
\usepackage[normalem]{ulem}
\usepackage[backend=biber, style=numeric-comp, sorting=none]{biblatex}
\addbibresource{references.bib}

\AtEveryBibitem{
  \clearfield{doi}
  \clearfield{url}
  \clearfield{issn}
}

% Define a custom command to remove end tags
\newcommand{\algorithmwithoutend}{%
  \algtext*{EndWhile}%
  \algtext*{EndFor}%
  \algtext*{EndLoop}%
  \algtext*{EndIf}%
  \algtext*{EndProcedure}%
  \algtext*{EndFunction}%
}

\definecolor{lightgray}{gray}{0.92}
\definecolor{lightgreen}{rgb}{0.4019607843137255, 0.7862745098039216, 0.3901960784313726}
\setstcolor{red}

\title{\LARGE \bf
Pushing Through Clutter With Movability Awareness of \\ Blocking Obstacles
}

\author{Joris J. Weeda, Saray Bakker, Gang Chen and Javier Alonso-Mora% <-this % stops a space
\thanks{This project has received funding from the European Union through ERC, INTERACT, under Grant 101041863. Views and opinions expressed are however those of the author(s) only and do not necessarily reflect those of the European Union. Neither the European Union nor the granting authority can be held responsible for them.} 
\thanks{The authors are with the Cognitive Robotics Department, TU Delft,
2628 CD Delft, The Netherlands 
        {\tt\small \{jjweeda, s.bakker-7, g.chen-5}\}@tudelft.nl}%
}


\begin{document}

\maketitle
\thispagestyle{empty}
\pagestyle{empty}

%%%%%%%%%%%%%%%%%%%%%%%%%%%%%%%%%%%%%%%%%%%%%%%%%%%%%%%%%%%%%%%%%%%%%%%%%%%%%%%%
\begin{abstract}
Navigation Among Movable Obstacles (NAMO) poses a challenge for traditional path-planning methods when obstacles block the path, requiring push actions to reach the goal. 
We propose a framework that enables movability-aware planning to overcome this challenge without relying on explicit obstacle placement.
Our framework integrates a global Semantic Visibility Graph and a local Model Predictive Path Integral (SVG-MPPI) approach to efficiently sample rollouts, taking into account the continuous range of obstacle movability. A physics engine is adopted to simulate the interaction result of the rollouts with the environment, and generate trajectories that minimize contact force.
In qualitative and quantitative experiments, SVG-MPPI outperforms the existing paradigm that uses only binary movability for planning, achieving higher success rates with reduced cumulative contact forces.
Our code is available at: \href{https://github.com/tud-amr/SVG-MPPI}{https://github.com/tud-amr/SVG-MPPI}


\end{abstract}

\begin{keywords}
    Motion and Path Planning, 
    Collision Avoidance,
    Integrated Planning and Control
\end{keywords}

%%%%%%%%%%%%%%%%%%%%%%%%%%%%%%%%%%%%%%%%%%%%%%%%%%%%%%%%%%%%%%%%%%%%%%%%%%%%%%%%

\section{Introduction}

% State of the world (robots for creative activites)
The term ``robot,'' originally signifying `forced labor,' has long been associated with labor and work. Robots have demonstrated their utility in various automated productive and social contexts, where the primary goals are improving productivity, safety, and fostering social interactions with humans~\cite{simoes2022designing, weidemann2021role, honig2018understanding}. However, an increasing number of cases feature using of robots in creative settings. Unlike productive contexts, where the focus is on efficiency and task completion~\cite{arents2022smart}, or social contexts, where communication and trust are prioritized~\cite{nam2020trust, saunderson2019robots}, creative environments prioritize artistic innovation and expression~\cite{hsueh2024counts}. This shift fundamentally alters the dynamics of human-robot interaction, redefining the roles and expectations for both humans and robots.

For instance, robots’ social behaviors are leveraged to support the generation and expression of creative ideas~\cite{hu2021exploring, sandoval2022human, alves2020creativity}, and programmable robotic movements and trajectories are employed to inspire artistic activities such as sketching~\cite{lin2020your}. These studies often engage participants from creative fields who possess limited prior experience with robotics, and are typically conducted in short-term, experimental settings. Consequently, the findings from these studies remain constrained since much can be learned from professional practitioners' experiences to inform system design such as digital fabrication~\cite{hirsch2023nothing}. There is a notable gap in research examining the long-term, active, and practical experience of integrating robotic systems into the creative processes. As a result, the deeper insights into how robots facilitate and shape creative processes, beyond simply augmenting human creativity, remain underexplored. In this study, we aim to better understand the impacts of robots on creative processes and outcomes.

As early as Leonardo da Vinci's 16th century ``Automaton,'' artists have explored the creative affordances of robotic systems~\cite{shanken2002cybernetics, pagliarini2009development, jeon2017robotic}. The artistic creation process typically encompasses various stages, including the exploration of materials and techniques, ongoing experimentation and iteration, and the continual refinement of the artists' insights into their creative subjects~\cite{lewis2023art, sturdee2022state}. Therefore, investigating the artistic process involving robots offers an opportunity to gain deeper insights into robots' creative potential. Robotic art, in particular, provides a compelling case for this exploration.

We define robotic art as artworks that utilize robotic or automated machines to create artistic experiences and tangible artifacts. One example is robotic installation art, in which robots are programmed to follow specific rules that embody the artist’s expression (\autoref{fig:teaser} (a)). Another example is responsive art, in which robots react to their environment, with behaviors that change over time or in response to spectators (\autoref{fig:teaser} (b)). Additionally, there are robotic creators, which possess a degree of agency, allowing them to collaborate with human artists and produce works that extend beyond mere replication of human-created art (\autoref{fig:teaser} (c) and (d)). As such, robotic art becomes a rich case for exploring human-machine interactions in creative contexts. Gaining a deeper understanding of how robots facilitate artistic expression can provide insights for designing computing systems to support creative activities~\cite{gomez2021robot}.

% Therefore, we did...
We draw on semi-structured, in-depth interviews with renowned professional robotic artists to investigate the use of robots in artistic practice. Specifically, our goal is to understand how artistic exploration of robotic systems challenges conventional assumptions about the functions of robots, such as their roles in automating repetitive tasks or serving human needs. We also explore the implications of robots in the artistic process and examine how creativity may emerge within robotic art. To address these interrelated inquiries, our study focuses on the practice of robotic art, posing the research question: \textit{How do robotic artists utilize robots in their artistic practice?} We approach this inquiry through the perspectives and experiences of robotic artists, who creatively design, modify, and repurpose robotic systems for artistic expression and exploration.

% The key findings are...
Our findings highlight the social, material, and temporal dimensions of artists' practices that shape their creativity and artistic outcomes. The creation of robotic art is largely a social process, as artists receive both explicit and implicit feedback through the audience's reactions and reception of their work. Simultaneously, the embodiment and malfunctions inherent to robotic systems drive artistic experimentation. The temporal processes of creation and exhibition, beyond just the final product, further enhance the creative value. Our empirical analysis presents how creativity emerges through the interplay of social, material, and temporal interactions among artists, robots, audiences, and the environment.

% The contributions of this work are...
We make two main contributions to HCI in this study. 
First, we elucidate the interactive mechanisms among key actors---human creators, machines, audiences, and environments---within the practice of robotic art, a topic that remains underexplored in HCI. Our findings reveal the significance of sociality (e.g., interactions between artists and audiences), materiality (e.g., the embodiment and malfunctions of robots), and temporality (e.g., the processes of creation and exhibition) in shaping creative values. We propose that these three facets are central to the creative process and facilitate the emergence of creativity in robotic art.
Second, drawing from the findings, we offer implications for \textit{socially informed}, \textit{material-attentive}, and \textit{process-oriented} creation with computing systems. We suggest leveraging these three aspects to enhance creativity and the creative experience. Specifically, we discuss the value of incorporating implicit audience feedback, designing with technical malfunctions, and focusing on the post-creation process to foster alternative creative experiences with machines~\cite{alter2010designing, juarez2022glitch}.




\section{Preliminaries}
\label{sec:sec-Preliminary}




\begin{table}[t]
  \caption{
  Space complexity of various approximate matrix multiplication algorithms in the sliding-window setting. 
  %Note that the final complexity is scaled by \(O(d_x + d_y)\).
  %Space complexity for various AMM algorithms under the sliding-window setting.
  }
  \vspace{-2mm}
  \label{tab:table-comparsion}
  \renewcommand{\arraystretch}{1.5}
  \begin{tabular}{|c|c|c|}
    \hline
    Algorithm & Normalized model & Unnormalized model \\
    \hline
    Sampling \cite{efraimidis2006weighted, drineas2006fast,babcock2001sampling} & $O(\frac{d_x + d_y}{\epsilon^2}\log{N})$ & $O\left(\frac{d_x + d_y}{\epsilon^2}\log{NR}\right)$ \\
    \hline
    DI-COD \cite{YaoLCWC24} & $O(\frac{d_x + d_y}{\epsilon}\log^2{\frac{1}{\epsilon}})$ & $O\left(\frac{(d_x + d_y)\cdot R}{\epsilon}\log^2{\frac{R}{\epsilon}}\right)$\\
    \hline
    EH-COD \cite{YaoLCWC24} & $O(\frac{d_x + d_y}{\epsilon^2}\log{\epsilon N})$ & $O\left(\frac{d_x + d_y}{\epsilon^2}\log{\epsilon NR}\right)$ \\
    \hline
    % DI-SCOD \cite{YaoLCWC24} & $O(\frac{d_x + d_y}{\epsilon}\log^2{\frac{1}{\epsilon}})$ & $O\left(\frac{(d_x + d_y)\cdot R}{\epsilon}\log^2{\frac{R}{\epsilon}}\right)$\\
    % \hline
    % EH-SCOD \cite{YaoLCWC24} & $O(\frac{d_x + d_y}{\epsilon^2})$ & $O\left(\frac{d_x + d_y}{\epsilon^2}\log{R}\right)$ \\
    % \hline
    {\oursolution} (Ours) & $O(\frac{d_x + d_y}{\epsilon})$ & $O\left(\frac{d_x + d_y}{\epsilon}\log{R}\right)$ \\
    \hline
\end{tabular}
\vspace{-1mm}
\end{table}

\subsection{Notations and Basic Concepts}

In this paper, we use bold uppercase letters to denote matrices (e.g., $\boldsymbol{A}$), bold lowercase letters to denote vectors (e.g., $\boldsymbol{x}$), and lowercase letters to denote scalars (e.g., $w$).


Given a real matrix \(\boldsymbol{A} \in \mathbb{R}^{n\times m}\), its condensed singular value decomposition (SVD) is given by \(\boldsymbol{A} = \boldsymbol{U} \boldsymbol{\Sigma} \boldsymbol{V}^\top\), where \(\boldsymbol{U} \in \mathbb{R}^{n \times r}\) and \(\boldsymbol{V} \in \mathbb{R}^{m \times r}\) are matrices with orthonormal columns, and \(\boldsymbol{\Sigma} = \operatorname{diag}(\sigma_1, \sigma_2, \dots, \sigma_r)\) is a diagonal matrix containing the singular values of \(\boldsymbol{A}\) arranged in non-increasing order, i.e., \(\sigma_1 \geq \sigma_2 \geq \cdots \geq \sigma_r > 0\), with \(r\) denoting the rank of \(\boldsymbol{A}\). We define several matrix norms as follows: the Frobenius norm of \(\boldsymbol{A}\) is given by 
\(
\|\boldsymbol{A}\|_F = \sqrt{\sum_{i,j} A_{i,j}^2} = \sqrt{\sum_{i=1}^r \sigma_i^2},
\)
the spectral norm is 
\(
\|\boldsymbol{A}\|_2 = \sigma_1,
\)
and the nuclear norm is 
\(
\|\boldsymbol{A}\|_* = \sum_{i=1}^r \sigma_i.
\)
For a matrix \(\boldsymbol{A}\), we denote its \(i\)-th column by \(\boldsymbol{a}_i\); hence, if \(\boldsymbol{A} \in \mathbb{R}^{n \times l}\) and \(\boldsymbol{B} \in \mathbb{R}^{m \times l}\), then their product can be expressed as: 
\(
\boldsymbol{A}\boldsymbol{B}^T = \sum_{i=1}^l \boldsymbol{a}_i \boldsymbol{b}_i^T.
\)
Finally, \(\boldsymbol{I}_n\) denotes the \(n \times n\) identity matrix and \(\boldsymbol{0}^{n \times m}\) denotes the \(n \times m\) zero matrix.








\subsection{Problem Definition}
The Approximate Matrix Multiplication (AMM) problem has been extensively studied in the literature \cite{YeLZ16,MrouehMG17,drineas2006fast,FrancisR22}. Given two matrices \(\boldsymbol{X} \in \mathbb{R}^{d_x \times n}\) and \(\boldsymbol{Y} \in \mathbb{R}^{d_y \times n}\), the AMM problem aims to obtain two smaller matrices \(\boldsymbol{B}_X \in \mathbb{R}^{d_x \times l}\) and \(\boldsymbol{B}_Y \in \mathbb{R}^{d_y \times l}\), with \(l \ll n\), so that
\(
\left\|\boldsymbol{X}\boldsymbol{Y}^\top - \boldsymbol{B}_X \boldsymbol{B}_Y^\top\right\|_2
\) is sufficiently small. Next, we formally define the problem of AMM over a Sliding Window.

\vspace{-1mm}
\begin{definition}[AMM over a Sliding Window]\label{def:amm}
Let \(\{(\boldsymbol{x}_t, \boldsymbol{y}_t)\}_{t \ge 1}\) be a sequence of data items (a data stream), where for each time \(t\) we have \(\boldsymbol{x}_t \in \mathbb{R}^{d_x}\) and \(\boldsymbol{y}_t \in \mathbb{R}^{d_y}\). For a fixed window size \(N\) and for any time \(T \ge N\), define the sliding window matrices
\[
\boldsymbol{X}_W = \begin{bmatrix} \boldsymbol{x}_{T-N+1} & \boldsymbol{x}_{T-N+2} & \cdots & \boldsymbol{x}_T \end{bmatrix} \in \mathbb{R}^{d_x \times N},\]
\[
\boldsymbol{Y}_W = \begin{bmatrix} \boldsymbol{y}_{T-N+1} & \boldsymbol{y}_{T-N+2} & \cdots & \boldsymbol{y}_T \end{bmatrix} \in \mathbb{R}^{d_y \times N}.
\]
A streaming algorithm (or matrix sketch) \(\kappa\) is said to yield an \(\epsilon\)-approximation for AMM over the sliding window if, at every time \(T \ge N\), it outputs matrices \(\boldsymbol{A}_W \in \mathbb{R}^{d_x \times m}\) and \(\boldsymbol{B}_W \in \mathbb{R}^{d_y \times m}\) (with \(m \le N\), typically \(m \ll N\)) satisfying
\[
\left\|\boldsymbol{X}_W \boldsymbol{Y}_W^\top - \boldsymbol{A}_W \boldsymbol{B}_W^\top\right\|_2 \le \epsilon\, \|\boldsymbol{X}_W\|_F \, \|\boldsymbol{Y}_W\|_F.
\]
That said, the product \(\boldsymbol{A}_W \boldsymbol{B}_W^\top\) produced by the sketch \(\kappa\) approximates the true product \(\boldsymbol{X}_W \boldsymbol{Y}_W^\top\) with a spectral norm error that is at most an \(\epsilon\)-fraction of the product of the Frobenius norms of \(\boldsymbol{X}_W\) and \(\boldsymbol{Y}_W\).
\end{definition}
\vspace{-1mm}



In \cite{WeiLLSDW16}, it is shown that one may assume without loss of generality that the squared norms of the columns of \(\boldsymbol{X}\) and \(\boldsymbol{Y}\) are normalized to lie in the intervals \([1, R_x]\) and \([1, R_y]\), respectively, which is a mild assumption. We denote \(R = \max(R_x, R_y)\).







Next, we present two key techniques that underpin our solution: the Co-occurring Directions (COD) algorithm \cite{MrouehMG17} and the \(\lambda\)-snapshot method \cite{LeeT06}. While each technique was originally developed for a different problem, we show later that their careful and nontrivial integration yields a solution with optimal space to gain $\epsilon$-approximation guarantee.




\subsection{Co-Occurring Directions}


Co-occurring Directions (COD) \cite{MrouehMG17} is a deterministic streaming algorithm for approximate matrix multiplication. Given matrices \(\boldsymbol{X} \in \mathbb{R}^{d_x \times n}\) and \(\boldsymbol{Y} \in \mathbb{R}^{d_y \times n}\) whose columns arrive sequentially, COD maintains sketch matrices \(\boldsymbol{B}_X \in \mathbb{R}^{d_x \times l}\) and \(\boldsymbol{B}_Y \in \mathbb{R}^{d_y \times l}\) (with \(l \ll n\)) so that 
\(
\boldsymbol{X}\boldsymbol{Y}^\top \approx \boldsymbol{B}_X \boldsymbol{B}_Y^\top.
\)
Algorithm \ref{alg:cod} shows the pseudo-code of COD algorithm. In essence, each incoming column pair \((\boldsymbol{x}_i, \boldsymbol{y}_i)\) is inserted into an available slot in the corresponding sketch (Lines 3-4). When a sketch becomes full, a compression step is performed (Lines 5-12): the sketches are first orthogonalized via QR decomposition (Lines 6-7); then, an SVD is applied to the product of the resulting factors (Line 8); finally, the singular values are shrunk by a threshold \(\delta\) (typically chosen as the \(\ell/2\)-th singular value) to update the sketches (Lines 9-12). This process effectively discards less significant directions while preserving the dominant correlations, thereby controlling the approximation error. 

\begin{algorithm}[t]
    \caption{Co-occurring Directions (COD)}
    \label{alg:cod}
    \DontPrintSemicolon
    \KwInput{\(\boldsymbol{X}\in \mathbb{R}^{d_x\times n}\), \(\boldsymbol{Y}\in \mathbb{R}^{d_y\times n}\), sketch size \(l\)}
    \KwOutput{\(\boldsymbol{A}\in \mathbb{R}^{d_x\times l}\) and \(\boldsymbol{B}\in \mathbb{R}^{d_y\times l}\)}
    
    Initialize \(\boldsymbol{A} \leftarrow \boldsymbol{0}^{d_x\times l}\) and \(\boldsymbol{B} \leftarrow \boldsymbol{0}^{d_y\times l}\)\;
    
    \For{\(i = 1, \dots, n\)}{
        Insert \(\boldsymbol{x}_i\) into an empty column of \(\boldsymbol{A}\)\;
        
        Insert \(\boldsymbol{y}_i\) into an empty column of \(\boldsymbol{B}\)\;
        
        \If{\(\boldsymbol{A}\) or \(\boldsymbol{B}\) is full}{
            \((\boldsymbol{Q}_x, \boldsymbol{R}_x) \leftarrow \text{QR}(\boldsymbol{A})\)\;
            
            \((\boldsymbol{Q}_y, \boldsymbol{R}_y) \leftarrow \text{QR}(\boldsymbol{B})\)\;
            
            \([\boldsymbol{U},\boldsymbol{\Sigma},\boldsymbol{V}] \leftarrow \text{SVD}(\boldsymbol{R}_x \boldsymbol{R}_y^\top)\)\;
            
            \(\delta \leftarrow \sigma_{l/2}(\boldsymbol{\Sigma})\), \(\hat{\boldsymbol{\Sigma}} \leftarrow \max(\boldsymbol{\Sigma} - \delta\,\boldsymbol{I}_l, \boldsymbol{0})\)\;
            
            Update \(\boldsymbol{A} \leftarrow \boldsymbol{Q}_x\,\boldsymbol{U}\,\sqrt{\hat{\boldsymbol{\Sigma}}}\)\;
            Update \(\boldsymbol{B} \leftarrow \boldsymbol{Q}_y\,\boldsymbol{V}\,\sqrt{\hat{\boldsymbol{\Sigma}}}\)\;
        }
    }
    
    \Return \(\boldsymbol{A}, \boldsymbol{B}\)
\end{algorithm}






\vspace{-1mm}
\begin{theorem} \label{thm:cod}
The output of Co-occurring Directions (Algorithm~\ref{alg:cod}) provides correlation sketch matrices \((\boldsymbol{B}_X \in \mathbb{R}^{d_x \times l}, \boldsymbol{B}_Y \in \mathbb{R}^{d_y \times l})\) for \((\boldsymbol{X} \in \mathbb{R}^{d_x \times n}, \boldsymbol{Y} \in \mathbb{R}^{d_y \times n})\), where \(l \leq \min(d_x, d_y)\), satisfying:
\[
\left\|\boldsymbol{X} \boldsymbol{Y}^\top - \boldsymbol{B}_X \boldsymbol{B}_Y^\top\right\|_2 \leq \frac{2\|\boldsymbol{X}\|_F \|\boldsymbol{Y}\|_F}{l}.
\]
Algorithm~\ref{alg:cod} runs in \(O(n(d_x + d_y)l)\) time using \(O((d_x + d_y)l)\) space.
\end{theorem}

\subsection{$\boldsymbol{\lambda}$-Snapshot Method}

We explain the key idea of the $\lambda$-snapshot method \cite{LeeT06} by beginning with a bit stream
\(
f = \{b_1, b_2, b_3, \dots\}
\)
where 
\(
b_i \in \{0,1\},
\)
and the goal is to approximate the number of 1-bits in a sliding window. The \(\lambda\)-snapshot method achieves this by “sampling” every \(\lambda\)-th 1-bit. That is, if we index the 1-bits in order of appearance, the \(\lambda\)-th, \(2\lambda\)-th, \(3\lambda\)-th, etc., are stored. The stream is conceptually divided into blocks of \(\lambda\) consecutive positions (called \(\lambda\)-blocks), and a block is deemed \emph{significant} if it intersects the current sliding window and contains at least one sampled 1-bit. The algorithm maintains a \(\lambda\)-counter that tracks: {\em (i)} a queue \(Q\) holding the indices of significant \(\lambda\)-blocks; {\em (ii)} a counter \(\ell\) recording the number of 1-bits seen since the last sample; and {\em (iii)} auxiliary variables for the current block index and the offset within that block. When a new bit arrives, the offset is incremented and any block that no longer falls within the sliding window is removed from \(Q\). If the offset reaches \(\lambda\), the block index is incremented. Upon encountering a 1-bit, \(\ell\) is incremented; when \(\ell\) reaches \(\lambda\), that 1-bit is sampled (i.e., \(\ell\) is reset to 0) and the current block index is \underline{\em registered} and appended to \(Q\). The estimated count of 1-bits in the current window \(W_p\) is given by
\(
v(S) = |Q|\lambda + \ell.
\)
If the true count is \(m\), then it is guaranteed that:
\(
m \le v(S) \le m + 2\lambda,
\)
so that the error is bounded by \(2\lambda\). As the window slides, blocks falling entirely out of range are removed from \(Q\), ensuring that the estimate \(v(S) = |Q|\lambda + \ell\) remains valid with an error of at most \(2\lambda\). 

An example of how the $\lambda$-snapshot method works to count the number of 1-bits in a sliding window is provided in Appendix \ref{app:examples}.





To support frequency estimation over sliding window, a naive approach is to apply the $\lambda$-snapshot method to every distinct element in the stream. For any given element $e$, the stream can be represented as a bit stream, where each new bit is set to 1 if the stream element equals $e$ and 0 otherwise. However, maintaining a separate $\lambda$-snapshot structure for each element would lead to unbounded space usage. Lee et al.\ \cite{LeeT06} show that by maintaining only $O(1/\epsilon)$ such $\lambda$-snapshot structures and combine with the well known frequency estimation algorithm MG-sketch \cite{MisraG82}, an $\epsilon$-approximation for the frequency of each element can be achieved while using just $O(1/\epsilon)$ space. Although there are $O(1/\epsilon)$ $\lambda$-snapshot structures, they collectively track a stream containing at most $N$ ones, ensuring that the overall space cost remains bounded. For each element $e$, let $f(e)$ be its true frequency and $\hat{f}(e)$ be the estimated frequency derived using the $\lambda$-snapshot method, then
\(
f(e)-\hat{f}(e) \leq \epsilon N.
\)









\section{Methodology}
\begin{table*}
\centering
\begin{tblr}{|Q[l]|Q[4.5cm, c]|Q[8cm, l]|}
\hline
\textbf{Representation} & \textbf{Input} & \textbf{Output} \\
\hline
Markdown & \SetCell[r=4]{c}{Your **contribution** to Goodwill will mean more than you may know.} & - Donor: Your \linebreak - Recipient: to Goodwill\\
\hline
XML Tags &  & \texttt{<Donor>Your</Donor> contribution <Recipient>to Goodwill</Recipient> will mean more than you may know.} \\
\hline
JSON-Existing &  & \texttt{\{``Donor": ``Your", ``Recipient": ``to Goodwill"\}} \\
\hline
JSON-Complete &  & \texttt{\{``Donor": ``Your", ``Recipient": ``to Goodwill", ``Theme": ``", ``Place": ``", ...\}} \\
\hline
\end{tblr}
\caption{\label{tab:fe-representation}Representation formats for the given input and outputs.}
\end{table*}



\subsection{Input Representation Design}
\label{sec:input-representation}
Previous research has shown that large language models are sensitive to input formatting~\cite{Sclar2023QuantifyingLM} and that different representations can result in different model performance~\cite{tam-etal-2024-speak,textsql-eval-gao-2024,exploring-marcos-2024}. To study these effects on frame-semantics, we systematically evaluated multiple input-output representation formats to determine their impact on frame element extraction performance.

For all input formats, we wrap the target word or phrase in double asterisks, as shown in Table~\ref{tab:fe-representation}, to explicitly mark the token that evokes the frame. This marking helps focus the model's attention on the relevant part of the sentence when making frame element predictions, ensuring that the model identifies frame elements for the correct target.

We developed and tested four distinct representation formats. The Markdown format offers a simple, human-readable approach where frame elements are represented as a markdown list. Each list item contains a frame element name paired with its corresponding text span from the sentence. This format only includes frame elements that the model predicts are present in the input. The XML Tags format provides a structured approach that uses XML-style tags to wrap frame elements within the sentence text. The tag names correspond to frame element names, providing both semantic labeling and precise positional information without requiring additional processing.

We also developed two JSON-based formats. The JSON-Existing format uses frame element names as keys and their corresponding text spans from the sentence as values. Similar to the Markdown format, this only includes predicted frame elements. The JSON-Complete format provides an exhaustive representation different from previous representations that includes all possible frame elements as keys, with empty strings as values for elements not found in the sentence. This format was designed to test whether explicitly presenting all possible frame elements might improve model performance. Examples of each representation format are provided in Table~\ref{tab:fe-representation}, illustrating how they encode the same semantic information in different ways.

\subsection{Model Selection and Implementation}
To ensure a comprehensive evaluation across the current LLM landscape, we selected models varying in size, architecture, and accessibility. Our selection criteria focused on three key dimensions. In terms of model scale, we included models ranging from 0.5B to 78B parameters, categorizing them into small-scale (0-14B parameters) and large-scale (14B+ parameters) groups to analyze the impact of model size on performance. For architecture diversity, we selected top-performing models from the HuggingFace LLM leaderboard, with particular focus on Qwen 2.5 and Llama 3.2, which have shown strong performance on various tasks.

We included both open-source models (Qwen 2.5, Llama 3, and Phi-4) and closed-source systems (GPT-4o and GPT-4o-mini) to compare performance across different levels of model accessibility. For the open-source models, we implemented fine-tuning using LoRA~\cite{hu2021loralowrankadaptationlarge}. We used $r=16$ for all models except Llama 3.3 and Qwen 2.5 (72B) where we used $r=32$, according to best practices. This approach allowed us to optimize model performance while maintaining reasonable computational requirements.

\subsection{Evaluation}
Our evaluation framework was designed to comprehensively assess model performance across different scenarios and conditions. We began by testing each representation's effectiveness using controlled experiments with GPT-4o-mini. Model performance was evaluated using standard metrics including precision, recall, F1 score, and accuracy with exact match criteria.

To understand data requirements and efficiency, we analyzed performance with varying amounts of training data to understand data efficiency and saturation points. We also conducted extensive testing of model performance on unseen frames, unseen frame elements, and out-of-domain samples. Finally, we analyzed the distribution of argument extraction performance for each frame to gain a granular understanding. This evaluation framework enables us to systematically evaluate LLMs' capabilities in frame-semantic parsing while providing insights into the impact of different design choices and implementation strategies.


% We use common representations from our knowledge, including Markdown, XML tags, and two types of JSON representations. Examples of each of these representations can be found in Table~\ref{tab:fe-representation}. 

% The \textit{Markdown} representation is a very simple representation which represents a typical approach for instructing an LLM. The output is expected as a markdown list which contain a frame element name and its corresponding value within the sentence. In this representation, only frame elements which the model predicts exist will be included in the output. 

% The \textit{XML Tags} representation contains XML tags wrapped around the frame elements in the sentence. The names of these XML tags are determined by the frame element name. One additional benefit this approach provides is a positional understanding of the frame elements without additional post-processing. 

% The JSON representations are similar to the Markdown outputs, but in JSON format instead, where keys are the names of frame elements and values are the substring for those frame elements in the sentence. The \textit{JSON-Complete} representation, unlike other representations, includes all frame elements from the input frame as keys. For the frame elements which do not appear in the sentence, they are left blank. We include this option to identify whether alleviating the need to recall the frame elements from the input may improve the performance.

% \subsection{Models}

% We explore several state-of-the-art large language models for our evaluation. We particularly focus on diversity in a few key areas, namely, model size, model architecture, and model availability. For model size, we explore sizes ranging from 0.5B up to 78B, loosely categorizing them into small- (0-14B parameters) and large-scale models (14B+ parameters). We selected top-performing model families from the HuggingFace LLM leaderboard and found that Qwen 2.5 and Llama 3.2 are very common among top-performing systems. Finally, we also explored closed-source systems like GPT-4o and 4o-mini.

% We fine-tuned the open-source models (Qwen 2.5, Llama 3, and Phi-4) using LoRA~\cite{hu2021loralowrankadaptationlarge}. The details of our implementation are reported in Appendix~\ref{app:reproducibility}. 


\section{Empirical Evaluation}
\begin{table*}[!ht]
    \centering
    \resizebox{0.88\textwidth}{!}{    
    \begin{tabular}{r|cccccc|cccccc}
        \toprule 
        & \multicolumn{6}{c}{\textbf{LLaVA-1.5-7B}} & \multicolumn{6}{c}{\textbf{LLaVA-1.5-13B}} \\ 
        \cmidrule(lr){2-7}\cmidrule(lr){8-13}
        & \multicolumn{3}{c}{\textbf{MM-SafetyBench}} & \multicolumn{3}{c|}{\textbf{MOSSBench}} & \multicolumn{3}{c}{\textbf{MM-SafetyBench}} & \multicolumn{3}{c}{\textbf{MOSSBench}} \\
        \textbf{Method} & \textbf{DSR}$\uparrow$ & \textbf{RR}$\uparrow$ & \textbf{Avg}$\uparrow$ & \textbf{DSR}$\uparrow$ & \textbf{RR}$\uparrow$ & \textbf{Avg}$\uparrow$ & \textbf{DSR}$\uparrow$ & \textbf{RR}$\uparrow$ & \textbf{Avg}$\uparrow$ & \textbf{DSR}$\uparrow$ & \textbf{RR}$\uparrow$ & \textbf{Avg}$\uparrow$\\
        \midrule
        w/o Defense          & 0.06  & 0.98  & 0.52  & 0.14  & 0.97  & 0.55  & 0.10  & 0.97  & 0.53  & 0.30  & 0.96  & 0.63  \\
        \midrule
        \multicolumn{13}{c}{Baseline} \\
        \midrule
        Responsible          & 0.12  & 0.96  & 0.54  & 0.32  & 0.96  & 0.64  & 0.18  & 0.96  & 0.57  & 0.47  & 0.92  & 0.70  \\
        Policy               & 0.08  & 0.96  & 0.52  & 0.18  & 0.98  & 0.58  & 0.12  & 0.97  & 0.55  & 0.34  & 0.97  & 0.65  \\
        Demonstration        & 0.15  & 0.97  & 0.56  & 0.37  & 0.95  & 0.66  & 0.25  & 0.96  & 0.60  & 0.52  & 0.92  & \textbf{0.72}  \\
        SFT                  & 0.20  & 0.95  & 0.58  & 0.50  & 0.88  & 0.69  & 0.13  & 0.98  & 0.55  & 0.49  & 0.88  & 0.68 \\
        SafeDecoding         & 0.08  & 0.97  & 0.53  & 0.31  & 0.94  & 0.62  & 0.12  & 0.96  & 0.54  & 0.42  & 0.93  & 0.68  \\
        Caption              & 0.09  & 0.98  & 0.53  & 0.21  & 0.98  & 0.60  & 0.12  & 0.97  & 0.55  & 0.27  & 0.94  & 0.60  \\
        Caption (w/o image)  & 0.16  & 0.95  & 0.55  & 0.34  & 0.94  & 0.64  & 0.22  & 0.93  & 0.57  & 0.45  & 0.89  & 0.67 \\
        Intention            & 0.07  & 0.98  & 0.53  & 0.20  & 0.99  & 0.59  & 0.11  & 0.96  & 0.54  & 0.26  & 0.97  & 0.61  \\
        \midrule
        % \multicolumn{13}{c}{} \\
        % \midrule
        \midrule
        \multicolumn{13}{c}{SR++} \\
        \midrule        
        Responsible-Demonstration & 0.18 & 0.95 & 0.57 & 0.40 & 0.94 & 0.67 & 0.29 & 0.96 & 0.62 & 0.58 & 0.85 & \textbf{0.72} \\
        Responsible-Policy & 0.12 & 0.96 & 0.54 & 0.27 & 0.97 & 0.62 & 0.18 & 0.96 & 0.57 & 0.46 & 0.94 & 0.70 \\
        Policy-Demonstration & 0.13 & 0.96 & 0.55 & 0.37 & 0.97 & 0.67 & 0.20 & 0.96 & 0.58 &0.51 & 0.93 & \textbf{0.72}\\
        Responsible-Policy-Demonstration & 0.15 & 0.96 & 0.55 & 0.38 & 0.95 & 0.66 & 0.25 & 0.97 & 0.61 & 0.53 & 0.88 & 0.70\\
        \midrule
        \multicolumn{13}{c}{SR+MO} \\
        \midrule     
        Responsible-SFT & 0.56 & 0.93 & \textbf{0.75} & 0.61 & 0.72 & 0.67 & 0.35 & 0.96 & 0.65 & 0.74 & 0.62 & 0.68 \\
        Responsible-SafeDecoding & 0.30 & 0.96 & 0.63 & 0.54 & 0.87 & \underline{0.70} & 0.23 & 0.96 & 0.59 & 0.63 & 0.79 & 0.71\\
        Demonstration-SFT & 0.60 & 0.90 & \textbf{0.75} & 0.65 & 0.77 & \textbf{0.71} & 0.56 & 0.92 & \textbf{0.74} & 0.67 & 0.70 & 0.68\\
        Demonstration-SafeDecoding & 0.38 & 0.96 & \underline{0.67} & 0.55 & 0.87 & \textbf{0.71} & 0.40 & 0.96 & \underline{0.68} & 0.62 & 0.78 & 0.70\\
        \midrule
        \multicolumn{13}{c}{QR++} \\
        \midrule   
        Caption-Intention & 0.09 & 0.97 & 0.53 & 0.20 & 0.98 & 0.59 & 0.14 & 0.95 & 0.55 & 0.26 & 0.96 & 0.61\\
        % Caption-Intention (w/o image) & 0.18 & 0.96 & 0.57 & 0.32 & 0.95 & 0.64 & 0.25 & 0.92 & 0.59 & 0.45 & 0.92 & 0.68\\
        \midrule
        % \multicolumn{13}{c}{} \\
        % \midrule
        \midrule
        \multicolumn{13}{c}{QR\textbar{}SR} \\
        \midrule   
        Caption-Responsible & 0.34 & 0.96 & 0.65 & 0.53 & 0.79 & 0.66 & 0.33 & 0.96 & 0.65 & 0.50 & 0.82 & 0.66\\
        Intention-Responsible & 0.36 & 0.97 & \underline{0.67} & 0.51 & 0.86 & 0.68 & 0.27 & 0.96 & 0.61 & 0.49 & 0.90 & 0.70\\
        Caption-Responsible (w/o image) & 0.96 & 0.25 & 0.60 & 0.93 & 0.16 & 0.55 & 0.60 & 0.80 & \underline{0.70} & 0.72 & 0.72 & \textbf{0.72}\\
        % Responsible-Intention (w/o image) & 0.99 & 0.06 & 0.52 & 0.95 & 0.17 & 0.56 & 0.61 & 0.81 & 0.71 & 0.68 & 0.77 & 0.72\\
        \midrule
        \multicolumn{13}{c}{QR\textbar{}MO} \\
        \midrule
        Caption-SafeDecoding & 0.20 & 0.96 & 0.58 & 0.39 & 0.88 & 0.64 & 0.33 & 0.94 & 0.63 & 0.40 & 0.90 & 0.65 \\
        Intention-SFT & 0.28 & 0.97 & 0.62 & 0.43 & 0.78 & 0.61 & 0.25 & 0.96 & 0.60 & 0.50 & 0.88 & 0.69\\
        Caption-SafeDecoding (w/o image) & 0.24 & 0.95 & 0.60 & 0.41 & 0.89 & 0.65 & 0.36 & 0.85 & 0.61 & 0.56 & 0.84 & 0.70\\
        \bottomrule
    \end{tabular}}
    \caption{Comparison results of ensemble strategies with the corresponding individual defenses. \textbf{Bold} indicates the best overall performance, while \underline{underlined} highlights the top three methods.} % and the full score is 100\%
    \label{tab:en_inter_results}
\end{table*}


\subsection{Experimental Setup}
We empirically evaluate various defense methods and their ensemble strategies on LLaVA-1.5-7B and LLaVA-1.5-13B~\cite{liu2024visual} to validate their effectiveness in standard settings. Using MM-SafetyBench and MOSSBench datasets, we assess safety and helpfulness by measuring defense success rate (DSR) on harmful queries and response rate (RR) on benign queries. We evaluate 28 defense methods, including system reminders, optimization techniques, query refactoring, and noise injection, as well as inter- and intra-mechanism ensembles. Detailed descriptions of defense methods and experimental setups are provided in Appendix~\ref{sec:defense strategies} and~\ref{sec:experiment_detail}. 
For a broader evaluation, we add more experiments in Appendix~\ref{sec:utility}, ~\ref{sec:diverse_attacks} and~\ref{sec:time}, including evaluation with the MM-Vet dataset for testing the quality of model's response on general queries, tests on JailbreakV-28K for more diverse and complex attack scenarios, and a comparison of inference time for different defense methods.

\subsection{Individual Defense Results}

Table~\ref{tab:indi_results} shows results of individual defense methods across four categories. Most methods, except for noise injection, effectively improve model safety across different models and datasets, as evidenced by increased defense success rates. This aligns with our analysis in Figure~\ref{fig:analysis results} where system reminder, model optimization and query refactoring lead to an overall increase in refusal probabilities. 

\paragraph{Safety shift defenses compromise helpfulness.} System reminder and model optimization methods generally reduce response rates on the benign subset while increasing defense success rates on the harmful subset. This confirms that safety shift tend to compromise helpfulness. This is more pronounced in MOSSBench than MM-SafetyBench due to the more apparent harmfulness and concealed harmlessness in MOSSBench queries.

\paragraph{Harmfulness discrimination defenses mitigate over-defense.} Query refactoring methods, except for Caption (w/o image), generally achieve the highest response rates on the benign subset, particularly for MOSSBench with misleadingly benign queries. This validates that harmfulness discrimination improves the model's ability to distinguish between truly harmful and benign queries. Notably, the removal of images in the Caption (w/o image) significantly reduces response rates for both harmful and benign queries, highlighting the crucial role images play in jailbreaking LVLMs.
% \paragraph{Image matters.} The removal of images in the Caption (w/o image) and Intention (w/o image) defenses leads to significant improvements in DSR compared to their image-included counterparts, underscoring the crucial role that images play in jailbreaking LVLMs.

\paragraph{Multimodal defense is challenging.}
However, all individual defense methods still exhibit limited defense success rates. While larger-scale LVLMs (i.e., LLaVA-1.5-13B) tend to achieve slightly higher success rates, they are also more susceptible to over-defense. This underscores the inherent challenges of jailbreak defense for LVLMs, especially when relying on individual defense methods. 

\subsection{Ensemble Defense Results}
Table~\ref{tab:en_inter_results} provides the empirical evaluation of both inter-mechanism and intra-mechanism ensemble strategies, leading to the following insights:

\paragraph{Ensembles improve safety.} Compared to individual methods, most ensemble strategies effectively enhance safety across both datasets and model sizes, showing increased defense success rates, especially in \textit{SR+MO} and \textit{QR\textbar{}SR} methods.

\paragraph{Inter-mechanism ensembles amplify.} Our evaluation shows most \textit{SR++} and \textit{SR+MO} ensembles improve defense success rates while reducing responses rates, whereas the \textit{QR++} ensemble better maintain responses rates. This confirms that inter-mechanism ensembles can amplify a single defense mechanism. Specifically, safety shift ensembles would further enhance model safety at the expense of helpfulness, while harmfulness discrimination ensemble better preserves helpfulness. Among inter-mechanism ensembles, those combining different types of specific methods (e.g., SR+MO) show a more pronounced amplification effect than those combining the same type (e.g., SR++). 
Notably, the Demonstration-SFT method excels in defense strength, utility, and response rate. Its success comes from combining two strong safety shift defenses, Demonstration and SFT, which complement each other and boost overall performance.

\paragraph{Intra-mechanism ensembles complement.} Compared to inter-mechanism ensembles, most \textit{QR\textbar{}SR} and \textit{QR\textbar{}MO} methods—except those without input images—can simultaneously maintain decent defense success rates and stable response rates,
compared to the undefended model and individual defense methods. This demonstrates that intra-mechanism ensemble can complement each other to achieve a more balanced trade-off. Additionally, the removal of input images offering a most conservative ensemble for multimodal defense while still maintaining certain helpfulness.
% In contrast, the defenses in intra-mechanism ensemble complement each other, strengthening safety while maintaining a stable level of helpfulness.
% In contrast, intra-mechanism ensembles combine the strengths of both mechanisms to achieve a more balanced trade-off. Specifically, \textit{QR\textbar{}SR} and \textit{QR\textbar{}MO} increase the refusal probability for harmful queries, while maintaining or even decreasing the refusal probability for benign queries, thereby improving the model's ability to distinguish between benign and harmful queries. This makes them a better choice for general scenarios where balancing safety and helpfulness is essential. 


\subsection{How Do Fine-tuning Affect Model Safety?}
We examine how different fine-tuning methods impact the safety of LVLMs by training LLaVA-1.5-7B using DPO and SFT with two datasets: SPA-VL~\cite{zhang2024spa} and VLGuard~\cite{zong2024safety}. SPA-VL focuses on safety discussions, while VLGuard emphasizes query rejection. We also test the effect of adding 5000 general instruction-following data from LLaVA.  

Table~\ref{tab:training_dataset_results} shows that DPO with SPA-VL and LLaVA provides a slight safety boost without significantly changing response behavior. In contrast, SFT has a stronger impact, but its effectiveness depends on the dataset. SPA-VL improves safety while maintaining helpfulness, though it may miss some harmful cases. VLGuard, however, makes the model overly defensive, rejecting too many queries. Adding LLaVA data helps balance safety and helpfulness, reducing excessive refusals.  


\begin{table}[ht]
    \centering
    \resizebox{0.49\textwidth}{!}{
    \begin{tabular}{r|cccccc}
        \toprule 
        & \multicolumn{3}{c}{\textbf{MM-SafetyBench}} & \multicolumn{3}{c}{\textbf{MOSSBench}} \\
        \textbf{Method} & \textbf{DSR}$\uparrow$ & \textbf{RR}$\uparrow$ & \textbf{Avg}$\uparrow$ & \textbf{DSR}$\uparrow$ & \textbf{RR}$\uparrow$ & \textbf{Avg}$\uparrow$ \\
        \midrule
        w/o Defense          & 0.06  & 0.98  & 0.52  & 0.14  & 0.97  & 0.55 \\
        \midrule
        \multicolumn{7}{c}{DPO} \\
        \midrule
        \multicolumn{1}{l|}{SPA-VL + LLaVA}          & 0.06  & 0.97  & 0.52  & 0.28  & 0.97  & 0.63  \\
        \midrule
        \multicolumn{7}{c}{SFT} \\
        \midrule
        \multicolumn{1}{l|}{SPA-VL}          & 0.24  & 0.96  & 0.60  & 0.58  & 0.78  & 0.68  \\
        + LLaVA     & 0.20  & 0.95  & 0.58  & 0.50  & 0.88  & 0.69  \\
        \midrule
        \multicolumn{1}{l|}{VLGuard}          & 1.00  & 0.09  & 0.55  & 0.90  & 0.21  & 0.55  \\
        + LLaVA     & 0.97  & 0.43  & 0.70  & 0.76  & 0.58  & 0.67  \\
        \bottomrule
    \end{tabular}}
    \caption{Comparison of varying fine-tuning settings.} % and the full score is 100\%
    \label{tab:training_dataset_results}
\end{table}


\section{Conclusion}
We introduced \methodname, an effective training framework defending against MIAs for LLMs. The extensive experiments demonstrate its robustness in protecting privacy while maintaining strong language modeling performance across various datasets and architectures. Although our study focuses on fine-tuning due to computational constraints, \methodname can be seamlessly applied to large-scale pretraining, as done in prior selective pretraining work~\cite{lin2024not}. By categorizing tokens and treating them appropriately, \methodname opens a novel pathway for MIA defense. Future work can explore improved token selection strategies and multi-objective training approaches.

%%%%%%%%%%%%%%%%%%%%%%%%%%%%%%%%%%%%%%%%%%%%%%%%%%%%%%%%%%%%%%%%%%%%%%%%%%%%%%%%

% This project has received funding from the European Union through ERC, INTERACT, under Grant 101041863. Views and opinions expressed are however those of the author(s) only and do not necessarily reflect those of the European
% Union. Neither the European Union nor the granting authority can be held responsible for them.

\printbibliography[heading=bibintoc,title=References]

\end{document}
