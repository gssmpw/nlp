%%%%%%%%%%%%%%%%%%%%%%%%%%%%%%%%%%%%%%%%%%%%%%%%%%%%%%%%%%%%%%%%%%%%%%%%%%%%%%%%
%2345678901234567890123456789012345678901234567890123456789012345678901234567890
%        1         2         3         4         5         6         7         8

\documentclass[letterpaper, 10 pt, conference]{ieeeconf}  % Comment this line out if you need a4paper

%\documentclass[a4paper, 10pt, conference]{ieeeconf}      % Use this line for a4 paper

\IEEEoverridecommandlockouts                              % This command is only needed if 
                                                          % you want to use the \thanks command

\overrideIEEEmargins                                      % Needed to meet printer requirements.

%In case you encounter the following error:
%Error 1010 The PDF file may be corrupt (unable to open PDF file) OR
%Error 1000 An error occurred while parsing a contents stream. Unable to analyze the PDF file.
%This is a known problem with pdfLaTeX conversion filter. The file cannot be opened with acrobat reader
%Please use one of the alternatives below to circumvent this error by uncommenting one or the other
%\pdfobjcompresslevel=0
%\pdfminorversion=4

% See the \addtolength command later in the file to balance the column lengths
% on the last page of the document

\RequirePackage{mathtools}  % Mathematical tools to use with amsmath
\RequirePackage{amssymb}    % Extended symbol collection
\RequirePackage{siunitx}    % Comprehensive (SI) units package

\RequirePackage{tabularx}   % Tabulars with adjustable-width columns
\RequirePackage{booktabs}   % Publication quality tables
\RequirePackage{longtable}  % Allow tables to flow over page boundaries
\RequirePackage{multirow}   % Create tabular cells spanning multiple rows
\RequirePackage{colortbl}
\RequirePackage{xcolor}

\RequirePackage{graphicx}   % Enhanced support for images
\RequirePackage{float}      % Improved interface for floating objects
\RequirePackage[labelfont=bf,justification=centering, footnotesize]{caption} % Captions
\RequirePackage{subcaption}         % Support for sub-captions
\RequirePackage{algpseudocode}      % Include algorithm pseudocode
\RequirePackage{algorithm}          % Include algorithm
\usepackage{hyperref}
\captionsetup{justification   = justified,
              singlelinecheck = false}

\usepackage{soul,xcolor}
\usepackage[normalem]{ulem}
\usepackage[backend=biber, style=numeric-comp, sorting=none]{biblatex}
\addbibresource{references.bib}

\AtEveryBibitem{
  \clearfield{doi}
  \clearfield{url}
  \clearfield{issn}
}

% Define a custom command to remove end tags
\newcommand{\algorithmwithoutend}{%
  \algtext*{EndWhile}%
  \algtext*{EndFor}%
  \algtext*{EndLoop}%
  \algtext*{EndIf}%
  \algtext*{EndProcedure}%
  \algtext*{EndFunction}%
}

\definecolor{lightgray}{gray}{0.92}
\definecolor{lightgreen}{rgb}{0.4019607843137255, 0.7862745098039216, 0.3901960784313726}
\setstcolor{red}

\title{\LARGE \bf
Pushing Through Clutter With Movability Awareness of \\ Blocking Obstacles
}

\author{Joris J. Weeda, Saray Bakker, Gang Chen and Javier Alonso-Mora% <-this % stops a space
\thanks{This project has received funding from the European Union through ERC, INTERACT, under Grant 101041863. Views and opinions expressed are however those of the author(s) only and do not necessarily reflect those of the European Union. Neither the European Union nor the granting authority can be held responsible for them.} 
\thanks{The authors are with the Cognitive Robotics Department, TU Delft,
2628 CD Delft, The Netherlands 
        {\tt\small \{jjweeda, s.bakker-7, g.chen-5}\}@tudelft.nl}%
}


\begin{document}

\maketitle
\thispagestyle{empty}
\pagestyle{empty}

%%%%%%%%%%%%%%%%%%%%%%%%%%%%%%%%%%%%%%%%%%%%%%%%%%%%%%%%%%%%%%%%%%%%%%%%%%%%%%%%
\begin{abstract}
Navigation Among Movable Obstacles (NAMO) poses a challenge for traditional path-planning methods when obstacles block the path, requiring push actions to reach the goal. 
We propose a framework that enables movability-aware planning to overcome this challenge without relying on explicit obstacle placement.
Our framework integrates a global Semantic Visibility Graph and a local Model Predictive Path Integral (SVG-MPPI) approach to efficiently sample rollouts, taking into account the continuous range of obstacle movability. A physics engine is adopted to simulate the interaction result of the rollouts with the environment, and generate trajectories that minimize contact force.
In qualitative and quantitative experiments, SVG-MPPI outperforms the existing paradigm that uses only binary movability for planning, achieving higher success rates with reduced cumulative contact forces.
Our code is available at: \href{https://github.com/tud-amr/SVG-MPPI}{https://github.com/tud-amr/SVG-MPPI}


\end{abstract}

\begin{keywords}
    Motion and Path Planning, 
    Collision Avoidance,
    Integrated Planning and Control
\end{keywords}

%%%%%%%%%%%%%%%%%%%%%%%%%%%%%%%%%%%%%%%%%%%%%%%%%%%%%%%%%%%%%%%%%%%%%%%%%%%%%%%%

\section{Introduction}

Despite the remarkable capabilities of large language models (LLMs)~\cite{DBLP:conf/emnlp/QinZ0CYY23,DBLP:journals/corr/abs-2307-09288}, they often inevitably exhibit hallucinations due to incorrect or outdated knowledge embedded in their parameters~\cite{DBLP:journals/corr/abs-2309-01219, DBLP:journals/corr/abs-2302-12813, DBLP:journals/csur/JiLFYSXIBMF23}.
Given the significant time and expense required to retrain LLMs, there has been growing interest in \emph{model editing} (a.k.a., \emph{knowledge editing})~\cite{DBLP:conf/iclr/SinitsinPPPB20, DBLP:journals/corr/abs-2012-00363, DBLP:conf/acl/DaiDHSCW22, DBLP:conf/icml/MitchellLBMF22, DBLP:conf/nips/MengBAB22, DBLP:conf/iclr/MengSABB23, DBLP:conf/emnlp/YaoWT0LDC023, DBLP:conf/emnlp/ZhongWMPC23, DBLP:conf/icml/MaL0G24, DBLP:journals/corr/abs-2401-04700}, 
which aims to update the knowledge of LLMs cost-effectively.
Some existing methods of model editing achieve this by modifying model parameters, which can be generally divided into two categories~\cite{DBLP:journals/corr/abs-2308-07269, DBLP:conf/emnlp/YaoWT0LDC023}.
Specifically, one type is based on \emph{Meta-Learning}~\cite{DBLP:conf/emnlp/CaoAT21, DBLP:conf/acl/DaiDHSCW22}, while the other is based on \emph{Locate-then-Edit}~\cite{DBLP:conf/acl/DaiDHSCW22, DBLP:conf/nips/MengBAB22, DBLP:conf/iclr/MengSABB23}. This paper primarily focuses on the latter.

\begin{figure}[t]
  \centering
  \includegraphics[width=0.48\textwidth]{figures/demonstration.pdf}
  \vspace{-4mm}
  \caption{(a) Comparison of regular model editing and EAC. EAC compresses the editing information into the dimensions where the editing anchors are located. Here, we utilize the gradients generated during training and the magnitude of the updated knowledge vector to identify anchors. (b) Comparison of general downstream task performance before editing, after regular editing, and after constrained editing by EAC.}
  \vspace{-3mm}
  \label{demo}
\end{figure}

\emph{Sequential} model editing~\cite{DBLP:conf/emnlp/YaoWT0LDC023} can expedite the continual learning of LLMs where a series of consecutive edits are conducted.
This is very important in real-world scenarios because new knowledge continually appears, requiring the model to retain previous knowledge while conducting new edits. 
Some studies have experimentally revealed that in sequential editing, existing methods lead to a decrease in the general abilities of the model across downstream tasks~\cite{DBLP:journals/corr/abs-2401-04700, DBLP:conf/acl/GuptaRA24, DBLP:conf/acl/Yang0MLYC24, DBLP:conf/acl/HuC00024}. 
Besides, \citet{ma2024perturbation} have performed a theoretical analysis to elucidate the bottleneck of the general abilities during sequential editing.
However, previous work has not introduced an effective method that maintains editing performance while preserving general abilities in sequential editing.
This impacts model scalability and presents major challenges for continuous learning in LLMs.

In this paper, a statistical analysis is first conducted to help understand how the model is affected during sequential editing using two popular editing methods, including ROME~\cite{DBLP:conf/nips/MengBAB22} and MEMIT~\cite{DBLP:conf/iclr/MengSABB23}.
Matrix norms, particularly the L1 norm, have been shown to be effective indicators of matrix properties such as sparsity, stability, and conditioning, as evidenced by several theoretical works~\cite{kahan2013tutorial}. In our analysis of matrix norms, we observe significant deviations in the parameter matrix after sequential editing.
Besides, the semantic differences between the facts before and after editing are also visualized, and we find that the differences become larger as the deviation of the parameter matrix after editing increases.
Therefore, we assume that each edit during sequential editing not only updates the editing fact as expected but also unintentionally introduces non-trivial noise that can cause the edited model to deviate from its original semantics space.
Furthermore, the accumulation of non-trivial noise can amplify the negative impact on the general abilities of LLMs.

Inspired by these findings, a framework termed \textbf{E}diting \textbf{A}nchor \textbf{C}ompression (EAC) is proposed to constrain the deviation of the parameter matrix during sequential editing by reducing the norm of the update matrix at each step. 
As shown in Figure~\ref{demo}, EAC first selects a subset of dimension with a high product of gradient and magnitude values, namely editing anchors, that are considered crucial for encoding the new relation through a weighted gradient saliency map.
Retraining is then performed on the dimensions where these important editing anchors are located, effectively compressing the editing information.
By compressing information only in certain dimensions and leaving other dimensions unmodified, the deviation of the parameter matrix after editing is constrained. 
To further regulate changes in the L1 norm of the edited matrix to constrain the deviation, we incorporate a scored elastic net ~\cite{zou2005regularization} into the retraining process, optimizing the previously selected editing anchors.

To validate the effectiveness of the proposed EAC, experiments of applying EAC to \textbf{two popular editing methods} including ROME and MEMIT are conducted.
In addition, \textbf{three LLMs of varying sizes} including GPT2-XL~\cite{radford2019language}, LLaMA-3 (8B)~\cite{llama3} and LLaMA-2 (13B)~\cite{DBLP:journals/corr/abs-2307-09288} and \textbf{four representative tasks} including 
natural language inference~\cite{DBLP:conf/mlcw/DaganGM05}, 
summarization~\cite{gliwa-etal-2019-samsum},
open-domain question-answering~\cite{DBLP:journals/tacl/KwiatkowskiPRCP19},  
and sentiment analysis~\cite{DBLP:conf/emnlp/SocherPWCMNP13} are selected to extensively demonstrate the impact of model editing on the general abilities of LLMs. 
Experimental results demonstrate that in sequential editing, EAC can effectively preserve over 70\% of the general abilities of the model across downstream tasks and better retain the edited knowledge.

In summary, our contributions to this paper are three-fold:
(1) This paper statistically elucidates how deviations in the parameter matrix after editing are responsible for the decreased general abilities of the model across downstream tasks after sequential editing.
(2) A framework termed EAC is proposed, which ultimately aims to constrain the deviation of the parameter matrix after editing by compressing the editing information into editing anchors. 
(3) It is discovered that on models like GPT2-XL and LLaMA-3 (8B), EAC significantly preserves over 70\% of the general abilities across downstream tasks and retains the edited knowledge better.

\section{Preliminaries}
\label{sec:preliminaries}

Let $S$ be a set of vertices of a graph $G=(V, E)$.
We denote the total degree of vertices in $S$ by $d(S) = \sum_{u \in S} d_u$, where $d_u$ represents the degree of vertex $u$.
A simple observation is that the size of the edge cut $\partial(S)$ is determined by the degree of the vertices in $S$ and the edges between vertices of $S$.
If there are $k$ edges between vertices of $S$, then $\size{\partial(S)}=d(S)- 2k$.
%
Since the number of edges in $S$ may vary from $0$ to $\binom{\size{S}}{2}$, a necessary condition for the realizability of an \GRC{} instance is as follows.

\begin{remark}
\label{thm:feasible_cut_sizes}
    A \GRC{} instance $(\texttt{d}, \call)$ is realizable only if, for each cut $(S, \ell) \in \call$, we have $\ell \in \set{ d(S) - 2k : 0 \leq k \leq \binom{\size{S}}{2} }$.
\end{remark}

Since this condition is easily verifiable, we assume henceforth that it holds for any \GRC{} instance. In particular, for cuts of size two, this observation implies that only two feasible values are possible, determining whether an edge must exist between the corresponding vertices, as detailed below.

\begin{remark}
\label{thm:fixed_forbidden_edges}
    Given an instance $I = (\texttt{d}, \call)$ of \GRC{}, in any realization $G$ of $I$, if $(\set{u, v}, d_u + d_v - 2) \in \call$, then $uv \in E(G)$, and if $(\set{u, v}, d_u + d_v) \in \call$, then $uv \notin E(G)$.
\end{remark}

Based on this, we say that an edge $uv$ is \textit{fixed} if $(\set{u, v}, d_u + d_v - 2) \in \call$ and is \textit{forbidden} if $(\set{u, v}, d_u + d_v) \in \call$.
We apply similar terminology when constructing an instance of \GRC{}.
Given an instance $(\texttt{d}, \call)$ of \GRC{}, to \textit{fix} or \textit{forbid} an edge $uv$ means adding the cut $(\set{u, v}, d_u + d_v - 2)$ or $(\set{u, v}, d_u + d_v)$ to $\call$, respectively.

\cref{thm:fixed_forbidden_edges} implies that the \GRC{} problem, when limited to cuts of size two, is equivalent to the \GR{} problem with added constraints: a subset of edges is fixed, and another disjoint one is forbidden. Moreover, we can simplify the problem by focusing only on forbidden edges by reducing the degree of vertices incident to fixed edges and then marking those edges as forbidden.
%
Formally, given an instance $(\texttt{d}, \call)$ and a cut $(\set{u, v}, d_u + d_v - 2) \in \call$, in which case the edge $uv$ is fixed, we can produce an equivalent instance $(\texttt{d}', \call')$ as follows.  For all $i \notin \{u, v\}$  set $d'_i = d_i$. Reduce $d'_u = d_u - 1$ and   $d'_v = d_v - 1$;  
%\begin{align*}
    % \begin{cases}
    %     d'_i = d_i, &\text{ if  $i \notin \{u, v\}$ }\\
    %     d'_u = d_u - 1 \\
    %     d'_v = d_v - 1 
    % \end{cases}
    %
    % d'_i &= d_i, &\mbox{ if  }i \notin \{u, v\};\\
    %d'_i &= d_i, &\forall\ i \notin \{u, v\};\\
    %d'_u &= d_u - 1 ;\\
    %d'_v &= d_v - 1 ;
%\end{align*}
and $\call'$ is obtained from $\call$ by replacing $(\set{u, v}, d_u + d_v - 2)$ with $(\set{u, v}, d_u + d_v)$.


The resulting instance $(\texttt{d}', \call')$ has a realization if and only if $(\texttt{d}, \call)$ has a realization. If $G = (V, E)$ is a realization of $(\texttt{d}, \call)$, then, as discussed above, we must have $uv \in E$, and $G - uv$ is a realization of $\call'$. Conversely, if $G' = (V, E')$ is a realization of $(\texttt{d}', \call')$, then necessarily $uv \notin E'$ due to the cut $(\set{u, v}, d_u + d_v)$, and $G' + uv$ is a realization of $(\texttt{d}, \call)$.

Thus, cut restrictions involving sets of size two can be simply reinterpreted as forbidding edges.
%
Let $F$ be the set of all forbidden edges that cannot appear in any realization of instance $(\texttt{d}, \call)$. Then $\calg = K_n - F$ is what we call the \emph{possibility graph}, which must be a supergraph of any valid realization of~$(\texttt{d}, \call)$.


\begin{figure*}[ht]
    \centering
    \includegraphics[width=0.95\textwidth]{quantitative_1_2.jpg}
    \vspace{-5mm}
    \caption{Quantitative comparisons for Scenario 1 and Scenario 2. The simulation setup is illustrated on the left. We compare the prediction accuracy using ADE, and the efficiency and safety in resulting robot plans using Detour and Minimum Distance.}
    % AToM achieves the best accuracy and results in improving efficiency while maintaining safety.}
    \vspace{-5mm}
    \label{fig:quantitative_12}
\end{figure*}

\section{Methodology}
\textbf{Problem Definition.} 
Let the state and control of an agent be $\boldsymbol{x} \in \mathds{R}^{n}$ and $\boldsymbol{u} \in \mathds{R}^{m}$, where $n, m$ are state and control dimensions. 
We use subscripts $H, R, HR$ to represent human, robot, and their joint system, and superscript $k$ to represent timesteps.
The dynamics can be described by $\boldsymbol{x}^{k+1} = \Dynamics(\boldsymbol{x}^{k}, \boldsymbol{u}^{k})$.
The trajectories from timestep $i$ to timestep $j$ can be represented by $\boldsymbol{X}^{i:j} = [\boldsymbol{x}^{i}, \dots, \boldsymbol{x}^{j}]$.
In a long-term interaction setting, we assume that the goal positions and environmental obstacles are known as $\boldsymbol{g}_{HR}$ and $\mathcal{O}$, respectively.
At each timestep $t$, given the observed historical trajectories $\boldsymbol{X}_{HR}^{0:t}$ and known information $\boldsymbol{g}_{HR}$ and $\mathcal{O}$, our goal is to predict future human trajectories $\hat{\boldsymbol{X}}_{H}^{t+1:t+T_{f}}$ with prediction horizon $T_{f}$, which the robot can use for collision-free predictive planning. 
For simplicity, we will omit time superscripts when no confusion is aroused. Our method is planner-agnostic and the robot planning algorithm is outside the scope of this work. 
% We omit the time superscripts for predicted human trajectories $\hat{\boldsymbol{X}}_{H}$ in the rest of the paper for simplicity. Our method is planner-agnostic and the robot planning algorithm is outside the scope of this work. 

% Let the state of the human and the robot be $\boldsymbol{x}_{H} \in \mathds{R}^{n_{H}}$ and $\boldsymbol{x}_{R} \in \mathds{R}^{n_{R}}$, where $n_{H}, n_{R}$ are state dimensions. 
% Let their joint state be $\boldsymbol{x}_{HR} \in \mathds{R}^{n_{H} + n_{R}}$.
% Let the control of the human and the robot be $\boldsymbol{u}_{H} \in \mathds{R}^{m_{H}}$ and $\boldsymbol{u}_{R} \in \mathds{R}^{m_{R}}$, where $m_{H}, m_{R}$ are control dimensions. 
% The dynamics can be described by $\dot{\boldsymbol{x}}_{H} = \Dynamics_{H}(\boldsymbol{x}_{H}, \boldsymbol{u}_{H})$ and $\dot{\boldsymbol{x}}_{R} = \Dynamics_{R}(\boldsymbol{x}_{R}, \boldsymbol{u}_{R})$. 
% In a long-term interaction setting, we assume the goal positions are known for both agents as $g_{H}$ and $g_{R}$. 
% The human and robot share the knowledge of all agents' historical trajectories up to the current step $\boldsymbol{Z}_{HR} = [\boldsymbol{x}_{HR}^{0}, \dots, \boldsymbol{x}_{HR}^{t}]$, as well as the environmental obstacles $\mathcal{O}$. 
% At each step $t$, our goal is to predict future human trajectories $\hat{\boldsymbol{X}}_{H} = [\hat{\boldsymbol{x}}_{H}^{t+1}, \dots, \hat{\boldsymbol{x}}_{H}^{t+T_{pred}}]$ with prediction horizon $T_{pred}$, which the robot can use for collision-free predictive planning. Our method is planner-agnostic and the robot planning algorithm is outside the scope of this work. 

\textbf{Overview.} 
We first present the original nested ToM formulation in robot predictive planning. We then propose our reformulated adaptive ToM (AToM) which detaches the human prediction model from the recursive structure. The human model consists of a game-theoretic solver and a parameter update mechanism, which captures dynamic human behaviours in long-term interactions.

\subsection{Original ToM in Robot Planning}
Following prior works mentioned in Sec. \ref{sec:ToM}, we formulate the original ToM in robot predictive planning task and identify the potential issues with this formulation.

The robot plans with a finite horizon $T_{p}$, where at each timestep t the robot optimises its controls $\boldsymbol{U}_{R} = [\boldsymbol{u}_{R}^{t+1}, \dots, \boldsymbol{u}_{R}^{t+T_{p}}]$ to minimize the cumulative cost:
\begin{equation}
\label{eq:robot_planning}
\begin{aligned}
    \min &\; \Cost_{R}(\boldsymbol{x}_{R}^{t+1:t+T_{p}}, \boldsymbol{x}_{R}^{t}, \hat{\boldsymbol{X}}_{H}, \boldsymbol{g}_{R}, \mathcal{O}), \\
    \text{s.t.} \quad &\boldsymbol{x}_{R}^{k+1} = \Dynamics_{R}(\boldsymbol{x}_{R}^{k}, \boldsymbol{u}_{R}^{k}), \\
                      &\Constraint_{R}(\boldsymbol{x}_{R}^{k}, \boldsymbol{u}_{R}^{k}, \boldsymbol{x}_{H}^{k}, \mathcal{O}) \leq 0, \\
                      &k = t+1, \dots, t+T_{p},
\end{aligned}
\end{equation}
where $\Cost_{R}$ is the cost function that captures the robot's performance goals, 
$\Constraint_{R}$ contains constraints such as collision-avoidance, 
and predicted human trajectories $\hat{\boldsymbol{X}}_{H}$ can be obtained from human prediction models such as the Social Force or prediction neural networks, as described in Sec. \ref{seq:traj_prediction}.

% Human predictions $\hat{X}_{H}$ can be obtained from historical observations and other additional information such as human goal and environmental obstacles: ({\color{blue} can introduce the parameter here and remove equation \eqref{eq:humanpred2}})
% \begin{equation}
% \label{eq:human_pred}
%     \tilde{X}_{H} = f_{pred} (X_{HR}, g_{H}, \mathcal{O}),
% \end{equation}
% where $f_{pred}$ is the prediction method such as the Social Force model or neural networks. 

To obtain $\hat{\boldsymbol{X}}_{H}$ using a ToM model, predicted human actions need to be solved from predicted robot trajectories:
\begin{equation}
\label{eq:original_ToM}
\begin{aligned}
    \boldsymbol{\hat{U}}_{H} = \argmin_{\boldsymbol{u}_H^{t+1:t+T_{f}}} &\Cost_{H}(\boldsymbol{x}_{H}^{t+1:t+T_{f}}, \boldsymbol{x}_{HR}^{t}, \hat{\boldsymbol{X}}_{R}, \boldsymbol{g}_{H}, \mathcal{O}), \\
    \text{s.t.} \quad &\boldsymbol{x}_{H}^{k+1} = \Dynamics_{H}(\boldsymbol{x}_{H}^{k}, \boldsymbol{u}_{H}^{k}), \\
                      &\boldsymbol{\hat{x}}_{R}^{k+1} = \Dynamics_{R}(\boldsymbol{\hat{x}}_{R}^{k}, \boldsymbol{u}_{R}^{k}), \\
                      &\Constraint_{H}(\boldsymbol{x}_{H}^{k}, \boldsymbol{u}_{H}^{k}, \boldsymbol{x}_{R}^{k}, \mathcal{O}) \leq 0, \\
                      &k = t+1, \dots, t+T_{f}. \\
\end{aligned}
\end{equation}
% In the original ToM formulation, future human actions are optimised similarly to robot predictive planning:
% \begin{equation}
% \label{eq:original_ToM}
% \begin{aligned}
%     \min_{u_H^{t+1:t+T_{pred}}} &\sum_{i=t+1}^{t+T_{pred}} J_{H}(x_{H}^{i}, g_{H}, \tilde{X}_{R}, \mathcal{O}), \\
%     \text{s.t.} \quad &x_{H}^{i+1} = x_{H}^{i} + f_{H}(x_{H}^{i}, u_{H}^{i}), \\
%                       &x_{R}^{i+1} = x_{R}^{i} + f_{R}(x_{R}^{i}, u_{R}^{i}), \\
%                       &C(x_{H}^{i}, u_{H}^{i}, x_{R}^{i}) \leq 0, \\
%                       &i = t+1, \dots, t+T_{pred},
% \end{aligned}
% \end{equation}
Note that the predicted robot trajectories are obtained from the robot plans $\boldsymbol{U}_{R}$ solved from Problem \eqref{eq:robot_planning}. 
If we substitute Eq. \eqref{eq:original_ToM} and human dynamics $\Dynamics_{H}$ back to the cost in Problem \eqref{eq:robot_planning}, it results in a recursive optimisation problem for the robot. 
Solving such optimisation is tractable for simple settings as discussed in Sec. \ref{sec:ToM}, but complex and computationally expensive for robot planning with continuous and infinite action space. 
Therefore, we propose a reformulation of the original ToM to detach the human prediction model from the nested optimisation problem.

% \begin{figure}
%     \centering
%     \includegraphics[width=\linewidth]{quantitative_3.jpg}
%     \vspace{-8mm}
%     \caption{Quantitative comparison between AToM and SF in Scenario 3. We compare the number of steps the robot takes to reach the goal which reflects the efficiency. We mark the rounds where collisions happens using red crosses.}
%     % AToM enables the robot to pass through the doorway safely and efficiently in all rounds of experiments.}
%     \vspace{-5mm}
%     \label{fig:quantitative_3}
% \end{figure}

\subsection{Reformulation and Adaptive ToM (AToM)}
Instead of using the true robot plan as human-predicted robot actions, we argue that the human maintains an internal model of the navigation problem and they predict the optimal actions for all agents based on a dynamic belief about each agent's behavioural pattern. 
We formulate the human internal model as a multi-player general-sum differential game with finite horizon $T_{f}$. 
In a navigation scenario, the strategy for player $i$ is a control sequence $\boldsymbol{U}_{i} = [\boldsymbol{u}_{i}^{t+1}, \dots, \boldsymbol{u}_{i}^{t+T_{f}}]$. 
Each player seeks to minimize its cost while respecting constraints and dynamics:
\begin{equation}
\label{eq:game_cost}
\begin{aligned}
    \min_{U} \quad &\sum_{k=t+1}^{t+T_{f}} 
    (\boldsymbol{x}_{i}^{k} - \boldsymbol{g})^\top Q (\boldsymbol{x}_{i}^{k} - \boldsymbol{g}) 
    + \boldsymbol{u}_{i}^{k}{}^\top R \boldsymbol{u}_{i}^{k} \\
    +  &w_{s}(\sum_{n \neq i} \max (0, \|\boldsymbol{x}^{k}_{i} - \boldsymbol{x}^{k}_{n}\|_{2} - d_{s}))) \\
    + &w_{o}(\max (0, D(\boldsymbol{x}^{k}_{i}, \mathcal{O}) - d_{o}))\\
    \text{s.t.} \quad &\boldsymbol{x}_{min} \leq \boldsymbol{x}_{i}^{k} \leq \boldsymbol{x}_{max}, \\
                &\boldsymbol{u}_{min} \leq \boldsymbol{u}_{i}^{k} \leq \boldsymbol{u}_{max}, \\
                &\boldsymbol{x}_{i}^{k+1} = \Dynamics_{i}(\boldsymbol{x}_{i}^{k}, \boldsymbol{u}_{i}^{k}), \\
                &k = t+1, \dots, t+T_{f}, \\
\end{aligned}
\end{equation}
% \begin{equation}
% \label{eq:game_cost}
% \begin{aligned}
%     \min_{U} \quad & \Cost_{goal}(\boldsymbol{X}, \boldsymbol{g}, \theta_{1}) + \Cost_{speed}(\boldsymbol{X}, \theta_{2}) \\
%                    &+ \Cost_{obs}(\boldsymbol{X}, \mathcal{O}, \theta_{3}) + \Cost_{social}(\boldsymbol{X}_{joint}, \theta_{4}), \\
%     \text{s.t.} \quad &\Constraint(\boldsymbol{X}_{joint}, \boldsymbol{U}, \mathcal{O}, \theta_{5}) \leq 0, \\
%                 &\boldsymbol{x}^{k+1} = \Dynamics(\boldsymbol{x}^{k}, \boldsymbol{u}^{k}), \\
%                 &k = t+1, \dots, t+T_{f}, \\
% \end{aligned}
% \end{equation}
% where the 4 cost components are $\Cost_{goal}$ for goal reaching, $\Cost_{speed}$ for speed regulation, $\Cost_{obs}$ for obstacle avoidance, and $\Cost_{social}$ for social distancing. 
% $\boldsymbol{\theta} = [\theta_{1}, \theta_{2}, \theta_{3}, \theta_{4}, \theta_{5}]^\top$ are the parameters from the costs and constraints 
% such as the preferred speed in $\Cost_{speed}$, and preferred distance from neighbours in $\Cost_{social}$. 
where $Q \geq 0$ and $R > 0$ are the weights for the state and control, $w_{s}$ and $w_{o}$ are the weights for social and obstacle avoidance, $D$ calculates the closest distance to obstacle $\mathcal{O}$, 
$d_{s}$ and $d_{o}$ are the preferred social and obstacle distances, $\boldsymbol{x}_{min}$, $\boldsymbol{x}_{max}$, $\boldsymbol{u}_{min}$, and $\boldsymbol{u}_{max}$ are the state and control limits.
We represent these behavioural parameters for all players using $\boldsymbol{\theta}$.
% $\boldsymbol{\theta} = [w_{s}, d, w_{o}, d_{o}, \boldsymbol{x}_{min}, \boldsymbol{x}_{max}, \boldsymbol{u}_{min}, \boldsymbol{u}_{max}]^\top$ are the parameters from the costs and constraints such as the weights, preferred distance from other agents and obstacles, and limits for the states and controls.
% These parameters model the behaviours of each player in a game-theoretic setting.

\begin{figure}
    \centering
    \includegraphics[width=\linewidth]{quantitative_3.jpg}
    \vspace{-8mm}
    \caption{Quantitative comparison between AToM and SF in Scenario 3. We compare the number of steps the robot takes to reach the goal which reflects the efficiency. SF leads to collisions in 7 rounds, which we highlighted using red crosses.}
    % We mark the rounds where collisions happen using red crosses. AToM enables the robot to pass through the doorway efficiently and safely in all rounds of experiments.}
    \vspace{-5mm}
    \label{fig:quantitative_3}
\end{figure}


We then find a Nash equilibrium solution $\tilde{\boldsymbol{U}}_{1}, \dots, \tilde{\boldsymbol{U}}_{n}$ for $n$ players with each player's cost defined in Problem \eqref{eq:game_cost}.
At a Nash equilibrium point, no player $i$ can further decrease its cost by unilaterally changing its strategy $\tilde{\boldsymbol{U}}_{i}$ \cite{facchinei2010generalized}. 
Solving this generalized Nash equilibrium problem can be done using existing dynamic game solvers. In this work, we use ILQSolver \cite{fridovich2020efficient} which iteratively solves linear-quadratic approximations of the original differential game in multi-player settings. 
We consider it an algorithmic module $\GameSolver$ parameterized by $\boldsymbol{\theta}$ that takes as input the current joint state of the system, the known goal positions and environmental obstacles, and returns an open-loop joint strategy that satisfies global Nash equilibrium:
\begin{equation}
\label{eq:our_ToM}
    \tilde{\boldsymbol{U}}_{H}, \tilde{\boldsymbol{U}}_{R} = \GameSolver(\boldsymbol{x}_{HR}^{t}, \boldsymbol{g}_{HR}, \mathcal{O}, \hat{\boldsymbol{\theta}}),
\end{equation}
where $\tilde{\boldsymbol{U}}_{H}$ is the predicted human action and equivalent to $\boldsymbol{\hat{U}}_{H}$, 
$\tilde{\boldsymbol{U}}_{R}$ is the human-predicted robot action, 
and $\hat{\boldsymbol{\theta}}$ is the human-predicted behavioural parameters. 
With this reformulation, we can now easily obtain $\hat{\boldsymbol{X}}_{H}$ from Eq. \eqref{eq:our_ToM} and human dynamics $\Dynamics_{H}$, and therefore solve Problem \eqref{eq:robot_planning} without any recursive structure.
% With this reformulation, we can now substitute Eq. \eqref{eq:our_ToM} and human dynamics back to the robot predictive planning problem in Eq. \eqref{eq:robot_planning} without any recursive structure.

Existing game-theoretic robot planners use similar formulations to solve for the optimal robot plans using known cost functions for each agent \cite{tian2022safety, le2021lucidgames, hu2023emergent}. 
A fundamental difference in this work is that our human internal model $\GameSolver$ is parameterized by $\hat{\boldsymbol{\theta}}$, which represents a dynamic human belief instead of the true agent parameters.

After obtaining predicted human trajectories $\hat{\boldsymbol{X}}_{H}$, the downstream robot planner can now perform predictive planning and execute the actual robot plans. 
As we do not assume any leader-follower structure, the human simultaneously performs her/his actions. 
Up to this point, we have obtained four sets of trajectories: 
1) Predicted Human $\hat{\boldsymbol{X}}_{H}$, 
2) Human-predicted Robot $\hat{\boldsymbol{X}}_{R}$, which can be obtained from $\tilde{\boldsymbol{U}}_{R}$
3) Observed Human $\boldsymbol{X}_{H}$, 
4) Observed Robot $\boldsymbol{X}_{R}$. 
These predicted-observed pairs can be used to correct the estimated behavioural parameters $\hat{\boldsymbol{\theta}}$ in the human internal model.

We use Unscented Kalman Filter (UKF) \cite{wan2000unscented} as the update mechanism. We allow $\hat{\boldsymbol{\theta}}$ to evolve using a random walk as the process model $\ProcessModel$. The measurement states are the agents' trajectories and the measurement model $\MeasurementModel$ is therefore the game solver $\GameSolver$ combined with agent dynamics.
\begin{equation}
\begin{aligned}
    \boldsymbol{\theta}^{t+1} &= \ProcessModel(\boldsymbol{\theta}^{t}, \delta^{t}) = \boldsymbol{\theta}^{t} + \delta^{t}, \quad \delta^t \sim \mathcal{N}(0, Q_t), \\
    \boldsymbol{x}^{t+1} &= \MeasurementModel(\boldsymbol{x}^{t}, \boldsymbol{\theta}^{t}) \\
            &= \Dynamics(\boldsymbol{x}^{t}, \GameSolver(\boldsymbol{x}^{t}, \boldsymbol{\theta}^{t})) + \epsilon^{t}, \quad \epsilon^t \sim \mathcal{N}(0, R),
\end{aligned}
\end{equation}
where $Q_{t}$ and $R$ are the covariance for the process model and measurement model.
In this way, our adaptive ToM human model can be adjusted dynamically to improve its prediction accuracy, and to reflect how humans update their beliefs on others. The complete procedure is detailed in Algorithm \ref{alg:our_method}, where $\Sigma$ is the covariance of $\theta$ estimation.
At the observation step, the robot executes the planned action from the predictive planner and the true human motion is observed.

% \vspace{-1mm}
\begin{algorithm}
\caption{Predict-Observe-Update Procedures with AToM}
\label{alg:our_method}
\begin{algorithmic}[1]
\State \textbf{Inputs:} $\boldsymbol{x}_{HR}^{t}, \hat{\boldsymbol{\theta}}^{t}, \Sigma^{t}, \boldsymbol{g}_{HR}, \mathcal{O}$
\State \textbf{for} $k = t, t+1, \dots$ \textbf{do}
    \State \quad $\tilde{\boldsymbol{U}}_{HR} \gets \GameSolver(\boldsymbol{x}_{HR}^{k}, \boldsymbol{g}_{HR}, \mathcal{O}, \hat{\boldsymbol{\theta}}^{k})$ 
    \State \quad $\hat{\boldsymbol{X}}_{HR} \gets \Dynamics_{HR}(\boldsymbol{x}_{HR}^{k}, \tilde{\boldsymbol{U}}_{HR})$\Comment{Predict}
    \State \quad $\boldsymbol{x}_{R}^{k+1} \gets \text{RobotPlanner}(\boldsymbol{x}_{R}^{k}, \hat{\boldsymbol{X}}_{H}, \boldsymbol{g}_{R}, \mathcal{O})$
    % \State \quad $\boldsymbol{x}_{H}^{k+1} \gets \text{Human}$ \Comment{Observation}
    \State \quad observe human state $\boldsymbol{x}_{H}^{k+1}$ \Comment{Observe}
    \State \quad $\hat{\boldsymbol{\theta}}^{k+1}, \Sigma^{k+1} \gets $
    \Statex \quad $\text{UKF}(\boldsymbol{x}_{HR}^{k+1}, \hat{\boldsymbol{x}}_{HR}^{k+1}, \hat{\boldsymbol{\theta}}^{k}, \Sigma^{k}, \ProcessModel, \MeasurementModel)$ \Comment{Update}
\State \textbf{end for}
\end{algorithmic}
\end{algorithm}
% \vspace{-2mm}

% \begin{equation}
% \begin{aligned}
%     \theta^{t+1} &= \ProcessModel(\theta^{t}, \delta^{t}) \\
%                  &= \theta^{t} + \delta^{t}, \quad \delta^t \sim \mathcal{N}(0, Q_t), \\
%     X^{t+1} &= \MeasurementModel(X^{t}, \theta^{t}) \\
%             &= X^{t} + F(X^{t}, GameSolver_{\theta^{t}}(X^{t})).
% \end{aligned}
% \end{equation}





\section{Experiments}
\seclabel{experiments}
Our experiments are designed to test a) the extent to which open loop execution is an issue for precise mobile manipulation tasks, b) how effective are blind proprioceptive correction techniques, c) do object detectors and point trackers perform reliably enough in wrist camera images for reliable control, d) is occlusion by the end-effector an issue and how effectively can it be mitigated through the use of video in-painting models, and e) how does our proposed \name methodology compare to large-scale imitation learning? 


\subsection{Tasks and Experimental Setup}
We work with the Stretch RE2 robot. Stretch RE2 is a commodity mobile manipulator with a 5DOF arm mounted on top of a non-holomonic base. We upgrade the robot to use the Dex Wrist 3, which has an eye-in-hand RGB-D camera (Intel D405). 
We consider 3 task families for a total
of 6 different tasks: a) holding a knob to pull open a cabinet or drawer, b) holding a
handle to pull open a cabinet, and c) pushing on objects (light buttons, books
in a book shelf, and light switches). Our focus is on generalization. {\it
Therefore, we exclusively test on previously unseen instances, not used during
development in any way.} 
\figref{tasks} shows the instances that we test on. 

All tasks involve some precise manipulation, followed by execution of a motion
primitive. {\bf For the pushing tasks}, the precise motion is to get the
end-effector exactly at the indicated point and the motion primitive is to push
in the direction perpendicular to the surface and retract the end-effector 
upon contact. The robot is positioned such
that the target position is within the field of view of the wrist camera. A user
selects the point of pushing via a mouse click on the wrist camera image. The
goal is to push at the indicated location. Success is determined by whether the
push results in the desired outcome (light turns on / off or book gets pushed in). 
The original rubber gripper bends upon contact, we use a rigid known tool
that sticks out a bit. We take the geometry of the tool into account while servoing.

{\bf For the opening articulated object tasks}, the precise manipulation is grasping the
knob / handle, while the motion primitive is the whole-body motion that opens
the cupboard. Computing and executing this full body motion is difficult. We
adopt the modular approach to opening articulated objects (MOSART) from Gupta \etal~\cite{gupta2024opening} and invoke it
after the gripper has been placed around the knob / handle. The whole tasks 
starts out with the robot about 1.5m way from the target object, with the 
target object in view
from robot's head mounted camera. We use MOSART to compute articulation
parameters and convey the robot to a pre-grasp
location with the target handle in view of the wrist camera. At this point,
\name (or baseline) is used to center the gripper around the knob / handle, 
before resuming MOSART: extending the gripper till contact, close the gripper, and play rest of the predicted motion plan. Success is 
determined by whether the cabinet opens by more than $60^\circ$
or the drawer is pulled out by more than $24cm$, similar to the criteria used in \cite{gupta2024opening}.


For the precise manipulation part, all baselines consume the current and
previous RGB-D images from the wrist camera and output full body motor
commands.

% % Please add the following required packages to your document preamble:
% % \usepackage{graphicx}
% \begin{table*}[!ht]
% \centering
% \caption{}
% \label{tab:my-table}
% \resizebox{\textwidth}{!}{%
% \begin{tabular}{lcccccc}
% \toprule
%  & \multicolumn{2}{c}{ours} & \multicolumn{2}{c}{Gurobi} & \multicolumn{2}{c}{MOSEK} \\
%  & \multicolumn{1}{l}{time (s)} & \multicolumn{1}{l}{optimality gap (\%)} & \multicolumn{1}{l}{time (s)} & \multicolumn{1}{l}{optimality gap (\%)} & \multicolumn{1}{l}{time (s)} & \multicolumn{1}{l}{optimality gap (\%)} \\ \hline
% \begin{tabular}[c]{@{}l@{}}Linear Regression\\ Synthetic \\ (n=16000, p=16000)\end{tabular} & 57 & 0.0 & 3351 & - & 2148 & - \\ \hline
% \begin{tabular}[c]{@{}l@{}}Linear Regression\\ Cancer Drug Response\\ (n=822, p=2300)\end{tabular} & 47 & 0.0 & 1800 & 0.31 & 212 & 0.0 \\ \hline
% \begin{tabular}[c]{@{}l@{}}Logistic Regression\\ Synthetic\\ (n=16000, p=16000)\end{tabular} & 271 & 0.0 & N/A & N/A & 1800 & - \\ \hline
% \begin{tabular}[c]{@{}l@{}}Logistic Regression\\ Dorothea\\ (n=1150, p=91598)\end{tabular} & 62 & 0.0 & N/A & N/A & 600 & 0.0 \\
% \bottomrule
% \end{tabular}%
% }
% \end{table*}

% Please add the following required packages to your document preamble:
% \usepackage{multirow}
% \usepackage{graphicx}
\begin{table*}[]
\centering
\caption{Certifying optimality on large-scale and real-world datasets.}
\vspace{2mm}
\label{tab:my-table}
\resizebox{\textwidth}{!}{%
\begin{tabular}{llcccccc}
\toprule
 &  & \multicolumn{2}{c}{ours} & \multicolumn{2}{c}{Gurobi} & \multicolumn{2}{c}{MOSEK} \\
 &  & time (s) & opt. gap (\%) & time (s) & opt. gap (\%) & time (s) & opt. gap (\%) \\ \hline
\multirow{2}{*}{Linear Regression} & \begin{tabular}[c]{@{}l@{}}synthetic ($k=10, M=2$)\\ (n=16k, p=16k, seed=0)\end{tabular} & 79 & 0.0 & 1800 & - & 1915 & - \\ \cline{2-8}
 & \begin{tabular}[c]{@{}l@{}}Cancer Drug Response ($k=5, M=5$)\\ (n=822, p=2300)\end{tabular} & 41 & 0.0 & 1800 & 0.89 & 188 & 0.0 \\ \hline
\multirow{2}{*}{Logistic Regression} & \begin{tabular}[c]{@{}l@{}}Synthetic ($k=10, M=2$)\\ (n=16k, p=16k, seed=0)\end{tabular} & 626 & 0.0 & N/A & N/A & 2446 & - \\ \cline{2-8}
 & \begin{tabular}[c]{@{}l@{}}DOROTHEA ($k=15, M=2$)\\ (n=1150, p=91598)\end{tabular} & 91 & 0.0 & N/A & N/A & 634 & 0.0 \\
 \bottomrule
\end{tabular}%
}
% \vspace{-3mm}
\end{table*}

\begin{figure*}
\insertW{1.0}{figures/figure_6_cropped_brighten.pdf}
\caption{{\bf Comparison of \name with the open loop (eye-in-hand) baseline} for opening a cabinet with a knob. Slight errors in getting to the target cause the end-effector to slip off, leading to failure for the baseline, where as our method is able to successfully complete the task.}
\figlabel{rollout}
\end{figure*}

\begin{table}
\setlength{\tabcolsep}{8pt}
  \centering
  \resizebox{\linewidth}{!}{
  \begin{tabular}{lcccg}
  \toprule
                              & \multicolumn{2}{c}{\bf Knobs} & \bf Handle & \bf \multirow{2}{*}{\bf Total} \\
                              \cmidrule(lr){2-3} \cmidrule(lr){4-4}
                              & \bf Cabinets & \bf Drawer & \bf Cabinets & \\
  \midrule
  RUM~\cite{etukuru2024robot}  & 0/3    & 1/4         & 1/3         & 2/10 \\
  \name (Ours) & 2/3    & 2/4         & 3/3     &  7/10 \\
  \bottomrule
  \end{tabular}}
  \caption{Comparison of \name \vs RUM~\cite{etukuru2024robot}, a recent large-scale end-to-end imitation learning method trained on 1200 demos for opening cabinets and 525 demos for opening drawers across 40 different environments. Our evaluation spans objects from three environments across two buildings.}
  \tablelabel{rum}
\end{table}

\subsection{Baselines}
We compare against three other methods for the precise manipulation part of
these tasks. 
\subsubsection{Open Loop (Eye-in-Hand)} To assess the precision requirements of
the tasks and to set it in context with the manipulation capabilities of the
robot platform, this baseline uses open loop execution starting from estimates
for the 3D target position from the first wrist camera image.
\subsubsection{MOSART~\cite{gupta2024opening}}
The recent modular system for opening cabinets and drawers~\cite{gupta2024opening}
reports impressive performance with open-loop control (using the head camera from 1.5m away), combined with proprioception-based feedback to 
compensate for errors in perception and control when interacting with handles. 
We test if such correction is also sufficient for interacting with knobs. Note 
that such correction is not possible for the smaller buttons and pliable books.

\subsubsection{\name (no inpainting)} To understand how much of an issue
occlusion due to the end-effector is during manipulation, we ablate the use of
inpainting. %

\subsubsection{Robot Utility Models (RUM)~\cite{etukuru2024robot}}
For the opening articulated object tasks, we also compare to Robot Utility Models (RUM), 
a closed-loop imitation learning method recently proposed by Etukuru et al. \cite{etukuru2024robot}.
RUM is trained on a substantial dataset comprising expert demonstrations, including 
1,200 instances of cabinet opening and 525 of drawer opening, gathered from roughly 
40 different environments.
This dataset stands as the most extensive imitation 
learning dataset for articulated object manipulation to date, establishing RUM as a 
strong baseline for our evaluation.

Similar to our method, we use MOSART to compute articulation
parameters and convey the robot to a pre-grasp location
with the target handle in view of the wrist camera.
One of the assumptions of RUM is a good view of the handle.
To benefit RUM, we try out three different heights of the wrist camera,
and \textit{report the best result for RUM.}

\begin{figure*}
\insertW{1.0}{figures/figure_9_cropped_brighten.pdf}
\caption{{\bf \name \vs open loop (eye-in-hand) baseline for pushing on user-clicked points}. Slight errors in getting to the target cause failure, where as \name successfully turns the lights off. Note the quality of CoTracker's track ({\color{blue} blue dot}).}
\figlabel{rollout_v2}
\end{figure*}

\begin{figure*}
\insertW{1.0}{figures/figure_5_v2_cropped_brighten.pdf}
\caption{{\bf Comparison of \name with and without inpainting}. Erroneous detection without inpainting causes execution to fail, where as with inpainting the target is correctly detected leading to a successful grasp and a successful execution.}
\figlabel{rollouts2}
\end{figure*}


\subsection{Results}
\tableref{results} presents results from our experiments. 
Our training-free approach \name successfully 
solves over 85\% of task instances that we test on.
As noted, all these
tests were conducted on unseen object instances in unseen
environments that were not used for development in any way. We discuss our key
experimental findings below.

\subsubsection{Closing the loop is necessary for these precise tasks} 
While the proprioception-based strategies proposed in MOSART~\cite{gupta2024opening}
work out for handles, they are inadequate for targets like knobs and just
don't work for tasks like pushing buttons. Using estimates from the wrist
camera is better, but open loop execution still fails for knobs and pushing
buttons. 

\subsubsection{Vision models work reasonably well even on wrist camera images}
Inpainting works well on wrist camera images (see \figref{occlusion} and \figref{inpainting}).
Closing the loop using feedback from vision detectors and point trackers on
wrist camera images also work well, particularly when we use in-painted images.
See some examples detections and point tracks in \figref{rollout} and \figref{rollout_v2}. 
Detic~\cite{zhou2022detecting} was able to reliably detect the knobs and
handles and CoTracker~\cite{karaev2023cotracker} was able to successfully track
the point of interaction letting us solve 24/28 task instances.

\subsubsection{Erroneous detections without inpainting hamper performance on 
handles and our end-effector out-painting strategy effectively mitigates it} 
As shown in \figref{rollouts2}, presence of the end-effector caused the object
detector to miss fire leading to failed execution. Our out painting approach
mitigates this issue leading to a higher success rate than the 
approach without out-painting. Interestingly, CoTracker~\cite{karaev2023cotracker} is quite robust
to occlusion (possibly because it tracks multiple points) and doesn't benefit
from in-painting. 


\subsubsection{Closed-loop imitation learning struggles on novel objects}
As presented in \tableref{rum}, \name significantly outperforms RUM in a paired evaluation on unseen objects across three novel environments. A common failure mode of RUM is its inability to grasp the object's handle, even when it approaches it closely.
Another failure mode we observe is RUM misidentifying keyholes or cabinet edges as handles, also resulting in failed grasp attempts.
These result demonstrate that a modular approach that leverages the broad generalization capabilities of vision foundation models is able to generalize much better than an end-to-end imitation learning approach trained on 1000+ demonstrations, which must learn all aspects of the task from scratch.




\section{Conclusion }
This paper introduces the Latent Radiance Field (LRF), which to our knowledge, is the first work to construct radiance field representations directly in the 2D latent space for 3D reconstruction. We present a novel framework for incorporating 3D awareness into 2D representation learning, featuring a correspondence-aware autoencoding method and a VAE-Radiance Field (VAE-RF) alignment strategy to bridge the domain gap between the 2D latent space and the natural 3D space, thereby significantly enhancing the visual quality of our LRF.
Future work will focus on incorporating our method with more compact 3D representations, efficient NVS, few-shot NVS in latent space, as well as exploring its application with potential 3D latent diffusion models.


%%%%%%%%%%%%%%%%%%%%%%%%%%%%%%%%%%%%%%%%%%%%%%%%%%%%%%%%%%%%%%%%%%%%%%%%%%%%%%%%

% This project has received funding from the European Union through ERC, INTERACT, under Grant 101041863. Views and opinions expressed are however those of the author(s) only and do not necessarily reflect those of the European
% Union. Neither the European Union nor the granting authority can be held responsible for them.

\printbibliography[heading=bibintoc,title=References]

\end{document}
