\begin{abstract}
Task specification for robotic manipulation in open-world environments is challenging, requiring flexible and adaptive objectives that align with human intentions and can evolve through iterative feedback. We introduce \algname (\algabrvname), a visually grounded, Python-based reward function that serves as a dynamic task specification. Our framework leverages VLMs to generate and refine these reward functions for multi-step manipulation tasks. Given RGB-D observations and free-form language instructions, we sample keypoints in the scene and generate a reward function conditioned on these keypoints. IKER operates on the spatial relationships between keypoints, leveraging commonsense priors about the desired behaviors, and enabling precise SE(3) control. We reconstruct real-world scenes in simulation and use the generated rewards to train reinforcement learning (RL) policies, which are then deployed into the real world—forming a real-to-sim-to-real loop. Our approach demonstrates notable capabilities across diverse scenarios, including both prehensile and non-prehensile tasks, showcasing multi-step task execution, spontaneous error recovery, and on-the-fly strategy adjustments. The results highlight \algabrvname's effectiveness in enabling robots to perform multi-step tasks in dynamic environments through iterative reward shaping. Project Page: \url{https://iker-robot.github.io/}
\end{abstract}
