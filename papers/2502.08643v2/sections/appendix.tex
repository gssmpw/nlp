\begin{figure*}[h!]
    \setlength{\abovecaptionskip}{0pt}
    \includegraphics[width=\linewidth]{figures/unrolled.pdf}
    \vspace{0.1em}
    \caption{\small{\textbf{Examples of keypoint-marked images with corresponding predicted codes.} The top row represents the starting point, with subsequent rows illustrating the progression step by step.  The VLM first predicts to push the box to create space, followed by sequential placement of the shoes.}
    \label{fig:unrolled}}
    
\end{figure*}

\subsection{Grasping Subroutine}
\label{sec:grasping}

During training, the gripper fingers open only in the \textit{grasp} mode, where the end-effector approaches the object with open fingers and then closes them to grasp the object. We employ a heuristic-based grasp for faster training. In real-world, the gripper fingers remain closed until the grasp mode is triggered. AnyGrasp predicts an appropriate grasp pose and the fingers close at the predicted position. To address the sim-to-real gap, we add randomization to the heuristic grasp pose during simulation. This allows the policy to generalize more effectively, resulting in more robust and reliable policies in the real-world.

\subsection{Domain Randomization Parameters}
\label{sec:domain_randomization}
To enhance the robustness of our policies for effective real-to-sim-to-real transfer, we apply domain randomization to various object properties and initial conditions. Table~\ref{tab:domain_randomization} details the key randomized parameters and their respective ranges. These variations ensure that our learned policies generalize effectively to real-world conditions, mitigating the discrepancies between simulation and real-world.

\begin{table}[h]
    \centering
    \resizebox{0.75\linewidth}{!}{ %
    \begin{tabular}{l c}
        \toprule
        \textbf{Parameter} & \textbf{Range} \\
        \midrule
        Object Scale & [0.8, 1.2] \\
        Mass & [0.3, 2.0] \\
        Friction & [0.3, 1.8] \\
        Restitution & [0.0, 1.0] \\
        Compliance & [0.0, 1.0] \\
        Center of Mass Perturbation & [-0.05, 0.05] \\
        Initial Position Perturbation & [-0.02, 0.02] \\
        Initial Orientation Perturbation & [-0.05, 0.05] \\
        Grasp Position Noise & [-0.01, 0.01] \\
        Grasp Orientation Noise & [-0.2, 0.2] \\
        \bottomrule
    \end{tabular}
    } %
    \caption{\small{Domain randomization ranges for key object properties and initial conditions in simulation.}}
    \label{tab:domain_randomization}
    \vspace{-2em}
\end{table}




\subsection{VLM Prompts}
\label{sec:prompt}
The VLM receives the image overlaid with keypoints ${1, \ldots, K}$, along with the task description as text. These are given to the VLM, along with the prompt. We do not provide any in-context examples with the prompt. Our prompt for single-step tasks is as follows:



\lstinputlisting{sections/prompt1.md}

The prompt for multi-step tasks is as follows:

\lstinputlisting{sections/prompt2.md}




The prompt for baseline that uses pose input for single-step tasks is as follows:

\lstinputlisting{sections/prompt3.md}


\subsection{Case study of a complex task}
\begin{figure}[h]
    \setlength{\abovecaptionskip}{0pt}
    \includegraphics[width=\linewidth]{figures/case_study.pdf}
    \caption{\small{\textbf{Case study of a complex task with in-context examples.} The robot uses the environment to regrasp and stow the book. Then, the human updates the instructions to place it on the other shelf.}}
    \label{fig:case_study}
\end{figure}

We present results on a complex 3D understanding task. The task involves stowing a book on a shelf, where the book is initially positioned with only its shorter edge graspable. The instruction is to place the book on the shelf. However, the robot cannot place the book directly with the shorter edge grasped, as this would result in a collision between the book and the table due to the position of its arm. To complete this task, the robot must perform multiple steps: first, it needs to regrasp the book along its longer edge using some part of the environment, and only then can it stow the book on the shelf. After the robot places the book on the initial shelf, a human intervenes by adding an instruction to move the book to a different shelf.


Given the complexity of this long-horizon task, we employ in-context examples to guide the VLM. With this change,  our system is able to successfully perform the task. \figref{fig:case_study} illustrates the progression of the task.

