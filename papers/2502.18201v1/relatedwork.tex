\section{Related Work}
\subsection{Computer-Mediated Communication}
With the proliferation of computers, computer-mediated communication (CMC) has become ubiquitous. Video conferencing systems like Zoom~\footnote{\url{https://zoom.us/}} and text chat tools such as Messenger~\footnote{\url{https://www.messenger.com/}} and WeChat~\footnote{\url{https://www.wechat.com/}} are widely used in both personal and professional settings. Consequently, the characteristics of CMC have been extensively studied in previous research.

One of the most prominent theories related to CMC is the Media Richness Theory~\cite{MediaRichness1986}. This theory posits that different media used for communication convey varying amounts of information, impacting the effectiveness of communication. For example, video conferencing transmits messages through diverse verbal and nonverbal cues such as facial expressions, gestures, and vocal tone, providing rich information from the speaker and real-time feedback from the listener. Conversely, text chats lack most of the nonverbal cue available in richer mediums, which can potentially hinder the communicative process leading to more messages having to be shared to achieve similar relational outcomes~\cite{WALTHER1992h,WALTHER1994a}. Furthermore, the level of media richness influences impression formation between interlocutors. The Hyperpersonal Model~\cite{HyperPersonal1996} suggests in the absence of rich verbal and nonverbal cues message recipients tend to project idealized impressions onto the message sender. The scarcity of information about the message sender and the context of the message can also lead to misunderstandings regarding the intended emotional tone of the message~\cite{Byron2008-km}. Thus, media choice in CMC is not merely a means of information transmission but a crucial factor that significantly affects interpersonal impression formation.

\subsection{AI-Mediated Communication}

%The importance of the CMC systems becomes even more significant as AI technology pervades daily life. 
Against the backdrop of CMC systems' pervasiveness in daily life, the emerging field of AI-Mediated Communication (AIMC) has gained attention as an extension of CMC.Hancock et al.~\cite{Hancock2020} define AIMC as "mediated communication between people in which a computational agent operates on behalf of a communicator by modifying, augmenting, or generating messages to accomplish communication or interpersonal goals". They emphasize AIMC's rapid expansion, and advocate for its examination across various dimensions, including the impact of AIMC on language, interpersonal dynamics, self-presentation, impression formation, trust, feedback, as well as long-term relationship formation and maintenance. Hancock et al. propose five key dimensions to describe AIMC systems: magnitude, autonomy, media type, optimization goal, and role orientation. In this research, we focus particularly on two of these dimensions: \textit{magnitude}, which reflects "the extent of the changes that AI enacts on messages", and \textit{autonomy}—"the degree to which AI can operate on messages without the sender's supervision".

The magnitude and autonomy dimensions provide a framework for understanding various AIMC applications. While magnitude primarily pertains to the technical capacity of the AI system to alter messages, autonomy refers to the interaction design, specifically the level of control that users have over the transformations applied by AI. These two dimensions create a matrix of possible AIMC systems with distinct characteristics. (Table \ref{table:aimc_matrix})

\begin{itemize}
    
    \item \textbf{Low Magnitude, Low Autonomy}: Examples of this combination include tools such as Grammarly~\cite{Grammarly}, which make small, incremental changes to messages and allow users to retain control over the content. Previous research, such as that by Fu et al.~\cite{Fu2024FromText}, has explored the short-term and long-term effects of AI-powered writing assistance. Studies by Jakesch et al.~\cite{Jakesch2019AIMC}, Hohenstein et al.~\cite{Hohenstein2023AIMC}, and Mieczkowski et al.~\cite{Mieczkowski2021AIMC} have shown that these types of AIMC systems can introduce subtle biases and impact perceptions of trustworthiness.

    \item \textbf{High Magnitude, Low Autonomy}: Here, AIMC systems enact substantial transformations but maintain user oversight. This configuration includes technologies for complete voice or facial reconstruction and applications that entirely rewrite messages or transform input content into a different format. Examples include tools such as Gmail's Smart Reply~\cite{GmailSmartReply}, which can significantly alter message phrasing while allowing users to retain control. Algouzi et al.~\cite{Algouzi2023Gmail} investigated the sociocultural implications of using such tools. Harashima's work on intellectual coding~\cite{harashima1991intelligent} has also explored significant transformation based on encoding and decoding facial expressions and voice audios. While these systems can perform large modifications, they rarely employ high autonomy, and users usually retain control over significant alterations.
    
    \item \textbf{Low Magnitude, High Autonomy}: In this quadrant, the AI applies relatively minor modifications autonomously, without requiring user supervision for each transformation. For example, Arias-Sarah et al. ~\cite{sarah2024face}  modified the facial expressions in video conferences without notifying the participants. Certain creative applications have used simple algorithms to rephrase or adjust chat messages dynamically (e.g.,~\cite{mwitm}). Systems with this configuration are characterized by a limited degree of message transformation but function independently of constant user input, thus requiring higher autonomy than the previous category. Moreover, AI Clone Agents, as depicted in the science fiction series "Black Mirror~\footnote{\url{https://www.imdb.com/title/tt20247352/}}" and currently under development by various startups (e.g., Synthesia~\footnote{\url{https://www.synthesia.io/}}), can also be considered to be situated within this quadrant.
    
    \item \textbf{High Magnitude, High Autonomy}: In this category, the system autonomously performs extensive modifications on messages, often without user involvement in the process. In this case, each user experiences a significant transformation to their messages, and the notion of a single objective world shared by all participants is replaced by distinct subjective environments mediated by AI. As such, this configuration offers considerable potential for AI to mediate communication. However, research in this area remains limited. Our study proposes the \textbf{Intersubjective Model} as a framework for this quadrant, exploring the effects of high magnitude, high autonomy transformations within the context of text-based communication.
\end{itemize}
\begin{table}[h]
\centering
\begin{tabular}{c|c|c}
% \multicolumn{2}{|c|}{} & \textbf{Magnitude} \\ \hline
\textbf{} & \textbf{Low Magnitude} & \textbf{High Magnitude} \\ \hline
\textbf{Low Autonomy} 
& \begin{tabular}[c]{@{}c@{}}Grammarly~\cite{Grammarly} \\ Fu et al.~\cite{Fu2024FromText} \\ Jakesch et al.~\cite{Jakesch2019AIMC} \\ Hohenstein et al.~\cite{Hohenstein2023AIMC} \\ Mieczkowski et al.~\cite{Mieczkowski2021AIMC} \end{tabular} 
& \begin{tabular}[c]{@{}c@{}}Gmail's Smart Reply~\cite{GmailSmartReply} \\ Algouzi et al.~\cite{Algouzi2023Gmail} \\ Harashima~\cite{harashima1991intelligent} \end{tabular} \\ \hline

\textbf{High Autonomy} 
& \begin{tabular}[c]{@{}c@{}}Arias-Sarah et al.~\cite{sarah2024face} \\ MWITM~\cite{mwitm} \end{tabular} 
& \begin{tabular}[c]{@{}c@{}}Intersubjective Model \\ \end{tabular} \\ 
\end{tabular}
\caption{AIMC Systems Categorized by Autonomy and Magnitude}
\label{table:aimc_matrix}
\end{table}


%ここで、まだautonomy高めなやつにも多少言及したい。zoomにおけるフィルタとか通訳とか。それでもあくまでも現実を共有しつつ少しずらしているだけ、と主張する。

%十分に探索されていないが、autonomy高いやつにはこういうポテンシャルがあると言える。みたいな話をして話を盛り上げていく。

%あとobjective/intersubjectiveに繋げていかないと3章が浮く。