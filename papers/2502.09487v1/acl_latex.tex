% This must be in the first 5 lines to tell arXiv to use pdfLaTeX, which is strongly recommended.
\pdfoutput=1
% In particular, the hyperref package requires pdfLaTeX in order to break URLs across lines.

\documentclass[11pt]{article}

% Change "review" to "final" to generate the final (sometimes called camera-ready) version.
% Change to "preprint" to generate a non-anonymous version with page numbers.
\def\rethicsInfo{The study was approved by UCL's Research Ethics Committee (Project ID: 16639/001)}
% \def\rethicsInfo{[Research Ethics information blinded]}
\def\acknInfo{JO was supported by the International Max Planck Research School on Computational Methods in Psychiatry and Ageing Research (IMPRS COMP2PSYCH) fellowship (577749/D-CON/186534). We are thankful for useful comments from the members of Applied Computation Psychiatry Lab. We are also grateful to Dr Andrew Saxe for discussions regarding our approach.}
% \def\acknInfo{[Acknowledgements information blinded]}
 % \usepackage[review]{acl}
 
\usepackage[preprint]{acl}

% \pagenumbering{gobble}

% Standard package includes
\usepackage{times}
\usepackage{latexsym}

% For proper rendering and hyphenation of words containing Latin characters (including in bib files)
\usepackage[T1]{fontenc}
% For Vietnamese characters
% \usepackage[T5]{fontenc}
% See https://www.latex-project.org/help/documentation/encguide.pdf for other character sets

% This assumes your files are encoded as UTF8
\usepackage[utf8]{inputenc}

% This is not strictly necessary, and may be commented out,
% but it will improve the layout of the manuscript,
% and will typically save some space.
\usepackage{microtype}

% This is also not strictly necessary, and may be commented out.
% However, it will improve the aesthetics of text in
% the typewriter font.
\usepackage{inconsolata}

%Including images in your LaTeX document requires adding
%additional package(s)
\usepackage{graphicx}
\usepackage{adjustbox}
% If the title and author information does not fit in the area allocated, uncomment the following
%
%\setlength\titlebox{<dim>}
%
% and set <dim> to something 5cm or larger.

\title{Objective quantification of mood states using large language models}

% Author information can be set in various styles:
% For several authors from the same institution:
% \author{Author 1 \and ... \and Author n \\
%         Address line \\ ... \\ Address line}
% if the names do not fit well on one line use
%         Author 1 \\ {\bf Author 2} \\ ... \\ {\bf Author n} \\
% For authors from different institutions:
% \author{Author 1 \\ Address line \\  ... \\ Address line
%         \And  ... \And
%         Author n \\ Address line \\ ... \\ Address line}
% To start a separate ``row'' of authors use \AND, as in
% \author{Author 1 \\ Address line \\  ... \\ Address line
%         \AND
%         Author 2 \\ Address line \\ ... \\ Address line \And
%         Author 3 \\ Address line \\ ... \\ Address line}

% \author{Jakub Onysk \\
%   Affiliation / Address line 1 \\
%   Affiliation / Address line 2 \\
%   Affiliation / Address line 3 \\
%   \texttt{jakub.onysk.22@ucl.ac.uk} \\\And
%   Quentin Huys \\
%   Affiliation / Address line 1 \\
%   Affiliation / Address line 2 \\
%   Affiliation / Address line 3 \\
%   \texttt{q.huys@ucl.ac.uk} \\}

\author{
 \textbf{Jakub Onysk\textsuperscript{1}},
 \textbf{Quentin Huys\textsuperscript{1}}
%  \textbf{Third T. Author\textsuperscript{1}},
%  \textbf{Fourth Author\textsuperscript{1}},
%\\
%  \textbf{Fifth Author\textsuperscript{1,2}},
%  \textbf{Sixth Author\textsuperscript{1}},
%  \textbf{Seventh Author\textsuperscript{1}},
%  \textbf{Eighth Author \textsuperscript{1,2,3,4}},
%\\
%  \textbf{Ninth Author\textsuperscript{1}},
%  \textbf{Tenth Author\textsuperscript{1}},
%  \textbf{Eleventh E. Author\textsuperscript{1,2,3,4,5}},
%  \textbf{Twelfth Author\textsuperscript{1}},
%\\
%  \textbf{Thirteenth Author\textsuperscript{3}},
%  \textbf{Fourteenth F. Author\textsuperscript{2,4}},
%  \textbf{Fifteenth Author\textsuperscript{1}},
%  \textbf{Sixteenth Author\textsuperscript{1}},
%\\
%  \textbf{Seventeenth S. Author\textsuperscript{4,5}},
%  \textbf{Eighteenth Author\textsuperscript{3,4}},
%  \textbf{Nineteenth N. Author\textsuperscript{2,5}},
%  \textbf{Twentieth Author\textsuperscript{1}}
\\
\\
 \textsuperscript{1}Applied Computational Psychiatry Lab, Max Planck UCL Centre for Computational\\Psychiatry and Ageing Research, Queen Square Institute of Neurology and Mental Health\\Neuroscience Department, Division of Psychiatry, University College London
 % \textsuperscript{1}Applied Computational Psychiatry Lab, Max Planck UCL Centre for Computational Psychiatry\\and Ageing Research, Queen Square Institute of Neurology, University College London
%  \textsuperscript{2}Affiliation 2,
%  \textsuperscript{3}Affiliation 3,
%  \textsuperscript{4}Affiliation 4,
%  \textsuperscript{5}Affiliation 5
\\
 \small{
   \textbf{Correspondence:} \href{mailto:jakub.onysk.22@ucl.ac.uk}{jakub.onysk.22@ucl.ac.uk}
 }
}

% Variables --------------
\def\Ntot{422}
\def\Nv{148}
\def\Nvd{49}
\def\Nvdd{225}
\def\Nhs{381}
\def\ProlficCompletionTime{21 minutes}
\def\itemLCorr{0.52}
\def\itemLHCorr{0.84}

% CHECK THESE
\usepackage[super]{nth}
% \usepackage{amsmath}
% \usepackage{amssymb,amsthm}
% \usepackage{bm}

% \renewcommand{\vec}[1]{\underline{\bm{#1}}}
% \DeclareMathOperator{\E}{\mathbb{E}}
% \DeclareMathOperator{\Var}{\mathrm{Var}}
% \DeclareMathOperator{\Cov}{\mathrm{Cov}}
% \DeclareMathOperator{\corr}{\mathrm{corr}}
% \DeclareMathOperator{\Tr}{\mathrm{Tr}}
% \DeclareMathOperator{\diag}{\mathrm{diag}}
% \DeclareMathOperator{\od}{\mathrm{offdiag}}
% \DeclareMathOperator{\ecdf}{\mathrm{ecdf}}
% \DeclareMathOperator{\sign}{sign}

\usepackage{listings} %for line break in verbatim
\lstset{
basicstyle=\footnotesize\ttfamily,
columns=flexible,
breaklines=true
}

%%%%%%%%%%% DOCUMENT

\begin{document}
\maketitle
\begin{abstract}
Emotional states influence human behaviour and cognition, leading to diverse thought trajectories. Similarly, Large Language Models (LLMs) showcase an excellent level of response consistency across wide-ranging contexts (prompts). We leverage these parallels to establish a framework for quantifying mental states. Our approach utilises self-report questionnaires that reliably assess these states due to their inherent sensitivity to patterns of co-occurring responses. Specifically, we recruited a large sample of participants (N=\Ntot) to investigate how well an LLM (Mistral-7B-OpenOrca) quantifies a heterogenous set of depressive mood states measured with participants' open-ended responses to a depression questionnaire. We show LLM responses to held-out multiple-choice questions, given participants' open-ended answers, correlate strongly (r: \itemLCorr-\itemLHCorr) with true questionnaire scores, demonstrating LLM's generalisation from mood representations. We explore a link between these representations and factor analysis. Using ridge regression, we find depression-related subspaces within LLM hidden states. We show these subspaces to be predictive of participants' "Depression" and "Somatic \& Emotional Distress" factor scores, as well as suicidality severity. Overall, LLMs can provide quantitative measures of mental states. The reliability of these hinges upon how informative the questions we ask participants are. Used correctly, this approach could supplement mental state assessment in a variety of settings.
% These results warrant a further pursuit of quantifying mental states with LLMs - an approach that could supplement mental state assessment in standard and wide-scale online settings.
% but also in sensitive contexts (suicidal ideation) by mapping relevant LLM-subspace onto neural activity space for decoding.
% 
% and suicidal ideation. 
% exhibit a similar relation as humans, whereby different contexts (prompts) lead to different responses, retaining an excellent level of consistency. 
% In fact, LLMs exhibit many similarities to humans in generating language and provide a promising model of neural activity during story listening and internally generated speech. 
\end{abstract}

\section{Introduction}
%%%%%%%% INCLUDE ----------- START
Moods, emotions, and mental states more broadly, are fascinating. They guide us in major life decisions such as buying a house, choosing a partner or a career path.
%and have a significant effect on the evolution of stock market \citep{kuhnen_influence_2011},
On the other hand, they influence our thoughts and cognition \citep{andrews-hanna_conceptual_2022, raffaelli_think_2021, bellana_narrative_2022}. 
For example, feeling joyful and happy may promote positive thinking: "I was great today", "I think they like me". Conversely, feeling depressed, worthless and suicidal may constrain the space of thoughts a person entertains to ones such as: "Life is not worth living", "I am to blame for this". 
% Let's label such a dependency as \texttt{mental state => thought \& behaviour}. 
In extreme cases, this may lead someone to engage in suicidal behaviours - a global issue resulting in premature deaths at a rate of 700,000 a year \citep{who_suicide_nodate}. Unfortunately, research to-date has mainly focused on understanding the algorithmic thought processes governing how people solve hierarchical planning problems \citep{correa_exploring_2023,correa_humans_2023} or simple risky decision-making tasks \citep{russek_heuristics_2024}. While some attempted to spell out the effects mood may have on such processes \citep{huys_bonsai_2012, huys_interplay_2015, russek_opportunities_2020}, these approaches are still limited in explaining the intricacies of thought and their mechanisms in psychopathology \citep{coppersmith_mapping_2023, millner_advancing_2020}. 
To shed light onto this problem, we take a novel research avenue that ties closely with the advent of Large Language Models (LLMs). Notably, state-of-the-art LLM-based systems have demonstrated a striking similarity between how they consistently generate language and converse in response to a set of prompts and how humans communicate \citep{openai_gpt-4_2023}, think, and report mental states using language \citep{colombatto_folk_2024, dillion_can_2023,van_duijn_theory_2023, hagendorff_machine_2023}. 
% We label this dependency in LLMs \texttt{context => text}. 
Interestingly, LLMs provide a promising model of neural activity during language generation, story listening and internal speech generation \cite{goldstein_shared_2022, tikochinski_incremental_2025, tang_semantic_2023} and offer means of stimulus design to drive specific neural activity \citep{tuckute_driving_2024}. By virtue of the parallels between the human mental state consistency and LLM context consistency, a fundamental question arises.
% \texttt{mental state => thought \& behaviour} dependency in humans and the \texttt{context => text} dependency in LLMs, 
% two fundamental questions arise - 
What is the nature of this correspondence and can we leverage our understanding thereof to model psychological processes, such as thought, both in health and psychopathology? Answering these questions has a potential to: 1) offer a framework to understand the structure of language related to thoughts; 2) quantify relevant dimensions associated with mental health conditions to inform, refine, personalise and scale treatment; 3) expand the literature on parallels between LLMs and human brains, contributing to the development of more advanced LLMs that are better aligned with human thought processes. 

\section{Problem statement and related work}
We take the first step towards understanding and leveraging the apparent parallels between LLMs and human mental states and ask whether an LLM can be used to quantify human depressive mood states, maintaining consistency of responses given a mood state.
% with respect to the things people would think, say or write.
We examine this by turning to standardised self-report psychiatric questionnaires. These questionnaires deal with agreed-upon constructs (e.g. depression, anxiety, apathy), and are widely used to aid clinical diagnosis due to their specificity and sensitivity \citep{kroenke_phq-9_2001, ang_distinct_2017, spitzer_brief_2006}. Such questionnaires are highly sensitive to context, whereby individuals with diverse mental states (acting as context) provide sets of responses consistent with these mental states. Importantly, the way people respond to the questions reveals relevant structure, with certain responses co-occurring together and thereby pointing at an underlying relationship between thoughts tied together by an emotional state. For example, answering "I feel hopeless about the future" tends to co-occur participants answering "I feel that I am neither useful nor needed" \citep{romera_factor_2008}. This co-occurrence highlights two important aspects: 1) Given a depressed mental state, people tend to report statements consistent with this state; 2) Given a statement expressing hopelessness, a statement expressing irrelevance is likely to follow. While seemingly obvious, the second observation in combination with LLMs' promising predictive utility in cognitive science, may offer a powerful approach to objective quantification of human mental states, including psychopathology. 

Often, self-report questionnaires employ multiple-choice scales. This approach constrains mental state representations to patterns of responses (scores), precluding predictions in high-dimensional settings (everyday thinking, behaviour, neural activity, treatment).
% Usually, self-report questionnaires require participants to answer questions on a predefined multiple-choice scale. A mental state representation derived from patterns of responses is then constrained to the space of multiple-choice questionnaires, precluding predictions in high-dimensional settings (everyday thinking, behaviour, neural activity, treatment). 
To overcome this, we extend the standard self-report questionnaires to have an open-ended form \citep{hur_language_2024}, retaining both the important structure of such questionnaires but also allowing for a richer sample from participants' dimensions of mental state that can then be represented in an LLM space.

Notably, there is an existing body of work, showcasing the capability of LLMs to represent human-like personality profiles \citep{rutinowski_self-perception_2023,serapio-garcia_personality_2023}, as well as mental states relating to psychopathology \cite{coda-forno_inducing_2024} and more broadly \citep{hagendorff_machine_2023,jiang_personallm_2024}. As a result, different criteria have been established 
for using psychological tests on LLMs \citep{lohn_is_2024}, including questionnaire scales \citep{huang_reliability_2024}. What emerges is a broad consistency of LLMs across psychological measurements, justifying our approach. We believe our contribution is novel, as we go beyond the focus on whether LLMs demonstrate something like a psychology. In contrast, we leverage this apparent ability to then quantify and model human depressive states. To our knowledge, this approach has only been explored in a limited capacity to extract sentiment of open-ended questionnaire responses to predict depression severity \citep{hur_language_2024}.

\begin{figure}[!t]
    \centering
    \includegraphics[width=1\linewidth]{figs/design4.pdf}
    \caption{\textbf{A: Study design}. The study consists of three levels of questions of increasing specificity. At each level, open-ended and equivalent multiple-choice questions were asked. Following Level 3 open-ended questions, participants completed multiple-choice PHQ-9, GAD-7 and SDS questionnaires, assessing depression and anxiety. \textbf{B: LLM prompt design and sampling}. For each level, we take the open-ended question, $OQ_q$, and participant's open-ended answer, $A_{q,i}$, forming participant's QA-pair. We append the corresponding multiple-choice question $CQ_{q'}$ and pass the prompt to MistralO to sample responses, $R_{q',i}$, on a scale.}
    \label{fig:study_des}
\end{figure}


Given the promising demonstration of LLMs psychology, we take the idea of open-ended responses as samples from participants' mental states further. We recruit participants to provide responses to open-ended depression questionnaires and investigate how well an LLM quantifies a diverse set of depressive mood states. We use participants' additional multiple-choice responses to such questionnaires as validation metrics. Crucially, we evaluate whether LLM representations of said mood states generalise, allowing to make predictions across held-out questionnaires responses. Lastly, we showcase an exploration of LLM hidden states to capture a latent structure relevant to depression and find it predictive of severity in different mood state dimensions, including suicidality.


%%%%%%%% INCLUDE ----------- END


%%%%%%%% IN PROGRESS ------------- START



% I establish such an approach as follows. In the first study, I ascertain whether the responses from LLMs to a range of self-report psychiatric questionnaires display reliability and context sensitivity similar to humans. I also compare the similarities between the structures of responses from LLMs and humans. 




% Given this limitation, I build on the work by Hur et al.\citep{hur_language_2024} that gathers participants' responses to open-ended questions derived from PHQ-9 to predict symptom changes. 

% In a similar vein, I collect a large ground-truth dataset of responses to three sets of open-ended questions derived from PHQ-9 of varying level of specificity. Additionally, I collect participant multiple-choice responses to PHQ-9, GAD-7, SDS questionnaire for generalisably analyses. Collecting open-ended responses to a depression questionnaire provides a richer, individual specific description of participant's mental state. Importantly, given the vast utility of LLMs, we can represent such open-ended responses with the LLMs' internal states and transformations thereof, and subsequently look for common structure across responses and how these are related to more standardised measures such the validated self-report questionnaires - PHQ-9, SDS, GAD-7. 

%%%%%%%%IN PROGRESS ------------- END


% Motivated by the promising consistency and reliability of LLMs in answering self-report questionnaires, 

% \section{Problem statement}

%%%%%%%% STUDY DESIGN ------------- START
\section{Study design}
We recruited N=$\Ntot$ fluent English speakers (mean age 37$\pm$12 years, 216 females) with diverse depression severity (see distributions in Appendix \ref{apdx:depression_totals}) on \citealp{noauthor_prolific_nodate}. \rethicsInfo. Participants provided consent and were reimbursed £8.21/h. See more details in Appendix \ref{apdx:ppt_info}.


Participants' main task was to provide written responses to open-ended questions that we derived from the multiple-choice depression Patient Health Questionnaire (PHQ-9; \citealp{kroenke_phq-9_2001, hur_language_2024}). See study design in Figure \ref{fig:study_des}A. We created three levels of open-ended questions of increasing specificity, each containing one, three and eight questions, respectively, that corresponded to the original PHQ-9 questions. Broadly, the questions target participants': "Overall wellbeing", "Interest and pleasure in activities",  "Mood and feelings", "Sleep", "Energy levels", "Appetite", "Self-worth", "Focus", "Slowness/restlessness" (see Appendix \ref{apdx:open_qs} for the exact phrasing of the questions - note that we removed the suicidality Question 9). At each level, participants had 1.5 minutes to answer each open-ended question with a written response of at least 30 words. 
% Additionally, after Question 1 at level 2, and Question 6 at level 3, we asked participants to select emotional labels (out of 28 listed in Appendix \ref{apdx:emos}) to answer a question "Which of the following emotions best describes what you are currently feeling?" 
Following a set of open-ended questions at Level 1 and 2, we asked participants to respond on a scale (Very Bad, Bad, Good, Very Good) to answer "Please select a response to the following statement that best describes you over the last 2 weeks" regarding a statement that was phrased similarly to the corresponding open-ended question (see the exact phrasing in Appendix \ref{apdx:closed_qs}). After Level 3 open-ended questions, participants completed the standard PHQ-9 questionnaire, as well as two additional questionnaires assessing anxiety - Generalized Anxiety Disorder scale (GAD-7) \cite{spitzer_brief_2006} - and depression - Self-rating Depression Scale (SDS) \cite{zung_self-rating_1965}. At the very end of the study, we asked participants to again answer the open-ended Question 1 at Level 2. 
% All of the responses were time-limited, with the experiment median completion time of \ProlficCompletionTime.

We used Mistral-7B-OpenOrca (MistralO) LLM \cite{lian2023mistralorca1, jiang_mistral_2023} for the analyses (see details in Appendix \ref{apdx:llm_dets}). We report Spearman's rank correlations and Bonferroni p-values (defaulting to $5\%$-$\alpha$ level), unless otherwise stated.
% We now move on to our research questions to establish how well we can predict questionnaire scores given open-ended responses as context using LLMs. Following the LLM evaluations in Study 1, w

\begin{figure*}[!t]
    \centering
    \includegraphics[width=1\linewidth]{figs/item/scores_correlations_spec-general_v4-v4_d-v4_dd.pdf}
    % \vspace{-1.75em}
    \caption{Violin, box and scatter plots, with correlations (p-val<0.001) of subject's item-level scores vs MistralO' scores for the item given the corresponding open-ended QA-pair for Level 1 (\textbf{A}), Level 2 (\textbf{B-D}) and Level 3 (\textbf{E-L}).}
    \label{fig:study2_llm_item_level_scores}
\end{figure*}
%%%%%%%% STUDY DESIGN ------------- END


%%%%%%%% RESULTS ------------- START
\section{Results}

\subsection{Item-level sampling}
Firstly, we show we can use MistralO to predict item-level multiple-choice question scores given participants' item-level open-ended responses.

For each participant, at each level, we create a set of item-level prompts, where a prompt consists of: a QA-pair (open-ended question and participant's open-ended answer), as well as the corresponding multiple-choice question (see an example in Appendix \ref{apdx:item_prompt}). We then perform a forward pass using MistralO, for each of participant's QA-pair and multiple-choice question to sample a response $50$ times. See an illustration of this process in Figure \ref{fig:study_des}B. We instruct the model to answer using the questionnaire scale. In the following, we average across the samples for each QA-pair response.

In Figure \ref{fig:study2_llm_item_level_scores}, for each level and each question we plot subject's score on the multiple choice question against MistralO predicted score to the same question, given participant's QA-pair. Overall, MistralO scores for each question given participants' corresponding QA-pair as context are highly correlated with the true participants' scores ($0.52-0.84$). This indicates that MistralO is able to represent the sample from participant's mental state (in the form of open-ended responses) to then provide expected responses on the multiple-choice question. This finding emphasises the ability of the model to provide consistent responses across a diverse set of mood states in a human population. Furthermore, we observe a significant level of variability of MistralO responses, particularly when participants report a lack of the symptom in a given Level 3 question ("Not at all" answer for multiple-choice PHQ-9 questions). This may indicate that the content of participant's answer doesn't give the model enough evidence to claim the lack of a symptom given that participants don't explicitly state they are not bothered by something. 
% As a result MistralO seems to be avoiding the fallacy of "denying the antecedent" at the expense of providing higher scores.
We also notice that overall model responses are biased towards higher scores, which may be a result of an inherent bias from the model's initial training dataset.

Overall, MistralO quantifies mood states of participants as expressed through the QA-pairs. We validated this by comparing item-level score predictions to true participant scores. While the model may be biased towards providing more 'depressed' answers, the clear trend of score severity is a promising sign that the requisite consistency, indicative of mental state, is represented.
% \begin{center}
% \[
% \begin{array}{ r l }
% & \mbox{If P then Q}  \\
% & \mbox{\hspace{3em}("I report I'm bothered by something" then "I'm bothered by something")} \\
% & \mbox{not P}\\
% & \mbox{\hspace{3em}("I don't provide a report that I'm bothered by soemthing")} \\
%  \cline{2-2}
% & \therefore \mbox{Therefore not Q (fallacy)}\\
% & \mbox{\hspace{3em}("Therefore I'm not bothered by something")}
% \end{array}
% \]
% \end{center}

% This potential bias is further observed when we perform an additional check of how item-level responses may correspond to total scores. Here, we group MistralO item-level responses (random permutation) into sets of questions corresponding to an entire questionnaire so as to approximate the total score for each level (averaged across $S=50$ MistralO samples). We then plot the approximate total score distribution for each level and compare with participants' in Figure \ref{fig:study2_sub_llm_totals_perm_item}.

% \begin{figure*}[t]
%     \centering
%     \includegraphics[width=1\linewidth]{figs/item/totals_comparison_llm_sub.pdf}
%     \vspace{-1em}
%     \caption{Histograms and density estimates of participants' total scores for each level vs MistralO estimated total scores from permuted item sets.}
%     \label{fig:study2_sub_llm_totals_perm_item}
% \end{figure*}

% \paragraph{Interim discussion}\mbox{}\\

\subsection{Item-level generalisation sampling}
% In this study, we ask the following questions. Using MistralO and given participants item-level open-ended PHQ-9 responses: 
% \begin{enumerate}
%     \item Can we predict item-level scores to other questionnaires such as GAD-7, SDS?
%     \item Can we characterise a latent structure within LLM's hidden state representations of participant's open-ended responses to predict participant's SDS factor scores?
% \end{enumerate}
% Furthermore,
% \begin{enumerate}
%   \setcounter{enumi}{3}
%     \item Does the item-level PHQ-9 score prediction, improve given a full set of open-ended responses?
%     \item Specifically, can we predict suicidality item severity?
% \end{enumerate}
Next, we evaluate MistralO's response generalisation to multiple-choice questions beyond those implied by QA-pairs. We show we can predict item-level scores to other depression (SDS) and anxiety (GAD-7) questionnaires using Level 3 QA-pairs.

We create a new set of item-level prompts, where a prompt consists of: one of participant's eight Level 3 QA-pairs and one of all of the multiple-choice questions from the SDS, GAD-7 and PHQ-9 questionnaires. 
% For each participant, we have sets of QA-pair question prompts of sizes: $\{8\} \times \{20, 7, 9\}$. 
Here, we evaluate how well MistralO is able to recover multiple-choice scores given each of the Level 3 QA-pairs (mental state samples in a specific dimension). We pass each of the participant's QA-pair question prompt through MistralO in order to sample and then average across 20 responses to the multiple-choice question.
\begin{figure*}[!t]
    \centering
    \includegraphics[width=0.9\linewidth]{figs/gen/gen_qs_total_item_corrs.pdf}
    \caption{\textit{Column I} (\textbf{A, D, G}): Subject's total scores for each questionnaire against the MistralO total score estimate (Pearson correlation reported). \textit{Column II} (\textbf{B, E, H}): Subjects’ pairwise correlations between PHQ-9 item scores and SDS, GAD-7 and PHQ-9 scores. \textit{Column III} (\textbf{C, F, I}): Correlations between participant's true item score for each questionnaire (x-axis) and MistralO recovered score for that question given each of the Level 3 QA-pairs (y-axis). Significant correlations are in cool-warm colour. The correlations not reaching significance are in gray scale.}
    \label{fig:study2_llm_gen_item_qs}
\end{figure*}

Additionally, in Equation \ref{eq:gen_total_def}, we approximate MistralO's total score (overall symptom severity) for each of the sampled self-report questionnaire, $j$, by summing up scores, $s_{q_{j},c}$, across questionnaire items, $\{q_{j}\}$, for each of the Level 3 QA-pairs as context, $c$, and then averaging across them.
\begin{equation}
\label{eq:gen_total_def}
	\hat{T_j} = \frac{1}{C}\sum_{c=1}^{C}\sum_{q_{j}=1}^{Q_j} s_{q_{j},c} 
\end{equation}

In Figure \ref{fig:study2_llm_gen_item_qs} (first column), we plot participants' total scores for each questionnaire (SDS, GAD-7, PHQ-9) against MistralO estimated total score. While the model captures the expected trend in total scores (Pearson's correlation 0.73-0.85), we observe a bias towards higher scores for PHQ-9 and GAD-7 in contrast with the true participant's total scores. Interestingly, for the SDS questionnaire, the bias appears to go in the opposite direction in the higher range of total scores, where participant's total scores are higher than MistralO's predictions.


At the item-level view, we plot the correlations between participants' scores for each PHQ-9 question and each question from SDS, GAD-7, and PHQ-9 (second column). We use this as a benchmark questionnaire response structure that participant's demonstrate and compare with the correlations between the true item score on each of SDS, GAD-7, PHQ-9 questionnaires (x-axis) and MistralO recovered score for that question given each of the Level 3 open-ended QA-pairs (y-axis). We observe that the covariance structure between PHQ-9 and generalisation questionnaire scores (column 2) match some of the block structure of the correlation map between the true generalisation questionnaire scores and Mistral0 recovered scores (column 3) from the corresponding mental state description.

These results are promising and indicate MistralO represents a short open-ended mental state report derived from PHQ-9 items sufficiently to generalise from this state to respond to other depression and anxiety questions in a manner reflecting the true underlying mental structure. This further indicates MistralO's ability to encode a consistent representation akin to a mental state resulting in expected responses. Importantly, these consistencies are seen both at an item and questionnaire level.

\subsection{Predicting SDS factor scores with ridge regression from hidden states}
Next, we show that can we characterise a latent structure within LLM's hidden state representations of participants' QA-pairs to predict participant's severity on latent dimensions derived from SDS depression questionnaire scores (factor scores).
\begin{figure*}[!t]
    \centering
    \includegraphics[width=0.925\linewidth]{figs/latent_regression/acl_joint.pdf}
    % \vspace{-2em}
    \caption{\textit{Top}: Correlations between all subjects' factor scores and multiple-choice scores for each question (significant correlations are in orange, not significant correlations in greyscale). \textit{Middle} (\textbf{A-D}): Scatter plot of the held-out predicted factor scores against the true factor scores using average QA-pair PCA-reduced hidden state (with Pearson correlations, a regression line and 95\% CIs). \textit{Bottom}: Pearson's correlations between the held-out predicted factor scores and true factor scores, using each QA-pair PCA-reduced hidden state (significant correlations are in orange, short-of-significance correlations are annotated, with corresponding p-values in purple scale).}
    \label{fig:latent_ridge_all}
\end{figure*}

\subsubsection{SDS factor analysis}
% \paragraph{Factor analysis}\mbox{}\\
We make use of factor analysis \citep{mulaik_foundations_2013}, which allows to characterise the co-occurrences of responses across a questionnaire. Factor analyses assumes a latent space that represents cognitive-behavioural dimensions represented by weighted combinations of questions. The coordinates within this latent space (expressing severity within each dimension) are expressed as factor scores.

% \begin{equation}
% p(\vec{z})=\mathcal{N}(\vec{z}|\vec{0}, I)
% \end{equation}
% The questionnaire responses, $\vec{x}$, are assumed to be conditioned on this latent vector:
% \begin{align}
% p(\vec{x}|\bm{z}) = \mathcal{N}(\vec{x}|W\vec{z} + \vec{\mu}, \Psi)
% \end{align}
% We can marginalise the above distribution to express the distribution of the observed questionnaire responses as:
% \begin{align}
% p(\vec{x}) = \mathcal{N}(\vec{x}|\vec{\mu}, WW^T + \Psi)
% \end{align}
% The assumptions are that $\Psi$ is a diagonal matrix of factor-unique variances with diagonal elements, $\psi_{i}$, that captures both factor-specific variance and random noise. $W$ is a matrix of common factor coefficients (factor loading matrix). As such:
% \begin{align}
% \Cov[\vec{x}]=WW^T + \Psi   
% \end{align}
% where
% \begin{align}
%     \Var[x_i] =\sum_{j=1}^{m}w_{i,j}^2+\psi_i
% \end{align}
% For the purposes of factor analysis of questionnaire responses here, dataset and model responses are standardised, therefore $\E(\vec{x})= \vec{\mu} = \vec{0}$ and $\Var(x_i)=1 \, \forall i$. 

% The goal of factor analysis is to approximate the covariance of responses by factorising it into the estimated factor loading matrix $W$, that specifies how much each question loads onto each factor. Given the estimated factor loading matrix $W$, and assuming questionnaire responses are standardised, the solution for covariance $\Psi$ is given by \citep[see pp.~193-196]{mulaik_foundations_2013}:
%  \begin{align}
%  \diag(\Psi) = \diag(\Cov[\vec{x}]-WW^T])\\
%  \text{where}\quad
%  \psi_i = 1 - \sum_{j=1}^{m}w_{i,j}^2. 
%  \end{align}

% I use Python library FactorAnalyzer \citep{factor_analyzer} with promax rotation and least-squares estimation of the correlation matrix using the factorised covariance $WW^T+\Psi$, to obtain factor loading matrix $W_m$ ($Q \times F$; Q - \# of questions, F - \# of factors) for a set of LLM responses $\{\vec{x}_{i,m}\}_i$, as well as the covariance matrix of unique variances, $\Psi_m$($Q \times Q$, see above).

% Importantly, given the factor loading matrix, we want to infer how an individual set of responses maps onto each of the cognitive-behavioural dimension (coordinates in that space, commonly known as factor scores). We can calculate factor scores, $\vec{f}_{i,m}$, for each set of model response, $\vec{x}_{i,m}$, such that: 
% \begin{align}
%     \vec{x}_{i,m} = \vec{\mu}_m + W_m \vec{f}_{i,m} + \vec{\epsilon}_{i,m}\\
%     \text{where}\quad \vec{\epsilon}_{i,m}\sim \mathcal{N}(\bm{0},\Psi_m)
% \end{align}
% This is done using Weighted Least Squares (WLS) solution to the linear regression problem above, yielding the factor score, $\vec{f}_{i,m}$, for each model response, $\vec{x}_{i,m}$:
% \begin{align}
% 	\vec{f}_{i,m} = L_m\vec{x}_{i,m}\\
%     \text{where}\quad L_m = (W_m^T\Psi_m^{-1}W_m)^{-1}W_m^T\Psi_m^{-1}
% \end{align}

% For this analysis, we exclude participants who provided less than 20 words in any of open-ended responses or didn't provide a full set of SDS responses. Following this exclusion we have 412 participants remaining. 

% In Figure \ref{fig:study2_factor_load_counts_sds}A we plot the covariance matrix of participants' responses, which we then use to perform factor analysis. 
To calculate SDS factor scores for each participant, we assume a four factor model \cite{romera_factor_2008} with a promax rotation for the SDS responses. 
% We don't perform a full search for the best factor analysis model, as we are not interested (at this stage) in finding the best description of the underlying structure, but .
At this stage, we are interested in examining the patterns between an assumed participant (rather than best-fitting one) and LLM-represented structure. We use SDS responses from all participants to fit the factor analysis model. 

% and obtain factor loading matrix in Figure \ref{fig:study2_factor_load_counts_sds}B. Some questions are reverse scored where appropriate, such that higher score means higher depression severity, see \ref{apdx:sds_questions} for a full list of SDS questions and corresponding short labels. Additionally, in Figure \ref{fig:study2_factor_load_counts_sds}C we illustrate the distribution of responses - for each SDS question we plot the number of responses for each score. Finally, for each participant we calculate their factor score vector (see Eq. \ref{eq:factor_score}). 
% \begin{center}
% \begin{figure}[!hbt]
% 	\centering
%     % \hspace{-9mm} 
% 	\includegraphics[width=1\linewidth]{figs/study2/tmp_results/latent_regression/factor_loading_and_counts_cov_sds.pdf}
% 	% \vspace{-2em}
% 	\caption{\textbf{A}: Covariance of participant SDS responses. \textbf{B}: Factor loading matrix from the factor analysis model with four factors and promax rotation for SDS questionnaire response. \textbf{C}: Heatmap of counts of scores for each SDS question.}
% 	\label{fig:study2_factor_load_counts_sds}
% \end{figure}
% \end{center}
% \newpage

We find the following factors that consist of questions with highest loading for each - 1) Depression: Hopelessness \& Apathy; 2) Somatic \& Emotional Distress; 3 Cognitive Functioning; 4) Appetite \& Weight Changes. See Appendix \ref{apdx:fa_results} for the content of each factor and the factor loading matrix.
% \begin{itemize}
%     \item \textit{Factor 1} - \textbf{Depression: Hopelessness \& Apathy}: Sad, Morning Best, Sex Enjoyment, Clear Thinking, Task Ease, Hopeful, Decisive, Needed, Full Life, Enjoyment.
%     \item \textit{Factor 2} - \textbf{Somatic and Emotional Distress}: Sad, Crying, Sleep Issues, Constipation, Fast Heart, Tired, Restless, Irritable, Suicidal.
%     \item \textit{Factor 3} - \textbf{Cognitive Functioning}: Task Ease, Clear Thinking.
%     \item \textit{Factor 4} - \textbf{Appetite and Weight Changes}: Eating Normal, Weight Loss.
% \end{itemize}

The first two factors are well represented loading onto the majority of the questions, while Factor 3 and 4 are less defined. This may be because Factor 3 overlaps with Factor 1 as reflected by the loadings and the co-occurrence of Factor 3 questions with other questions in the covariance matrix (see Figure \ref{fig:study2_factor_load_counts_sds}A in Appendix \ref{apdx:fa_results}), while the "Appetite \& Weight Changes" factor consists of two questions, which don't co-vary with other questions.

To see the relationship between factor scores and multiple-choice questions at each level, we calculate and plot (Figure \ref{fig:latent_ridge_all} top) pairwise correlations between each factor score and each multiple-choice question score. We label questions to indicate their content (a full list of questions is found in Appendix \ref{apdx:open_qs}).
% \begin{figure*}[!hbt]
% 	\centering
% 	\includegraphics[width=1\linewidth]{figs/latent_regression/fscores_vs_qs_scores_sds.pdf}
% 	% \vspace{-2em}
% 	\caption{Orange scale: significant (at 5\% level, Bonferroni corrected) correlations between true scores for each factor and true scores for each closed question. Gray scale: not significant correlations. Closed question labels indicate the rough content of each question.}
% 	\label{fig:fscores_vs_sds_scores}
% \end{figure*}
Given that only Factor 1 and Factor 2 are well defined, we see strong relationship between multiple-choice questions and these two factors. We notice that Level 3 questions regarding focus, slowness/restlessness correlate mostly with the "Somatic \& Emotional Distress" factor scores, while the Level 2 question regarding mood, feelings and self-worth correlate most with the "Depression: Hopelessness \& Apathy" factor. While these are expected and offer a way to validate the content of PHQ-9 derived questions and SDS questionnaire, we can't fully ascertain the relationship due to the underdetermined Factor 3 and 4, which could explain some of the remaining variance.
% variance related to the multiple-choice questions.

% \subsubsection{Level 2 questions representation}
\subsubsection{All questions average representation}
We are interested in establishing whether MistralO is able to sufficiently represent participant's mental state using all of the QA-pairs to be able to generalise to SDS factor scores. 
% We chose these questions due to their level of generality as compared to Level 3, but also because they could offer a much faster assessment of someone's mental state. 
To that end, we extract MistralO hidden states at every \nth{3} layer for the last token of each open QA-pair and calculate the average hidden state, representing participant's overall mental state. We then split participants data into train and test set (75\%/25\%; $n=$ 283/97 after applying exclusion criteria).
To find the most informative subspace, we reduce the dimensionality of train-set participants' average hidden state using PCA with $221$ features (explaining 95\% variance). We then fit a ridge regression model, to predict participants' factor scores from these representations.
% We combine all train set participants' ($N_{tr}$) factor score vectors into matrix $F (N_{tr} \times 4)$ and average PCA-reduced hidden states into matrix $H (N_{tr}\times221)$. 

Using the train set, we ran 20-fold leave-one-out cross validation of the ridge regression 
% using\texttt{sklearn RidgeCV}
\cite{scikit-learn} to find the best regularisation strength, $\alpha$, from 20 values on the log-scale between $[10^{-2},10^{2}]$ and the best layer. We find layer 17 and $\alpha=3.360$ provide the highest overall $R^2$. We use this parametrisation to predict factor scores on the held-out test set. For each predicted factor score, for each participant we plot the true value and calculate the correlation in Figure \ref{fig:latent_ridge_all} (middle).


\begin{figure*}[t]
    \centering
    \includegraphics[width=0.85\linewidth]{figs/whole_qs/totals_and_scores_correlations_whole_and_suid_new.pdf}
    % \includegraphics[width=0.95\linewidth]{figs/whole_qs/totals_and_scores_correlations_whole_and_suid_new.pdf}
    % \vspace{-1.75em}
    \caption{\textbf{A}: Scatter plot of subjects' PHQ-9 total scores vs MistralO total scores given a full set of cumulative Level 3 QA-pairs, with Pearson correlations. \textbf{B-J}: Violin, box and scatter plots of subjects' item-level scores against MistralO' scores for the same question given a full set of cumulative Level 3 QA-pairs. \textbf{K}: Scatter plot of regression held-out predictions for PHQ-9 suicidality question against true scores using the average across all cumulative QA-pairs PCA-reduced hidden state (purple) and raw scores (red). Correlations are significant, p-value$<0.001$.}
    \label{fig:study2_llm_wholeqs_level_scores}
\end{figure*}

% We then run the cross-validated ridge regression model using these low-dimensional average representations and find that again layer 17 provides the best fit with regularisation strength $\alpha=3.360$. We then predict factor scores on the held-out test set and plot the results as before in Figure \ref{fig:ridge_reg_avg_pca_hs_scatter}.




% For each of the three $4096$-dimensional vectors representing a question-answer pair, we stack them into $12,288$-dimensional vector, $\mathbf{h}$ approximating participant's mental state across these three questions. 

% The goal is to fit a ridge regression model, $W$($12,288 \times 4$), to predict participant's 4-dimensional factor score vector, $\mathbf{f}$ calculated earlier in the factor analysis, using the $12,288$-dimensional vector representation, $\mathbf{h}$. We combine all train set participants' ($N_{tr}$) factor score vectors into matrix $F (N_{tr} \times 4)$ and hidden states into matrix $H (N_{tr}\times12,888)$. 

% Using the train set, we ran 20-fold leave-one-out cross validation of the ridge regression using \texttt{sklearn RidgeCV}\cite{scikit-learn} to find the best regularisation strength, $\alpha$, from 20 values on the log-scale between $[10^{-2},10^{2}]$ and the best layer. We find layer 17 and $\alpha=2.069$ provide the highest overall $R^2$ value across the folds. Using this parametrisation for the ridge regression, we then predict factor scores on the held-out test set. For each predicted factor score, for each participant we plot the true value and calculate the correlation in Figure \ref{fig:ridge_reg_hs_scatter}.
% This yields a regression model: 
% % ({\color{red}add more details for ridge regression?)}:
% \begin{equation}
%     F = HW + \mathcal{E}
% \end{equation}
% where the noise covariance $\mathcal{E} \sim \mathcal{N}(0, \sigma^2\mathbb{I})$. We can then write the L2 loss, which regularises the regression weights to avoid over-fitting due the feature space being much larger than the number of data points \citep{bishop_pattern_2006}):
% \begin{equation}
%     \mathcal{L} = \frac{1}{N_{tr}}\|F- HW \|_F^2 + \alpha \|W\|_F^2 \\
%     = \frac{1}{N_tr} \Tr((F-HW)^T(F-HW)) + \alpha\Tr(W^TW) 
% \end{equation}
% % (given that we only have real-valued matrices, the conjugate transpose is just the transpose)
% We then calculate the gradient of the loss w.r.t $W$:
% \begin{equation}
%     \nabla_{W}\mathcal{L} = -\frac{2}{N_{tr}}H^T(F-HW) + 2\alpha W.
% \end{equation}
% By setting the gradient to zero, we find the solution to the ridge regression problem:
% \begin{equation}
%     \hat{W} = (H^TH+N_{tr}\alpha\mathbb{I})^{-1}H^TF)
% \end{equation}

% 
% Lastly, we investigate how the prediction of SDS factors scores improves when we use the hidden states from the entire set of open-ended responses across all levels to represent each participant's mental state. 

% For each participant in the train set, we average their hidden states across all questions. 

% \begin{figure*}
%     \centering
%     \includegraphics[width=1\linewidth]{figs/latent_regression/scatter_avg_pca_results_sds.pdf}
%     % \vspace{-2em}
%     \caption{Scatter plot of the predicted factor score against the true value for each factor from average across all questions PCA-reduced hidden states, including Pearson correlation and Bonferroni corrected p-values. $<0.001$}
%     \label{fig:ridge_reg_avg_pca_hs_scatter}
% \end{figure*}% 

The model is able to predict factor scores for Factor 1 and 2 quite well (as measured with correlation between true and predicted factor scores). This suggest that the internal states of the model encode the requisite representations indicative of the factor space related to depression. The remaining factors are less well captured, which may indicate the limitations of these under-defined factors. 


% The inclusion of the entire set of open-ended answers for each participants, overall improves the performance of the regression model, specifically for the Factor 3 and 4, which still fare worse than Factor 1 and 2. It seems that specific questions that force participants to describe their mental state in a given dimension provide additional information that allows the model to predict the worse performing factor scores from before.

We aim to quantify how individual QA-pair (with respect to the average representation) contributes to the factor score prediction. We apply the PCA-model from above to the held-out hidden states for each open QA-pair and use the regression model to predict factor scores for each question. We calculate pairwise correlations between the predicted vs true factor scores for each factor score and question in Figure \ref{fig:latent_ridge_all} (bottom panel).

% \begin{figure*}[!hbt]
% 	\centering
% 	\includegraphics[width=1\linewidth]{figs/latent_regression/reg_avg_pca_factor_score_from_each_q_sds.pdf}
% %	\vspace{-1em}
% 	\caption{Orange scale: significant (at 5\% level) correlations between true and ridge regression predicted factor scores using PCA-reduced single question hidden state. Purple: short-of significance correlations - purple colour indicates p-value while annotation the correlation coefficient.}
% 	\label{fig:ridge_reg_avg_pca_factor}
% \end{figure*}
We see prediction performance using Level 2 Question 2 (sleep, energy, cognition) and Level 3 Questions 7, 8 (focus, slowness/restlessness) corresponds well with the "Somatic and Emotional Distress" factor (Sad, Crying, Sleep Issues, Constipation, Fast Heart, Tired, Restless, Irritable, Suicidal) while the predictions using Level 3 Question 6 (Self-worth) significantly correlate with the "Depression: Hopelessness \& Apathy" factor scores as expected. However, the remaining correlation patterns are less clear and may be a limitation of the relatively small sample size of the held-out test, as well as the under-defined Factors 3 and 4.
%which may affect the representation of other factors.

Overall, we show that MistralO hidden states encode the requisite information that allows for the representation of dimensions related to depression. The prediction of participant factors scores from hidden states representations is promising and attests to the emergence of such dimensions in the model space. We also explored and illustrated the potential of a novel approach to mapping open-ended responses that approximate participant mental state onto factor scores. A partial match between the content of open-ended answers and the factors as measured with correlation maps points to the mood-state like consistency of these representations, reflecting the consistency we established at the text response level in the previous sections.
% This is very promising as it could allow for a momentary measure of participants' severity in specific dimensions of psychopathology based on any written/spoken text or self-generated reports. Most importantly, it could offer a way to establish such dimensional severity based on decoded information from neural activity in response to passive exposure to semantically rich information (as hinted at in the introduction) - offering a method for mental state assessment in challenging cases (suicidal ideation).
% \begin{figure*}
%     \centering
%     \includegraphics[width=1\linewidth]{figs/latent_regression/scatter_level2_results_sds.pdf}
%     % \vspace{-2em}
%     \caption{Scatter plot of the predicted factor scores against the true value for each factor, including Pearson correlation and Bonferroni corrected p-values.}
%     \label{fig:ridge_reg_hs_scatter}
% \end{figure*}

% The model is able to predict factor scores for Factor 1 and 2 well (as measured with correlation between true and predicted factor scores). This suggest that the internal states of the model encode the requisite representations indicative of the factor space related to depression. However, the remaining factors are not well captured, which may indicate the previously mentioned limitations of these under-defined factors. 

% \newpage
% To investigate which hidden units in the model contribute to the prediction of factor scores, we take the regression model weights $W$ and select the top 5\% active weights (absolute value) setting the rest to zero. Within the set of hidden state dimensions corresponding to each question ($4096$), we sum the remaining 5\% of absolute values for each factor and divide by total absolute activity (100\%) for that question, yielding a 3-dimensional activity profile across open-ended questions per factor. This approximates MistralO hidden unit sensitivity to each factor as depicted in Figure \ref{fig:fscores_top_act}. 
% \begin{figure}
% 		\centering
% 		\includegraphics[width=1\linewidth]{figs/latent_regression/top_0.05_active_weights_fraction_level2_sds.pdf}
% 		% \vspace{-1em}
% 		\caption{Absolute activation of top 5\% regression weights summed across hidden dimension-bins corresponding to each open question-answer for each of the four factors.}
% 		\label{fig:fscores_top_act}
% \end{figure}

% The results suggest that units activated for Question 3 (Self-worth) corresponds most strongly with the scores for "Depression: Hopelessness \& Apathy" factor (Sad, Morning Best, Sex Enjoyment, Clear Thinking, Task Ease, Hopeful, Decisive, Needed, Full Life, Enjoyment).
% representing\footnote{you mean activated during Question 3?}
% the contents of Question 3 maps the most onto factor 1 (sexual enjoyment, clarity of mind, ease of doing and deciding things, hopefulness, feeling of usefulness and fullness of life, and enjoyment of activities. 

% Furthermore, the contents of Question 2 (Sleep, energy and cognition) corresponds most strongly with the scores for the "Somatic and Emotional Distress" (Sad, Crying, Sleep Issues, Constipation, Fast Heart, Tired, Restless, Irritable, Suicidal) and "Cognitive Functioning" factors (Task Ease, Clear Thinking). Lastly, Question 1 (Mood and feelings) corresponds mostly with the "Appetite and Weight Changes" factor. Interestingly, the activation patterns partly match the correlation map between scores for Level 2 multiple-choice question 2 and 3, and factor scores in the dataset (Figure \ref{fig:fscores_vs_sds_scores}). The results are promising as the units activated for Question 2 and 3 relate to very similar symptoms that these questions pertain to, indicating that a relevant factor subspace in the hidden states space may exist. However, given the limitation of the factor analysis results for Factor 3 and 4, a further examination on a more diverse/larger dataset may help to tease these apart and potentially explain the unexpected correspondence of "Mood and feelings" Question 1 with the "Appetite \& Weight" factor. Moreover, the 5\% absolute activity approach is crude, and may discard detailed information regarding across-unit correlations in the hidden states, as well disregard the positive/negative contribution of units to factor score prediction. We partly address this in the next approach.


%\begin{figure}[!hbt]
% \begin{minipage}{0.475\linewidth}
%\centering
%\includegraphics[width=1\linewidth]{figs/study2/tmp_results/latent_regression/top_0.05_active_weights_sds.pdf}
%\vspace{-2em}
%\caption{Absolute activation of top 5\% regression weights summed across hidden dimension-bins corresponding to each open question-answer for each of the four factors.}
%\label{fig:fscores_top_act}
% \end{minipage}
%\hfill
% \begin{minipage}{0.475\linewidth}
%\centering
%\includegraphics[width=1\linewidth]{figs/study2/tmp_results/latent_regression/fscores_vs_qs_scores_sds.pdf}
%\vspace{-2em}
%\caption{Correlations between true scores for each factor and true scores for each Level 2 question.}
%\label{fig:fscores_vs_sds_scores}
% \end{minipage}
%\end{figure}
\subsection{Whole questionnaire sampling}
Lastly, we investigate how providing a whole set of Level 3 QA-pairs in the prompt as context, improves MistralO's item-level multiple-choice question responses as well as the total score on each of questionnaire. In essence, we provide MistralO with a complete mental state to predict responses given LLM representations of those states.

We provide all the Level 3 QA-pairs in order (cumulative), append a PHQ-9 question and sample MistralO's response one by one, appending each as we go. We obtain 50 samples from the model for each question, which we then average across.

In Figure \ref{fig:study2_llm_wholeqs_level_scores}A, we plot participants' PHQ-9 total scores (overall depression severity) vs. MistralO's PHQ-9 total scores given a full set of cumulative Level 3 QA-pairs. We report a strong correlation between the total scores ($r=0.844$), as well as a correction of the bias we observed when estimating total scores from item-level QA-pairs before. 

In Figure \ref{fig:study2_llm_wholeqs_level_scores}B-J, we plot item-level predictions against true scores for each PHQ-9 question given a full set of participants' Level 3 QA-pairs. We observe an overall improvement in predicting item-level scores. Importantly, we observe less bias toward negative scores, especially in the no symptom response, indicating item-level de-biasing. Both total and item-level results indicate that providing a full set of QA-pairs leads to a richer representation of participant's mental state, allowing the model to make more accurate predictions.

Notably, the model showed good prediction of suicidality scores (Question 9, $r=0.547$). Given this, we wondered whether we can skip the costly process of response sampling and predict suicidality severity using an average representation across all Level 3 QA-pairs hidden states (here the QA-pair contexts accumulate with each pair). We repeated the latent analysis process and fit a ridge regression model to predict suicidality severity (z-scored) from PCA-reduced average across Level-3 QA-pairs hidden state. We compare this to a baseline approach, where we take raw participants' Question 1-8 scores and fit a cross-validated ridge regression to predict Question 9 scores. We plot the PHQ-9 Question 9 held-out predictions from both approaches in Figure \ref{fig:study2_llm_wholeqs_level_scores}K. Correlation between hidden-states predicted and true scores, $r=0.499$ (in purple), is comparable to that obtained with the sampling approach. Interestingly, we see some false positives, which are worth investigating as they may indicate participant's unwillingness to disclose suicidality - a widespread issue preventing timely intervention \citep{hallford_disclosure_2023}. Additionally, while predicting Question 9 scores from the remaining scores (in red) performs bettter ($r=0.595$) than hidden states approach, it does so at a cost of using less rich representations which can't be used to generate or steer open-ended responses - the main promise of LLMs. Overall, this result suggests the existence of the relevant subspace predictive of suicidality severity, offering an efficient and equally good sampling-free method.

%%%%%%%% RESULTS ------------- END

%%%%%%%% DISCUSSION ------------- START
\section{Discussion and conclusion}
% We explored the use Large Language Models (LLMs) to understand the structure of human thought in health and mental illness. The core idea revolves around the parallels between how LLMs generate text given a context (\texttt{context => text}) and how humans express thoughts and behaviours based on their mental state (\texttt{mental state => thought \& behaviour}). 

% This research investigates if LLMs can quantify and model human mood states, and further stipulates that these representations

% can be linked with behavioural measures of semantic structure, as well as used to decode moods states from neural activity during passive listening.

% Moving forward, several key analyses will be conducted to solidify these initial findings. A larger dataset ($N\approx1000$) will be collected to enhance the robustness and generalizability of our results. We will also explore how well MistralO responds to the whole questionnaire given different combinations of open-ended responses as context. This should allow for a better approximation of the symptom severity as measured with total scores, but also a better representation of the underlying mental state, improving the covariance structure match. Critically, hidden state measurements will be collected throughout the questionnaire process, providing a dynamic view of how these representations evolve, allowing for a validation of our latent structure measures in a dynamic setting. 

%%%%%%%% MAYBE ------------- START


%%%%%%%% MAYBE ------------- END

We explored whether an LLM (MistralO) can quantify diverse depressive mood states from open-ended responses to a depression questionnaire and generalise this quantification. Participants provided written answers to open-ended questions derived from the PHQ-9, and responses to standard multiple-choice questionnaires (PHQ-9, GAD-7, SDS). LLM-predicted responses to multiple-choice questions, given the open-ended answers, correlated strongly (0.52-0.84) with participant scores, demonstrating LLM's ability to represent mental states from open-ended descriptions, albeit with a bias towards "depressed" scores. Furthermore, MistralO generalised from open-ended responses to predict scores on other questionnaires (GAD-7, SDS), capturing trends in total scores and reproducing some of the covariance structure between questionnaires. 
% The study also investigated whether LLM hidden states could predict SDS factor scores using ridge regression. 
The study demonstrated promise in predicting certain factors scores (e.g., "Depression: Hopelessness \& Apathy," "Somatic \& Emotional Distress") from the model hidden state representations of open-ended responses, suggesting that these representations contain information about the underlying mental state dimensions. Lastly, we showed that providing the model with a full-set of participants' open-ended responses to represent their mental state improves model performance compared with item-level representations and corrects the severity bias. Importantly, we were able to predict suicidality scores both by sampling responses and by reading it out from a hidden states subspace representing participant's mental state.

To further refine our understanding of the LLM's latent structure, our future work will explore applying supervised, low-dimensional Autoencoders \citep{le_supervised_2018}, constrained with participants' true factor scores during training. This could provide a more constrained representation of the relevant dimensions of psychopathology. Importantly, we will then investigate perturbing the represented factor scores in a specific direction, thereby inducing targeted changes in responses to questionnaires and beyond. This approach could offer a robust measure of mental states that we hope to evaluate as predictors of neural activity to semantically and emotionally rich naturalistic stories \citep{goldstein_shared_2022,tang_semantic_2023,tang_semantic_2025}. It also stands as a precursor for an experimental stimulus design approach allowing us to create targeted semantic stimuli that drive the neural activity in the relevant mood-subspace \citep{tuckute_driving_2024}.

% This lays ground for an examination of how to manipulate LLM-represented latent structures so as to generate text that loads onto specific dimensions of psychopathology.
% \section{Conclusion}
Overall, the research demonstrates the potential of LLMs to quantify and model human mood states from both structured questionnaire data and unstructured open-ended responses. The ability to generalise from one type of mood assessment (open-ended responses) to predict scores on other assessments (multiple-choice questionnaires) is promising and indicative of LLMs ability to represent relevant dimensions of mood states, as is further demonstrated by successfully utilising hidden state representations to predict factor scores and suicidality severity. The reliability of these results depends both on how informative the questions and the participant responses are. Nevertheless, if used correctly, this approach could supplement mental state assessment in the clinic and within scalable online platforms. Notably, this approach could help in identifying those at higher risk of suicide in situations where they may be reluctant to disclose suicidality \citep{hallford_disclosure_2023}.

% These promising results warrant a further pursuit of this approach to define measures that can be used to better describe mental states in standard settings but also in sensitive settings (suicidal ideation) with the help of neural decoders. 

% Lastly, LLM-based approach also offer a way to experiment in-silico to help define small and easily scalable personalised interventions to shift participants' underlying latent structure resulting in desirable, less distressing thoughts.

%%%%%%%% DISCUSSION ------------- END




% \clearpage
% \section*{Acknowledgments}
% \acknInfo

\section*{Limitations}
There are several limitation to our study. Firstly, we only focused on three self-report questionnaires (SDS, GAD-7, PHQ-9). We also used mental state descriptions based on a specific, structured PHQ-9 questionnaire. It's unclear how well the LLM would perform with other questionnaires and if the findings would generalise to other types of assessments. The limitation is further underlined by the findings that many of the depression questionnaires weakly overlap, threatening a consistent and generalisable measurement of depression severity \cite{fried_52_2017}. However, the LLM-based approach may offer a partial solution, as it could allow for a very rich representation of mental states given a sufficiently informative input, particularly if supplemented with neural activity representations. 

Secondly, we acknowledge that the quality and informativeness of open-ended responses may have varied across participants (especially the slow-typers) due to 1.5 minutes time-limit and a minimum 30-words requirement. We also didn't control for confounding variables such as current mood due to the most recent events, which may not be informative about a more persistent depressive state. Furthermore, we recruited a specific online population on Prolfic, selecting for those with depressive symptoms. Further analyses would have to be performed to establish whether LLMs are able to quantify other mental states across a diverse subsets of populations, including healthy controls as well as participants in lab environments. 

Moreover, in the factor-analysis approach, we assumed a specific model, which might not be optimal for our dataset. Finding the best-fitting factor analysis model, in combination with collecting a larger dataset to potentially better represent the under-defined factors 3 and 4, could alleviate some of these concerns and potentially allow drawing more meaningful conclusions about the mapping between hidden states and these factors. Future analyses will consider deriving a factor space from a combination of questionnaires, which might offer a more robust factor quantification.

Additionally, we performed our analysis using one LLM (Mistral-7B-OpenOrca \citep{lian2023mistralorca1, jiang_mistral_2023}). A replication using different models would ensure the generalisability of our approach, especially with smaller models that may be better positioned for researchers and clinicians to apply at scale. Our future work will consider other models, such as Gemma-2-2b \citep{team_gemma_2024}. Given how quickly the field of LLMs progresses, we are confident our results would replicate.

Lastly, due to the sensitive nature of the collected data we won't be releasing the dataset at this stage.

\section*{Ethical considerations}
As with any work involving computing resources, there are environmental impacts. Experiments were conducted using NVIDIA L4 and T4 GPUs from Google Cloud Platform in region us-west1, which has a carbon efficiency of 0.3 kgCO$_2$eq/kWh. A cumulative of 500 hours of computation contributed to emissions of approximately 10.8 kg of CO$_2$eq. We aimed to use a relatively small model for our analyses, which minimised our emissions. Estimations were conducted using the \href{https://mlco2.github.io/impact#compute}{MachineLearning Impact calculator} presented in \citealp{lacoste2019quantifying}.

Given the sensitive nature of the questions asked, we took necessary measures to ensure participants are comfortable in answering the questions and provided them with mental health resources. Participants were also given the option to withdraw at any point. We see our research to have a potentially high positive societal impact, aiming at mental health assessment and improved, personalised treatment. The translation of our research could be applied as scale for example by integrating with mental health chatbots or with smartphone ambulatory data, helping as many as possible.

We also see potential future risks, when similar methods could be used to infer aspects of someone's mental state that they wish not to disclose. This is pertinent in the realm of vast social media data. Someone's mental state representations combined with robust methods of altering it to steer thoughts of behaviours may be hijacked by bad actors with goals of maximising profits at any cost. It is particularity important to put guardrails and regulations in place to prevent such activities, especially when we consider vulnerable populations being the target such as depressed individuals or gambling addicts. Equally, a potential risk arises in situations where employers, with access to employee's vast set of written communication, would be interested in predicting for example sick leave due to mental health reasons. In cases where employers would be motivated by minimising absence costs, they could resort to unethical practices of issuing redundancies to flagged individuals based on the predictions from the mental state representations.
% Bibliography entries for the entire Anthology, followed by custom entries
%\bibliography{anthology,custom}
% Custom bibliography entries only
\bibliography{custom}

% \clearpage
\appendix
\section{Participant information}\label{apdx:ppt_info}
\subsection{Recruitment}
The study was approved by \rethicsInfo. All participants (18 or above) provided online consent after reading provided Participant Information Sheet. Participants were informed that they would be asked questions about their mood and feelings and we have provided information about ways to seek help should they feel affected by the issues raised by these questions. We reimbursed participants at a rate of £8.21/h, which as rate approved by the Ethics Committee (appropriate for the demographic). For the analyses, we have pseudo-anonymised the data (changed the Prolific ID, which in itself ensures participants anonymity), but cannot ensure no identifiable is contained in the open-ended responses. Participants were free to withdraw at any point, and we have emphasised that throughout the study. 

Participants were instructed that they will have to provide written responses as well as multiple-choice responses. They answered one open-ended question at a time, while the multiple-choice questions were displayed on one scrollable page. See detailed experimental instructions in Figures \ref{fig:instr_welcome}-\ref{fig:instr_closed_qs}


\subsection{Inclusion/exclusion criteria}
We recruited a mixed sample of participants from Prolific to represent diverse range of depressive symptoms. In addition to our first general population sample of $\Nv$ participants, we recruited $\Nvd$ participants, who answered that they experience depression, as well as $\Nvdd$ participants, who additionally reported having been diagnosed with depression. We screened for Prolific participants who were willing to participate in a study about sensitive topics (mental health, emotions, feelings) and potentially harmful content, who had approval rate of 95-100\%, at least 5 previous submissions.

We excluded participants who meet at least one of the criteria:
\begin{itemize}
    \item timed out or provided less than 30-word response to any of the questions more than 4 times in total,
    \item failed two attention checks.
\end{itemize}
For the hidden states analysis, we excluded 41 participants who missed at least one response to any of the multiple-choice questions or provided less than 20 words in any single open-ended response, resulting in $\Nhs$ participants.

\subsection{Instructions}
See detailed experimental instructions in Figures \ref{fig:instr_welcome}-\ref{fig:instr_closed_qs}.
\begin{figure*}[!t]
    \centering
    \includegraphics[width=1\linewidth]{figs/instr/welcome.png}
    % \vspace{-2.25em}
    \caption{Experiment welcome screen}
    \label{fig:instr_welcome}
\end{figure*}

\begin{figure*}[!t]
    \centering
    \includegraphics[width=1\linewidth]{figs/instr/instr1.png}
    % \vspace{-2.25em}
    \caption{First page of the experiment instructions}
    \label{fig:instr_p1}
\end{figure*}

\begin{figure*}[!t]
    \centering
    \includegraphics[width=1\linewidth]{figs/instr/instr2.png}
    % \vspace{-2.25em}
    \caption{Second page of the experiment instructions}
    \label{fig:instr_p2}
\end{figure*}


\begin{figure*}[!t]
    \centering
    \includegraphics[width=1\linewidth]{figs/instr/instr3.png}
    % \vspace{-2.25em}
    \caption{Third page of the experiment instructions}
\vspace{3em}
    \label{fig:instr_p3}
\end{figure*}

\begin{figure*}[!t]
    \centering
    \includegraphics[width=1\linewidth]{figs/instr/instr4.png}
    % \vspace{-2.25em}
    \caption{Fourth page of the experiment instructions}
    \label{fig:instr_p4}
\end{figure*}

\begin{figure*}[!t]
    \centering
    \includegraphics[width=1\linewidth]{figs/instr/open_question.png}
    % \vspace{-2.25em}
    \caption{Example open-ended question screen}
    \label{fig:instr_open_qs}
\end{figure*}

\begin{figure*}[!t]
    \centering
    \includegraphics[width=1\linewidth]{figs/instr/multiple_choice.png}
    % \vspace{-2.25em}
    \caption{Example multiple-choice question screen}
\vspace{3em}
    \label{fig:instr_closed_qs}
\end{figure*}


\subsection{Depression total scores}\label{apdx:depression_totals}
We plot the distributions of total scores for PHQ-9, GAD-7, SDS as well as Level 1 and 2 multiple-choice questions in Figure \ref{fig:study2_totals_hist}.
\begin{figure*}[!t]
    \centering
    \includegraphics[width=1\linewidth]{figs/prelim/totals_histograms.pdf}
    % \vspace{-2.25em}
    \caption{\textbf{A-E}: Total score histograms and density estimate plots for each questionnaire (PHQ9, GAD7, SDS) and Level 1, 2 multiple-choice questions. We also report the mean, standard deviation and median of the distribution for each questionnaire.}
    \label{fig:study2_totals_hist}
\end{figure*}

\subsection{Text writing statistics}\label{apdx:write_stats}
 We plot the distributions of participant's item-level word rates (word/second) and word count, as well as item-level time-to-write (seconds) in Figure \ref{fig:study2_word_stats}.
\begin{figure*}[!t]
    \centering
    \includegraphics[width=1\linewidth]{figs/prelim/word_stats.pdf}
    % \vspace{-2.25em}
    \caption{\textbf{A-B}: Histogram and density estimates for item-level response word-rate (word/second) and word-count. \textbf{C-E} Violin plots of times-to-write for each level's question.}
    \label{fig:study2_word_stats}
\end{figure*}

\section{Open-ended questions}\label{apdx:open_qs}
\subsection{Open-ended questions - Level 1}
\begin{enumerate}
    \item In the past two weeks, how would you describe your overall emotional well-being, including your mood, feelings, thoughts about yourself, and any changes in your behaviour or daily functioning? Please provide examples of situations or experiences that best illustrate these aspects.
\end{enumerate}

% \vspace{12em}
\subsection{Open-ended questions - Level 2}
\begin{enumerate}
    \item Can you describe how you felt when doing activities and how your mood has been generally in the past two weeks? Can you provide typical examples that would best capture your feelings?
    \item In the past two weeks, how have your sleep patterns, energy levels, appetite, focus and cognitive abilities been? Are there any occasions that you could share where they have been affected in any way?
    \item Please provide typical examples of your thoughts and feelings about yourself and your sense of self-worth that you've had over the last two weeks.
\end{enumerate}

\subsection{Open-ended questions - Level 3}
\begin{enumerate}
    \item Could you share any activities or events from the past two weeks that made you feel bothered because of a lack of interest or pleasure in doing them?
    \item For the past two weeks, have you been bothered by your mood and how you felt generally? Were there any situations when you felt down, depressed, or hopeless?
    \item Can you provide examples of how your sleep has been in the past two weeks? Have you been bothered by challenges with falling asleep, staying asleep, or even sleeping too much?
    \item Have you been bothered by your energy levels over the past two weeks? Can you recall situations when it comes to feeling tired/lively or low/high on energy?
    \item In the past two weeks, have you been bothered about your appetite? Can you describe your typical attitude towards food - maybe you have noticed something unusual, like changes in how much you're eating or not eating?
    \item In the past two weeks, have you been bothered by feelings about yourself? In what situations did you feel proud or like a failure? Did you feel you met your own and your family's expectations, or let them down?
    \item In the past two weeks, have you been bothered by your ability to concentrate and focus? Please describe how it felt to do things that require you to concentrate for a while, like working, reading, or watching movies?
    \item Can you describe situations over the past two weeks when you were bothered by feeling slower than usual in terms of thinking, speaking, or just acting - or situations where you felt fidgety and restless?
\end{enumerate}



\section{Multiple-choice questions}\label{apdx:closed_qs}
For Level 1 and 2, participants were instructed to "Please select a response to the following statement that best describes you over the last 2 weeks.". These are answered on a scale - Very Good (0), Good(1), Bad (2), Very Bad (3).

\subsection{Closed questions - Level 1}
\begin{enumerate}
    \item My overall emotional well-being, including my mood feelings, thoughts about myself, and any changes in my behaviour or daily functioning have been
\end{enumerate}

\subsection{Multiple-choice questions - Level 2}
\begin{enumerate}
    \item My feelings when doing activities and my overall mood have been
    \item My sleep patterns, energy levels, appetite, focus and cognitive abilities have been
    \item My thoughts and feelings about myself and my sense of self-worth have been
\end{enumerate}

\subsection{PHQ-9 items - Level 3}
Participants were instructed to consider "Over the last 2 weeks, how often have you been bothered by any of the following problems?". These are answered on a scale - Not at all (0), Several days (1), More than half the days (2), Nearly every day (3).
\begin{enumerate}
    \item Little interest or pleasure in doing things.
    \item Feeling down, depressed, or hopeless.
    \item Trouble falling or staying asleep, or sleeping too much.
    \item Feeling tired or having little energy
    \item Poor appetite or overeating.
    \item Feeling bad about yourself - or that you are a failure or have let yourself or your family down.
    \item Trouble concentrating on things, such as reading the newspaper or watching television.
    \item Moving or speaking so slowly that other people could have noticed? Or the opposite - being so fidgety or restless that you have been moving around a lot more than usual.
    \item Thoughts that you would be better off dead or of hurting yourself in some way.
\end{enumerate}

\subsection{SDS items} \label{apdx:sds_questions}
These are answered on a scale - A little of the time (1), Some of the time (2), Good part of the time (3), Most of time (4). Questions 2, 5, 6, 11, 12, 14, 16, 17, 18 and 20 are reverse scored. We also include a short name for each question.
\begin{enumerate}

\item I feel down-hearted and blue - Sad
\item Morning is when I feel the best - Morning Best
\item I have crying spells or feel like it - Crying
\item I have trouble sleeping at night - Sleep Issues
\item I eat as much as I used to - Eating Normal
\item I still enjoy sex - Sex Enjoyment
\item I notice that I am losing weight - Weight Loss
\item I have trouble with constipation - Constipation
\item My heart beats faster than usual - Fast Heart
\item I get tired for no reason - Tired
\item My mind is as clear as it used to be - Clear Thinking
\item I find it easy to do the things I used to - Task Ease
\item I am restless and can’t keep still - Restless
\item I feel hopeful about the future - Hopeful
\item I am more irritable than usual - Irritable
\item I find it easy to make decisions - Decisive
\item I feel that I am useful and needed - Needed
\item My life is pretty full - Full Life
\item I feel that others would be better off if I were dead - Suicidal
\item I still enjoy the things I used to do - Enjoyment
\end{enumerate}

\section{Example item-level prompt}\label{apdx:item_prompt}
% \begin{shadedquotation}
\begin{lstlisting}
<start_of_turn>user
Please answer the following question in detail.

Question:
For the past two weeks, have you been bothered by your mood and how you felt generally? Were there any situations when you felt down, depressed, or hopeless?

Answer:<end_of_turn>
<start_of_turn>model
 *** Participant's response redacted*** <end_of_turn>
<start_of_turn>user
Please answer the following question.

Over the last 2 weeks, how often have you been bothered the following problem?

Problem: Feeling down, depressed, or hopeless.

Please answer by selecting only one answer from the scale below:

- Not at all 
- Several days
- More than half the days
- Nearly every day

Answer:<end_of_turn>
<start_of_turn>model
\end{lstlisting}

% \end{shadedquotation}

% \label{sec:appendix}


\section{Factor analysis}\label{apdx:fa_results}
We used \texttt{factor-analyzer v0.5.1} Python package to fit the four-factor with promax rotation factor analysis model. We found and named the following four factors based on all participants' SDS responses. 
\begin{itemize}
    \item \textit{Factor 1} - \textbf{Depression: Hopelessness \& Apathy}: Sad, Morning Best, Sex Enjoyment, Clear Thinking, Task Ease, Hopeful, Decisive, Needed, Full Life, Enjoyment.
    \item \textit{Factor 2} - \textbf{Somatic and Emotional Distress}: Sad, Crying, Sleep Issues, Constipation, Fast Heart, Tired, Restless, Irritable, Suicidal.
    \item \textit{Factor 3} - \textbf{Cognitive Functioning}: Task Ease, Clear Thinking.
    \item \textit{Factor 4} - \textbf{Appetite and Weight Changes}: Eating Normal, Weight Loss.
\end{itemize}
\begin{figure*}[!t]
	\centering
    % \hspace{-9mm} 
	\includegraphics[width=1\linewidth]{figs/latent_regression/factor_loading_and_counts_cov_sds.pdf}
	% \vspace{-2em}
	\caption{\textbf{A}: Covariance of participant SDS responses. \textbf{B}: Factor loading matrix from the factor analysis model with four factors and promax rotation for SDS questionnaire response. \textbf{C}: Heatmap of counts of scores for each SDS question.}
	\label{fig:study2_factor_load_counts_sds}
\end{figure*}

We plot the SDS covariance of responses, factor loading matrix and question-response distribution in Figure \ref{fig:study2_factor_load_counts_sds}.
% \newpage
\section{Packages}
We used the following packages for LLM sampling and analysis:
\begin{itemize}
    \item \texttt{python v3.12.4}
    \item \texttt{transformers v4.37.2}
    \item \texttt{torch v2.4.1}
    \item \texttt{scikit-learn v1.5.2}
    \item \texttt{scipy v1.14.1}
    \item \texttt{factor-analyzer v0.5.1}
    \item \texttt{pandas v2.2.3}
    \item \texttt{numpy v2.2.1}
    \item \texttt{seaborn v0.13.2}
    
\end{itemize}

\section{LLM details}\label{apdx:llm_dets}
We used Mistral-7B-OpenOrca large language model \cite{lian2023mistralorca1} (referred as MistralO). The model is a fine-tuned version of Mistral-7B-v0.1 model \citep{jiang_mistral_2023}. The model has approximately 7.23 billion parameters. The original model was finetuned on a rich collection of augmented FLAN data aligns \citep{longpre_flan_2023} based on Orca model \citep{mukherjee_orca_2023}. Essentially the model was fine-tuned on rich signals from GPT-4 model \citep{openai_gpt-4_2023} that include step-by-step thought processes and complex instructions guided by teacher assistance.

To obtain responses from unaltered token distribution, we set the sampling parameters to Hugging Face \texttt{transformers v4.37.2} library's default values for: temperature$=1$, top-p$=1$.

The original model was released under Apache 2.0 license allowing us to use the fine-tuned version without restrictions, which is also under the Apache 2.0 license.

Through the development of this project, including the preliminary stages, we relied on NVIDIA T4 and L4 GPU instances.  In total we used approximately 500 GPU hours as estimated with Google Cloud billing report.
\end{document}


