\section{Literature Review}
Several studies have demonstrated the efficacy of AI algorithms in analyzing medical imaging and genomic data to improve diagnostic accuracy and prognostic assessments. For instance, research by ____ revealed a novel methodology for analyzing genomic activity across three laterality classes: left, right, and bilateral. Utilizing a dataset of 450 samples, the researchers successfully identified key genes that serve as indicators for each class, achieving an impressive accuracy rate of nearly $99\%$. Among the significant differentially expressed genes are RTN1, HLA-DMB, and MRI1, which effectively distinguish between the classes and correlate with disease progression____.Additionally, ____ investigated the variability in aggressiveness of prostate cancer by analyzing a dataset of 106 RNA-Seq samples. The study employed machine learning techniques to identify 44 differentially expressed transcripts associated with cancer progression. Notably, the transcripts USP13 and PTGFR were found to have reduced expression in advanced stages, correlating with findings in other cancers like breast and ovarian ____.

The authors in ____, introduced a novel histopathological dataset for prostate cancer detection, comprising over 2.6 million tissue patches from $430$ fully annotated scans. The dataset included $4675$ scans with binary diagnoses and 46 scans evaluated by histopathologists. They also presented a machine learning framework for identifying cancerous tissue regions and predicting scan-level diagnoses, employing thresholding to handle uncertain cases. Their approach, which utilized ensembles of deep neural networks at various scales, achieved $94.6\%$ accuracy in patch-level recognition and demonstrated high statistical agreement with nine human histopathologists in scan-level diagnosis____.
____ examined the role of AI and machine learning in managing prostate cancer, highlighting the current trends and future possibilities. AI has improved digital pathology, facilitating quicker and more precise diagnoses, enhancing lesion detection, and predicting patient outcomes. It has also contributed to predicting radiotherapy toxicity and increasing the autonomy of surgical robots for independent problem-solving ____. 

AI has revolutionized the detection of prostate cancer with MRI images, ____ presented a fully automated deep learning system designed for the analysis of MRI-visible lesions. This system leveraged data collected from two distinct institutions. The lesions were meticulously annotated by a qualified radiologist, and subsequent biopsies were performed to establish a reliable ground truth, which was utilized for training the UNet and AH-Net architectures. The dataset comprised 525 patients, split into training ($n=368$), validation ($n=79$), and test ($n=78$) cohorts. Results indicated that AHNet outperformed UNet in validation sensitivity ($74.4\%$ vs. $70.9\%$) and false positives ($0.87$ vs. $1.41$). In the test cohort, UNet achieved a sensitivity of  $72.8\%$ compared to AHNet's $63.0\%$. Overall, the DL-based AI system shows promise for aiding radiologists in detecting prostate cancer lesions, despite challenges with false positives____.
    
\begin{figure}
\centering
\tikz \node [scale=0.9, inner sep=0] {
\begin{tikzpicture}
  \path[mindmap,concept color=black,text=white]
    node[concept] {Prostate \\Cancer \\Staging}
    [clockwise from=0]
    child[concept color=green!50!black, ] {
      node[concept] {AJCC TNM}
      [clockwise from=90]
      child { node[concept] {Tumor  \\ \textit{(T)}}
        child[concept color=red!50!black] { node[concept] {Clinical} }
        child[concept color=red!50!black] { node[concept] {Pathological} }
      }
      child { node[concept] {Nodes  \\ \textit{(N)}} }
      child[] { node[concept] {Metastasis \\ \textit{(M)}} }
    }
    child[concept color=blue]{node[concept] {Gleason Score}}
    child[concept color=orange]{node[concept] {Prostate-specific Antigen \textit{(PSA)}}};
\end{tikzpicture}
};
\caption{Three methods of prostate cancer staging: the Prostate-specific Antigen (PSA), Gleason Score, and the AJCC TNM system, where AJCC stands for the American Joint Committee on Cancer. This system categorizes the cancer based on Tumor (T) size and extent, Nodes (N) for lymph node involvement, and Metastasis (M) for the presence of distant spread.}
\label{fig:enter-label}
\end{figure}