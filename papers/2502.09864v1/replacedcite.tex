\section{Related Work}
\label{sec:related}

\PARAGRAPH{Machine clears for performance degradation}
%
____ introduce HyperDegrade, a novel technique that combines previous
approaches to performance degradation with simultaneous multithreading (SMT)
architectures. It investigates the root causes of performance degradation using
cache eviction, largely attributed to machine clears. The slowdown produced by
HyperDegrade is significantly higher than previous approaches, leading to an
increased time granularity for \FR{} attacks.

\PARAGRAPH{Constructing transient execution gadgets via machine clears}
%
____ investigate the root causes of a subset of transient execution
attacks, specifically focusing on machine clears. The authors identify previously
unexplored types of machine clears induced by different functionality such as
floating point operations, self-modifying code detection, memory ordering, and
memory disambiguation. These machine clear events create new transient execution
windows, widening the horizon for known attacks and introducing new attack
gadgets.

\PARAGRAPH{Cache flushing as a microarchitectural timing attack vector}
%
The Flush+\-Flush technique by ____
exploits the \texttt{clflush} instruction's
execution time variance due to early abort behavior.
%
The key insight of the Flush+\-Flush attack is that the execution time of the
\texttt{clflush} instruction varies depending on whether the data is cached or
not. If the data is not cached, the execution time is shorter; if the data is
cached, it takes longer. While \MCHAMMER{} shares some similarity to the
Flush+\-Flush technique, the root causes are significantly different:
(i) According to the authors \cite[Section~3]{2016:flfl},
      ``[Flush+\-Flush] builds upon the observation that the \texttt{clflush}
      instruction can abort early in case of a cache miss'',
      while the \MCHAMMER{} root cause is linked to machine clears;
(ii) The Flush+Flush technique yields timing differences between 9--12 clock
      cycles \cite[Figure~1]{2016:flfl}, while \MCHAMMER{} yields timing differences
      of roughly 157 clock cycles (see \autoref{fig:latencies_cdf}),
      a huge improvement in terms of trace granularity and error-handling ability.
 %