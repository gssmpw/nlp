\documentclass[11pt]{article}

\def\withcolors{0}
\def\withnotes{1}


\usepackage{hyperref}
\hypersetup{
    unicode=false,          % non-Latin characters in Acrobat’s bookmarks
    colorlinks=true,        % false: boxed links; true: colored links
    linkcolor=blue,          % color of internal links (change box color with linkbordercolor)
    citecolor=darkgreen,        % color of links to bibliography
    filecolor=magenta,      % color of file links
    urlcolor=cyan           % color of external links
}

\usepackage{xspace}
\usepackage{graphicx}
\usepackage{waingarten}
\usepackage{paralist}
%\usepackage[linesnumbered,ruled]{algorithm2e}
\usepackage{thmtools}
\usepackage{thm-restate}
\usepackage{color}
\usepackage{setspace}


% Proof environment: redefine
\def\FullBox{\hbox{\vrule width 8pt height 8pt depth 0pt}}
\newcommand{\QED}{\;\;\;\FullBox}
\renewenvironment{proof}{\noindent{{\textbf{Proof:}~}}} {\hfill\QED}
\providecommand{\email}[1]{\href{mailto:#1}{\nolinkurl{#1}\xspace}}
\newenvironment{proofof}[1]{\noindent{\bf Proof of {#1}:~}}{\hfill\(\QED\)}


\def\FullBox{\hbox{\vrule width 8pt height 8pt depth 0pt}}

\newcommand{\new}[1]{{\color{purple} {#1}}} % new
\newcommand{\strikeout}[1]{{\color{gray} #1}}
\newcommand{\restrict}[1]{\upharpoonright_{#1}}

%
% --- inline annotations
%
\newcommand{\red}[1]{{\color{red}#1}}
\newcommand{\todo}[1]{{\color{red}#1}}
\newcommand{\TODO}[1]{\textbf{\color{red}[TODO: #1]}}
% --- disable by uncommenting  
% \renewcommand{\TODO}[1]{}
% \renewcommand{\todo}[1]{#1}



\newcommand{\VLM}{LVLM\xspace} 
\newcommand{\ours}{PeKit\xspace}
\newcommand{\yollava}{Yo’LLaVA\xspace}

\newcommand{\thisismy}{This-Is-My-Img\xspace}
\newcommand{\myparagraph}[1]{\noindent\textbf{#1}}
\newcommand{\vdoro}[1]{{\color[rgb]{0.4, 0.18, 0.78} {[V] #1}}}
% --- disable by uncommenting  
% \renewcommand{\TODO}[1]{}
% \renewcommand{\todo}[1]{#1}
\usepackage{slashbox}
% Vectors
\newcommand{\bB}{\mathcal{B}}
\newcommand{\bw}{\mathbf{w}}
\newcommand{\bs}{\mathbf{s}}
\newcommand{\bo}{\mathbf{o}}
\newcommand{\bn}{\mathbf{n}}
\newcommand{\bc}{\mathbf{c}}
\newcommand{\bp}{\mathbf{p}}
\newcommand{\bS}{\mathbf{S}}
\newcommand{\bk}{\mathbf{k}}
\newcommand{\bmu}{\boldsymbol{\mu}}
\newcommand{\bx}{\mathbf{x}}
\newcommand{\bg}{\mathbf{g}}
\newcommand{\be}{\mathbf{e}}
\newcommand{\bX}{\mathbf{X}}
\newcommand{\by}{\mathbf{y}}
\newcommand{\bv}{\mathbf{v}}
\newcommand{\bz}{\mathbf{z}}
\newcommand{\bq}{\mathbf{q}}
\newcommand{\bff}{\mathbf{f}}
\newcommand{\bu}{\mathbf{u}}
\newcommand{\bh}{\mathbf{h}}
\newcommand{\bb}{\mathbf{b}}

\newcommand{\rone}{\textcolor{green}{R1}}
\newcommand{\rtwo}{\textcolor{orange}{R2}}
\newcommand{\rthree}{\textcolor{red}{R3}}
\usepackage{amsmath}
%\usepackage{arydshln}
\DeclareMathOperator{\similarity}{sim}
\DeclareMathOperator{\AvgPool}{AvgPool}

\newcommand{\argmax}{\mathop{\mathrm{argmax}}}     




\title{Monotonicity Testing of High-Dimensional Distributions with Subcube Conditioning\footnote{This work was partially completed while authors were visiting the Simons Institute for the Theory of Computing at UC Berkeley. DC is supported by National Science Foundation (NSF) under grants CCF-2041920, CCF-2402571.
XC is supported   by NSF grants CCF-2106429, CCF-2107187. CS is supported by NSF grants CCF-1740850, CCF-1839317, CCF-2402572, and DMS-2023495.
EW is supported by NSF grant  CCF-2337993.}}

%%
%% The "author" command and its associated commands are used to define
%% the authors and their affiliations.
%% Of note is the shared affiliation of the first two authors, and the
%% "authornote" and "authornotemark" commands
%% used to denote shared contribution to the research.
\author{
Deeparnab Chakrabarty\\
Dartmouth College\\
{\tt deeparnab@dartmouth.edu}
\and
Xi Chen\\
Columbia University\\
{\tt xichen@cs.columbia.edu}
\and
Simeon Ristic\\
University of Pennsylvania\\
{\tt sristic@seas.upenn.edu}
\and
C. Seshadhri\\
University of California, Santa Cruz\\
{\tt sesh@ucsc.edu}
\and
Erik Waingarten \\
University of Pennsylvania\\
{\tt ewaingar@seas.upenn.edu}
}





    %\author {
      %Deeparnab Chakrabarty\thanks{Dartmouth College}
      %\and
      %Xi Chen\thanks{Columbia University}
      %\and
      %Simeon Ristic\thanks{}
      %\and
      %C. Seshadhri\thanks{University of California, Santa Cruz}
      %\and
      %Erik Waingarten\thanks{University of Pennsylvania.}
   % }
   \date{}
    %
\begin{document}
\maketitle

\begin{abstract}
\ignore{Distribution testing has led to a plethora of results, but for most interesting properties, the query\xnote{Should this be sample complexity?} complexity
is polynomial in the domain size. For high-dimensional domains, such complexities are infeasible.
The subcube conditional model (Canonne-Ron-Servedio, SICOMP 2015 and Bhattacharyya-Chakraborty, TOCT 2018)
allows for sampling subcubes of the domain, often leading to queries polynomial in the \emph{dimension}.

We study the classic property of monotonicity of distributions over $\{0,1\}^n$, under the subcube conditional model.
Consider an unknown input distribution $p$ over $\{0,1\}^n$. A distribution is monotone if 
$p(x) \leq p(y)$ for any $x \preceq y$, where $\preceq$ denotes the standard coordinate-wise partial order over $\{0,1\}^n$.
The aim is to distinguish monotone $p$ from $p$ being $\eps$-far (in TV distance) from monotone.
Up to logarithmic factors, we resolve the query complexity to be $\Theta(n/\eps^2)$.

For the upper bound, we take inspiration from the literature of monotonicity testing of Boolean functions.
We run an ``edge tester" for testing distribution monotonicity. To bound the query complexity, we
rely on directed isoperimetric theorems that relate the "directed boundary" of functions
to their distance to monotonicity. We prove a real-valued analogue of the directed Talagrand theorem (Khot-Minzer-Safra, SICOMP 2018),
which is a key tool in our proof.
For the lower bound, we reduce to dealing with product distributions under the standard sampling model. 
We develop lower bounds for determining the bias of coordinates, which are then applied to distributional monotonicity testing.

Applying our methods, we also prove an $\tilde{\Omega}(\sqrt{n}/\eps^2)$ lower bound for uniformity testing
with subcube conditional queries, under the promise that the distribution is monotone. This matches the 
upper bound up to logarithmic factors, which holds for arbitrary distributions. Thus, it proves that monotonicity
does not help for uniformity testing with subcube conditional queries. }

We study monotonicity testing of high-dimensional distributions on $\{-1,1\}^n$ in the model of subcube conditioning, suggested and studied by Canonne, Ron, and Servedio~\cite{CRS15} and Bhattacharyya and Chakraborty~\cite{BC18}. Previous work shows that the \emph{sample complexity} 
of monotonicity testing must be exponential in $n$ %is at most $2^{n - \Omega_{\eps}(n^{1/5})}$ 
(Rubinfeld, Vasilian~\cite{RV20}, and %at least $2^{n(1 - O(\sqrt{\eps}-o(1))}$ (
Aliakbarpour, Gouleakis, Peebles, Rubinfeld, Yodpinyanee~\cite{AGPRY19}). We show that the subcube \emph{query complexity} is $\tilde{\Theta}(n/\eps^2)$,
by proving nearly matching upper and lower bounds. Our work is the first to use directed isoperimetric inequalities (developed for function monotonicity testing) for analyzing a distribution testing algorithm. Along the way, we generalize an inequality of Khot, Minzer, and Safra~\cite{KMS18} to real-valued functions on $\{-1,1\}^n$.

We also study uniformity testing of distributions that are promised to be monotone, a problem introduced by Rubinfeld, Servedio~\cite{RS09}%(Rubinfeld and Servedio, Rand. Struc. and Alg. 2009)
, using subcube conditioning. We show that the query complexity is $\tilde{\Theta}(\sqrt{n}/\eps^2)$. Our work proves the lower bound, which matches (up to poly-logarithmic factors) the uniformity testing upper bound for general distributions (Canonne, Chen, Kamath, Levi, Waingarten~\cite{CCKLW21}).
Hence, we show that monotonicity does not help, beyond logarithmic factors, in testing uniformity of distributions with subcube conditional queries.

\end{abstract}
\thispagestyle{empty}
\newpage
\begin{spacing}{0.75}
\tableofcontents
\end{spacing}
\thispagestyle{empty}


\newpage


\pagenumbering{arabic}
\setcounter{page}{1}


\section{Introduction}


\begin{figure}[t]
\centering
\includegraphics[width=0.6\columnwidth]{figures/evaluation_desiderata_V5.pdf}
\vspace{-0.5cm}
\caption{\systemName is a platform for conducting realistic evaluations of code LLMs, collecting human preferences of coding models with real users, real tasks, and in realistic environments, aimed at addressing the limitations of existing evaluations.
}
\label{fig:motivation}
\end{figure}

\begin{figure*}[t]
\centering
\includegraphics[width=\textwidth]{figures/system_design_v2.png}
\caption{We introduce \systemName, a VSCode extension to collect human preferences of code directly in a developer's IDE. \systemName enables developers to use code completions from various models. The system comprises a) the interface in the user's IDE which presents paired completions to users (left), b) a sampling strategy that picks model pairs to reduce latency (right, top), and c) a prompting scheme that allows diverse LLMs to perform code completions with high fidelity.
Users can select between the top completion (green box) using \texttt{tab} or the bottom completion (blue box) using \texttt{shift+tab}.}
\label{fig:overview}
\end{figure*}

As model capabilities improve, large language models (LLMs) are increasingly integrated into user environments and workflows.
For example, software developers code with AI in integrated developer environments (IDEs)~\citep{peng2023impact}, doctors rely on notes generated through ambient listening~\citep{oberst2024science}, and lawyers consider case evidence identified by electronic discovery systems~\citep{yang2024beyond}.
Increasing deployment of models in productivity tools demands evaluation that more closely reflects real-world circumstances~\citep{hutchinson2022evaluation, saxon2024benchmarks, kapoor2024ai}.
While newer benchmarks and live platforms incorporate human feedback to capture real-world usage, they almost exclusively focus on evaluating LLMs in chat conversations~\citep{zheng2023judging,dubois2023alpacafarm,chiang2024chatbot, kirk2024the}.
Model evaluation must move beyond chat-based interactions and into specialized user environments.



 

In this work, we focus on evaluating LLM-based coding assistants. 
Despite the popularity of these tools---millions of developers use Github Copilot~\citep{Copilot}---existing
evaluations of the coding capabilities of new models exhibit multiple limitations (Figure~\ref{fig:motivation}, bottom).
Traditional ML benchmarks evaluate LLM capabilities by measuring how well a model can complete static, interview-style coding tasks~\citep{chen2021evaluating,austin2021program,jain2024livecodebench, white2024livebench} and lack \emph{real users}. 
User studies recruit real users to evaluate the effectiveness of LLMs as coding assistants, but are often limited to simple programming tasks as opposed to \emph{real tasks}~\citep{vaithilingam2022expectation,ross2023programmer, mozannar2024realhumaneval}.
Recent efforts to collect human feedback such as Chatbot Arena~\citep{chiang2024chatbot} are still removed from a \emph{realistic environment}, resulting in users and data that deviate from typical software development processes.
We introduce \systemName to address these limitations (Figure~\ref{fig:motivation}, top), and we describe our three main contributions below.


\textbf{We deploy \systemName in-the-wild to collect human preferences on code.} 
\systemName is a Visual Studio Code extension, collecting preferences directly in a developer's IDE within their actual workflow (Figure~\ref{fig:overview}).
\systemName provides developers with code completions, akin to the type of support provided by Github Copilot~\citep{Copilot}. 
Over the past 3 months, \systemName has served over~\completions suggestions from 10 state-of-the-art LLMs, 
gathering \sampleCount~votes from \userCount~users.
To collect user preferences,
\systemName presents a novel interface that shows users paired code completions from two different LLMs, which are determined based on a sampling strategy that aims to 
mitigate latency while preserving coverage across model comparisons.
Additionally, we devise a prompting scheme that allows a diverse set of models to perform code completions with high fidelity.
See Section~\ref{sec:system} and Section~\ref{sec:deployment} for details about system design and deployment respectively.



\textbf{We construct a leaderboard of user preferences and find notable differences from existing static benchmarks and human preference leaderboards.}
In general, we observe that smaller models seem to overperform in static benchmarks compared to our leaderboard, while performance among larger models is mixed (Section~\ref{sec:leaderboard_calculation}).
We attribute these differences to the fact that \systemName is exposed to users and tasks that differ drastically from code evaluations in the past. 
Our data spans 103 programming languages and 24 natural languages as well as a variety of real-world applications and code structures, while static benchmarks tend to focus on a specific programming and natural language and task (e.g. coding competition problems).
Additionally, while all of \systemName interactions contain code contexts and the majority involve infilling tasks, a much smaller fraction of Chatbot Arena's coding tasks contain code context, with infilling tasks appearing even more rarely. 
We analyze our data in depth in Section~\ref{subsec:comparison}.



\textbf{We derive new insights into user preferences of code by analyzing \systemName's diverse and distinct data distribution.}
We compare user preferences across different stratifications of input data (e.g., common versus rare languages) and observe which affect observed preferences most (Section~\ref{sec:analysis}).
For example, while user preferences stay relatively consistent across various programming languages, they differ drastically between different task categories (e.g. frontend/backend versus algorithm design).
We also observe variations in user preference due to different features related to code structure 
(e.g., context length and completion patterns).
We open-source \systemName and release a curated subset of code contexts.
Altogether, our results highlight the necessity of model evaluation in realistic and domain-specific settings.





% !TEX root = main.tex

\section{Testing Monotonicity}

In this section, we show that using the directed and real-valued version of Talagrand's inequality, we may design an ``edge tester'' for testing monotonicity of distributions using subcube conditioning. In particular, we give the following theorem.
\begin{theorem}\label{thm:mon-ub}
    There exists an algorithm that receives as input subcube conditioning access to an unknown distribution $p$ supported on $\{-1,1\}^n$, as well as an accuracy parameter $\eps$. The algorithm makes $\tilde{O}(n/\eps^2)$ subcube conditioning queries and satisfies the following guarantees:
    \begin{itemize}
        \item If $p$ is monotone, the algorithm outputs ``accept'' with probability at least $0.9$.
        \item If $p$ is $\eps$-far from monotone, the algorithm outputs ``reject'' with probability at least $0.9$.
    \end{itemize}
\end{theorem}

The algorithm referred to in Theorem~\ref{thm:mon-ub} is given in Figure~\ref{fig:alg}.
% where we take the variable $c_0$ to be a small enough constant factor of $1/\sqrt{\log n}$. 
We break up the proof of \Thm{mon-ub}
into a few parts. First, we argue about the running time.

\begin{figure}
\begin{framed}

\textbf{Algorithm for Testing Monotonicity of Distributions}. We receive as input subcube conditioning access to an unknown distribution $p$ which is supported on $\{-1,1\}^n$. Furthermore, we receive the accuracy parameter $\eps \in (0,1)$. We let $c_0$ denote a sufficiently small constant. 
%     The algorithm proceeds in the following manner: 
\begin{enumerate}
    \item For all integers $w\geq0$ such that $2^w = \tilde{O}(n/\eps^2)$, repeat the following $t = O(2^{w} \log(n/\eps))$ times:
    \begin{itemize}
        \item Sample $\bx \sim p$ and $\bi \sim [n]$, and consider the restriction $\brho \in \{-1,1,*\}^n$ given by $\brho_j = \bx_j$ if $j \neq \bi$, and $\brho_{\bi} = *$. 
        \item Let $\eta = c_0^2\eps^2 \cdot 2^{w} / (16n \cdot \log(n/\eps) \cdot \log n)$ and take $m = O(\log (n/\eps)/\eta)$ subcube conditioning queries with restriction $\brho$ while counting the number of $1$'s and $-1$'s in coordinate $\bi$ observed. If the number of $-1$'s observed is larger than $m\left(1/2 + \sqrt{\eta}/2\right)$, output ``reject.''
    \end{itemize}
    \item If the algorithm has not rejected, output ``accept.''
\end{enumerate}

\end{framed}
\caption{Algorithm for Testing Monotonicity of Distributions} \label{fig:alg}
\end{figure}

\begin{claim}
    The query complexity is $\tilde{O}(n / \eps^2)$.
\end{claim}
\begin{proof}
    We simply upper bound the query complexity by inspecting Figure~\ref{fig:alg}. We have that (disregarding constant factors) the query complexity is the sum over all integers $w \geq 0$ such that $2^{w} = \tilde{O}(n/\eps^2)$ of 
    \[ O(2^{w} \cdot \log(n/\eps)) \cdot O\left(\dfrac{n\cdot \log^2(n/\eps) \cdot \log n}{c_0^2 \eps^2 \cdot 2^{w}} \right) = \tilde{O}(n / \eps^2). \]
    There are $O(\log(n/\eps))$ such settings of $w$, so the total complexity is still $\tilde{O}(n / \eps^2)$.
\end{proof}

\begin{lemma}
    Whenever $p$ is monotone, the algorithm outputs ``accept'' with probability at least $0.9$.
\end{lemma}

\begin{proof}
    Note that if $p$ is monotone, then for any restriction $\rho \in \{-1, 1, *\}^n$, which has one coordinate $i$ with $\rho_i = *$, a sample $\by$ from $p_{|\rho}$ must have the probability that $\by_i$ is 1 is at least the probability that it is $-1$. A standard Hoeffding bound implies that if one takes $m = O(\log(n/\eps)/\eta)$ samples of some event which is more likely to be $1$ than $-1$, the probability that the number of $1$'s observed is smaller than $m (1/2 - \sqrt{\eta}/2)$ is smaller than $\poly(\eps / n)$, for an arbitrarily large polynomial. Note that the number of times we may wrongfully reject is at most the query complexity, which is at most $\tilde{O}(n/\eps^2)$. So we may union bound as desired.
\end{proof}

\begin{lemma}\label{lem:far-case-reject}
    Whenever $p$ is $\eps$-far from monotone, the algorithm outputs ``reject'' with probability at least $0.9$.
\end{lemma}

\begin{proof}
    We show that whenever $p$ is $\eps$-far from monotone, there exists some $\gamma \in \{0, \dots, h\}$ with $h = O(\log(n/\eps))$ and a setting of $\ell \in \{ 0, \dots, r\}$ where $r = O(\log(n/\eps))$ which satisfies $2^{2\gamma + r +1} = \tilde{O}(n / \eps^2)$ and
    \begin{align*}
        \Prx_{ \substack{\bx \sim p \\ \bi \sim [n]}}\left[\left(\dfrac{(p(\bx^{(\bi\to-1)}) - p(\bx^{(\bi\to1)}))^+}{p(\bx^{(\bi\to-1)}) + p(\bx^{(\bi\to1)})}\right)^2 \geq \eta\right] \geq \frac{1}{r \cdot 2^{\gamma + \ell}}.
    \end{align*}
    for $\eta = c_0^2 \eps^2 \cdot 2^{2\gamma+\ell} / (16 h \cdot n \cdot \log n)$. When the algorithm iterates over all $w \geq 0$ such that $2^w = \tilde{O}(n/\eps^2)$, it will eventually consider $w = 2\gamma + \ell$.
    This implies that except with probability $0.01$, one of the $t$ samples $\bx \sim p$ and $\bi \sim [n]$ satisfy the above bound, since we repeat $t = O(2^{w}\log(n/\eps))$ times and $2^{w} = 2^{2\gamma+\ell}$ is larger than $2^{\gamma+\ell}$. Once that is set, with probability except $0.01$, the algorithm outputs reject; the subcube conditioning query $\brho$ is exactly sampling from $\{-1,1\}$ whose probability of being $-1$ is at least $1/2 + \sqrt{\eta}$. By a Hoeffding bound, the probability that the number of $-1$'s is smaller than $m(1/2 + \sqrt{\eta}/2)$ is at most $0.01$. From Corollary~\ref{cor:l1-tal}, for 
    small enough constant $c_0$. the fact that $p$ is $\eps$-far from monotone implies,
    \begin{align*}
        \frac{c_0\eps}{\sqrt{\log n}} &\leq \sum_{x \in \{-1,1\}^n} \sqrt{\sum_{i:x_i = -1} \left( \left(p(x^{(i\to-1)}) - p(x^{(i\to1)} \right)^+ \right)^2} \\
        &= \Ex_{\bx \sim p}\left[ \sqrt{\sum_{i:\bx_i=-1} \left(\dfrac{(p(\bx^{(i\to-1)}) - p(\bx^{(i\to1)}))^+}{p(\bx^{(i\to-1)})} \right)^2} \right].
    \end{align*}
    Furthermore, $p(\bx^{(i\to-1)}) - p(\bx^{(i\to1)}) \geq 0$ implies $p(\bx^{(i\to-1)}) + p(\bx^{(i\to 1)}) \leq 2 p(\bx^{(i\to-1)})$. So we may lower bound
    \begin{align}
        \frac{c_0 \eps}{4\sqrt{\log n}} \leq \Ex_{\bx \sim p}\left[ \sqrt{\sum_{i:\bx_i=-1} \left(\dfrac{(p(\bx^{(i\to-1)}) - p(\bx^{(i\to1)}))^+}{p(\bx^{(i\to-1)}) + p(\bx^{(i\to1)})} \right)^2} \right] \label{eq:exp1}
    \end{align}
    Notice that the maximum quantity within the expectation in (\ref{eq:exp1}) is $\sqrt{n}$, since each of the terms being added is between $0$ and $1$. Therefore, there must exist some $\gamma \in \{0,\dots, h \}$ with $h = \lceil \log_2(4\sqrt{n}/(c_0\eps))\rceil + 1 = O(\log(n/\eps))$ which satisfies
    \begin{align}
     \Prx_{\bx \sim p}\left[\sum_{i:\bx_i=-1} \left(\dfrac{(p(\bx^{(i\to-1)}) - p(\bx^{(i\to1)}))^+}{p(\bx^{(i\to-1)}) + p(\bx^{(i\to1)})} \right)^2 \geq \frac{c_0^2 \cdot \eps^2 \cdot 2^{2\gamma}}{16h\log n} \right] \geq \frac{1}{2^{\gamma}}. \label{eq:good-1}
    \end{align}
    Thus, consider any one of those values of $x$, and in order to simplify the notation, we define 
    \[ \xi \eqdef \frac{c_0^2 \cdot \eps^2 \cdot  2^{2\gamma}}{16h\log n} \qquad \nu_i = \left(\dfrac{(p(x^{(i\to-1)}) - p(x^{(i\to1)}))^+}{p(x^{(i\to-1)}) + p(x^{(i\to1)})} \right)^2,\]
    so that we assume to fix $x$ such that $\sum_{i=1}^n \nu_i \geq \xi$, and each $\nu_i \in [0, 1]$. Consider a partition of the coordinates of $[n]$ into groups $B_1, \dots, B_{r}$, such that $i \in B_{\ell}$ whenever the $i$-th coordinate contributes between $\xi 2^{\ell}/ n$ and $\xi 2^{\ell+1} / n$, and $r$ is chosen is the value $\xi 2^{r+1} / n$ is between $1$ and $2$ (note that, since $\nu_i \in [0, 1]$, $B_{r'}$ for $r' > r$ must be empty), so $r = O(\log(n/\xi))$. Then, there must be some $\ell$ with $|B_{\ell}| \geq n / (r \cdot 2^{\ell+1})$, and this implies
    \begin{align}
        \Prx_{\bi \sim [n]}\left[\left(\dfrac{(p(\bx^{(\bi\to-1)}) - p(\bx^{(\bi\to1)}))^+}{p(\bx^{(\bi\to-1)}) + p(\bx^{(\bi\to1)})} \right)^2 \geq \frac{\xi \cdot 2^{\ell}}{n}  \right] \geq \frac{1}{r \cdot 2^{\ell}}.\label{eq:good-2}
    \end{align}
    The desired bound then follows from the setting of $\eta$, and lower bounding the probability that $\bx \sim p$ satisfies the event of (\ref{eq:good-1}), and then $\bi \sim [n]$ satisfies the event of (\ref{eq:good-2}).
\end{proof}

\subsection{The Real-Valued Directed Talagrand Inequality} \label{sec:tal}

We will prove a ``directed isoperimetric theorem" for real-valued functions. This is an important
tool used for the analysis of the monotonicity tester. We define notions of the
directed boundary for Boolean functions. 

Let $f:\{-1,1\}^n \to [0,1]$ be a function defined on the $n$-dimensional hypercube. The $L_1$-distance 
of $f$ from monotonicity is defined as
\begin{equation*}
	\dist_1(f) \eqdef \min_{g~:~\text{monotone}} ~~\Ex_{\bx \sim \{-1,1\}^n}\left[ |f(\bx) - g(\bx)| \right]
\end{equation*}
where the expectation is over the uniform distribution over $\{-1,1\}^n$.
For a point $x\in \{-1,1\}^n$, define the directed derivative $\grad^-f(x)$ to be the $n$-dimensional vector defined as 
\begin{equation}\label{eq:defgrad}
	\left(\grad^-f(x)\right)_i \eqdef \begin{cases}
		0 & \textrm{if $x_i = 1$} \\
		\left(f(x) - f(x+2e_i)\right)^+ & \text{otherwise}
	\end{cases}
\end{equation}
where $(z)^+$ is a shorthand for $\max(z,0)$. For Boolean-valued $f:\{-1,1\}^n \to \{0,1\}$, the distance $\dist_1(f)$ corresponds to the ``normal'' Hamming distance notion, $\dist_0(f)$.
Based on isoperimetric theorems of Talagrand~\cite{Tal93}, the quantity $\Exp_{\bx} \norm{\grad^-f (\bx)}_2$ can be thought
of as a ``directed surface area" for the function $f$. A deep isoperimetric theorem of Khot, Minzer, and Safra~\cite{KMS18} (see, also~\cite{PRW22}, who showed how to remove the final logarithmic factor)
lower bounds this surface area by the distance to monotonicity.

\begin{theorem}[\cite{KMS18, PRW22}]\label{thm:booliso}
	There exists a universal constant $C > 0$ such that for every $f \colon \{-1,1\}^n \to \{0,1\}$,
$\Exp_{\bx} \norm{\grad^-f (\bx)}_2 \geq C\cdot \dist_0(f)$.
\end{theorem}

Theorem~\ref{thm:l1-talagrand} gives a real-valued generalization of the above theorem, with a $\sqrt{\log n}$ loss in the bound. The proof appears in Subsection~\ref{sec:proof-tal}, but we state the following corollary used in the tester's analysis.
\begin{corollary} \label{cor:l1-tal} Let $p$ be a distribution over $\{-1,1\}^n$ that is $\eps$-far
from monotone. Then 
\[
\sum_{x \in \{-1,1\}^n} \sqrt{\sum_{i:x_i = -1} \left( \left(p(x^{(i\to-1)}) - p(x^{(i\to1)} \right)^+ \right)^2} = \Omega\left(\frac{\eps}{\sqrt{\log n}}\right).
\]
\end{corollary}

\begin{proof} Let $\eps(p)$ be the distance of $p$ to monotonicity. Note that this is the distance
over distributions, while \Thm{l1-talagrand} refers to $L_1$-distance between arbitrary functions.
So we need an extra calculation to apply \Thm{l1-talagrand}.

Let $\cM$ be the set of monotone distributions. Then, $\eps(p) = \min_{q \in \cM} \dtv(p,q)
= \min_{q \in \cM} \|p-q\|_1/2$. On the other hand, $\dist_1(p) = \min_{g: \textrm{monotone}} \Exp_{\bx}|p(\bx) - g(\bx)|
= 2^{-n} \min_{g: \textrm{monotone}} \|p-g\|_1$. Note that the minimizer $g$ is non-negative, since $p$
is non-negative. Hence, the function $f = g/\|g\|_1$ is a distribution. 

By triangle inequality,
\begin{eqnarray*}
\eps(p) \leq \|p - f\|_1 \leq \|p-g\|_1 + \|f-g\|_1 = \|p-g\|_1 + \|g - g/\|g\|_1 \|_1
\end{eqnarray*}
Observe that $\|g - g/\|g\|_1 \|_1 = \sum_x |g(x) - g(x)/\|g\|_1| = |1 - 1/\|g\|_1| \cdot \sum_x |g(x)|
= | 1 - \|g\|_1|$. Since $p$ is a distribution, this expression is equal to $| \|p\|_1 - \|g\|_1|$.
And finally, $| \sum_x (|p(x)| - |g(x)|)| \leq \sum_x |p(x) - g(x)| = \|p-g\|_1$. 
Overall, we deduce that $\eps(p) \leq 2\|p-g\|_1$. Recall that $\dist_1(p)$ is defined
using an expectation over the domain, so $\eps(p) \leq 2 \cdot 2^n \dist_1(p)$. 

With our lower bound for $\dist_1(g)$, we can apply \Thm{l1-talagrand}. 
So $\Exp_{\bx} \norm{\grad^-p(\bx)} = \Omega(\dist_1(p)/\sqrt{\log n}) = \Omega(2^{-n} \eps(p)/\sqrt{\log n})$.
We expand out the expression for $\grad^-p(x)$ to wrap up the proof.
\begin{eqnarray*}
\Ex_{\bx\sim\{-1,1\}^n}\left[ \norm{\grad^-p(\bx)} \right] = 2^{-n} \sum_{x \in \{-1,1\}^n} \norm{\grad^-p(x)}
= 2^{-n}  \sum_{x \in \{-1,1\}^n} \sqrt{\sum_{i:x_i = -1} \left( (p(x) - p(x+2e_i) )^+ \right)^2}
\end{eqnarray*}
As argued above, this expression is lower bounded by $\Omega(2^{-n} \eps(p)/\sqrt{\log n})$.
The $2^{-n}$ terms ``cancel out", and noting that $\eps(p) \geq \eps$, we get the desired bound.
%$\sum_{x \in \{-1,1\}^n} \sqrt{\sum_{i:x_i = -1} \left( \left(p(x^{(i\to-1)}) - p(x^{(i\to1)} \right)^+ \right)^2} = \Omega(\eps/\sqrt{\log n})$.
\end{proof}


\subsection{The proof of \Thm{l1-talagrand}} \label{sec:proof-tal}

By a simple translation and rescaling argument, we reduce the function range to $[0,1]$.
This will make subsequent calculations easier.

\begin{claim} \label{clm:rescale} Consider $f:\{-1,1\}^n \to \R$. For positive $\alpha \in \R^+$ and
any $\beta \in \R$, define the function $\hat{f}$ where $\hat{f}(x) = \alpha f(x) + \beta$. Then,
$\EX_x[\|\nabla^- \hat{f}\|_2]/\dist_1(\hat{f}) =\EX_x[\|\nabla^- f\|_2]/\dist_1(f) $.
\end{claim}

\begin{proof} The monotonicity violations in $f$ and $\hat{f}$ are identical.
For any point $x$ and coordinate $i$, $ (\nabla^- \hat{f}(x))_i = \alpha (\nabla^- f(x))_i$.
Hence, $\EX_x[\|\nabla^- \hat{f}\|_2] = \alpha \EX_x[\|\nabla^- f\|_2]$.
For a function $g$, let $\alpha g + \beta$ be the function whose value at $x$
is $\alpha g(x) + \beta$.
\begin{align*}
 \dist_1(\hat{f}) = \min_{\hat{g}: \textrm{monotone}} \|\hat{f}-\hat{g}\|_1 &= \min_{\hat{g}: \textrm{monotone}} \| (\alpha f + \beta) - \hat{g}\|_1
\min_{\hat{g}: \textrm{monotone}} \|(\alpha f + \beta) - (\alpha (\alpha^{-1}(\hat{g} - \beta)) + \beta)\|
\end{align*}
Monotonicity is preserved by positive scaling and translation, so $\hat{g}$ is monotone iff $(\alpha^{-1}(\hat{g} - \beta))$
is monotone.
Hence,
\begin{align*}
 \dist_1(\hat{f}) = \min_{g: \textrm{monotone}} \|(\alpha f + \beta) - (\alpha g + \beta)\|_1
= \min_{g: \textrm{monotone}} \|\alpha f - \alpha g\|_1 = \alpha \ \dist_1(f)
\end{align*}
We conclude that  
$\EX_x[\nabla^- \hat{f}\|_2]/\dist_1(\hat{f}) =\EX_x[\|\nabla^- f\|_2]/\dist_1(f) $.
\end{proof}

Given $f$, we technically work with the function $\hat{f} = f/2M + 1/2$,
where $M = \max_x |f(x)|$. Observe that $\hat{f}$ has range in $[0,1]$,
and by \Clm{rescale}, the statement of \Thm{l1-talagrand} for $\hat{f}$ implies
the statement for $f$.

Abusing notation, we just assume that $f:\{-1,1\}^n \to [0,1]$.
We use the technique of Berman, Raskhodnikova, and Yaroslavtsev~\cite{BeRaYa14} of using threshold Boolean functions to relate the real-valued $f$ to Boolean functions. 

\noindent
Given $t\in [0,1]$ consider the following Boolean function (Definition 2.1 in~\cite{BeRaYa14}) $f_t : \{-1,1\}^n \to \{0,1\}$
\[
	f_t(x) = \begin{cases}
		1 & \text{if}~ f(x) \geq t; \\
		0 & \text{if}~ f(x) < t
	\end{cases}
\]
\noindent
It is easy to see that for any $x\in \{-1,1\}^n$,
\[
	f(x) = \int_0^{f(x)} dt  =  \int_0^1 f_t(x)~dt ~= \Ex_{\bt \sim [0,1]} \left[ f_{\bt}(x) \right]
\]
where the expectation is over $t$ uniformly distributed over $[0,1]$.
One can perform analogous calculations to relate the $L_1$ distance of (the real valued)
$f$ to the $L_0$ distance of (the Boolean) $f_t$s.  

\begin{lemma}[Lemma 2.1~\cite{BeRaYa14}]\label{lem:bry}
	\noindent
	For any function $f:\{-1,1\}^n \to [0,1]$, $\dist_1(f) = \int_0^1 \dist_0(f_t) dt = \Exp_{\bt} \left[ \dist_0(f_{\bt})\right]$.
\end{lemma}

The main work is in relating the (directed) gradients of $f$ to the corresponding
gradients of $f_t$. This is where we suffer a $\sqrt{\log n}$ loss.
 
% One can also similarly check that for any $x\in \{0,1\}^n$,
% we have 
% \begin{equation}\label{eq:exp-grad}
% 	\grad^-f(x) = \Exp_t \grad^- f_t(x)
% \end{equation}
% In particular, since $\grad^-f(x)$ and $\grad^-f_t(x)$ are non-negative vectors, by linearity of expectation we get
% \begin{equation}\label{eq:l1-norm}
% 	\norm{\grad^-f(x)}_1 = \norm{\Exp_t \grad^- f_t(x)}_1 = \Exp_t \norm{\grad^- f_t(x)}_1
% \end{equation}
% BRY provide the following characterization
% 
% Using~\Lem{bry} and \eqref{eq:exp-grad}, one immediately obtains the Poincare version of~\Thm{booliso}:
% \[
% 	\dist_1(f) =  \Exp_t  \dist_0(f_t)  \underbrace{\leq}_{\Thm{booliso}} \frac{1}{C}\cdot \Exp_t \Exp_x \norm{\grad^- f_t(x)}_1 \underbrace{=}_{\eqref{eq:l1-norm}} \frac{1}{C}\cdot \Exp_x \norm{\Exp_t \grad^- f_t(x)}_1 \underbrace{=}_{\eqref{eq:exp-grad}}\frac{1}{C}\cdot \Exp_x \norm{\grad^- f(x)}_1
% \]
% ndent
% 
% Talagrand's inequality doesn't {\em immediately} follow because (as usual) ``Jensen is in the wrong direction''. \eqref{eq:exp-grad} and the fact that the $\ell_2$ norm is a convex function implies that $\Exp_t \norm{\grad^- f_t(x)}_2 \geq \norm{\Exp_t \grad^- f_t(x)}_2 = \norm{\grad^- f(x)}_2$ which is counter to what would've been nice. Nevertheless one can prove the following lemma which, using the same derivation as above, proves~\Thm{l1-talagrand}.
% 

\begin{lemma}
	For all $x\in \{-1,1\}^n$, $\norm{\grad^- f(x)}_2 = \Omega(1/\sqrt{\log n}) \Exp_t \norm{\grad^- f_t(x)}_2 $.	
\end{lemma}
\begin{proof}
Fix any $x \in \{-1,1\}^n$. Let $y_1, \ldots, y_d \in \{-1,1\}^n$ denote the ``up''-neighbors of $x$ which satisfy $f(x) > f(y_j)$. In particular, there are at most $d \leq n$ points $y_1 ,\dots, y_d$ such that, for every $j \in [d]$, $y_j = x + 2e_i$ for some $i$, and in addition, $f(x) > f(y_j)$.
%(so there exists some coordinate $i$ such that $y_j = x + 2e_i$).
%Note that $d\leq n$.
Order the indices so as to assume $f(y_1) \leq f(y_2) \leq \cdots \leq f(y_d)$ and let $a_j := f(x) - f(y_j)$ (and so $a_1 \geq a_2 \geq \cdots \geq a_d$). By definition, we have defined $a_1,\dots, a_d$ to have $\norm{\grad^- f(x)}_2 = (\sum_{j=1}^d a^2_j)^{1/2}$.

For $t \in [0, 1]$, consider the function $f_t$, and let edge $(x,y_j)$ be called a violation in $f_t$ if $f_t(x) = 1$ and $f_t(y_j) = 0$.
Observe that only violated edges contribute to $\norm{\grad^-f_t(x)}$. 
Notice that for any $t \in (f(y_i), f(y_{i+1})]$, the edge $(x,y_j)$ is a violation in $f_t$ iff $j \leq i$.
Hence, if $t \in (f(y_i), f(y_{i+1})]$, then the vector $\grad^-f_t(x)$ 
has exactly $i$ non-zeros and $\norm{\grad^- f_t(x)}_2 = \sqrt{i}$. For $i < d$, the probability that $t \in (f(y_i), f(y_{i+1})]$
is exactly $y_{i+1} - y_i = a_i - a_{i+1}$. The probability that $t \in (f(y_d), x]$
is exactly $a_d$.
% 	
% Next note that $\grad^-f_t(x)$ looks as follows:
% 	\begin{equation}\label{eq:gorilla}
% 		\grad^-f_t(x) =  \begin{cases}
% 			(0,0,\ldots, 0, 0) & \text{if} ~ 0\leq t\leq f(y_1) \\
% 			(1,0,\ldots, 0, 0) & \text{if} ~ f(y_1) <  t\leq f(y_2) \\
% 			(1,1,\ldots, 0, 0) & \text{if} ~ f(y_2) <  t\leq f(y_3) \\
% 			\vdots & \\
% 			(1,1,\ldots, 1, 0) & \text{if} ~ f(y_{d-1}) <  t\leq f(y_d) \\
% 			(1,1,\ldots, 1, 1) & \text{if} ~ f(y_d) <  t\leq f(x) \\
% 			(0,0,\ldots, 0, 0) & \text{if} ~ f(x) <  t\leq 1 \\
% 		\end{cases}
% 	\end{equation}
Thus,
\[
	\Ex_{\bt\sim[0,1]}\left[ 	\norm{\grad^- f_{\bt}(x)}_2 \right] = \sum_{i=1}^{d-1} \left(a_i - a_{i+1}\right)\sqrt{i} + a_d \sqrt{d} = \sum_{i=1}^d a_i \cdot \left(\sqrt{i} - \sqrt{i-1}\right)
\]
By Cauchy-Schwarz and the following calculation, we complete the proof
\begin{equation*}
	\Ex_{\bt\sim [0,1]}\left[ 	\norm{\grad^- f_{\bt}(x)}_2 \right]\leq \norm{\grad^- f(x)}_2 \cdot \sqrt{\sum_{i=1}^d \left(\sqrt{i} - \sqrt{i-1}\right)^2}\leq O(\sqrt{\log n})\cdot \norm{\grad^- f(x)}_2
\end{equation*}
since $\sqrt{i} - \sqrt{i-1} = 1/(\sqrt{i}+\sqrt{i-1}) \leq 1/\sqrt{i}$, and so $\sum_{i \leq d} (\sqrt{i} - \sqrt{i-1})^2 \leq \sum_{i \leq d} 1/i = O(\log d)$.
% 
% To see the above, note that
% \[
% (\sqrt{i}-\sqrt{i-1})^2 = (2i - 1) - 2\sqrt{i(i-1)} = 2i\cdot \left(1 - \left(1 - \frac{1}{i}\right)^{1/2}\right) - 1
% \]
% By Taylor approximation, we get that there exists constant $A > 0$ (indeed, $A = 1$ suffices) such that
% \[
% \left(1 - \frac{1}{i}\right)^{1/2} \geq 1 - \frac{1}{2i} - \frac{A}{i^2}
% \]
% \noindent
% and so substituting above we get that
% \[
% \sum_{i=1}^d \left(\sqrt{i} - \sqrt{i-1}\right)^2 \leq \sum_{i=1}^d \frac{2A}{i} = O(\log d)\qedhere
% \]
\end{proof}

We now complete the proof of \Thm{l1-talagrand}. By the above lemma,
\begin{align*}
\Ex_{\bx\sim\{-1,1\}^n}\left[ \norm{\grad^- f(\bx)}_2\right] &= \Omega(1/\sqrt{\log n}) \Ex_{\substack{\bx \sim \{-1,1\}^n \\ \bt \sim [0,1]}} \left[ \norm{\grad^- f_{\bt}(\bx)}_2\right] \\
    &= \Omega(1/\sqrt{\log n}) \Ex_{\bt \sim [0,1]} \left[  \Ex_{\bx\sim\{-1,1\}^n}\left[ \norm{\grad^- f_{\bt}(\bx)}_2 \right] \right].
\end{align*}
By the directed Boolean isoperimetric statement of \Thm{booliso}, $\Exp_{\bx} \norm{\grad^- f_t(\bx)}_2 = \Omega(\dist_0(f_t))$.
We apply this bound and then \Lem{bry} to relate back to $f$.
\begin{align*}
\Ex_{\bx\sim\{-1,1\}^n}\left[ \norm{\grad^- f(\bx)}_2 \right] = \Omega(1/\sqrt{\log n}) \Ex_{\bt\sim[0,1]} \left[\dist_0(f_{\bt})\right] = \Omega(1/\sqrt{\log n}) \cdot \dist_1(f)
\end{align*}

\section{The lower bound}\label{sec:lb}
In this section, we provide a sample complexity lower bound proof with regard to the error tolerance $\eps$. In fact, we prove \Cref{thm:main}, which is a formal version of \Cref{thm:main-lb}. %All distributions constructed in this section will satisfy \Cref{assump:moment} and \Cref{assump:smooth}.

% A sampling algorithm here is a procedure with query access to $f$, $\grad f$ or $\grad^2 f$, which outputs a sample point based on a series of queries. We aim to prove the following theorem in this note.
% \htodo{Here $c$ is the constant in \Cref{lem:disjointcap}.}

Recall that we use $\@D_{L,M}$ to denote the collection of distributions which are $L$-log-smooth and have second moment at most $M$. 
\begin{theorem}\label{thm:main}
    % There exist a universal constant $C>0$ such that 
    For any $L,M>0$ satisfying $LM\ge d$ and for any $\eps\in(0,1/32)$, $d\geq 5$, if a sampling algorithm $\+A$ always terminates within 
\[
    \frac{\eps (d-2)^{\frac{3}{2}}}{8}\cdot \tp{\frac{9}{256} \cdot \frac{LM}{d\eps} \cdot \frac{1}{\log \frac{LM}{d\eps}}}^{\frac{d-1}{2}}
    % \frac{\eps}{4}\cdot \tp{\frac{C\cdot LM}{2 d\eps} \cdot \frac{1}{\log \frac{LM}{d\eps}}}^{\frac{d-1}{2}}% \approx \frac{\eps}{4}\exp\set{\frac{d}{2}\cdot \Omega\tp{\log \frac{LM}{d\eps}}}
\]
queries on every input instance in $\@D_{L,M}$, then there must exist some distribution $\mu\in \@D_{L,M}$ such that when the underlying instance is $\mu$, the distribution of $\+A$'s output, denoted as $\tilde \mu$, is $\eps$ away from $\mu$ in total variation distance, i.e., $\DTV(\mu,\tilde \mu)\geq \eps$.
\end{theorem}
%\htodo{Actually we consider those distributions with second moment $O(M)$ and $O(L)$-smooth. Not exactly $M$ and $L$.}

%\mn{When $d$ is even, $\Gamma\tp{\frac{d}{2}+1} = \tp{\frac{d}{2}}!$. When $d$ is odd, $\Gamma\tp{\frac{d}{2}+1} = \frac{\sqrt{\pi}}{2^d}\cdot \frac{d!}{\tp{\frac{d-1}{2}}!}$.}

\subsection{The base instance}

We first construct a base distribution $\mu_0$. Let $R= \tp{\frac{M}{\eps}}^{\frac{1}{2}}$, and let
\[
    \mathfrak{g}_{[\frac{R}{4},\frac{R}{2}]}(x)=q_{\!{mol}}\tp{\frac{ \|x\|^2- \frac{R^2}{16}}{\frac{R^2}{4} - \frac{R^2}{16}}} \quad \mbox{and} \quad \mathfrak{g}_{[R,2R]}(x)=q_{\!{mol}}\tp{\frac{\|x\|^2-R^2}{4R^2-R^2}},
\]
% Let $\alpha\in (0,2)$ be a constant to be determined later.
With constant function $h_1 \equiv \log \tp{\!{vol}(\+B_{3 R})} + \log \frac{1}{\eps}$ and function $h_0(x)=\frac{d\|x\|^2}{2M} + \frac{d}{2}\log \frac{2\pi M}{d}$, define the function $f_0$ as
\[
    \forall x\in \bb R^d, f_0(x) = \begin{cases} h_0(x), & \|x\|\leq \frac{R}{4} \\
    \mathfrak{g}_{[\frac{R}{4},\frac{R}{2}]}(x)\cdot h_1 + \tp{1-\mathfrak{g}_{[\frac{R}{4},\frac{R}{2}]}(x)}\cdot h_0(x), & \frac{R}{4}<\|x\| \leq \frac{R}{2}\\
    h_1, & \frac{R}{2} <\|x\|\leq R\\
    \mathfrak{g}_{[R,2R]}(x)\cdot h_0(x) + \tp{1-\mathfrak{g}_{[R,2R]}(x)}\cdot h_1, & R<\|x\|\leq 2R \\
    h_0(x), & \|x\|>2R
    \end{cases}.
\]

%\mn{Here we use a constant function $h_1$ rather than using Gaussian directly. This is crucial.}
Consider the distribution $\mu_0$ with density $p_{\mu_0} \propto \exp\tp{-f_0(x)}$ and its normalizing factor $Z_0 = \int_{\bb R^d} \exp\tp{-f_0(x)} \d x$.
\begin{lemma}\label{lem:Z_0}
    The normalizing constant $1-16\eps \leq Z_0 \leq  1+\eps$. 
\end{lemma}
\begin{proof}
    On one hand, from Markov's inequality,
    $$
        Z_0\geq \int_{\+B_{\frac{R}{4}}} e^{-f_0(x)} \dd x = \int_{\+B_{\frac{R}{4}}} e^{-h_0(x)} \dd x = 1- \Pr[X\sim \+N\tp{0,\frac{M}{d}\cdot \!{Id}_d}]{\|X\|^2\geq \frac{R^2}{16}} \geq 1-16\eps.
    $$
    On the other hand, 
    $$
        Z_0\leq \int_{\bb R^d} e^{-h_0(x)} \dd x + \vol (\+B_{2R})\cdot e^{-h_1} \leq 1 + \eps\cdot \frac{\vol (\+B_{2R})}{\vol(\+B_{3R})}\leq 1+\eps.
    $$
    
    % Note that $h_1=\log \frac{1}{\eps} + \frac{d}{2}\log\tp{\pi \alpha^2 R^2} - \log \Gamma\tp{\frac{d}{2}+1}$.
    
    % Note that for each fixed $x\in \bb R^d$, $f_0$ is a non-decreasing function wrt $\alpha$. So $Z_0$ is decreasing as $\alpha$ increases. When $\alpha=1$,
\end{proof}

% We fixed the value of $\alpha$ to be the one in \Cref{lem:Z_0}.
\begin{lemma}\label{lem:propertymu0}
    The distribution $\mu_0$ is $\+O(L)$-log-smooth and has second moment $\+O(M)$.
\end{lemma}
%\htodo{$O(M)$ and $O(L)$, not exactly $M$ and $L$} 
\begin{proof}
    We first calculate the second moment of $\mu_0$. We have
    $$
        \E[\mu_0]{\|X\|^2} \leq \frac{\E[X\sim \+N\tp{0,\frac{M}{d}\cdot \!{Id}_d}]{\|X\|^2} + \vol (\+B_{2R})\cdot e^{-h_1}\cdot 4R^2}{Z_0} \leq \frac{M + \eps\cdot \frac{\vol (\+B_{2R})}{\vol (\+B_{3R})}\cdot 4R^2}{Z_0} \leq \frac{3M}{Z_0}\leq 6M,
    $$
    where the last inequality is due to \Cref{lem:Z_0} and the fact that $\eps<\frac{1}{32}$.
    % \htodo{We may need $\eps<\frac{1}{32}$.}

    For the smoothness, we only need to check $\|\grad^2 f_0(x)\|$ for those $x$ with $\|x\|\in (\frac{R}{4},\frac{R}{2}]$ and $\|x\|\in (R,2R]$ since clearly $f_0\in C^2(\bb R^d)$. %\ctodo{need $f_0\in C^2$ here.}

    First, for $\|x\|\in (\frac{R}{4},\frac{R}{2}]$,
    \[
        \grad f_0(x) = \grad \mathfrak{g}_{[\frac{R}{4},\frac{R}{2}]}(x) \cdot (h_1 - h_0(x) ) + (1-\mathfrak{g}_{[\frac{R}{4},\frac{R}{2}]}(x))\cdot \grad h_0(x), 
    \]
    and
    \begin{align*}
        \grad^2 f_0(x) &= \underbrace{\grad^2 \mathfrak{g}_{[\frac{R}{4},\frac{R}{2}]}(x) \cdot (h_1 - h_0(x))}_{\mbox{(a)}} - \underbrace{\tp{\grad \mathfrak{g}_{[\frac{R}{4},\frac{R}{2}]}(x) \cdot \grad h_0(x)^{\top} +  \grad h_0(x)\cdot \grad \mathfrak{g}_{[\frac{R}{4},\frac{R}{2}]}(x)^{\top}}}_{\mbox{(b)}}\\
        &\quad + \underbrace{ \tp{1-\mathfrak{g}_{[\frac{R}{4},\frac{R}{2}]}(x)}\cdot \grad^2 h_0(x)}_{\mbox{(c)}}.
        \end{align*}
    Recall that $LM\geq d$, so it is easy to get $0\preceq \mbox{(c)}\preceq L\cdot \!{Id}_d$. Since 
    \[
        \grad \mathfrak{g}_{[\frac{R}{4},\frac{R}{2}]}(x) = \frac{2x}{\frac{R^2}{4} - \frac{R^2}{16}} \cdot q_{\!{mol}}'\tp{\frac{ \|x\|^2- \frac{R^2}{16}}{\frac{R^2}{4} - \frac{R^2}{16}}},
    \]
    we have
    \[
        \mbox{(b)} = \frac{4d \cdot xx^{\top}}{M\tp{\frac{R^2}{4} - \frac{R^2}{16}}} \cdot q_{\!{mol}}'\tp{\frac{ \|x\|^2- \frac{R^2}{16}}{\frac{R^2}{4} - \frac{R^2}{16}}}.
    \]
    % \mn{For $x,y\in \bb R^d$, we can show that $\|x\|\|y\|\!{Id}_d - xy^T\succeq 0$: for any $z\in \bb R^d$,
    % \begin{align*}
    %     &\phantom{{}={}}z^T\tp{\|x\|\|y\|\!{Id}_d - xy^T}z \\
    %     & =\|x\|\|y\|\|z\|^2 - (z^Tx)(y^Tz)\\
    %     &\geq \|x\|\|y\|\|z\|^2 - \|x\|\|y\|\|z\|^2\\
    %     &= 0.
    % \end{align*}}
    Therefore, $-\+O(L)\cdot\!{Id}_d \preceq \mbox{(b)} \preceq \+O(L)\cdot\!{Id}_d$.
    By direct calculation,
    \[
        \grad^2 \mathfrak{g}_{[\frac{R}{4},\frac{R}{2}]}(x) = \frac{4xx^{\top}}{\tp{\frac{R^2}{4} - \frac{R^2}{16}}^2} \cdot q''_{\!{mol}}\tp{\frac{ \|x\|^2- \frac{R^2}{16}}{\frac{R^2}{4} - \frac{R^2}{16}}} + \frac{2\!{Id}_d}{\frac{R^2}{4} - \frac{R^2}{16}} \cdot q'_{\!{mol}}\tp{\frac{ \|x\|^2- \frac{R^2}{16}}{\frac{R^2}{4} - \frac{R^2}{16}}}
    \]
    and 
    \begin{align}
        \abs{h_1 - h_0(x)} &= \abs{\log \frac{1}{\eps} + \frac{d}{2}\log\tp{9\pi R^2} - \log \Gamma\tp{\frac{d}{2}+1} - \frac{d\|x\|^2}{2M} - \frac{d}{2}\log \frac{2\pi M}{d}} \notag\\
        &= \abs{\log \frac{1}{\eps} + \frac{d}{2}\log\frac{9d}{2\eps} - \log \Gamma\tp{\frac{d}{2}+1} - \frac{d\|x\|^2}{2M}} \label{eq:1}
        % &\leq \log \frac{1}{\eps} + \frac{d}{2}\log\tp{9\pi R^2} - \log \Gamma\tp{\frac{d}{2}+1} + \frac{R^2 d}{8M} + \frac{d}{2}\log \frac{2\pi M}{d}.
    \end{align}
    % \htodo{Here we need $R^2=\frac{M}{\eps}\log \frac{M}{\eps}$?}
    From Stirling's formula, we know that for any $d>0$,
    \[
        \log\sqrt{\pi d} + \frac{d}{2}\log \frac{d}{2e} \leq \log \Gamma\tp{\frac{d}{2}+1}\leq \log\sqrt{\pi d} + \frac{d}{2}\log \frac{d}{2e} + 1.
    \]
    % \[
    %     \log \Gamma\tp{\frac{d}{2}+1} \leq \begin{cases}
    %         \log\sqrt{\pi d} + \frac{d}{2}\log \frac{d}{2e} + 1, & d \mbox{ is even}\\
    %         \log \sqrt{\frac{2\pi d}{d-1}} + \frac{d+1}{2}\log \frac{d}{2e} + 1, & d \mbox{ is odd}
    %     \end{cases},
    % \]
    % and 
    % \[
    %     \log \Gamma\tp{\frac{d}{2}+1} \geq \begin{cases}
    %         \log\sqrt{\pi d} + \frac{d}{2}\log \frac{d}{2e} , & d \mbox{ is even}\\
    %         \log \sqrt{\frac{2\pi d}{d-1}} + \frac{d+1}{2}\log \frac{d}{2e} , & d \mbox{ is odd}
    %     \end{cases}.
    % \]
    Back to \Cref{eq:1}, we have
    \begin{align*}
        \abs{h_1 - h_0(x)}&\leq \abs{\log \frac{1}{\eps} + \frac{d}{2}\log\frac{9e}{\eps} - \frac{d\|x\|^2}{2M} -\log\sqrt{\pi d}} + 1 \\
        &\leq \log \frac{1}{\eps} + \frac{d}{2}\log\frac{9e}{\eps} + \frac{dR^2}{8M} + \log\sqrt{\pi d} +1.
    \end{align*}
    Since $LM\geq d$, we have $-\+O(L)\cdot\!{Id}_d \preceq \mbox{(a)}\preceq \+O(L)\cdot\!{Id}_d$.
    
    For $\|x\|\in (R,2R]$, 
    \[
        \grad^2 f_0(x) = \grad^2 \mathfrak{g}_{[R,2R]}(x)(h_0(x)-h_1) + 2\grad \mathfrak{g}_{[R,2R]}(x) \grad h_0(x)^{\top} + \mathfrak{g}_{[R,2R]}(x) \cdot \grad^2 h_0(x).
    \]
    The remaining calculations are similar. %\ctodo{Maybe say more here.}
\end{proof}


\subsection{Perturb the base instance}\label{sec:hardinstance}
We then construct instances via perturbing $\mu_0$. Let $r_1= \sqrt{\frac{d}{L}\log \frac{LM}{d\eps}}$, $r_2=\sqrt{2}r_1$. Let $h_2\defeq h_1 - \gamma$, where $\gamma$ is a value to be determined later.
% \htodo{Assume the value of $\eps,L,M,d$ satisfy $4r_2\leq R$ and $r_1=\Omega(1)$.}

Note that when $\eps<1/32$, we have $4r_2\leq R$. For a point $v\in \bb R^d$ with $\|v\|=\frac{3R}{4}$, let $\mathfrak{g}_v(x) = q_{\!{mol}}\tp{\frac{\|x-v\|^2-r_1^2}{r_2^2-r_1^2}}$ and $f_v(x)=\mathfrak{g}_v(x)f_0(x)+(1-\mathfrak{g}_v(x))h_2$. This means that, outside the ball $\+B_{r_2}(v)$, $f_v\equiv f_0$, and inside the ball $\+B_{r_1}(v)$, $f_v\equiv h_2$. Define density of the distribution $\mu_v$ over $\bb R^d$ as $p_{\mu_v}\propto e^{-f_v}$. Let $Z_v = \int_{\bb R^d} e^{-f_v(x)} \dd x$.

\begin{lemma}\label{lem:gamma}
    There exists a $\gamma>0$ such that the following holds at the same time:
    \begin{itemize}
        \item  $\int_{\+B_{r_2}(v)} \tp{e^{-f_v(x)} - e^{-h_1}}  \dd x = 9\eps$;
        %\item $9 \tp{\frac{3R}{r_2}}^d \leq e^{\gamma}\leq 18\tp{\frac{3R}{r_1}}^d $;
        % \item $\eps \leq \DTV(\mu_0,\mu_v)\leq 10\eps$;
        \item $\frac{9\eps e^{h_1}}{\!{vol}(\+B_{r_2})}\le e^\gamma-1 \le \frac{9\eps e^{h_1}}{\!{vol}(\+B_{r_1})}$.
        \item $Z_0\leq Z_v\leq 1+10\eps$.
    \end{itemize}
    % There exists a $\gamma$ with $\eps\abs{h_1-\gamma} \leq ()$, such that $\eps \leq \DTV(\mu_0,\mu_v)\leq 10\eps$ for each $\|v\| = \frac{3R}{4}$. This $\gamma$ also satisfies that $Z_0\leq Z_v\leq 1+10\eps$
\end{lemma}

Before proving the lemma, let us examine the information it brings. Recall that $f_0(x)\equiv h_1$ when $x\in \+B_{r_2}(v)$ and we would like to perturb $h_1$ by amount of $\gamma$ to obtain $f_v$ so that there will be $\Theta(\eps)$ more probability mass in $\+B_{r_2}(v)$. For fixed $r_1$ and $r_2$, the lemma says that the order of $\gamma$ is roughly proportional to $h_1$. 

\begin{proof}[Proof of \Cref{lem:gamma}]
    Consider the value $\int_{\+B_{r_2}(v)} \tp{e^{-f_v(x)} - e^{-h_1}}\dd x$. It is continuous and increasing in $\gamma$ when $\gamma\ge 0$. When $\gamma = 0$, $\int_{\+B_{r_2}(v)} \tp{e^{-f_v(x)} - e^{-h_1}}\dd x = 0$. When $\gamma \to \infty $, this value goes to $\infty$. So we can find a $\gamma$ such that $\int_{\+B_{r_2}(v)} \tp{e^{-f_v(x)} - e^{-h_1}}\dd x = 9\eps$ holds exactly.

    For such a $\gamma$, we have
    \[
        Z_v = \int_{\bb R^d} e^{-f_v(x)} \dd x \leq \int_{\bb R^d} e^{-f_0(x)} \dd x + \int_{\+B_{r_2}(v)} \tp{ e^{-f_v(x)} - e^{-h_1}}\dd x = Z_0+9\eps \leq 1+10\eps.
    \]
    Also
    $$
        Z_v = \int_{\bb R^d} e^{-f_v(x)} \dd x \geq \int_{\bb R^d} e^{-f_0(x)} \dd x = Z_0.
    $$
    
    % Then we caculate $\DTV(\mu_0,\mu_v)$. On one hand, since $\+B(v,r_1)\subseteq \+B(v,r_2)\subseteq \set{x\in \bb R^d:\ \|x\| \in (\frac{R}{2},R]}$,
    % \begin{align*}
    %     \DTV(\mu_0,\mu_v) &= \frac{1}{2}\int_{\bb R^d} \abs{\frac{e^{-f_0(x)}}{Z_0} - \frac{e^{-f_v(x)}}{Z_v}} \dd x \\ 
    %     &\geq \frac{1}{2}\int_{\+B(v,r_2)} \abs{\frac{e^{-f_0(x)}}{Z_0} - \frac{e^{-f_v(x)}}{Z_v}} \dd x \\
    %     &=  \frac{1}{2}\int_{\+B(v,r_2)}  \abs{\frac{e^{-f_0(x)}}{Z_v} - \frac{e^{-f_v(x)}}{Z_v}} - \abs{\frac{e^{-f_0(x)}}{Z_0} -  \frac{e^{-f_0(x)}}{Z_v}}  \dd x  \\
    %     &= \frac{1}{2}\tp{\frac{9\eps}{Z_v} - e^{-h_1}\cdot \abs{\frac{1}{Z_0} - \frac{1}{Z_v}} \cdot \vol\tp{\+B(v,r_2)}} \\
    %     &\geq \eps.
    % \end{align*}

    % On the other hand, \begin{align*}
    %     \DTV(\mu_0,\mu_v) &= \frac{1}{2}\int_{\bb R^d} \abs{\frac{e^{-f_0(x)}}{Z_0} - \frac{e^{-f_v(x)}}{Z_v}} \dd x \\ 
    %     &\leq \frac{1}{2}\int_{\bb R^d }  \abs{\frac{e^{-f_0(x)}}{Z_v} - \frac{e^{-f_v(x)}}{Z_v}} + \abs{\frac{e^{-f_0(x)}}{Z_0} -  \frac{e^{-f_0(x)}}{Z_v}}  \dd x \\
    %     &\leq \frac{1}{2}\tp{\frac{9\eps}{Z_v} + Z_0\cdot \abs{\frac{1}{Z_0} - \frac{1}{Z_v}}}\\
    %     &\leq 10\eps.
    % \end{align*}
    
    It remains to calculate $e^\gamma$. We have that
    \[
        \vol\tp{\+B_{r_1}(v)} \cdot e^{-h_1} \tp{e^\gamma - 1}\leq \int_{\+B_{r_2}(v)} \tp{e^{-f_v(x)} - e^{-h_1}}  \dd x =9\eps \leq \vol\tp{\+B_{r_2}(v)} \cdot e^{-h_1} \tp{e^\gamma - 1}.
    \]
\end{proof}

\begin{corollary}\label{cor:gamma-bound}
    For our choice of $h_1$, $r_1$ and $r_2$, it holds that
    \[
        9\tp{\frac{3R}{r_2}}^d \leq e^{\gamma}\leq 18\tp{\frac{3R}{r_1}}^d.
    \]
\end{corollary}
\begin{proof}
    Recall that $e^{h_1}=\eps^{-1}\vol\tp{\+B_{3R}}$ and $\vol\tp{\+B_r} = \frac{\tp{\pi R^2}^{\frac{d}{2}}}{\Gamma\tp{\frac{d}{2}+1}}$. We have
    \[
        e^\gamma-1 \leq  9\tp{\frac{3R}{r_1}}^d \mbox{ and } e^\gamma-1 \geq 9 \tp{\frac{3R}{r_2}}^d.
    \]
    Therefore
    \[
        18\tp{\frac{3R}{r_1}}^d \geq e^\gamma\geq 9 \tp{\frac{3R}{r_2}}^d.
    \]
\end{proof}

\subsection{Properties of the perturbed distributions}
For every $v$ with $\|v\|= \frac{3R}{4}$, we first analyze the smoothness and second moment of the distribution $\mu_v$.

\begin{lemma}\label{lem:moment}
    For $\|v\|=\frac{3R}{4}$, $\E[\mu_v]{\|X\|^2}= \+O\tp{M}$.
\end{lemma}
\begin{proof}
    Direct calculation gives
    \begin{align*}
        \E[\mu_v]{\|X\|^2} & \leq \frac{\E[X\sim \+N\tp{0,\frac{M}{d}\cdot \!{Id}_d}]{\|X\|^2} + \vol (B_{2R})\cdot e^{-h_1}\cdot 4R^2 + \int_{\+B_{r_2}(v)} \tp{e^{-f_v(x)} - e^{-h_1}} \|X\|^2 \dd x}{Z_v} \\
        &\leq \frac{3M + R^2\cdot 9\eps}{Z_v} \leq 24 M.
    \end{align*}
    % \htodo{Here only $\tilde O(M)$. It seems that we cannot guarantee the $O(L)$-smooth of $f_0$ and $O(M)$ second moment of $\mu_v$. Or equivalently, we can choose $M=\frac{M_0}{\log \frac{M_0}{\eps}}$. We require $LM_0\geq d\log\frac{M_0}{\eps}$.}
\end{proof}

\begin{lemma}\label{smooth}
  For $\|v\|=\frac{3R}{4}$, the function $f_v$ is $\+O(L)$-smooth. 
\end{lemma}
\begin{proof}
    We only need to consider those $x\in \+B_{r_2}(v)\setminus \+B_{r_1}(v)$. For such $x$, $f_0(x)=h_1$. Therefore,
    \[
        \grad^2 f_v(x) = \gamma\cdot \grad^2 g_v(x).
    \]
    Note that
    \[
        \grad g_v(x) = \frac{2(x-v)}{r_2^2-r_1^2}\cdot q_{\!{mol}}'\tp{\frac{\|x-v\|^2-r_1^2}{r_2^2-r_1^2}}
    \]
    and
    \[
        \grad^2 g_v(x) = \frac{4(x-v)(x-v)^{\top}}{\tp{r_2^2-r_1^2}^2}\cdot q''_{\!{mol}}\tp{y=\frac{\|x-v\|^2-r_1^2}{r_2^2-r_1^2}} + \frac{2\!{Id}_d}{r_2^2-r_1^2}\cdot q'_{\!{mol}}\tp{\frac{\|x-v\|^2-r_1^2}{r_2^2-r_1^2}}.
    \]
    From \Cref{cor:gamma-bound}, 
    \[
        \gamma \leq \log 18 + \frac{d}{2}\log\frac{9LM}{\eps d} - \frac{d}{2}\log\log \frac{LM}{d\eps}.
    \]
    Therefore, $-\+O(L)\cdot\!{Id}_d \preceq \grad^2 f_v(x) \preceq \+O(L)\cdot\!{Id}_d $.
\end{proof}

We remark that the constants hidden in the $\+O(\cdot)$ in the above two lemmas are universal constants and do not depend on $d$ and $\eps$. 


\begin{lemma}\label{lem:TV}
    For $u,v\in \bb R^d$ such that $\|v\|=\|u\|=\frac{3R}{4}$ and $\+B_{r_2}(u)\cap \+B_{r_2}(v)=\emptyset$, $\DTV\tp{\mu_u,\mu_v}> 4\eps$.
\end{lemma}
\begin{proof}
    By the definition of total variation distance, 
    \begin{align*}
        \DTV(\mu_u,\mu_v) & = \frac{1}{2}\int_{\bb R^d} \abs{\frac{e^{-f_u(x)}}{Z_u} - \frac{e^{-f_v(x)}}{Z_v}} \dd x \\
        \mr{$Z_u=Z_v$} & =\frac{1}{2Z_v} \tp{\int_{\+B_{r_2}(u)}\abs{e^{-f_u(x)} - e^{-h_1}} \dd x + \int_{\+B_{r_2}(v)}\abs{e^{-f_v(x) }- e^{-h_1}} \dd x} \\
        &= \frac{9\eps}{Z_v} > 4\eps.
    \end{align*}
\end{proof}

\subsection{The number of disjoint $\+B_{r_2}(v)$'s}

\begin{lemma}\label{lem:disjointcap}
    Suppose $d\geq 5$. There exist $n=\frac{(d-1)\sqrt{d-2}}{2}\cdot \tp{\frac{3R}{8\sqrt{2}r_2}}^{d-1}$ vectors $v_1,v_2,\dots,v_n \in \bb R^d$ such that
    \begin{itemize}
        \item for each $i\in[n]$, $\|v_i\| = \frac{3R}{4}$;
        \item for each $i,j\in [n]$, if $i\ne j$, then $\+B_{r_2}(v_i)\cap \+B_{r_2}(v_j)=\emptyset$.
    \end{itemize}
\end{lemma}
\begin{proof}
    Let $S$ be the sphere $\set{x\in \bb R^d: \|x\| = \frac{3R}{4}}$. For two vectors $x,y\in \bb R^d$, let $\theta(x,y)$ represent the angle between $x$ and $y$. We first try to find $n$ disjoint caps $C_1,C_2,\dots,C_n$ on $S$. Denoting $v_i$ as the central vector of cap $C_i$, $C_i=\set{x\in \bb R^d: \|x\|=\frac{3R}{4}, \cos(\theta(x,v_i))\geq \ell}$ with $\ell=\frac{\sqrt{(\tp{\frac{3R}{4}}^2-2r_2^2)}}{\frac{3R}{4}}$. 
    
    In contrast, suppose we can only find $n'<\frac{(d-2)^{\frac{3}{2}}}{2}\cdot \tp{\frac{3R}{8\sqrt{2}r_2}}^{d-1}$ such disjoint caps $\set{C_i}_{1\leq i\leq n'}$ with central vectors $\set{v_i}_{1\leq i\leq n'}$. Then for any $w\in S$, there exist $i\in[n']$ and $x\in S$ such that $\cos(\theta(x,v_i))\geq \ell$ and $\cos(\theta(x,w))\geq \ell$. Otherwise we can find a new cap with center $w$.
    
    Via \Cref{lem:cos}, we know that $\cos(\theta(w,v_i))\geq \ell^2 - \tp{1-\ell^2}  \ell'$ with $\ell'=\frac{\tp{\frac{3R}{4}}^2-4r_2^2}{\tp{\frac{3R}{4}}^2}$. 
    This means we can find $n'$ larger caps with central vectors $\set{v_i}_{1\leq i\leq n'}$ and angle $\arccos(\ell')$ to cover the sphere.
    From \cite{L11}, however, the area of a cap with angle $\theta = \arccos(\ell')$ is $\frac{\Gamma\tp{\frac{d-1}{2}}}{\sqrt{\pi}\Gamma\tp{\frac{d}{2}}} \int_{0}^{\theta} \sin^{d-2}(\phi)\d \phi$ times of the total sphere. We have
    % \ctodo{Find the constant $c$.}
    \[
         \int_{0}^{\theta} \sin^{d-2}(\phi)\d \phi \leq \frac{1}{\ell'} \int_{0}^{\theta} \sin^{d-2}(\phi)\cos (\phi)\d \phi = \frac{1}{\ell'} \int_0^{\sin(\theta)} s^{d-2} \d s = \frac{1}{\ell'} \frac{\tp{\sin \theta}^{d-1}}{d-1}.
    \]
    Since $\sin \theta = \sqrt{1-\tp{\ell'}^2}\leq \frac{2\sqrt{2}r_2}{\frac{3R}{4}}$, the ratio between the area of this larger cap and the sphere can be bounded by
    \begin{align*}
        \frac{\Gamma\tp{\frac{d-1}{2}}}{\sqrt{\pi}\Gamma\tp{\frac{d}{2}}} \int_{0}^{\theta} \sin^{d-2}(\phi)\d \phi & \leq \frac{1}{\ell' (d-1)}\cdot \frac{\Gamma\tp{\frac{d-1}{2}}}{\sqrt{\pi}\Gamma\tp{\frac{d}{2}}}\cdot \tp{\frac{2\sqrt{2}r_2}{\frac{3R}{4}}}^{d-1} \\
        \mr{$R\geq 4r_2$}
        &\leq \frac{9}{5\sqrt{\pi}(d-1)} \cdot \frac{\Gamma\tp{\frac{d-1}{2}}}{\Gamma\tp{\frac{d}{2}}} \cdot \tp{\frac{8\sqrt{2}r_2}{3R}}^{d-1} \\
        \mr{Gautschi's inequality}
        &\leq \frac{9\sqrt{2}}{5\sqrt{\pi}}\cdot \frac{1}{(d-1)\sqrt{d-2}}\cdot \tp{\frac{8\sqrt{2}r_2}{3R}}^{d-1}\\
        &\leq \frac{2}{(d-1)\sqrt{d-2}} \tp{\frac{8\sqrt{2}r_2}{3R}}^{d-1}.
    \end{align*}
    
    % So there exists some universal constant $c'$ such that the ratio between the area of this larger cap and the sphere is no larger than $\tp{\frac{c' r_2}{R}}^{d-1}$. 
    % By choosing $c= \frac{1}{c'}$, 
    This will lead to a conflict since $n'$ such caps cannot cover the sphere. Therefore, we can find $n= \frac{(d-1)\sqrt{d-2}}{2}\cdot \tp{\frac{3R}{8\sqrt{2}r_2}}^{d-1}$ such $C_i$'s.
    
    Furthermore, from \Cref{lem:cosinBall}, $\+B_{r_2}(v_i)\cap \+B_{r_2}(v_j)=\emptyset$.
\end{proof}

\subsection{Proof of the lower bound}
\begin{theorem}[\Cref{thm:main} restated]
    % There exist a universal constant $C>0$ such that 
    For any $L,M>0$ satisfying $LM\ge d$ and for any $\eps\in(0,1/32)$, $d\geq 5$, if a sampling algorithm $\+A$ always terminates within 
    \[
        \frac{\eps (d-2)^{\frac{3}{2}}}{8}\cdot \tp{\frac{9}{256} \cdot \frac{LM}{d\eps} \cdot \frac{1}{\log \frac{LM}{d\eps}}}^{\frac{d-1}{2}}
        % \frac{\eps}{4}\cdot \tp{\frac{C\cdot LM}{2 d\eps} \cdot \frac{1}{\log \frac{LM}{d\eps}}}^{\frac{d-1}{2}}
        % % \approx \frac{\eps}{4}\exp\set{\frac{d}{2}\cdot \Omega\tp{\log \frac{LM}{d\eps}}}
    \]
    queries on every input instance in $\@D_{L,M}$, then there must exist some distribution $\mu\in \@D_{L,M}$ such that when the underlying instance is $\mu$, the distribution of $\+A$'s output, denoted as $\tilde \mu$, is $\eps$ away from $\mu$ in total variation distance, i.e., $\DTV(\mu,\tilde \mu)\geq \eps$.
\end{theorem}
\begin{proof}
    Let $v_1,v_2,\dots,v_n$ be the $n$ vectors in \Cref{lem:disjointcap}. For each $i\in [n]$, construct a distribution $\mu_{v_i}$ with density $p_i \propto e^{-f_{v_i}}$ as described in \Cref{sec:hardinstance}. From the discussion in previous sections, we can assume that every $\mu_{v_i}$ as well as the base instance are $L$-log-smooth and have second moment at most $M$. For simplicity, we write $\mu_{v_i}$ as $\mu_i$.

    We use $\Pr[\mu]{\cdot}$ and $\E[\mu]{\cdot}$ to denote the probability and expectation when the underlying instance is some distribution $\mu$. Let $\+E_{k,i}$ be the event that the algorithm $\+A$ queries a value in zone $\+B_{r_2}(v_i)$ in the $k$-th query and it is the first time that $\+A$ queries the points in $\+B_{r_2}(v_i)$. If the algorithm terminates before the $k$-th query, we regard $\+E_{k,i}$ as an impossible event.

    Assume $\+A$ always terminates in $N$ queries for some 
    $$
        N< \frac{\eps (d-2)^{\frac{3}{2}}}{8}\cdot \tp{\frac{9}{256}\cdot \frac{LM}{d\eps} \cdot \frac{1}{\log \frac{LM}{d\eps}}}^{\frac{d-1}{2}} < \frac{\eps}{4}\cdot \frac{(d-1)\sqrt{d-2}}{2}\cdot \tp{\frac{3R}{8\sqrt{2}r_2}}^{d-1} = \frac{\eps n}{4}.
    $$ 
    Let $\+E_{i}$ be the event that $\+A$ queries the points in $\+B_{r_2}(v_i)$ at least once. Then
    \begin{align*}
        \sum_{i=1}^n \Pr[\mu_0]{\+E_i}& = \sum_{i=1}^n \sum_{k=1}^N \Pr[\mu_0]{\+E_{k,i}} = \sum_{k=1}^N \sum_{i=1}^n \Pr[\mu_0]{\+E_{k,i}} \leq N<\frac{\eps n}{4}.
    \end{align*}
    So there exists some $i_0,j_0\in[n]$ and $i_0\neq j_0$ such that $\Pr[\mu_0]{\+E_{i_0}}<\frac{\eps}{3}$ and $\Pr[\mu_0]{\+E_{j_0}}<\frac{\eps}{3}$. Otherwise $\sum_{i=1}^n \Pr[\mu_0]{\+E_i}\geq \frac{\eps(n-1)}{3}\geq \frac{\eps n}{4}$ since $n\geq 4$ when $d\geq 5$. From union bound, $\Pr[\mu_0]{\ol{\+E_{i_0}}\cap \ol{\+E_{j_0}}}> 1-\frac{2\eps}{3}$. 
    
    We know that on $\bb R^d\setminus \+B_{r_2}(v_i)$, $f_{v_i}(x)=f_0(x)$ for each $i\in[n]$. Therefore, via coupling arguments,
    \[
        \Pr[\mu_{i_0}]{\ol{\+E_{i_0}}\cap \ol{\+E_{j_0}}} = \Pr[\mu_{j_0}]{\ol{\+E_{i_0}}\cap \ol{\+E_{j_0}}} = \Pr[\mu_0]{\ol{\+E_{i_0}}\cap \ol{\+E_{j_0}}}> 1-\frac{2\eps}{3}.
    \]

    Let $\+E=\+E_{i_0}\cup \+E_{j_0}$. Let $\tilde \mu_0^{\ol{\+E}}$ be the output distribution of $\+A$ with input distribution $\mu_0$ when $\ol{\+E}$ happens. 
    Since
    \begin{align*}
        4\eps\leq \DTV(\mu_{i_0},\mu_{j_0}) \leq \DTV(\mu_{i_0},\tilde \mu_0^{\ol{\+E}}) + \DTV(\mu_{j_0},\tilde \mu_0^{\ol{\+E}}),
    \end{align*}
    we have either $\DTV(\mu_{i_0},\tilde \mu_0^{\ol{\+E}})>2\eps$ or $\DTV(\mu_{j_0},\tilde \mu_0^{\ol{\+E}})>2\eps$. W.l.o.g., assume $\DTV(\mu_{i_0},\tilde \mu_0^{\ol{\+E}})>2\eps$. Assume the output distribution of $\+A$ is $\tilde \mu_{i_0}$ when the input is $\mu_{i_0}$. Let $\tilde \mu_{i_0}^{\+E}$ and $\tilde \mu_{i_0}^{\ol{\+E}}$ be $\tilde \mu_{i_0}$ conditioned on $\+E$ and $\ol{\+E}$ respectively and denote their density functions as $\tilde p_{i_0}^{\+E}$, $\tilde p_{i_0}^{\ol{\+E}}$ and $\tilde p_{i_0}$. Then $\tilde \mu_{i_0}^{\ol{\+E}}=\tilde \mu_0^{\ol{\+E}}$. 
    Then we have
    \begin{align*}
        \DTV(\mu_{i_0},\tilde \mu_{i_0}) &= \frac{1}{2}\int_{\bb R^d} \abs{p_{i_0}(x) - \tilde p_{i_0} (x)} \dd x\\
        &=\frac{1}{2}\int_{\bb R^d} \abs{p_{i_0}(x) - \Pr[\mu_{i_0}]{\+E}\tilde p_{i_0}^{\+E}(x) - \Pr[\mu_{i_0}]{\ol{\+E}}\tilde p_{i_0}^{\ol{\+E}}(x)} \dd x \\
        &\geq \Pr[\mu_{i_0}]{\ol{\+E}}\cdot \frac{1}{2} \int_{\bb R^d}\abs{p_{i_0}(x) - \tilde p_{i_0}^{\ol{\+E}}(x)} \dd x  - \Pr[\mu_{i_0}]{\+E}\cdot \frac{1}{2} \int_{\bb R^d}\abs{p_{i_0}(x) - \tilde p_{i_0}^{\+E}(x)} \dd x \\
        & = \Pr[\mu_{i_0}]{\ol{\+E}}\cdot \DTV(\mu_{i_0}, \tilde \mu_{i_0}^{\ol{\+E}}) - \Pr[\mu_{i_0}]{\+E}\cdot \DTV(\mu_{i_0}, \tilde \mu_{i_0}^{\+E}) \\
        & = \Pr[\mu_{i_0}]{\ol{\+E}}\cdot \DTV(\mu_{i_0}, \tilde \mu^{\ol{\+E}}_0) - \Pr[\mu_{i_0}]{\+E}\cdot \DTV(\mu_{i_0}, \tilde \mu_{i_0}^{\+E}) \\
        &\geq \tp{1-\frac{2\eps}{3}}\cdot 2\eps - \frac{2\eps}{3}\cdot 1\\
        &>\eps.
    \end{align*}
    This means, on instance $\mu_{i_0}$, the algorithm $\+A$ will fail to output a distribution which is $\eps$-close to $\mu_{i_0}$ in total variation distance.
\end{proof}



% !TEX root = main.tex

\section{Testing Uniformity of Monotone Distributions}

In this section, we prove the following theorem, which gives a query lower bound on testing uniformity of a distribution which is promised to be monotone using subcube conditioning queries. 
\begin{theorem}[Uniformity Testing of Monotone Distributions -- Lower Bound] For any $\eps > 0$,~any $\eps$-test for uniformity of distributions that are promised to be monotone must make $\tilde{\Omega}(\sqrt{n}/\eps^2)$ queries.
\end{theorem}
Similar to Section~\ref{sec:mon-lb}, we describe a distribution $\Dno$ supported on monotone product distributions over $\{-1,1\}^n$. Importantly, a distribution $\bp \sim \Dno$ will be $\Omega(\eps)$-far from the uniform distribution with probability $1-o_n(1)$ (Lemma \ref{lem:distancelemma}). Then, we show that for  $q$ which is at most $c \sqrt{n} / (\eps^2 \log^4 n)$, for a small enough constant $c > 0$, any function $\Alg \colon \{-1,1\}^{nq} \to \{\text{``accept''}, \text{``reject''} \}$ cannot output ``accept'' with probability at least $0.99$ when samples are drawn from the uniform distribution, and output ``reject'' with probability at least $0.99$ when samples are drawn from  $\Dno$.

\subsection{The distribution $\Dno$}\label{sec:dno-def}

A draw of $\bp \sim \Dno$ is generated as follows:
\begin{flushleft}\begin{itemize}
\item First, we let $\calD$ denote the distribution over vectors $\bmu$ where we 
independently set $\bmu_i$ to be $\eps/n^{1/4}$ with probability $1/\sqrt{n}$ and $0$ otherwise.
%sample a subset $\bN$ of size $\sqrt{n}$ uniformly at 
%random from $[n]$ and set $\bmu_i = 1 / n^{1/4}$ when $i\in \bN$ and $\bmu=0$ when $i\notin \bN$.}
  %{\color{red} and $\bmu_i=0$ otherwise}.
\item Then, we let $\bp$ be the monotone product distribution on $\{-1,1\}^n$ whose mean vector is $\bmu$.
\end{itemize}\end{flushleft}
The fact that $\bp \sim \Dno$ is far from the uniform distribution follows from the subsequent lemma.
\begin{lemma}\label{lem:distancelemma}
With probability at least $1 - o_n(1)$, $\bp \sim \Dno$ is $\Omega(\eps)$-far from the uniform distribution.
\end{lemma}
\begin{proof}
We consider a draw of $\bmu \sim \calD$, and note that with probability at least $1-o_n(1)$, $\bp \sim \Dno$ has a set $\bN \subset [n]$ of at least $\Omega(\sqrt{n})$ coordinates $i$ with $\bmu_i = \eps/n^{1/4}$. Fix such a draw and let $\bp$ denote the corresponding distribution. The total variation distance from $\bp$, generated with mean vector $\mu$, to the uniform distribution can be lower bounded by considering strings $x \in \{-1,1\}^n$ which have fewer $1$'s than $-1$'s in coordinates of $\bN$. 
For each such string $x$, letting $t(x)$ denote the number of coordinates in $\bN$ with $x_i = 1$, the probability of $x$ in $\bp$ is
$$
\frac{1}{2^n}\cdot \left(1 + \frac{\eps}{n^{1/4}} \right)^{t(x)} \left( 1 - \frac{\eps}{n^{1/4}}\right)^{|\bN| - t(x)} 
=\frac{1}{2^n}\cdot \left(1 - \frac{\eps^2}{\sqrt{n}}\right)^{t(x)} \left(1 - \frac{\eps}{n^{1/4}}\right)^{|\bN| - 2t(x)}< \frac{1}{2^n}.
$$
As a result, we can bounded $\dtv(\bp,\calU_n)$ from below as follows:
\begin{align}
\dtv(\bp, \calU_n) %&\geq \frac{1}{2^n} \sum_{\substack{x \in \{-1,1\}^n \\ t(x)\le |\bN|/2}} \left( 1 - \prod_{i=1}^n \left( 1 + x_i \cdot \mu_i \right)\right) \nonumber \\
     &\geq \frac{1}{2^n} \sum_{\substack{x \in \{-1,1\}^n \\ t(x) \leq |\bN|/2}} \left( 1 - \left(1 - \frac{\eps^2}{\sqrt{n}}\right)^{t(x)} \left(1 - \frac{\eps}{n^{1/4}}\right)^{|\bN| - 2t(x)}\right) \nonumber\\[0.5ex]
    &\ge
     \frac{1}{2^n} \sum_{\substack{x \in \{-1,1\}^n \\ t(x) \leq |\bN|/2}} \left( 1 -  \left(1 - \frac{\eps}{n^{1/4}}\right)^{|\bN| - 2t(x)}\right).\label{eq:hehe1}
\end{align}
%In (\ref{eq:simp-1}), we expanded the product over $i \in [n]$ using the definition of $t(x)$. Note that the fact $t(x) < |\bN| / 2$ implies $t(x) \leq |\bN| - t(x)$ and thus,
On the other hand, using $|\bN| = \Omega(\sqrt{n})$, there is a constant probability over a uniform $\bx \sim \{-1,1\}^n$ that $t(\bx)$ is bounded away from $|\bN|/2$ by $\Omega(n^{1/4})$.
For every such $\bx$, we have 
\[ %\left(1 + \frac{\eps}{n^{1/4}}\right)^{t(x)} \left(1 - \frac{\eps}{n^{1/4}}\right) = 
 \left(1 - \frac{\eps}{n^{1/4}}\right)^{|\bN| - 2t(\bx)} \leq \left(1 - \frac{\eps}{n^{1/4}}\right)^{\Omega(n^{1/4})}\le e^{-\Omega(\eps)}\le 1-\Omega(\eps). \]
Combining everything we have from (\ref{eq:hehe1}) that
$$
\dtv(\bp,\calU_n)\ge \frac{1}{2^n}\cdot \Omega(2^n)\cdot \Omega(\eps)=\Omega(\eps).
$$
This finishes the proof of the lemma.
%We now lower bound (\ref{eq:simp-1}) using the above inequality, were we have (\ref{eq:simp-%1}) is at least
%\[ \frac{1}{2^n} \sum_{x \in \{-1,1\}^n} \frac{\eps}{n^{1/4}} \left(|\bN| - 2t(x) \right)^+ %= \frac{\eps}{n^{1/4}} \Ex_{\bx \sim \{-1,1\}^n}\left[ \left(|\bN| - 2t(\bx) %\right)^+\right], \]
%which is $\Omega(\eps)$, 
\end{proof}

\def\br{\mathbf{r}}

\subsection{Indistinguishability of $\Dno$ from the uniform distribution}

Consider the task of distinguishing via $q$ independent samples, $\bx_1, \dots, \bx_q \in \R^n$, whether these samples were drawn from the standard $n$-dimensional Gaussian $\calN(0, I)$, or an $n$-dimensional Gaussian $\calN(\bmu, I)$, where $\bmu \sim \calD$. We consider the above problem because of the following simple claim, which shows how to generate a product distribution whose mean vector has $i$-th coordinate $\Omega(\bmu_i)$.

\begin{claim}
Let $\sign\colon \R^n \to \{-1,1\}^n$ denote the function which applies $\sign(\cdot)$  coordinate-wise. 
\begin{flushleft}\begin{itemize}
\item The uniform distribution over $\{-1,1\}^n$ can be generated by sampling $\bx \sim \calN(0, I)$ and outputting $\sign(\bx)$.
\item For a fixed $\mu \in [0, 1/2]^n$, consider the product distribution over $\{-1,1\}^n$ generated by sampling $\bx \sim \calN( \mu, I)$ and outputting $\sign(\bx)$. Then, the mean vector of such a distribution has the $i$-th coordinate set to $\Omega(\mu_i)$.
\end{itemize}\end{flushleft}
\end{claim}

\begin{proof}
The first condition is by symmetry of the Gaussian distribution, and the second condition is by standard Gaussian anti-concentration, whenever $\mu \in [0, 1/2]$.
\end{proof}

Hence, it suffices to prove the following lemma.

\begin{lemma}\label{lem:mainlemma1}
% Let $\calD_n'$ denote the distribution over $n$-dimensional Gaussian  $\calN(\bmu, I)$ where $\bmu \sim \calD$. 
Consider an algorithm that takes $q$ samples $\bx_1, \ldots, \bx_q\in \R^n$ and satisfies the following guarantees:
\begin{flushleft}\begin{itemize}
\item \textbf{\emph{Standard Case}}: If $\bx_1,\dots, \bx_q \sim \calN(0, I)$ and the algorithm receives those samples, then the algorithm outputs ``standard'' with probability at least $0.99$.
\item \textbf{\emph{Non-Standard Case}}: We sample $\bmu \sim \calD$,\footnote{The distribution $\calD$ is defined in the first bullet point of Subsection~\ref{sec:dno-def}.} then $\bx_1,\dots, \bx_q \sim \calN(\bmu, I)$, and the algorithm receives those samples, the algorithm outputs ``not standard'' with probability at least $0.99$.
\end{itemize}\end{flushleft}
Then, the number of samples must satisfy $q = \tilde{\Omega}(\sqrt{n}/\eps^2)$.
\end{lemma}

\subsection{Proof of \Lem{mainlemma1}} \label{sec:proof-mono-lb}

We prove by contradiction. {We assume that $q\le  \sqrt{n}/(c\eps^2 \log^4n)$ for some sufficiently large constant $c$} and show below
that the algorithm cannot distinguish between the two cases.

We set up some notation for the proof. We use $i \in [q]$ as an index over the set of queries,
while $j \in [n]$ indexes the dimension/coordinate.
We write $f_y, f_n \colon (\R^n)^q \to \R_{\geq 0}$ to denote the probability density functions of a tuple of $q$ independent samples from $\calN(0, I)$ (for $f_y$) or $\calN(\bmu, I)$ with $\bmu \sim \calD$ (for $f_n$) given by
\begin{align*}
f_y(x_1, \dots, x_q) &= \frac{1}{(2\pi)^{n/2}} \prod_{j=1}^n  \exp\left(-\frac{1}{2} \sum_{i=1}^q x_{ij}^2 \right) \qquad \text{and}\quad \\[1ex]
f_{n}(x_1,\dots, x_q) &= \frac{1}{(2\pi)^{n/2}} \prod_{j=1}^n \exp\left(-\frac{1}{2} \sum_{i=1}^q x_{ij}^2 \right) \left(1 - \frac{1}{\sqrt{n}} + \frac{1}{\sqrt{n}} \cdot \exp\left( \frac{\eps}{n^{1/4}}\sum_{i=1}^q x_{ij} -  \frac{q\eps^2}{2\sqrt{n}} \right)\right).
\end{align*}
The definition of $f_y$ comes from the product of $q$ many $n$-dimensional Gaussian p.d.fs; the definition of $f_n$ comes from the fact that each coordinate $j$ behaves independently under the draw of $\bmu \sim \calD$: with probability $1/\sqrt{n}$, $\bmu_i = \eps / n^{1/4}$ and is otherwise 0.

%We use $\bx_{ij}$ to denote the $j$th coordinate
%of $\bx_i$. Note that $\bmu$ is a vector in $\R^n$, where the $j$th coordinate is set according
%to the distribution described earlier.
%We use $f_y, f_n \colon (\R^n)^q \to \R_{\geq 0}$ to denote the probability density functions of a tuple of $q$ independent samples from $\calN(0, I)$ (for $f_y$) or $\calN(\bmu, I)$ for $\bmu \sim \calD$ (for $f_n$). We have
%\begin{align*}
%f_y(\bx_1, \dots, \bx_q) &= \frac{1}{(2\pi)^{n/2}} \prod_{i=1}^q  \exp\left(-\|\bx_i\|^2_2/2 \right) \qquad \text{and}\\[0.5ex]
%f_n(\bx_1,\dots, \bx_q) &= \frac{1}{(2\pi)^{n/2}} \prod_{i=1}^q \exp\left(-\|\bx_i - \bmu\|^2_2/2 \right)
%\end{align*}
%Note that $\bmu$ is a random vector, based on the distribution $\cD$. For a fixed choice of $\bmu$,
%we have the above probability distribution defined over $(\bx_1, \bx_2, \ldots, \bx_q)$.
The main lemma below shows that these pdfs are nearly the same with high probability over draws $\bx_1,\ldots,\bx_q\sim \cN(0,I)$.
(Technically, we only need to lower bound $f_n$ by $f_y$.)

\begin{lemma} \label{lem:gaussian-calc} Consider $q$ independent draws of $\bx_i \sim \cN(0,I)$.
%(the $n$-dimensional
%Gaussian distribution).
With probability at least $1- o_n(1)$, 
$$\frac{f_n(\bx_1, \ldots, \bx_q)}{f_y(\bx_1, \ldots, \bx_q)} \geq 1-o_n(1) .$$
\end{lemma}

\begin{proof} 
%Let us take the ratio and perform some simple manipulations.
%\begin{align}
%&\frac{f_n(\bx_1, \bx_2, \ldots, \bx_q)}{f_y(\bx_1, \bx_2, \ldots, \bx_q)}
%= \frac{\prod_{i=1}^q \exp(-\|\bx_i - \bmu\|^2_2/2 )}{\prod_{i=1}^q  \exp(-\|\bx_i\|^2_2/2 ) }
%= \prod_{i=1}^q \exp(\bx_i \cdot \bmu - \|\bmu\|^2/2) \\
%&= \prod_{i=1}^q \exp(\sum_{j=1}^n x_{ij}\mu_j - \sum_{j=1}^n \mu^2_j/2) 
%= \exp(\sum_{i=1}^q \sum_{j=1}^n x_{ij} \mu_j - \sum_{i=1}^q \sum_{j=1}^n \mu^2_j/2)\\
%&= \exp\Big(\sum_{j=1}^n (\sum_{i=1}^q x_{ij}) \mu_j - \sum_{j=1}^n q\mu^2_j/2\Big)
%= \prod_{j=1}^n \exp(X_j \mu_j - q\mu^2_j/2)
%\end{align}
We set $\bX_j := \sum_{i=1}^q \bx_{ij}$ for each $j\in [n]$. %, which is the sum of $j$th coordinates over all the pdf domain (which corresponds to the
%queries). We now take the expectation over the bias vector $\bmu$. Recall that each coordinate $\mu_i$
%is generated independently. With probability $1/\sqrt{n}$, $\mu_i = \eps/n^{1/4}$ and zero otherwise.
%Using these facts, we get the following.
Then the ratio can be written as
\begin{align}
%\EX_{\bmu}
\frac{f_n(\bx_1, \ldots, \bx_q)}{f_y(\bx_1, \ldots, \bx_q)}
%&= \prod_{j=1}^n \EX_{\mu_j}\exp(X_j \mu_j - q\mu^2_j/2) \\
&= \prod_{j=1}^n\left( 1 + \frac{1}{\sqrt{n}}\left( \exp\left(\frac{\eps\bX_j}{n^{1/4}}  - \frac{q\eps^2}{2\sqrt{n}}\right)-1\right)\right)
%\\
%&= \prod_{j=1}^n \Big[ 1 + (1/\sqrt{n})(\exp(\eps X_j/n^{1/4} - q\eps^2/2\sqrt{n}) - 1)\Big]
\end{align}
and thus, 
\begin{align}
 \ln\left( \frac{f_n(\bx_1, \ldots, \bx_q)}{f_y(\bx_1,  \ldots, \bx_q)}\right) 
&= %\sum_{j=1}^n \ln\Big[ 1 + (1/\sqrt{n})(\exp(\eps X_j/n^{1/4} - q\eps^2/2\sqrt{n}) - 1)\Big] \\
  \sum_{j=1}^n \ln\left[ 1 + \frac{\bW_j}{\sqrt{n}}\right],
  \quad \text{with\ }\bW_j \eqdef \exp\left(\frac{\eps \bX_j}{n^{1/4}} - \frac{q\eps^2}{2\sqrt{n}}\right) - 1.
  \label{eq:wj}
\end{align}
%
%We define $$.
At this point, we use the distributional information of $\bx_1, \ldots, \bx_n\sim \cN(0,I)$:

\begin{claim} \label{clm:wj} With probability at least $1-1/n$, we have 
$|\bW_j| \leq 1/\log n$ for all $j \in [n]$.
\end{claim}

\begin{proof} Note that $\bX_j = \sum_{i=1}^q \bx_{ij}$ where each $\bx_{ij} \sim \cN(0,1)$. Hence, $\bX_j \sim \cN(0,q)$.
With probability at least $1-1/n^2$, we have $|\bX_j| \le 4\sqrt{q}\log n$. By a union bound over all coordinates,~with probability at least $1-1/n$, we have $|\bX_j| \le  4\sqrt{q}\log n$ for all $j\in [n]$.

When this is the case, using $ {q\le \sqrt{n}/(c\eps^2 \log^4n)}$
  we have (when $c$ is sufficiently large)
$$\frac{\eps |\bX_j|}{n^{1/4}} \le \frac{4\eps \sqrt{q} \log n }{n^{1/4}} \le \frac{1}{4\log n}\quad\text{and}\quad
\frac{q\eps^2}{2\sqrt{n}} \le \frac{1}{4\log n}.$$ 
%We deduce that $|\eps X_j/n^{1/4} - q\eps^2/2\sqrt{n}| \leq \eps|X_j|/n^{1/4} + q\eps^2/2\sqrt{n}
%< 1/2\log n$. 
Using $1/(1-z) \geq \exp(z) \geq 1+z$ for $|z|\le 1$, we have
$$ \exp\left(\frac{\eps \bX_j}{n^{1/4}} - \frac{q\eps^2}{2\sqrt{n}}\right) \geq 1-\frac{1}{2\log n}$$
and
$$ \exp\left(\frac{\eps \bX_j}{n^{1/4}} - \frac{q\eps^2}{2\sqrt{n}}\right) \leq \frac{ 1}{1-1/(2\log n)} \leq 1 + \frac{1}{\log n}.$$
So $|\bW_j|  \leq 1/\log n$ for all $j$ and the claim follows.
\end{proof}

We go back to \Eqn{wj}, and apply the inequality $\ln(1+z) \geq z-z^2$ for $|z| \leq 1/2$.
With probability at least $1-1/n$ over $\bx_1, \ldots, \bx_n\sim \cN(0,I)$, we have $|\bW_j|\le 1/\log n$ for all $j$ and thus,
\begin{align}
\ln\left(\frac{f_n(\bx_1,  \ldots, \bx_q)}{f_y(\bx_1,  \ldots, \bx_q)} \right)\nonumber
&\geq \frac{1}{\sqrt{n}}\sum_{j=1}^n \bW_j - \frac{1}{n} \sum_{j=1}^n \bW^2_j \\
&\geq \frac{1}{\sqrt{n}}\sum_{j=1}^n \bW_j - \frac{1}{\log^2 n} \ \ \ ~~~~~~~ \textrm{(by \Clm{wj}, $\bW^2_j \leq 1/\log^2 n$)}\nonumber \\
&= \frac{1}{\sqrt{n}} \left[ \exp\left(-\frac{q \eps^2}{2\sqrt{n}}\right) \sum_{j=1}^n \exp\left(\frac{\eps \bX_j}{n^{1/4}}\right) - n \right]  - \frac{1}{\log^2 n}. \label{eq:gaussian}
\end{align}
The heart of the matter is the
next claim on the distribution of sum of exponentials of Gaussians.

\begin{claim} \label{clm:exp-gauss} Let each $\bX_j \sim \cN(0,q)$ be independent.
With probability at least $ 1-1/\sqrt{\log n}$,  we have$$\sum_{j=1}^n \exp\left(\frac{\eps \bX_j}{n^{1/4}}\right) \geq n \cdot\exp\left(\frac{q \eps^2}{2\sqrt{n}}\right) - \frac{\sqrt{n}}{\log^{0.25} n}.$$
\end{claim}

\begin{proof} Denote $\bY_j = \eps \bX_j/n^{1/4}$ and consider the random variable $\bZ_j = \exp(\bY_j)$. 
Observe~that~$\bY_j$ $\sim \cN(0, q\eps^2/\sqrt{n})$. Using the formula for the moment generating function
    of the Gaussian~\cite{gaussian-wiki}, we have $\EX[\exp(t\bY_j)] = \exp(q\eps^2 t^2/(2\sqrt{n}))$.
Hence, $$\EX[\bZ_j] = \EX[\exp(\bY_j)] = \exp\left(\frac{q\eps^2}{2\sqrt{n}}\right)\quad\text{and}\quad \EX[\bZ^2_j] = \EX[\exp(2\bY_j)] = \exp\left(\frac{2q\eps^2}{\sqrt{n}}\right).$$
So $\textrm{var}[\bZ_j] = \exp(2q\eps^2/\sqrt{n}) - \exp(q\eps^2/\sqrt{n})\leq 1/\log n$ using  $q\eps^2/\sqrt{n} = o (1/\log n)$.
Overall, we have 
$$\EX\left[\sum_{j=1}^n \bZ_j\right] = n\cdot \exp\left(\frac{q\eps^2}{ 2\sqrt{n}}\right)\quad\text{and}\quad\var\left[\sum_{j=1}^n \bZ_j\right] \leq \frac{n}{\log n}$$ since
all $\bZ_j$'s are independent. By Chebyshev's inequality, we have 
$$\Pr\left[\left|\sum_{j=1}^n \bZ_j - n\cdot \exp\left(\frac{q\eps^2}{2\sqrt{n}}\right)\right| > \frac{\sqrt{n}}{\log^{0.25} n}\right] \leq \frac{1}{\sqrt{\log n}}.$$
This finishes the proof of the claim.
\end{proof}

%We pick up from \Eqn{gaussian}, and apply \Clm{exp-gauss}.  
In particular, with probability at least $1-1/n - 1/\sqrt{\log n} = 1-o_n(1)$, we get that
\begin{align*}
\exp\left(-\frac{q \eps^2}{2\sqrt{n}}\right) \sum_{j=1}^n \exp\left(\frac{\eps \bX_j}{n^{1/4}}\right) & \geq \exp\left(-\frac{q \eps^2}{2\sqrt{n}}\right)\cdot \left( n \cdot \exp\left(\frac{q \eps^2}{2\sqrt{n}}\right) - \frac{\sqrt{n}}{\log^{0.25} n}\right)  >  n - \frac{\sqrt{n}}{\log^{0.25} n} 
    \end{align*}
where in the last inequality we used $\exp(-q \eps^2/2\sqrt{n}) < 1$. Substituting in \Eqn{gaussian}, we get
\begin{align}
\ln\left(\frac{f_n(\bx_1, \ldots, \bx_q)}{f_y(\bx_1,  \ldots, \bx_q)} \right)
& > -\frac{ 1}{    \log^{0.25} n}  - \frac{1}{\log^2 n} \label{eq:18}
\end{align}
Hence, with probability at least $1-o_n(1)$,  we have
$$\frac{f_n(\bx_1,  \ldots, \bx_q)}{f_y(\bx_1, \ldots, \bx_q)} \geq \exp\left(-\frac{ 1}{ \log^{0.25} n} - \frac{1}{\log^2n}\right) = 1-o_n(1).$$
This finishes the proof of the lemma.
\end{proof}

We can now complete the proof of \Lem{mainlemma1}. 
Consider the set $Y \subset (\R^n)^{q}$ of tuples that lead  the algorithm to output ``standard.''
Then we must have $$\Prx_{ \bx_i \sim \cN(0,I)}\big[(\bx_1, \dots, \bx_q) \in Y\big] \geq 0.99.$$ Let $Y' \subseteq Y$
be the set of tuples that also satisfy the condition of \Lem{gaussian-calc}.
By a union bound $$\Prx_{ \bx_i \sim \cN(0,I)}\big[(\bx_1, \dots, \bx_q) \in Y'\big] \geq 0.99 - o_n(1) \geq 0.98.$$
Thus, $\int_{Y'} f_y(\bx_1, \dots, \bx_q) d \bx_1 d \bx_2 \ldots d \bx_q \geq 0.98$. 
By the condition of \Lem{gaussian-calc}, we have  $$\int_{Y'}  f_n(\bx_1, \dots, \bx_q)  d \bx_1 \ldots d \bx_q \geq (1-o_n(1))\cdot0.98 \geq 0.97.$$
%Since $f_n(\cdot)$ is absolutely bounded, by Fubini's theorem, we can switch the expectation and integral. So
%$\EX_{\mu} [\int_{Y'} f_n(\bx_1, \dots, \bx_q) d \bx_1 
This is exactly the probability that we see a tuple in $Y'$,
when we generate the samples $\bx_1, \cdots, \bx_q$ from the non-standard case. Thus,
with probability at least $0.97$, the algorithm outputs ``standard'' when the samples are generated
from the non-standard case. This completes the contradiction.

\begin{comment}
\subsection{The old proof}

\begin{proof} 
We write $f_y, f_n \colon (\R^n)^q \to \R_{\geq 0}$ to denote the probability density functions of a tuple of $q$ independent samples from $\calN(0, I)$ (for $f_y$) or $\calN(\bmu, I)$ for $\bmu \sim \calD$ (for $f_n$) given by
\begin{align*}
f_y(x_1, \dots, x_q) &= \frac{1}{(2\pi)^{n/2}} \prod_{k=1}^n  \exp\left(-\frac{1}{2} \sum_{i=1}^q x_{ik}^2 \right) \qquad \text{and}\qquad \\[1ex]
f_{n}(x_1,\dots, x_q) &= \frac{1}{(2\pi)^{n/2}} \prod_{k=1}^n \exp\left(-\frac{1}{2} \sum_{i=1}^q x_{ik}^2 \right) \left(1 - \frac{1}{\sqrt{n}} + \frac{1}{\sqrt{n}} \cdot \exp\left( \frac{\eps}{n^{1/4}}\sum_{i=1}^q x_{ik} -  \frac{q\eps^2}{2\sqrt{n}} \right)\right).
\end{align*}
The definition of $f_y$ comes from the product of $q$ many $n$-dimensional Gaussian p.d.fs, and the definition of $f_n$ comes from the fact that each coordinate $k \in [n]$ behaves independently under the draw of $\bmu \sim \calD$, and with probability $1/\sqrt{n}$, $\bmu_i = \eps / n^{1/4}$ and is otherwise 0.

Letting $s = \sum_{i=1}^q x_i$ denote the vector given by the sum and $\xi = \exp(-q\eps^2 / (2\sqrt{n}))$, we write
\begin{align*}
\dfrac{f_n(x_1, \dots, x_q)}{f_y(x_1, \dots, x_q)} &= \prod_{k=1}^n \left(1 + \frac{\xi}{\sqrt{n}} \left( \exp\left( \frac{\eps s_k}{n^{1/4}} \right) - \exp\left(\frac{q\eps^2}{2\sqrt{n}}\right)  \right) \right).
\end{align*}
Applying Taylor expansion, we have
$$
\dfrac{f_n(x_1, \dots, x_q)}{f_y(x_1, \dots, x_q)}  = \prod_{k=1}^n \left(1 + \frac{\xi}{\sqrt{n}} \left( 
\sum_{\text{$\ell$ odd}} \frac{1}{\ell!}\left(\frac{\eps s_k}{n^{1/4}}\right)^\ell
+\sum_{\ell=1}^\infty \left(\frac{1}{(2\ell)!}\left(\frac{\eps s_k}{n^{1/4}}\right)^{2\ell}
-\frac{1}{\ell!}\left(\frac{q\eps^2}{2\sqrt{n}}\right)^\ell
\right)\right)\right).
$$
For each $k$, we write $A_k$ and $B_k$ to denote
\begin{align*}
A_k  \eqdef \frac{\xi}{\sqrt{n}}\sum_{\ell \text{ odd}} \frac{1}{\ell!} \left( \frac{\eps s_k}{n^{1/4}}\right)^{\ell} \quad\text{and}\quad 
B_k  \eqdef \frac{\xi}{\sqrt{n}} \sum_{\ell = 1}^{\infty} \left(\frac{\eps^2}{\sqrt{n}} \right)^{\ell} \left(\frac{ s_k^{2\ell}}{(2\ell)!} - \frac{ q^{\ell}}{2^{\ell}\ell!} \right),
\end{align*}
and $A=\sum_{k=1}^n A_k$ and $B=\sum_{k=1}^n B_k$:
\begin{align*}
    A  \eqdef \frac{\xi}{\sqrt{n}}\sum_{\ell \text{ odd}} \frac{1}{\ell!} \left( \frac{\eps}{n^{1/4}}\right)^{\ell} \sum_{k=1}^n s_k^{\ell}\quad\text{and}\quad
    B  \eqdef \frac{\xi}{\sqrt{n}} \sum_{\ell = 1}^{\infty} \left(\frac{\eps^2}{\sqrt{n}} \right)^{\ell} \left(\frac{1}{(2\ell)!}\sum_{k=1}^{n} s_k^{2\ell} - \frac{nq^{\ell}}{2^{\ell}\ell!} \right).
\end{align*}
Then we have 
\begin{equation}\label{eq:haha2}
\dfrac{f_n(x_1, \dots, x_q)}{f_y(x_1, \dots, x_q)}
=\prod_{k=1}^n \left(1+A_k+B_k\right)
\le \exp\big(A+B\big)  
\end{equation}
using $1+x\le e^x$.
We finish the proof using the following claim, which
  shows that $A$ and $B$ are both small with high probability when $\bx_1,\ldots,\bx_q$ are drawn
  from the non-standard case:

\begin{claim}\label{mainclaim1}
When $\bmu\sim \calD$ and $\bx_1,\ldots,\bx_q\sim \calN(\bmu, I)$ independently,
  we have both $A$ and $B$ are $o_n(1)$ with probability at least $1-o_n(1)$.
\end{claim}

We delay the proof of Claim \ref{mainclaim1} to the end and use it to 
  finishes the proof of Lemma \ref{mainlemma1}.

Let the set $N \subset (\R^n)^{q}$ denote the subset of tuples of samples $(x_1, \dots, x_q)$ such that, if the algorithm observes samples $(x_1, \dots, x_q)\in N$, it declares ``not standard.'' Then, we note that by assumption of the algorithm, a draw of $\bx_1, \dots, \bx_q \sim \calN(\bmu, I)$ with $\bmu\sim \calD$ must have $(\bx_1,\dots, \bx_q) \in N$ with probability at least $0.99$. Furthermore, let $N' \subseteq N$ be the set of tuples $(x_1, \dots, x_q)$ in $N$ such that $A$ and $B$ derived from it satisfy the condition of  Claim~\ref{mainclaim1}. On the one hand, we have from Claim~\ref{mainclaim1} and 
  a union bound that 
\begin{align*}
\Prx_{\substack{\bmu\sim \calD\\
\bx_1,\dots, \bx_q \sim \calN(\bmu, I)}}\left[ (\bx_1,\dots, \bx_q) \in N' \right] \geq 0.99-o_n(1)>0.98.
\end{align*}
On the other hand, for every $(x_1,\ldots,x_q)\in N'$ we have from (\ref{eq:haha2}) that
$$
\frac{f_n(x_1, \dots, x_n) } {f_y(x_1, \dots, x_n)}\le 
\exp\big(1+o_n(1)\big)\le 1+o_n(1).
$$ 
As a result, we have 
$$
\Prx_{\bx_1,\ldots,\bx_q\sim \calN(0,I)}\left[(\bx_1,\ldots,\bx_q)\in N'\right]
\ge (1-o_n(1))\cdot \Prx_{\substack{\bmu\sim \calD\\
\bx_1,\dots, \bx_q \sim \calN(\bmu, I)}}\left[ (\bx_1,\dots, \bx_q) \in N' \right] 
>0.97.
$$
However, this means
%but since $f_n(x_1, \dots, x_n) \geq f_y(x_1, \dots, x_n) (1 - o(1))$ for every $(x_1, \dots, x_n) \in A'$, we must have
%\begin{align*}
%\Prx_{\substack{\bmu \sim \calD \\ \bx_1,\dots, \bx_q \sim \calN(\bmu, I)}}\left[ (\bx_1,\dots, \bx_q) \in A' \right] \geq 0.98 \cdot (1 - o(1)) \geq 0.97.
%\end{align*}
  that the algorithm outputs ``not standard'' with probability at least $0.97$ when samples are drawn from $\calN(0,I)$, which contradicts the assumption that the algorithm outputs ``not standard'' with probability $0.99$ if the distribution is $\calN(0,I)$.
  %for $\bmu \sim \calD$.
%We will divide the computation in the following way. We show that under a certain event (which occurs often over draw of $\bx_1,\dots, \bx_q$), we have the following two inequalities holding:
%Once we establish both, then we can lower bound the ratio by expanding the Taylor expansion of both instances of $\exp(\cdot)$, and taking a union bound
%\begin{align*}
%    \dfrac{f_n(x_1, \dots, x_q)}{f_y(x_1, \dots, x_q)} \geq 1 - |A| - |B| \geq 1 - o(1).
%\end{align*}
\end{proof}

Finally we prove Claim \ref{mainclaim1}:

\begin{proof}[Proof of Claim \ref{mainclaim1}]
First, with probability at least $1-o_n(1)$, we have that the number of $i\in [n]$
  with $\bmu_i=\eps/n^{1/4}$ is at most $O(\sqrt{n})$.
Fix such a $\mu$ and without less of generality, we assume that 
  $\mu_1,\ldots,\mu_m=0$ and $\mu_{m+1},\ldots, \mu_n=\eps/n^{1/4}$
  for some $m$ with $n-m=O(\sqrt{n})$.
Then $\bs_k$ is drawn from $\calN(0,q)$ for $k\le m$ and 
  $\bs_{k}$ is drawn from $\calN(q\eps/n^{1/4},q)$ for $k>m$.
We also break $A,B$ into $A',A''$ and $B',B''$ accordingly, where
\begin{align*}
A'  &\eqdef \frac{\xi}{\sqrt{n}}\sum_{\ell \text{ odd}} \frac{1}{\ell!} \left( \frac{\eps}{n^{1/4}}\right)^{\ell} \sum_{k=1}^m \bs_k^{\ell}\\[0.6ex]
A''  &\eqdef \frac{\xi}{\sqrt{n}}\sum_{\ell \text{ odd}} \frac{1}{\ell!} \left( \frac{\eps}{n^{1/4}}\right)^{\ell} \sum_{k=m+1}^n \bs_k^{\ell} \\[0.6ex]
B'  &\eqdef \frac{\xi}{\sqrt{n}} \sum_{\ell = 1}^{\infty} \left(\frac{\eps^2}{\sqrt{n}} \right)^{\ell} \left(\frac{1}{(2\ell)!}\sum_{k=1}^{m} \bs_k^{2\ell} - \frac{mq^{\ell}}{2^{\ell}\ell!} \right)\quad \text{and}\qquad\\[0.6ex]
B''  &\eqdef \frac{\xi}{\sqrt{n}} \sum_{\ell = 1}^{\infty} \left(\frac{\eps^2}{\sqrt{n}} \right)^{\ell} \left(\frac{1}{(2\ell)!}\sum_{k=m+1}^{n} \bs_k^{2\ell} - \frac{(n-m)q^{\ell}}{2^{\ell}\ell!} \right).
\end{align*}
It suffices to show that $|A'|,|A''|,|B'|$ and $|B''|$ are all 
  $o_n(1)$ with probability $1-o_n(1)$.
For $A''$ and $B''$, we have from a standard union bound that with probability
  at least $1-o_n(1)$, every $\bs_k$ is 
  at most 
$$
\frac{q\eps}{n^{1/4}}+O\left(\sqrt{q\log n}\right)=O\left(\sqrt{q\log n}\right)
$$
for every $k>m$ using our choice of $q$.
When this holds for every $\bs_k$, $k>m$, we have
$$
|A''|\le \frac{1}{\sqrt{n}}\sum_{\text{$\ell$ odd}}\frac{1}{\ell!}
  \left(\frac{\eps}{n^{1/4}}\right)^\ell \cdot O(\sqrt{n})\cdot 
  \left(O\left(\sqrt{q\log n}\right)\right)^\ell=o_n(1),
$$
and 
$$
|B''| \le  \frac{1}{\sqrt{n}} \sum_{\ell = 1}^{\infty} \left(\frac{\eps^2}{\sqrt{n}} \right)^{\ell} \left(\frac{1}{(2\ell)!}\cdot O(\sqrt{n})\cdot 
\left(O\left(\sqrt{q\log n}\right)\right)^\ell+ \frac{1}{2^\ell \ell!}\cdot O(\sqrt{n})\cdot  q^{\ell} \right)=o_n(1).
$$

We bound $|A'|$ and $|B'|$ in the rest of the proof, where 
  we need to be more careful and take advantage of cancellations in the sums.
Below $k$ will always denote an index in $[m]$ and $\bs_k$ is always 
  drawn independently from $\calN(0,q)$.
We start with the following claim about expectations of $\bs_k^\ell$.

\begin{claim}\label{cl:exp}
For every $k\in [m]$ and $\ell \in \N$, the expectation of $\bs_k^{\ell}$ is zero if $\ell$ is odd, and
\begin{align*}
    \frac{1}{(2\ell)!} \Ex_{\bs_k\sim \calN(0, q)}\left[ \bs_k^{2\ell} \right] = \dfrac{q^{\ell}}{2^{\ell} \ell!}.
\end{align*}
\end{claim}

\begin{proof}
    The fact that odd moments are zero follows from the fact $\calN(0, q)$ is symmetric. Then, recall that the $2\ell$-th moment of the Gaussian distribution with variance $q$ is exactly $q^{\ell} (2\ell-1)!!$,    where the double-factorial $n!!$ is the product of all numbers from $1$ to $n$ which share the same parity as $n$. Evaluating $(2\ell-1)!! / (2\ell)!$, we obtain $1/(2\ell)!!$, which is $2 \cdot 4 \cdot 6 \cdot \dots \cdot 2\ell = \ell! \cdot 2^{\ell}$.
\end{proof}

We use the following claim to bound $|A'|$ and $|B'|$, where we note that $m\le n$:

\begin{claim}\label{cl:good-s}
With probability $1-o_n(1)$, 
$\bs_1,\ldots,\bs_m\sim \calN(0,q)$ satisfy the following two conditions:
\begin{itemize}
    \item For every odd $\ell \geq 1$, we have
\begin{align*}
\left| \sum_{k=1}^m \bs_k^{\ell} \right| \leq \left(\sqrt{n} \cdot q^{\ell / 2} \cdot \sqrt{(2\ell - 1)!!}\right) \cdot (\log n)^{\ell};
\end{align*}
\item For every $\ell \geq 1$, we have
\begin{align*}
    \left|\frac{1}{(2\ell)!} \sum_{k=1}^{m} \bs_k^{2\ell} - \frac{mq^{\ell}}{2^{\ell} \ell!} \right| \leq \left(\sqrt{n} \cdot q^{\ell} \cdot 2^{\ell} \right)\cdot (\log n)^{\ell}.
\end{align*}
\end{itemize}
\end{claim}

\begin{proof}
Since $\bs_k\sim \calN(0,q)$, each sum of $\bs_k^{\ell}$ has expectation $0$ by Claim~\ref{cl:exp}. We apply Chebyshev's inequality, and we have
\begin{align*}
\Prx_{\bs_1, \dots, \bs_m \sim \calN(0, q)}\left[ \left| \sum_{k=1}^m \bs_k^{\ell} \right| \geq v \right] \leq \frac{n}{v^2}  \cdot \Ex_{\bs \sim \calN(0, q)}\left[ \bs^{2\ell} \right] = \frac{n \cdot q^{\ell}}{v^2} \cdot (2\ell - 1)!! \leq \frac{1}{(\log n)^{2\ell}},
\end{align*}
for the setting of $$v = \left(\sqrt{n} \cdot q^{\ell/2} \cdot \sqrt{(2\ell - 1)!!}\right) \cdot (\log n)^{\ell}.$$ 

Now for the second item,
  consider any $\ell \in \N$, and note that by Claim~\ref{cl:exp}, the expectation of the expression is also zero, so we apply Chebyshev's inequality once more. We have
\begin{align*}
    \Prx_{\bs_1,\dots, \bs_m \sim \calN(0,q)}\left[ \left|\frac{1}{(2\ell)!} \sum_{k=1}^m \bs_k^{2\ell} - \frac{mq^{\ell}}{2^{\ell} \ell!}\right|  \geq v \right] &\leq \frac{n}{v^2} \cdot \Ex_{\bs \sim \calN(0,q)}\left[\left(\frac{\bs^{2\ell}}{(2\ell)!} - \frac{q^{\ell}}{2^{\ell}\ell!} \right)^2\right] \\[0.3ex]
        &\leq \frac{n}{v^2} \cdot \left(\frac{1}{(2\ell)!}\right)^2 \Ex_{\bs\sim\calN(0,q)}\left[\bs^{4\ell} \right] \\[0.6ex]
        &\leq \frac{n}{v^2} \cdot \left(\frac{1}{(2\ell)!}\right)^2 q^{2\ell} \cdot (4\ell-1)!! \leq \frac{nq^{2\ell}}{v^2} \cdot 2^{2\ell} \leq \frac{1}{(\log n)^{2\ell}},
\end{align*}
for the setting of 
$$
v=\left(\sqrt{n} \cdot q^{\ell} \cdot 2^{\ell} \right)\cdot (\log n)^{\ell}.
$$
The final simplification comes from the fact the square of $(2\ell)!$ multiplies every integer between $1$ and $2\ell$ twice, and $(4\ell - 1)!!$ multiplies every odd integer between $1$ and $4\ell-1$ (a total of at most $2\ell$ of them), which is at most twice an integer in multiplied in $(2\ell)!$; a very loose bound gives $(4
\ell-1)!! \leq ((2\ell)!)^2 \cdot 2^{2\ell}$.
%Setting $v = \left(\sqrt{n} \cdot q^{\ell} \cdot 2^{\ell} \right) \cdot (\log n)^{\ell}$. 
By a union bound over all terms, the probability that some inequality fails to be satisfied is at most $\sum_{\ell \in \N} 2 / (\log n)^{2\ell} = o_n(1)$.
\end{proof}

Consider any setting of $s_1,\dots, s_m$ where the event of Claim~\ref{cl:good-s} is satisfied, then
\begin{align*}
    |A'|  &\leq \frac{\xi}{\sqrt{n}} \sum_{\ell \text{ odd}} \frac{1}{\ell!}\left(\frac{\eps}{n^{1/4}} \right)^{\ell} \left(\sqrt{n} \cdot q^{\ell/2} \sqrt{(2\ell-1)!!} \right) \cdot (\log n)^{\ell} \leq  \sum_{\ell \text{ odd}} \left(\frac{2\eps \sqrt{q} \cdot \log n}{n^{1/4}} \right)^{\ell}\quad \text{and} \\[0.8ex]
    |B'| &\leq \frac{\xi}{\sqrt{n}} \sum_{\ell=1}^{\infty} \left(\frac{\eps^2}{\sqrt{n}}\right)^{\ell} \cdot \left( \sqrt{n} \cdot q^{\ell} \cdot 2^{\ell} \right) \cdot (\log n)^{\ell} \leq \sum_{\ell=1}^{\infty} \left(\frac{2\eps^2 q\cdot  \log n}{\sqrt{n}} \right)^{\ell},
\end{align*}
and both are $o_n(1)$ when $q$ is a sufficiently small constant smaller than $\sqrt{n} / (\eps^2 \log^2 n)$. 
\end{proof}
\end{comment}



\bibliographystyle{alpha}
\bibliography{waingarten,monotonicity-full}

\end{document}
