\documentclass[11pt]{article}

\def\withcolors{0}
\def\withnotes{1}


\usepackage{hyperref}
\hypersetup{
    unicode=false,          % non-Latin characters in Acrobat’s bookmarks
    colorlinks=true,        % false: boxed links; true: colored links
    linkcolor=blue,          % color of internal links (change box color with linkbordercolor)
    citecolor=darkgreen,        % color of links to bibliography
    filecolor=magenta,      % color of file links
    urlcolor=cyan           % color of external links
}

\usepackage{xspace}
\usepackage{graphicx}
\usepackage{waingarten}
\usepackage{paralist}
%\usepackage[linesnumbered,ruled]{algorithm2e}
\usepackage{thmtools}
\usepackage{thm-restate}
\usepackage{color}
\usepackage{setspace}


% Proof environment: redefine
\def\FullBox{\hbox{\vrule width 8pt height 8pt depth 0pt}}
\newcommand{\QED}{\;\;\;\FullBox}
\renewenvironment{proof}{\noindent{{\textbf{Proof:}~}}} {\hfill\QED}
\providecommand{\email}[1]{\href{mailto:#1}{\nolinkurl{#1}\xspace}}
\newenvironment{proofof}[1]{\noindent{\bf Proof of {#1}:~}}{\hfill\(\QED\)}


\def\FullBox{\hbox{\vrule width 8pt height 8pt depth 0pt}}

\newcommand{\new}[1]{{\color{purple} {#1}}} % new
\newcommand{\strikeout}[1]{{\color{gray} #1}}
\newcommand{\restrict}[1]{\upharpoonright_{#1}}

\newcommand{\CG}{\mathcal{G}\xspace}
\newcommand{\CV}{\mathcal{V}\xspace}
\newcommand{\CE}{\mathcal{E}\xspace}
\newcommand{\CA}{\mathcal{A}\xspace}
\newcommand{\CF}{\mathcal{F}\xspace}
\newcommand{\CR}{\mathcal{R}\xspace}
\newcommand{\CB}{\mathcal{B}\xspace}
\newcommand{\CX}{\mathcal{X}\xspace}
\newcommand{\CK}{\mathcal{K}\xspace}
\newcommand{\CM}{\mathcal{M}\xspace}
\newcommand{\CC}{\mathcal{C}\xspace}
\newcommand{\CL}{\mathcal{L}\xspace}
\newcommand{\CI}{\mathcal{I}\xspace}
\newcommand{\CQ}{\mathcal{Q}\xspace}
\newcommand{\CO}{\mathcal{O}\xspace}
\newcommand{\CP}{\mathcal{P}\xspace}
\newcommand{\CS}{\mathcal{S}\xspace}
\newcommand{\CT}{\mathcal{T}\xspace}
\newcommand{\CJ}{\mathcal{J}\xspace}
\usepackage[para]{footmisc}
\usepackage{subfig}
% \usepackage{subcaption}
% \usepackage{array}
% \usepackage{colortbl}




\title{Monotonicity Testing of High-Dimensional Distributions with Subcube Conditioning\footnote{This work was partially completed while authors were visiting the Simons Institute for the Theory of Computing at UC Berkeley. DC is supported by National Science Foundation (NSF) under grants CCF-2041920, CCF-2402571.
XC is supported   by NSF grants CCF-2106429, CCF-2107187. CS is supported by NSF grants CCF-1740850, CCF-1839317, CCF-2402572, and DMS-2023495.
EW is supported by NSF grant  CCF-2337993.}}

%%
%% The "author" command and its associated commands are used to define
%% the authors and their affiliations.
%% Of note is the shared affiliation of the first two authors, and the
%% "authornote" and "authornotemark" commands
%% used to denote shared contribution to the research.
\author{
Deeparnab Chakrabarty\\
Dartmouth College\\
{\tt deeparnab@dartmouth.edu}
\and
Xi Chen\\
Columbia University\\
{\tt xichen@cs.columbia.edu}
\and
Simeon Ristic\\
University of Pennsylvania\\
{\tt sristic@seas.upenn.edu}
\and
C. Seshadhri\\
University of California, Santa Cruz\\
{\tt sesh@ucsc.edu}
\and
Erik Waingarten \\
University of Pennsylvania\\
{\tt ewaingar@seas.upenn.edu}
}





    %\author {
      %Deeparnab Chakrabarty\thanks{Dartmouth College}
      %\and
      %Xi Chen\thanks{Columbia University}
      %\and
      %Simeon Ristic\thanks{}
      %\and
      %C. Seshadhri\thanks{University of California, Santa Cruz}
      %\and
      %Erik Waingarten\thanks{University of Pennsylvania.}
   % }
   \date{}
    %
\begin{document}
\maketitle

\begin{abstract}
\ignore{Distribution testing has led to a plethora of results, but for most interesting properties, the query\xnote{Should this be sample complexity?} complexity
is polynomial in the domain size. For high-dimensional domains, such complexities are infeasible.
The subcube conditional model (Canonne-Ron-Servedio, SICOMP 2015 and Bhattacharyya-Chakraborty, TOCT 2018)
allows for sampling subcubes of the domain, often leading to queries polynomial in the \emph{dimension}.

We study the classic property of monotonicity of distributions over $\{0,1\}^n$, under the subcube conditional model.
Consider an unknown input distribution $p$ over $\{0,1\}^n$. A distribution is monotone if 
$p(x) \leq p(y)$ for any $x \preceq y$, where $\preceq$ denotes the standard coordinate-wise partial order over $\{0,1\}^n$.
The aim is to distinguish monotone $p$ from $p$ being $\eps$-far (in TV distance) from monotone.
Up to logarithmic factors, we resolve the query complexity to be $\Theta(n/\eps^2)$.

For the upper bound, we take inspiration from the literature of monotonicity testing of Boolean functions.
We run an ``edge tester" for testing distribution monotonicity. To bound the query complexity, we
rely on directed isoperimetric theorems that relate the "directed boundary" of functions
to their distance to monotonicity. We prove a real-valued analogue of the directed Talagrand theorem (Khot-Minzer-Safra, SICOMP 2018),
which is a key tool in our proof.
For the lower bound, we reduce to dealing with product distributions under the standard sampling model. 
We develop lower bounds for determining the bias of coordinates, which are then applied to distributional monotonicity testing.

Applying our methods, we also prove an $\tilde{\Omega}(\sqrt{n}/\eps^2)$ lower bound for uniformity testing
with subcube conditional queries, under the promise that the distribution is monotone. This matches the 
upper bound up to logarithmic factors, which holds for arbitrary distributions. Thus, it proves that monotonicity
does not help for uniformity testing with subcube conditional queries. }

We study monotonicity testing of high-dimensional distributions on $\{-1,1\}^n$ in the model of subcube conditioning, suggested and studied by Canonne, Ron, and Servedio~\cite{CRS15} and Bhattacharyya and Chakraborty~\cite{BC18}. Previous work shows that the \emph{sample complexity} 
of monotonicity testing must be exponential in $n$ %is at most $2^{n - \Omega_{\eps}(n^{1/5})}$ 
(Rubinfeld, Vasilian~\cite{RV20}, and %at least $2^{n(1 - O(\sqrt{\eps}-o(1))}$ (
Aliakbarpour, Gouleakis, Peebles, Rubinfeld, Yodpinyanee~\cite{AGPRY19}). We show that the subcube \emph{query complexity} is $\tilde{\Theta}(n/\eps^2)$,
by proving nearly matching upper and lower bounds. Our work is the first to use directed isoperimetric inequalities (developed for function monotonicity testing) for analyzing a distribution testing algorithm. Along the way, we generalize an inequality of Khot, Minzer, and Safra~\cite{KMS18} to real-valued functions on $\{-1,1\}^n$.

We also study uniformity testing of distributions that are promised to be monotone, a problem introduced by Rubinfeld, Servedio~\cite{RS09}%(Rubinfeld and Servedio, Rand. Struc. and Alg. 2009)
, using subcube conditioning. We show that the query complexity is $\tilde{\Theta}(\sqrt{n}/\eps^2)$. Our work proves the lower bound, which matches (up to poly-logarithmic factors) the uniformity testing upper bound for general distributions (Canonne, Chen, Kamath, Levi, Waingarten~\cite{CCKLW21}).
Hence, we show that monotonicity does not help, beyond logarithmic factors, in testing uniformity of distributions with subcube conditional queries.

\end{abstract}
\thispagestyle{empty}
\newpage
\begin{spacing}{0.75}
\tableofcontents
\end{spacing}
\thispagestyle{empty}


\newpage


\pagenumbering{arabic}
\setcounter{page}{1}


\section{Introduction}
\label{sec:intro}

\begin{figure*}[tb]
    \centering
    \includegraphics[width=0.848\linewidth]{figs/circuitnn.pdf} 
    \caption{Illustration of differentiable CircuitNN. CircuitNN is designed based on differentiable NAND gates. After DAS is guided by PI and PO pairs of the truth table, CircuitNN can get the precise circuit architecture logic equivalent to the truth table.}
    \label{fig:circuitnn}
\end{figure*}

% 1. Describe the importance of logic synthesis
% 2. Existing Problems
% (a) Neural Architecture Search: Unstable, Predefined Setting, etc.
% (b) Circuit Generation: Probabilistic Model, Logic Equivalence

With the rapid advancement of technology, the scale of integrated circuits (ICs) has expanded exponentially. 
This expansion has introduced significant challenges in chip manufacturing, particularly concerning power and area metrics.
A primary objective in IC design is achieving the same circuit function with fewer transistors, thereby reducing power usage and area occupancy.

Logic synthesis~\cite{hachtel2005logicsynth}, a critical step in electronic design automation (EDA), transforms behavioral-level circuit designs into optimized gate-level circuits, ultimately yielding the final IC layout. 
The primary goal of logic synthesis is to identify the physical implementation with the fewest gates for a given circuit function. 
This task constitutes a challenging NP-hard combinatorial optimization problem. 
Current logic synthesis tools~\cite{brayton2010abc, wolf2013yosys} rely on human-designed heuristics, often leading to sub-optimal outcomes.

Differentiable architecture search (DAS) techniques~\cite{liu2018darts, chu2020darts} offer novel perspectives on addressing challenges in this problem.
Circuit functions can be represented through truth tables, which map binary inputs to their corresponding outputs. 
Truth tables provide a precise representation of input-output relationships, ensuring the design of functionally equivalent circuits.
Inspired by this, researchers~\cite{deepmind2024ai4sys, wang2024tnet} have begun exploring the application of DAS to synthesize circuits directly from truth tables.
Specifically, \citet{deepmind2024ai4sys} proposed CircuitNN, a framework that learns differentiable connection structures with logic gates, enabling the automatic generation of logic circuits from truth tables.
This approach significantly reduces the complexity of traditional circuit generation. 
Building on this, \citet{wang2024tnet} introduced T-Net, a triangle-shaped variant of CircuitNN, incorporating regularization techniques to enhance the efficiency of DAS.

Despite these advancements, several challenges remain. 
The computational complexity of DAS grows quadratically with the number of gates, posing scalability issues.
Although triangle-shaped architecture~\cite{wang2024tnet} partially mitigates this problem, redundancy persists. 
%Additionally, DAS is susceptible to converging to local optima, limiting the ability to search architectures that satisfy the given truth tables~\cite{liu2018darts}. 
%Furthermore, hyperparameters (network depth and layer width) require extensive searches, introducing complexity and prolonging the synthesis process. 
Additionally, DAS is susceptible to converging to local optima~\cite{liu2018darts} and hyperparameters (network depth and layer width) require extensive searches. 
The challenges arise from the vast search space in DAS. 
% Even with predefined settings for CircuitNN, finding a configuration that meets the truth table requires extensive trial and error during the DAS process. 
Intuitively, limiting the search space through predefined parameters (network depth, gates per layer, and connection probabilities) can significantly reduce the complexity.

Recent advances~\cite{openai2023gpt4, abramson2024alphafold3, esser2024sd3, li2024mar} in conditional generative models have demonstrated remarkable performance across language, vision, and graph generation tasks. 
Motivated by these developments, we propose a novel approach to circuit generation that generates preliminary circuit structures to guide DAS in generating refined circuits matching specified truth tables. 
Firstly, we introduce CircuitVQ, a tokenizer with a discrete codebook for circuit tokenization. 
Built upon our Circuit AutoEncoder framework~\cite{hou2022graphmae,li2023maskgae,wu2025mgvga}, CircuitVQ is trained through a circuit reconstruction task. 
Specifically, the CircuitVQ encoder encodes input circuits into discrete tokens using a learnable codebook, while the decoder reconstructs the circuit adjacency matrix based on these tokens.
Subsequently, the CircuitVQ encoder serves as a circuit tokenizer for CircuitAR pretraining, which employs a masked autoregressive modeling paradigm~\cite{chang2022maskgit, li2023mage}. 
In this process, the discrete codes function as supervision signals. 
After training, CircuitAR can generate discrete tokens progressively, which can be decoded into initial circuit structures by the decoder of the CircuitVQ. 
These prior insights can guide DAS in producing refined circuits that match the target truth tables precisely.

Our key contributions can be summarized as follows:
\begin{itemize}
\item We introduce CircuitVQ, a circuit tokenizer that facilitates graph autoregressive modeling for circuit generation, based on our Circuit AutoEncoder framework;
\item Develop CircuitAR, a model trained using masked autoregressive modeling, which generates initial circuit structures conditioned on given truth tables;
\item Propose a refinement framework that integrates differentiable architecture search to produce functionally equivalent circuits guided by target truth tables;
\item Comprehensive experiments demonstrating the scalability and capability emergence of our CircuitAR and the superior performance of the proposed circuit generation approach.
\end{itemize}

% Motivation
% (a) Diffusion (Vision, Graph), Autoregressive (Language, Vision)
% (b) Circuit Generation for Predefined Setting
% (c) Neural Architecture Search for Strict Logic Equivalence

% Contribution
% (a) Circuit Tokenizer (new transformer arch, training strategy)
% (b) CircuitAR (train and gen strategies, post-ar strategy)
% (c) Extensive Evaluation including BitD (Bit Distance) for Scalability

% !TEX root = main.tex

\section{Testing Monotonicity}

In this section, we show that using the directed and real-valued version of Talagrand's inequality, we may design an ``edge tester'' for testing monotonicity of distributions using subcube conditioning. In particular, we give the following theorem.
\begin{theorem}\label{thm:mon-ub}
    There exists an algorithm that receives as input subcube conditioning access to an unknown distribution $p$ supported on $\{-1,1\}^n$, as well as an accuracy parameter $\eps$. The algorithm makes $\tilde{O}(n/\eps^2)$ subcube conditioning queries and satisfies the following guarantees:
    \begin{itemize}
        \item If $p$ is monotone, the algorithm outputs ``accept'' with probability at least $0.9$.
        \item If $p$ is $\eps$-far from monotone, the algorithm outputs ``reject'' with probability at least $0.9$.
    \end{itemize}
\end{theorem}

The algorithm referred to in Theorem~\ref{thm:mon-ub} is given in Figure~\ref{fig:alg}.
% where we take the variable $c_0$ to be a small enough constant factor of $1/\sqrt{\log n}$. 
We break up the proof of \Thm{mon-ub}
into a few parts. First, we argue about the running time.

\begin{figure}
\begin{framed}

\textbf{Algorithm for Testing Monotonicity of Distributions}. We receive as input subcube conditioning access to an unknown distribution $p$ which is supported on $\{-1,1\}^n$. Furthermore, we receive the accuracy parameter $\eps \in (0,1)$. We let $c_0$ denote a sufficiently small constant. 
%     The algorithm proceeds in the following manner: 
\begin{enumerate}
    \item For all integers $w\geq0$ such that $2^w = \tilde{O}(n/\eps^2)$, repeat the following $t = O(2^{w} \log(n/\eps))$ times:
    \begin{itemize}
        \item Sample $\bx \sim p$ and $\bi \sim [n]$, and consider the restriction $\brho \in \{-1,1,*\}^n$ given by $\brho_j = \bx_j$ if $j \neq \bi$, and $\brho_{\bi} = *$. 
        \item Let $\eta = c_0^2\eps^2 \cdot 2^{w} / (16n \cdot \log(n/\eps) \cdot \log n)$ and take $m = O(\log (n/\eps)/\eta)$ subcube conditioning queries with restriction $\brho$ while counting the number of $1$'s and $-1$'s in coordinate $\bi$ observed. If the number of $-1$'s observed is larger than $m\left(1/2 + \sqrt{\eta}/2\right)$, output ``reject.''
    \end{itemize}
    \item If the algorithm has not rejected, output ``accept.''
\end{enumerate}

\end{framed}
\caption{Algorithm for Testing Monotonicity of Distributions} \label{fig:alg}
\end{figure}

\begin{claim}
    The query complexity is $\tilde{O}(n / \eps^2)$.
\end{claim}
\begin{proof}
    We simply upper bound the query complexity by inspecting Figure~\ref{fig:alg}. We have that (disregarding constant factors) the query complexity is the sum over all integers $w \geq 0$ such that $2^{w} = \tilde{O}(n/\eps^2)$ of 
    \[ O(2^{w} \cdot \log(n/\eps)) \cdot O\left(\dfrac{n\cdot \log^2(n/\eps) \cdot \log n}{c_0^2 \eps^2 \cdot 2^{w}} \right) = \tilde{O}(n / \eps^2). \]
    There are $O(\log(n/\eps))$ such settings of $w$, so the total complexity is still $\tilde{O}(n / \eps^2)$.
\end{proof}

\begin{lemma}
    Whenever $p$ is monotone, the algorithm outputs ``accept'' with probability at least $0.9$.
\end{lemma}

\begin{proof}
    Note that if $p$ is monotone, then for any restriction $\rho \in \{-1, 1, *\}^n$, which has one coordinate $i$ with $\rho_i = *$, a sample $\by$ from $p_{|\rho}$ must have the probability that $\by_i$ is 1 is at least the probability that it is $-1$. A standard Hoeffding bound implies that if one takes $m = O(\log(n/\eps)/\eta)$ samples of some event which is more likely to be $1$ than $-1$, the probability that the number of $1$'s observed is smaller than $m (1/2 - \sqrt{\eta}/2)$ is smaller than $\poly(\eps / n)$, for an arbitrarily large polynomial. Note that the number of times we may wrongfully reject is at most the query complexity, which is at most $\tilde{O}(n/\eps^2)$. So we may union bound as desired.
\end{proof}

\begin{lemma}\label{lem:far-case-reject}
    Whenever $p$ is $\eps$-far from monotone, the algorithm outputs ``reject'' with probability at least $0.9$.
\end{lemma}

\begin{proof}
    We show that whenever $p$ is $\eps$-far from monotone, there exists some $\gamma \in \{0, \dots, h\}$ with $h = O(\log(n/\eps))$ and a setting of $\ell \in \{ 0, \dots, r\}$ where $r = O(\log(n/\eps))$ which satisfies $2^{2\gamma + r +1} = \tilde{O}(n / \eps^2)$ and
    \begin{align*}
        \Prx_{ \substack{\bx \sim p \\ \bi \sim [n]}}\left[\left(\dfrac{(p(\bx^{(\bi\to-1)}) - p(\bx^{(\bi\to1)}))^+}{p(\bx^{(\bi\to-1)}) + p(\bx^{(\bi\to1)})}\right)^2 \geq \eta\right] \geq \frac{1}{r \cdot 2^{\gamma + \ell}}.
    \end{align*}
    for $\eta = c_0^2 \eps^2 \cdot 2^{2\gamma+\ell} / (16 h \cdot n \cdot \log n)$. When the algorithm iterates over all $w \geq 0$ such that $2^w = \tilde{O}(n/\eps^2)$, it will eventually consider $w = 2\gamma + \ell$.
    This implies that except with probability $0.01$, one of the $t$ samples $\bx \sim p$ and $\bi \sim [n]$ satisfy the above bound, since we repeat $t = O(2^{w}\log(n/\eps))$ times and $2^{w} = 2^{2\gamma+\ell}$ is larger than $2^{\gamma+\ell}$. Once that is set, with probability except $0.01$, the algorithm outputs reject; the subcube conditioning query $\brho$ is exactly sampling from $\{-1,1\}$ whose probability of being $-1$ is at least $1/2 + \sqrt{\eta}$. By a Hoeffding bound, the probability that the number of $-1$'s is smaller than $m(1/2 + \sqrt{\eta}/2)$ is at most $0.01$. From Corollary~\ref{cor:l1-tal}, for 
    small enough constant $c_0$. the fact that $p$ is $\eps$-far from monotone implies,
    \begin{align*}
        \frac{c_0\eps}{\sqrt{\log n}} &\leq \sum_{x \in \{-1,1\}^n} \sqrt{\sum_{i:x_i = -1} \left( \left(p(x^{(i\to-1)}) - p(x^{(i\to1)} \right)^+ \right)^2} \\
        &= \Ex_{\bx \sim p}\left[ \sqrt{\sum_{i:\bx_i=-1} \left(\dfrac{(p(\bx^{(i\to-1)}) - p(\bx^{(i\to1)}))^+}{p(\bx^{(i\to-1)})} \right)^2} \right].
    \end{align*}
    Furthermore, $p(\bx^{(i\to-1)}) - p(\bx^{(i\to1)}) \geq 0$ implies $p(\bx^{(i\to-1)}) + p(\bx^{(i\to 1)}) \leq 2 p(\bx^{(i\to-1)})$. So we may lower bound
    \begin{align}
        \frac{c_0 \eps}{4\sqrt{\log n}} \leq \Ex_{\bx \sim p}\left[ \sqrt{\sum_{i:\bx_i=-1} \left(\dfrac{(p(\bx^{(i\to-1)}) - p(\bx^{(i\to1)}))^+}{p(\bx^{(i\to-1)}) + p(\bx^{(i\to1)})} \right)^2} \right] \label{eq:exp1}
    \end{align}
    Notice that the maximum quantity within the expectation in (\ref{eq:exp1}) is $\sqrt{n}$, since each of the terms being added is between $0$ and $1$. Therefore, there must exist some $\gamma \in \{0,\dots, h \}$ with $h = \lceil \log_2(4\sqrt{n}/(c_0\eps))\rceil + 1 = O(\log(n/\eps))$ which satisfies
    \begin{align}
     \Prx_{\bx \sim p}\left[\sum_{i:\bx_i=-1} \left(\dfrac{(p(\bx^{(i\to-1)}) - p(\bx^{(i\to1)}))^+}{p(\bx^{(i\to-1)}) + p(\bx^{(i\to1)})} \right)^2 \geq \frac{c_0^2 \cdot \eps^2 \cdot 2^{2\gamma}}{16h\log n} \right] \geq \frac{1}{2^{\gamma}}. \label{eq:good-1}
    \end{align}
    Thus, consider any one of those values of $x$, and in order to simplify the notation, we define 
    \[ \xi \eqdef \frac{c_0^2 \cdot \eps^2 \cdot  2^{2\gamma}}{16h\log n} \qquad \nu_i = \left(\dfrac{(p(x^{(i\to-1)}) - p(x^{(i\to1)}))^+}{p(x^{(i\to-1)}) + p(x^{(i\to1)})} \right)^2,\]
    so that we assume to fix $x$ such that $\sum_{i=1}^n \nu_i \geq \xi$, and each $\nu_i \in [0, 1]$. Consider a partition of the coordinates of $[n]$ into groups $B_1, \dots, B_{r}$, such that $i \in B_{\ell}$ whenever the $i$-th coordinate contributes between $\xi 2^{\ell}/ n$ and $\xi 2^{\ell+1} / n$, and $r$ is chosen is the value $\xi 2^{r+1} / n$ is between $1$ and $2$ (note that, since $\nu_i \in [0, 1]$, $B_{r'}$ for $r' > r$ must be empty), so $r = O(\log(n/\xi))$. Then, there must be some $\ell$ with $|B_{\ell}| \geq n / (r \cdot 2^{\ell+1})$, and this implies
    \begin{align}
        \Prx_{\bi \sim [n]}\left[\left(\dfrac{(p(\bx^{(\bi\to-1)}) - p(\bx^{(\bi\to1)}))^+}{p(\bx^{(\bi\to-1)}) + p(\bx^{(\bi\to1)})} \right)^2 \geq \frac{\xi \cdot 2^{\ell}}{n}  \right] \geq \frac{1}{r \cdot 2^{\ell}}.\label{eq:good-2}
    \end{align}
    The desired bound then follows from the setting of $\eta$, and lower bounding the probability that $\bx \sim p$ satisfies the event of (\ref{eq:good-1}), and then $\bi \sim [n]$ satisfies the event of (\ref{eq:good-2}).
\end{proof}

\subsection{The Real-Valued Directed Talagrand Inequality} \label{sec:tal}

We will prove a ``directed isoperimetric theorem" for real-valued functions. This is an important
tool used for the analysis of the monotonicity tester. We define notions of the
directed boundary for Boolean functions. 

Let $f:\{-1,1\}^n \to [0,1]$ be a function defined on the $n$-dimensional hypercube. The $L_1$-distance 
of $f$ from monotonicity is defined as
\begin{equation*}
	\dist_1(f) \eqdef \min_{g~:~\text{monotone}} ~~\Ex_{\bx \sim \{-1,1\}^n}\left[ |f(\bx) - g(\bx)| \right]
\end{equation*}
where the expectation is over the uniform distribution over $\{-1,1\}^n$.
For a point $x\in \{-1,1\}^n$, define the directed derivative $\grad^-f(x)$ to be the $n$-dimensional vector defined as 
\begin{equation}\label{eq:defgrad}
	\left(\grad^-f(x)\right)_i \eqdef \begin{cases}
		0 & \textrm{if $x_i = 1$} \\
		\left(f(x) - f(x+2e_i)\right)^+ & \text{otherwise}
	\end{cases}
\end{equation}
where $(z)^+$ is a shorthand for $\max(z,0)$. For Boolean-valued $f:\{-1,1\}^n \to \{0,1\}$, the distance $\dist_1(f)$ corresponds to the ``normal'' Hamming distance notion, $\dist_0(f)$.
Based on isoperimetric theorems of Talagrand~\cite{Tal93}, the quantity $\Exp_{\bx} \norm{\grad^-f (\bx)}_2$ can be thought
of as a ``directed surface area" for the function $f$. A deep isoperimetric theorem of Khot, Minzer, and Safra~\cite{KMS18} (see, also~\cite{PRW22}, who showed how to remove the final logarithmic factor)
lower bounds this surface area by the distance to monotonicity.

\begin{theorem}[\cite{KMS18, PRW22}]\label{thm:booliso}
	There exists a universal constant $C > 0$ such that for every $f \colon \{-1,1\}^n \to \{0,1\}$,
$\Exp_{\bx} \norm{\grad^-f (\bx)}_2 \geq C\cdot \dist_0(f)$.
\end{theorem}

Theorem~\ref{thm:l1-talagrand} gives a real-valued generalization of the above theorem, with a $\sqrt{\log n}$ loss in the bound. The proof appears in Subsection~\ref{sec:proof-tal}, but we state the following corollary used in the tester's analysis.
\begin{corollary} \label{cor:l1-tal} Let $p$ be a distribution over $\{-1,1\}^n$ that is $\eps$-far
from monotone. Then 
\[
\sum_{x \in \{-1,1\}^n} \sqrt{\sum_{i:x_i = -1} \left( \left(p(x^{(i\to-1)}) - p(x^{(i\to1)} \right)^+ \right)^2} = \Omega\left(\frac{\eps}{\sqrt{\log n}}\right).
\]
\end{corollary}

\begin{proof} Let $\eps(p)$ be the distance of $p$ to monotonicity. Note that this is the distance
over distributions, while \Thm{l1-talagrand} refers to $L_1$-distance between arbitrary functions.
So we need an extra calculation to apply \Thm{l1-talagrand}.

Let $\cM$ be the set of monotone distributions. Then, $\eps(p) = \min_{q \in \cM} \dtv(p,q)
= \min_{q \in \cM} \|p-q\|_1/2$. On the other hand, $\dist_1(p) = \min_{g: \textrm{monotone}} \Exp_{\bx}|p(\bx) - g(\bx)|
= 2^{-n} \min_{g: \textrm{monotone}} \|p-g\|_1$. Note that the minimizer $g$ is non-negative, since $p$
is non-negative. Hence, the function $f = g/\|g\|_1$ is a distribution. 

By triangle inequality,
\begin{eqnarray*}
\eps(p) \leq \|p - f\|_1 \leq \|p-g\|_1 + \|f-g\|_1 = \|p-g\|_1 + \|g - g/\|g\|_1 \|_1
\end{eqnarray*}
Observe that $\|g - g/\|g\|_1 \|_1 = \sum_x |g(x) - g(x)/\|g\|_1| = |1 - 1/\|g\|_1| \cdot \sum_x |g(x)|
= | 1 - \|g\|_1|$. Since $p$ is a distribution, this expression is equal to $| \|p\|_1 - \|g\|_1|$.
And finally, $| \sum_x (|p(x)| - |g(x)|)| \leq \sum_x |p(x) - g(x)| = \|p-g\|_1$. 
Overall, we deduce that $\eps(p) \leq 2\|p-g\|_1$. Recall that $\dist_1(p)$ is defined
using an expectation over the domain, so $\eps(p) \leq 2 \cdot 2^n \dist_1(p)$. 

With our lower bound for $\dist_1(g)$, we can apply \Thm{l1-talagrand}. 
So $\Exp_{\bx} \norm{\grad^-p(\bx)} = \Omega(\dist_1(p)/\sqrt{\log n}) = \Omega(2^{-n} \eps(p)/\sqrt{\log n})$.
We expand out the expression for $\grad^-p(x)$ to wrap up the proof.
\begin{eqnarray*}
\Ex_{\bx\sim\{-1,1\}^n}\left[ \norm{\grad^-p(\bx)} \right] = 2^{-n} \sum_{x \in \{-1,1\}^n} \norm{\grad^-p(x)}
= 2^{-n}  \sum_{x \in \{-1,1\}^n} \sqrt{\sum_{i:x_i = -1} \left( (p(x) - p(x+2e_i) )^+ \right)^2}
\end{eqnarray*}
As argued above, this expression is lower bounded by $\Omega(2^{-n} \eps(p)/\sqrt{\log n})$.
The $2^{-n}$ terms ``cancel out", and noting that $\eps(p) \geq \eps$, we get the desired bound.
%$\sum_{x \in \{-1,1\}^n} \sqrt{\sum_{i:x_i = -1} \left( \left(p(x^{(i\to-1)}) - p(x^{(i\to1)} \right)^+ \right)^2} = \Omega(\eps/\sqrt{\log n})$.
\end{proof}


\subsection{The proof of \Thm{l1-talagrand}} \label{sec:proof-tal}

By a simple translation and rescaling argument, we reduce the function range to $[0,1]$.
This will make subsequent calculations easier.

\begin{claim} \label{clm:rescale} Consider $f:\{-1,1\}^n \to \R$. For positive $\alpha \in \R^+$ and
any $\beta \in \R$, define the function $\hat{f}$ where $\hat{f}(x) = \alpha f(x) + \beta$. Then,
$\EX_x[\|\nabla^- \hat{f}\|_2]/\dist_1(\hat{f}) =\EX_x[\|\nabla^- f\|_2]/\dist_1(f) $.
\end{claim}

\begin{proof} The monotonicity violations in $f$ and $\hat{f}$ are identical.
For any point $x$ and coordinate $i$, $ (\nabla^- \hat{f}(x))_i = \alpha (\nabla^- f(x))_i$.
Hence, $\EX_x[\|\nabla^- \hat{f}\|_2] = \alpha \EX_x[\|\nabla^- f\|_2]$.
For a function $g$, let $\alpha g + \beta$ be the function whose value at $x$
is $\alpha g(x) + \beta$.
\begin{align*}
 \dist_1(\hat{f}) = \min_{\hat{g}: \textrm{monotone}} \|\hat{f}-\hat{g}\|_1 &= \min_{\hat{g}: \textrm{monotone}} \| (\alpha f + \beta) - \hat{g}\|_1
\min_{\hat{g}: \textrm{monotone}} \|(\alpha f + \beta) - (\alpha (\alpha^{-1}(\hat{g} - \beta)) + \beta)\|
\end{align*}
Monotonicity is preserved by positive scaling and translation, so $\hat{g}$ is monotone iff $(\alpha^{-1}(\hat{g} - \beta))$
is monotone.
Hence,
\begin{align*}
 \dist_1(\hat{f}) = \min_{g: \textrm{monotone}} \|(\alpha f + \beta) - (\alpha g + \beta)\|_1
= \min_{g: \textrm{monotone}} \|\alpha f - \alpha g\|_1 = \alpha \ \dist_1(f)
\end{align*}
We conclude that  
$\EX_x[\nabla^- \hat{f}\|_2]/\dist_1(\hat{f}) =\EX_x[\|\nabla^- f\|_2]/\dist_1(f) $.
\end{proof}

Given $f$, we technically work with the function $\hat{f} = f/2M + 1/2$,
where $M = \max_x |f(x)|$. Observe that $\hat{f}$ has range in $[0,1]$,
and by \Clm{rescale}, the statement of \Thm{l1-talagrand} for $\hat{f}$ implies
the statement for $f$.

Abusing notation, we just assume that $f:\{-1,1\}^n \to [0,1]$.
We use the technique of Berman, Raskhodnikova, and Yaroslavtsev~\cite{BeRaYa14} of using threshold Boolean functions to relate the real-valued $f$ to Boolean functions. 

\noindent
Given $t\in [0,1]$ consider the following Boolean function (Definition 2.1 in~\cite{BeRaYa14}) $f_t : \{-1,1\}^n \to \{0,1\}$
\[
	f_t(x) = \begin{cases}
		1 & \text{if}~ f(x) \geq t; \\
		0 & \text{if}~ f(x) < t
	\end{cases}
\]
\noindent
It is easy to see that for any $x\in \{-1,1\}^n$,
\[
	f(x) = \int_0^{f(x)} dt  =  \int_0^1 f_t(x)~dt ~= \Ex_{\bt \sim [0,1]} \left[ f_{\bt}(x) \right]
\]
where the expectation is over $t$ uniformly distributed over $[0,1]$.
One can perform analogous calculations to relate the $L_1$ distance of (the real valued)
$f$ to the $L_0$ distance of (the Boolean) $f_t$s.  

\begin{lemma}[Lemma 2.1~\cite{BeRaYa14}]\label{lem:bry}
	\noindent
	For any function $f:\{-1,1\}^n \to [0,1]$, $\dist_1(f) = \int_0^1 \dist_0(f_t) dt = \Exp_{\bt} \left[ \dist_0(f_{\bt})\right]$.
\end{lemma}

The main work is in relating the (directed) gradients of $f$ to the corresponding
gradients of $f_t$. This is where we suffer a $\sqrt{\log n}$ loss.
 
% One can also similarly check that for any $x\in \{0,1\}^n$,
% we have 
% \begin{equation}\label{eq:exp-grad}
% 	\grad^-f(x) = \Exp_t \grad^- f_t(x)
% \end{equation}
% In particular, since $\grad^-f(x)$ and $\grad^-f_t(x)$ are non-negative vectors, by linearity of expectation we get
% \begin{equation}\label{eq:l1-norm}
% 	\norm{\grad^-f(x)}_1 = \norm{\Exp_t \grad^- f_t(x)}_1 = \Exp_t \norm{\grad^- f_t(x)}_1
% \end{equation}
% BRY provide the following characterization
% 
% Using~\Lem{bry} and \eqref{eq:exp-grad}, one immediately obtains the Poincare version of~\Thm{booliso}:
% \[
% 	\dist_1(f) =  \Exp_t  \dist_0(f_t)  \underbrace{\leq}_{\Thm{booliso}} \frac{1}{C}\cdot \Exp_t \Exp_x \norm{\grad^- f_t(x)}_1 \underbrace{=}_{\eqref{eq:l1-norm}} \frac{1}{C}\cdot \Exp_x \norm{\Exp_t \grad^- f_t(x)}_1 \underbrace{=}_{\eqref{eq:exp-grad}}\frac{1}{C}\cdot \Exp_x \norm{\grad^- f(x)}_1
% \]
% ndent
% 
% Talagrand's inequality doesn't {\em immediately} follow because (as usual) ``Jensen is in the wrong direction''. \eqref{eq:exp-grad} and the fact that the $\ell_2$ norm is a convex function implies that $\Exp_t \norm{\grad^- f_t(x)}_2 \geq \norm{\Exp_t \grad^- f_t(x)}_2 = \norm{\grad^- f(x)}_2$ which is counter to what would've been nice. Nevertheless one can prove the following lemma which, using the same derivation as above, proves~\Thm{l1-talagrand}.
% 

\begin{lemma}
	For all $x\in \{-1,1\}^n$, $\norm{\grad^- f(x)}_2 = \Omega(1/\sqrt{\log n}) \Exp_t \norm{\grad^- f_t(x)}_2 $.	
\end{lemma}
\begin{proof}
Fix any $x \in \{-1,1\}^n$. Let $y_1, \ldots, y_d \in \{-1,1\}^n$ denote the ``up''-neighbors of $x$ which satisfy $f(x) > f(y_j)$. In particular, there are at most $d \leq n$ points $y_1 ,\dots, y_d$ such that, for every $j \in [d]$, $y_j = x + 2e_i$ for some $i$, and in addition, $f(x) > f(y_j)$.
%(so there exists some coordinate $i$ such that $y_j = x + 2e_i$).
%Note that $d\leq n$.
Order the indices so as to assume $f(y_1) \leq f(y_2) \leq \cdots \leq f(y_d)$ and let $a_j := f(x) - f(y_j)$ (and so $a_1 \geq a_2 \geq \cdots \geq a_d$). By definition, we have defined $a_1,\dots, a_d$ to have $\norm{\grad^- f(x)}_2 = (\sum_{j=1}^d a^2_j)^{1/2}$.

For $t \in [0, 1]$, consider the function $f_t$, and let edge $(x,y_j)$ be called a violation in $f_t$ if $f_t(x) = 1$ and $f_t(y_j) = 0$.
Observe that only violated edges contribute to $\norm{\grad^-f_t(x)}$. 
Notice that for any $t \in (f(y_i), f(y_{i+1})]$, the edge $(x,y_j)$ is a violation in $f_t$ iff $j \leq i$.
Hence, if $t \in (f(y_i), f(y_{i+1})]$, then the vector $\grad^-f_t(x)$ 
has exactly $i$ non-zeros and $\norm{\grad^- f_t(x)}_2 = \sqrt{i}$. For $i < d$, the probability that $t \in (f(y_i), f(y_{i+1})]$
is exactly $y_{i+1} - y_i = a_i - a_{i+1}$. The probability that $t \in (f(y_d), x]$
is exactly $a_d$.
% 	
% Next note that $\grad^-f_t(x)$ looks as follows:
% 	\begin{equation}\label{eq:gorilla}
% 		\grad^-f_t(x) =  \begin{cases}
% 			(0,0,\ldots, 0, 0) & \text{if} ~ 0\leq t\leq f(y_1) \\
% 			(1,0,\ldots, 0, 0) & \text{if} ~ f(y_1) <  t\leq f(y_2) \\
% 			(1,1,\ldots, 0, 0) & \text{if} ~ f(y_2) <  t\leq f(y_3) \\
% 			\vdots & \\
% 			(1,1,\ldots, 1, 0) & \text{if} ~ f(y_{d-1}) <  t\leq f(y_d) \\
% 			(1,1,\ldots, 1, 1) & \text{if} ~ f(y_d) <  t\leq f(x) \\
% 			(0,0,\ldots, 0, 0) & \text{if} ~ f(x) <  t\leq 1 \\
% 		\end{cases}
% 	\end{equation}
Thus,
\[
	\Ex_{\bt\sim[0,1]}\left[ 	\norm{\grad^- f_{\bt}(x)}_2 \right] = \sum_{i=1}^{d-1} \left(a_i - a_{i+1}\right)\sqrt{i} + a_d \sqrt{d} = \sum_{i=1}^d a_i \cdot \left(\sqrt{i} - \sqrt{i-1}\right)
\]
By Cauchy-Schwarz and the following calculation, we complete the proof
\begin{equation*}
	\Ex_{\bt\sim [0,1]}\left[ 	\norm{\grad^- f_{\bt}(x)}_2 \right]\leq \norm{\grad^- f(x)}_2 \cdot \sqrt{\sum_{i=1}^d \left(\sqrt{i} - \sqrt{i-1}\right)^2}\leq O(\sqrt{\log n})\cdot \norm{\grad^- f(x)}_2
\end{equation*}
since $\sqrt{i} - \sqrt{i-1} = 1/(\sqrt{i}+\sqrt{i-1}) \leq 1/\sqrt{i}$, and so $\sum_{i \leq d} (\sqrt{i} - \sqrt{i-1})^2 \leq \sum_{i \leq d} 1/i = O(\log d)$.
% 
% To see the above, note that
% \[
% (\sqrt{i}-\sqrt{i-1})^2 = (2i - 1) - 2\sqrt{i(i-1)} = 2i\cdot \left(1 - \left(1 - \frac{1}{i}\right)^{1/2}\right) - 1
% \]
% By Taylor approximation, we get that there exists constant $A > 0$ (indeed, $A = 1$ suffices) such that
% \[
% \left(1 - \frac{1}{i}\right)^{1/2} \geq 1 - \frac{1}{2i} - \frac{A}{i^2}
% \]
% \noindent
% and so substituting above we get that
% \[
% \sum_{i=1}^d \left(\sqrt{i} - \sqrt{i-1}\right)^2 \leq \sum_{i=1}^d \frac{2A}{i} = O(\log d)\qedhere
% \]
\end{proof}

We now complete the proof of \Thm{l1-talagrand}. By the above lemma,
\begin{align*}
\Ex_{\bx\sim\{-1,1\}^n}\left[ \norm{\grad^- f(\bx)}_2\right] &= \Omega(1/\sqrt{\log n}) \Ex_{\substack{\bx \sim \{-1,1\}^n \\ \bt \sim [0,1]}} \left[ \norm{\grad^- f_{\bt}(\bx)}_2\right] \\
    &= \Omega(1/\sqrt{\log n}) \Ex_{\bt \sim [0,1]} \left[  \Ex_{\bx\sim\{-1,1\}^n}\left[ \norm{\grad^- f_{\bt}(\bx)}_2 \right] \right].
\end{align*}
By the directed Boolean isoperimetric statement of \Thm{booliso}, $\Exp_{\bx} \norm{\grad^- f_t(\bx)}_2 = \Omega(\dist_0(f_t))$.
We apply this bound and then \Lem{bry} to relate back to $f$.
\begin{align*}
\Ex_{\bx\sim\{-1,1\}^n}\left[ \norm{\grad^- f(\bx)}_2 \right] = \Omega(1/\sqrt{\log n}) \Ex_{\bt\sim[0,1]} \left[\dist_0(f_{\bt})\right] = \Omega(1/\sqrt{\log n}) \cdot \dist_1(f)
\end{align*}

\section{Statistical efficiency of CARROT} 
\label{sec:lower-bound}

In this section we establish that, under certain conditions, the plug-in approach to routing is minimax optimal. To show this, we follow two steps:
\begin{itemize}
    \item First we establish an information theoretic lower bound on the sample complexity for learning the oracle routers (\cf\ Theorem \ref{thm:lower-bound}). 
    \item Next, establish an upper bound for the minimax risk of plug-in routers (\cf\ Theorem \ref{thm:upper-bound}). We show that under sufficient conditions on the estimates of $\Ex[Y\mid X]$ the sample complexity in the upper bound matches the lower bound. Together, they imply the statistical efficiency of the plug-in approach.  
    % We also suggest an estimate for $\Ex[Y\mid X]$ that meets the needed conditions for CARROT to be rate optimal.  
\end{itemize} 


%For our minimax analysis we begin with some notation. For the probability distribution $P $ defined on the space $\cX \times \reals^{M \times K}$, we denote the marginal distribution of $X$ by $P_X$. Let us denote $\supp(\cdot)$ as the support of a probability distribution. Within the space $\reals^d$, we denote $\Lambda_d$ as the Lebesgue measure, $\|\cdot\|_2$ and $\|\cdot\|_\infty$ as the $\ell_2$ and $\ell_\infty$-norms, and $\cB(x, r, \ell_2)$ and $\cB(x, r, \ell_\infty)$ as closed balls of radius $r$ and centered at $x$ with respect to the $\ell_2$ and $\ell_\infty$-norms. 

% We denote $\Ex_P[\ell \{ Y, f_m(X)\}\mid X]$ as $\Phi^\star_m(X)$. Then, following eq. \eqref{eq:reg-decomposition} the regression function $\eta_{\lambda, m}^\star(X) = \Ex_P[\eta_\lambda(X, Y)]$ has the decomposition $\eta_{\lambda, m}^\star(X) = \lambda \Phi^\star_m(X) + (1 - \lambda) \kappa_m(X)$. In the following lemma, we provide a formulation of oracle routers using this decomposition, which will be useful for developing their computationally efficient estimates. 
% \begin{lemma} \label{lemma:oracle-router}
%     For any $0 \le \lambda \le 1$  the oracle router $g_\lambda^\star$ that minimizes the loss $\cL_P(g, \lambda)$ is 
%     \begin{equation} \label{eq:oracle-router-2}
%         \textstyle g_\lambda^\star(X) = \argmin_m ~ \eta_{\lambda, m} ^\star(X) = \argmin_m ~ \{ \lambda \Phi^\star_m(X) + (1 - \lambda) \kappa_m(X)\}\,.
%     \end{equation}
% \end{lemma}

We begin with a notational convention for $g_\mu^\star(X)$. If the minimum is attained at multiple $m$'s, we consider $g_\mu^\star(X)$ as a subset of $[M]$. On the contrary, if the minimum is uniquely attained, then $g_\mu^\star(X)$ refers to both the index $m_X$ where the minimum is attained and the singleton set $\{m_X\} \subset [M]$. The distinction should be clear from the context.

We also generalize slightly to the setting where the last $K_2$ metrics are known functions of $X$, \ie\ for $m \in [M], k \in \{K - K_2 +1 , \dots K\}$ there exist known functions $f_{m, k}: \cX \to \reals$ such that $[Y]_{m, k} = f_{m, k}(X)$. Since $\Ex[[Y]_{m, k}\mid X] = f_{m, k}(X)$ are known for $k \ge K - K_2 +1 $ they don't need to be estimated. 
% We shall see the presence of known metrics has consequences for the sample complexity (\cf\ Remark \ref{remark:difficulty-routing}). We also define $K_1 = K - K_2$ as the number of known metrics.    

\subsection{Technical Assumptions}

%Let us discuss a notational convention for $g_\lambda^\star(X)$. The minimum can be attained at multiple $m$'s. In that case, $g_\lambda^\star(X) \subset [M]$. However, when the minimum is uniquely attained, the $g_\lambda^\star(X)$ refers to both the index $m_X$ where the minimum is attained and the singleton set $\{m_X\} \subset [M]$. The distinctions should be clear from the contexts.


%We also assume that the last $K_2$ many metrics are known functions of $X$, \ie\ for $m \in [M], k \in \{K - K_2 +1 , \dots K\}$ there exist known functions $f_{m, k}: \cX \to \reals$ such that $[Y]_{m, k} = f_{m, k}(X)$. Since $\Ex[[Y]_{m, k}\mid X] = f_{m, k}(X)$ are known for $k \ge K - K_2 +1 $ within the Algorithm \ref{alg:pareto-routers}, they don't need to be estimated. We shall see its consequence in the study sample complexity. Define $K_1 = K - K_2$.    



The technical assumptions of our minimax study are closely related to those in investigations of non-parametric binary classification problems with $0/1$ loss functions, \eg\  \citet{cai2019Transfer,kpotufe2018Marginal,maity2022minimax,audibert2007Fast}. In fact, our setting generalizes the classification settings considered in these papers on multiple fronts: (i) we allow for general loss functions, (ii) we allow for more than two classes, and (iii) we allow for multiple objectives. %So, before we describe the assumptions, 

To clarify this, we discuss how binary classification is a special case of our routing problem. %This connection will be later used for adapting the standard assumptions considered in these papers to our setting. 

\begin{example}[Binary classification with $0/1$-loss] \label{example:binary-classification}
    Consider a binary classification setting with $0/1$-loss: we have the pairs $(X, Z) \in \cX \times \{0, 1\}$ and we want to learn a classifier $h: \cX \to\{0, 1\} $ to predict $Z$ using $X$. This is a special case of our setting with $M = 2$ and $K= 1$, where for $m \in \{0, 1\}$ the $[Y]_{m, 1} = \bbI\{Z \neq m\}$. Then the risk for the classifier $h$, which can also be thought of as a router, is 
\begin{align*}
\textstyle \cR_P(h) & \textstyle = \Ex\big[\sum_{m \in \{0, 1\}}[Y]_{m, 1} \bbI\{h(X) = m\} \big]\\ 
& = \Ex\big[ \bbI\{h(X) \neq Z\} \big]\,,
\end{align*} the standard misclassification risk for binary classification. 
\end{example}

% \SM{Mention that for $\lambda = 0$ the oracle router is precisely known. Thus, we only focus on the cases of $\lambda > 0$.}

We assume that $\supp(P_X)$ is a compact set in $\reals^d$. This is a standard assumption in minimax investigations for non-parametric classification problems \citep{audibert2007Fast,cai2019Transfer,kpotufe2018Marginal,maity2022minimax}. 
Next,  we place H\"older smoothness conditions on the functions $\Phi_m^\star$. This controls the difficulty of their estimation. For a tuple $s = (s_1 , \dots, s_d) \in (\bN \cup \{0\})^d$ of $d$ non-negative integers  define $|s| = \sum_{j = 1}^d s_j$ and for a function $\phi: \reals^d\to \reals$ and $x = (x_1, \dots, x_d) \in \reals^d$ define the differential operator: 
\begin{equation}
  \textstyle  D_s(\phi, x) = \frac{\partial^{|s|}\phi(x)}{\partial x_1^{s_1} \dots \partial x_d^{s_d}}\,, 
\end{equation} assuming that such a derivative exists. Using this differential operator we now define the H\"older smoothness condition: 

\begin{definition}[H\"older smoothness]
   For $\beta, K_\beta >0$ we say that $\phi:\reals^d \to \reals$ is $(\beta, K_\beta)$-H\"older smooth on a set $ A \subset \reals^d$ if it is $\lfloor \beta \rfloor$-times continuously differentiable on $A$ and for any $x, y \in A $ 
   \begin{equation}
       |\phi(y) - \phi_x ^{(\lfloor \beta \rfloor)}(y)| \le K_\beta \|x - y\|_2^\beta\,,
   \end{equation} where 
$\phi_x ^{(\lfloor \beta \rfloor)}(y) = \sum_{|s| \le \lfloor \beta \rfloor} D_s(\phi, x) \{\prod_{j = 1}^d(y_j - x_j)^{s_j}\} $ is the $\lfloor \beta \rfloor$-order Taylor polynomial approximation of $\phi(y)$ around $x$. 
\end{definition}
With this definition, we assume the following:
\begin{assumption}\label{assmp:smooth}
    For $m \in [M]$ and $k \in [K_1]$ the 
    % functions $\kappa_m$ and 
    $[\Phi(X)]_{m, k}$ is $(\gamma_{k}, K_{\gamma, k})$-H\"older smooth. 
\end{assumption} 
%The H\"older smoothness assumption controls how well a non-parametric function can be estimated \citep{fan1997local}, where higher smoothness parameters lead to a smaller error in estimation. 
This smoothness parameter will appear in the sample complexity of our  plug-in router. Since the $[\Phi(X)]_{m, k}$ are known for $k \ge K_1 + 1$ we do not require any smoothness assumptions on them.

% Note that the assumption implies for any $\lambda\in [0, 1]$ and $m \in [M]$ the $\eta_{\lambda, m}^\star = \lambda \Phi_m^\star + (1 - \lambda) \kappa_m$ are also $(\beta, K_\beta)$-H\"older smooth.
% The smoothness on the $\Phi_m^\star$ functions controls their complexity: the higher smoothness implies a lower complexity and is easier to estimate. \SM{Talk about how different complexity would lead to the rate at lowest smoothness. Argue it through the estimation of the differences. And also talk about how smoothness in $\kappa_m$ is not necessary.}
% Before we move on, we want to make a few remarks about this smoothness assumption. As discussed in Section \ref{sec:setup} the core idea behind our approach is to plug-in an estimate of $[\Phi(X)]_{m, k}$ into the oracle router \eqref{eq:oracle-router-2}
%  \[
%     \textstyle g_\mu^\star(X)  = \argmin_m \big\{ \sum_{k = 1}^K \mu_k [\Phi (X)]_{m, k} \big\} \,.
%     \]
% A similar idea can also be found  in the context of binary classification in non-parametric settings: for $(X, Y) \in \cX \times \{0, 1\}$ drawn from the distribution $P$, they plug-in an estimator of $\eta(X) = P(Y = 1\mid X )$ into the Bayes classifier $f^\star(X) = \bbI\{\eta(X) \ge \nicefrac{1}{2}\}$ to obtain a minimax rate optimal classifier, which they call ``plug-in classifier''. In their context, the smoothness in $\eta$ controls how well it can be estimated from a dataset, which later affects the misclassification error for this plug-in classifier. Drawing a parallel to our context,
% \begin{enumerate}
%     \item Within the regression function $\eta_{\lambda, m}^\star(X) = \lambda \Phi^\star _m(X) + (1 - \lambda) \kappa_m(X)$ the $\kappa_m$ are already known and need not be estimated. Thus, we do not require any smoothness assumption for $\kappa_m$ and only require a smoothness condition for the unknown $\Phi_m^\star$.  
%     \item To make the setting more general, we could assume different smoothness parameters for different $\Phi_m^\star$; \eg\ $\Phi_m^\star$ is $\beta_m$ H\"older smooth, in which case it can be estimated at a minimax optimal $\ell_1$-error rate $\cO_P(n^{-{\beta_m}/{(2\beta_m + d)}})$ \citep{fan1997local}. But then the differences \[
%     \textstyle \eta_{\lambda, m_1}^\star(X) - \eta_{\lambda, m_2}^\star(X) = \lambda \big\{ \Phi^\star _{m_1}(X) -\Phi^\star _{m_2}(X)\big\}  + (1 - \lambda) \big \{ \kappa_{m_1}(X) - \kappa_{m_2}(X)\big \}\,, 
%     \] which are crucial for the prediction of oracle routers, will be estimated at a rate 
%     \[
%     \textstyle \cO_P\big(n^{-\frac{\beta_{m_1}}{2\beta_{m_1} + d}} \vee n^{-\frac{\beta_{m_2}}{2\beta_{m_2} + d}}\big) = \cO_P\big(n^{-\frac{\beta_{m_1} \wedge \beta_{m_2}}{2(\beta_{m_1} \wedge \beta_{m_2}) + d}} \big)\,,
%     \]
%     and in the worst case, at a rate $\cO_P(n^{-{\beta_{\min} }/{(2\beta_{\min}  + d})} )$ with respect to the smallest smoothness parameter $\beta_{\min} = \min_m \beta_m$. Since accurately estimating all of these pairwise differences is important to obtain a prediction similar to that of the oracle routers, the final rate of convergence rate for excess risk will be determined by the smallest smoothness parameter, in which case, the other smoothness parameters become irrelevant. Therefore, for a simpler exposition of the problem setting, we assume that the smoothness parameters of $\Phi_m^\star$ are all identical. 
% \end{enumerate}




Next, we introduce \emph{margin condition}, which quantifies the difficulty in learning the oracle router.  For a given $\mu$ define the margin as the difference between the minimum and second minimum of the risk values: 
{ \begin{equation}\label{eq:margin}
    \begin{aligned}
        & \textstyle \Delta_\mu(x) =  
    \begin{cases}
       \min\limits_{m \notin g_\mu(x)} \eta_{\mu, m}(x) - \min\limits_m \eta_{\mu, m}(x) & \text{if} ~ g_\mu^\star(x) \neq [M]\\ 
       0 & \text{otherwise}.
       \end{cases} 
    \end{aligned}
\end{equation}}

% At an $x$, the margin is simply the gap between the second-lowest and lowest coordinate values of $\eta_{\lambda, m}$. If all the coordinates are the same, then we set the margin at zero. 
\noindent Our definition of a margin generalizes the usual definition of the margin considered for binary classification with $0/1$ loss in \citet{audibert2007Fast}. Recall the binary classification example in \ref{example:binary-classification}, in which case, 
$[\Phi(X)]_{m , 1} =  P(Z \neq m\mid X) $. Since $K = 1$ we have 
$\eta_{\mu, m}(X) = P(Z \neq m\mid X) $, which further implies $\eta_{\mu, 0}(X) + \eta_{\mu, 1}(X) = 1$.
Thus for binary classification with $0/1$ loss, our definition of margin simplifies to 
\begin{align*}
\textstyle \min\limits_{m \notin g_\mu^\star(x)} \eta_{\mu, m}(x) - \min\limits_m \eta_{\mu, m}(x)
=  |\eta_{\mu, 1}(X) - \eta_{\mu, 0}(X)| = 2 |\eta_{\mu, 0}(X) - \nicefrac{1}{2}| \,,
\end{align*}
which is a constant times the margin $  |P(Y = 1\mid X) - \nicefrac{1}{2}| = |\eta_{\mu, 0}(X) - \nicefrac{1}{2}| $ in \citet{audibert2007Fast}. 


% Relating their framework to ours, for them $M = 2$ with the class indices $\{0, 1\}$. Moreover, for $m \in \{0, 1\}$ the loss regression function for classifying a sample $X$ as class $m$ is  $\eta_{\lambda, m}^\star(X) = P(Y \neq m\mid X) $, which satisfies $\eta^\star_{\lambda, 0}(X) + \eta^\star_{\lambda, 1}(X) = 1$. In this case, our definition of margin simplifies to $|\eta_{\lambda, 1}(X) - \eta_{\lambda, 0}(X)| = 2 |\eta_{\lambda, 1}(X) - \nicefrac{1}{2}|$, which is a constant multiplication of their definition of margin $  |P(Y = 1\mid X) - \nicefrac{1}{2}| = |\eta_{\lambda, 1}(X) - \nicefrac{1}{2}| $. 



Clearly, the margin determines the difficulty in learning the oracle router. A query $X$ with a small margin gap is difficult to route, because to have the same prediction as the oracle, \ie\  $\argmin_{m} \hat \eta_{\mu, m}(X) = \argmin_{m} \eta_{\mu, m}^\star(X)$ we need to estimate $ \eta_{\mu, m}^\star(X)$ with high precision. In the following assumption, we control the probability of drawing these ``difficult to route'' queries.

\begin{assumption}[Margin condition]\label{assmp:margin}
    For $\alpha, K_\alpha >0$ and any $t > 0$ the margin $\Delta_{\mu}$ \eqref{eq:margin} satisfies: \begin{equation}
        P_X \big\{0 < \Delta _\mu(X) \le t\big \}  \le K_\alpha t^{\alpha}\,. 
    \end{equation}
\end{assumption}
%From Proposition 3.4 of \cite{audibert2007Fast}, if $\alpha  (1 \wedge \gamma_{k})  \ge d$ for some $k$ then if $\mu = e_k$ for which the $g_\mu^\star$ never changeswe argue that when $\alpha  (1 \wedge \gamma_{k})  \ge d$ for some $k$ then for $\mu = e_k$ for which the $g_\mu^\star$ never changes its decision within the interior of $\supp(P_X)$. 
Following \citet{audibert2007Fast}, we focus on the cases where $\alpha < d$ and for every $k$ the $\alpha \gamma_k < d$. This helps to avoid trivial cases where routing decisions are constant over $P_X$ for some $\mu$.  %These are trivial cases, which we ignore. Thus, throughout our paper, we assume that $\alpha   < d$ and for every $k$ the $\alpha \gamma_k < d$.  
Next, we assume that $P_X$ has a density $p_X$ that satisfies a strong density condition described below.
% We start by formalizing the problem setup. We assume that the covariate space is $\cX$ is a compact set in $\reals^d$. Next, we assume that the density exists for the marginal probability $P_X$ and satisfies a strong density condition, which is formalized below. 
\begin{assumption}[Strong density condition] \label{assmp:strong-density}
Fix constants $c_0, r_0> 0$ and $0 \le \mu_{\min}  \le \mu_{\max} < \infty$. We say $P_X$ satisfies the strong density condition if its support is a compact $(c_0, r_0)$-regular set and it has density $p_X$ which is bounded: $\mu_{\min} \le p_X (x) \le \mu_{\max} $ for all $x$ within $\supp(P_X)$. A set $A \subset \reals^d$ is Lebesgue measurable and %\ie\ for the Lebesgue measure $\Lambda_d$ on $\reals^d$, any Lebesgue measurable set $A \subset \reals^d$ and 
$\text{for any} ~ 0 < r \le r_0, ~  x \in A$ it satisfies
\begin{equation}
    \Lambda_d (A \cap \cB(x, r, \ell_2)) \ge c_0 \Lambda_d(\cB(x, r, \ell_2)). %~ \text{for any} ~ 0 < r \le r_0, ~  x \in A,
\end{equation} %and the density $p_X$ is bounded as: $\mu_{\min} \le p_X (x) \le \mu_{\max} $ for all $x$ within $\supp(P_X)$. 
\end{assumption}
This is another standard assumption required for minimax rate studies in nonparametric classification problems \citep{audibert2007Fast,cai2019Transfer}. All together, we define $\cP(c_0, r_0, \mu_{\min}, \mu_{\max}, \beta_{m ,k}, K_{\beta, m, k}, \alpha, K_\alpha)$, or simply $\cP$, as the class of probabilities $P$ defined on the space $\cX \times \cY$ for which $P_X$  is  compactly supported and satisfies the strong density assumption \ref{assmp:strong-density} with parameters $(c_0, r_0, \mu_{\min}, \mu_{\max})$, and the H\"older smoothness assumption \ref{assmp:smooth} and the $(\alpha, K_\alpha)$-margin condition in Assumption \ref{assmp:margin} hold. We shall establish our minimax rate of convergence within this probability class. 










% In this section we provide a ``mini-max'' investigation on learning of the oracle $g^\star$. For this purpose, assume that 
% \begin{itemize}
    
%     \item {\bf Strong density condition:} $X$ is distributed in the space $[0, 1]^d$ that satisfies the strong density condition \citep{audibert2007Fast}.
%     \item {\bf H\"older smoothness:} The functions $\kappa_\lambda(x, l) \triangleq\ell (f^\star(x), f_l(x)) + \lambda c_l (x)$ are $\alpha$-H\"older smooth. 
% \item {\bf Noise condition:} For $t > 0$ there exists a $\gamma > 0$ such that 
% \[
% \textstyle P_X \big ( 0 <\max_l \big | \kappa_\lambda(x, l) - \min_{l' \neq l} \kappa_\lambda (x, l') \big | \le t \big ) = \cO(t^\gamma)
% \]
% \end{itemize}
% Denote $\cP$ as the class of all probabilities that satisfies the above conditions. 

% \SM{revise the lower bound in light of $\lambda  \ge c n^{\frac{\beta - \nicefrac d\alpha }{2\beta +d }}$ requirement.}
\subsection{The lower bound} 
Rather than the actual risk $\cR_P(\mu, g)$, we establish a lower bound on the excess risk:
\begin{equation}\label{eq:excess-risk}
    \cE_P(\mu, g) = \cR_P(\mu, g) - \cR_P(\mu, g_\mu^\star)\,,
\end{equation} that compares the risk of a proposed router to the oracle one. We denote $\Gamma = \{g: \cX \to [M]\}$ as the class of all routers. For an $n \in \bN$ we refer to the map $A_n: \cZ^n \to \Gamma$, which takes the dataset $\cD_n $ as an input and produces a router $A_n(\cD_n): \cX \to [M]$, as an algorithm. Finally, call the class of all algorithms that operate on $\cD_n$ as $\cA_n$. The following Theorem describes a lower bound on the minimax risk for any such algorithm $A_n$. 
\begin{theorem}\label{thm:lower-bound}
    For an $n \ge 1$  and $A_n \in \cA_n$ define  $\cE_P(\mu, A_n) = \Ex_{\cD_n}\big[\cE_P\big(\mu, A_n(\cD_n)\big)\big]$ as the excess risk of an algorithm $A_n$. There exists a constant $c> 0$ that is independent of both $n$ and $\mu$ such that for any $n\ge 1$ and $\mu\in \Delta^{K-1}$ we have the lower bound
    \begin{equation}\label{eq:lower-bound}
      \textstyle  \min\limits_{A_n \in \cA_n} \max\limits_{P \in \cP} ~~ \cE_P(\mu, A_n) \ge c \big \{\sum_{k = 1}^{K_1} \mu_k n^{- \frac{\gamma_k}{2\gamma_k + d}}\big\}^{1+\alpha} \,.
    \end{equation}
\end{theorem} 
This result is a generalization of that in \citet{audibert2007Fast}, which considers binary classification. 
\begin{remark} \label{remark:minimax-lower-bound}
    Consider the binary classification in Example \ref{example:binary-classification}. Since $K = 1$, the lower bound simplifies to $\cO(n^{-\nicefrac{\gamma_1 (1+ \alpha)}{2\gamma_1 + d}})$,  which matches with the rate in \citet[Theorem 3.5]{audibert2007Fast}. 
    Beyond $0/1$ loss, our lower bound also establishes that the rate remains identical for other classification loss functions as well. 
    
    % The case of $\lambda = 1$ is closely related to the usual classification tasks with a single objective function. In fact, binary classification with $0/1$-loss is a special case. To make this connection clear, consider $M = 2$ and the index set for the classes as $\{0, 1\}$. Moreover, assume that the loss is $0/1$, \ie\ for $m\in \{0, 1\}$ the  $Z_m = \ell\{ Y, f_m(X)\} \in \{0, 1\}$ and $Z_0 + Z_1 = 1$, in which case the loss of a router $g:\cX \to \{0, 1\}$ is 
    % \[
    % \begin{aligned}
    %      \textstyle \ell\{g; X, Y\} & \textstyle= \bbI\{ g(X) = 0\} Z_0 +  \bbI\{ g(X) = 1\} Z_1 \\
    %      & \textstyle= \bbI\{ g(X) = 0\} Z_0 +  \bbI\{ g(X) = 1\} (1 - Z_0) = \bbI \{g(X) \neq Z_0\} \,.
    % \end{aligned}
    % \] Thus, it is not surprising that for $\lambda = 1$ our rate of convergence $\cO(n^{\frac{-\beta(1 + \alpha)}{2\beta +d}})$ for the lower bound is exactly the same as in \citet[Theorem 3.5]{audibert2007Fast}. As such, {we broaden the framework of the minimax lower bound study for non-parametric classification tasks to (1) more than two classes, and (2) general loss functions.} To understand this, consider a classification task with $M$ classes and the loss function of a classifier $g: \cX \to [M]$ is $\ell\{ g; X, Y\} = \sum_{m = 1}^M \bbI \{g(X) = m\} \ell_m(X, Y)$ where $\ell_m(X, Y)$ is the loss incurred when a sample $(X, Y)$ is predicted as the class $m$. In that case, simply letting $\lambda = 1$ and $Z_m = \ell_{m}(X, Y)$ within the lower bound analysis, we obtain a rate of convergence $\cO(n^{\nicefrac{-\beta(1 + \alpha)}{(2\beta +d)}})$. Moreover, the analysis of the upper bound in Section \ref{sec:upper-bound} reveals that this rate is minimax optimal. 
\end{remark}


% \SM{mention how the true difficulty of the problem lies when the $\lambda$ are bounded away from zero because in that case one needs to accurately estimate the $\Phi_m^\star$. At $\lambda \to 0$ the importance is not on the unknown function.}

% \begin{remark}\label{remark:diff-in-lambda-lb} \label{remark:lower-bound-lambda}
%     The lower bound also highlights that ``it is easier to route for a smaller value $\lambda $''. Within the oracle router, we know the $(1 - \lambda) \kappa_m(X)$ part within the function $\eta_{\lambda, m}^\star(X) = \lambda \Phi_m^\star(X) + (1 - \lambda) \kappa_m(X)$. Thus, at an intuitive level, for smaller values of $\lambda$ there is less importance on the unknown $\lambda \Phi_m^\star(X)$ part, and thus the plug-in router in eq. \eqref{eq:plugin-router} can tolerate a larger noise in their estimations. Our lower bound makes this intuition precise: for a smoothness parameter $\beta$ the $\Phi_m^\star$ are estimated at a $\cO_P(n^{-\nicefrac{\beta}{(2\beta+d)}})$-rate, and thus both $\lambda \Phi_m^\star(X)$ and the whole $\eta_{\lambda, m}^\star(X)$ are estimated at a $\cO_P(\lambda n^{-\nicefrac{\beta}{(2\beta+d)}})$-rate. This intuition is more formally exposed in the upper bound study of excess risk in Theorem \ref{thm:upper-bound}, where we precisely quantify the relationship between the error in the estimation of $\Phi_m^\star$ and the convergence rate for excess risk (\cf\ Remark \ref{remark:upper-bound-lambda}).
%     Following this intuition, we can understand that learning a router for a smaller $\lambda$ is easier and for $\lambda = 1$ it is the hardest.
% \end{remark}

% This is the exact same rate of convergence obtained in \citet[Theorem 3.5]{audibert2007Fast}. Intuitively, for a fixed $\lambda$ the task of routing is equivalent to multi-class classification, so, it is not surprising that they have the same rate. 


% \SM{remark about $\lambda$}. 

% \begin{remark}
%     For a fixed $n$ our lower bound analysis in Theorem \ref{thm:lower-bound} is only valid for $\lambda \ge c_1 n^{\frac{\beta - \nicefrac{d}{\alpha}}{2\beta + d}}$. To address this gap, we note: 
%     \begin{itemize}
%         \item Firstly, because we know the cost functions $\kappa_m$, we can precisely find the oracle router $g^\star_{0}(X) = \argmin_m \kappa_m(X)$ at $\lambda = 0$. Furthermore, since we consider $\alpha \beta < d$ the $c_1 n^{\frac{\beta - \nicefrac{d}{\alpha}}{2\beta + d}}$ decreases to zero as $n$ grows to infinity. Thus, the gap in $\lambda$, where this lower bound is invalid, vanishes as the sample size increases to infinity. 
%         \item Regardless of this gap, we argue in Remark \ref{} that we can estimate the Pareto frontier for the performance-cost trade-off efficiently. 
%     \end{itemize}
% \end{remark}


% \SM{compare it to usual rate of convergence for classification when $\lambda = 1$}. 


% \subsection{Efficient learning of oracle routers}
% \label{sec:efficient-learning}
% Let us quickly recall our core idea behind it. For a $\lambda \in [0, 1]$ the true loss regression function $\eta_{\lambda, m}^\star(X)$ for the oracle router $g_\lambda^\star (X) = \argmin_m \eta_{\lambda, m}^\star(X)$ is  decomposed as: 
% \begin{equation} \label{eq:oracle-router}
%     \eta_{\lambda, m}^\star(X) = \lambda \Phi^\star_m(X) + (1 - \lambda) \kappa_m(X), ~ \Phi^\star _m(X) = \Ex_P[\ell\{ Y, f_m(X)\} \mid X ]\,. 
% \end{equation} Since we already know $\kappa_m(X)$ at a new $X$ the only unknown is the $\Phi^\star_m(X)$. Thus, we can plug-in its estimate  $\widehat \Phi_m(X)$ within eq. \eqref{eq:oracle-router} and estimate the oracle router as:
% \begin{equation}\label{eq:plugin-router}
%     \widehat g_\lambda(X) = \argmin_m \hat \eta_{\lambda, m}(X), ~~ \hat \eta_{\lambda, m}(X) = \lambda \widehat \Phi_m(X) + (1 - \lambda) \kappa_m(X)  
% \end{equation} 
% Moreover, we evaluate their prediction errors and costs on a test split of the dataset as: 
% \begin{equation}
%    \textstyle  \hat \cE_\lambda  =  \frac1{n_\text{test}} \sum_{i = 1}^{n_\text{test}}  \ell\{Y_i', f_{\widehat g_\lambda(X_i')}(X_i')\}, ~~ \hat \cC _\lambda =  \frac1{n_\text{test}} \sum_{i = 1}^{n_\text{test}}  \kappa_{\widehat g_\lambda(X_i')}(X_i')\,. 
% \end{equation} 
% This plug-in approach is computationally efficient, as we can estimate oracle routers for all $\lambda$ in one go, instead of minimizing \ref{eq:ERM} at different $\lambda$'s. In addition to being computationally efficient, through a study on the minimax upper bound on excess risk in the next section we establish that this plug-in approach is also statistically efficient. Furthermore, we extend this approach to a general multi-objective classification problem. For now, we end this section by summarizing our steps in Algorithm \ref{alg:pareto-routers}. 


% \begin{algorithm}
%     \begin{algorithmic}[1]
% \Require Dataset $\cD_n$
% \State Randomly split the dataset into training and test splits: $\cD_n = \cD_{\text{tr}} \cup \cD_{\text{test}}$. 
% \State  Learn an estimate $\widehat \Phi_m (X)$ of $\Phi_m^\star(X)$ using the training split $\cD_{\text{tr}}$. 
% \For{$\lambda \in [0, 1]$}
% \State  Define $\hat \eta_{\lambda, m}(X) =  \lambda \widehat \Phi_m(X) + (1 - \lambda) \kappa_m(X)  $ and 
%  $\widehat g_\lambda(X) = \argmin_m \hat \eta_{\lambda, m}(X)$. If there is a tie within the $\argmin$, break the tie randomly.
%  \State Calculate $\hat \cE_\lambda  =  \frac1{|\cD_{\text{test}}|} \sum_{(X, Y) \in \cD_{\text{test}}}  \ell\{Y, f_{\widehat g_\lambda(X)}(X)\}$
%  \State \quad\quad  and $\hat \cC_\lambda  =  \frac1{|\cD_{\text{test}}|} \sum_{(X, Y) \in \cD_{\text{test}}}  \kappa_{\widehat g_\lambda(X)}(X)$
% \EndFor

% \Return $\{g_\lambda: \lambda \in [0, 1]\}$ and $\hat\cF = \{(\hat \cE_\lambda, \hat \cC_\lambda): \lambda \in [0, 1]\}$. 
% \end{algorithmic}
% \caption{Learning of oracle routers}
% \label{alg:pareto-routers}
% \end{algorithm}


% \SM{talk about the computational efficiency, that we can obtain all the routers in one go. Also, provide a teaser that in the next section we shall establish that its statistically efficient as well.}





% Given a router $\widehat g:[0, 1]^d \to \Delta^ L$ we define the excess risk as:
% \begin{equation}
%    \textstyle \cE_P (\widehat g) = \cR_P(\widehat g) - \cR_P(g^\star) = \Ex_P\big[\sum_{l = 1}^L\{g_l(x_0)- g_l^\star(X_0)\}\kappa_\lambda(X_0 , l) \big]
% \end{equation} We shall show that 
% \begin{equation}
%    \inf_{\widehat g} \sup_{P\in \cP} \textstyle \Ex_P \big[ \cE_\lambda (\widehat g) \big ] \asymp n ^{- \frac{\alpha(1 + \gamma)}{2\alpha + d}}\,.
% \end{equation}



\subsection{The upper bound }\label{sec:upper-bound}
Next, we show that if algorithm the $A_n$ corresponds to CARROT, the performance of $\hat{g}_{\mu}$ matches the lower bound in Theorem \ref{thm:lower-bound} (\cf\ equation \ref{eq:lower-bound}). En-route to attaining $\hat{g}_{\mu}$, we need an estimate $\widehat \Phi(X)$ of $\Phi(X) = \Ex_P[Y \mid X ]$. %In this section, we ask what is the needed performance of this estimate? More importantly, we also study if the plug-in approach leads to statistically efficient estimates of the oracle routers, or in other words does the performance of $\hat{g}_{\mu}$ match \ref{eq:lower-bound} 
Our strategy will consist of two steps: 
\begin{itemize}
    \item First,  we establish an upper bound on the rate of convergence for excess risk \eqref{eq:excess-risk} for the plug-in router in terms of the rate of convergence for $\widehat \Phi(X)$. 
    \item Then we discuss the desired rate of convergence in $\widehat \Phi(X)$ so that the upper bound has the identical rate of convergence to the lower bound \eqref{eq:lower-bound}. Later in Appendix \ref{sec:reg-fn-estimate} we provide an estimate $\widehat \Phi(X)$ that has the required convergence rate. 
\end{itemize}
These two steps, together with the lower bound in \eqref{eq:lower-bound} establish that our plug-in router achieves the best possible rate of convergence in excess risk. 

We begin with an assumption that specifies a rate of convergence for $[\widehat \Phi(X)]_{m, k}$. 
\begin{assumption} \label{assmp:convergence}
    For some constants $\rho_1, \rho_2 > 0$ and any $n \ge 1$ and $t > 0$ and almost all $X$ with respect to the distribution $P_X$ we have the following concentration bound:
    \begin{align}\label{eq:concentration-phi}
        \max_{P\in \cP} P \big \{ \max_{m, k} a_{k, n}^{-1}\big |[\widehat \Phi (X)]_{m, k} - [\Phi  (X)]_{m, k}\big |
        \ge t\big \}  
        \le  \rho_1 \exp\big (- \rho_2  t^2 \big )\,,  
    \end{align}  where for each $k$ the  $\{a_{k,n}; n \ge 1\}\subset (0, \infty)$ is a sequence that decreases to zero. 
\end{assumption}
Using this high-level assumption, in the next theorem, we establish an upper bound on the minimax excess risk for CARROT that depends on both $a_{k, n}$ and $\mu$.  
\begin{theorem}[Upper bound]\label{thm:upper-bound}
  Assume \ref{assmp:convergence}.   If all the $P\in \cP$ satisfy the margin condition \ref{assmp:margin} with the parameters $(\alpha, K_\alpha)$ then there exists a $K> 0$ such that for any $n \ge 1$ and $\mu\in \Delta^{K-1} $ the excess risk for the router $\widehat g_\mu$ in Algorithm \ref{alg:pareto-routers} is upper bounded as 
    \begin{equation}
        \max_{P\in \cP} \Ex_{\cD_n}\big [\cE_P(\widehat g_\lambda,\lambda)\big ] \le\textstyle K \big \{\sum_{k = 1}^{K_1} \mu_k a_{k, n}\big\}^{1+\alpha} \,.
    \end{equation}
\end{theorem} 
% \begin{remark} \label{remark:upper-bound-lambda}
%     Recall the Remark \ref{remark:lower-bound-lambda}, where we discuss that it is easier to route for smaller $\lambda$. Indeed, under Assumption \ref{assmp:convergence} the same holds for $\hat \eta_{\lambda, m}(X) -  \eta_{\lambda, m}^\star(X) = \lambda \{ \widehat \Phi_m(X) - \Phi^\star_m(X)\}$ at a rate $\lambda a_n^{-1}$, which, under the $\alpha$-margin condition (Assumption \ref{assmp:margin}), reveals such a dependence for the minimax upper bound on $\lambda$. 
% \end{remark}


\begin{remark}[Rate efficient routers] \label{cor:efficient-routers}
    When $a_{k, n} = n^{-\nicefrac{\gamma_k}{(2\gamma_k +d)}}$ the upper bound in Theorem \ref{thm:upper-bound} has the $\cO(\{\sum_{k = 1}^{K_1}\mu_k n^{-\nicefrac{\gamma_k}{(2\gamma_k+d)}}\}^{1+\alpha})$-rate, which is identical to the rate in the lower bound (\cf\ Theorem \ref{thm:lower-bound}), suggesting that the minimax optimal rate of convergence for the routing problem is 
\begin{equation}
    \label{eq:minimax-rate}
     \textstyle  \min\limits_{A_n \in \cA_n} \max\limits_{P \in \cP} ~~ \cE_P(A_n, \lambda) \asymp \textstyle  \cO\big ( \big \{\sum_{k = 1}^{K_1} \mu_k n^{- \frac{\gamma_k}{2\gamma_k + d}}\big\}^{1+\alpha}\big ) \,.
\end{equation}
   Following this, we conclude: When $a_{k, n} = n^{-\nicefrac{\gamma_k}{(2\gamma_k +d)}}$ the plug-in approach in Algorithm \ref{alg:pareto-routers}, in addition to being computationally efficient, is also minimax rate optimal. 
%\begin{enumerate}
   % \item When $a_{k, n} = n^{-\nicefrac{\gamma_k}{(2\gamma_k +d)}}$ the plug-in approach in Algorithm \ref{alg:pareto-routers}, in addition to being computationally efficient, is also minimax rate optimal in excess risk. 
   % \item  Following up on the Remark  \ref{remark:minimax-lower-bound}, this result generalizes the minimax study of non-parametric binary classification to (a) more than two classes, and (b) classification loss functions beyond $0/1$-loss. 
%\end{enumerate}
\end{remark} 
An example of an estimator $\widehat{\Phi}$ that meets the needed conditions for $a_{k, n} = n^{-\nicefrac{\gamma_k}{(2\gamma_k +d)}}$ to hold is described in Appendix \ref{sec:reg-fn-estimate}. 
% \begin{remark} \label{remark:difficulty-routing}
% The optimal rate of convergence in \eqref{eq:minimax-rate} is particularly small for small values of $\sum_{k = 1}^{K_1} \mu_k$; in fact, it's identical to zero when $\sum_{k = 1}^{K_1} \mu_k = 0$. This is quite intuitive from the expression of $g_\mu^\star$ in Lemma \ref{lemma:oracle-router}.  For $\sum_{k = 1}^{K_1} \mu_k = 0$ the $g_\mu^\star(X) = \argmin_{m} \{\sum_{k\ge K_1+1} \mu_k [\Phi(X)]_{m, k}\}$ is precisely known to us, thus the excess risk for routing is identical to zero.
    
% \end{remark}

   
%     This leads to the two following conclusions: 
%     \begin{enumerate}
%         \item For any $\lambda$ the minimax optimal rate of convergence for the routing problem is 
%         \[
%         \textstyle \textstyle  \min_{A_n \in \cA_n} \max_{P \in \cP} ~~ \cE_P(A_n, \lambda) \asymp  \cO\big ( \big\{\lambda n^{-\nicefrac{\beta}{(2\beta+d)}}\big\}^{1+\alpha}\big ) \,.
%         \] 
%         This reveals 
        
        
%         Unsurprisingly, this is identical to the optimal rate of convergence in non-parametric classification problems \citep{audibert2007Fast}. Repeating the intuition again, for a fixed $\lambda$ the routing problem 
%         is essentially a multiclass identical minimax optimal rate of convergence for excess risk. 
%         \item  A more interesting observation is that if $\widehat \Phi_m$ converges to $\Phi_m^\star$ at a rate $a_n^{-\frac{1}{2}} = n^{-\frac{\beta}{2\beta +d}}$ then our one-shot approach in Algorithm \ref{alg:pareto-routers} achieves this optimal minimax rate of convergence in excess risk and thus is a \emph{rate efficient} estimator for Pareto routers. We shall show later in Section \ref{sec:reg-fn-estimate} that a local polynomial regression estimator with a suitable bandwidth achieves this desired rate. 
%         \end{enumerate}



% \begin{remark}
%     Remark about the importance of $\lambda$. 
% \end{remark}

% \begin{remark}
%     Let us quickly recall the gap in the validity in our lower bound in Theorem \ref{thm:lower-bound}, that the rate in the lower bound is not true when $0 < \lambda < c_1 n ^{\frac{\beta - \nicefrac{d}{\alpha}}{2\beta +d}}$. However, the rate in the upper bound continues to hold for all $0 \le \lambda \le 1$. 
% \end{remark}


% Now, we make this statement precise and show that for such an $\widehat \Phi $ the excess risk \eqref{eq:excess-risk} achieves the rate in the lower bound \eqref{eq:lower-bound}. 

% \begin{theorem}[Upper bound]\label{thm:upper-bound}
%     Suppose that for some $\rho_1, \rho_2 > 0$ and any $n \ge 1$ and $t > 0$ and almost all $x$ with respect to $P_X$ we have the following concentration bound for $\widehat \Phi$:
%     \begin{equation}\label{eq:concentration-phi}
%         \max_{P\in \cP} P_{\cD_n} \big \{ \|\widehat \Phi(x) - \Phi^\star (x)\|_1 \ge t\big \} \le  \rho_1 \exp\big (- \rho_2 a_n t^2 \big )\,, 
%     \end{equation}
%     where $\{a_n; n \ge 1\}\subset \reals$ is a sequence that increases to $\infty$.  
%     Fix a $\lambda \in [0, 1]$.  Then, if all the $P\in \cP$ satisfies the margin condition \ref{assmp:margin} with the parameters $(\alpha, K_\alpha)$ then there exists a $K> 0$ such that for any $n \ge 1$ the excess risk for the router $\widehat g_\lambda$ in \eqref{eq:eff-estimate-router} is upper bounded as 
%     \begin{equation}
%         \max_{P\in \cP} \Ex_{\cD_n}\big [\cE_P(\widehat g_\lambda,\lambda)\big ] \le K a_n^{-\frac{1+ \alpha}{2}}\,. 
%     \end{equation}
%     Thus, as long as $a_n = n^{{  2\beta/(2\beta + d)}}$  for any $\lambda \in [0, 1]$ the excess risk for the router $\widehat g_\lambda$ in \eqref{eq:eff-estimate-router} has the rate of convergence $a_n^{- {(1 + \alpha)}/{2}} = n^{- {\beta(1 + \alpha)}/{(2\beta + d)}}$ that matches with the lower bound rate in \eqref{eq:lower-bound}. 
% \end{theorem}






% \subsection{Efficiency of the estimation Pareto frontier}

% As mentioned in Section \ref{sec:efficient-learning} and Algorithm \ref{alg:pareto-routers} we estimate the performance-cost trade-off for Pareto routers in a very simple manner: first estimate the Pareto routers $\widehat g_\lambda$ in a training split of the dataset, then evaluate their average performances and costs in the remaining test split as
% \[
% \textstyle \hat \cE_\lambda  =  \frac1{n_\text{test}} \sum_{i = 1}^{n_\text{test}}  \ell\{Y_i', f_{\widehat g_\lambda(X_i')}(X_i')\}, ~~ \hat \cC _\lambda =  \frac1{n_\text{test}} \sum_{i = 1}^{n_\text{test}}  \kappa_{\widehat g_\lambda(X_i')}(X_i')\,. 
% \] In this section we evaluate the efficiency of estimating the Pareto front $\{ (\hat \cE_\lambda, \hat \cC_\lambda): 0 \le \lambda \le 1\}$ in this manner. Note that the true Pareto front is $\{ ( \cE_\lambda^\star,  \cC_\lambda^\star): 0 \le \lambda \le 1\}$ where 
% \[
% \textstyle \cE^\star_\lambda = \Ex_P [\ell\{ Y, f_{g_\lambda^\star(X)}(X)\}], ~~ \cC_\lambda^\star =  \Ex_P [\kappa _{g_\lambda^\star(X)}(X)]\,. 
% \] To evaluate their differences, focus only on $\hat \cE_\lambda - \cE_\lambda^\star$, as the analysis of $\hat \cC_\lambda - \cC_\lambda^\star$ is very similar. Defining $\widetilde \cE_\lambda = \Ex_P [\ell\{ Y, f_{\widehat g_\lambda(X)}(X)\}]$, we can decompose the difference as 
% \begin{equation}
%     \textstyle \hat \cE_\lambda - \cE_\lambda^\star = \{\hat \cE_\lambda - \widetilde \cE_\lambda\}  + \{\widetilde \cE_\lambda-  \cE_\lambda^\star\} 
% \end{equation} The first term is $\cO (n^{-\nicefrac12})$. Now, to bound the second term, notice that 
% \[
% \textstyle \cE_P(\widehat g_\lambda,\lambda) \le 
% \]


% that for $h = n ^{-1/(2\beta + d)}$ the concentration bound in \eqref{eq:concentration-phi} is satisfied with $a_n = n^{-2\beta/(2\beta + d)}$, therefore the router derived from such $\hat\Phi$ using \eqref{eq:eff-estimate-router} achieves the same rate of convergence as in the lower bound \eqref{eq:lower-bound}, \ie\ rate optimal. 
% !TEX root = main.tex

\section{Testing Uniformity of Monotone Distributions}

In this section, we prove the following theorem, which gives a query lower bound on testing uniformity of a distribution which is promised to be monotone using subcube conditioning queries. 
\begin{theorem}[Uniformity Testing of Monotone Distributions -- Lower Bound] For any $\eps > 0$,~any $\eps$-test for uniformity of distributions that are promised to be monotone must make $\tilde{\Omega}(\sqrt{n}/\eps^2)$ queries.
\end{theorem}
Similar to Section~\ref{sec:mon-lb}, we describe a distribution $\Dno$ supported on monotone product distributions over $\{-1,1\}^n$. Importantly, a distribution $\bp \sim \Dno$ will be $\Omega(\eps)$-far from the uniform distribution with probability $1-o_n(1)$ (Lemma \ref{lem:distancelemma}). Then, we show that for  $q$ which is at most $c \sqrt{n} / (\eps^2 \log^4 n)$, for a small enough constant $c > 0$, any function $\Alg \colon \{-1,1\}^{nq} \to \{\text{``accept''}, \text{``reject''} \}$ cannot output ``accept'' with probability at least $0.99$ when samples are drawn from the uniform distribution, and output ``reject'' with probability at least $0.99$ when samples are drawn from  $\Dno$.

\subsection{The distribution $\Dno$}\label{sec:dno-def}

A draw of $\bp \sim \Dno$ is generated as follows:
\begin{flushleft}\begin{itemize}
\item First, we let $\calD$ denote the distribution over vectors $\bmu$ where we 
independently set $\bmu_i$ to be $\eps/n^{1/4}$ with probability $1/\sqrt{n}$ and $0$ otherwise.
%sample a subset $\bN$ of size $\sqrt{n}$ uniformly at 
%random from $[n]$ and set $\bmu_i = 1 / n^{1/4}$ when $i\in \bN$ and $\bmu=0$ when $i\notin \bN$.}
  %{\color{red} and $\bmu_i=0$ otherwise}.
\item Then, we let $\bp$ be the monotone product distribution on $\{-1,1\}^n$ whose mean vector is $\bmu$.
\end{itemize}\end{flushleft}
The fact that $\bp \sim \Dno$ is far from the uniform distribution follows from the subsequent lemma.
\begin{lemma}\label{lem:distancelemma}
With probability at least $1 - o_n(1)$, $\bp \sim \Dno$ is $\Omega(\eps)$-far from the uniform distribution.
\end{lemma}
\begin{proof}
We consider a draw of $\bmu \sim \calD$, and note that with probability at least $1-o_n(1)$, $\bp \sim \Dno$ has a set $\bN \subset [n]$ of at least $\Omega(\sqrt{n})$ coordinates $i$ with $\bmu_i = \eps/n^{1/4}$. Fix such a draw and let $\bp$ denote the corresponding distribution. The total variation distance from $\bp$, generated with mean vector $\mu$, to the uniform distribution can be lower bounded by considering strings $x \in \{-1,1\}^n$ which have fewer $1$'s than $-1$'s in coordinates of $\bN$. 
For each such string $x$, letting $t(x)$ denote the number of coordinates in $\bN$ with $x_i = 1$, the probability of $x$ in $\bp$ is
$$
\frac{1}{2^n}\cdot \left(1 + \frac{\eps}{n^{1/4}} \right)^{t(x)} \left( 1 - \frac{\eps}{n^{1/4}}\right)^{|\bN| - t(x)} 
=\frac{1}{2^n}\cdot \left(1 - \frac{\eps^2}{\sqrt{n}}\right)^{t(x)} \left(1 - \frac{\eps}{n^{1/4}}\right)^{|\bN| - 2t(x)}< \frac{1}{2^n}.
$$
As a result, we can bounded $\dtv(\bp,\calU_n)$ from below as follows:
\begin{align}
\dtv(\bp, \calU_n) %&\geq \frac{1}{2^n} \sum_{\substack{x \in \{-1,1\}^n \\ t(x)\le |\bN|/2}} \left( 1 - \prod_{i=1}^n \left( 1 + x_i \cdot \mu_i \right)\right) \nonumber \\
     &\geq \frac{1}{2^n} \sum_{\substack{x \in \{-1,1\}^n \\ t(x) \leq |\bN|/2}} \left( 1 - \left(1 - \frac{\eps^2}{\sqrt{n}}\right)^{t(x)} \left(1 - \frac{\eps}{n^{1/4}}\right)^{|\bN| - 2t(x)}\right) \nonumber\\[0.5ex]
    &\ge
     \frac{1}{2^n} \sum_{\substack{x \in \{-1,1\}^n \\ t(x) \leq |\bN|/2}} \left( 1 -  \left(1 - \frac{\eps}{n^{1/4}}\right)^{|\bN| - 2t(x)}\right).\label{eq:hehe1}
\end{align}
%In (\ref{eq:simp-1}), we expanded the product over $i \in [n]$ using the definition of $t(x)$. Note that the fact $t(x) < |\bN| / 2$ implies $t(x) \leq |\bN| - t(x)$ and thus,
On the other hand, using $|\bN| = \Omega(\sqrt{n})$, there is a constant probability over a uniform $\bx \sim \{-1,1\}^n$ that $t(\bx)$ is bounded away from $|\bN|/2$ by $\Omega(n^{1/4})$.
For every such $\bx$, we have 
\[ %\left(1 + \frac{\eps}{n^{1/4}}\right)^{t(x)} \left(1 - \frac{\eps}{n^{1/4}}\right) = 
 \left(1 - \frac{\eps}{n^{1/4}}\right)^{|\bN| - 2t(\bx)} \leq \left(1 - \frac{\eps}{n^{1/4}}\right)^{\Omega(n^{1/4})}\le e^{-\Omega(\eps)}\le 1-\Omega(\eps). \]
Combining everything we have from (\ref{eq:hehe1}) that
$$
\dtv(\bp,\calU_n)\ge \frac{1}{2^n}\cdot \Omega(2^n)\cdot \Omega(\eps)=\Omega(\eps).
$$
This finishes the proof of the lemma.
%We now lower bound (\ref{eq:simp-1}) using the above inequality, were we have (\ref{eq:simp-%1}) is at least
%\[ \frac{1}{2^n} \sum_{x \in \{-1,1\}^n} \frac{\eps}{n^{1/4}} \left(|\bN| - 2t(x) \right)^+ %= \frac{\eps}{n^{1/4}} \Ex_{\bx \sim \{-1,1\}^n}\left[ \left(|\bN| - 2t(\bx) %\right)^+\right], \]
%which is $\Omega(\eps)$, 
\end{proof}

\def\br{\mathbf{r}}

\subsection{Indistinguishability of $\Dno$ from the uniform distribution}

Consider the task of distinguishing via $q$ independent samples, $\bx_1, \dots, \bx_q \in \R^n$, whether these samples were drawn from the standard $n$-dimensional Gaussian $\calN(0, I)$, or an $n$-dimensional Gaussian $\calN(\bmu, I)$, where $\bmu \sim \calD$. We consider the above problem because of the following simple claim, which shows how to generate a product distribution whose mean vector has $i$-th coordinate $\Omega(\bmu_i)$.

\begin{claim}
Let $\sign\colon \R^n \to \{-1,1\}^n$ denote the function which applies $\sign(\cdot)$  coordinate-wise. 
\begin{flushleft}\begin{itemize}
\item The uniform distribution over $\{-1,1\}^n$ can be generated by sampling $\bx \sim \calN(0, I)$ and outputting $\sign(\bx)$.
\item For a fixed $\mu \in [0, 1/2]^n$, consider the product distribution over $\{-1,1\}^n$ generated by sampling $\bx \sim \calN( \mu, I)$ and outputting $\sign(\bx)$. Then, the mean vector of such a distribution has the $i$-th coordinate set to $\Omega(\mu_i)$.
\end{itemize}\end{flushleft}
\end{claim}

\begin{proof}
The first condition is by symmetry of the Gaussian distribution, and the second condition is by standard Gaussian anti-concentration, whenever $\mu \in [0, 1/2]$.
\end{proof}

Hence, it suffices to prove the following lemma.

\begin{lemma}\label{lem:mainlemma1}
% Let $\calD_n'$ denote the distribution over $n$-dimensional Gaussian  $\calN(\bmu, I)$ where $\bmu \sim \calD$. 
Consider an algorithm that takes $q$ samples $\bx_1, \ldots, \bx_q\in \R^n$ and satisfies the following guarantees:
\begin{flushleft}\begin{itemize}
\item \textbf{\emph{Standard Case}}: If $\bx_1,\dots, \bx_q \sim \calN(0, I)$ and the algorithm receives those samples, then the algorithm outputs ``standard'' with probability at least $0.99$.
\item \textbf{\emph{Non-Standard Case}}: We sample $\bmu \sim \calD$,\footnote{The distribution $\calD$ is defined in the first bullet point of Subsection~\ref{sec:dno-def}.} then $\bx_1,\dots, \bx_q \sim \calN(\bmu, I)$, and the algorithm receives those samples, the algorithm outputs ``not standard'' with probability at least $0.99$.
\end{itemize}\end{flushleft}
Then, the number of samples must satisfy $q = \tilde{\Omega}(\sqrt{n}/\eps^2)$.
\end{lemma}

\subsection{Proof of \Lem{mainlemma1}} \label{sec:proof-mono-lb}

We prove by contradiction. {We assume that $q\le  \sqrt{n}/(c\eps^2 \log^4n)$ for some sufficiently large constant $c$} and show below
that the algorithm cannot distinguish between the two cases.

We set up some notation for the proof. We use $i \in [q]$ as an index over the set of queries,
while $j \in [n]$ indexes the dimension/coordinate.
We write $f_y, f_n \colon (\R^n)^q \to \R_{\geq 0}$ to denote the probability density functions of a tuple of $q$ independent samples from $\calN(0, I)$ (for $f_y$) or $\calN(\bmu, I)$ with $\bmu \sim \calD$ (for $f_n$) given by
\begin{align*}
f_y(x_1, \dots, x_q) &= \frac{1}{(2\pi)^{n/2}} \prod_{j=1}^n  \exp\left(-\frac{1}{2} \sum_{i=1}^q x_{ij}^2 \right) \qquad \text{and}\quad \\[1ex]
f_{n}(x_1,\dots, x_q) &= \frac{1}{(2\pi)^{n/2}} \prod_{j=1}^n \exp\left(-\frac{1}{2} \sum_{i=1}^q x_{ij}^2 \right) \left(1 - \frac{1}{\sqrt{n}} + \frac{1}{\sqrt{n}} \cdot \exp\left( \frac{\eps}{n^{1/4}}\sum_{i=1}^q x_{ij} -  \frac{q\eps^2}{2\sqrt{n}} \right)\right).
\end{align*}
The definition of $f_y$ comes from the product of $q$ many $n$-dimensional Gaussian p.d.fs; the definition of $f_n$ comes from the fact that each coordinate $j$ behaves independently under the draw of $\bmu \sim \calD$: with probability $1/\sqrt{n}$, $\bmu_i = \eps / n^{1/4}$ and is otherwise 0.

%We use $\bx_{ij}$ to denote the $j$th coordinate
%of $\bx_i$. Note that $\bmu$ is a vector in $\R^n$, where the $j$th coordinate is set according
%to the distribution described earlier.
%We use $f_y, f_n \colon (\R^n)^q \to \R_{\geq 0}$ to denote the probability density functions of a tuple of $q$ independent samples from $\calN(0, I)$ (for $f_y$) or $\calN(\bmu, I)$ for $\bmu \sim \calD$ (for $f_n$). We have
%\begin{align*}
%f_y(\bx_1, \dots, \bx_q) &= \frac{1}{(2\pi)^{n/2}} \prod_{i=1}^q  \exp\left(-\|\bx_i\|^2_2/2 \right) \qquad \text{and}\\[0.5ex]
%f_n(\bx_1,\dots, \bx_q) &= \frac{1}{(2\pi)^{n/2}} \prod_{i=1}^q \exp\left(-\|\bx_i - \bmu\|^2_2/2 \right)
%\end{align*}
%Note that $\bmu$ is a random vector, based on the distribution $\cD$. For a fixed choice of $\bmu$,
%we have the above probability distribution defined over $(\bx_1, \bx_2, \ldots, \bx_q)$.
The main lemma below shows that these pdfs are nearly the same with high probability over draws $\bx_1,\ldots,\bx_q\sim \cN(0,I)$.
(Technically, we only need to lower bound $f_n$ by $f_y$.)

\begin{lemma} \label{lem:gaussian-calc} Consider $q$ independent draws of $\bx_i \sim \cN(0,I)$.
%(the $n$-dimensional
%Gaussian distribution).
With probability at least $1- o_n(1)$, 
$$\frac{f_n(\bx_1, \ldots, \bx_q)}{f_y(\bx_1, \ldots, \bx_q)} \geq 1-o_n(1) .$$
\end{lemma}

\begin{proof} 
%Let us take the ratio and perform some simple manipulations.
%\begin{align}
%&\frac{f_n(\bx_1, \bx_2, \ldots, \bx_q)}{f_y(\bx_1, \bx_2, \ldots, \bx_q)}
%= \frac{\prod_{i=1}^q \exp(-\|\bx_i - \bmu\|^2_2/2 )}{\prod_{i=1}^q  \exp(-\|\bx_i\|^2_2/2 ) }
%= \prod_{i=1}^q \exp(\bx_i \cdot \bmu - \|\bmu\|^2/2) \\
%&= \prod_{i=1}^q \exp(\sum_{j=1}^n x_{ij}\mu_j - \sum_{j=1}^n \mu^2_j/2) 
%= \exp(\sum_{i=1}^q \sum_{j=1}^n x_{ij} \mu_j - \sum_{i=1}^q \sum_{j=1}^n \mu^2_j/2)\\
%&= \exp\Big(\sum_{j=1}^n (\sum_{i=1}^q x_{ij}) \mu_j - \sum_{j=1}^n q\mu^2_j/2\Big)
%= \prod_{j=1}^n \exp(X_j \mu_j - q\mu^2_j/2)
%\end{align}
We set $\bX_j := \sum_{i=1}^q \bx_{ij}$ for each $j\in [n]$. %, which is the sum of $j$th coordinates over all the pdf domain (which corresponds to the
%queries). We now take the expectation over the bias vector $\bmu$. Recall that each coordinate $\mu_i$
%is generated independently. With probability $1/\sqrt{n}$, $\mu_i = \eps/n^{1/4}$ and zero otherwise.
%Using these facts, we get the following.
Then the ratio can be written as
\begin{align}
%\EX_{\bmu}
\frac{f_n(\bx_1, \ldots, \bx_q)}{f_y(\bx_1, \ldots, \bx_q)}
%&= \prod_{j=1}^n \EX_{\mu_j}\exp(X_j \mu_j - q\mu^2_j/2) \\
&= \prod_{j=1}^n\left( 1 + \frac{1}{\sqrt{n}}\left( \exp\left(\frac{\eps\bX_j}{n^{1/4}}  - \frac{q\eps^2}{2\sqrt{n}}\right)-1\right)\right)
%\\
%&= \prod_{j=1}^n \Big[ 1 + (1/\sqrt{n})(\exp(\eps X_j/n^{1/4} - q\eps^2/2\sqrt{n}) - 1)\Big]
\end{align}
and thus, 
\begin{align}
 \ln\left( \frac{f_n(\bx_1, \ldots, \bx_q)}{f_y(\bx_1,  \ldots, \bx_q)}\right) 
&= %\sum_{j=1}^n \ln\Big[ 1 + (1/\sqrt{n})(\exp(\eps X_j/n^{1/4} - q\eps^2/2\sqrt{n}) - 1)\Big] \\
  \sum_{j=1}^n \ln\left[ 1 + \frac{\bW_j}{\sqrt{n}}\right],
  \quad \text{with\ }\bW_j \eqdef \exp\left(\frac{\eps \bX_j}{n^{1/4}} - \frac{q\eps^2}{2\sqrt{n}}\right) - 1.
  \label{eq:wj}
\end{align}
%
%We define $$.
At this point, we use the distributional information of $\bx_1, \ldots, \bx_n\sim \cN(0,I)$:

\begin{claim} \label{clm:wj} With probability at least $1-1/n$, we have 
$|\bW_j| \leq 1/\log n$ for all $j \in [n]$.
\end{claim}

\begin{proof} Note that $\bX_j = \sum_{i=1}^q \bx_{ij}$ where each $\bx_{ij} \sim \cN(0,1)$. Hence, $\bX_j \sim \cN(0,q)$.
With probability at least $1-1/n^2$, we have $|\bX_j| \le 4\sqrt{q}\log n$. By a union bound over all coordinates,~with probability at least $1-1/n$, we have $|\bX_j| \le  4\sqrt{q}\log n$ for all $j\in [n]$.

When this is the case, using $ {q\le \sqrt{n}/(c\eps^2 \log^4n)}$
  we have (when $c$ is sufficiently large)
$$\frac{\eps |\bX_j|}{n^{1/4}} \le \frac{4\eps \sqrt{q} \log n }{n^{1/4}} \le \frac{1}{4\log n}\quad\text{and}\quad
\frac{q\eps^2}{2\sqrt{n}} \le \frac{1}{4\log n}.$$ 
%We deduce that $|\eps X_j/n^{1/4} - q\eps^2/2\sqrt{n}| \leq \eps|X_j|/n^{1/4} + q\eps^2/2\sqrt{n}
%< 1/2\log n$. 
Using $1/(1-z) \geq \exp(z) \geq 1+z$ for $|z|\le 1$, we have
$$ \exp\left(\frac{\eps \bX_j}{n^{1/4}} - \frac{q\eps^2}{2\sqrt{n}}\right) \geq 1-\frac{1}{2\log n}$$
and
$$ \exp\left(\frac{\eps \bX_j}{n^{1/4}} - \frac{q\eps^2}{2\sqrt{n}}\right) \leq \frac{ 1}{1-1/(2\log n)} \leq 1 + \frac{1}{\log n}.$$
So $|\bW_j|  \leq 1/\log n$ for all $j$ and the claim follows.
\end{proof}

We go back to \Eqn{wj}, and apply the inequality $\ln(1+z) \geq z-z^2$ for $|z| \leq 1/2$.
With probability at least $1-1/n$ over $\bx_1, \ldots, \bx_n\sim \cN(0,I)$, we have $|\bW_j|\le 1/\log n$ for all $j$ and thus,
\begin{align}
\ln\left(\frac{f_n(\bx_1,  \ldots, \bx_q)}{f_y(\bx_1,  \ldots, \bx_q)} \right)\nonumber
&\geq \frac{1}{\sqrt{n}}\sum_{j=1}^n \bW_j - \frac{1}{n} \sum_{j=1}^n \bW^2_j \\
&\geq \frac{1}{\sqrt{n}}\sum_{j=1}^n \bW_j - \frac{1}{\log^2 n} \ \ \ ~~~~~~~ \textrm{(by \Clm{wj}, $\bW^2_j \leq 1/\log^2 n$)}\nonumber \\
&= \frac{1}{\sqrt{n}} \left[ \exp\left(-\frac{q \eps^2}{2\sqrt{n}}\right) \sum_{j=1}^n \exp\left(\frac{\eps \bX_j}{n^{1/4}}\right) - n \right]  - \frac{1}{\log^2 n}. \label{eq:gaussian}
\end{align}
The heart of the matter is the
next claim on the distribution of sum of exponentials of Gaussians.

\begin{claim} \label{clm:exp-gauss} Let each $\bX_j \sim \cN(0,q)$ be independent.
With probability at least $ 1-1/\sqrt{\log n}$,  we have$$\sum_{j=1}^n \exp\left(\frac{\eps \bX_j}{n^{1/4}}\right) \geq n \cdot\exp\left(\frac{q \eps^2}{2\sqrt{n}}\right) - \frac{\sqrt{n}}{\log^{0.25} n}.$$
\end{claim}

\begin{proof} Denote $\bY_j = \eps \bX_j/n^{1/4}$ and consider the random variable $\bZ_j = \exp(\bY_j)$. 
Observe~that~$\bY_j$ $\sim \cN(0, q\eps^2/\sqrt{n})$. Using the formula for the moment generating function
    of the Gaussian~\cite{gaussian-wiki}, we have $\EX[\exp(t\bY_j)] = \exp(q\eps^2 t^2/(2\sqrt{n}))$.
Hence, $$\EX[\bZ_j] = \EX[\exp(\bY_j)] = \exp\left(\frac{q\eps^2}{2\sqrt{n}}\right)\quad\text{and}\quad \EX[\bZ^2_j] = \EX[\exp(2\bY_j)] = \exp\left(\frac{2q\eps^2}{\sqrt{n}}\right).$$
So $\textrm{var}[\bZ_j] = \exp(2q\eps^2/\sqrt{n}) - \exp(q\eps^2/\sqrt{n})\leq 1/\log n$ using  $q\eps^2/\sqrt{n} = o (1/\log n)$.
Overall, we have 
$$\EX\left[\sum_{j=1}^n \bZ_j\right] = n\cdot \exp\left(\frac{q\eps^2}{ 2\sqrt{n}}\right)\quad\text{and}\quad\var\left[\sum_{j=1}^n \bZ_j\right] \leq \frac{n}{\log n}$$ since
all $\bZ_j$'s are independent. By Chebyshev's inequality, we have 
$$\Pr\left[\left|\sum_{j=1}^n \bZ_j - n\cdot \exp\left(\frac{q\eps^2}{2\sqrt{n}}\right)\right| > \frac{\sqrt{n}}{\log^{0.25} n}\right] \leq \frac{1}{\sqrt{\log n}}.$$
This finishes the proof of the claim.
\end{proof}

%We pick up from \Eqn{gaussian}, and apply \Clm{exp-gauss}.  
In particular, with probability at least $1-1/n - 1/\sqrt{\log n} = 1-o_n(1)$, we get that
\begin{align*}
\exp\left(-\frac{q \eps^2}{2\sqrt{n}}\right) \sum_{j=1}^n \exp\left(\frac{\eps \bX_j}{n^{1/4}}\right) & \geq \exp\left(-\frac{q \eps^2}{2\sqrt{n}}\right)\cdot \left( n \cdot \exp\left(\frac{q \eps^2}{2\sqrt{n}}\right) - \frac{\sqrt{n}}{\log^{0.25} n}\right)  >  n - \frac{\sqrt{n}}{\log^{0.25} n} 
    \end{align*}
where in the last inequality we used $\exp(-q \eps^2/2\sqrt{n}) < 1$. Substituting in \Eqn{gaussian}, we get
\begin{align}
\ln\left(\frac{f_n(\bx_1, \ldots, \bx_q)}{f_y(\bx_1,  \ldots, \bx_q)} \right)
& > -\frac{ 1}{    \log^{0.25} n}  - \frac{1}{\log^2 n} \label{eq:18}
\end{align}
Hence, with probability at least $1-o_n(1)$,  we have
$$\frac{f_n(\bx_1,  \ldots, \bx_q)}{f_y(\bx_1, \ldots, \bx_q)} \geq \exp\left(-\frac{ 1}{ \log^{0.25} n} - \frac{1}{\log^2n}\right) = 1-o_n(1).$$
This finishes the proof of the lemma.
\end{proof}

We can now complete the proof of \Lem{mainlemma1}. 
Consider the set $Y \subset (\R^n)^{q}$ of tuples that lead  the algorithm to output ``standard.''
Then we must have $$\Prx_{ \bx_i \sim \cN(0,I)}\big[(\bx_1, \dots, \bx_q) \in Y\big] \geq 0.99.$$ Let $Y' \subseteq Y$
be the set of tuples that also satisfy the condition of \Lem{gaussian-calc}.
By a union bound $$\Prx_{ \bx_i \sim \cN(0,I)}\big[(\bx_1, \dots, \bx_q) \in Y'\big] \geq 0.99 - o_n(1) \geq 0.98.$$
Thus, $\int_{Y'} f_y(\bx_1, \dots, \bx_q) d \bx_1 d \bx_2 \ldots d \bx_q \geq 0.98$. 
By the condition of \Lem{gaussian-calc}, we have  $$\int_{Y'}  f_n(\bx_1, \dots, \bx_q)  d \bx_1 \ldots d \bx_q \geq (1-o_n(1))\cdot0.98 \geq 0.97.$$
%Since $f_n(\cdot)$ is absolutely bounded, by Fubini's theorem, we can switch the expectation and integral. So
%$\EX_{\mu} [\int_{Y'} f_n(\bx_1, \dots, \bx_q) d \bx_1 
This is exactly the probability that we see a tuple in $Y'$,
when we generate the samples $\bx_1, \cdots, \bx_q$ from the non-standard case. Thus,
with probability at least $0.97$, the algorithm outputs ``standard'' when the samples are generated
from the non-standard case. This completes the contradiction.

\begin{comment}
\subsection{The old proof}

\begin{proof} 
We write $f_y, f_n \colon (\R^n)^q \to \R_{\geq 0}$ to denote the probability density functions of a tuple of $q$ independent samples from $\calN(0, I)$ (for $f_y$) or $\calN(\bmu, I)$ for $\bmu \sim \calD$ (for $f_n$) given by
\begin{align*}
f_y(x_1, \dots, x_q) &= \frac{1}{(2\pi)^{n/2}} \prod_{k=1}^n  \exp\left(-\frac{1}{2} \sum_{i=1}^q x_{ik}^2 \right) \qquad \text{and}\qquad \\[1ex]
f_{n}(x_1,\dots, x_q) &= \frac{1}{(2\pi)^{n/2}} \prod_{k=1}^n \exp\left(-\frac{1}{2} \sum_{i=1}^q x_{ik}^2 \right) \left(1 - \frac{1}{\sqrt{n}} + \frac{1}{\sqrt{n}} \cdot \exp\left( \frac{\eps}{n^{1/4}}\sum_{i=1}^q x_{ik} -  \frac{q\eps^2}{2\sqrt{n}} \right)\right).
\end{align*}
The definition of $f_y$ comes from the product of $q$ many $n$-dimensional Gaussian p.d.fs, and the definition of $f_n$ comes from the fact that each coordinate $k \in [n]$ behaves independently under the draw of $\bmu \sim \calD$, and with probability $1/\sqrt{n}$, $\bmu_i = \eps / n^{1/4}$ and is otherwise 0.

Letting $s = \sum_{i=1}^q x_i$ denote the vector given by the sum and $\xi = \exp(-q\eps^2 / (2\sqrt{n}))$, we write
\begin{align*}
\dfrac{f_n(x_1, \dots, x_q)}{f_y(x_1, \dots, x_q)} &= \prod_{k=1}^n \left(1 + \frac{\xi}{\sqrt{n}} \left( \exp\left( \frac{\eps s_k}{n^{1/4}} \right) - \exp\left(\frac{q\eps^2}{2\sqrt{n}}\right)  \right) \right).
\end{align*}
Applying Taylor expansion, we have
$$
\dfrac{f_n(x_1, \dots, x_q)}{f_y(x_1, \dots, x_q)}  = \prod_{k=1}^n \left(1 + \frac{\xi}{\sqrt{n}} \left( 
\sum_{\text{$\ell$ odd}} \frac{1}{\ell!}\left(\frac{\eps s_k}{n^{1/4}}\right)^\ell
+\sum_{\ell=1}^\infty \left(\frac{1}{(2\ell)!}\left(\frac{\eps s_k}{n^{1/4}}\right)^{2\ell}
-\frac{1}{\ell!}\left(\frac{q\eps^2}{2\sqrt{n}}\right)^\ell
\right)\right)\right).
$$
For each $k$, we write $A_k$ and $B_k$ to denote
\begin{align*}
A_k  \eqdef \frac{\xi}{\sqrt{n}}\sum_{\ell \text{ odd}} \frac{1}{\ell!} \left( \frac{\eps s_k}{n^{1/4}}\right)^{\ell} \quad\text{and}\quad 
B_k  \eqdef \frac{\xi}{\sqrt{n}} \sum_{\ell = 1}^{\infty} \left(\frac{\eps^2}{\sqrt{n}} \right)^{\ell} \left(\frac{ s_k^{2\ell}}{(2\ell)!} - \frac{ q^{\ell}}{2^{\ell}\ell!} \right),
\end{align*}
and $A=\sum_{k=1}^n A_k$ and $B=\sum_{k=1}^n B_k$:
\begin{align*}
    A  \eqdef \frac{\xi}{\sqrt{n}}\sum_{\ell \text{ odd}} \frac{1}{\ell!} \left( \frac{\eps}{n^{1/4}}\right)^{\ell} \sum_{k=1}^n s_k^{\ell}\quad\text{and}\quad
    B  \eqdef \frac{\xi}{\sqrt{n}} \sum_{\ell = 1}^{\infty} \left(\frac{\eps^2}{\sqrt{n}} \right)^{\ell} \left(\frac{1}{(2\ell)!}\sum_{k=1}^{n} s_k^{2\ell} - \frac{nq^{\ell}}{2^{\ell}\ell!} \right).
\end{align*}
Then we have 
\begin{equation}\label{eq:haha2}
\dfrac{f_n(x_1, \dots, x_q)}{f_y(x_1, \dots, x_q)}
=\prod_{k=1}^n \left(1+A_k+B_k\right)
\le \exp\big(A+B\big)  
\end{equation}
using $1+x\le e^x$.
We finish the proof using the following claim, which
  shows that $A$ and $B$ are both small with high probability when $\bx_1,\ldots,\bx_q$ are drawn
  from the non-standard case:

\begin{claim}\label{mainclaim1}
When $\bmu\sim \calD$ and $\bx_1,\ldots,\bx_q\sim \calN(\bmu, I)$ independently,
  we have both $A$ and $B$ are $o_n(1)$ with probability at least $1-o_n(1)$.
\end{claim}

We delay the proof of Claim \ref{mainclaim1} to the end and use it to 
  finishes the proof of Lemma \ref{mainlemma1}.

Let the set $N \subset (\R^n)^{q}$ denote the subset of tuples of samples $(x_1, \dots, x_q)$ such that, if the algorithm observes samples $(x_1, \dots, x_q)\in N$, it declares ``not standard.'' Then, we note that by assumption of the algorithm, a draw of $\bx_1, \dots, \bx_q \sim \calN(\bmu, I)$ with $\bmu\sim \calD$ must have $(\bx_1,\dots, \bx_q) \in N$ with probability at least $0.99$. Furthermore, let $N' \subseteq N$ be the set of tuples $(x_1, \dots, x_q)$ in $N$ such that $A$ and $B$ derived from it satisfy the condition of  Claim~\ref{mainclaim1}. On the one hand, we have from Claim~\ref{mainclaim1} and 
  a union bound that 
\begin{align*}
\Prx_{\substack{\bmu\sim \calD\\
\bx_1,\dots, \bx_q \sim \calN(\bmu, I)}}\left[ (\bx_1,\dots, \bx_q) \in N' \right] \geq 0.99-o_n(1)>0.98.
\end{align*}
On the other hand, for every $(x_1,\ldots,x_q)\in N'$ we have from (\ref{eq:haha2}) that
$$
\frac{f_n(x_1, \dots, x_n) } {f_y(x_1, \dots, x_n)}\le 
\exp\big(1+o_n(1)\big)\le 1+o_n(1).
$$ 
As a result, we have 
$$
\Prx_{\bx_1,\ldots,\bx_q\sim \calN(0,I)}\left[(\bx_1,\ldots,\bx_q)\in N'\right]
\ge (1-o_n(1))\cdot \Prx_{\substack{\bmu\sim \calD\\
\bx_1,\dots, \bx_q \sim \calN(\bmu, I)}}\left[ (\bx_1,\dots, \bx_q) \in N' \right] 
>0.97.
$$
However, this means
%but since $f_n(x_1, \dots, x_n) \geq f_y(x_1, \dots, x_n) (1 - o(1))$ for every $(x_1, \dots, x_n) \in A'$, we must have
%\begin{align*}
%\Prx_{\substack{\bmu \sim \calD \\ \bx_1,\dots, \bx_q \sim \calN(\bmu, I)}}\left[ (\bx_1,\dots, \bx_q) \in A' \right] \geq 0.98 \cdot (1 - o(1)) \geq 0.97.
%\end{align*}
  that the algorithm outputs ``not standard'' with probability at least $0.97$ when samples are drawn from $\calN(0,I)$, which contradicts the assumption that the algorithm outputs ``not standard'' with probability $0.99$ if the distribution is $\calN(0,I)$.
  %for $\bmu \sim \calD$.
%We will divide the computation in the following way. We show that under a certain event (which occurs often over draw of $\bx_1,\dots, \bx_q$), we have the following two inequalities holding:
%Once we establish both, then we can lower bound the ratio by expanding the Taylor expansion of both instances of $\exp(\cdot)$, and taking a union bound
%\begin{align*}
%    \dfrac{f_n(x_1, \dots, x_q)}{f_y(x_1, \dots, x_q)} \geq 1 - |A| - |B| \geq 1 - o(1).
%\end{align*}
\end{proof}

Finally we prove Claim \ref{mainclaim1}:

\begin{proof}[Proof of Claim \ref{mainclaim1}]
First, with probability at least $1-o_n(1)$, we have that the number of $i\in [n]$
  with $\bmu_i=\eps/n^{1/4}$ is at most $O(\sqrt{n})$.
Fix such a $\mu$ and without less of generality, we assume that 
  $\mu_1,\ldots,\mu_m=0$ and $\mu_{m+1},\ldots, \mu_n=\eps/n^{1/4}$
  for some $m$ with $n-m=O(\sqrt{n})$.
Then $\bs_k$ is drawn from $\calN(0,q)$ for $k\le m$ and 
  $\bs_{k}$ is drawn from $\calN(q\eps/n^{1/4},q)$ for $k>m$.
We also break $A,B$ into $A',A''$ and $B',B''$ accordingly, where
\begin{align*}
A'  &\eqdef \frac{\xi}{\sqrt{n}}\sum_{\ell \text{ odd}} \frac{1}{\ell!} \left( \frac{\eps}{n^{1/4}}\right)^{\ell} \sum_{k=1}^m \bs_k^{\ell}\\[0.6ex]
A''  &\eqdef \frac{\xi}{\sqrt{n}}\sum_{\ell \text{ odd}} \frac{1}{\ell!} \left( \frac{\eps}{n^{1/4}}\right)^{\ell} \sum_{k=m+1}^n \bs_k^{\ell} \\[0.6ex]
B'  &\eqdef \frac{\xi}{\sqrt{n}} \sum_{\ell = 1}^{\infty} \left(\frac{\eps^2}{\sqrt{n}} \right)^{\ell} \left(\frac{1}{(2\ell)!}\sum_{k=1}^{m} \bs_k^{2\ell} - \frac{mq^{\ell}}{2^{\ell}\ell!} \right)\quad \text{and}\qquad\\[0.6ex]
B''  &\eqdef \frac{\xi}{\sqrt{n}} \sum_{\ell = 1}^{\infty} \left(\frac{\eps^2}{\sqrt{n}} \right)^{\ell} \left(\frac{1}{(2\ell)!}\sum_{k=m+1}^{n} \bs_k^{2\ell} - \frac{(n-m)q^{\ell}}{2^{\ell}\ell!} \right).
\end{align*}
It suffices to show that $|A'|,|A''|,|B'|$ and $|B''|$ are all 
  $o_n(1)$ with probability $1-o_n(1)$.
For $A''$ and $B''$, we have from a standard union bound that with probability
  at least $1-o_n(1)$, every $\bs_k$ is 
  at most 
$$
\frac{q\eps}{n^{1/4}}+O\left(\sqrt{q\log n}\right)=O\left(\sqrt{q\log n}\right)
$$
for every $k>m$ using our choice of $q$.
When this holds for every $\bs_k$, $k>m$, we have
$$
|A''|\le \frac{1}{\sqrt{n}}\sum_{\text{$\ell$ odd}}\frac{1}{\ell!}
  \left(\frac{\eps}{n^{1/4}}\right)^\ell \cdot O(\sqrt{n})\cdot 
  \left(O\left(\sqrt{q\log n}\right)\right)^\ell=o_n(1),
$$
and 
$$
|B''| \le  \frac{1}{\sqrt{n}} \sum_{\ell = 1}^{\infty} \left(\frac{\eps^2}{\sqrt{n}} \right)^{\ell} \left(\frac{1}{(2\ell)!}\cdot O(\sqrt{n})\cdot 
\left(O\left(\sqrt{q\log n}\right)\right)^\ell+ \frac{1}{2^\ell \ell!}\cdot O(\sqrt{n})\cdot  q^{\ell} \right)=o_n(1).
$$

We bound $|A'|$ and $|B'|$ in the rest of the proof, where 
  we need to be more careful and take advantage of cancellations in the sums.
Below $k$ will always denote an index in $[m]$ and $\bs_k$ is always 
  drawn independently from $\calN(0,q)$.
We start with the following claim about expectations of $\bs_k^\ell$.

\begin{claim}\label{cl:exp}
For every $k\in [m]$ and $\ell \in \N$, the expectation of $\bs_k^{\ell}$ is zero if $\ell$ is odd, and
\begin{align*}
    \frac{1}{(2\ell)!} \Ex_{\bs_k\sim \calN(0, q)}\left[ \bs_k^{2\ell} \right] = \dfrac{q^{\ell}}{2^{\ell} \ell!}.
\end{align*}
\end{claim}

\begin{proof}
    The fact that odd moments are zero follows from the fact $\calN(0, q)$ is symmetric. Then, recall that the $2\ell$-th moment of the Gaussian distribution with variance $q$ is exactly $q^{\ell} (2\ell-1)!!$,    where the double-factorial $n!!$ is the product of all numbers from $1$ to $n$ which share the same parity as $n$. Evaluating $(2\ell-1)!! / (2\ell)!$, we obtain $1/(2\ell)!!$, which is $2 \cdot 4 \cdot 6 \cdot \dots \cdot 2\ell = \ell! \cdot 2^{\ell}$.
\end{proof}

We use the following claim to bound $|A'|$ and $|B'|$, where we note that $m\le n$:

\begin{claim}\label{cl:good-s}
With probability $1-o_n(1)$, 
$\bs_1,\ldots,\bs_m\sim \calN(0,q)$ satisfy the following two conditions:
\begin{itemize}
    \item For every odd $\ell \geq 1$, we have
\begin{align*}
\left| \sum_{k=1}^m \bs_k^{\ell} \right| \leq \left(\sqrt{n} \cdot q^{\ell / 2} \cdot \sqrt{(2\ell - 1)!!}\right) \cdot (\log n)^{\ell};
\end{align*}
\item For every $\ell \geq 1$, we have
\begin{align*}
    \left|\frac{1}{(2\ell)!} \sum_{k=1}^{m} \bs_k^{2\ell} - \frac{mq^{\ell}}{2^{\ell} \ell!} \right| \leq \left(\sqrt{n} \cdot q^{\ell} \cdot 2^{\ell} \right)\cdot (\log n)^{\ell}.
\end{align*}
\end{itemize}
\end{claim}

\begin{proof}
Since $\bs_k\sim \calN(0,q)$, each sum of $\bs_k^{\ell}$ has expectation $0$ by Claim~\ref{cl:exp}. We apply Chebyshev's inequality, and we have
\begin{align*}
\Prx_{\bs_1, \dots, \bs_m \sim \calN(0, q)}\left[ \left| \sum_{k=1}^m \bs_k^{\ell} \right| \geq v \right] \leq \frac{n}{v^2}  \cdot \Ex_{\bs \sim \calN(0, q)}\left[ \bs^{2\ell} \right] = \frac{n \cdot q^{\ell}}{v^2} \cdot (2\ell - 1)!! \leq \frac{1}{(\log n)^{2\ell}},
\end{align*}
for the setting of $$v = \left(\sqrt{n} \cdot q^{\ell/2} \cdot \sqrt{(2\ell - 1)!!}\right) \cdot (\log n)^{\ell}.$$ 

Now for the second item,
  consider any $\ell \in \N$, and note that by Claim~\ref{cl:exp}, the expectation of the expression is also zero, so we apply Chebyshev's inequality once more. We have
\begin{align*}
    \Prx_{\bs_1,\dots, \bs_m \sim \calN(0,q)}\left[ \left|\frac{1}{(2\ell)!} \sum_{k=1}^m \bs_k^{2\ell} - \frac{mq^{\ell}}{2^{\ell} \ell!}\right|  \geq v \right] &\leq \frac{n}{v^2} \cdot \Ex_{\bs \sim \calN(0,q)}\left[\left(\frac{\bs^{2\ell}}{(2\ell)!} - \frac{q^{\ell}}{2^{\ell}\ell!} \right)^2\right] \\[0.3ex]
        &\leq \frac{n}{v^2} \cdot \left(\frac{1}{(2\ell)!}\right)^2 \Ex_{\bs\sim\calN(0,q)}\left[\bs^{4\ell} \right] \\[0.6ex]
        &\leq \frac{n}{v^2} \cdot \left(\frac{1}{(2\ell)!}\right)^2 q^{2\ell} \cdot (4\ell-1)!! \leq \frac{nq^{2\ell}}{v^2} \cdot 2^{2\ell} \leq \frac{1}{(\log n)^{2\ell}},
\end{align*}
for the setting of 
$$
v=\left(\sqrt{n} \cdot q^{\ell} \cdot 2^{\ell} \right)\cdot (\log n)^{\ell}.
$$
The final simplification comes from the fact the square of $(2\ell)!$ multiplies every integer between $1$ and $2\ell$ twice, and $(4\ell - 1)!!$ multiplies every odd integer between $1$ and $4\ell-1$ (a total of at most $2\ell$ of them), which is at most twice an integer in multiplied in $(2\ell)!$; a very loose bound gives $(4
\ell-1)!! \leq ((2\ell)!)^2 \cdot 2^{2\ell}$.
%Setting $v = \left(\sqrt{n} \cdot q^{\ell} \cdot 2^{\ell} \right) \cdot (\log n)^{\ell}$. 
By a union bound over all terms, the probability that some inequality fails to be satisfied is at most $\sum_{\ell \in \N} 2 / (\log n)^{2\ell} = o_n(1)$.
\end{proof}

Consider any setting of $s_1,\dots, s_m$ where the event of Claim~\ref{cl:good-s} is satisfied, then
\begin{align*}
    |A'|  &\leq \frac{\xi}{\sqrt{n}} \sum_{\ell \text{ odd}} \frac{1}{\ell!}\left(\frac{\eps}{n^{1/4}} \right)^{\ell} \left(\sqrt{n} \cdot q^{\ell/2} \sqrt{(2\ell-1)!!} \right) \cdot (\log n)^{\ell} \leq  \sum_{\ell \text{ odd}} \left(\frac{2\eps \sqrt{q} \cdot \log n}{n^{1/4}} \right)^{\ell}\quad \text{and} \\[0.8ex]
    |B'| &\leq \frac{\xi}{\sqrt{n}} \sum_{\ell=1}^{\infty} \left(\frac{\eps^2}{\sqrt{n}}\right)^{\ell} \cdot \left( \sqrt{n} \cdot q^{\ell} \cdot 2^{\ell} \right) \cdot (\log n)^{\ell} \leq \sum_{\ell=1}^{\infty} \left(\frac{2\eps^2 q\cdot  \log n}{\sqrt{n}} \right)^{\ell},
\end{align*}
and both are $o_n(1)$ when $q$ is a sufficiently small constant smaller than $\sqrt{n} / (\eps^2 \log^2 n)$. 
\end{proof}
\end{comment}



\bibliographystyle{alpha}
\bibliography{waingarten,monotonicity-full}

\end{document}
