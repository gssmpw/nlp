\section{Discussion}\label{sec::conclusion}

%\noindent\textbf{Conclusion.} In this paper, 
We introduce \method, a feed-forward model that can infer high-quality camera poses, geometry, and appearance from sparse-view uncalibrated images within 0.5 seconds.
%
We propose a novel cascade learning paradigm that progressively estimates camera poses, geometry, and appearance, leading to substantial improvements over previous methods.
%
Our model, trained on a set of public datasets, learns strong reconstruction priors that generalize robustly to challenging scenarios, such as very sparse views captured in real-world settings, enabling photo-realistic novel view synthesis.


% \subsection{Limitations and Future Work.}
% Our approach cannot reconstruct thin structures perfectly. Additionally, camera pose estimation becomes not very accurate when dealing with out-of-distribution camera trajectories. For future work, we plan to explore advanced multi-scale geometry representations to better handle thin structures, and incorporate out-of-distribution algorithms in pose estimation to improve robustness across diverse camera trajectories.

\section*{Acknowledgements}
We gratefully acknowledge Tao Xu for his assistance with the evaluation, Yuanbo Xiangli for insightful discussions, and Xingyi He for his Blender visualization code.
This work was partially supported by NSFC (No. U24B20154), Zhejiang Provincial Natural Science Foundation of China (No. LR25F020003), Ant Group, and Information Technology Center and State Key Lab of CAD\&CG, Zhejiang University.