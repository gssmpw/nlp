\documentclass[10pt,conference]{IEEEtran}
\IEEEoverridecommandlockouts
% The preceding line is only needed to identify funding in the first footnote. If that is unneeded, please comment it out.
%Template version as of 6/27/2024

\usepackage{cite}
\usepackage{amsmath,amssymb,amsfonts}
\usepackage{algorithmic}
\usepackage{graphicx}
\usepackage{textcomp}
\usepackage{tcolorbox}
\usepackage{todonotes}
\usepackage{url}
\usepackage{listings}
\usepackage{booktabs}
\usepackage{xspace}
\usepackage{hyperref}

\newcommand{\change}[1]{\textcolor{red}{#1}\xspace}

\lstdefinestyle{mystyle}{
%    backgroundcolor=\color{backcolour},   
%    commentstyle=\color{codegreen},
 %   keywordstyle=\color{magenta},
 %   numberstyle=\tiny\color{codegray},
 %   stringstyle=\color{codepurple},
    basicstyle=\ttfamily\footnotesize,
 %   breakatwhitespace=false,         
 %   breaklines=true,                 
    captionpos=b,                    
 %   keepspaces=true,                 
    numbers=left,                    
    numbersep=5pt,                  
 %   showspaces=false,                
 %   showstringspaces=false,
 %   showtabs=false,                  
    tabsize=2
}
\lstset{style=mystyle}
\newcommand{\urlRepo}[0]{\url{https://osf.io/nw8vp/?view_only=a4654139dce54584ab5e1549e56dff3c}\xspace} 

\usepackage{xcolor}
\def\BibTeX{{\rm B\kern-.05em{\sc i\kern-.025em b}\kern-.08em
    T\kern-.1667em\lower.7ex\hbox{E}\kern-.125emX}}

    
\begin{document}

\title{Studying How Configurations Impact Code Generation in LLMs: the Case of ChatGPT
%{\footnotesize \textsuperscript{*}Note: Sub-titles are not captured for https://ieeexplore.ieee.org  and
%should not be used}
%\thanks{Identify applicable funding agency here. If none, delete this.}
}
\author{\IEEEauthorblockN{Benedetta Donato}
\IEEEauthorblockA{
\textit{University of Milano-Bicocca}\\
Milano, Italy \\
benedetta.donato@unimib.it}
\and
\IEEEauthorblockN{Leonardo Mariani}
\IEEEauthorblockA{
\textit{University of Milano-Bicocca}\\
Milano, Italy \\
leonardo.mariani@unimib.it}
 \and
%\linebreakand
%\hspace{-1.0cm} % Adjust spacing between authors if needed
\IEEEauthorblockN{Daniela Micucci}
\IEEEauthorblockA{
\textit{University of Milano-Bicocca}\\
Milano, Italy \\
daniela.micucci@unimib.it}
\and
\IEEEauthorblockN{Oliviero Riganelli}
\IEEEauthorblockA{
\textit{University of Milano-Bicocca}\\
Milano, Italy \\
oliviero.riganelli@unimib.it}

}
%\author{\IEEEauthorblockN{Anonymous Authors}}
%\author{\IEEEauthorblockN{1\textsuperscript{st} Given Name Surname}
%\IEEEauthorblockA{\textit{dept. name of organization (of Aff.)} \\
%\textit{name of organization (of Aff.)}\\
%City, Country \\
%email address or ORCID}
%\and
%\IEEEauthorblockN{2\textsuperscript{nd} Given Name Surname}
%\IEEEauthorblockA{\textit{dept. name of organization (of Aff.)} \\
%\textit{name of organization (of Aff.)}\\
%City, Country \\
%email address or ORCID}
%\and
%\IEEEauthorblockN{3\textsuperscript{rd} Given Name Surname}
%\IEEEauthorblockA{\textit{dept. name of organization (of Aff.)} \\
%\textit{name of organization (of Aff.)}\\
%City, Country \\
%email address or ORCID}
%\and
%\IEEEauthorblockN{4\textsuperscript{th} Given Name Surname}
%\IEEEauthorblockA{\textit{dept. name of organization (of Aff.)} \\
%\textit{name of organization (of Aff.)}\\
%City, Country \\
%email address or ORCID}
%\and
%\IEEEauthorblockN{5\textsuperscript{th} Given Name Surname}
%\IEEEauthorblockA{\textit{dept. name of organization (of Aff.)} \\
%\textit{name of organization (of Aff.)}\\
%City, Country \\
%email address or ORCID}
%\and
%\IEEEauthorblockN{6\textsuperscript{th} Given Name Surname}
%\IEEEauthorblockA{\textit{dept. name of organization (of Aff.)} \\
%\textit{name of organization (of Aff.)}\\
%City, Country \\
%email address or ORCID}
%}

\maketitle


\begin{abstract}
Leveraging LLMs for code generation is becoming increasingly common, as tools like ChatGPT can suggest method implementations with minimal input, such as a method signature and brief description. Empirical studies further highlight the effectiveness of LLMs in handling such tasks, demonstrating notable performance in code generation scenarios.
%Writing code assisted by an LLM is becoming common practice. For instance, tools like ChatGPT can suggest the implementation of a method by simply accessing the method's signature and a short description of the expected behavior. Several empirical studies report the interesting performance of LLMs in code generation tasks.

However, LLMs are inherently non-deterministic, with their output influenced by parameters such as temperature, which regulates the model's level of creativity, and top-p, which controls the choice of the tokens that shall appear in the output. Despite their significance, the role of these parameters is often overlooked.%The role of these parameters is unfortunately underrated. 

This paper systematically studies the impact of these parameters, as well as the number of prompt repetitions required to account for non-determinism,
%of times a prompt should be repeatedly submitted to account for non-determinism, 
in the context of 548 Java methods. We observe significantly different performances across different configurations of ChatGPT, with temperature having a marginal impact compared to the more prominent influence of the top-p parameter. Additionally, we show how creativity can enhance code generation tasks. Finally, we provide concrete recommendations for addressing the non-determinism of the model.
%revealing a marginal impact of the temperature, in favor of a much more prominent role of parameter top-p. Further, we show how creativity can be beneficial in code generation tasks. Finally, we provide concrete recommendations on how to deal with the non-determinism of the model. 

\end{abstract}

\begin{IEEEkeywords}
LLMs, code generation, ChatGPT, temperature, top-p, repetitions
\end{IEEEkeywords}

\section{Introduction} \label{sec:introduction}
\section{Introduction}
\label{sec:introduction}
The business processes of organizations are experiencing ever-increasing complexity due to the large amount of data, high number of users, and high-tech devices involved \cite{martin2021pmopportunitieschallenges, beerepoot2023biggestbpmproblems}. This complexity may cause business processes to deviate from normal control flow due to unforeseen and disruptive anomalies \cite{adams2023proceddsriftdetection}. These control-flow anomalies manifest as unknown, skipped, and wrongly-ordered activities in the traces of event logs monitored from the execution of business processes \cite{ko2023adsystematicreview}. For the sake of clarity, let us consider an illustrative example of such anomalies. Figure \ref{FP_ANOMALIES} shows a so-called event log footprint, which captures the control flow relations of four activities of a hypothetical event log. In particular, this footprint captures the control-flow relations between activities \texttt{a}, \texttt{b}, \texttt{c} and \texttt{d}. These are the causal ($\rightarrow$) relation, concurrent ($\parallel$) relation, and other ($\#$) relations such as exclusivity or non-local dependency \cite{aalst2022pmhandbook}. In addition, on the right are six traces, of which five exhibit skipped, wrongly-ordered and unknown control-flow anomalies. For example, $\langle$\texttt{a b d}$\rangle$ has a skipped activity, which is \texttt{c}. Because of this skipped activity, the control-flow relation \texttt{b}$\,\#\,$\texttt{d} is violated, since \texttt{d} directly follows \texttt{b} in the anomalous trace.
\begin{figure}[!t]
\centering
\includegraphics[width=0.9\columnwidth]{images/FP_ANOMALIES.png}
\caption{An example event log footprint with six traces, of which five exhibit control-flow anomalies.}
\label{FP_ANOMALIES}
\end{figure}

\subsection{Control-flow anomaly detection}
Control-flow anomaly detection techniques aim to characterize the normal control flow from event logs and verify whether these deviations occur in new event logs \cite{ko2023adsystematicreview}. To develop control-flow anomaly detection techniques, \revision{process mining} has seen widespread adoption owing to process discovery and \revision{conformance checking}. On the one hand, process discovery is a set of algorithms that encode control-flow relations as a set of model elements and constraints according to a given modeling formalism \cite{aalst2022pmhandbook}; hereafter, we refer to the Petri net, a widespread modeling formalism. On the other hand, \revision{conformance checking} is an explainable set of algorithms that allows linking any deviations with the reference Petri net and providing the fitness measure, namely a measure of how much the Petri net fits the new event log \cite{aalst2022pmhandbook}. Many control-flow anomaly detection techniques based on \revision{conformance checking} (hereafter, \revision{conformance checking}-based techniques) use the fitness measure to determine whether an event log is anomalous \cite{bezerra2009pmad, bezerra2013adlogspais, myers2018icsadpm, pecchia2020applicationfailuresanalysispm}. 

The scientific literature also includes many \revision{conformance checking}-independent techniques for control-flow anomaly detection that combine specific types of trace encodings with machine/deep learning \cite{ko2023adsystematicreview, tavares2023pmtraceencoding}. Whereas these techniques are very effective, their explainability is challenging due to both the type of trace encoding employed and the machine/deep learning model used \cite{rawal2022trustworthyaiadvances,li2023explainablead}. Hence, in the following, we focus on the shortcomings of \revision{conformance checking}-based techniques to investigate whether it is possible to support the development of competitive control-flow anomaly detection techniques while maintaining the explainable nature of \revision{conformance checking}.
\begin{figure}[!t]
\centering
\includegraphics[width=\columnwidth]{images/HIGH_LEVEL_VIEW.png}
\caption{A high-level view of the proposed framework for combining \revision{process mining}-based feature extraction with dimensionality reduction for control-flow anomaly detection.}
\label{HIGH_LEVEL_VIEW}
\end{figure}

\subsection{Shortcomings of \revision{conformance checking}-based techniques}
Unfortunately, the detection effectiveness of \revision{conformance checking}-based techniques is affected by noisy data and low-quality Petri nets, which may be due to human errors in the modeling process or representational bias of process discovery algorithms \cite{bezerra2013adlogspais, pecchia2020applicationfailuresanalysispm, aalst2016pm}. Specifically, on the one hand, noisy data may introduce infrequent and deceptive control-flow relations that may result in inconsistent fitness measures, whereas, on the other hand, checking event logs against a low-quality Petri net could lead to an unreliable distribution of fitness measures. Nonetheless, such Petri nets can still be used as references to obtain insightful information for \revision{process mining}-based feature extraction, supporting the development of competitive and explainable \revision{conformance checking}-based techniques for control-flow anomaly detection despite the problems above. For example, a few works outline that token-based \revision{conformance checking} can be used for \revision{process mining}-based feature extraction to build tabular data and develop effective \revision{conformance checking}-based techniques for control-flow anomaly detection \cite{singh2022lapmsh, debenedictis2023dtadiiot}. However, to the best of our knowledge, the scientific literature lacks a structured proposal for \revision{process mining}-based feature extraction using the state-of-the-art \revision{conformance checking} variant, namely alignment-based \revision{conformance checking}.

\subsection{Contributions}
We propose a novel \revision{process mining}-based feature extraction approach with alignment-based \revision{conformance checking}. This variant aligns the deviating control flow with a reference Petri net; the resulting alignment can be inspected to extract additional statistics such as the number of times a given activity caused mismatches \cite{aalst2022pmhandbook}. We integrate this approach into a flexible and explainable framework for developing techniques for control-flow anomaly detection. The framework combines \revision{process mining}-based feature extraction and dimensionality reduction to handle high-dimensional feature sets, achieve detection effectiveness, and support explainability. Notably, in addition to our proposed \revision{process mining}-based feature extraction approach, the framework allows employing other approaches, enabling a fair comparison of multiple \revision{conformance checking}-based and \revision{conformance checking}-independent techniques for control-flow anomaly detection. Figure \ref{HIGH_LEVEL_VIEW} shows a high-level view of the framework. Business processes are monitored, and event logs obtained from the database of information systems. Subsequently, \revision{process mining}-based feature extraction is applied to these event logs and tabular data input to dimensionality reduction to identify control-flow anomalies. We apply several \revision{conformance checking}-based and \revision{conformance checking}-independent framework techniques to publicly available datasets, simulated data of a case study from railways, and real-world data of a case study from healthcare. We show that the framework techniques implementing our approach outperform the baseline \revision{conformance checking}-based techniques while maintaining the explainable nature of \revision{conformance checking}.

In summary, the contributions of this paper are as follows.
\begin{itemize}
    \item{
        A novel \revision{process mining}-based feature extraction approach to support the development of competitive and explainable \revision{conformance checking}-based techniques for control-flow anomaly detection.
    }
    \item{
        A flexible and explainable framework for developing techniques for control-flow anomaly detection using \revision{process mining}-based feature extraction and dimensionality reduction.
    }
    \item{
        Application to synthetic and real-world datasets of several \revision{conformance checking}-based and \revision{conformance checking}-independent framework techniques, evaluating their detection effectiveness and explainability.
    }
\end{itemize}

The rest of the paper is organized as follows.
\begin{itemize}
    \item Section \ref{sec:related_work} reviews the existing techniques for control-flow anomaly detection, categorizing them into \revision{conformance checking}-based and \revision{conformance checking}-independent techniques.
    \item Section \ref{sec:abccfe} provides the preliminaries of \revision{process mining} to establish the notation used throughout the paper, and delves into the details of the proposed \revision{process mining}-based feature extraction approach with alignment-based \revision{conformance checking}.
    \item Section \ref{sec:framework} describes the framework for developing \revision{conformance checking}-based and \revision{conformance checking}-independent techniques for control-flow anomaly detection that combine \revision{process mining}-based feature extraction and dimensionality reduction.
    \item Section \ref{sec:evaluation} presents the experiments conducted with multiple framework and baseline techniques using data from publicly available datasets and case studies.
    \item Section \ref{sec:conclusions} draws the conclusions and presents future work.
\end{itemize}



\section{Methodology} \label{sec:methodology}
\section{Methodology}
\subsection{Preliminary}
\label{sec:preliminary}
\mypara{Architecture of MLLM.}
% The MLLM architectures generally consist of three components: a visual encoder, a modality projector, and a LLM. The visual encoder, typically a pre-trained image encoder like CLIP's vision model, converts input images into visual tokens. The projector module aligns these visual tokens with the LLM's word embedding space, enabling the LLM to process visual data effectively. The LLM then integrates the aligned visual and textual information to generate responses.
The architecture of Multimodal Large Language Models (MLLMs) typically comprises three core components: a visual encoder, a modality projector, and a language model (LLM). Given an image $I$, the visual encoder and a subsequent learnable MLP are used to encode $I$ into a set of visual tokens $e_v$. These visual tokens $e_v$ are then concatenated with text tokens $e_t$ encoded from text prompt $p_t$, forming the input for the LLM. The LLM decodes the output tokens $y$ sequentially, which can be formulated as:
\begin{equation}
\label{eq1}
    y_i = f(I, p_t, y_0, y_1, \cdots, y_{i-1}).
\end{equation}

\mypara{Computational Complexity.}  
To evaluate the computational complexity of MLLMs, it is essential to analyze their core components, including the self-attention mechanism and the feed-forward network (FFN). The total floating-point operations (FLOPs) required can be expressed as:  
\begin{equation}
\text{Total FLOPs} = T \times (4nd^2 + 2n^2d + 2ndm),
\end{equation}  
where $T$ denotes the number of transformer layers, $n$ is the sequence length, $d$ represents the hidden dimension size, and $m$ is the intermediate size of the FFN.  
This equation highlights the significant impact of sequence length $n$ on computational complexity. In typical MLLM tasks, the sequence length is defined as: 
\begin{equation}
    n = n_S + n_I + n_Q, 
\end{equation}
where $n_I$, the tokenized image representation, often dominates, sometimes exceeding other components by an order of magnitude or more.  
As a result, minimizing $n_I$ becomes a critical strategy for enhancing the efficiency of MLLMs.

\subsection{Beyond Token Importance: Questioning the Status Quo}
Given the computational burden associated with the length of visual tokens in MLLMs, numerous studies have embraced a paradigm that utilizes attention scores to evaluate the significance of visual tokens, thereby facilitating token reduction.
Specifically, in transformer-based MLLMs, each layer performs attention computation as illustrated below:
\begin{equation}
   \text{Attention}(\mathbf{Q}, \mathbf{K}, \mathbf{V}) = \text{softmax}\left(\frac{\mathbf{Q} \cdot \mathbf{K}^\mathbf{T}}{\sqrt{d_k}}\right)\cdot \mathbf{V},
\end{equation}
where $d_k$ is the dimension of $\mathbf{K}$. The result of $\text{Softmax}(\mathbf{Q}\cdot \mathbf{K}^\mathbf{T}/\sqrt{d_k})$ is a square matrix known as the attention map.
Existing methods extract the corresponding attention maps from one or multiple layers and compute the average attention score for each visual token based on these attention maps:
\begin{equation}
    \phi_{\text{attn}}(x_i) = \frac{1}{N} \sum_{j=1}^{N} \text{Attention}(x_i, x_j),
\end{equation}
where $\text{Attention}(x_i, x_j)$ denotes the attention score between token $x_i$ and token $x_j$, $\phi_{\text{attn}}(x_i)$ is regarded as the importance score of the token $x_i$, $N$ represents the number of visual tokens.
Finally, based on the importance score of each token and the predefined reduction ratio, the most significant tokens are selectively retained:
\begin{equation}
    \mathcal{R} = \{ x_i \mid (\phi_{\text{attn}}(x_i) \geq \tau) \},
\end{equation}
where $\mathcal{R}$ represents the set of retained tokens, and $\tau$ is a threshold determined by the predefined reduction ratio.

\noindent{\textbf{Problems:}} Although this paradigm has demonstrated initial success in enhancing the efficiency of MLLMs, it is accompanied by several inherent limitations that are challenging to overcome.

First, when it comes to leveraging attention scores to derive token importance, it inherently lacks full compatibility with Flash Attention, resulting in limited hardware acceleration affinity and diminished acceleration benefits.

Second, does the paradigm of using attention scores to evaluate token importance truly ensure the effective retention of crucial visual tokens? Our empirical investigations reveal that it is not the optimal approach.

% As illustrated in Figure~\ref{fig:random_vs_others}, performance evaluations on certain benchmarks show that methods meticulously designed based on this paradigm sometimes underperform compared to randomly retaining the same number of visual tokens.
Performance evaluations on certain benchmarks, as illustrated in Figure~\ref{fig:random_vs_others}, demonstrate that methods meticulously designed based on this paradigm sometimes underperform compared to randomly retaining the same number of visual tokens.

% As depicted in Figure~\ref{fig:teaser_curry}, which visualizes the results of token reduction, the selection of visual tokens based on attention scores exhibits a noticeable bias, favoring tokens located in the lower-right region of the image—those positioned later in the visual token sequence. However, it is evident that the lower-right region is not always the most significant in every image.
% Furthermore, in Figure~\ref{fig:teaser_curry}, we present the outputs of the original LLaVA-1.5-7B, FastV, and our proposed \algname. Notably, FastV introduces more hallucinations compared to the vanilla model, while \algname demonstrates a noticeable trend of reducing hallucinations.
% We suppose that this phenomenon arises because the important-based method, which relies on attention scores, tends to retain visual tokens that are concentrated in specific regions of the image due to the inherent bias in attention scores. As a result, relying on only a portion of the image often leads to outputs that are inconsistent with the overall image content. In contrast, \algname primarily removes highly duplication tokens and retains tokens that are more evenly distributed across the entire image, enabling it to make more accurate and consistent judgments.
%--------------- shorter version ---------------------
Figure~\ref{fig:teaser_curry} visualizes the results of token reduction, revealing that selecting visual tokens based on attention scores introduces a noticeable bias toward tokens in the lower-right region of the image—those appearing later in the visual token sequence. However, this region is not always the most significant in every image. Additionally, we present the outputs of the original LLaVA-1.5-7B, FastV, and our proposed \algname. Notably, FastV generates more hallucinations compared to the vanilla model, while \algname effectively reduces them. 
We attribute this to the inherent bias of attention-based methods, which tend to retain tokens concentrated in specific regions, often neglecting the broader context of the image. In contrast, \algname removes highly duplication tokens and preserves a more balanced distribution across the image, enabling more accurate and consistent outputs.

\subsection{Token Duplication: Rethinking Reduction}
Given the numerous drawbacks associated with the paradigm of using attention scores to evaluate token importance for token reduction, \textit{what additional factors should we consider beyond token importance in the process of token reduction?}
Inspired by the intuitive ideas mentioned in \secref{sec:introduction} and the phenomenon of tokens in transformers tending toward uniformity (i.e., over-smoothing)~\citep{nguyen2023mitigating, gong2021vision}, we propose that token duplication should be a critical focus.

Due to the prohibitively high computational cost of directly measuring duplication among all tokens, we adopt a paradigm that involves selecting a minimal number of pivot tokens. 
\begin{equation}
    \mathcal{P} = \{p_1, p_2, \dots, p_k\}, \quad k \ll n,
\end{equation}
where $p_i$ denotes pivot token, $\mathcal{P}$ represents the set of pivot tokens and $n$ means the length of tokens.

Subsequently, we compute the cosine similarity between these pivot tokens and the remaining visual tokens:
\begin{equation}
    dup (p_i, x_j) = \frac{p_i \cdot x_j}{\|p_i\| \cdot \|x_j\|}, \quad p_i \in \mathcal{P}, \, x_j \in \mathcal{X},
\end{equation}
where $dup (p_i, x_j)$ represents the token duplication score between $i$-th pivot token $p_i$ and $j$-th visual token $x_j$,
ultimately retaining those tokens that exhibit the lowest duplication with the pivot tokens.
\begin{equation}
    \mathcal{R} = \{ x_j \mid \min_{p_i \in \mathcal{P}} dup (p_i, x_j) \leq \epsilon \}.
\end{equation}
Here, $\mathcal{R}$ denotes the set of retained tokens, and $\epsilon$ is a threshold determined by the reduction ratio.

Our method is orthogonal to the paradigm of using attention scores to measure token importance, meaning it is compatible with existing approaches. Specifically, we can leverage attention scores to select pivot tokens, and subsequently incorporate token duplication into the process.

However, this approach still does not fully achieve compatibility with Flash Attention. To this end, we explored alternative strategies for selecting pivot tokens, such as using K-norm, V-norm\footnote{Here, the K-norm and V-norm refer to the L1-norm of K matrix and V matrix in attention computing, respectively.}, or even random selection. Surprisingly, we found that all these methods achieve competitive performance across multiple benchmarks. This indicates that our token reduction paradigm based on token duplication is not highly sensitive to the choice of pivot tokens. Furthermore, it suggests that removing duplicate tokens may be more critical than identifying ``important tokens'', highlighting token duplication as a potentially more significant factor to consider in token reduction.
The selection of pivot tokens is discussed in greater detail in \secref{pivot_token_selection}.
% 加个总结


\section{Results} \label{sec:results}
\section{Results}
\label{sec:Results}

In this section, we present various analysis results that demonstrate the adoption of code obfuscation in Google Play.

\subsection{Overall Obfuscation Trends} 
\label{sec:obstrend}

\subsubsection{Presence of obfuscation} Out of the 548,967 Google Play Store APKs analyzed, we identified 308,782 obfuscated apps, representing approximately 56.25\% of the total. In Figure~\ref{fig:obfuscated_percentage}, we show the year-wise percentage of obfuscated apps for 2016-2023. There is an overall obfuscation increase of 13\% between 2016 and 2023, and as can be seen, the percentage of obfuscated apps has been increasing in the last few years, barring 2019 and 2020. As explained in Section~\ref{subsec:dataset}, 2019 and 2020 contain apps that are more likely to be abandoned by developers, and as such, they may not use advanced development practices.

\begin{figure}[h!]
\centering
    \includegraphics[width=\linewidth]{Figures/Only_obfuscation_trendV2.pdf}
    \caption{Percentage of obfuscated apps by year} \vspace{-4mm}
    \label{fig:obfuscated_percentage}
\end{figure}


From 2016 to 2018, the obfuscation levels were relatively stable at around 50-55\%, while from 2021 to 2023, there was a marked rise, reaching approximately 66\% in 2023. This indicates a growing focus on app protection measures among developers, likely driven by heightened security and IP concerns and the availability of advanced obfuscation tools.


\subsubsection{Obfuscation tools} Among the obfuscated APKs, our tool detector identified that 40.92\% of the apps use Proguard, 36.64\% use Allatori, 1.01\% use DashO, and 21.43\% use other (i.e., unknown) tools. We show the yearly trends in Figure~\ref{fig:ofbuscated_tool}. Note that we omit results in 2019 and 2020 ({\bf cf.} Section~\ref{subsec:dataset}).

ProGuard and Allatori are the most consistently used obfuscation tools, with ProGuard showing a slight overall increase in popularity and Allatori demonstrating variability. This inclination could be attributed to ProGuard being the default obfuscator integrated into Android Studio, a widely used development environment for Android applications. Notably, ProGuard usage increased by 13\% from 2018 to 2021, likely due to the introduction of R8 in April 2019~\cite{release_note_android}, which further simplified ProGuard integration with Android apps.

\begin{figure}[h]
\centering
    \includegraphics[width=\linewidth]{Figures/Initial_Tool_Trend_2019_dropV2.pdf} 
    \caption{Yearly obfuscation tool usage}
    \label{fig:ofbuscated_tool}
\end{figure}


DashO consistently remains low in usage, likely due to its high cost. The use of other obfuscation tools decreased until 2018 but has shown a resurgence from 2021 to 2023. This suggests that developers might be using other or custom tools, or our detector might be predicting some apps obfuscated with Proguard or Allatori as `other.' To investigate, we manually checked a sample of apps from the `other' category and confirmed they are indeed obfuscated. However, we could not determine which obfuscation tools the developers used. We discuss this potential limitation further in Section~\ref{sec:limitations}.


\subsubsection{Obfuscation techniques} We show the year-wise breakdown of obfuscation technique usage in Figure~\ref{fig:obfuscated_tech}. Among the various obfuscation techniques, Identifier Renaming emerged as the most prevalent, with 99.62\% of obfuscated apps using it alone or in combination with other methods (Categories of Only IR, IR and CF, IR and SE, or All three). Furthermore, 81.04\% of obfuscated apps used Control Flow Modification, and 62.76\% used String Encryption. The pervasive use of Identifier Renaming (IR) can be attributed to the fact that all obfuscation tools support it ({\bf cf.} Table~\ref{tab:ob_tool_cap}). Similarly, lower adoption of Control Flow Modification and String Encryption can be attributed to Proguard not supporting it. 

\begin{figure}[h]
\centering
    \includegraphics[width=\linewidth]{Figures/Initial_Tech_Trend_2019_dropV2.pdf} 
    \caption{Yearly obfuscation technique usage}
    \label{fig:obfuscated_tech}
\end{figure}



Next, we investigate the adoption of obfuscation on Google Play Store from various perspectives. Same as earlier, due to the smaller dataset size and possible bias ({\bf cf.} Section~\ref{subsec:dataset}), we exclude the APKs from 2019 and 2020 from this analyses.


\subsection{App Genre}
\label{sec:app_genre}

First, we investigate whether the obfuscation practices vary according to the App genre. Initially, we analysed all the APKs together before separating them into two snapshots.


\begin{figure*}[h]
    \centering
    \includegraphics[width=\linewidth]{Figures/AppGenreObfuscationV3.pdf}
    \caption{Obfuscated app percentage by genre (overall)}
    \label{fig:app_genre_overall}
\end{figure*}

Figure~\ref{fig:app_genre_overall} shows the genre-wise obfuscated app percentage. We note that 19 genres have more than 60\% of the apps obfuscated, and almost all the genres have more than 40\% obfuscation percentage. \textit{Casino} genre has the highest obfuscation percentage rate at 80\%, and overall, game genres tend to be more obfuscated than the other genres. The higher obfuscation usage in casino apps is logical due to their nature. These apps often simulate or involve gambling activities and handle monetary transactions and sensitive data related to in-game purchases, making them attractive targets for reverse engineering and hacking. This necessitates robust security measures to prevent fraud and protect user data. 


\begin{figure}[h]
    \centering
    \includegraphics[width=\linewidth]{Figures/AppGenre2018_2023ComparisonV3.pdf}
    \caption{Percentage of obfuscated apps by genre (2018-2023)}
    \label{fig:app_genre_comparison}
\end{figure}



\subsubsection{Genre-wise obfuscation trends in the two snapshots} To investigate the adoption of obfuscation over time, we study the two snapshots of Google Play separately, i.e., APKs from 2016-2018 as one group and APKs from 2021-2023 as another. 

Figure~\ref{fig:app_genre_comparison} illustrates the change in obfuscation levels by app genre between 2016-2018 to 2021-2023. Notably, app categories such as Education, Weather, and Parenting, which had obfuscation levels below the 2018 average, have increased to above the 2023 average by 2023. One possible reason for this in Education and Parenting apps can be the increase in online education activities during and after COVID-19 and the developers identifying the need for app hardening.

There are some genres, such as Casino and Action, for which the percentage of obfuscated apps didn't change across the two snapshots (i.e., purple and orange circles are close together in Figure~\ref{fig:app_genre_comparison}). This is because these genres are highly obfuscated from the beginning. Finally, the four genres, including Simulation and Role Playing, have a lower percentage of obfuscated apps in the 2021-2023 dataset. Our manual analysis didn't result in a conclusion as to why.


\begin{figure}[!h]
    \centering
    \includegraphics[width=\linewidth]{Figures/AppGenreTechAllV5.pdf}
    \caption{Obfuscation technique usage by genre (overall)}
    \label{fig:app_genre_all_tech}
\end{figure}


\subsubsection{Obfuscation techniques in different app genres} In Figure~\ref{fig:app_genre_all_tech}, we show the prevalence of key obfuscation techniques among various genres. As expected, almost all obfuscated apps in all genres used  Identifier Renaming. Also, it can be noted that genres with more obfuscated app percentages tend to use all three obfuscation techniques. Notably, more than 85\% of \textit{Casino} genre apps employ multiple obfuscation techniques

\subsubsection{Obfuscation tool usage in different app genres} We also investigated whether specific obfuscation tools are favoured by developers in different genres. However, apart from the expected observation that  ProGuard and Allatori being the most used tools, we didn't find any other interesting patterns. Therefore, we haven't included those measurement results.

\subsection{App Developers}
Next, we investigate individual developer-wise code obfuscation practices. From the pool of analyzed APKs, we identified the number of apps associated with each developer. Subsequently, we sorted the developers according to the number of apps they had created and selected the top 100 developers with the highest number of APKs for the 2016-2018 and 2021-2023 datasets. For the 2018 snapshot, we had 8,349 apps among the top 100 developers, while for the 2023 snapshot, we had 11,338 apps among the top 100 developers.

We then proceeded to detect whether or not these developers obfuscate their apps and, if so, what kind of tools and techniques they use. We present our results in five levels; developer obfuscating over 80\% of their apps, 60\%--80\% of apps, 40\%--60\% of apps, less than 40\%, and no obfuscation.

Figure~\ref{fig:developer_trend_my_apps_all} compares the two datasets in terms of developer obfuscation adoption. It shows that more developers have moved to obfuscate more than 80\% of their apps in the 2021-2023 dataset (76\%) compared to the 2016-2018 dataset (48\%).

We also found that among developers who obfuscate more than 80\% of their apps, 73\% in 2018 and 93\% in 2023 used the same obfuscation tool. Additionally, these top developers employ Control Flow Modification (CF) and String Encryption (SE) above the average values discussed in Section~\ref{sec:obstrend}. Specifically, in 2018, top developers used CF in 81.3\% of cases and SE in 66.7\%, while in 2023, these figures increased to 88.2\% and 78.9\%. This results in two insights: 1) Most top developers obfuscate all their apps with advanced techniques, possibly due to concerns about IP and security, and 2) Developers stick to a single tool, possibly due to specialized knowledge or because they bought a commercial licence.

\begin{figure}[]
    \centering
    \includegraphics[width=\linewidth]{Figures/Developer_Analysed_Comparison.pdf}
    \caption{Obfuscation usage (Top-100 developers)}
    \label{fig:developer_trend_my_apps_all}
\end{figure}


Finally, we investigate the obfuscation practices of developers with only one app in Table~\ref{tab:my-table}. According to the table, from those developers, 45.5\% of them obfuscated their apps in the 2016-2018 dataset and 57.2\% obfuscated their apps in the 2021-2023 dataset, showing a clear increase. However, these percentages are approximately 10\% lower than the average obfuscation rate in both cohorts discussed in Section~\ref{sec:obstrend}. This indicates that single-app developers may be less aware or concerned about code protection.


\begin{table}[]
\caption{Developers with only one app}
\label{tab:my-table}
\resizebox{\columnwidth}{!}{%
\begin{tabular}{cccccc}
\hline
\textbf{Year} & \textbf{\begin{tabular}[c]{@{}c@{}}Non\\ Obfuscated\end{tabular}} & \multicolumn{4}{c}{\textbf{Obfuscated}} \\ \hline
\multirow{3}{*}{\textbf{\begin{tabular}[c]{@{}c@{}}2018 \\ Snapshot\end{tabular}}} & \multirow{3}{*}{\begin{tabular}[c]{@{}c@{}}26,581 \\ (54.5\%)\end{tabular}} & \multicolumn{4}{c}{\begin{tabular}[c]{@{}c@{}}22,214 (45.5\%)\end{tabular}} \\ \cline{3-6} 
 &  & \textbf{ProGuard} & \textbf{Allatori} & \textbf{DashO} & \textbf{Other} \\ \cline{3-6} 
 &  & 6,131 & 8,050 & 658 & 7,375 \\ \hline
\multirow{3}{*}{\textbf{\begin{tabular}[c]{@{}c@{}}2023 \\ Snapshot\end{tabular}}} & \multirow{3}{*}{\begin{tabular}[c]{@{}c@{}}19,510 \\ (42.8\%)\end{tabular}} & \multicolumn{4}{c}{\begin{tabular}[c]{@{}c@{}}26,084 (57.2\%)\end{tabular}} \\ \cline{3-6} 
 &  & \textbf{ProGuard} & \textbf{Allatori} & \textbf{DashO} & \textbf{Other} \\ \cline{3-6} 
 &  & 12,697 & 9,672 & 234 & 3,581 \\ \hline
\end{tabular}%
}
\end{table}

\subsection{Top-k Apps}

Next, we investigate the obfuscation practices of top apps in Google Play Store. First, we rank the apps using the same criterion used by our previous work~\cite{rajasegaran2019multi, karunanayake2020multi, seneviratne2015early}. That is, we sort the apps in descending order of number of downloads, average rating, and rating count, with the intuition that top apps have high download numbers and high ratings, even when reviewed by a large number of users. Then, we investigated the percentage of obfuscated apps and obfuscation tools and technique usage as summarized in Table~\ref{tab:top_k_apps_2018_2023}.

When considering the highly ranked applications (i.e., top-1,000), the obfuscation percentage is notably higher, at around 93\%, in both datasets, which is significantly higher than the average percentage of obfuscation we observed in Section~\ref{sec:obstrend}. Top-ranked apps, likely due to their higher visibility and potential revenue, invest more in obfuscation to safeguard their intellectual property and enhance security. 

The obfuscation percentage decreases when going from the top 1,000 apps to the top 30,000 apps. Nonetheless, the obfuscation percentage in both datasets remains around similar values until the top 30,000 (e.g., $\sim$74\% for top-30,000). This indicates that the major increase in obfuscation in the 2021-2023 dataset comes from apps beyond the top 30,000.

When observing the tools used, the usage of ProGuard increases as we move from top to lower-ranked apps in both datasets. This may be because ProGuard is free and the default in Android Studio, while commercial tools like Allatori and DashO are expensive. There is a notable increase in the use of Allatori among the top apps in the 2021-2023 dataset. Regarding obfuscation techniques, the top 1,000 apps utilize all three techniques more frequently than lower-ranked apps in both snapshots. This indicates that the top 1,000 apps are more heavily protected compared to lower-ranked ones.

\begin{table*}[]
\caption{Summary of analysis results for Top-k apps in 2018 and 2023}
\label{tab:top_k_apps_2018_2023}
\resizebox{\textwidth}{!}{%
\begin{tabular}{lccccccccc}
\hline
\multicolumn{1}{c}{\begin{tabular}[c]{@{}c@{}}Top k apps - \\ Year\end{tabular}} & \begin{tabular}[c]{@{}c@{}}Total \\ Apps\end{tabular} & \begin{tabular}[c]{@{}c@{}}Obfuscation\\ Percentage\end{tabular} & \begin{tabular}[c]{@{}c@{}}ProGuard\\ Percentage\end{tabular} & \begin{tabular}[c]{@{}c@{}}Allatori\\ Percentage\end{tabular} & \begin{tabular}[c]{@{}c@{}}DashO\\ Percentage\end{tabular} & \begin{tabular}[c]{@{}c@{}}Other\\ Percentage\end{tabular} & \begin{tabular}[c]{@{}c@{}}IR\\ Percentage\end{tabular} & \begin{tabular}[c]{@{}c@{}}CF\\ Percentage\end{tabular} & \begin{tabular}[c]{@{}c@{}}SE\\ Percentage\end{tabular} \\ \hline
1k (2018) & 1,000 & 93.40 & 29.98 & 28.48 & 0.64 & 40.90 & 99.90 & 88.76 & 65.42 \\
10k (2018) & 10,000 & 85.19 & 25.55 & 35.32 & 0.47 & 38.65 & 99.90 & 88.76 & 71.91 \\
20k (2018) & 20,000 & 78.42 & 26.31 & 36.76 & 0.57 & 36.36 & 99.87 & 87.37 & 71.49 \\
30k (2018) & 30,000 & 74.40 & 27.30 & 37.71 & 0.64 & 34.36 & 99.82 & 86.75 & 71.11 \\
30k+ (2018) & 314,568 & 53.36 & 36.72 & 34.70 & 1.33 & 27.24 & 99.34 & 83.54 & 63.11 \\ \hline
1k (2023) & 1,000 & 92.50 & 24.00 & 51.89 & 1.95 & 22.16 & 100.0 & 92.54 & 83.68 \\
10k (2023) & 10,000 & 81.88 & 26.03 & 56.20 & 1.03 & 16.74 & 99.89 & 89.40 & 82.01 \\
20k (2023) & 20,000 & 76.62 & 30.48 & 52.92 & 0.96 & 15.64 & 99.93 & 85.80 & 78.01 \\
30k (2023) & 30,000 & 73.72 & 33.87 & 50.34 & 0.89 & 14.90 & 99.95 & 83.31 & 75.34 \\
30k+ (2023) & 206,216 & 61.90 & 46.56 & 38.21 & 0.64 & 14.59 & 99.97 & 77.51 & 62.50 \\ \hline
\end{tabular}%
}
\end{table*}


\section{Related Work} \label{sec:related}
\subsubsection{Conditioned Diffusion Models}

By operating the data in latent space instead of pixel space, conditioned diffusion models have gained promising development \cite{rombach2022latentDiff}. MM-Diffusion \cite{ruan2023mmdi} designed for joint audio and video generation took advantage of coupled denoising autoencoders to generate aligned audio-video pairs from Gaussian noise. Extending the scalability of diffusion models, diffusion Transformers treat all inputs, including time, conditions, and noisy image patches, as tokens, leveraging the Transformer architecture to process these inputs \cite{bao2023ViTDiff}. In DiT \cite{peebles2023DiT}, William et al. emphasized the potential for diffusion models to benefit from Transformer architectures, where conditions were tokenized along with image tokens to achieve in-context conditioning. 

\subsubsection{Diffusion Models in Robotics}

Recently, a probabilistic multimodal action representation was proposed by Cheng Chi et al. \cite{chi2023diffusionpolicy}, where the robot action generation is considered as a conditional diffusion denoising process. Leveraging the diffusion policy, Ze et al. \cite{ze20243d} conditioned the diffusion policy on compact 3D representations and robot poses to generate coherent action sequences. Furthermore, GR-MG combined a progress-guided goal image generation model with a multimodal goal-conditioned policy, enabling the robot to predict actions based on both text instructions and generated goal images \cite{li2025grmg}. BESO used score-based diffusion models to learn goal-conditioned policies from large, uncurated datasets without rewards. Score-based diffusion models progressively add noise to the data and then reverse this process to generate new samples, making them suitable for capturing the multimodal nature of play data \cite{reuss2023md}. RDT-1B employed a scalable Transformer backbone combined with diffusion models to capture the complexity and multimodality of bimanual actions, leveraging diffusion models as a foundation model to effectively represent the multimodality inherent in bimanual manipulation tasks \cite{liu2024rdt-1b}. NoMaD exploited the diffusion model to handle both goal-directed navigation and task-agnostic exploration in unfamiliar environments, using goal masking to condition the policy on an optional goal image, allowing the model to dynamically switch between exploratory and goal-oriented behaviors \cite{sridhar2023nomad}. The aforementioned insights grounded the significant advancements of diffusion models in robotic tasks.

\subsubsection{VLM-based Autonomous Driving}

End-to-end autonomous driving introduces policy learning from sensor data input, resulting in a data-driven motion planning paradigm \cite{chen2024vadv2}. As part of the development of VLMs, they have shown significant promise in unifying multimodal data for specific downstream tasks, notably improving end-to-end autonomous driving systems\cite{ma2024dolphins}. DriveMM can process single images, multiview images, single videos, and multiview videos, and perform tasks such as object detection, motion prediction, and decision making, handling multiple tasks and data types in autonomous driving \cite{huang2024drivemm}. HE-Drive aims to create a human-like driving experience by generating trajectories that are both temporally consistent and comfortable. It integrates a sparse perception module, a diffusion-based motion planner, and a trajectory scorer guided by a Vision Language Model to achieve this goal \cite{wang2024hedrive}. Based on current perspectives, a differentiable end-to-end autonomous driving paradigm that directly leverages the capabilities of VLM and a multimodal action representation should be developed. 









\section{Conclusion} \label{sec:conclusions}
We present RiskHarvester, a risk-based tool to compute a security risk score based on the value of the asset and ease of attack on a database. We calculated the value of asset by identifying the sensitive data categories present in a database from the database keywords. We utilized data flow analysis, SQL, and Object Relational Mapper (ORM) parsing to identify the database keywords. To calculate the ease of attack, we utilized passive network analysis to retrieve the database host information. To evaluate RiskHarvester, we curated RiskBench, a benchmark of 1,791 database secret-asset pairs with sensitive data categories and host information manually retrieved from 188 GitHub repositories. RiskHarvester demonstrates precision of (95\%) and recall (90\%) in detecting database keywords for the value of asset and precision of (96\%) and recall (94\%) in detecting valid hosts for ease of attack. Finally, we conducted an online survey to understand whether developers prioritize secret removal based on security risk score. We found that 86\% of the developers prioritized the secrets for removal with descending security risk scores.

\section*{Acknowledgment}
This work was partially supported by the MUR under the grant “Dipartimenti di Eccellenza 2023-2027" of the Department of Informatics, Systems and Communication of the University of Milano-Bicocca, Italy; and by the Engineered MachinE Learning-intensive IoT systems (EMELIOT) national research project, which has been funded by the MUR under the PRIN 2020 program (Contract 2020W3A5FY).



\bibliographystyle{IEEEtran}
  \bibliography{main}

\vspace{12pt}


\end{document}
