\documentclass[10pt,conference]{IEEEtran}
\IEEEoverridecommandlockouts
% The preceding line is only needed to identify funding in the first footnote. If that is unneeded, please comment it out.
%Template version as of 6/27/2024

\usepackage{cite}
\usepackage{amsmath,amssymb,amsfonts}
\usepackage{algorithmic}
\usepackage{graphicx}
\usepackage{textcomp}
\usepackage{tcolorbox}
\usepackage{todonotes}
\usepackage{url}
\usepackage{listings}
\usepackage{booktabs}
\usepackage{xspace}
\usepackage{hyperref}

\newcommand{\change}[1]{\textcolor{red}{#1}\xspace}

\lstdefinestyle{mystyle}{
%    backgroundcolor=\color{backcolour},   
%    commentstyle=\color{codegreen},
 %   keywordstyle=\color{magenta},
 %   numberstyle=\tiny\color{codegray},
 %   stringstyle=\color{codepurple},
    basicstyle=\ttfamily\footnotesize,
 %   breakatwhitespace=false,         
 %   breaklines=true,                 
    captionpos=b,                    
 %   keepspaces=true,                 
    numbers=left,                    
    numbersep=5pt,                  
 %   showspaces=false,                
 %   showstringspaces=false,
 %   showtabs=false,                  
    tabsize=2
}
\lstset{style=mystyle}
\newcommand{\urlRepo}[0]{\url{https://osf.io/nw8vp/?view_only=a4654139dce54584ab5e1549e56dff3c}\xspace} 

\usepackage{xcolor}
\def\BibTeX{{\rm B\kern-.05em{\sc i\kern-.025em b}\kern-.08em
    T\kern-.1667em\lower.7ex\hbox{E}\kern-.125emX}}

    
\begin{document}

\title{Studying How Configurations Impact Code Generation in LLMs: the Case of ChatGPT
%{\footnotesize \textsuperscript{*}Note: Sub-titles are not captured for https://ieeexplore.ieee.org  and
%should not be used}
%\thanks{Identify applicable funding agency here. If none, delete this.}
}
\author{\IEEEauthorblockN{Benedetta Donato}
\IEEEauthorblockA{
\textit{University of Milano-Bicocca}\\
Milano, Italy \\
benedetta.donato@unimib.it}
\and
\IEEEauthorblockN{Leonardo Mariani}
\IEEEauthorblockA{
\textit{University of Milano-Bicocca}\\
Milano, Italy \\
leonardo.mariani@unimib.it}
 \and
%\linebreakand
%\hspace{-1.0cm} % Adjust spacing between authors if needed
\IEEEauthorblockN{Daniela Micucci}
\IEEEauthorblockA{
\textit{University of Milano-Bicocca}\\
Milano, Italy \\
daniela.micucci@unimib.it}
\and
\IEEEauthorblockN{Oliviero Riganelli}
\IEEEauthorblockA{
\textit{University of Milano-Bicocca}\\
Milano, Italy \\
oliviero.riganelli@unimib.it}

}
%\author{\IEEEauthorblockN{Anonymous Authors}}
%\author{\IEEEauthorblockN{1\textsuperscript{st} Given Name Surname}
%\IEEEauthorblockA{\textit{dept. name of organization (of Aff.)} \\
%\textit{name of organization (of Aff.)}\\
%City, Country \\
%email address or ORCID}
%\and
%\IEEEauthorblockN{2\textsuperscript{nd} Given Name Surname}
%\IEEEauthorblockA{\textit{dept. name of organization (of Aff.)} \\
%\textit{name of organization (of Aff.)}\\
%City, Country \\
%email address or ORCID}
%\and
%\IEEEauthorblockN{3\textsuperscript{rd} Given Name Surname}
%\IEEEauthorblockA{\textit{dept. name of organization (of Aff.)} \\
%\textit{name of organization (of Aff.)}\\
%City, Country \\
%email address or ORCID}
%\and
%\IEEEauthorblockN{4\textsuperscript{th} Given Name Surname}
%\IEEEauthorblockA{\textit{dept. name of organization (of Aff.)} \\
%\textit{name of organization (of Aff.)}\\
%City, Country \\
%email address or ORCID}
%\and
%\IEEEauthorblockN{5\textsuperscript{th} Given Name Surname}
%\IEEEauthorblockA{\textit{dept. name of organization (of Aff.)} \\
%\textit{name of organization (of Aff.)}\\
%City, Country \\
%email address or ORCID}
%\and
%\IEEEauthorblockN{6\textsuperscript{th} Given Name Surname}
%\IEEEauthorblockA{\textit{dept. name of organization (of Aff.)} \\
%\textit{name of organization (of Aff.)}\\
%City, Country \\
%email address or ORCID}
%}

\maketitle


\begin{abstract}
Leveraging LLMs for code generation is becoming increasingly common, as tools like ChatGPT can suggest method implementations with minimal input, such as a method signature and brief description. Empirical studies further highlight the effectiveness of LLMs in handling such tasks, demonstrating notable performance in code generation scenarios.
%Writing code assisted by an LLM is becoming common practice. For instance, tools like ChatGPT can suggest the implementation of a method by simply accessing the method's signature and a short description of the expected behavior. Several empirical studies report the interesting performance of LLMs in code generation tasks.

However, LLMs are inherently non-deterministic, with their output influenced by parameters such as temperature, which regulates the model's level of creativity, and top-p, which controls the choice of the tokens that shall appear in the output. Despite their significance, the role of these parameters is often overlooked.%The role of these parameters is unfortunately underrated. 

This paper systematically studies the impact of these parameters, as well as the number of prompt repetitions required to account for non-determinism,
%of times a prompt should be repeatedly submitted to account for non-determinism, 
in the context of 548 Java methods. We observe significantly different performances across different configurations of ChatGPT, with temperature having a marginal impact compared to the more prominent influence of the top-p parameter. Additionally, we show how creativity can enhance code generation tasks. Finally, we provide concrete recommendations for addressing the non-determinism of the model.
%revealing a marginal impact of the temperature, in favor of a much more prominent role of parameter top-p. Further, we show how creativity can be beneficial in code generation tasks. Finally, we provide concrete recommendations on how to deal with the non-determinism of the model. 

\end{abstract}

\begin{IEEEkeywords}
LLMs, code generation, ChatGPT, temperature, top-p, repetitions
\end{IEEEkeywords}

\section{Introduction} \label{sec:introduction}
\documentclass[../main.tex]{subfiles}
\graphicspath{{../images/}}
\makeatletter
\def\input@path{{../images/}}
\makeatother
\begin{document}
\section{Introduction}
\begin{figure}
\centering
\begin{tikzpicture}
\node[inner sep=0pt] (ws) at (0, 0) {
\includegraphics[height=.4\textwidth, trim={10cm 0 10cm 0},clip]{world_space.png}};
\node[inner sep=0pt] (cs) at (6,0) {\includegraphics[height=.4\textwidth, trim={10cm 1cm 10cm 4cm},clip]{conf_space.png}};
\end{tikzpicture}
\vspace{-5pt}
\label{fig:pbrm_intro}
\caption{\textbf{Left}: Shows world space obstacles as grey spheres. Robots start and goal configuration is colored red and green, respectively. Configurations along the computed path are colored transparent blue. \textbf{Right:} Mapped world space scenario to configuration space. Obstacle region is the grey mesh. Red spheres are collision-free regions computed by the neural SCDF. The optimized shortest path in the convex corridor is the blue curve.}
\vspace{-25pt}
\end{figure}
Motion planning is the problem of finding a collision-free trajectory that connects a given start and goal configuration. The planning takes place in the configuration space of the robot. For single body robots, like mobile robots or drones, the configuration space and the world space are usually the same. This simplifies the planning, since explicit obstacle representations are available which enables geometrical tools like separating hyperplanes, smallest distance to obstacles etc., to be used when designing motion planning algorithms. For multi-body robots like manipulators, the situation is completely different. The world space obstacles are usually mapped to non-convex regions, and to make the problem even harder, the mapping is usually not known. Forming explicit representations of the obstacle region in the configuration space is usually too expensive or intractable. Despite all of this, sampling based planners are used with great success, which mainly is due to their use of implicit representations of the obstacle region. The basic idea is to construct a graph in the configuration space that covers and connects the collision-free region. From this graph, a path can be extracted that connects a given start and goal configuration. The approach is computationally expensive, since the graph is constructed with the smallest geometrical building block available, points, which represents a collision-check. Furthermore, the extracted paths from the graph are non-smooth and jagged due to the stochastic nature of the approach. This adds an additional post-processing step to the process, where the paths are shortcutted and smoothened, before the path can be used for tracking. Clearly a lot of time is invested to form this graph and produce smooth paths. Thus, if the obstacles start to move, then all of this work is done in no use, since all points that make up this graph need to be re-verified, which is simply too time consuming to be done in real time.
\\\\
In this work, we want to address the existing drawbacks of the sampling based planners. Our main contribution is an improved motion planner where each vertex in the graph covers a collision-free region in the form of a sphere instead of a point and where the edges are formed with neighboring intersecting spheres. This representation has the advantage of instead of returning piecewise linear paths, returning a sequence of overlapping spheres, i.e. a convex corridor, that connects a given start and goal configuration, illustrated in Figure \ref{fig:pbrm_intro}. This convex corridor allows us to use convex optimization to produce smooth trajectories, instead of computationally expensive post-processing methods. The representation further allows us to estimate the coverage of the collision-free space, which gives us awareness and feedback in the offline roadmap construction phase. Finally, our representation is simple to adapt to moving obstacles, simply requery for the new radii and recheck for intersections. 
\\\\
The spherical collision-free regions are formed using a signed distance function (SDF), which is a function that returns the smallest distance from an arbitrary point to the boundary of an obstacle. As the name implies, the distance is signed, thus if the point is inside the obstacle it is negative otherwise positive. If the distance is positive, a sphere with radius equal to the distance is guaranteed to cover a collision-free region. Using an SDF in motion planning is not new, but what is novel about our approach is that we express the distance in the configuration space instead of the world space and by doing so allows us to form these convex collision-free regions. We refer to the resulting SDF as a signed configuration distance function (SCDF). Computing an SCDF analytically is non-trivial, our approach is therefore to parameterize the SCDF with a deep neural network and learn the mapping by supervised learning. Our resulting neural SCDF can compute distances for different parameter values of obstacle shapes and we also show how multiple distances can be combined, thus making our approach flexible.
\section{Related work}
Motion planning algorithms can roughly be divided into three families, grid-based, sampling based and optimization based methods. Grid-based methods (GBM) discretize the planning space from which a graph is then compiled. A standard search method is A$^\star$ \citep{a_star}, which is classified as an \textit{informed} search method, since it employs a heuristic function to speed up the search. A$^\star$ guarantees to return an optimal path at the level of discretization used. GBMs usually discretize the planning space by a regular lattice and this limits the GBMs to problems with low dimensionality due to the curse of dimensionality. Thus, GBMs are usually limited to single-body robots where the degrees of freedom (DOF) are low. To overcome the inherent scaling problem with the GBMs, stochastic methods are usually used for multi-body robots. These methods are termed as sampling-based methods (SBM) and core members within this family are the rapidly-exploring random trees (RRT) \citep{rrt} and the probabilistic roadmap (PRM) \citep{prm}. RRT grows a tree from the start configuration and explores the collision-free region in a rapid way until it is able to connect to the goal region. RRT is usually improved by bi-directional planning \citep{rrt_connect}, i.e. an additional tree is grown from the goal configuration and the trees are tested for connection after any tree has been expanded. RRT is a single-query method, thus it searches for a path from scratch each time it is queried. Contrary to this, PRM is a multi-query method, which solves for multiple queries without starting from scratch. PRM does this by creating a roadmap (graph) that covers the collision-free space as an offline step. The graph is then used to solve for multiple queries. PRMs are used in cases where the environment does not change since the extra offline step is too computationally costly and needs to be re-done if the environment is changed. In our work, we address this inherent issue by using a different roadmap representation. Our vertices in the graph cover a collision-free region in the form of spheres and we form the edges by checking for intersecting spheres. If something in the environment changes, we recompute the spheres radii and recheck the intersections, without relying on collision detection. We use a trained neural network to compute the sphere radius, therefore querying for the radius can be done fast, hence our representation enables the PRM for dynamic environments.
\\\\
In the recent decades, optimization based methods (OBM) \citep{chomp, schulman, itomp, stomp} have been introduced as an alternative to SBM for multi-body robots. Like the SBM, the OBMs scale well to higher dimensional problems and produce smoother motion. It is common to use a SDF in the optimization since it is a smooth function, thus enabling gradient-based methods. However, the standard way of expressing the SDF is in world space. The distance therefore needs to be mapped to the configuration space by the forward kinematics. This mapping makes the optimization problem a non-linear program (NLP), which is computationally expensive to solve. Recently, a different approach has been proposed. In \cite{mp_gcs} motion planning is formulated as a convex optimization problem by using the graph of convex sets framework \citep{gcs}. The underlying idea is to decompose the collision-free space into intersecting convex sets from which a convex optimization problem is formulated. In cases where an explicit representation of the obstacles in the configuration space exists, like for single-body robots, creating collision-free convex regions can be done fast \citep{iris}. For multi-body robots, this is non-trivial. Existing work does this successfully \citep{iris_nlp, iris_c} by an optimization based approach, but the methods are still too time consuming to be used in the presence of moving obstacles. Our approach is instead to use deep learning to learn an SDF expressed in the configuration space. With this, we can query for shortest distances to the collision boundary, which allows us to expand spherical regions which are collision-free. Our approach is fast and therefore enables our suggested roadmap planner to be used in dynamic environments.
\\\\
Recent research has focused on learning collision detection \citep{fk_kernel_distance, diffco, graphdistnet} by predicting the signed distance between the robot links and the surrounding obstacles in the world space. The learned SDF is used in trajectory optimization but since the distance is expressed in the world space, the problem becomes an NLP and therefore takes a long time to solve. We take a novel approach and suggest to instead express the signed distance in the configuration space. This allows us to improve the PRM at the same time as it enables convex optimization for trajectory optimization, which runs faster and is more reliable than NLP solvers. In \cite{cspf} a learned signed distance function in the configuration space is proposed similar to our approach. However, their approach is restricted to point cloud representations, while we propose to represent the obstacles as parameterized geometric shapes, e.g. spheres. Furthermore, we also show how to use our learned SCDF to improve an existing roadmap planner.
\section{Problem formulation}
A robot is located in the world space, $\W \subset \R^3 $. The unique location of the robot is given by its configuration $\q \in \C$, where $\C$ is the configuration space. The set of points covered by the robots bodies at a certain configuration is expressed as $\B(\q) \subset \W$. The robot is surrounded by $\NrObst$ obstacles $\O = \bigcup_{i=1}^{\NrObst} \O_i$, where  $\O_i \subset \W$. The representation of the obstacle in the configuration space is the set $\C\O_i = \{\q \in \C \: |\: \B(\q) \cap \O_i \neq \emptyset \}$. The obstacle space is formed as $\Co = \bigcup_{i=1}^{\NrObst} \C \O_i$. The complement is referred to as the free space, $\Cf = \C \setminus \Co$. The path planning problem is a tuple, ($\Cf$, $\qStart$, $\qGoal$), where we want to connect a query pair, consisting of a start, $\qStart$, and goal configuration, $\qGoal$, with a geometric path, $\q(s): [0, 1] \mapsto \Cf$, such that $\q(0)=\qStart$ and $\q(1)=\qGoal$, or report correctly when such a path does not exist.
\end{document}




\section{Methodology} \label{sec:methodology}
\section{Methodology}
In this section, we outline the key research questions driving this study, followed by a detailed description of the methodology used to design and conduct the survey.
\subsection{Research Questions}
\begin{enumerate}
    \item[\textbf{RQ1:}] How do developers allocate their time during a typical workweek, and how does this compare to their perception of an \textbf{ideal workweek?}
    \item[\textbf{RQ2:}] How are developer's satisfaction and productivity affected by \textbf{deviations} from their ideal workweek?
     \item[\textbf{RQ3:}] For which tasks do developers prefer using \textbf{AI tools}, and how does the frequency of AI tool usage \textbf{influence} their satisfaction and productivity?
\end{enumerate}

\subsection{Survey Design}
% Describe how the survey was conducted, survey structure, sample size, which activities were selected and how, incentives, etc. 

To gain insights into the types of activities developers engage in during a typical work week, we conducted a series of exploratory interviews with 12 randomly selected participants. These semi-structured interviews provided a qualitative foundation, allowing us to iteratively develop a comprehensive list of higher-level activities that reflect both ideal and actual workweek allocations. The findings from these interviews were instrumental in refining our survey questions and design.

% - When was it distributed
% - How many people were invited
% - how was the survey advertised
% - incentive provided to participants
% - how many responses received (with response rates)
% - Board of ethics description \& instruments
% - Describe the main questions asked in the survey

The survey was distributed in \textcolor{blue}{May 2024} to software engineers working in Microsoft teams across India and the United States. A total of 6000 developers were invited to participate via email. Framed as a study aimed at boosting developer productivity by understanding how they allocate their time in a workday, the survey received 510 complete responses (responses rate of 8.5\%). After finishing the survey, the participants could enter a sweepstake to win one out of ten \$50 Amazon.com Gift Cards.
\textcolor{blue}{description of ethics}.

The main questions in the survey were as follows:
\begin{enumerate}
    \item Their roles and years of experience in the industry/team
    \item The hours spent on various activities in their typical workweek
    \item Ideally, the percentage of time they would want to allocate to each activity in a workweek
    \item How productive and satisfied were they by their past workweek
    \item Activities they find most cognitively challenging
    \item How often do they use AI tools to assist in their daily activities
    \item Two open-ended questions about the activities they would want to automate using AI tools, and advice for new hires to boost their productivity and satisfaction levels 
\end{enumerate}



\subsection{Data Analysis \& Exploration}
% Here, we could start with discussing the survey group:
% - demographic observations
% - distribution of participants (based on the years experience in the industry/team), 

From the exploratory interviews, we identified sixteen key activities, which were subsequently used to quantify the developers' time allocation across their work week. 

\subsection{Limitations}

\section{Results} \label{sec:results}

% \begin{figure*}[htpb!]
% \label{}
% \centering

%     {{\label{ROCIowaCedar} \includegraphics[width=\textwidth/3]{figures/IowaCedar_roc.png}}}%
%     \qquad
%     {{\label{ROCIowaDesMoines} \includegraphics[width=\textwidth/3]{figures/IowaDesMoines_roc.png} }%
%   \captionsetup{justification=centering}
%   \caption{\Acf{ROC} curves for \acf{RW} Iowa (CR) and  \acf{RW} Iowa (DM) dataset. Dummy model here represents a model whose output is solely a ``no Flood'' for all pixels.}
%   \label{fig:RW_ROC_Curves}%
% \end{figure*}



\section{Results and Discussions}
\label{sec:Results}

In this section, we aim to answer three main questions. First, we want to validate our hypothesis that \ac{SYN} data is a viable proxy for \ac{RW} data when training ML models for downscaling. Secondly, we seek to assess how much more skillful ML-based downscaling is compared to classical, non-data-driven techniques, such as our baseline methods, \textit{i.e.}, thresholded bicubic and Lanczos interpolation. Finally, we would like to appraise the extent to which data-driven models like ours are transferable (in terms of usefulness) to other regions without major performance degradations.  
To assess the quality of the models, we conduct a multiple comparison test --namely the Holm-Bonferroni procedure \cite{HolmBonferroni1979} -- that is designed to control the \ac{FWER}. We notice that, with a \ac{FWER} of $10^{-3}$, all the differences in model performance are significant. The only exception to this trend was observed in \ac{RW}-GH for whom the pairwise differences between \ac{RCAN} and \ac{ESRT}, Lanczos and Bicubic were not significant with the aforementioned \ac{FWER}. 

%Finally, we aim to find out the factors influencing the transferability of our models from one region to another.

\subsection{Potential of using SYN Data for RW downscaling}

In order to evaluate the utility of synthetic data for training, we compare performances of our candidate models on both \ac{SYN} and \ac{RW} Iowa data whose results are presented in Table \ref{tab:IowaResults}. We notice that 
\textbf{(i)} For the Iowa datasets, there is a drop in performance of all the models when going from \ac{SYN} to \ac{RW} datasets, 
\textbf{(ii)} for the \ac{RW}-IA (CR) as well as \ac{RW}-IA (DM) datasets, both bicubic and Lanczos interpolation have accuracies and MCC up to 70.89\% and 0.42 respectively while the deep learning models have accuracies and MCC up to 73.34\% and 0.46 respectively, 
\textbf{(iii)} There is a roughly 6\% accuracy improvement for the \ac{SYN} data for the deep learning models compared to the bicubic and lanczos models and this improvement drops to about 3\% for \ac{RW} data,  
\textbf{(iv)} the performance of all the models remain consistent across both \ac{RW}-IA datasets and \textbf{(v)} in \figref{fig:RW_ROC_Curves}, we observe that there is a high degree of overlap among the \ac{ROC} curves for the data-driven models.

From (i) and (iv) we can conclude that \ac{SYN} data is more intricate than \ac{RW} data. This implies that the benefits yielded by training with \ac{SYN} dataset, while significant, is not as prominent in the \ac{RW} Iowa datasets. 
% This may be due to sensor noise prevalent in the \ac{RW} Landsat-8 data that can be harder to reproduce in the synthetically generated examples. 
(i), (iii) and (v) implies that while \ac{SYN} data is not an exact replacement for \ac{RW} data, it provides a rather significant edge, which is all the more important when there is insufficient \ac{RW} for training. From (ii) we can conclude that the three proposed data driven models outperform classical super-resolution techniques such as bicubic and lanczos, conclusion supported by the \ac{ROC} curves in Figure \ref{fig:RW_ROC_Curves} for whom the data-driven models, in general, lie above the non-data-driven alternatives. Observation (iv) shows that  for the climatically similar \ac{RW}-Iowa(CR) and \ac{RW}-Iowa(DM) regions, training on \ac{SYN} Iowa data does indeed provide an edge. 

% have similar climate. 

\begin{figure*}[t!]
    \centering
    \begin{subfigure}[t]{0.5\textwidth}
        \centering
        \includegraphics[width=\textwidth/2]{figures/IowaCedar_roc.png}
        \caption{}
    \end{subfigure}%
    ~ 
    \begin{subfigure}[t]{0.5\textwidth}
        \centering
        \includegraphics[width=\textwidth/2]{figures/IowaDesMoines_roc.png}
        \caption{}
    \end{subfigure}
    \vspace*{0.5cm}
    \caption{    \label{fig:RW_ROC_Curves} \Acf{ROC} curves for (a) RW-IA (CR) and (b) RW-IA (DM) dataset. Na\"ive model here represents a model whose output is solely a ``no Flood'' for all pixels. Star here represents the pixel-wise classifier with a threshold of 0.5.}
\end{figure*}


\subsection{Effectiveness of data-driven approaches}

In order to evaluate the effectiveness of ML models in the downscaling task, we compare performances of our candidate models to Lanczos and bicubic interpolation methods by looking at figures of some sample predictions from Iowa (Figure \ref{fig:RWIowaDesMoines}), performance comparison in the region of Iowa in Table \ref{tab:IowaResults} and the ROC curves in Figure \ref{fig:RW_ROC_Curves} for \ac{RW} data. We notice that 
\textbf{(vi)} For RW-IA (DM) samples, the deep learning models maintain a higher degree of spatial continuity in the predicted \ac{FIM}, 
\textbf{(vii)} We observe that  bicubic and Lanczos interpolation produces over-smoothed \ac{FIM} reconstructions, while the plain \ac{RDN}, \ac{RCAN} and \ac{ESRT} models are more detail-inclusive. Similar conclusions can be drawn upon inspecting the \ac{ROC} curves in Figure \ref{fig:RW_ROC_Curves} and 
\textbf{(viii)} For RW-IA (CR), the ML models show a performance improvement of 3.06\% when comparing the best ML model and non-data-driven method and, while for RW-IA (DM) there is a performance improvement of 2.45\%.


Figures \ref{fig:EUSamples} and \ref{fig:RWIowaDesMoines} show the spatial disparity among the models whose details are often obscured in aggregated metrics such as accuracy. (vi) This implies that these data-driven models are better are recognizing an underlying stream network geometry than the classical methods. However, when it comes to narrow river streams, all the models struggle capturing the nuances of the \ac{FIM} resultant from localized high elevation features such as small islands within rivers or man-made structures. (vii) shows a clear advantage of our data-driven approaches over the non-data-driven alternatives. (viii) indicates the benefits of the data-driven models when evaluated over Iowa. 



\subsection{Applicability of our models to external regions}

To evaluate how transferable our models are, we draw conclusions from figures of the sample predictions from Western Europe (Figure \ref{fig:EUSamples}) and Ghana (Figure \ref{fig:GhanaSamples}) as well as the performance comparison in Table \ref{tab:ExternalResults}. We notice that 
\textbf{(ix)} for Ghana all of the models fail to adequately inundate the pixels over separated areas on account of several disconnected regions of inundation in the chosen area,
\textbf{(x)} the ML models outperform non-data driven methods for RW-EU, 
\textbf{(xi)} for the RW-EU dataset, there is an improvement of 4.89\% when comparing the accuracy of the best data- and non-data-driven methods, 
\textbf{(xii)} For RW-RR and RW-GH, there is marginal improvement (up to 0.77\% in accuracy) of the ML methods over the non-data driven methods and 
\textbf{(xiii)} For RW-EU, we notice that the ML models produce more connected streams over the non-data-driven models. 

(x) and (xi) implies that the models are transferable when considering hydroclimaticalogically similar regions since Iowa and the Meuse river in Europe lie within mid temperate zones. Similar to the observation (vi) for RW-IA (DM), (xiii) implies that the benefits of the ML model in identifying underlying network streams is also transferable to hydroclimatologically similar regions. In contrast, (xii) and (ix) both imply that the trained ML models struggle to generalize to RW-RR \& RW-GH. We speculate that this may be due to the significant differences in geography and climate when compared to Iowa. 

% More specifically, we notice that Ghana has a lot of disconnected regions when compared to Iowa and Western Europe, possibly indicating a geomorphological dissimilarity. Additionally, in the case of Red River and Ghana, we also speculate that they include drivers to flood inundation that are different from Iowa and Western Europe, which lie within mild temperate zones. Ghana on the other hand has a tropical (dry and hot) climate.

Our study directly implies that good quality synthetic data can be useful surrogates for downscaling low-resolution \acp{WFM} to high-resolution \acp{FIM} in regions, where such data are hard to come by, even when downscaling by a factor of 10. We noticed that such models were readily transferable to climatically similar regions as the region of training. However, Such derived ML models did not feature significantly different transferability when evaluated over hydroclimatologically dissimilar regions, which we attribute to different flood inundation characteristics, primarily at finer scales. A possible avenue to circumvent such issues is to explore additional training approaches that fall under the general area of domain adaptation. Nevertheless, data-driven models are still advantageous (and, hence, preferable) over non-data-driven alternatives in transfer scenarios like the one we considered here. 


%%%%%%%%%%%%%%%%%%%%%%%%%%%%%%% unused text %%%%%%%%%%%%%%%%%%%%%%%%%%%%%%%%%%%%%%%



% \tabref{tab:AccuracyResults} depicts test accuracies obtained by our models on both \ac{SYN} and \ac{RW} data. For Iowan floods, a comparison of \ac{SYN} and \ac{RW} results shows \textbf{(i)} bicubic and Lanczos interpolations remarkably gaining about $3\%$ in accuracy, as well as \textbf{(ii)} \ac{RDN} and \ac{RCAN} remaining relatively stable, while \textbf{(iii)} topography-aware models loosing $2.7\%$ in performance. From (i) one can conclude that \ac{SYN} data are morphologically slightly more intricate than \ac{RW} data. Also, (i) and (ii) likely imply that \ac{SYN} data, excluding topography, can serve as satisfactory surrogates of \ac{RW} data. However, as implied by (iii), our topography-dependent models seems to be particularly sensitive to distributional shifts of their combined inputs (\acp{WFM} and topographic features). More specifically, the topography-informed models' performance edge, while still statistically significant, is extremely marginal, even when compared to our non-data-driven approaches. Next, when comparing results between the cases of Iowan and Ghanaian \ac{RW} data, one observes that \textbf{(iv)} the accuracy of bicubic and Lanczos interpolations drops by almost $5\%$ due to over-smoothing. This may imply that Ghanaian \acp{FIM} bare a more complex morphology, when compared to Iowan \acp{FIM}. Also, \textbf{(v)} our topography-agnostic, data-driven models' performance degrades more gracefully (by about $2\%$), while \textbf{(vi)} our topography-aware models perform, virtually, as bad as our non-data-driven approaches. Hence, the differences in the data populations of the two regions we considered are significant enough to render our topography-dependent models noncompetitive. 




\section{Related Work} \label{sec:related}
\section{Related works}
Implicit Neural Representations are designed to learn continuous representations of target functions by taking advantages of the approximation power of neural networks.
%
Their inherent continuous property can beneficial in many cases like video compression~\citep{chen2021nerv,strumpler2022implicit}, 3D modeling~\citep{park2019deepsdf,atzmon2020sal,9010266,gropp2020implicit,sitzmann2019scene} and volume rendering~\citep{pumarola2021d, barron2021mip,martin2021nerf,barron2023zip}.
%
However, simply employing MLPs may result in spectral bias, where oversmoothed outputs are generated due to the inherent tendency of MLPs to prioritize learning low-frequency components first. Consequently, many studies have focused on these drawbacks and explored various methods to address this issue.
%
The most straightforward way to address this issue is by projecting the coordinates into the higher dimension~\citep{tancik2020fourier, wang2021spline}.
%
However, these methods can lead to noisy outputs if there is a mismatch in the embeddings variance.
%
To address this, \citet{landgraf2022pins} propose dividing the Random Fourier Features into multiple levels of detail, allowing the MLPs to disregard unnecessary high-frequency components. Another type of approach to mitigating the spectral bias introduced by the ReLU activation function, as proposed by \citet{sitzmann2020implicit}, \citet{ramasinghe2022beyond}, \citet{saragadam2023wire}, and \citet{shenouda2024relus}, is to modify the activation function itself by using alternatives such as the Sine function, Wavelets, or a combination of ReLU with other functions. There are also efforts to modify network structures to mitigate spectral bias~\citep{mujkanovic2024neural}. 
%
\citet{lindell2022bacon} introduce a network design that treats MLPs as filters applied to the input of the next layer, known as Multiplicative Filter Networks (MFNs). 
%
Additionally, based on the discrete nature of signals like images and videos, grid-based approaches (e.g., Grid Tangent Kernel~\citep{zhao2024grounding}, DINER~\citep{xie2023diner}, and Fourier Filter Bank~\citep{wu2023neural}) have been proposed to address spectral bias, as the grid property allows for sharp changes in features, which facilitates learning fine details.
Even though, there are some prior works trying to solve the inherent problems of Fourier features embeddings ~\citep{landgraf2022pins, yuce2022structured, hertz2021sape, saratchandran2024sampling}, limited research has addressed both the underlying causes of high-frequency noise and provides a non-heuristic solution even if these embeddings are widely employed into many downstream tasks.

\section{Conclusion} \label{sec:conclusions}
Software development is increasingly conceived as a collaboration activity between developers and AIs. Indeed, IDEs already implement features to enable interactive development, with AI suggesting implementations that are reused by developers.

Although multiple studies show this interaction can be successful, there is still limited understanding of how the models must be configured and used in the context of code generation tasks. This study addresses this gap, systematically investigating the impact of several key parameters, including the repeated submission of a prompt to accommodate for the non-deterministic nature of the models.

Our study reveals several key findings about the usage of ChatGPT. In particular, we discovered how creativity, although up to a limited extent, is useful to increase the range of methods whose code can be generated correctly. A major role is played by parameter top-p, which is commonly underrated, and instead has a major impact on the correctness of the results, with lower values producing better results. Finally, prompts should be submitted multiple times, with $5$ repetitions combined with a temperature of $1.2$ resulting in an effective configuration in our experiments.  

Future work concerns two main research directions. One is about replicating this experiment with other AI assistants, to validate our findings in multiple contexts. The second research direction concerns finding strategies to deal with the need to submit the same prompt multiple times to obtain a useful result, and thus developing approaches able to select or merge multiple responses automatically. 

\section*{Acknowledgment}
This work was partially supported by the MUR under the grant “Dipartimenti di Eccellenza 2023-2027" of the Department of Informatics, Systems and Communication of the University of Milano-Bicocca, Italy; and by the Engineered MachinE Learning-intensive IoT systems (EMELIOT) national research project, which has been funded by the MUR under the PRIN 2020 program (Contract 2020W3A5FY).



\bibliographystyle{IEEEtran}
  \bibliography{main}

\vspace{12pt}


\end{document}
