\section{Related work}
\label{sec:related}

The notion of agony score for  ranking was first proposed by \citet{gupte2011finding} for
the unweighted, directed graphs. The authors provided a
polynomial-time, exact algorithm in which agony minimization problem has been
formulated as an integer linear program. They exploits primal-dual techniques
where the primal problem tries to minimize the agony and  the dual  finds the
maximum eulerian subgraph. Later, Tatti extended agony minimization problem
to handle weights and the cardinality constraint~\cite{nikolaj2017tiers}. At the same time, the author  introduced a
greedy heuristic  to find hierarchies provably  faster.
We should stress that these algorithms do not use any label information whereas
we propose an optimization problem and a heuristic algorithm in which the node
labels are exploited.


The appeal of using agony is that it can be solved in polynomial time.
Alternatively, if we use
constant penalty function for backward edges, the minimization problem is equivalent to \textsc{feedback arc set}
(\fasprb), which is known to be \apx-hard with a coefficient of $c =
1.3606$~\cite{dinur2005hardness}, no known constant-ratio
approximation algorithm with the best known approximation algorithm
yielding a ratio $\bigO{\log n \log \log n}$~\cite{even1998approximating}. Interestingly, while
Tatti~\cite{nikolaj2017tiers} showed
minimizing agony for a convex penalty, that is, $\score{G, r}$ can be done
in polynomial time; however, in this paper, we show that finding label tree
minimizing $\labelscore{G, T}$ turns out to be \np-hard.

Let us now look at alternative methods for ranking nodes in a graph.
We should stress that these methods do not use any label information.

A popular and a classic method for ranking players in competitions, is Elo
ranking, originally proposed as a rating method to rank chess players, based on
the outcomes of tournament games~\cite{elo1978rating}. 

\citet{maiya2009inferring} inferred  rankings, by first inducing interaction
model to the network and then searching maximum likelihood hierarchy  for each
candidate model, with a greedy heuristic.

Another way of finding hierarchies is to use graph theoretic centrality
measures. The notion of  eigenvector centrality score and degree has been
proposed by \citet{memon2008retracted}, assuming that high scores imply  highly
ranked entities. Utilizing a weighted combination of such graph theoretic
measures, including  number of cliques and betweenness  centrality was proposed
by~\citet{rowe2007automated}.
Highly ranked nodes in directed graphs typically have many outgoing edges.
Finding such nodes can be also done using a classic HITS method introduced
by~\citet{kleinberg1999authoritative}, where the score of a hub is based on the quality
of the nodes it points.