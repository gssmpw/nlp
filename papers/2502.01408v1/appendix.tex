\section{Proofs}


\begin{proof}[Proposition~\ref{prop:np}]
We will prove the hardness from $k$-\prbcover, a problem where we are given a
family of subsets $\fm{C}$ of a universe $U$ and we asked if there are at most $k$
sets covering $U$.

Assume that we are given an instance of \prbcover. Our graph is as follows: the
vertices $V$ consists of the items in $U = u_1, \ldots, u_n$ plus one additional
vertex $w$.  We connect each $u_i$ to $w$. The labels $L(u_i)$ are the indices
of the sets in which $u_i$ is included. Finally, $L(w) = \emptyset$.

We claim that there is a label tree $T$ with at most $k + 1$ leaves yielding
$\labelscore{G, T} = 0$ if and only if there is a $k$-cover.

Assume there is a cover $\fm{Z}$. Then let $T$ be a tree with $\abs{\fm{Z}}$ non-leaves.
Each of these nodes have a leaf as the left child, and one node has two leaves
as children. Each non-leaf has a label corresponding to a set in $\fm{Z}$
guiding the nodes in $G$ with the label to the left leaf.
The tree has at most $k + 1$ leaves, moreover the last leaf contains only
$w$ since it does not have any labels and $\fm{Z}$ is a cover. Consequently, all 
edges are forward, and $\labelscore{G, T} = 0$.

On the other hand, if $\labelscore{G, T} = 0$, then the last leaf contains only $w$
since otherwise there is a backward edge. Let $\fm{Z}$ be the sets corresponding
to the labels occurring in $T$. Then $\fm{Z}$ is a cover since otherwise there would
be a node with the same rank as $w$, violating the assumption. Moreover, $\abs{\fm{Z}} \leq k$
proving the claim.
\end{proof}

\begin{proof}[Proposition~\ref{prop:np2}]
We will prove the hardness from $k$-\prbcover.
Assume that we are given an instance of \prbcover:
a universe $U$ of $n$ items, a
family of $m$ subsets $\fm{C}$, and an integer $k$.

Our graph is as follows: the
vertices $V$ consists of the items in $U = u_1, \ldots, u_m$, $m$ vertices representing
the sets $S = s_1, \ldots, s_m$, 
and one additional
vertex $x$.  We connect each $u_i$ to $x$ with a weight $w(e) = m + k(k + 1)$.
We connect $x$ to each $s_i$ with a weight $w(e) = 1$.

The labels $L(u_i)$ are the indices of the sets in which $u_i$ is included. The
single label for $s_i$ is the index of the $i$th set.  Finally, $L(w) =
\emptyset$.

We claim that there is a label tree $T$ yielding
$\labelscore{G, T} \leq m + k(k + 1)/2$ if and only if there is a $k$-cover.

Assume there is a cover $\fm{Z}$. Then let $T$ be a tree with $\abs{\fm{Z}}$ non-leaves.
Each of these nodes have a leaf as the left child, and one node has two leaves
as children. Each non-leaf has a label corresponding to a set in $\fm{Z}$
guiding the nodes in $G$ with the label to the left leaf.
Since $\fm{Z}$ is a cover all 
edges $(u_i, x)$ are forward. Moreover, $k$ vertices in $S$ have ranks $1, \ldots, k$
and the remaining $m - k$ vertices in $S$ have the same rank as $x$, that is $k + 1$. Consequently,
\[
	\labelscore{G, T} = m + k(k + 1)/2\quad.
\]



On the other hand, if $\labelscore{G, T} \leq m + k(k + 1)/2$, then the last leaf contains only $x$
and vertices from $S$, since $w(u_i, x) > m + k(k + 1)/2$.

Let $\fm{Z}$ be the $o$ sets corresponding to the labels occurring in the path to
$x$ from the root, in order. Then $\fm{Z}$ is a cover since otherwise there is a
node in $U$ with the same rank as $x$, violating the assumption. 
Let $z_1, \ldots, z_o$ be the corresponding vertices in $S$.
Then $r(x) - r(z_i) \geq o - i + 1$ since there are at least $o - i$ leaves between $z_i$ and $x$.
Since $x$ has the largest rank, we also have $r(x) \geq r(s_i)$. Consequently,
\[
\begin{split}
	 m + \frac{k(k + 1)}{2} & \geq \labelscore{G, T} = \sum_{i = 1}^m 1 + r(x) - r(s_i) \\
	 & \geq m - o + \sum_{i = 1}^o (o - i + 2) = m + \frac{o(o + 1)}{2}, \\
\end{split}
\]
which shows that $o \leq k$, proving the result.
\end{proof}


\begin{proof}[Proposition~\ref{prop:time}]
$\algpartition(\alpha)$ first finds the best candidate by calling
repeatedly $\algsplit(\alpha, t)$,
and then performs the split by calling $\algleft(\alpha)$.

Since $\algsplit(\alpha, t)$ runs in $\bigO{\abs{V(\alpha, t)}}$ time,
the total time needed to find the candidate for splitting $\alpha$
is $\bigO{\sum_t \abs{V(\alpha, t)}} \subseteq \bigO{R}$.

Next we bound the time needed to update the counters during the split.
Let us define $i_{e\alpha} = 1$ if an edge $e$ is visited during $\algleft(\alpha)$,
and $i_{e\alpha} = 0$ otherwise.
Define also $i_{v\alpha} = 1$ if a node $v$ is visited during for-loop,
and $i_{v\alpha} = 0$ otherwise.

%Similarly, define $i_{e\alpha} = 1$ if and only if an edge $e$ is visited.
Write $n_\alpha = \sum i_{v\alpha}$ and $m_\alpha = \sum i_{e\alpha}$ to be the total number of nodes and edges visited during the split.

Let us define $c_\alpha = \abs{E(\alpha)} + \abs{V(\alpha)}$.
Testing whether $c_\beta \leq c_\gamma$ can be done in $\bigO{n_\alpha + m_\alpha}$, and
splitting $\alpha$ can be done in $\bigO{\abs{V(\alpha, t)} + n_\alpha + m_\alpha}$ time.

First note that $\abs{V(\alpha, t)} \leq R$.
The proposition then follows if we can prove that $\sum_\alpha n_\alpha + m_\alpha \in \bigO{(n + m) \log n}$.
Since 
\[
	\sum_\alpha n_\alpha + m_\alpha = \sum_v \sum_{\alpha} i_{v\alpha} + \sum_e \sum_{\alpha} i_{e\alpha},
\]
we will prove the claim by showing that $\sum_{\alpha} i_{e\alpha} \in \bigO{\log n}$
and  $\sum_{\alpha} i_{v\alpha} \in \bigO{\log n}$

Fix edge $e$, and let $\alpha$ and $\alpha'$ be two nodes for which $i_{e\alpha}
= i_{e\alpha'} = 1$. Then $\alpha'$ is a descendant of $\alpha$, or otherwise.
Assume the former. Then $\alpha'$ is either a child of $\alpha$ or
is a descendant of that child. Let us denote this child by $\beta$ and let $\gamma$
be the other child.

Since $e \in E(\alpha') \subseteq E(\beta)$, we have
$c_\beta \leq c_\gamma$ as otherwise $e$ is not visited
when $\alpha$ is split.
Consequently,
\[
	2 c_{\alpha'} \leq 2 c_{\beta} \leq c_\beta + c_\gamma \leq c_\alpha \quad.
\]

To conclude, let $\set{\alpha_j}$ be the nodes for which $i_{e\alpha_j} = 1$. We have shown that we can
safely assume that $\alpha_{j + 1}$ is a descendant of $\alpha_{j + 1}$. Moreover,
$c_{\alpha_j} \geq 2c_{\alpha_{j + 1}}$. Since $c_{\alpha_1} \leq m + n$,
there can be at most $\bigO{\log m} \subseteq \bigO{\log n}$ nodes.
The argument for $\sum_{\alpha} i_{v\alpha} \in \bigO{\log n}$ is similar.
\end{proof}

\section{Additional experiments}

%%%%%%%%%%%%%%%%%%%%%%%%%%%%%%% given a tree...

\textbf{Synthetic data:}
As an additional experiment with synthetic data, our goal was to showcase how well 
the algorithm finds the ground truth for the dataset where there are no labels
that match the ranks exactly. Instead we considered a ground truth tree
given in Figure~\ref{fig:tree-gt}. 

Note that here
thick lines indicate the availability of the label and
dashed lines represent the unavailability.

We generated a new dataset with $n =
9\,000$, $m = 9\,600$, $h = 9$, and $\eta = 0.75$, in accordance with the true
label-sets derived from the given tree.  The discovered label trees with
noiseless~($\theta = 0$ and  $\mu = 0$)
and
noisy~($\theta = 0.3$ and  $\mu = 0.3$)
parameters are shown in Figure~\ref{fig:tree-noiseless} and
Figure~\ref{fig:tree-noisy} respectively.
When compared to the ground truth,
we achieved the value $1$ for
Kendall's $\tau$ measure without the noise and the
$0.928$ with the noise. Note that the
discovered tree in Figure~\ref{fig:tree-noiseless} is not the same as the ground truth tree
but it will yield the same ranking.


%%%%%FIGURE-4%%%%%%% ground truth

\begin{figure}[ht!]
\begin{center}
% \begin{figure}[t!]
\setlength{\tabcolsep}{0pt}
\begin{tabular}{ll}

\begin{tikzpicture}[baseline = 0pt, yscale=0.7, xscale=1]
\node (n1) at (0, 0) {$\shortstack{L-1}$};

\node (n2)  at (-2, -1.7) {$\shortstack{L-2}
 $};
\node (n3) at (2, -1.7) {$\shortstack{L-3}
 $};

\node (n4) at (-3, -3.4) {$\shortstack{L-4}$};

\node (n10) at (-3.7, -5.0) {$ \shortstack{ L-6}$};
\node (n14) at (-4.0, -6.4) {$ \shortstack{ \color{purple} R-1}$};
\node (n15) at (-3.4, -6.4) {$ \shortstack{ \color{purple} R-2}$};

\node (n11) at (-2.3, -5.0) {$ \shortstack{ L-7}$};
\node (n16) at (-2.6, -6.4) {$ \shortstack{ \color{purple} R-3}$};
\node (n17) at (-2.0, -6.4) {$ \shortstack{ \color{purple} R-4}$};

\node (n5) at (-1, -3.4) {$\shortstack{L-5}$};
\node (n12) at (-.5, -5.0) {$ \shortstack{ \color{purple} R-6}$};
\node (n13) at (-1.5, -5.0) {$ \shortstack{ \color{purple} R-5}$};

\node (n6) at (1, -3.4) {$\shortstack{L-8}$};
\node (n7) at (3, -3.4) {$\shortstack{ \color{purple} R-9}$};

\node (n8) at (0.5, -5) {$ \shortstack{ \color{purple} R-7}$};
\node (n9) at (1.5, -5) {$ \shortstack{ \color{purple} R-8}$};


\draw[yafcolor3, ->, >=latex, thick, dashed, in=90, out=-90] (n1) edge (n2);
\draw[yafcolor3, ->, >=latex, thick, in=90, out=-90] (n1) edge (n3);

\draw[yafcolor3, ->, >=latex, thick, in=90, out=-90] (n2) edge (n4);
\draw[yafcolor3, ->, >=latex, thick, dashed, in=90, out=-90] (n2) edge (n5);

\draw[yafcolor3, ->, >=latex, thick, dashed, in=90, out=-90] (n3) edge (n6);
\draw[yafcolor3, ->, >=latex, thick, in=90, out=-90] (n3) edge (n7);

\draw[yafcolor3, ->, >=latex, thick, in=90, out=-90] (n4) edge (n10);
\draw[yafcolor3, ->, >=latex, thick, dashed, in=90, out=-90] (n4) edge (n11);

\draw[yafcolor3, ->, >=latex, thick, dashed, in=90, out=-90] (n5) edge (n12);
\draw[yafcolor3, ->, >=latex, thick, in=90, out=-90] (n5) edge (n13);

\draw[yafcolor3, ->, >=latex, thick, dashed, in=90, out=-90] (n6) edge (n8);
\draw[yafcolor3, ->, >=latex, thick, in=90, out=-90] (n6) edge (n9);

\draw[yafcolor3, ->, >=latex, thick, in=90, out=-90] (n10) edge (n14);
\draw[yafcolor3, ->, >=latex, thick, dashed, in=90, out=-90] (n10) edge (n15);

\draw[yafcolor3, ->, >=latex, thick, in=90, out=-90] (n11) edge (n16);
\draw[yafcolor3, ->, >=latex, thick, dashed, in=90, out=-90] (n11) edge (n17);


\end{tikzpicture}
\end{tabular}
\caption{Ground truth label tree.}
\label{fig:tree-gt}
\end{center}
\end{figure}


%%%%%%%%%%%%%%%%%%%%%%%%%%%%%%%

% \begin{table}[t]\label{table-tag}
% \centering
% \begin{tabular}{c c}
% \hline
% rank & set of labels  \\
% \hline
% $R-1$  & $\{L-2, L-4, L-6\}$  \\
% $R-2$ &  $\{L-2, L-4\}$  \\
% $R-3$ &  $\{L-2, L-7\}$ \\
% $R-4$ &  $\{L-2\}$ \\
% $R-5$ &  $\{L-5\}$ \\
% $R-6$ & $\{L-9\}$ \\
% $R-7$ & $\{L-1\}$ \\
% $R-8$ & $\{L-1, L-8\}$ \\
% $R-9$ & $\{L-1, L-3\}$ \\
% \hline
% \end{tabular}
% \label{table1}
%  \caption{Sets of ground truth labels corresponding to each rank.}
% \end{table}


%%%%%FIGURE-5%%%%%%% Noiseless

\begin{figure}[ht!]
\begin{center}
% \begin{figure}[t!]
\setlength{\tabcolsep}{0pt}
\begin{tabular}{ll}

\begin{tikzpicture}[baseline = 0pt,yscale=0.75, xscale=0.9]
\node (n0) at (-5, 1.5) {$\shortstack{L-6}$};

\node (n1) at (-4, 0) {$\shortstack{L-2}$};

\node (n18) at (-6, 0) {$\shortstack{ \color{purple} R-1}$};

\node (n2)  at (-2, -1.7) {$\shortstack{L-1}
 $};
\node (n3) at (-6, -1.7) {$\shortstack{L-4}
 $};

\node (n4) at (-3, -3.4) {$\shortstack{L-5}$};

\node (n10) at (-3.6, -5.0) {$ \shortstack{ \color{purple} R-5}$};


\node (n11) at (-2.4, -5.0) {$ \shortstack{ \color{purple} R-6}$};
\node (n16) at (-1.9, -6.4) {$ \shortstack{ \color{purple} R-7}$};
\node (n17) at (-1.1, -6.4) {$ \shortstack{ \color{purple} R-8}$};

\node (n5) at (-1, -3.4) {$\shortstack{L-3}$};
\node (n12) at (-.5, -5.0) {$ \shortstack{ \color{purple} R-9}$};
\node (n13) at (-1.5, -5.0) {$ \shortstack{  L-8}$};

\node (n6) at (-5, -3.4) {$\shortstack{L-7}$};
\node (n7) at (-7, -3.4) {$\shortstack{ \color{purple} R-2}$};

\node (n8) at (-5.5, -5) {$ \shortstack{ \color{purple} R-3}$};
\node (n9) at (-4.5, -5) {$ \shortstack{ \color{purple} R-4}$};


\draw[yafcolor3, ->, >=latex, thick, dashed, in=90, out=-90] (n1) edge (n2);
\draw[yafcolor3, ->, >=latex, thick, in=90, out=-90] (n1) edge (n3);

\draw[yafcolor3, ->, >=latex, thick, dashed, in=90, out=-90] (n2) edge (n4);
\draw[yafcolor3, ->, >=latex, thick, in=90, out=-90] (n2) edge (n5);

\draw[yafcolor3, ->, >=latex, thick, dashed, in=90, out=-90] (n3) edge (n6);
\draw[yafcolor3, ->, >=latex, thick, in=90, out=-90] (n3) edge (n7);

\draw[yafcolor3, ->, >=latex, thick, in=90, out=-90] (n4) edge (n10);
\draw[yafcolor3, ->, >=latex, thick, dashed, in=90, out=-90] (n4) edge (n11);

\draw[yafcolor3, ->, >=latex, thick, in=90, out=-90] (n5) edge (n12);
\draw[yafcolor3, ->, >=latex, thick, dashed, in=90, out=-90] (n5) edge (n13);

\draw[yafcolor3, ->, >=latex, thick, in=90, out=-90] (n6) edge (n8);
\draw[yafcolor3, ->, >=latex, thick, dashed, in=90, out=-90] (n6) edge (n9);


\draw[yafcolor3, ->, >=latex, thick, dashed, in=90, out=-90] (n13) edge (n16);
\draw[yafcolor3, ->, >=latex, thick, in=90, out=-90] (n13) edge (n17);

\draw[yafcolor3, ->, >=latex, thick, in=90, out=-90] (n0) edge (n18);
\draw[yafcolor3, ->, >=latex, thick, dashed, in=90, out=-90] (n0) edge (n1);

\end{tikzpicture}
\end{tabular}
\caption{Discovered label tree obtained from noiseless network~($\theta = 0$ and  $\mu = 0$).
% When $\theta = 0$ and  $\mu = 0$,  Kendall's $\tau$ measure of $1$ is obtained.
}
\label{fig:tree-noiseless}
\end{center}
\end{figure}


%%%%%FIGURE-6%%%%%%%%%% Noisy

\begin{figure}[ht!]
\begin{center}
% \begin{figure}[t!]
\setlength{\tabcolsep}{0pt}
\begin{tabular}{ll}

\begin{tikzpicture}[baseline = 0pt,yscale=0.7, xscale=0.9]
\node (n0) at (-6.5, 1.7) {$\shortstack{L-2}$};

\node (n1) at (-4, 0) {$\shortstack{L-3}$};

\node (n18) at (-9, 0) {$\shortstack{  L-6}$};
\node (n19) at (-10, -1.7) {$\shortstack{  L-40}$};
\node (n20) at (-8, -1.7) {$\shortstack{  L-4}$};

\node (n21) at (-10.5, -3.4) {$\shortstack{ \color{purple} R-1}$};
\node (n22) at (-9.5, -3.4) {$\shortstack{ \color{purple} R-2}$};

\node (n23) at (-8.5, -3.4) {$\shortstack{ \color{purple} R-3}$};
\node (n24) at (-7.5, -3.4) {$\shortstack{  L-7}$};

\node (n25) at (-8, -5) {$ \shortstack{ \color{purple} R-4}$};
\node (n26) at (-7, -5) {$ \shortstack{ \color{purple} R-5}$};



\node (n2)  at (-2.7, -1.7) {$\shortstack{L-13}
 $};
\node (n3) at (-5.3, -1.7) {$\shortstack{L-8}
 $};

\node (n4) at (-3.2, -3.4) {$\shortstack{L-22}$};


\node (n10) at (-3.6, -5.0) {$ \shortstack{ \color{purple} R-10}$};
\node (n11) at (-2.7, -5.0) {$ \shortstack{ \color{purple} R-11}$};


\node (n5) at (-2.2, -3.4) {$\shortstack{\color{purple} R-12}$};


\node (n6) at (-4.8, -3.4) {$\shortstack{ \color{purple} R-9}$};
\node (n7) at (-5.8, -3.4) {$\shortstack{ L-1}$};

\node (n8) at (-6.2, -5) {$ \shortstack{  L-5}$};
\node (n27) at (-6.6, -6.6) {$ \shortstack{ \color{purple} R-6}$};
\node (n28) at (-5.8, -6.6) {$ \shortstack{ \color{purple} R-7}$};
\node (n9) at (-5.3, -5) {$ \shortstack{ \color{purple} R-8}$};


\draw[yafcolor3, ->, >=latex, thick, in=90, out=-90] (n1) edge (n2);
\draw[yafcolor3, ->, >=latex, thick, dashed, in=90, out=-90] (n1) edge (n3);

\draw[yafcolor3, ->, >=latex, thick, dashed, in=90, out=-90] (n2) edge (n4);
\draw[yafcolor3, ->, >=latex, thick, in=90, out=-90] (n2) edge (n5);

\draw[yafcolor3, ->, >=latex, thick, in=90, out=-90] (n3) edge (n6);
\draw[yafcolor3, ->, >=latex, thick, dashed, in=90, out=-90] (n3) edge (n7);

\draw[yafcolor3, ->, >=latex, thick, dashed, in=90, out=-90] (n4) edge (n10);
\draw[yafcolor3, ->, >=latex, thick, in=90, out=-90] (n4) edge (n11);


\draw[yafcolor3, ->, >=latex, thick, dashed, in=90, out=-90] (n7) edge (n8);
\draw[yafcolor3, ->, >=latex, thick, in=90, out=-90] (n7) edge (n9);


\draw[yafcolor3, ->, >=latex, thick, in=90, out=-90] (n0) edge (n18);
\draw[yafcolor3, ->, >=latex, thick, dashed, in=90, out=-90] (n0) edge (n1);

\draw[yafcolor3, ->, >=latex, thick, in=90, out=-90] (n18) edge (n19);
\draw[yafcolor3, ->, >=latex, thick, dashed, in=90, out=-90] (n18) edge (n20);

\draw[yafcolor3, ->, >=latex, thick, in=90, out=-90] (n19) edge (n21);
\draw[yafcolor3, ->, >=latex, thick, dashed, in=90, out=-90] (n19) edge (n22);

\draw[yafcolor3, ->, >=latex, thick, in=90, out=-90] (n20) edge (n23);
\draw[yafcolor3, ->, >=latex, thick, dashed, in=90, out=-90] (n20) edge (n24);

\draw[yafcolor3, ->, >=latex, thick, in=90, out=-90] (n24) edge (n25);
\draw[yafcolor3, ->, >=latex, thick, dashed, in=90, out=-90] (n24) edge (n26);

\draw[yafcolor3, ->, >=latex, thick, in=90, out=-90] (n8) edge (n27);
\draw[yafcolor3, ->, >=latex, thick, dashed, in=90, out=-90] (n8) edge (n28);

\end{tikzpicture}
\end{tabular}
\caption{Discovered label tree obtained from noisy network~($\theta = 0.3$ and  $\mu = 0.3$).
% When $\theta = 30$ and  $\mu = 30$,  Kendall's $\tau$ measure of $0.928$ is obtained. We see $12$ number of hierarchy levels are discovered w.r.t $9$ of ground-truth hierarchy levels.
}
\label{fig:tree-noisy}
\end{center}
\end{figure}

%%%%%%%%%%%%%%%%%%%%%%%%%%%%%%%
\iffalse

\begin{figure}[ht!]
\begin{center}
% \begin{figure}[t!]
\setlength{\tabcolsep}{0pt}
\begin{tabular}{ll}

\begin{tikzpicture}[baseline = 0pt]
\node (n1) at (0, 0) {$\shortstack{Theory}$};

\node (n2)  at (-2, -1.7) {$\shortstack{Braneworld}
 $};
\node (n3) at (2, -1.7) {$\shortstack{Conjecture}
 $};

\node (n4) at (-3, -3.4) {$\shortstack{Holographic}$};

\node (n10) at (-3.7, -5.0) {$ \shortstack{ Gravity}$};
\node (n14) at (-4.0, -6.4) {$ \shortstack{ \color{purple} R-1}$};
\node (n15) at (-3.4, -6.4) {$ \shortstack{ \color{purple} R-2}$};

\node (n11) at (-2.3, -5.0) {$ \shortstack{ Duality}$};
\node (n16) at (-2.6, -6.4) {$ \shortstack{ \color{purple} R-3}$};
\node (n17) at (-2.0, -6.4) {$ \shortstack{ \color{purple} R-4}$};

\node (n5) at (-1, -3.4) {$\shortstack{World}$};
\node (n12) at (-.5, -5.0) {$ \shortstack{ \color{purple} R-6}$};
\node (n13) at (-1.5, -5.0) {$ \shortstack{ \color{purple} R-5}$};

\node (n6) at (1, -3.4) {$\shortstack{Various}$};
\node (n7) at (3, -3.4) {$\shortstack{ \color{purple} R-9}$};

\node (n8) at (0.5, -5) {$ \shortstack{ \color{purple} R-7}$};
\node (n9) at (1.5, -5) {$ \shortstack{ \color{purple} R-8}$};


\draw[yafcolor3, ->, >=latex, dashed, in=90, out=-90] (n1) edge (n2);
\draw[yafcolor3, ->, >=latex, thick, in=90, out=-90] (n1) edge (n3);

\draw[yafcolor3, ->, >=latex, thick, in=90, out=-90] (n2) edge (n4);
\draw[yafcolor3, ->, >=latex, dashed, in=90, out=-90] (n2) edge (n5);

\draw[yafcolor3, ->, >=latex, dashed, in=90, out=-90] (n3) edge (n6);
\draw[yafcolor3, ->, >=latex, thick, in=90, out=-90] (n3) edge (n7);

\draw[yafcolor3, ->, >=latex, thick, in=90, out=-90] (n4) edge (n10);
\draw[yafcolor3, ->, >=latex, dashed, in=90, out=-90] (n4) edge (n11);

\draw[yafcolor3, ->, >=latex, dashed, in=90, out=-90] (n5) edge (n12);
\draw[yafcolor3, ->, >=latex, thick, in=90, out=-90] (n5) edge (n13);

\draw[yafcolor3, ->, >=latex, dashed, in=90, out=-90] (n6) edge (n8);
\draw[yafcolor3, ->, >=latex, thick, in=90, out=-90] (n6) edge (n9);

\draw[yafcolor3, ->, >=latex, thick, in=90, out=-90] (n10) edge (n14);
\draw[yafcolor3, ->, >=latex, dashed, in=90, out=-90] (n10) edge (n15);

\draw[yafcolor3, ->, >=latex, thick, in=90, out=-90] (n11) edge (n16);
\draw[yafcolor3, ->, >=latex, dashed, in=90, out=-90] (n11) edge (n17);


\end{tikzpicture}
\end{tabular}
\caption{Label tree for \dtname{Physics-citation} dataset using Algorithm \ref{alg:parttion-fast}. Each node represents the name of the optimal label, $t_{min}$ for the next split. Thick lines indicate a set of vertices which contains $t_{min}$ and dashed lines represents vice a set of vertices without $t_{min}$.}
\label{fig:tree-physics}
\end{center}
\end{figure}

\fi
%%%%%%%%%%%%%%%%%%%%%%%%%%%%%%%
\iffalse

\begin{figure}[ht!]
\begin{center}
% \begin{figure}[t!]
\setlength{\tabcolsep}{0pt}
\begin{tabular}{ll}

\begin{tikzpicture}[baseline = 0pt]
\node (n1) at (0, 0) {$\shortstack{Large}$};

\node (n2)  at (-2, -1.7) {$\shortstack{Recommendation}
 $};
\node (n3) at (2, -1.7) {$\shortstack{Noise}
 $};

\node (n4) at (-3.2, -3.4) {$\shortstack{Attention}$};

\node (n10) at (-4.1, -5.0) {$ \shortstack{ Personalized}$};
\node (n14) at (-4.8, -6.4) {$ \shortstack{ Knowledge}$};
\node (n18) at (-5.4, -7.6) {$ \shortstack{\color{purple} R-1}$};
\node (n19) at (-4.2, -7.6) {$ \shortstack{ \color{purple} R-2}$};

\node (n15) at (-3.4, -6.4) {$ \shortstack{ \color{purple} R-3}$};

\node (n11) at (-2.3, -5.0) {$ \shortstack{ Exploit}$};
\node (n16) at (-2.7, -6.4) {$ \shortstack{ \color{purple} R-4}$};
\node (n17) at (-1.9, -6.4) {$ \shortstack{ \color{purple} R-5}$};

\node (n5) at (-.8, -3.4) {$\shortstack{Based}$};
\node (n12) at (-.3, -5.0) {$ \shortstack{ \color{purple} R-7}$};
\node (n13) at (-1.3, -5.0) {$ \shortstack{ \color{purple} R-6}$};

\node (n6) at (1, -3.4) {$\shortstack{Hypertext}$};
\node (n7) at (3, -3.4) {$\shortstack{ \color{purple} R-10}$};

\node (n8) at (0.5, -5) {$ \shortstack{ \color{purple} R-8}$};
\node (n9) at (1.5, -5) {$ \shortstack{ \color{purple} R-9}$};


\draw[yafcolor3, ->, >=latex, dashed, in=90, out=-90] (n1) edge (n2);
\draw[yafcolor3, ->, >=latex, thick, in=90, out=-90] (n1) edge (n3);

\draw[yafcolor3, ->, >=latex, thick, in=90, out=-90] (n2) edge (n4);
\draw[yafcolor3, ->, >=latex, dashed, in=90, out=-90] (n2) edge (n5);

\draw[yafcolor3, ->, >=latex, dashed, in=90, out=-90] (n3) edge (n6);
\draw[yafcolor3, ->, >=latex, thick, in=90, out=-90] (n3) edge (n7);

\draw[yafcolor3, ->, >=latex, dashed, in=90, out=-90] (n4) edge (n10);
\draw[yafcolor3, ->, >=latex, thick, in=90, out=-90] (n4) edge (n11);

\draw[yafcolor3, ->, >=latex, dashed, in=90, out=-90] (n5) edge (n12);
\draw[yafcolor3, ->, >=latex, thick, in=90, out=-90] (n5) edge (n13);

\draw[yafcolor3, ->, >=latex, dashed, in=90, out=-90] (n6) edge (n8);
\draw[yafcolor3, ->, >=latex, thick, in=90, out=-90] (n6) edge (n9);

\draw[yafcolor3, ->, >=latex, thick, in=90, out=-90] (n10) edge (n14);
\draw[yafcolor3, ->, >=latex, dashed, in=90, out=-90] (n10) edge (n15);

\draw[yafcolor3, ->, >=latex, thick, in=90, out=-90] (n11) edge (n16);
\draw[yafcolor3, ->, >=latex, dashed, in=90, out=-90] (n11) edge (n17);

\draw[yafcolor3, ->, >=latex, thick, in=90, out=-90] (n14) edge (n18);
\draw[yafcolor3, ->, >=latex, dashed, in=90, out=-90] (n14) edge (n19);


\end{tikzpicture}
\end{tabular}
\caption{Label tree for \dtname{DBLP} dataset using Algorithm \ref{alg:parttion-fast}. Each node represents the name of the optimal label, $t_{min}$ for the next split. Thick lines indicate a set of vertices which contains $t_{min}$ and dashed lines represents vice a set of vertices without $t_{min}$.}
\label{fig:tree-dblp}
\end{center}
\end{figure}

\fi
%%%%%%%%%%%%%%%%%%%%%%%%%%%%%%%

\textbf{Real-world data:}
Let us look at the trees obtained \dtname{Patent-citation} dataset together with their labels. This tree shows $7$ levels. The label for the very first  partition has become a patent class \emph{** Classification Undetermined **}. Then, the tree has been further split to the left side, based on the label $Computers \& communications$ which is a sub category.

\begin{figure}[ht!]
\begin{center}
% \begin{figure}[t!]
\setlength{\tabcolsep}{0pt}
\begin{tabular}{ll}

\begin{tikzpicture}[baseline = 0pt]
\node (n1) at (0, 0) {$\shortstack{** Classification  
\\Undetermined **}$};

\node (n2)  at (-2, -1.7) {$\shortstack{Computers  \& 
\\Communications}
 $};
\node (n3) at (2, -1.7) {$\shortstack{Synthetic Resins \\Natural Rubbers}
 $};

\node (n4) at (-3, -3.4) {$\shortstack{Telephonic \\Comm.}$};
\node (n10) at (-3.5, -5.0) {$ \shortstack{ \color{purple} R-1}$};
\node (n11) at (-2.5, -5.0) {$ \shortstack{ \color{purple} R-2}$};

\node (n5) at (-1, -3.4) {$\shortstack{Transport, \\Mech.}$};
\node (n12) at (-.5, -5.0) {$ \shortstack{ \color{purple} R-4}$};
\node (n13) at (-1.5, -5.0) {$ \shortstack{ \color{purple} R-3}$};

\node (n6) at (1, -3.4) {$\shortstack{Org. Com., \\Chem.}$};
\node (n7) at (3, -3.2) {$\shortstack{ \color{purple} R-7}$};
\node (n8) at (0.5, -5) {$ \shortstack{ \color{purple} R-5}$};
\node (n9) at (1.5, -5) {$ \shortstack{ \color{purple} R-6}$};


\draw[yafcolor3, ->, >=latex, dashed, in=90, out=-90] (n1) edge (n2);
\draw[yafcolor3, ->, >=latex, thick, in=90, out=-90] (n1) edge (n3);

\draw[yafcolor3, ->, >=latex, thick, in=90, out=-90] (n2) edge (n4);
\draw[yafcolor3, ->, >=latex, dashed, in=90, out=-90] (n2) edge (n5);

\draw[yafcolor3, ->, >=latex, thick, in=90, out=-90] (n3) edge (n6);
\draw[yafcolor3, ->, >=latex, dashed, in=90, out=-90] (n3) edge (n7);

\draw[yafcolor3, ->, >=latex, thick, in=90, out=-90] (n4) edge (n10);
\draw[yafcolor3, ->, >=latex, dashed, in=90, out=-90] (n4) edge (n11);

\draw[yafcolor3, ->, >=latex, thick, in=90, out=-90] (n5) edge (n12);
\draw[yafcolor3, ->, >=latex, dashed, in=90, out=-90] (n5) edge (n13);

\draw[yafcolor3, ->, >=latex, dashed, in=90, out=-90] (n6) edge (n8);
\draw[yafcolor3, ->, >=latex, thick, in=90, out=-90] (n6) edge (n9);

\end{tikzpicture}
\end{tabular}
\caption{Label tree for \dtname{Patent-citation} dataset using Algorithm \ref{alg:parttion-fast}. Solid lines indicate the branch for the nodes with the corresponding label.}
\label{fig:tree-patent}
\end{center}
\end{figure}

%%%%%%%%%%%%%%%%%%%%%%%%%%%%%%%

Figure~\ref{fig:tree-flickr} shows the label tree obtained with \dtname{Flickr} dataset. This tree shows $5$ hierarchy levels in total. We can observe that the nodes with label \emph{G-104} is ranked higher with respect to the nodes which include \emph{G-134} as its label.

\begin{figure}[ht!]
\begin{center}
% \begin{figure}[t!]
\setlength{\tabcolsep}{0pt}
\begin{tabular}{ll}

\begin{tikzpicture}[baseline = 0pt, yscale=0.7, xscale=1]
\node (n1) at (0, 0) {$\shortstack{G-85}$};

\node (n2)  at (-2, -1.7) {$\shortstack{G-134}
 $};
\node (n3) at (2, -1.7) {$\shortstack{G-104}
 $};
 \node (n6) at (1, -3.4) {$\shortstack{\color{purple} R-4}$};
 \node (n7) at (3, -3.4) {$\shortstack{\color{purple} R-5}$};

\node (n4) at (-3, -3.4) {$\shortstack{\color{purple} R-1}$};


\node (n5) at (-1, -3.4) {$\shortstack{G-150}$};
\node (n12) at (0, -5.0) {$ \shortstack{ \color{purple} R-3}$};
\node (n13) at (-2, -5.0) {$ \shortstack{ \color{purple} R-2}$};


\draw[yafcolor3, ->, >=latex, thick, in=90, out=-90] (n1) edge (n2);
\draw[yafcolor3, ->, >=latex, dashed, in=90, out=-90] (n1) edge (n3);

\draw[yafcolor3, ->, >=latex, thick, in=90, out=-90] (n2) edge (n4);
\draw[yafcolor3, ->, >=latex, dashed, in=90, out=-90] (n2) edge (n5);

\draw[yafcolor3, ->, >=latex, dashed, in=90, out=-90] (n3) edge (n6);
\draw[yafcolor3, ->, >=latex, thick, in=90, out=-90] (n3) edge (n7);

\draw[yafcolor3, ->, >=latex, dashed, in=90, out=-90] (n5) edge (n12);
\draw[yafcolor3, ->, >=latex, thick, in=90, out=-90] (n5) edge (n13);



\end{tikzpicture}
\end{tabular}
\caption{Label tree for \dtname{Flickr} dataset using Algorithm \ref{alg:parttion-fast}. Solid lines indicate the branch for the nodes with the corresponding label.}
\label{fig:tree-flickr}
\end{center}
\end{figure}

%%%%%%%%%%%%%%%%%%%%%%%%%%%%%%%

Let us look at Figure~\ref{fig:tree-cora} which demonstrates the label tree obtained for \dtname{Cora} dataset. This tree shows $3$ hierarchy levels in total. Based on this tree, we can argue that the  case-based papers are ranked higher with respect to the theoretical papers.


\begin{figure}[ht!]
\begin{center}
% \begin{figure}[t!]
\setlength{\tabcolsep}{0pt}
\begin{tabular}{ll}

\begin{tikzpicture}[baseline = 0pt, yscale=0.7, xscale=1]
\node (n1) at (0, 0) {$\shortstack{Case based}$};

\node (n2)  at (-2, -1.7) {$\shortstack{Theory}
 $};
\node (n3) at (2, -1.7) {$\shortstack{\color{purple} Rank 3}
 $};

\node (n4) at (-3, -3.4) {$\shortstack{\color{purple} Rank 1}$};
\node (n5) at (-1, -3.4) {$\shortstack{\color{purple} Rank 2}$};



\draw[yafcolor3, ->, >=latex, dashed, in=90, out=-90] (n1) edge (n2);
\draw[yafcolor3, ->, >=latex, thick, in=90, out=-90] (n1) edge (n3);

\draw[yafcolor3, ->, >=latex, thick, in=90, out=-90] (n2) edge (n4);
\draw[yafcolor3, ->, >=latex, dashed, in=90, out=-90] (n2) edge (n5);




\end{tikzpicture}
\end{tabular}
\caption{Label tree for \dtname{Cora} dataset using Algorithm \ref{alg:parttion-fast}. Solid lines indicate the branch for the nodes with the corresponding label.}
\label{fig:tree-cora}
\end{center}
\end{figure}


%%%%%%%%%%%%%%%%%%%%%%%%%%%%%%%
Next we observe Figure~\ref{fig:tree-opal} which provides the label tree constructed using \dtname{EIES} dataset. This tree shows $4$ hierarchy levels in total. Based on the interactions between researchers, our algorithm classifies the researchers with citation index greater than $100$ in higher rank with respect to the researchers with citation index between $10$ and $20$.

\begin{figure}[ht!]
\begin{center}
% \begin{figure}[t!]
\setlength{\tabcolsep}{0pt}
\begin{tabular}{ll}

\begin{tikzpicture}[baseline = 0pt, yscale=0.7, xscale=1]
\node (n1) at (0, 0) {$\shortstack{Citation Index 10-20}$};

\node (n2)  at (-2, -1.7) {$\shortstack{\color{purple} Rank 1}
 $};
\node (n3) at (2, -1.7) {$\shortstack{Citation Index $>$ 100}$};

\node (n4) at (0.8, -3.4) {$\shortstack{Citation Index $>$ 50}$};
\node (n6) at (0.0, -5.0) {$\shortstack{\color{purple} Rank 2}$};
\node (n7) at (1.6, -5.0) {$\shortstack{\color{purple} Rank 3}$};

\node (n5) at (3.2, -3.4) {$\shortstack{\color{purple} Rank 4}$};



\draw[yafcolor3, ->, >=latex, thick, in=90, out=-90] (n1) edge (n2);
\draw[yafcolor3, ->, >=latex, dashed, in=90, out=-90] (n1) edge (n3);

\draw[yafcolor3, ->, >=latex, dashed, in=90, out=-90] (n3) edge (n4);
\draw[yafcolor3, ->, >=latex, thick, in=90, out=-90] (n3) edge (n5);

\draw[yafcolor3, ->, >=latex, dashed, in=90, out=-90] (n4) edge (n6);
\draw[yafcolor3, ->, >=latex, thick, in=90, out=-90] (n4) edge (n7);


\end{tikzpicture}
\end{tabular}
\caption{Label tree for \dtname{EIES} dataset using Algorithm \ref{alg:parttion-fast}. Solid lines indicate the branch for the nodes with the corresponding label.}
\label{fig:tree-opal}
\end{center}
\end{figure}



%%%%%%%%%%%%%%%%%%%%%%%%%%%%%%%

Next, Figure~\ref{fig:tree-citeseer} demonstrates the label tree obtained using \dtname{Citeseer} dataset. This tree shows $3$ hierarchy levels in total. We can see that
the papers in \emph{Agents} category are cited more than the papers which are categorized under \emph{Machine Learning}.

\begin{figure}[ht!]
\begin{center}
% \begin{figure}[t!]
\setlength{\tabcolsep}{0pt}
\begin{tabular}{ll}

\begin{tikzpicture}[baseline = 0pt, yscale=0.7, xscale=1]
\node (n1) at (0, 0) {$\shortstack{Agents}$};

\node (n2)  at (-2, -1.7) {$\shortstack{Machine Learning}
 $};
\node (n3) at (2, -1.7) {$\shortstack{\color{purple} Rank 3}
 $};

\node (n4) at (-3, -3.4) {$\shortstack{\color{purple} Rank 1}$};
\node (n5) at (-1, -3.4) {$\shortstack{\color{purple} Rank 2}$};



\draw[yafcolor3, ->, >=latex, dashed, in=90, out=-90] (n1) edge (n2);
\draw[yafcolor3, ->, >=latex, thick, in=90, out=-90] (n1) edge (n3);

\draw[yafcolor3, ->, >=latex, thick, in=90, out=-90] (n2) edge (n4);
\draw[yafcolor3, ->, >=latex, dashed, in=90, out=-90] (n2) edge (n5);




\end{tikzpicture}
\end{tabular}
\caption{Label tree for \dtname{Citeseer} dataset using Algorithm \ref{alg:parttion-fast}. Solid lines indicate the branch for the nodes with the corresponding label.}
\label{fig:tree-citeseer}
\end{center}
\end{figure}


%%%%%%%%%%%%%%%%%%%%%%%%%%%%%%%
Next we focus on Figure~\ref{fig:tree-student} which shows the label tree obtained \dtname{Student} dataset. This tree shows $3$ hierarchy levels in total. Based on the contacts between students, it is evident that  mathematics and physics class students are highly sorted than biology class students.

\begin{figure}[ht!]
\begin{center}
% \begin{figure}[t!]
\setlength{\tabcolsep}{0pt}
\begin{tabular}{ll}

\begin{tikzpicture}[baseline = 0pt, yscale=0.7, xscale=1]
\node (n1) at (0, 0) {$\shortstack{Biology-1}$};

\node (n2)  at (-2, -1.7) {$\shortstack{\color{purple} Rank 1}
 $};
\node (n3) at (2, -1.7) {$\shortstack{Mathematics and Physics}$};

\node (n4) at (0.8, -3.4) {$\shortstack{\color{purple} Rank 2}$};


\node (n5) at (3.2, -3.4) {$\shortstack{\color{purple} Rank 3}$};



\draw[yafcolor3, ->, >=latex, thick, in=90, out=-90] (n1) edge (n2);
\draw[yafcolor3, ->, >=latex, dashed, in=90, out=-90] (n1) edge (n3);

\draw[yafcolor3, ->, >=latex, dashed, in=90, out=-90] (n3) edge (n4);
\draw[yafcolor3, ->, >=latex, thick, in=90, out=-90] (n3) edge (n5);


\end{tikzpicture}
\end{tabular}
\caption{Label tree for \dtname{Student} dataset using Algorithm \ref{alg:parttion-fast}. Solid lines indicate the branch for the nodes with the corresponding label.}
\label{fig:tree-student}
\end{center}
\end{figure}
