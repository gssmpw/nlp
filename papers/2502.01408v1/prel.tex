\section{Preliminary notation and problem definition}\label{sec:prel}


The main input to our problem is a \emph{weighted, directed, and labeled graph} which we 
denote by $G = (V, E, w, L)$, where $V$ is a set of nodes, $E$ is a set of edges, $w$ is a function mapping an edge to a real
positive number. If $w$ is not provided, we assume that each edge has a weight
of 1. $L$ is a \emph{label} function which assigns a set of labels from $U$ to each vertex, where $U$ is the label universe. Note that, each node can have multiple labels. We  often denote $n = \abs{V}$ and $m = \abs{E}$. Given a subset of nodes $W$, we will write $E(W)$ to be the edges
having both endpoints in $W$. Given two subsets of nodes $W_1$ and $W_2$, we write $E(W_1, W_2)$
to be the edges that go from $W_1$ to $W_2$.

Given $G= (V, E, w, L)$, our goal is to rank a set of vertices $V$.  We
express this ranking with a \emph{rank assignment} $r$; a function
mapping a vertex to an integer. 

Given a graph $G$ and a rank assignment $r$, we  say that an edge
$(u, v)$ is \emph{forward} if $r(u) < r(v)$, otherwise edge is
\emph{backward}, even if $r(u) = r(v)$.
Ideally, rank assignment $r$ should not have backward edges, meaning that, for any
$(u, v) \in E$ we should have $r(u) < r(v)$. 

Next, we define the
penalty function  $\pen{}$ which penalizes the backward edges based on the severity of the violation. The penalty
for a single edge $(u, v)$ is  equal to $\pen{d} = \max(0, d + 1)$, where $d = r(u) - r(v)$. The forward edges will always receive $0$ penalty. Whenever $d \geq 0$, then the backward edges will receive $d+1$ amount of penalty. 
For example, an edge $(u, v)$ with $r(u) = r(v)$ has a penalty score of $1$. if $r(u) = r(v) + 1$, then the penalty becomes  $2$, and so on.

Next, we define a score for the ranking, to be the  sum of all backward edge penalties multiplied by its weight as follows.
\begin{definition}
Assume a weighted directed graph $G = (V, E, w)$, a rank assignment $r$, and a penalty function $\pen{}$.
We define \emph{agony} score $\score{G, r}$ as 
\[
	\score{G, r} = \sum_{e = (u, v) \in E} w(e) \pen{r(u) - r(v)}\quad.
\]
\end{definition}

Finding rank assignment minimizing agony can be done in polynomial time~\citep{gupte2011finding,nikolaj2017tiers}.

As mentioned earlier we are interested in explainable rankings. Let us now formalize this notion.

\begin{definition}
We define a \emph{label tree}, $T$ as follows. 
Label tree is an ordered, binary tree, where each leaf node represents a rank, which is an index starting from the leftmost leaf to the rightmost leaf. 
Each of the non-leaf node has a label $t$
and a Boolean value $c$.
\end{definition}

We use the label tree $T$ to partition the vertices of $G$ by traversing them
from the root to a leaf. Let $\alpha$ be a non-leaf node in $T$ with a label $t$ and a boolean value $c$.
If $c$ is true and, then a node $v$ with $t \in L(v)$ traverses to the left branch,
otherwise to the right. If $c$ is false, the branch roles are reversed.
%The index of the leaf can be found while  traversing the label tree from left to right.

\begin{figure}
\begin{tikzpicture}

\draw[thick,fill=yafcolor2!50!white, draw=yafcolor2] (0,0) circle (7pt);
\draw[thick,fill=yafcolor1!50!white, draw=yafcolor1] (0,0) --  +(0:7pt) arc (0:180:7pt) -- cycle;
\node[text=white, circle, inner sep=2pt] (v1) at (0, 0) {$v_1$};

\draw[thick,fill=yafcolor2!50!white, draw=yafcolor2] (0,-1) circle (7pt);
\draw[thick,fill=yafcolor1!50!white, draw=yafcolor1] (0,-1) --  +(0:7pt) arc (0:180:7pt) -- cycle;
\node[text=white, circle, inner sep=2pt] (v2) at (0, -1) {$v_2$};

\draw[thick,fill=yafcolor2!50!white, draw=yafcolor2] (1.5,0) circle (7pt);
\draw[thick,fill=yafcolor1!50!white, draw=yafcolor1] (1.5,0) --  +(0:7pt) arc (0:180:7pt) -- cycle;
\node[text=white, circle, inner sep=2pt] (v3) at (1.5, 0) {$v_3$};

\draw[thick,fill=yafcolor1!50!white, draw=yafcolor1] (1.5,-1) circle (7pt);
%\draw[thick,fill=yafcolor1!50!white, draw=yafcolor1] (1.5,-1) --  +(0:7pt) arc (0:180:7pt) -- cycle;
\node[text=white, circle, inner sep=2pt] (v4) at (1.5, -1) {$v_4$};

\draw[thick,fill=yafcolor2!50!white, draw=yafcolor2] (0.75,-1.5) circle (7pt);
\draw[thick,fill=yafcolor3!50!white, draw=yafcolor3] (0.75,-1.5) --  +(0:7pt) arc (0:180:7pt) -- cycle;
\node[text=white, circle, inner sep=2pt] (v5) at (0.75, -1.5) {$v_5$};

\draw[thick, ->, >=latex] (v1) edge[bend left=10] (v2);
\draw[thick, <-, >=latex] (v1) edge[bend left=10] (v3);
\draw[thick, ->, >=latex] (v2) edge[bend left=10] (v3);
\draw[thick, ->, >=latex] (v2) edge[bend left=10] (v4);
\draw[thick, ->, >=latex] (v3) edge[bend left=10] (v4);
\draw[thick, ->, >=latex] (v4) edge[bend left=10] (v5);
\draw[thick, ->, >=latex] (v4) edge[bend right=10] (v1);

\begin{scope}[xshift=4cm]
\node[diamond, thick,fill=yafcolor1!50!white, draw=yafcolor1] (d1) at (0,0) {};
\node[diamond, thick,fill=yafcolor2!50!white, draw=yafcolor2] (d2) at (-0.8,-0.8) {};
\node[circle, fill, inner sep=0.5pt] (d3) at (-1.6,-1.6) {};
\node[circle, fill, inner sep=0.5pt] (d4) at (0,-1.6) {};
\node[circle, fill, inner sep=0.5pt] (d5) at (0.8,-0.8) {};

\draw[thick, ->] (d1) edge[out=180, in=90] node[pos=0.5, font=\tiny, sloped, auto, inner sep=0.5pt] {Yes} (d2);
\draw[thick, ->] (d2) edge[out=180, in=90] node[pos=0.5, font=\tiny, sloped, auto, inner sep=0.5pt] {Yes}  (d3);
\draw[thick, ->] (d2) edge[out=0, in=90] node[pos=0.5, font=\tiny, sloped, auto, inner sep=0.5pt] {No}  (d4);
\draw[thick, ->] (d1) edge[out=0, in=90] node[pos=0.5, font=\tiny, sloped, auto, inner sep=0.5pt] {No}  (d5);


\draw[thick,fill=yafcolor2!50!white, draw=yafcolor2] (d3) ++(-16pt, -8pt)  circle (7pt);
\draw[thick,fill=yafcolor1!50!white, draw=yafcolor1] (d3) ++(-16pt, -8pt) --  +(0:7pt) arc (0:180:7pt) -- cycle;
\draw (d3) ++(-16pt, -8pt) node[text=white, circle, inner sep=2pt] {$v_1$};

\draw[thick,fill=yafcolor2!50!white, draw=yafcolor2] (d3) ++(0, -8pt)  circle (7pt);
\draw[thick,fill=yafcolor1!50!white, draw=yafcolor1] (d3) ++(0, -8pt) --  +(0:7pt) arc (0:180:7pt) -- cycle;
\draw (d3) ++(0, -8pt) node[text=white, circle, inner sep=2pt] {$v_2$};

\draw[thick,fill=yafcolor2!50!white, draw=yafcolor2] (d3) ++(16pt, -8pt)  circle (7pt);
\draw[thick,fill=yafcolor1!50!white, draw=yafcolor1] (d3) ++(16pt, -8pt) --  +(0:7pt) arc (0:180:7pt) -- cycle;
\draw (d3) ++(16pt, -8pt) node[text=white, circle, inner sep=2pt] {$v_3$};


\draw[thick,fill=yafcolor1!50!white, draw=yafcolor1] (d4) ++(0, -8pt) circle (7pt);
\draw (d4) ++(0, -8pt) node[text=white, circle, inner sep=2pt] {$v_4$};

\draw[thick,fill=yafcolor2!50!white, draw=yafcolor2] (d5) ++(0, -8pt)  circle (7pt);
\draw[thick,fill=yafcolor3!50!white, draw=yafcolor3] (d5) ++(0, -8pt) --  +(0:7pt) arc (0:180:7pt) -- cycle;
\draw (d5) ++(0, -8pt) node[text=white, circle, inner sep=2pt] {$v_5$};

\end{scope}

\begin{scope}[xshift=6.1cm]
\draw[thick,fill=yafcolor2!50!white, draw=yafcolor2] (-0.2,0) circle (7pt);
\draw[thick,fill=yafcolor1!50!white, draw=yafcolor1] (-0.2,0) --  +(0:7pt) arc (0:180:7pt) -- cycle;
\node[text=white, circle, inner sep=2pt] (v1) at (-0.2, 0) {$v_1$};


\draw[thick,fill=yafcolor2!50!white, draw=yafcolor2] (0.75,0) circle (7pt);
\draw[thick,fill=yafcolor1!50!white, draw=yafcolor1] (0.75,0) --  +(0:7pt) arc (0:180:7pt) -- cycle;
\node[text=white, circle, inner sep=2pt] (v2) at (0.75, 0) {$v_2$};

\draw[thick,fill=yafcolor2!50!white, draw=yafcolor2] (1.7,0) circle (7pt);
\draw[thick,fill=yafcolor1!50!white, draw=yafcolor1] (1.7,0) --  +(0:7pt) arc (0:180:7pt) -- cycle;
\node[text=white, circle, inner sep=2pt] (v3) at (1.7, 0) {$v_3$};

\draw[thick,fill=yafcolor1!50!white, draw=yafcolor1] (0.75,-1) circle (7pt);
\node[text=white, circle, inner sep=2pt] (v4) at (0.75, -1) {$v_4$};

\draw[thick,fill=yafcolor2!50!white, draw=yafcolor2] (0.75,-2) circle (7pt);
\draw[thick,fill=yafcolor3!50!white, draw=yafcolor3] (0.75,-2) --  +(0:7pt) arc (0:180:7pt) -- cycle;
\node[text=white, circle, inner sep=2pt] (v5) at (0.75, -2) {$v_5$};

\node[anchor=east, font=\tiny, inner sep=1pt] at (v1.west) {$r = 1$};
\node[anchor=east, font=\tiny, inner sep=1pt] at (v4.west) {$r = 2$};
\node[anchor=east, font=\tiny, inner sep=1pt] at (v5.west) {$r = 3$};

\draw[thick, ->, >=latex, yafcolor4] (v1) edge[bend left=0] (v2);
\draw[thick, <-, >=latex, yafcolor4] (v1) edge[bend left=24] (v3);
\draw[thick, ->, >=latex, yafcolor4] (v2) edge[bend left=0] (v3);
\draw[thick, ->, >=latex] (v2) edge[bend left=0] (v4);
\draw[thick, ->, >=latex] (v3) edge[bend left=0] (v4);
\draw[thick, ->, >=latex] (v4) edge[bend left=0] (v5);
\draw[thick, ->, >=latex, yafcolor4] (v4) edge[bend right=0] (v1);
\end{scope}


\end{tikzpicture}
\caption{Graph $G$, label tree $T$, and the resulting hierarchy. The agony $\labelscore{G, T} =5$.}
\label{fig:toy}
\end{figure}



The following notations will be useful in the next section. Given a graph $G$,
a label tree $T$, and a node $\alpha$ in $T$, we write $V(\alpha)$ to be the nodes in $G$
that traverse through or end up in $\alpha$. We also write $E(\alpha) = E(V(\alpha))$.
Given a label $t$, we write $V(\alpha, t)$ to be the nodes in $V(\alpha)$ that have label $t$.
Finally, given a label $t$ and Boolean value $c$, we write $V(\alpha, t, c)$ to
be $V(\alpha, t)$ if $c$ is true, and $V(\alpha) \setminus V(\alpha, t)$ if $c$ is false.

Next, we define the score for the label tree.

\begin{definition}
Let $r(T, X)$ denote the index of the leaf after  traversing the label tree $T$ using $X$, where  $X$ is the label set. We define the \emph{agony score} $\labelscore{G, T}$ as
\[
	\labelscore{G, T} = \sum_{e = (u, v) \in E} w(e) \pen{r(T, L(u)) - r(T, L(v))}\quad.
\]
\end{definition}

Finally, we state our main optimization problem.

\begin{problem}[\prblagy]
Given a weighted, directed, labeled graph  $G = (V, E, w, L)$ and an integer $k$, 
find a label tree $T$, with at most $k$ number of leaves, which minimizes the label agony $\labelscore{G, T}$.
\end{problem}

An example of a graph and a tree is given in Fig.~\ref{fig:toy}.

We should point out that the cardinality constraint $k$ is an optional parameter,
the problem makes sense even if we set $k = \infty$.
