\section{Concluding remarks}\label{sec:conclusions}

We introduced a novel tree-based, hierarchy mining problem for vertex-labeled,
directed graphs. Here the goal is to rank the nodes into tiers so that
ideally the edges are directing to the lower ranks. The ranking is done with
a decision tree that can only use node labels.


%We proposed an efficient divide-and-conquer heuristic
%algorithm which constructed an ordered binary tree called as \emph{label tree}
%where each leaf represented a rank. Each non-leaf contains a label and a criterion
%which were used to traverse the nodes in the input graph, and consequently rank the nodes.


%where agony score was equal to $0$. We further showed the \np-completeness of the decision version of our problem which implied the inapproximability  due to the fact that if any approximation algorithm existed it could have found the optimal solution.

The goal was to minimize a penalty score known as agony which penalized the edges from higher ranks to lower ranks.
We showed that the construction of such a label tree optimally is \np-hard, or even inapproximable when we limit the number of leaves.
Therefore, we presented a heuristic algorithm which runs in  $\bigO{(n + m)
\log n + \ell R}$, where $R = \sum_v \abs{L(v)}$ is the number of node-label
pairs in given graph, $\ell$ is the number of nodes in the resulting label
tree, and $n$ and $m$ are the number of nodes and edges respectively.  To
enforce the cardinality constraint for number of ranks $k$, we pruned the label
tree such that tree had only $k$ leaves, exploiting dynamic programming
techniques.

The synthetic experiments showed that our approach accurately recovers the
latent hierarchy. The experiments on  real-world networks confirmed that our
proposed label-driven algorithm achieved a good quality ranks
which can be explained by node labels. We discovered
hierarchies reasonably fast in practice. The notion of discovering hierarchies
in labeled networks opens up several lines of work. For example, 
instead of requiring that the nodes in each rank group match exactly to the
label tree we can require that only a large portion of the nodes match the label tree.



%As future work, one can extend our
%work for finding descriptive, \emph{overlapping} hierarchical groups despite of
%the disjoint hierarchical groups.
