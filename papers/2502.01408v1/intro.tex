\section{Introduction}

Many real-world networks naturally inherit a hierarchical structure.
Often times, these interactions among the entities in the network, provide the
means of finding underlying hierarchical organization of the network.
Examples include ranking teams in a football league~\cite{Neumann2018RankingTT},
exploring  rankings in social
networks~\cite{maiya2009inferring,gupte2011finding}, terrorist
networks~\cite{memon2008retracted}, and animal
networks~\cite{jameson1999finding}.


The problem of ranking an individual,  based on its interactions with others,
has drawn attention over past
decades~\cite{gupte2011finding,nikolaj2017tiers,Neumann2018RankingTT}.
More formally, given a graph $G$ we are looking to assign each node $i$, a rank $r(i)$, an integer that minimizes a penalty.
Here the score, called \emph{agony},  penalizes the backward edges $(u, v)$ based on the difference between the ranks $r(u) - r(v)$.

Many practical networks contain information beyond the network
structure. Such labels can be used in
numerous applications such as community
detection and clustering~\cite{pool2014description,
bothorel2015clustering,galbrun2014overlapping,bai2020towards,falih2018community}.
Here, the labels are used to explain the obtained results,
which ideally increase the usability of these results to the practitioners.

In this paper, we propose an approach to infer \emph{explainable} hierarchies in labeled, weighted, directed  networks.
More specifically, we look for a decision tree-like structure, which we refer as label tree. The tree uses labels
to rank the nodes. As a quality of measure we use the agony score.

We show that our problem is \np-hard, and inapproximable even if we limit the
number of leaves in the label tree. This is in contrast with the original
optimization problem: a ranking minimizing agony can be discovered in
polynomial time.

Due to the \np-hardness of our problem, we resort to heuristics
and propose a divide-and-conquer heuristic algorithm in Section~\ref{sec:algorithm} that runs in
$\bigO{(n + m) \log n + \ell R}$ time, where $R = \sum_v \abs{L(v)}$
is the number of node-label pairs in the given graph, $\ell$ is the number of
nodes in the resulting label tree, and $n$ and  $m$ denote the number of nodes
and edges respectively.

The algorithm starts with an empty tree and finds the best split minimizing the
score. This process is continued until we have reached the maximum number of
nodes or the score can no longer be increased. It is important to point out
that the bottleneck here is computing the score while looking for the optimal
split as a naive implementation would require to enumerate over all edges for
every possible candidate.  We avoid unnecessary  enumerating over the edges by
repurposing some results from~\citet{nikolaj2017tiers} and maintaining certain
counters.  Such bookkeeping allows a fast, practical method to find label trees
with low agony score.

Furthermore, we show experimentally in Section~\ref{sec:exp} that the algorithm performs well in
practice, finds the hidden hierarchical structure in synthetic data, and
finds explainable hierarchies in real-world datasets.
Moreover, our algorithm is reasonably fast in practice as we are able to
process large networks with over hundreds of thousands of edges in less than 30 minutes.