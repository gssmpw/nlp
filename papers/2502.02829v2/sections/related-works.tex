%!TEX root = ../main.tex

\section{Related Works}
\label{sec:relatedworks}

\subsection{Contact-rich planning: global methods}
To achieve global optimality in contact-rich planning, two obstacles should be moved: (1) the combinatorial nature of contact mode assignment; (2) nonconvexity introduced by nonlinear dynamics and geometric constraints. To handle (1), mixed-integer convex programming~\cite{ding2020iros-motionplanning-multilegged-mixedinteger,marcucci2020arxiv-warmstart-mixedinteger-mpc}; to handle (2), linearize and smoothing techniques~\cite{pang2023tro-global-planning-contact-rich-quasi-dynamic-contact-models} are adopted. They either sacrifice modelling accuracy, or consumes exponential computational resources in terms of the planning horizon. 
Recently, semidefinite-relaxation based techniques are gradually attracting people's attention, for their modelling flexibility and polynomial complexity. Typically, graph of convex sets

\subsection{Contact-rich planning: other methods}

\subsection{Polynomial optimization and Moment-SOS Hierarchy}

\subsection{Minimizer extraction}