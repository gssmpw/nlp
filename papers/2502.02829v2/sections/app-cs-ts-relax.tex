%!TEX root = ../main.tex

\subsection{Moment-SOS Hierarchy with CS-TS}
\label{app:sec:cs-ts-relax}

Given a graph $G(V, E)$, define:
\begin{align}
\mbS_G = \cbrace{Q\in \mbS^{|V|}\mymid Q_{\beta,\gamma} = Q_{\gamma,\beta} = 0, \forall \beta\ne\gamma, (\beta, \gamma)\notin E}
\end{align}
where the rows and columns of $Q \in \mathbf{S}_G$ are indexed by $V$. Let $\Pi_G$ be the projection form $\mbS^{|V|}$ to the subspace $\mbS_G$. Specifically, forall $Q\in \mbS^{|V|}$:
\begin{align}
	\Pi_G(Q) = \begin{cases}
		Q_{\beta,\gamma}, & \beta = \gamma \text{ or } (\beta, \gamma)\in E\\
		0, & \text{otherwise}
	\end{cases}
\end{align}
If we further define $\Pi_G(\mathbf{S}^{|V|}_+)$ as $\cbrace{\Pi_G(Q) \mymid Q \in \mathbf{S}^{|V|}_+}$, the image of PSD cone with under projection $\Pi_G(\cdot)$, then the Moment-SOS Hierarchy with combined CS and TS can be concretely written as:
\begin{eqnarray}
    \label{eq:gs-cs-ts}
    \min & L_y(f) \\
    \text{s.t.} & L_y\left( M_d(g_j, I_l) \right) \circ B_{d,l,j}^g \in \Pi_{G'_{d,l,j}}(S^{|V_{d,l,j}|}_{+}), \nonumber \\
    & \forall j\in \cbrace{0} \cup \calG_l, l\in [p] \\
    & L_y\left( H_d(h_j, I_l) \right) \circ B_{d,l,j}^h = 0, \forall j\in \calH_l, l\in [p] \\
    & y_{\mathbf{0}} = 1 
\end{eqnarray} 

The above procedure outlines the CS-TS Moment-SOS Hierarchy with a sparse order of $k = 1$. As shown in~\cite{wang2022tms-cs-tssos}, one can further iteratively apply support extension and chordal extension within term sparsity. This process generates new sets $B_{d,l,j}^g$ and $B_{d,l,j}^h$, leading to a two-level hierarchy:  
\begin{enumerate}
    \item The outer level is governed by CS's relaxation order $d$.  
    \item The inner level is controlled by TS's sparse order $k$, which corresponds to the number of iterations used to generate new $B_{d,l,j}^g$ and $B_{d,l,j}^h$.  
\end{enumerate}
Define the optimal value of~\eqref{eq:gs-cs-ts} as $\rho_d^k$. The sequence $\{\rho_d^k\}_{k \geq 1}$ is monotonically non-decreasing and satisfies $\rho_d^k \leq \rho_d$ for all $k$. What's more:

\begin{theorem}[Theorem 4.26 in~\cite{wang2023book-introduction-pop}]
    If we use maximal chordal extension (\ie block closure) for term sparsity, $\rho_d^k \rightarrow \rho_d$ as $k \rightarrow \infty$.
\end{theorem}