%!TEX root = ../main.tex

% \vspace{4mm}
\section{Introduction}
\label{sec:introduction}
\vspace{-1mm}

Contact-rich planning plays a fundamental role in robotics tasks ranging from manipulation to locomotion~\cite{mason1986ijrr-mechanics-planning-pushing,di2018iros-dynamic-locomotion,hirai1998icra-development-honda-humanoid}. At the heart of such planning problems lie two interrelated challenges: (a) \textit{Contact mode selection}: determining when and where to establish or break contact is critical, yet the number of possible contact sequences grows exponentially with the number of contact modes and the planning horizon; (b) \textit{Nonlinear dynamics and nonconvex geometric constraints}: the planned trajectory must satisfy the system's nonlinear dynamics and geometric constraints such as avoiding self and obstacle collisions. Together, these challenges exacerbate the problem's nonconvexity and computational complexity.

\textbf{Problem statement.} Let $N$ represent the planning horizon with $\enum{N} := \{ 1, 2, \dots, N \}$. Define the state trajectory $\{ x_k \}_{k=0}^N \subset \mathbb{R}^{n_x}$ and the control input trajectory $\{ u_k \}_{k=0}^{N-1} \subset \mathbb{R}^{n_u}$. For contact-rich planning problems, we introduce ``\textit{contact variables}'' $\{ \lambda_k \}_{k=0}^{N-1} \subset \mathbb{R}^{n_\lambda}$, which can be interpreted either as a set of binary contact modes or as continuous contact forces (see examples in \S\ref{sec:exp}). With these definitions, we focus on the following general contact-rich planning problem
\begin{subequations} \label{eq:intro:contact-rich}
    \begin{eqnarray}
     \hspace{-5mm} \min_{\substack{ \left\{ x_k \right\}_{k=0}^N, \left\{ u_k \right\}_{k=0}^{N-1} \\ \left\{ \lam{k} \right\}_{k=0}^{N-1} } } & \displaystyle \ell_N(x_N) + \sum_{k=0}^{N-1} \ell_k(x_k, u_k, \lam{k}) \\
     \subject & x_0 = x_\init \\
        & F_k(x_{k-1}, u_{k-1}, \lam{k-1}, x_{k}) = 0, \ k \in \enum{N}  \\
        & (u_{k-1}, \lam{k-1}, x_k) \in \calC_k, \ k \in \enum{N} 
    \end{eqnarray}
\end{subequations}
where $\ell_k, k = 0, \dots , N$ represents instantaneous loss and terminal loss functions. $F_k$ represents the discretized system dynamics obtained from differential algebraic equations and multiple shooting, possibly involving explicit or implicit contact mode switching. $\calC_k$ imposes various types of constraints on $u_{k-1}, \lam{k-1}, x_k$, including (a) control limits; (b) geometric constraints such as collision avoidance; (c) complementarity constraints related to contact. 
A well-known special case of~\eqref{eq:intro:contact-rich} occurs when the dynamics are linear when fixing $\lam{k}$'s. In such case,~\eqref{eq:intro:contact-rich} can be modeled either as mixed-integer linear/quadratic programming~\cite{ding2020iros-motionplanning-multilegged-mixedinteger,marcucci2020arxiv-warmstart-mixedinteger-mpc} or as linear complementarity problems~\cite{aydinoglu2021tro-stabilization-complementary, yunt2006-opttraj-planning-structure-variant}.

In this paper, we do not assume linearity or convexity but assume (a) $\ell_k$ and $F_k$ are polynomial functions; (b) $\calC_k$ is basic semi-algebraic (\ie described by polynomial constraints). Thus,~\eqref{eq:intro:contact-rich} becomes a \textit{polynomial optimization problem} (POP). 

\textbf{Previous methods.}
We briefly review five different methods for solving the contact-rich planning~\eqref{eq:intro:contact-rich}.
(a) \textit{Hybrid MPC}: These methods alternate between contact sequence generation using discrete search~\cite{chen2021iros-traopt-tree-search-multi-contact,wu2020icra-r3t-nonlinear-hybrid,cheng2022icra-contact-mode-quasidynamic,mastalli2020icra-crocoddyl} and continuous-state planning with a fixed sequence.
(b) \textit{Mixed-integer programming}: Contact modes are modeled as binary variables, leading to mixed-integer convex programming (with linear dynamics)~\cite{ding2020iros-motionplanning-multilegged-mixedinteger,marcucci2020arxiv-warmstart-mixedinteger-mpc} or mixed-integer nonconvex programming (with nonlinear dynamics)~\cite{koolen2020arxiv-balance-control-humanoid-nonlinear-centroidal}. These methods scale poorly with the planning horizon, as the worst-case computational complexity grows exponentially with the number of binary variables.
(c) \textit{Dynamics smoothing}: This approach approximates nonsmooth complementarity constraints with smooth surrogate functions, simplifying the problem into a smooth nonlinear programming formulation suitable for local solvers~\cite{chatzinikolaidis2021ral-traopt-contact-rich-implicit-ddp,tassa2014icra-control-limitted-ddp,mordatch2012tog-discovery-complex-behaviors-contact-invariant,tassa2012iros-synthesis-stabilization-online-traopt}. Convex smoothing methods~\cite{pang2023tro-global-planning-contact-rich-quasi-dynamic-contact-models} also exist, at the cost of locally linearizing the dynamics. 
(d) \textit{Contact-implicit methods}: These mainstream methods encode contact modes implicitly through contact forces and complementarity constraints. Numerous local solvers are based on this framework~\cite{aydinoglu2023icra-realtime-multicontact-mpc-admm,yunt2007isdc-combined-continuation-penalty,posa2014ijrr-traopt-directmethod-contact,manchester2020isrr-variational-contact-implicit,yang2024rss-dynamic-on-plam-control-sliding,le2024tro-fast-contact-implicit-mpc}. 
% \hy{Did you cite Zac's bilevel work, which can also be viewed as smoothing?}
However, it is well known that contact-implicit planning problems fail the common constraint qualifications that are crucial for the convergence of numerical solvers.
(e) \textit{Graph of convex sets (GCS)}: As a recently proposed planning framework that explicitly models both discrete and continuous actions, GCS has been extended to contact-rich tasks~\cite{graesdal2024arxiv-tightconvexrelax-contactrich,yang2024arxiv-sdp-linear-piecewise-affine-optimal-control}. These methods can be viewed as an extension of mixed-integer nonconvex optimization with two-level convex relaxations where level one is a semidefinite relaxation and level two involves inequality multiplication. However, when applied to contact-rich planning, they struggle to guarantee both the tightness of the relaxation and fast solution time. Moreover, it is unclear how to simultaneously model the nonlinear dynamics and the geometric constraints in the current GCS framework.

% \emph{Can we solve the general contact-rich planning problem~\eqref{eq:intro:contact-rich} to (near) global optimality efficiently?}

\begin{quote}
    \textit{Is it possible to solve the contact-rich planning problem~\eqref{eq:intro:contact-rich} to (near) global optimality efficiently?}
\end{quote}


\textbf{``Sparse'' Moment-SOS hierarchy}. 
Modeling contact-implicit planning as polynomial optimization (POP) in~\eqref{eq:intro:contact-rich} brings both opportunities and challenges. 
\begin{itemize}
    \item \textbf{Opportunities.} Lasserre's hierarchy of moment and sums-of-squares (SOS) relaxations~\cite{lasserre2001siopt-global} provides a principled and powerful machinery for global optimization of POPs through convex relaxations. Particularly, the Moment-SOS hierarchy generates a series of convex \textit{semidefinite programs} (SDPs) with growing sizes whose optimal values provide nondecreasing \emph{lower bounds} that asymptotically converge to the global minimum of~\eqref{eq:intro:contact-rich}. Combined with a feasible (or locally optimal) solution of~\eqref{eq:intro:contact-rich} that provides an \emph{upper bound} to the global minimum, one can compute increasingly tight (small) \emph{(sub)optimality certificates} by measuring the relative gap between the lower bound and the upper bound. To enhance scalability of the hierarchy, ``\emph{sparse}'' Moment-SOS hierarchy has been proposed to exploit sparsity in the POPs, including correlative sparsity~\cite{lasserre2006msc-correlativesparse,huang2024arxiv-sparsehomogenization} and term sparsity~\cite{wang2021siam-tssos, magron23book-sparse} (see more details in \S\ref{sec:general-sparsity-ksc}). Notably, the Julia package \tssos~\cite{magron2021arxiv-julia-tssos} supports automatic sparsity exploitation as long as the user provides a POP formulation. The recent work~\cite{ teng2024ijrr-convex-geometric-motion-planning} in robotics applied \tssos to several motion planning problems and demonstrated that sparse relaxations can deliver small suboptimality gaps. Furthermore, \cite{kang2024wafr-strom} has shown that all trajectory optimization problems exhibit a generic \emph{chain-like} correlative sparsity pattern and designed a GPU-based ADMM SDP solver that achieves significant speedup than off-the-shelf SDP solvers. \emph{Can we directly apply sparse Moment-SOS relaxations to the contact-implicit planning problem~\eqref{eq:intro:contact-rich}?}
    
    \item \textbf{Challenges.} The answer is unfortunately NO, due to three challenges. First, multiple contact modes will make the chain-like correlative sparsity pattern introduced in~\cite{kang2024wafr-strom} too large to be solved efficiently. Second, \tssos allows exploiting more flexible sparsity patterns but its automatic sparsity exploitation operates like a black box---it does not visualize the sparsity patterns being exploited and it is unclear whether robotics-specific domain knowledge can lead to customized sparsity. Third, as reported in~\cite{teng2024ijrr-convex-geometric-motion-planning}, the suboptimality gaps when using \tssos for many smooth planning problems are already large (above $20\%$), not to mention the extra nonsmoothness and combinatorial complexity brought by contact-implicit planning. As shown in~\cite{teng2023arxiv-geometricmotionplanning-liegroup}, \tssos can fail to extract feasible solutions for certain difficult instances of~\eqref{eq:intro:contact-rich}.

\end{itemize}

% have effectively addressed these challenges.

% for polynomial optimization problems (POPs), providing a series of \textit{semidefinite programs} (SDPs) that progressively tighten this gap.  

% \textit{Convex relaxation} offers a robust and efficient approach to solving general nonconvex problems by computing a \textit{suboptimality gap} through a \textit{lower bound} from the convex relaxation and an \textit{upper bound} from a feasible solution of the nonconvex problem. Lasserre's Moment and Sum-of-Squares (SOS) Hierarchy~\cite{lasserre2001siopt-global} is a primary method for polynomial optimization problems (POPs), providing a series of \textit{semidefinite programs} (SDPs) that progressively tighten this gap. 

% However, scalability becomes problematic for planning problems, as polynomial variable counts can surge into the thousands. Advances in sparse Moment-SOS Hierarchy, including correlative sparsity~\cite{lasserre2006msc-correlativesparse} and term sparsity~\cite{wang2021siam-tssos, magron23book-sparse}, have effectively addressed these challenges. 



% \textbf{Sparse Moment-SOS Hierarchy: Challenges.} 

\textbf{Contributions.} 
In this paper, we tackle the aforementioned challenges and show that it is indeed possible to solve many instances of the contact-implicit planning problem~\eqref{eq:intro:contact-rich} to near global optimality. The key strategy is to build ``sparsity-rich'' semidefinite relaxations from the ground up, for robotics. 

We summarize our contributions as follows. 
% globally solve the contact-rich planning problem described in Equation~\eqref{eq:intro:contact-rich} using sparsity-rich semidefinite relaxation. 
\begin{enumerate}[label=(\Roman*)]
    \item \textbf{White-box sparsity exploitation.} 
    We provide a tutorial-style review of the fundamental mathematical concepts underpinning correlative and term sparsity for POPs, and further ground our discussion in a concrete contact-implicit planning problem. 
    We build a new C++ Sparse Polynomial Optimization Toolbox (\spot), interfacing both Matlab and Python, that (a) is faster than \tssos, (b) offers richer relaxation options, and (c) visualizes the automatically discovered sparsity patterns (see Fig.~\ref{fig:demos}).
    % We comprehensively review the theorems and algorithms for correlative and term sparsity clique generation. As a result, a new C++ sparse POP conversion package, \spot, is developed, offering faster conversion speeds and richer conversion options compared to TSSOS~\cite{magron2021arxiv-julia-tssos}.
    \item \textbf{Robotics-specific sparsity.} Beyond automatic exploitation of the generic correlative and term sparsity, we show that it is possible and crucial to exploit robotics-specific sparsity patterns. Particularly, we investigate sparsities derived from robot kinematic chains and separable contact modes, and demonstrate that robotics-specific sparsity patterns achieve both tighter lower bounds and reduced computation times compared to automatically generated ones in large-scale problems such as Planar Hand. 
    \item \textbf{Robust minimizer extraction.} 
    An important but often overlooked problem in SDP relaxations is how to extract good solutions to the nonconvex optimization from optimal SDP solutions, especially when the relaxation is not tight (\ie the suboptimality gap is large).  
    Inspired by the recent advances in Gelfand-Naimark-Segal (GNS) construction~\cite{klep2018siopt-minimizer-extraction-robust}, we develop a new minimizer extraction routine for sparse Moment-SOS relaxations that demonstrates superior robustness over naive extraction methods previously implemented in~\cite{kang2024wafr-strom,magron2021arxiv-julia-tssos}.
    \item \textbf{Extensive case studies.} We test our sparse semidefinite relaxations on five contact-rich planning problems: Push Bot, Push Box, Push T, Push Box with Obstacles, and Planar Hand. 
    Of independent interests, some of our polynomial modeling techniques also appear to be new in the planning literature. 
    Thanks to rich sparsity, the generated small-to-medium scale SDP relaxations can be solved in \textit{seconds} while achieving decent tightness and certified global optimality.
    Furthermore, we showcase robust push-T performance of our SDP relaxations using a real-world robotic manipulator. In fact, with global optimization, model predictive control is so robust that it succeeds the task even under severe environment disturbances that effectively make the ``model wrong'' (see Fig.~\ref{fig:demos}). 
\end{enumerate}

\textbf{Paper organization.} We present correlative and term sparsity in \S\ref{sec:general-sparsity-ksc}, robotics-specific sparsity in \S\ref{sec:robotics-specific}. We give numerical and real-world experiments in \S\ref{sec:exp}, and conclude in \S\ref{sec:conclusion}.

\textbf{Notations.} Let $\mvx = (x_1, \dots, x_n)$ be a tuple of variables and $\R[\mvx] = \R[x_1,\dots,x_n]$ be the set of polynomials in $\mvx$ with real coefficients. A monomial is defined as $\mvx^\mva = x_1^{\alpha_ 1}x_2^{\alpha_2} \cdots x_n^{\alpha_n}$. A polynomial in $\mvx$ can be written as $f(\mvx) = \sum_{\mva \in \bbN^n} f_\mva \mvx^\mva$ with coefficients $f_\mva \in \R$. We denote the set of all polynomials with degree less than or equal to $d$ as $\R_{d}[\mvx]$. 
The support of $f$ is defined by $\text{supp}(f) = \left\{ \mva\in \bbN^n\mid f_\mva \ne 0 \right\}$, i.e., the set of exponents with nonzero coefficients. The set of all variables contained in $f$ is defined by $\text{var}(f)$.
Let $\mvx^{\bbN_d^n}$ be the \emph{standard monomial basis}, abbreviated as ${[\mvx]}_d$. Given an index set $I\subseteq [n]$, let $\mvx(I)= (x_i, i\in I)$ and ${[\mvx(I)]}_d$ denote the standard monomial basis of the subspace spanned by the variables $x_i, i\in I$.
An undirected graph $G(V, E)$ consists of a vertex set $V={v_1, v_2, \dots, v_n}$ and an edge set $E \subseteq \{(v_i, v_j) \mid v_i, v_j\in V, v_i\ne v_j \}$. 
Let $\mbS^n$ denote the space of $n \times n$ symmetric matrices, and $\mbS^n_+$ denote the cone of $n \times n$ symmetric positive semidefinite (PSD) matrices.










