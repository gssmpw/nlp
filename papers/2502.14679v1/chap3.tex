\section{Results}
\label{sec:3}
We first verify the performance of the three autoencoder models
 in enforcing a disentangled latent space for model reduction of the data extracted from a periodic flow benchmark case suggested by Solera-Rico et al.~\cite{solera-rico:2024, solera_rico_2024_dataset}. 
%
Subsequently, we apply all models to reduce the load data extracted from aircraft ditching simulations. This more complex industrial test case uses the same data recently used in~\cite{schwarz:2024}. 
%
All models are implemented in PyTorch~\cite{NEURIPS2019_9015} and trained on a workstation equipped with an NVIDIA GeForce RTX 3090 GPU. 


\subsection{2D Periodic Benchmark Flow}
\label{sec:periodicflow}
A 2D benchmark case suggested in~\cite{solera-rico:2024} {and previously described in~\cite{Asztalos:2024}} is used for the verification of the models. The configuration is illustrated in Fig. \ref{fig:periodic_flow_scheme}.
\begin{figure}[!h]
     \centering
    \includegraphics{figures/periodicflow.pdf}
    \caption{{Illustration of the 2D periodic flow configuration employed for the verification analysis \cite{solera-rico:2024, Asztalos:2024, solera_rico_2024_dataset}}.}
    \label{fig:periodic_flow_scheme}
\end{figure}
The simulation data includes 1000 equidistant
time steps of a periodic flow over two flat plates of length $c$ 
whose centers are offset by $2c$ in the vertical ($y$-) direction
 and approached at an angle of attack of 90°. The first 900 time steps are used for training and the subsequent 100 time steps are used for validation.
The homogeneous, unidirectional horizontal inflow velocity $U_x$ results in a Reynolds number of $Re_{c,U_x}=40$. 
 The computational domain is of size $96c\times28c$, and the spatial resolution of the data refers to $300 \times 88$ homogeneously spaced, equidistant  
grid points in the horizontal ($x$-) and vertical ($y$-) direction, respectively. The temporal resolution of the data refers to $\Delta t= c/U_x$. 

Attention will be restricted to the reduction of the horizontal and vertical velocities by the autoencoders.  The data of this case is particularly suitable for comparison between machine learning and classical model reduction strategies, since it can be accurately represented with a latent space dimension of $m=2$. Therefore, the flow field was previously used to compare the results obtained from a $\beta$-VAE with a POD approach~\cite{solera-rico:2024}. The ability of the models to reduce to two modes and the behavior of the two modes is the focus of this subsection. To safe space, we limit ourselves to the reconstruction of the 11th snapshot of the validation cycle as an exemplary reference snapshot.


\subsubsection{Autoencoder Configuration}
\label{sec:periodicaeconf}
The employed $\beta$-VAE agrees with that of Solera-Rico et al.~\cite{solera-rico:2024}, and the decoder structure of the $\beta$-VAE was also used for the decoder of the OAE and the UAE, see Table~\ref{tab:cae_periodic_flow}. 
%
The encoder structure of the OAE and UAE is generally consistent with that of the $\beta$-VAE, with one exception being the last layer, which has been modified to have $m$ neurons instead of $2\cdot m$, since unlike the $\beta$-VAE, the OAE and UAE do not require two outputs for the mean and log-variance.
%
Similar to Solera-Rico et al.~\cite{solera-rico:2024}, \textsf{ELU}~\cite{clevert2016fastaccuratedeepnetwork} is the activation function used for the hidden layers in the encoder and the decoder. The output layers of the encoder and the decoder are linear. 


\begin{table}[h!]
\caption{Structure of encoder and decoder. The kernel size is $3\times 3$, and the stride is 2 for all Conv2d and ConvTranspose2d layers. For the $\beta$-VAE, one Linear(4) layer representing $\boldsymbol{\mu}$ and $\log (\boldsymbol{\sigma}^2)$ and the calculation of $\mathbf{z} = \boldsymbol{\mu} + \boldsymbol{\sigma} \odot \boldsymbol{\epsilon}$ replace the single Linear(2) layer in the encoder.}
\centering
\begin{tabular}{cccc}
    \hline
    Encoder & Output shape & Decoder & Output shape\\
    \hline
 Input & (300,88,2) & Linear(256) & (256)\\ 
     Conv2d(out\_channels=8) & (150,44,8) & Linear(2560) & (2560)\\
     Conv2d(out\_channels=16) & (76,22,16) & Unflatten(5,2,256) & (5,2,256)\\
     Conv2d(out\_channels=32) & (38,12,32) & ConvTranspose2d(out\_channels=128) & (10,4,128) \\
     Conv2d(out\_channels=64) & (20,6,64) &  ConvTranspose2d(out\_channels=64) & (20,6,64) \\
     Conv2d(out\_channels=128) & (10,4,128) & ConvTranspose2d(out\_channels=32) & (38,12,32) \\
     Conv2d(out\_channels=256) & (5,2,256) & ConvTranspose2d(out\_channels=16) & (76,22,16)\\
     Flatten() & (2560)  & ConvTranspose2d(out\_channels=8) & (150,44,8)\\  
     Linear(256) & (256) & ConvTranspose2d(out\_channels=2) & (300,88,2) \\
     Linear(2) & (2) \\
    \hline
\end{tabular}
\label{tab:cae_periodic_flow}
\end{table}

The models are trained using Adam~\cite{Kingma:2015} with a mini batch size of 256. The training employs 1000 epochs using a 1-cycle learning rate~\cite{smith:2018} that starts at $1\cdot 10^{-4}$, increases to $2\cdot 10^{-4}$ until the 200th epoch and subsequently decreases to $5\cdot 10^{-6}$ until the 1000th epoch as in~\cite{solera-rico:2024}. In all cases, no pretraining for the autoencoder with just the reconstruction loss is performed, meaning reconstruction loss and latent space loss are both used in every epoch. 


\subsubsection{Influence of Loss Function Weights}
\label{sec:periodicflow_loss_function_weights}
To compare how the models perform when changing the weight for the disentanglement contribution to the loss function, we train each model with 6 different weighting values $10^{-5}$, $10^{-4}$, $10^{-3}$, $10^{-2}$, $10^{-1}$ and $10^{0}$ for $\beta$ (VAE), $\lambda$ (OAE) and $\nu$ (UAE), respectively. 
%
Note that for the $\beta$-VAE, the value $\beta=10^{-3}$ used herein does not correspond to the value of $\beta=10^{-3}$ also used in~\cite{solera-rico:2024}, since the KL divergence was divided by the latent space dimension of $m=2$. The latter is concluded from the accompanying code of~\cite{solera-rico:2024}. Our value of $\beta=10^{-3}$ is thus two times larger. 

We compare the results in terms of (a) the achieved reconstruction losses and (b) the achieved correlation coefficients between the two latent variables.
For this comparison, each model is trained five times for each of the six weighting values to include the influence of the randomness in the training process. 
Figure~\ref{fig:1cycleLR_loss_factors_vs_rec_losses_corr_coeff} shows the reconstruction errors (left) that are obtained on the validation set as well as the absolute values of the correlation coefficients between the two latent variables (right). 

In general, the variance of the reconstruction error between the five training runs is small for all autoencoders considered.
%
As the green symbols in the left graph show, the reconstruction error of the $\beta$-VAE was not minimized at all during training 
for the three larger $\beta$-values, i.e.,  $\beta=10^{0}, 10^{-1}, 10^{-2}$.  This indicates too high $\beta$-values that lead to a dominating behavior of the KL divergence. Reducing the weight to $\beta=10^{-3}$, this is not the case anymore and the average reconstruction error decreases by about two orders of magnitude. For $\beta=10^{-4}$, the error decreases even further and is similarly low as for $\beta=10^{-5}$.  {We also performed the same experiments with a warm-up phase for 50 epochs, in which $\beta$ is initially set to 0 and then gradually increased to the specific value. The warm-up strategy did return similar results, although the reconstruction loss  for $\beta=10^{-2}$  did minimize to values around $10^{-2}$ in a few cases. In combination with  $\beta=10^{-1}$ and $\beta=10^0$ the reconstruction errors obtained with a warm-up strategy at the end of the training  are at the same level as depicted in Fig. \ref{fig:1cycleLR_loss_factors_vs_rec_losses_corr_coeff}.}
The reconstruction errors of the other two models show a more robust behavior for the choice of weighting values, with the UAE (blue) delivering lower average reconstruction errors than the OAE (orange) in all cases presented. In none of the cases investigated did the reconstruction loss of the UAE and the OAE models remain at the initial level.


With regard to the correlation coefficients, we can observe a larger variance between the different training runs, especially for the OAE results displayed in orange. 
The UAE results displayed in blue yield the lowest average correlation coefficient for the four largest values of the weight, i.e., $\nu=10^{0}, 10^{-1}, 10^{-2}, 10^{-3}$. 
Although in these cases the second lowest value is returned by the $\beta$-VAE, this must be considered under the limitation that the reconstruction error was not minimized for the three largest values of $\beta$. 
For the two smallest weighting values, i.e., $10^{-4}$ and $10^{-5}$, the $\beta$-VAE and the OAE, respectively, yield the lowest average correlation coefficient, with the UAE providing the second lowest value for both weights.

\begin{figure}[h!]
    \centering
    \includegraphics[width=0.8\textwidth]{figures/All_trainings_Case1_1cycleLR_loss_factors_vs_rec_losses_corr_coeff.pdf}
    \caption{Reconstruction errors obtained for the validation set and resulting absolute values of correlation coefficients between the two latent variables returned by the three different autoencoder models. The errorbars correspond to average, lowest and highest values obtained from five trainings.}
    \label{fig:1cycleLR_loss_factors_vs_rec_losses_corr_coeff}
\end{figure}



\subsubsection{Mode Analysis for Small Latent Space Dimension}
\label{sec:modean-2Dm2}
To analyze the impact of the latent variables, we generate an exemplary snapshot, verify its reconstruction quality and subsequently change one latent variable, while keeping the other latent variable fixed. This procedure is done for both latent variables ($z_1, z_2$) and all three models, i.e., UAE, OAE and $\beta$-VAE.
%
Fixing the second latent variable can be done in two ways.
On the one hand, one can use the value corresponding to the exemplary snapshot, as in~\cite{Higgins2016betaVAELB, burgess2018understandingdisentanglingbetavae, kang:2022}. On the other hand, the other latent variable can simply be set to zero.
%
However, because the investigated periodic benchmark flow is quite simple and does not contain many significantly different states, we choose to keep the exemplary value of one latent variable and observe how the other latent variable changes the state. 

Figures~\ref{fig:case1_reconstruction}-\ref{fig:case1_v_modes2} display the baseline results obtained in combination with $\nu=10^{-2}$ for the UAE, and $\lambda=\beta=10^{-4}$ for the OAE and $\beta$-VAE for the  reference snapshot.
This choice of weights should ensure a satisfactory reconstruction error and a significant level of disentanglement.
%
 Figure \ref{fig:case1_reconstruction} compares the true values with the 
autoencoder-based reconstructions for the horizontal $u$-velocity (left) 
 and the vertical $v$-velocity (right) in an upstream portion of the domain. 
 The color coding used is indicated in the figure and is consistent for all $u$- and $v$-velocity images, respectively.
The figure reveals that only small reconstruction errors occur in conjunction with any of the three models for the chosen exemplary snapshot. 

\begin{figure}[h!]
    \centering
    \includegraphics[width=\linewidth]{figures/case1_reconstruction.pdf}
    \caption{True $u$- and $v$-velocity fields, model reconstructions and related errors for the exemplary reference snapshot of the verification case.}
    \label{fig:case1_reconstruction}
\end{figure}
 
 In Figs. \ref{fig:case1_u_modes1}-\ref{fig:case1_v_modes2}, one of the two latent variables, outlined by the dark background color, is changed, while the other latent variable, indicated by the light background color, remains unchanged.   The imposed changes of the variable values refer to the upper, lower and mean value of the entire value range.
Displayed results are confined to a spatial window ranging from {$-1\le x \le 15$ and $-5\le y\le 6$} to focus on the region of the most significant influences, and
the employed color codes agree.
%
Looking at the $u$-velocity depicted in Figs.~\ref{fig:case1_u_modes1}-\ref{fig:case1_u_modes2}, all models show a similar behavior. One latent variable, i.e., the first for the UAE and the OAE and the second for the $\beta$-VAE seems to induce a shift of the $u$-velocity pattern in horizontal ($x$-) direction, while the other seems to induce a shift in vertical ($y$-) direction including a corresponding wave shaped transition that is observed during the initial phase, cf. center graphs in Fig. \ref{fig:case1_u_modes2}. 
With attention given to the $v$-velocity depicted in Figs. \ref{fig:case1_v_modes1}-\ref{fig:case1_v_modes2}, no vertical displacement is observed in response to any changes of the latent variables. Instead, only horizontal shifts can be observed.
The changes induced by  $z_1$ (UAE, OAE) and $z_2$ ($\beta$-VAE) are much lower than in the other case. 


\begin{figure}[h!]
    \centering
    \includegraphics[width=0.8\textwidth]{figures/case1_u_modes1.pdf}
    \caption{Impact of  changing $z_1$ for the UAE/OAE and $z_2$ for the $\beta$-VAE on the  $u$-velocity in the reference snapshot of the verification case using $m=2$ latent variables. The changed latent variable takes equidistant values ranging from the minimum to the maximum obtained for that variable on the validation set. The other latent variable is fixed to the corresponding latent space representation of the reference snapshot. The color code is the same for all examples, ranging from -1.87 (dark blue) to 1.90 (dark red).}
    \label{fig:case1_u_modes1}
\end{figure}

\begin{figure}[h!]
    \centering
    \includegraphics[width=0.8\textwidth]{figures/case1_u_modes2.pdf}
    \caption{Impact of changing $z_2$ for the UAE/OAE and $z_1$ for the $\beta$-VAE on the $u$-velocity in the reference snapshot of the verification case using $m=2$ latent variables. The changed latent variable takes equidistant values ranging from the minimum to the maximum obtained for that variable on the validation set. The other latent variable is fixed to the corresponding latent space representation of the reference snapshot. The color code is the same for all examples, ranging from -1.87 (dark blue) to 1.90 (dark red).}
    \label{fig:case1_u_modes2}
\end{figure}

\begin{figure}[h!]
    \centering
    \includegraphics[width=0.8\textwidth]{figures/case1_v_modes1.pdf}
    \caption{Impact of changing $z_1$ for the UAE/OAE and $z_2$ for the $\beta$-VAE on the $v$-velocity in the reference snapshot of the verification case using $m=2$ latent variables. The changed latent variable takes equidistant values ranging from the minimum to the maximum obtained for that variable on the validation set. The other latent variable is fixed to the corresponding latent space representation of the reference snapshot. The color code is the same for all examples, ranging from -11.58 (dark blue) to 10.41 (dark red).}
    \label{fig:case1_v_modes1}
\end{figure}

\begin{figure}[h!]
    \centering
    \includegraphics[width=0.8\textwidth]{figures/case1_v_modes2.pdf}
    \caption{Impact of changing $z_2$ for the UAE/OAE and $z_1$ for the $\beta$-VAE on the $v$-velocity in the reference snapshot of the verification case using $m=2$ latent variables. The changed latent variable takes equidistant values ranging from the minimum to the maximum obtained for that variable on the validation set. The other latent variable is fixed to the corresponding latent space representation of the reference snapshot. The color code is the same for all examples, ranging from -11.58 (dark blue) to 10.41 (dark red).}
    \label{fig:case1_v_modes2}
\end{figure}


\subsubsection{Physics Awareness for Higher Latent Space Dimension}
\label{sec:case1:higher_latent_dimension}
For the periodic flow at hand, two modes or latent variables are enough to capture (most of) the underlying physics~\cite{solera-rico:2024}. Solera-Rico et al.~\cite{solera-rico:2024} emphasize that a $\beta$-VAE with a larger latent space dimension $m>2$ still only finds two meaningful modes. This is in line with the findings by Eivazi et al.~\cite{eivazi:2022} and Kang et al.~\cite{kang:2022}, where the $\beta$-VAE is shown to find physics-aware latent variables. 
To assess the physics-awareness properties of the deterministic autoencoder models, we train all three models with a latent space dimension of $m=10$ and analyze whether they can learn using only two disentangled latent variables.  
Inspired by Kang et al.~\cite{kang:2022}, we employ the standard deviation 
(for all models) and the KL divergence (only for the $\beta$-VAE) of the latent variables to judge which latent variable  is active, i.e., physically significant. 
Subsequently, we generate and compare the modes obtained from the latent variables featuring the four highest standard deviations for an appropriate weighting value along the route outlined in Sec. \ref{sec:modean-2Dm2}.

\smallskip
Figure~\ref{fig:case1_std} depicts the standard deviations of the considered ten latent variables in combination with three different weights of the disentanglement contribution to the loss function. 
The investigated weighting values were taken from the successful studies for $m=2$ latent variables, cf. Fig.~\ref{fig:1cycleLR_loss_factors_vs_rec_losses_corr_coeff}.  
%
 They refer to $\nu = 10^{0}$, $\nu = 10^{-1}$ and $\nu = 10^{-2}$ for the UAE, $\lambda = 10^{-3}$, $\lambda = 10^{-4}$ and $\lambda = 10^{-5}$ for the OAE, as well as $\beta = 10^{-3}$, $\beta = 10^{-4}$ and $\beta = 10^{-5}$ for the $\beta$-VAE.


\begin{figure}[h!]
    \centering
    \includegraphics[width=\textwidth]{figures/Case1_STD.pdf}
    \caption{Standard deviations of latent variables for all three models obtained in the verification case for a latent space dimension of $m=10$. 
    Each model is trained with
    three different weights of the disentanglement contribution to the loss function.}
    \label{fig:case1_std}
\end{figure}

\begin{figure}[h!]
    \centering
    \includegraphics[width=\textwidth]{figures/Case1_VAE_KLD.pdf}
    \caption{KL divergence of the latent variables for the $\beta$-VAE
    obtained in the verification case for a latent space dimension of $m=10$. 
    The model is trained with
    three different values of $\beta$ in the loss function.}
    \label{fig:case1_kld}
\end{figure}


Considering the UAE results on the left of Fig.~\ref{fig:case1_std}, we observe that for $\nu=10^0=1$ and $\nu=10^{-1}$ two latent variables --i.e., $z_1$ and $z_5$, as well as $z_1$ and $z_7$, respectively-- have  noticeably higher standard deviations than the others. In conclusion, these two latent variables change their values more for different inputs to the encoder and are therefore more active than the other latent variables. This in turn suggests that the model has successfully extracted the two essential variables required to describe this flow, although this remains to be verified by analyzing their modes.
For $\nu=10^{-2}$ the correlation penalty is less strongly involved during training, and four latent variables are more active than the rest. 
This behavior is probably explained by the {balance between the numerator and the denominator} inside the loss function~\eqref{eq:our_loss} forcing a latent variable to be close to its mean over all inputs. A smaller $\nu/m^2$ allows more active latent variables as they are less strongly forced to be close to their mean over all inputs. For ($m=10$, $\nu=1$) and 
($m=10$, $\nu=10^{-1}$), the MSE and the correlation penalty are balanced in a way that the UAE uses a sufficient amount of two latent variables for the reconstruction. 
%
%
For the OAE depicted in the center graphs of Fig.~\ref{fig:case1_std}, the standard deviations of the different latent variables are generally closer to each other. For the three displayed values of $\lambda$, two latent variables with a clearly higher standard deviation than the others cannot be identified, indicating that the OAE has difficulties to extract the two most crucial latent variables when trained with a latent space dimension of $m=10$ in combination with these $\lambda$ values. In combination with $\lambda=10^{-4}$ the largest standard deviation refers to $z_1, z_3$ and $z_5$.  %
%
Results for the $\beta$-VAE are shown in the right graphs of Fig.~\ref{fig:case1_std}. For the large value $\beta=10^{-3}$, the standard deviation is close to 1 and the KL divergence shown in Fig. \ref{fig:case1_kld} is approximately 0 for every latent variable. Moreover, 
 the latent variables with largest standard deviation and the largest KL divergence disagree and partly contradict each other.
This makes  $\beta=10^{-3}$ seem questionable in this case. 
For $\beta=10^{-4}$, $z_4$ and $z_9$ have a slightly higher standard deviation than the other latent variables, which mostly have standard deviations close to 1. Accordingly, the KL divergence for these two variables also increases.
Moreover, $z_{10}$ noticeably deviates from a unit standard deviation in conjunction with $\beta=10^{-4}$ and thus also displays an increased KL divergence. 
For $\beta=10^{-5}$, there are three latent variables {($z_1$, $z_5$ and $z_9$)} that are most active. Two latent variables $z_6$ and $z_8$ [$z_5$/$z_1$ and $z_9$] 
have a noticeably lower [higher] standard deviation than the rest. However, the KL divergence is visibly greater than 0 for many more variables, suggesting that $\beta$ is too small to separate the two crucial modes.

\smallskip
To verify how the modes for the active and inactive latent variables differ, we  generate modes for the four most active latent variables of the latent space dimension $m=10$ in combination with the intermediate weight values, i.e., the red curves in Fig.~\ref{fig:case1_std}. The standard deviation is used for identification in combination with the UAE and OAE models. For the $\beta$-VAE ($\beta=10^{-4}$) we employ the three latent variables that display the largest standard deviation ($z_4, z_7, z_9$) in Fig.~\ref{fig:case1_std} together with $z_{10}$ that displays a large KL divergence in Fig. \ref{fig:case1_kld}.  
%
To safe space, we restrict ourselves to the influence of the modes on the horizontal $u$-velocity fields in a confined spatial domain as outlined in Figs.~\ref{fig:Case1_UAE_Latent10_modes}-~\ref{fig:Case1_VAE_Latent10_modes}. Consideration of the vertical velocity leads to the same conclusions.


\begin{figure}[h!]
    \centering
    \includegraphics[width=0.8\textwidth]{figures/case1_latent10_UAE_modes.pdf}
    \caption{Impact of single latent variables on the $u$-velocity in the reference snapshot of the verification case for the UAE using a latent space dimension of $m=10$ and $\nu=10^{-1}$.  
    In each column, nine latent variables are fixed to the corresponding latent space representation of the snapshot and one variable takes equidistant values ranging from the minimum to the maximum value obtained
    for the validation set.
    Color code is the same for all examples in this figure, ranging from -1.65 (dark blue) to 1.66 (dark red).}
    \label{fig:Case1_UAE_Latent10_modes}
\end{figure}

\begin{figure}[h!]
    \centering
    \includegraphics[width=0.8\textwidth]{figures/case1_latent10_OAE_modes.pdf}
    \caption{Impact of single latent variables on the $u$-velocity in the reference snapshot  of the verification case for the OAE using a latent space dimension of $m=10$ and $\lambda=10^{-4}$. 
    In each column, nine latent variables are fixed to the corresponding latent space representation of the snapshot and one variable takes equidistant values ranging from the minimum to the maximum value obtained
    for the validation set.
    Color code is the same for all examples in this figure, ranging from -1.65 (dark blue) to 1.74 (dark red).}
    \label{fig:Case1_OAE_Latent10_modes}
\end{figure}

\begin{figure}[h!]
    \centering
    \includegraphics[width=0.8\textwidth]{figures/case1_latent10_VAE_modes.pdf}
    \caption{Impact of single latent variables on the $u$-velocity in the reference snapshot  of the verification case for the $\beta$-VAE using a latent space dimension of $m=10$ and $\beta=10^{-4}$.
    In each column, nine latent variables are fixed to the corresponding latent space representation of the snapshot and one variable takes equidistant values ranging from the minimum to the maximum value obtained
    for the validation set.
    Color code is the same for all examples in this figure, ranging from -1.67 (dark blue) to 1.94 (dark red).}
    \label{fig:Case1_VAE_Latent10_modes}
\end{figure}

Figure~\ref{fig:Case1_UAE_Latent10_modes} shows the UAE modes obtained for  $\nu=10^{-1}$, where default values for $m=10$ read {$z_1=-0.47$, $z_7=0.19$, $z_{10}=-0.10$ and $z_8=0.20$.}
The modes for the most active latent variable $z_1$ displayed in the first column clearly induce a shift in vertical direction. Similarly, the modes of the second most active variable $z_7$ displayed in the second column  introduce a horizontal shift. 
 The behavior is similar to that observed in the first columns of  Figs.~\ref{fig:case1_u_modes1} and~\ref{fig:case1_u_modes2}.
%
For the less active latent variables $z_{10}$ and $z_8$, the outputs are nearly identical for every latent value tested. 
This supports the assumption that the UAE model successfully identified the two crucial latent variables while being trained with a latent dimension of $m=10$. 
The inactiveness of latent variables in the UAE is in line with the findings by Kim et al.~\cite{KIM2021148}, in which the correlation penalty leads to inactive channels of encoded feature maps.

Figure~\ref{fig:Case1_VAE_Latent10_modes} displays the modes for the four most active latent variables of the $\beta$-VAE in combination with $\beta=10^{-4}$, where default values for $m=10$ are {$z_9=-1.90$, $z_4=0.89$, $z_{7}=2.14$ and $z_{10}=0.84$}. Again, the modes for the most active latent variables $z_9$ and $z_4$ display strong similarities to the modes investigated for $m=2$ in Figs.~\ref{fig:case1_u_modes1} and ~\ref{fig:case1_u_modes2}. 
Varying the value of the third most active latent variable $z_7$ does not significantly change the output of the decoder. Note that although $z_{10}$ has the lowest standard deviation, we also show the modes for $z_{10}$ because it has the second highest KL divergence. Interestingly, the output of the decoder also does not significantly alter in response to the value of $z_{10}$. Hence a large KL divergence without considering the standard deviation seems no clear indicator.

For the OAE we obtain slightly different results, as Fig.~\ref{fig:Case1_OAE_Latent10_modes} shows for the mode study based on $\lambda=10^{-4}$. Here default values for $m=10$ refer to {$z_3=-0.01$, $z_5=0.10$, $z_{1}=-0.03$ and $z_{10}=-0.03$}.
Horizontal shifts induced by the highest ranked latent variable $z_3$ and the third highest ranked latent variable $z_1$ are in close agreement. They are supplemented by vertical shifts introduced through changing the value of the second highest ranked variable $z_5$. The influence of $z_{10}$ on the encoder data is minor. Although the results are consistent with the data shown for $m=2$, the reduction to two modes is not so obvious in this case. 


\subsubsection{Pruning}
In contrast to separating active from inactive latent variables during a post-processing step after the training, inactive latent variables could also be identified and pruned during the training process. 
%
The $i$th latent variable is pruned by setting the $i$th row of the weight matrix and the $i$th component of the bias of the final layer in the encoder to zero. By pruning inactive latent variables during training, these variables do not take part in the reconstruction of the input anymore and the models learn to only use the active latent variables. We tested such an approach for the UAE with a latent space dimension of $m=10$, in which an experiment-based threshold value of $0.07$ for the standard deviation of the normalized latent variables was employed to identify inactive latent variables, starting from the 500th epoch. A normalized threshold value slightly above $0.05$ can also be used to identify the active latent variables in a post-processing step.

The results were compared to a post-processing approach, where the two most active latent variables are identified after the training and the other latent variables are set to their mean, as for example suggested in~\cite{kang:2022}. 
We can confirm that the pruning approach successfully identified the two most active latent variables in several cases. 
However, no significant improvement of the reconstruction accuracy and the correlation coefficient was achieved compared to the post-processing approach. Further work on the pruning approach will be conducted in future research.

\subsection{Industrial Application: Aircraft Ditching Loads}
The second case serves to validate the use of the autoencoder models for more complex problems.  The particular application refers to the data-driven surrogate model of ditching load and deformation predictions. In a previous study~\cite{schwarz:2024}, different combinations of CAE and LSTM networks well as a Koopman autoencoder~\cite{lusch:2018} were assessed to predict the spatio-temporal evolution of ditching loads, with a \textit{convolutional encoder--LSTM--convolutional decoder} combination yielding the best results. In this paper, the focus lies on the disentanglement and analysis of the latent variables. 

The data set contains simulated ditching loads on the fuselage of a DLR-D150 aircraft, which is of similar size to an Airbus A320. The simulations were performed with the 3 degrees of freedom method \emph{ditch}~\cite{bensch:2003}, which considers horizontal, vertical and pitch motion of the aircraft in response to hydrodynamic loads, aerodynamic loads, inertia forces and thrust. Due to the resulting symmetry, only one half of the fuselage is used. A detailed description of the data generation and processing is given in~\cite{schwarz:2024}. 
The training set contains data from 323 simulations, which differ in the horizontal and vertical initial velocities. A validation set containing data from 20 simulations is used to compare the models for different weighting values of the disentanglement contributions to the loss functions.
For each simulation, around 18-35 time step images of size $128\times 128$, representing equidistantly spaced circumferential and longitudinal coordinates of the fuselage, are included in the data sets. The equidistant time step corresponds to 0.1 seconds.

The assessment of the autoencoder models is based on typical results for a temporal load series that were obtained from the trained models, see also \cite{schwarz:2024}. An exemplary temporal development of the hull loads is shown in Fig.~\ref{fig:examplary_ditching_loads} and shows the five typical phases. 
%
\begin{figure}[h!]
    \centering
    \includegraphics[width=\linewidth]{figures/ExemplaryLoads.pdf}
    \caption{Exemplary spatio-temporal evolution of simulated ditching loads.}
    \label{fig:examplary_ditching_loads}
\end{figure}
%
The first impact phase usually results in an initially  confined  circumferential load area (initial phase; 1st step), followed by a widening towards a larger longitudinal and circumferential load area (early phase; 5th, 9th step). For the subsequent phases, positive load signatures are usually slender crescent-shaped and limited to a smaller spatial area (late phase; 13th, 17th step). Moreover, they are followed by a more pronounced under pressure region in downstream direction. The final loads of rear fuselage are often small in amplitude and area (final phase; 21st step).


\subsubsection{Autoencoder Configuration}
The structure of the encoder and the decoder is based on the previous study~\cite{schwarz:2024} and outlined in  Table~\ref{tab:cae_ditch}. The baseline study employs a latent space dimension of $m=10$, which lead to satisfactory results in the previous research~\cite{schwarz:2024}, while keeping the latent space compact to allow for future research on interpretability. 
\begin{table}[h!]
\caption{Structure of encoder and decoder. The kernel size is $3\times 3$, and the stride is 2 for all Conv2d and ConvTranspose2d layers. For the $\beta$-VAE, two Linear(10) layers representing $\boldsymbol{\mu}$ and $\log (\boldsymbol{\sigma}^2)$ and the calculation of $\mathbf{z} = \boldsymbol{\mu} + \boldsymbol{\sigma} \odot \boldsymbol{\epsilon}$ replace the single Linear(10) layer in the encoder.}
\centering
\begin{tabular}{cccc}
    \hline
    Encoder & Output shape & Decoder & Output shape\\
    \hline
 Input & (3,128,128,1) & Linear(4096) & (4096)\\ 
     Conv2d(out\_channels=8) & (64,64,8) & Unflatten(8,8,64) & (8,8,64)\\
     Conv2d(out\_channels=16) & (32,32,16) & ConvTranspose2d(out\_channels=32) & (16,16,32)\\
     Conv2d(out\_channels=32) & (16,16,32) & ConvTranspose2d(out\_channels=16) & (32,32,16) \\
     Conv2d(out\_channels=64) & (8,8,64) &  ConvTranspose2d(out\_channels=8) & (64,64,8) \\
     Flatten() & (4096)  & ConvTranspose2d(out\_channels=1) & (128,128,1)\\  
     Linear(10) & (10)  \\
    \hline
\end{tabular}
\label{tab:cae_ditch}
\end{table}
A \textsf{LeakyReLU}~\cite{maas:2013} activation function  with slope parameter $\alpha=0.01$ was employed for the hidden layers in the encoder and the decoder. The output layers of the encoder and the decoder are linear. 
Models are trained for 500 epochs using Adam with the default learning rate of $1\cdot10^{-3}$ and a minibatch size of 128.


\subsubsection{Influence of Loss Function Weights}
Similar to the verification study in Sec. \ref{sec:periodicflow_loss_function_weights} we compare the reconstruction losses on the validation set and use the determinant of the correlation matrix $\det(\mathbf{R})$ to simultaneously  measure the disentanglement of the latent variables for different weights $\beta$, $\lambda$ and $\nu$ of the disentanglement contribution to the loss
function.  Again, both contributions to the loss function are used in every epoch.  To obtain some guidance on judging the results for the different $\beta$, $\lambda$ and $\nu$ values, we additionally trained a standard convolutional autoencoder without any disentanglement constraints in the latent space, and obtained $\det(\mathbf{R})=0.014$ and a reconstruction error of $2.53\cdot10^{-6}$. 

Figure~\ref{fig:D150_loss_factors_vs_rec_losses_detR} depicts the influence of the weights on the reconstruction error (left) and the determinant of the correlation matrix (right).
%
\begin{figure}[h!]
    \centering
    \includegraphics[width=0.8\textwidth]{figures/D150_validation_loss_factors_vs_rec_losses_detR.pdf}
    \caption{Reconstruction errors on the validation set and resulting values of $\det(\mathbf{R})$ obtained by the different models for the ditching application using $m=10$ latent variables. 
    }
    \label{fig:D150_loss_factors_vs_rec_losses_detR}
\end{figure}
%
The two deterministic models again reveal a robust behavior. For the investigated range of values for $\nu$ and $\lambda$, the minimization of the reconstruction loss does not get stuck in the beginning, although higher values 
usually yield higher reconstruction errors. The UAE, represented by the blue symbols, achieves the lowest reconstruction error and returns $\det(\mathbf{R}) \approx 1$ except for $\nu=10^{-6}$ and $\nu=10^{-7}$, where $\det(\mathbf{R})$ is close to zero, suggesting that $\nu\le 10^{-6}$ is too small to enforce uncorrelated latent variables. For the OAE, marked in orange, $\det(\mathbf{R})$ is mostly close to 0.9 and smaller than $\det(\mathbf{R})$ for the UAE when $\lambda\ge10^{-4}$. Although  $\det(\mathbf{R})$ values associated with $\lambda\le10^{-6}$ are higher for the OAE than for the corresponding UAE, the determinant only reaches values about 0.4 and 0.25. These values are quite small, indicating an undesirable correlation level between the latent variables.
%
On the contrary, one can observe that the $\beta$-VAE represented by the green symbols, only minimizes the reconstruction loss when $\beta$ is set smaller than $\beta=10^{-5}$, 
 which is the largest value, for which the minimization of the reconstruction loss does not get stuck in the beginning. For $\beta=10^{-4}$, we also tested a warm-up phase, in which $\beta$ gradually increased to $10^{-4}$ at the beginning of training. Without going into detail we note that, although the reconstruction loss is minimized at first, it eventually increases to around $1.2\cdot10^{-4}$ again, corresponding to the same reconstruction error level as for the greater values for $\beta$.
In all cases, the reconstruction error for the $\beta$-VAE exceeds the reconstruction error of the deterministic models. 
The values for $\det(\mathbf{R})$ are greater than $0.9$ for $\beta=10^{-5}$ and $\beta=10^{-6}$. For $\beta=10^{-7}$, it is below $0.7$, indicating a small interval of suitable $\beta$-values that achieve both, a disentanglement of the latent variables and a satisfactory reconstruction capability.


\subsubsection{Mode Analysis}
\label{sec:modean_D150}
Based on the studies from Sec. \ref{sec:modean-2Dm2}, Figs. \ref{fig:D150_Latent10_std} and \ref{fig:D150_Latent10_vae_KLD} depict the standard deviations of the latent variables of the three autoencoder models as well as the KL divergence of the $\beta$-VAE. The investigated three weight values for  $\nu$, $\lambda$ and $\beta$ were extracted from the results displayed in Fig. \ref{fig:D150_loss_factors_vs_rec_losses_detR}.

\smallskip
Considering the UAE on the left, we can observe that many latent variables appear to be active for all three investigated values of $\nu$.
For the large value $\nu=10^{-3}$ displayed in blue, which promises the best disentanglement whilst still delivering a fair amount of reconstruction accuracy, we can structure the latent space into two higher-ranked contributions $z_1, z_5$ and two second-ranked contributions
$z_8, z_9$.
As regards the OAE, the situation is similar. Due to the results shown in Fig. \ref{fig:D150_loss_factors_vs_rec_losses_detR}, we discard the lower values and focus on the largest meaningful  value $\lambda=10^{-3}$ displayed in blue.
 The related latent space can also be grouped into a dominating mode $z_2$ and two second-ranked modes $z_8, z_5$ followed by $z_3$.
For the $\beta$-VAE, Fig.
\ref{fig:D150_loss_factors_vs_rec_losses_detR} suggests to employ $\beta=10^{-6}$ as a baseline value, where the four most important latent variables refer to $z_7$ (top rank) and  $z_9$ slightly ahead of $z_6, z_8, z_{10}$, cf.  red curves in Fig. \ref{fig:D150_Latent10_std}. The KL divergence depicted in Fig.~\ref{fig:D150_Latent10_vae_KLD} suggests to assign $z_{6}$ to a lower priority. An interesting phenomenon occurs when comparing the standard deviation in Fig. \ref{fig:D150_Latent10_std} with the
KL divergence in Fig. \ref{fig:D150_Latent10_vae_KLD}. For the chosen  value $\beta=10^{-6}$, $z_2$ clearly shows the largest KL divergence, but also by far the smallest standard deviation. It turns out, that this latent variable is inactive with respect to the decoder output. This again underlines that a high KL divergence is only a good indicator of an active variable in combination with a sufficiently large standard deviation. 


\begin{figure}[h!]
    \centering
    \includegraphics[width=\textwidth]{figures/D150_Latent10_STD.pdf}
    \caption{Standard deviations of the latent variables 
    obtained in the application case for a latent space dimension of $m=10$. Results for three different weights of the disentanglement contribution to the loss function are shown corresponding to Figure \ref{fig:D150_loss_factors_vs_rec_losses_detR}.}
    \label{fig:D150_Latent10_std}
\end{figure}

\begin{figure}[h!]
    \centering
    \includegraphics[width=\textwidth]{figures/D150_Latent10_VAE_KLD.pdf}
    \caption{KL divergence of the latent variables for the $\beta$-VAE  obtained in the application case for a latent space dimension of $m=10$. Results for three different values of $\beta$ in the loss function are shown corresponding to Figure \ref{fig:D150_loss_factors_vs_rec_losses_detR}.}
    \label{fig:D150_Latent10_vae_KLD}
\end{figure}

For each autoencoder model, we employ the four most active latent variables
to assess the associated modes. In contrast to the 
periodic flow example, we set all other latent variables to zero during the assessment, except the one we are analyzing. The reason is that the ditching data involves more different states and we believe that the choice of a suitable baseline state is not obvious. This also means that the studies do not consider a reference snapshot, since all but one latent variable are inactive.



\begin{figure}[h!]
    \centering
    \includegraphics[width=0.6\textwidth]{figures/UAE_Latent10_modes_10e-3.pdf}
    \caption{Impact of four latent variables with the highest standard deviation on the loads obtained from the UAE for the application case using a latent space dimension of $m=10$ and $\nu=10^{-3}$. 
    In each column,  one latent variable takes equidistant values ranging from the minimum to the maximum obtained on that variable on the validation set while the other latent variables are set to zero.
    Color code is the same for all examples in this figure, ranging from lowest (dark blue) to highest (dark red) values.}
    \label{fig:D150_UAE_Latent10_modes}
\end{figure}

\begin{figure}[h!]
    \centering
    \includegraphics[width=0.6\textwidth]{figures/OAE_Latent10_modes10e-3.pdf}
    \caption{Impact of the four latent variables with the highest standard deviation on the loads obtained from the OAE for the application case using a latent space dimension of $m=10$ and $\lambda=10^{-3}$. 
    In each column, one latent variable takes equidistant values ranging from the minimum to the maximum obtained on that variable on the validation set while the other latent variables are set to zero.
    Color code is the same for all examples in this figure, ranging from lowest (dark blue) to highest (dark red) values.}
    \label{fig:D150_OAE_Latent10_modes}
\end{figure}

\begin{figure}[h!]
    \centering
    \includegraphics[width=0.6\textwidth]{figures/VAE_Latent10_modes_10e-6.pdf}
    \caption{Impact of the four latent variables with the highest standard deviation on the loads obtained from the $\beta$-VAE for the application case using a latent space dimension of $m=10$ and $\beta=10^{-6}$. 
    %
    In each column, one latent variable takes equidistant values ranging from the minimum to the maximum obtained on that variable on the validation set while the other latent variables are set to zero.
    Color code is the same for all examples in this figure, ranging from lowest (dark blue) to highest (dark red) values.}
    \label{fig:D150_VAE_Latent10_modes}
\end{figure}

Figure~\ref{fig:D150_UAE_Latent10_modes} illustrates the response of the load to the most 
active latent variables for the UAE. 
Considering the most active variable $z_1$, we can see that negative $z_1$-values only have a small impact on the output. Positive values seem to be related to the late phase of the impact, cf. steps 13 and 17 in Fig. \ref{fig:examplary_ditching_loads}, featuring a typical slender crescent-type  pressure load followed by a more pronounced suction load regime.
The second most active variable $z_5$ is contributing to the initial impact phase for positive values, cf. step 1 in Fig. \ref{fig:examplary_ditching_loads}, with a confined small area being exposed to high pressure loads. Negative $z_5$-values resemble the subsequent early phase, cf. steps 5 and 9 in Fig. \ref{fig:examplary_ditching_loads}, in which a large area of the fuselage is exposed to smaller pressure loads. 
Negative values in $z_8$ are associated to the final phase of the impact, in which  loads occur at the rear of the fuselage and begin to vanish, cf. step 21 in Fig. \ref{fig:examplary_ditching_loads}. For positive values, $z_8$ is contributing to the early impact phase, cf. steps 5 and 9 in Fig. \ref{fig:examplary_ditching_loads}. Small values of $z_9$ are related to the late phase and larger values to final phase of the ditching, cf. steps 13, 17 and 21 in Fig. \ref{fig:examplary_ditching_loads}.

The outputs generated by an investigation of the four most active latent variables obtained from the OAE are shown in Fig.~\ref{fig:D150_OAE_Latent10_modes}. The OAE results are generally more noisy than the corresponding UAE results. 
%
Although it is evident that certain latent variables contribute to the loads occurring in a particular phase of the ditching event,
for example, the spatial expansion of the load regime in the early impact phase captured by positive values of $z_5, z_8$ and a footprint of final phase loads represented by negative values of $z_2$, in most cases the patterns are not so clearly linked to a loading situation.


Looking at the modes associated with the four most active variables of the $\beta$-VAE  in Fig.~\ref{fig:D150_VAE_Latent10_modes}, we can see more distinct links with realistic load pattern again. The most active variable $z_7$ is associated to the final phase of the impact for negative values and to the early phase for positive values. Typical load pattern for the late phase resemble the structures reported by $z_9, z_8$, whereas $z_{10}$ supplements the load widening in circumferential direction.  
%
It is interesting to note, that the fourth highest ranked variable $z_6$ does not yield any meaningful output, although the fifth highest ranked variable $z_8$ does. The standard deviations of these two latent variables are nearly the same, cf. Fig. \ref{fig:D150_Latent10_std}. However, the KL divergence for $z_8$ is the second highest of all latent variables and the one for $z_6$ is nearly zero. For all remaining latent variables including $z_2$, which has the highest KL divergence in combination with the smallest standard deviation, the outputs are similar to the ones for $z_6$ and, thus, nearly identical. That is in line with the findings in Sec. \ref{sec:case1:higher_latent_dimension}, in which changes in the latent variable with the lowest standard deviation and highest KL divergence returned nearly constant outputs.  

\smallskip
The resulting modes of the ditching loads are in general harder to analyze compared to the periodic flow case, as the temporal dynamics is more complex and over 300 different simulations with different input parameters were used to train the models. However, distinct influences of different active latent variables on different phases of a ditching event can be observed when the latent variables are disentangled.
%
One can verify that a latent variable is indeed associated to a specific impact phase by reconstructing a particular snapshot of this phase after setting this variable to 0. For this purpose, we reuse the second test case from~\cite{schwarz:2024}, which is simulated with the full-order model using a horizontal velocity of 80.75 m/s, a vertical velocity of 1.91 m/s and a pitch angle of $6^\circ$. 
As an example, we illustrate this for the two most active latent variables of the UAE, i.e., $z_1, z_5$, cf. Fig.~\ref{fig:D150_UAE_Latent10_modes}. Figure~\ref{fig:D150_UAE_reconstructions} shows snapshots for two particular time steps of the test case, their UAE-reconstruction using $m=10$ latent variables and the corresponding UAE-reconstruction obtained from setting either the most active latent variable $z_1$ or the second most active $z_5$ to zero. The selected time steps correspond to impact phases associated to the two most active latent variables, i.e., one from the early and one from the late phase, cf. Fig.~\ref{fig:D150_UAE_Latent10_modes}. 
Comparing the true values in the first column with the reconstructions using all $m=10$ latent variables shown in the second column displays a satisfactory agreement. If $z_1=0$ is set, the reconstruction of the 16th time step depicted in the first row is considered insufficient. However, if $z_5=0$ is set, the reconstruction is again very accurate.
A similar observation can be made for the 10th time step, which is shown in the second row. Removing the most active latent variable does not have a significant impact on the reconstruction quality. If $z_5=0$ is set, the reconstruction deteriorates significantly. 
It is evident that the corresponding latent variables have a significant influence on the quality of the reconstruction in the respective distinct phases. 

\begin{figure}[h!]
    \centering
    \includegraphics[width=0.8\textwidth]{figures/UAE_Reconstructions.pdf}
    \caption{UAE reconstructions of the 16th (first row) and the 10th (second row) time step of a particular ditching case
    using a latent space dimension of $m=10$ and $\nu=10^{-3}$. 
    The suppressed variables in column three and four, i.e., $z_1$ and $z_5$, correspond to the most active and second most active latent variable, respectively.}
    \label{fig:D150_UAE_reconstructions}
\end{figure}

\subsubsection{Physics Awareness for Higher Latent Space Dimension}
Similar to Sec. \ref{sec:case1:higher_latent_dimension}, 
we train the models with a latent space dimension of 100 to see, how well the models work at finding a smaller number of active latent variables. Again, we use different weighting values for the contribution of the disentanglement part to the loss function, i.e., $\nu = \lambda = 10^{-3}, 10^{-4}, 10^{-5}$ as well as $\beta = 10^{-5}, 10^{-6}, 10^{-7}$ to analyze the impact of the loss term in the latent space. 


Figure~\ref{fig:D150_std} shows the standard deviations of the different latent variables for all three autoencoder models using the different weighting values. We can observe a similar behavior as for the periodic flow example in Fig.~\ref{fig:case1_std}. Looking at the UAE, we can identify two distinct active latent variables in combination with the most pronounced disentanglement $\nu=10^{-3}$. Four more latent variables have a noticeably higher standard deviation than the remaining ones, from which four further variables stand out. By reducing the weight to $\nu=10^{-4}$ and $\nu=10^{-5}$, the number of active latent variables increases significantly. 
%
Observing the behavior of the OAE model shown in the second column, the standard deviations of the latent variables are again generally much closer to each other for $\lambda=10^{-3}$. For $\lambda=10^{-4}$ and $\lambda=10^{-5}$, the differences between the standard deviations become higher, but are never  as pronounced as for the UAE. 
%
Considering the $\beta$-VAE in the right column, we can observe that the standard deviations are also fairly close to each other for $\beta=10^{-5}$ and $\beta=10^{-6}$. For $\beta=10^{-7}$, there are three distinct latent variables with a clearly higher standard deviation.  
The corresponding KL divergences are depicted in Fig.~\ref{fig:D150_Latent100_vae_KLD}. As expected, higher KL divergences are returned for lower values of $\beta$. For $\beta=10^{-7}$, the three highest values are obtained for the three latent variables with the highest standard deviation in Fig \ref{fig:D150_std}. 


\begin{figure}[h!]
    \centering
    \includegraphics[width=\textwidth]{figures/D150_STD.pdf}
    \caption{Standard deviations of the latent variables for all three autoencoder models using a latent space dimension of $m=100$. Each model is trained with three different weights of the disentanglement
contribution to the loss function.}
    \label{fig:D150_std}
\end{figure}

\begin{figure}[h!]
    \centering
    \includegraphics[width=\textwidth]{figures/D150_Latent100_VAE_KLD.pdf}
    \caption{KL divergence for the latent variables of the $\beta$-VAE using a latent space dimension of $m=100$. The model is trained with
    three different values of $\beta$ in the loss function.}
    \label{fig:D150_Latent100_vae_KLD}
\end{figure}
  
Regarding $\beta$-VAE, Kang et al.~\cite{kang:2022} noted that considering $\det(\mathbf{R})$ using all latent variables is not sensible when the latent space is sparse, and suggested to consider combinations of a smaller number of latent variables. 
Similarly, we compute $\det(\mathbf{R})$ for the twenty most active latent variables to verify their independence. The corresponding results are illustrated in Fig.~\ref{fig:D150_UAE_Latent100_detR}. For all three models, an increased weight of the disentanglement contribution to the loss function yields higher values of $\det(\mathbf{R})$. The only exception is the $\beta$-VAE, for which the differences between $\beta=10^{-5}$ and $\beta=10^{-6}$ are negligible. Analyzing the UAE behavior for $\nu=10^{-3}$, one finds that the five most active latent variables are all approximately at $\det(\mathbf{R})\approx 1$, and $\det(\mathbf{R})$ decreases only slightly when using additional active variables.
For the 10 most active latent variables, $\det(\mathbf{R})$ is still greater than $0.9$. When reducing $\nu$, the value of $\det(\mathbf{R})$ decreases more rapidly, starting with five ($\nu=10^{-4}$) or even three ($\nu=10^{-5}$) variables.
The OAE gives similar results to the UAE for $\lambda=10^{-3}$. Using $\lambda=10^{-4}$ and $\lambda=10^{-5}$ induces a more rapid decrease in $\det(\mathbf{R})$ when increasing the amount of latent variables. 
For the $\beta$-VAE, $\beta=10^{-5}$ and $\beta=10^{-6}$ return similar values of $\det(\mathbf{R})$ as $\nu=\lambda=10^{-3}$ for the UAE and OAE, respectively, although the five most active latent variables yield slightly smaller values than the five most active for the UAE. From the seventh ranked variable, $\det(\mathbf{R})$ deviates from the higher values of $\beta$ for $\beta=10^{-7}$ and takes values smaller than $\approx0.83$.




\begin{figure}[h!]
    \centering
    \includegraphics[width=0.9\textwidth]{figures/Latent100_detR.pdf}
    \caption{Values for $\det(\mathbf{R})$ using different amounts of the most active latent variables.}
    \label{fig:D150_UAE_Latent100_detR}
\end{figure}


In Fig.~\ref{fig:D150_UAE_Latent100_modes}, the modes for the four most active latent variables of the UAE using $\nu=10^{-3}$ are displayed in combination with $m=100$. The modes are generated exactly as described in Sec. \ref{sec:modean_D150}. For the two most active latent variables $z_{64}$ and $z_{23}$ results agree with the modes associated with $z_1$ and $z_5$ in the $m=10$ configuration, cf. Fig.~\ref{fig:D150_UAE_Latent10_modes}. The main differences are the flipped order of the first column, which is again related to the late phase,  as well as differences in the intensities of the loads.  The two following latent variables of the $m=100$ configuration $z_{13}$ and $z_{39}$, however, have a smaller impact on the output. This could also be expected from the different standard deviations depicted in Fig. \ref{fig:D150_std}.
Without going into detail, we report that the $\beta$-VAE also returned reasonable leading modes for $\beta=10^{-7}$ and $m=100$.

\begin{figure}[h!]
    \centering
    \includegraphics[width=0.6\textwidth]{figures/UAE_Latent100_modes_10e-3.pdf}
    \caption{Impact of four latent variables with the highest standard deviation on the loads obtained from the UAE trained with a latent space dimension of $m=100$. In each column,  one latent variable takes equidistant values ranging from the minimum to the maximum obtained on that variable on the validation set while the other latent variables are set to zero.
    Color code is the same for all examples in this figure, ranging from lowest (dark blue) to highest (dark red) values.}
    \label{fig:D150_UAE_Latent100_modes}
\end{figure}