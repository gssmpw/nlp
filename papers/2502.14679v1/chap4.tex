\section{Conclusion}
\label{sec:4}
The paper compares the performance of two deterministic autoencoder models and the $\beta$-variational autoencoder 
for the dimension reduction of two-mode periodic benchmark flow data and multi-mode aircraft ditching load data.  Emphasis is placed on the disentanglement of the latent variables and the analysis of the resulting modes. Considering the reconstruction accuracy and the level of disentanglement in the latent space, the uncorrelated autoencoder (UAE) outperformed the $\beta$-VAE, while being deterministic and easier to train with respect to the choice of the hyperparameter that balances the reconstruction loss and latent space loss in both test cases.
Therefore, if only reconstruction accuracy and disentanglement of latent variables are desired in a surrogate model, but not specifically a normally distributed latent space, the UAE may more easily provide satisfactory results.
%
Similarly to $\beta$-VAE, the UAE can identify a small and limited number of truly active latent variables when the model is trained with a larger latent space dimension than required. Such active latent variables can be easily identified by their standard deviation. 
%
The second deterministic autoencoder model tested, the orthogonal autoencoder (OAE), was also able to return small correlation coefficients between the latent variables. However, it was not able to identify the active latent variables in this study when trained with a higher latent space dimension.
%
The analysis of the ditching case provided a better understanding of how the different latent variables contribute to the reconstruction. It was observed that different latent variables concentrate at different temporal phases of the impact. This can be of importance in future work, when deformation-induced load changes are to be included in the model.