\begin{figure*}[h!]
\newcommand{\x}{0.2}
\newcommand{\y}{0pt}

\centering
    \subfigure{
        \includegraphics[width=\x\linewidth]{images/exp-state-dense/SeqOfColors3-v0-state.pdf}
    }
    \subfigure{
        \includegraphics[width=\x\linewidth]{images/exp-state-dense/SeqOfColors5-v0-state.pdf}
    }
    \subfigure{
        \includegraphics[width=\x\linewidth]{images/exp-state-dense/SeqOfColors7-v0-state.pdf}
    }

    \subfigure{
        \includegraphics[width=\x\linewidth]{images/exp-state-dense/BunchOfColors3-v0-state.pdf}
    }
    \subfigure{
        \includegraphics[width=\x\linewidth]{images/exp-state-dense/BunchOfColors5-v0-state.pdf}
    }
    \subfigure{
        \includegraphics[width=\x\linewidth]{images/exp-state-dense/BunchOfColors7-v0-state.pdf}
    }

    \subfigure{
        \includegraphics[width=\x\linewidth]{images/exp-state-dense/ChainOfColors3-v0-state.pdf}
    }
    \subfigure{
        \includegraphics[width=\x\linewidth]{images/exp-state-dense/ChainOfColors5-v0-state.pdf}
    }
    \subfigure{
        \includegraphics[width=\x\linewidth]{images/exp-state-dense/ChainOfColors7-v0-state.pdf}
    }

% \captionsetup{labelformat=default}
\caption{Demonstration of PPO-MLP performance on MIKASA-Robo benchmark when trained with oracle-level \texttt{state} information. Results are shown for memory capacity (\texttt{SeqOfColors[3,5,7]-v0}, \texttt{BunchOfColors[3,5,7]-v0}) and sequential memory (\texttt{ChainOfColors[3,5,7]-v0}).}
\label{fig:all-environments-group-2}
% }

\end{figure*}