\begin{table*}[t!]
\small
\centering
\caption{Description of available modes of agent training on tasks from the proposed ManiSkill-Memory benchmark. An important feature is that any combination of these modes is allowed, however, \texttt{rgb}+\texttt{joints} is canonical for testing the agent's memory. When training in \texttt{state} mode, the problem is solved in pure MDP.}
\vspace{-5pt}
\label{tab:include-description}
\setlength{\tabcolsep}{1mm}
\renewcommand{\arraystretch}{0.1}
\setlength{\tabcolsep}{1mm}
\resizebox{\textwidth}{!}{

\begin{tabular}{lp{16cm}}

\toprule

\makecell[lc]{\textbf{Supported Training} \\ \textbf{Regimes}}
& \textbf{Description} \\

\midrule

\texttt{state}
& All the environment and agent information in vector form needed to successfully solve the task, including additional oracle info and Tool Center Point (TCP) pose (end effector center coordinate). For example, coordinates and velocities of all objects in the environment, robot joins, etc.

\\

\texttt{rgb}
& Two images: an image of the robot and table from above and an image of the robot gripper. When using this mode, the TCP pose information is additionally used.

\\

\texttt{joints}
& Information about robot joint positions and joint velocities as well as additional information about TCP pose in vector form without any information about the environment.

\\

\texttt{oracle}
& Information about the environment that defines the memory-intensive task. For example, target cap number in \texttt{ShellGame-v0} or ball velocity in \texttt{Intercept-v0}. This information is built into \texttt{state} mode and is not available in other modes, but can be connected additionally for debugging.

\\
\midrule
\texttt{prompt}
& Additional information that doesn't change over time passed to the agent at each step in case if not None. such as in \texttt{RotateLenient-v0} and \texttt{RotateStrict-v0} is the angle by which to rotate the peg relative to its original position.

\\

\bottomrule
\end{tabular}
}
\vspace{-5pt}
\end{table*}