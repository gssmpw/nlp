\section{Classification of memory-intensive tasks}
\label{sec:tasks-classification}

The evaluation of memory capabilities in RL faces two major challenges. First, as shown in~\autoref{tab:behcmark-baseline}, research studies use different sets of environments with minimal overlap, making it difficult to compare memory-enhanced agents across studies. Second, even within individual studies, benchmarks may focus on testing similar memory aspects (e.g., remembering object locations) while neglecting others (e.g., reconstructing sequential events), leading to incomplete evaluation of agents’ memory.

Different architectures may exhibit varying performance across memory tasks. For instance, an architecture optimized for long-term object property recall might struggle with sequential memory tasks, yet these limitations often remain undetected due to the narrow focus of existing evaluation approaches.

To address these challenges, we propose a systematic approach to memory evaluation in RL. Given the impracticality of testing agents on every possible memory-intensive environment, \textbf{we aim to identify a minimal diagnostic set that comprehensively covers different memory requirements}. Drawing from established research in developmental psychology and cognitive science, where similar memory challenges have been extensively studied in humans, we develop a categorization framework consisting of four distinct memory task classes, detailed in~\autoref{sec:mem-class}.

\subsection{Memory: From Cognitive Science to RL}
In developmental psychology and cognitive science, memory is classified into categories based on cognitive processes. Key concepts include object permanence~\citep{piaget1952origins}, which involves remembering the existence of objects out of sight, and categorical perception~\citep{liberman1957discrimination}, where objects are grouped based on attributes like color or shape. Working memory \citep{baddeley1992working} and memory span~\citep{daneman1980individual} refer to the ability to hold and manipulate information over time, while causal reasoning~\citep{kuhn2012development} and \mbox{transitive inference}~\citep{heckers2004hippocampal} involve understanding cause-and-effect relationships and deducing hidden relationships, respectively.

The RL field has attempted to utilize these concepts in the design of specific memory-intensive environments~\cite{mra,hcam}, but these have been limited at the task design level. Of particular interest, however, is how existing memory-intensive tasks can be categorized using these concepts to develop a benchmark on which to test the greatest number of memory capabilities of memory-enhanced agents, and it is this problem that we address in this paper.
Thus, we aim to provide a balanced framework that covers important aspects of memory for real-world applications while maintaining practical simplicity (see ~\autoref{fig:mikasa}).

\subsection{Taxonomy of Memory Tasks}
\label{sec:mem-class}
\begin{figure*}[t!]
    \vspace{-5pt}
    \centering
    \includegraphics[width=\textwidth]{images/mikasa.pdf}
    \vspace{-15pt}
    \caption{MIKASA bridges the gap between human-like memory complexity and RL agents requirements. While agents tasks don’t require the full spectrum of human memory capabilities, they can’t be reduced to simple spatio-temporal dependencies. MIKASA provides a balanced framework that captures essential memory aspects for agents tasks while maintaining practical simplicity.}
    \label{fig:mikasa}
    \vspace{-15pt}
\end{figure*}

We introduce a comprehensive task classification framework for evaluating memory mechanisms in RL. Our framework categorizes memory-intensive tasks into four fundamental types, each targeting distinct aspects of memory capabilities:

\begin{enumerate}
    \item \textbf{Object Memory.} Tasks that evaluate an agent's ability to maintain object-related information over time, particularly when objects become temporarily unobservable. These tasks align with the cognitive concept of object permanence, requiring agents to track object properties when occluded, maintain object state representations, and recognize encountered objects.
    \item \textbf{Spatial Memory.} Tasks focused on environmental awareness and navigation, where agents must remember object locations, maintain mental maps of environment layouts, and navigate based on previously observed spatial information.
    \item \textbf{Sequential Memory.} Tasks that test an agent's ability to process and utilize temporally ordered information, similar to human serial recall and working memory. These tasks require remembering action sequences, maintaining order-dependent information, and using past decisions to inform future actions.
    \item \textbf{Memory Capacity. }Tasks that challenge an agent's ability to manage multiple pieces of information simultaneously, analogous to human memory span. These tasks evaluate information retention limits and multi-task information processing.
\end{enumerate}

This classification framework enables systematic evaluation of memory-enhanced RL agents across diverse scenarios. By providing a structured approach to memory task categorization, we establish a foundation for comprehensive benchmarking that spans the wide spectrum of memory requirements. In the following section, we present a carefully curated set of tasks based on this classification, forming the basis of our proposed MIKASA benchmark.