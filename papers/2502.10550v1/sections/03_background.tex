\section{Background}
\label{sec:background}

\subsection{Partially Observable Markov Decision Process}
\label{app:pomdp}
Partially Observable Markov Decision Process (POMDP)~\citep{pomdp_new} extend MDP to account for partial observability, where an agent observes only noisy or incomplete information about the true environments state. POMDP defined by a tuple $(S,A,T,R,\Omega,O,\gamma)$, where: \begin{wraptable}{r}{0.5\textwidth}
\small
\centering
\caption{Key memory-intensive environments from the reviewed studies for evaluating agent memory. The Atari~\citep{atari} environment with frame stacking is included to illustrate that many memory-enhanced agents are tested solely in MDP. \colorbox{LightViolet}{Benchmark first introduced in the same work}. \textcolor{LightGreen!100}{Benchmark is open-sourced}.}
\vspace{-10pt}
\label{tab:behcmark-baseline}
\begin{adjustbox}{width=0.5\textwidth}
\begin{tabular}{lccccccccccccccccccc}

\toprule




& \rotatebox[origin=c]{90}{DRQN~\citep{drqn}}
& \rotatebox[origin=c]{90}{DTQN~\citep{esslinger2022dtqn}}
& \rotatebox[origin=c]{90}{HCAM~\citep{hcam}}
& \rotatebox[origin=c]{90}{AMAGO~\citep{amago2024}}
& \rotatebox[origin=c]{90}{GTrXL~\citep{gtrxl}}
& \rotatebox[origin=c]{90}{R2I~\citep{r2i}}
& \rotatebox[origin=c]{90}{RATE~\citep{rate2024}}
& \rotatebox[origin=c]{90}{R2A~\citep{goyal2022retrieval}}
& \rotatebox[origin=c]{90}{Modified S5~\citep{modified_s5}}
& \rotatebox[origin=c]{90}{\makecell{Neural Map \\ ~\citep{neural_map}}}
& \rotatebox[origin=c]{90}{GBMR~\citep{gbmr}}
& \rotatebox[origin=c]{90}{EMDQN~\citep{emdqn}}
& \rotatebox[origin=c]{90}{MRA~\citep{mra}}
& \rotatebox[origin=c]{90}{FMRQN~\citep{memnns}}
& \rotatebox[origin=c]{90}{ADRQN~\citep{adrqn}}
& \rotatebox[origin=c]{90}{DCEM~\citep{dcem}}
& \rotatebox[origin=c]{90}{R2D2~\citep{r2d2}}
& \rotatebox[origin=c]{90}{ERLAM~\citep{erlam}}
% & \rotatebox[origin=c]{90}{\todo{RMA}}
& \rotatebox[origin=c]{90}{AdaMemento~\citep{adamemento}}

\\

\midrule

Atari w/o FrameStack
& \cellcolor{LightViolet}\textcolor{LightGreen}{\ding{51}} % DRQN
& % DTQN
& % HCAM
& % AMAGO
& % GTrXL
& % R2I
& % RATE
& % R2A
& % Modified S5
& % Neural Map
& % GBMR
& % EMDQN
& % MRA
& % FMRQN
& \textcolor{LightGreen}{\ding{51}} % ADRQN
& % DCEM
& % R2D2
& % ERLAM
& \textcolor{LightGreen}{\ding{51}} % AdaMemento

\\

Atari with FrameStack
& % DRQN
& % DTQN
& % HCAM
& % AMAGO
& % GTrXL
& \textcolor{LightGreen}{\ding{51}} % R2I
& \textcolor{LightGreen}{\ding{51}} % RATE
& \textcolor{LightGreen}{\ding{51}} % R2A
& % Modified S5
& % Neural Map
& \textcolor{LightGreen}{\ding{51}} % GBMR
& \textcolor{LightGreen}{\ding{51}} % EMDQN
& % MRA
& % FMRQN
& % ADRQN
& % DCEM
& \textcolor{LightGreen}{\ding{51}} % R2D2
& \textcolor{LightGreen}{\ding{51}} % ERLAM
& % AdaMemento

\\
\midrule

gym-gridverse
& % DRQN
& \textcolor{LightGreen}{\ding{51}} % DTQN
& % HCAM
& % AMAGO
& % GTrXL
& % R2I
& % RATE
& % R2A
& % Modified S5
& % Neural Map
& % GBMR
& % EMDQN
& % MRA
& % FMRQN
& % ADRQN
& % DCEM
& % R2D2
& % ERLAM
& % AdaMemento

\\

car flag
& % DRQN
& \textcolor{LightGreen}{\ding{51}} % DTQN
& % HCAM
& % AMAGO
& % GTrXL
& % R2I
& % RATE
& % R2A
& % Modified S5
& % Neural Map
& % GBMR
& % EMDQN
& % MRA
& % FMRQN
& % ADRQN
& % DCEM
& % R2D2
& % ERLAM
& % AdaMemento

\\

memory card
& % DRQN
& \cellcolor{LightViolet}\textcolor{LightGreen}{\ding{51}} % DTQN
& % HCAM
& % AMAGO
& % GTrXL
& % R2I
& % RATE
& % R2A
& % Modified S5
& % Neural Map
& % GBMR
& % EMDQN
& % MRA
& % FMRQN
& % ADRQN
& % DCEM
& % R2D2
& % ERLAM
& % AdaMemento

\\

Hallway
& % DRQN
& \textcolor{LightGreen}{\ding{51}} % DTQN
& % HCAM
& % AMAGO
& % GTrXL
& % R2I
& % RATE
& % R2A
& % Modified S5
& % Neural Map
& % GBMR
& % EMDQN
& % MRA
& % FMRQN
& % ADRQN
& % DCEM
& % R2D2
& % ERLAM
& % AdaMemento

\\

HeavenHell
& % DRQN
& \textcolor{LightGreen}{\ding{51}} % DTQN
& % HCAM
& % AMAGO
& % GTrXL
& % R2I
& % RATE
& % R2A
& % Modified S5
& % Neural Map
& % GBMR
& % EMDQN
& % MRA
& % FMRQN
& % ADRQN
& % DCEM
& % R2D2
& % ERLAM
& % AdaMemento

\\

Ballet
& % DRQN
& % DTQN
& \cellcolor{LightViolet}\textcolor{LightGreen}{\ding{51}} % HCAM
& % AMAGO
& % GTrXL
& % R2I
& % RATE
& % R2A
& % Modified S5
& % Neural Map
& % GBMR
& % EMDQN
& % MRA
& % FMRQN
& % ADRQN
& % DCEM
& % R2D2
& % ERLAM
& % AdaMemento

\\

Object Permanence
& % DRQN
& % DTQN
& \cellcolor{LightViolet}\ding{51} % HCAM
& % AMAGO
& % GTrXL
& % R2I
& % RATE
& % R2A
& % Modified S5
& % Neural Map
& % GBMR
& % EMDQN
& % MRA
& % FMRQN
& % ADRQN
& % DCEM
& % R2D2
& % ERLAM
& % AdaMemento

\\

DMLab-30
& % DRQN
& % DTQN
& \textcolor{LightGreen}{\ding{51}} % HCAM
& % AMAGO
& \textcolor{LightGreen}{\ding{51}} % GTrXL
& % R2I
& % RATE
& % R2A
& % Modified S5
& % Neural Map
& % GBMR
& % EMDQN
& % MRA
& % FMRQN
& % ADRQN
& % DCEM
& \textcolor{LightGreen}{\ding{51}} % R2D2
& % ERLAM
& % AdaMemento

\\

POPGym
& % DRQN
& % DTQN
&  % HCAM
& \textcolor{LightGreen}{\ding{51}} % AMAGO
& % GTrXL
& \ding{51} % R2I
& % RATE
& % R2A
& \textcolor{LightGreen}{\ding{51}} % Modified S5
& % Neural Map
& % GBMR
& % EMDQN
& % MRA
& % FMRQN
& % ADRQN
& \textcolor{LightGreen}{\ding{51}} % DCEM
& % R2D2
& % ERLAM
& % AdaMemento

\\

Passive T-Maze
& % DRQN
& % DTQN
&  % HCAM
& \textcolor{LightGreen}{\ding{51}} % AMAGO
& % GTrXL
& % R2I
& \textcolor{LightGreen}{\ding{51}} % RATE
& % R2A
& % Modified S5
& % Neural Map
& % GBMR
& % EMDQN
& % MRA
& % FMRQN
& % ADRQN
& % DCEM
& % R2D2
& % ERLAM
& % AdaMemento

\\

ViZDoom-Two-Colors
& % DRQN
& % DTQN
&  % HCAM
& % AMAGO
& % GTrXL
& % R2I
& \textcolor{LightGreen}{\ding{51}} % RATE
& % R2A
& % Modified S5
& % Neural Map
& % GBMR
& % EMDQN
& % MRA
& % FMRQN
& % ADRQN
& % DCEM
& % R2D2
& % ERLAM
& % AdaMemento

\\

Numpad
& % DRQN
& % DTQN
&  % HCAM
& % AMAGO
& \ding{51} % GTrXL
& % R2I
& % RATE
& % R2A
& % Modified S5
& % Neural Map
& % GBMR
& % EMDQN
& % MRA
& % FMRQN
& % ADRQN
& % DCEM
& % R2D2
& % ERLAM
& % AdaMemento

\\

Memory Maze
& % DRQN
& % DTQN
&  % HCAM
& % AMAGO
& % GTrXL
& \textcolor{LightGreen}{\ding{51}} % R2I
& \textcolor{LightGreen}{\ding{51}} % RATE
& % R2A
& % Modified S5
& % Neural Map
& % GBMR
& % EMDQN
& % MRA
& % FMRQN
& % ADRQN
& % DCEM
& % R2D2
& % ERLAM
& % AdaMemento

\\

Memory Maze (apples)
& % DRQN
& % DTQN
&  % HCAM
& % AMAGO
& \cellcolor{LightViolet}\ding{51} % GTrXL
& % R2I
& % RATE
& % R2A
& % Modified S5
& % Neural Map
& % GBMR
& % EMDQN
& % MRA
& % FMRQN
& % ADRQN
& % DCEM
& % R2D2
& % ERLAM
& % AdaMemento

\\

Minigrid-Memory
& % DRQN
& % DTQN
&  % HCAM
& % AMAGO
& % GTrXL
& % R2I
& \textcolor{LightGreen}{\ding{51}} % RATE
& % R2A
& % Modified S5
& % Neural Map
& % GBMR
& % EMDQN
& % MRA
& % FMRQN
& % ADRQN
& % DCEM
& % R2D2
& % ERLAM
& % AdaMemento

\\

BSuite
& % DRQN
& % DTQN
&  % HCAM
& % AMAGO
& % GTrXL
& \textcolor{LightGreen}{\ding{51}} % R2I
& % RATE
& % R2A
& \textcolor{LightGreen}{\ding{51}} % Modified S5
& % Neural Map
& % GBMR
& % EMDQN
& % MRA
& % FMRQN
& % ADRQN
& % DCEM
& % R2D2
& % ERLAM
& % AdaMemento

\\

Goal-Search
& % DRQN
& % DTQN
&  % HCAM
& % AMAGO
& % GTrXL
& % R2I
& % RATE
& % R2A
& % Modified S5
& \cellcolor{LightViolet}\ding{51} % Neural Map
& % GBMR
& % EMDQN
& % MRA
& % FMRQN
& % ADRQN
& % DCEM
& % R2D2
& % ERLAM
& % AdaMemento

\\

Doom Maze
& % DRQN
& % DTQN
&  % HCAM
& % AMAGO
& % GTrXL
& % R2I
& % RATE
& % R2A
& % Modified S5
& \cellcolor{LightViolet}\ding{51} % Neural Map
& % GBMR
& % EMDQN
& % MRA
& % FMRQN
& % ADRQN
& % DCEM
& % R2D2
& % ERLAM
& % AdaMemento

\\

PsychLab
& % DRQN
& % DTQN
&  % HCAM
& % AMAGO
& % GTrXL
& % R2I
& % RATE
& % R2A
& % Modified S5
& % Neural Map
& % GBMR
& % EMDQN
& \textcolor{LightGreen}{\ding{51}} % MRA
& % FMRQN
& % ADRQN
& % DCEM
& % R2D2
& % ERLAM
& % AdaMemento

\\

Spot the Difference
& % DRQN
& % DTQN
&  % HCAM
& % AMAGO
& % GTrXL
& % R2I
& % RATE
& % R2A
& % Modified S5
& % Neural Map
& % GBMR
& % EMDQN
& \cellcolor{LightViolet}\textcolor{LightGreen}{\ding{51}} % MRA
& % FMRQN
& % ADRQN
& % DCEM
& % R2D2
& % ERLAM
& % AdaMemento

\\

Goal Navigation
& % DRQN
& % DTQN
&  % HCAM
& % AMAGO
& % GTrXL
& % R2I
& % RATE
& % R2A
& % Modified S5
& % Neural Map
& % GBMR
& % EMDQN
& \cellcolor{LightViolet}\textcolor{LightGreen}{\ding{51}} % MRA
& % FMRQN
& % ADRQN
& % DCEM
& % R2D2
& % ERLAM
& % AdaMemento

\\

Transitive Inference
& % DRQN
& % DTQN
&  % HCAM
& % AMAGO
& % GTrXL
& % R2I
& % RATE
& % R2A
& % Modified S5
& % Neural Map
& % GBMR
& % EMDQN
& \cellcolor{LightViolet}\textcolor{LightGreen}{\ding{51}} % MRA
& % FMRQN
& % ADRQN
& % DCEM
& % R2D2
& % ERLAM
& % AdaMemento

\\

I-Maze
& % DRQN
& % DTQN
&  % HCAM
& % AMAGO
& % GTrXL
& % R2I
& % RATE
& % R2A
& % Modified S5
& % Neural Map
& % GBMR
& % EMDQN
& % MRA
& \cellcolor{LightViolet}\ding{51} % FMRQN
& % ADRQN
& % DCEM
& % R2D2
& % ERLAM
& % AdaMemento

\\

Pattern Matching
& % DRQN
& % DTQN
&  % HCAM
& % AMAGO
& % GTrXL
& % R2I
& % RATE
& % R2A
& % Modified S5
& % Neural Map
& % GBMR
& % EMDQN
& % MRA
& \cellcolor{LightViolet}\ding{51} % FMRQN
& % ADRQN
& % DCEM
& % R2D2
& % ERLAM
& % AdaMemento

\\

Random Maze
& % DRQN
& % DTQN
&  % HCAM
& % AMAGO
& % GTrXL
& % R2I
& % RATE
& % R2A
& % Modified S5
& % Neural Map
& % GBMR
& % EMDQN
& % MRA
& \cellcolor{LightViolet}\ding{51} % FMRQN
& % ADRQN
& % DCEM
& % R2D2
& % ERLAM
& % AdaMemento

\\

Unity Fast-Mapping Task
& % DRQN
& % DTQN
&  % HCAM
& % AMAGO
& % GTrXL
& % R2I
& % RATE
& % R2A
& % Modified S5
& % Neural Map
& % GBMR
& % EMDQN
& % MRA
& % FMRQN
& % ADRQN
& \cellcolor{LightViolet}\ding{51} % DCEM
& % R2D2
& % ERLAM
& % AdaMemento

\\

Action Associative Retrieval
& % DRQN
& % DTQN
&  % HCAM
& % AMAGO
& % GTrXL
& % R2I
& \cellcolor{LightViolet}\textcolor{LightGreen}{\ding{51}} % RATE
& % R2A
& % Modified S5
& % Neural Map
& % GBMR
& % EMDQN
& % MRA
& % FMRQN
& % ADRQN
& % DCEM
& % R2D2
& % ERLAM
& % AdaMemento

\\

BabyAI
& % DRQN
& % DTQN
& % HCAM
& % AMAGO
& % GTrXL
& % R2I
& % RATE
& \textcolor{LightGreen}{\ding{51}} % R2A
& % Modified S5
& % Neural Map
& % GBMR
& % EMDQN
& % MRA
& % FMRQN
& % ADRQN
& % DCEM
& % R2D2
& % ERLAM
& % AdaMemento
\\



\bottomrule
\end{tabular}
% }
\end{adjustbox}
\vspace{-30pt}
\end{wraptable}$S$ is the set of states representing the complete environment configuration; $A$ is the action space; $T(s'|s,a): S \times A \times S \to [0,1]$ is the transition function defining the probability of reaching state $s'$ from state $s$ after taking action $a$; $R(s,a): S \times A \to \mathbb{R}$ is the reward function specifying the immediate reward for taking action $a$ in state $s$; $\Omega$ is the observation space containing all possible observations; $O(o|s,a): S \times A \times \Omega \to [0,1]$ is the observation function defining the probability of observing $o$ after taking action $a$ and reaching state $s$; $\gamma \in [0,1)$ is the discount factor determining the importance of future rewards. The objective is to find a policy $\pi$ that maximizes the expected discounted cumulative reward: $\mathbb{E}_\pi\left[\sum_{t=0}^{\infty} \gamma^t R(s_t,a_t)\right]$, where $a_t \sim \pi(\cdot|o_{1:t})$ depends on the history of observations rather than the true state. Relying on partial observations makes POMDPs harder to solve than MDPs.

\subsection{Memory-intensive environments}
Memory-intensive environment is an environment where agents must leverage past experiences to make decisions, often in problems with long-term dependencies or delayed rewards. More formally, following~\citet{memory_rl}, a memory-intensive task is a POMDP where there exists a correlation horizon $\xi>1$, representing the minimum number of timesteps between an event critical for decision-making and when that information must be recalled. Popular memory-intensive environments in RL are listed in~\autoref{tab:behcmark-baseline}. One way to solving memory-intensive environments is to augment agents with memory mechanisms (see~\autoref{app:memory-mechanisms}). 

\subsection{Robotic Tabletop Manipulation}
\label{app:tabletop}

Robotic tabletop manipulation~\citep{shridhar2022cliport} involves robots manipulating objects on flat surfaces through actions like grasping, pushing, and picking. While crucial for real-world applications~\citep{levine2018learning}, most existing simulators treat these tasks as MDPs without memory requirements, failing to capture the spatio-temporal dependencies present in real scenarios. This limitation hinders the development of memory-enhanced agents for practical applications.