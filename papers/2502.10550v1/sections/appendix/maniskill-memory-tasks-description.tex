\newpage

\section{MIKASA-Robo Detailed Tasks Description}
\label{app:tasks-description}

In this section, we provide comprehensive descriptions of the 32 memory-intensive tasks that comprise the MIKASA-Robo benchmark. Each task is designed to evaluate specific aspects of memory capabilities in robotic manipulation, ranging from object tracking and spatial memory to sequential decision-making. For each task, we detail its objective, memory requirements, observation space, reward structure, and success criteria. Additionally, we explain how task complexity increases across different variants and discuss the specific memory challenges they present. The following subsections describe each task category and its variants in detail.

Each of the proposed environment supports multiple observation modes:
\begin{itemize}
    \item \textbf{State}: Full state information including ball position
    \item \textbf{RGB+joints}: Two camera views (top-down and gripper) plus robot joint states
    \item \textbf{RGB}: Only visual information from two cameras
\end{itemize}

In the case of \texttt{RotateLenient-v0} and \texttt{RotateStrict-v0}, the prompt information available at each step is additionally added to each observation.

\newpage
\begin{figure*}[h!]
    \centering
    \includegraphics[width=\textwidth]{images/appedix-img/app-shell-game-touch.pdf}
    \vspace{-15pt}
    \caption{\texttt{ShellGameTouch-v0}: The robot observes a ball in front of it. next, this ball is covered by a mug and then the robot has to touch the mug with the ball underneath.}
    \label{fig:app-shell-game-touch}
    \vspace{-15pt}
\end{figure*}
%%%%%%%%%%%%%%%%%%%%%%%% SHELL-GAME %%%%%%%%%%%%%%%%%%%%%%%%%%%%%%
\subsection{ShellGame-v0}
\label{app:shell-game}

The \texttt{ShellGame-v0} task (\autoref{fig:app-shell-game-touch}) is inspired by a simplified version of the classic shell game, which tests a person's ability to remember object locations when they become occluded. This task evaluates an agent's capacity for object permanence and spatial memory, crucial skills for real-world robotic manipulation where objects frequently become temporarily hidden from view.

\paragraph{Environment Description} The environment consists of three identical mugs placed on a table and a red ball. The task proceeds in three phases:
\begin{enumerate}
    \item \textbf{Observation Phase} (steps 0-4): The ball is placed at one of three positions, and the agent can observe its location.
    \item \textbf{Occlusion Phase} (step 5): The ball and positions are covered by three identical mugs.
    \item \textbf{Action Phase} (steps 6+): The agent must interact with the mug covering the ball's location. The type of target interaction depends on the selected mode: \texttt{Touch}, \texttt{Push} and \texttt{Pick}.
\end{enumerate}

\paragraph{Task Modes} The task includes three variants of increasing difficulty:
\begin{itemize}
    \item \texttt{Touch}: The agent only needs to touch the correct mug
    \item \texttt{Push}: The agent must push the correct mug to a designated area
    \item \texttt{Pick}: The agent must pick and lift the correct mug above a specified height
\end{itemize}

\paragraph{Success Criteria} Success is determined by:
\begin{itemize}
    \item \texttt{Touch}: Contact between the gripper and the correct mug
    \item \texttt{Push}: Moving forward the correct mug to the target zone
    \item \texttt{Pick}: Elevating the correct mug above 0.1m
\end{itemize}

\paragraph{Reward Structure} The environment provides both sparse and dense reward variants:
\begin{itemize}
    \item \textbf{Sparse}: Binary reward (1.0 for success, 0.0 otherwise)
    \item \textbf{Dense}: Continuous reward based on:
    \begin{itemize}
        \item Distance between gripper and target mug
        \item Robot's motion smoothness (static reward based on joint velocities)
        \item Task completion status (additional reward when the task is solved)
    \end{itemize}
\end{itemize}


\newpage
\begin{figure*}[h!]
    \centering
    \includegraphics[width=\textwidth]{images/appedix-img/app-remember-color-9.pdf}
    \vspace{-15pt}
    \caption{\texttt{RememberColor9-v0}: The robot observes a colored cube in front of it, then this cube disappears and an empty table is shown. Then 9 cubes appear on the table, and the agent must touch a cube of the same color as the one it observed at the beginning of the episode.}
    \label{fig:app-remember-color}
    \vspace{-15pt}
\end{figure*}
%%%%%%%%%%%%%%%%%%%%%%%% REMEMBER-COLOR %%%%%%%%%%%%%%%%%%%%%%%%%%%%%%
\subsection{RememberColor-v0}
\label{app:remember-color}

The \texttt{RememberColor-v0} task (\autoref{fig:app-remember-color}) tests an agent's ability to remember and identify objects based on their visual properties. This capability is essential for real-world robotics applications where agents must recall and match object characteristics across time intervals.

\paragraph{Environment Description} The environment presents a sequence of colored cubes on a table. The task proceeds in three phases:
\begin{enumerate}
    \item \textbf{Observation Phase} (steps 0-4): A cube of a specific color is displayed, and the agent must memorize its color.
    \item \textbf{Delay Phase} (steps 5-9): The cube disappears, leaving an empty table.
    \item \textbf{Selection Phase} (steps 10+): Multiple cubes of different colors appear (3, 5, or 9 depending on difficulty), and the agent must identify and interact with the cube matching the original color.
\end{enumerate}

\paragraph{Task Modes} The task includes three complexity levels:
\begin{itemize}
    \item \texttt{3} (easy): Choose from 3 different colors (red, lime, blue)
    \item \texttt{5} (Medium): Choose from 5 different colors (red, lime, blue, yellow, magenta)
    \item \texttt{9} (Hard): Choose from 9 different colors (red, lime, blue, yellow, magenta, cyan, maroon, olive, teal)
\end{itemize}

\paragraph{Success Criteria} Success is determined by:
\begin{itemize}
    \item Correctly identifying and touching the cube that matches the color shown in the observation phase
    \item Maintaining contact with the correct cube for at least 0.1 seconds
\end{itemize}

\paragraph{Reward Structure} The environment provides both sparse and dense reward variants:
\begin{itemize}
    \item \textbf{Sparse}: Binary reward (1.0 for success, 0.0 otherwise)
    \item \textbf{Dense}: Continuous reward based on:
    \begin{itemize}
        \item Distance between gripper and target cube
        \item Robot's motion smoothness (static reward based on joint velocities)
        \item Additional reward for robot being static while touching the correct cube
        \item Task completion status (additional reward when the task is solved)
    \end{itemize}
\end{itemize}


\newpage
\begin{figure*}[h!]
    \centering
    \includegraphics[width=\textwidth]{images/appedix-img/app-remember-shape-9.pdf}
    \vspace{-15pt}
    \caption{\texttt{RememberShape9-v0}: The robot observes an object with specific shape in front of it, then the object disappears and an empty table appears. Then 9 objects of different shapes appear on the table, and the agent must touch an object of the same shape as the one it observed at the beginning of the episode.}
    \label{fig:app-remember-shape}
    \vspace{-15pt}
\end{figure*}
%%%%%%%%%%%%%%%%%%%%%%%% REMEMBER-SHAPE %%%%%%%%%%%%%%%%%%%%%%%%%%%%%%
\subsection{RememberShape-v0}
\label{app:remember-shape}

The \texttt{RememberShape-v0} task (\autoref{fig:app-remember-shape}) evaluates an agent's ability to remember and identify objects based on their geometric properties. This capability is crucial for robotic applications where shape recognition and recall are essential for successful manipulation.

\paragraph{Environment Description} The environment presents a sequence of geometric shapes on a table. The task proceeds in three phases:
\begin{enumerate}
    \item \textbf{Observation Phase} (steps 0-4): A shape (cube, sphere, cylinder, etc.) is displayed, and the agent must memorize its geometry.
    \item \textbf{Delay Phase} (steps 5-9): The shape disappears, leaving an empty table.
    \item \textbf{Selection Phase} (steps 10+): Multiple shapes appear (3, 5, or 9 depending on difficulty), and the agent must identify and interact with the shape matching the original geometry.
\end{enumerate}

\paragraph{Task Modes} The task includes three complexity levels:
\begin{itemize}
    \item \texttt{3} (Easy): Choose from 3 different shapes (cube, sphere, cylinder)
    \item \texttt{5} (Medium): Choose from 5 different shapes (cube, sphere, cylinder cross, torus)
    \item \texttt{9} (Hard): Choose from 9 different shapes (cube, sphere, cylinder cross, torus, star, pyramid, t-shape, crescent)
\end{itemize}

\paragraph{Success Criteria} Success is determined by:
\begin{itemize}
    \item Correctly identifying and touching the object with the same shape shown in the observation phase
    \item Maintaining contact with the correct shape for at least 0.1 seconds
\end{itemize}

\paragraph{Reward Structure} The environment provides both sparse and dense reward variants:
\begin{itemize}
    \item \textbf{Sparse}: Binary reward (1.0 for success, 0.0 otherwise)
    \item \textbf{Dense}: Continuous reward based on:
    \begin{itemize}
        \item Distance between gripper and target object
        \item Robot's motion smoothness (static reward based on joint velocities)
        \item Additional reward for maintaining static position when touching correct object
        \item Task completion status (additional reward when the task is solved)
    \end{itemize}
\end{itemize}

\newpage
\begin{figure*}[h!]
    \centering
    \includegraphics[width=\textwidth]{images/appedix-img/app-remember-shape-and-color-5x3.pdf}
    \vspace{-15pt}
    \caption{\texttt{RememberShapeAndColor5x3-v0}: An object of a certain shape and color appears in front of the agent. Then the object disappears and the agent sees an empty table. Then objects of 5 different shapes and 3 different colors appear on the table and the agent has to touch what it observed at the beginning of the episode.}
    \label{fig:app-remember-shape-and-color-5x3}
    \vspace{-15pt}
\end{figure*}
%%%%%%%%%%%%%%%%%%%%%%%% REMEMBER-SHAPE-AND-COLOR %%%%%%%%%%%%%%%%%%%%%%%%%%%%%%
\subsection{RememberShapeAndColor-v0}
\label{app:remember-shape-and-color}

The \texttt{RememberShapeAndColor-v0} task (\autoref{fig:app-remember-shape-and-color-5x3}) evaluates an agent's ability to remember and identify objects based on multiple visual properties simultaneously. This task combines shape and color recognition, testing the agent's capacity to maintain and match multiple object features across time intervals.

\paragraph{Environment Description} The environment presents a sequence of colored geometric shapes on a table. The task proceeds in three phases:
\begin{enumerate}
    \item \textbf{Observation Phase} (steps 0-4): An object with specific shape and color is displayed, and the agent must memorize both properties.
    \item \textbf{Delay Phase} (steps 5-9): The object disappears, leaving an empty table.
    \item \textbf{Selection Phase} (steps 10+): Multiple objects with different combinations of shapes and colors appear, and the agent must identify and interact with the object matching both the original shape and color.
\end{enumerate}

\paragraph{Task Modes} The task includes three complexity levels based on the number of shape-color combinations:
\begin{itemize}
    \item \texttt{3x2} (Easy): Choose from 6 objects (3 shapes × 2 colors); shapes: cube, sphere, t-shape; colors: red, green
    \item \texttt{3x3} (Medium): Choose from 9 objects (3 shapes × 3 colors); shapes: cube, sphere, t-shape; colors: red, green, blue
    \item \texttt{5x3} (Hard): Choose from 15 objects (5 shapes × 3 colors); shapes: cube, sphere, t-shape, cross, torus; colors: red, green, blue
\end{itemize}

\paragraph{Success Criteria} Success is determined by:
\begin{itemize}
    \item Correctly identifying and touching the object that matches both the shape and color shown in the observation phase
    \item Maintaining contact with the correct object for at least 0.1 seconds
\end{itemize}

\paragraph{Reward Structure} The environment provides both sparse and dense reward variants:
\begin{itemize}
    \item \textbf{Sparse}: Binary reward (1.0 for success, 0.0 otherwise)
    \item \textbf{Dense}: Continuous reward based on:
    \begin{itemize}
        \item Distance between gripper and target object
        \item Robot's motion smoothness (static reward based on joint velocities)
        \item Additional reward for maintaining static position while touching correct object
        \item Task completion status (additional reward when the task is solved)
    \end{itemize}
\end{itemize}


\newpage
\begin{figure*}[h!]
    \centering
    \includegraphics[width=\textwidth]{images/appedix-img/app-intercept-medium.pdf}
    \vspace{-15pt}
    \caption{\texttt{InterceptMedium-v0}: A ball rolls on the table in front of the agent with a random initial velocity, and the agent's task is to intercept this ball and direct it at the target zone.}
    \label{fig:app-intercept-medium}
    \vspace{-15pt}
\end{figure*}
%%%%%%%%%%%%%%%%%%%%%%% INTERCEPT %%%%%%%%%%%%%%%%%%%%%%%%%%%%%%
\subsection{Intercept-v0}
\label{app:intercept}

The \texttt{Intercept-v0} task (\autoref{fig:app-intercept-grab-medium}) evaluates an agent's ability to predict and intercept a moving object based on its initial trajectory. This task tests the agent's capacity for motion prediction and spatial-temporal reasoning, which are essential skills for dynamic manipulation tasks in robotics.

\paragraph{Environment Description} The environment consists of a red ball moving across a table and a target zone. The task requires the agent to:
\begin{enumerate}
    \item Observe the ball's initial position and velocity
    \item Predict the ball's trajectory
    \item Guide the ball to reach a designated target zone
\end{enumerate}

\paragraph{Task Modes} The task includes three variants with increasing ball velocities:
\begin{itemize}
    \item \texttt{Slow}: Ball velocity range of 0.25-0.5 m/s
    \item \texttt{Medium}: Ball velocity range of 0.5-0.75 m/s
    \item \texttt{Fast}: Ball velocity range of 0.75-1.0 m/s
\end{itemize}

\paragraph{Success Criteria} Success is determined by:
\begin{itemize}
    \item Guiding the ball to enter the target zone
    \item The ball must come to rest within the target area (radius 0.1m)
\end{itemize}

\paragraph{Reward Structure} The environment provides both sparse and dense reward variants:
\begin{itemize}
    \item \textbf{Sparse}: Binary reward (1.0 for success, 0.0 otherwise)
    \item \textbf{Dense}: Continuous reward based on:
    \begin{itemize}
        \item Distance between gripper and ball
        \item Distance between ball and target zone
        \item Robot's motion smoothness (static reward based on joint velocities)
        \item Task completion status (additional reward when the task is solved)
    \end{itemize}
\end{itemize}


\newpage
\begin{figure*}[h!]
    % \vspace{-20pt}
    \centering
    \includegraphics[width=\textwidth]{images/appedix-img/app-intercept-grab-medium.pdf}
    \vspace{-15pt}
    \caption{\texttt{InterceptGrabMedium-v0}: A ball rolls on the table in front of the agent with a random initial velocity, and the agent's task is to intercept this ball with a gripper and lift it up.}
    \label{fig:app-intercept-grab-medium}
    \vspace{-15pt}
\end{figure*}
%%%%%%%%%%%%%%%%%%%%%%%% INTERCEPT-GRAB %%%%%%%%%%%%%%%%%%%%%%%%%%%%%%
\subsection{InterceptGrab-v0}
\label{app:intercept-grab}

The \texttt{InterceptGrab-v0} task (\autoref{fig:app-intercept-grab-medium}) extends the \texttt{Intercept-v0} task by requiring the agent to not only predict the trajectory of a moving object but also grasp it while in motion. This task evaluates the agent's ability to combine motion prediction with precise manipulation timing, simulating real-world scenarios where robots must catch or intercept moving objects.

\paragraph{Environment Description} The environment consists of a red ball moving across a table. The task requires the agent to:
\begin{enumerate}
    \item Observe the ball's initial position and velocity
    \item Predict the ball's trajectory
    \item Position the gripper to intercept the ball's path
    \item Time the grasping action correctly to catch the ball
    \item Maintain a stable grasp while bringing the ball to rest
\end{enumerate}

\paragraph{Task Modes} The task includes three variants with increasing ball velocities:
\begin{itemize}
    \item \texttt{Slow}: Ball velocity range of 0.25-0.5 m/s
    \item \texttt{Medium}: Ball velocity range of 0.5-0.75 m/s
    \item \texttt{Fast}: Ball velocity range of 0.75-1.0 m/s
\end{itemize}

\paragraph{Success Criteria} Success is determined by:
\begin{itemize}
    \item Successfully grasping the moving ball
    \item Maintaining a stable grasp until the ball comes to rest
    \item The robot must be static with the ball firmly grasped
\end{itemize}

\paragraph{Reward Structure} The environment provides both sparse and dense reward variants:
\begin{itemize}
    \item \textbf{Sparse}: Binary reward (1.0 for success, 0.0 otherwise)
    \item \textbf{Dense}: Continuous reward based on:
    \begin{itemize}
        \item Distance between gripper and ball
        \item Grasping reward
        \item Robot's motion smoothness (static reward based on joint velocities)
        \item Task completion status (additional reward when the task is solved)
    \end{itemize}
\end{itemize}


\newpage
\begin{figure*}[h!]
    % \vspace{-20pt}
    \centering
    \includegraphics[width=\textwidth]{images/appedix-img/app-rotate-lenient-pos.pdf}
    \vspace{-15pt}
    \caption{\texttt{RotateLenientPos-v0}: A randomly oriented peg is placed in front of the agent. The agent's task is to rotate this peg by a certain angle (the center of the peg can be shifted).}
    \label{fig:app-rotate-lenient-pos}
    \vspace{-15pt}
\end{figure*}
%%%%%%%%%%%%%%%%%%%%%%%% ROTATE-LENIENT %%%%%%%%%%%%%%%%%%%%%%%%%%%%%%
\subsection{RotateLenient-v0}
\label{app:rotate-lenient}

The \texttt{RotateLenient-v0} task (\autoref{fig:app-rotate-lenient-pos}) evaluates an agent's ability to remember and execute specific rotational movements. This task tests the agent's capacity to maintain and reproduce angular information, which is crucial for manipulation tasks requiring precise orientation control. This task tests the agent's ability to hold information in memory about how far peg has already rotated at the current step 
relative to its initial position.

\paragraph{Environment Description} The environment consists of a blue-colored peg on a table that must be rotated by a specified angle. The task proceeds in one phase, but the static prompt information about the target angle is available to the agent at each timestep:
\begin{enumerate}
    \item \textbf{Action Phase}: The agent must rotate the peg to match the target angle
\end{enumerate}

\paragraph{Task Modes} The task includes two variants with different rotation requirements:
\begin{itemize}
    \item \texttt{Pos}: Rotate by a positive angle between 0 and $\pi/2$
    \item \texttt{PosNeg}: Rotate by either positive or negative angle between $-\pi/4$ and $\pi/4$
\end{itemize}

\paragraph{Success Criteria} Success is determined by:
\begin{itemize}
    \item Rotating the peg to within the angle threshold (±0.1 radians) of the target angle
    \item Maintaining the final orientation in a stable position
    \item The robot must be static with the peg at the correct orientation
\end{itemize}

\paragraph{Reward Structure} The environment provides both sparse and dense reward variants:
\begin{itemize}
    \item \textbf{Sparse}: Binary reward (1.0 for success, 0.0 otherwise)
    \item \textbf{Dense}: Continuous reward based on:
    \begin{itemize}
        \item Distance between gripper and peg
        \item Angular distance to target rotation
        \item Stability of the peg's orientation
        \item Robot's motion smoothness (static reward based on joint velocities)
        \item Task completion status (additional reward when the task is solved)
    \end{itemize}
\end{itemize}


\newpage
\begin{figure*}[h!]
    % \vspace{-20pt}
    \centering
    \includegraphics[width=\textwidth]{images/appedix-img/app-rotate-strict-pos.pdf}
    \vspace{-15pt}
    \caption{\texttt{RotateStrictPos-v0}: A randomly oriented peg is placed in front of the agent. The agent's task is to rotate this peg by a certain angle (it is not allowed to move the center of the peg)}
    \label{fig:app-rotate-strict-pos}
    \vspace{-15pt}
\end{figure*}
%%%%%%%%%%%%%%%%%%%%%%%% ROTATE-STRICT %%%%%%%%%%%%%%%%%%%%%%%%%%%%%%
\subsection{RotateStrict-v0}
\label{app:rotate-strict}

The \texttt{RotateStrict-v0} task (\autoref{fig:app-rotate-strict-pos}) extends the \texttt{RotateLenient-v0} task with more stringent requirements for precise rotational control.

\paragraph{Environment Description} The environment consists of a blue-colored peg on a table that must be rotated by a specified angle while maintaining its position. The task proceeds in one phase, but the static prompt information about the target angle is available to the agent at each timestep:
\begin{enumerate}
    \item \textbf{Action Phase}: The agent must rotate the peg to match the target angle while keeping it centered
\end{enumerate}

\paragraph{Task Modes} The task includes two variants with different rotation requirements:
\begin{itemize}
    \item \texttt{Pos}: Rotate by a positive angle between 0 and $\pi/2$
    \item \texttt{PosNeg}: Rotate by either positive or negative angle between $-\pi/4$ and $\pi/4$
\end{itemize}

\paragraph{Success Criteria} Success is determined by:
\begin{itemize}
    \item Rotating the peg to within the angle threshold (±0.1 radians) of the target angle
    \item Maintaining the peg's position within 5cm of its initial XY coordinates
    \item The robot must be static with the peg at the correct orientation
    \item No significant deviation in other rotation axes
\end{itemize}

\paragraph{Reward Structure} The environment provides both sparse and dense reward variants:
\begin{itemize}
    \item \textbf{Sparse}: Binary reward (1.0 for success, 0.0 otherwise)
    \item \textbf{Dense}: Continuous reward based on:
    \begin{itemize}
        \item Distance between gripper and peg
        \item Angular distance to target rotation
        \item Position deviation from initial location
        \item Stability of the peg's orientation
        \item Robot's motion smoothness (static reward based on joint velocities)
        \item Task completion status (additional reward when the task is solved)
    \end{itemize}
\end{itemize}

\newpage
\begin{figure*}[h!]
    % \vspace{-20pt}
    \centering
    \includegraphics[width=\textwidth]{images/appedix-img/app-take-it-back.pdf}
    \vspace{-15pt}
    \caption{\texttt{TakeItBack-v0}: The agent observes a green cube in front of him. The agent's task is to move the green cube to the red target, and as soon as it lights up violet, return the cube to its original position (the agent does not observes the original position of the cube).}
    \label{fig:app-take-it-back}
    \vspace{-15pt}
\end{figure*}
%%%%%%%%%%%%%%%%%%%%%%%% TAKE-IT-BACK %%%%%%%%%%%%%%%%%%%%%%%%%%%%%%
\subsection{TakeItBack-v0}
\label{app:take-it-back}

The \texttt{TakeItBack-v0} task (\autoref{fig:app-take-it-back}) assesses the agent's ability to perform sequential tasks and memorize the starting position. This task tests the agent's capacity for sequential memory and spatial reasoning, requiring it to maintain information about past locations and achievements while executing a multi-step plan.

\paragraph{Environment Description} The environment consists of a green cube and two target regions (initial and goal) on a table. The task proceeds in two phases:
\begin{enumerate}
    \item \textbf{First Phase}: The agent must move the cube from its initial position to a goal region
    \item \textbf{Second Phase}: After reaching the goal, goal region change it's color from red to magenta, and the agent must return the cube to its original position (marked by the initial region and invisible for the agent)
\end{enumerate}

\paragraph{Success Criteria} Success is determined by:
\begin{itemize}
    \item First reaching the goal region with the cube
    \item Then returning the cube to the initial region
    \item Both goals must be achieved in sequence
\end{itemize}

\paragraph{Reward Structure} The environment provides both sparse and dense reward variants:
\begin{itemize}
    \item \textbf{Sparse}: Binary reward (1.0 for success, 0.0 otherwise)
    \item \textbf{Dense}: Continuous reward based on:
    \begin{itemize}
        \item Distance between gripper and cube
        \item Distance to current target region
        \item Progress through the task sequence
        \item Stability of cube manipulation
        \item Robot's motion smoothness (static reward based on joint velocities)
        \item Task completion status (additional reward when the task is solved)
    \end{itemize}
\end{itemize}

\newpage
\begin{figure*}[h!]
    \centering
    \includegraphics[width=\textwidth]{images/appedix-img/app-seq-of-colors-7.pdf}
    \vspace{-15pt}
    \caption{\texttt{SeqOfColors7-v0}: In front of the agent, 7 cubes of different colors appear sequentially. After the last cube is shown, the agent observes an empty table. Then 9 cubes of different colors appear on the table and the agent has to touch the cubes that were shown at the beginning of the episode in any order.}
    \label{fig:app-seq-of-colors}
    \vspace{-15pt}
\end{figure*}
%%%%%%%%%%%%%%%%%%%%%%%% SEQ-OF-COLORS %%%%%%%%%%%%%%%%%%%%%%%%%%%%%%
\subsection{SeqOfColors-v0}
\label{app:seq-of-colors}

The \texttt{SeqOfColors-v0} task (\autoref{fig:app-seq-of-colors}) evaluates an agent's ability to remember and reproduce an unordered sequence of colors. This task tests memory capacity capabilities essential for robotic tasks that require following specific patterns or sequences.

\paragraph{Environment Description} The environment presents a sequence of colored cubes that must be reproduced in any order. The task proceeds in two phases:
\begin{enumerate}
    \item \textbf{Observation Phase} (steps 0-($5N-1$)): A sequence of N colored cubes is shown one at a time, with each cube visible for 5 steps.
    \item \textbf{Delay Phase} (steps ($5N$)-($5N+4$)): All cubes disappear
    \item \textbf{Selection Phase} (steps ($5N+5$)+): A larger set of cubes appears, and the agent must identify and touch all previously shown cubes in any order
\end{enumerate}

\paragraph{Task Modes} The task includes three complexity levels:
\begin{itemize}
    \item \texttt{3} (Easy): Remember 3 colors demonstrated sequentially
    \item \texttt{5} (Medium): Remember 5 colors demonstrated sequentially
    \item \texttt{7} (Hard): Remember 7 colors demonstrated sequentially
\end{itemize}

\paragraph{Success Criteria} Success is determined by:
\begin{itemize}
    \item Correctly identifying and touching all cubes from the observation phase
    \item Order of selection doesn't matter
    \item Each cube must be touched for at least 0.1 seconds
    \item The demonstrated set must be touched without any mistakes
\end{itemize}

\paragraph{Reward Structure} The environment provides both sparse and dense reward variants:
\begin{itemize}
    \item \textbf{Sparse}: Binary reward (1.0 for success, 0.0 otherwise)
    \item \textbf{Dense}: Continuous reward based on:
    \begin{itemize}
        \item Distance between gripper and next target cube
        \item Number of correctly identified cubes
        \item Static reward for stable contact
        \item Robot's motion smoothness (static reward based on joint velocities)
        \item Task completion status (additional reward when the task is solved)
    \end{itemize}
\end{itemize}

\newpage
\begin{figure*}[h!]
    \centering
    \includegraphics[width=\textwidth]{images/appedix-img/app-bunch-of-colors-7.pdf}
    \vspace{-15pt}
    \caption{\texttt{BunchOfColors7-v0}: 7 cubes of different colors appear simultaneously in front of the agent. After the agent observes an empty table. Then, 9 cubes of different colors appear on the table and the agent has to touch the cubes that were shown at the beginning of the episode in any order.}
    \label{fig:app-bunch-of-colors}
    \vspace{-15pt}
\end{figure*}
%%%%%%%%%%%%%%%%%%%%%%%% BUNCH-OF-COLORS %%%%%%%%%%%%%%%%%%%%%%%%%%%%%%
\subsection{BunchOfColors-v0}
\label{app:bunch-of-colors}

The \texttt{BunchOfColors-v0} task (\autoref{fig:app-bunch-of-colors}) tests an agent's memory capacity by requiring it to remember multiple objects simultaneously. This capability is crucial for tasks requiring parallel processing of multiple items.

\paragraph{Environment Description} The environment presents multiple colored cubes simultaneously. The task proceeds in three phases:
\begin{enumerate}
    \item \textbf{Observation Phase} (steps 0-4): Multiple colored cubes are displayed simultaneously
    \item \textbf{Delay Phase} (steps 5-9): All cubes disappear
    \item \textbf{Selection Phase} (steps 10+): A larger set of cubes appears, and the agent must identify and touch all previously shown cubes in any order 
\end{enumerate}

\paragraph{Task Modes} The task includes three complexity levels:
\begin{itemize}
    \item \texttt{3} (Easy): Remember 3 colors demonstrated simultaneously
    \item \texttt{5} (Medium): Remember 5 colors demonstrated simultaneously
    \item \texttt{7} (Hard): Remember 7 colors demonstrated simultaneously
\end{itemize}

\paragraph{Success Criteria} Success is determined by:
\begin{itemize}
    \item Correctly identifying and touching all cubes from the observation phase
    \item Order of selection doesn't matter
    \item Each cube must be touched for at least 0.1 seconds
    \item The demonstrated set must be touched without any mistakes
\end{itemize}

\paragraph{Reward Structure} The environment provides both sparse and dense reward variants:
\begin{itemize}
    \item \textbf{Sparse}: Binary reward (1.0 for success, 0.0 otherwise)
    \item \textbf{Dense}: Continuous reward based on:
    \begin{itemize}
        \item Distance between gripper and next target cube
        \item Static reward for stable contact
        \item Number of correctly touched cubes
        \item Robot's motion smoothness (static reward based on joint velocities)
        \item Task completion status (additional reward when the task is solved)
    \end{itemize}
\end{itemize}

\newpage
\begin{figure*}[h!]
    \centering
    \includegraphics[width=\textwidth]{images/appedix-img/app-chain-of-colors-7.pdf}
    \vspace{-15pt}
    \caption{\texttt{ChainOfColors7-v0}: In front of the agent, 7 cubes of different colors appear sequentially. After the last cube is shown, the agent sees an empty table. Then 9 cubes of different colors appear on the table and the agent must unmistakably touch the cubes that were shown at the beginning of the episode, in the same strict order.}
    \label{fig:app-chain-of-colors}
    \vspace{-15pt}
\end{figure*}
%%%%%%%%%%%%%%%%%%%%%%%% CHAIN-OF-COLORS %%%%%%%%%%%%%%%%%%%%%%%%%%%%%%
\subsection{ChainOfColors-v0}
\label{app:chain-of-colors}

The \texttt{ChainOfColors-v0} task (\autoref{fig:app-chain-of-colors}) evaluates the agent's ability to store and retrieve ordered information. This task simulates scenarios where the agent must track changing relationships between objects over time.

\paragraph{Environment Description} The environment presents am ordered sequence (chain) of colored cubes that must be followed. The task proceeds in multiple phases:
\begin{enumerate}
    \item \textbf{Observation Phase} (steps 0-($5N-1$)): A sequence of N colored cubes is shown one at a time, with each cube visible for 5 steps.
    \item \textbf{Delay Phase} (steps ($5N$)-($5N+4$)): All cubes disappear
    \item \textbf{Selection Phase} (steps ($5N+5$)+): A larger set of cubes appears, and the agent must identify and touch all previously shown cubes in the exact order as demonstrated
\end{enumerate}

\paragraph{Task Modes} The task includes three complexity levels:
\begin{itemize}
    \item \texttt{3} (Easy): Remember 3 colors demonstrated sequentially
    \item \texttt{5} (Medium): Remember 5 colors demonstrated sequentially
    \item \texttt{7} (Hard): Remember 7 colors demonstrated sequentially
\end{itemize}

\paragraph{Success Criteria} Success is determined by:
\begin{itemize}
    \item Correctly identifying and touching all cubes from the observation phase in the exact order
    \item Each cube must be touched for at least 0.1 seconds
    \item The demonstrated set must be touched without any mistakes
\end{itemize}

\paragraph{Reward Structure} The environment provides both sparse and dense reward variants:
\begin{itemize}
    \item \textbf{Sparse}: Binary reward (1.0 for success, 0.0 otherwise)
    \item \textbf{Dense}: Continuous reward based on:
    \begin{itemize}
        \item Distance between gripper and next target cube
        \item Static reward for stable contact
        \item Number of correctly touched cubes
        \item Robot's motion smoothness (static reward based on joint velocities)
        \item Task completion status (additional reward when the task is solved)
    \end{itemize}
\end{itemize}