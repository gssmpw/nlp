\begin{lstlisting}[caption={Example code for running \texttt{MemoryLength-v0} environment.}, label={lst:memory}]
import mikasa_base
import gymnasium as gym

# use pre-defined env
# env_id = "MemoryLengthEasy-v0"
# env_kwargs = None

# create env using custom parameters 
env_id = "MemoryLength-v0"
env_kwargs = {"memory_length": 10, "num_bits": 1}
seed = 123

env = gym.make(env_id, env_kwargs)

obs, _ = env.reset(seed=seed)

for i in range(11):
    action = env.action_space.sample()
    next_obs, reward, terminations, truncations, infos = env.step(action)
env.close()
\end{lstlisting}


% \begin{lstlisting}[caption={Example code for running \texttt{MemoryLength-v0} environment.}, label={lst:memory}]
% import mikasa
% import gymnasium as gym

% def make_env(env_id, idx, capture_video, run_name, env_kwargs):
%     def thunk():
%         if capture_video and idx == 0:
%             env = gym.make(env_id, render_mode="rgb_array", **env_kwargs)
%             env = gym.wrappers.RecordVideo(env, f"videos/{run_name}")
%         else:
%             env = gym.make(env_id, **env_kwargs)
%         env = gym.wrappers.RecordEpisodeStatistics(env)
%         return env
%     return thunk

% # create env using custom parameters 
% num_envs = 8
% env_id = "MemoryLength-v0"
% env_kwargs = {"memory_length": 10, "num_bits": 1}

% # use pre-defined env
% # env_id = "MemoryLengthEasy-v0"
% # env_kwargs = None

% envs = gym.vector.AsyncVectorEnv(
%     [make_env(env_id, i, False, "demonstration", env_kwargs) for i in range(num_envs)]
% )

% obs, _ = envs.reset(seed=1)

% for i in range(11):
%     action = envs.action_space.sample()
%     next_obs, reward, terminations, truncations, infos = envs.step(action)
% \end{lstlisting}