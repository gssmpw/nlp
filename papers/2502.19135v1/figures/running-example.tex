% \documentclass[tikz,border=2mm]{standalone}
% \begin{document}
% \usetikzlibrary{shapes.geometric}

\begin{tikzpicture}[scale=1]
\footnotesize

\def\cubex{0.5} % Cube x-dimension
\def\cubey{0.5} % Cube y-dimension
\def\cubez{0.5} % Cube z-dimension

% Command to draw a 3D cube
\newcommand{\drawcube}[3]{
\draw[red,fill=yellow] (#1,#2,#3) -- ++(-\cubex,0,0) -- ++(0,-\cubey,0) -- ++(\cubex,0,0) -- cycle;
\draw[red,fill=yellow] (#1,#2,#3) -- ++(0,0,-\cubez) -- ++(0,-\cubey,0) -- ++(0,0,\cubez) -- cycle;
\draw[red,fill=yellow] (#1,#2,#3) -- ++(-\cubex,0,0) -- ++(0,0,-\cubez) -- ++(\cubex,0,0) -- cycle;
}


\node[trapezium, draw, trapezium left angle=-120, trapezium right angle=-60, trapezium stretches=false, minimum width=4cm, fill=white]    at (1.5,1) {};
\node[trapezium, draw, trapezium left angle=-120, trapezium right angle=-60, trapezium stretches=false, minimum width=4cm, fill=white]    at (7.5,1) {};


% Draw the cubes on the table
\drawcube{1}{1}{0};
\drawcube{2}{1}{0};

\drawcube{7.7}{1.5}{0};
\drawcube{7.7}{2}{0};

% Arrow between configurations
\draw[->,thick] (3.5,1) -- (5.5,1);

% References
\draw[->] (-1.2,-0.2) -- (3,-0.2);
\draw[->] (-1.2,-0.2) -- (0.25, 2.2);

\draw[-] (-0.25, -0.3) -- (-0.15, -0.1);
\node[] at (-0.25, -0.55) {$0$};
\draw[-] (0.3, -0.3) -- (0.4, -0.1);
\node[] at (0.3, -0.55) {$1$};
\draw[-] (0.85, -0.3) -- (0.95, -0.1);
\node[] at (0.85, -0.55) {$2$};
\draw[-] (1.4, -0.3) -- (1.5, -0.1);
\node[] at (1.4, -0.55) {$3$};

\draw[-] (-1.15, 0.05) -- (-0.95, 0.05);
\node[] at (-1.3, 0.05) {$0$};
\draw[-] (-0.8, 0.65) -- (-0.6, 0.65);
\node[] at (-0.95, 0.65) {$1$};
\draw[-] (-0.45, 1.2) -- (-0.25, 1.2);
\node[] at (-0.6, 1.2) {$2$};


\end{tikzpicture}

% \end{document}
