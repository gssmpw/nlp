
% Vectors
\def\x{{\mathbf x}}

% Sets
\def\R{\mathbb{R}}
\def\N{\mathbb{N}}
\def\Z{\mathbb{Z}}

% Stochastic
\def\Pr#1{\ensuremath{\text{Pr}\left [#1 \right ]}}
\def\E#1{\ensuremath{\text{E}\left \{#1 \right \}}}

% Matrix measures
\newcommand{\tr}[1]{{\mbox{Trace}\left (#1\right) }}

% CBS
\newcommand{\cbs}{\textsf{CBS}}

% Math operations
\newcommand{\ceil}[1]{\left\lceil #1 \right\rceil}
\newcommand{\floor}[1]{\left\lfloor #1 \right\rfloor}

% Frames
\newcommand{\frm}[1]{\langle #1\rangle}


% Theorems
%% as per the requirement new theorem styles can be included as shown below
%%\newtheorem{theorem}{Theorem}[section]% meant for sectionwise numbers
%% optional argument [theorem] produces theorem numbering sequence instead of independent numbers for Proposition

\theoremstyle{thmstyleone}
\newtheorem{theorem}{Theorem}
\newtheorem{acknowledgement}[theorem]{Acknowledgement}
%\newtheorem{algorithm}[theorem]{Algorithm}
\newtheorem{axiom}[theorem]{Axiom}
\newtheorem{case}[theorem]{Case}
\newtheorem{claim}[theorem]{Claim}
\newtheorem{conclusion}[theorem]{Conclusion}
\newtheorem{condition}[theorem]{Condition}
\newtheorem{conjecture}[theorem]{Conjecture}
\newtheorem{corollary}[theorem]{Corollary}
\newtheorem{criterion}[theorem]{Criterion}
\newtheorem{exercise}[theorem]{Exercise}
% \newtheorem{lemma}[LemCounter]{Lemma}
\newtheorem{notation}[theorem]{Notation}
\newtheorem{problem}[theorem]{Problem}
\newtheorem{solution}[theorem]{Solution}
\newtheorem{summary}[theorem]{Summary}
% \newtheorem{assumption}[Ass_counter]{Assumption}
% \newtheorem{property}[PropCounter]{Property}
\newtheorem{proposition}[theorem]{Proposition}% 
%%\newtheorem{proposition}{Proposition}% to get separate numbers for theorem and proposition etc.
\theoremstyle{thmstyletwo}%
\newtheorem{example}{Example}%
\newtheorem{remark}{Remark}%
\theoremstyle{thmstylethree}%
\newtheorem{definition}{Definition}%


\newcommand{\aStart}[1]{
\ifmmode
\texttt{$#1$}_\vdash
\else 
$\texttt{$#1$}_\vdash$
\fi
}
\newcommand{\aEnd}[1]{
\ifmmode
\texttt{$#1$}_\dashv
\else 
$\texttt{$#1$}_\dashv$
\fi
}
\newcommand{\pc}[1]{pre(#1)}
\newcommand{\eff}[1]{eff(#1)}
\newcommand{\fl}[1]{fl(#1)}
\newcommand{\ach}[1]{\text{Enbl}(#1)}
\newcommand{\tn}[1]{\textnormal{#1}}


% Comments
\newcommand{\dan}[1]{\color{red}{{\bf #1}}\color{black}}
\newcommand{\lui}[1]{\color{green}{{\bf #1}}\color{black}}
\newcommand{\enr}[1]{\color{blue}{{#1~}}\color{black}}
\newcommand{\enrcom}[1]{\todo[inline,color=blue!70,textcolor=white]{Enrico: #1}}
\newcommand{\ahmcom}[1]{\todo[inline,color=red!70,textcolor=white]{#1}}
\newcommand{\edocom}[1]{\todo[inline,color=green!70,textcolor=black]{Edo: #1}}
\newcommand{\luicom}[1]{\todo[inline,color=green!70,textcolor=black]{Luigi: #1}}
\newcommand{\mrcom}[1]{\todo[inline,color=purple,textcolor=yellow]{Marco: #1}}
%\newcommand{\Enr}[2]{\st{#1}~\enr{#2}}
\newcommand{\revv}[1]{\color{red}{{\bf #1}}\color{black}}

\newtcolorbox{textbox}[2][]{
    % enhanced jigsaw,breakable,pad at break*=1mm,
    colback=gray!5!white,colframe=gray!75!black,
    left=1.5mm, lefttitle=4mm, right=1.5mm, 
    title=#2,
    #1
}

\newtcolorbox{textboxerror}[1][]{
    % enhanced jigsaw,breakable,pad at break*=1mm,
    colback=red!5!white,colframe=red!75!black,
    #1
}

\newtcblisting[list inside=loe]{codebox}[2]{
    listing engine=minted, minted language={#1}, listing only,
    minted options={breaklines, autogobble, linenos, fontsize=\footnotesize, numbersep=3mm},
    % enhanced jigsaw,breakable,pad at break*=1mm,
    colback=gray!5!white,colframe=gray!75!black,
    left=5.5mm, lefttitle=5mm, right=1.5mm, 
    bottom=0mm, top=0mm,
    title=#2,   
}

% Cartesian product
\global\long\def\cart{\operatorname*{\diagup\hspace{-9pt}\diagdown}}

\DeclareMathOperator*{\argmax}{arg\,max}

\newcommand{\prolog}[1]{\mintinline{prolog}{#1}}

\makeatletter
\newcommand{\writings}[1]{%
    #1%
    \@ifnextchar.{}{%
        \@ifnextchar,{}{%
            \@ifnextchar'{}{%
                \@ifnextchar){}{\ }%
            }%
        }%
    }%
}
\newcommand{\llevel}[0]{\protect\writings{low-level}}
\newcommand{\hlevel}[0]{\protect\writings{high-level}}
\newcommand{\Llevel}[0]{\protect\writings{Low-level}}
\newcommand{\Hlevel}[0]{\protect\writings{High-level}}
\newcommand{\LL}[0]{\protect\writings{low-level}}
\newcommand{\HL}[0]{\protect\writings{high-level}}
\newcommand{\llm}[0]{\protect\writings{LLM}}
\newcommand{\llms}[0]{\protect\writings{LLMs}}
\newcommand{\largelm}[0]{\protect\writings{Large Language Model}}
\newcommand{\largelms}[0]{\protect\writings{Large Language Models}}
\newcommand{\kb}[0]{\protect\writings{KB}}
\newcommand{\kbs}[0]{\protect\writings{KBs}}
\newcommand{\kbase}[0]{\protect\writings{knowledge-base}}
\newcommand{\Kbase}[0]{\protect\writings{Knowledge-base}}
\newcommand{\kbases}[0]{\protect\writings{knowledge-bases}}
\newcommand{\Kbases}[0]{\protect\writings{Knowledge-bases}}
\newcommand{\ucase}[0]{\protect\writings{use-case}}
\newcommand{\Ucase}[0]{\protect\writings{Use-case}}
\newcommand{\ucases}[0]{\protect\writings{use-cases}}
\newcommand{\Ucases}[0]{\protect\writings{Use-cases}}
\newcommand{\bt}[0]{\protect\writings{BT}}
\newcommand{\bts}[0]{\protect\writings{BTs}}
\newcommand{\btree}[0]{\protect\writings{behavior tree}}
\newcommand{\btrees}[0]{\protect\writings{behavior trees}}
\newcommand{\Btree}[0]{\protect\writings{Behavior Tree}}
\newcommand{\Btrees}[0]{\protect\writings{Behavior Trees}}
\newcommand{\storm}[0]{\protect\writings{Storm}}
\newcommand{\prism}[0]{\protect\writings{PRISM}}

\newcommand{\cm}[0]{\textbf{\checkmark}}

\makeatother

