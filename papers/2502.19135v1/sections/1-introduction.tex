The introduction of advanced language models (\llms) represents a
paradigm shift for a wide range of AI applications. In
robotics, \llms could drive disruptive advancements in human-robot
interaction, adaptive behaviours, and multi-agent collaboration,
bringing the widespread integration of robots into everyday life
closer to reality in the coming years.
To achieve these ambitious goals, robot decision-making processes must
go beyond initial user input, continuously incorporating feedback
loops for real-time task adaptation~\cite{ugur2025neuro}.

Despite the remarkable efficiency of \llms in interpreting natural
language and providing semantic information about the real world, this
technology lacks a crucial feature for robotic applications: the
predictability and explainability of the decision-making process.
These qualities are at the heart of more traditional approaches to
knowledge representation and management, such as logic or functional
languages~\cite{PAULIUS201913}. For instance,
Prolog~\cite{clocksin2003programming} enables the expression of a
Knowledge Base (\kb) through facts and rules, linking them to feasible
actions. This capability has already been successfully leveraged in
robotics to construct knowledge representations~\cite{Ten13}, address
planning and reasoning challenges~\cite{prologPlanning}, and, when
combined with natural language processing, enhance human-robot
interaction~\cite{NLPProlog1,NLPProlog2}.

The use of Prolog makes the \kb inherently compositional and reusable,
allowing for the deduction of new concepts and the execution of
queries to verify consistency, infer new knowledge, or update existing
knowledge. Logical reasoning facilitates the evaluation of these
updates, ensuring coherence. For example, a new robotic implementation
of the same task can be achieved by refining the relevant predicates
and actions.  However, populating a \kb is not a straightforward task
for non-experts, as it requires mastery of a highly specific formalism
and its associated rules.
%\todo[inline]{MR: Add here something to justify the role of prolog}

In this work, we bridge the gap between these two worlds by
harnessing the power of \llms to generate a human-readable
logic \kb. This process is carried out using natural language
descriptions of the environment, the goals, and the robot’s
capabilities, which can be easily derived from technical documentation
or the verbalised experiences of workers and operators. The \kb\/ is
then utilised to generate executable plans with a high degree of
predictability and robustness.

Specifically, this manuscript introduces a novel framework for knowledge generation, management, and planning in multi-agent systems, which:  
\begin{enumerate*}[label=\roman*)]  
    \item employs a semi-automatic procedure that leverages \llms to populate a Prolog-based \kb;  
    \item enables the seamless generation of plans incorporating temporal parallelism, thus allowing multiple agents to execute concurrent actions;  
    \item automatically translates the generated plan into the widely used Behaviour Tree formalism, facilitating integration into ROS-based execution environments for robotic agents.  
\end{enumerate*}  



To evaluate the proposed framework, we conducted a series of
experiments focusing on its two main components: \kb generation and
planning capabilities. These experiments were carried out in a
multi-robot scenario involving multiple robots. We considered assembly
and construction tasks in two distinct proof-of-concept environments:
a classical block world and an arch-building setup. To further assess
the generalisability of the framework, we designed multiple examples
of increasing complexity for each environment.

The \kb generation process was examined by testing whether \llm
architectures could produce accurate and consistent \kbs based on the
given input queries.  The results demonstrate that: 1. \llms are a
promising choice for interpreting natural language inputs and
generating comprehensive \kbs, albeit still requiring some manual
corrections.  2. The framework successfully generates plans that
integrate the logical aspects captured by the \kb\/ with resource
availability and constraints.

The paper is organised as follows. In Section~\ref{sec:probdesc} we
describe our problem in detail and provide an overview of the
framework, In Section~\ref{sec:background}, we describe the main
background technologies used in thsi paper and formalise the planning
problem.  In Section~\ref{sec:kb} we describe the approach for the
creation and the population of the knowledge base.  In
Section~\ref{sec:plangen} we detail our algorithmic solution to
generate an executable plan from the \kb. In
Section~\ref{sec:experiments}, we give a full account of the
experimental validation of the framework.  Finally, in
Section~\ref{sec:relatedwork} we discuss the related work, while in
Section~\ref{sec:conclusions} we offer our conclusions and discuss
future work directions.









