% In this work we have presented \frameworkname a framework for generating Prolog \kbs using \llms, extracting feasible plans and executing them.  

In this work, we introduce a framework that leverages \llms\  
and logic programming to address a fundamental challenge in modern  
robotics: generating explainable and reliable plans from natural language  
specifications. The core idea of this paper is to use  
\llms\ to generate robot-oriented logic \kb. By  
combining \llm-driven \kb\ generation with Prolog’s symbolic reasoning  
and a final mixed-integer linear optimisation step, our approach  
produces fully executable plans that are compositional, reusable,  
and suitable for parallel execution. Furthermore, converting the final plan into a Behaviour Tree  
ensures compatibility with ROS2, enabling straightforward deployment  
on different robotic platforms.

The experimental results in block-world and arch-building scenarios  
suggest that \llms\ can reliably capture domain knowledge with only  
modest guidance and verification. Although some manual consistency  
checks remain necessary, the underlying Prolog-based structure  
guarantees plan correctness, explainability, and reusability. In  
general, this research highlights the synergy between advanced  
language models and symbolic methods, paving the way towards more  
intuitive and adaptable robotic systems.  

Future work will focus on further reducing manual overhead for \kb\  
validation. This task is probably within reach by  
exploiting the fine-tuning capabilities of the most recent \llms.  
We also aim to test the approach in a broad range of applications  
involving diverse robotic resources.  
To improve efficiency in generating the total-order plan, we are  
considering the application of state-of-the-art PDDL planners (e.g.,  
OPTIC~\cite{DBLP:conf/aips/BentonCC12} or  
FastDownward~\cite{DBLP:journals/jair/Helmert06}), which integrate  
several sophisticated heuristics to efficiently compute solution  
plans. Moreover, we will investigate the adoption of techniques  
integrating task and motion planning to develop a more precise and  
high-performance algorithm for extracting feasible plans in  
multi-agent robotics applications.  
