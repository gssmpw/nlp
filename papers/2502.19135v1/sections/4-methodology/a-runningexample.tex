We now introduce a running example, which will be used throughout this work to expose the interplay between the different components of the framework. 
This scenario is taken from the blocks-world domain~\cite{blocksworld}, which is frequently used in task planning. In particular, in this scenario we consider a table, blocks, which may either be directly on the table or stack on top of each other, and robotics arms, which move the blocks around. Each block is also associated with a position in the 2D space. 
In this particular example, we start from a situation in which we have two blocks, \verb|b1| and \verb|b2|, which are sat on the table in position (1,1) and (3,1) respectively. The goal is to move \verb|b1| in position (2,2) and then put \verb|b2| on top of it. An iconography of the example can be seen in Figure~\ref{fig:running-example}.

\begin{figure}
    \centering
    \section{Running Example}

\begin{figure}
    \centering
    \includegraphics[width=1\linewidth]{Images/running-example.jpg}
    \caption{Real-Time Object Detection Example Frame}
    \label{fig:running-example}
\end{figure}

We illustrate our EdgeMLBalancer approach on a scenario that concerns real-time traffic data analysis for object detection on Indian roads, chosen for their diverse and complex traffic conditions. 
% Indian roads, characterized by heterogeneous traffic comprising various vehicle types and lane-less traffic patterns, present a challenging environment for object-detection systems. This complexity makes them an ideal test bed to evaluate the robustness and adaptability of self-adaptive frameworks in real-world conditions, especially resource-constrained edge devices. 
Rather than relying on a single model, which may perform well under specific conditions but fails to adapt to dynamic changes in traffic density or vehicle behavior, self-adaptive systems dynamically adjust model selection based on runtime conditions. This adaptability is particularly critical in environments such as Indian roads, where traffic patterns are highly variable due to factors such as time of day, location, and road conditions. For instance, traffic density can peak during rush hours and significantly drop during off-peak times. A fixed-model approach may either overutilize system resources during low-traffic periods or fail to maintain the required accuracy during high-density periods, leaving to inefficiencies in both performance and resource usage.

In such scenarios, the choice of edge devices over centralized cloud systems becomes essential. Edge devices enable real-time inference directly on the device, local processing of data, minimizing latency and reducing reliance on stable internet connectivity-factors that are particularly important in regions with inconsistent network infrastructure, unlike cloud systems that add transmission delays and high energy and operational costs. 
% However, given the limited computational and energy resources of edge devices, robust adaptive frameworks are necessary to balance the trade-off between performance and resource constraints. 

The EdgeMLBalancer system uses the \textit{Image Capture Module} as shown in \textit{Figure}\ref{fig:architecture}, where a smartphone camera streams real-time traffic frames such as in \textit{Figure}\ref{fig:running-example}, emulating real-world scenarios. These frames are then passed to the \textit{Pre-processing Module}, which resizes and normalizes the data to meet the requirements of collection of models deployed on edge where each model \(m_i\) in \textit{M} represents different configurations, including EfficientDet Lite0, Lite1, Lite2, and SSD MobileNet V1 \cite{b36}\cite{b37}. Some of the models offer high accuracy demanding more computational power while other demanding less with moderate accuracy. The preprocessed frames are processed in the \textit{Object Detection} component, where the system processed each frame by selecting a model based on the trade-off between real-time CPU usage, and inference accuracy. Finally, the \textit{Post-Processing Module} refines detection results by filtering low-confidence predictions and overlays bounding boxes and confidence scores onto the original frames for visualization, as shown in \textit{Figure}\ref{fig:running-example}. The entire system effectively demonstrates the feasibility of self-adaptive systems in real-time applications, to balance trade-off between efficient resource utilization and accuracy, making it a sustainable and robust solution for resource-constrained edge devices. 
    \caption{A scheme showing the running example. Two blocks must be moved from their initial position to a new position in which they are also stacked.}
    \label{fig:running-example}
\end{figure}

While this is a trivial example, it highlights very well the
capability of the knowledge management system to generate complex
predicates that can be used for planning and it also shows the
cooperative abilities of the framework. Indeed, while using a single
robotic arm generates a straight-forward plan solution, coordinating
two robotics arms to do the same task reduces the completion time at the price
of a higher planning complexity.
