\section{Conclusion}
In this paper, we explore and mitigate the limitations of current knowledge editing methods in Large Language Models (LLMs) when the edited content is present in the context. 
Our investigations reveal the \textbf{Specificity Failure} issue that the attention mechanisms in these models overly focus on the edited entities, leading to a neglect of other relevant information in the context. 
Based on the preliminary study, such a Specificity Failure issue stems from the \textbf{Attention Drift} phenomenon: the edited model assigns excessive attention scores to the entities related to the edited knowledge, thereby overly concentrating on specific snippets within the context.
Thus, based on the previous knowledge editing methods, we propose the \textbf{S}elective \textbf{A}ttention \textbf{D}rift \textbf{R}estriction (\textbf{SADR}) method, which introduces a regularization term during the editing process, dynamically constraining the weight of partial selected attention heads and preventing excessive focus on the edited entities. 
Our experiment indicates that SADR can significantly reduce Specificity Failures while preserving high rates of editing success. 

\section*{Reproducibility Statement}
Our work is based on open-source models and datasets. In Section~\ref{sec:experiment} and Appendix~\ref{app:implementation_details}, we provide detailed descriptions of data processing, method implementation, and ablation experiments. Additionally, in the supplementary materials, we include the complete code for our method as well as the processed datasets. 

\section*{Acknowledgments}
We want to thank all the anonymous reviewers for their valuable comments. 
This work was supported by the National Science Foundation of China (NSFC No. 62206194), the Natural Science Foundation of Jiangsu Province, China (Grant No. BK20220488), the Young Elite Scientists Sponsorship Program by CAST (2023QNRC001), the Key Laboratory of Data Intelligence and Advanced Computing in Provincial Universities, Soochow University, and the Priority Academic Program Development of Jiangsu Higher Education Institutions.
