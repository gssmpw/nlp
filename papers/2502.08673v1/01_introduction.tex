\vspace{-0.25cm}
\section{Introduction}

High-throughput SAT samplers play a crucial role in advancing the state of the art in software and hardware verification methodologies \cite{dutra2018quicksampler}.
Generating a set of random solutions to logical constraints is critical in the verification, testing, and synthesis.
In software verification, SAT samplers enable efficient exploration of diverse execution paths, addressing the scalability challenges inherent in symbolic execution~ \cite{Clarke1976testing, King1976symbolic, Avgerinos2014dynamic, Anand2011test, Anand2007test, Artzi2008test, Burnim2008test, Cadar2008test, Chipounov2012test, Godefroid2005test, Jayaraman2009jFuzzAC, yoshida2017test, corina2008test, saxena2010test, dawn2008test, Tillmann2008test}. In hardware verification, they support the generation of varied input patterns, ensuring a rigorous and effective verification process~\cite{Kitchen2007crv, Zhao2009crv, Naveh2013crv, Naveh2006crv}. 

%% In software, traditional symbolic execution \cite{Clarke1976testing, King1976symbolic} and dynamic symbolic execution methods \cite{Avgerinos2014dynamic, Anand2011test, Anand2007test, Artzi2008test, Burnim2008test, Cadar2008test, Chipounov2012test, Godefroid2005test, Jayaraman2009jFuzzAC, yoshida2017test, corina2008test, saxena2010test, dawn2008test, Tillmann2008test} generate a path constraint for each feasible execution path within a program. A solver is then used to compute a solution for each constraint. However, these methods often face scalability challenges due to the exponential growth in the number of potential paths as program complexity increases. To address this issue, an alternative approach called constraint-based fuzzing has been developed \cite{Marcel2016fuzzing, Fraser2011fuzzing, Holler2012fuzzing, householder_2012, Pacheco2007fuzzing, Yang2011fuzzing}. This approach generates multiple solutions to randomly test paths that share the same prefix, significantly accelerating symbolic execution while incorporating the advantages of random testing.

%% Similarly, in hardware verification, constrained-random verification (CRV) has been proposed and implemented as a strategy to generate high-quality inputs for hardware designs \cite{Kitchen2007crv, Zhao2009crv, Naveh2013crv, Naveh2006crv}. In CRV, verification engineers define preconditions and constraints based on specific domain knowledge. A constraint sampler is then used to generate multiple random inputs that satisfy these constraints. This method ensures comprehensive coverage by sampling diverse inputs that meet the specified criteria, thereby enhancing the reliability and robustness of hardware designs.

%% The integration of Boolean satisfiability (SAT) samplers in both software and hardware verification highlights their utility in generating multiple solutions that facilitate thorough testing. In software verification, SAT samplers enable efficient exploration of diverse execution paths, addressing the scalability challenges inherent in symbolic execution. In hardware verification, they support the generation of varied input patterns, ensuring a rigorous and effective verification process. Thus, high-throughput SAT samplers play a crucial role in advancing the state of the art in both software and hardware verification methodologies \cite{dutra2018quicksampler}.


%% The SAT sampling process begins by formulating the logical constraints of the target application into a special form of Boolean logic known as conjunctive normal form (CNF) \cite{Biere2009SAT}. Converting logical relationships into CNF is a critical step for effectively using SAT samplers. CNF is the specific format required by most SAT samplers, where the logical formula is expressed as a conjunction of clauses, with each clause being a disjunction of literals. This conversion ensures that the resulting CNF accurately represents the original behavior of the target application.

The SAT sampling process begins by formulating the logical constraints of the target application into conjunctive normal form (CNF) \cite{Biere2009SAT}.
CNF is the specific format required by most SAT samplers, where the logical formula is expressed as a conjunction of clauses, with each clause being a disjunction of literals.
%Therefore, converting logical relationships into CNF is a critical step for effectively using SAT samplers. 
%This conversion ensures that the resulting CNF accurately represents the original behavior of the target application.
The complexity of the logical constraints in the target application can result in a CNF that is not always concise. Nevertheless, CNF remains the preferred format due to the strong performance of SAT samplers and solvers. The complexity of the CNF can, however, affect the efficiency of these solvers.

SAT solvers employ various strategies to find a satisfying assignment for the variables in the CNF. Modern SAT solvers \cite{Niklas2003SAT, Moskewicz2001Chaff, Audemard2018Glucose} often use the conflict-driven clause learning (CDCL) algorithm \cite{Silva1996CDCL, silva2021CDCL}, that relies heavily on heuristics such as conflict-driven backtracking and clause learning. These heuristics effectively guide the CDCL algorithm in finding a satisfying assignment. Due to the sequential nature of these heuristics, that involve branching and backtracking, the latest SAT solvers are typically executed on CPUs. Consequently, state-of-the-art SAT samplers, which incorporate SAT solvers within their algorithms, also rely on a sequential process and are optimized for CPU execution.

%GPU acceleration has shown significant performance improvements over CPUs in a wide range of applications, particularly in machine learning \cite{Krizhevsky2012AlexNet}.
%GPU computing is well-suited to problems that involve large amounts of data parallelism.
Generating multiple satisfying solutions to the SAT problem is a good match to GPU computing, provided that a sampling method is available that performs consistent, data-parallel computations.
To address this opportunity, we propose a transformation algorithm that converts logical constraints encoded as a CNF into a simplified multi-level, multi-output Boolean function while maintaining the original logical constraints. This transformation significantly reduces the number of bit-wise operations and thus lowers the complexity of the sampling process.
We then formulate the resulting simplified multi-level, multi-output Boolean function as a multi-output regression task, where each logic gate is represented probabilistically, enabling the use of gradient descent (GD) to learn diverse solutions. This approach enables the parallel generation of independent SAT problem solutions, allowing for effective GPU acceleration.
We demonstrate the superior performance of our sampling method across $60$ instances from a publicly available benchmark suite \cite{meel_2020_benchmark} used in previous studies. The code of our sampler is available at \url{https://github.com/arashardakani/High-Throughput-SAT-Sampler}.















