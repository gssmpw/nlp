\section{Acoustic variation within-between wards}
\label{sec:diff_loc}

Building on the findings from \Cref{sec:diff_mic}, acoustic differences between and within wards were further analysed using only the binaural channels. The analysis focused on three bed positions: HD-A, NICU-A, and NICU-B, omitting HD-B due to the limited duration of data collection, as noted in \Cref{tab:bedmovement}. For each A- and C-weighted metric, LME-ART-ANOVA was performed with bed position as a fixed effect, and microphones and time as random effects. Post-hoc contrast tests were conducted for results with significance at the \SI{1}{\percent} level.

\subsection{A- and C-weighted metrics}

The LME-ART-ANOVA revealed statistically significant differences between bed positions for all A- and C-weighted metrics at the \SI{1}{\percent} significance level. Large effect sizes were consistently observed across most metrics, except for \Lx{ASmax} and \Lx{CSmax}, which exhibited small and medium effect sizes, respectively, as summarised in \Cref{tab:artloc}. In the post-hoc contrast tests for A-weighted metrics, significant differences were found across all bed position pairs with large effect sizes, except for HD-A vs. NICU-B in \Lx{AS10} and HD-A vs. NICU-A in \Lx{ASmax}. Similarly, post-hoc tests for C-weighted metrics indicated significant differences with large effect sizes for all bed position pairs, except for HD-A vs. NICU-B in both \Lx{CS} and \Lx{CS50}.

In general, NICU-A exhibited significantly higher A- and C-weighted metrics compared to both HD-A and NICU-B, as shown in \Cref{fig:byhourmetrics}, Between HD-A and NICU-B, \Lx{AS}, \Lx{AS50}, and \Lx{CS50} levels were higher at NICU-B, while \Lx{ASmax}, \Lx{CS10}, and \Lx{CSmax} were higher at HD-A.

\begin{figure*}[h]
    \centering
    \includegraphics[width=\textwidth]{Figures/all_metrics_24h.pdf}
    \caption{A- and C-weighted decibel metrics averaged by hour of the day across the entire measurement duration at NICU-A, NICU-B and HD-A measurement points.}
    \label{fig:byhourmetrics}
\end{figure*}

\begin{figure}[h]
    \centering
    \includegraphics[width=\linewidth]{Figures/or_dBA_tuHMS.pdf}
    \caption{Occurrence rate of $\textit{OR}^h_\text{SNR}(5)$ and $\textit{OR}_{T}^h(0.4)$ averaged over the same daily 1-h period throughout the entire measurement campaign. A-weighted decibel metrics and tonality metrics averaged by hour of the day across the entire measurement duration at NICU-A, NICU-B and HD-A measurement points.}
    \label{fig:ORbyhour}
\end{figure}


\subsection{Acoustic guidelines}

Measurements of \Lx{AS10} and \Lx{AS50} across HD-A, NICU-A, and NICU-B were evaluated primarily using the 9\textsuperscript{th} and 10\textsuperscript{th} editions of the CC NICU design recommendations, as outlined in \Cref{tab:sound-standards}. For comparison, additional assessments were conducted based on \Lx{AS} and \Lx{ASmax} using the 8\textsuperscript{th} edition guidelines.

At HD-A, compliance with the \Lx{AS10} guideline was achieved for up to \SI{97}{\percent} of the $N=981$ 1-hour measurement periods, while the \Lx{AS50} limits were up to \SI{40}{\percent} compliant, as shown in \Cref{tab:reg_subloc}. Compliance rates for \Lx{AS10} and \Lx{AS50} varied depending on the microphone type, with binaural measurements showing up to \SI{8}{\percent} higher compliance for \Lx{AS10} and up to \SI{9}{\percent} higher compliance for \Lx{AS50}.

At NICU-A, compliance with the \Lx{AS10} guideline was achieved for up to \SI{99}{\percent} of the $N=492$ 1-hour measurement periods, while the \Lx{AS50} limits were consistently exceeded. Notably, microphone type influenced compliance, with binaural measurements resulting in up to \SI{8}{\percent} lower compliance for \Lx{AS10}.

At NICU-B, a similar trend to NICU-A was observed regarding adherence to \Lx{AS10} and \Lx{AS50} guidelines across the $N=98$ 1-hour measurement periods, although compliance rates remained consistent across different microphone types.

The \Lx{ASmax} and \Lx{AS} guidelines from the 8\textsuperscript{th} edition were exceeded almost consistently at across HD-A, NICU-A and NICU-B, suggesting the presence of very loud events near neonates (\Lx{ASmax} $\ge \SI{65}{\decibelA}$), though these events were infrequent, occurring less than \SI{10}{\percent} of the time. 

\chadded[]{
While \Lx{AS} exceedance patterns were consistent with previous studies irrespective of microphone placement \cite{andy_systematic_2025}, \Lx{AS} levels recorded near neonates' ears in this study aligned closely with those reported in similar NICU environments \cite{Krueger2007,Smith2018}. However, \Lx{AS} levels measured within enclosed incubators in prior studies were generally up to \dba{6} higher \cite{Romeu2016,Parra2017}, as summarized in \Cref{tab:niculevels}.}

\chadded[comment=R4.3]{Compared to \cite{Krueger2007,Parra2017}, \Lx{ASmax} levels in this study were up to \dba{23} lower, while \Lx{AS10} levels were comparable. The discrepancy in \Lx{ASmax} levels may reasonably be attributed to the absence of bedside medical devices during our measurements, though the specific cause cannot be definitively determined.}

\setlength{\LTpost}{0mm}
\begin{table*}[ht]
\scriptsize
\centering
\caption{Summary of mean A-weighted metrics and percentage of time where the metrics were within \textit{CC} guidelines over $N=981$, $N=392$, and $N=98$ 1-\si{\hour} periods at HD-A, NICU-A and NICU-B bed positions, respectively.}
\label{tab:reg_subloc}
\begin{tabularx}{\linewidth}{%
>{\raggedright\arraybackslash}p{0.04\linewidth}%
*{11}{>{\RaggedLeft}X}
}
\toprule
& \multicolumn{3}{c}{\textbf{HD-A}} 
& \multicolumn{4}{c}{\textbf{NICU-A}} 
& \multicolumn{4}{c}{\textbf{NICU-B}} \\ 

\cmidrule(lr){2-4} \cmidrule(lr){5-8} \cmidrule(lr){9-12}
& \GRASOut\textsuperscript{\textit{1}} 
& \binL\textsuperscript{\textit{1}} 
& \binR\textsuperscript{\textit{1}} 
& \GRASIn\textsuperscript{\textit{2}} 
& \GRASOut\textsuperscript{\textit{2}} 
& \binL\textsuperscript{\textit{2}} 
& \binR\textsuperscript{\textit{2}} 
& \GRASIn\textsuperscript{\textit{3}} 
& \GRASOut\textsuperscript{\textit{3}} 
& \binL\textsuperscript{\textit{3}} 
& \binR\textsuperscript{\textit{3}} \\

\midrule\addlinespace[2.5pt]
\Lx{AS} & 57.18 (3.64) & 56.05 (3.87) & 56.09 (3.48) & 56.45 (1.93) & 56.78 (2.09) & 59.33 (1.90) & 59.38 (1.90) & 55.05 (1.88) & 55.87 (1.78) & 57.47 (2.01) & 58.29 (1.81) \\ 
\Lx{AS10} & 60.25 (4.09) & 58.74 (3.85) & 58.98 (3.87) & 58.54 (2.73) & 58.96 (2.86) & 61.33 (2.65) & 61.40 (2.70) & 56.73 (2.54) & 57.55 (2.33) & 59.15 (2.74) & 59.89 (2.27) \\ 
\Lx{AS50} & 52.48 (3.71) & 51.35 (3.30) & 51.59 (3.36) & 54.53 (1.43) & 54.79 (1.55) & 57.61 (1.43) & 57.63 (1.38) & 53.47 (1.10) & 54.60 (1.27) & 55.97 (1.44) & 56.80 (1.15) \\ 
\Lx{ASmax} & 74.85 (4.14) & 73.97 (5.28) & 73.76 (4.20) & 70.96 (4.00) & 71.55 (3.99) & 73.94 (4.21) & 73.40 (4.12) & 68.44 (4.85) & 68.78 (4.92) & 70.71 (5.05) & 71.47 (4.81) \\ 
\midrule
\multicolumn{12}{l}{\Lx{AS10} $\le$ \dba{65} (\textit{CC 9\textsuperscript{th}/10\textsuperscript{th} ed})}\\ 
    Yes & 877 (89\%) & 953 (97\%) & 942 (96\%) & 387 (99\%) & 386 (98\%) & 360 (92\%) & 353 (90\%) & 98 (100\%) & 98 (100\%) & 97 (99\%) & 97 (99\%) \\ 
    No & 104 (11\%) & 28 (2.9\%) & 39 (4.0\%) & 5 (1.3\%) & 6 (1.5\%) & 32 (8.2\%) & 39 (9.9\%) & 0 (0\%) & 0 (0\%) & 1 (1.0\%) & 1 (1.0\%) \\ 
\midrule
\multicolumn{12}{l}{\Lx{AS50} $\le$ \dba{50} (\textit{CC 9\textsuperscript{th}/10\textsuperscript{th} ed})}\\
    Yes & 309 (31\%) & 389 (40\%) & 368 (38\%) & 0 (0\%) & 0 (0\%) & 0 (0\%) & 0 (0\%) & 0 (0\%) & 0 (0\%) & 0 (0\%) & 0 (0\%) \\ 
    No & 672 (69\%) & 592 (60\%) & 613 (62\%) & 392 (100\%) & 392 (100\%) & 392 (100\%) & 392 (100\%) & 98 (100\%) & 98 (100\%) & 98 (100\%) & 98 (100\%) \\ 
\midrule
\multicolumn{12}{l}{\Lx{ASmax} $\le$ \dba{65} (\textit{CC 8\textsuperscript{th} ed})}\\ 
    Yes & 8 (0.8\%) & 21 (2.1\%) & 19 (1.9\%) & 24 (6.1\%) & 14 (3.6\%) & 6 (1.5\%) & 4 (1.0\%) & 24 (24\%) & 22 (22\%) & 12 (12\%) & 10 (10\%) \\ 
    No & 973 (99\%) & 960 (98\%) & 962 (98\%) & 368 (94\%) & 378 (96\%) & 386 (98\%) & 388 (99\%) & 74 (76\%) & 76 (78\%) & 86 (88\%) & 88 (90\%) \\ 
\midrule
\multicolumn{12}{l}{\Lx{AS} $\le$ \dba{45} (\textit{CC 8\textsuperscript{th} ed})}\\
    Yes & 0 (0\%) & 0 (0\%) & 0 (0\%) & 0 (0\%) & 0 (0\%) & 0 (0\%) & 0 (0\%) & 0 (0\%) & 0 (0\%) & 0 (0\%) & 0 (0\%) \\ 
    No & 981 (100\%) & 981 (100\%) & 981 (100\%) & 392 (100\%) & 392 (100\%) & 392 (100\%) & 392 (100\%) & 98 (100\%) & 98 (100\%) & 98 (100\%) & 98 (100\%) \\ 
\bottomrule
\end{tabularx}
\begin{minipage}{\linewidth}
\textsuperscript{\textit{1}}$N=981$; \textsuperscript{\textit{2}}$N=392$; 
\textsuperscript{\textit{3}}$N=98$;
Mean (SD); n (\%)\\
\end{minipage}
\end{table*}


\subsection{Occurrence rates}

Significant differences in the frequency of loud events ($\textit{OR}_\text{SNR}^h(5)$) were found across HD-A, NICU-A, and NICU-B. Post-hoc contrasts revealed that loud events occurred significantly more frequently at HD-A than at both NICU-A and NICU-B, as depicted in \Cref{fig:ORbyhour}. Although slightly higher at NICU-A than NICU-B, the difference in $\textit{OR}_\text{SNR}^h(5)$ was still statistically significant.

The hourly variation in loud event occurrences remained relatively stable throughout the measurement period, as indicated by the standard error in \Cref{fig:ORbyhour}. At HD-A, the average occurrence remained consistent throughout the day, whereas at NICU-A and NICU-B, loud events were more frequent between 7 a.m. and 9 p.m.

Significant differences were also noted in the occurrence of tonal events ($\textit{OR}_\text{T}^h(0.4)$) across HD-A, NICU-A, and NICU-B. Post-hoc tests showed that the $\textit{OR}_\text{T}^h(0.4)$ values for NICU-A and NICU-B were statistically similar, but both were significantly higher than those for HD-A.

At NICU-A and NICU-B, tonal or modulated signals were present almost continuously ($T>0.4$), while at HD-A, such signals occurred consistently only between 7 a.m. and 9 p.m.