\section{Method}
\label{sec:Method}


\subsection{Site characteristics and administration}

\iftodo
\begin{itemize}[leftmargin=*]
    \checkedbox{Measurement site}
    \begin{itemize}
        \checkedbox{Floor plan of NICU and HD}
        \checkedbox{Capacity and features e.g. open area, no of beds, isolation wards etc} See: \citep{Mayhew2022}
        \checkedbox{Current noise mitigation measures: SoundSign}
    \end{itemize}
    \checkedbox{NICU \& HD ward occupancy}
    \begin{itemize}
        \checkedbox{Bed occupancy rate (preferably daily/weekly)}
        \checkedbox{Staff availability rate (preferably daily/weekly)}
    \end{itemize}
    \checkedbox{NICU \& HD ward schedule}
    \begin{itemize}
        \checkedbox{Visitation schedule}
        \checkedbox{Feeding schedule}
        \checkedbox{Staff shift schedule}
    \end{itemize}
\end{itemize}
\fi

Singapore General Hospital serves as the largest tertiary hospital in Singapore and houses the Department of Neonatal and Development Medicine, offering a comprehensive spectrum of services for newborns, ranging from standard to intensive care. Within this department, there is a dedicated newborn nursery, a neonatal high dependency (HD) unit, and a neonatal intensive care unit (NICU).

This study specifically focused on the neonatal HD and NICU units. The HD unit comprises two rooms housing a total of 18 cots, where neonates classified as level 2 under the Provincial Council for Maternal and Child Health Standardized Levels of Care infant acuity levels receive specialized care. Simultaneously, the NICU encompasses 10 incubators or warmers, tending to neonates categorized as level 3.

The units were equipped with a comprehensive array of medical resources, including a central monitoring station, refrigerators (1 for medication and another 2 for breast milk (fresh and frozen)), computers on wheels, ventilators, warmers, infusion pumps, oxygen-air proportioners, phototherapy lights, ultrasound machine, EEG, cooling machines, and nitric oxide delivery systems. These resources collectively facilitate extensive care services for both term and preterm newborns, with each NICU bed equipped with a bedside monitoring device. 

\chadded[comment=R3.2]{To prioritize patient care, the least utilized bedspaces were chosen as measurement locations. In both the NICU and HD wards, a primary bedspace was designated for measurements and labeled with the suffix ``A,'' as depicted in \Cref{fig:wardlayout}. Secondary bedspaces were labeled sequentially, e.g., NICU-B. Potential noise sources, such as wash basins, telephones, nurse stations, storage cabinets, and pneumatic tubes, were identified and marked in \Cref{fig:wardlayout}. Due to the critical condition of NICU patients, medical equipment sounds—such as ventilators and alarms—were more prevalent in the NICU compared to the HD wards. It is important to note that measurements captured only noises external to the bedspace, i.e. no medical device sounds were simulated.}

Maintaining a nurse-to-patient ratio of 1:1.5 for NICU neonates and 1:4 for HD neonates, the unit sustains a robust staffing structure with 10 nurses during day and evening shifts and 7 nurses during the night shift. Additionally, the space accommodates various healthcare professionals not exclusively stationed in the unit but utilizing the facility to assess and administer treatments to the infants.

The average patient occupancy rate for both the NICU and HD wards was approximately \SI{60}{\percent}. Notably, there were no restrictions imposed on the visitation schedule, even during the COVID-19 pandemic. However, it is important to highlight that only the parents of the patients were permitted to visit the wards. Additionally, the feeding regimen followed a schedule with intervals of three hours, each session lasting approximately one hour.

In a previous effort to reduce operational noise in the NICU, two SPL-activated warning signs named ``SoundSign'' (Cirrus Research plc, North Yorkshire, UK) was installed at high visibility areas on the wall in the NICU and HD wards, as shown in \Cref{fig:wardlayout}. The SoundSign would light up continously for \SI{30}{s} upon triggering at \Lx{AS}$=\dba{65}$, which corresponds to the \Lx{ASmax} limit in the previous editions of the \textit{Recommended Design Standards for Advanced Neonatal Care}. Currently, there is no implementation of scheduled daily "Quiet Time" \chadded[comment=3.2]{or structured programs to reduce environmental noise} in the NICU or HD wards.

\begin{figure*}[h]
    \centering
    \includegraphics[width=\textwidth]{Figures/wardlayout.pdf}
    \caption{Floor plan depicting the layout of the neonatal intensive care unit (NICU) and high dependency (HD) ward. Measurement positions within the NICU are highlighted in violet, while the measurement point in the HD ward is marked in green. In both the NICU and HD wards, a SoundSign device was strategically positioned high on the wall.}
    \label{fig:wardlayout}
\end{figure*}

\subsection{Acoustic measurement: equipment and procedure}

\iftodo
\begin{itemize}[leftmargin=*]
    \checkbox{Equipment}
    \begin{itemize}
        \highlightlist{\checkedbox{NICU incubator \& HD basinet}}
        \checkedbox{Acoustic}
    \end{itemize}
    \checkbox{Procedure}
    \begin{itemize}
        \checkedbox{Measurement period}
        \checkbox{Movement of measurement points during measurement period}
        \checkbox{Measurement positions}
        \checkedbox{Calibration}
        \checkedbox{Figure of setup}
    \end{itemize}
\end{itemize}
\fi

Measurements were carried out over a period of 20 consecutive days in March 2022, using a single incubator within the NICU ward. Similarly, within the HD ward, measurements were conducted using a single basinet for 41 consecutive days between March and April 2022.

\begin{table}[ht]
    \scriptsize
    \centering
    \caption{Bed movement schedule during the measurement period}
    \label{tab:bedmovement}
    \begin{tabularx}{\linewidth}{
    lllll%
    }
    \toprule
         Ward
         & Date
         & Time 
         & Location
         & Remarks\\
    \midrule
         NICU
         & 2022-03-03
         & 15:15 
         & NICU-A
         & Designated bedspace\\

         NICU
         & 2022-03-17
         & 11:20
         & NICU-B
         & NICU-A $\rightarrow$ NICU-B\\

         NICU
         & 2022-03-21 
         & 12:45
         & NICU-A
         & NICU-B $\rightarrow$ NICU-A\\

    \midrule
        HD
         & 2022-03-03 
         & 15:15
         & HD-A
         & Designated bedspace\\

        HD
         & 2022-04-03 
         & 09:00
         & HD-B
         & HD-A $\rightarrow$ HD-B\\

        HD
         & 2022-04-03 
         & 10:00
         & HD-A
         & HD-B $\rightarrow$ HD-A\\
    \bottomrule
    
    \end{tabularx}
\end{table}

To simulate how sound is perceived at the ears of neonates without capturing self-noise, calibrated binaural microphones (TYPE~4101\nobreakdash-B, Hottinger Brüel \& Kjær A/S, Virum, Denmark) were affixed on the ears of two neonate dolls with surgical tape. One doll was placed in an open-box incubator (CosyCot\textsuperscript{\textsc{tm}}, Fisher \& Paykel Healthcare Limited, Auckland, New Zealand), and the other was placed in a bassinet (Huntleigh Healthcare Limited, Wales, United Kingdom), as shown in \Cref{fig:setup}\subref{fig:setup-incubator} and \subref{fig:setup-basinet}. The left and right microphone channels of the binaural microphone would be referenced as $m_\text{L}$ and $m_\text{R}$, respectively. \chadded[comment=3.2]{Both the incubator and bassinet were open-type configurations, representative of those commonly used in the NICU and HD wards under investigation.} 

In addition, both the incubator and bassinet were fitted with a IEC~61672\nobreakdash-1 Class~1 compliant sound pressure acquisition system: IEC~61094\nobreakdash-4 WS2F microphone (146AE, GRAS Sound \& Vibration A/S, Holte, Denmark) connected to a data acquisition system (SQoBold, HEAD acoustics GmbH, Herzogenrath, Germany). To capture the sound levels around the incubator, two 146AE microphones were attached to the incubator, one secured to the IV pole \SI{1.6}{\meter} from the ground ($m_\text{out}$) and the other \SI{30}{\centi\meter} from the incubator bed, above the raised walls ($m_\text{in}$). Due to resource constraints, only one 146AE microphone was secured to the bassinet behind the head area \SI{1.6}{\meter} from the ground ($m_\text{out}$).

Before and after the measurements, each acoustic measurement device was examined to be within $\pm\SI{0.1}{\decibel}$ of \SI{94}{\decibel} and \SI{114}{\decibel} at both \SI{1}{\kilo\hertz} and \SI{250}{\hertz} using a calibrator (42AG, GRAS Sound \& Vibration A/S, Holte, Denmark). 

\begin{figure}[ht]
    \centering
    \subfigure[]{
    \includegraphics[width=0.45\linewidth]{Figures/incubator.jpeg}
    \label{fig:setup-incubator}}
    \subfigure[]{
    \includegraphics[width=0.45\linewidth]{Figures/openincubator.pdf}
    \label{fig:setup-incubator-mic}}
    \subfigure[]{
    \includegraphics[width=0.45\linewidth]{Figures/basinet.jpeg}
    \label{fig:setup-basinet}}
    \subfigure[]{
    \includegraphics[width=0.45\linewidth]{Figures/hdbassinet.pdf}
    \label{fig:setup-basinet-mic}}
    \caption{(a) Photo and (b) diagram of the in-situ measurement setup of the incubator in the NICU, and (c) a photo and (d) diagram of bassinet setup in the high-dependency ward.}
    \label{fig:setup}
\end{figure}

\subsection{(Psycho)acoustic metrics}

To evaluate the influence of microphone and bed positions on the acoustics, slow time weighted, A- and C-weighted sound pressure metrics along with their summary statistics were computed. Equivalent sound pressure levels (\Lh{AS}{1}, \Lh{CS}{1}), along with summary statistics of maximum levels (\Lh{ASmax}{1}, \Lh{CSmax}{1}), \SI{10}{\percent} exceedance levels (\Lh{AS10}{1}, \Lh{CS10}{1}), and \SI{50}{\percent} exceedance levels (\Lh{AS50}{1}, \Lh{CS50}{1}) were the primary focus. The inclusion of C-weighted metrics examines the presence of low-frequency noise, which could be heard by neonates as low as \SI{250}{\hertz} \citep{novitski_neonatal_2007}. The summary statistics, allows for a nuanced understanding of temporal variations influenced by specific microphone positions during the measurement period. Subscript reference to the 1-\si{\hour} averaging is henceforth dropped for brevity. 

The occurrence rate, \textit{OR(N)} \citep{Bliefnick2019,Ryherd2008}, which describes the percentage of time a threshold level \textit{N} was exceeded, was adopted to evaluate the possibility that the neonate would be awakened by loud sounds. A conservative threshold of \SI{5}{\decibelA} signal-to-noise ratio (SNR) above the background noise ($\textit{SNR}<5$) was employed to mitigate the chance of awakening to less than \SI{30}{\percent} \citep{kuhn_evaluating_2011}. The $\textit{OR}_\text{SNR}^h(5)$ was computed for each one hour period in a day, $h\in\{0,1,\cdots,23\}$, where the SNR was determined as \citep{Smith2018}
\begin{equation}
    \textit{SNR} = \textit{L}_\text{ASmax,1min}^h - \textit{L}_\text{AS50}^h.
\end{equation} 
Hence, $\textit{OR}_\text{SNR}^h(5)=75$ indicates that the SNR exceeded \SI{5}{\decibelA} at least once in 45 out of 60 one minute intervals in the $h$-th hour.  

Designed to quantify tonal and modulated sounds, the tonality metric, $T$, based on Sottek’s hearing model as described in the ECMA-418-2 \citep{International2020}, could indicate the prominence of alarm sounds and potentially speech. Therefore, the prominence of tonal sounds were determined by the occurrence rate where the tonality $T$ exceeded 0.4 tuHMS \citep{International2020}, within one minute periods of the $h$-th hour, i.e. $\textit{OR}^h_T(0.4)$.

\subsection{Data analysis}

The acoustic and psychoacoustic indices were computed with a commercial software package (ArtemiS \textsc{suite}, HEAD acoustics GmbH, Herzogenrath, Germany). Decibel-based metrics and their associated statistics were computed in accordance with ISO 1996-1 guidelines \citep{ISO1996-1}. Tonality was assessed using the hearing-model variant specified in ECMA-418-2 \citet{International2020}.

Due to non-normal residual distributions as indicated by the Anderson-Darling test, differences in each decibel metric across all microphones were evaluated separately for each location using a linear mixed-effects aligned ranks transformation ANOVA (LME-ART-ANOVA). Microphone type (\HDGRAS, \HDbinL, \HDbinR, \NICUGRASIn, \NICUGRASOut, \NICUbinL, \NICUbinR) was treated as a fixed effect, while the \num{1}-\si{\hour} time intervals served as a random intercept. Similarly, within- and between-ward differences for each A- and C-weighted metric were analyzed with LME-ART-ANOVA, with bed position as the fixed effect and both the \num{1}-\si{\hour} time intervals and binaural microphone positions as random effects. Differences in occurrence rates for SNR and tonality exceedances were also assessed using LME-ART-ANOVA, with bed position as the fixed effect and the \num{1}-\si{\hour} time intervals as a random effect.

All data analyses were conducted with the R programming language (R version 4.4.1) \citep{RCoreTeam2023} on a 64-bit ARM environment. The analyses were performed with these specific R packages: KS test and BH correction, with \texttt{stats} (Version 4.4.1, \citet{RCoreTeam2023}), LME-ART-ANOVA with \texttt{ARTool} (Version 0.11.1, \citet{Kay2021}), partial omega squared effect size with \texttt{effectsize} (Version 0.8.3, \citet{Ben-Shachar2020}), and acoustic data analyses with \texttt{timetk} (Version 2.9.0, \citet{dancho_timetk_2023}) and \texttt{seewave} (Version 2.2.3, \citet{sueur_seewave_2008}).


