\section{Acoustic variation between microphone positions}
\label{sec:diff_mic}

This section analyses the influence of microphone positions on the assessment of noise exposure in neonatal critical care, directly addressing the first research question (RQ1). A detailed examination of disparities in A- and C-weighted decibel metrics was conducted, following the prescribed standards in \Cref{tab:sound-standards}. The computations, employing slow time-weighting and aggregation over 1-hour intervals, spanned a continuous monitoring period of 329.5 hours, ranging from 03/03/2022 at 17:00:00 to 17/03/2023 at 10:30:00 concurrently at both NICU-A and HD-A. Crucially, this time frame remained free from disruptions related to bed changes or data collection.

Differences were evaluated for each metric with the microphone positions as the repeated measures factor in the LME-ART-ANOVA and subsequent post-hoc contrast tests, as summarised in \Cref{tab:artmicHD} and \Cref{tab:artmicNICU} for HD and NICU wards, respectively. The differences in metrics between microphone positions were further examined by computing the mean differences by
\begin{equation}
    \mu^{k,x-y}_{z}=\frac{1}{N}\sum^{N}_{n=1}(L_{z,m_x^k}(n) - L_{z,m_y^k}(n)),   
\end{equation}
where $L_{z,m_x^k}(n)$ and $L_{z,m_y^k}(n)$ refer to the decibel level indices at microphone $m_x^k$ and $m_y^k$, respectively, at the $n^\text{th}$ 1-\si{\hour} period. Here, $L_z \in \{L_\text{AS}, L_\text{CS}, L_\text{ASmax}, L_\text{CSmax}, L_\text{AS10}, L_\text{CS10}, L_\text{AS50}, \\L_\text{CS50}\}$, $\{x,y\}\in\{\text{L},\text{R},\text{out},\text{in}\}$ for $k=\text{NICU}$, $\{x,y\}\in\{\text{L},\text{R},\text{out}\}$ for $k=\text{HD}$, and $N$ is the total number of 1-\si{\hour} periods. For reference, $\mu^{{\text{HD,L}-\text{out}}}_{\text{AS50}}$ refers to the mean of the differences in \Lx{AS50} between the left binaural microphone \HDbinL\ and the standard microphone \HDGRAS\ in the HD ward.

\begin{table*}[!ht]
\scriptsize
\caption{Mean and standard deviation of the differences in 1-\si{\hour} slow time-weighted metrics at the HD-A and NICU-A bed positions measured from 03/03/2022 17:00:00 to 17/03/2022 10:30:00. Difference pairs with significant ART contrasts are indicated in \textbf{bold}.}\label{tab:mean_sd_1hr}
\begin{tabularx}{1\textwidth}{
>{\RaggedRight}p{2em}
*{9}{>{\RaggedLeft}X}
}
\toprule
 & \multicolumn{3}{c}{HD-A} & \multicolumn{6}{c}{NICU-A} \\ 
\cmidrule(lr){2-4} \cmidrule(lr){5-10}

& $m_\text{L}-m$
& $m_\text{L}-m_\text{R}$ 
& $m_\text{R}-m$ 
& $m_\text{in}-m_\text{out}$
& $m_\text{L}-m_\text{in}$
& $m_\text{L}-m_\text{out}$ 
& $m_\text{L}-m_\text{R}$
& $m_\text{R}-m_\text{in}$
& $m_\text{R}-m_\text{out}$ \\

\midrule\addlinespace[2.5pt]
\Lx{AS10} 
& \textbf{-1.62 (0.97)} 
& \textbf{-0.17 (0.50)} 
& \textbf{-1.45 (0.87)} 
& \textbf{-0.47 (0.46)} 
& \textbf{2.90 (0.56) }
& \textbf{2.44 (0.54)} 
& -0.00 (0.65) 
& \textbf{2.91 (0.46)} 
& \textbf{2.44 (0.61)} \\ 

\Lx{AS50} 
& \textbf{-1.24 (0.71)} 
& \textbf{-0.20 (0.24)}
& \textbf{-1.04 (0.66)} 
& \textbf{-0.36 (0.42)} 
& \textbf{3.29 (0.58)}
& \textbf{2.93 (0.57)}
& \textbf{0.11 (0.70)}
& \textbf{3.17 (0.43)}
& \textbf{2.82 (0.56)} \\ 

\Lx{AS} 
& \textbf{-1.34 (1.18)} 
& -0.07 (0.82) 
& \textbf{-1.27 (0.79)} 
& \textbf{-0.38 (0.40)} 
& \textbf{3.02 (0.54)} 
& \textbf{2.64 (0.50)}
& 0.04 (0.55)
& \textbf{2.98 (0.38)}
& \textbf{2.60 (0.51)} \\ 

\Lx{ASmax} 
& \textbf{-1.27 (2.77)} 
& 0.04 (2.13) 
& \textbf{-1.31 (1.83)}
& \textbf{-0.59 (1.20)} 
& \textbf{3.04 (1.62)} 
& \textbf{2.44 (1.61)}
& \textbf{0.54 (1.47)} 
& \textbf{2.50 (1.39)} 
& \textbf{1.90 (1.62)} \\ 

\Lx{CS10} 
& \textbf{-0.52 (0.44)} 
& \textbf{-0.92 (0.16)}
& \textbf{0.40 (0.54)} 
& \textbf{-0.33 (0.11)} 
& \textbf{1.28 (0.35)}
& \textbf{0.95 (0.38)}
& \textbf{0.41 (0.24)}
& \textbf{0.87 (0.46)}
& \textbf{0.54 (0.50)} \\

\Lx{CS50} 
& \textbf{-0.16 (0.16)} 
& \textbf{-1.08 (0.06)} 
& \textbf{0.92 (0.19)} 
& \textbf{-0.36 (0.06)} 
& \textbf{0.94 (0.13)} 
& \textbf{0.57 (0.14)} 
& \textbf{0.54 (0.15)}
& \textbf{0.40 (0.17)} 
& 0.04 (0.19) \\ 

\Lx{CS} 
& \textbf{-0.29 (0.64)} 
& \textbf{-0.96 (0.55)} 
& \textbf{0.67 (0.36)} 
& \textbf{-0.35 (0.07)} 
& \textbf{1.09 (0.19)} 
& \textbf{0.73 (0.20)} 
& \textbf{0.48 (0.16)} 
& \textbf{0.60 (0.25)} 
& \textbf{0.25 (0.27)} \\ 

\Lx{CSmax} 
& \textbf{-1.16 (2.94)} 
& \textbf{-0.28 (1.76)} 
& \textbf{-0.88 (1.93)} 
& \textbf{-0.39 (0.79)} 
& \textbf{2.19 (0.84)} 
& \textbf{1.81 (0.98)} 
& \textbf{0.31 (0.71)} 
& \textbf{1.88 (0.87)} 
& \textbf{1.50 (1.14)} \\ 

\bottomrule
\end{tabularx}
\end{table*}

\subsection{HD ward}

For measurements at HD-A, LME-ART-ANOVA also revealed significant differences across all A-weighted decibel metrics at \SI{0.01}{\percent} significance level. Except between \binL\ and \binR\ in \Lx{AS} and \Lx{ASmax}, significant differences were found between all microphone pairs in the post-hoc contrast tests. 

A reverse trend was observed in HD-A, where the measurement microphones were louder than the binaural microphones across all A-weighted metrics between all but one binaural-measurement microphone pair. Except between \binR\ and \GRASOut\ (\mudiffLo{R}{out}{AS10}{1.45} \dba{}), mean differences across all A-weighted metrics in the rest of the binaural-measurement microphone pairs ranged from \num{-1.62} to \dba{-1.04}, as shown in \Cref{tab:mean_sd_1hr}. Hence, on average, the sound level that was exceeded \SI{10}{\percent} of the time was louder at the right ear than the measurement microphone \SI{1.6}{\meter} from the ground above the bassinet in HD-A. 

For each C-weighted decibel metric, main effects reached significance with LME-ART-ANOVA at a \SI{0.01}{\percent} significance level, indicating a large effect size. Post-hoc contrast tests further demonstrated significant differences across all microphone pairs for each C-weighted metric at HD-A, also at a \SI{0.01}{\percent} significance level.

Mean differences between \binL\ and \binR\ in C-weighted metrics were significant but small: \mudiffLo{L}{R}{CS}{-0.96}, \mudiffLo{L}{R}{CSmax}{-0.29}, \mudiffLo{L}{R}{CS10}{-0.92}, and \mudiffLo{L}{R}{CS50}{-1.08} \dbc{}. Additionally, the variability in mean differences between the binaural microphones and \GRASOut\ was more pronounced, ranging from \dLhgeC{CSmax}{1}{1.91}{6.41} and \dLhgeC{CS10}{1}{0.56}{1.56}, while remaining comparable in \dLhgeC{CS}{1}{0.46}{0.71} and \dLhgeC{CS50}{1}{0.92}{1.09}. 

Between binaural channels, low-frequency sounds were slightly louder at the right ear (\binL) across all C-weight metrics: \mudiffLo{L}{R}{CS}{-0.96}, \mudiffLo{L}{R}{CSmax}{-0.29}, \mudiffLo{L}{R}{CS10}{-0.92}, and \mudiffLo{L}{R}{CS50}{-1.08} \dbc{}. As compared to the standard measurement microphone, C-weighted metrics were slightly higher than the left but slightly lower than the right binaural channel, with the exception of the mean difference in \Lx{CSmax} levels between \binR\ and \GRASOut: \mudiffLo{R}{out}{CSmax}{-0.88} \dbc{}.

\subsection{NICU ward}

The non-parametric LME-ART-ANOVA revealed significant differences across \Lx{AS}, \Lx{ASmax}, \Lx{AS10} and \Lx{AS50} at \SI{0.01}{\percent} significance level. Subsequent pairwise post-hoc ART contrast tests indicated significant differences among all microphone pairs, except between the left and right binaural channels (\binL\ and \binR) in \Lx{AS} and \Lx{AS10} at NICU-A.  

Overall, sound levels were higher at the binaural microphones than the standard measurement microphones, indicated by mean differences across all A-weighted metrics. Mean differences in \Lx{AS} among all binaural-standard microphone pairs ranged between \mudiffLo{R}{out}{AS}{2.60} and \mudiffLo{L}{in}{AS}{3.02} \dba{}. For maximum levels, mean difference varied from \mudiffLo{R}{out}{ASmax}{1.90} to \mudiffLo{L}{in}{ASmax}{3.04} \dba{}. Likewise, mean differences in the \SI{10}{\percent} and \SI{50}{\percent} exceedance levels fell within the range of \mudiffLo{L}{out}{AS10}{2.44} to \mudiffLo{R}{in}{AS10}{2.91} \dba{}, and \mudiffLo{R}{out}{AS50}{2.82} to \mudiffLo{L}{in}{AS50}{3.29} \dba{}, respectively.

Sound levels recorded at the standard measurement microphone outside the incubator (\GRASOut) were generally significantly louder than the standard microphone inside the incubator (\GRASIn) across all A-weighted metrics,  indicating elevated external sound sources. However, mean differences across all A-weighted metrics were notably smaller than those between binaural and measurement microphones. For instance, mean differences ranged between \mudiffLo{in}{out}{AS50}{-0.36} and \mudiffLo{in}{out}{ASmax}{-0.60} \dba{}.

Although the 1-h aggregated sound levels of higher frequencies (A-weighted) were similar across both ears (\mudiffLo{L}{R}{AS}{0.039}), temporal variation between \binL\ and \binR\ is evident in the significant differences \SI{50}{\percent} exceedance levels \mudiffLo{L}{R}{AS50}{0.11} \dba{}. \chreplaced[comment=R3.3]{The significant differences between \binL\ and \binR\ in the loudest events (\mudiffLo{L}{R}{ASmax}{0.54}) contrasted with }{Interestingly, the significant differences between \binL\ and \binR\ in the loudest events (\mudiffLo{L}{R}{ASmax}{0.54}) were contrasted by} the lack of difference in loud events that occurred for about \SI{10}{\percent} of the time (\mudiffLo{L}{R}{AS10}{-0.00}), which indicate the infrequent transient nature of the loud sounds.

Statistically significant differences emerged across all C-weighted decibel metrics through LME-ART-ANOVA at a significance level of \SI{0.01}{\percent}, with a large effect size. Subsequent post-hoc contrast tests revealed significant differences among all microphone pairs for all C-weighted decibel metrics at the \SI{0.01}{\percent} significance level, with exceptions for comparisons between \binL\ and \binR, and between \GRASOut\ and \binR, both occurring at \Lx{CS50}. In these two instances at \Lx{CS50}, differences were identified at a significance level of \SI{1}{\percent} between \binL\ and \binR, while no significant differences manifested between \GRASOut\ and \binR. 

Significant differences between \binL\ and \binR\ across all C-weighted metrics indicate that left ear experienced slightly higher low-frequency sounds on average at NICU-A: \mudiffLo{L}{R}{CS}{0.48}, \mudiffLo{L}{R}{CS10}{0.41}, \mudiffLo{L}{R}{CS50}{0.54}, and \mudiffLo{L}{R}{CSmax}{0.31} \dbc{}. Similarly, low-frequency sound levels were slightly louder outside ($m_\text{out}$) than near to ($m_\text{in}$) the infant tray on average: \mudiffLo{in}{out}{CS}{-0.35} , \mudiffLo{in}{out}{CSmax}{-0.39}, \mudiffLo{in}{out}{CS10}{-0.36}, and \mudiffLo{in}{out}{CS50}{-0.33} \dbc{}. 

Differences between the binaural microphones and the standard measurement microphones for low-frequency sounds were most pronounced for the loudest sounds, while differences were small for the rest of the C-weighted metrics. Mean difference in \Lx{Cmax} levels of binaural-standard microphone pairs ranged from \mudiffLo{R}{out}{CSmax}{1.50} to \mudiffLo{L}{in}{CSmax}{2.19} \dbc{}, whereas other C-weighted metrics ranged from \mudiffLo{R}{out}{CS}{0.25} to \mudiffLo{L}{in}{CS10}{1.28} \dbc{}. Peculiarly, sound levels that were exceeded \SI{50}{\percent} of the time were similar between the right ear and the standard microphone \SI{1.6}{\meter} above the ground: \mudiffLo{R}{out}{CS50}{0.04} \dbc{}. 



