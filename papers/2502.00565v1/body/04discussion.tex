\section{Results and Discussion}
\label{sec:discussion}

The following discussion seeks to research questions established in \Cref{sec:rq} in the context of providing insight into the implications for clinical practice and NICU design guidelines. \Cref{sec:rq1micpos} examines how microphone positions influences the measurement of noise levels (RQ1). \chreplaced[]{
\Cref{sec:rq2bedward} explores the influence of bed positions and ward layout on neonatal auditory perception of the environment (RQ2). \Cref{sec:rq3other} reviews the suitability of non-traditional (psycho)acoustic parameters in assess the neonatal critical care acoustic environment. Finally, limitations and future work are presented in \Cref{sec:limits}
}
{
The research questions in \Cref{sec:rq} are discussed sequentially in \Cref{sec:rq1micpos}, \ref{sec:rq2bedward}, and \ref{sec:rq3other}, followed by limitations and future work in \Cref{sec:limits}.
} 

\subsection{Do different microphone positions influence the measurement of noise levels in neonatal critical care environments?} \label{sec:rq1micpos}

The results indicate that microphone positioning significantly impacts the assessment of noise exposure in neonatal critical care environments. In NICU-A, overall sound levels were consistently higher at the binaural microphones compared to the standard microphones, whereas in HD-A, the opposite trend was observed, with standard microphones capturing higher sound levels than the binaural ones. This contrast suggests that the spatial location of microphones relative to the neonate’s ears plays a crucial role in accurately capturing their auditory experience, especially in reverberant environments and dynamic noise sources.

When comparing the left and right binaural microphones, A-weighted metrics — which capture higher frequency sounds — were generally similar between both ears in both NICU-A and HD-A. However, a notable difference emerged during the loudest events (\Lx{ASmax}) in NICU-A, where the higher frequency sounds were more pronounced in one ear, indicating potential directional sound sources, such as cardiac alarms. On the other hand, C-weighted metrics, which focus on lower frequency sounds, showed consistent disparities between the left and right binaural microphones. These differences were more pronounced in HD-A than in NICU-A, suggesting that low-frequency sounds are perceived differently across both ears and are more variable in open ward environments like HD-A.

The distinctions observed in C-weighted metrics can be attributed to the significant mean differences in low-frequency sound levels, as highlighted in \Cref{tab:mean_sd_1hr}. This finding highlights the importance of placing microphones at or near ear level within open incubators, particularly in the NICU, to accurately capture fluctuations in low-frequency sounds over time. Given that neonates can discriminate sounds as low as \SI{250}{\hertz}, monitoring C-weighted metrics is crucial for a comprehensive assessment of their auditory environment.

The influence of microphone positions is further evident in the varying compliance rates in the CC guidelines. In NICU-A and NICU-B, lower compliance was noted with binaural microphones, whereas in HD-A, the standard microphones showed reduced compliance. These findings highlight the need for future guidelines to adopt standardized protocols and equipment that more accurately reflect the soundscape experienced by neonates. For instance, ISO 12913-2 \citep{iso12913-2} mandates the use of artificial head and torso simulators (HATS) with traceable calibration for soundscape assessments. However, since HATS that replicate neonatal hearing mechanisms are not yet available, standardized neonatal mannequins equipped with calibrated binaural microphones, as employed in this study, could provide a suitable alternative. 

\chadded[comment=R3.4]{Additionally, it is crucial to acknowledge that the observations in this study pertain only to external noise sources, as self-noise, such as sounds from medical equipment, caregiving activities, or crying infants, was not captured. Here, ``crying infants'' refers specifically to the sounds that would have been recorded if the microphones were affixed to actual infants rather than neonatal dolls. This distinction ensures that the findings focus on environmental noise external to the measurement setup.} 

\subsection{What role do bed position and ward layout play in
shaping neonatal noise exposure in critical care units?}
\label{sec:rq2bedward}

Measurements revealed that bed position and ward layout considerably influence neonatal noise exposure. The differences between NICU-A, NICU-B, and HD-A bed positions were statistically significant across both A- and C-weighted metrics, with large effect sizes in most cases. NICU-A consistently had the highest noise levels across metrics, likely due to its proximity and frequency of noise sources such as cardiac monitor alarms and personnel activities of the reception area in the vicinity. This suggests that NICU design and bed placement can play a significant role in either mitigating or exacerbating noise exposure for neonates.

In contrast, NICU-B exhibited lower noise levels than NICU-A across most metrics but was still louder than HD-A in specific instances, such as for \Lx{AS}, \Lx{AS50}, and \Lx{CS50}. The relative quietness in HD-A may be attributed to its spatial configuration, which is more isolated from noise sources, or differences in staff activity levels, or reduced reverberance. These findings emphasise the need for strategic bed placement and ward design to minimise harmful noise exposure, considering both the location of noise sources and the acoustic properties of the ward environment.

\subsection{How suitable are additional (psycho)acoustic metrics for assessing the complex sound environment in
neonatal critical care?} \label{sec:rq3other}

While traditional A-weighted metrics remain useful, the results suggest that incorporating additional (psycho)acoustic metrics such as tonality and signal-to-noise ratio can offer a more comprehensive evaluation of the sound environment in neonatal critical care. The differences observed in the frequency of loud events and tonal occurrences across bed positions and wards highlights the variability in how noise manifests, both temporally and spectrally.

For instance, the NICU wards exhibited continuous tonal signals ($T>0.4$), which were more frequent than in the HD ward. Such signals, though not necessarily loud, could contribute to sensory overload for neonates and affect their development if persistent \citep{lejeune_sound_2016}. Loud event occurrence rates were higher at HD-A than at both NICU-A and NICU-B, suggesting that while average sound levels may be lower, transient loud noises are more frequent, potentially disrupting sleep and physiological outcomes \citep{kuhn_evaluating_2011}. 

These findings indicate that integrating (psycho)acoustic metrics with standard decibel-based measurements could lead to a more nuanced understanding of the neonatal soundscapes. This is particularly relevant when designing interventions or making decisions on NICU layouts, as it aligns more closely with the perceptual impact of noise on neonates, potentially leading to better health outcomes.

\subsection{Limitations and future work} \label{sec:limits}

This study provides valuable insights into noise assessment in neonatal critical care units (NICUs), yet several limitations warrant consideration. Although binaural microphones offer a closer approximation of what neonates might experience, they do not fully replicate the unique auditory anatomy of neonates. The absence of standardised HATS designed specifically for neonates limits the generalisability and replicability of these findings, given that existing HATS are calibrated for adult sound perception. Developing neonatal-specific HATS or alternative simulation models could bridge this gap in future research.

Second, owing to operational constrains, the study only assessed soundscapes in three bed positions across two different wards, which may not fully capture the variability in noise exposure across diverse neonatal intensive care unit (NICU) layouts and configurations. Expanding this research \chreplaced[comment=R4.1]{to encompass varied ward layouts and larger number of measurement points is needed to enhance generalizability of these findings and}{to include more varied environments and larger sample sizes} would provide a broader understanding of how ward design, equipment placement, and caregiving activities influence noise exposure.

Lastly, while this study incorporated (psycho)acoustic metrics such as tonality and transient event occurrences, it is important to note that the thresholds used were primarily derived from adult hearing models (e.g., $T>0.4$ from \citep{International2020}) or limited empirical evidence (e.g., $\text{SNR}>\dba{5}$ for possible waking events in \citep{kuhn_evaluating_2011}). Future research should focus on establishing neonatal-specific thresholds and models for (psycho)acoustic metrics, taking into account their unique auditory profiles and developmental stages.