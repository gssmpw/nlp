\newcommand*{\papertitle}{Do neonates hear what we measure? Assessing neonatal ward soundscapes at the neonates' ears}
% Short title
\shorttitle{\papertitle}    
% Short author
\shortauthors{Lam et al.}  

\title[mode=title]{\papertitle}

\author[eee]{Bhan Lam}[orcid=0000-0001-5193-6560, degree=Ph.D.]
\ead{blam002@e.ntu.edu.sg}
\corref{c}\cortext[c]{Corresponding authors}
\credit{Conceptualization, Methodology, Software, Validation, Formal analysis, Investigation, Project administration, Data Curation, Writing - Original Draft, Writing - Review \& Editing, Visualization, Supervision}

\author[sgh]{Peijin Esther Monica Fan}[orcid=0000-0002-6325-154X, degree=RN]
%\ead{esther.monica.fan.p.j@sgh.com.sg}
\credit{Conceptualization, Methodology, Resources, Writing - Review \& Editing, Project administration}

\author[sgh]{Yih Yann Tay}[degree=RN]
%\ead{tay.yih.yann@sgh.com.sg}
\credit{Conceptualization, Investigation, Resources, Writing - Review \& Editing, Project administration}

\author[sghndm]{Woei Bing Poon}[degree=MRCPCH{,} FAMS]
%\ead{tay.yih.yann@sgh.com.sg}
\credit{Conceptualization, Investigation, Resources, Writing - Review \& Editing, Project administration}

\author[eee]{Zhen-Ting Ong}[orcid=0000-0002-1249-4760]
%\ead{ztong@ntu.edu.sg}
\credit{Resources, Investigation, Data Curation, Project administration}

\author[eee]{Kenneth Ooi}[orcid=0000-0001-5629-6275, degree=Ph.D.]
%\ead{wooi002@e.ntu.edu.sg}
\credit{Formal analysis, Resources, Writing - Review \& Editing}

\author[eee]{Woon-Seng Gan}[orcid=0000-0002-7143-1823, degree=Ph.D.]
%\ead{ewsgan@ntu.edu.sg}
\credit{Resources, Writing - Review \& Editing, Supervision, Supervision}

\author[sgh]{Shin Yuh Ang}[orcid=0000-0001-9614-7012,degree=MBA{,} RN]
\ead{ang.shin.yuh@singhealth.com.sg}
\corref{c}
%\cortext[c]{Corresponding authors}
\credit{Conceptualization, Resources, Investigation, Writing - Review \& Editing, Supervision}

\affiliation[eee]{
    organization={%%
        School of Electrical and Electronic Engineering, 
        Nanyang Technological University%
    },
    addressline={50 Nanyang Ave}, 
    postcode={639798}, 
    country={Singapore}
}

\affiliation[sgh]{
    organization={%%
        Nursing Division, 
        Singapore General Hospital%
    },
    addressline={Outram Rd, 169608}, 
    country={Singapore}
}

\affiliation[sghndm]{
    organization={%%
        Department of Neonatal and Developmental Medicine, 
        Singapore General Hospital%
    },
    addressline={Outram Rd, 169608}, 
    country={Singapore}
}

\begin{abstract}
%% Text of abstract
Acoustic guidelines for neonatal intensive care units (NICUs) aim to protect vulnerable neonates from noise-induced physiological harm. However, the lack of recognised international standards for measuring neonatal soundscapes has led to inconsistencies in instrumentation and microphone placement in existing literature, raising concerns about the relevance and effectiveness of these guidelines. This study addresses these gaps through long-term acoustic measurements in an operational NICU and a high-dependency ward. We investigate the influence of microphone positioning, bed placement, and ward layout on the assessment of NICU soundscapes. Beyond traditional A-weighted decibel metrics, this study evaluates C-weighted metrics for low-frequency noise, the occurrence of tonal sounds (e.g., alarms), and transient loud events known to disrupt neonates' sleep. Using linear mixed-effects models with aligned ranks transformation ANOVA (LME-ART-ANOVA), our results reveal significant differences in measured noise levels based on microphone placement, highlighting the importance of capturing sound as perceived directly at the neonate's ears. Additionally, bed position and ward layout significantly impact noise exposure, with a NICU bed position consistently exhibiting the highest sound levels across all (psycho)acoustic metrics. These findings support the adoption of binaural measurements along with the integration of additional (psycho)acoustic metrics, such as tonality and transient event occurrence rates, to reliably characterise the neonatal auditory experience.
\end{abstract}

%%Graphical abstract
% \begin{graphicalabstract}
% \includegraphics[width=\linewidth]{figures/grabs_placeholder.pdf} 
% \end{graphicalabstract}

\ifarxiv\else
%%Research highlights
\begin{highlights}
\item Binaural microphone positioning reveals disparities in neonatal noise exposure
\item Ward layout and bed positioning significantly impacts neonatal soundscapes
\item Frequent loud transients despite lower averages highlight need for advanced metrics
\item C-weighted metric differences suggest the need for low-frequency monitoring
\end{highlights}
\fi

\begin{keywords}
%% keywords here, in the form: keyword \sep keyword
neonatal intensive care \sep indoor noise \sep hospital soundscape \sep binaural \sep hospital acoustics \sep building acoustics


%% PACS codes here, in the form: \PACS code \sep code
%\PACS 0000 \sep 1111
%% MSC codes here, in the form: \MSC code \sep code
%% or \MSC[2008] code \sep code (2000 is the default)
%\MSC 0000 \sep 1111
\end{keywords}

\maketitle