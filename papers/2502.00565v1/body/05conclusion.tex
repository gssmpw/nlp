\section{Conclusion and recommendations}

Despite the existence of specific acoustic guidelines for neonatal intensive care units (NICU), no internationally recognised standards for acoustic measurements in healthcare settings currently exist. This has inadvertently resulted in notable disparities in how NICU soundscapes are assessed across the literature, particularly in terms of instrument type and placement, resulting in reduced reliability and comparability of the results. Through a detailed measurement campaign in an operational NICU and high dependency (HD) ward, this work investigated the impact of microphone type and placement, bed positioning, and ward layout on the assessment of neonatal soundscapes. To investigate the potential limitations of ubiquitous A-weighted decibel metrics, neonatal soundscapes were further assessed with other (psycho)acoustic parameters such as C-weighted sound pressure level, tonality and signal-to-noise ratio.

Assessment of microphone positions across all (psycho)acoustic parameters within each ward revealed significant differences between standard measurement and binaural microphones, which were affixed to the ears of a neonate doll. With differences also occurring for loud events (\Lx{ASmax}) and low-frequency sounds (C-weighted metrics) between binaural channels, it further indicates that binaural microphone placements would provide a more representative aural experience for neonates. Furthermore, significant differences between bed positions in NICU and across wards lend further support to binaural monitoring (with artificial simulators) for more holistic assessment in dynamic neonatal critical care environments. Notably, the need for additional (psycho)acoustic metric assessment is evidenced in the high occurrence rates for loud events occurring \dba{5} above the background noise levels and prominent tonal events that were not captured by A-weighted decibel metrics.

Evidence gathered in this study point towards a pressing need for further investigation and development of \chadded[comment=R3.5 R4.5]{standardized international} acoustic measurement protocols and instruments that accurately capture the aural experience of vulnerable neonates \chreplaced[comment=R3.5 R4.5]{This includes employing binaural measurements with standardized neonate dolls and incorporating additional metrics, such as C-weighted levels and tonality. These enhancements offer actionable insights for tailoring acoustic treatments and care protocols, aiming to minimize loud events, reduce low-frequency noise propagation, and improve the accessibility and audibility of maternal voice, which is essential for neonatal development.}{such as the use binaural measurement with standardised neonate dolls}.

\chadded[]{Implementation of the proposed measurement techniques and guidelines would require close collaboration between engineers and clinicians during initial design and construct of the NICU environment, in order to optimize positioning of cots and equipment, balancing between acoustic quality, infection prevention measures and work efficiency. Collaboration between medical device companies, hospital engineers and clinicians is also crucial in designing future devices or modules suitable for the NICU patient population.} 

\chadded[comment=R4.6]{In addition is a need for management commitment to invest in monitoring equipment and regular conduct of such exercises to allow for continuous improvement in a timely manner.}