\section{Introduction}
\label{sec:Introduction}

\subsection{Background and Motivation}

\iftodo
\begin{itemize}[leftmargin=*]
  \checkedbox{neonate hearing biology}
  \begin{itemize}
      \checkbox{frequency range}
  \end{itemize}
  \checkedbox{impact of noise on neonates}
\end{itemize}
\fi

The development of the auditory system begins very early in gestation. By 26 weeks of gestation, the human fetus begins to react to auditory stimuli. The auditory system then continues to develop to be fine-tuned to specific frequencies, to process acoustic stimuli into electric signals sent to the brainstem, to discern different speech phonemes and to facilitate learning and memory formation \cite{mcmahon_auditory_2012}. \chadded[comment=R3.1]{Following birth, neonates become increasingly attuned to their acoustic surroundings. They can distinguish auditory cues from background noise and differentiate human speech from non-biological sounds as early as 28 weeks \cite{kuhn_auditory_2017}.}
 
While sensory input due to the acoustic environment has an impact on the structure and function of the neural systems throughout life, its influence is most pronounced during infancy \cite{Dahmen2007}. Hence, the auditory environment of a developing infant is extremely important to its development. 

In the uterus, the fetus receives auditory input via bone conduction and is surrounded by fluid and maternal tissues which attenuates high frequency sounds more so than low frequency sounds \cite{Lahav2015}. Preterm exit from the optimal environment of the womb means that preterm infants are exposed to hearing via air conduction, as well as sounds of higher frequencies and higher volumes (especially in the Neonatal Intensive Care Unit (NICU)). \chreplaced[]{Prolonged exposure to high-frequency noise, especially in NICUs, can disrupt physiological and developmental processes. Studies have shown that such noise may lead to alterations in blood pressure, respiratory rates, and sleep cycles, all of which could be detrimental to neural growth and auditory development \cite{Lahav2015, lahav_acoustic_2014, el-metwally_potential_2020, retsa_longstanding_2024}.}{This may be harmful to infants as high frequency noise exposure has shown to alter blood pressure, decrease respiratory rate and disrupt sleep \cite{Lahav2015}.}

\chadded[]{In the NICU, loud transient sounds can induce immediate physiological responses, including elevated heart rates, altered respiratory patterns, and disrupted sleep cycles \cite{wachman_effects_2011, kuhn_infants_2012, shimizu_sound_2016}. These sudden noises are likely to trigger startle reflexes and stress-mediated defensive response, which may interfere with homeostatic regulation and delay neurological development \cite{kuhn_infants_2012, Philbin2017}.}

\chadded[comment=R3.1 R4.2]{The absence of \textit{in utero} maternal sounds increases the risk of long-term speech and language deficits in preterm neonates \cite{mcmahon_auditory_2012, webb_mothers_2015}.  Evidence suggests that interventions involving the introduction of spoken or recorded maternal sounds, can mitigate these effects, promoting improved auditory and visual orientation as well as enhanced cognitive and language outcomes later in life \cite{philpott-robinson_impact_2017}. Consequently, optimizing the NICU soundscape by minimizing non-biological noise is essential, as it may enhance the efficacy of maternal voice interventions \cite{kuhn_auditory_2017,el-metwally_potential_2020}. However, a recent systematic review found that no NICU meets the American Academy of Pediatrics guidelines of \dba{45}, indicating a need for improved NICU designs to support auditory and developmental health \cite{andy_systematic_2025}.}

\subsection{Acoustic design guidelines and regulations in NICUs}

\iftodo
\begin{itemize}[leftmargin=*]
  \checkedbox{WHO and AAP}
  \checkedbox{NICU design guidelines 9th ed}
  \checkedbox{Singapore guidelines}
  \checkedbox{Summary table}
  \checkedbox{incubator maximum sound level and standard}
\end{itemize}
\fi

In the context of neonatal ward soundscapes, it is crucial to adhere to established and updated guidelines and regulations to ensure an optimal acoustic environment to protect the well-being of vulnerable neonates. Among the various guidelines available, the World Health Organisation (WHO) followed by American Academy of Pediatrics (AAP) guidelines are frequently referenced in the literature \citep{DeLimaAndrade2021}. The WHO recommends stringent acoustic standards for hospital wards, including an A-weighted equivalent sound pressure level (SPL) indoors of \Lhle{AF}{24}{30} and an instantaneous maximum SPL level of \Lle{AFmax}{40} \citep{Berglund1999a}, where the subscript $\{{\cdot}\}_\text{F}$ refers to the `\textit{fast}' time response (\SI{125}{\milli\second}) according to IEC~61672\nobreakdash‑1 \citep{IEC6167212013}. Similarly, the AAP Guidelines suggests maintaining sound pressure levels (SPL) below a day-night averaged SPL (\dba{10} penalty to nighttime levels) of \Lle{dn}{45} \citep{AAP1997}, based on the U.S. Environmental Protection Agency guidelines for hospitals \citep{EPA1974}. 

While acknowledging the widespread referencing of the WHO (\citeyear{Berglund1999a}) and AAP (\citeyear{AAP1997}) guidelines in the literature, it is important to recognise the possibility that these guidelines may now be outdated. Consequently, in response to the need for more intuitive and realistic standards, the Consensus Committee (CC) on \textit{Recommended Design Standards for Advanced Neonatal Care} has consistently revised the ``\textit{Recommended Standards for Newborn ICU Design}''. The most recent update, the 10\textsuperscript{th} edition in 2023 \citep{altimier_recommended_2023}, reflects these changes aimed at enhancing the standard's practicality and alignment with current research and practices in neonatal care.

According to the CC, the recommended one-hour SPLs are \Lhle{AS50}{1}{50} and \Lhle{AS10}{1}{65}, as measured three feet from an infant bed or other relevant listening position. Here, subscript $\{{\cdot}\}_\text{S}$ refers to the `\textit{slow}' (\SI{1}{\second}) time response, while $\{{\cdot}\}_{50}$ and $\{{\cdot}\}_{10}$ represent the percentage exceedance levels as described in ISO~1966\nobreakdash-1 \cite{ISO1996-1}. Notably, the $9^\text{th}$ edition update replaced the equivalent SPL guidelines of \Lhle{AS}{1}{45} to the \SI{50}{\percent} exceedance levels of \Lhle{AS50}{1}{50}; increased the $L_\text{AS10,\SI{1}{\hour}}$ from \dba{50} to \dba{65}; and removed the maximum SPL requirement of \Lle{ASmax}{65}. Various portions of the CC NICU design standards have been adopted in American Institute of Architects/Facilities Guidelines Institute Guidelines \citep{FGI2022}, AAP/American College of Obstetricians and Gynecologists’ (ACOG) Guidelines for Perinatal Care \citep{AAPACOG2017}, and forms the basis for noise emission limits in infant incubators in IEC 60601-2-19 \citep{IEC60601-2-19}.

In Singapore, there are no specific guidelines or regulations governing the acoustic environment in NICUs or buildings in general. However, under the Environmental Protection and Management Act of 1999, two statutory regulations control noise levels emitted from construction sites and factory premises, with specific limits for noise sensitive buildings such as hospitals. For construction noise \citep{AGCConst2008}, the maximum permissible levels for hospitals as measured \SI{1}{\meter} from the facade, are as follows: (1) \Lhle{AF}{12}{60} between 7am to 7pm, (2) \Lhle{AF}{12}{50} between 7pm to 7am, (3) \Lminle{AF}{5}{75} between 7am to 7pm, and (4) \Lminle{AF}{5}{55} between 7pm to 7am. Regarding factory premises, the maximum permissible levels measured on the boundary facing the hospital are: (1) \Lhle{AF}{12}{65} between 7 am and 7 pm, (2) \Lhle{AF}{3}{55} between 7 pm and 11 pm, and (3) \Lhle{AF}{9}{50} between 11 pm and 7 am \citep{AGCFactoryNoise2008}. 

Another point of reference for NICU acoustic design requirements in Singapore can be found in the BCA Green Mark 2021 (GM: 2021) green building certification for healthcare facilities. The acoustic requirements in GM:2021 are derived from guidelines provided by the United Kingdom (UK) \citep{HTM0801_2013} and a joint standard established by Australia and New Zealand (ANZ) \citep{ASNZS2107_2016}. The Healthcare Technical Memorandum (HTM~08\nobreakdash-01) from the UK, sets limits for external sound sources within multi-bed wards, with recommended levels of \Lhle{AF}{1}{40} during the day and \Lhle{AF}{1}{35} at night (11pm to 7am), as well as \Lle{AFmax}{45}. Noise rating (NR) limits were also specified for mechanical and electrical services within the ward (i.e. $\text{NR}\le30$). In contrast to the broader guidelines applicable to hospital wards in general, the ANZ standard (AS/NZS~2107) provides specific interior noise limits tailored for NICUs and pediatric ICUs (PICU). These standards define acoustic targets such as \Lle{ASmax}{50} for external transient noise, as well as interior ambient noise levels of \Lhle{AS}{1}{45} and \Lhle{AS10}{1}{50} for NICU/PICU wards. The recommended acoustic limits and guidelines discussed are summarised in \Cref{tab:sound-standards}.

%\footnotesize
%\tiny
\begin{table}[ht]
\footnotesize
  \centering
  \caption{Acoustic standards and guidelines for Neonatal Intensive Care Units (NICU) / Pediatric Intensive Care Units (PICU)}
  \label{tab:sound-standards}
  \begin{tabularx}{\linewidth}{p{1cm}X*{2}{>{\raggedright\arraybackslash}p{2.55cm}}}
    \toprule
    \textbf{Source}\textsuperscript{\textit{1}}
    & \textbf{NICU/\newline PICU} 
    & \textbf{Ambient Sound Level} 
    & \textbf{External Sound Intrusion} \\
    \midrule
    WHO \citep{Berglund1999a}
    & No 
    & \Lhle{AF}{24}{30}\newline \Lle{AFmax}{40}
    & - \\
    \midrule
    AAP \citep{AAP1997}
    & Yes 
    & \Lle{dn}{45}
    & - \\
    \midrule
    CC (8\textsuperscript{th} ed) \citep{White2013}
    & Yes
    & \Lhle{AS}{1}{45}\newline \Lhle{AS10}{1}{50}\newline \Lle{ASmax}{65}
    & - \\
    \midrule
    CC (9 \& 10\textsuperscript{th} ed) \citep{White2020,altimier_recommended_2023}
    & Yes
    & \Lhle{AS50}{1}{50}\newline \Lhle{AS10}{1}{65}
    & - \\
    \midrule
    HTM 08-01 \citep{HTM0801_2013}
    & No 
    & $\text{NR}\le30$ 
    & \Lhle{AF}{1}{40}\newline (7am to 11pm)\newline \Lhle{AF}{1}{35}\newline (11pm to 7am)\newline \Lle{AFmax}{45}\\
    \midrule
    AS/NZS 2107 \citep{ASNZS2107_2016}
    & Yes 
    & \Lhle{AS}{1}{45}\newline\Lhle{AS10}{1}{50}
    & \Lle{ASmax}{50}\\
    \midrule
    IEC 60601-2-19 \citep{IEC60601-2-19}
    & Yes
    & \Lle{ASmax}{60} (ambient sound inside incubator)\newline \Lle{ASmax}{80} (alarm sounding inside incubator)
    & -\\
    \bottomrule
  \end{tabularx}
  \noindent
  \begin{minipage}{1\linewidth}
    \scriptsize
    \textsuperscript{\textit{1}}World Health Organisation (WHO), American Academy of Pediatrics (AAP), Consensus Committee (CC) on recommended design standards for advanced neonatal care, HTM (Health Technical Memoranda), Australian/New Zealand Standard (AS/NZS), International Electrotechnical Commission (IEC) 
  \end{minipage}
\end{table}


\iftodo
\subsection{Neonatal intensive care unit (NICU) soundscape}
\begin{itemize}[leftmargin=*]
  \checkedbox{state of noise in neonatal wards}
  \checkedbox{incubator acoustics}
\end{itemize}
\fi

Despite notable advancements in noise control technology, a growing body of literature highlights the increasing noise levels within hospitals \cite{DeLimaAndrade2021,Busch-Vishniac2019, Busch-Vishniac2023,Lam2022c}. Within neonatal intensive care units (NICUs), the role of acoustics becomes even more critical as it directly impacts the survival rates and recovery process of vulnerable neonates. Previous measurements of A-weighted equivalent sound pressure levels (SPL) near the ears of infants in open-box incubators have revealed values ranging from \num{54.7} to \SI{60.44}{\decibelA}, surpassing the recommended guidelines outlined in \Cref{tab:sound-standards}. 

 Furthermore, neonates are frequently exposed to loud transient sounds, reaching levels between \num{62}--\SI{90.6}{\decibelA}, as indicated by variations in the \Lx{Amax} indicators presented in \Cref{tab:niculevels}. These transient noises can be attributed to various sources, including medical-staff activities, medical devices, or alarm peaks \citep{Bertsch2020}. Such high sound levels pose potential risks to the delicate hearing system of neonates, with \Lx{Amax} levels reaching up to \SI{94.8}{\decibelA} inside incubators \citep{Parra2017}, potentially resulting in permanent cochlear damage \citep{Surenthiran2003} or temporary hearing threshold shifts \citep{McCullagh1979}. Moreover, sudden sounds can trigger startle responses in neonates, interrupting their rest and recovery \citep{Philbin1999}. It is worth noting that the \Lh{AS10}{1} measurements reported by \citeauthor{Krueger2007} fall well within the NICU design guidelines of the 9\textsuperscript{th} edition (\Lhle{AS10}{1}{65}) ranging from 59.26--\SI{60.6}{\decibelA}. 


\begin{table*}[ht]
  \centering
  \caption{A summary of acoustic measurement studies in neonatal intensive care units (NICU) with reported acoustic indicators and measurement equipment standards.}
  \label{tab:niculevels}
  \begin{tabularx}{\textwidth}{%
  >{\raggedright\arraybackslash}p{0.1\textwidth}%
  >{\raggedright\arraybackslash}p{0.1\textwidth}%
  >{\raggedright\arraybackslash}p{0.1\textwidth}%
  >{\raggedright\arraybackslash}p{0.12\textwidth}%
  >{\raggedright\arraybackslash}p{0.28\textwidth}X}
    \toprule
    \textbf{Source} & \textbf{Indicator} & \textbf{Value} [\si{\decibelA}] & \textbf{Measurement Period} & \textbf{Measurement Equipment Placement} & \textbf{Measurement Equipment} \\
    \midrule
    \multirow[t]{6}{0.1\textwidth}{\citeauthor{Krueger2007} \citeyear{Krueger2007}} 
    & \Lh{AS}{1} 
    & 60.44 
    & \multirow[t]{3}{0.12\textwidth}{Before: \SI{8}{\hour}/day for 9 days}
    & \multirow[t]{6}{0.28\textwidth}{Behind bed spaces 2--3 feet above countertop}
    & \multirow[t]{6}{*}{SLM (Class 1)} 
    \\
    \cline{2-3}
    & \Lh{AS10}{1} 
    & 59.26 &  &  & 
    \\
    \cline{2-3}
    & \Lh{ASmax}{1} & 78.39 &  &  & 
    \\
    \cline{2-4}
    & \Lh{AS}{1} 
    & 56.4 
    & \multirow[t]{3}{0.12\textwidth}{After: \SI{8}{\hour}/day for 2 days}
    &  
    & 
    \\
    \cline{2-3}
    & \Lh{AS10}{1} & 60.6 &  &  & 
    \\
    \cline{2-3}
    & \Lh{ASmax}{1} & 90.6 &  &  & 
    \\
    \midrule
    \citeauthor{Darcy2008} \citeyear{Darcy2008}
    & \Ls{AS}{1}
    & 53.9--60.6
    & 5/\si{\hour} in \SI{2}{\hour} 
    & Center of bay 
    & SLM (Class 2) 
    \\
    \midrule
    \multirow[t]{2}{0.1\textwidth}{\citeauthor{Lahav2015} \citeyear{Lahav2015}} 
    & \Lh{A}{24} 
    & 60.05
    & \multirow[t]{2}{*}{5 days} 
    & \multirow[t]{2}{*}{Center of bay}
    & \multirow[t]{2}{*}{SLM (Class 1)} 
    \\
    \cline{2-3}
    & \Lh{A}{24} 
    & 58.67 
    & 
    & 
    &  
    \\
    \midrule
    \multirow[t]{2}{0.1\textwidth}{\citeauthor{Romeu2016} \citeyear{Romeu2016}} 
    & \Ls{A}{1} 
    & 53.4--62.5
    & \SI{56}{\hour}
    & Inside incubator (roof)
    & \multirow[t]{2}{*}{SLM (Class 1)} 
    \\
    \cline{2-5}
    & \Ls{A}{1}
    & 60.4--65.5 
    & \SI{56}{\hour}
    & Outside incubator
    &  
    \\
    \midrule
    \multirow[t]{3}{0.1\textwidth}{\citeauthor{Shoemark2016} \citeyear{Shoemark2016}} 
    & \Lx{AS} 
    & 58 
    & \multirow[t]{3}{*}{720/\si{\hour} in \SI{24}{\hour}} 
    & \multirow[t]{3}{*}{Near head-end of bed} 
    & \multirow[t]{3}{*}{SLM (Class 1)} 
    \\
    \cline{2-3}
    & \Lx{ASmax} & 62 &  &  &  
    \\
    \cline{2-3}
    & \Lx{ASmin} & 55 &  &  &  
    \\
    \midrule
    \multirow[t]{2}{0.1\textwidth}{\citeauthor{Park2017neonatal} \citeyear{Park2017neonatal} } 
    & \Lmin{A}{10}
    & 56.7 
    & \SI{24}{\hour}
    & Inside incubator 
    & \multirow[t]{2}{*}{IEC 61094-4 WS2P} 
    \\
    \cline{2-5}
    & \Lmin{A}{10} 
    & 57.1
    & \SI{24}{\hour}
    & Outside incubator
    &  
    \\
    \hline
    \multirow[t]{6}{0.1\textwidth}{\citeauthor{Parra2017} \citeyear{Parra2017}} 
    & \Lx{A} 
    & 59.5 
    & \multirow[t]{3}{*}{3600/\si{\hour} in \SI{24}{\hour}} 
    & \multirow[t]{3}{*}{Center of bay}
    & \multirow[t]{6}{*}{Dosimeter (Class 2)} \\
    \cline{2-3}
    & \Lx{A10} & 61.8 &  &  &  \\
    \cline{2-3}
    & \Lx{Amax} & 85.2 &  &  &  \\
    \cline{2-5}
    & \Lx{A} 
    & 65.8 
    & \multirow[t]{3}{*}{3600/\si{\hour} in \SI{24}{\hour}}
    & \multirow[t]{3}{*}{Inside incubator, \SI{30}{\centi\meter} from the ears}
    &  
    \\
    \cline{2-3}
    & \Lx{A10} & 68.1 &  &  &  \\
    \cline{2-3}
    & \Lx{Amax} & 94.8 &  &  &  \\
    \midrule
    \multirow[t]{2}{0.1\textwidth}{\citeauthor{Smith2018} \citeyear{Smith2018}} 
    & \Ls{AF}{1} 
    & 58.1
    & 60/\si{\hour} in \SI{4}{\hour}
    & Open-pod; between two Isolette  
    & \multirow[t]{2}{*}{Dosimeter (Class 2)} \\
    \cline{2-5}
    & \Ls{AF}{1} 
    & 54.7 
    & 60/\si{\hour} in \SI{4}{\hour}
    & Private room; near ears 
    & \\
    \midrule
    \multirow[t]{2}{0.1\textwidth}{This study}
    & \Lx{AS,1h} & 56.09 (3.48) & \multirow[t]{4}{*}{\SI{981}{\hour}} & \multirow[t]{4}{0.28\textwidth}{Binaural microphone affixed to ears on neonate doll in a open bassinet (HD-A)} &  IEC 61094-4 WS2F \\
    \cline{2-3}
    & \Lx{AS10,1h} & 58.98 (3.87) &  &  &  \\
    \cline{2-3}
    & \Lx{AS50,1h} & 51.59 (3.36)  &  &  &  \\
    \cline{2-3}
    & \Lx{ASmax,1h} & 73.76 (4.20) &  &  &  \\
    \cline{2-5}
    & \Lx{AS,1h} & 59.38 (1.90) & \multirow[t]{4}{*}{\SI{392}{\hour}} & \multirow[t]{4}{0.28\textwidth}{Binaural microphone affixed to ears on neonate doll in a open-type incubator (NICU-A)} &  \\
    \cline{2-3}
    & \Lx{AS10,1h} & 61.40 (2.70) &  &  &  \\
    \cline{2-3}
    & \Lx{AS50,1h} & 57.63 (1.38)  &  &  &  \\
    \cline{2-3}
    & \Lx{ASmax,1h} & 73.4 (4.12) &  &  &  \\
    \cline{2-5}
    & \Lx{AS,1h} & 58.29 (1.81) & \multirow[t]{4}{*}{\SI{98}{\hour}} & \multirow[t]{4}{0.28\textwidth}{Binaural microphone affixed to ears on neonate doll in a open-type incubator (NICU-B)} &  \\
    \cline{2-3}
    & \Lx{AS10,1h} & 59.89 (2.27) &  &  &  \\
    \cline{2-3}
    & \Lx{AS50,1h} & 56.80 (1.15)  &  &  &  \\
    \cline{2-3}
    & \Lx{ASmax,1h} & 71.47 (4.81) &  &  &  \\
    \bottomrule
\end{tabularx}
\end{table*}

\subsection{NICU soundscape assessment}

\iftodo
\begin{itemize}[leftmargin=*]
  \checkedbox{current practices and measurements used to assess noise in the NICU}
  \checkedbox{limitations of traditional SPL measurements}
\end{itemize}
\fi

Unlike established standards for acoustic characterisation of performance spaces \citep{ISO3382-1} and open-plan offices \citep{ISO3382-3}, as well as airborne \citep{ISO16283-1}, impact \citep{ISO16283-2} and facade sound insulation \citep{iso12913-3}, there is currently a lack of internationally recognised standards for acoustic measurements in healthcare institutions. This absence of established measurement guidelines, including accuracy and placement of acoustic measurement equipment, adds to the ambiguity surrounding sound level guidelines specific to NICUs \citep{Philbin2017}. 

In the literature, NICU sound levels are commonly measured in the centre of the room \citep{Darcy2008,Lahav2015,Parra2017}, which may not accurately represent the sound levels experienced at the ears of the neonates, as the sound sources are typically situated near their head-area \cite{Darbyshire2019}. To estimate the sound levels experienced by neonates more effectively while minimising disruption to clinical operations, some studies have placed sound level meters (SLMs) \citep{Krueger2007,Romeu2016,Shoemark2016}, noise dosimeters \citep{Parra2017,Smith2018}, or microphones \citep{Park2017neonatal,Bertsch2020,Reuter2023} near the neonates' head area at varying distances. However, due to the dynamic nature of the critical care environment, the sound field surrounding the measurement instruments may inadvertently be influenced by reflective surfaces and proximity to noise sources, such as respirators and CPAP machines. Consequently, the most accurate measurement of neonates' noise exposure can be obtained by measuring sound levels very close to the opening of their ears or within the ear canal itself on actual patients \citep{Surenthiran2003} or on neonate simulators \citep{Bertsch2020,Reuter2023}. 

 Regarding measurement accuracy, traceable calibration to international standards ensures repeatability and reliability of acoustic measurements. This involves adhering to standards such as IEC~61672\nobreakdash-1 classification for SLMs and noise dosimeters \citep{IEC6167212013}, as well as the IEC~61094\nobreakdash-1 and IEC~61094\nobreakdash-4 standards for laboratory and working standard measurement microphones, respectively \citep{IEC61904-1,IEC6109441996}. Achieving similar accuracy with SLMs and dosimeters in the measurement of SPLs in a random-incidence field requires the use of measurement microphone diaphragms that are as small as possible ($\le\SI{13}{\milli\meter}$) or up to \SI{26}{\milli\meter} with random-incidence correction \citep{ISO3382-1}.

\subsection{Research questions} \label{sec:rq}

To address the gaps in determining sensible acoustic guidelines for NICUs, we investigated the microphone placements through a long-term measurement campaign in an operational neonatal unit. Specifically, the following research questions were addressed:
\begin{itemize}
    \item[RQ1] Do different microphone positions influence the measurement of noise levels in neonatal critical care environments?
    \item[RQ2] What role do bed position and ward layout play in shaping neonatal noise exposure in critical care units?
    \item[RQ3] How suitable are additional (psycho)acoustic metrics for assessing the complex sound environment in neonatal critical care?
\end{itemize}

\iftodo
\begin{itemize}[leftmargin=*]
  \checkbox{novelty and significance of measuring and analyzing sound at the ears of a neonate} 
  \checkbox{no self-noise from infant by using the simulator doll}
  \checkbox{expanded analysis on acoustic indices to analyse low-frequencies (C-weighting), Tonality indices (e.g. regression with SPL indices to understand contribution/correlation with alarms), OR(N) to understand frequency of threshold exceeded for loud events}
\end{itemize}
\fi