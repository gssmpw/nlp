\section{Preliminaries}\label{sec:prelims}


We follow the common convention of representing a dataset $D = \{x_1, \ldots, x_n\}$, whose elements are from the domain $\cX$, by the corresponding histogram vector $\bh^D \in \Z_{\ge 0}^{\cX}$, 
where $\bh^D_x$ is $|\{ i \in [n] : x_i = x \}|$, as well as interpreting $f: \cX \to \{0, 1\}$ as a vector in $\{0,1\}^{\cX}$. 
It is immediate to see that
$
f(D) = \ang{\bh^D, f} = \sum_{x \in \cX} h^D_x \cdot f(x)
$.
For simplicity, we use $\bh$ to denote the dataset itself and skip the superscript $D$, and we use $f(\bh)$ to denote $\ang{\bh, f}$. For any $S \subseteq \cX$, we use $\ind_S : \cX \to \{0, 1\}$ to denote the function $\ind_S(x) = \mathds{1}\{x \in S\}$. And so, $\ind_S(\bh) = \sum_{x \in S} h_x$. And we use $\bh|_S$ to denote the histogram $\bh' \in \Z_{\ge 0}^{\cX}$ with $h_x' = h_x \cdot \ind_S(x)$.

We consider mechanisms $\cB:\Z_{\ge 0}^{\cX}\mapsto \R_{\ge 0}^{\cX}$ that generate synthetic histograms $\hat{\bh} = \cB(\bh)$, from which the estimates $(\hat{e}_f := f(\hat{\bh}))_{f \in \cF}$ can be derived. We refer to such mechanisms as 
{\em synthetic data generators}.
Note that, we allow the synthetic histogram to be real-valued (instead of integer-valued). 
 In what follows, we will not distinguish between a synthetic histogram and a synthetic dataset, for the following reason.
 If a (nonnegative real-valued) histogram is relatively accurate with respect to a query family, sampling a dataset i.i.d.~from the probability distribution induced by the histogram will also be relatively accurate, with slightly worse parameters. We omit these details, but they are implicit in the proof of \Cref{prop:existence_approx_histogram}.

\paragraph{Differential Privacy.}
Two data histograms $\bh, \bh' \in \Z_{\ge 0}^{\cX}$ are said to be {\em adjacent} if
$\|\bh\|_1 = \|\bh'\|_1$ and $\|\bh - \bh'\|_1 = 2$; 
this is equivalent to replacing some record in the dataset corresponding to $\bh$ by a different record in the dataset corresponding to $\bh'$. We view randomized mechanisms $\cM$ as mapping datasets $\bh$ to a random variable $\cM(\bh)$ over some output space $\cO$.

\begin{definition}[{\boldmath $(\eps, \delta)$-DP}]
    A randomized mechanism $\cM$ with output space $\cO$ satisfies \emph{$(\eps, \delta)$-DP} (referred to as \emph{approximate-DP}) if, for all (measurable) events $E\subseteq \cO$, and for all adjacent $\bh, \bh'$ it holds that
    $
    \Pr[\cM(\bh) \in E] \le e^\eps \cdot \Pr[\cM(\bh') \in E] + \delta
    $.
    The special case of $\delta = 0$ is denoted as $\eps$-DP (referred to as \emph{pure-DP}).
\end{definition}



\begin{algorithm}[t]
\small
\caption{\textsc{$\OutsideIntervalMonitor^{\mathrm{apx}}_{\bh, a, \cY}$ (Approximate-DP version)}}
\label{alg:outside-thresholds}

\textbf{Initialization:}
{\footnotesize $\bullet$ } $\bh \in \Z_{\ge 0}^{\cX}$ : private histogram \qquad \qquad 
{\footnotesize $\bullet$ } $a > 0$ : noise parameter\\
\phantom{\textbf{Initialization:}\,}
{\footnotesize $\bullet$ } $\cS_{\acti} \gets \cY$ : initial active set (for $\cY \subseteq \cX$)\\[-2mm]

\textbf{On input} $(f: \cX \to \{0, 1\}, \tau_\ell, \tau_u \in \R_{\ge 0})$ {\bf :}\\[-2mm]

$\hf \gets f(\bh|_{\cS_{\acti}}) + \Lap(1/a)$\;

\If{{\em $\tau_\ell < \hf < \tau_u$}} {
    \Return{\Between}\;
}
\Else{
    ${\cal S}_{\acti} \gets {\cal S}_{\acti} \smallsetminus f^{-1}(1)$\;\algcomment{Note: $\Sactive$ is a persistent state across inputs.}\\
    \If{$\hf \ge \tau_u$} {
        \Return{\Above}
    } \Else {
        \Return{\Below}
    }
}
\end{algorithm}

We use the compositional properties of DP and the well-studied Laplace mechanism; we include the details in \Cref{sec:basic_DP_properties} for completeness.
%

\paragraph{$\OutsideIntervalMonitor$.}
We use the $\OutsideIntervalMonitor$, an iterative mechanism that determines if a given statistical query evaluates to a value within a specified interval or falls above or below it. The key property is that the privacy cost of the entire mechanism on a sequence of queries only degrades with the number of times the value is {\em not} in the interval. We use this technique in two flavors: one that satisfies approximate-DP (\Cref{alg:outside-thresholds}) and other that satisfies pure-DP (\Cref{alg:outside-thresholds-pure-dp}). The analysis of the pure-DP version follows from the standard analysis of the so-called {\em sparse vector technique (SVT)} (\Cref{sec:OIM_pure_DP}). The approximate-DP version is analyzed using the target charging technique~\citep{Cohen:2023} and individual privacy accounting (\Cref{app:TCT}).


\begin{proposition}[{\boldmath $\OutsideIntervalMonitor$ (Approx-DP) guarantees}]\label{prop:oim-approx-dp} 
For any $0< a\leq 1$, $\delta>0$ and integer $R > 0$, $\OutsideIntervalMonitor^{\mathrm{apx}}_{\bh, a, \cY}$ (\Cref{alg:outside-thresholds}), after $R$ rounds of queries, satisfies $(\eps, \delta)$-DP for $\eps = O\left(a \log \frac{1}{\delta}\right)$.  Let $\beta > 0$, $C := \frac1a \log \frac{R}{\beta}$, and $\Sactive$ be the state of the set before the query is made.  Then, with probability at least $1-\beta$, on any query $(f, \tau_\ell, \tau_u),$
\begin{itemize}[topsep=3pt,itemsep=-3pt]
    \item If $\Between$ is returned, then $\tau_\ell - C \le f(\bh|_{\cS_{\acti}}) \le \tau_u + C$.
    \item If $\Above$ is returned, then $f(\bh|_{\cS_{\acti}}) \ge \tau_u - C$.
    \item If $\Below$ is returned, then $f(\bh|_{\cS_{\acti}}) \le \tau_\ell + C$.
\end{itemize}
\end{proposition}
