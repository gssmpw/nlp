\newcommand{\bin}{\mathrm{bin}}
\newcommand{\rval}{\mathrm{real}}
\newcommand{\bha}{\bh^{\ast}}

\section{From Binary Queries to Real-Valued Queries}

\label{app:real-v-bin}

We refer to queries of the form $f: \cX \to \{0, 1\}$ as \emph{binary queries} and those of the form $f: \cX \to [0, 1]$ as \emph{real-valued queries}. (Sometimes these are referred to as \emph{counting queries} and \emph{linear queries}, respectively, in literature.) In this section, we prove the following theorem, which translates the accuracy guarantee for binary query families to real-valued query families in a black box manner with almost no additional overhead.

\begin{theorem}\label{thm:bin-to-real}
Suppose that for any binary query family $\cF_{\bin} \subseteq \{0, 1\}^{\cX}$, there exists a synthetic data generator $\cB_{\bin} : \cX^* \to \cZ_{\ge 0}^{\cX}$ that is $(\zeta, \alpha_{\bin}, \beta, \cF_{\bin})$-accurate, for some error bound $\alpha_{\bin} := \alpha_{\bin}\prn{\eps, \delta, n, |\cF_{\bin}|, |\cX|, \zeta, \beta}$. Then, for any $\zeta \leq 0.5$, there is a synthetic data generator $\cB_{\rval} : \cX^* \to \cZ_{\ge 0}^{\cX}$ for any real-valued query family $\cF_{\rval} \subseteq [0, 1]^{\cX}$ such that it is $(\zeta, \alpha_{\rval}, \beta, \cF_{\rval})$-accurate for
\[\textstyle
\alpha_{\rval} := O\left(1 + \alpha_{\bin}\left(\eps, \delta, n, O\left(\frac{|\cF_{\rval}| \cdot \log n}{\zeta}\right), |\cX|, \frac\zeta{10}, \beta\right)\right).
\]
\end{theorem}

\begin{proof}
In the argument below, we will translate each real-valued query to binary queries using ``thresholding'' construction. For this purpose, for every real-valued query $f: \cX \to [0, 1]$ and $\tau \in \R$, let $f_{\geq \tau}: \cX \to \{0, 1\}$ be defined by
\begin{align*}
f_{\geq \tau}(x) :=
\begin{cases}
1 & \text{ if } f(x) \geq \tau, \\
0 & \text{ otherwise,}
\end{cases}
\qquad \forall x \in \cX.
\end{align*}

Let $\cF_{\rval}$ be any family of $k$ real-valued queries. The algorithm $\cB_{\rval}$ works as follows:
\begin{itemize}
\item Let $\zeta' = 0.1\zeta, L := \lceil \log_{1+\zeta'} n \rceil + 1$ and, for all $i \in \{0, \dots, L\}$, let $\tau_i := \frac{1}{(1 + \zeta')^i}$. For convenience, also let $\tau_{L+1} = 0$.
\item Let $\cF_{\bin}$ be the set of binary queries defined by $\cF_{\bin} := \{f_{\geq \tau_i} \mid i \in \{0, \dots, L+1\}, f \in \cF_{\rval}\}$.
\item Run $\cB_{\bin}$ with the query family $\cF_{\bin}$ and relative approximation parameter $\zeta'$ (with the other parameters remaining the same) to produce a synthetic data $\hbh$.
\item Output $\hbh$.
\end{itemize}
Note that $|\cF_{\bin}| = |\cF_{\rval}| \cdot \left(L + 1\right) = O\left(\frac{k \log n}{\zeta}\right)$. Thus, $\cB_{\bin}$ is $(\zeta', \alpha', \beta, \cF_{\bin})$-accurate for $\alpha' = \alpha\left(\eps, \delta, n, O\left(\frac{k \log n}{\zeta}\right), |\cX|, \zeta', \beta\right)$. That is, with probability $1 - \beta$, we have that for all $f \in \cF$ and $i \in \{0, \ldots, L+1\}$,
\begin{align} \label{eq:acc-bin}
(1 - \zeta') \cdot f_{\geq \tau_i}(\bh^{\ast}) - \alpha' \leq  f_{\geq \tau_i}(\hbh) \leq (1 + \zeta') \cdot f_{\geq \tau_i}(\bh^{\ast}) + \alpha'.
\end{align}
We will assume that this event holds for the rest of the proof and analyze the accuracy of $\cB_{\rval}$. Consider any $f \in \cF$. Let $\tf: \cX \to \R$ be defined as $\tf(x) := \sum_{i=0}^L (\tau_i - \tau_{i + 1}) f_{\geq \tau_i}(x)$. We begin by proving the following lemma:
\begin{lemma}
For any $\bh \in \R_{\geq 0}^{\cX}$, we have 
\begin{align} \label{eq:real-to-bin-sandwich}
\tf(\bh) \leq  f(\bh) \leq (1 + \zeta') \tf(\bh) + \tau_L \cdot\|\bh\|_1.
\end{align}
\end{lemma}
\begin{proof}
For every $x \in \cX$, let $i(x) \in \{0, \dots, L+1\}$ denote the smallest index $i$ such that $f(x) \geq \tau_i$. We can bound $f(\bh)$ from below as follows:
\begin{align*}
f(\bh) &\textstyle= \sum_{x \in \cX} h(x) f(x) \\
&\textstyle\geq \sum_{x \in \cX} h(x) \tau_{i(x)} \\
&\textstyle= \sum_{x \in \cX} h(x) \sum_{i = i(x)}^{L} (\tau_{i(x)} - \tau_{i(x) + 1}) \\
&\textstyle= \sum_{i=0}^L (\tau_i - \tau_{i + 1}) \sum_{x \in \cX} h(x) \cdot \ind[i \geq i(x)] \\
&\textstyle= \sum_{i=0}^L (\tau_i - \tau_{i + 1}) \cdot f_{\geq \tau_i}(\bh) \\
&\textstyle= \tf(\bh).
\end{align*}
For the upper bound, note that our definition of $i(x)$ implies that $f(x) \leq (1 + \zeta') \tau_{i(x)} + \tau_L$. This implies that
\begin{align*}
f(\bh) &\textstyle~=~ \sum_{x \in \cX} h(x) f(x)
\textstyle~\leq~ \sum_{x \in \cX} h(x) \left((1 + \zeta') \tau_{i(x)} + \tau_L\right)
\textstyle~=~ (1 + \zeta') \tf(\bh) + \tau_L \cdot \|\bh\|_1.
\end{align*}
\end{proof}
We now continue with our accuracy analysis using the above lemma together with \eqref{eq:acc-bin}. We note here that $\|\bh\|_1 = \langle\bh, f_{\geq \tau_{L+1}}\rangle$. Note also that, since $\tf$ is a convex combination of $f_{\geq \tau_i}$'s, \eqref{eq:acc-bin} implies that
\begin{align} \label{eq:acc-lin-comb}
(1 - \zeta') \cdot \tf(\bh^{\ast}) - \alpha' \leq  \tf(\hbh) \leq (1 + \zeta') \cdot \tf(\bh^{\ast}) + \alpha'.
\end{align}
Also note that $\tau_{L} \cdot n \le 1$. With this ready, we can upper bound $f(\hbh)$ as follows:
\begin{align*}
f(\hbh) &\overset{\eqref{eq:real-to-bin-sandwich}}{\leq}  (1 + \zeta') \tf(\hbh) + \tau_L \cdot \|\hbh\|_1 \\
&\!\!\!\overset{\eqref{eq:acc-lin-comb}, \eqref{eq:acc-bin}}{\leq}  (1 + \zeta') \left((1 + \zeta') \cdot \tf(\bha) + \alpha'\right) + \tau_L \cdot \left((1 + \zeta') \|\bha\|_1 + \alpha'\right) \\
&\leq (1 + \zeta) \tf(\bha) + 3\alpha' + 1 \\
&\overset{\eqref{eq:real-to-bin-sandwich}}{\leq} (1 + \zeta) f(\bha) + 3\alpha' + 1.
\end{align*}
Finally, it can be similarly lower bounded as
\begin{align*}
f(\hbh) &\overset{\eqref{eq:real-to-bin-sandwich}}{\geq}  \tf(\hbh) \\
&\overset{\eqref{eq:acc-lin-comb}}{\geq} (1 - \zeta') \cdot \tf(\bh^{\ast}) - \alpha' \\
&\overset{\eqref{eq:real-to-bin-sandwich}}{\geq} (1 - \zeta') \cdot \left(\frac{1}{1 + \zeta'} \cdot (f(\bh^{\ast}) - \tau_L \cdot n) \right) - \alpha' \\
&\geq (1 - \zeta) f(\bh^{\ast}) - \alpha' - 1. %\qedhere
\end{align*}
\end{proof} 