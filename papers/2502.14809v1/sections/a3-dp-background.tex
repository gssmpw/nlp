\section{Further Background on DP and the Target Charging Technique}

\subsection{Basic Mechanisms and Properties in DP}

\label{sec:basic_DP_properties}

A key property of DP is its closure under adaptive composition; namely for mechanisms $\cM_1$ and $\cM_2$, the adaptive composition refers to the mechanism that outputs $(a_1 = \cM_1(\bh), a_2 = \cM_2(a_1, \bh))$ (composition of $k$ mechanisms can be defined analogously).

\begin{proposition}\label{prop:dp-composition}
The adaptive composition of $k$ mechanisms, each satisfying $(\eps, \delta)$-DP, satisfies $(k\eps, k\delta)$-DP. Furthermore, for any $0<\delta'\leq 1$ it also satisfies $(\eps',k\delta+\delta')$-DP, where $\eps'=\eps[\sqrt{2k\ln(1/\delta')}+k\frac{e^{\eps}-1}{e^{\eps}+1}]$.
\end{proposition}

The Laplace distribution with parameter $b$ is supported over $\R$ and has the probability density function $P_{\Lap(b)}(x) = \frac{1}{2b} e^{-|x|/b}$. The Laplace mechanism\footnote{
Since we only use the Laplace mechanism to estimate functions which take integer values, we could also use the discrete Laplace mechanism that is supported over $\Z$ with probability mass function $P_{\mathrm{DLap}(a)}(x) = \frac{e^a-1}{e^a+1} \cdot e^{-a|x|}$.
} is defined as follows.

\begin{proposition}[Laplace mechanism]\label{prop:laplace}
For any function $g: \Z_{\ge 0}^{\cX} \to \R$ such that $|g(\bh) - g(\bh')| \le 1$ for any adjacent $\bh$, $\bh'$, the mechanism that returns $g(\bh) + \Lap(1/\eps)$ satisfies $\eps$-DP.
\end{proposition}

\subsection{$\OutsideIntervalMonitor$ with Pure DP}\label{sec:OIM_pure_DP}

We provide a pure DP version of the $\OutsideIntervalMonitor$. For this, we resort on the classical {\em sparse vector technique} for Above Threshold queries.

\begin{algorithm}[t]
\caption{$\OutsideIntervalMonitor^{\mathrm{pure}}_{\bh, a, \Gamma, \cY}$ (Pure-DP version)}
\label{alg:outside-thresholds-pure-dp}

\textbf{Initialization:}
{\footnotesize $\bullet$ } $\bh \in \Z_{\ge 0}^{\cX}$ : private histogram \qquad \qquad
{\footnotesize $\bullet$ } $a > 0$ : noise parameter \\ 
\phantom{\textbf{Initialization:}\,}
{\footnotesize $\bullet$ } $\Gamma > 0$ : maximum number of rounds when $\Above$ or $\Below$ is returned\\
\phantom{\textbf{Initialization:}\,}
{\footnotesize $\bullet$ } $\Sactive \gets \cY$ : initial active set (for $\cY \subseteq \cX$)\\
\phantom{\textbf{Initialization:}\,}
{\footnotesize $\bullet$ } $\counter \gets 0$ \qquad \qquad \qquad\qquad \qquad \qquad 
{\footnotesize $\bullet$ } $\eta \sim \Lap(2/a)$ : noise for threshold\\[-2mm]


\textbf{On input} $(f: \cX \to \{0, 1\}, \tau_{\ell},\tau_u\in\R_{\geq 0})$ {\bf :}\\[-2mm]

\If{{\em $\counter \geq  \Gamma$}}{
    \textbf{halt}
}


\If{$f(\bh|_{\Sactive}) + \Lap(4/a)\geq \tau_u+\eta$}{
    $\counter\gets \counter+1$\\
    $\eta\sim\Lap(2/a)$\\
    $\Sactive \gets \Sactive \smallsetminus f^{-1}(1)$\algcomment{Update state for future rounds before returning}\\
    \Return $\Above$\;
}
\ElseIf{$f(\bh|_{\Sactive}) + \Lap(4/a)\leq \tau_{\ell}-\eta$}{
    $\counter\gets \counter+1$\\
    $\eta\sim\Lap(2/a)$\\
    $\Sactive \gets \Sactive \smallsetminus f^{-1}(1)$\algcomment{Update state for future rounds before returning}\\
    \Return $\Below$\;
}
\Else{
    \Return $\Between$\;
}
\end{algorithm}




\begin{proposition}[$\OutsideIntervalMonitor$ (Pure-DP) guarantees]\label{prop:oim-pure-dp}
For any integer $R > 0$, the mechanism
$\OutsideIntervalMonitor^{\mathrm{pure}}_{\bh, a, \Gamma, \cY}$ (\Cref{alg:outside-thresholds-pure-dp}), after $R$ rounds of queries, satisfies $\eps$-DP for $\eps = a\Gamma$.
Let $\beta \in (0, 1)$, $C := \frac6a \log(\frac{R}{\beta})$, and $\Sactive$ be the state of the set before the query is made. Then with probability at least $1-\beta$, on any query $(f, \tau_{\ell}, \tau_{u})$,
\begin{itemize}[topsep=3pt,itemsep=-3pt]
    \item If $\Between$ is returned then $\tau_{\ell} - C \le f(\bh|_{\Sactive}) \le \tau_u + C$.
    \item If $\Above$ is returned then $f(\bh|_{\Sactive}) \ge \tau_u - C$.
    \item If $\Below$ is returned then $f(\bh|_{\Sactive}) \le \tau_{\ell} + C$.
\end{itemize}
\end{proposition}



\begin{proof}
The privacy of the algorithm follows by reduction from the sparse vector technique with \textsc{AboveThreshold} queries with a fixed threshold (e.g.~\cite{DworkR14}).
Namely, every $\Above$ query in \Cref{alg:outside-thresholds-pure-dp} corresponds to an \textsc{AboveThreshold} query with threshold 0 for the function
\[ g(x)=f(x)-\frac{\tau_u}{|\Sactive|},\]
whereas the $\Below$ query in the algorithm correspond to \textsc{AboveThreshold} query with threshold 0 for the function
\[ g(x)=\frac{\tau_{\ell}}{|\Sactive|}-f(x).\]
This reduction implies that the standard analysis of SVT with \textsc{AboveThreshold} queries yields the privacy guarantee of $(a\Gamma)$-DP.

The second part of the result follows from standard concentration of Laplace random variables with a union bound over the $R$ rounds (see \Cref{prop:Laplace_concentration}).
\end{proof}

\subsection{The Target Charging Technique}

\label{app:TCT}

We present here some of the technical background behind the target charging technique and its application to the $\OutsideIntervalMonitor$ with individual privacy accounting, \Cref{alg:outside-thresholds}.

We recall that two probability distributions $\mathbb{P},\mathbb{Q}$ are $(\eps,\delta)$-indistinguishable, denoted $A\stackrel{\eps,\delta}{\approx} B$ if for any event $E$, $\mathbb{P}[E]\leq e^{\eps}\mathbb{Q}[E]+\delta$. Hence, an algorithm $\cA$ is $(\eps,\delta)$-DP iff for all pair of neighboring datasets $D^0,D^1$, $\cA(D^0)\stackrel{\eps,\delta}{\approx}\cA(D^1)$.

\begin{definition}[\boldmath $q$-target with $(\eps,\delta)$]
Given $0< q\leq 1$ and $\eps\geq 0$, $0\leq \delta\leq 1$, 
let $Z^0,Z^1$ be probability distributions supported on a set $\cY$. We say that $\top$ is a $q$-target of $(Z^0,Z^1)$ with $(\eps,\delta)$ if there exists $p\in[0,1]$ and five probability distributions $C,B^b,E^b$ (where $b\in\{0,1\}$) such that $Z^0$ and $Z^1$ can be written as the mixtures
\begin{align*}
Z^0 &= (1-\delta)(p C+(1-p)B^0)+\delta E^0\\
Z^1 &= (1-\delta)(p C+(1-p)B^1)+\delta E^1,
\end{align*}
where $B^0$ and $B^1$ are $\eps$-indisginguishable, and $\min\{\mathbb{P}[B^0\in \top],\, \mathbb{P}[B^1\in \top]\} \geq q$.

Let $\cA:\cX^n \to \cY$ be a randomized algorithm. We say that $\top\subseteq \cY$ is a $q$-target of $\cA$ with $(\eps,\delta)$, if for any pair of neighboring datasets $D^0,D^1$, $\top$ is a $q$-target of $(\cA(D^0),\cA(D^1))$ with $(\eps,\delta)$.
\end{definition}

We illustrate the utility of the concept of $q$-targets with an example. 
For threshold-style mechanisms, outcomes are only useful when they reach a target set. Otherwise, it suffices to return a ``prior'' output $\perp$. 

\begin{lemma}[Lemma C.1 in \cite{Cohen:2023}]
\label{lem:target}
Let $\cA:\cX\mapsto\cY\cup\{\perp\}$, where $\perp\notin\cY$, be an $(\eps,\delta)$-DP algorithm. Then $\cY$ is a $\frac{1}{1+\exp(\eps)}$-target of $\cA$ with $(\eps,\delta)$.
\end{lemma}



The Target Charging Technique ensures that for adaptive sequences of private mechanisms with corresponding $q$-targets, one can release only the outcomes that reach the targets, paying for privacy accumulation only accounting for the steps where targets are reached.

\begin{theorem}[Simplification of Theorem B.4 in \cite{Cohen:2023}] \label{thm:TCT}
Let $(\cA_i,\top_i)_{i\geq 1}$ a sequence of pairs of an $\eps$-DP mechanism $\cA_i$ and set $\top_i$ that is a $q$-target of $\cA_i$ with $\eps$. Let $\tau\geq 1$.
Then, the transcript that only releases the computations where the target is reached until $\tau$ target hits occur satisfies $(\eps',\delta')$-DP with 
\[\eps'=\frac{2\tau}{q}\eps,\qquad \delta'\leq \exp(-\tau/4).\]
\end{theorem}

We conclude with the privacy analysis of $\OutsideIntervalMonitor$, which closely follows the one  for $\SVT$ in \cite{Cohen:2023}. In principle, for individual privacy accounting we would only need $\tau=1$ in the target hit counter; however, to get the desired privacy parameters from the previous result, we will choose $\tau=\Omega(\log(1/\delta))$.

\begin{proofof}{\Cref{prop:oim-approx-dp}}
For simplicity, we prove for add/remove neighboring datasets a DP guarantee. To turn this into a replace-one neighboring condition it suffices to compose the previous argument twice. 

Consider datasets $D$, $D'=D\circ\{x\}$. 
We perform a simulation-based analysis \citep{CohenSTOC:23,Cohen:2023}. Note first that the updates on the active set can be obtained by post-processing (only based on the query and the output), and therefore it suffices to focus on the privacy of the returned output. 
If any query $f$ is such that $f(x)=0$, the answers on both datasets are identically distributed, and no interaction with the data holder is needed. By contrast, if $f(x)=1$, the simulator can run the Laplace mechanism and provide its output to the simulator. 
Furthermore, upon an $\Above$ or $\Below$ output in this latter case, $x$ is removed from the active set, and then the outcome distributions are identical for any further query, therefore no further access to the data holder is needed. 

To conclude, by \Cref{lem:target}, $\Between$ is a $q$-target of the Laplace mechanism with $a$, where $q=1/[1+\exp(a)]=\Omega(1)$ (since $0\leq a\leq 1$). By \Cref{thm:TCT}, the entire transcript is $(\eps,\delta)$-DP for $\eps=O(a\log(1/\delta))$.
\end{proofof}


