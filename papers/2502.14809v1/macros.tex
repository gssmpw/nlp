\usepackage{bm,dsfont}
\usepackage{multirow}
\usepackage{booktabs}
\usepackage{nicefrac}

\usepackage{enumitem}
\setitemize[1]{leftmargin=*,itemsep=0pt}

% djhsu magic
\def\ddefloop#1{\ifx\ddefloop#1\else\ddef{#1}\expandafter\ddefloop\fi}
% mathbb e.g. \R
\def\ddef#1{\expandafter\def\csname #1\endcsname{\ensuremath{\mathbb{#1}}}}
\ddefloop ABCDFGHIJKLMNOQRTUVWXYZ\ddefloop  % exceptions: E, P, S
\DeclareMathOperator*{\E}{\mathbb{E}}
% mathcal e.g. \cA
\def\ddef#1{\expandafter\def\csname c#1\endcsname{\ensuremath{\mathcal{#1}}}}
\ddefloop ABCDEFGHIJKLMNOPQRSTUVWXYZ\ddefloop  % exceptions:
% bold e.g. \bA, \bx
\def\ddef#1{\expandafter\def\csname b#1\endcsname{\ensuremath{\bm #1}}}
\ddefloop ABCDEFGHIJKLMNOPQRSTUVWXYZabcdeghijklnopqrstuvwxyz\ddefloop  % exceptions: f, m
% hat e.g. \hc, \hN
\def\ddef#1{\expandafter\def\csname h#1\endcsname{\ensuremath{\hat{#1}}}}
\ddefloop Ncefghnu \ddefloop  % Can define for more letters as needed.
% tilde e.g. \tg, \tO
\def\ddef#1{\expandafter\def\csname t#1\endcsname{\ensuremath{\tilde{#1}}}}
\ddefloop GOefguw \ddefloop  % Can define for more letters as needed.
% tilde{bold} e.g. \tba, \tN
\def\ddef#1{\expandafter\def\csname tb#1\endcsname{\ensuremath{\tilde{\bm #1}}}}
\ddefloop ghwz \ddefloop  % Can define for more letters as needed.

\newcommand{\bra}[1]{\left \{ #1 \right \}}
\newcommand{\prn}[1]{\left ( #1 \right )}
\newcommand{\ang}[1]{\left \langle #1 \right \rangle}
\newcommand{\sq}[1]{\left [ #1 \right ]}

%%% Custom Colors
\definecolor{Gred}{RGB}{219, 50, 54}
\definecolor{Ggreen}{RGB}{60, 186, 84}
\definecolor{Gblue}{RGB}{72, 133, 237}
\definecolor{Gyellow}{RGB}{247, 178, 16}
\definecolor{ToCgreen}{RGB}{0, 128, 0}
\definecolor{myGold}{RGB}{231,141,20}
\definecolor{myBlue}{rgb}{0.19,0.41,.65}
\definecolor{myPurple}{RGB}{175,0,124}

%%% ToDos
\providecommand{\Comments}{0}
\ifnum\Comments>0
    \usepackage[colorinlistoftodos,prependcaption,textsize=scriptsize]{todonotes}
    %\setlength{\marginparwidth}{4.5cm}
    \paperwidth=\dimexpr \paperwidth + 2cm\relax
    \oddsidemargin=\dimexpr\oddsidemargin + 1cm\relax
    \evensidemargin=\dimexpr\evensidemargin + 1cm\relax
    \marginparwidth=\dimexpr\marginparwidth + 3.6cm\relax
\else
    \usepackage[disable]{todonotes}
\fi
\newcommand{\mytodo}[1]{\ifnum\Comments<2{#1}\fi}
\newcommand{\mytodoTwo}[1]{\ifnum\Comments<3{#1}\fi}
\newcommand{\mytodoThree}[1]{\ifnum\Comments<4{#1}\fi}
\newcommand{\mynote}[1]{\mytodo{{\color{Gred}{#1}}}}
\newcommand{\todoinline}[1]{\ifnum\Comments<4\todo[inline,linecolor=Gred,backgroundcolor=Gred!25,bordercolor=Gred]{#1}\fi}
\newcommand{\badih}[1]{\todo[linecolor=Gred,backgroundcolor=Gred!25,bordercolor=Gred]{Badih: #1}}
\newcommand{\cristobal}[1]{\todo[linecolor=Gred,backgroundcolor=Gred!25,bordercolor=Gred]{Cristobal: #1}}
\newcommand{\pasin}[1]{\todo[linecolor=Gblue,backgroundcolor=Gblue!40,bordercolor=Gblue]{Pasin: #1}}
\newcommand{\pritishimp}[1]{\mytodoThree{\todo[linecolor=Gred,backgroundcolor=Gred!25,bordercolor=Gred]{Pritish: #1}}}
\newcommand{\pritish}[1]{\mytodoTwo{\todo[linecolor=myGold,backgroundcolor=myGold!25,bordercolor=myGold]{Pritish: #1}}}
\newcommand{\pritishinfo}[1]{\mytodo{\todo[linecolor=Ggreen,backgroundcolor=Ggreen!25,bordercolor=Ggreen]{Pritish: #1}}}
\newcommand{\ravi}[1]{\todo[linecolor=cyan,backgroundcolor=cyan!25,bordercolor=cyan]{Ravi: #1}}
\newcommand{\sasha}[1]{\todo[linecolor=Gblue,backgroundcolor=Gblue!40,bordercolor=Gblue]{Sasha: #1}}
\newcommand{\sushant}[1]{\todo[linecolor=Gred,backgroundcolor=Gred!25,bordercolor=Gred]{Sushant: #1}}
\newcommand{\change}[1]{\todo[linecolor=Gred,backgroundcolor=Gred!25,bordercolor=Gred]{#1}}
\newcommand{\tableoftodos}{\ifnum\Comments=1 \listoftodos[Comments/To Do's] \fi}

%%%% Other macros
\newcommand{\eps}{\varepsilon}

\newcommand{\SVT}{\mathrm{SVT}}

\newcommand{\PREM}{\mathsf{PREM}}

\DeclareMathOperator{\offdiag}{T}

\DeclareMathOperator{\reg}{\mu}

\DeclareMathOperator{\optsolunregSDP}{Q_{s,0}}

\DeclareMathOperator{\optsolregSDP}{Q_{s,\reg}}

\newcommand{\PrivateGraph}{\mathbf{w}^{\ast}}

\newcommand{\SyntheticGraph}{\widehat{\mathbf{w}}}

\newcommand{\CutSize}{H}

\newcommand{\bzero}{\mathbf{0}}
\newcommand{\bhp}{\hat{\bp}}
\newcommand{\bpsi}{\bm{\psi}}
\renewcommand{\phi}{\varphi}
\newcommand{\bphi}{\bm{\phi}}
\newcommand{\anon}{\mathrm{anon}}
\newcommand{\tot}{\mathrm{tot}}
\newcommand{\myin}{\mathrm{in}}
\newcommand{\myout}{\mathrm{out}}
\newcommand{\mystart}{\mathrm{start}}
\newcommand{\myend}{\mathrm{end}}

\newcommand{\hbg}{\hat{\bg}}
\newcommand{\obg}{\overline{\bg}}
\newcommand{\tOmega}{\tilde{\Omega}}
\newcommand{\bbw}{\bar{\bw}}
\DeclareMathOperator{\Geo}{Geo}
\DeclareMathOperator{\Lap}{Lap}
\DeclareMathOperator{\botr}{bot_r}
\DeclareMathOperator{\rnn}{\mbox{$r$}-nnd}
\DeclareMathOperator{\argmin}{argmin}
\DeclareMathOperator{\bit}{bit}
\DeclareMathOperator{\lsbit}{lsbit}
\DeclareMathOperator{\powbit}{powbit}
\DeclareMathOperator{\disc}{disc}
\DeclareMathOperator{\dist}{dist}
\DeclareMathOperator{\tdist}{\widetilde{dist}}
\DeclareMathOperator{\hdist}{\widehat{dist}}
\DeclareMathOperator{\herdisc}{herdisc}
\DeclareMathOperator{\supp}{supp}
\DeclareMathOperator{\conv}{conv}
\DeclareMathOperator{\proj}{Proj}
\DeclareMathOperator{\winnow}{\mathcal{A}_{\mathrm{winnow}}}
\DeclareMathOperator{\acti}{active}
\DeclareMathOperator{\argmax}{argmax}
\DeclareMathOperator{\equal}{\mbox{\sc Approx}}
\DeclareMathOperator{\plus}{+1}
\DeclareMathOperator{\minus}{--1}
\DeclareMathOperator{\final}{\mbox{\tiny final}}
\newcommand{\hbh}{\hat{\bh}}
\newcommand{\counter}{\mbox{counter}}
\newcommand{\OutsideThresholds}{\textsc{OutsideThresholds}}
\newcommand{\OutsideIntervalMonitor}{\textsc{RangeMonitor}}
\newcommand{\OIM}{\mathsf{RM}}
\newcommand{\FindMarginExample}{\textsc{FindMarginExample}}
\newcommand{\Above}{\textsc{Above}}
\newcommand{\Below}{\textsc{Below}}
\newcommand{\Between}{\textsc{Inside}}

\newcommand{\ind}{\mathds{1}} %{\chi} % Or \mathds{1} ?

\newcommand{\Factive}{\cF_{\acti}}
\newcommand{\Sactive}{\cS_{\acti}}
\newcommand{\Splus}{\cS^{+1}}
\newcommand{\Sminus}{\cS^{-1}}

\newcommand{\algcomment}[1]{\hfill\textcolor{black!50}{$\ldots$ #1}}

\renewcommand{\setminus}{\smallsetminus}
