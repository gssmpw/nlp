\section{Related Work}
\label{rel_work}

Our study provides a blockchain-specific analysis of developers' activities and network structures. As shown in the following, previous studies have analyzed aspects of community interaction, event impact and network dynamics. Our work, on the other hand, focuses on a set of major events analysing how these positively influence the collaborative and technical aspects of Etheruem projects, a particular example of blockchain OSS. 
\paragraph{Studies on OSS Development}
The analysis of GitHub issues and comments in OSS projects has been explored by Mumtaz et al. \cite{Mumtaz202261}, who introduce features like "assign issues to issue commenters" and analyze social dynamics within software teams. Our study extends this approach by focusing on blockchain-specific OSS projects, examining not just social interactions but also the technical contributions through commits, thereby providing a more holistic view of developer behavior. Similarly, Jamieson et al. \cite{10548732} study the decentralized web communities by considering commits and the impact of project values on team dynamics. Our work diverges by employing network analysis techniques to scrutinize the structure of developer interactions over time, offering a novel perspective on how values and events influence OSS development processes.
Santos et al. \cite{Santos2023611} focus on facilitating task selection in OSS projects through textual data analysis. Their approach to understanding developer interactions and sentiments mirrors our methodology, which also seeks to unravel the complexities of developer communication in blockchain projects. However, our study extends beyond textual analysis to incorporate statistical and network analysis techniques, providing a more multidimensional view of developer behavior.
\paragraph{OSS Blockchain-Specific Studies}
Das et al. \cite{Das2022211} conduct an empirical analysis of interactions on commits within blockchain software repositories. Our methodology differs in that we not only analyze commits but also integrate issue resolution patterns and developer network dynamics to understand the broader implications of major blockchain events. Chakraborty et al. \cite{Chakraborty2018} provide an overview of software engineering practices in blockchain projects through a survey approach. In contrast, our study applies quantitative analysis of actual project data to uncover how events specifically impact these practices.
Ortu et al. \cite{Ortu201932} provide a comparative statistical characterization of traditional software systems versus blockchain-oriented software projects, exploring potential differences using a set of ten software metrics. Our research complements their findings by digging deeper into the blockchain-specific dynamics, especially how significant events influence these metrics within OSS development environments. 

\begin{table*}[h!]
\centering
\caption{Overview of the selected Ethereum repositories, sorted by total activity.}
\label{tab:repo_stats}
\resizebox{\textwidth}{!}{%
\begin{tabular}{l l r r r r }
\hline
\textbf{Repository} & \textbf{Description} & \textbf{Commits} & \textbf{Issues} & \textbf{Comments} \\ \hline
\textit{MetaMask} &  A widely used browser extension for managing Ethereum wallets and interacting with dApps  & 21121 & 11248 & 92676 \\ 
\textit{Solidity}  &    The main programming language for writing smart contracts on Ethereum  & 24617  & 5984  & 49744 \\ 
\textit{Go-ethereum} &  The primary client for running Ethereum nodes, crucial for the core operation of the blockchain  & 15373 & 8071  & 56109 \\ 
\textit{Chainlink} & A decentralized oracle network that allows smart contracts to connect with real-world data, essential for many DeFi applications  & 24060 & 430   & 26184 \\ 
\textit{Truffle}  & A development framework for compiling, deploying, and testing smart contracts  & 16047 & 2926  & 18546 \\ 
\textit{Web3-js}  & Another library for blockchain interaction, widely used in building decentralized applications   & 4005  & 3888  & 21189 \\
\textit{Hardhat} &  Contains the specifications for Ethereum 2, focusing on the shift to proof-of-stake and scalability  & 10650 & 2548  & 15715 \\ 
\textit{OpenZeppelin} & Provides a library of secure and reusable smart contracts, widely used in the ecosystem for security and development best practices   & 3648  & 1901  & 14359 \\ 
\textit{Consensus-Specs} & Contains the specifications for Ethereum 2, focusing on the shift to proof-of-stake and scalability   & 9685  & 920   & 8164  \\ 
\textit{Ethers-js} & A lightweight library for interacting with the Ethereum blockchain, popular among developers for its simplicity  & 678  & 2634  & 13961 \\ \hline
\textbf{Total} & -  & 129884  &  40550 & 316647 \\ \hline
\end{tabular}%
}
\end{table*}



We build on their foundational work by applying a similar analytical rigor to more granular, event-driven data, enhancing our understanding of how specific blockchain events impact software development practices.
\paragraph{Network Analysis in Software Engineering}
Network analysis has been employed to study community structures and interactions within software projects \cite{bartolucci2020butterfly, ferretti2020ethereum, kleinberg2000small, la2023game, louridas2008power, lucchini2020code, potanin2005scale, valverde2003hierarchical}.
Ao et al. \cite{Ao2021Temporal} and Said et al. \cite{Said2021Detailed} utilize network science to explore the community dynamics on the Ethereum blockchain. Our research builds upon these studies by specifically focusing on how major events such as market crashes or protocol upgrades affect the temporal and structural dynamics of developer networks, employing novel statistical techniques such as motif analysis and survival analysis.
\paragraph{Event Impact Analysis}
Major events and their impacts on software development processes are highlighted by Treude et al. \cite{Treude2017Unusual}, who examine unusual activities in GitHub projects. Our study expands on this by systematically defining event windows and employing rigorous statistical methods to assess the impact of these events on multiple facets of OSS development in blockchain projects. Additionally, studies by Pejić et al. \cite{Pejić2023Analyzing} and Kapengut et al. \cite{Kapengut2022An} focus on the impacts of COVID-19 and the transition to proof of stake on GitHub activities and network concentration, respectively. 


