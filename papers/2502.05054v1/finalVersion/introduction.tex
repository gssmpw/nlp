\section{Introduction}
\label{intro}

Open-source software (OSS) projects play an important role in the development and maintenance of numerous technological infrastructures. Understanding the dynamics of these projects, particularly during significant events, can provide insights into their resilience, adaptability, and overall health \cite{malgonde2023resilience}. 
In recent years, blockchain technologies have emerged as a new frontier for OSS, presenting unique challenges and opportunities for collaborative development that extend and redefine traditional OSS paradigms \cite{Berdik2021}. 
Blockchain's convergence with OSS principles has created a distinct ecosystem with several notable characteristics. Unlike traditional OSS projects often led by a -- possibly distributed -- core team or a foundation\cite{Oreg2008Exploring}, blockchain projects adopt decentralized governance models\cite{Liu2021A}, aligning with their underlying philosophy but introducing new complexities in decision-making. Many projects incorporate native cryptocurrencies or tokens, introducing direct economic incentives for contributors and significantly impacting contribution patterns\cite{Petryk2023Impact}.Consensus-critical aspects of blockchain development are further impacted by complex technical challenges while managing community expectations, security concerns, and regulatory pressures  \cite{Chu2019The,TOUFAILY2021103444}. Major events -- such as security incidents, protocol changes, political attention, or even ordinary social media backlash are amplified by intense public scrutiny and can dramatically impact the project's trajectory\cite{Iqbal2020Asymmetric, ortu2022cryptocurrency}.
Ethereum\footnote{Ethereum is a decentralized, open-source blockchain platform that enables the creation of smart contracts, decentralized applications (dApps) \cite{Aufiero2024,ibba2024mindthedapp}, generate and exchange NFTs (non-fungible tokens), and many others. 
Ethereum's codebase is publicly available under open-source licenses, encouraging global collaboration and modification by a diverse community of developers.} serves as an ideal case study for this research due to its position as the leading smart contract platform, its significant market capitalization, and its history of major network upgrades and community-driven changes. The project's scale, complexity, and diverse ecosystem of developers and users provide a rich environment for examining OSS dynamics in the blockchain context.
Ethereum has faced numerous of the above-mentioned events throughout its history, which have tested the resilience, adaptability, and cohesion of its development community and have shaped its development path \cite{li2023development}. The dynamics and the effects of these events have, to the best of our knowledge, not been studied and understood.

In our study, we focus on analyzing the impact of major Ethereum events\footnote{A major Ethereum event represents a significant milestone or occurrence that fundamentally impacts matters belonging to at least two of the following categories: (1) ecosystem infrastructure (encompassing protocol-level changes and technical modifications); (2) market or community (including significant market reactions and community-driven changes); (3) development trajectory (covering coordinated development efforts and ecosystem-wide adaptations). Changes that affect only a single aspect of the ecosystem or require limited coordination across the network (e.g., minor bug fixes) are not classified as major. See Sec. \ref{meth} for more details.} on the Ethereum ecosystem by examining a set of ten key repositories. These repositories cover different aspects of Ethereum’s infrastructure, including core client implementations, smart contract languages, development tools, and libraries that enable interaction with the blockchain. Together, these repositories form the backbone of Ethereum's ecosystem, with interdependencies that reflect the interconnected nature of the platform. Studying them provides a view of how different components of the ecosystem respond to major events, offering insights into the overall adaptability and resilience of Ethereum's OSS community.

The nature resulting from the convergence of OSS and blockchain technologies drives us to investigate the broader impacts of ten of the major Ethereum events (described in greater detail in Sec. \ref{meth}).
By examining how these events influence team organization, developer response rates, commit activities\cite{Soto2017Analyzing}, and overall project health across key repositories, we aim to study how the Ethereum community adapts to these changes and provide insights into the ecosystem’s development dynamics\cite{Wang2019Unveiling}.
Among these metrics, commit activity serves as an important indicator, referring to the frequency of commits made by developers to a repository, capturing the level of ongoing development work and engagement within the project. 
Understanding commit activity helps identify periods of heightened or reduced activity, revealing how teams respond to major events, such as protocol upgrades or security incidents, and informing decisions about resource allocation and workload planning \cite{10.1145/3510003.3510205}.

In our study, we pose the following research questions:
\newline
\textbf{RQ 1: What is the time window during which the effects of major Ethereum events on developer collaboration in the Ethereum ecosystem are observable, before the network of collaboration returns to its normal state?}
Since our goal is to understand how specific major events have impacted the Ethereum developer collaboration ecosystem, it is essential to clearly define the \textit{before} and \textit{after} of each event, determining the time window beyond which the effects of the event are no longer observable. Through temporal analysis, we empirically established a 90-day window as the typical duration for event impacts to manifest and stabilize.
\newline
\textbf{RQ 2. How do major Ethereum events impact commit activity across key repositories in the Ethereum ecosystem?}
By analyzing fluctuations in commit activity around major events, we can identify when developers are most actively engaged or when contributions decrease. This understanding helps reveal how development efforts are coordinated, whether on a planned or voluntary basis, and provides insights into how developers respond to major changes in the ecosystem \cite{8115619}.
\newline
\textbf{RQ 3. How do major Ethereum events impact issue resolution time?}
Changes in issue resolution time indicate the efficiency and responsiveness of the development team during critical periods. This question helps assess the operational challenges faced by the team and whether developer actions are more structured (planned/allocated) or occur on a best-effort/volunteer basis \cite{1205177}.
\newline
\textbf{RQ 4. How do developer collaboration networks change during major Ethereum events?}
Examining shifts in collaboration networks reveals how team dynamics, roles, and communication patterns adapt during major events. This question is important because it helps us understand the underlying social structure of the developer community, which supports or constrains their collective response to changes in the ecosystem. We provide a complete replication package including all datasets, analysis scripts, results  at  \href{https://figshare.com/articles/journal_contribution/_b_Mining_a_Decade_of_Event_Impacts_on_Contributor_Dynamics_in_Ethereum_A_Longitudinal_Study_b_-_MSR_2025/27629559/1}{\textbf{this link}} to support reproducibility and verification \cite{Vaccargiu2025}.





