\section{Conclusion}
\label{conclusion}


Our analysis of the Ethereum ecosystem offers insights into open-source software development dynamics, particularly in contexts where community-driven development intersects with market pressures. Traditional OSS projects rarely experience such direct market influences, making it challenging to study how economic forces shape development patterns. The Ethereum ecosystem, however, provides a natural experiment where market events, technical upgrades, and community decisions create conditions similar to those faced by commercial software development teams.
The observed distinction between responses to planned technical changes and unexpected market events parallels challenges faced in commercial software development. Through our temporal analysis, we found that event impacts persist for approximately three months before activity patterns normalize, with key events like the Frontier Release and DAO hack showing the strongest effects. Our findings suggest that OSS communities, typically insulated from market pressures, can maintain development momentum through both planned and unplanned changes when proper coordination mechanisms exist.

Network analysis reveals how different types of events influence collaboration patterns, with technical transitions promoting stronger core team interactions and market events driving broader community engagement. This pattern, particularly evident in Go-ethereum's central role during both technical and market events, offers lessons about team structure during different types of organizational changes. 
Core repositories' rapid issue resolution alongside varying patterns in development tools shows OSS communities can balance market pressures with technical excellence, challenging assumptions about market versus community-driven development.








