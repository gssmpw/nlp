\begin{table}[h!]
\caption{Metrics across repositories}
\resizebox{\columnwidth}{!}{%
\begin{tabular}{c cc cc cc }
\hline
\multicolumn{1}{l}{
\textbf{Repository}}
  &
  \multicolumn{2}{c}{\textbf{Issue resol. time}} &
  \multicolumn{2}{c}{\textbf{N° of Comments}} &
  \multicolumn{2}{c}{\textbf{N° of Commits}} \\ 
 &
  \multicolumn{1}{r}{Mean} &
  \multicolumn{1}{r}{Median} &
  \multicolumn{1}{r}{Mean} &
  \multicolumn{1}{r}{Median} &
  \multicolumn{1}{r}{Mean} &
  \multicolumn{1}{r}{Median} \\ \hline
\multicolumn{1}{l}{\textbf{Overall}} & \multicolumn{1}{r}{143.8} & \multicolumn{1}{r}{16.3} & \multicolumn{1}{r}{2473.8} & \multicolumn{1}{r}{2854.0} & \multicolumn{1}{r}{1006.8} & \multicolumn{1}{r}{1078.0} \\
\multicolumn{1}{l}{\textit{MetaMask}} &
  \multicolumn{1}{r}{146.9} &
  \multicolumn{1}{r}{23.9} &
  \multicolumn{1}{r}{891.1} &
  \multicolumn{1}{r}{733.0} &
  \multicolumn{1}{r}{193.8} &
  \multicolumn{1}{r}{171.0} \\
\multicolumn{1}{l}{\textit{Solidity}} &
  \multicolumn{1}{r}{254.6} &
  \multicolumn{1}{r}{33.2} &
  \multicolumn{1}{r}{456.4} &
  \multicolumn{1}{r}{453.0} &
  \multicolumn{1}{r}{192.3} &
  \multicolumn{1}{r}{176.5} \\
\multicolumn{1}{l}
{\textit{Go-ethereum}} &
  \multicolumn{1}{r}{122.5} &
  \multicolumn{1}{r}{5.4} &
  \multicolumn{1}{r}{438.3} &
  \multicolumn{1}{r}{441.0} &
  \multicolumn{1}{r}{119.2} &
  \multicolumn{1}{r}{82.0} \\
\multicolumn{1}{l} {\textit{Chainlink}} &
  \multicolumn{1}{r}{85.8} &
  \multicolumn{1}{r}{12.7} &
  \multicolumn{1}{r}{323.3} &
  \multicolumn{1}{r}{220.0} &
  \multicolumn{1}{r}{293.4} &
  \multicolumn{1}{r}{278.0} \\
\multicolumn{1}{l}{\textit{Truffle}} &
  \multicolumn{1}{r}{116.9} &
  \multicolumn{1}{r}{24.2} &
  \multicolumn{1}{r}{178.3} &
  \multicolumn{1}{r}{175.5} &
  \multicolumn{1}{r}{160.5} &
  \multicolumn{1}{r}{141.0} \\
  \multicolumn{1}{l} {\textit{Web3-js}} &
  \multicolumn{1}{r}{127.8} &
  \multicolumn{1}{r}{29.8} &
  \multicolumn{1}{r}{178.1} &
  \multicolumn{1}{r}{165.0} &
  \multicolumn{1}{r}{33.4} &
  \multicolumn{1}{r}{20.0} \\
\multicolumn{1}{l}{\textit{Hardhat}} &
  \multicolumn{1}{r}{103.9} &
  \multicolumn{1}{r}{17.0} &
  \multicolumn{1}{r}{206.8} &
  \multicolumn{1}{r}{190.0} &
  \multicolumn{1}{r}{138.3} &
  \multicolumn{1}{r}{123.0} \\ 
\multicolumn{1}{l}{\textit{OpenZeppelin}} &
  \multicolumn{1}{r}{109.8} &
  \multicolumn{1}{r}{7.3} &
  \multicolumn{1}{r}{149.6} &
  \multicolumn{1}{r}{139.0} &
  \multicolumn{1}{r}{37.6} &
  \multicolumn{1}{r}{31.0} \\ 
\multicolumn{1}{l}{\textit{Consensus-Specs}}  & \multicolumn{1}{r}{105.3} & \multicolumn{1}{r}{14.0} & \multicolumn{1}{r}{113.4}  & \multicolumn{1}{r}{70.5}   & \multicolumn{1}{r}{134.5}  & \multicolumn{1}{r}{98.5}   \\ 
\multicolumn{1}{l}{\textit{Ethers-js}} &
  \multicolumn{1}{r}{69.1} &
  \multicolumn{1}{r}{6.0} &
  \multicolumn{1}{r}{143.9} &
  \multicolumn{1}{r}{145.0} &
  \multicolumn{1}{r}{28.2} &
  \multicolumn{1}{r}{26.0} \\ 
 \hline
\end{tabular}%
}
\label{tab:metriche}
\end{table}

\begin{figure}[t!]   
    \centering \includegraphics[scale=0.3]{images/seasonality_resolution_time.pdf}
    \caption{Analysis for the resolution times of issues}
    \label{fig: season resol time}
\end{figure}

\begin{figure}[t!]
\centering
    \centering \includegraphics[scale=0.3]{images/autocorr_commits.pdf} 
    \caption{ACF function for commits} \label{fig: ACF commits}
\end{figure}


\section{Analyzing the Temporal Impact of Significant Events on Developer Collaboration}\label{RQ1}

To understand the impact of significant events on developer collaboration, we define clear \textit{before} and \textit{after} periods for each event. This allows us to identify the time window beyond which the effects of the event are no longer noticeable. 


\textbf{Issue resolution time}, defined as the time between an issue's creation and closure \cite{ortu2015bullies} ($M = 143$, $SD = 288$, $median = 16$ days). This distribution is skewed, with most issues resolved quickly and a smaller subset remaining unresolved for extended periods.\\
\textbf{Number of comments created monthly} ($M = 2473$, $SD = 1488$, $median = 2854$ comments).\\
\textbf{Number of commits created monthly} ($M = 1006$, $SD = 463$, $median = 1078$ commits)

For each of these metrics, we compute the overall mean and median values, as well as for each individual repository, as summarized in Table \ref{tab:metriche}. The distribution of issue resolution times is consistently skewed, with the median significantly lower than the mean across all repositories. In contrast, the number of comments and commits show closer alignment between the mean and median, indicating more balanced distributions for these metrics.
To analyze the temporal patterns in these metrics, we use the median values of resolution times, comments created, and commits on a monthly basis. This approach helps account for any potential seasonality in the data. We selected the median over the mean due to the skewed distribution of resolution times, as the median provides a more robust measure of central tendency and is less affected by outliers or extreme values. 

Fig. \ref{fig: season resol time} presents the seasonal decomposition of issue resolution times into trend, seasonality, and residuals. The original time series (top panel) shows raw data from 2015-2024, with notable variations in median resolution times. The trend component (second panel) reveals long-term patterns, while the seasonal component (third panel) captures recurring cycles in resolution times throughout each year. The residuals (bottom panel) show the remaining variance after accounting for both trend and seasonal effects, with their random distribution suggesting the model has effectively captured the main temporal patterns.
The bottom section displays the residuals, which represent the remaining variance after accounting for both trend and seasonality. The residuals reflect random fluctuations in the data that are not explained by the long-term trend or seasonal patterns. The random distribution of residuals suggests that the model has captured most of the meaningful variations.
To confirm the randomness of the residuals, we compute the autocorrelation function (ACF). The ACF measures how a time series is correlated with its past values at different time lags, helping us understand the relationship between an observation at time $t$ and earlier observations at times $t - 1$, $t - 2$, and so on. Ideally, the residuals should resemble white noise — random, without discernible patterns — indicating that the model (trend and seasonality) has adequately captured the time series dynamics. If the residuals exhibit significant autocorrelation, the model has not fully captured the underlying patterns, and some structure remains.

We compute the autocorrelation function (ACF) analysis for each metric - resolution times, comments, and commits. For resolution times, we observe an inverse relationship at lag 3, while comment patterns show no statistically significant autocorrelation. 
Fig. \ref{fig: ACF commits} shows the ACF analysis for commits, where bars outside the blue confidence interval indicate statistically significant temporal correlations. The analysis reveals significant lags at 1 (positive) and 4 (negative), indicating that high commit activity in one month predicts similar activity the next month, but tends to decrease after four months. The positive autocorrelation at lag 1 suggests persistent complexity, where certain issues span multiple days or indicate development process bottlenecks. In contrast, the negative autocorrelation at lag 4 suggests that teams typically resolve longer issues or adapt processes within this timeframe, leading to decreased commit activity.
Given our dataset of 129884 commits and 40550 issues, commits provide the most detailed measure of developer activity and are our primary focus. Through this ACF analysis, we define the \textit{post-event window} as the third month after an event, when the final effects on commits are observed, with normal activity resuming by the fourth month. For consistency, we consider the 90 days before an event the \textit{pre-event} and the 90 days immediately following as the \textit{post-event}.



\begin{tcolorbox}[right=0.1cm,left=0.1cm,top=0.1cm,bottom=0.1cm]
\textbf{Answer to RQ1}: The effects of major Ethereum events on developer collaboration are observable for 90 days (approximately three months) after the event. By the fourth month, activity levels typically stabilize. 
\end{tcolorbox}