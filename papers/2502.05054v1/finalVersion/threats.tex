\section{Threats to Validity}
\label{threats}

\textbf{Construct Validity:} Primary threats involve our operationalization of key concepts. Our classification of major Ethereum events involves subjectivity, though mitigated by selecting widely acknowledged events with cross-domain impact. Using commit activity and issue resolution time as development metrics may not capture all contributions or team dynamics, while network analysis through shared comments might miss other collaboration forms. We address these limitations through multiple metrics and cross-repository validation. Repository selection bias is minimized by using objective criteria (activity levels, ecosystem roles) to ensure complete coverage of the Ethereum ecosystem.
To ensure broad impact, we define major events as those spanning at least two categories (infrastructure, market, or development trajectory). This criterion avoids overestimating the influence of isolated events. Our classification is based on historical records beyond our dataset, including Ethereum Foundation communications, network upgrades, market shifts, and security reports from 2014–2024. While a constructed definition, it provides consistency in assessing event significance. Future work could refine this approach by exploring alternative classification methods.


\textbf{Internal Validity:} Several factors could affect the relationship between events and observed developer activity. First, concurrent events or changes not included in our analysis might influence our results. We addressed this through our 90-day window analysis and by validating findings against random time periods.
Repository sizes vary significantly (from 678 to 24617 commits). However, our key findings and primary statistical inferences are drawn from the major repositories ($>15K$ commits): Go-ethereum, MetaMask, and Solidity. While we analyze medium ($5-15K$) and smaller ($<5K$) repositories for completeness, their results primarily serve to complement our main conclusions. This approach ensures our statistical inferences remain robust despite the size variations across the dataset.\\
\textbf{External Validity:} Our study focuses specifically on the Ethereum ecosystem during 2014–2024, examining how different types of events impact development patterns across its diverse repositories. While our findings provide insights into Ethereum's development dynamics, we acknowledge that these patterns are shaped by Ethereum's unique characteristics as a leading smart contract platform. Factors such as its decentralized governance, token incentives, and community-driven protocol upgrades differentiate Ethereum from traditional OSS projects.
Our repository selection spans multiple layers of the Ethereum ecosystem—from core infrastructure (Go-ethereum) to development tools (Hardhat, Truffle) and user-facing applications (MetaMask)—providing a broad view of development patterns within this blockchain platform. This diversity strengthens our findings regarding how different components of the Ethereum ecosystem respond to various types of events. However, we recognize that event-response patterns may differ in non-Ethereum projects, where governance models, incentive structures, and development workflows vary. 
Future research could explore whether similar trends hold across other blockchain and non-blockchain OSS ecosystems, particularly in projects with distinct governance mechanisms or without direct market exposure.\\ 
\textbf{Conclusion Validity:} We ensured statistical reliability by selecting appropriate tests based on data distributions, calculating effect sizes, and conducting multiple complementary analyses. Our survival and network analyses are based on assumptions of censoring independence and comment co-occurrence as a proxy for collaboration. To address multiple comparisons, we applied the Benjamini-Hochberg procedure to control the false discovery rate (FDR).  
Data completeness may be influenced by GitHub API limitations, particularly for older events. However, to support validation and reproducibility, we provide a complete replication package containing all data, code, and analysis scripts \cite{Vaccargiu2025}.


