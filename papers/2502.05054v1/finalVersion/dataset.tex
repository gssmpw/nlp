\section{Dataset}
\label{dataset}

This study examines a set of 10 repositories from the Ethereum ecosystem. These repositories were selected because they represent various core components of the platform, including client implementations, smart contract programming, developer tools, and libraries for interacting with the blockchain. The goal is to understand how different parts of the ecosystem respond to major Ethereum events.
The Selected Repositories are Go-Ethereum (Geth), MetaMask, Ethers.js, Web3.js, Truffle, Solidity, Chainlink, Hardhat, Consensus-Specs and OpenZeppelin Contracts, which are described in detail below and in Table \ref{tab:repo_stats}.
The selected repositories represent different functional components within the Ethereum ecosystem, each serving distinct purposes while maintaining operational independence. While there are theoretical interdependencies in terms of functionality - for instance, developer tools like Truffle and Hardhat need to adapt to changes in Go-Ethereum specifications - the actual development processes and teams operate independently. Although Truffle has been officially sunset, as announced by ConsenSys, its historical significance and the transition towards Hardhat justify its inclusion in our analysis. The transition period from Truffle to Hardhat provides valuable insights into how the community adapts to evolving tools, demonstrating how independent development teams respond to ecosystem-wide changes.

Updates to the Solidity programming language can impact tools like Ethers.js, Web3.js, and OpenZeppelin Contracts, which must stay compatible. Such interconnections mean changes in one area, like a protocol update, quickly ripple through and require adjustments across tools. 
MetaMask, a bridge between users and decentralized applications, relies on Web3.js and Ethers.js for Ethereum interactions. Changes in MetaMask or these libraries significantly impact user experience and development practices. Chainlink, providing external data to smart contracts, is also vital for projects requiring secure data feeds.
These interconnections mean that changes from major Ethereum events propagate through the ecosystem, impacting various repositories and the developer community. Examining these linked repositories provides insight into how significant changes influence Ethereum’s development and operation. This interconnected dataset enables a thorough analysis of Ethereum’s systemic responses, including transitional phases like Truffle to Hardhat.
Including repositories of varying sizes and activity levels captures a broad spectrum of responses within the Ethereum community. Larger repositories like MetaMask and Solidity represent core infrastructure, while smaller or transitioning ones, like Truffle’s shift to Hardhat, demonstrate adaptability. Analyzing both stable and evolving repositories reveals how different parts of the ecosystem respond, whether by maintaining stability or adapting to new technologies.


