\section{Impact of Major Ethereum Events on Issue Resolution Time}\label{RQ3}

To evaluate whether development efforts are planned or reactive to major events, we analyzed issue activity before and after each event. Daily issue data was collected from each repository to observe trends in issue openings and closures, focusing on shifts in resolution patterns. Kaplan-Meier estimation was used to model resolution time distributions, with event dates serving as covariates to compare resolution behavior before and after each event. Using the list of the major events from RQ2, we visualized changes in issue activity within the 1-month and 3-month timeframes identified in previous sections. Notable time-based fluctuations in issue activity across repositories highlight the importance of dynamically analyzing these patterns to capture the temporal impact of major events on resolution times.

\paragraph{Issue Opening and Resolution Near Key Events}
To examine how issue activity aligns with major Ethereum events, we plotted issue resolution times across all repositories, with each line representing an issue. Resolved issues are marked with an orange dot at the end, while unresolved issues extend in blue to the end of the observation period. Sorting issues by creation date allows us to track resolution patterns before and after specific events. Here, only the plot for MetaMask is shown (Fig.~\ref{fig:lifelines-metamask}).
Across repositories, a common trend is the spike in issue resolutions around major events. In MetaMask, we observe notable resolution activity around the Beacon Chain Launch (event 5) and the Arrow Glacier Update (event 7). Similar patterns are seen in other repositories. For instance, Go-Ethereum shows increased resolution activity near key events and an additional spike in older issue resolutions before 2019. Solidity also displays surges around the Beacon Chain Launch (event 5), The Merge (event 9), and the Shanghai Upgrade (event 10). 

\begin{figure}
    \centering
    \includegraphics[width=\linewidth]{images/pdf2/metamask-extension-issues-lifelines-1.pdf}
    \caption{MetaMask's issue resolution and events}
    \label{fig:lifelines-metamask}
\end{figure}

For the other repositories, similar resolution spikes are evident. For example, Web3.js, Hardhat, and Ether.js show increased resolution activity around the Shanghai Upgrade (event 10), while the Arrow Glacier Update (event 7) coincides with significant resolution efforts in Ether.js, OpenZeppelin, and Chainlink. The Merge (event 9) also influences activity in Hardhat and Chainlink.
While not all resolution spikes directly correspond to major events—some relate to each repository’s unique milestones—these findings illustrate a general pattern of heightened issue resolution, often in anticipation of or response to major Ethereum events. 

\paragraph{Characterizing Issue Resolution Through Survival Analysis}
We applied the Kaplan-Meier estimator to analyze resolution times for both open and closed issues, using the last collection date as the endpoint for unresolved issues. Fig.~\ref{fig:survival} shows the survival curve for Go-Ethereum, our most efficient repository, resolving 40\% of issues within days and 75\% within a year. Other repositories show varying patterns: MetaMask and Solidity resolve 50--60\% within a year, while OpenZeppelin and Consensus-Specs match Go-Ethereum's efficiency. Truffle, Ethers.js, and particularly Hardhat show slower resolution times. The anomaly around 400 days in Go-Ethereum suggests systematic closure of backlogged issues.

\begin{figure}
    \centering
    \includegraphics[width=\linewidth]{images/pdf2/go-ethereum-issues-survival-1.pdf}
    \caption{Survival analysis of issue resolution times in Go-Ethereum. The line represents the probability of issue resolution after each day since the issue was opened.}
    \label{fig:survival}
\end{figure}

\paragraph{Impact of Events on Survival Analysis}
To assess event influence on issue resolution, we split issues into groups based on their opening time relative to each event and compared their survival curves. Fig.~\ref{fig:survival-split-4} demonstrates this approach using the Crypto Boom (event 3) in MetaMask, where diverging curves indicate event impact on resolution patterns. 
We used log-rank tests to determine statistical significance of these differences, applying the Benjamini-Hochberg procedure across the 10 events for each repository, consistent with our approach in RQ2.
Table~\ref{tab:survival-events} presents the test results across all repositories. 



\begin{figure}
    \centering
    \includegraphics[width=\linewidth]{images/pdf3/metamask-extension-issues-survival-1-ev-split-4.png}
    \caption{Comparison of issue resolution patterns in MetaMask before and after the Crypto Boom event. The two lines representing issue resolution probabilities before and after the event, can highlight changes.}
    \label{fig:survival-split-4}
\end{figure}


\begin{table*}[h]
\caption{$p$ for log-rank test of survival curves when issues are split into two groups (before and after an event)}
\label{tab:survival-events}
\resizebox{\textwidth}{!}{
\begin{tabular}{lrrrrrrrrrrrr}
\hline
\textit{\textbf{Repository} } & \textbf{1 - Frontier} & \textbf{2 - DAO Hack} &  \textbf{3 - Crypto Boom}  & \textbf{4 - COVID Crash}& \textbf{5 - Beacon} & \textbf{6 - London} & \textbf{7 - Arrow Gl.} & \textbf{8 - Ropsten} & \textbf{9 - Merge} & \textbf{10 - Shanghai}\\
 \hline
\textit{MetaMask} & - & \cellcolor{lightgray}$.249$ & $<.001$ & \cellcolor{lightgray}$.067$ & \cellcolor{lightgray}$.173$ & \cellcolor{lightgray}$.067$ & \cellcolor{lightgray}$.149$ & $<.001$ & $<.001$ & $<.001$\\
\textit{Solidity} & - & \cellcolor{lightgray}$.721$ & $.003$ & \cellcolor{lightgray}$.534$ & \cellcolor{lightgray}$.534$ & \cellcolor{lightgray}$.534$ & \cellcolor{lightgray}$.534$ & \cellcolor{lightgray}$.534$ & \cellcolor{lightgray}$.534$ & \cellcolor{lightgray}$.534$ \\
\textit{Go-ethereum} & $<.001$ & $.003$ & $<.001$ & $<.001$ & $<.001$ & $<.001$ & $<.001$ & $<.001$ & $<.001$ & $<.001$ \\
\textit{Chainlink} & - & - & - & $<.001$ & $<.001$ & $<.001$ & $<.001$ & $<.001$ & $<.001$ & $<.001$ \\
\textit{Truffle} & \cellcolor{lightgray}$.171$ & $<.001$ & $<.001$ & $<.001$ & $<.001$ & $.009$ & $.004$ & $<.001$ & $.002$ & \cellcolor{lightgray}$.218$  \\
\textit{Hardhat} & - & - & - & $<.001$ & $<.001$  & $<.001$ & $<.001$ & $<.001$ & $<.001$ & $<.001$  \\
\textit{Web3-js} & $<.001$ & \cellcolor{lightgray}$.310$  & $<.001$ & $<.001$ & $<.001$ & $<.001$ & $<.001$ & $<.001$ & $<.001$ & $<.001$ \\
\textit{OpenZeppelin} & - & - & \cellcolor{lightgray}$.560$ & \cellcolor{lightgray}$.836$ & \cellcolor{lightgray}$.764$ & \cellcolor{lightgray}$.317$ & $.035$ & $<.001$ & $<.001$ & $<.001$\\
\textit{Consensus-Specs} & - & - & - & $<.001$ & $<.001$ & $<.001$ & $<.001$ & $<.001$ & $<.001$ & $<.001$ \\
\textit{Ethers-js} & - & - & \cellcolor{lightgray}$.344$ & $<.001$ & \cellcolor{lightgray}$.225$ & $.058$ & $.007$ & $<.001$ & $<.001$ & $<.001$ \\

 \hline
\end{tabular}
}
\end{table*}

While many events significantly impacted resolution times across repositories, some repositories like Solidity showed fewer significant changes, suggesting different dynamics in their issue resolution processes. For instance, Go-Ethereum displayed significant differences before and after all the events. MetaMask showed significant changes after the Crypto Boom and the last 3 events.

\begin{tcolorbox}[right=0.1cm,left=0.1cm,top=0.1cm,bottom=0.1cm]
\textbf{Answer to RQ3:} Major Ethereum events impact issue resolution times. Issue creation spikes around key events are often followed by bursts of resolutions as developers prepare for or respond to these changes. While resolution speeds vary across repositories, most issues are resolved quickly, though some repositories appear to manage the number of outstanding issues differently. Overall, issue resolution times change in a statistically significant way before and after major events for most repositories, reflecting increased activity and responsiveness. However, some repositories exhibit resilience to event effects, indicating diverse development dynamics within the ecosystem.
\end{tcolorbox}





