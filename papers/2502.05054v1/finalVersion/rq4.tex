\section{Changes in Developer Collaboration Networks During Significant Ethereum Events}\label{RQ4}




We constructed a network to capture interactions within the Ethereum developer community. Each node represents a contributor, and an edge represents a collaboration between two contributors who comment on the same issue. We focus on how collaboration networks change in response to events, so we do not track who created the issue, only who commented on it. Since our goal is to examine collaboration, the issue creator's role is not considered hierarchical, and the networks are undirected.
Across all repositories, we identified 17829 unique comment authors, with 2774 (15\%) contributing to more than one repository. Over half of these overlapping authors (1816) are concentrated in just two repositories. This indicates that each repository largely functions as a distinct network of interactions. Therefore, for this analysis, we treat each repository as an individual, self-contained network.
Let us focus on MetaMask, the largest repository by total activity. We constructed an adjacency matrix where two comment authors are connected if they commented on the same issue, with the connection strength determined by the number of shared issues. This results in an undirected, weighted network of interactions among MetaMask contributors.
The network, spanning from October 22, 2014, to August 28, 2024, consists of 6828 nodes (comment authors), all of whom have collaborated with at least one other contributor. The degree distribution of the network is highly right-skewed, following a long-tailed distribution. Most nodes have a low degree, meaning contributors engage with a small number of others by commenting on a few issues. However, a few nodes exhibit very high degrees, indicating involvement in numerous issues and collaboration with many contributors.
This distribution is typical of a scale-free network, where a few highly connected nodes (hubs) dominate \cite{barabasi2009scale}. These hubs likely represent core contributors who play a key role in MetaMask's development, while the majority of contributors participate sporadically, a common pattern in open-source communities.
Additional metrics further illustrate the network structure: the network density is .0027, indicating sparsity, and the average clustering coefficient is .0003, reflecting low local clustering \cite{newman2012communities}. The network contains 24 connected components, indicating some fragmentation.


Following Squartini et al. (2013) \cite{squartini2013early}, we analyze the network evolution by examining topological signatures in monthly intervals from January 26, 2016 to August 2024 (104 months). This monthly resolution aligns with findings from Sec.~\ref{RQ1}, where event impacts manifest within 90-day periods. We represent collaboration as symmetric, weighted networks where nodes are developers and edge weights indicate shared issue comments, yielding monthly $N \times N$ adjacency matrices $A_t$, with $(A_t)_{ij}$ representing interaction strength between developers $i$ and $j$ at month $t$ (where $t = 1, \cdots, 104$).
Beyond network size and density, we analyze higher-order topological properties through motifs, which reveal complex interaction patterns and emerging trends. We examine the relative frequency of undirected network motifs: reciprocated dyads (\(D^{\Leftrightarrow}\)), open triads (V-shape, two nodes connected to a common node), and closed triads (triangles). Using the Configuration Model (CM) as a null model, we calculate Z-scores to quantify deviations from random expectations, comparing each measured quantity $X$ to its expected value $\langle X \rangle$. 
Unlike the Erdős–Rényi model, CM preserves degree distribution while randomizing link weights.
Following findings from Sec.~\ref{RQ2}, we analyze three key events: the COVID-19 crash (March 13, 2020), London Hard Fork (August 5, 2021), and Arrow Glacier Update (December 9, 2021). We examine Z-scores of triadic motifs, which better capture complex collaboration structures than dyadic ones, during six-month windows around each event, with grey highlighting marking the three-month impact period. 
As shown in Fig.~\ref{fig: geth triadic motifs}, Go-ethereum exhibits the strongest patterns, with technical events triggering significant drops in open triads and spikes in triangles. 
While MetaMask and Solidity show similar but muted effects, the COVID-19 crash had minimal impact across repositories compared to technical events' substantial influence on core infrastructure collaboration patterns. The market crash in March 2020 had the least impact on the interaction structure. In contrast, the London Hard Fork and Arrow Glacier Update had more significant effects, particularly on closed triads, suggesting tighter collaboration during technical events. 
Go-Ethereum consistently shows higher Z-scores across both open and closed triads, indicating its role as a central hub during both financial and technical shifts, likely due to its importance in defining smart contracts. MetaMask and Geth were more reactive during market events, reflecting greater sensitivity to market conditions, while Solidity remained stable, showing its consistent role within the ecosystem.







\begin{tcolorbox}[right=0.1cm,left=0.1cm,top=0.1cm,bottom=0.1cm]
\textbf{Answer to RQ4:} The March 2020 market crash had the least impact on interaction structures, while the London Hard Fork and Arrow Glacier Update had a more significant effect, suggesting increased collaboration during technical events. Go-ethereum showed its central role during financial and technical changes, likely due to its importance in smart contract development. MetaMask and Go-ethereum were more reactive to market events, while Solidity remained stable, showing resilience in the ecosystem.
\end{tcolorbox}

\begin{figure}[h]
    \begin{minipage}{0.5\textwidth}
        \centering \includegraphics[width=\textwidth]{images/Geth_open_tryads.pdf} \subcaption{Open tryads}\label{fig: geth open tryads}
    \end{minipage}%
    \hspace{0.05\textwidth}
    \begin{minipage}{0.5\textwidth}
        \centering \includegraphics[width=\textwidth]{images/Geth_triangles.pdf} \subcaption{Triangles}\label{fig: geth triangles}
    \end{minipage}
    \caption{Go-ethereum repository. The black vertical line marks the event, while the grey shaded area represents the three months before and after the event. The blue line shows the trend of the Z-score for the triadic motifs over time} \label{fig: geth triadic motifs}
\end{figure}