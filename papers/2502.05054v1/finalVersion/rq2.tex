\section{Impact Analysis of Major Ethereum Events on Developer Commit Activity}\label{RQ2}

Understanding how major Ethereum events impact development activity is important for assessing how developer teams respond to network changes. By tracking commit activity before and after these events, we can determine whether development efforts are planned or reactive. This analysis highlights periods of increased or decreased engagement, offering a clearer view of how updates or disruptions affect the ecosystem’s overall activity.

We observed significant variations in commit activity across repositories in response to major Ethereum events. By comparing the number of commits before and after each event, we can identify whether developers tend to be more active before or after these occurrences.

For the Frontier Release, we observed that the number of commits generally decreases post-event in Go-ethereum, Solidity, and Web3.js, while MetaMask Extension remains stable. The DAO Creation and Hack led to an increase in commits for Solidity and MetaMask Extension, indicating post-event code updates. The COVID-19 Crash resulted in more commits for Go-ethereum, Hardhat, and MetaMask Extension. Notably, Hardhat showed significant increases in commit activity after the Beacon Chain Launch, The Merge, and Shanghai Upgrade.

\begin{figure}
    \centering
    \includegraphics[width=0.9\linewidth]{images/PrePostNormCommitCount.png}
    \caption{Pre vs. Post event commit counts for each repository and event}
    \label{fig:PrePostNormCommitCount}
\end{figure}

Fig. \ref{fig:PrePostNormCommitCount} highlights the changes in pre- and post-event activity that fall outside the gray confidence area of $r-value = .30$. The only event with a significant increase in commits is the Cryptocurrency Boom for Chainlink. Conversely, there is a notable decrease in commits for the Frontier Release in Solidity and Go-ethereum, and for the Beacon Chain Launch in Hardhat.

\begin{figure}
    \centering
    \includegraphics[width=0.9\linewidth]{images/EventPrePostCount.png}
    \caption{Normalized count pre (circle) and post (cross) event commits across all repositories}
    \label{fig:EventPrePostCount}
\end{figure}

Fig.~\ref{fig:EventPrePostCount} reveals varying impacts of events on commit activity. Market-driven events (COVID-19 Crash, DAO Creation) trigger post-event commit peaks, while planned technical changes (Beacon Chain Launch, The Merge, Frontier Release) show pre-event activity concentration, suggesting proactive development. Technical milestones tend to slow subsequent commit activity during protocol adaptation, whereas market events stimulate reactive development responding to economic shifts.
Repository responses varied by type: infrastructure repositories (Go-ethereum, Solidity) concentrate activity around core updates; development tools (Hardhat, Truffle) show post-upgrade compatibility spikes; market-sensitive repositories (Chainlink, MetaMask) demonstrate rapid responses to external pressures; and libraries (Web3.js, Ethers.js) maintain consistent activity with targeted increases during technical transitions.
Shapiro-Wilk tests ($p < .05$) showed non-normal distributions for all repository-event pairs, necessitating the use of non-parametric methods.

We, hence, employed the Wilcoxon signed-rank test for our Hypothesis on comparing pre- and post-event commit activity within the same repository, as samples are paired.
\begin{table*}[t]
\rowcolors{2}{gray!15}{white}
\footnotesize
\caption{Significant Results for Wilcoxon Tests after Benjamini-Hochberg Correction ($p < .05$)}
\resizebox{\textwidth}{!}{
\begin{tabular}{lcccccccccccc}
\hline
\textit{\textbf{Repository} } & \textbf{1 - Frontier} & \textbf{2 - DAO Hack} &  \textbf{3 - Crypto Boom}  & \textbf{4 - COVID Crash}& \textbf{5 - Beacon} & \textbf{6 - London} & \textbf{7 - Arrow Gl.} & \textbf{8 - Ropsten} & \textbf{9 - Merge} & \textbf{10 - Shanghai}\\
\hline
\textit{MetaMask} &  & x &  & x &  &  & x &  &  &  \\
\textit{Solidity} & x & x &  &  &  & x &  &  &  &  \\
\textit{Go-ethereum} & x &  &  & x &  &  & x &  &  &  \\
\textit{Chainlink} &  &  &  &  & x &  & x &  &  &  \\
\textit{Truffle} &  &  & x &  &  & x &  &  & x &  \\
\textit{Web3.js} & x &  &  &  & x &  &  &  &  &  \\
\textit{Hardhat} &  &  &  & x & x & x &  &  & x & x \\
\textit{OpenZeppelin} &  &  &  &  &  &  &  &  &  &  \\
\textit{Consensus-Specs} &  &  &  &  &  &  & x &  & x &  \\
\textit{Ethers.js} &  &  &  &  &  &  &  &  &  &  \\
\hline
\end{tabular}
}
\label{tab:wilcoxon-results}
\end{table*}

Table~\ref{tab:wilcoxon-results} shows significant results ($p < .05$) from the Wilcoxon tests after applying the Benjamini-Hochberg correction. Core repositories (Go-ethereum, Solidity) show significant changes during foundational events like the Frontier Release, while development tools (Hardhat, Truffle) display significant activity around major upgrades, indicating efforts to ensure compatibility. MetaMask exhibits heightened activity around security incidents and market crashes.
We calculated effect sizes (r-values) for the Wilcoxon signed-rank test (the complete effect size values list is available in our replication package).
Negative r-values were common, indicating a general decrease in commit activity after major events, suggesting a period of adjustment. The main decreases with ($r = -.86$) are for in the Web3-js repository for The DAO Creation and Hack, Beacon Chain Launch and London Hard Fork events. Further major changes are for The Merge ($r = -.83$) and Shanghai Upgrade ($r = -.82$) in Ethers-js and Frontier Release ($r = -.83$) for Go-ethereum and for solidity. 
Finally, market-related events such as Ethereum Market Crash During COVID-19 ($r = -.82$) and Beacon Chain Launch ($r = -.83$) have an interesting impact for OpenZeppelin. 

To validate our findings, we repeated the same analysis using 100 random events and reshuffled time series for the repositories. The analysis yielded 0 significant results after correction, compared to the observed significant events in Table~\ref{tab:wilcoxon-results}. This confirms that the observed effects are due to the selected major Ethereum events and not random chance.

\begin{tcolorbox}[right=0.1cm,left=0.1cm,top=0.1cm,bottom=0.1cm]
\textbf{Answer to RQ2:} 
Major Ethereum events significantly impact commit activity across key repositories, with the most significant events being the Frontier Release, The DAO Creation and Hack, and the Beacon Chain Launch. Core repositories show increased activity around foundational and security events, while development tools respond to major upgrades. 
\end{tcolorbox}
