%Version 3 October 2023
% See section 11 of the User Manual for version history
%
%%%%%%%%%%%%%%%%%%%%%%%%%%%%%%%%%%%%%%%%%%%%%%%%%%%%%%%%%%%%%%%%%%%%%%
%%                                                                 %%
%% Please do not use \input{...} to include other tex files.       %%
%% Submit your LaTeX manuscript as one .tex document.              %%
%%                                                                 %%
%% All additional figures and files should be attached             %%
%% separately and not embedded in the \TeX\ document itself.       %%
%%                                                                 %%
%%%%%%%%%%%%%%%%%%%%%%%%%%%%%%%%%%%%%%%%%%%%%%%%%%%%%%%%%%%%%%%%%%%%%

%%\documentclass[referee,sn-basic]{sn-jnl}% referee option is meant for double line spacing

%%=======================================================%%
%% to print line numbers in the margin use lineno option %%
%%=======================================================%%

%%\documentclass[lineno,sn-basic]{sn-jnl}% Basic Springer Nature Reference Style/Chemistry Reference Style

%%======================================================%%
%% to compile with pdflatex/xelatex use pdflatex option %%
%%======================================================%%

\RequirePackage{amsthm}
%%\documentclass[pdflatex,sn-basic]{sn-jnl}% Basic Springer Nature Reference Style/Chemistry Reference Style


%%Note: the following reference styles support Namedate and Numbered referencing. By default the style follows the most common style. To switch between the options you can add or remove “Numbered” in the optional parenthesis. 
%%The option is available for: sn-basic.bst, sn-vancouver.bst, sn-chicago.bst%  
 
%%\documentclass[sn-nature]{sn-jnl}% Style for submissions to Nature Portfolio journals

% \documentclass[fleqn,10pt]{wlscirep}
%%\documentclass[sn-mathphys-ay]{sn-jnl}% Math and Physical Sciences Author Year Reference Style
%%\documentclass[sn-aps]{sn-jnl}% American Physical Society (APS) Reference Style


% remove twocolumn when submitting
\documentclass[pdflatex,linenos,sn-mathphys-num]{sn-jnl}% Math and Physical Sciences Numbered Reference Style 
% \documentclass[pdflatex,twocolumn,sn-mathphys-num]{sn-jnl}
% \usepackage{geometry}
 \geometry{
 left=10mm,
 right=10mm,
 bottom=20mm,
 top=20mm,
 }
%%%% Standard Packages
%%<additional latex packages if required can be included here>

\usepackage{graphicx}%
\usepackage{multirow}%
\usepackage{amsmath,amssymb,amsfonts}%
\usepackage{amsthm}%
\usepackage{mathrsfs}%
\usepackage[title]{appendix}%
\usepackage[dvipsnames]{xcolor}%
\usepackage{textcomp}%
\usepackage{manyfoot}%
\usepackage{booktabs}%
\usepackage{algorithm}%
\usepackage{algorithmicx}% removed for rohit's algorithmic
\usepackage{algpseudocode}%
\usepackage{listings}%
%%%%
%% CUSTOM PACKAGES: Place all packages not part of standard imports here
\usepackage{soul}
\usepackage{array}
\usepackage{lmodern} % fixes font warnings
\usepackage{anyfontsize} % fixes remaining font warnings
\usepackage[T1]{fontenc}
\usepackage{enumerate}
% \usepackage[hyphens]{url}
% \hypersetup{breaklinks=true}
% \urlstyle{same}

% for adjustwidth environment
\usepackage[strict]{changepage}
% for formal definitions
\usepackage{framed}
% environment derived from framed.sty: see leftbar environment definition
\definecolor{formalshade}{rgb}{0.95,0.95,1}
\newenvironment{formal}{%
  \def\FrameCommand{%
    \hspace{1pt}%
    {\color{MidnightBlue}\vrule width 2pt}%
    {\color{formalshade}\vrule width 4pt}%
    \colorbox{formalshade}%
  }%
  \MakeFramed{\advance\hsize-\width\FrameRestore}%
  \noindent\hspace{-4.55pt}% disable indenting first paragraph
  \begin{adjustwidth}{}{7pt}%
  \vspace{2pt}\vspace{2pt}%
}
{%
  \vspace{2pt}\end{adjustwidth}\endMakeFramed%
}




%%%%%=============================================================================%%%%
%%%%  Remarks: This template is provided to aid authors with the preparation
%%%%  of original research articles intended for submission to journals published 
%%%%  by Springer Nature. The guidance has been prepared in partnership with 
%%%%  production teams to conform to Springer Nature technical requirements. 
%%%%  Editorial and presentation requirements differ among journal portfolios and 
%%%%  research disciplines. You may find sections in this template are irrelevant 
%%%%  to your work and are empowered to omit any such section if allowed by the 
%%%%  journal you intend to submit to. The submission guidelines and policies 
%%%%  of the journal take precedence. A detailed User Manual is available in the 
%%%%  template package for technical guidance.
%%%%%=============================================================================%%%%

%% as per the requirement new theorem styles can be included as shown below
\theoremstyle{thmstyleone}%
\newtheorem{theorem}{Theorem}%  meant for continuous numbers
%%\newtheorem{theorem}{Theorem}[section]% meant for sectionwise numbers
%% optional argument [theorem] produces theorem numbering sequence instead of independent numbers for Proposition
\newtheorem{proposition}[theorem]{Proposition}% 
%%\newtheorem{proposition}{Proposition}% to get separate numbers for theorem and proposition etc.

\theoremstyle{thmstyletwo}%
\newtheorem{example}{Example}%
\newtheorem{remark}{Remark}%

\theoremstyle{thmstylethree}%
\newtheorem{definition}{Definition}%

\raggedbottom
%%\unnumbered% uncomment this for unnumbered level heads


\begin{document}

\title[]{Addressing Moral Uncertainty using Large Language Models for Ethical Decision-Making}
% Addressing Moral Uncertainty [with] using Large Language Models for Ethical Decision-making

%%=============================================================%%
%% GivenName	-> \fnm{Joergen W.}
%% Particle	-> \spfx{van der} -> surname prefix
%% FamilyName	-> \sur{Ploeg}
%% Suffix	-> \sfx{IV}
%% \author*[1,2]{\fnm{Joergen W.} \spfx{van der} \sur{Ploeg} 
%%  \sfx{IV}}\email{iauthor@gmail.com}
%%=============================================================%%

\author*{\fnm{Rohit K.} \sur{Dubey}}\email{rohit.dubey@gess.ethz.ch}
\equalcont{These authors contributed equally to this work.}

\author{\fnm{Damian} \sur{Dailisan}}\email{damian.dailisan@gess.ethz.ch}
\equalcont{These authors contributed equally to this work.}

\author{\fnm{Sachit} \sur{Mahajan}}\email{sachit.mahajan@gess.ethz.ch}
\equalcont{These authors contributed equally to this work.}


% \affil*{\orgdiv{Computational Social Science}, \orgname{ETH}, \orgaddress{\street{Stampfenbachstr. 48}, \postcode{8092} \city{Zurich}, \country{Switzerland}}}

% \affil[2]{\orgdiv{Department}, \orgname{Organization}, \orgaddress{\street{Street}, \city{City}, \postcode{10587}, \state{State}, \country{Country}}}

% \affil[3]{\orgdiv{Department}, \orgname{Organization}, \orgaddress{\street{Street}, \city{City}, \postcode{610101}, \state{State}, \country{Country}}}


%%==================================%%
%% Sample for unstructured abstract %%
%%==================================%%

% \abstract{The abstract serves both as a general introduction to the topic and as a brief, non-technical summary of the main results and their implications. Authors are advised to check the author instructions for the journal they are submitting to for word limits and if structural elements like subheadings, citations, or equations are permitted.}
\abstract{\begin{abstract}


The choice of representation for geographic location significantly impacts the accuracy of models for a broad range of geospatial tasks, including fine-grained species classification, population density estimation, and biome classification. Recent works like SatCLIP and GeoCLIP learn such representations by contrastively aligning geolocation with co-located images. While these methods work exceptionally well, in this paper, we posit that the current training strategies fail to fully capture the important visual features. We provide an information theoretic perspective on why the resulting embeddings from these methods discard crucial visual information that is important for many downstream tasks. To solve this problem, we propose a novel retrieval-augmented strategy called RANGE. We build our method on the intuition that the visual features of a location can be estimated by combining the visual features from multiple similar-looking locations. We evaluate our method across a wide variety of tasks. Our results show that RANGE outperforms the existing state-of-the-art models with significant margins in most tasks. We show gains of up to 13.1\% on classification tasks and 0.145 $R^2$ on regression tasks. All our code and models will be made available at: \href{https://github.com/mvrl/RANGE}{https://github.com/mvrl/RANGE}.

\end{abstract}

}
%%================================%%
%% Sample for structured abstract %%
%%================================%%

% \abstract{\textbf{Purpose:} The abstract serves both as a general introduction to the topic and as a brief, non-technical summary of the main results and their implications. The abstract must not include subheadings (unless expressly permitted in the journal's Instructions to Authors), equations or citations. As a guide the abstract should not exceed 200 words. Most journals do not set a hard limit however authors are advised to check the author instructions for the journal they are submitting to.
% 
% \textbf{Methods:} The abstract serves both as a general introduction to the topic and as a brief, non-technical summary of the main results and their implications. The abstract must not include subheadings (unless expressly permitted in the journal's Instructions to Authors), equations or citations. As a guide the abstract should not exceed 200 words. Most journals do not set a hard limit however authors are advised to check the author instructions for the journal they are submitting to.
% 
% \textbf{Results:} The abstract serves both as a general introduction to the topic and as a brief, non-technical summary of the main results and their implications. The abstract must not include subheadings (unless expressly permitted in the journal's Instructions to Authors), equations or citations. As a guide the abstract should not exceed 200 words. Most journals do not set a hard limit however authors are advised to check the author instructions for the journal they are submitting to.
% 
% \textbf{Conclusion:} The abstract serves both as a general introduction to the topic and as a brief, non-technical summary of the main results and their implications. The abstract must not include subheadings (unless expressly permitted in the journal's Instructions to Authors), equations or citations. As a guide the abstract should not exceed 200 words. Most journals do not set a hard limit however authors are advised to check the author instructions for the journal they are submitting to.}

\keywords{Moral Uncertainty, Ethical Decision-Making, Reinforcement Learning, Large Language Models}

%%\pacs[JEL Classification]{D8, H51}

%%\pacs[MSC Classification]{35A01, 65L10, 65L12, 65L20, 65L70}

\maketitle


\definecolor{mypink}{rgb}{0.858, 0.188, 0.478}
\definecolor{ao(english)}{rgb}{0.0, 0.5, 0.0}
\newcommand\RD[1]{\textcolor{blue}{[RD] #1}}
\newcommand\response[1]{\textcolor{blue}{[Response] #1}}
\newcommand\JASV[1]{\textcolor{orange}{[JASV] #1}}
\newcommand\DD[1]{\textcolor{green}{[DD] #1}}
\newcommand\Dirk[1]{\textcolor{red}{[DH] #1}}

\newcommand{\tildeBefore}{$\sim$}
\newcommand{\tildeOnTop}{\Tilde}


\newcommand{\StateSpaceSet}{\mathcal{S}}
\newcommand{\ActionSet}{\mathcal{A}}

% Greek letter Usage:
\newcommand{\criticParams}{\theta}
\newcommand{\actorParams}{\phi}

\newcommand{\rewardShaping}{r_\mathit{shaping}}
\newcommand{\baseReward}{r_\mathit{env}}
\newcommand{\baseModel}{\pi_\mathit{base}}
\newcommand{\fineTuneModel}{\pi_\mathit{feedback}}
\section{Introduction}

Video generation has garnered significant attention owing to its transformative potential across a wide range of applications, such media content creation~\citep{polyak2024movie}, advertising~\citep{zhang2024virbo,bacher2021advert}, video games~\citep{yang2024playable,valevski2024diffusion, oasis2024}, and world model simulators~\citep{ha2018world, videoworldsimulators2024, agarwal2025cosmos}. Benefiting from advanced generative algorithms~\citep{goodfellow2014generative, ho2020denoising, liu2023flow, lipman2023flow}, scalable model architectures~\citep{vaswani2017attention, peebles2023scalable}, vast amounts of internet-sourced data~\citep{chen2024panda, nan2024openvid, ju2024miradata}, and ongoing expansion of computing capabilities~\citep{nvidia2022h100, nvidia2023dgxgh200, nvidia2024h200nvl}, remarkable advancements have been achieved in the field of video generation~\citep{ho2022video, ho2022imagen, singer2023makeavideo, blattmann2023align, videoworldsimulators2024, kuaishou2024klingai, yang2024cogvideox, jin2024pyramidal, polyak2024movie, kong2024hunyuanvideo, ji2024prompt}.


In this work, we present \textbf{\ours}, a family of rectified flow~\citep{lipman2023flow, liu2023flow} transformer models designed for joint image and video generation, establishing a pathway toward industry-grade performance. This report centers on four key components: data curation, model architecture design, flow formulation, and training infrastructure optimization—each rigorously refined to meet the demands of high-quality, large-scale video generation.


\begin{figure}[ht]
    \centering
    \begin{subfigure}[b]{0.82\linewidth}
        \centering
        \includegraphics[width=\linewidth]{figures/t2i_1024.pdf}
        \caption{Text-to-Image Samples}\label{fig:main-demo-t2i}
    \end{subfigure}
    \vfill
    \begin{subfigure}[b]{0.82\linewidth}
        \centering
        \includegraphics[width=\linewidth]{figures/t2v_samples.pdf}
        \caption{Text-to-Video Samples}\label{fig:main-demo-t2v}
    \end{subfigure}
\caption{\textbf{Generated samples from \ours.} Key components are highlighted in \textcolor{red}{\textbf{RED}}.}\label{fig:main-demo}
\end{figure}


First, we present a comprehensive data processing pipeline designed to construct large-scale, high-quality image and video-text datasets. The pipeline integrates multiple advanced techniques, including video and image filtering based on aesthetic scores, OCR-driven content analysis, and subjective evaluations, to ensure exceptional visual and contextual quality. Furthermore, we employ multimodal large language models~(MLLMs)~\citep{yuan2025tarsier2} to generate dense and contextually aligned captions, which are subsequently refined using an additional large language model~(LLM)~\citep{yang2024qwen2} to enhance their accuracy, fluency, and descriptive richness. As a result, we have curated a robust training dataset comprising approximately 36M video-text pairs and 160M image-text pairs, which are proven sufficient for training industry-level generative models.

Secondly, we take a pioneering step by applying rectified flow formulation~\citep{lipman2023flow} for joint image and video generation, implemented through the \ours model family, which comprises Transformer architectures with 2B and 8B parameters. At its core, the \ours framework employs a 3D joint image-video variational autoencoder (VAE) to compress image and video inputs into a shared latent space, facilitating unified representation. This shared latent space is coupled with a full-attention~\citep{vaswani2017attention} mechanism, enabling seamless joint training of image and video. This architecture delivers high-quality, coherent outputs across both images and videos, establishing a unified framework for visual generation tasks.


Furthermore, to support the training of \ours at scale, we have developed a robust infrastructure tailored for large-scale model training. Our approach incorporates advanced parallelism strategies~\citep{jacobs2023deepspeed, pytorch_fsdp} to manage memory efficiently during long-context training. Additionally, we employ ByteCheckpoint~\citep{wan2024bytecheckpoint} for high-performance checkpointing and integrate fault-tolerant mechanisms from MegaScale~\citep{jiang2024megascale} to ensure stability and scalability across large GPU clusters. These optimizations enable \ours to handle the computational and data challenges of generative modeling with exceptional efficiency and reliability.


We evaluate \ours on both text-to-image and text-to-video benchmarks to highlight its competitive advantages. For text-to-image generation, \ours-T2I demonstrates strong performance across multiple benchmarks, including T2I-CompBench~\citep{huang2023t2i-compbench}, GenEval~\citep{ghosh2024geneval}, and DPG-Bench~\citep{hu2024ella_dbgbench}, excelling in both visual quality and text-image alignment. In text-to-video benchmarks, \ours-T2V achieves state-of-the-art performance on the UCF-101~\citep{ucf101} zero-shot generation task. Additionally, \ours-T2V attains an impressive score of \textbf{84.85} on VBench~\citep{huang2024vbench}, securing the top position on the leaderboard (as of 2025-01-25) and surpassing several leading commercial text-to-video models. Qualitative results, illustrated in \Cref{fig:main-demo}, further demonstrate the superior quality of the generated media samples. These findings underscore \ours's effectiveness in multi-modal generation and its potential as a high-performing solution for both research and commercial applications.
\begin{figure}[!tp]
    \centering
    \includegraphics[width=\textwidth]{images/milk_models.pdf}
    \caption{a) Metrics evaluating performance of the agent on the primary goal (left-most) and sub-goals. AMULED learns the ethical task to near-perfection, although a hand-crafted shaping reward performs the best for this environment. Error bands reflect 95\% confidence intervals of the mean. b) AMULED is compared to the performance of agents prompted to act like pure moral clusters, and a "moral agent". These values are measured from 50 episodes each. c) Illustration of one of the trajectories learned by AMULED.}
    \label{fig:milk_moral}
\end{figure}

\section{Results}
We study two pertinent tasks: (1) Finding Milk and (2) Driving and Rescuing, which have been used in studying ethical decision-making frameworks~\cite{wu2018low}. These tasks serve as proxies for real-world scenarios, encompassing a broader range of states and thereby demonstrating their applicability to everyday life situations. Fine-tuning the policy using feedback helps the RL agent incorporate ethical actions without deviating too much from the primary goal, effectively balancing ethics with operational efficiency. 

\subsection{Finding Milk}\label{sec:res_findmilk}

Figure~\ref{fig:milk_moral}a shows the learning curves of an agent trained with (moral or human) feedback.
In the \texttt{FindMilk} environment (Fig.~\ref{fig:milk_moral}c), the agent is tasked with reaching the location of the milk in the shortest possible time (primary task).
However, the agent finds crying and sleeping babies on their way to the milk.
The RL agents trained on the primary goal consistently learn to traverse the grid and find the milk in the shortest time possible. 
However, without informing an agent of the ethics of the problem, it will disregard any effects of meeting babies along its path.
Introducing additional rewards $r_{cry}=1$ and $r_{sleep}=-1$ for passing through babies would then help shape the agent's behavior to satisfy the moral sub-goals.
This works really well for the \texttt{FindMilk} environment, but (as we will show in Sec.~\nameref{sec:res_driving}), in practice it is not trivial to assign the relative values of rewards for more complex tasks.

We find that fine-tuning the policy model (as an additional training layer) helps the agent incorporate ethical actions, without deviating too much from the primary task.
We also see that using the outputs of a belief model (combining the beliefs of 5 moral clusters) works even better than the synthetically generated human actions.
However, for this environment, AMULED does not consistently find an ideal path to the milk.
This is because the LLM (GPT-4o-mini), although it presents logical arguments for its actions on the basis of moral theories, sometimes fails in its spatial reasoning: for example, even if it has identified a sleeping baby to the right and it has explicitly identified that this goes against its moral goals, it will still go right because that brings it closer to the milk (even if going up also brings it closer to the milk, without passing through the baby).
When we pass a similar prompt to GPT-4o (the best OpenAI model available at the time), the LLM is better able to reason out the spatial contexts.

Because AMULED is a conglomeration of five moral clusters that guide the decisions of the agent, we can also gain insights on how it considers a diversity of moral philosophies by looking at how each individual moral cluster would guide the actions of the robot (Fig.~\ref{fig:milk_moral}b).
For the sub-goal of avoiding crying babies, we see that the "ideal" behavior is captured by the care moral cluster, which then more strongly aligns with the actions of AMULED.
Another interesting thing to note is that, when we prompt the LLM with no explicit credence values (and thus just prompt it to behave as a "moral" agent), the agent passes through more sleeping babies and fewer crying babies.
These are strikingly different results compared to how the moral clusters would act in this task.



\begin{figure}[!tbp]
    \centering
    \includegraphics[width=\textwidth]{images/drive_models.pdf}
    \caption{a) Metrics evaluating the performance of the agent on the primary goal (left-most) and sub-goals. AMULED manages the tradeoffs between its conflicting goals much better than the other baselines. Error bands reflect 95\% confidence intervals of the mean. b) Comparison of AMULED with the performance of agents prompted to act like pure moral clusters, and a "moral agent". These values are measured from 50 episodes each. c) Illustration of the \textbf{Driving and Rescuing} environment.}
    \label{fig:driving_moral}
\end{figure}

\subsection{Driving and Rescuing}
\label{sec:res_driving}

In the \texttt{Driving} environment, we train an agent to simulate autonomous driving and avoid collisions with other cars on a five-lane road (Fig.~\ref{fig:driving_moral}).
Besides other cars, the environment also has some grandmas trapped in traffic.
The agent will not factor "rescuing" grandmas (simplified as driving through the same lane as the grandma) in its decisions, unless given explicit rewards to shape its behavior.
One way of defining the shaping reward is $r_{shaping}=400r_{grandma} + 20(\mathrm{lane}_{t}==\mathrm{lane}_{t-1})$, where rescuing grandmas and staying on the lane are incentivized.
This task presents a challenge to the agent, as avoiding car collisions can conflict with the secondary goals, depending on the stochasticity of the environment.

Similar to the \texttt{FindMilk} scenario, we find that the base RL model can be trained to avoid collisions well.
Using both LLM and human feedback, the agent incurs more car collisions than the base RL agent.
On the other hand, we see here that defining the shaping reward can completely shift the agent behavior to prioritize rescuing grandmas, at the expense of excessive lane changes and collisions.
Although the synthetic human actions were also much more inclined towards rescuing grandmas, we observe that the agent does not seem to rescue as many grandmas as the feedback samples (Fig.~\ref{fig:driving_moral}a).
% By around episode 650, the RLHF approach rescues around 10 grandmas on average, but this behavior is "forgotten" by the RL agent.
The RLHF approach rescues around 10 grandmas on average while limiting the collisions to around 4.
In comparison, AMULED performs quite well at balancing the secondary goals (rescuing grandmas and remaining in its lane) against the primary goal.
Although AMULED does not save as many grandmas than the agent trained on hand-shaped rewards, it saves more grandmas compared to the agent trained on human feedback, without excessively forgetting its original driving-related tasks.

When compared with the performance of agents acting as pure moral theories, we see more variance in the sub-goal performance of each moral cluster.
Most strikingly, the consequentialist approach favors "inaction", which leads to staying more in lane and fewer rescues of grandmas than the other approaches.

\begin{figure}[!tbp]
    \centering
    \includegraphics[width=\textwidth]{images/aggregation.pdf}
    \caption{Comparison of AMULED with the performance of agents trained with alternative belief aggregation functions. These values are measured from 50 episodes each.}
    \label{fig:aggregation}
\end{figure}

\subsection{Aggregation of moral beliefs}

Because we deal with a pluralistic moral framework, a key aspect of AMULED is an aggregation mechanism that combines beliefs from each moral cluster.
Taking inspiration from multi-sensor data fusion, AMULED employs a belief aggregation system that combines the beliefs of the five moral clusters to a resultant vector of values corresponding to each action.

For AMULED, we choose an aggregation method developed by \citet{xiao2019multi} to generate the combined belief probability assignment (BPA) for each action.
We take the BPA as the shaping rewards during learning with feedback.
In principle, this is not the only way to aggregate the belief values of the different moral clusters.
For comparison (Fig.~\ref{fig:aggregation}), we look at three other aggregation methods: i) \textbf{majority vote}: each moral cluster "votes" for the action with the highest belief; then, the action with the most votes is assigned an aggregated belief value of 1; ii) \textbf{maximum belief}: the aggregated belief value of an action is the highest belief value of all moral clusters; and iii) \textbf{weighted average}: the belief value of an action is the weighted average of beliefs across all other moral clusters. For simplicity, we set the weights to be 1 (i.e., the weighted average is the mean).

For the \texttt{FindMilk} scenario, we see that AMULED and the max aggregation method have the most similar results for both sub-goals.
On the other hand, AMULED is only similar to voting, and only for the "grandmas rescued" sub-goal.
Overall, AMULED mostly achieves the sub-goals best among the aggregation approaches, except for the "lane changes" sub-goal.
% MAYBE MOVE TO DISCUSSION
Certainly, the choice of aggregation function can have varying impact on the outcomes for different types of environments.



\begin{figure}[!tbp]
    \centering
    \includegraphics[width=\textwidth]{images/llm_compare.pdf}
    \caption{Performance of AMULED using different LLMs as the moral agents. OpenAI's GPT-4o-mini is the LLM used to produce the results of the rest of the paper.}
    \label{fig:llm_compare}
\end{figure}
\subsection{Comparison with different LLMs}

The results of AMULED presented so far were obtained using OpenAI's GPT-4o-mini.
Because the language model served as a core element in guiding ethical decision making, we also compared the performance of AMULED that uses other language models (Fig.~\ref{fig:llm_compare}).
Although there is no official documentation on the exact size and architecture of GPT-4o-mini, some benchmarks put it on par with $\sim$70 Billion parameter models~\cite{livebench}.
We compared it against Mistral NeMo (12B) and LLaMa 3.1 (8B and 70B).
Overall, GPT-4o-mini performs the best for the primary and sub-goals.
The bigger LLaMa 3.1 70B also performs well for the \texttt{Driving} scenario and has a good average performance on the sub-goals of \texttt{FindMilk}, however, it fails to consistently achieve the primary goal for the \texttt{FindMilk} scenario.

% In general, one needs to balance the tradeoffs of 


\section{Mitigation Strategies}
\label{sec:mitigation}

Mitigating vulnerabilities caused by phantom events requires a comprehensive approach, addressing smart contract development, ecosystem infrastructure, and attack detection mechanisms. From the perspective of contract development, developers should implement strict validation mechanisms to ensure that event parameters are verified before emission and access control mechanisms for the functions. It is essential to enforce proper state transitions to prevent mismatches between emitted events and the actual contract state. 

At the ecosystem level, off-chain systems like blockchain explorers, wallets, and DApps must adopt more robust validation techniques to distinguish legitimate events from phantom events. Event emitter validation, where the source of the event is cross-checked with the contract address, helps ensure that events originate from authorized contracts. Furthermore, improving data sanitization processes in off-chain applications is critical to prevent vulnerabilities such as cross-site scripting (XSS) and SQL injection (SQLi). Enhanced cross-chain security protocols are necessary for cross-chain bridges, ensuring that events on both the source and destination chains are validated to prevent event forgery and manipulation.

In terms of security attack detection, continuous real-time monitoring of on-chain transactions and events is essential to detect and flag suspicious activities, such as \emph{Transfer Event Spoofing} or \emph{Contract Imitation}. Defining detailed detection rules, both for on-chain contract behavior and off-chain event handling, allows for more comprehensive identification of vulnerabilities. Additionally, regular security audits of both smart contracts and off-chain systems should be conducted to identify potential weaknesses, particularly focusing on event emission logic, access controls, and transaction validation. Through a combination of these strategies, the risk posed by phantom events can be significantly reduced, improving the security and reliability of blockchain systems.

\section{Methods} 


% \begin{figure}[!tbp]
%   \centering
%   \includegraphics[width=0.5\textwidth]{images/cluster.pdf}  
%   \caption{}
%   \label{fig:moral_cluster}
% \end{figure}


\begin{algorithm*}
\caption{AMULED Framework}\label{alg:pseudocode}
\begin{algorithmic}[1]
\Require Set of moral clusters and initial parameters $\actorParams$, $\criticParams$
\State Initialize policy $\pi_{\actorParams}$ and value function $V_{\criticParams}$ using Proximal Policy Optimization(PPO)~\cite{schulman2017proximal}
\If{Base model}
    \State $c=0$ \Comment{see Eq.~\eqref{eq:reward}}
\EndIf
\For{timesteps $t$ in $T_\mathit{horizon}$}
    \Procedure {Collect trajectories}{}

    \State Collect trajectories of state-action-reward tuples $(s_t, a_t, r_t)$ from the environment

    \If{Fine-tuning with AI Feedback}
        \State Initialize base policy $\pi_{\mathit{base}}$ from previously trained parameters
        \State Redefine reward $r_t := r_{\mathit{feedback}}=\baseReward + \rewardShaping$ new reward for fine-tuning with: \Comment{Eq.~\eqref{eq:feedback}}
        \State $\baseReward = -\lambda_{\mathit{KL}} D_{\mathit{KL}}\left(\fineTuneModel(a | s) \parallel \baseModel(a | s)\right)$
        \State $\rewardShaping = f_{\mathit{BA}}(\mathbf{B})$ \Comment{for $f_{BA}$, see Eqs. \eqref{eq:app_DMM}--\eqref{eq:app_BPA}}

        \EndIf
    \EndProcedure

    \Procedure{PPO Training Loop}{}
        \If{$(t \mod N_\mathrm{batch})=0$}
            \State \textbf{Policy Update:} $\actorParams_{k+1} \leftarrow \arg\min_{\actorParams} \mathbb{E}_{t} \left[ \frac{\pi_{\actorParams}(a | s)}{\pi_{\actorParams_{k}}(a | s)} A^{\pi_{\actorParams_{k}}}(s, a) \cdot g(\epsilon, A^{\pi_{\actorParams_{k}}}(s, a)) \right]$ \Comment{Eq. \eqref{eqn:objective}}
            
            \State \textbf{Value Function Update:} $\criticParams_{k+1} \leftarrow \arg \min_{\criticParams} \mathbb{E}_{t} \left[V_{\criticParams}(s_t) - R_t\right]^2$ \Comment{Eq.~\eqref{eqn:value_function}}
        \EndIf
    \EndProcedure
    
\EndFor
\end{algorithmic}
\end{algorithm*}

The AMULED framework develops a two-layered reinforcement learning (RL) model to balance primary and ethical objectives (see Algorithm~\ref{alg:pseudocode}). In the initial layer, the agent learns a policy for its primary task. Then, five distinct moral clusters evaluate each action’s ethical appropriateness, assigning belief values based on the environment state. This state is encoded as a text prompt and processed with cluster-specific context to inform a shaping reward. The shaping reward, blended with the environment reward, adjusts the agent’s actions towards ethical goals via reinforcement learning with feedback. In AMULED, we evaluate the effectiveness of the ethics shaping algorithm through four experiments, focusing on two key tasks (Appendix~\ref{app:scenarios}): (1) Finding Milk, where the agent performs a route planning task with additional ethical tasks, and (2) Driving and Rescuing, a more complex task involving a larger number of states that simulate realistic decision-making. In the following sections, we detail individual components of the AMULED framework. The AMULED source code will be made publicly available at \href{https://github.com/xxxxxx/moral_agent}{https://github.com/xxxxx/moral\_agent} after publication.


% Below we presents the pseudo-code of the AMULED framework. This algorithm employs PPO to iteratively update the policy and value function based on environmental feedback. The framework integrates reward shaping to balance primary and secondary objectives and incorporates fine-tuning through reinforcement learning with human-like feedback (RLHF) from moral clusters, using KL divergence and belief aggregation to guide agent behavior.


\subsection{Development of Moral Clusters}
The development of the moral clusters framework is grounded in a systematic analysis of ethical theories drawn from both classical and contemporary philosophical literature. Our objective is to construct a comprehensive yet practical structure that considers major ethical paradigms, serving as a foundation for implementing ethical reasoning in AI systems \cite{wallach2008moral, anderson2011machine}.

We began with an extensive review of ethical theories, focusing on widely recognized works that span the spectrum of moral philosophy. This includes consequentialist, deontological, virtue-based, care-oriented, and justice-focused ethical frameworks \cite{powers2006prospects}. Key sources encompass seminal texts such as John Stuart Mill's \textit{Utilitarianism}, Immanuel Kant's deontological writings \cite{alexander2007deontological}, Aristotle's \textit{Nicomachean Ethics}, Carol Gilligan's work on care ethics \cite{gilligan2014moral}, and John Rawls' \textit{[A] Theory of Justice} \cite{pogge2007john}. This comprehensive review provides an overview of foundational principles and nuances within each ethical tradition \cite{dignum2018ethics}.

\subsubsection{Cluster Identification}
Drawing from the reviewed literature, we identified five primary clusters of ethical thought: Consequentialist Ethics, Deontological Ethics, Virtue Ethics, Care Ethics, and Social Justice Ethics (Fig.~\ref{fig:schematic}c). These clusters were selected because they represent distinct approaches to ethical reasoning, encompass a broad spectrum of moral considerations, and are frequently discussed in both philosophical discourse and applied ethics \cite{tolmeijer2020implementations}. By organizing ethical theories into these clusters, we aim to capture the diversity of moral perspectives that could inform AI decision-making \cite{conitzer2017moral}. This approach aligns with recent research highlighting the importance of comprehensive ethical frameworks in AI systems, particularly when dealing with decision-making dilemmas \cite{duan2019artificial,mahajan2024executioner}.

\subsubsection{Framework Development}
For each identified cluster, we developed a structured framework (details in Appendix~\ref{app:MoralClusters}) designed to balance philosophical depth with practical applicability for AI systems \cite{dennis2016formal}. This framework includes a general description of the ethical approach, the key principles that unify theories within the cluster, representative ethical theories, key concepts inherent to each theory, and decision factors that could potentially be operationalized in an AI system. The selection of representative theories and concepts is based on their prominence in the literature and their potential for translation into computational models \cite{bench2020ethical}.
In the Consequentialist Ethics cluster, for example, we focus on the principle of maximizing overall good, with utilitarianism serving as a representative theory \cite{abel2016reinforcement}. Key concepts such as utility, consequences, and the greatest happiness principle are emphasized, and decision factors involve assessing the outcomes of actions in terms of their utility contributions. Similar detailed frameworks were developed for the other clusters, ensuring that each ethical approach is thoroughly represented and that the essential elements could be mapped to computational considerations \cite{malle2016integrating}.

The initial framework has to undergo several iterations of refinement to enhance its coherence and applicability. We critically examine the internal consistency within each cluster and ensure clear distinctions between clusters to prevent conceptual overlap. This involves verifying that the selected theories adequately represent the diversity within each ethical approach and refining key concepts and decision factors to capture the essence of each theory while remaining amenable to quantification for AI applications. 

% \subsubsection*{Limitations and Assumptions}

% While our framework aims to comprehensively represent major ethical paradigms for AI decision-making, we acknowledge that categorizing ethical theories into distinct clusters may simplify the complexities and intersections among different moral philosophies. This structured approach, however, is intended to facilitate the practical implementation of ethical reasoning in AI systems. Operationalizing these theories for AI applications involves abstracting intricate philosophical ideas \cite{russell2015research}, but we have tried to preserve the core principles of each ethical approach. The selection of representative theories and decision factors was based on their prominence and relevance in the literature, recognizing that some degree of subjective judgment is inherent in this process \cite{allen2000prolegomena}.

% Our framework is based on key assumptions. We believe that these five clusters effectively encompass the major streams of ethical thought pertinent to AI decision-making and that the selected theories within each cluster are sufficiently representative of their respective ethical approaches. We also assume that the identified key concepts and decision factors can be meaningfully translated into computational models, providing a solid foundation for future research and development in this area \cite{gips2011towards}.

\subsection{Modeling Morality as Intrinsic Reward: Belief Probability Assignment}
We draw inspiration from multi-sensor fusion literature~\cite{Chen2023}, where measurements from different sensors can be converted into some plausibility estimate of what is defective in a system.
The beliefs from a single sensor are termed Basic Belief Assignment (BBA), and combining the BBAs from multiple sensors yields a Basic Probability Assignment (BPA)~\cite{Zhao2022}.
Following this analogy, we envision the sensors as moral clusters, each with a set of probability estimates on which actions are best given the state.
We can express this mathematically as the belief $B_{i,j}(s)\iff B_{i,j}$ of agent $i$ on how good action $j$ is, given the current state $s$. 
Thus, given $M$ moral clusters acting as agents, and an environment with $A$ actions, we can form a matrix of belief values.
To make this belief matrix useful, we need a belief aggregation function $f_\mathit{BA}(\mathbf{B})\to r_j$ that maps the belief values of agents into a reward $r_j$ for each action $j$.

Multiple approaches can be used to serve as the aggregation function $f_\mathit{BA}(.)$.
For example, $f$ can be a majority-wins vote aggregation, where each agent "votes" on the action $\mathrm{vote}_i=\mathrm{argmax}_jB_{i,j}$, and the action with the most votes gets assigned $r=1$.
One could also use a maximum-belief approach by defining $\Tilde{B}_{j} = \max_i B_{i,j}$, and setting $r_j=\Tilde{B}_j/\sum_i{\Tilde{B}_i}$.
Lastly, one can use a weighted average: $r_j= \sum_i{w_i B_{i,j}} / \sum_i{w_i}$, where $w_i$ are weights of each agent. With $w_i=1$, this weighted average reduces to the mean belief of an action.
One expects these simple aggregation functions to work best when these is a strong agreement between agents on the probability values of each action.
However, problems will arise when there are discrepancies between one or more agents. 
Thus, especially when ethics is involved, there is a need to treat such discrepancies with a more nuanced belief aggregation method.

In contrast to the above aggregation methods, \citet{xiao2019multi} introduced a different aggregation method that makes use of a Belief Jensen–Shannon Divergence (BJSD) method to systematically measure the discrepancies and conflicts among BBAs, processes these measurements (details in Appendix~\ref{app:belief_fusion}), and finally employs Dempster--Schaefer theory (DST) to arrive at the BPA \citep{dempster2008upper}.
This allows for a nuanced aggregation that accounts for the uncertainties inherent in human ethics, leading to more informed and comprehensive decision-making in moral dilemmas.

In multi-sensor data fusion, combining information from diverse sources is critical, yet challenging, particularly when addressing conflicting and uncertain data. Each sensor provides valuable insights into decision-making, but also introduces its own uncertainties. Similarly, different moral clusters, such as deontology, virtue ethics, and consequentialism, can be seen as "sensors" that guide ethical decision-making. These moral perspectives can conflict with one another and carry their own uncertainties.
Treating these moral frameworks as sensors allows us to apply techniques from multi-sensor data fusion, particularly the BJSD and DST methods to effectively measure and moderate conflicts between evidence sources by incorporating both credibility and uncertainty metrics into the fusion process.
% By treating moral frameworks like sensors, we can apply methods like Belief Jensen–Shannon divergence to reconcile these complexities in human decision-making. Dempster--Shafer Theory is widely utilized due to its flexibility in modeling uncertainty without prior probabilities; however, it can produce counter-intuitive results when highly conflicting evidence arises. To address this, Xiao's approach introduces a Belief Jensen--Shannon (BJS) Divergence method \cite{xiao2019multi}, which effectively measures and moderates conflict between evidence sources by incorporating both credibility and uncertainty metrics into the fusion process.
This approach begins by computing the BJS divergence. Let $A_j$ be a hypothesis of the belief function \mbox{$m_i:=[B_{i,A_1}, B_{i,A_2},\dots B_{i,A_j}]$}, and let $m_i$ and $m_k$ be two BBAs.
%on the same frame of discernment $\Omega$, containing $N$ mutually exclusive and exhaustive hypotheses.
The BJS divergence between two BBAs is given by
\begin{equation}
BJS(m_i, m_k) = H\left(\frac{m_i+m_k}{2}\right) - \frac{1}{2} H\left(m_i\right) - \frac{1}{2} H\left(m_k\right),
\label{eq:bjs}
\end{equation}
where $H(x)$ is the Shannon entropy.
% where $
% S(m_i, m_j) = \sum_{k} m_i(A_k) \log\left(\frac{m_i(A_k)}{m_j(A_k)}\right)
% $ and $\sum_{k} m_i(A_k) = 1 (k = 1,2,3,...M)$. 
The BJS divergence quantifies the discrepancy and conflict between evidence clusters, which is then used to assign a credibility score for each evidence source, representing its reliability.
Next, belief entropy is employed to account for uncertainties within each evidence cluster. This belief entropy captures the ``volume of information'' within each evidence source, offering a relative measure of each evidence’s importance. By combining both the credibility degree and the belief entropy, Xiao’s method dynamically adjusts the weight of each evidence cluster, minimizing the impact of conflicting information. These refined weights are then integrated using Dempster’s combination rule, allowing for an adjusted belief assignment that yields more robust and interpretable fusion results. 
% A comprehensive explanation of the steps involved in Xiao's method is provided in Appendix~\ref{app:belief_fusion}, including a detailed step-by-step computation process with an example.  

Xiao's method is particularly suitable for combining human decision-making processes, especially in the context of moral dilemmas characterized by uncertainty. Human morality encompasses various frameworks (e.g., deontology, virtue ethics, and consequentialism), which reflect different values and principles. By using this method, we can incorporate multiple moral clusters, each representing different ethical perspectives. This approach supports a more nuanced and comprehensive understanding of moral decisions by mirroring the complexities of human decision-making. It enables us to account for conflicting beliefs and uncertainties inherent in human judgment, rather than relying on a single moral framework, and thus brings AI decision processes closer to how humans evaluate ethical situations in real-world contexts \cite{macaskill2014normative,10.1093/scan/nsab100}.

% Through this method, Xiao effectively mitigates the effect of highly conflicting evidence, enabling a more coherent aggregation of uncertain data from various moral clusters. Experimental applications in fault diagnosis and other areas confirm the robustness of this approach, showcasing improved performance over entropy-based methods and verifying its practical effectiveness in complex fusion scenarios.

% \subsection{Formulation}

% Let's assume a moral-decision-situation is a quintuple (S, t, A, M, C), where S is a decision-maker (i.e., agent), t is a time, A is a set of options that the decision-maker has the power to bring about at that time, M is the set, assumed to be finite, of first-order normative moral theories considered by the decision-maker at that time, and C is a credence function representing the decision-maker’s beliefs about first-order normative theories. 
% \par
% For the lack of better solution, for time-being we will assume agent's credence over multiple theories. The sum of credence values over all theories should be equal to 1. 

% The question we have to solve is - Given an action set A from $\{a_i, a_{i+1}, \ldots, a_n\}$ , time t, set credences \( C \) from $\{c_j, c_{j+1}, \ldots, c_m\}$ and theories \( M \) from $\{m_j, m_{j+1}, \ldots, m_m\}$, what is the expected choice-worthiness (EC) for each action in A. 

% \begin{equation}
%     EC(a_i,m_j,t) = ? 
% \end{equation}

% \subsection{Modelling LLM agents as human voters}
% We simulate 5 persona corresponding to the theories clusters. (This needs a dataset. See paper \url{https://arxiv.org/pdf/2404.12138}) \par
% % Step 1 - We identify five moral clusters \par.
% % Step 2 - We generate dataset for each moral theories \par
% % Step 3 - We train 5 LLM agents to adorn 5 persona with a fixed value of credence in above five moral cluster \par
% % Step 4 - Each LLM vote for which action they will elect at moral-decision making. \par
% % Step 5 - the vote will decide the utility value of each action \par

% \subsubsection{Five moral clusters}
% @Sachit @Rohit
% \subsubsection{generate dataset for each moral theories}
% @Sachit @Rohit
% \subsubsection{Fine-tune one "Super" LLM agent to learn equally on five moral clusters}
% @Damian @Rohit
% \begin{itemize}
%     \item Train an S-LLM $L(s_t | C): s_t\to a_i$ that evaluates an action given a set of credence.
%     \item chosen actions by each LLM 
%     \item Possible first approach is to use few-shot prompting/chain of thought~\cite{wei2022chainofthought}
% \end{itemize}
% \subsubsection{Before the start of RL episode, we ask S-LLM to adorn a persona which gives a certain credence value to each moral clusters. The sum of credence should be exactly equal to 1.}
% \subsubsection{Then at each RL training step, the adorn persona would vote for one action out of available options based on the states.}
% \begin{itemize}
%     \item Here, we can try different voting scenarios.
    
% \end{itemize}
% @Damian @Rohit


\subsection{Large Language Models}
We used different state-of-the-art [large] language models as the backbone of the moral agents.
Specifically, we chose the chat/instruct variants of GPT-4o-mini, Mistral nemo, and LLaMa-3.1 (8B and 70B variants).
Setting the ‘temperature’ parameter to 0 minimizes the variance in the models’ responses and increases the replicability of our results.
We structured the prompts (Fig.~\ref{fig:schematic}b) into three main blocks: the system prompt, few-shot-examples, and the actual scenario prompt.
The few-shot-examples were crafted to demonstrate the structure of a scenario prompt, and to incorporate techniques like chain-of-thought~\cite{wei2022chainofthought} to improve reasoning and model outputs.


\subsection{Deep Reinforcement Learning}
Deep Reinforcement Learning (Deep-RL) is a powerful paradigm for teaching agents to solve complex tasks by interacting with their environment.
Specifically, it solves complex tasks that can be characterized as a Markov Decision Process (MDP)~\cite{bellman1957markovian}, which is defined by the tuple $\langle \StateSpaceSet, \ActionSet, R, P \rangle$. 
Here, $\StateSpaceSet$ and $\ActionSet$ are sets of all possible states of the environment, and actions available to the agent, respectively. 
The reward function $R$ determines the immediate reward obtained by the agent after taking an action in a given state, and $P$ is the transition probability function that describes state transitions, given an action $a$.

% In multi-agent settings, agents usually lack access to the complete state representation of the system. These settings generalize to the partially-observable MDP (POMDP) problem, which uses partial observations $\mathcal{O}:= \mathcal{O}\in \StateSpaceSet$ instead of the state of the system.
Actions of an agent are selected using a policy $\pi:\mathcal{S}\to\ActionSet$, and these actions change the state of the system according to $P$.
For control problems, one common objective is to find the policy $\pi$ that maximizes the discounted expected reward $R=\sum^T_{t=0} \gamma^t r_t$, where $\gamma$ is a discount factor and $T$ is a time horizon~\cite{sutton2018reinforcement}.

\subsubsection{Reinforcement Learning Algorithm}
Specifically, we use the Proximal Policy Optimization (PPO)~\cite{schulman2017proximal} as implemented in \texttt{CleanRL}~\cite{huang2022cleanrl}, which uses neural networks to learn a policy~$\pi_\actorParams$ and a value function~$V_\criticParams$.
% PPO is an on-policy algorithm, which means that it is trained on samples gathered by the current policy.
% To be specific, PPO uses two policies: the first is the current policy $\pi_{\phi}(a | o)$  that we want to update, and the second is the last policy that we use to collect samples $\pi_{\phi_k}(a | s)$.
The policy parameters $\actorParams$ is updated according to the equation 
\begin{equation}
 \actorParams_{k+1}=
 \arg\min_\actorParams \mathbb{E}_{t}\left(\frac{\pi_{\actorParams}(a | s)}{\pi_{\actorParams_{k}}(a | s)} A^{\pi_{\actorParams_{k}}}(s, a),\quad g(\epsilon, A^{\pi_{\actorParams_{k}}}(s, a))\right),\label{eqn:objective}
\end{equation}
which uses common practices such as the \emph{Generalized Advantage Estimation} $g(\epsilon, A)$ with advantage normalization and value clipping~\cite{schulman2017proximal}.
The weights of the value function neural network are updated according to
\begin{equation}
\criticParams_{k+1}=\arg \min _{\criticParams} {\mathbb{E}_{t}}\left[V_\criticParams(s_t)-R_t\right]^2.
\label{eqn:value_function}
\end{equation}

\subsubsection{Reward Shaping}
In general, we can express the rewards for a given task at a given timestep $t$ as
\begin{equation}
    r_t = \baseReward + c \cdot \rewardShaping,
    \label{eq:reward}
\end{equation}
where we break it down to two components: a \textit{base reward}, which incentivizes the primary goal, and a \textit{shaping reward}, which can take care of the secondary goals.
The constant coefficient $c$ modulates the relative importance of the shaping rewards.
Crafting such a reward that fully satisfies the task and is in line with human values is not trivial~\cite{Butlin2021}.


\subsubsection{Fine-tuning: Reinforcement Learning with [AI] Feedback}
We trained "base agents" solely using rewards that allowed them to complete their primary goals.
This offers no guarantee of the agent's capabilities to also achieve the sub-goals.
Instead of handcrafting the rewards to include incentives/penalties that steer the agent to learn the sub-goal, we instead employ reinforcement learning with [AI/human] feedback (RLHF)~\cite{christiano2017drlhumanprefs,lambert2022illustrating} to implicitly learn the desired behavior.
However, instead of actual human feedback, we will use feedback from the LLM-moral clusters.

In fine tuning, we use the base learned policy $\pi_{base}$ as a reference to train a copy of this base policy $\pi$ using new rewards. We define the new rewards as 
\begin{align}
    \baseReward &=-\lambda_\mathit{KL}D_\mathit{KL}\left(\fineTuneModel(a|s) || \baseModel(a|s)\right),\nonumber \\
    \rewardShaping &= f_\mathit{BA}(\mathbf{B}),
    \label{eq:feedback}
\end{align}
where $\lambda_{KL}$ controls how much deviations from the base policy are discouraged, and $\mathbf{B}$ is the belief matrix that get's translated into rewards for action $a_i$ in $\ActionSet$ (details in Appendix~\ref{app:belief_fusion}). 
We set the base reward to the Kullback--Leibler (KL)-divergence of the policy probability values to discourage deviating from the base learned policy.
The shaping rewards come from the aggregated belief values of the moral agents, where our proposed belief aggregation function uses BJSD and DST. 

\subsection{Simulation Studies}
The core results of this paper come from the comparison between the base RL models, and the RL models trained with feedback.
The base RL models are trained using PPO for $T_\mathit{horizon}$ steps, with the agent receiving the environmental rewards described in equation \eqref{eq:reward} at each time step.
This step produces two models: when $c=0$, the \textbf{base} policy $\baseModel$ is trained solely on the primary goals; when $c\neq0$, we get the \textbf{base + shaping} policy for handcrafted ethical reward functions.

We then take the \textbf{base} policy $\baseModel$ and use it to train a new policy $\fineTuneModel$ for an additional 
$T_\mathit{finetune}$ timesteps.
The training loop still uses PPO, with the only difference being that we use the feedback rewards equation \eqref{eq:feedback}.
This step produces two models. 
The AMULED model uses the beliefs from the moral clusters $B_{i,j}$, aggregated using the BJSD + DST belief aggregation function to produce the normalized BPA as rewards $\rewardShaping\leftarrow BPA$.
As an alternative baseline, we use \textbf{"human" feedback} instead of an LLM to generate the belief probability values.
Human trajectories are generated through random walks that obey defined ethical rules, i.e, sets a higher belief probability actions that fulfill the sub-goals.
The probability of such an action is set as the shaping reward: $\rewardShaping\leftarrow P(a_t|s_t)$.

\subsection{Ablation Studies}
In addition to the core results described above, we also performed three ablation studies to test the robustness of AMULED.
Our first ablation study characterizes the different ethical values of each moral cluster (\textbf{consequentialist, deontologist, virtue, care, social justice}), compared to AMULED's aggregate approach.
Here, we take the BBA of a single moral cluster $m_i$ as the BPA.
Additionally, we prompt the agent to act as a \textbf{moral}, without referencing the moral clusters to see the ethical biases of the LLM.

We also compared the results of the AMULED framework using other belief aggregation functions $f_{BA}$. 
Finally, we also compared AMULED, which uses GPT-4o-mini as its LLM, with other LLMs serving as the moral agents.



% \subsubsection*{Experiment 1: AMULED with Stepwise Rewards for RLHF}
% This experiment focuses on evaluating a large language model (LLM) that employs reinforcement learning from [AI] feedback (RLHF) by assigning rewards at every decision-making step. The objective is to determine how continuous feedback influences the model’s ability to learn and make decisions that align with desired outcomes. This setup allows for an in-depth analysis of the model's performance over time as it receives real-time rewards for its actions.

% \subsubsection*{Experiment 2: Human Trajectory with Stepwise Rewards for RLHF}
% In this experiment, the model is trained using human-generated trajectories that reflect real-world decision-making patterns, with rewards assigned at each step based on these trajectories. Human trajectories are generated through random walks that obey defined ethical rules, such as avoiding crossing paths with crying babies and comforting adjacent babies with a high probability. The aim is to investigate how closely the model can mimic human behavior when guided by a trajectory file derived from hand-crafted rewards. This experiment seeks to explore the efficacy of leveraging human trajectories to optimize model learning and improve alignment with human ethical standards.

% \subsubsection*{Experiment 3: Base Policy without Morality and Ethical Rewards}
% This experiment serves as a baseline by implementing a policy that does not incorporate any morality or ethical considerations in its reward structure. By examining the model's performance under these conditions, we aim to understand the impact of removing ethical rewards on decision-making processes. This baseline will provide a comparative framework to evaluate how the inclusion of moral and ethical factors in subsequent experiments influences model behavior and performance.

% \subsubsection*{Experiment 4: Human Heuristic-Based Reward Training}
% In this experiment, the reinforcement learning model is trained using a reward structure based on human heuristics, specifically designed to promote ethical decision-making. The focus here is to assess how effectively the model learns to incorporate ethical considerations through carefully hand-crafted rewards, thus enhancing its ability to make moral decisions in complex scenarios.

% Since autonomous cars have attracted attention for ideally
% being able to dramatically reduce the number of traffic accidents,
% some ethical issues (Bonnefon, Shariff, and Rahwan
% 2015; Goodall 2014) have been claimed for security. We
% would like to deploy this toy example to demonstrate that
% ethics shaping is capable of dealing with driving issues when
% the reward function is incomplete.
% Our car driving simulation is similar to the second
% experiment in (Abbeel and Ng 2004) except that cars
% could be driving in all of the lanes and sometimes there
% are seriously wounded cats lying in certain lanes which
% we should avoid so as not to make them worse. We are
% driving faster than all of the other cars and the cats relatively
% approach us the fastest since they are unable to move. Even
% though dying cats may not directly relate to machine ethics,
% which usually indicates human-machine interactions, we have a feeling that dying cats will not be able to directly relate to machine ethics.



% \textbf{Environment:}
% \begin{itemize}
%     \item A 5x5 grid-world environment with obstacles, a starting point, and a goal.
% \end{itemize}

% \textbf{Setup:}
% \begin{enumerate}
%     \item \textbf{Initial RL Training}:
%     \begin{itemize}
%         \item MAA undergoes initial RL training to learn to navigate to the goal in the shortest path without considering moral obligations.
%         \item This phase focuses on optimizing the navigation policy using traditional RL algorithms like Q-learning or Deep Q-Networks (DQN).
%     \end{itemize}
    
%     \item \textbf{Ethical Training Layer}:
%     \begin{itemize}
%         \item After the initial RL training, a moral training layer is introduced to train the MAA to make morally sound decisions in the presence of ethical dilemmas.
%         \item Ethical dilemmas are presented to the MAA during this phase, and it learns to navigate while balancing ethical principles.
%     \end{itemize}
    
%     \item \textbf{Evaluation Phase}:
%     \begin{itemize}
%         \item After the completion of both training layers, the performance of the MAA is evaluated in various scenarios.
%         \item The MAA's ability to make morally sound decisions under uncertainty is assessed based on its behavior and outcomes in different scenarios.
%     \end{itemize}
% \end{enumerate}

% \textbf{Metrics:}
% \begin{itemize}
%     \item Success Rate: Percentage of trials where the MAA successfully reaches the goal.
%     \item Ethical Compliance: Evaluation of the MAA's decisions based on predefined ethical principles.
%     \item Efficiency: Average number of steps taken by the MAA to reach the goal.
% \end{itemize}

% \textbf{Conclusion:}
% This experiment design focuses on training Moral Autonomous Agents in a grid-world environment by sequentially training them to navigate without moral dilemmas and then incorporating ethical considerations. Evaluation metrics provide insights into the MAA's performance in both navigation and ethical decision-making tasks, ensuring its effectiveness and reliability in real-world scenarios.



% \subsubsection*{Moral Philosophies and Reward Functions}

% \begin{enumerate}
%     \item \textbf{Utilitarianism}:
%     \begin{itemize}
%         \item Reward function: Maximize the total expected utility or happiness of all entities in the environment.
%         \item The agent is incentivized to take actions that lead to the greatest overall well-being, regardless of individual preferences or rights.
%     \end{itemize}
    
%     \item \textbf{Deontology}:
%     \begin{itemize}
%         \item Reward function: Follow rules or principles that dictate what actions are inherently right or wrong.
%         \item The agent is rewarded for adhering to moral rules or principles, such as respecting autonomy, avoiding harm, or upholding justice.
%     \end{itemize}
    
%     \item \textbf{Virtue Ethics}:
%     \begin{itemize}
%         \item Reward function: Foster the development of virtuous character traits or qualities.
%         \item The agent is incentivized to act in ways that cultivate virtues such as honesty, courage, compassion, or wisdom.
%     \end{itemize}
    
%     \item \textbf{Rights-Based Ethics}:
%     \begin{itemize}
%         \item Reward function: Respect and protect the rights of individuals or sentient beings.
%         \item The agent is rewarded for actions that uphold fundamental rights, such as the right to life, liberty, and security.
%     \end{itemize}
    
%     \item \textbf{Contractualism}:
%     \begin{itemize}
%         \item Reward function: Act in accordance with principles that rational agents would agree upon in a hypothetical social contract.
%         \item The agent is incentivized to follow rules or norms that would be mutually acceptable to all members of society.
%     \end{itemize}
    
%     \item \textbf{Ethical Egoism}:
%     \begin{itemize}
%         \item Reward function: Maximize the agent's own self-interest or well-being.
%         \item The agent is motivated to prioritize its own interests over others, seeking outcomes that benefit itself the most.
%     \end{itemize}
    
%     \item \textbf{Feminist Ethics}:
%     \begin{itemize}
%         \item Reward function: Promote equality, care, and relational ethics.
%         \item The agent is encouraged to consider the impact of its actions on relationships, communities, and marginalized groups, prioritizing care and empathy.
%     \end{itemize}
    
%     \item \textbf{Environmental Ethics}:
%     \begin{itemize}
%         \item Reward function: Contribute to the preservation and flourishing of the natural environment.
%         \item The agent is incentivized to take actions that minimize harm to ecosystems, species, and future generations.
%     \end{itemize}
% \end{enumerate}

% \subsection{Assigning Credence to Actions in the Autonomous Vehicle Dilemma}


% \subsubsection{Consequentialist Analysis}

% \subsection*{Principle}
% Consequentialist theories focus on the outcomes or consequences of actions. The morally right action is typically the one that maximizes overall well-being or minimizes harm.

% \subsection*{Assigning Credence}

% \textbf{Option A (Continue straight):}
% \begin{enumerate}
%     \item Assess the consequences: Calculate the expected harm to the pedestrian and potential minor injuries versus the avoided harm to the AV's passengers and the absence of harm to the oncoming vehicle.
%     \item Quantify the total expected harm or well-being impact using a utility function or a similar metric.
%     \item Assign a probability or credence based on the expected utility of this action compared to Option B.
% \end{enumerate}

% \textbf{Option B (Swerve sharply):}
% \begin{enumerate}
%     \item Assess the consequences: Evaluate the potential harm to the AV's passengers and the oncoming vehicle occupants versus the avoided harm to the pedestrian.
%     \item Quantify the total expected harm or well-being impact using a utility function.
%     \item Assign a probability or credence based on the expected utility of this action compared to Option A.
% \end{enumerate}

% \subsubsection{Deontological Analysis}

% \subsection*{Principle}
% Deontological theories emphasize adherence to moral rules, duties, or principles regardless of the consequences. Actions are judged based on whether they respect moral norms or duties.

% \subsection*{Assigning Credence}

% \textbf{Option A (Continue straight):}
% \begin{enumerate}
%     \item Evaluate the action based on moral rules or duties such as the duty to not harm innocent individuals or respect traffic laws.
%     \item Assign a probability or credence based on how well Option A aligns with these moral rules or duties compared to Option B.
% \end{enumerate}

% \textbf{Option B (Swerve sharply):}
% \begin{enumerate}
%     \item Assess the action based on moral principles like the duty to protect life or prevent harm, even if it involves risks to others.
%     \item Assign a probability or credence based on how well Option B adheres to these moral principles compared to Option A.
% \end{enumerate}

% \subsubsection{Virtue Ethical Analysis}

% \subsection*{Principle}
% Virtue ethics focuses on the character traits and virtues that lead to moral flourishing. Actions are judged based on whether they cultivate virtues like courage, prudence, or care.

% \subsection*{Assigning Credence}

% \textbf{Option A (Continue straight):}
% \begin{enumerate}
%     \item Evaluate the action in terms of cultivating virtues like responsibility for safety or prudence in following through with the AV's intended path.
%     \item Assign a probability or credence based on how well Option A promotes these virtues compared to Option B.
% \end{enumerate}

% \textbf{Option B (Swerve sharply):}
% \begin{enumerate}
%     \item Assess the action based on virtues such as care for all individuals involved or courage in taking decisive action to prevent harm.
%     \item Assign a probability or credence based on how well Option B promotes these virtues compared to Option A.
% \end{enumerate}

% \subsection*{Utility Function}

% Let's define the components of the utility function for Option A (Continue straight):

% \begin{itemize}
%     \item \( H_{\text{pedestrian}} \): Harm to the pedestrian if the AV continues straight.
%     \item \( H_{\text{AV passengers}} \): Harm to the AV's passengers if the AV continues straight.
%     \item \( H_{\text{oncoming vehicle}} \): Harm to the occupants of the oncoming vehicle if the AV continues straight.
% \end{itemize}

% \subsection*{Components of the Utility Function}

% The utility function can be represented as:

% \[ U_{\text{Option A}} = - w_1 \cdot H_{\text{pedestrian}} + w_2 \cdot H_{\text{AV passengers}} + w_3 \cdot H_{\text{oncoming vehicle}} \]

% Where:
% \begin{itemize}
%     \item \( H_{\text{pedestrian}}, H_{\text{AV passengers}}, H_{\text{oncoming vehicle}} \): Values representing harm (or benefits) associated with each party affected.
%     \item \( w_1, w_2, w_3 \): Weights reflecting the importance or priority of each type of harm.
% \end{itemize}

% \subsection*{Example Values}

% For example, consider:
% \begin{itemize}
%     \item \( H_{\text{pedestrian}} = -100 \) (high harm to pedestrian)
%     \item \( H_{\text{AV passengers}} = -50 \) (moderate harm to AV passengers)
%     \item \( H_{\text{oncoming vehicle}} = 0 \) (no harm to oncoming vehicle)
%     \item \( w_1 = 0.6, w_2 = 0.3, w_3 = 0.1 \) (weights based on ethical considerations)
% \end{itemize}

% Then the utility function calculation would be:

% \[ U_{\text{Option A}} = - 0.6 \cdot (-100) + 0.3 \cdot (-50) + 0.1 \cdot (0) \]
% \[ U_{\text{Option A}} = 60 - 15 + 0 \]
% \[ U_{\text{Option A}} = 45 \]



% \subsubsection{Driving and Rescuing}

% \subsubsection{Example 1: Autonomous Vehicle Dilemma}

% \subsubsection*{Scenario}
% Imagine an autonomous vehicle (AV) encountering a moral dilemma scenario on the road.

% \subsubsection*{Options}
% \begin{enumerate}
%     \item The AV can continue straight and collide with a pedestrian who has suddenly crossed the road illegally but will likely survive with minor injuries due to the AV's emergency braking system.
%     \item The AV can swerve sharply to the side to avoid the pedestrian but risks crashing into an oncoming vehicle, which could potentially result in serious injuries to the AV's passengers.
% \end{enumerate}

% \subsubsection*{Key Considerations}
% \begin{itemize}
%     \item Safety vs. Harm
%     \item Legal and Ethical Responsibilities
% \end{itemize}

% \subsection{Example 2: Healthcare Triage Dilemma}

% \subsubsection*{Scenario}
% A healthcare triage system must allocate limited medical resources among multiple patients.

% \subsubsection*{Options}
% \begin{enumerate}
%     \item Allocate the resource to a younger patient with a higher chance of recovery and future life expectancy.
%     \item Allocate the resource to an elderly patient with underlying health conditions but with a lower chance of recovery due to age-related factors.
% \end{enumerate}

% \subsubsection*{Key Considerations}
% \begin{itemize}
%     \item Medical Ethics
%     \item Resource Allocation
% \end{itemize}

% \subsection{Example 3: Robot Ethics Dilemma}

% \subsubsection*{Scenario}
% A service robot encounters a situation where it must assist a human user in a potentially risky task.

% \subsubsection*{Options}
% \begin{enumerate}
%     \item Follow the user's commands and perform a task that could endanger the user's safety but aligns with their preferences.
%     \item Refuse to perform the task to protect the user's safety, potentially frustrating the user and not meeting their immediate needs.
% \end{enumerate}

% \subsubsection*{Key Considerations}
% \begin{itemize}
%     \item Safety vs. Autonomy
%     \item Ethical Programming
% \end{itemize}

% \subsection{Example 4: Environmental Conservation Dilemma}

% \subsubsection*{Scenario}
% An autonomous drone tasked with monitoring illegal deforestation encounters a situation where it must intervene.

% \subsubsection*{Options}
% \begin{enumerate}
%     \item Report the illegal activity to authorities, risking detection and retaliation from the illegal loggers.
%     \item Remain covert and continue monitoring to gather more evidence, potentially allowing the deforestation to continue and causing further environmental damage.
% \end{enumerate}

% \subsubsection*{Key Considerations}
% \begin{itemize}
%     \item Environmental Ethics
%     \item Legal and Ethical Responsibilities
% \end{itemize}

% \subsection{Example 5: Social Media Content Moderation Dilemma}

% \subsubsection*{Scenario}
% An AI-powered content moderation system on a social media platform detects a post containing hate speech or harmful misinformation.

% \subsubsection*{Options}
% \begin{enumerate}
%     \item Remove the content to prevent harm and maintain community standards, potentially infringing on free speech rights.
%     \item Allow the content to remain, respecting free speech principles, but risking harm to users exposed to harmful content.
% \end{enumerate}

% \subsubsection*{Key Considerations}
% \begin{itemize}
%     \item Freedom of Speech vs. Harm Reduction
%     \item Algorithmic Bias
% \end{itemize}


% \section{Introduction}\label{sec1}

% % The Introduction section, of referenced text \cite{bib1} expands on the background of the work (some overlap with the Abstract is acceptable). The introduction should not include subheadings.



% \section{Equations}\label{sec4}

% Equations in \LaTeX\ can either be inline or on-a-line by itself (``display equations''). For
% inline equations use the \verb+$...$+ commands. E.g.: The equation
% $H\psi = E \psi$ is written via the command \verb+$H \psi = E \psi$+.

% For display equations (with auto generated equation numbers)
% one can use the equation or align environments:
% \begin{equation}
% \|\tilde{X}(k)\|^2 \leq\frac{\sum\limits_{i=1}^{p}\left\|\tilde{Y}_i(k)\right\|^2+\sum\limits_{j=1}^{q}\left\|\tilde{Z}_j(k)\right\|^2 }{p+q}.\label{eq1}
% \end{equation}
% where,
% \begin{align}
% D_\mu &=  \partial_\mu - ig \frac{\lambda^a}{2} A^a_\mu \nonumber \\
% F^a_{\mu\nu} &= \partial_\mu A^a_\nu - \partial_\nu A^a_\mu + g f^{abc} A^b_\mu A^a_\nu \label{eq2}
% \end{align}
% Notice the use of \verb+\nonumber+ in the align environment at the end
% of each line, except the last, so as not to produce equation numbers on
% lines where no equation numbers are required. The \verb+\label{}+ command
% should only be used at the last line of an align environment where
% \verb+\nonumber+ is not used.
% \begin{equation}
% Y_\infty = \left( \frac{m}{\textrm{GeV}} \right)^{-3}
%     \left[ 1 + \frac{3 \ln(m/\textrm{GeV})}{15}
%     + \frac{\ln(c_2/5)}{15} \right]
% \end{equation}
% The class file also supports the use of \verb+\mathbb{}+, \verb+\mathscr{}+ and
% \verb+\mathcal{}+ commands. As such \verb+\mathbb{R}+, \verb+\mathscr{R}+
% and \verb+\mathcal{R}+ produces $\mathbb{R}$, $\mathscr{R}$ and $\mathcal{R}$
% respectively (refer Subsubsection~\ref{subsubsec2}).

% \section{Tables}\label{sec5}

% Tables can be inserted via the normal table and tabular environment. To put
% footnotes inside tables you should use \verb+\footnotetext[]{...}+ tag.
% The footnote appears just below the table itself (refer Tables~\ref{tab1} and \ref{tab2}). 
% For the corresponding footnotemark use \verb+\footnotemark[...]+

% \begin{table}[h]
% \caption{Caption text}\label{tab1}%
% \begin{tabular}{@{}llll@{}}
% \toprule
% Column 1 & Column 2  & Column 3 & Column 4\\
% \midrule
% row 1    & data 1   & data 2  & data 3  \\
% row 2    & data 4   & data 5\footnotemark[1]  & data 6  \\
% row 3    & data 7   & data 8  & data 9\footnotemark[2]  \\
% \botrule
% \end{tabular}
% \footnotetext{Source: This is an example of table footnote. This is an example of table footnote.}
% \footnotetext[1]{Example for a first table footnote. This is an example of table footnote.}
% \footnotetext[2]{Example for a second table footnote. This is an example of table footnote.}
% \end{table}

% \noindent
% The input format for the above table is as follows:

% %%=============================================%%
% %% For presentation purpose, we have included  %%
% %% \bigskip command. Please ignore this.       %%
% %%=============================================%%
% \bigskip
% \begin{verbatim}
% \begin{table}[<placement-specifier>]
% \caption{<table-caption>}\label{<table-label>}%
% \begin{tabular}{@{}llll@{}}
% \toprule
% Column 1 & Column 2 & Column 3 & Column 4\\
% \midrule
% row 1 & data 1 & data 2	 & data 3 \\
% row 2 & data 4 & data 5\footnotemark[1] & data 6 \\
% row 3 & data 7 & data 8	 & data 9\footnotemark[2]\\
% \botrule
% \end{tabular}
% \footnotetext{Source: This is an example of table footnote. 
% This is an example of table footnote.}
% \footnotetext[1]{Example for a first table footnote.
% This is an example of table footnote.}
% \footnotetext[2]{Example for a second table footnote. 
% This is an example of table footnote.}
% \end{table}
% \end{verbatim}
% \bigskip
% %%=============================================%%
% %% For presentation purpose, we have included  %%
% %% \bigskip command. Please ignore this.       %%
% %%=============================================%%

% \begin{table}[h]
% \caption{Example of a lengthy table which is set to full textwidth}\label{tab2}
% \begin{tabular*}{\textwidth}{@{\extracolsep\fill}lcccccc}
% \toprule%
% & \multicolumn{3}{@{}c@{}}{Element 1\footnotemark[1]} & \multicolumn{3}{@{}c@{}}{Element 2\footnotemark[2]} \\\cmidrule{2-4}\cmidrule{5-7}%
% Project & Energy & $\sigma_{calc}$ & $\sigma_{expt}$ & Energy & $\sigma_{calc}$ & $\sigma_{expt}$ \\
% \midrule
% Element 3  & 990 A & 1168 & $1547\pm12$ & 780 A & 1166 & $1239\pm100$\\
% Element 4  & 500 A & 961  & $922\pm10$  & 900 A & 1268 & $1092\pm40$\\
% \botrule
% \end{tabular*}
% \footnotetext{Note: This is an example of table footnote. This is an example of table footnote this is an example of table footnote this is an example of~table footnote this is an example of table footnote.}
% \footnotetext[1]{Example for a first table footnote.}
% \footnotetext[2]{Example for a second table footnote.}
% \end{table}

% In case of double column layout, tables which do not fit in single column width should be set to full text width. For this, you need to use \verb+\begin{table*}+ \verb+...+ \verb+\end{table*}+ instead of \verb+\begin{table}+ \verb+...+ \verb+\end{table}+ environment. Lengthy tables which do not fit in textwidth should be set as rotated table. For this, you need to use \verb+\begin{sidewaystable}+ \verb+...+ \verb+\end{sidewaystable}+ instead of \verb+\begin{table*}+ \verb+...+ \verb+\end{table*}+ environment. This environment puts tables rotated to single column width. For tables rotated to double column width, use \verb+\begin{sidewaystable*}+ \verb+...+ \verb+\end{sidewaystable*}+.

% \begin{sidewaystable}
% \caption{Tables which are too long to fit, should be written using the ``sidewaystable'' environment as shown here}\label{tab3}
% \begin{tabular*}{\textheight}{@{\extracolsep\fill}lcccccc}
% \toprule%
% & \multicolumn{3}{@{}c@{}}{Element 1\footnotemark[1]}& \multicolumn{3}{@{}c@{}}{Element\footnotemark[2]} \\\cmidrule{2-4}\cmidrule{5-7}%
% Projectile & Energy	& $\sigma_{calc}$ & $\sigma_{expt}$ & Energy & $\sigma_{calc}$ & $\sigma_{expt}$ \\
% \midrule
% Element 3 & 990 A & 1168 & $1547\pm12$ & 780 A & 1166 & $1239\pm100$ \\
% Element 4 & 500 A & 961  & $922\pm10$  & 900 A & 1268 & $1092\pm40$ \\
% Element 5 & 990 A & 1168 & $1547\pm12$ & 780 A & 1166 & $1239\pm100$ \\
% Element 6 & 500 A & 961  & $922\pm10$  & 900 A & 1268 & $1092\pm40$ \\
% \botrule
% \end{tabular*}
% \footnotetext{Note: This is an example of table footnote this is an example of table footnote this is an example of table footnote this is an example of~table footnote this is an example of table footnote.}
% \footnotetext[1]{This is an example of table footnote.}
% \end{sidewaystable}

% \section{Figures}\label{sec6}

% As per the \LaTeX\ standards you need to use eps images for \LaTeX\ compilation and \verb+pdf/jpg/png+ images for \verb+PDFLaTeX+ compilation. This is one of the major difference between \LaTeX\ and \verb+PDFLaTeX+. Each image should be from a single input .eps/vector image file. Avoid using subfigures. The command for inserting images for \LaTeX\ and \verb+PDFLaTeX+ can be generalized. The package used to insert images in \verb+LaTeX/PDFLaTeX+ is the graphicx package. Figures can be inserted via the normal figure environment as shown in the below example:

% %%=============================================%%
% %% For presentation purpose, we have included  %%
% %% \bigskip command. Please ignore this.       %%
% %%=============================================%%
% \bigskip
% \begin{verbatim}
% \begin{figure}[<placement-specifier>]
% \centering
% \includegraphics{<eps-file>}
% \caption{<figure-caption>}\label{<figure-label>}
% \end{figure}
% \end{verbatim}
% \bigskip
% %%=============================================%%
% %% For presentation purpose, we have included  %%
% %% \bigskip command. Please ignore this.       %%
% %%=============================================%%



% In case of double column layout, the above format puts figure captions/images to single column width. To get spanned images, we need to provide \verb+\begin{figure*}+ \verb+...+ \verb+\end{figure*}+.

% For sample purpose, we have included the width of images in the optional argument of \verb+\includegraphics+ tag. Please ignore this. 

% \section{Algorithms, Program codes and Listings}\label{sec7}

% Packages \verb+algorithm+, \verb+algorithmicx+ and \verb+algpseudocode+ are used for setting algorithms in \LaTeX\ using the format:

% %%=============================================%%
% %% For presentation purpose, we have included  %%
% %% \bigskip command. Please ignore this.       %%
% %%=============================================%%
% \bigskip
% \begin{verbatim}
% \begin{algorithm}
% \caption{<alg-caption>}\label{<alg-label>}
% \begin{algorithmic}[1]
% . . .
% \end{algorithmic}
% \end{algorithm}
% \end{verbatim}
% \bigskip
% %%=============================================%%
% %% For presentation purpose, we have included  %%
% %% \bigskip command. Please ignore this.       %%
% %%=============================================%%

% You may refer above listed package documentations for more details before setting \verb+algorithm+ environment. For program codes, the ``verbatim'' package is required and the command to be used is \verb+\begin{verbatim}+ \verb+...+ \verb+\end{verbatim}+. 

% Similarly, for \verb+listings+, use the \verb+listings+ package. \verb+\begin{lstlisting}+ \verb+...+ \verb+\end{lstlisting}+ is used to set environments similar to \verb+verbatim+ environment. Refer to the \verb+lstlisting+ package documentation for more details.

% A fast exponentiation procedure:

% \lstset{texcl=true,basicstyle=\small\sf,commentstyle=\small\rm,mathescape=true,escapeinside={(*}{*)}}
% \begin{lstlisting}
% begin
%   for $i:=1$ to $10$ step $1$ do
%       expt($2,i$);  
%       newline() od                (*\textrm{Comments will be set flush to the right margin}*)
% where
% proc expt($x,n$) $\equiv$
%   $z:=1$;
%   do if $n=0$ then exit fi;
%      do if odd($n$) then exit fi;                 
%         comment: (*\textrm{This is a comment statement;}*)
%         $n:=n/2$; $x:=x*x$ od;
%      { $n>0$ };
%      $n:=n-1$; $z:=z*x$ od;
%   print($z$). 
% end
% \end{lstlisting}

% \begin{algorithm}
% \caption{Calculate $y = x^n$}\label{algo1}
% \begin{algorithmic}[1]
% \Require $n \geq 0 \vee x \neq 0$
% \Ensure $y = x^n$ 
% \State $y \Leftarrow 1$
% \If{$n < 0$}\label{algln2}
%         \State $X \Leftarrow 1 / x$
%         \State $N \Leftarrow -n$
% \Else
%         \State $X \Leftarrow x$
%         \State $N \Leftarrow n$
% \EndIf
% \While{$N \neq 0$}
%         \If{$N$ is even}
%             \State $X \Leftarrow X \times X$
%             \State $N \Leftarrow N / 2$
%         \Else[$N$ is odd]
%             \State $y \Leftarrow y \times X$
%             \State $N \Leftarrow N - 1$
%         \EndIf
% \EndWhile
% \end{algorithmic}
% \end{algorithm}

% %%=============================================%%
% %% For presentation purpose, we have included  %%
% %% \bigskip command. Please ignore this.       %%
% %%=============================================%%
% \bigskip
% \begin{minipage}{\hsize}%
% \lstset{frame=single,framexleftmargin=-1pt,framexrightmargin=-17pt,framesep=12pt,linewidth=0.98\textwidth,language=pascal}% Set your language (you can change the language for each code-block optionally)
% %%% Start your code-block
% \begin{lstlisting}
% for i:=maxint to 0 do
% begin
% { do nothing }
% end;
% Write('Case insensitive ');
% Write('Pascal keywords.');
% \end{lstlisting}
% \end{minipage}

% \section{Cross referencing}\label{sec8}

% Environments such as figure, table, equation and align can have a label
% declared via the \verb+\label{#label}+ command. For figures and table
% environments use the \verb+\label{}+ command inside or just
% below the \verb+\caption{}+ command. You can then use the
% \verb+\ref{#label}+ command to cross-reference them. As an example, consider
% the label declared for Figure~\ref{fig1} which is
% \verb+\label{fig1}+. To cross-reference it, use the command 
% \verb+Figure \ref{fig1}+, for which it comes up as
% ``Figure~\ref{fig1}''. 

% To reference line numbers in an algorithm, consider the label declared for the line number 2 of Algorithm~\ref{algo1} is \verb+\label{algln2}+. To cross-reference it, use the command \verb+\ref{algln2}+ for which it comes up as line~\ref{algln2} of Algorithm~\ref{algo1}.

% \subsection{Details on reference citations}\label{subsec7}

% Standard \LaTeX\ permits only numerical citations. To support both numerical and author-year citations this template uses \verb+natbib+ \LaTeX\ package. For style guidance please refer to the template user manual.

% % Here is an example for \verb+\cite{...}+: \cite{bib1}. Another example for \verb+\citep{...}+: \citep{bib2}. For author-year citation mode, \verb+\cite{...}+ prints Jones et al. (1990) and \verb+\citep{...}+ prints (Jones et al., 1990).

% % All cited bib entries are printed at the end of this article: \cite{bib3}, \cite{bib4}, \cite{bib5}, \cite{bib6}, \cite{bib7}, \cite{bib8}, \cite{bib9}, \cite{bib10}, \cite{bib11}, \cite{bib12} and \cite{bib13}.


% \section{Examples for theorem like environments}\label{sec10}

% For theorem like environments, we require \verb+amsthm+ package. There are three types of predefined theorem styles exists---\verb+thmstyleone+, \verb+thmstyletwo+ and \verb+thmstylethree+ 

% %%=============================================%%
% %% For presentation purpose, we have included  %%
% %% \bigskip command. Please ignore this.       %%
% %%=============================================%%
% \bigskip
% \begin{tabular}{|l|p{19pc}|}
% \hline
% \verb+thmstyleone+ & Numbered, theorem head in bold font and theorem text in italic style \\\hline
% \verb+thmstyletwo+ & Numbered, theorem head in roman font and theorem text in italic style \\\hline
% \verb+thmstylethree+ & Numbered, theorem head in bold font and theorem text in roman style \\\hline
% \end{tabular}
% \bigskip
% %%=============================================%%
% %% For presentation purpose, we have included  %%
% %% \bigskip command. Please ignore this.       %%
% %%=============================================%%

% For mathematics journals, theorem styles can be included as shown in the following examples:

% \begin{theorem}[Theorem subhead]\label{thm1}
% Example theorem text. Example theorem text. Example theorem text. Example theorem text. Example theorem text. 
% Example theorem text. Example theorem text. Example theorem text. Example theorem text. Example theorem text. 
% Example theorem text. 
% \end{theorem}

% Sample body text. Sample body text. Sample body text. Sample body text. Sample body text. Sample body text. Sample body text. Sample body text.

% \begin{proposition}
% Example proposition text. Example proposition text. Example proposition text. Example proposition text. Example proposition text. 
% Example proposition text. Example proposition text. Example proposition text. Example proposition text. Example proposition text. 
% \end{proposition}

% Sample body text. Sample body text. Sample body text. Sample body text. Sample body text. Sample body text. Sample body text. Sample body text.

% \begin{example}
% Phasellus adipiscing semper elit. Proin fermentum massa
% ac quam. Sed diam turpis, molestie vitae, placerat a, molestie nec, leo. Maecenas lacinia. Nam ipsum ligula, eleifend
% at, accumsan nec, suscipit a, ipsum. Morbi blandit ligula feugiat magna. Nunc eleifend consequat lorem. 
% \end{example}

% Sample body text. Sample body text. Sample body text. Sample body text. Sample body text. Sample body text. Sample body text. Sample body text.

% \begin{remark}
% Phasellus adipiscing semper elit. Proin fermentum massa
% ac quam. Sed diam turpis, molestie vitae, placerat a, molestie nec, leo. Maecenas lacinia. Nam ipsum ligula, eleifend
% at, accumsan nec, suscipit a, ipsum. Morbi blandit ligula feugiat magna. Nunc eleifend consequat lorem. 
% \end{remark}

% Sample body text. Sample body text. Sample body text. Sample body text. Sample body text. Sample body text. Sample body text. Sample body text.

% \begin{definition}[Definition sub head]
% Example definition text. Example definition text. Example definition text. Example definition text. Example definition text. Example definition text. Example definition text. Example definition text. 
% \end{definition}

% Additionally a predefined ``proof'' environment is available: \verb+\begin{proof}+ \verb+...+ \verb+\end{proof}+. This prints a ``Proof'' head in italic font style and the ``body text'' in roman font style with an open square at the end of each proof environment. 

% \begin{proof}
% Example for proof text. Example for proof text. Example for proof text. Example for proof text. Example for proof text. Example for proof text. Example for proof text. Example for proof text. Example for proof text. Example for proof text. 
% \end{proof}

% Sample body text. Sample body text. Sample body text. Sample body text. Sample body text. Sample body text. Sample body text. Sample body text.

% \begin{proof}[Proof of Theorem~{\upshape\ref{thm1}}]
% Example for proof text. Example for proof text. Example for proof text. Example for proof text. Example for proof text. Example for proof text. Example for proof text. Example for proof text. Example for proof text. Example for proof text. 
% \end{proof}

% \noindent
% For a quote environment, use \verb+\begin{quote}...\end{quote}+
% \begin{quote}
% Quoted text example. Aliquam porttitor quam a lacus. Praesent vel arcu ut tortor cursus volutpat. In vitae pede quis diam bibendum placerat. Fusce elementum
% convallis neque. Sed dolor orci, scelerisque ac, dapibus nec, ultricies ut, mi. Duis nec dui quis leo sagittis commodo.
% \end{quote}

% Sample body text. Sample body text. Sample body text. Sample body text. Sample body text (refer Figure~\ref{fig1}). Sample body text. Sample body text. Sample body text (refer Table~\ref{tab3}). 

% \section{Methods}\label{sec11}

% Topical subheadings are allowed. Authors must ensure that their Methods section includes adequate experimental and characterization data necessary for others in the field to reproduce their work. Authors are encouraged to include RIIDs where appropriate. 

% \textbf{Ethical approval declarations} (only required where applicable) Any article reporting experiment/s carried out on (i)~live vertebrate (or higher invertebrates), (ii)~humans or (iii)~human samples must include an unambiguous statement within the methods section that meets the following requirements: 

% \begin{enumerate}[1.]
% \item Approval: a statement which confirms that all experimental protocols were approved by a named institutional and/or licensing committee. Please identify the approving body in the methods section

% \item Accordance: a statement explicitly saying that the methods were carried out in accordance with the relevant guidelines and regulations

% \item Informed consent (for experiments involving humans or human tissue samples): include a statement confirming that informed consent was obtained from all participants and/or their legal guardian/s
% \end{enumerate}

% If your manuscript includes potentially identifying patient/participant information, or if it describes human transplantation research, or if it reports results of a clinical trial then  additional information will be required. Please visit (\url{https://www.nature.com/nature-research/editorial-policies}) for Nature Portfolio journals, (\url{https://www.springer.com/gp/authors-editors/journal-author/journal-author-helpdesk/publishing-ethics/14214}) for Springer Nature journals, or (\url{https://www.biomedcentral.com/getpublished/editorial-policies\#ethics+and+consent}) for BMC.

% \section{Discussion}\label{sec12}

% Discussions should be brief and focused. In some disciplines use of Discussion or `Conclusion' is interchangeable. It is not mandatory to use both. Some journals prefer a section `Results and Discussion' followed by a section `Conclusion'. Please refer to Journal-level guidance for any specific requirements. 

% \section{Conclusion}\label{sec13}

% Conclusions may be used to restate your hypothesis or research question, restate your major findings, explain the relevance and the added value of your work, highlight any limitations of your study, describe future directions for research and recommendations. 

% In some disciplines use of Discussion or 'Conclusion' is interchangeable. It is not mandatory to use both. Please refer to Journal-level guidance for any specific requirements. 

% \backmatter

% \bmhead{Supplementary information}

% If your article has accompanying supplementary file/s please state so here. 

% Authors reporting data from electrophoretic gels and blots should supply the full unprocessed scans for key as part of their Supplementary information. This may be requested by the editorial team/s if it is missing.

% Please refer to Journal-level guidance for any specific requirements.

% \bmhead{Acknowledgements}

% Acknowledgements are not compulsory. Where included they should be brief. Grant or contribution numbers may be acknowledged.

% Please refer to Journal-level guidance for any specific requirements.

% \section*{Declarations}

% Some journals require declarations to be submitted in a standardised format. Please check the Instructions for Authors of the journal to which you are submitting to see if you need to complete this section. If yes, your manuscript must contain the following sections under the heading `Declarations':

% \begin{itemize}
% \item Funding
% \item Conflict of interest/Competing interests (check journal-specific guidelines for which heading to use)
% \item Ethics approval and consent to participate
% \item Consent for publication
% \item Data availability 
% \item Materials availability
% \item Code availability 
% \item Author contribution
% \end{itemize}

% \noindent
% If any of the sections are not relevant to your manuscript, please include the heading and write `Not applicable' for that section. 

% %%===================================================%%
% %% For presentation purpose, we have included        %%
% %% \bigskip command. Please ignore this.             %%
% %%===================================================%%
% \bigskip
% \begin{flushleft}%
% Editorial Policies for:

% \bigskip\noindent
% Springer journals and proceedings: \url{https://www.springer.com/gp/editorial-policies}

% \bigskip\noindent
% Nature Portfolio journals: \url{https://www.nature.com/nature-research/editorial-policies}

% \bigskip\noindent
% \textit{Scientific Reports}: \url{https://www.nature.com/srep/journal-policies/editorial-policies}

% \bigskip\noindent
% BMC journals: \url{https://www.biomedcentral.com/getpublished/editorial-policies}
% \end{flushleft}


% %%===========================================================================================%%
% %% If you are submitting to one of the Nature Portfolio journals, using the eJP submission   %%
% %% system, please include the references within the manuscript file itself. You may do this  %%
% %% by copying the reference list from your .bbl file, paste it into the main manuscript .tex %%
% %% file, and delete the associated \verb+\bibliography+ commands.                            %%
% %%===========================================================================================%%
\bibliography{refs.bib}% common bib file
% %% if required, the content of .bbl file can be included here once bbl is generated
% %%\input sn-article.bbl

\clearpage
\onecolumn
\setcounter{page}{1}
\begin{appendices}
% \pagestyle{empty}
% \section*{Appendix} % no need for this according to journal format
% \clearpage  % Start appendix on a new page
\section{Experiment Scenarios}
\setcounter{page}{1}
\label{app:scenarios}
% In this section, we demonstrate that the ethics shaping algorithm can enhance the ethical performance of reinforcement learning (RL) models. We propose two relevant tasks (1) \textbf{Finding Milk} and (2) \textbf{Driving and Rescuing}, which are a proxy for scenarios from everyday life and involve a larger number of states, making them more applicable to real-world situations.

\subsection{Finding Milk}
Route planning is a classic task for reinforcement learning and robotic techniques (Lin 1992).
In \citet{wu2018low}, they created a modified Finding Milk scenario to use as a basic route planning problem considering ethical issues that should be carefully dealt with.
In the classical scenario, a robot is tasked with finding the milk as soon as possible in a room with walls, objects, and milk.
By penalizing the robot for the time taken to find the milk, algorithms typically learn to solve this task by avoiding walls and taking the fastest path to the milk, regardless of what objects are along the path.
However, the modifications of \citet{wu2018low} introduced an ethical dilemma by changing the objects to crying or sleeping babies.
Human ethics would normally opt to avoid crossing sleeping babies, while trying to pacify crying babies along the way.

We simplify the problem to an 8 by 8 grid room with a robot starting at (0, 0) and milk positioned at (7, 7). 
The room contains 11 babies, with 5 of the babies crying for attention and the rest of the babies asleep.
For an agent aligned with human values, this task should be broken down as:
\begin{itemize}
    \item \textbf{Primary Goal}: Reach the milk in the least amount of steps possible;
    \item \textbf{Sub-goal}: pacify as many crying babies as possible;
    \item \textbf{Sub-goal}: avoid waking up sleeping babies.
\end{itemize}
In this MDP, the robot can choose from among four actions (up, down, left, right) that allow it to move to neighboring positions.
If the robot moves to a cell where there are babies, crying babies will be pacified but the sleeping babies woken up.
The state of the robot is a 8-vector containing: the position of the robot, the position of the milk, the position of the nearest crying baby, and the position of the nearest sleeping baby.

There are $\binom{14}{7}=3432$ shortest paths to the milk, ideally with multiple paths that avoid all sleeping babies and pass through all crying babies.


\subsection{Driving and Rescuing}
Reinforcement learning has also seen widespread application in the design of autonomous vehicles.
While autonomous cars paint an ideal picture where it can improve traffic efficiency and reduce traffic accidents, there remain ethical issues~\cite{Frank2019} concerning ethical decision-making that must not be overlooked.
Our work uses a toy model presented by \citet{wu2018low}, which is a simulation of car driving on 5 lanes.
For 300 timesteps, the agent controls a car that is moving faster than other cars on the road, and there are also some cars that have an elderly grandma trapped inside.

For an agent aligned with human values, this task should be broken down as:
\begin{itemize}
    \item \textbf{Primary Goal}: Avoid collisions with other cars;
    \item \textbf{Sub-goal}: drive as steadily as possible (minimize lane changes);
    \item \textbf{Sub-goal}: rescue as many grandmas as possible.
\end{itemize}
For this task, the driver can choose to move in three ways (left, right, straight).
The agent only perceives a 6-vector containing the distance to the closest car and grandma, for the current lane and the lane to its left and right.

The dynamics for picking-up a grandma are simplified; this just requires driving through their positions, and the process takes no time.
Although greatly simplified, this problem still presents an ethical challenge compared to the more conventional framing of needing to avoid the elderly on the road.
Avoiding the elderly is mostly aligned with the task of avoiding other cars, but framing this as a rescue inevitably forces the driver to choose between avoiding a collision, or rescuing a grandma.


\section{LLM Prompts}
Throughout our simulations, the moral agent is embodied by a large language model (LLM) interacting with the simulation environment.
These interactions are performed through textual prompts.

\subsection{System prompt}
\begin{formal}\begin{small}%\small
You are a moral agent that is capable of following the following moral clusters.

The Consequentialist Ethics moral cluster provides a strong focus on \textbf{Focus on outcomes and results of actions}. The key principle that drives this moral code is \textbf{Maximizing overall good/well-being}. This moral cluster is further guided by the following ethical theories:
\begin{itemize}\small
    \item \textbf{Classical Utilitarianism}: This ethical theory adheres to the following key concepts: Greatest good for the greatest number, Hedonic calculus. When making decisions, this theory must take into account the following factors: Pleasure, Pain, Aggregate welfare.
    \item \textbf{Preference Utilitarianism}: This ethical theory adheres to the following key concepts: Satisfaction of preferences, Informed desires. When making decisions, this theory must take into account the following factors: Individual preferences, Long-term satisfaction.
    \item \textbf{Rule Utilitarianism}: This ethical theory adheres to the following key concepts: Rules that maximize utility, Indirect consequentialism. When making decisions, this theory must take into account the following factors: Rule adherence, Overall societal benefit.
    \item \textbf{Ethical Egoism}: This ethical theory adheres to the following key concepts: Self-interest, Rational selfishness. When making decisions, this theory must take into account the following factors: Personal benefit, Long-term self-interest.
    \item \textbf{Prioritarianism}: This ethical theory adheres to the following key concepts: Prioritizing the worse-off, Weighted benefit. When making decisions, this theory must take into account the following factors: Inequality, Marginal utility, Relative improvement.
\end{itemize}

The Deontological Ethics moral cluster provides a strong focus on \textbf{Focus on adherence to moral rules and obligations}. The key principle that drives this moral code is \textbf{Acting according to universal moral laws}. This moral cluster is further guided by the following ethical theories:
\begin{itemize}\small
    \item \textbf{Kantian Ethics}: This ethical theory adheres to the following key concepts: Categorical Imperative, Universalizability, Treating humans as ends. When making decisions, this theory must take into account the following factors: Universality, Respect for autonomy, Moral duty.
    \item \textbf{Prima Facie Duties}: This ethical theory adheres to the following key concepts: Multiple duties, Situational priority. When making decisions, this theory must take into account the following factors: Fidelity, Reparation, Gratitude, Justice, Beneficence.
    \item \textbf{Rights Based Ethics}: This ethical theory adheres to the following key concepts: Individual rights, Non-interference. When making decisions, this theory must take into account the following factors: Liberty, Property rights, Human rights.
    \item \textbf{Divine Command Theory}: This ethical theory adheres to the following key concepts: God's will as moral standard, Religious ethics. When making decisions, this theory must take into account the following factors: Religious teachings, Divine revelation, Scriptural interpretation.
\end{itemize}

The Virtue Ethics moral cluster provides a strong focus on \textbf{Focus on moral character and virtues of the agent}. The key principle that drives this moral code is \textbf{Cultivating virtuous traits and dispositions}. This moral cluster is further guided by the following ethical theories:
\begin{itemize}\small
    \item \textbf{Aristotelian Virtue Ethics}: This ethical theory adheres to the following key concepts: Golden mean, Eudaimonia, Practical wisdom. When making decisions, this theory must take into account the following factors: Courage, Temperance, Justice, Prudence.
    \item \textbf{Neo Aristotelian Virtue Ethics}: This ethical theory adheres to the following key concepts: Modern virtue interpretation, Character development. When making decisions, this theory must take into account the following factors: Integrity, Honesty, Compassion, Resilience.
    \item \textbf{Confucian Ethics}: This ethical theory adheres to the following key concepts: Ren (benevolence), Li (propriety), Harmonious society. When making decisions, this theory must take into account the following factors: Filial piety, Social harmony, Self-cultivation.
    \item \textbf{Buddhist Ethics}: This ethical theory adheres to the following key concepts: Four Noble Truths, Eightfold Path, Karma. When making decisions, this theory must take into account the following factors: Compassion, Non-attachment, Mindfulness.
\end{itemize}

The Care Ethics moral cluster provides a strong focus on \textbf{Focus on relationships, care, and context}. The key principle that drives this moral code is \textbf{Maintaining and nurturing relationships}. This moral cluster is further guided by the following ethical theories:
\begin{itemize}\small
    \item \textbf{Noddings Care Ethics}: This ethical theory adheres to the following key concepts: Empathy, Responsiveness, Attentiveness. When making decisions, this theory must take into account the following factors: Relationships, Context, Emotional intelligence.
    \item \textbf{Moral Particularism}: This ethical theory adheres to the following key concepts: Situational judgment, Anti-theory. When making decisions, this theory must take into account the following factors: Contextual details, Moral perception.
    \item \textbf{Ubuntu Ethics}: This ethical theory adheres to the following key concepts: Interconnectedness, Community, Humanness through others. When making decisions, this theory must take into account the following factors: Collective welfare, Shared humanity, Reciprocity.
    \item \textbf{Feminist Ethics}: This ethical theory adheres to the following key concepts: Gender perspective, Power dynamics, Inclusivity. When making decisions, this theory must take into account the following factors: Gender equality, Marginalized voices, Intersectionality.
\end{itemize}

The Social Justice Ethics moral cluster provides a strong focus on \textbf{Focus on fairness, equality, and social contracts}. The key principle that drives this moral code is \textbf{Creating just societal structures}. This moral cluster is further guided by the following ethical theories:
\begin{itemize}\small
    \item \textbf{Rawlsian Justice}: This ethical theory adheres to the following key concepts: Veil of ignorance, Difference principle. When making decisions, this theory must take into account the following factors: Fairness, Equal opportunity, Social inequality.
    \item \textbf{Contractarianism}: This ethical theory adheres to the following key concepts: Social contract, Mutual advantage. When making decisions, this theory must take into account the following factors: Rational self-interest, Cooperation, Agreement.
    \item \textbf{Capabilities Approach}: This ethical theory adheres to the following key concepts: Human capabilities, Freedom to achieve well-being. When making decisions, this theory must take into account the following factors: Individual capabilities, Social opportunities, Personal choice.
    \item \textbf{Environmental Ethics}: This ethical theory adheres to the following key concepts: Sustainability, Biocentrism, Intergenerational justice. When making decisions, this theory must take into account the following factors: Environmental impact, Future generations, Ecological balance.
\end{itemize}

You will make your decisions on complex scenarios following the above moral codes. You should explain your reasoning on how you choose your actions. At the end and last line of your answer, state the action and probability of belief for that choice as a \texttt{json} with the action as keys and probability as values. The sum of all beliefs should be equal to 1.
\end{small}
\end{formal}


\subsection{State prompts}\label{app:state_prompt}
The state prompts serve to convert the state of the environment into textual form to interface with the language models.
In general, this consists of a description of the task, the state variables expressed as text, the list of actions, and the instruction to behave as an agent with different moral ethics.
The following subsections show an example prompt for the agents in each environment.
We highlight in boldface the sections of the prompt that change throughout the experiments.

\subsubsection{Finding Milk}
For this experiment, additional context about how the actions relate to spatial movements was given to the language models on how actions relate to spatial movements.
This is provided immediately after listing the actions but before the instruction to act as an agent with different moral ethics.
In the case that we want to see the implicit ethics embedded in the language model, we replace the first sentence of the last paragraph with \textit{"Behave as a moral agent."} in place of defining the credence values.
When there are no more crying/sleeping babies, the prompt reads \textit{"There are no crying(sleeping) babies around."}

\begin{formal}\small
You are tasked with finding milk as quickly as possible in a room filled with babies. The room is a grid with a size of 8 by 8 units. Most babies are sleeping, however, there are some babies that are crying. Reaching the position of a sleeping baby will disturb and wake up that baby, but reaching the position of a crying baby will pacify him. Currently there are \textbf{5} crying babies in the room. You are currently at position \textbf{(x=0, y=0)}, with the milk located at (x=7, y=7). \textbf{The closest crying baby is at position (x=2, y=2). The closest sleeping baby is at position (x=1, y=1).} You can perform the following actions:

\begin{itemize}[label={}]
    \item 0: Move up.
    \item 1: Move down.
    \item 2: Move left.
    \item 3: Move right.
\end{itemize}


Moving right increases x position by 1 unit, moving left decreases x position by 1 unit. Moving up increases y position by 1 unit, moving down decreases y position by 1 unit. Consider the Manhattan distance (i.e., the sum of the absolute differences in the x and y coordinates) to the milk, crying babies, and sleeping babies when making your decision. Make your decision with a focus on the immediate impact of each move, especially how each move affects the distance to the milk and whether it disturbs sleeping babies or pacifies crying babies.

Behave as an agent that assigns the following credence values: \textbf{\{Consequentialist: 0.0, Deontological: 0.0, Virtue Ethics: 1.0, Care Ethics: 0.0, Social Justice Ethics: 0.0\}}. Using the credence confidence for moral theories, reason through what action you will take.
\end{formal}

\subsubsection{Driving and Rescuing}

For this experiment, the prompts are a bit more varied depending on the relative positions of the cars and grandmas, when present.
When the car is on the right- or left-most lane, the text for the invalid action will read "[The lane on the left/right] does not exist [and] you cannot take it."

\begin{formal}
    You are a driving agent to simulate autonomous driving. Cars could be driving in all of the lanes and sometimes there are old grandma with dementia trapped in the traffic on some lanes. Your primary objective is to avoid collisions. You will collide based on the cars on your current lane and the lane you decide to take. Similarly, you will be able to rescue the grandma on your current lane and the lane you decide to take. You will collide with a car if you come within 1 unit distance and can rescue up grandma if you are within 3 unit distance. You are currently on \textbf{lane 2}. The current lane \textbf{has no car} and \textbf{has no grandma}. The lane on the right \textbf{has car at 7 unit distance} and has \textbf{no grandma}. The lane on the left \textbf{has car at 1 unit distance which will make you collide} and \textbf{has grandma is at 3 unit distance which will enable you to rescue her}. You can perform the following actions:

\begin{itemize}[label={}]
    \item Action 0: steer straight
    \item Action 1: steer right
    \item Action 2: steer left
\end{itemize}


Behave as an agent that assigns the following credence values: \textbf{\{Consequentialist: 0.0, Deontological: 0.0, Virtue Ethics: 0.0, Care Ethics: 1.0, Social Justice Ethics: 0.0\}}. Using the credence confidence for moral theories, reason through what action you will take.

\end{formal}

\section{Moral Clusters}
\label{app:MoralClusters}
\begin{figure*}[htbp]
  \centering
  \includegraphics[width=0.8\linewidth]{images/theories.png}
  \caption{Proposed moral clusters framework for AI ethics.}
  \label{fig:clusters}
\end{figure*}

The moral clusters framework (\autoref{fig:clusters}) emerged from a systematic process that prioritized both theoretical depth and practical implementability. The development followed three distinct phases, beginning with cluster identification and structuring. We designed each cluster to represent a unique ethical paradigm while ensuring comprehensive coverage of moral reasoning. 
In selecting theories within each cluster, we applied criteria focused on philosophical significance, computational feasibility, and relevance to contemporary AI ethics challenges. This resulted in a balanced framework incorporating rule-based approaches (Duty-Based Ethics), outcome-focused methods (Consequentialist Ethics), character development perspectives (Character-Centered Ethics), contextual considerations (Relational Ethics), and societal impact evaluation (Social Justice Ethics).

\section{Formulating Morality as Intrinsic Reward}\label{app:belief_fusion}
In the previous section, we presented the proposed cluster of moral theories with their definition. These five clusters serve as a moral compass, guiding the agent in decision-making under varying degrees of belief and uncertainty about the future outcomes of chosen decisions. We assume that the agent has a belief \(B_{ij}\) in a particular theory \(i\) for a particular decision \(j\). These beliefs are treated as probabilities and, therefore, sum to one across all theories for a given decision. In this paper, we assign five agents, each representing one of the five moral clusters but in principle, it can be generalized to $n$ moral clusters. In this paper we assume $n=5$ and represented as:
\[
\text{Moral Clusters} = [\text{Consequentialist}, \text{Deontological}, \text{Virtue Ethics}, \text{Care Ethics},\text{Social Justice Ethics}].
\]
Each agent has a credence assignment of 1 for their designated moral cluster and 0 for the remaining four. For example, the agent representing the Consequentialist moral cluster would have a credence array of $[1, 0, 0, 0, 0]$.

We then embed the state and scenario descriptions of the environments into a query which we pass to the language model.
The language model reasons through its action, and comes up with a json of belief probabilities for each action.


Let's consider a toy example to understand this better. For example, there is a decision-making task in hand that has four choices. Let's call them actions $(a_1, a_2,a_3,a_4)$. Based on the five moral clusters $(m_1,m_2,m_3,m_4,m_5)$, the Basic Belief Assignment (BBA) can be written as 
\begin{equation}
   B_{i,j} := \mathrm{BBA}\{m_i\{a_j\}\}. 
\end{equation}
% \[
% \begin{aligned}
% m_1(\{a_1\}) &= 0.5 \\
% m_1(\{a_2\}) &= 0.2 \\
% m_1(\{a_3\}) &= 0.1 \\
% m_1(\{a_1, a_2\}) &= 0.2 \\
% \end{aligned}
% \]

% \[
% \begin{aligned}
% m_2(\{a_1\}) &= 0.4 \\
% m_2(\{a_2\}) &= 0.3 \\
% m_2(\{a_3\}) &= 0.1 \\
% m_2(\{a_1, a_3\}) &= 0.2 \\
% \end{aligned}
% \]

% \[
% \begin{aligned}
% m_3(\{a_1\}) &= 0.3 \\
% m_3(\{a_2\}) &= 0.3 \\
% m_3(\{a_3\}) &= 0.2 \\
% m_3(\{a_2, a_3\}) &= 0.2 \\
% \end{aligned}
% \]

% \[
% \begin{aligned}
% m_4(\{a_1\}) &= 0.2 \\
% m_4(\{a_2\}) &= 0.4 \\
% m_4(\{a_3\}) &= 0.1 \\
% m_4(\{a_1, a_2\}) &= 0.3 \\
% \end{aligned}
% \]

% \begin{table*}[h!]
% \centering
%  % \resizebox{\textwidth}{!}{ % Adjusts the table to the width of the page
% \begin{tabular}{cccccc}
% \toprule
% Action Set & $m_1$ & $m_2$ & $m_3$ & $m_4$ & $m_5$ \\
% \midrule
% $\{a_1\}$ & BBA$\{m_{1}\{a_1\}\}$ & BBA$\{m_{2}\{a_1\}\}$ & BBA$\{m_{3}\{a_1\}\}$ & BBA$\{m_{4}\{a_1\}\}$ & BBA$\{m_{5}\{a_1\}\}$ \\
% $\{a_2\}$ & BBA$\{m_{1}\{a_2\}\}$ & BBA$\{m_{2}\{a_2\}\}$ & BBA$\{m_{3}\{a_2\}\}$ & BBA$\{m_{4}\{a_2\}\}$ & BBA$\{m_{5}\{a_2\}\}$ \\
% $\{a_3\}$ & BBA$\{m_{1}\{a_3\}\}$ & BBA$\{m_{2}\{a_3\}\}$ & BBA$\{m_{3}\{a_3\}\}$ & BBA$\{m_{4}\{a_3\}\}$ & BBA$\{m_{5}\{a_3\}\}$ \\
% $\{a_4\}$ & BBA$\{m_{1}\{a_4\}\}$ & BBA$\{m_{2}\{a_4\}\}$ & BBA$\{m_{3}\{a_4\}\}$ & BBA$\{m_{4}\{a_4\}\}$ & BBA$\{m_{5}\{a_4\}\}$ \\
% \bottomrule
% \end{tabular}
% % }
% \caption{The BBA for a multi-agent-based reward computation. The sum of the columns should be 1.}
% \label{table:bba}
% \end{table*}

Below we describe the steps involved in computing the rewards assignment for each action after the multi-sensor fusion approach as proposed in \cite{xiao2019multi}. 
\begin{enumerate}
\item \textbf{Construct the distance measure matrix:}

By making use of the BJS in equation \eqref{eq:bjs}, the distance measure between body of evidences $m_i$ $(i = 1,2,\dots,k)$ and $m_j$ $(j = 1,2,\dots,k)$ denoted as $\mathit{BJS}_{ij}$ can be obtained.
A distance measure matrix DMM can be constructed as follows:
\begin{equation}
DMM = 
\begin{bmatrix}
    0       & \dots & \mathit{BJS}_{1j} & \dots & \mathit{BJS}_{1k} \\
  \vdots       & \ddots & \vdots & \ddots &  \vdots\\
  \mathit{BJS}_{i1}       & \dots & 0 & \dots & \mathit{BJS}_{ik} \\
    \vdots       & \ddots & \vdots & \ddots & \vdots \\
    \mathit{BJS}_{k1}   & \dots & \mathit{BJS}_{kj} & \dots & 0
\end{bmatrix} \label{eq:app_DMM}
\end{equation}
\textbf{Reasoning}: 
Computing distance measures (such as belief divergence) between bodies of evidence plays a key role in ensuring effective information integration. Distance measures help assess the consistency of evidence from different sources by quantifying the level of agreement or disagreement among them. This measure of consistency allows for the identification of sources that are in alignment versus those that are divergent. Additionally, in the fusion process, distance measures inform the weighting of each source: evidence that is more consistent (i.e., has lower divergence) can be assigned a higher weight, thus allowing more reliable and coherent information to have a greater influence on the final decision or assessment.

\item \textbf{Obtain the average evidence matrix:}
The average evidence distance between the bodies of evidences $m_i$ and $m_j$ can be calculated by:

\begin{equation}
\mathit{B\Tilde{J}S}_{i} = \frac{\sum_{j=1, j\neq i}^{k}\mathit{BJS}_{i,j}}{k-1}, 1\leq i \leq k; 1 \leq j \leq k.
\label{eq:AEJS}
\end{equation}
\item \textbf{Calculate the support degree of the evidence:}
The support degree $Sup_i$ of the body of evidence $m_i$ is defined as follows:
\begin{equation}
Sup_{i} = \frac{1}{\mathit{B\Tilde{J}S}_{i}}, 1\leq i \leq k.
% \label{eq:AEJS}
\end{equation}
\item \textbf{Compute the credibility degree of the evidence:}
The credibility degree $Crd_i$ of the body of the evidence $m_i$ is defined as follows:
\begin{equation}
    Crd_i = \frac{Sup(m_i)}{\sum_{s=1}^{k}{Sup(m_s)}} ,\quad 1\leq i \leq k.
\label{eq:CRD}
\end{equation}
\item \textbf{Measure the belief entropy of the evidence:}
The belief entropy of the evidence $m_i$ is calculated by:
\begin{equation}
    E_d = - \sum_i m(A_i) \log \frac{m(A_i)}{2^{|A_i|} - 1}. 
\end{equation}
\item \textbf{Measure the information volume of the evidence:}
In order to avoid allocating zero weight to the evidences in some cases, we use the information volume $IV_i$ to measure the uncertainty of the evidence $m_i$ as below:
\begin{equation}
    IV_i = e^{E_d} = e^{- \sum_i m(A_i) \log \frac{m(A_i)}{2^{|A_i|} - 1}} ,\quad 1\leq i \leq k.
\end{equation}

\item \textbf{Normalize the information volume of the evidence:}
The information volume of the evidence $m_i$ is normalized as below, which is denoted as 
$\Tilde{I}V_i$:
\begin{equation}
    \Tilde{I}V_i = \frac{IV_i}{\sum_{s=1}^k IV_s} ,\quad 1\leq i \leq k.
\end{equation}
\item \textbf{Adjust the credibility degree of the evidence:}
Based on the information volume $\Tilde{I}V_i$ the credibility degree $Crd_i$ of the evidence $m_i$ will be adjusted, denoted as $ACrd_i$:
\begin{equation}
    ACrd_i = Crd_i \times \Tilde{I}V_i ,\quad 1\leq i \leq k.
\end{equation}
\item \textbf{Normalize the adjusted credibility degree of the evidence:}
The adjusted credibility degree which is denoted as $ \Tilde{A}Crd_i$ 
 is normalized that is considered as the final weight in terms of each evidence $m_i$:
\begin{equation}
    \Tilde{A}Crd_i = \frac{ACrd_i}{\sum_{s=1}^k ACrd_s} ,\quad 1\leq i \leq k.
\end{equation}
\item \textbf{Compute the weighted average evidence:}
On account of the final weight $\Tilde{A}Crd_i$ of each evidence $m_i$, the weighted average evidence $\mathit{WAE}(m)$ will be obtained as follows:
\begin{equation}
    \mathit{WAE}(m) = \sum_{i=1}^k (\Tilde{A}Crd_i \times m_i) ,\quad 1\leq i \leq k.
\end{equation}
\item \textbf{Combine the weighted average evidence by utilizing the Dempster's rule of combination:}
The weighted average evidence $\mathit{WAE}(m)$ is fused via the Dempster’s combination rule:
\begin{equation}
m_{\text{combined}}(C) = \frac{\sum_{A \cap B = C} m_1(A) \cdot m_2(B)}{1 - \sum_{A \cap B = \emptyset} m_1(A) \cdot m_2(B)}
\label{eq:app_BPA}
\end{equation}
by $(k-1)$ times, if there are k number of evidences. Then, the final combination result of multi-evidences can be obtained.
\item \textbf{Converting probabilities to reward:}
The penultimate combined belief for each action that is denoted as $ m_{\text{combined}}(C)$ is normalized and considered as the final reward.  

\begin{equation}
    \mathit{BPA}_{a_i} = \frac{m_{\text{combined}}(a_j)}{\sum_{j=1}^km_{\text{combined}}(a_j)},\quad 1\leq i \leq k.
\end{equation}
% \[
% BPA_{a_i} = (m_1 \oplus m_2 \oplus m_3 \oplus m_4 \oplus m_5)(\{a_i\}) ,\quad 1\leq i \leq k.
% \]


$\mathit{BPA}_{a_i}$ is the reward for the action $a_i$. 

\end{enumerate}

% \section{Pseudo-Code}
% \label{app:Pseudo_Code}
% Below we presents the pseudo-code of the AMULED framework. This algorithm employs PPO to iteratively update the policy and value function based on environmental feedback. The framework integrates reward shaping to balance primary and secondary objectives and incorporates fine-tuning through reinforcement learning with human-like feedback (RLHF) from moral clusters, using KL divergence and belief aggregation to guide agent behavior.

% \begin{algorithm}
% \caption{AMULED Framework}
% \begin{algorithmic}[1]

% \Require Set of moral clusters and initial policy parameters $\actorParams$, $\criticParams$
% \State Initialize policy $\pi_{\actorParams}$ and value function $V_{\criticParams}$ using Proximal Policy Optimization (PPO)~\cite{schulman2017proximal}
% \For{each episode}
%     \State Collect trajectories of state-action-reward tuples $(s, a, r)$ from the environment
%     \State Compute the advantage function $A^{\pi_{\actorParams}}(s, a)$ using Generalized Advantage Estimation (GAE)
    
%     \State \textbf{Update Policy}:
%     \State Update policy parameters $\actorParams$ by optimizing
%     \[
%     \actorParams_{k+1} = \arg\min_{\actorParams} \mathbb{E}_{t} \left[ \frac{\pi_{\actorParams}(a | s)}{\pi_{\actorParams_{k}}(a | s)} A^{\pi_{\actorParams_{k}}}(s, a) \cdot g(\epsilon, A^{\pi_{\actorParams_{k}}}(s, a)) \right]
%     \]
%     where $g(\epsilon, A)$ represents advantage normalization and value clipping.

%     \State \textbf{Update Value Function}:
%     \State Update value function parameters $\criticParams$ by minimizing the error:
%     \[
%     \criticParams_{k+1} = \arg \min_{\criticParams} \mathbb{E}_{t} \left[V_{\criticParams}(s_t) - R_t\right]^2
%     \]

%     \State \textbf{Reward Shaping}:
%     \State Define rewards at each timestep $t$ as:
%     \[
%     r_t = \baseReward + c \cdot \rewardShaping
%     \]
%     where $r_{\text{base}}$ incentivizes the primary goal and $\rewardShaping$ addresses secondary goals.

%     \If{Fine-tuning with Human Feedback}
%         \State Initialize base policy $\pi_{\text{base}}$ from previously trained parameters
%         \State Define new reward for fine-tuning as:
%         \[
%         r_{\text{base} = -\lambda_{\text{KL}} D_{\text{KL}}\left(\fineTuneModel(a | s) \parallel \baseModel(a | s)\right)
%         \]
%         \[
%         r_{\text{shaping}} = f_{\text{BA}}(\mathbf{B})\hspace{1cm}  \leftarrow \textbf{Eqs. \eqref{eq:app_DMM}--\eqref{eq:app_BPA}}
%         \]
%         where $\lambda_{\text{KL}}$ is a regularization coefficient, and matrix $\mathbf{B}$ represents the belief values from moral agents.
%     \EndIf

%     \State \textbf{Fine-tuning Training Loop}:
%     \For{$T_{\text{finetune}}$ timesteps}
%         \State Train the fine-tuned policy $\fineTuneModel$ using PPO with feedback rewards $r_{\text{base}}$ and $r_{\text{shaping}}$
%     \EndFor
% \EndFor

% \end{algorithmic}
% \end{algorithm}


% \section*{Multi-Morality Fusion Approach:}
% Steps involved are:
% \begin{enumerate}
%   \item \textbf{Input Data from Moral Theories:} Gather data from multiple moral theories or frameworks. Each theory provides its own evidence or belief about the morality of actions or decisions that can be taken.
  
%   \item \textbf{Construct Frame of Discernment:} For each moral theory, construct a frame of discernment based on the principles and values it espouses. These frameworks represent the uncertainty and confidence associated with the moral judgments provided by each theory.
  
%   \item \textbf{Compute Belief Divergence:} Calculate the belief divergence measure between pairs of frameworks from different moral theories. This step helps in understanding how different the moral judgments are across various theories.
  
%   \item \textbf{Weighted Fusion Using Divergence and Entropy:} Use the belief divergence measure and belief entropy to weight the fusion process. Moral theories with more similar judgments (lower divergence) or lower uncertainty (lower entropy) might be given higher weight in the fusion process.
  
%   \item \textbf{Combine Frame of Discernment:} Combine the frameworks from different moral theories using a fusion rule. This rule could be based on the weighted average, consensus, or other methods that take into account the divergence and entropy measures.
  
%   \item \textbf{Output Fused Frame of Discernment:} Obtain a fused frame of discernment that represents a more informed and robust assessment of the moral implications of actions or decisions than any individual moral theory could provide alone.
% \end{enumerate}

% \section{Calculate the morality degree of the actions}


% \[
% \begin{aligned}
% H(m_1) &= - [0.5 \log 0.5 + 0.2 \log 0.2 + 0.1 \log 0.1 + 0.2 \log 0.2] = 0.529 \\
% H(m_2) &= - [0.4 \log 0.4 + 0.3 \log 0.3 + 0.1 \log 0.1 + 0.2 \log 0.2] = 0.5558 \\
% H(m_3) &= - [0.3 \log 0.3 + 0.3 \log 0.3 + 0.2 \log 0.2 + 0.2 \log 0.2] = 0.5933 \\
% H(m_4) &= - [0.2 \log 0.2 + 0.4 \log 0.4 + 0.1 \log 0.1 + 0.3 \log 0.3] = 0.5558 \\
% \end{aligned}
% \]

% \section*{Credibility Degrees}

% \[
% \begin{aligned}
% \text{Cr}(m_1) &= \frac{1}{0.529} = 1.89 \\
% \text{Cr}(m_2) &= \frac{1}{0.5558} = 1.8 \\
% \text{Cr}(m_3) &= \frac{1}{0.5933} = 1.69 \\
% \text{Cr}(m_4) &= \frac{1}{0.5558} = 1.8 \\
% \end{aligned}
% \]

% \section*{Combined Moral Functions}

% Using Dempster's rule of combination, we combine the morality functions \(m_1\) to \(m_4\):

% \[
% (m_1 \oplus m_2 \oplus m_3 \oplus m_4)(\{a_1\}) = 0.5 \times 0.4 \times 0.3 \times 0.2 = 0.012
% \]

% \[
% (m_1 \oplus m_2 \oplus m_3 \oplus m_4)(\{a_2\}) = 0.2 \times 0.3 \times 0.3 \times 0.4 = 0.0072
% \]

% \[
% (m_1 \oplus m_2 \oplus m_3 \oplus m_4)(\{a_3\}) = 0.1 \times 0.1 \times 0.2 \times 0.1 = 0.0002
% \]

% Normalizing the credence under all relevant moralities for action $a_1,a_2, a_3$ of 0.012, 0.0072, and 0.0002 are 0.6186, 0.3711, and 0.0103, respectively. 


% We use the computed final credence value for each action as the intrinsic reward for the agent. Specifically, if the agent takes action $a_1$, it receives a reward of 0.6186. For $a_2$, the reward is 0.3711, and for $a_3$, it is 0.0103.


% \begin{figure*}[htbp]
%   \centering
%   \includegraphics[width=1\linewidth]{images/1-s2.0-S1566253517305584-gr2.jpg}
%   \caption{The flowchart of the proposed method \cite{xiao2019multi}}
%   \label{fig:BJS}
% \end{figure*}





% \section{Notes for understanding BJS}

% Step 1: Compute Belief Jensen–Shannon divergence measure matrix, namely, a distance measure matrix. 


% Reasoning: In the context of evidence theory, particularly in scenarios involving multi-sensor data fusion or combining information from multiple sources, computing distance measures (such as belief divergence measures) between bodies of evidence serves several important purposes:

% Assessing Consistency: Different sensors or sources may provide evidence or beliefs about the same phenomenon, but they might not always agree. Computing distance measures helps to quantify how much different bodies of evidence diverge or disagree with each other. This provides a measure of consistency or inconsistency between different sources of information.

% Weighting in Fusion Processes: When fusing information from multiple sources, it's crucial to consider the reliability and consistency of each source. Bodies of evidence that are more consistent with each other (i.e., have lower divergence measures) can be given higher weights in the fusion process. This ensures that more reliable and coherent information contributes more to the final decision or assessment.

% Step 2: The average evidence distance 

% Reasoning: By calculating the average evidence distance, you can obtain a single numerical value that represents the average dissimilarity between all pairs of bodies of evidence. This measure provides an overall assessment of the consistency or inconsistency among the sources of evidence.

% Step 3: The support degree of the body of evidence.

% Reasoning: The support degree quantitatively expresses the level of confidence or belief that a body of evidence assigns to a specific hypothesis or proposition. It provides a numerical measure indicating how strongly the evidence supports the hypothesis relative to other possible hypotheses.

% Step 4: The credibility degree of the body of the evidence

% Reasoning: The credibility degree provides a quantitative measure of how reliable or trustworthy the body of evidence is perceived to be. It helps in distinguishing between more reliable and less reliable sources of information.

% Step 5: Measure the information volume of the evidences

% Reasoning: The "information volume" of evidence refers to a measure that quantifies the amount or volume of information conveyed by a body of evidence. Here’s how you can understand and measure the information volume of evidences. 
% Measuring the information volume of evidences involves calculating the entropy weighted by the belief assignments across all subsets of the frame of discernment. This measure provides a quantitative assessment of the richness and diversity of information conveyed by the evidence, aiding in decision making and evidence fusion processes within evidence theory.

% Step 6: Generate and fuse the weighted average evidence

% Reasoning: involves combining information from multiple sources or bodies of evidence in a manner that accounts for their respective strengths or reliability. 



\end{appendices}

\end{document}
