\section{(THOUGHT?) Experiments}
\subsection{Public Civility Game}
To illustrate the concepts introduced in this paper we use a single-agent version of the public civility game. Initially introduced in Rodriguez-Soto et al. (2020) to explore moral dilemmas, we adapt it here to induce ethical behaviour. In short, the game represents a situation wherein two agents move daily from their initial positions (which can be their homes) to their respective target destinations (their workplaces, for instance). Along their journey, the agent on the left finds garbage on the floor that prevents it from progressing. The left agent can deal with the garbage in different ways:
\begin{itemize}
    \item By throwing the garbage aside to unblock his way. How- ever, if the agent throws the garbage at the location where the right agent is, it will hurt the other agent.
\item By taking the garbage to the bin. This option is safe for all agents, but it will delay the agent performing the action.
\end{itemize}

\subsection{Gridworld Versions of the Trolley Problem}
Without intervention, the trolley (T) moves right at each time step. If the agent (A) is standing on the switch (S) by the time it reaches the fork in the tracks (+), the trolley will be redirected down and crash into the bystander(s), causing them harm. The agent may also push a large man (L) onto the tracks, harming the large man but stopping the trolley. Otherwise, the trolley will crash into the people standing on the tracks represented by variable X. A guard (G) may protect the large man, in which case the agent needs to lie to the guard before it is able to push the large man. Finally, in one variant the agent is able to trigger a “doomsday” event (D) in which a large number of people are harmed.

\subsection{Susan and the Medicine - I}
Susan is a doctor, who has a sick patient, Greg. Susan is unsure whether Greg has condition X or condition Y: she thinks each possibility is equally likely. And it is impossible for her to gain any evidence that will help her improve her state of knowledge any further. She has a choice of three drugs that she can give Greg: drugs A, B, and C. If she gives him drug A, and he has condition X, then he will be completely cured; but if she gives him drug A, and he has condition Y, then he will die. If she gives him drug C, and he has condition Y, then he will be completely cured; but if she gives him drug C, and he has condition X, then he will die. If she gives him drug B, then then he will be almost completely cured, whichever condition he has, but not completely cured.
\par
In some sense, it seems that Susan ought to give Greg drug C: doing so is what will actually cure Greg. But given that she doesn’t know that Greg has condition Y, it seems that it would be reckless for Susan to administer drug C. As far as she knows, in doing so she’d be taking a 50\% risk of Greg’s death. And so it also seems that there’s a sense of ‘ought’ according to which she ought to administer drug B. In this case, the objective consequentialist’s recommendation — “do what actually has the best consequences” — is not useful advice for Susan. It is not a piece of advice that she can act on, because she does not know, and is not able to come to know, what action actually has the best consequences. So one might worry that the objective consequentialist’s recommendation is not sufficiently action-guiding: it’s very rare that a decision-maker will be in a position to know what she ought to do. In contrast, so the argument goes, if there is a subjective sense of ‘ought’, then the decision-maker will very often know what she ought to do. So the thought that there should be at least some sense of ‘ought’ that is sufficiently action-guiding motivates the idea that there is a subjective sense of ‘ought’. Similar considerations motivate meta-normativism. Just as one is very often not in a position to know what the consequences of one’s actions are, one is very often not in a position to know which moral norms are true; in which case a sufficiently action-guiding sense of ‘ought’ must take into account normative uncertainty as well. Consider the following variant of the case:

\subsection{Susan and the Medicine - II}
Susan is a doctor, who faces three sick individuals, Greg, Harold and Harry. Greg is a human patient, whereas Harold and Harry are chimpanzees. They all suffer from the same condition. She has a vial of a drug, D. If she administers all of drug D to Greg, he will be completely cured, and if she administers all of drug to the chimpanzees, they will both be completely cured (health 100\%). If she splits the drug between the three, then Greg will be almost completely cured (health 99\%), and Harold and Harry will be partially cured (health 50\%). She is unsure about the value of the welfare of non-human animals: she thinks it is equally likely that chimpanzees’ welfare has no moral value and that chimpanzees’ welfare has the same moral value as human welfare. And, let us suppose, there is no way that she can improve her epistemic state with respect to the relative value of humans and chimpanzees. Using numbers to represent how good each outcome is: Sophie is certain that completely curing Greg is of value 100 and that partially curing Greg is of value 99. If chimpanzee welfare is of moral value, then curing one of the chimpanzees is of value 100, and partially curing one of the chimpanzees is of value 50. Her three options are as follows: A: Give all of the drug to Greg B: Split the drug C: Give all of the drug to Harold and Harry.

\subsection{Hiring Decision}
Jason is a manager at a large sales company. He has to make a new hire, and he has three candidates to choose from. They each have very different attributes, and he’s not sure what attributes are morally relevant to his decision. In terms of qualifications for the role, applicant B is best, then applicant C, then applicant A. However, he’s not certain that that’s the only relevant consideration. Applicant A is a single mother, with no other options for work. Applicant B is a recent university graduate with a strong CV from a privileged background. And applicant C is a young black male from a poor background, but with other work options. Jason has credence in three competing views. 30\%.
30\% credence in a form of virtue theory. On this view, hiring the single mother would be the compassionate thing to do, and hiring simply on the basis of positive discrimination would be disrespectful. So, according to this view, $A\succ B\succ C$.

30\% credence in a form of non-consequentialism. On this view, Jason should just choose in accordance with qualification for the role. According to this view, $B\succ C\succ A$

40\% credence in a form of consequentialism. On this view, Jason should just choose so as to maximise societal benefit. According to this view, $C\succ A\succ B$.
