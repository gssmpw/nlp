\section{Modeling Morality as Intrinsic Reward}
In the previous section, we presented the proposed cluster of moral theories. These five clusters serve as a moral compass, guiding the agent in decision-making under varying degrees of credence and uncertainty about the future outcomes of chosen decisions. We assume that the agent has a belief or credence \(c_{ij}\) in a particular theory \(i\) for a particular decision \(j\). These credences are treated as probabilities and, therefore, sum up to one across all theories for a given decision.

For now, we assume the credences of these theories are set and fixed, either by the system designer’s beliefs or by surveying relevant stakeholders. However, for future research, it would be extremely beneficial to derive the credence values for a moral theory based on the task context and the type of agent.

Let's consider a toy example to understand this better. For example, there is a decision-making task at hand that has three choices. Let's call them actions $(a_1, a_2,a_3)$. Based on the expert beliefs, four moral clusters $(m_1,m_2,m_3,m_4)$ are relevant and should be considered for decision-making. Expert credence assignment would look something like this - 
\[
\begin{aligned}
m_1(\{a_1\}) &= 0.5 \\
m_1(\{a_2\}) &= 0.2 \\
m_1(\{a_3\}) &= 0.1 \\
m_1(\{a_1, a_2\}) &= 0.2 \\
\end{aligned}
\]

\[
\begin{aligned}
m_2(\{a_1\}) &= 0.4 \\
m_2(\{a_2\}) &= 0.3 \\
m_2(\{a_3\}) &= 0.1 \\
m_2(\{a_1, a_3\}) &= 0.2 \\
\end{aligned}
\]

\[
\begin{aligned}
m_3(\{a_1\}) &= 0.3 \\
m_3(\{a_2\}) &= 0.3 \\
m_3(\{a_3\}) &= 0.2 \\
m_3(\{a_2, a_3\}) &= 0.2 \\
\end{aligned}
\]

\[
\begin{aligned}
m_4(\{a_1\}) &= 0.2 \\
m_4(\{a_2\}) &= 0.4 \\
m_4(\{a_3\}) &= 0.1 \\
m_4(\{a_1, a_2\}) &= 0.3 \\
\end{aligned}
\]

\begin{table}[h!]
\centering
\begin{tabular}{|c|c|c|c|c|}
\hline
Action Set & $m_1$ & $m_2$ & $m_3$ & $m_4$ \\
\hline
$\{a_1\}$ & 0.5 & 0.4 & 0.3 & 0.2 \\
$\{a_2\}$ & 0.2 & 0.3 & 0.3 & 0.4 \\
$\{a_3\}$ & 0.1 & 0.1 & 0.2 & 0.1 \\
$\{a_1, a_2\}$ & 0.2 & - & - & 0.3 \\
$\{a_1, a_3\}$ & - & 0.2 & - & - \\
$\{a_2, a_3\}$ & - & - & 0.2 & - \\
\hline
\end{tabular}
\caption{Credence Assignments for Actions}
\label{table:credences}
\end{table}

\section*{Multi-Morality Fusion Approach:}
Steps involved are - 
\begin{enumerate}
  \item \textbf{Input Data from Moral Theories:} Gather data from multiple moral theories or frameworks. Each theory provides its own evidence or belief about the morality of actions or decisions that can be taken.
  
  \item \textbf{Construct Frame of Discernment:} For each moral theory, construct a frame of discernment based on the principles and values that it espouses. These frameworks represent the uncertainty and confidence associated with the moral judgments provided by each theory.
  
  \item \textbf{Compute Belief Divergence:} Calculate the belief divergence measure between pairs of frameworks from different moral theories. This step helps to understand how different moral judgments are across various theories.
  
  \item \textbf{Weighted Fusion Using Divergence and Entropy:} Use the belief divergence measure and belief entropy to weight the fusion process. Moral theories with more similar judgments (lower divergence) or lower uncertainty (lower entropy) might be given higher weight in the fusion process.
  
  \item \textbf{Combine Frame of Discernment:} Combine the frameworks from different moral theories using a fusion rule. This rule could be based on the weighted average, consensus, or other methods that take into account the above divergence and entropy measures.
  
  \item \textbf{Output Fused Frame of Discernment:} Obtain a fused frame of discernment that represents a more informed and robust assessment of the moral implications of actions or decisions than what any individual moral theory could provide alone.
\end{enumerate}

\subsection{Calculate the Morality Degree of Actions}


\[
\begin{aligned}
H(m_1) &= - [0.5 \log 0.5 + 0.2 \log 0.2 + 0.1 \log 0.1 + 0.2 \log 0.2] = 0.529 \\
H(m_2) &= - [0.4 \log 0.4 + 0.3 \log 0.3 + 0.1 \log 0.1 + 0.2 \log 0.2] = 0.5558 \\
H(m_3) &= - [0.3 \log 0.3 + 0.3 \log 0.3 + 0.2 \log 0.2 + 0.2 \log 0.2] = 0.5933 \\
H(m_4) &= - [0.2 \log 0.2 + 0.4 \log 0.4 + 0.1 \log 0.1 + 0.3 \log 0.3] = 0.5558 \\
\end{aligned}
\]

\section*{Credibility Degrees}

\[
\begin{aligned}
\text{Cr}(m_1) &= \frac{1}{0.529} = 1.89 \\
\text{Cr}(m_2) &= \frac{1}{0.5558} = 1.8 \\
\text{Cr}(m_3) &= \frac{1}{0.5933} = 1.69 \\
\text{Cr}(m_4) &= \frac{1}{0.5558} = 1.8 \\
\end{aligned}
\]

\section*{Combined Moral Functions}

Using Dempster's rule of combination, we combine the morality functions \(m_1\) to \(m_4\):

\[
(m_1 \oplus m_2 \oplus m_3 \oplus m_4)(\{a_1\}) = 0.5 \times 0.4 \times 0.3 \times 0.2 = 0.012
\]

\[
(m_1 \oplus m_2 \oplus m_3 \oplus m_4)(\{a_2\}) = 0.2 \times 0.3 \times 0.3 \times 0.4 = 0.0072
\]

\[
(m_1 \oplus m_2 \oplus m_3 \oplus m_4)(\{a_3\}) = 0.1 \times 0.1 \times 0.2 \times 0.1 = 0.0002
\]

Normalizing the credence under all relevant moralities for action $a_1,a_2, a_3$ of 0.012, 0.0072, and 0.0002 are 0.6186, 0.3711, and 0.0103, respectively. 


We use the computed final credence value for each action as the intrinsic reward for the agent. Specifically, if the agent takes action $a_1$, it receives a reward of 0.6186. For $a_2$, the reward is 0.3711, and for $a_3$, it is 0.0103.


\begin{figure*}[htbp]
  \centering
  \includegraphics[width=1\textwidth]{images/1-s2.0-S1566253517305584-gr2.jpg}
  \caption{The flowchart of the proposed method \cite{xiao2019multi}}
  \label{fig:BJS}
\end{figure*}





\subsection{Notes for Understanding BJS}

Step 1: Compute the Belief Jensen–Shannon divergence measure matrix, namely, a distance measure matrix. 


Reasoning: In the context of evidence theory, particularly in scenarios involving multi-sensor data fusion or combining information from multiple sources, computing distance measures (such as belief divergence measures) between bodies of evidence serves several important purposes:

Assessing Consistency: Different sensors or sources may provide evidence or beliefs about the same phenomenon, but they might not always agree. Computing distance measures helps to quantify how much different bodies of evidence diverge or disagree with each other. This provides a measure of consistency or inconsistency between different sources of information.

Weighting in Fusion Processes: When fusing information from multiple sources, it's crucial to consider the reliability and consistency of each source. Bodies of evidence that are more consistent with each other (i.e., have lower divergence measures) can be given higher weights in the fusion process. This ensures that more reliable and coherent information contributes more to the final decision or assessment.

Step 2: Compute the average evidence distance 

Reasoning: By calculating the average evidence distance, one can obtain a single numerical value that represents the average dissimilarity between all pairs of bodies of evidence. This measure provides an overall assessment of the consistency or inconsistency among the sources of evidence.

Step 3: Determine the support degree by the body of evidence.

Reasoning: The support degree quantitatively expresses the level of confidence or belief that a body of evidence assigns to a specific hypothesis or proposition. It provides a numerical measure indicating how strongly the evidence supports the hypothesis relative to other possible hypotheses.

Step 4: Calculate the credibility degree of the body of the evidence

Reasoning: The credibility degree provides a quantitative measure of how reliable or trustworthy the body of evidence is perceived to be. It helps to distinguish between more reliable and less reliable sources of information.

Step 5: Measure the information volume of the evidences

Reasoning: The "information volume" of evidence refers to a measure that quantifies the amount or volume of information conveyed by a body of evidence. Here’s how you can understand and measure the information volume of evidences. 
Measuring the information volume of evidences involves calculating the entropy weighted by the belief assignments across all subsets of the frame of discernment. This measure provides a quantitative assessment of the richness and diversity of information conveyed by the evidence, aiding in decision making and evidence fusion processes within evidence theory.

Step 6: Generate and fuse the weighted average evidence

Reasoning: This involves combining information from multiple sources or bodies of evidence in a manner that accounts for their respective strengths or reliability. 



