%%%%%%%%%my own package
\usepackage[framemethod=tikz]{mdframed}
\definecolor{mycolor}{rgb}{0.122, 0.435, 0.698}
%\newmdenv[innerlinewidth=0.5pt, linecolor=mycolor,innerleftmargin=6pt,
%innerrightmargin=6pt,innertopmargin=6pt,innerbottommargin=6pt,topline=true]{notebox}
\usepackage{tcolorbox}
\newtcolorbox{notebox}{
	colback=red!5!white,
	colframe=red!100!black,
	coltext=red!100!black,arc=0pt,halign=center,}
\DeclareMathOperator*{\argmax}{arg\,max}
\DeclareMathOperator*{\argmin}{arg\,min}
%%%%%%%%%my own package

% updated with editorial comments 8/9/2021

\DeclareMathOperator{\corr}{corr}
\usepackage{tikz}

\usepackage{amsthm}
%\newtheorem{prop}{Proposition}[section]
\newtheorem{prop}{Proposition}
\newtheorem{lem}[prop]{Lemma}
\newtheorem{cor}[prop]{Corollary}
\newtheorem{thm}[prop]{Theorem}
\newtheorem{fact}[prop]{Fact}

\theoremstyle{definition}
\newtheorem{defi}[prop]{Definition}
\newtheorem*{question}{Question}
\newtheorem*{rem}{Remark}
\newtheorem*{ex}{Example}
\newtheorem*{assumption}{Assumption}
\newtheorem*{hypo}{Hypothesis Statement}


\newcommand{\R}{\mathbb{R}}
\newcommand{\N}{\mathbb{N}}
\newcommand{\Z}{\mathbb{Z}}
\newcommand{\Q}{\mathbb{Q}}
\newcommand{\LR}{\mathcal{L}_\R}
\newcommand{\sumdata}{\sum_{i=1}^{n}}
\newcommand{\sumfeat}{\sum_{i=1}^{d}}
