\pdfoutput=1
\documentclass[journal]{IEEEtran}
%\documentclass[]{report}
\usepackage{amsmath,amsfonts}
\usepackage{algorithmic}
\usepackage{algorithm}
\usepackage{array}
\usepackage[caption=false,font=normalsize,labelfont=sf,textfont=sf]{subfig}
\usepackage{textcomp}
\usepackage{stfloats}
\usepackage{url}
\usepackage{verbatim}
\usepackage{graphicx}
\usepackage{cite}
\usepackage{hyperref}
\usepackage{fontawesome5}
\hyphenation{op-tical net-works semi-conduc-tor IEEE-Xplore}


%%%%%%%%%%%%%%%%%%%%%%%%%%%%%%%%%%%%%%%%%%%%%%%%%%%%%%%%%%%%%%%%%%
%%% Comment out to use biblatex instead of bibtex
%%%%%%%%%%%%%%%%%%%%%%%%%%%%%%%%%%%%%%%%%%%%%%%%%%%%%%%%%%%%%%%%%%
%\def\UseBibLatex{1}

%%%%%%%%%%%%%%%%%%%%%%%%%%%%%%%%%%%%%%%%%%%%%%%%%%%%%%%%%%%%%%%%%
% Put all your private style files/class style files in the styles/
% subdirectory. The following command guarantee that latex would find
% it.
%%%%%%%%%%%%%%%%%%%%%%%%%%%%%%%%%%%%%%%%%%%%%%%%%%%%%%%%%%%%%%%%%

\makeatletter
\def\input@path{{styles/}}
\makeatother


%%%%%%%%%%%%%%%%%%%%%%%%%%%%%%%%%%%%%%%%%%%%%%%%%%%%%%%%%%%%%%%%%%
% A modified usepackge command that checks for style files in the
% styles/ subdirectory.
%%%%%%%%%%%%%%%%%%%%%%%%%%%%%%%%%%%%%%%%%%%%%%%%%%%%%%%%%%%%%%%%%% 
\newcommand{\UsePackage}[1]{%
  \IfFileExists{styles/#1.sty}{%
      \usepackage{styles/#1}%
   }{%
      \IfFileExists{../styles/#1.sty}{%
         \usepackage{../styles/#1}%
      }{%
         \usepackage{#1}%
      }%
   }%
}


\usepackage{natbib}
\usepackage[T1]{fontenc}
\usepackage{lmodern}
\usepackage{textcomp}

\usepackage{amsmath}%
\usepackage{amssymb}%
\usepackage[table]{xcolor}%

\setlength{\marginparwidth}{6cm} 
\usepackage{todonotes}
\usepackage[in]{fullpage}%

\usepackage[amsmath,thmmarks]{ntheorem}%
\theoremseparator{.}%

\usepackage{titlesec}%
\titlelabel{\thetitle. }%
\usepackage{xcolor}%
\usepackage{mleftright}%
\usepackage{xspace}%
\usepackage{graphicx}
\usepackage{hyperref}%
\usepackage[parfill]{parskip}
\usepackage{bbm}

\newcommand{\hrefb}[3][black]{\href{#2}{\color{#1}{#3}}}%

\usepackage{hyperref}%
\hypersetup{%
      unicode,
      breaklinks,%
      colorlinks=true,%
      urlcolor=[rgb]{0.25,0.0,0.0},%
      linkcolor=[rgb]{0.5,0.0,0.0},%
      citecolor=[rgb]{0,0.2,0.445},%
      filecolor=[rgb]{0,0,0.4},
      anchorcolor=[rgb]={0.0,0.1,0.2}%
}
\usepackage[ocgcolorlinks]{ocgx2}

%%%%%%%%%%%%%%%%%%%%%%%%%%%%%%%%%%%%%%%%%%%%%%%%%%%%%%%%%%%%%%%%%%%%%%%%
% Defining theorem like environments
%

\theoremseparator{.}%

\theoremstyle{plain}%
\newtheorem{theorem}{Theorem}

\newtheorem{lemma}{Lemma}
\newtheorem{conjecture}[theorem]{Conjecture}
\newtheorem{corollary}[theorem]{Corollary}
\newtheorem{claim}[theorem]{Claim}%
\newtheorem{fact}[theorem]{Fact}
\newtheorem{observation}[theorem]{Observation}
\newtheorem{invariant}[theorem]{Invariant}
\newtheorem{question}[theorem]{Question}
\newtheorem{proposition}[theorem]{Proposition}
\newtheorem{prop}[theorem]{Proposition}
\newtheorem{openproblem}[theorem]{Open Problem}

\theoremstyle{plain}%
\theoremheaderfont{\sf} \theorembodyfont{\upshape}%
\newtheorem*{remark:unnumbered}[theorem]{Remark}%
\newtheorem*{remarks}[theorem]{Remarks}%
\newtheorem{remark}[section]{Remark}%
\newtheorem{definition}[theorem]{Definition}
\newtheorem{defn}[theorem]{Definition}
\newtheorem{example}[theorem]{Example}
\newtheorem{exercise}[theorem]{Exercise}
\newtheorem{problem}[theorem]{Problem}
\newtheorem{xca}[theorem]{Exercise}
\newtheorem{exercise_h}[theorem]{Exercise}
\newtheorem{assumption}[theorem]{Assumption}%

% Proof environment
\newcommand{\myqedsymbol}{\rule{2mm}{2mm}}

\theoremheaderfont{\em}%
\theorembodyfont{\upshape}%
\theoremstyle{nonumberplain}%
\theoremseparator{}%
\theoremsymbol{\myqedsymbol}%
\newtheorem{proof}{Proof:}%

\newtheorem{proofof}{Proof of\!}%

% theorem block end
%%%%%%%%%%%%%%%%%%%%%%%%%%%%%%%%%%%%%%%%%%%%%%%%%%%%%%%%%%%%%%%%%%%%


%%%%%%%%%%%%%%%%%%%%%%%%%%%%%%%%%%%%%%%%%%%%%%%%%%%%%%%%%%%%%%%%%% 5
% Color emph

\providecommand{\emphind}[1]{}%
\renewcommand{\emphind}[1]{\emph{#1}\index{#1}}

\definecolor{blue25emph}{rgb}{0, 0, 11}

\providecommand{\emphic}[2]{}
\renewcommand{\emphic}[2]{\textcolor{blue25emph}{%
      \textbf{\emph{#1}}}\index{#2}}

\providecommand{\emphi}[1]{}%
\renewcommand{\emphi}[1]{\emphic{#1}{#1}}

\definecolor{almostblack}{rgb}{0, 0, 0.3}

\providecommand{\emphw}[1]{}%
\renewcommand{\emphw}[1]{{\textcolor{almostblack}{\emph{#1}}}}%

\providecommand{\emphOnly}[1]{}%
\renewcommand{\emphOnly}[1]{\emph{\textcolor{blue25}{\textbf{#1}}}}

% Color emph - end
%%%%%%%%%%%%%%%%%%%%%%%%%%%%%%%%%%%%%%%%%%%%%%%%%%%%%%%%%%%%%%%%%% 5

%%%%%%%%%%%%%%%%%%%%%%%%%%%%%%%%%%%%%%%%%%%%%%%%%%%%%%%%%%%%%%%%%%%
% Authors thanks
%%%%%%%%%%%%%%%%%%%%%%%%%%%%%%%%%%%%%%%%%%%%%%%%%%%%%%%%%%%%%%%%%%%

\newcommand{\JamesThanks}[1]{%
   \thanks{%
      Department of Computer Science; %
      University of Moochi; %
      102 S. Bad St; %
      Blackstone, SF, 12345, USA; %
      \href{mailto:spam@spam.edu}{spam@spam.edu}; %
      \url{http://spammer.org/}. %
   #1%
   }%
}

%%%%%%%%%%%%%%%%%%%%%%%%%%%%%%%%%%%%%%%%%%%%%%%%%
\newcommand{\james}[1]{%   
\todo[author=James,inline,color=blue!25]{#1}}
\newcommand{\ford}[1]{%   
\todo[author=Ford,inline,color=red!25]{#1}}


%%%%%%%%%%%%%%%%%%%%%%%%%%%%%%%%%%%%%%%%%%%%%%%%%%%%%%%%%%%%%%%%%%%%%%
%    Handling references
%%%%%%%%%%%%%%%%%%%%%%%%%%%%%%%%%%%%%%%%%%%%%%%%%%%%%%%%%%%%%%%%%%%%%%

\newcommand{\HLink}[2]{\hyperref[#2]{#1~\ref*{#2}}}
\newcommand{\HLinkSuffix}[3]{\hyperref[#2]{#1\ref*{#2}{#3}}}

\newcommand{\figlab}[1]{\label{fig:#1}}
\newcommand{\figref}[1]{\HLink{Figure}{fig:#1}}

\newcommand{\thmlab}[1]{{\label{theo:#1}}}
\newcommand{\thmref}[1]{\HLink{Theorem}{theo:#1}}

\newcommand{\remlab}[1]{\label{rem:#1}}
\newcommand{\remref}[1]{\HLink{Remark}{rem:#1}}%

\newcommand{\corlab}[1]{\label{cor:#1}}
\newcommand{\corref}[1]{\HLink{Corollary}{cor:#1}}%

\providecommand{\deflab}[1]{}
\renewcommand{\deflab}[1]{\label{def:#1}}
\newcommand{\defref}[1]{\HLink{Definition}{def:#1}}

\newcommand{\lemlab}[1]{\label{lemma:#1}}
\newcommand{\lemref}[1]{\HLink{Lemma}{lemma:#1}}%

\providecommand{\eqlab}[1]{}%
\renewcommand{\eqlab}[1]{\label{equation:#1}}
\newcommand{\Eqref}[1]{\HLinkSuffix{Eq.~(}{equation:#1}{)}}

%%%%%%%%%%%%%%%%%%%%%%%%%%%%%%%%%%%%%%%%%%%%%%%%%%%%%%%%%%%%%%%%%%%

\newcommand{\remove}[1]{}%
\newcommand{\Set}[2]{\left\{ #1 \;\middle\vert\; #2 \right\}}

\newcommand{\pth}[1]{\mleft(#1\mright)}%

\newcommand{\ProbC}{{\mathbb{P}}}
\newcommand{\ExC}{{\mathbb{E}}}
\newcommand{\VarC}{{\mathbb{V}}}

\newcommand{\Prob}[1]{\ProbC\mleft[ #1 \mright]}
\newcommand{\Ex}[1]{\ExC\mleft[ #1 \mright]}
\newcommand{\Var}[1]{\VarC\mleft[ #1 \mright]}


\newcommand{\ceil}[1]{\mleft\lceil {#1} \mright\rceil}
\newcommand{\floor}[1]{\mleft\lfloor {#1} \mright\rfloor}

\newcommand{\brc}[1]{\left\{ {#1} \right\}}
\newcommand{\set}[1]{\brc{#1}}%

\newcommand{\cardin}[1]{\left\lvert {#1} \right\rvert}%

\renewcommand{\th}{th\xspace}
\newcommand{\ds}{\displaystyle}%

\renewcommand{\Re}{\mathbb{R}}%
\newcommand{\reals}{\Re}%


%%%%%%%%%%%%%%%%%%%%%%%%%%%%%%%%%%%%%%%%%%%%%%%%%%%%%%%%%%%%%%%%%%%%%%%%%
% Defining comptenum environment using enumitem
\usepackage[inline]{enumitem}

\newlist{compactenumA}{enumerate}{5}%
\setlist[compactenumA]{topsep=0pt,itemsep=-1ex,partopsep=1ex,parsep=1ex,%
   label=(\Alph*)}%

\newlist{compactenuma}{enumerate}{5}%
\setlist[compactenuma]{topsep=0pt,itemsep=-1ex,partopsep=1ex,parsep=1ex,%
   label=(\alph*)}%

\newlist{compactenumI}{enumerate}{5}%
\setlist[compactenumI]{topsep=0pt,itemsep=-1ex,partopsep=1ex,parsep=1ex,%
   label=(\Roman*)}%

\newlist{compactenumi}{enumerate}{5}%
\setlist[compactenumi]{topsep=0pt,itemsep=-1ex,partopsep=1ex,parsep=1ex,%
   label=(\roman*)}%

\newlist{compactitem}{itemize}{5}%
\setlist[compactitem]{topsep=0pt,itemsep=-1ex,partopsep=1ex,parsep=1ex,%
   label=\ensuremath{\bullet}}%


%%%%%%%%%%%%%%%%%%%%%%%%%%%%%%%%%%%%%%%%%%%%%%%%%%%%%%%%%%%%%%%%%%%%%%%%%%

%%%%%%%%%%%%%%%%%%%%%%%%%%%%%%%%%%%%%%%%%%%%%%%%%%%%%%%%%%%%%%%%%%%
% Biblatex....
%
\providecommand{\BibLatexMode}[1]{}
\providecommand{\BibTexMode}[1]{}

\ifx\UseBibLatex\undefined%
  \renewcommand{\BibLatexMode}[1]{}
  \renewcommand{\BibTexMode}[1]{#1}
\else
  \renewcommand{\BibLatexMode}[1]{#1}
  \renewcommand{\BibTexMode}[1]{}
\fi


% Bib latex stuff
\BibLatexMode{%
   \usepackage[bibencoding=utf8,style=alphabetic,backend=biber]{biblatex}%
   \UsePackage{my_biblatex}%
}

%
%%%%%%%%%%%%%%%%%%%%%%%%%%%%%%%%%%%%%%%%%%%%%%%%%%%%%%%%%%%%%%%%%%%

\numberwithin{figure}{section}%
\numberwithin{table}{section}%
% \numberwithin{equation}{section}%



%%%%%%%%%%%%%%%%%%%%%%%%%%%%%%%%%%%%%%%%%%%%%%%%%%%%%%%%%%%%%%%%%%%
%%%%%%%%%%%%%%%%%%%%%%%%%%%%%%%%%%%%%%%%%%%%%%%%%%%%%%%%%%%%%%%%%%%
% Papers specific commands...
%%%%%%%%%%%%%%%%%%%%%%%%%%%%%%%%%%%%%%%%%%%%%%%%%%%%%%%%
%%%%%%%%%%%%%%%%%%%%%%%%%%%%%%%%%%%%%%%%%%%%%%%%%%%%%%%%



%%%%%%%%%%%%%%%%%%%%%%%%%%%%%%%%%%%%%%%%%%%%%%%%%%%%%%%%
%%BeginIpePreamble
%%%%%%%%%%%%%%%%%%%%%%%%%%%%%%%%%%%%%%%%%%%%%%%%%%%%%%%%


%%%%%%%%%%%%%%%%%%%%%%%%%%%%%%%%%%%%%%%%%%%%%%%%%%%%%%%%
%%EndIpePreamble
%%%%%%%%%%%%%%%%%%%%%%%%%%%%%%%%%%%%%%%%%%%%%%%%%%%%%%%%


\begin{document}

\title{Some Insights of Construction of Feature Graph to Learn Pairwise Feature Interactions with Graph Neural Networks}

% \author{Phaphontee Yamchote, Saw Nay Htet Win, Chainarong Amornbunchornvej,~\IEEEmembership{Member,~IEEE},  Thanapon Noraset
%         % <-this % stops a space
%         }
\author{Phaphontee Yamchote, Saw Nay Htet Win, Chainarong Amornbunchornvej,  Thanapon Noraset
        % <-this % stops a space
        }

%\author{Thanapon Noraset}
%
%
%\author{IEEE Publication Technology,~\IEEEmembership{Staff,~IEEE,}
%        % <-this % stops a space
%}

% The paper headers
\markboth{Journal of \LaTeX\ Class Files,~Vol.~14, No.~8, August~2021}%
{Shell \MakeLowercase{\textit{et al.}}: A Sample Article Using IEEEtran.cls for IEEE Journals}

%\IEEEpubid{0000--0000/00\$00.00~\copyright~2021 IEEE}
% Remember, if you use this you must call \IEEEpubidadjcol in the second
% column for its text to clear the IEEEpubid mark.
%\setcounter{page}{0}
%\begin{figure*}[t]
%	\tableofcontents
%\end{figure*}

\maketitle


\begin{abstract}
	Feature interaction is crucial in predictive machine learning models, as it captures the relationships between features that influence model performance. In this work, we focus on pairwise interactions and investigate their importance in constructing feature graphs for Graph Neural Networks (GNNs). Rather than proposing new methods, we leverage existing GNN models and tools to explore the relationship between feature graph structures and their effectiveness in modeling interactions. Through experiments on synthesized datasets, we uncover that edges between interacting features are important for enabling GNNs to model feature interactions effectively. We also observe that including non-interaction edges can act as noise, degrading model performance. Furthermore, we provide theoretical support for sparse feature graph selection using the Minimum Description Length (MDL) principle. We prove that feature graphs retaining only necessary interaction edges yield a more efficient and interpretable representation than complete graphs, aligning with Occam's Razor.
	Our findings offer both theoretical insights and practical guidelines for designing feature graphs that improve the performance and interpretability of GNN models.
\end{abstract}

\begin{IEEEkeywords}
Feature interaction, Graph neural network
\end{IEEEkeywords}

\begin{figure}[H]
	\centering
	\includegraphics[width=1\linewidth]{firstFigure}
	\caption{High-level figure to show the overall objective of our study: Given a dataset consisting of pairwise feature interaction, which structure of feature graph should we use as an input data graph structure for the given GNN model? Specifically, which graph structure can give the highest performance to GNNs? And what are the dependencies between these features that the optimal graph structure tells us?
	}
	\label{fig:firstfigure}
\end{figure}

\section{Introduction}\label{section:intro}

\textbf{Feature Interaction}, one of the key challenges in predictive modeling, is the behavior of datasets whose two or more independent variables have a common influence on their dependent variable.
Mathematically, it refers to operations between features in a conjunctive manner, such as multiplication, rather than treating features in isolation or through simple linear combinations.
For example, if we have a dataset generated by the equation $y = x_1x_2 + x_3$, then we say that $x_1$ and $x_2$ interact with each other in the model of $y$.

Modeling interactions can significantly improve the accuracy of predictions and reduce the complexity of models by augmenting datasets with interaction features for the models to learn from.
In the best-case scenario, even linear models can be used effectively.
For example, it is easier to approximate the previous function by including interaction terms $x_1x_2$ and $x_3$ in the dataset, rather than using more complex models with only the features $x_1, x_2$ and $x_3$ with more complex models.
Although this approach is not straightforward, it improves the interpretability of downstream tasks.

Applications of feature interaction modeling span a wide range of domains, particularly in recommendation systems, where it plays a critical role in improving predictions.
The click-through rate (CTR) problem in recommendation systems is an example of an application highly based on feature interactions.
For example, the age of the user and the type of product may interact in different ways on various platforms.
Younger users may prefer to buy fashion products on one platform, while older users might lean towards formal products on all platforms.

Although many models, including traditional machine learning methods like decision trees, are capable of modeling feature interactions through their hierarchical structures.
For example, we may consider that, in the context of meal preferences, a decision tree could represent the interaction between a type of wine and a kind of food (steak or fish) by placing the wine-related node as a descendant of the food-related node. 
Although this structure provides a clear representation of feature interactions, these models face significant limitations when dealing with complex or high-dimensional interactions.
As the complexity of the interactions increases, the trees often become too large, leading to reduced interpretability.

Hence, we focus on graph-based methods which are more expressive in interpretability with edges between nodes.
Specifically, graph-based methods represent data using graphs and model interactions through Graph Neural Networks (GNNs).
Many graph-based methods have been developed to model feature interactions.
Feature graphs, where nodes represent features, play an important role in GNN-based models for modeling interactions.
However, a major challenge arises because most methods rely on a complete graph, where every possible interaction, i.e., every pair of features, is included.
In addition, no insight and analysis about structures of feature graphs concerning features interactions are studied to make us understand the effect of graph structure to model it.
One of the key questions that arises is the following: What characteristics should a feature graph have to optimally model interactions, instead of relying only on complete graphs?

Understanding the relationship between feature graph structures and their ability to model feature interactions is essential to building efficient and interpretable representations.
Using a complete graph often introduces unnecessary complexity and increases computational overhead.
Identifying and selectively including only the most meaningful edges in a feature graph can be considered one of the feature selection techniques for graph models that can avoid the overfitting issue.

In this work, we attempt to discover this insight, focusing on pairwise interactions, interactions between two features.
We do not propose a new method; instead, we utilize existing methods in our experiments as tools for investigation.
Our key contributions are as follows.
\begin{enumerate}
	
	\item \textbf{Theoretical Foundation for Sparse Feature Graphs:}
	We establish a theoretical foundation for sparse feature graphs using the Minimum Description Length (MDL) principle. By framing the feature graph selection problem in the context of MDL, we prove that retaining only necessary interaction edges minimizes the description length, making it a better choice than using complete graphs. (Section \ref{sec:theoResult})  
	
	\item \textbf{Insights into Feature Graph Structures:}
	%	เน้นที่ finding ที่พบ (theoretical result)
	We provide insights into the structures of feature graphs pertaining the given feature interactions instead of using complete graphs to be used as input graphs for GNN models.
	Our experiment result shows that the edges corresponding to the pair of interacting features are significant and must remain in the feature graphs to maintain the prediction performance. (Section \ref{sec:insight-results})
	
	%EMP_ THEORY and ALGORITHM
	\item \textbf{Empirical Validation on Real-World Datasets:}
	We empirically show that, on real-world datasets, the feature graphs constructed by the pairwise interaction can improve the prediction performance compared with baselines including tree-based algorithms and linear-feature-based algorithms. (Section \ref{sec:caseStudy}) %(แล้วอัลกอจะหาได้ไหมว่ากันอีกเรื่อง)
	
	
	
\end{enumerate}

\section{Preliminary}\label{sec:prelim}
\begin{table}[!h]
	\centering
	\begin{tabular}{|c|l|}
		\hline
		Notation   & Description \\ \hline
		$\R$       & the set of real numbers   \\ \hline
		$\R^d$     & the set of d-dimensional real-valued vector   \\ \hline
		$x$        & a real feature value  \\ \hline
		$\vec{x}$  & a real-valued feature vector   \\ \hline
		$G$        & a graph   \\ \hline
		$V$ & a set of nodes of a graph $G$   \\ \hline
		$E$ & a set of edges of a graph $G$   \\ \hline
		$A$ & an adjacent matrix of a graph $G$   \\ \hline
		$X$ & an embedding matrix of a graph $G$   \\ \hline
		$\mathcal{N}(v)$ & a set of nodes adjacent to the node $v$ in a graph $G$   \\ \hline
		$\mathcal{G}(d)$ & a set of feature graphs of $d$ nodes $v_1, \dots, v_d$    \\ \hline
		% $\mathfrak{G}(d)$ & a quotient set of $\mathcal{G}(d)$ by reachable relation    \\ \hline
		$v$ & a node of a graph    \\ \hline
		$e_{uv}$ & an edge from a node $u$ to a node $v$    \\ \hline
		$\vec{h}_i^{l}$ & an embedding of a node $v_i$ in the layer $l$  \\ \hline
		$\vec{e}_{uv}$ & an embedding of an edge $e_{uv}$  \\ \hline
		$\vec{m}_i^l$ & a message aggregated from $\mathcal{N}(v_i)$ in the layer $l$   \\ \hline
		$\mathcal{L}(H)$ & description length for a model $H$  \\ \hline
		$\mathcal{L}(D|H)$ & description length for a dataset $D$ encoded\\
		& by a model $H$  \\ \hline
		$\mathcal{L}(D)$ & minimum description length for a dataset $D$  \\ \hline
		$\LR(x)$ & the number of bits of  $x\in\R$ by some encoding  \\ \hline
		\end{tabular}
		\caption{All related terminology}
\end{table}

\paragraph*{Feature Interactions} Feature interaction refers to the operation between two or more features concerning the output variable. Linear combinations or additive models represent zero-order interactions, in other words, no interaction \cite{fignn}. % cite FI-GNN
For example, in a ground truth model, $y = 5x_1 + x_2x_3 - 7\sin(x_4 - 3x_5x_6)$, we observe interactions between $x_2$ and $x_3$, as well as between $x_4, x_5$ and $x_6$ with respect to $y$.


\paragraph*{Feature Graphs and Graph Neural Networks}
For graphs, we denote a tuple $G = (V, E, A,X)$ of nodes $V$, a set of edges $E$, an adjacency matrix $A$ and an embedding matrix $X$.
%Sometimes we use notations $ V(G), E(G), A(G) $ and $ X(G) $ without defining $G$ explicitly. 
For any node $ v \in V $, we denote by $ \mathcal{N}(v) $ the set of nodes adjacent to $ v $, called the neighborhood of the node $ v $.
If only the structure of a graph is considered, we may use only $G(V,E)$.

Since we represent the interaction between features by graph edges, we model instances with the graphs whose nodes correspond to features of a dataset.
According to Fi-GNN \cite{fignn}, this graph is called a \textbf{feature graph}. (see the example in Figure \ref{fig:experiment})
Mathematically, given a dataset of features $x_1,\dots, x_d$, a feature graph structure is a graph $G(V,E)$ where $V = \{v_1,\dots, v_d\}$.

%Graph structures are increasingly popular, allowing us to represent relationships among objects using graph edges.
% Much work utilizes graphs to model tabular data, assuming that graph edges can represent underlying relationships or interactions. Graph Neural Networks (GNNs), a deep learning framework for graph data structures, are gaining popularity.

There are many tools that can be used to process graphs. Among these GNN models have gained significant attention due to their effectiveness in handling complex, non-Euclidean data.
GNNs are a category of neural models specifically designed to operate on graph data structures, where the order of nodes should not influence the computation of predicted values, a property referred to as ``permutation invariance.'' There are various approaches to implement GNNs, but one of the most prominent paradigms is based on ``message passing.''

In our context,
message passing can be described through the following equations:
\begin{align}
	\vec{m}_{i}^{(k)} &= \bigoplus_{j\in \mathcal{N}(i)} \phi^{(k)} \left(\vec{h}_{i}^{(k-1)},\vec{h}_{j}^{(k-1)}, \vec{e}_{j,i}\right)\label{message}
	\\
	\vec{h}_{i}^{(k)} &= \gamma^{(k)} \left(\vec{h}_{i}^{(k-1)}, \vec{m}_{i}^{(k-1)}\right)\label{updating}
\end{align}
where $\bigoplus$ represent a differentiable, permutation-invariant function, such as summation, averaging, or maximum operation, while $\gamma$ and $\phi$ denote differentiable functions, often implemented as neural network layers.
In Equation \ref{updating}, the model updates the node information using both existing information (from the previous layer) and information shared among neighboring nodes (messages), which is computed as Equation \ref{message}. The latter equation pertains to the aggregation of messages from neighboring nodes.

Message passing layers are not the sole focus in GNN models; pooling layers also play a crucial role, particularly in tasks like graph classification. Pooling involves summarizing local information in a graph, such as node embeddings, to derive a global representation of the entire graph. There are various methods for graph pooling, including sum pooling and mean pooling \cite{SelfAttentionGP}. %Graph Pooling for Graph Neural Networks: Progress, Challenges, and Opportunities (Chuang Liu)

\paragraph*{Minimum Description Length (MDL)}
Our problem can be framed as the selection of models, specifically the representation of graphs.
MDL \cite{MDL, Rissanen1978ModelingBS} is a paradigm of model selection that attempts to find the model that minimizes the number of bits that represent information.
This information may include the complexity of models and the performance score of the model on given data.

In our work, we use MDL to study the model whose complexity is minimized while maintaining the well-performed accuracy of the prediction.
It is the two-stage code description approach \cite{Grnwald2007TheMD}. 
Specifically, it encodes the data $D$ with the description length $\mathcal{L}_\mathcal{H}(D)$ by first encoding a hypothesis $H$ from the set of given hypotheses $\mathcal{H}$ followed by encoding $D$ with the help of $H$.
The objective is to minimize the total description length as follows:
$$
\mathcal{L}_{\mathcal{H}}(D) := \min_{H\in\mathcal{H}}(\mathcal{L}(H)+ \mathcal{L}(D|H)).
$$
In other words, $\mathcal{L}(H)$ includes any descriptions related to the model, such as coefficients in linear regression or the number of edges in the representation of the feature graph (as used in this work).
The latter term, $\mathcal{L}(D|H)$, typically includes the code length of the prediction errors and the code length of the data.

\section{Related Work}\label{sec:related}

\subsection{Feature Interaction Modeling}

Feature interactions in terms of nonadditive terms play a crucial role in model-based predictive modeling.
Many works attempted to develop methods to indicate the interaction.
Beginning with the statistical-based measure called Friedman's H statistic \cite{Hstat} involving the theory of partial dependency decomposition.
However, it is computationally expensive and is dependent on predictive models.

From the perspective of machine learning, many models were proposed to model feature interactions, especially variants of factorization models inspired by FM \cite{FM}.
FM relies on the inner product between the embedding of the feature-value motivated by the factorization of matrices.
Several variants of FM have been proposed to fill the gaps.
FFM \cite{FFM} takes into account the embedding of feature field (i.e. columns in a table of data).
AFM \cite{AFM} considers the weight of the interaction by adding attention coefficients to its model.

The success of deep nets in various domains motivates researchers to use them in modeling interactions.
For example, Factorization Machine supported Neural Network (FNN) \cite{FNN} proposed a deep net architecture that can automatically learn effective patterns from categorical feature interactions in CTR tasks.
In addition, DeepFM \cite{DeepFM} proposed a kind of similar idea but included a special deep net layer that performs the FM task.

The attention mechanisms \cite{Bahdanau} are the method designed to model the importance between different components.
In AFM, the attention mechanism was applied because the authors believe that all interactions should not be treated equally likely with respect to prediction values.
Attention allows the model to weight feature interactions differently based on their importance.
After introducing the attention mechanism to interaction tasks, many works use attention to model the interaction of features based on the factorization machine framework \cite{Sarkar2022DualAH, Cheng2019AdaptiveFN,Wang2020AdnFMAA,Wen2020NeuralAM,Li2021GraphFMGF}.

One of the crucial challenges of interactions in traditional machine learning models is the explainability of which features have interaction, although some of them outperform in prediction performance.
So we need an explicit representation that can correspond to pairwise interaction, where a graph is a representation that can fill this gap.
In addition, GNN is a tool that can be used to learn information from graphs.

\subsection{Graph Neural Networks for Feature Interactions}
GNN are becoming more interesting for use in feature interaction problems.
Since feature interactions can be seen as relationships between features, most of the works rely on feature graphs for each individual instance to represent the interactions via aggregation of representation from neighbor nodes.

Motivated by the click-through rate (CTR) prediction, which is believed to influence the rate of clicking on a product, such as age and gender, many works attempted to model this prediction by using GNN applied feature graphs.
To the best of our knowledge, Fi-GNN \cite{fignn} is the first work to use GNN to model the interaction of features in the feature graph for the prediction of CTR.
It applied the field-aware embedding layer to compute the latent representation of nodes of features before feeding the feature graph with the computed representations into a stack of message passing layers.

After the introduction of Fi-GNN, many subsequent works have built on this idea.
Cross-GCN \cite{Feng2020CrossGCNEG} used a simple graph convolutional network with cross-feature transformation.
GraphFM \cite{GraphFM} seamlessly combined the idea of FM and GNN using the neural matrix factorization \cite{NeuralMF} based function to estimate the edge weight. It also adopts the attentional aggregation strategy to compute feature representation.
Table2Graph \cite{Table2Graph} applied attention mechanisms to compute the probability adjacency matrix instead of edge weight. It also used the reinforcement learning policy to capture key feature interactions by sampling the edges within feature graphs.

All mentioned works chronically improve the methods of feature interaction learning.
However, no work provides information about the effect of the construction of feature graphs on the learning capacity.
Most of the works utilize attention mechanisms or edge weights to let models learn graph structures via these parameters.

\section{Problem Formulation: Pairwise Interaction Feature Graph Problem} \label{sec:problemformulation}

An important open research question in graph representation for data without explicit connections is that {\em What is a suitable graph structure?} While current research highlighted the potential of feature graphs, a feature graph that is too dense, for example, a complete graph, may lead to learning issues, as shown and discussed in previous work \cite{fignn, Kipf2016SemiSupervisedCW, Rong2019DropEdgeTD} and in Section \ref{subsec:complete-graph}.
Thus, finding an optimal feature graph structure for any particular prediction task is the common goal of the community.
Consequently, we can formulate the ``feature-graph-finding'' problem as follows:
\begin{equation*}
	G^* = \argmin_{G\in\mathcal{G}(n)} \ell(\vec{y}, \text{GNN}(X; G)),
\end{equation*}
where $\text{GNN}(\ \cdot\ ;G)$ is a given GNN model trained for the feature graph $G$, and $\ell$ is an arbitrary but fixed loss function for the prediction task. The above task is not traceable as there are exponentially many possible graph structures ($\mathcal{O}(2^{n^2})$.) Many works attempted to solve this problem using soft learnable interactions \cite{fignn,GraphFM} or hard interactions through reinforcement learning \cite{Table2Graph}. However, they did not provide any insight into the characteristics of a suitable feature graph structure related to the interaction between features.

Consequently, instead of solving the ``feature-graph-finding'' problem, this work aims to study the relationship between feature graph structures and their performance in modeling feature interactions.
% Maybe like this? Too strong?
Specifically, we would like to theoretically show that (1) feature graphs correspond to feature interactions and (2) keeping only edges between interacting feature nodes is sufficient under MDL assumptions.
% Specifically, we want to ideally classify the unary predicate $P(G)$ for a feature graph $G\in \mathcal{G}$ such that:
% $$
% \forall G\in \mathcal{G}, P(G) \leftrightarrow \text{good result}.
% $$
% Notably, we develop neither new models nor new frameworks in this work.
% We use existing models to investigate behavior of them.
% Moreover, we provide theoretical perspective result related to this objective in the next section. 

\section{Theoretical Results}\label{sec:theoResult}
There are two questions that we should address.
The first is ``Is the space of the functions and the space of feature graphs a correspondence?''
It is significant because we aim to guarantee that we can use some graph to represent any of the given functions and vice versa.
The second question is that of ``Why do we need sparse graphs rather than complete graphs?''
Here, we consider the second problem to be the problem of model selection according to the concept of MDL.
Proofs of results in this section are in the Appendix.

\subsection{Correspondence Problem}\label{subsubsec:corresQues}

It is worth considering the correspondence between expressions and graph structures.
Even though we primarily focus on pairwise interactions, in this correspondence, we go towards the arbitrary number of variables interacting, not only pairwise.
For convenience of stating, we discuss only the multiplicative interactions, which is a special case of the non-additive interaction.
Throughout this, let us call the kind of expression disjoint multilinear interaction which is defined as follows:

\begin{defi}[Disjoint Multilinear Interaction Expression]
	The predictive model $y(\vec{x})$ is said to be a \textbf{disjoint multilinear interaction expression} (DMIE) if it can be expressed as 
	\begin{equation}
		y(\vec{x}) = \sum_{i=1}^n \prod_{j=1}^{m_i} x_{i_j}
	\end{equation}
	where $x_{i_j}$ and $x_{k_l}$ are distinct variables when $i\neq k$ or $j\neq l$.
	We denote by $\mathcal{DM}(n)$ the set of all DMIE expressions of $n$ variables.
\end{defi}


Multiple graphs can infer the same expression considering the perspective of connected components. For example, the expression $x_0 x_1 x_2 + x_3 x_4 + x_5 + x_6$ corresponds to the graphs $E_1 = \{\{0,1\},\{0,2\},\{1,2\},\{3,4\}\}$ and also $E_2 = \{\{0,1\},\{1,2\},\{3,4\}\}$. Consequently, we use equivalence classes of graphs (based on connectedness) instead of individual graphs. To this end, we define an equivalence class of feature graphs.


\begin{defi}
	We define a binary relation $\sim$ in $\mathcal{G}(n)$ as follows: for any graph $G_1,G_2 \in \mathcal{G}(n)$, $G_1\sim G_2$ if for any $i, j\in N_n$, they are reachable in $G_1$ if and only if they are reachable in $G_2$.
	
	Obviously, $\sim$ is an equivalence relation to $\mathcal{G}(n)$.
	We then denote the set of equivalence classes from $(\mathcal{G}(n),\sim)$ by $\mathfrak{G}(n)$:
	\begin{equation}
		\mathfrak{G}(n) := \left\{ \left[G\right]_\sim : G \in \mathcal{G}(n) \right\}
	\end{equation}
\end{defi}

As a result, we can prove the correspondence, as presented in Theorem \ref{mainThm}.

\begin{thm}\label{mainThm}
	$\mathcal{DM}(n)$ one-to-one corresponds to $\mathfrak{G}(n)$.
\end{thm}


\subsection{MDL for graph representation of pairwise interaction}\label{subsec:MDL}

This problem can be considered as a model selection problem for selection of representations.
%Supported by the insight mentioned in Section \ref{sec:insight},
We adopt the idea of Minimum Description Length (MDL) to shape the problem in the manner of selection of graph representation.
Generally speaking, a selected feature graph should not be so dense that it captures an uninformative characteristic.
On the other hand, it should not be such a light graph that the informative edges to learn interactions are omitted.
In this work, we focus on the pairwise interaction case, which is formalized below.

Given a graph $G=(V,E)$ whose nodes are of degree at most 1, 
a function containing pairwise interaction induced from $G$ is defined as
\begin{equation}\label{eq:pairwisefunction}
	f_G(x) = \sum_{\deg(i)=0} c_i x_i + \sum_{\{i,j\}\in E} c_{ij}x_i x_j
\end{equation}
Note that we represent pairwise interaction by bilinear interaction for convenience in describing by mathematical notation.
Therefore, the problem of selecting the graph for this function $f_G$ is framed as Problem \ref{problem:MDLInteractionCapture}.

\begin{algorithm}
	\floatname{algorithm}{Problem}
	\caption{MDL Pairwise Interaction Graph Problem******}\label{problem:MDLInteractionCapture}
	\textbf{Input:} A dataset $\mathcal{S} = \{(x_i,y_i)\}_{i=1}^n$, and a pre-configured GNN model $\mathcal{F}:\mathfrak{G}\times\R^d  \to \R$\\
	\textbf{Output:} The edge set $E^*$ for an input feature graph $G^*(V,E^*) \in \mathfrak{G}$ so that
	\begin{equation}
		G^* = \argmin_{G\in\mathfrak{G}}{\mathcal{L}(f_G) + \mathcal{L}(S|f_G)}
	\end{equation}
\end{algorithm}
\noindent In this problem, we denote $\mathcal{L}(S,f_G) = \mathcal{L}(S|f_G) + \mathcal{L}(f_G) $, where
\begin{align}
	\mathcal{L}(f_G) &= \LR (|E(G)|) + \sum_{\deg(i)=0}\LR(c_i) + \sum_{\{i,j\}\in E(G)}\LR(c_{ij})\\
	\mathcal{L}(S|f_G) &= \sumdata\LR(x_i) + \sumdata \LR(y_i - f_G(x_i))
\end{align}
\noindent We denote by $\LR:\R\to\N$ a function of the number of bits $\LR(x)$ we need to encode a real number $x$.
In the next part, we will show that the description length of the optimal graph $G^*$ is bounded by the lengths of both the complete graph $K$ and the null graph $N$.

\paragraph*{Assumptions}
	Our consequent result in this paper is based on these assumptions about $\LR$:
	\begin{enumerate}
		\item $\forall x_1,x_2\in\R,\LR (x_1+x_2) \leq \LR(x_1) + \LR(x_2)$
		\item $ \forall x_1,x_2\in\R, |x_1|\leq |x_2|\iff\LR(x_1)\leq\LR(x_2) $
		%	\item $\LR(0) = 0$ and $\forall x\in \R, |x|>0 \Rightarrow \LR(x)\geq 1$
		\item $
		\exists D >0\forall x,y\in \R,|x-y| \leq 1 \Rightarrow |\LR(x) - \LR(y)| \leq D
		$
	\end{enumerate}
	

		We assume that the functions that generate $Y$ from $X$ in a dataset, denoted by $S_{G^*} := \{(X_i,Y_i)\}_{i=1}^n$, are induced by some feature graph $G^*$ as Equation \ref{eq:pairwisefunction}, i.e. $$Y = f_{G^*}(X) + \mathcal{E}$$ where $\mathcal{E}$ is a noise random variable from some truncated distribution, i.e. there exists $\epsilon^*>0$ with $|\mathcal{E}|\leq\epsilon^*$.
		
		We also denote $\hat{f}_{G}$ a function $f_G$ corresponding to a feature graph $G$ whose parameters are approximated by $S_{G^*}$.
		Moreover, we assume that the parameters of actual additive terms can be learned well (by some learning algorithm).
		Specifically, if the term $c_ix_i$ exists in the ground truth function, then the parameters $d_i$ of the feature graph can be learned so that $\LR(d_i - c_i)\approx 0$. It is similar for the interaction term $c_{ij}x_ix_j$, the parameter $d_{ij}$ can be learned so that $\LR(d_{ij} - c_{ij})\approx 0$.
		
		We also assume further that if the features $a$ and $b$ interact with each other in the expression of ground truth, then $\LR(\hat{d}_a) + \LR(\hat{d}_b) >> \LR(\hat{c}_{ab})$ where $\hat{d}_a$ and $\hat{d}_b$ are the coefficients learned by the expression that these features are additive and $\hat{c}_{ab}$ are those learned by the expression consisting of the interaction term between $x_a$ and $x_b$.
		So, it is safe to assume that $\LR(\hat{d}_a) + \LR(\hat{d}_b) - \LR(\hat{c}_{ab}) \geq D$ which is a uniform bound of difference between two real numbers that is not further than 1.
		On the other hand, we also assume that if $a$ and $b$ do not interact with each other, then $\LR(\hat{c}_{ab}) >> \LR(\hat{d}_a) + \LR(\hat{d}_b)$.
		
		For the description length of error term $\sumdata \LR(y_i - f_G(x_i))$, we need the assumption that fitting a pairwise interaction term $c_{ab}x_ax_b$ by the additive terms $c_ax_a + c_bx_b$ always yields dramatic length with respect to the noise term $\epsilon_i$ so that
		$$
		\LR(c_{ab}x_ax_b -(\hat{c}_ax_a + \hat{c}_bx_b) + \epsilon_i) >> \LR(\epsilon_i).
		$$
		Similarly, we also need a similar assumption to fit the additive terms $c_ax_a + c_bx_b$ by the interaction term $c_{ab}x_ax_b$ as follows:
		$$
		\LR((c_ax_a + c_bx_b) - \hat{d}_{ab}x_ax_b + \epsilon_i) >> \LR(\epsilon_i).
		$$



First, we begin with the most intuitive result motivated by our empirical result that we must keep the edges of interactions and we should prune non-interaction edges out in the desired feature graph.


\begin{lem}\label{lemma:error_term}
	If $\{a,b \}\in E(G^*)$, then we have $$\sumdata \LR(y_i - f_{G^*\setminus\{a,b\}}(x_i)) \geq \sumdata \LR(y_i - f_{G^*}(x_i)).$$
	If $\{a,b \}\notin E(G^*)$, then $$\sumdata \LR(y_i - f_{G^*\cup\{a,b\}}(x_i)) \geq \sumdata \LR(y_i - f_{G^*}(x_i)).$$
\end{lem}


\begin{prop} \label{prop:add-itr-drop-nonitr-from-good}
	If $\{a,b \}\in E(G^*)$, then we have $ \mathcal{L}(S_{G^*},f_{G^*\setminus\{a,b\}})\geq \mathcal{L}(S_{G^*},f_{G^*})$. On the other hand, if $\{a,b \}\notin E(G^*)$, then $ \mathcal{L}(S,f_{G^*\cup\{a,b\}})\geq \mathcal{L}(S_{G^*},f_{G^*})$.
\end{prop}


By Lemma \ref{lemma:error_term}, it is easy to show that:
\begin{cor}\label{cor:MDLerrorComplete}
	Let $K$ be the complete feature graph of the features of $S_{G^*}$.	Then
	$$
	\sumdata \LR(y_i - \hat{f}_K(x_i)) \geq \sumdata \LR(y_i - \hat{f}_{G^*}(x_i))
	$$ 
	Consequently, we obtain the desired result.
	
	$$\mathcal{L}(S_{G^*},f_{G^*}) \leq \mathcal{L}(S_{G^*},f_K).$$
\end{cor}

Not only for the complete graph, but also for the graph without edges we aim a similar result.

\begin{cor}\label{lem:MDLerrorNull}
	Let $N$ be the feature graph without edges.
	Then
	$$
	\sumdata \LR(y_i - \hat{f}_N(x_i)) \geq \sumdata \LR(y_i - \hat{f}_{G^*}(x_i))
	$$
	Hence, we also have $\mathcal{L}(S_{G^*},f_{G^*}) \leq \mathcal{L}(S_{G^*},f_N)$.
\end{cor}

\begin{thm} \label{theorem:complete-null-bad}
	For any feature graph $G$ containing all interaction edges, we have
	$$
	\mathcal{L}(S_{G^*},f_{G}) \leq \mathcal{L}(S_{G^*},f_K)
	$$
	$$
	\mathcal{L}(S_{G^*},f_{G}) \leq  \mathcal{L}(S_{G^*},f_N)
	$$
\end{thm}


\paragraph*{Implication}

%Let us assume to have all of the above results for this talk.
The theoretical results shown in Section \ref{subsubsec:corresQues} are just to ensure us that we can confide in learning feature graph structures to infer an expression containing feature interactions.
In other words, given any function containing feature interactions of at most one occurrence for each feature, there always exist feature graph structures that can infer backward to the function and vice versa.
Therefore, to find such graph structures, we need some insight into graph construction related to feature interactions.

Consequently, we reformulate these observations into more general arguments.
Generally speaking from empirical results in Section \ref{subsec:itr-vs-nonitr}, it is better to maintain all the edges of the interaction in the feature graph to maximize the learning interaction in a given dataset.
Also, it is better to prune non-interaction edges out as much as possible to avoid fitting noise information.
Finally, these results are supported by Proposition \ref{prop:add-itr-drop-nonitr-from-good}.

In addition, we also show the empirical result about using complete feature graphs to learn from larger datasets in Subsection \ref{subsec:complete-graph}.
Although the result in Subsection \ref{subsec:itr-vs-nonitr} cannot infer the limitation of complete graph obviously,
the result in Subsection \ref{subsec:complete-graph} reveals that the complete feature graphs are worse when datasets are larger in feature size.
Finally, we can show that, in the manner of MDL which is a framework of model selection, the complete feature graph is worse than any feature graphs containing all interaction edges.
This tells us that it is not a good idea to use the complete feature graph to learn from datasets.
Not only learning hidden noise, but it also face with the computation and memory issues.
In the next section, we will focus on the empirical results in more details.

\section{Methodology}\label{sec:methodology}

\begin{figure*}[t!]
	\centering
	\includegraphics[width=\linewidth]{experiment1}
	\caption{Experimental procedure: data generation based on a ground truth function and variation of feature graph structures as an input.}
	\label{fig:experiment}
\end{figure*}

Our experiment provides empirical evidence regarding the relationship between feature graph structures and pairwise interactions in datasets. We train and evaluate several feature graph structures for each dataset and report the results.
However, due to the combinatorial number of possible feature graph structures and the stochastic nature of the learning algorithm, we adopt a strategy of generating random edge sets to explore various graph configurations, as illustrated in Figure \ref{fig:experiment}. 

\subsection{Datasets}
Real-world datasets rarely provide information about interactions between features, which disables our analysis.
To this end, we use synthesized datasets to control interaction characteristics in datasets such as the number of interaction pairs of features, strength of interactions, or kinds of interactions. We assume that there is only one mode of interaction per feature graph.
If real-world datasets had many interaction modes, we could apply weighted ensemble feature graphs proposed Li et al. \cite{Li2021GraphFMGF}

Construction of the simulated dataset (the first arrow in Figure \ref{fig:experiment}) can be formally described as follows.
Let $f:\R^d\to\R$ be a hidden ground truth function containing pairwise interaction terms. A synthetic dataset of $n$ instances with respect to $f$ if $\{(\vec{X_i},f(\vec{X_i}) + 0.1\mathcal{E}_i)\}_{i=1}^n$ where $\vec{X_i}\sim\mathcal{N}(0,1)$ and $\mathcal{E}_i\sim\mathcal{N}(0,\text{var}(f(\vec{X_i})))$. We varied the number of features and interaction pairs to aid our analyses.
Note that for each function $f$, we generated one set of data and used it for all variations of feature graphs.

\subsection{Feature graphs} 
For a given function $y=f(x_1,x_2,\dots, x_d)$ of $d$ features, a set of feature graphs is constructed by randomly assigning edges between feature nodes (see Figure \ref{fig:experiment} after the second arrow).
In other words, given a dataset of $d$ features and $n$ samples $\{ (\vec{x}_i, y_i) \}_{i=1}^n$ and an edge set $E$ for a graph of $d$ nodes, the feature graph dataset of edge set $E$ is a graph dataset
$
\left\{(G(V,E,\text{concat}(\vec{x},X)),y_i)\right\}_{i=1}^n,
$
where $X$ is a learnable embedding for a feature graph for nodes.
The initial node embedding matrix is denoted by
\begin{equation}
    H^{(0)}=\text{concat}(\vec{x},X). \label{eq:initEmb}
\end{equation}

\subsection{GNN Architectures}
Our GNN model utilizes transformer convolution layers (TransformerConv \cite{TransformerConv}) for the message passing function, incorporating attention mechanisms to capture the strength of pairwise interactions, as depicted in Figure \ref{fig:overview}. Then, a mean pooling layer averages the embedding of all nodes for the prediction layer.

\begin{figure*}[h]
	\centering
	\includegraphics[width=0.65\linewidth]{overview}
	\caption{(will be fixed for used notation) Detail of the GNN model: from feature graph construction to output prediction.}
	\label{fig:overview}
\end{figure*}
%\begin{notebox}
%	\faCheck tell more why BatchNorm instead of LayerNorm
%\end{notebox}

As Fi-GNN does, this work adopts the idea from Fi-GNN, but simpler architecture.
We use TransformerConv \cite{TransformerConv} which is a message-passing layer based on the attention mechanism whose feature interaction learning capability is shown by many works.
For the computation, we start from the initial node embedding $H^{(0)}$ defined in Equation \ref{eq:initEmb}.
Then a feature graph with the node embedding matrix is fed to TransformerConv with BatchNorm:
\begin{equation} \label{eq:transformerConv}
	\begin{split}
    	Z^{(l)} &= \text{TransformerConv}(H^{(l-1)},E),\\
    	H^{(l)} &= \text{ReLU}(\text{BatchNorm}(Z^{(l)})),
	\end{split}
\end{equation}
where $H^{(l)}$ is an intermediate node embedding matrix at the layer $l$.
After applying all message passing layers, we calculate the average pooling of node embeddings from the last GNN layer, and we use this vector as a graph embedding which is fed into the projection layer to get prediction:
\begin{align} 
    	\vec{p} &= \frac{1}{n}\sum_{i}H^{(L)}[i],\label{eq:pooling}\\
    	\hat{y}  &= \vec{w}\cdot \vec{p} + b,\label{eq:finalProjection}
\end{align}
where $H^{(L)}$ is a node embedding matrix from the last GNN layer and $H^{(L)}[i]$ is the row $i$ of the matrix which corresponds to the node embedding of the node $i$ and $b$ is a biased term.
Equation \ref{eq:pooling} is the calculation of mean pooling layer and Equation \ref{eq:finalProjection} is the final projection to calculate the prediction value.

Equation \ref{eq:transformerConv} is a bit different from the original work \cite{TransformerConv} which uses LayerNorm after aggregation of message like as the original transformer in NLP.
In the first phase of experiment, we found that the GNN model with LayerNorm cannot capture interaction behavior while BatchNorm can.

In training the GNN model, we adopt an usual supervised learning approach, where the loss function measures the difference between the predicted value, $\hat{y}$, and the true target value, $y$.
Specifically, we use the mean squared error (MSE) loss, which is defined as
$$
\mathcal{L}_{\text{MSE}} = \frac{1}{N} \sum_{i=1}^{N} (\hat{y}_i - y_i)^2,
$$
where $N$ is the number of samples.


\section{Experiments and Results}\label{sec:insight-results}

Results shown in this section are from the dataset generated by the equation $f(\vec{x}) = \sum_{i=1}^{p} x_{2i-1}x_{2i} + \sum_{i=1}^{q}x_{2p+i} + \epsilon$.
We present the experiment results in the following insights:
\begin{enumerate}
	\item The efficient structures of feature graphs (Section \ref{subsec:itr-vs-nonitr})
	\item The dependency between edges and interaction pairs (Section \ref{subsec:remove-itr})
%	\item \nameref{subsec:add-nonitr}
	\item The relationship between multi-hops and number of message passing layers (Section \ref{subsec:reach})
	\item Scalability of efficient feature graphs (Section \ref{subsec:complete-graph})
	% \item Transformer attention learns to capture efficient feature graph structures in small numbers of nodes (Section \ref{subsec:attentionWeight})
\end{enumerate}
Since Section \ref{subsec:itr-vs-nonitr} - \ref{subsec:reach} consider the variants of graph structures w.r.t. the given dataset, we show only results from an equation containing 2 interaction terms and 2 unary terms, i.e. $p =2, q=2$.
In Section \ref{subsec:complete-graph}, $p$ is varied while $q=2$.
%Section \ref{subsec:attentionWeight} is a result from an equation with $p=7$ and $q=10$.
\subsection{Experimental Setting}
\paragraph*{Synthetic dataset generating}

Each synthetic dataset contains 10,000 samples generated by distribution described in Methodology section (Section \ref{sec:methodology}).
In every randomization, we use seed number $n$ for feature $x_n$ and seed number 0 for noise term.
We split it into first 7,000 rows for training and next 3,000 rows for testing sets.

\paragraph*{Random sampling of feature graph structures}
Ideally, we want results of all possible feature graph structures.
However, there are up to $\mathcal{O}(2^{n^2})$ feature graphs for a dataset of $n$ features.
This leads us to the combinatorial issue,
so we set up an experiment by sampling in random stratified by (1) the number of edges, (2) the number of interaction edges, (3) the number of non-interaction edges, and (4) the number of hops between interacting nodes.
Since we also want to study about subgraphs with respect to existence of interaction edges (for example, edges $\{0,1\}$ and $\{2,3\}$ for the dataset generated by $y=x_0x_1 + x_2x_3 + x_4 + x_5$), we want these collections (1) graphs having both edges, (2) graphs having $\{0,1\}$ but not $\{2,3\}$ (3) vice versa and (4) graphs having no these edges. To have full collections to compare, once any feature graph is sampled, all other three graphs are also sampled.

\paragraph*{Model setting and training}
We also vary the number of message passing layers from 1 layer to 3 layers where all models' hyperparameters are set to have the equally likely number of parameters. In all setting, we use the 16-dimensional node embedding $X$. The hidden size in TransformerConv $H^{(i)}$ for 1, 2 and 3 MP layers is 26, 20, and 16, respectively.
In training of models, we use Adam optimizer with automatically schedule of learning rate.
% The loss function for training is mean square error (MSE) loss.
% For performance comparison, we measure their results by mean absolute error (MAE).
% The best result of model on testing set is recorded.
%
%\paragraph*{Hardware specification}
%..............

\subsection{Interaction Edges vs. Non-interaction Edges} \label{subsec:itr-vs-nonitr}
We are first interested in the existence of interaction edges and that of non-interaction edges.
Intuitively, the existence of interaction edges should be significant for learning the interaction between features by attention-based GNNs since they allow paasing of the message between their associated nodes to model interactions.



\begin{figure}[h]
	\centering
%	\begin{notebox}
%		Between option 1 and 2, which one should I use in report?\\
%		But the paragraph describe result based in the option 1.
%	\end{notebox}
	\includegraphics[width=\linewidth]{result1_1}
	\includegraphics[width=\linewidth]{result1_2}
	\includegraphics[width=\linewidth]{result1_3}
	\caption{the average of MAE (y-axis) in prediction against the number of non-interaction edges (the edges corresponding to the nodes of features not interacting in equation, e.g. the edge $\{0,4\}$ or the edge $\{4,5\}$) in x-axis where different lines in the same figure indicated by color are for the number of interaction edges (there are two possible interaction edges in this equation: $\{0,1\}$ and $\{2,3\}$)}
	\label{fig:result1-1}
\end{figure}

%\begin{figure}[h]
%	\centering(option 2)
%	\includegraphics[width=\linewidth]{result1_1_2}
%	\includegraphics[width=\linewidth]{result1_2_2}
%	\includegraphics[width=\linewidth]{result1_3_2}
%	\caption{the average of MAE (y-axis) in prediction against the proportion of the number of interaction edges (the edges corresponding to the nodes of features not interacting in equation, e.g. the edge $\{0,4\}$ or the edge $\{4,5\}$) to the number of edges in x-axis where different lines in the same figure indicated by color are for the number of interaction edges (there are two possible interaction edges in this equation: $\{0,1\}$ and $\{2,3\}$)}
%	\label{fig:result1-2}
%\end{figure}

Figure \ref{fig:result1-1} illustrates the average MAE in predicting each number of edges without interaction along the $x$ axis.
Each line corresponds to the number of interaction edges.
We see that among the graphs that have the same number of non-interaction, feature graphs containing all interaction edges give significantly lower errors than the graphs missing some interaction edges give for all of the number of non-interaction edges.
Especially, we also see that even though graphs with all interaction edges contain any number of other noise edges, they still give good performance compared to the entire line of the graphs missing some interaction edges. 
From this result, we may infer that the interaction edges are the most important in input feature graphs to learn a dataset containing pairwise interaction.
Although we do not know how to construct feature graphs used by GNN models for the task that we believe that there must be interaction terms, at least the interaction edges (if we know) should be kept.

This result also tells us that when the feature graph can maintain all interaction edges, the increasing number of non-interaction edges leads to poorer performance.
It seems that the non-interaction edges make a model capture non-underlying information in a dataset and interfere with the learning of pairwise interactions of interaction edges.

However, if we consider the results of the 0-interaction-edge and 1-interaction-edge graphs, we notice a different trend from the 2-interaction-edge graphs.
It starts with the worst result when there are no edges in a graph.
In this case, the more noninteraction edges it has, the better the performance.
One hypothesis of us at first sight is that the more number of edges (even though they are not interaction edges), the higher chance that interaction nodes can pass messages in GNNs' computation via multi-hop connectivity rather than direct interaction edges.
This leads us to consider the connectivity of interacting nodes rather than the direct edges between them.

\subsection{Removing Interaction Edges from a Graph} \label{subsec:remove-itr}
The above figure shows the result in average.
However, the information about subgraphs is not shown in them, while it is another interesting property that can motivate us to construct an appropriate feature graph.
For the subgraph property, we consider the effect of removing the interaction edge from a graph.
For example, if we consider a graph $G$ containing all interaction edges, i.e. $\{\{0,1\},\{2,3\}\}\subseteq E(G)$ w.r.t. $x_0x_1 + x_2x_3 + x_4 + x_5$, the other considered subgraphs compared to this graph are $E(G)\setminus \{\{0,1\},\{2,3\}\}$, $E(G)\setminus \{\{0,1\}\}$ and $E(G)\setminus \{\{2,3\}\}$.
Figure \ref{fig:subgraphLattice} shows an abstract lattice of the subgraphs.
Our claim is that the performance would drop along this lattice from top to bottom of lattice.
\begin{figure}[h!]
	\centering
	\includegraphics[width=0.7\linewidth]{subgraphLattice}
	\caption{a lattice of subgraph of a given graph $G$ containing all interaction edges for $x_0x_1+x_2x_3+x_4+x_5$}
	\label{fig:subgraphLattice}
\end{figure}

To this end, we perform a statistical analysis by the confidence interval of the paired difference of MAE where the pair of graphs is paired with the subgraphs by removing the interaction edges: $\text{MAE}(G\setminus\{a,b\}) > \text{MAE}(G)$
%\begin{align*}
%	H_0:\hspace{5pt}&\text{MAE}(G\setminus\{a,b\}) = \text{MAE}(G)\\
%	H_a:\hspace{5pt} &\text{MAE}(G\setminus\{a,b\}) > \text{MAE}(G)
%\end{align*}
where an edge $\{a,b\}$ is an interaction edge in a graph $G$.

The lower-sided confidence interval 95\% is shown in Table \ref{tab:pval-remove-itrEdge}.
It ensures us that interaction edges are significantly important for input feature graph in attention-based GNNs.
Figure \ref{fig:RemovingItrEdge} shows the MAE of some random sample generated, sorted by the MAE of $E(G)\setminus \{\{0,1\},\{2,3\}\}$ from the 3 message passing layer model to illustrate how they are.

\begin{table}[h!]
	\centering
	\begin{tabular}{c c c c c }
		\hline
		graph   & dropped & 1MP & 2 MPs & 3 MPs \\ \hline
		$G$       & $\{0,1\}$  &  $>0.124$  &  $>0.173$  &  $>0.160$ \\
		$G$       & $\{2,3\}$  &  $>0.129$  &  $>0.186$  &  $>0.163$ \\ 
		$G\setminus\{0,1\}$ & $\{2,3\}$   &  $>0.114$  &  $>0.132$   &    $>0.137$ \\ 
		$G\setminus\{2,3\}$ & $\{0,1\}$  &  $>0.114$  &  $>0.156$   &  $>0.166$ \\ \hline
	\end{tabular}
	\caption{95\% lower-sided confidence interval of paired-difference of MAE between a graph and interaction-edge-removed graph}\label{tab:pval-remove-itrEdge}
\end{table}

\begin{figure}[h!]
	\centering
	\includegraphics[width=\linewidth]{result-subgraph}
	\caption{Samples of MAE score colored by the number of interaction edges and sorted by in x-axis by MAE of no-interaction-edge graphs to show the increasing of MAE when an interaction edge is removed, where our $G$ in this subsection corresponds to the green points}
	\label{fig:RemovingItrEdge}
\end{figure}


\subsection{Reachability of Interaction Nodes and the Number of Message Passing Layers through a Longer Path}\label{subsec:reach}
\begin{figure*}[t!]
	\centering
	\includegraphics[width=\linewidth]{result2}
	\caption{Heatmap of average MAE for the number of hops between interaction nodes (between node 0 and node 1 in vertical and between node 2 and node 3 in horizontal) for a model consists of (a) 1 MP layer, (b) 2 MP layers and (c) 3 MP layers: 99 means not reachable}
	\label{fig:result2}
\end{figure*}
In the previous result, we discussed the effect of the occurrence and absence of interaction and noninteraction edges.
We observed that the feature graphs without interaction edges can predict the test set better with the help of increasing non-interaction edges.
So, it comes up with the question about connectivity whether GNN models can capture interaction patterns via paths, not necessarily to be adjacent, and whether the length of a shortest path relates or not.

In Figure \ref{fig:result2} (a),
we see that although the interacting nodes are not adjacent, these graphs can still be used at some level if they are reachable.
In other words, connectivity through a sufficient number of hops can also yield satisfactory results.

We also see that the longer the shortest path between interacting nodes, the worse the score.
This result shows a considerable increase in error when the number of hops increases from 1 hop to 2 hops in both directions.
This might be because we used only one message passing layer in the model.
This allows the passing of messages between nodes only for one hop.

Since the passing message allows nodes to exchange their messages via edges, the greater number of layers should allow further nodes to do so.
So we show, in Figure \ref{fig:result2} (b)-(c), an alternative result experiment in which a GNN model consists of 2 and 3 message passing layers whose hyperparameters are set to have the same number of parameters as that of the 1-layer model.
The result is analyzed similarly to that shown in Figure \ref{fig:result2} (a).

In Figure \ref{fig:result2} (b) compared to (a), we see that the 2-layer model can be trained using a feature graph to ensure that the interaction nodes are reachable to each other up to 2 hops.
Similarly, Figure \ref{fig:result2} (c) of a 3-layer GNN reveals that the feature graph of up to 3 hops of interaction nodes can be used to train the GNN model.
This may infer that the more messages that pass through, the weaker the condition of reachability of the interaction node.

\subsection{Limitation of Complete Feature Graphs}\label{subsec:complete-graph}

From the result in Figure \ref{fig:result1-1}, the complete feature graph can give errors almost similar to that of the graph that contains only interacting edges.
This may sound like a good result to tell us that it is not necessary to find the optimal graph structures.
However, the demonstrating dataset is very small in number of features.
It contains only 6 features; then the complete graph consists of only 15 edges, which is small.
Therefore, we ask the question of how it will be when datasets contain more and more features.

\begin{figure}[h!]
	\centering
	a result from simulated datasets of equation $y=x_1x_2+\cdots + x_{2p-1}x_{2p} + x_{2p+1} + x_{2p+2}$
	\includegraphics[width=\linewidth]{result3}
	\caption{Predictive performance of lightGBM, GNN with complete graphs and GNN with ground truth feature graphs measured by MAE (y-axis) when the number of pairwise terms is varied (x-axis)}
	\label{fig:result3}
\end{figure}

Figure \ref{fig:result3} presents the results tested in synthetic datasets with 2 linear terms, that is, $y=x_1x_2+\cdots + x_{2p-1}x_{2p} + x_{2p+1} + x_{2p+2}$.
The results for complete graphs and the ground truth graphs are represented by the red and blue lines, respectively.
In addition, we depict the theoretical standard deviation of the datasets with a black dot line. The thick black line indicates the noise term $\epsilon$ added to the training label, which is $y + 0.1\sigma$.

It is evident that greater noise, resulting from an increase in the number of features, leads to poorer prediction performance.
When our GNN model is trained on complete feature graphs, it outperforms lightGBM when there are only a few pairwise terms (up to 10 in both datasets). Beyond this point, it dramatically increases, and the GNN model performs less effectively than lightGBM.

Notably, the red line plot in the right figure terminates at 40 pairwise terms (95 features or 4465 edges). The dataset in the right figure experiences memory leakage during the construction of complete feature graphs and computational processes of the model. This is why it is essential to avoid training the entire dataset using complete feature graphs when dealing with a larger number of features.

On the other hand, considering the performance of the GNN model trained on the ground truth feature graphs, the results show the lowest error compared to complete graphs and lightGBM. This underscores the value of pruning the edges of feature graphs when training models for prediction. Complete graphs not only lead to memory leaks but also capture uninformative hidden data. In conclusion, pairwise feature interaction is a beneficial property that should be retained in feature graphs to enhance the predictive performance of tabular data using GNNs.

% \subsection{Attention Weight of Message Passing Layer w.r.t. Pairwise Interaction}\label{subsec:attentionWeight}

% The result we have already shown indicates the necessity of interaction edge or at least connectivity of them.
% It relies on the assumption of us that it is because of attention coefficients in Transformer Convolution (Equation \ref{equation:transformerConv} and \ref{attention}).

% Last but not least, we assess the attention coefficients from the Transformer Convolution message passing layer of 1-layer GNN model train and test by the complete feature graph to see how interaction pairs and attention coefficients are related.
% We define a score matrix $\mathcal{M}$ to represent attention coefficients of all instance as follows:
% \begin{align}
% 	(M_i)_{u,v} &= \alpha^{(i)}_{u,v} + \alpha^{(i)}_{v,u},\\
% 	\mathcal{M} &=  \text{MinMaxNormalizer}\left(\sum_{\text{instance } i} M_i \right),\label{equation:attention_score_matrix}
% \end{align}
% where $\alpha^{(i)}_{u,v}$ is the attention coefficient from node $u$ to node $v$ of instance $i$.
% In this context, for a matrix $M$, we denote $(M)_{u,v}$ indicates the $ij$-entry of the matrix $M$.

% \begin{figure}[h!]
% 	\centering
% 	\includegraphics[width=0.8\linewidth]{result5}
% 	\caption{}
% 	\label{fig:result_attention_weight}
% \end{figure}

% Figure \ref{fig:result_attention_weight} illustrates the heatmap matrix of the attention score matrix $\mathcal{M}$ defined in Equation \ref{equation:attention_score_matrix} from the synthetic dataset $y=x_1x_2+\cdots+x_{13}x_{14} + x_{15}+\cdots + x_{24}$.
% This matrix clearly reveals that the score effectively captures the pairwise interaction pairs.
% This result confirms our belief about attention mechanisms to conclude that attention can learn interaction of features from each feature like it learn importance between words in NLP tasks.

\subsection{Discussion}
All results we show in this section mainly support the argument about the necessity of interaction edges in the case of a bilinear (pairwise interaction by multiplication) interaction.
Moreover, it also reveals that the non-interaction edges tend to behave like noise edges to capture non-underlying information in the given dataset when feature graphs contain all interaction edges.
So, if possible, we need to prune out such uninformative edges that avoid capturing noise in the dataset.
However, in some cases, we may not know explicit interaction pairs to be able to keep the correct edges.
Our result shows that the adjacency between interacting nodes can be weakened to only reachability via non-interaction edges in the limited number of hops by increasing the number of layers of message passing in the GNN models.

However, the results shown in this section are from only one simulated dataset.
Even though the results seem obviously to infer the conclusions, we need some theoretical results to guarantee the consistency of the results in any dataset that we are focusing on.
The results from Sections \ref{subsec:itr-vs-nonitr} and \ref{subsec:remove-itr} motivate us to reformulate the problem into a theoretical perspective which is discussed in the next section.

\section{Case Study: Click-through-rate Prediction}\label{sec:caseStudy}

We observed that it is ideal to include all known interaction edges in feature graphs. Non-interaction edges can be considered noise, capturing irrelevant information. When all interaction edges cannot be preserved, adjacency constraints can be relaxed to focus on reachability or connectedness, aided by more message passing layers.
In this section, we move from simulated data to the real world problems.
One of the real world problems that interactions between features are often concerned is the click-through rate (CTR) problem as a recommendation system in various application.
 
\begin{figure}[h!]
	\centering
	\includegraphics[width=1\linewidth]{AUC_Score.png}
	\caption{Comparison of AUC scores for different graph structures on the Criteo dataset. Randomly constructed sparse graphs (blue) exhibit varying performance based on connection percentage, with an optimal score of 0.8072 at 24\% connectivity. The manually constructed sparse graph (orange) outperforms both the fully connected graph (yellow) and FiGNN (red) with an AUC of 0.8076 at just 13\% connectivity.}
	\label{fig:resultctr}
\end{figure}

In this experiment, we utilized the Criteo dataset \cite{criteo2014} to evaluate the performance of different graph structures using the AUC score as the primary metric.

FiGNN, a widely recognized graph model for CTR prediction, achieves an AUC score of 0.8062. Initially, we generated random sparse graphs with varying percentages of connections and evaluated their performance using a simple GNN model. The optimal AUC score for randomly constructed graphs was approximately 0.8072, achieved with 24\% graph connections.
Subsequently, we experimented with a fully connected graph, utilizing 100\% of the graph connections. The AUC score for the fully connected graph was 0.8073.
Additionally, we manually constructed a sparse graph inspired by the GDCN heatmap \cite{GDCN}. This manually constructed graph achieved an AUC score of 0.8076 with only 13 \% of the connections.

The findings of this experiment demonstrate that a simple GNN model with a randomly constructed sparse graph can outperform the FiGNN model and achieve results comparable to a fully connected graph. Moreover, the manually constructed sparse graph, designed with deliberate and meaningful connections, outperforms both the fully connected graph and FiGNN in performance. This supports the results found in our simulation.

\section{Conclusion and Future Work} \label{sec:discussion}

In this study, we investigated how the structure of feature graphs relates to their capacity to represent pairwise feature interactions within GNNs. We offered both theoretical and empirical insights on the development of sparse feature graphs. Our findings reveal that interaction edges are crucial, whereas non-interaction edges can introduce noise, which hinders the learning process of GNN models. From a theoretical standpoint, employing the Minimum Description Length (MDL) principle, we showed that feature graphs that retain only essential interaction edges provide a more efficient and interpretable representation compared to complete graphs. Empirical tests on synthetic datasets underscored the significance of interaction edges for improved prediction performance, while non-interaction edges were found to introduce noise that reduces model accuracy. Additionally, we discovered that the connection between interacting nodes, even without direct interaction edges, can enhance learning via multi-hop message passing in GNNs.

Our findings emphasize the significance of feature graph sparsity for both computational efficiency and prediction accuracy, particularly in large-scale datasets where complete graphs become useless.
Furthermore, we validated the idea of our insights through a case study on click-through rate prediction, further highlighting the practical value of constructing well-designed feature graphs for real-world tasks.

For future work, our aim is to extend this study by:
\begin{enumerate}
	\item Investigating algorithm for constructing optimal feature graphs from raw datasets, especially when prior knowledge about feature interactions is unknown.
	\item Exploring feature interactions beyond pairwise relationships, such as higher-order interactions, and their implications for feature graph design.
	\item Integrating dynamic graph construction methods to adapt feature graphs during model training for each instance, allowing better handling of evolving datasets and relationships.
	\item Design new message-passing mechanisms specifically tailored for interaction modeling, enabling GNNs to better capture interaction patterns and relationships in feature graphs.
\end{enumerate}

By addressing these directions, we hope to further advance the understanding and utility of feature graphs in GNN-based models, contributing to improved model performance and interpretability across a broad range of applications.

%%#################################################################################
%%#################################################################################
%%#################################################################################

%% Use IEEE style for the bibliography
\bibliographystyle{IEEEtran}
%\bibliography{references}  % This assumes references.bib is in the same directory
\documentclass{article}
\usepackage{pgfplots}
\pgfplotsset{compat=1.18}
\usepackage{arxiv}
\usepackage{graphicx}   % For including images
\usepackage{hyperref}   % For PDF metadata and hyperlinks
\usepackage{ifpdf}      % To handle both LaTeX and PDFLaTeX
\usepackage{caption}    % Better figure captions
% \usepackage[a4paper,margin=1in]{geometry}  % Adjust page layout
\usepackage{float}      % To control figure placement
\usepackage{amsmath}    % For cases environment and math alignment
\usepackage{booktabs}

\usepackage{pgfplots}
\usepackage{csvsimple}
\usepackage{subcaption}

% Set PDFLaTeX output flag
\ifpdf
    \pdfoutput=1
\fi

\title{Real-Time Moving Flock Detection in Pedestrian Trajectories Using Sequential Deep Learning Models}

\author{
Amartaivan Sanjjamts\\
Department of Information and Physical Sciences\\
Graduate School of Information Science and Technology\\
Osaka University, Osaka, Japan\\
\texttt{s.amaru@ist.osaka-u.ac.jp}\\
\And
Morita Hiroshi\\
Department of Information and Physical Sciences\\
Graduate School of Information Science and Technology\\
Osaka University, Osaka, Japan\\
\texttt{morita@ist.osaka-u.ac.jp}\\
\And
Enkhtogtokh Togootogtokh\\
Department of Research and Development\\
Voizzr Technologies\\
Bavaria, Germany\\
\texttt{enkhtogtokh.java@gmail.com}\\
}


\date{\today}

\begin{document}
\maketitle

\begin{abstract}
Understanding collective pedestrian movement is crucial for applications in crowd management, autonomous navigation, and human-robot interaction. This paper investigates the use of sequential deep learning models, including Recurrent Neural Networks (RNNs), Long Short-Term Memory (LSTM) networks, and Transformers, for real-time flock detection in multi-pedestrian trajectories. Our proposed approach consists of a two-stage process: first, a pre-trained binary classification model is used for pairwise trajectory classification, and second, the learned representations are applied to identify multi-agent flocks dynamically. 

We validate our method using real-world group movement datasets, demonstrating its robustness across varying sequence lengths and diverse movement patterns. Experimental results indicate that our model consistently detects pedestrian flocks with high accuracy and stability, even in dynamic and noisy environments. Furthermore, we extend our approach to identify other forms of collective motion, such as convoys and swarms, paving the way for more comprehensive multi-agent behavior analysis. 

\end{abstract}


\section{Introduction}

The analysis of pedestrian trajectories has become an essential aspect of understanding human mobility patterns in various environments such as urban spaces, transportation systems, and public gatherings. In particular, the identification of pedestrian groups or "flocks" moving together in real-time is a challenging but crucial task. A flock can be defined as a group of individuals whose movements are highly correlated over time, often indicating a shared goal or destination. Detecting such flocks is not only important for crowd management and safety but also for enhancing the effectiveness of autonomous systems, such as self-driving cars, and improving human-robot interaction.

Collective motion in trajectory data can be categorized into different formats, including \textbf{flocks}, \textbf{convoys}, and \textbf{swarms} \cite{wang2020big}. A \textbf{flock} is a set of agents moving together within a limited spatial region over a specific time interval. A \textbf{convoy} extends this definition by maintaining the same group structure over longer periods, making it more stable in dynamic environments. A \textbf{swarm} represents a more loosely connected group, where individuals exhibit similar movement patterns but do not necessarily maintain fixed spatial relationships. In this study, we focus on \textbf{moving flock detection}, where groups of pedestrians dynamically form and dissolve while moving together over short time intervals.

Traditional approaches to flock detection have largely relied on density-based methods, such as clustering algorithms and graph-based representations, to identify groupings based on spatial proximity or movement patterns. While effective in some scenarios, these methods often struggle in dynamic environments with changing densities, irregular movement, or occlusions.

In contrast, modern data-driven approaches, particularly deep learning models, offer more flexibility and robustness in detecting real-time flocks. Techniques such as Recurrent Neural Networks (RNNs), Long Short-Term Memory (LSTM) networks, and Transformers have shown great promise in capturing the temporal dependencies in trajectory data, making them well-suited for flock detection in dynamic, multi-agent settings. These models can learn complex patterns from raw trajectory data and make predictions on the likelihood of group formation, even in challenging environments.

This paper investigates the use of deep learning models for real-time flock detection in multiple pedestrian trajectories, comparing the performance of modern sequential deep learning models like RNNs, LSTMs, and Transformers for pair detection and applying the pre-trained model to multiple trajectories to identify flock size and members.


\section{Literature Review}

Understanding collective pedestrian movement has been an active area of research, with various methodologies proposed to detect and analyze groups of individuals moving together. These approaches can broadly be categorized into traditional model-based methods and modern data-driven techniques, with recent advancements exploring hybrid strategies to improve detection accuracy.

\subsection{Traditional Approaches}

Early research in flock detection primarily relied on model-based and clustering techniques. One of the most widely used methods is Density-Based Spatial Clustering of Applications with Noise (DBSCAN), which identifies pedestrian groups based on spatial proximity and density \cite{ester1996density}. While effective in many cases, DBSCAN and other density-based techniques face challenges in dynamic environments where pedestrian group sizes and movement patterns change over time.

Graph-based methods have also been extensively used, where pedestrians are represented as nodes and edges denote spatial or temporal interactions \cite{helbing2000social}. These methods often employ social force models to simulate pedestrian dynamics, considering factors like inter-agent distance and collision avoidance. While useful for understanding local interactions, such approaches struggle with long-range dependencies and occlusions, limiting their applicability in real-world, large-scale crowd scenarios.

Another line of research involves rule-based and statistical models, such as Hidden Markov Models (HMMs) and Kalman filters, to estimate pedestrian movement patterns. These models typically assume linear motion patterns and rely on handcrafted features, making them less adaptable to complex and unstructured pedestrian flows \cite{li2008learning}. 

\subsection{Modern Data-Driven Approaches}

With the rise of deep learning, trajectory-based modeling has shifted towards data-driven methods that learn patterns directly from large-scale pedestrian datasets. Recurrent Neural Networks (RNNs) and Long Short-Term Memory (LSTM) networks have been widely used for pedestrian trajectory prediction and group detection due to their ability to capture sequential dependencies \cite{zhang2016deep}. These models have demonstrated improved accuracy in detecting pedestrian flocks by learning motion tendencies over time.

More recently, Transformer-based architectures have emerged as state-of-the-art models in sequence learning tasks. Unlike RNNs and LSTMs, Transformers rely on self-attention mechanisms, allowing them to model long-range dependencies more effectively without suffering from vanishing gradient issues \cite{vaswani2017attention}. This has led to improved performance in large-scale pedestrian behavior analysis, where long-term dependencies and complex interactions are prevalent.

Additionally, attention-based models, such as Social-LSTM and Social-GAN, have been proposed to model social interactions among pedestrians, dynamically adapting their predictions based on surrounding agents \cite{gupta2018social}. These models leverage learned representations to classify pedestrian groups, offering a more flexible alternative to traditional rule-based approaches.

\subsection{Hybrid Approaches}

Recent studies have explored the integration of traditional and deep learning-based methods to enhance flock detection capabilities. Hybrid models, such as DBSCAN combined with LSTMs, have been used to detect groups first through clustering and then refine predictions using sequential learning models \cite{jia2020trajectory}. Similarly, Graph Neural Networks (GNNs) have been applied to pedestrian movement analysis, capturing both spatial and temporal dependencies by integrating graph structures with deep learning architectures \cite{zhang2020graph}.

The combination of these approaches has demonstrated promising results, particularly in dynamic and large-scale scenarios where traditional clustering methods alone may fail. By leveraging both spatial clustering techniques and deep sequence modeling, hybrid models provide a robust framework for real-time multi-agent flock detection.



\section{Methodology}

In this study, we leverage sequential deep learning models to classify trajectory pairs as either forming a flock or not. The task is based on the premise that identifying patterns of movement within groups of pedestrians can offer valuable insights into pedestrian dynamics, such as crowd behavior, space utilization, and evacuation strategies. The problem is framed as a binary classification task, where each trajectory pair is classified as belonging to a flock (pair) or not (non-pair).

To achieve this, we employ a variety of sequential deep learning models. Specifically, we focus on architectures such as Recurrent Neural Networks (RNNs), Long Short-Term Memory (LSTM) networks, and Transformer-based models. Each of these models is trained to predict pairwise relationships between pedestrians based on their movement characteristics. The core idea is to capture temporal dependencies in the movement data, allowing the models to understand how pairs of pedestrians interact over time and how these interactions evolve.

\begin{figure}[H] 
    \centering
    \ifpdf
        \includegraphics[width=0.75\textwidth]{figure1.jpg}  % Use PDFLaTeX-compatible formats
    \fi
    \caption{Proposed Method Overview}
    \label{fig:proposed_method}
\end{figure}

As shown in Figure \ref{fig:proposed_method}, our proposed method consists of the following steps, each contributing to a comprehensive approach for accurate flock detection:

\begin{itemize} \item \textbf{Binary Classification:} We train a deep learning model specifically designed to predict whether a given pair of pedestrians belongs to a flock or not. This model is trained using trajectory data, where each pedestrian's movement features such as position, velocity, motion angle, and inter-agent distance are considered to make the classification decision. The binary classification framework allows the model to focus on detecting pairwise interactions, essential for identifying potential group behaviors. \end{itemize}

\begin{itemize} \item \textbf{Pairwise Evaluation:} Once the model is trained, it is applied across the entire dataset, evaluating each possible pair of pedestrians. By performing pairwise classification, we assess the temporal and spatial interactions between all pedestrians in the scene. This step is crucial because it helps identify potential pairings that might form a larger group or flock. It effectively transforms the problem of group detection into multiple binary decisions, enabling the model to efficiently handle dynamic pedestrian environments with varying numbers of individuals. \end{itemize}

\begin{itemize} \item \textbf{Flock Filtering:} After the pairwise evaluations are completed, a filtering mechanism is applied to refine the detected flocks. This process involves using predefined criteria such as a minimum flock size (e.g., a minimum of two pedestrians) and temporal consistency (e.g., ensuring that pairs that are detected as a flock maintain their relationship over time). Flock filtering ensures that the detected groups represent meaningful pedestrian formations rather than isolated or spurious pairings. The filtering mechanism also helps in removing false positives and reduces the likelihood of detecting non-flocking pairs, thereby improving the accuracy of the model's output. \end{itemize}

\begin{itemize} \item \textbf{Flock Aggregation:} After filtering, the final step involves aggregating the results of pairwise evaluations into larger flock formations. By clustering pairs that are consistently detected as belonging to the same flock, we can estimate the size and structure of the group. This aggregation step is key in translating the pairwise predictions into actionable insights, allowing us to identify the overall structure of pedestrian groups, such as convoys, clusters, or swarms, in the scene. The resulting groups are then analyzed for their behavior, density, and movement patterns. \end{itemize}

\subsection{Binary Classification Problem}

We formulate real-time flock detection as a binary classification problem. Given a pair of pedestrian trajectories, our goal is to determine whether the two pedestrians form a flock at a given time.  

Let $\mathbf{T}_i = \{(x_t^i, y_t^i) \mid t = 1, \dots, T\}$ and $\mathbf{T}_j = \{(x_t^j, y_t^j) \mid t = 1, \dots, T\}$ represent the trajectories of pedestrians $i$ and $j$, where $(x_t, y_t)$ denotes the position of a pedestrian at timestamp $t$.  

We define a binary function $\mathcal{F}$ to classify whether the pair $(i, j)$ belongs to the same flock:  

\begin{equation}
    \mathcal{F}(\mathbf{T}_i, \mathbf{T}_j) =
    \begin{cases}
      1, & \text{if } (i, j) \text{ belong to the same flock}, \\
      0, & \text{otherwise}.
    \end{cases}
\end{equation}

To learn this function, we train a deep learning model $\mathcal{M}_{\theta}$ with parameters $\theta$ such that:  

\begin{equation}
    \hat{y}_{i,j} = \mathcal{M}_{\theta}(\mathbf{T}_i, \mathbf{T}_j),
\end{equation}

where $\hat{y}_{i,j} \in \{0,1\}$ represents the predicted label indicating whether the pair belongs to a flock.
\subsection{Group Data Preparation}\label{Group_Data_Preperation}


To analyze pedestrian flocking behavior, we preprocess trajectory data by classifying trajectories based on their starting points under the assumption that pedestrians who start moving at similar times may belong to the same flock. The preparation process consists of two main steps: time-bin classification and sequence data selection.

\subsubsection{Time-Bin Classification}
First, the function read the preprocessed pedestrian trajectory data, which includes fields for timestamp, agent ID, position $(X,Y)$, velocity, motion angle, and face angle. To ensure uniform timestamp representation, the timestamps are converted into datetime format and adjusted to the appropriate time zone.

Next, we filter out agents with fewer trajectory points than the required sequence length ($L$). For each remaining agent, we identify the first recorded trajectory point (i.e., the minimum timestamp for each agent). These first trajectory points are then grouped into discrete time bins of size $T$, using:

\begin{equation}
\text{time\_bin} = \left\lfloor \frac{\text{TIMESTAMP} - \text{min}(\text{TIMESTAMP})}{T} \right\rfloor.
\end{equation}

Each bin represents a cohort of agents that started moving within the same time interval. The number of agents in each bin is recorded, and bins with zero agents are excluded. The resulting groups of agent IDs are stored in a file for further processing.

\subsubsection{Sequential Data Selection}
Once the agents are grouped into time bins, the trajectory data for each agent is extracted to form sequences of length $L$. For each agent, we ensure that a sufficient number of consecutive data points exist to meet the sequence length requirement. The data is then sorted by timestamp and agent ID.

The trajectory sequences for all agents within a given time bin are structured into blocks, where each block contains multiple agents' trajectories for a shared time interval. These structured blocks of sequential data are then exported as scene data for further analysis.


This method enables the identification of potential pedestrian flocks by examining agents who start moving at similar times, facilitating a deeper analysis of collective movement behaviors.


\subsection{Multi-agent flock identification}\label{multi-agent}

To identify multi-agent flocks from pairwise classification results, we employ a union-find-based clustering approach. This method efficiently groups pedestrians into cohesive flocks based on their classified pairwise relationships. The algorithm processes an input list of pedestrian pairs that have been determined to be part of the same flock and iteratively merges them into larger clusters. By leveraging the union-find data structure, we ensure that the clustering process remains computationally efficient, even for large-scale pedestrian datasets. The final output is a set of detected flocks, where each flock represents a cohesive group of individuals moving together within the observed environment.

\subsubsection{Input and Output}

\begin{itemize}
    \item \textbf{Input:} A list of tuples, where each tuple contains:
    \begin{itemize}
        \item A pair of pedestrian identifiers $(p_1, p_2)$.
        \item A binary classification result: 1 if the pair belongs to the same flock, 0 otherwise.
    \end{itemize}
    \item \textbf{Output:} A list of detected flocks, each represented as:
    \begin{itemize}
        \item The number of pedestrians in the flock.
        \item A sorted list of pedestrian identifiers forming the flock.
    \end{itemize}
\end{itemize}

\subsubsection{Algorithm Description}

The algorithm operates as follows:

\begin{enumerate}
    \item \textbf{Union-Find Initialization:} Each pedestrian is initially considered its own flock.
    \item \textbf{Pairwise Flock Merging:} For each detected flock pair:
    \begin{enumerate}
        \item Find the representative (root) of each pedestrian's flock using the \textit{find\_root} function.
        \item If the pedestrians belong to different flocks, merge them by updating their root.
    \end{enumerate}
    \item \textbf{Flock Organization:} After processing all pairs, pedestrians belonging to the same root are grouped into final flock structures.
    \item \textbf{Output Generation:} The function returns a list where each detected flock is represented by its size and the sorted list of pedestrian identifiers.
\end{enumerate}

\subsubsection{Union-Find with Path Compression}

To efficiently manage flock merging, we implement a union-find data structure with path compression. Given a pedestrian $p$, the function \texttt{find\_root(p)} recursively finds the root representative:

\begin{equation}
    \text{flocks}[p] =
    \begin{cases}
      p, & \text{if } p \text{ is its own root}, \\
      \text{find\_root}(\text{flocks}[p]), & \text{otherwise}.
    \end{cases}
\end{equation}

Path compression ensures that subsequent searches for the same pedestrian are efficient, reducing the overall complexity of the algorithm to nearly $\mathcal{O}(N)$ in practice.



\section{Experiments and Results}

We compare RNN, LSTM, and Transformer models in terms of accuracy and performance.

\subsection{Data and Model Preperation}

\subsubsection{Dataset}

We use a real-world pedestrian group-identified trajectory dataset \cite{zanlungo2015pedestrian} from an indoor environment recorded at the Asian Trade Center in Osaka, Japan.

Each row in the CSV file corresponds to a single tracked individual at a specific time instant and includes the following fields: \textit{time [ms] (Unix time + milliseconds/1000)}, \textit{person ID}, \textit{position x [mm]}, \textit{position y [mm]}, \textit{velocity [mm/s]}, \textit{angle of motion [rad]}, and \textit{facing angle [rad]}.


The group files contain annotations for the groups present on a given day. Only pedestrians in groups are listed, while those walking alone are excluded. Each row represents a single tracked pedestrian within a group and includes the following fields (space-separated values): $PEDESTRIAN-ID$, \textit{$GROUP-SIZE$}, \textit{$PARTNER-ID-1$}... (list of IDs of all other pedestrians in the group), \textit{$NUMBER-OF-INTERACTING-PARTNERS$}, and \textit{$INTERACTING-PARTNER-ID-1$}... (list of all socially interacting partners).

For the binary classification model training, we select only data where \textit{$GROUP-SIZE$} = 2, focusing on pairs of pedestrians from a single day. The pre-trained model is then tested on data from different days, ensuring that the model is evaluated on entirely unseen data.

\subsubsection*{Addressing Class Imbalance}

One challenge in training the binary classification model is class imbalance, as natural pedestrian movement data may contain significantly more non-flocking samples than flocking ones. To mitigate this issue, we apply a combination of:

Weighted Loss Functions: Assigning higher penalties to misclassified minority class samples.
Oversampling \& Undersampling: Augmenting minority class sequences and reducing redundant majority class samples.
Synthetic Data Generation: Creating additional flocking sequences by interpolating existing trajectories.
By employing these techniques, we ensure that the model does not develop a bias toward the majority class and maintains high predictive accuracy across both categories.
We split the dataset into training and testing sets with an 80/20 ratio. Depending on the sequence length (the number of records in a single sample), the number of samples vary.


\begin{table}[H]
   
    \label{tab:data_summary}
    \centering
    
    \begin{tabular}{@{}cccc@{}} 
     \toprule
     \textbf{Sequence Length} & \textbf{\# of total samples} & 
     \textbf{\# of training samples} & \textbf{\# of excluded agents} \\
     \midrule
    30 & 2794 & 2235  & 102 \\
    60 & 2758 &  2206 &  138\\
    100 & 2684 & 2147 &  212\\
    150 & 2612 & 2090 &  284\\
    200 & 2542 & 2034 & 354\\
    300 & 2404 & 1923 & 492\\
    500 & 2022 & 1618 &  874\\
     \bottomrule
    \end{tabular}  
    \caption{Sample Sequence Length vs. Number of Samples on 2013/05/05}
    \label{tab:seq_vs_sample}
\end{table}

As shown in Table \ref{tab:seq_vs_sample}, we extract sequence lengths of 30, 60, 100, 150, 200, 300, and 500. Based on the prepared pair pedestrian IDs, we select trajectory data where the current pedestrian meets the required sequence length, with a label of '1' indicating a flock. We then randomly select two single pedestrian datasets and add them to the dataset, labeling them '0' to indicate they are not part of a flock. As a result, the \textit{\# of total samples} include both pair and non-pair data, maintaining a 50-50 ratio. 


\subsubsection{Deep Learning Model Preparation}

\subsubsection*{Feature Engineering}

In accordance with the definition of a flock, we extract features that capture spatial and behavioral interactions between pedestrians. These include the center of the pair, the inter-distance between pedestrians, and absolute differences in motion properties such as velocity, motion angle, and facing angle. Additionally, we incorporate trajectory-based similarity metrics like Dynamic Time Warping (DTW) to enhance the model's ability to distinguish between temporary co-movement and sustained flocking behavior.

To ensure robust performance, we preprocess the extracted features through normalization techniques, such as min-max scaling and standardization, preventing bias due to differences in measurement units. Furthermore, we explore additional features related to pedestrian interaction, such as relative velocity angle, trajectory curvature, and acceleration differences.

For our binary classification sequential model, each sample consists of multiple sequential data points, where each instance corresponds to a label of '1' (indicating that the pair forms a flock) or '0' (indicating that they do not). The final set of features, selected based on their predictive importance, is listed in Table \ref{tab:feature_table}

\begin{table}[H]
    
    \centering
    \resizebox{\textwidth}{!}{
    \begin{tabular}{@{}llp{0.6\textwidth}@{}} % Adjust column widths as needed
    \toprule
    \textbf{Input (Features)} & \textbf{Feature Name}     & \textbf{Description} \\ 
    \midrule
    1 & \texttt{centerX}           & X-coordinate of the center between the two pedestrians in the pair. \\
    2 & \texttt{centerY}     & Y-coordinate of the center between the two pedestrians in the pair. \\
    3 & \texttt{interDistance}            & Euclidean distance between the two pedestrians in the pair. \\
    4 & \texttt{timeDifference}       & Absolute difference in timestamps between the two pedestrians in the pair. \\
    5 & \texttt{velocityDifference}       & Absolute difference in velocity between the two pedestrians in the pair. \\
    6 & \texttt{motionAngleDifference} & Absolute difference in motion angles between the two pedestrians in the pair. \\    
    7  & \texttt{faceAngleDifference}     & Absolute difference in facing angles between the two pedestrians in the pair. \\
    8  & \texttt{dtwValues}     & Pair trajectory similarity metric calculated using Dynamic Time Warping (DTW). \\
    
    \bottomrule
    \textbf{Output (Label)} & \texttt{label} &  Predicted label indicating whether the pair is a flock (1) or not (0). \\ 
    \bottomrule
    \end{tabular}
    }    
    \caption{Feature Description for Binary Classification Sequential Models (Inputs and Output)}
    \label{tab:feature_table}
      
\end{table}


\subsubsection*{Binary Model Parameter Selection}

In the process of selecting the optimal hyperparameters for our binary classification sequential model, we primarily draw from established best practices in the field. The model's hyperparameters, including batch size and hidden layer sizes, play a critical role in ensuring effective learning and generalization. We initially follow guidelines found in key research papers, which focus on the importance of carefully tuning these parameters to achieve robust performance.

For batch sizes and hidden layer sizes, we refer to studies such as \cite{deepsurrogate} and \cite{batchsizepaper}, which explore the impact of various batch sizes on model stability and convergence rates. Another important paper is \cite{hiddenlayers}, which discusses the effects of hidden layer sizes and their impact on model expressiveness and training efficiency.

To select the best-performing hyperparameters, we conduct tests across different combinations of batch sizes and hidden layer sizes. Specifically, we test the following combinations:
\begin{itemize}
    \item \textbf{Batch Sizes}: [64, 32, 16, 8]
    \item \textbf{Hidden Layer Sizes}: [256, 128, 64, 32, 16]
\end{itemize}

These combinations are chosen to balance the trade-off between computational efficiency and model performance. Larger batch sizes may speed up training but could lead to poorer generalization, while smaller batch sizes might improve generalization at the cost of longer training times. Similarly, different hidden layer sizes can affect the model's ability to capture complex patterns in the data while maintaining computational feasibility.

The result sections will describe the results from conducting experiments with these parameter combinations, comparing their performance based on validation accuracy and other relevant metrics.


\subsubsection{Multi-agent group data preparation}

As described in Section \ref{Group_Data_Preperation}, we prepare timestamp bin data by aligning the starting point of each pedestrian within the current timestamp range. 


\begin{figure}[H] 
    \centering
    \ifpdf
        \includegraphics[width=0.75\textwidth]{figure2.jpg}  % Use PDFLaTeX-compatible formats
    \fi
    \caption{Sample Member Count for Each Bin on 2013/05/05}
    \label{fig:member_count_chart}
\end{figure}

The bar chart shown in Figure \ref{fig:member_count_chart} visualizes the \textbf{member count} across distinct \textbf{time ranges} from the dataset. The data is grouped into four primary time periods that represent significant hours of pedestrian activity, as described in \url{https://dil.atr.jp/sets/groups/}:

\begin{itemize} \item 10:00 - 11:00 \item 12:00 - 13:00 \item 15:00 - 16:00 \item 19:00 - 20:00 \end{itemize}

Each time range is subdivided into 1-minute intervals, with the height of the bars indicating the number of pedestrians observed in each interval. The distribution of pedestrians across these time ranges allows for a more detailed analysis of pedestrian activity within each hour. Given that the sequence length used in the analysis was 100, the data is processed in 1-minute bins to capture a finer temporal resolution of pedestrian movement.

The total number of bins in the dataset is calculated to be 255. These bins serve as the basis for further analysis, including the detection of flocking behavior and the execution of pairwise interactions, which are crucial for understanding collective pedestrian dynamics. 

Depending on the duration of the time interval and the sequence length of the sample, the number of bins and the member count in each bin may vary. In this study, we set the sequence length to 100 and the time interval to 1-minute for the experiment.

\subsection{Computational Environment}

The computational environment for the present study consists of a high-performance workstation equipped with an Intel(R) Core(TM) i9-13900 processor (32 CPUs), 64 GB of RAM, and a powerful 32GB NVIDIA GeForce RTX 4070 Ti GPU. The system is configured to support intensive computational tasks and is running on Python 3.12. For deep learning tasks, we leverage PyTorch, with TabNet-4.1 being employed for regression and classification tasks.

The following software libraries and frameworks were utilized in this study:

\begin{itemize}
    \item \textbf{Deep Learning Models}: We implemented various models using \textit{PyTorch}, including Recurrent Neural Networks (RNN), Long Short-Term Memory networks (LSTM), and Transformer-based models. Key hyperparameters include \texttt{learning\_rate} = 0.001 and \texttt{num\_epochs} = 1000.
\end{itemize}

\begin{itemize}
    \item \textbf{Similarity Measurement}: For calculating the similarity between trajectories, we utilized the \textit{fastdtw} package, which efficiently computes the Euclidean distance between time-series data.
\end{itemize}

\begin{itemize}
    \item \textbf{Visualization}: We used the \textit{Matplotlib} library for high-quality visualizations of the flocking patterns, allowing for clear and efficient presentation of the agent trajectories and group dynamics.
\end{itemize}


\subsection{Main Results}

In this section, we present the results of the binary classification model by evaluating and comparing the performance of RNN, LSTM, and Transformer models in flock detection. The experiments were conducted on multiple datasets with varying sequence lengths (i.e., number of samples) to assess the impact of sequence size on model performance. The results indicate that both runtime and classification accuracy vary significantly depending on the model architecture, even when applied to the same dataset.



\subsubsection{Binary Classification Accuracy by Data Length and Models}

The following reported accuracy values represent the \textbf{average of 20-independent runs} (combination of batch  and hidden layer size) for each model. The results highlight the increasing performance of all three models (RNN, LSTM, Transformer) as sequence length increases.


\begin{figure}[H] 
    \centering
    \ifpdf
        \includegraphics[width=0.75\textwidth]{figure4.jpg}  % Use PDFLaTeX-compatible formats
    \fi
    \caption{Sequential models accuracy for different sequences}
    \label{fig:model_accuracy}
\end{figure}

\begin{itemize}
    \item Transformer Model Performance. 
    The Transformer consistently outperforms both RNN and LSTM across all sequence lengths. Notably, at \textbf{sequence length = 100}, the Transformer model exceeds \textbf{90\% accuracy} in some runs, averaging \textbf{88.06\%}. It continues to improve with longer sequences, reaching \textbf{95.70\% at sequence length = 500}.
    \item LSTM Model Performance.
    The LSTM model demonstrates \textbf{strong performance for longer sequences}, surpassing \textbf{80\% accuracy} at \textbf{150 sequences} and achieving \textbf{86.96\% at 500 sequences}. It consistently outperforms RNN across all sequence lengths, showing its ability to better capture long-term dependencies.
    \item RNN Model Performance.
    The RNN model lags behind the LSTM and Transformer, especially for longer sequences. While it shows improvement with increased sequence length, it struggles to exceed \textbf{70\% accuracy} even at \textbf{500 sequences}, peaking at \textbf{71.43\%}.
    
    
\end{itemize}

\textbf{Key Observations}
\begin{itemize}
    \item \textbf{Longer sequences benefit all models}, with the Transformer making the most significant gains.
    \item \textbf{LSTM handles longer sequences better than RNN}, suggesting that its gating mechanisms effectively capture dependencies.
    \item \textbf{The Transformer is the best-performing model} across all tested sequence lengths, achieving over \textbf{90\% accuracy} for sequences \textbf{$\geq$ 150}.
\end{itemize}


\subsubsection{RunTime Comparison}

The reported runtime values represent the \textbf{average of 20 independent runs}, considering different combinations of batch size and hidden layer size for each model. The results highlight the increasing runtime of all three models (RNN, LSTM, Transformer) as sequence length increases. Since actual training involved early stopping, the runtime could vary across runs. To ensure a fair comparison of model training efficiency, we set the number of epochs to 1000 for all training instances. 

\begin{figure}[H] 
    \centering
    \ifpdf
        \includegraphics[width=0.75\textwidth]{figure5.jpg}  % Use PDFLaTeX-compatible formats
    \fi
    \caption{Model training time for different sequences}
    \label{fig:model_training_runtime}
\end{figure}

The runtime comparison graph indicates that the RNN model has the shortest training time, making it the fastest among the three models due to its simpler structure. In contrast, the LSTM model has the longest training time across all sequence lengths, likely due to its more complex memory mechanisms.

The training time also varies with sequence length. Interestingly, the fastest training occurred at sequence lengths 100 and 300, while sequence length 150 took the longest to train. This suggests that certain sequence lengths may introduce computational inefficiencies.

\textbf{Key Observations}
\begin{itemize}
    \item RNN is the fastest model across all sequence lengths due to its simpler architecture.
    \item LSTM has the longest training time, peaking at 57.1 minutes for sequence length 150.
    \item Transformer runtimes closely follow LSTM but are slightly shorter in most cases.
    \item Sequence lengths 100 and 300 resulted in the shortest runtimes across all models, while 150 took the longest to train.

\end{itemize}

These results highlight the trade-offs between model complexity and training efficiency. While LSTM and Transformers are more powerful for sequence learning, they require significantly longer training times than RNNs. Choosing the right sequence length can also impact training efficiency, with some lengths (e.g., 100, 300) being computationally favorable.

\subsubsection{Comparison of Accuracy and Runtime Across Models}

\begin{figure}[H] 
    \centering
    \ifpdf
        \includegraphics[width=0.75\textwidth]{figure3.jpg}  % Use PDFLaTeX-compatible formats
    \fi
    \caption{Model training time for different sequences}
    \label{fig:model_training_accuracy_runtime}
\end{figure}

The figure illustrates the trade-off between \textbf{accuracy (\%)} and \textbf{runtime (minutes)} for RNN, LSTM, and Transformer models across different sequence lengths.

\begin{itemize}
    \item \textbf{Accuracy:} Transformer achieves the highest accuracy, followed by LSTM, while RNN performs the worst.
    \item \textbf{Runtime:} RNN is the fastest, while LSTM takes the longest time. Transformer is slightly faster than LSTM.
    \item \textbf{Sequence Length:} Training is more efficient at sequence lengths \textbf{100 and 300}, while \textbf{150 takes the longest}.
    \item \textbf{Key Trade-off:} RNN is best for speed, while Transformer offers a better balance between accuracy and runtime.
\end{itemize}


\subsubsection{Sample Screens for pair prediction}

Below are sample pair detection results for a sequence length of 100. The model accurately predicts complex trajectories as pairs in many cases. However, in some instances, it may incorrectly detect two individual pedestrians moving in similar directions as a pair.

\begin{figure}[H]
    \centering
    % First row of images
    \ifpdf
        \begin{subfigure}[c]{0.45\textwidth} 
            \centering
            \includegraphics[width=\textwidth]{figure6.jpg}
            \caption{The pair prediction result for straight trajectories.}
            \label{fig:pair1}
        \end{subfigure}
    \fi
    \hfill
    \begin{subfigure}[c]{0.45\textwidth}
        \centering
        \includegraphics[width=\textwidth]{figure7.jpg}
        \caption{The pair prediction result for complex trajectories.}
        \label{fig:pair2}
    \end{subfigure}

    \caption{Sample pair detection results for a sequence length of 100.}
    \label{fig:pair_prediction}
\end{figure}

\subsubsection{Flock Validation}

We validate our flock detection algorithm using the group files described in \cite{zanlungo2015pedestrian}. These group files contain group-labeled data for 6 days. After performing data preprocessing and cleaning, we found that most flocks have sizes of 2, 3, at the most between 11 and 12 across all the group files. We validated our method by training the model on data from different days, introducing unseen data to assess the model's generalization ability.

As we described in \ref{multi-agent} our flock detection algorithm iterates through all combinations to predict pairs and then filters the results to determine the flock size based on pair predictions. 

We further describe an example of flock validation results for the groups on 2013/02/17, using a pre-trained Transformer model with a sequence length = 100, batch size = 32, hidden layer size=64 and model accuracy 91.35\%. We set the confidence score for pair prediction to 0.9. As a result, all the flocks were correctly detected, with the following "group size":"count" distribution:

\textit{{"2": 1528, "4": 143, "3": 448, "5": 36, "7": 4, "6": 6, "11": 1, "9": 1, "8": 2}}


\subsubsection{Sample Screens for flock detection}

\begin{figure}[H]
    \centering
    % First row of images
    \ifpdf
        \begin{subfigure}[c]{0.45\textwidth} 
            \centering
            \includegraphics[width=\textwidth]{figure8.jpg}
            \caption{Detected flock size with 2.}
            \label{fig:flock2}
        \end{subfigure}
    \fi
    \hfill
    \begin{subfigure}[c]{0.45\textwidth}
        \centering
        \includegraphics[width=\textwidth]{figure9.jpg}
        \caption{Detected flock size with 3.}
        \label{fig:flock3}
    \end{subfigure}

    % Second row of images
    \begin{subfigure}[c]{0.45\textwidth}
        \centering
        \includegraphics[width=\textwidth]{figure10.jpg}
        \caption{Detected flock size with 5.}
        \label{fig:flock5}
    \end{subfigure}
    \hfill
    \begin{subfigure}[c]{0.45\textwidth}
        \centering
        \includegraphics[width=\textwidth]{figure11.jpg}
        \caption{Detected flock size with 7.}
        \label{fig:flock7}
    \end{subfigure}

    \caption{Sample flock detection screen steps with 4 different sizes}
    \label{fig:flock_detection}
\end{figure}


\section{Discussion}

In this section, we discuss the results of our flock detection algorithm and highlight key observations regarding the impact of data sequence length and the dependency of features used in the model.

\subsection*{Impact of Data Sequence Length}
The performance of our model demonstrates that using a sequence length of 100 steps results in over 90\% accuracy for pair detection with a set of 8 features. While this sequence length yields high accuracy, it is worth noting that 100 steps might seem relatively long from a human perspective when identifying pairs in a trajectory. Human perception of trajectories typically involves recognizing patterns and interactions over shorter durations. Despite this, our results indicate that using longer sequences provides the model with more comprehensive information about the agents' movements, which is crucial for accurately predicting pairings. However, a balance between sequence length and model performance must be considered, as excessively long sequences could lead to increased computational complexity and diminishing returns in predictive accuracy beyond a certain point.

\subsection*{Feature Dependency}
The current feature set for the binary classification model, consisting of center coordinates (centerx, centery), inter-agent distance, absolute values of the timestamp, velocity, motion angle, and face angle, has shown promising results in detecting pairs. With a total of 8 features, the model achieved high performance. Interestingly, in some instances, reducing the number of features to 6 (excluding centerx and centery) still yielded similar accuracy results. This suggests that the model can effectively detect pairs based on more dynamic features, such as inter-agent distance, velocity, and motion angles. The less significant contribution of static features like center coordinates in certain instances could inform future efforts to optimize the feature set, potentially simplifying the model without sacrificing performance. In practice, reducing the feature space could improve computational efficiency while maintaining high accuracy, especially for real-time applications.

In both of these cases, we further analyze and conduct additional experiments to explore the optimal balance between sequence length and feature set for improving model efficiency and generalization across various scenarios.

\section{Conclusion}

We proposed a robust flock detection method leveraging a pre-trained binary classification model, which excels in identifying flocks in dynamic environments with varying group sizes. One of the key advantages of our approach is its ability to handle highly dynamic scenes, where the number of groups can fluctuate over time. This flexibility is essential for accurately detecting groups under different conditions, making it well-suited for real-world applications. Our method has shown its potential in real-time flock detection, with applications spanning pedestrian movement analysis, space utilization, crowd management, and surveillance systems.

Additionally, the method demonstrates versatility by extending to more complex group behaviors, such as the identification of convoys or swarms. This enhancement is particularly valuable in scenarios where agent interactions form intricate patterns that need to be understood for effective crowd control or behavioral analysis.

Our current implementation uses a sequence length of 100, which delivers high accuracy in group detection. However, we recognize that in real-time systems, reducing the sequence length could be beneficial to enhance computational efficiency and allow for faster decision-making processes. Future work will explore ways to optimize this sequence length to achieve a balance between accuracy and efficiency. Shorter sequence lengths will enable broader applications, especially in environments that demand quick responses, such as smart city traffic management or disaster response systems.

Furthermore, the method can be enhanced by experimenting with different sequential deep learning models, such as bidirectional LSTMs or attention-based models like the Transformer. These models could potentially improve accuracy in identifying group dynamics by leveraging both past and future trajectory data. Exploring different model architectures and incorporating hybrid models could also enhance the system’s adaptability to various environments and increase its ability to generalize across unseen data. In addition to sequence length optimization, the integration of more sophisticated model structures may further contribute to the development of scalable and efficient flock detection methods.

In summary, while our current model performs effectively with the existing feature set and sequence length, further research will focus on optimizing both the model architecture and input data to ensure scalability and efficiency for a wider range of real-time applications.

\bibliographystyle{unsrt}
\bibliography{references}

\end{document}


% \section{Biography Section}

% \begin{IEEEbiography}[{\includegraphics[width=1in,height=1.25in,clip,keepaspectratio]{Toey.jpg}}]{Phaphontee Yamchote}
% 	received B.Sc. and M.Sc. degrees in Mathematics from Chulalongkorn University, Thailand. He is currently pursuing a Ph.D. degree in Computer Science at the Faculty of Information and Communication Technology, Mahidol University, Thailand. His research interests include mathematical logic, computational logic in artificial intelligence, algorithmic machine learning, and data science.
% \end{IEEEbiography}
% \begin{IEEEbiography}[{\includegraphics[width=1in,height=1.25in,clip,keepaspectratio]{fig1}}]{Saw Nay Htet Win}
% 	Ba-ba-ba-ba-ba-na-na
% \end{IEEEbiography}
% \begin{IEEEbiography}[{\includegraphics[width=1in,height=1.25in,clip,keepaspectratio]{Chai.jpg}}]{Chainarong Amornbunchornvej}
% 	received the bachelor of engineering degree (with honor) in computer engineering in 2011 and the master's degree in telecommunications engineering in 2013, both from King Mongkut’s Institute of Technology Ladkrabang, Bangkok, Thailand. He received the Ph.D. degree in computer science from the University of Illinois Chicago, IL, USA, in 2018. He is currently a researcher at National Electronics and Computer Technology Center, Thailand. He focuses on data science and statistical inference in general especially in time series analysis, causal inference, social network analysis, theoretical computer science, as well as bioinformatics
% \end{IEEEbiography}
% \begin{IEEEbiography}[{\includegraphics[width=1in,height=1.25in,clip,keepaspectratio]{noras-3363879-small}}]{Thanapon Noraset} received the bachelor’s degree from the Faculty of Information and Communication Technology, in 2007, and the Ph.D. degree in computer science from Northwestern University, USA, in 2017. He is currently a Faculty Member with the Faculty of Information and Communication Technology, Mahidol University, Thailand. His research work and interests include natural language processing and machine learning.
% \end{IEEEbiography}


% \section{appendix}
{
\clearpage % Ensures the layout change happens on a new page
\onecolumn
\appendices
	\section*{Proof of the Correspondence Question}
	To prove Theorem \ref{mainThm}, we need some prerequisites as follows:
	\begin{prop}\label{represent_equi}
		Let $C_1$ and $C_2$ be cluster graphs in $\mathcal{G}(n)$ for $n \in \N$. Then
		$[C_1]_\sim = [C_2]_\sim$ if and only if $C_1 = C_2$.
	\end{prop}
	\begin{proof}
		The backward direction is trivial. We need to prove only the forward direction.
		Assume that $[C_1]_\sim = [C_2]_\sim$. That is, $C_1 \sim C_2$.
		To prove that $C_1 = C_2$, we prove $E(C_1) = E(C_2)$, and by symmetry, we prove only $E(C_1) \subseteq E(C_2)$.
		
		To prove that $E(C_1) \subseteq E(C_2)$, let $\{u,v\} \in E(C_1)$ for nodes $u,v \in N_n$. Then $u$ and $v$ are trivially reachable in $C_1$.
		Since $C_1\sim C_2$, they also are reachable in $C_2$.
		Since $C_2$ is a cluster graphs, $\{u, v\} \in E(C_2)$.
		Hence $E(C_1) \subseteq E(C_2)$, and also $E(C_1) = E(C_2)$. Therefore, $C_1 = C_2$.
	\end{proof}
	\begin{lem} \label{lem1}
		The set of partitions of the set $N_n$, named $P(n)$, one-to-one corresponds to $\mathfrak{G}(n)$\end{lem}
	\begin{proof}
		Observe the function $F:P(n)\to \mathfrak{G}(n)$ defined by 
		$
		F(\mathfrak{A}) = \left[ C(\mathfrak{A}) \right]_\sim
		$ for any $\mathfrak{A}\in P(n)$.
		
		%	For convenience, let $C_1$ and $C_2$ be cluster graphs
		%	\begin{align*}
			%		C_1 &= \{ \{x,y \} : x,y \in A, x\neq y, A\in \mathfrak{A}_1, \text{ and } |A|>1 \}\\
			%		C_2 &= \{ \{x,y \} : x,y \in A, x\neq y, A\in \mathfrak{A}_2, \text{ and } |A|>1 \}.
			%	\end{align*}
		
		First we show that $F$ is well-defined.
		Let $\mathfrak{A}_1$ and $\mathfrak{A}_2$ be two partition of $N_n$ such that $\mathfrak{A}_1 = \mathfrak{A}_2$.
		Since $\mathfrak{A}_1 = \mathfrak{A}_2$, we have $C(\mathfrak{A}_1) = C(\mathfrak{A}_2)$.
		Then
		$F(\mathfrak{A}_1) = [C(\mathfrak{A}_1)]_\sim = [C(\mathfrak{A}_2)]_\sim = F(\mathfrak{A}_2).$
		Hence $F$ is well-defined.
		
		Next, we prove injection.
		Let $\mathfrak{A}_1$ and $\mathfrak{A}_2$ be two partitions of $N_n$ such that
		%	$F(\mathfrak{A}_1) = F(\mathfrak{A}_2)$. Then $C(\mathfrak{A}_1) = C(\mathfrak{A}_2)$ because $C(\mathfrak{A}_i)$ is a cluster graph.
		%	Assume to the contrary that
		$\mathfrak{A}_1\neq\mathfrak{A}_2$.
		Then, without loss of generality, there must be $A \subseteq N_n$ such that $A \in \mathfrak{A}_1$ but $A \notin \mathfrak{A}_2$.
		
		\underline{Case 1}: $|A|=1$. Let say $A = \{a\}$.
		Since $\{a\} \in \mathfrak{A}_1$ which is a partition, we obtain that $a\notin e$ for all edge $e\in C(\mathfrak{A}_1)$.
		Since $\{a\} \notin \mathfrak{A}_2$, there must be some $k\in N_n$ such that $a,k \in Y$ for some $Y\in \mathfrak{A}_2$.
		Then $\{a,k\} \in C(\mathfrak{A}_2)$.
		Hence $C(\mathfrak{A}_1)\neq C(\mathfrak{A}_2)$ in this case.
		
		\underline{Case 2}: Since $A\notin \mathfrak{A}_2$ and $|A|>1$, there are two possible sub-cases as follows:
		\begin{enumerate}
			\item There is $a \in N_n\setminus A$ such that $A \cup \{a\}\subseteq B$ for some partition $B \in \mathfrak{A}_2$.
			\item There are $a, b \in A$ such that $a\neq b$ and they are in different partition in $\mathfrak{A}_2$.
		\end{enumerate}
		~~~~~~\underline{Case 2.1}: there is $a \in N_n\setminus A$ such that $A \cup \{a\}\subseteq B$ for some partition $B \in \mathfrak{A}_2$. Let $x,y \in A$. Then $\{x,a\}\notin C(\mathfrak{A}_1)$ but $\{x,a\}\in C(\mathfrak{A}_2)$.
		Hence $C(\mathfrak{A}_1)\neq C(\mathfrak{A}_2)$ in this case.
		
		~~~~~~\underline{Case 2.2}: there are $a, b \in A$ such that $a\neq b$ and they are in different partition in $\mathfrak{A}_2$. Then $\{a,b\}\in C(\mathfrak{A}_1)$ but $\{a,b\}\notin C(\mathfrak{A}_2)$.
		Hence $C(\mathfrak{A}_1)\neq C(\mathfrak{A}_2)$ in this case.
		
		Therefore, by Proposition \ref{represent_equi}, we can conclude that $F(\mathfrak{A}_1) \neq F(\mathfrak{A}_2)$, and hence $F$ is injective.
		
		Lastly, surjectivity of $F$ is trivial by considering the partition $\mathfrak{A}$ of sets which is constructed by any cluster graph $C$ and is defined as:
		For any nodes $u,v\in N_n$, they are in the same set in $\mathfrak{A}$ if $\{u,v\} \in E(C)$, and
		if a node $u \in N_n$ is isolated, then $\{ u \} \in \mathfrak{A}$.
		Then $F(\mathfrak{A}) =[C]_\sim $. Therefore $F$ is one-to-one correspondence between $P(n)$ and $\mathfrak{G}(n)$. Now, we are done for the correspondence question.
	\end{proof}
	\begin{lem} \label{lem2}
		$\mathcal{DM}(n)$ one-to-one corresponds to $P(n)$.
	\end{lem}
	\begin{proof}[\textbf{Proof of Lemma \ref{lem2}}]
		Observe the function $f:\mathcal{DM}\to P(n)$ defined by
		$
		f\left( \sum_{i=1}^{n} \prod_{j=1}^{m_i} x_{i_j} \right) = \left\{\left\{x_{i_j}\right\}_{j=1}^{m_i}\right\}_{i=1}^n
		$.
		It is obvious that $f$ is well-defined and surjective.\\
		
		To prove injectivity, we proceed the proof by induction on $n$.
		For $n=1$, suppose that $f\left( \prod_{j=1}^{m} x_j \right) =f\left( \prod_{j=1}^{m'} y_j \right)  $. Then $\left\{\left\{x_{j}\right\}_{j=1}^{m}\right\} = \left\{\left\{y_{j}\right\}_{j=1}^{m'}\right\}$.
		Since they are singleton sets, we have that $\left\{x_{j}\right\}_{j=1}^{m} = \left\{y_{j}\right\}_{j=1}^{m'}$.
		Then $m = m'$, and also 
		$\prod_{j=1}^{m} x_j = \prod\left\{x_{j}\right\}_{j=1}^{m} = \prod\left\{y_{j}\right\}_{j=1}^{m'} = \prod_{j=1}^{m'} y_j$.
		Next, for $n>1$, we assume the induction hypothesis which is stated as $\sum_{i=1}^{k} \prod_{j=1}^{m_i} x_{i_j} = \sum_{i=1}^{k} \prod_{j=1}^{m_i} y_{i_j}$ provided $f(\sum_{i=1}^{k} \prod_{j=1}^{m_i} x_{i_j}) = f(\sum_{i=1}^{k} \prod_{j=1}^{m_i} y_{i_j})$ for all $k\leq n-1$.
		Suppose that $f(\sum_{i=1}^{n} \prod_{j=1}^{m_i} x_{i_j}) = f(\sum_{i=1}^{n} \prod_{j=1}^{m_i} y_{i_j})$.
		Then $\left\{\left\{x_{i_j}\right\}_{j=1}^{m_i}\right\}_{i=1}^n = \left\{\left\{y_{i_j}\right\}_{j=1}^{m_i}\right\}_{i=1}^n$.
		Thus $\{x_{n_j}\}_{j=1}^{m_n} \in \left\{\left\{y_{i_j}\right\}_{j=1}^{m_i}\right\}_{i=1}^n$. Without loss of generality, let say that $\{x_{n_j}\}_{j=1}^{m_n} = \{y_{n_j}\}_{j=1}^{m_n}$.
		Thus $\prod_{j=1}^{m_n} x_{n_j} = \prod_{j=1}^{m_n} y_{i_j}$ and also
		$\left\{\left\{x_{i_j}\right\}_{j=1}^{m_i}\right\}_{i=1}^{n-1} = \left\{\left\{y_{i_j}\right\}_{j=1}^{m_i}\right\}_{i=1}^{n-1}$. By induction hypothesis, we have $\sum_{i=1}^{n-1} \prod_{j=1}^{m_i} x_{i_j} = \sum_{i=1}^{n-1} \prod_{j=1}^{m_i} y_{i_j}$. Thus
		$
		\sum_{i=1}^{n} \prod_{j=1}^{m_i} x_{i_j} = \sum_{i=1}^{n-1} \prod_{j=1}^{m_i} x_{i_j} + \prod_{j=1}^{m_n} x_{n_j}
		= \sum_{i=1}^{n-1} \prod_{j=1}^{m_i} y_{i_j} + \prod_{j=1}^{m_n} y_{n_j}
		= \sum_{i=1}^{n} \prod_{j=1}^{m_i} y_{i_j}
		$.
		Hence $f$ is injective. Therefore, $f$ is one-to-one correspondence between $\mathcal{DM}$ and $P(n)$.
	\end{proof}
	
	Finally, from Lemma \ref{lem1} and Lemma \ref{lem2}, Theorem \ref{mainThm} is obviously obtained:
	the set of all DMIE expession of $n$ features one-to-one corresponds to the set of feature graph up to connected component, as we needed.
	
	
	
	\section*{Proof of the MDL for Feature Graphs}
	\subsection*{Proof of Lemma \ref{lemma:error_term}}
	\begin{proof}
		For this proof, unless indicate otherwise, we denote by $\deg(\cdot)$ the degree of node with respect to the graph $G*$.
		Let us denote $\hat{c}_{i}$ and $\hat{c}_{ij}$ parameters to be learned by a function $f_{G*}$.
		As well, we let $\hat{d}_{i}$ and $\hat{d}_{ij}$ for that of $f_{G^*\setminus\{i,j\}}$.
		Note first that $f_{G*}$ contains the term $\hat{c}_{ab}x_a x_b$, but $f_{G^*\setminus\{i,j\}}$ does not.
		Similarly, $f_{G^*\setminus\{i,j\}}$ contains the terms $\hat{d}_ax_a$ and $\hat{d}_bx_b$.
		
		It is obvious that
		$
		\sumdata\LR(y_i - f_{G^*}(x_i))\leq \sumdata\LR(\epsilon_i)\leq\sumdata\LR(\epsilon^*) = n\LR(\epsilon^*).
		$
		Now we claim that $\sumdata \LR(y_i - f_{G^*\setminus\{a,b\}}(x_i)) \geq n\LR(\epsilon^*)$.
		We observe that
		$
		\sumdata\LR\left(\left|y_i - f_{G^*\setminus\{a,b\}}(x_i)\right|\right)
		= \sumdata\LR\left(\left|
		\sum_{\deg(i)=0}(c_i-\hat{d}_i)x_i + \sum_{\{i,j\}\in E(G^*)\setminus ab}(c_{ij}-\hat{d}_{ij})x_ix_j + \epsilon_i +c_{ab}x_ax_b - \hat{d}_ax_a - \hat{d}_bx_b
		\right|\right)
		= \sumdata\LR\left(\left|
		c_{ab}x_ax_b - \hat{d}_ax_a - \hat{d}_bx_b + \epsilon_i
		\right|\right) \geq \sumdata\LR\left(\left|\epsilon_i\right|\right) \geq n\LR(\epsilon^*)$
		
%		\begin{notebox}
%			$ \sumdata\LR\left(\left|
%			c_{ab}x_ax_b - \hat{d}_ax_a - \hat{d}_bx_b + \epsilon_i
%			\right|\right) \geq \sumdata\LR\left(\left|\epsilon_i\right|\right)$\\can use the argument of truncated error (no need to assume)
%		\end{notebox}
		as we desired.
		Therefore, $\sumdata\LR\left(y_i - f_{G^*\setminus\{a,b\}}(x_i)\right) \geq n\LR(\epsilon^*) \geq 	\sumdata\LR(y_i - f_{G^*}(x_i)).$
		By similar argument, for $\{a,b \}\notin E(G^*)$, we obtain that
		$
		\sumdata\LR\left(\left|y_i - f_{G^*\cup\{a,b\}}(x_i)\right|\right) = \sumdata\LR\left(\left|
		c_ax_a + c_bx_b - \hat{d}_{ab}x_ax_b + \epsilon_i
		\right|\right)\geq n\LR(\epsilon^*)
		$
		Therefore, $\sumdata\LR\left(y_i - f_{G^*\cup\{a,b\}}(x_i)\right) \geq n\LR(\epsilon^*) \geq 	\sumdata\LR(y_i - f_{G^*}(x_i)).$
	\end{proof}
	
	\subsection*{Proof of Proposition \ref{prop:add-itr-drop-nonitr-from-good}}
	\begin{proof}
		For the removing interaction edge, by our initial assumptions, we have 
		$
		\sum_{\deg(i)=0}\LR(\hat{c}_i)\approx\sum_{\deg(i)=0}\LR(\hat{d}_i)
		$
		and
		$
		\sum_{\{i,j\}\in E(G*)}\LR(\hat{c}_{ij})\approx \LR(\hat{c}_{ab})+\sum_{\{i,j\}\in E(G*) \setminus \{\{a,b\}\}}\LR(\hat{d}_{ij}).
		$
		It is obvious that the term $\sumdata\LR(x_i)$ does not depend on representation.
		Then we have
		$
		\mathcal{L}(S_{G^*},f_{G^*\setminus\{a,b\}}) - \mathcal{L}(S_{G^*},f_{G^*})
		= \LR(|E(G*)|-1) - \LR(|E(G*)|) + \LR(\hat{d}_a) + \LR(\hat{d}_b) - \LR(\hat{c}_{ab})
		+ \sumdata \LR(y_i - f_{G^*\setminus\{a,b\}}(x_i))-\sumdata \LR(y_i - f_{G^*}(x_i))
		\geq \LR(\hat{d}_a) + \LR(\hat{d}_b) - \LR(\hat{c}_{ab}) - D \geq 0
		$, 
		by what we assume above and Lemma \ref{lemma:error_term}.
		
		Next, we prove the argument about adding a non-interaction edge $\{a,b \}\notin E(G^*)$. Similarly, we have that
		$
		\mathcal{L}(S_{G^*},f_{G^*\cup\{a,b\}}) - \mathcal{L}(S_{G^*},f_{G^*})
		= \LR(|E(G*)|+1) - \LR(|E(G*)|) + \LR(\hat{d}_{ab}) - \LR(\hat{c}_{a}) - \LR(\hat{c}_{b})
		+ \sumdata \LR(y_i - f_{G^*\cap\{a,b\}}(x_i))-\sumdata \LR(y_i - f_{G^*}(x_i))
		\geq D + \LR(\hat{d}_{ab}) - \LR(\hat{c}_{a}) - \LR(\hat{c}_{b}) \geq 0
		$
	\end{proof}

}

\end{document}


