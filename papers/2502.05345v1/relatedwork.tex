\section{Related Work}
\subsubsection{\textbf{Features Extraction}}

The selection of features to be used for IR drop analysis is a very important part of the procedure, as it will define the accuracy and duration of the simulation. The first step for the commercial tool is to generate a netlist of the design under test, which contains information about the components of the circuit, their connections and the power rails of the circuit [1]. Then, the software extracts geometrical information from the layout, including parasitics [1]. The three main areas from which these features are extracted are: power, timing and physical information. Power features may include the target cell's toggle rate ($T_R$), average current drawn by a cell instance ($I_{avg}$) or total power consumption ($P_{total}$), which includes switching, leakage and internal power. Timing features may include the switching timing of the target cell minimum and/or maximum rising/falling time. Finally, physical features are related to actual x,y coordinate of a cell instance on physical layout or their distances from the nearest via.\\ 

In order to create a high-performance and efficient ML model, features to be used need to be carefully selected as input during the training phase. Chun Fang et al. [3] created an ML model to predict IR drop after ECO based on the original circuit and some input features. The input to the machine learning model consisted of 17 inputs consisting power, timing and geometrical information in order to be more accurate during the prediction phase. After defining the area around a cell that they want to test, they extracted features of the target cell, such as \textit{cell type}, \textit{loading capacitance}, \textit{toggle rate} and \textit{peak current} of a cell during switching time. They also imported information from neighboring cells in a form of a density map. Specifically, they selected cell instances which have the same timing window of the target cell, and they constructed density maps of average and peak current, toggle rate and power consumption. The use of a fast ML algorithm XGBoost, resulted in simulation time of 2 minutes, tens of times faster compared to a commercial tool.\\

Another interesting selection of features input to ML model was developed in PowerNet [12]. In this experiment, CNN was used to predict the dynamic vector-based and vectorless IR drop. They formed power maps which included features such as internal, switching, and leakage power and a time power map, which contributed in the model by ensuring that only cells that could switch at the same time were considered in the training of the model. Similar to PowerNet[12], Xian Chen et al. at [14], imported two types of features: \textit{raw features} and \textit{density map features}. The first one is not much different compared to similar studies as they also included features about x,y coordinates of a cell, power and current consumption. However, the \textit{density map features} used here include information about the target cell, as well as their neighboring cells around it.\\

\subsubsection{\textbf{ML models used for IR drop}}
Examples of practical applications can be linear regression, Support vector Machine (SVM),XGBoost, ANN or CNN. An example of the design flow of an ML model for IR prediction can be seen in Figure \ref{fig:ML model for ir drop}. \\


\begin{figure*}[t]
\centering
\includegraphics[width=4.5in]{Figures/ml model.png}
\caption{Indicative flow of ML model for IR drop prediction}
\label{fig:ML_model_for_ir_drop}
\end{figure*}