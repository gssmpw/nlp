% CVPR 2025 Paper Template; see https://github.com/cvpr-org/author-kit

\documentclass[10pt,twocolumn,letterpaper]{article}

%%%%%%%%% PAPER TYPE  - PLEASE UPDATE FOR FINAL VERSION
\usepackage{cvpr}              % To produce the CAMERA-READY version
% \usepackage[review]{cvpr}      % To produce the REVIEW version
% \usepackage[pagenumbers]{cvpr} % To force page numbers, e.g. for an arXiv version

% Import additional packages in the preamble file, before hyperref
%
% --- inline annotations
%
\newcommand{\red}[1]{{\color{red}#1}}
\newcommand{\todo}[1]{{\color{red}#1}}
\newcommand{\TODO}[1]{\textbf{\color{red}[TODO: #1]}}
% --- disable by uncommenting  
% \renewcommand{\TODO}[1]{}
% \renewcommand{\todo}[1]{#1}



\newcommand{\VLM}{LVLM\xspace} 
\newcommand{\ours}{PeKit\xspace}
\newcommand{\yollava}{Yo’LLaVA\xspace}

\newcommand{\thisismy}{This-Is-My-Img\xspace}
\newcommand{\myparagraph}[1]{\noindent\textbf{#1}}
\newcommand{\vdoro}[1]{{\color[rgb]{0.4, 0.18, 0.78} {[V] #1}}}
% --- disable by uncommenting  
% \renewcommand{\TODO}[1]{}
% \renewcommand{\todo}[1]{#1}
\usepackage{slashbox}
% Vectors
\newcommand{\bB}{\mathcal{B}}
\newcommand{\bw}{\mathbf{w}}
\newcommand{\bs}{\mathbf{s}}
\newcommand{\bo}{\mathbf{o}}
\newcommand{\bn}{\mathbf{n}}
\newcommand{\bc}{\mathbf{c}}
\newcommand{\bp}{\mathbf{p}}
\newcommand{\bS}{\mathbf{S}}
\newcommand{\bk}{\mathbf{k}}
\newcommand{\bmu}{\boldsymbol{\mu}}
\newcommand{\bx}{\mathbf{x}}
\newcommand{\bg}{\mathbf{g}}
\newcommand{\be}{\mathbf{e}}
\newcommand{\bX}{\mathbf{X}}
\newcommand{\by}{\mathbf{y}}
\newcommand{\bv}{\mathbf{v}}
\newcommand{\bz}{\mathbf{z}}
\newcommand{\bq}{\mathbf{q}}
\newcommand{\bff}{\mathbf{f}}
\newcommand{\bu}{\mathbf{u}}
\newcommand{\bh}{\mathbf{h}}
\newcommand{\bb}{\mathbf{b}}

\newcommand{\rone}{\textcolor{green}{R1}}
\newcommand{\rtwo}{\textcolor{orange}{R2}}
\newcommand{\rthree}{\textcolor{red}{R3}}
\usepackage{amsmath}
%\usepackage{arydshln}
\DeclareMathOperator{\similarity}{sim}
\DeclareMathOperator{\AvgPool}{AvgPool}

\newcommand{\argmax}{\mathop{\mathrm{argmax}}}     



\usepackage{amsmath,amssymb}

% コード表示用
\usepackage{listings}
\usepackage{xcolor}
\lstset{
  language=Python,
  basicstyle=\ttfamily\small,
  keywordstyle=\color{blue},
  commentstyle=\color{gray},
  stringstyle=\color{red},
  showstringspaces=false,
  breaklines=true,
  frame=single,
}

% 箇条書きの設定
\usepackage{enumitem}


% It is strongly recommended to use hyperref, especially for the review version.
% hyperref with option pagebackref eases the reviewers' job.
% Please disable hyperref *only* if you encounter grave issues, 
% e.g. with the file validation for the camera-ready version.
%
% If you comment hyperref and then uncomment it, you should delete *.aux before re-running LaTeX.
% (Or just hit 'q' on the first LaTeX run, let it finish, and you should be clear).
\definecolor{cvprblue}{rgb}{0.21,0.49,0.74}
\usepackage[pagebackref,breaklinks,colorlinks,allcolors=cvprblue]{hyperref}

%%%%%%%%% PAPER ID  - PLEASE UPDATE
\def\paperID{*****} % *** Enter the Paper ID here
\def\confName{CVPR}
\def\confYear{2025}

%%%%%%%%% TITLE - PLEASE UPDATE
\title{2.5D U-Net with Depth Reduction for 3D CryoET Object Identification}

%%%%%%%%% AUTHORS - PLEASE UPDATE
\author{Yusuke Uchida\\
GO Inc.\\
Tokyo, Japan\\
%{\tt\small firstauthor@i1.org}
% For a paper whose authors are all at the same institution,
% omit the following lines up until the closing ``}''.
% Additional authors and addresses can be added with ``\and'',
% just like the second author.
% To save space, use either the email address or home page, not both
\and
Takaaki Fukui\\
GO Inc.\\
Tokyo, Japan\\
%{\tt\small secondauthor@i2.org}
}

\begin{document}
\maketitle
\begin{abstract}


The choice of representation for geographic location significantly impacts the accuracy of models for a broad range of geospatial tasks, including fine-grained species classification, population density estimation, and biome classification. Recent works like SatCLIP and GeoCLIP learn such representations by contrastively aligning geolocation with co-located images. While these methods work exceptionally well, in this paper, we posit that the current training strategies fail to fully capture the important visual features. We provide an information theoretic perspective on why the resulting embeddings from these methods discard crucial visual information that is important for many downstream tasks. To solve this problem, we propose a novel retrieval-augmented strategy called RANGE. We build our method on the intuition that the visual features of a location can be estimated by combining the visual features from multiple similar-looking locations. We evaluate our method across a wide variety of tasks. Our results show that RANGE outperforms the existing state-of-the-art models with significant margins in most tasks. We show gains of up to 13.1\% on classification tasks and 0.145 $R^2$ on regression tasks. All our code and models will be made available at: \href{https://github.com/mvrl/RANGE}{https://github.com/mvrl/RANGE}.

\end{abstract}

    
\section{Introduction}
Backdoor attacks pose a concealed yet profound security risk to machine learning (ML) models, for which the adversaries can inject a stealth backdoor into the model during training, enabling them to illicitly control the model's output upon encountering predefined inputs. These attacks can even occur without the knowledge of developers or end-users, thereby undermining the trust in ML systems. As ML becomes more deeply embedded in critical sectors like finance, healthcare, and autonomous driving \citep{he2016deep, liu2020computing, tournier2019mrtrix3, adjabi2020past}, the potential damage from backdoor attacks grows, underscoring the emergency for developing robust defense mechanisms against backdoor attacks.

To address the threat of backdoor attacks, researchers have developed a variety of strategies \cite{liu2018fine,wu2021adversarial,wang2019neural,zeng2022adversarial,zhu2023neural,Zhu_2023_ICCV, wei2024shared,wei2024d3}, aimed at purifying backdoors within victim models. These methods are designed to integrate with current deployment workflows seamlessly and have demonstrated significant success in mitigating the effects of backdoor triggers \cite{wubackdoorbench, wu2023defenses, wu2024backdoorbench,dunnett2024countering}.  However, most state-of-the-art (SOTA) backdoor purification methods operate under the assumption that a small clean dataset, often referred to as \textbf{auxiliary dataset}, is available for purification. Such an assumption poses practical challenges, especially in scenarios where data is scarce. To tackle this challenge, efforts have been made to reduce the size of the required auxiliary dataset~\cite{chai2022oneshot,li2023reconstructive, Zhu_2023_ICCV} and even explore dataset-free purification techniques~\cite{zheng2022data,hong2023revisiting,lin2024fusing}. Although these approaches offer some improvements, recent evaluations \cite{dunnett2024countering, wu2024backdoorbench} continue to highlight the importance of sufficient auxiliary data for achieving robust defenses against backdoor attacks.

While significant progress has been made in reducing the size of auxiliary datasets, an equally critical yet underexplored question remains: \emph{how does the nature of the auxiliary dataset affect purification effectiveness?} In  real-world  applications, auxiliary datasets can vary widely, encompassing in-distribution data, synthetic data, or external data from different sources. Understanding how each type of auxiliary dataset influences the purification effectiveness is vital for selecting or constructing the most suitable auxiliary dataset and the corresponding technique. For instance, when multiple datasets are available, understanding how different datasets contribute to purification can guide defenders in selecting or crafting the most appropriate dataset. Conversely, when only limited auxiliary data is accessible, knowing which purification technique works best under those constraints is critical. Therefore, there is an urgent need for a thorough investigation into the impact of auxiliary datasets on purification effectiveness to guide defenders in  enhancing the security of ML systems. 

In this paper, we systematically investigate the critical role of auxiliary datasets in backdoor purification, aiming to bridge the gap between idealized and practical purification scenarios.  Specifically, we first construct a diverse set of auxiliary datasets to emulate real-world conditions, as summarized in Table~\ref{overall}. These datasets include in-distribution data, synthetic data, and external data from other sources. Through an evaluation of SOTA backdoor purification methods across these datasets, we uncover several critical insights: \textbf{1)} In-distribution datasets, particularly those carefully filtered from the original training data of the victim model, effectively preserve the model’s utility for its intended tasks but may fall short in eliminating backdoors. \textbf{2)} Incorporating OOD datasets can help the model forget backdoors but also bring the risk of forgetting critical learned knowledge, significantly degrading its overall performance. Building on these findings, we propose Guided Input Calibration (GIC), a novel technique that enhances backdoor purification by adaptively transforming auxiliary data to better align with the victim model’s learned representations. By leveraging the victim model itself to guide this transformation, GIC optimizes the purification process, striking a balance between preserving model utility and mitigating backdoor threats. Extensive experiments demonstrate that GIC significantly improves the effectiveness of backdoor purification across diverse auxiliary datasets, providing a practical and robust defense solution.

Our main contributions are threefold:
\textbf{1) Impact analysis of auxiliary datasets:} We take the \textbf{first step}  in systematically investigating how different types of auxiliary datasets influence backdoor purification effectiveness. Our findings provide novel insights and serve as a foundation for future research on optimizing dataset selection and construction for enhanced backdoor defense.
%
\textbf{2) Compilation and evaluation of diverse auxiliary datasets:}  We have compiled and rigorously evaluated a diverse set of auxiliary datasets using SOTA purification methods, making our datasets and code publicly available to facilitate and support future research on practical backdoor defense strategies.
%
\textbf{3) Introduction of GIC:} We introduce GIC, the \textbf{first} dedicated solution designed to align auxiliary datasets with the model’s learned representations, significantly enhancing backdoor mitigation across various dataset types. Our approach sets a new benchmark for practical and effective backdoor defense.



\section{Proposed Approach\label{sec:proposed}}

\subsection{Insights from Existing Designs\label{ssec:IED}}
We first discuss some insights from analyzing some existing open-source designs. 
We investigated 22 open-source cryptography IPs, 13 Peripheral Interface IPs, and 12 GPIO IPs (a subset of the family types that are defined in~\cite{accellera}) to gain insights into their design characteristics and commonalities, with a particular focus on 
recurring textual features that can be leveraged to develop an automated potential primary asset detection tool. 
These designs were collected from several sources\footnote{%
To support the community's efforts in this work, we make the details of our identified asset list and the associated IPs available here: \url{https://github.com/CalgaryISH/Asset_Dataset_using_PKG}
}, including GitHub, OpenCores, and OpenTitan, some of which have been used in prior work (e.g.,~\cite{Ahmad_2022}).
We manually examined design documentation and source code to get an intuitive sense of commonalities across IPs in each family. 

We applied the manual CSA method~\cite{ieee_p3164_working_group_asset_2024} to identify potential security assets. We also observed that most open-source designs involved ``sensible'' names for elements in the design (e.g., signals, sub-modules), and that frequently occurring signal/variable names in a given IP family provide indicators of ``important'' elements that point to assets. 

From this manual analysis of different \ac{IP} families and their source code, we curated a list of ``partial keywords'' for each \ac{IP} family that was associated with security-relevant Inputs, Outputs, and Reg/Wire/Logic(Net) signals, and we excluded ``typical'' names or tokens that were usually irrelevant to security, to increase the robustness of ``important RTL elements'' detection in the matching stage of our tool. 
This extends the similar concept of ``keyword-matching'' described in prior work~\cite{Ahmad_2022}.

For example, ``en'' represents the ``Enable'' partial keyword group ($PKG_{enable}$), which can be usable to detect signals named ``write\_en'', ``write\_enable'', ``wen'', ``gpio\_oen'', ``gpio\_out\_ena'' etc.
We sometimes added ``partial keywords'' that were especially meaningful for a given IP family even though they were not the most frequently appearing based on our manual code analysis. 
For instance, we added a partial keyword ``text'' in the list for Crypto to detect signals like ``plain\_text'', ``text\_in'', and ``text\_out'', which are important but not very common. 
\autoref{fig:explanations_of_general_image} illustrates the frequency with which the elements of our list of ``partial keywords'' appear throughout all files in an IP class. 
We also considered the following factors in the curation of partial keywords.

\subsubsection{Location and Frequency}
The location of the signals is crucial to understanding their nature. Potential important signal names often appear in logical/conditional expressions, assignments in sequential blocks, and combinational blocks. 
The frequency depends on the file size, but frequent signal names should be considered for further processing. These can be potential candidates if they frequently appear in these aforementioned locations.


\subsubsection{Keywords and Tokens of the Language}
Every language has reserved keywords, tokens, and constant names, which are special identifiers reserved to define the language constructs (e.g., \textit{wire}). We ignored everything that completely matched with these special ``words''.


\subsubsection{Types and Signal Width} We considered Input, Output, and internal Net-type signals in this study. Also, for signal width, we considered three categories: 1-bit, 2-bit to 8-bit, and larger signals up to 256-bit. A partial keyword group with a similar signal type(Input, Output, Net) and width is more desirable as it reduces the rule formulation complexities for asset classification.


\subsubsection{Overlap in Names} Due to naming conventions and functionalities of the designs, we observed some partial and full overlap in the signal names for different projects in the same IP class. As we discussed before, we used this overlapping portion of the name to create a partial keyword group like ($PKG_{enable}$). However, signal type and width should be taken into consideration as well.


\subsubsection{Commonalities in Roles} In some cases, we could not find any similarities in signal names. In that case, we created a partial keyword group according to the signal type, width, and role in the design. In this case, we used our design knowledge and ideas from the behavioral patterns. A partial keyword group for the status signal can be a good example. We have mentioned the most common status signals in \autoref{sssec:status}.


\subsection{Behavioral Patterns\label{ssec:BP}}

As simple name matching is insufficient to identify potential assets, we also manually examined how the potential candidate signals identified ``behave'' and ``function'' inside a design (e.g., if they often appear in \texttt{if-else}, \texttt{always} or \texttt{assign} statements, what are the ``width'' of the signals, in which operations they are associated to) to classify a set of signals' behavior patterns. 
Our approach does not solely depend on the keywords. 
Keywords help to identify potential candidate signals for the behavioral pattern detection stage. The behavioral pattern, including the attributes, functionalities, and location of a signal, indicates the overall structure, which is crucial to detecting structural assets~\cite{ieee_p3164_working_group_asset_2024}.

We observed four common signal behavioral patterns in the open-source hardware designs. 
Based on our manual analysis, attributes of the signals, and how they function in a design, we developed an algorithm denoted by \autoref{alg:one} to detect all four types of signals. 
\autoref{fig:code-example} provides an example of Verilog code that we use next as a running example to explain behavioral patterns. 
Each type of signal behavior can be described as relevant to confidentiality, availability, and/or integrity security properties. 

\subsubsection{Control Signals} are typically single-bit wide input signals or nets/variables (assigned or instantiated with input port in the module) that appear inside the conditional expression of \texttt{if-else} blocks, \texttt{case} blocks, and \texttt{ternary operation} statements. 
They are responsible for enabling/disabling functionality, controlling data flows and value assignment, and are often used to clear or load information from or into a memory register. Control signals are responsible for managing and controlling one or more blocking or non-blocking assignment statements inside \texttt{if-else} block, \texttt{case} block, and \texttt{ternary operation}. 
A control signal of a module can be connected with a status signal from a different module through instantiation when an interdependent sequential process occurs between two modules.
The \textit{Availability} security objective is commonly relevant to the Control signals. 
In the example (\autoref{fig:code-example}), \texttt{load} is a 1-bit input signal and controls the 128-bit data loading to \texttt{data\_in\_reg} register on line number 15. If the \texttt{load} becomes 0, no data will be loaded to the \texttt{data\_in\_reg} register. 


\subsubsection{Configuration Signals} are typically 2-bit to a few bits wide (in most cases, up to 8-bit) input signals or nets/variables (assigned or instantiated with input port in the module) that appear mostly in \texttt{case} expression, \texttt{ternary operation}, and conditional expression in multi-statements containing \texttt{if-else} blocks. 
These signals configure the operational flow of a module, data read and write direction, data splitting and loading into multiple memory registers or clear from the memory registers, select multiplexer's outputs, and control state transitions. 
\textit{Availability} and \textit{Integrity} security objectives are usually associated with the Configuration signals.
In the example, the 2-bit \texttt{bank\_selector} is a Configuration signal. According to code line 18, \texttt{bank\_selector} works as a Multiplexer output selector in the \texttt{case} expression for loading the split data into four memory banks.


\begin{figure}
\centering
\begin{minipage}{0.9\columnwidth}
\begin{minted}[breaklines=true,fontsize=\scriptsize, linenos=true]{verilog}
module data_splitter (
    input clk,
    input load,
    input [1:0] bank_selector,
    input [127:0] data,
    output reg [31:0] bank0,
    output reg [31:0] bank1,
    output reg [31:0] bank2,
    output reg [31:0] bank3,
    output reg done
);
  reg [127:0] data_in_reg;
  reg done0, done1, done2, done3;
  always @(posedge clk) begin
    if (load) data_in_reg <= data;
  end
  always @(data_in_reg or bank_selector) begin
    case (bank_selector)
      2'b00: begin
        bank0 <= data_in_reg[31:0];
        done0 <= 1'b1;
      end
      2'b01: begin
        bank1 <= data_in_reg[63:32];
        done1 <= 1'b1;
      end
      2'b10: begin
        bank2 <= data_in_reg[95:64];
        done2 <= 1'b1;
      end
      2'b11: begin
        bank3 <= data_in_reg[127:96];
        done3 <= 1'b1;
      end
      default: begin
      end
    endcase
  end
  always @(posedge clk) begin
    if (done0 && done1 && done2 && done3) 
        done <= 1'b1;
    else done <= 1'b0;
  end
endmodule
\end{minted}
\end{minipage}

\caption{A simple 128-bit to four banks of 32-bit data splitter code example.}
\label{fig:code-example}

\end{figure}

\begin{algorithm}[t!]
\footnotesize

\caption{Algorithm to detect different behavioral patterns in signals}\label{alg:one}

\KwData{RTL Code Written in Verilog or SystemVerilog, input\_ports[\ ], output\_ports[\ ], net\_variables[\ ];}
\KwResult{Control\_Signals[\ ], Configuration\_Signals[\ ], Status\_Signals[\ ], Data\_Signals[\ ];}
\vspace{1mm}
\For{$line\leftarrow 1$ \KwTo max\_line\_number of RTL\_Code} {

\If{$line\ contains\ a\ conditional\ expression$} {
\tcc{$x$ is the signal name inside the conditional expression}
\If{Width of $x ==\ 1bit$}{

\If{$x \in input\_ports[\ ]\ ||\ net\_variables[\ ] $} {append $x$ to Control\_Signals[\ ];}




}
\ElseIf{Width of $x \geq 2bit\ \&\&$ The conditional block contains multiple statements }{

\If{$x \in input\_ports[\ ]\ ||\ net\_variables[\ ] $} {append $x$ to Configuration\_Signals[\ ];}



}

}

\ElseIf{$line\ contains\ a\ blocking\ or\ non-blocking\ assignment$}{
\tcc{$l$ and $r$ are the signal names that appear on the left-hand side and right-hand side of an assignment statement, respectively}
\If{Width of $l ==\ 1bit\ \&\&\ l \in output\_ports[\ ] $}{ 
{append $l$ to Status\_Signals[\ ];}
}
\ElseIf{Width of $l \geq 2bit\ \&\&\ l \in output\_ports[\ ]$}{append $l$ to Data\_Signals[\ ];}
\ElseIf{Width of $r \geq 2bit\ \&\&\ r \in input\_ports[\ ]$}{append $r$ to Data\_Signals[\ ];}

}
\Else{do nothing;} 
}

\end{algorithm}



\subsubsection{Status Signals\label{sssec:status}} are typically single bit-width output signals or nets/variables (assigned or instantiated with output port in the module) that appear on the left-hand side of blocking/non-blocking assignments, statements under conditional code segments (e.g., \texttt{if-else} blocks, \texttt{case} blocks, and \texttt{ternary operation}). 
Usually, a status signal lets other modules know the status of a process or operation from the module to which it belongs. 
Common Status signals include \texttt{finish}, \texttt{done}, \texttt{ready}, \texttt{success}, \texttt{alert}, and \texttt{error}. 
If a status signal is connected with a control signal of another module, the \textit{Availability} objective is relevant; otherwise, \textit{Integrity} security objective is commonly relevant to the Status signals.
In the example, \texttt{done} is a Status signal. From code lines 40 to 42, the \texttt{if-else} block assigns the value of \texttt{done} signal based on \texttt{done0}, \texttt{done1}, \texttt{done2}, and \texttt{done3} signals. 

\subsubsection{Data Signals} can be inputs or outputs with a multi-bit width for assigning storage addresses, memory registers, information for processing, and processed information in a module.
Data signals propagate data and are processed in multiple modules in an \ac{IP}.
There is another kind of Data signal that does not get changed or processed during an operation but is involved in security-critical operations like encryption and decryption. ``Seed'' and ``Key'' are typical examples of this Data signal. 
Data signals appear in both blocking and non-blocking assignment-type statements in a module.
Data signals containing critical information are directly linked to \textit{Confidentiality}.
In our example, the 128-bit \texttt{data} and 32-bit \texttt{bank0}, \texttt{bank1}, \texttt{bank2}, and \texttt{bank3} are responsible for storing and carrying information; these all Data-type behavioral pattern.
The \texttt{data} may contain sensitive information like \textit{key} for a key-expansion operation in an encryption or decryption IP, or it may contain plain user information that needs to be protected equally. 
In line 15 of the \autoref{fig:code-example}, \texttt{data} appears on the right-hand side of the non-blocking assignment.
Conversely, all four banks appear on the left-hand side of the non-blocking assignment on lines 20, 24, 28, and 32, respectively. 
Such information, like which side of the assignment the signal appears, can be useful for automated asset identification. 

\begin{figure*}[t]
    \centering
    \includegraphics[width=\linewidth]{fig/overview-fig-crop.pdf}
    \caption{Overall view of our proposed automatic potential asset detection algorithm.}
    \label{fig:overview-figure}
\end{figure*}


\subsection{Automation}
In this section, we build on previously discussed observations and explain our proposed five-stage process for automated potential primary asset identification, as illustrated in \autoref{fig:overview-figure}. 

\subsubsection{Extraction Stage}
This stage uses RTL source files in Verilog or SystemVerilog. The tool iterates through all the available files in an \ac{IP}'s directory, extracting all the Input and Output Ports and Wire/Reg/Logic Nets. The tool stores these extracted ``RTL Elements'' in lists for the next stage. 

\subsubsection{Matching Stage}
In this stage, the tool finds partial matches between the RTL Elements and the IP-family-specific partial keyword lists to detect important signals that have the most potential to be assets. 
This stage outputs the subset of elements that are likely to be important from a security-perspective. 
Further pruning in the next stages narrows the potential candidate asset list. 
After this stage, the tool prepares a list of important\_rtl\_elements[ ] (reduced version of rtl\_elements[ ]) for the next stage.

\subsubsection{Width and Behavioral Pattern Detection Stage}
The matching step can sometimes result in 80\% of the RTL elements matching due to naming and multiple matches with partial keywords. 
For example, a signal \texttt{key\_rounding\_enable} in a Cryptographic \ac{IP} has three partial matches with keywords \texttt{en}, \texttt{round}, and \texttt{key}. 
The tool will detect \texttt{key\_rounding\_enable} for three partial keywords as a candidate for Important RTL Elements. 
However, not all of these should be considered structural assets (as explained in the IEEE P3164 white paper~\cite{ieee_p3164_working_group_asset_2024}). 
For this particular example, if the signal's width is single-bit and it is used to control any functionality or operation in the design (i.e., it has a ``control signal'' behavioral pattern), it can be a potential asset. 
Again, if the signal does not have the control pattern, just the partial keyword matching, it might not be a potential asset.
Therefore, we add another step to detect the signal width and behavioral patterns, as we previously described in \autoref{ssec:BP}. This will help the tool create lists of control\_signal[ ], config\_signal[ ], status\_signal[ ], and data\_signal[ ]. 

\subsubsection{IP-Specific Asset Classification and Filtering Stage}
Based on our manual analysis of the open-source designs, we designed a set of IP-family-specific classification rules, which narrow down the potential asset list. For example, for the crypto family, we consider a signal with the partial keyword \texttt{key} as the encryption/decryption key if it is a vectored signal (in most cases, the width starts from 64-bit) and has a ``Data Signal'' related behavioral pattern and the stored value needed for any encrypt/decrypt operation. 
In contrast, in the GPIO IP family, the partial keyword \texttt{data} helps to identify \textit{rdata}, \textit{wdata}, and just \textit{port\_data} or \textit{pad\_data} sometimes. The width for the GPIO port varies from 8-bit to the maximum bit at which the system operates (for instance, 32-bit for an ARM Cortex-M0 Microcontroller).
These all have \texttt{data} related behavioral patterns and store important values that are read/written to a register. 
In both cases, \texttt{key} and \texttt{data} can store potentially secret information that must be protected. Still, the detection and classification methods differ based on partial keyword group, width, signal types, and signal behavioral patterns. 
The filtered list is the set of candidate potential assets. 

\subsubsection{Refinement Stage\label{sssec:RS}}
Here, the tool identifies the ``root'' (original sources) of a candidate asset from the Input and Output ports of the TOP module. Any candidate potential asset that is related to the I/O Ports of the TOP module(for a complete IP) or in an important module (where the TOP module is not present) can be considered as a Potential Primary Asset.
At the beginning of this stage, the tool checks a candidate potential asset for its type (Input, Output, Net), behavioral patterns, and module information. Several situations can occur, as follows. 

\paragraph*{Case 1} If the candidate belongs to the Input or Output port of the TOP module, the tool appends it to the potential primary asset list.

\paragraph*{Case 2} If it does not belong to the TOP module, the tool finds the candidate's connection with the Input and output ports of the TOP module by traversing through the instantiations throughout the IP. Then, the tool appends the connected port from the TOP module to the potential primary asset list. If the tool does not find any interconnection between the candidate and the I/O ports of the TOP module, (1) the tool ignores the candidate (maybe a Secondary asset) when the candidate-containing module is instantiated inside the TOP module directly or indirectly through another submodule, or (2) the tool considers the candidate a primary asset when the candidate-containing module is not instantiated inside the TOP module directly or indirectly through another submodule.

\paragraph*{Case 3} If the candidate is a Net-type signal, then the tool identifies the assignments and connections through the instantiations related to the module to which the candidate belongs. The tool detects the candidate's connection(if any) with I/O ports. If the tool can detect any connection between the candidates and I/O ports throughout the IP, the tool repeats \textit{Case 1} and \textit{Case 2}.

After completing the process for each candidate, the tool removes duplicate potential primary assets that can be added multiple times due to the relationship with multiple candidates from multiple modules.
After this stage, the tool lists the final potential primary assets for an \ac{IP} as output.





{
    %\small
    \footnotesize
    %\bibliographystyle{ieeenat_fullname}
    \bibliographystyle{ieee}
    \bibliography{main}
}


% WARNING: do not forget to delete the supplementary pages from your submission 
% \clearpage
\pagenumbering{gobble}
\maketitlesupplementary

\section{Additional Results on Embodied Tasks}

To evaluate the broader applicability of our EgoAgent's learned representation beyond video-conditioned 3D human motion prediction, we test its ability to improve visual policy learning for embodiments other than the human skeleton.
Following the methodology in~\cite{majumdar2023we}, we conduct experiments on the TriFinger benchmark~\cite{wuthrich2020trifinger}, which involves a three-finger robot performing two tasks: reach cube and move cube. 
We freeze the pretrained representations and use a 3-layer MLP as the policy network, training each task with 100 demonstrations.

\begin{table}[h]
\centering
\caption{Success rate (\%) on the TriFinger benchmark, where each model's pretrained representation is fixed, and additional linear layers are trained as the policy network.}
\label{tab:trifinger}
\resizebox{\linewidth}{!}{%
\begin{tabular}{llcc}
\toprule
Methods       & Training Dataset & Reach Cube & Move Cube \\
\midrule
DINO~\cite{caron2021emerging}         & WT Venice        & 78.03     & 47.42     \\
DoRA~\cite{venkataramanan2023imagenet}          & WT Venice        & 81.62     & 53.76     \\
DoRA~\cite{venkataramanan2023imagenet}          & WT All           & 82.40     & 48.13     \\
\midrule
EgoAgent-300M & WT+Ego-Exo4D      & 82.61    & 54.21      \\
EgoAgent-1B   & WT+Ego-Exo4D      & \textbf{85.72}      & \textbf{57.66}   \\
\bottomrule
\end{tabular}%
}
\end{table}

As shown in Table~\ref{tab:trifinger}, EgoAgent achieves the highest success rates on both tasks, outperforming the best models from DoRA~\cite{venkataramanan2023imagenet} with increases of +3.32\% and +3.9\% respectively.
This result shows that by incorporating human action prediction into the learning process, EgoAgent demonstrates the ability to learn more effective representations that benefit both image classification and embodied manipulation tasks.
This highlights the potential of leveraging human-centric motion data to bridge the gap between visual understanding and actionable policy learning.



\section{Additional Results on Egocentric Future State Prediction}

In this section, we provide additional qualitative results on the egocentric future state prediction task. Additionally, we describe our approach to finetune video diffusion model on the Ego-Exo4D dataset~\cite{grauman2024ego} and generate future video frames conditioned on initial frames as shown in Figure~\ref{fig:opensora_finetune}.

\begin{figure}[b]
    \centering
    \includegraphics[width=\linewidth]{figures/opensora_finetune.pdf}
    \caption{Comparison of OpenSora V1.1 first-frame-conditioned video generation results before and after finetuning on Ego-Exo4D. Fine-tuning enhances temporal consistency, but the predicted pixel-space future states still exhibit errors, such as inaccuracies in the basketball's trajectory.}
    \label{fig:opensora_finetune}
\end{figure}

\subsection{Visualizations and Comparisons}

More visualizations of our method, DoRA, and OpenSora in different scenes (as shown in Figure~\ref{fig:supp pred}). For OpenSora, when predicting the states of $t_k$, we use all the ground truth frames from $t_{0}$ to $t_{k-1}$ as conditions. As OpenSora takes only past observations as input and neglects human motion, it performs well only when the human has relatively small motions (see top cases in Figure~\ref{fig:supp pred}), but can not adjust to large movements of the human body or quick viewpoint changes (see bottom cases in Figure~\ref{fig:supp pred}).

\begin{figure*}
    \centering
    \includegraphics[width=\linewidth]{figures/supp_pred.pdf}
    \caption{Retrieval and generation results for egocentric future state prediction. Correct and wrong retrieval images are marked with green and red boundaries, respectively.}
    \label{fig:supp pred}
\end{figure*}

\begin{figure*}[t]
    \centering
    \includegraphics[width=0.9\linewidth]{figures/motion_prediction.pdf}
    \vspace{-0.5mm}
    \caption{Motion prediction results in scenes with minor changes in observation.}
    \vspace{-1.5mm}
    \label{fig:motion_prediction}
\end{figure*}

\subsection{Finetuning OpenSora on Ego-Exo4D}

OpenSora V1.1~\cite{opensora}, initially trained on internet videos and images, produces severely inconsistent results when directly applied to infer future videos on the Ego-Exo4D dataset, as illustrated in Figure~\ref{fig:opensora_finetune}.
To address the gap between general internet content and egocentric video data, we fine-tune the official checkpoint on the Ego-Exo4D training set for 50 epochs.
OpenSora V1.1 proposed a random mask strategy during training to enable video generation by image and video conditioning. We adopted the default masking rate, which applies: 75\% with no masking, 2.5\% with random masking of 1 frame to 1/4 of the total frames, 2.5\% with masking at either the beginning or the end for 1 frame to 1/4 of the total frames, and 5\% with random masking spanning 1 frame to 1/4 of the total frames at both the beginning and the end.

As shown in Fig.~\ref{fig:opensora_finetune}, despite being trained on a large dataset, OpenSora struggles to generalize to the Ego-Exo4D dataset, producing future video frames with minimal consistency relative to the conditioning frame. While fine-tuning improves temporal consistency, the moving trajectories of objects like the basketball and soccer ball still deviate from realistic physical laws. Compared with our feature space prediction results, this suggests that training world models in a reconstructive latent space is more challenging than training them in a feature space.


\section{Additional Results on 3D Human Motion Prediction}

We present additional qualitative results for the 3D human motion prediction task, highlighting a particularly challenging scenario where egocentric observations exhibit minimal variation. This scenario poses significant difficulties for video-conditioned motion prediction, as the model must effectively capture and interpret subtle changes. As demonstrated in Fig.~\ref{fig:motion_prediction}, EgoAgent successfully generates accurate predictions that closely align with the ground truth motion, showcasing its ability to handle fine-grained temporal dynamics and nuanced contextual cues.

\section{OpenSora for Image Classification}

In this section, we detail the process of extracting features from OpenSora V1.1~\cite{opensora} (without fine-tuning) for an image classification task. Following the approach of~\cite{xiang2023denoising}, we leverage the insight that diffusion models can be interpreted as multi-level denoising autoencoders. These models inherently learn linearly separable representations within their intermediate layers, without relying on auxiliary encoders. The quality of the extracted features depends on both the layer depth and the noise level applied during extraction.


\begin{table}[h]
\centering
\caption{$k$-NN evaluation results of OpenSora V1.1 features from different layer depths and noising scales on ImageNet-100. Top1 and Top5 accuracy (\%) are reported.}
\label{tab:opensora-knn}
\resizebox{0.95\linewidth}{!}{%
\begin{tabular}{lcccccc}
\toprule
\multirow{2}{*}{Timesteps} & \multicolumn{2}{c}{First Layer} & \multicolumn{2}{c}{Middle Layer} & \multicolumn{2}{c}{Last Layer} \\
\cmidrule(r){2-3}   \cmidrule(r){4-5}  \cmidrule(r){6-7}  & Top1           & Top5           & Top1            & Top5           & Top1           & Top5          \\
\midrule
32        &  6.10           & 18.20             & 34.04               & 59.50             & 30.40             & 55.74             \\
64        & 6.12              & 18.48              & 36.04               & 61.84              & 31.80         & 57.06         \\
128       & 5.84             & 18.14             & 38.08               & 64.16              & 33.44       & 58.42 \\
256       & 5.60             & 16.58              & 30.34               & 56.38              &28.14          & 52.32        \\
512       & 3.66              & 11.70            & 6.24              & 17.62              & 7.24              & 19.44  \\ 
\bottomrule
\end{tabular}%
}
\end{table}

As shown in Table~\ref{tab:opensora-knn}, we first evaluate $k$-NN classification performance on the ImageNet-100 dataset using three intermediate layers and five different noise scales. We find that a noise timestep of 128 yields the best results, with the middle and last layers performing significantly better than the first layer.
We then test this optimal configuration on ImageNet-1K and find that the last layer with 128 noising timesteps achieves the best classification accuracy.

\section{Data Preprocess}
For egocentric video sequences, we utilize videos from the Ego-Exo4D~\cite{grauman2024ego} and WT~\cite{venkataramanan2023imagenet} datasets.
The original resolution of Ego-Exo4D videos is 1408×1408, captured at 30 fps. We sample one frame every five frames and use the original resolution to crop local views (224×224) for computing the self-supervised representation loss. For computing the prediction and action loss, the videos are downsampled to 224×224 resolution.
WT primarily consists of 4K videos (3840×2160) recorded at 60 or 30 fps. Similar to Ego-Exo4D, we use the original resolution and downsample the frame rate to 6 fps for representation loss computation.
As Ego-Exo4D employs fisheye cameras, we undistort the images to a pinhole camera model using the official Project Aria Tools to align them with the WT videos.

For motion sequences, the Ego-Exo4D dataset provides synchronized 3D motion annotations and camera extrinsic parameters for various tasks and scenes. While some annotations are manually labeled, others are automatically generated using 3D motion estimation algorithms from multiple exocentric views. To maximize data utility and maintain high-quality annotations, manual labels are prioritized wherever available, and automated annotations are used only when manual labels are absent.
Each pose is converted into the egocentric camera's coordinate system using transformation matrices derived from the camera extrinsics. These matrices also enable the computation of trajectory vectors for each frame in a sequence. Beyond the x, y, z coordinates, a visibility dimension is appended to account for keypoints invisible to all exocentric views. Finally, a sliding window approach segments sequences into fixed-size windows to serve as input for the model. Note that we do not downsample the frame rate of 3D motions.

\section{Training Details}
\subsection{Architecture Configurations}
In Table~\ref{tab:arch}, we provide detailed architecture configurations for EgoAgent following the scaling-up strategy of InternLM~\cite{team2023internlm}. To ensure the generalization, we do not modify the internal modules in InternML, \emph{i.e.}, we adopt the RMSNorm and 1D RoPE. We show that, without specific modules designed for vision tasks, EgoAgent can perform well on vision and action tasks.

\begin{table}[ht]
  \centering
  \caption{Architecture configurations of EgoAgent.}
  \resizebox{0.8\linewidth}{!}{%
    \begin{tabular}{lcc}
    \toprule
          & EgoAgent-300M & EgoAgent-1B \\
          \midrule
    Depth & 22    & 22 \\
    Embedding dim & 1024  & 2048 \\
    Number of heads & 8     & 16 \\
    MLP ratio &    8/3   & 8/3 \\
    $\#$param.  & 284M & 1.13B \\
    \bottomrule
    \end{tabular}%
    }
  \label{tab:arch}%
\end{table}%

Table~\ref{tab:io_structure} presents the detailed configuration of the embedding and prediction modules in EgoAgent, including the image projector ($\text{Proj}_i$), representation head/state prediction head ($\text{MLP}_i$), action projector ($\text{Proj}_a$) and action prediction head ($\text{MLP}_a$).
Note that the representation head and the state prediction head share the same architecture but have distinct weights.

\begin{table}[t]
\centering
\caption{Architecture of the embedding ($\text{Proj}_i$, $\text{Proj}_a$) and prediction ($\text{MLP}_i$, $\text{MLP}_a$) modules in EgoAgent. For details on module connections and functions, please refer to Fig.~2 in the main paper.}
\label{tab:io_structure}
\resizebox{\linewidth}{!}{%
\begin{tabular}{lcl}
\toprule
       & \multicolumn{1}{c}{Norm \& Activation} & \multicolumn{1}{c}{Output Shape}  \\
\midrule
\multicolumn{3}{l}{$\text{Proj}_i$ (\textit{Image projector})} \\
\midrule
Input image  & -          & 3$\times$224$\times$224 \\
Conv 2D (16$\times$16) & -       & Embedding dim$\times$14$\times$14    \\
\midrule
\multicolumn{3}{l}{$\text{MLP}_i$ (\textit{State prediction head} \& \textit{Representation head)}} \\
\midrule
Input embedding  & -          & Embedding dim \\
Linear & GELU       & 2048          \\
Linear & GELU       & 2048          \\
Linear & -          & 256           \\
Linear & -          & 65536     \\
\midrule
\multicolumn{3}{l}{$\text{Proj}_a$ (\textit{Action projector})} \\
\midrule
Input pose sequence  & -          & 4$\times$5$\times$17 \\
Conv 2D (5$\times$17) & LN, GELU   & Embedding dim$\times$1$\times$1    \\
\midrule
\multicolumn{3}{l}{$\text{MLP}_a$ (\textit{Action prediction head})} \\
\midrule
Input embedding  & -          & Embedding dim$\times$1$\times$1 \\
Linear & -          & 4$\times$5$\times$17     \\
\bottomrule
\end{tabular}%
}
\end{table}


\subsection{Training Configurations}
In Table~\ref{tab:training hyper}, we provide the detailed training hyper-parameters for experiments in the main manuscripts.

\begin{table}[ht]
  \centering
  \caption{Hyper-parameters for training EgoAgent.}
  \resizebox{0.86\linewidth}{!}{%
    \begin{tabular}{lc}
    \toprule
    Training Configuration & EgoAgent-300M/1B \\
    \midrule
    Training recipe: &  \\
    optimizer & AdamW~\cite{loshchilov2017decoupled} \\
    optimizer momentum & $\beta_1=0.9, \beta_2=0.999$ \\
    \midrule
    Learning hyper-parameters: &  \\
    base learning rate & 6.0E-04 \\
    learning rate schedule & cosine \\
    base weight decay & 0.04 \\
    end weight decay & 0.4 \\
    batch size & 1920 \\
    training iters & 72,000 \\
    lr warmup iters & 1,800 \\
    warmup schedule & linear \\
    gradient clip & 1.0 \\
    data type & float16 \\
    norm epsilon & 1.0E-06 \\
    \midrule
    EMA hyper-parameters: &  \\
    momentum & 0.996 \\
    \bottomrule
    \end{tabular}%
    }
  \label{tab:training hyper}%
\end{table}%

\clearpage


\end{document}
