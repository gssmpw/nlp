\section{Introduction}
\label{sec:introduction}

Protein complexes (such as oxygen-carrying hemoglobin, or keratin in hair, and thousands of others) are essential for cellular function, and understanding their interactions is crucial for our health and finding new disease treatments. Cryo-electron tomography (cryoET) generates 3D images, known as tomograms, with near-atomic resolution, showing proteins in their very complex and crowded natural environment~\cite{Young2023}. Therefore, cryoET has immense potential to unlock the mysteries of the cell.

Remarkably, a vast amount of data is publicly available in a standardized format through the cryoET data portal~\footnote{\url{https://cryoetdataportal.czscience.com/}}.
However, to fully utilize this data, it is necessary to automatically identify each protein molecule within these cryoET tomograms~\cite{Peck2024}.
To advance automated protein molecule recognition technology, the CZII - CryoET Object Identification competition was launched~\footnote{\url{https://www.kaggle.com/competitions/czii-cryo-et-object-identification}}.

In this competition, seven training samples, each with a size of $630 \times 630 \times 184$, are provided.
Each sample is annotated with the center positions of six types of particles~\cite{Peck2024}.  
Participants are required to predict the centers of these particles.  
The test dataset consists of approximately 500 samples to be predicted.
The predictions are evaluated using the $F_\beta$ metric with $\beta = 4$.
This metric emphasizes recall over precision, imposing heavy penalties on missed particles while being more tolerant of false positives.
Each particle type has a unique distance threshold for correct prediction, determined by its size.
The $F_\beta$ metric is calculated for each particle type, and the final score is obtained by computing the weighted sum using predefined weights based on the difficulty of each type.

This paper presents the detailed approach of the yu4u \& tattaka team, which achieved 4th place in the competition.
Our training source code is publicly available~\footnote{\url{https://github.com/yu4u/kaggle-czii-4th}}\footnote{\url{https://github.com/tattaka/czii-cryo-et-object-identification-public}}.
