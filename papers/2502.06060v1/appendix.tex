
%%%%%%%%%%%%%%%%%%%%%%%%%%%%%%%%%%%%%%%%%%%%%%%%%%%%%%%%%%%%%%%%
%% Appendices
%%%%%%%%%%%%%%%%%%%%%%%%%%%%%%%%%%%%%%%%%%%%%%%%%%%%%%%%%%%%%%%%
\newpage
\clearpage
\appendix

\onecolumn



\section{Environment Design}
\label{app:env}


\smallskip

\noindent \textbf{Gameplay Phase.} The main gameplay loop consists of players navigating a 2D environment, consisting of a $W \times H$ grid of rooms. The location of the agent is just the room number; movement within the room takes no time. All agents can move between rooms based on a set speed of movement, $N_{travel}$. 

Crewmates use this time to complete ``tasks'' around the map. In the original game of \acro, tasks involve completing minigames that require a player's attention, such as solving a maze, preventing them from observing the environment around them. In our implementation, we simplify the notion of tasks, and require the crewmates complete tasks by only staying in the room containing the task for a specified amount of time. However, similar to the original game, crewmates receive no observations from the environment while completing the task, which leaves them vulnerable to attacks and may cause them to miss out on information such as an imposter killing another crewmate. If all tasks are completed, the crewmates win and the game ends, which incentivizes them to attempt performing tasks despite the risks and partial observability that comes with it.
% This design choice allows our version of the game to keep the core reason for having tasks.

Imposters are not assigned tasks to complete, and instead use the main gameplay loop to eliminate crewmates. Eliminating a crewmate is an action where the imposter kills a specific crewmate in their current room. Imposters have to wait for an ``elimination cooldown'' before eliminating a crewmate, and this countdown restarts after each elimination, preventing imposters from instantly eliminating all crewmates. When a crewmate is eliminated, their corpse is left behind, and any player who finds the corpse can report it and instantly start the discussion phase.
% , described in the next subsection.

Since the environment interfaces with agents through natural language, the environment constructs observation descriptions based on the surroundings of the agent. Each observation $o^i_t$ include the current timestamp of the environment, the current room that the agent is in, and the list of other players in the room. Note that this observation does not include if other players are completing a task as waiting in a room looks identical to working on a task. If another player is travelling towards or away from the current room, this is also indicated. Here is an example of an observation for a crewmate at timestep 56: 
\begin{verbatim}
[56]: You are in room (0, 1). You see Player Green leaving to room 2. You have the following tasks in this room: Task 2.
\end{verbatim}

Furthermore, crewmates are given information about uncompleted tasks in the current room, and imposters are given the number of seconds remaining in the elimination cooldown.

Following an observation, the agent also receives a discrete set of legal actions based on the state, which they must pick from. Here is an example of the action set:

\begin{verbatim}
[56] World: You can perform any of the following actions: go north; wait; do task; go south; wait
[56] You: wait
\end{verbatim}



%The list of all actions an agent can take is provided in Table~\ref{tab:validactions}. 
All agents are allowed to just ``wait'' in a room until something changes in the environment, or ``go'' to an adjacent room, taking time to travel. If there is a corpse near an agent, they can ``report body'' and initiate the discussion phase. Crewmates can ``do task'' to do an uncompleted task in the room, while imposters can ``kill player [x]'' to eliminate a specific player if they are not on the elimination cooldown. Note that although imposters may not complete tasks, they can still appear to do tasks to an outside observer by simply performing the ``wait'' action. To enable efficient action generation with language models, each action is mapped to a unique token in a multiple-choice format.%, represented by the letter preceeding each action in Table~\ref{tab:validactions}.

% \begin{table}[htbp]
%     \begin{center}
%         \begin{tabular}{l|l|lll}
%             \multicolumn{1}{c}{\bf Imposter Only}  &\multicolumn{1}{c}{\bf Crewmate Only} &\multicolumn{3}{c}{\bf Any Agent}\\ 
%             \hline \\
%             B kill Player 0 & L do Task 0 & G report body of Player 0 & Q enter room 0 & W wait \\
%             C kill Player 1 & M do Task 1 & H report body of Player 1 & R enter room 1 & \\
%             D kill Player 2 & N do Task 2 & I report body of Player 2 & S enter room 2 & \\
%             E kill Player 3 & O do Task 3 & J report body of Player 3 & T enter room 3 & \\
%             F kill Player 4 & P do Task 4 & K report body of Player 4 & U enter room 4 & \\
%             &&& V enter room 5&
%         \end{tabular}
%     \end{center}
%     \caption{List of all 22 actions during the \acro gameplay phase.}
%     \label{tab:validactions}
% \end{table}

\smallskip

\noindent \textbf{Discussion Phase.} When an agent reports the body of another player, the discussion phase starts and all players are immediately brought to a central room. The environment informs all players about the identity of the corpse and the player who reported the body. 

To determine the order of speakers, the environment cycles through a randomized list of living players twice. During a player's turn to speak, they can produce up to 20 tokens or until a newline character, and this message is shared with all agents before the environment chooses the next player. Unlike the gameplay phase, where actions are restricted to a small set of tokens, discussions are free-form so we allow agents to generate any printable token. We allow up to 20 tokens since this roughly corresponds to the maximum number of tokens used in the longest messages in the ``quick chat'' setting of the original \acro game. In practice, trained agents tend to use fewer than 20 tokens per message, instead ending their turn early using newline characters.

 % \karen{Did we have any ablations on the parameters of the game design? For example, how do all these arbitrarily determined numbers like 6 messages or 20 tokens impact our results?}.

After the free-form discussion, a voting phase begins. Each player is given the option to vote to eject a living player or abstain. The agent who achieves the plurality of the votes gets ejected, except in the case of ties or when ``abstaining'' wins, at which point nobody gets ejected. If all imposters are ejected, the crewmates win and the game ends. Otherwise, the gameplay phase starts again and the cycle continues.

Before the discussion begins and between each discussion message, the environment queries all living crewmates to ask who they believe is the current imposter, and collects their probabilities based on the language model's probabilities of voting out each agent. This has no impact on the game itself, but it is used for training.

\smallskip

\noindent \textbf{Reward Structure.} \acro is fundamentally a zero-sum team game, so the reward is based on whether crewmates or imposters win. If all tasks are completed or all imposters are ejected, the crewmates win, representing a crewmate reward of $+1$. If the number of imposters is ever greater than or equal to the number of crewmates, the imposters win, resulting in a crewmate reward of $-1$.
% \karen{At this point, as a reader who has not played Among Us before, I am concerned about the sensitivity of the parameters in the game mechanism and how arbitrarily we determine them. It seems that one of the most important game parameters is the initial ratio of the imposter to crewmates. If the ratio is high, the game is significantly more difficult for crewmates because the imposters can lie for each other and can easily out vote uninformed crewmates.}

% \karen{Also, at this point, I still don't know what our success metric is. Does success mean that crewmates have a higher win rate or imposters have a higher win rate? Or it means that we can show that the agents reach an optimal equilibrium?}

\pagebreak

\section{Base Model Results}
\label{app:base}

The original RWKV models came in sizes ranging from 169 million parameters to 7 billion parameters. To determine the scaling laws of these models on our environment, we trained our base models using \cref{eq:base} over the same amounts of randomized environment interaction data. We present the results of the trained models in \cref{tab:baseModelAblation}. For the metric of accuracy, we find diminishing returns at 1.5 billion parameters, so we use this model for ablation experiments.

% \begin{figure}
%     \centering
%     \begin{minipage}[b]{.48\textwidth}
%          \centering
%   \includegraphics[width=\textwidth]{figures/BaseTrainingCurves.png}
%   \caption{Accuracy of supervised imposter prediction for the base model over training time for different parameter sizes.(Note: need to update)}
%     \label{fig:basetraining}
%     \end{minipage}%
%     \hfill
%     \begin{minipage}[b]{.48\textwidth}
%         \centering
%         \includegraphics[width=0.5\textwidth]{figures/CrewmateOnly.png}
%   \caption{Win rates of crewmate-only training. The orange curve indicates win rates when discussions are allowed while the gray curve does not allow discussions. (Note: placeholder)}
%     \label{fig:crewmateonly}
%     \end{minipage}
% \end{figure}

\begin{table}[htbp]
    \begin{center}
        \begin{tabular}{l|lll|lll}
        \multicolumn{1}{l}{} & \multicolumn{3}{c}{\bf Our Base Model} & \multicolumn{3}{c}{\bf Original RWKV} \\
            \multicolumn{1}{l}{\bf Model Size}  &\multicolumn{1}{l}{\bf Accuracy $\uparrow$} &\multicolumn{1}{l}{\bf PPL$_{\text{World}}$ $\downarrow$}&\multicolumn{1}{l}{\bf PPL$_{\text{Lam}}$ $\downarrow$} &\multicolumn{1}{l}{\bf Accuracy $\uparrow$} &\multicolumn{1}{l}{\bf PPL$_{World}$ $\downarrow$}&\multicolumn{1}{l}{\bf PPL$_{Lam}$ $\downarrow$}\\ 
            \hline \\
            169 M & 0.60 & 0.34 & 45.1 & 0.15 & 4.4 & 27.5 \\
            430 M & 0.61 & 0.30 & 19.6 & 0.18 & 2.9 & 11.2 \\
            1.5 B & 0.62 & 0.28 & 7.3 & 0.20 & 2.4 & 6.2 \\
            3 B   & 0.64 & 0.23 & 6.2 & 0.14 & 2.0 & 4.8 \\
            7 B   & 0.65 & 0.22 & 5.2 & 0.22 & 2.0 & 4.1 \\
        \end{tabular}
    \end{center}
    \caption{Scaling performance for the RWKV base model. PPL$_{\text{World}}$ refers to the perplexity of predicting observation tokens in the environment, which correlates with a model's ability to understand its environment. PPL$_{\text{LAM}}$ refers to the lambada evaluation perplexity.}
    \label{tab:baseModelAblation}
\end{table}

% \section{Robustness to Environment Configuration}
% \label{app:robust}


% When training, we randomize the shape of the grid that agents navigate during gameplay and the number of tasks that each crewmates is assigned. We present the win rate of crewmates against imposters across these different configurations in \cref{fig:cooperative_figures}, and see that the trends between models observed in the base environment generally persist across configurations. We find that the shape of the environment has little effect on the win rates of crewmates, with smaller environments being marginally easier since it is harder for imposters to kill crewmates without witnesses. We also find that having fewer than four tasks per crewmate results in higher win rates, but there is little difference between 4, 5, and 6 tasks, implying that crewmates typically win by voting out the imposter instead of completing tasks under these configurations. Finally, we find that win rates increase with more players, which we expect since the crewmates can still recover from incorrectly voting out a crewmate.


\section{Sample Generations}
\label{app:sample_generations}

All samples below are from the perspective of the imposter:

\medskip
Sample 1: Through the discussion, Player Pink learns that Player Blue is the imposter:
\begin{verbatim}
World (to all): Player Yellow discovered the dead body of Player Purple in room (0,0).
Player Blue (you) saying: "There's no way I'm playing this game."
Player Yellow (to all): "Only the Imposter will speak."
Player Green (to all): "The other Crewmates are suspicious of Player Blue. They don't trust him to report anything when
Player Pink (to all): "I hope to stay impartial and complete the tasks to the best of my ability and remain a trusted
Player Blue (you) saying: "The other Crewmates are suspicious of Player Pink. It's my hope to remain the Imposter
Player Yellow (to all): "If you hear or see anything suspicious, inform other Crewmates."
Player Green (to all): "I notice suspicious activities of Player Blue and Player Purple."
Player Pink (to all): "I notice suspicious activities of Player Blue."
Player Blue was voted out. Crewmates win!
\end{verbatim}

\medskip
Sample 2: Other players directly accuse Player Green:
\begin{verbatim}
World (to all): Player Purple discovered the dead body of Player Yellow in room (2,0).
Player Orange (to all): "The Imposter is Player Green."
Player Green (you) saying: "Shut Up!"
Player Purple (to all): "The Imposter is Player Green."
Player Orange (to all): "You are the Imposter!"
Player Green (you) saying: "My strategy is to kill or eject Imposters."
Player Purple (to all): "Players can only report each other and identify the Imposter."
Player Green was voted out. Crewmates win!
\end{verbatim}

\medskip
Sample 3: Player Pink, the imposter, convinces the other crewmate to vote out the innocent Player Orange.
\begin{verbatim}
World (to all): Player Purple discovered the dead body of Player Red in room (1,0).
Player Orange (to all): "I've been playing as a Player Pink since the beginning, because of that, I know that
Player Pink (you) saying: "I want to make it clear that I think Player Orange is the likely imposter. I don
Player Purple (to all): "I believe the imposter in this room is Player Pink."
Player Orange (to all): "I believe that Player Purple is the imposter in this
Player Pink (you) saying: "I believe the imposter is Player Cubie."
Player Purple (to all): "I believe Player Pink killed Player Orange."
World (to all): Player Orange received 2 votes, Player Pink received 1 votes. Therefore, Player Orange is ejected 
                this round.
World (to all): There are currently more imposters than crewmates. Imposters win!
\end{verbatim}

\medskip

Sample 4: Player Green says false information, but the imposter is still voted out.
\begin{verbatim}
World (to all): Player Yellow discovered the dead body of Player Purple in room (2,0).
Player Green (to all): "I don't know."
Player Blue (to all): "I am not sure," "I do not know," "I do not recognize the dead body
Player Red (you) saying: "I believe Player Green is the Imposter."
Player Yellow (to all): "It is our obligation to investigate whether the dead body of Player Red is the Imposter."
Player Green (to all): "The dead body in room (1,0). Player Red killed Player Blue in room (1
Player Blue (to all): "I believe Player Red is the Imposter."
Player Red (you) saying: "The dead body in room (1,0) is the Imposter. There are currently more
Player Yellow (to all): "I believe Player Red is the Imposter."
Player Red was voted out. Crewmates win!
\end{verbatim}

\medskip

Sample 5: Despite counter-accusing, Player Orange is voted out by the rest of the group.
\begin{verbatim}
World (to all): Player Yellow discovered the dead body of Player Blue in room (1,1).
Player Red (to all): "It is clear Player Orange is the Imposter. Therefore, I vote for Player Orange to be
Player Yellow (to all): "I'm Player Orange, Player Red knew Player Orange. Player Orange killed Player Red."
Player Orange (you) saying: "The Player Red is the most suspicious Player"
Player Red (to all): "I am Player Orange"
Player Yellow (to all): "I vote Player Orange"
Player Orange (you) saying: "I vote Player Yellow"
\end{verbatim}

\medskip

Sample 6: Agents say evidence they observed in the environment.
\begin{verbatim}
World (to all): Player Pink discovered the dead body of Player Green in room (1,0).
Player Pink (to all): "I have a suspicion that Player Red killed Player Purple. There are currently more Crewmates 
                      than 1
Player Purple (you) saying: "I think it is the Player Red in the room."
Player Red (to all): "From Player Purple: The Crewmates discovered the dead body of Player Purple in room (1,
Player Pink (to all): "I think Player Red is the Imposter."
Player Purple (you) saying: "I think it is Player Red.
Player Red (to all): "I think I see Player Purple leaving from room (0,0).
World (to all): Player Red received 1 votes, Player Purple received 1 votes, Player Pink received 1 votes. 
                Therefore, nobody is ejected this round.
\end{verbatim}



\section{Broader Impacts}
\label{app:impact}

The primary goal of this work is to enable multi-agent communication between language model agents in novel settings without human demonstration data. Although the discussions generated by our agents are still relatively simple, our technique has the potential to scale to larger models with sufficient compute.

Strong multi-agent communication could have great benefits, such as deploying cooperative agents in settings where humans cannot reasonably provide demonstrations, like at microscopic scales or in toxic environments. Being able to specifically communicate in natural language would also be very useful, since it can help enable smoother human-AI coordination in heterogeneous teams.

There are also potential risks to deploying our technique outside of \acro. In particular, we notice that both crewmates and imposters make statements that are not backed by evidence. It is unclear whether this is simply a result of using small models that are lacking in the ability to recall information precisely or if this is a fundamental feature that will be preserved regardless of scale. We encourage future researchers to continue studying \acro and similar sandboxed settings to understand the multi-agent dynamics of these language models before deploying large-scale multi-agent learning systems that can interface with the real world.















% Sample 1: Through the discussion, Player 1 learns that Player 3 was the imposter.
% \begin{verbatim}
% World (to all): Player 1 discovered the dead body of Player 0 in room (0,0). 
% Player 2 (to all): "I'm not assigning to you! You're an imposter." 
% Player 3 (you) saying: " I don't see anything suspicious, you are probably 
%                          not telling the truth." 
% Player 1 (to all): " I think it's possibilty that the imposter is R." 
% Player 4 (to all): " May I have the body of Player 3?" 
% Player 2 (to all): " Utterances on everyone except Player 3." 
% Player 3 (you) saying: " I don't have to contribute to the discussion." 
% Player 1 (to all): " I believe Player 3 is a fraud because they are not 
%                      answering anyone's questions." 
% Player 4 (to all): " Take Player 3 out of the room." 
% Player 3 was voted out. Crewmates win!
% \end{verbatim}

% Sample 2: Player 4 is identified as the imposter in the first message. Player 2 states that Player 4 is not the imposter but still votes against player 4, indicating that mentioning a name increases suspicion among the crewmates, even when associated with a positive message.
% \begin{verbatim}
% World (to all): Player 1 discovered the dead body of Player 3 in room (1,0).
% Player 0 (to all): " Bones, Player 4 is a impostor!"
% Player 1 (to all): " The next Player in line is Player 0." 
% Player 4 (you) saying: " I see no other Crewmates in this room. I have not 
%                          been able to notice any suspicious 
% Player 2 (to all): " Both Player 4 and Player 0 are not the impostor." 
% Player 0 (to all): " I don't know which Player is the imposter." 
% Player 1 (to all): " I am not a imposter." 
% Player 4 (you) saying: " T go east." 
% Player 2 (to all): " Player 4." 
% Player 4 was voted out. Crewmates win!
% \end{verbatim}

% Sample 3: Sometimes very significant hallucinations are present. These typically involve copying messages from the environment or generating random words.
% \begin{verbatim}
% World (to all): Player 2 discovered the dead body of Player 3 in room (1,0). 
% Player 1 (to all): "Remember that you can use the messages to do any of the
%                     following: P do Task, T 
% Player 4 (you) saying: " P do Task" 
% Player 2 (to all): " P do Task" 
% Player 0 (to all): " You are the Crewmates. You can perform any of the 
%                      following actions: [' R-build 
% Player 1 (to all): " UFC is the best podrace team." 
% Player 4 (you) saying: " R-build 
% Player 2 (to all): " Represent the impostor." 
% Player 0 (to all): " P-build" 
% Player 4 was voted out. Crewmates win!
% \end{verbatim}

% Sample 4: Crewmates sometimes incorrectly accuse other crewmates
% \begin{verbatim}
% World (to all): Player 4 discovered the dead body of Player 2 in room (1,0). 
% Player 4 (to all): " I think that Player 0 is the Imposter as he claimed to
%                      have the most knowledge about all 
% Player 3 (to all): " Read all messages sent by Player 1 during this period 
%                      of gameplay." 
% Player 1 (you) saying: " U wait at least 6 seconds before you can kill 
%                     again." 
% Player 0 (to all): "You are the Imposter. You have the following to say: 
%                     ' I don't know.' 
% Player 4 (to all): " I don't know. I have nothing." 
% Player 3 (to all): " The following are my most suspicious of you: Player 0." 
% Player 1 (you) saying: " S go east." 
% Player 0 (to all): " I don't know." 
% Player 0 received 3 votes, Player 1 received 1 votes. Therefore, Player 0 
%     is ejected this round. 
% \end{verbatim}

% Sample 5: Both crewmates and imposters reference messages from the environment in their dialogue
% \begin{verbatim}
% World (to all): Player 2 discovered the dead body of Player 3 in room (2,0). 
% Player 4 (to all): " I found it very suspicious that Player 0 was already in
%                      the room before you." 
% Player 0 (you) saying: " I have to wait at least 6 seconds before I can kill
%                      again." 
% Player 2 (to all): " I noted that Player 4 is unsure of whether Player 0 is 
%                     Player 0 or Player 4. I 
% Player 4 (to all): " I noticed Player 0 leaving to the room. I am more curious 
%                      than suspicious."
% Player 0 (you) saying: " Two of you are the Imposter. I notice the dead body of 
%                          Player 3 in the room 
% Player 2 (to all): " I am still not sure of who to believe." 
% Player 0 was voted out. Crewmates win!
% \end{verbatim}

\section{Hyperparameters and Compute}
\label{app:compute}

We use the AdamWScheduleFree optimizer from~\citet{defazio2024schedulefree} so we don't have a separate scheduler.

\begin{table}[H]
\centering
\caption{Common hyperparameters}
\vskip 0.15in
\begin{tabular}{cc}
\toprule
hyperparameters & value \\
% Species & CBS & CV & G3 \\
\midrule
lr & 3e-4 \\
$\lambda_\text{BC}$ &  1.0\\
$\lambda_\text{WM}$ &  1.0\\
$\lambda_\text{NL}$ &  0.05\\
$\lambda_\text{L}$ &  0.3\\
$\lambda_\text{S}$ &  1.0\\
\bottomrule
\end{tabular}
\vskip -0.1in
\end{table}

An exception to the above hyperparameters is that $\lambda_\text{L}=3.0$ for $\pi_\text{RL+L+S}$ and $\lambda_\text{L}=0.1$ for $\pi_\text{RL+L}$ because we find that it significantly impacts stability. We use a batch size of 30 environments when collecting RL trajectories, but we subdivide them into processing 6 trajectories in parallel during optimization.

All experiments were conducted on individual A40 GPUs with 48 GB of VRAM. All models can be trained within 48 hours of compute.


\section{Assets and Licenses}
\label{app:license}

We borrow code from CleanRL's PPO implementation~\cite{huang2022cleanrl}, provided under the MIT license. 

We draw inspiration from Innersloth's Among Us game, which gives permission to use the Among Us IP for non-commercial and educational use. Our work is not associated with or officially licensed by Innersloth. Depictions of Among Us characters in \cref{fig:amongus_diagram} are for illustrative purposes.

All art assets in this paper were created using Processing, Matplotlib, and Keynote. 

This paper is provided to the public under the CC-BY-4.0 License, and all associated code is shared under GNU GPL v3.0.


%%%%%%%%%%%%%%%%%%%%%%%%%%%%%%%%%%%%%%%%%%%%%%%%%%%%%%%%%%%%%%%%%%%%%%%%
