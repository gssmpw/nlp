\section{Conclusion}

In this work, we propose to collaborate deterministic models and probabilistic models for effective probabilistic urban spatiotemporal predictions. By decomposing spatiotemporal data into conditional mean predicted by deterministic models and residual learned by diffusion models, bridging the gap between deterministic and probabilistic predictions, our CoST achieves dual advantages: precise capture of primary patterns and non-linear variations. Experiments across seven real-world datasets demonstrate 20\% performance gains over state-of-the-art baselines. This study provides an efficient solution for urban spatiotemporal prediction that balances pattern learning and uncertainty modeling, holding significant theoretical and practical value for smart city decision support.
 